% Template rapport
% CASSAYRE Valentin

\documentclass[a4paper, 11pt]{article} % article
\usepackage[top=3cm, bottom=3cm, left = 2cm, right = 2cm]{geometry} % marges
\geometry{a4paper} % format de la page
\usepackage{ae,lmodern} % police vectorielle
\usepackage[utf8]{inputenc} % encodage du document brut
\usepackage[T1]{fontenc} % encodage d'export
\usepackage{textcomp} % symboles
\usepackage{amsmath,amsthm,amssymb,amsfonts} % polices mathématiques
\usepackage{bm}  
\usepackage[pdftex,bookmarks,colorlinks,breaklinks]{hyperref}  
\hypersetup{linkcolor=black,citecolor=black,filecolor=black,urlcolor=black} % liens en noir
\usepackage{memhfixc} 
\usepackage{pdfsync}  
\usepackage{fancyhdr}
\usepackage[round]{natbib}
\usepackage{color}
\usepackage{iftex}
\usepackage{xcolor}
\pagestyle{fancy}
\usepackage[french]{babel} % langue

% Images
\usepackage{graphicx} % image
\graphicspath{ {./images/} } % chemin des images
\usepackage{pgfplots} % images pgf (matplotlib ou matlab)
\usepackage{tikz}
\usepackage[european resistor, european voltage, european current]{circuitikz}
\usetikzlibrary{arrows,shapes,positioning}
\usetikzlibrary{decorations.markings,decorations.pathmorphing,
decorations.pathreplacing}
\usetikzlibrary{calc,patterns,shapes.geometric}
% Physique
\usepackage[squaren,Gray]{SIunits} % unités de physique
% Informatique
\usepackage{minted} % insérer du code général
\usepackage{pythonhighlight} % insérer du code python
\usepackage{pgf} % charger des plots matplotlib en pgf

\title{
Pointé automatique du temps d'arrivé des ondes P et S \\
\large Rapport terminal de projet informatique \\
1A EOST \\
}

\author{Bastien PITOISET et Valentin CASSAYRE}
%\date{}

\begin{document}
\maketitle
\tableofcontents

\newpage

\section{Introduction}

La libération d'énergie lors d'un séisme génère des ondes qui vont se propager différemment selon leurs caractéristiques. Les ondes de surfaces transmettent la plus grande partie de l'énergie relâchée, mais ne se propagent qu'à la surface de la Terre. Les ondes de volume, plus rapides, se propagent à l'intérieur de la Terre. Des ondes de volume, nous pouvons distinguer les ondes de compression, les ondes P, des ondes de cisaillement, les ondes S. Les ondes P sont les ondes les plus rapides avec une vitesse de propagation de $6 \kilo\meter\per\second$ près de la surface de la Terre.

L'ensemble de ces champs d'ondes peuvent être mesurées au niveau de capteurs sismiques, par exemple dans une station de surveillance sismique. Ces stations enregistrent en continu les déplacement du sol et ces mesurent permettent de détecter d'éventuels séismes à la suite d'analyses. Et la détermination du temps d'arrivée des différentes ondes est une étape essentiel dans l'analyse de ces signaux. Le temps d'arrivée d'une onde correspond au début de sa phase, période pendant laquelle l'onde est enregistrée. Cette identification est très importante dans divers domaines, il permet par exemple de localiser le foyer ou de récolter des données sur la structure intérieure de la Terre.

L'objet de ce projet est donc de réaliser un programme permettant de détecter un séisme puis de pointer automatiquement le début des phases P et S. 

Il existe trois grandes familles d’algorithmes de détection et de pointage automatique [d'après \cite{probatoire}] :
\begin{itemize}
  \item Les algorithmes de détection par calcul de l’énergie comparent chaque valeur avec la moyenne des valeurs qui la précèdent. 
  \item Les algorithmes basés sur des méthodes autorégressives recherchent les modélisations du bruit et du signal sismique. 
  \item Les algorithmes utilisant des réseaux de neurones artificiels, après une phase d’apprentissage, sont capables de détecter les différentes ondes sismiques. 
\end{itemize}

Pour ce projet nous avons choisi de travailler avec les algorithmes de détection par calcul de l’énergie. Ce sont historiquement ces algorithmes qui ont été utilisés en premier dans les années 80. Ils présentent l'avantage d'être les plus simples et de pouvoir à la fois détecter un potentiel séisme et de pointer le début des phases sismiques. 

Ces algorithmes sont basée sur un signal d'entrée qui peut-être corrigé ou modifié et calculent une fonction caractéristique de ce signal. Cette fonction qui caractérise le signal peut ensuite être utilisée pour détecter un potentiel séisme et pour pointer ses différentes phases.

\section{Acquisition des données}

Nous avons choisi de travailler essentiellement avec le langage Python, un langage haut niveau qui présente de nombreux modules notamment dans le domaine de la sismologie. Nous nous sommes basés sur des enregistrements de séismes locaux autour de Strasbourg par des stations à proximité. Les données sont celles du RéSiF et ont été chargées avec le module \cite{obspy}. Un traitement de base a été effectué au préalable consistant en un filtre passe bande entre les fréquences de 2 \hertz  et 10 \hertz. Le signal obtenu est ensuite stocké dans deux arrays Numpy, le temps dans $times$ et l'amplitude correspondante dans $data$ et le module \cite{obspy} n'est plus utilisé dans la suite, l'intérêt étant de présenter ces algorithmes.

\section{Fonction caractéristique}

La famille d'algorithme étudiée ici consiste en une analyse de l'énergie des champs d'onde incidents et de son évolution au cours du temps. \cite{allen1978} a introduit le concept de fonction caractéristique, qui caractérise un signal. Elle est obtenue par une ou plusieurs transformations non linéaires du sismogramme et doit augmenter brusquement au moment de l'arrivée d'un champ d'onde sismique. Elle se base sur un signal qui peut-être modifié ou corrigé. Dans notre cas, nous avons déjà corrigé le signal notamment en le filtrant. La modification du signal s'effectue par des transformations non linéaires. En général les signaux modifiés sont positifs et sont donc appelés enveloppe par abus de langage. 

\subsection{Modification du signal}

\subsubsection{Transformées simples}

\cite{vanderkulk1965} introduit l'utilisation de la valeur absolue du signal comme enveloppe. Cette approximation permettait de faire des économies de calcul significatives par rapport au carré des valeurs des amplitudes. Mais l'augmentation de la puissance de calculs a permis de retirer cette barrière. \cite{allen1978} généralise l'utilisation du carré des valeurs des amplitudes pour calculer la fonction caractéristique. Ces deux transformations correspondent aux transformations non linéaires les plus simples permettant d'aboutir à un signal positif, mais ne prennent pas en compte les valeurs voisines et donc les variations.

\subsubsection{Enveloppe de Stewart}

\begin{figure}[!ht]
    \centering
    \scalebox{.9}{%% Creator: Matplotlib, PGF backend
%%
%% To include the figure in your LaTeX document, write
%%   \input{<filename>.pgf}
%%
%% Make sure the required packages are loaded in your preamble
%%   \usepackage{pgf}
%%
%% Also ensure that all the required font packages are loaded; for instance,
%% the lmodern package is sometimes necessary when using math font.
%%   \usepackage{lmodern}
%%
%% Figures using additional raster images can only be included by \input if
%% they are in the same directory as the main LaTeX file. For loading figures
%% from other directories you can use the `import` package
%%   \usepackage{import}
%%
%% and then include the figures with
%%   \import{<path to file>}{<filename>.pgf}
%%
%% Matplotlib used the following preamble
%%   \usepackage{fontspec}
%%
\begingroup%
\makeatletter%
\begin{pgfpicture}%
\pgfpathrectangle{\pgfpointorigin}{\pgfqpoint{6.000000in}{3.000000in}}%
\pgfusepath{use as bounding box, clip}%
\begin{pgfscope}%
\pgfsetbuttcap%
\pgfsetmiterjoin%
\definecolor{currentfill}{rgb}{1.000000,1.000000,1.000000}%
\pgfsetfillcolor{currentfill}%
\pgfsetlinewidth{0.000000pt}%
\definecolor{currentstroke}{rgb}{1.000000,1.000000,1.000000}%
\pgfsetstrokecolor{currentstroke}%
\pgfsetdash{}{0pt}%
\pgfpathmoveto{\pgfqpoint{0.000000in}{0.000000in}}%
\pgfpathlineto{\pgfqpoint{6.000000in}{0.000000in}}%
\pgfpathlineto{\pgfqpoint{6.000000in}{3.000000in}}%
\pgfpathlineto{\pgfqpoint{0.000000in}{3.000000in}}%
\pgfpathlineto{\pgfqpoint{0.000000in}{0.000000in}}%
\pgfpathclose%
\pgfusepath{fill}%
\end{pgfscope}%
\begin{pgfscope}%
\pgfsetbuttcap%
\pgfsetmiterjoin%
\definecolor{currentfill}{rgb}{0.933333,0.933333,0.933333}%
\pgfsetfillcolor{currentfill}%
\pgfsetlinewidth{0.000000pt}%
\definecolor{currentstroke}{rgb}{0.000000,0.000000,0.000000}%
\pgfsetstrokecolor{currentstroke}%
\pgfsetstrokeopacity{0.000000}%
\pgfsetdash{}{0pt}%
\pgfpathmoveto{\pgfqpoint{0.649081in}{1.713187in}}%
\pgfpathlineto{\pgfqpoint{5.745833in}{1.713187in}}%
\pgfpathlineto{\pgfqpoint{5.745833in}{2.703703in}}%
\pgfpathlineto{\pgfqpoint{0.649081in}{2.703703in}}%
\pgfpathlineto{\pgfqpoint{0.649081in}{1.713187in}}%
\pgfpathclose%
\pgfusepath{fill}%
\end{pgfscope}%
\begin{pgfscope}%
\pgfpathrectangle{\pgfqpoint{0.649081in}{1.713187in}}{\pgfqpoint{5.096752in}{0.990516in}}%
\pgfusepath{clip}%
\pgfsetbuttcap%
\pgfsetroundjoin%
\pgfsetlinewidth{0.501875pt}%
\definecolor{currentstroke}{rgb}{0.698039,0.698039,0.698039}%
\pgfsetstrokecolor{currentstroke}%
\pgfsetdash{{1.850000pt}{0.800000pt}}{0.000000pt}%
\pgfpathmoveto{\pgfqpoint{0.649081in}{1.713187in}}%
\pgfpathlineto{\pgfqpoint{0.649081in}{2.703703in}}%
\pgfusepath{stroke}%
\end{pgfscope}%
\begin{pgfscope}%
\pgfsetbuttcap%
\pgfsetroundjoin%
\definecolor{currentfill}{rgb}{0.000000,0.000000,0.000000}%
\pgfsetfillcolor{currentfill}%
\pgfsetlinewidth{0.803000pt}%
\definecolor{currentstroke}{rgb}{0.000000,0.000000,0.000000}%
\pgfsetstrokecolor{currentstroke}%
\pgfsetdash{}{0pt}%
\pgfsys@defobject{currentmarker}{\pgfqpoint{0.000000in}{0.000000in}}{\pgfqpoint{0.000000in}{0.048611in}}{%
\pgfpathmoveto{\pgfqpoint{0.000000in}{0.000000in}}%
\pgfpathlineto{\pgfqpoint{0.000000in}{0.048611in}}%
\pgfusepath{stroke,fill}%
}%
\begin{pgfscope}%
\pgfsys@transformshift{0.649081in}{1.713187in}%
\pgfsys@useobject{currentmarker}{}%
\end{pgfscope}%
\end{pgfscope}%
\begin{pgfscope}%
\pgfpathrectangle{\pgfqpoint{0.649081in}{1.713187in}}{\pgfqpoint{5.096752in}{0.990516in}}%
\pgfusepath{clip}%
\pgfsetbuttcap%
\pgfsetroundjoin%
\pgfsetlinewidth{0.501875pt}%
\definecolor{currentstroke}{rgb}{0.698039,0.698039,0.698039}%
\pgfsetstrokecolor{currentstroke}%
\pgfsetdash{{1.850000pt}{0.800000pt}}{0.000000pt}%
\pgfpathmoveto{\pgfqpoint{1.498540in}{1.713187in}}%
\pgfpathlineto{\pgfqpoint{1.498540in}{2.703703in}}%
\pgfusepath{stroke}%
\end{pgfscope}%
\begin{pgfscope}%
\pgfsetbuttcap%
\pgfsetroundjoin%
\definecolor{currentfill}{rgb}{0.000000,0.000000,0.000000}%
\pgfsetfillcolor{currentfill}%
\pgfsetlinewidth{0.803000pt}%
\definecolor{currentstroke}{rgb}{0.000000,0.000000,0.000000}%
\pgfsetstrokecolor{currentstroke}%
\pgfsetdash{}{0pt}%
\pgfsys@defobject{currentmarker}{\pgfqpoint{0.000000in}{0.000000in}}{\pgfqpoint{0.000000in}{0.048611in}}{%
\pgfpathmoveto{\pgfqpoint{0.000000in}{0.000000in}}%
\pgfpathlineto{\pgfqpoint{0.000000in}{0.048611in}}%
\pgfusepath{stroke,fill}%
}%
\begin{pgfscope}%
\pgfsys@transformshift{1.498540in}{1.713187in}%
\pgfsys@useobject{currentmarker}{}%
\end{pgfscope}%
\end{pgfscope}%
\begin{pgfscope}%
\pgfpathrectangle{\pgfqpoint{0.649081in}{1.713187in}}{\pgfqpoint{5.096752in}{0.990516in}}%
\pgfusepath{clip}%
\pgfsetbuttcap%
\pgfsetroundjoin%
\pgfsetlinewidth{0.501875pt}%
\definecolor{currentstroke}{rgb}{0.698039,0.698039,0.698039}%
\pgfsetstrokecolor{currentstroke}%
\pgfsetdash{{1.850000pt}{0.800000pt}}{0.000000pt}%
\pgfpathmoveto{\pgfqpoint{2.347998in}{1.713187in}}%
\pgfpathlineto{\pgfqpoint{2.347998in}{2.703703in}}%
\pgfusepath{stroke}%
\end{pgfscope}%
\begin{pgfscope}%
\pgfsetbuttcap%
\pgfsetroundjoin%
\definecolor{currentfill}{rgb}{0.000000,0.000000,0.000000}%
\pgfsetfillcolor{currentfill}%
\pgfsetlinewidth{0.803000pt}%
\definecolor{currentstroke}{rgb}{0.000000,0.000000,0.000000}%
\pgfsetstrokecolor{currentstroke}%
\pgfsetdash{}{0pt}%
\pgfsys@defobject{currentmarker}{\pgfqpoint{0.000000in}{0.000000in}}{\pgfqpoint{0.000000in}{0.048611in}}{%
\pgfpathmoveto{\pgfqpoint{0.000000in}{0.000000in}}%
\pgfpathlineto{\pgfqpoint{0.000000in}{0.048611in}}%
\pgfusepath{stroke,fill}%
}%
\begin{pgfscope}%
\pgfsys@transformshift{2.347998in}{1.713187in}%
\pgfsys@useobject{currentmarker}{}%
\end{pgfscope}%
\end{pgfscope}%
\begin{pgfscope}%
\pgfpathrectangle{\pgfqpoint{0.649081in}{1.713187in}}{\pgfqpoint{5.096752in}{0.990516in}}%
\pgfusepath{clip}%
\pgfsetbuttcap%
\pgfsetroundjoin%
\pgfsetlinewidth{0.501875pt}%
\definecolor{currentstroke}{rgb}{0.698039,0.698039,0.698039}%
\pgfsetstrokecolor{currentstroke}%
\pgfsetdash{{1.850000pt}{0.800000pt}}{0.000000pt}%
\pgfpathmoveto{\pgfqpoint{3.197457in}{1.713187in}}%
\pgfpathlineto{\pgfqpoint{3.197457in}{2.703703in}}%
\pgfusepath{stroke}%
\end{pgfscope}%
\begin{pgfscope}%
\pgfsetbuttcap%
\pgfsetroundjoin%
\definecolor{currentfill}{rgb}{0.000000,0.000000,0.000000}%
\pgfsetfillcolor{currentfill}%
\pgfsetlinewidth{0.803000pt}%
\definecolor{currentstroke}{rgb}{0.000000,0.000000,0.000000}%
\pgfsetstrokecolor{currentstroke}%
\pgfsetdash{}{0pt}%
\pgfsys@defobject{currentmarker}{\pgfqpoint{0.000000in}{0.000000in}}{\pgfqpoint{0.000000in}{0.048611in}}{%
\pgfpathmoveto{\pgfqpoint{0.000000in}{0.000000in}}%
\pgfpathlineto{\pgfqpoint{0.000000in}{0.048611in}}%
\pgfusepath{stroke,fill}%
}%
\begin{pgfscope}%
\pgfsys@transformshift{3.197457in}{1.713187in}%
\pgfsys@useobject{currentmarker}{}%
\end{pgfscope}%
\end{pgfscope}%
\begin{pgfscope}%
\pgfpathrectangle{\pgfqpoint{0.649081in}{1.713187in}}{\pgfqpoint{5.096752in}{0.990516in}}%
\pgfusepath{clip}%
\pgfsetbuttcap%
\pgfsetroundjoin%
\pgfsetlinewidth{0.501875pt}%
\definecolor{currentstroke}{rgb}{0.698039,0.698039,0.698039}%
\pgfsetstrokecolor{currentstroke}%
\pgfsetdash{{1.850000pt}{0.800000pt}}{0.000000pt}%
\pgfpathmoveto{\pgfqpoint{4.046916in}{1.713187in}}%
\pgfpathlineto{\pgfqpoint{4.046916in}{2.703703in}}%
\pgfusepath{stroke}%
\end{pgfscope}%
\begin{pgfscope}%
\pgfsetbuttcap%
\pgfsetroundjoin%
\definecolor{currentfill}{rgb}{0.000000,0.000000,0.000000}%
\pgfsetfillcolor{currentfill}%
\pgfsetlinewidth{0.803000pt}%
\definecolor{currentstroke}{rgb}{0.000000,0.000000,0.000000}%
\pgfsetstrokecolor{currentstroke}%
\pgfsetdash{}{0pt}%
\pgfsys@defobject{currentmarker}{\pgfqpoint{0.000000in}{0.000000in}}{\pgfqpoint{0.000000in}{0.048611in}}{%
\pgfpathmoveto{\pgfqpoint{0.000000in}{0.000000in}}%
\pgfpathlineto{\pgfqpoint{0.000000in}{0.048611in}}%
\pgfusepath{stroke,fill}%
}%
\begin{pgfscope}%
\pgfsys@transformshift{4.046916in}{1.713187in}%
\pgfsys@useobject{currentmarker}{}%
\end{pgfscope}%
\end{pgfscope}%
\begin{pgfscope}%
\pgfpathrectangle{\pgfqpoint{0.649081in}{1.713187in}}{\pgfqpoint{5.096752in}{0.990516in}}%
\pgfusepath{clip}%
\pgfsetbuttcap%
\pgfsetroundjoin%
\pgfsetlinewidth{0.501875pt}%
\definecolor{currentstroke}{rgb}{0.698039,0.698039,0.698039}%
\pgfsetstrokecolor{currentstroke}%
\pgfsetdash{{1.850000pt}{0.800000pt}}{0.000000pt}%
\pgfpathmoveto{\pgfqpoint{4.896374in}{1.713187in}}%
\pgfpathlineto{\pgfqpoint{4.896374in}{2.703703in}}%
\pgfusepath{stroke}%
\end{pgfscope}%
\begin{pgfscope}%
\pgfsetbuttcap%
\pgfsetroundjoin%
\definecolor{currentfill}{rgb}{0.000000,0.000000,0.000000}%
\pgfsetfillcolor{currentfill}%
\pgfsetlinewidth{0.803000pt}%
\definecolor{currentstroke}{rgb}{0.000000,0.000000,0.000000}%
\pgfsetstrokecolor{currentstroke}%
\pgfsetdash{}{0pt}%
\pgfsys@defobject{currentmarker}{\pgfqpoint{0.000000in}{0.000000in}}{\pgfqpoint{0.000000in}{0.048611in}}{%
\pgfpathmoveto{\pgfqpoint{0.000000in}{0.000000in}}%
\pgfpathlineto{\pgfqpoint{0.000000in}{0.048611in}}%
\pgfusepath{stroke,fill}%
}%
\begin{pgfscope}%
\pgfsys@transformshift{4.896374in}{1.713187in}%
\pgfsys@useobject{currentmarker}{}%
\end{pgfscope}%
\end{pgfscope}%
\begin{pgfscope}%
\pgfpathrectangle{\pgfqpoint{0.649081in}{1.713187in}}{\pgfqpoint{5.096752in}{0.990516in}}%
\pgfusepath{clip}%
\pgfsetbuttcap%
\pgfsetroundjoin%
\pgfsetlinewidth{0.501875pt}%
\definecolor{currentstroke}{rgb}{0.698039,0.698039,0.698039}%
\pgfsetstrokecolor{currentstroke}%
\pgfsetdash{{1.850000pt}{0.800000pt}}{0.000000pt}%
\pgfpathmoveto{\pgfqpoint{5.745833in}{1.713187in}}%
\pgfpathlineto{\pgfqpoint{5.745833in}{2.703703in}}%
\pgfusepath{stroke}%
\end{pgfscope}%
\begin{pgfscope}%
\pgfsetbuttcap%
\pgfsetroundjoin%
\definecolor{currentfill}{rgb}{0.000000,0.000000,0.000000}%
\pgfsetfillcolor{currentfill}%
\pgfsetlinewidth{0.803000pt}%
\definecolor{currentstroke}{rgb}{0.000000,0.000000,0.000000}%
\pgfsetstrokecolor{currentstroke}%
\pgfsetdash{}{0pt}%
\pgfsys@defobject{currentmarker}{\pgfqpoint{0.000000in}{0.000000in}}{\pgfqpoint{0.000000in}{0.048611in}}{%
\pgfpathmoveto{\pgfqpoint{0.000000in}{0.000000in}}%
\pgfpathlineto{\pgfqpoint{0.000000in}{0.048611in}}%
\pgfusepath{stroke,fill}%
}%
\begin{pgfscope}%
\pgfsys@transformshift{5.745833in}{1.713187in}%
\pgfsys@useobject{currentmarker}{}%
\end{pgfscope}%
\end{pgfscope}%
\begin{pgfscope}%
\pgfpathrectangle{\pgfqpoint{0.649081in}{1.713187in}}{\pgfqpoint{5.096752in}{0.990516in}}%
\pgfusepath{clip}%
\pgfsetbuttcap%
\pgfsetroundjoin%
\pgfsetlinewidth{0.501875pt}%
\definecolor{currentstroke}{rgb}{0.698039,0.698039,0.698039}%
\pgfsetstrokecolor{currentstroke}%
\pgfsetdash{{1.850000pt}{0.800000pt}}{0.000000pt}%
\pgfpathmoveto{\pgfqpoint{0.649081in}{1.776830in}}%
\pgfpathlineto{\pgfqpoint{5.745833in}{1.776830in}}%
\pgfusepath{stroke}%
\end{pgfscope}%
\begin{pgfscope}%
\pgfsetbuttcap%
\pgfsetroundjoin%
\definecolor{currentfill}{rgb}{0.000000,0.000000,0.000000}%
\pgfsetfillcolor{currentfill}%
\pgfsetlinewidth{0.803000pt}%
\definecolor{currentstroke}{rgb}{0.000000,0.000000,0.000000}%
\pgfsetstrokecolor{currentstroke}%
\pgfsetdash{}{0pt}%
\pgfsys@defobject{currentmarker}{\pgfqpoint{0.000000in}{0.000000in}}{\pgfqpoint{0.048611in}{0.000000in}}{%
\pgfpathmoveto{\pgfqpoint{0.000000in}{0.000000in}}%
\pgfpathlineto{\pgfqpoint{0.048611in}{0.000000in}}%
\pgfusepath{stroke,fill}%
}%
\begin{pgfscope}%
\pgfsys@transformshift{0.649081in}{1.776830in}%
\pgfsys@useobject{currentmarker}{}%
\end{pgfscope}%
\end{pgfscope}%
\begin{pgfscope}%
\definecolor{textcolor}{rgb}{0.000000,0.000000,0.000000}%
\pgfsetstrokecolor{textcolor}%
\pgfsetfillcolor{textcolor}%
\pgftext[x=0.423000in, y=1.728636in, left, base]{\color{textcolor}\rmfamily\fontsize{10.000000}{12.000000}\selectfont \(\displaystyle {\ensuremath{-}5}\)}%
\end{pgfscope}%
\begin{pgfscope}%
\pgfpathrectangle{\pgfqpoint{0.649081in}{1.713187in}}{\pgfqpoint{5.096752in}{0.990516in}}%
\pgfusepath{clip}%
\pgfsetbuttcap%
\pgfsetroundjoin%
\pgfsetlinewidth{0.501875pt}%
\definecolor{currentstroke}{rgb}{0.698039,0.698039,0.698039}%
\pgfsetstrokecolor{currentstroke}%
\pgfsetdash{{1.850000pt}{0.800000pt}}{0.000000pt}%
\pgfpathmoveto{\pgfqpoint{0.649081in}{2.160757in}}%
\pgfpathlineto{\pgfqpoint{5.745833in}{2.160757in}}%
\pgfusepath{stroke}%
\end{pgfscope}%
\begin{pgfscope}%
\pgfsetbuttcap%
\pgfsetroundjoin%
\definecolor{currentfill}{rgb}{0.000000,0.000000,0.000000}%
\pgfsetfillcolor{currentfill}%
\pgfsetlinewidth{0.803000pt}%
\definecolor{currentstroke}{rgb}{0.000000,0.000000,0.000000}%
\pgfsetstrokecolor{currentstroke}%
\pgfsetdash{}{0pt}%
\pgfsys@defobject{currentmarker}{\pgfqpoint{0.000000in}{0.000000in}}{\pgfqpoint{0.048611in}{0.000000in}}{%
\pgfpathmoveto{\pgfqpoint{0.000000in}{0.000000in}}%
\pgfpathlineto{\pgfqpoint{0.048611in}{0.000000in}}%
\pgfusepath{stroke,fill}%
}%
\begin{pgfscope}%
\pgfsys@transformshift{0.649081in}{2.160757in}%
\pgfsys@useobject{currentmarker}{}%
\end{pgfscope}%
\end{pgfscope}%
\begin{pgfscope}%
\definecolor{textcolor}{rgb}{0.000000,0.000000,0.000000}%
\pgfsetstrokecolor{textcolor}%
\pgfsetfillcolor{textcolor}%
\pgftext[x=0.531025in, y=2.112562in, left, base]{\color{textcolor}\rmfamily\fontsize{10.000000}{12.000000}\selectfont \(\displaystyle {0}\)}%
\end{pgfscope}%
\begin{pgfscope}%
\pgfpathrectangle{\pgfqpoint{0.649081in}{1.713187in}}{\pgfqpoint{5.096752in}{0.990516in}}%
\pgfusepath{clip}%
\pgfsetbuttcap%
\pgfsetroundjoin%
\pgfsetlinewidth{0.501875pt}%
\definecolor{currentstroke}{rgb}{0.698039,0.698039,0.698039}%
\pgfsetstrokecolor{currentstroke}%
\pgfsetdash{{1.850000pt}{0.800000pt}}{0.000000pt}%
\pgfpathmoveto{\pgfqpoint{0.649081in}{2.544684in}}%
\pgfpathlineto{\pgfqpoint{5.745833in}{2.544684in}}%
\pgfusepath{stroke}%
\end{pgfscope}%
\begin{pgfscope}%
\pgfsetbuttcap%
\pgfsetroundjoin%
\definecolor{currentfill}{rgb}{0.000000,0.000000,0.000000}%
\pgfsetfillcolor{currentfill}%
\pgfsetlinewidth{0.803000pt}%
\definecolor{currentstroke}{rgb}{0.000000,0.000000,0.000000}%
\pgfsetstrokecolor{currentstroke}%
\pgfsetdash{}{0pt}%
\pgfsys@defobject{currentmarker}{\pgfqpoint{0.000000in}{0.000000in}}{\pgfqpoint{0.048611in}{0.000000in}}{%
\pgfpathmoveto{\pgfqpoint{0.000000in}{0.000000in}}%
\pgfpathlineto{\pgfqpoint{0.048611in}{0.000000in}}%
\pgfusepath{stroke,fill}%
}%
\begin{pgfscope}%
\pgfsys@transformshift{0.649081in}{2.544684in}%
\pgfsys@useobject{currentmarker}{}%
\end{pgfscope}%
\end{pgfscope}%
\begin{pgfscope}%
\definecolor{textcolor}{rgb}{0.000000,0.000000,0.000000}%
\pgfsetstrokecolor{textcolor}%
\pgfsetfillcolor{textcolor}%
\pgftext[x=0.531025in, y=2.496489in, left, base]{\color{textcolor}\rmfamily\fontsize{10.000000}{12.000000}\selectfont \(\displaystyle {5}\)}%
\end{pgfscope}%
\begin{pgfscope}%
\definecolor{textcolor}{rgb}{0.000000,0.000000,0.000000}%
\pgfsetstrokecolor{textcolor}%
\pgfsetfillcolor{textcolor}%
\pgftext[x=0.367444in,y=2.208445in,,bottom,rotate=90.000000]{\color{textcolor}\rmfamily\fontsize{12.000000}{14.400000}\selectfont Signal}%
\end{pgfscope}%
\begin{pgfscope}%
\definecolor{textcolor}{rgb}{0.000000,0.000000,0.000000}%
\pgfsetstrokecolor{textcolor}%
\pgfsetfillcolor{textcolor}%
\pgftext[x=0.649081in,y=2.745370in,left,base]{\color{textcolor}\rmfamily\fontsize{10.000000}{12.000000}\selectfont \(\displaystyle \times{10^{\ensuremath{-}5}}{}\)}%
\end{pgfscope}%
\begin{pgfscope}%
\pgfpathrectangle{\pgfqpoint{0.649081in}{1.713187in}}{\pgfqpoint{5.096752in}{0.990516in}}%
\pgfusepath{clip}%
\pgfsetrectcap%
\pgfsetroundjoin%
\pgfsetlinewidth{1.505625pt}%
\definecolor{currentstroke}{rgb}{0.498039,0.498039,0.498039}%
\pgfsetstrokecolor{currentstroke}%
\pgfsetdash{}{0pt}%
\pgfpathmoveto{\pgfqpoint{0.648656in}{2.160881in}}%
\pgfpathlineto{\pgfqpoint{0.794338in}{2.160769in}}%
\pgfpathlineto{\pgfqpoint{0.828317in}{2.160674in}}%
\pgfpathlineto{\pgfqpoint{0.906042in}{2.160791in}}%
\pgfpathlineto{\pgfqpoint{0.943843in}{2.160678in}}%
\pgfpathlineto{\pgfqpoint{1.037283in}{2.161427in}}%
\pgfpathlineto{\pgfqpoint{1.040257in}{2.162210in}}%
\pgfpathlineto{\pgfqpoint{1.042805in}{2.159811in}}%
\pgfpathlineto{\pgfqpoint{1.045353in}{2.158252in}}%
\pgfpathlineto{\pgfqpoint{1.047902in}{2.159842in}}%
\pgfpathlineto{\pgfqpoint{1.051300in}{2.161197in}}%
\pgfpathlineto{\pgfqpoint{1.055122in}{2.159422in}}%
\pgfpathlineto{\pgfqpoint{1.057670in}{2.158760in}}%
\pgfpathlineto{\pgfqpoint{1.059794in}{2.160790in}}%
\pgfpathlineto{\pgfqpoint{1.063192in}{2.163637in}}%
\pgfpathlineto{\pgfqpoint{1.066165in}{2.161445in}}%
\pgfpathlineto{\pgfqpoint{1.069138in}{2.160588in}}%
\pgfpathlineto{\pgfqpoint{1.072536in}{2.161722in}}%
\pgfpathlineto{\pgfqpoint{1.075934in}{2.162979in}}%
\pgfpathlineto{\pgfqpoint{1.078057in}{2.160649in}}%
\pgfpathlineto{\pgfqpoint{1.081455in}{2.157382in}}%
\pgfpathlineto{\pgfqpoint{1.083154in}{2.158935in}}%
\pgfpathlineto{\pgfqpoint{1.086552in}{2.162279in}}%
\pgfpathlineto{\pgfqpoint{1.088676in}{2.160211in}}%
\pgfpathlineto{\pgfqpoint{1.091649in}{2.158021in}}%
\pgfpathlineto{\pgfqpoint{1.093772in}{2.159968in}}%
\pgfpathlineto{\pgfqpoint{1.097170in}{2.162798in}}%
\pgfpathlineto{\pgfqpoint{1.099294in}{2.162183in}}%
\pgfpathlineto{\pgfqpoint{1.103966in}{2.159155in}}%
\pgfpathlineto{\pgfqpoint{1.107364in}{2.162789in}}%
\pgfpathlineto{\pgfqpoint{1.109912in}{2.163653in}}%
\pgfpathlineto{\pgfqpoint{1.111611in}{2.161589in}}%
\pgfpathlineto{\pgfqpoint{1.115009in}{2.156536in}}%
\pgfpathlineto{\pgfqpoint{1.116283in}{2.157495in}}%
\pgfpathlineto{\pgfqpoint{1.122229in}{2.165413in}}%
\pgfpathlineto{\pgfqpoint{1.123928in}{2.162960in}}%
\pgfpathlineto{\pgfqpoint{1.127751in}{2.155504in}}%
\pgfpathlineto{\pgfqpoint{1.129025in}{2.157958in}}%
\pgfpathlineto{\pgfqpoint{1.132848in}{2.168465in}}%
\pgfpathlineto{\pgfqpoint{1.134122in}{2.166971in}}%
\pgfpathlineto{\pgfqpoint{1.137095in}{2.156395in}}%
\pgfpathlineto{\pgfqpoint{1.139643in}{2.151282in}}%
\pgfpathlineto{\pgfqpoint{1.140917in}{2.153695in}}%
\pgfpathlineto{\pgfqpoint{1.145165in}{2.167605in}}%
\pgfpathlineto{\pgfqpoint{1.145589in}{2.167029in}}%
\pgfpathlineto{\pgfqpoint{1.147288in}{2.160043in}}%
\pgfpathlineto{\pgfqpoint{1.150261in}{2.147667in}}%
\pgfpathlineto{\pgfqpoint{1.151111in}{2.148533in}}%
\pgfpathlineto{\pgfqpoint{1.152810in}{2.155812in}}%
\pgfpathlineto{\pgfqpoint{1.157482in}{2.178748in}}%
\pgfpathlineto{\pgfqpoint{1.158331in}{2.177667in}}%
\pgfpathlineto{\pgfqpoint{1.160030in}{2.166969in}}%
\pgfpathlineto{\pgfqpoint{1.163003in}{2.148042in}}%
\pgfpathlineto{\pgfqpoint{1.163853in}{2.150054in}}%
\pgfpathlineto{\pgfqpoint{1.168100in}{2.172132in}}%
\pgfpathlineto{\pgfqpoint{1.168525in}{2.171638in}}%
\pgfpathlineto{\pgfqpoint{1.170224in}{2.164680in}}%
\pgfpathlineto{\pgfqpoint{1.174896in}{2.142310in}}%
\pgfpathlineto{\pgfqpoint{1.175745in}{2.142122in}}%
\pgfpathlineto{\pgfqpoint{1.177019in}{2.145617in}}%
\pgfpathlineto{\pgfqpoint{1.182966in}{2.175743in}}%
\pgfpathlineto{\pgfqpoint{1.183815in}{2.174940in}}%
\pgfpathlineto{\pgfqpoint{1.185939in}{2.167098in}}%
\pgfpathlineto{\pgfqpoint{1.188487in}{2.159499in}}%
\pgfpathlineto{\pgfqpoint{1.189337in}{2.160022in}}%
\pgfpathlineto{\pgfqpoint{1.190611in}{2.165735in}}%
\pgfpathlineto{\pgfqpoint{1.194009in}{2.186195in}}%
\pgfpathlineto{\pgfqpoint{1.194858in}{2.183451in}}%
\pgfpathlineto{\pgfqpoint{1.196557in}{2.164742in}}%
\pgfpathlineto{\pgfqpoint{1.199955in}{2.127537in}}%
\pgfpathlineto{\pgfqpoint{1.200804in}{2.130350in}}%
\pgfpathlineto{\pgfqpoint{1.202928in}{2.155855in}}%
\pgfpathlineto{\pgfqpoint{1.205052in}{2.176870in}}%
\pgfpathlineto{\pgfqpoint{1.205476in}{2.176718in}}%
\pgfpathlineto{\pgfqpoint{1.206750in}{2.165566in}}%
\pgfpathlineto{\pgfqpoint{1.210148in}{2.123232in}}%
\pgfpathlineto{\pgfqpoint{1.210998in}{2.127426in}}%
\pgfpathlineto{\pgfqpoint{1.216095in}{2.182959in}}%
\pgfpathlineto{\pgfqpoint{1.216944in}{2.181369in}}%
\pgfpathlineto{\pgfqpoint{1.219492in}{2.168435in}}%
\pgfpathlineto{\pgfqpoint{1.222041in}{2.158193in}}%
\pgfpathlineto{\pgfqpoint{1.222890in}{2.159573in}}%
\pgfpathlineto{\pgfqpoint{1.224589in}{2.171709in}}%
\pgfpathlineto{\pgfqpoint{1.226713in}{2.184348in}}%
\pgfpathlineto{\pgfqpoint{1.227562in}{2.181744in}}%
\pgfpathlineto{\pgfqpoint{1.231385in}{2.152134in}}%
\pgfpathlineto{\pgfqpoint{1.232234in}{2.153594in}}%
\pgfpathlineto{\pgfqpoint{1.235632in}{2.164509in}}%
\pgfpathlineto{\pgfqpoint{1.236906in}{2.162586in}}%
\pgfpathlineto{\pgfqpoint{1.239455in}{2.154234in}}%
\pgfpathlineto{\pgfqpoint{1.242852in}{2.137162in}}%
\pgfpathlineto{\pgfqpoint{1.243702in}{2.138379in}}%
\pgfpathlineto{\pgfqpoint{1.245401in}{2.150400in}}%
\pgfpathlineto{\pgfqpoint{1.248799in}{2.171938in}}%
\pgfpathlineto{\pgfqpoint{1.249648in}{2.172682in}}%
\pgfpathlineto{\pgfqpoint{1.250073in}{2.172104in}}%
\pgfpathlineto{\pgfqpoint{1.251772in}{2.163906in}}%
\pgfpathlineto{\pgfqpoint{1.254745in}{2.147494in}}%
\pgfpathlineto{\pgfqpoint{1.255594in}{2.149046in}}%
\pgfpathlineto{\pgfqpoint{1.257293in}{2.161997in}}%
\pgfpathlineto{\pgfqpoint{1.260266in}{2.182793in}}%
\pgfpathlineto{\pgfqpoint{1.261116in}{2.183639in}}%
\pgfpathlineto{\pgfqpoint{1.261541in}{2.183148in}}%
\pgfpathlineto{\pgfqpoint{1.262815in}{2.177415in}}%
\pgfpathlineto{\pgfqpoint{1.266637in}{2.138889in}}%
\pgfpathlineto{\pgfqpoint{1.267487in}{2.141233in}}%
\pgfpathlineto{\pgfqpoint{1.269610in}{2.170317in}}%
\pgfpathlineto{\pgfqpoint{1.271309in}{2.186238in}}%
\pgfpathlineto{\pgfqpoint{1.271734in}{2.186106in}}%
\pgfpathlineto{\pgfqpoint{1.273008in}{2.176321in}}%
\pgfpathlineto{\pgfqpoint{1.277680in}{2.124698in}}%
\pgfpathlineto{\pgfqpoint{1.278530in}{2.128362in}}%
\pgfpathlineto{\pgfqpoint{1.280653in}{2.160719in}}%
\pgfpathlineto{\pgfqpoint{1.282777in}{2.181926in}}%
\pgfpathlineto{\pgfqpoint{1.283626in}{2.178134in}}%
\pgfpathlineto{\pgfqpoint{1.287449in}{2.145339in}}%
\pgfpathlineto{\pgfqpoint{1.287874in}{2.145887in}}%
\pgfpathlineto{\pgfqpoint{1.289573in}{2.155103in}}%
\pgfpathlineto{\pgfqpoint{1.294245in}{2.187088in}}%
\pgfpathlineto{\pgfqpoint{1.294669in}{2.186679in}}%
\pgfpathlineto{\pgfqpoint{1.295944in}{2.180812in}}%
\pgfpathlineto{\pgfqpoint{1.299341in}{2.163580in}}%
\pgfpathlineto{\pgfqpoint{1.300616in}{2.163761in}}%
\pgfpathlineto{\pgfqpoint{1.302739in}{2.164206in}}%
\pgfpathlineto{\pgfqpoint{1.305712in}{2.159314in}}%
\pgfpathlineto{\pgfqpoint{1.310384in}{2.149484in}}%
\pgfpathlineto{\pgfqpoint{1.312083in}{2.152284in}}%
\pgfpathlineto{\pgfqpoint{1.313782in}{2.155038in}}%
\pgfpathlineto{\pgfqpoint{1.314207in}{2.154650in}}%
\pgfpathlineto{\pgfqpoint{1.315481in}{2.148528in}}%
\pgfpathlineto{\pgfqpoint{1.318879in}{2.123149in}}%
\pgfpathlineto{\pgfqpoint{1.319304in}{2.124516in}}%
\pgfpathlineto{\pgfqpoint{1.321003in}{2.142570in}}%
\pgfpathlineto{\pgfqpoint{1.323551in}{2.165461in}}%
\pgfpathlineto{\pgfqpoint{1.324400in}{2.164380in}}%
\pgfpathlineto{\pgfqpoint{1.326099in}{2.160474in}}%
\pgfpathlineto{\pgfqpoint{1.326524in}{2.161159in}}%
\pgfpathlineto{\pgfqpoint{1.327798in}{2.168718in}}%
\pgfpathlineto{\pgfqpoint{1.331621in}{2.194516in}}%
\pgfpathlineto{\pgfqpoint{1.333745in}{2.197233in}}%
\pgfpathlineto{\pgfqpoint{1.334594in}{2.195998in}}%
\pgfpathlineto{\pgfqpoint{1.336293in}{2.185317in}}%
\pgfpathlineto{\pgfqpoint{1.338841in}{2.169979in}}%
\pgfpathlineto{\pgfqpoint{1.339691in}{2.172120in}}%
\pgfpathlineto{\pgfqpoint{1.341814in}{2.179986in}}%
\pgfpathlineto{\pgfqpoint{1.342664in}{2.176242in}}%
\pgfpathlineto{\pgfqpoint{1.344788in}{2.147652in}}%
\pgfpathlineto{\pgfqpoint{1.347336in}{2.126005in}}%
\pgfpathlineto{\pgfqpoint{1.348185in}{2.126841in}}%
\pgfpathlineto{\pgfqpoint{1.352008in}{2.133633in}}%
\pgfpathlineto{\pgfqpoint{1.356680in}{2.138871in}}%
\pgfpathlineto{\pgfqpoint{1.358379in}{2.147239in}}%
\pgfpathlineto{\pgfqpoint{1.362626in}{2.173897in}}%
\pgfpathlineto{\pgfqpoint{1.364750in}{2.171343in}}%
\pgfpathlineto{\pgfqpoint{1.365175in}{2.171739in}}%
\pgfpathlineto{\pgfqpoint{1.366449in}{2.176379in}}%
\pgfpathlineto{\pgfqpoint{1.369422in}{2.190071in}}%
\pgfpathlineto{\pgfqpoint{1.370271in}{2.188532in}}%
\pgfpathlineto{\pgfqpoint{1.374094in}{2.174680in}}%
\pgfpathlineto{\pgfqpoint{1.374519in}{2.174923in}}%
\pgfpathlineto{\pgfqpoint{1.377492in}{2.180364in}}%
\pgfpathlineto{\pgfqpoint{1.378341in}{2.179031in}}%
\pgfpathlineto{\pgfqpoint{1.380465in}{2.168124in}}%
\pgfpathlineto{\pgfqpoint{1.383013in}{2.143565in}}%
\pgfpathlineto{\pgfqpoint{1.386411in}{2.110887in}}%
\pgfpathlineto{\pgfqpoint{1.387260in}{2.113351in}}%
\pgfpathlineto{\pgfqpoint{1.391508in}{2.137495in}}%
\pgfpathlineto{\pgfqpoint{1.392357in}{2.138580in}}%
\pgfpathlineto{\pgfqpoint{1.393631in}{2.147575in}}%
\pgfpathlineto{\pgfqpoint{1.398303in}{2.193661in}}%
\pgfpathlineto{\pgfqpoint{1.401277in}{2.194802in}}%
\pgfpathlineto{\pgfqpoint{1.402551in}{2.192204in}}%
\pgfpathlineto{\pgfqpoint{1.405524in}{2.183920in}}%
\pgfpathlineto{\pgfqpoint{1.407223in}{2.185012in}}%
\pgfpathlineto{\pgfqpoint{1.408072in}{2.181494in}}%
\pgfpathlineto{\pgfqpoint{1.409771in}{2.159152in}}%
\pgfpathlineto{\pgfqpoint{1.412744in}{2.121299in}}%
\pgfpathlineto{\pgfqpoint{1.413169in}{2.121376in}}%
\pgfpathlineto{\pgfqpoint{1.414443in}{2.128725in}}%
\pgfpathlineto{\pgfqpoint{1.417416in}{2.145716in}}%
\pgfpathlineto{\pgfqpoint{1.418266in}{2.143409in}}%
\pgfpathlineto{\pgfqpoint{1.421664in}{2.122625in}}%
\pgfpathlineto{\pgfqpoint{1.422513in}{2.125681in}}%
\pgfpathlineto{\pgfqpoint{1.424212in}{2.148530in}}%
\pgfpathlineto{\pgfqpoint{1.428034in}{2.200869in}}%
\pgfpathlineto{\pgfqpoint{1.428884in}{2.199954in}}%
\pgfpathlineto{\pgfqpoint{1.431008in}{2.185702in}}%
\pgfpathlineto{\pgfqpoint{1.434405in}{2.166822in}}%
\pgfpathlineto{\pgfqpoint{1.435255in}{2.168002in}}%
\pgfpathlineto{\pgfqpoint{1.437803in}{2.175584in}}%
\pgfpathlineto{\pgfqpoint{1.438228in}{2.175059in}}%
\pgfpathlineto{\pgfqpoint{1.439927in}{2.166274in}}%
\pgfpathlineto{\pgfqpoint{1.442051in}{2.157716in}}%
\pgfpathlineto{\pgfqpoint{1.443325in}{2.161467in}}%
\pgfpathlineto{\pgfqpoint{1.445024in}{2.166830in}}%
\pgfpathlineto{\pgfqpoint{1.445448in}{2.166132in}}%
\pgfpathlineto{\pgfqpoint{1.446723in}{2.156881in}}%
\pgfpathlineto{\pgfqpoint{1.450120in}{2.121159in}}%
\pgfpathlineto{\pgfqpoint{1.450545in}{2.121655in}}%
\pgfpathlineto{\pgfqpoint{1.451819in}{2.133555in}}%
\pgfpathlineto{\pgfqpoint{1.455642in}{2.191718in}}%
\pgfpathlineto{\pgfqpoint{1.456067in}{2.191244in}}%
\pgfpathlineto{\pgfqpoint{1.457341in}{2.177169in}}%
\pgfpathlineto{\pgfqpoint{1.460739in}{2.125130in}}%
\pgfpathlineto{\pgfqpoint{1.461163in}{2.125420in}}%
\pgfpathlineto{\pgfqpoint{1.462438in}{2.138625in}}%
\pgfpathlineto{\pgfqpoint{1.466685in}{2.206587in}}%
\pgfpathlineto{\pgfqpoint{1.467110in}{2.206129in}}%
\pgfpathlineto{\pgfqpoint{1.468384in}{2.193003in}}%
\pgfpathlineto{\pgfqpoint{1.473056in}{2.107813in}}%
\pgfpathlineto{\pgfqpoint{1.473480in}{2.110034in}}%
\pgfpathlineto{\pgfqpoint{1.475179in}{2.137935in}}%
\pgfpathlineto{\pgfqpoint{1.478577in}{2.188837in}}%
\pgfpathlineto{\pgfqpoint{1.479002in}{2.188768in}}%
\pgfpathlineto{\pgfqpoint{1.480276in}{2.180712in}}%
\pgfpathlineto{\pgfqpoint{1.483674in}{2.149599in}}%
\pgfpathlineto{\pgfqpoint{1.484099in}{2.150375in}}%
\pgfpathlineto{\pgfqpoint{1.485373in}{2.162515in}}%
\pgfpathlineto{\pgfqpoint{1.489195in}{2.208629in}}%
\pgfpathlineto{\pgfqpoint{1.490045in}{2.206125in}}%
\pgfpathlineto{\pgfqpoint{1.491744in}{2.184703in}}%
\pgfpathlineto{\pgfqpoint{1.495991in}{2.105816in}}%
\pgfpathlineto{\pgfqpoint{1.496416in}{2.106947in}}%
\pgfpathlineto{\pgfqpoint{1.497690in}{2.124811in}}%
\pgfpathlineto{\pgfqpoint{1.501513in}{2.185236in}}%
\pgfpathlineto{\pgfqpoint{1.502362in}{2.179667in}}%
\pgfpathlineto{\pgfqpoint{1.504486in}{2.140020in}}%
\pgfpathlineto{\pgfqpoint{1.507459in}{2.094530in}}%
\pgfpathlineto{\pgfqpoint{1.508308in}{2.100827in}}%
\pgfpathlineto{\pgfqpoint{1.510007in}{2.142529in}}%
\pgfpathlineto{\pgfqpoint{1.513405in}{2.222442in}}%
\pgfpathlineto{\pgfqpoint{1.513830in}{2.223181in}}%
\pgfpathlineto{\pgfqpoint{1.514255in}{2.221742in}}%
\pgfpathlineto{\pgfqpoint{1.515953in}{2.201334in}}%
\pgfpathlineto{\pgfqpoint{1.519351in}{2.161500in}}%
\pgfpathlineto{\pgfqpoint{1.519776in}{2.160722in}}%
\pgfpathlineto{\pgfqpoint{1.520201in}{2.161093in}}%
\pgfpathlineto{\pgfqpoint{1.521475in}{2.167982in}}%
\pgfpathlineto{\pgfqpoint{1.524023in}{2.183736in}}%
\pgfpathlineto{\pgfqpoint{1.524873in}{2.180920in}}%
\pgfpathlineto{\pgfqpoint{1.526572in}{2.160344in}}%
\pgfpathlineto{\pgfqpoint{1.529545in}{2.125976in}}%
\pgfpathlineto{\pgfqpoint{1.529969in}{2.125165in}}%
\pgfpathlineto{\pgfqpoint{1.530394in}{2.125524in}}%
\pgfpathlineto{\pgfqpoint{1.531668in}{2.133333in}}%
\pgfpathlineto{\pgfqpoint{1.535916in}{2.185753in}}%
\pgfpathlineto{\pgfqpoint{1.536765in}{2.181579in}}%
\pgfpathlineto{\pgfqpoint{1.538464in}{2.148081in}}%
\pgfpathlineto{\pgfqpoint{1.541437in}{2.088928in}}%
\pgfpathlineto{\pgfqpoint{1.541862in}{2.088904in}}%
\pgfpathlineto{\pgfqpoint{1.542711in}{2.097104in}}%
\pgfpathlineto{\pgfqpoint{1.544835in}{2.153702in}}%
\pgfpathlineto{\pgfqpoint{1.547808in}{2.225036in}}%
\pgfpathlineto{\pgfqpoint{1.548658in}{2.227453in}}%
\pgfpathlineto{\pgfqpoint{1.549932in}{2.217262in}}%
\pgfpathlineto{\pgfqpoint{1.553330in}{2.182216in}}%
\pgfpathlineto{\pgfqpoint{1.554179in}{2.183969in}}%
\pgfpathlineto{\pgfqpoint{1.557152in}{2.199584in}}%
\pgfpathlineto{\pgfqpoint{1.557577in}{2.197934in}}%
\pgfpathlineto{\pgfqpoint{1.558851in}{2.181650in}}%
\pgfpathlineto{\pgfqpoint{1.563948in}{2.069568in}}%
\pgfpathlineto{\pgfqpoint{1.564797in}{2.074916in}}%
\pgfpathlineto{\pgfqpoint{1.566921in}{2.120055in}}%
\pgfpathlineto{\pgfqpoint{1.570319in}{2.186723in}}%
\pgfpathlineto{\pgfqpoint{1.571593in}{2.191201in}}%
\pgfpathlineto{\pgfqpoint{1.572867in}{2.184310in}}%
\pgfpathlineto{\pgfqpoint{1.576690in}{2.157999in}}%
\pgfpathlineto{\pgfqpoint{1.577539in}{2.159833in}}%
\pgfpathlineto{\pgfqpoint{1.579663in}{2.174638in}}%
\pgfpathlineto{\pgfqpoint{1.584335in}{2.220385in}}%
\pgfpathlineto{\pgfqpoint{1.584760in}{2.219259in}}%
\pgfpathlineto{\pgfqpoint{1.586034in}{2.204437in}}%
\pgfpathlineto{\pgfqpoint{1.592405in}{2.087820in}}%
\pgfpathlineto{\pgfqpoint{1.592829in}{2.088935in}}%
\pgfpathlineto{\pgfqpoint{1.594104in}{2.101003in}}%
\pgfpathlineto{\pgfqpoint{1.599625in}{2.168079in}}%
\pgfpathlineto{\pgfqpoint{1.601749in}{2.174060in}}%
\pgfpathlineto{\pgfqpoint{1.606421in}{2.201708in}}%
\pgfpathlineto{\pgfqpoint{1.607270in}{2.200508in}}%
\pgfpathlineto{\pgfqpoint{1.610668in}{2.189281in}}%
\pgfpathlineto{\pgfqpoint{1.611093in}{2.189572in}}%
\pgfpathlineto{\pgfqpoint{1.613216in}{2.193080in}}%
\pgfpathlineto{\pgfqpoint{1.613641in}{2.192397in}}%
\pgfpathlineto{\pgfqpoint{1.614915in}{2.183666in}}%
\pgfpathlineto{\pgfqpoint{1.617464in}{2.140378in}}%
\pgfpathlineto{\pgfqpoint{1.620437in}{2.101110in}}%
\pgfpathlineto{\pgfqpoint{1.621711in}{2.098519in}}%
\pgfpathlineto{\pgfqpoint{1.622985in}{2.102954in}}%
\pgfpathlineto{\pgfqpoint{1.628082in}{2.128369in}}%
\pgfpathlineto{\pgfqpoint{1.629356in}{2.130210in}}%
\pgfpathlineto{\pgfqpoint{1.630630in}{2.137923in}}%
\pgfpathlineto{\pgfqpoint{1.633603in}{2.178328in}}%
\pgfpathlineto{\pgfqpoint{1.637001in}{2.215611in}}%
\pgfpathlineto{\pgfqpoint{1.642523in}{2.248215in}}%
\pgfpathlineto{\pgfqpoint{1.642948in}{2.248077in}}%
\pgfpathlineto{\pgfqpoint{1.643797in}{2.243137in}}%
\pgfpathlineto{\pgfqpoint{1.645496in}{2.211767in}}%
\pgfpathlineto{\pgfqpoint{1.650593in}{2.101548in}}%
\pgfpathlineto{\pgfqpoint{1.651017in}{2.101064in}}%
\pgfpathlineto{\pgfqpoint{1.651442in}{2.101682in}}%
\pgfpathlineto{\pgfqpoint{1.653990in}{2.110234in}}%
\pgfpathlineto{\pgfqpoint{1.654415in}{2.109886in}}%
\pgfpathlineto{\pgfqpoint{1.655689in}{2.103391in}}%
\pgfpathlineto{\pgfqpoint{1.659087in}{2.076800in}}%
\pgfpathlineto{\pgfqpoint{1.659512in}{2.077591in}}%
\pgfpathlineto{\pgfqpoint{1.660786in}{2.090356in}}%
\pgfpathlineto{\pgfqpoint{1.663759in}{2.163408in}}%
\pgfpathlineto{\pgfqpoint{1.666732in}{2.216505in}}%
\pgfpathlineto{\pgfqpoint{1.668007in}{2.219916in}}%
\pgfpathlineto{\pgfqpoint{1.669281in}{2.216657in}}%
\pgfpathlineto{\pgfqpoint{1.673103in}{2.206140in}}%
\pgfpathlineto{\pgfqpoint{1.675227in}{2.200773in}}%
\pgfpathlineto{\pgfqpoint{1.677351in}{2.184636in}}%
\pgfpathlineto{\pgfqpoint{1.682872in}{2.134190in}}%
\pgfpathlineto{\pgfqpoint{1.684996in}{2.129239in}}%
\pgfpathlineto{\pgfqpoint{1.685845in}{2.129608in}}%
\pgfpathlineto{\pgfqpoint{1.687119in}{2.133692in}}%
\pgfpathlineto{\pgfqpoint{1.690092in}{2.155555in}}%
\pgfpathlineto{\pgfqpoint{1.693490in}{2.173892in}}%
\pgfpathlineto{\pgfqpoint{1.694764in}{2.174631in}}%
\pgfpathlineto{\pgfqpoint{1.696463in}{2.170585in}}%
\pgfpathlineto{\pgfqpoint{1.703259in}{2.144611in}}%
\pgfpathlineto{\pgfqpoint{1.704533in}{2.140026in}}%
\pgfpathlineto{\pgfqpoint{1.704958in}{2.140121in}}%
\pgfpathlineto{\pgfqpoint{1.706232in}{2.146523in}}%
\pgfpathlineto{\pgfqpoint{1.710479in}{2.178959in}}%
\pgfpathlineto{\pgfqpoint{1.711329in}{2.177202in}}%
\pgfpathlineto{\pgfqpoint{1.713028in}{2.164871in}}%
\pgfpathlineto{\pgfqpoint{1.717275in}{2.130492in}}%
\pgfpathlineto{\pgfqpoint{1.718125in}{2.131619in}}%
\pgfpathlineto{\pgfqpoint{1.719824in}{2.145932in}}%
\pgfpathlineto{\pgfqpoint{1.723646in}{2.180078in}}%
\pgfpathlineto{\pgfqpoint{1.724496in}{2.178080in}}%
\pgfpathlineto{\pgfqpoint{1.727044in}{2.158604in}}%
\pgfpathlineto{\pgfqpoint{1.729592in}{2.145496in}}%
\pgfpathlineto{\pgfqpoint{1.730442in}{2.148246in}}%
\pgfpathlineto{\pgfqpoint{1.732141in}{2.168812in}}%
\pgfpathlineto{\pgfqpoint{1.735539in}{2.215752in}}%
\pgfpathlineto{\pgfqpoint{1.735963in}{2.215243in}}%
\pgfpathlineto{\pgfqpoint{1.737237in}{2.203589in}}%
\pgfpathlineto{\pgfqpoint{1.741485in}{2.154208in}}%
\pgfpathlineto{\pgfqpoint{1.742334in}{2.155440in}}%
\pgfpathlineto{\pgfqpoint{1.744458in}{2.168355in}}%
\pgfpathlineto{\pgfqpoint{1.746581in}{2.177558in}}%
\pgfpathlineto{\pgfqpoint{1.747431in}{2.175187in}}%
\pgfpathlineto{\pgfqpoint{1.749130in}{2.155225in}}%
\pgfpathlineto{\pgfqpoint{1.754227in}{2.073853in}}%
\pgfpathlineto{\pgfqpoint{1.755076in}{2.077858in}}%
\pgfpathlineto{\pgfqpoint{1.757200in}{2.108997in}}%
\pgfpathlineto{\pgfqpoint{1.763146in}{2.227353in}}%
\pgfpathlineto{\pgfqpoint{1.763571in}{2.225639in}}%
\pgfpathlineto{\pgfqpoint{1.765270in}{2.201203in}}%
\pgfpathlineto{\pgfqpoint{1.768667in}{2.152807in}}%
\pgfpathlineto{\pgfqpoint{1.769092in}{2.152158in}}%
\pgfpathlineto{\pgfqpoint{1.769517in}{2.152891in}}%
\pgfpathlineto{\pgfqpoint{1.770791in}{2.162626in}}%
\pgfpathlineto{\pgfqpoint{1.774614in}{2.207156in}}%
\pgfpathlineto{\pgfqpoint{1.775038in}{2.206548in}}%
\pgfpathlineto{\pgfqpoint{1.776313in}{2.194568in}}%
\pgfpathlineto{\pgfqpoint{1.780560in}{2.142446in}}%
\pgfpathlineto{\pgfqpoint{1.781834in}{2.141431in}}%
\pgfpathlineto{\pgfqpoint{1.784807in}{2.144176in}}%
\pgfpathlineto{\pgfqpoint{1.788630in}{2.151583in}}%
\pgfpathlineto{\pgfqpoint{1.789054in}{2.151047in}}%
\pgfpathlineto{\pgfqpoint{1.792028in}{2.147392in}}%
\pgfpathlineto{\pgfqpoint{1.794151in}{2.146471in}}%
\pgfpathlineto{\pgfqpoint{1.796700in}{2.143397in}}%
\pgfpathlineto{\pgfqpoint{1.797974in}{2.148654in}}%
\pgfpathlineto{\pgfqpoint{1.800522in}{2.179176in}}%
\pgfpathlineto{\pgfqpoint{1.802646in}{2.194931in}}%
\pgfpathlineto{\pgfqpoint{1.803495in}{2.190308in}}%
\pgfpathlineto{\pgfqpoint{1.806044in}{2.149459in}}%
\pgfpathlineto{\pgfqpoint{1.808167in}{2.131652in}}%
\pgfpathlineto{\pgfqpoint{1.809017in}{2.136113in}}%
\pgfpathlineto{\pgfqpoint{1.814538in}{2.191298in}}%
\pgfpathlineto{\pgfqpoint{1.814963in}{2.190808in}}%
\pgfpathlineto{\pgfqpoint{1.816662in}{2.183974in}}%
\pgfpathlineto{\pgfqpoint{1.820909in}{2.155009in}}%
\pgfpathlineto{\pgfqpoint{1.824732in}{2.131325in}}%
\pgfpathlineto{\pgfqpoint{1.825581in}{2.131304in}}%
\pgfpathlineto{\pgfqpoint{1.826855in}{2.137690in}}%
\pgfpathlineto{\pgfqpoint{1.829404in}{2.172773in}}%
\pgfpathlineto{\pgfqpoint{1.831952in}{2.194445in}}%
\pgfpathlineto{\pgfqpoint{1.832802in}{2.191121in}}%
\pgfpathlineto{\pgfqpoint{1.835350in}{2.162493in}}%
\pgfpathlineto{\pgfqpoint{1.838748in}{2.134587in}}%
\pgfpathlineto{\pgfqpoint{1.840022in}{2.133230in}}%
\pgfpathlineto{\pgfqpoint{1.841721in}{2.136714in}}%
\pgfpathlineto{\pgfqpoint{1.845119in}{2.148957in}}%
\pgfpathlineto{\pgfqpoint{1.847667in}{2.171535in}}%
\pgfpathlineto{\pgfqpoint{1.850215in}{2.190336in}}%
\pgfpathlineto{\pgfqpoint{1.851065in}{2.187687in}}%
\pgfpathlineto{\pgfqpoint{1.853613in}{2.160021in}}%
\pgfpathlineto{\pgfqpoint{1.855312in}{2.150186in}}%
\pgfpathlineto{\pgfqpoint{1.856162in}{2.152371in}}%
\pgfpathlineto{\pgfqpoint{1.858285in}{2.171942in}}%
\pgfpathlineto{\pgfqpoint{1.860409in}{2.186267in}}%
\pgfpathlineto{\pgfqpoint{1.861258in}{2.184789in}}%
\pgfpathlineto{\pgfqpoint{1.863382in}{2.167469in}}%
\pgfpathlineto{\pgfqpoint{1.865081in}{2.158169in}}%
\pgfpathlineto{\pgfqpoint{1.865506in}{2.158396in}}%
\pgfpathlineto{\pgfqpoint{1.866780in}{2.164635in}}%
\pgfpathlineto{\pgfqpoint{1.868904in}{2.175930in}}%
\pgfpathlineto{\pgfqpoint{1.869328in}{2.175341in}}%
\pgfpathlineto{\pgfqpoint{1.870602in}{2.165276in}}%
\pgfpathlineto{\pgfqpoint{1.875274in}{2.112407in}}%
\pgfpathlineto{\pgfqpoint{1.876124in}{2.114554in}}%
\pgfpathlineto{\pgfqpoint{1.878248in}{2.134324in}}%
\pgfpathlineto{\pgfqpoint{1.882070in}{2.170273in}}%
\pgfpathlineto{\pgfqpoint{1.885043in}{2.178074in}}%
\pgfpathlineto{\pgfqpoint{1.887592in}{2.181591in}}%
\pgfpathlineto{\pgfqpoint{1.889291in}{2.179375in}}%
\pgfpathlineto{\pgfqpoint{1.890989in}{2.178218in}}%
\pgfpathlineto{\pgfqpoint{1.892264in}{2.180828in}}%
\pgfpathlineto{\pgfqpoint{1.894812in}{2.187118in}}%
\pgfpathlineto{\pgfqpoint{1.895661in}{2.185154in}}%
\pgfpathlineto{\pgfqpoint{1.897360in}{2.170616in}}%
\pgfpathlineto{\pgfqpoint{1.901608in}{2.130111in}}%
\pgfpathlineto{\pgfqpoint{1.902457in}{2.133523in}}%
\pgfpathlineto{\pgfqpoint{1.906704in}{2.168466in}}%
\pgfpathlineto{\pgfqpoint{1.907554in}{2.166459in}}%
\pgfpathlineto{\pgfqpoint{1.909253in}{2.149879in}}%
\pgfpathlineto{\pgfqpoint{1.912226in}{2.119428in}}%
\pgfpathlineto{\pgfqpoint{1.912651in}{2.119413in}}%
\pgfpathlineto{\pgfqpoint{1.913500in}{2.124042in}}%
\pgfpathlineto{\pgfqpoint{1.915624in}{2.155704in}}%
\pgfpathlineto{\pgfqpoint{1.918597in}{2.193148in}}%
\pgfpathlineto{\pgfqpoint{1.919022in}{2.194106in}}%
\pgfpathlineto{\pgfqpoint{1.919446in}{2.193837in}}%
\pgfpathlineto{\pgfqpoint{1.920720in}{2.186700in}}%
\pgfpathlineto{\pgfqpoint{1.925393in}{2.148600in}}%
\pgfpathlineto{\pgfqpoint{1.926242in}{2.152160in}}%
\pgfpathlineto{\pgfqpoint{1.928366in}{2.178157in}}%
\pgfpathlineto{\pgfqpoint{1.930914in}{2.201045in}}%
\pgfpathlineto{\pgfqpoint{1.931763in}{2.197737in}}%
\pgfpathlineto{\pgfqpoint{1.933887in}{2.170114in}}%
\pgfpathlineto{\pgfqpoint{1.936860in}{2.136923in}}%
\pgfpathlineto{\pgfqpoint{1.937710in}{2.136105in}}%
\pgfpathlineto{\pgfqpoint{1.938984in}{2.142000in}}%
\pgfpathlineto{\pgfqpoint{1.942382in}{2.161239in}}%
\pgfpathlineto{\pgfqpoint{1.943656in}{2.156452in}}%
\pgfpathlineto{\pgfqpoint{1.948328in}{2.130935in}}%
\pgfpathlineto{\pgfqpoint{1.949602in}{2.130862in}}%
\pgfpathlineto{\pgfqpoint{1.950876in}{2.134842in}}%
\pgfpathlineto{\pgfqpoint{1.953000in}{2.152577in}}%
\pgfpathlineto{\pgfqpoint{1.958521in}{2.209099in}}%
\pgfpathlineto{\pgfqpoint{1.959371in}{2.208579in}}%
\pgfpathlineto{\pgfqpoint{1.960645in}{2.201740in}}%
\pgfpathlineto{\pgfqpoint{1.962769in}{2.173085in}}%
\pgfpathlineto{\pgfqpoint{1.966591in}{2.120276in}}%
\pgfpathlineto{\pgfqpoint{1.967016in}{2.120643in}}%
\pgfpathlineto{\pgfqpoint{1.968290in}{2.130306in}}%
\pgfpathlineto{\pgfqpoint{1.973387in}{2.191730in}}%
\pgfpathlineto{\pgfqpoint{1.973812in}{2.191554in}}%
\pgfpathlineto{\pgfqpoint{1.975086in}{2.184936in}}%
\pgfpathlineto{\pgfqpoint{1.978059in}{2.168136in}}%
\pgfpathlineto{\pgfqpoint{1.980183in}{2.166438in}}%
\pgfpathlineto{\pgfqpoint{1.981457in}{2.157161in}}%
\pgfpathlineto{\pgfqpoint{1.986554in}{2.104264in}}%
\pgfpathlineto{\pgfqpoint{1.986978in}{2.104437in}}%
\pgfpathlineto{\pgfqpoint{1.988252in}{2.111214in}}%
\pgfpathlineto{\pgfqpoint{1.990376in}{2.142437in}}%
\pgfpathlineto{\pgfqpoint{1.993774in}{2.190097in}}%
\pgfpathlineto{\pgfqpoint{1.995048in}{2.192841in}}%
\pgfpathlineto{\pgfqpoint{1.995473in}{2.192609in}}%
\pgfpathlineto{\pgfqpoint{1.997597in}{2.191258in}}%
\pgfpathlineto{\pgfqpoint{2.000145in}{2.192124in}}%
\pgfpathlineto{\pgfqpoint{2.001419in}{2.188266in}}%
\pgfpathlineto{\pgfqpoint{2.005242in}{2.173998in}}%
\pgfpathlineto{\pgfqpoint{2.006516in}{2.175929in}}%
\pgfpathlineto{\pgfqpoint{2.007790in}{2.177969in}}%
\pgfpathlineto{\pgfqpoint{2.008215in}{2.177763in}}%
\pgfpathlineto{\pgfqpoint{2.009489in}{2.172674in}}%
\pgfpathlineto{\pgfqpoint{2.012037in}{2.145295in}}%
\pgfpathlineto{\pgfqpoint{2.015010in}{2.122019in}}%
\pgfpathlineto{\pgfqpoint{2.015860in}{2.123771in}}%
\pgfpathlineto{\pgfqpoint{2.019682in}{2.145432in}}%
\pgfpathlineto{\pgfqpoint{2.020532in}{2.143245in}}%
\pgfpathlineto{\pgfqpoint{2.023930in}{2.125192in}}%
\pgfpathlineto{\pgfqpoint{2.024354in}{2.126120in}}%
\pgfpathlineto{\pgfqpoint{2.025629in}{2.135972in}}%
\pgfpathlineto{\pgfqpoint{2.031150in}{2.207619in}}%
\pgfpathlineto{\pgfqpoint{2.032000in}{2.205185in}}%
\pgfpathlineto{\pgfqpoint{2.034548in}{2.180558in}}%
\pgfpathlineto{\pgfqpoint{2.037096in}{2.167278in}}%
\pgfpathlineto{\pgfqpoint{2.038371in}{2.167383in}}%
\pgfpathlineto{\pgfqpoint{2.040069in}{2.170091in}}%
\pgfpathlineto{\pgfqpoint{2.043043in}{2.177309in}}%
\pgfpathlineto{\pgfqpoint{2.043467in}{2.176859in}}%
\pgfpathlineto{\pgfqpoint{2.044741in}{2.170704in}}%
\pgfpathlineto{\pgfqpoint{2.049413in}{2.139832in}}%
\pgfpathlineto{\pgfqpoint{2.051962in}{2.140885in}}%
\pgfpathlineto{\pgfqpoint{2.053661in}{2.142295in}}%
\pgfpathlineto{\pgfqpoint{2.056209in}{2.149686in}}%
\pgfpathlineto{\pgfqpoint{2.060881in}{2.166798in}}%
\pgfpathlineto{\pgfqpoint{2.063005in}{2.175334in}}%
\pgfpathlineto{\pgfqpoint{2.063854in}{2.173714in}}%
\pgfpathlineto{\pgfqpoint{2.065553in}{2.161056in}}%
\pgfpathlineto{\pgfqpoint{2.068526in}{2.139079in}}%
\pgfpathlineto{\pgfqpoint{2.069376in}{2.139328in}}%
\pgfpathlineto{\pgfqpoint{2.071075in}{2.147167in}}%
\pgfpathlineto{\pgfqpoint{2.074897in}{2.164851in}}%
\pgfpathlineto{\pgfqpoint{2.077021in}{2.166661in}}%
\pgfpathlineto{\pgfqpoint{2.078295in}{2.167856in}}%
\pgfpathlineto{\pgfqpoint{2.079569in}{2.172231in}}%
\pgfpathlineto{\pgfqpoint{2.082118in}{2.192578in}}%
\pgfpathlineto{\pgfqpoint{2.084241in}{2.203526in}}%
\pgfpathlineto{\pgfqpoint{2.085091in}{2.200867in}}%
\pgfpathlineto{\pgfqpoint{2.087214in}{2.178152in}}%
\pgfpathlineto{\pgfqpoint{2.090188in}{2.150849in}}%
\pgfpathlineto{\pgfqpoint{2.091037in}{2.150978in}}%
\pgfpathlineto{\pgfqpoint{2.093585in}{2.157402in}}%
\pgfpathlineto{\pgfqpoint{2.094010in}{2.156652in}}%
\pgfpathlineto{\pgfqpoint{2.095709in}{2.145631in}}%
\pgfpathlineto{\pgfqpoint{2.098682in}{2.127513in}}%
\pgfpathlineto{\pgfqpoint{2.099532in}{2.129289in}}%
\pgfpathlineto{\pgfqpoint{2.105053in}{2.149732in}}%
\pgfpathlineto{\pgfqpoint{2.107601in}{2.149776in}}%
\pgfpathlineto{\pgfqpoint{2.109725in}{2.148675in}}%
\pgfpathlineto{\pgfqpoint{2.110999in}{2.153072in}}%
\pgfpathlineto{\pgfqpoint{2.113123in}{2.175361in}}%
\pgfpathlineto{\pgfqpoint{2.115671in}{2.195520in}}%
\pgfpathlineto{\pgfqpoint{2.116521in}{2.192624in}}%
\pgfpathlineto{\pgfqpoint{2.121193in}{2.161761in}}%
\pgfpathlineto{\pgfqpoint{2.121617in}{2.162241in}}%
\pgfpathlineto{\pgfqpoint{2.123741in}{2.170110in}}%
\pgfpathlineto{\pgfqpoint{2.125865in}{2.175041in}}%
\pgfpathlineto{\pgfqpoint{2.127139in}{2.171051in}}%
\pgfpathlineto{\pgfqpoint{2.131386in}{2.149867in}}%
\pgfpathlineto{\pgfqpoint{2.132660in}{2.152888in}}%
\pgfpathlineto{\pgfqpoint{2.136908in}{2.169664in}}%
\pgfpathlineto{\pgfqpoint{2.137332in}{2.169486in}}%
\pgfpathlineto{\pgfqpoint{2.138607in}{2.164385in}}%
\pgfpathlineto{\pgfqpoint{2.143279in}{2.132195in}}%
\pgfpathlineto{\pgfqpoint{2.143703in}{2.132771in}}%
\pgfpathlineto{\pgfqpoint{2.145402in}{2.141563in}}%
\pgfpathlineto{\pgfqpoint{2.150924in}{2.178967in}}%
\pgfpathlineto{\pgfqpoint{2.152198in}{2.176519in}}%
\pgfpathlineto{\pgfqpoint{2.157719in}{2.158143in}}%
\pgfpathlineto{\pgfqpoint{2.158569in}{2.159171in}}%
\pgfpathlineto{\pgfqpoint{2.162391in}{2.169111in}}%
\pgfpathlineto{\pgfqpoint{2.162816in}{2.168563in}}%
\pgfpathlineto{\pgfqpoint{2.165789in}{2.164007in}}%
\pgfpathlineto{\pgfqpoint{2.168762in}{2.165564in}}%
\pgfpathlineto{\pgfqpoint{2.170461in}{2.159542in}}%
\pgfpathlineto{\pgfqpoint{2.176832in}{2.129670in}}%
\pgfpathlineto{\pgfqpoint{2.177682in}{2.130751in}}%
\pgfpathlineto{\pgfqpoint{2.178956in}{2.138072in}}%
\pgfpathlineto{\pgfqpoint{2.184053in}{2.177486in}}%
\pgfpathlineto{\pgfqpoint{2.185327in}{2.176694in}}%
\pgfpathlineto{\pgfqpoint{2.188300in}{2.168919in}}%
\pgfpathlineto{\pgfqpoint{2.191273in}{2.162236in}}%
\pgfpathlineto{\pgfqpoint{2.192123in}{2.164461in}}%
\pgfpathlineto{\pgfqpoint{2.194246in}{2.181940in}}%
\pgfpathlineto{\pgfqpoint{2.197219in}{2.201774in}}%
\pgfpathlineto{\pgfqpoint{2.198069in}{2.200433in}}%
\pgfpathlineto{\pgfqpoint{2.199768in}{2.189708in}}%
\pgfpathlineto{\pgfqpoint{2.205289in}{2.149020in}}%
\pgfpathlineto{\pgfqpoint{2.207838in}{2.140672in}}%
\pgfpathlineto{\pgfqpoint{2.210811in}{2.113446in}}%
\pgfpathlineto{\pgfqpoint{2.212934in}{2.101592in}}%
\pgfpathlineto{\pgfqpoint{2.213784in}{2.103994in}}%
\pgfpathlineto{\pgfqpoint{2.215483in}{2.121944in}}%
\pgfpathlineto{\pgfqpoint{2.219730in}{2.171461in}}%
\pgfpathlineto{\pgfqpoint{2.222703in}{2.179105in}}%
\pgfpathlineto{\pgfqpoint{2.223977in}{2.178493in}}%
\pgfpathlineto{\pgfqpoint{2.225676in}{2.172227in}}%
\pgfpathlineto{\pgfqpoint{2.228225in}{2.164119in}}%
\pgfpathlineto{\pgfqpoint{2.229074in}{2.165188in}}%
\pgfpathlineto{\pgfqpoint{2.231198in}{2.174830in}}%
\pgfpathlineto{\pgfqpoint{2.236294in}{2.207560in}}%
\pgfpathlineto{\pgfqpoint{2.237144in}{2.205720in}}%
\pgfpathlineto{\pgfqpoint{2.239268in}{2.187819in}}%
\pgfpathlineto{\pgfqpoint{2.242241in}{2.168190in}}%
\pgfpathlineto{\pgfqpoint{2.244364in}{2.158372in}}%
\pgfpathlineto{\pgfqpoint{2.250310in}{2.120814in}}%
\pgfpathlineto{\pgfqpoint{2.252009in}{2.121972in}}%
\pgfpathlineto{\pgfqpoint{2.257956in}{2.131154in}}%
\pgfpathlineto{\pgfqpoint{2.260079in}{2.145097in}}%
\pgfpathlineto{\pgfqpoint{2.265601in}{2.187434in}}%
\pgfpathlineto{\pgfqpoint{2.268149in}{2.192756in}}%
\pgfpathlineto{\pgfqpoint{2.270273in}{2.193354in}}%
\pgfpathlineto{\pgfqpoint{2.271972in}{2.191657in}}%
\pgfpathlineto{\pgfqpoint{2.276644in}{2.182085in}}%
\pgfpathlineto{\pgfqpoint{2.280891in}{2.160250in}}%
\pgfpathlineto{\pgfqpoint{2.281316in}{2.160778in}}%
\pgfpathlineto{\pgfqpoint{2.283864in}{2.168215in}}%
\pgfpathlineto{\pgfqpoint{2.284289in}{2.167590in}}%
\pgfpathlineto{\pgfqpoint{2.285563in}{2.159021in}}%
\pgfpathlineto{\pgfqpoint{2.291084in}{2.102357in}}%
\pgfpathlineto{\pgfqpoint{2.291509in}{2.102421in}}%
\pgfpathlineto{\pgfqpoint{2.292783in}{2.107255in}}%
\pgfpathlineto{\pgfqpoint{2.295757in}{2.134603in}}%
\pgfpathlineto{\pgfqpoint{2.300004in}{2.166735in}}%
\pgfpathlineto{\pgfqpoint{2.305101in}{2.194963in}}%
\pgfpathlineto{\pgfqpoint{2.307224in}{2.195806in}}%
\pgfpathlineto{\pgfqpoint{2.308923in}{2.197224in}}%
\pgfpathlineto{\pgfqpoint{2.311047in}{2.202805in}}%
\pgfpathlineto{\pgfqpoint{2.314445in}{2.216433in}}%
\pgfpathlineto{\pgfqpoint{2.314869in}{2.215581in}}%
\pgfpathlineto{\pgfqpoint{2.316144in}{2.202949in}}%
\pgfpathlineto{\pgfqpoint{2.318267in}{2.144299in}}%
\pgfpathlineto{\pgfqpoint{2.321240in}{2.068820in}}%
\pgfpathlineto{\pgfqpoint{2.321665in}{2.067875in}}%
\pgfpathlineto{\pgfqpoint{2.322514in}{2.074209in}}%
\pgfpathlineto{\pgfqpoint{2.328036in}{2.159493in}}%
\pgfpathlineto{\pgfqpoint{2.328885in}{2.156564in}}%
\pgfpathlineto{\pgfqpoint{2.331434in}{2.139674in}}%
\pgfpathlineto{\pgfqpoint{2.331859in}{2.139783in}}%
\pgfpathlineto{\pgfqpoint{2.332708in}{2.145258in}}%
\pgfpathlineto{\pgfqpoint{2.334832in}{2.184884in}}%
\pgfpathlineto{\pgfqpoint{2.337805in}{2.231287in}}%
\pgfpathlineto{\pgfqpoint{2.338229in}{2.231720in}}%
\pgfpathlineto{\pgfqpoint{2.339079in}{2.227025in}}%
\pgfpathlineto{\pgfqpoint{2.340778in}{2.198298in}}%
\pgfpathlineto{\pgfqpoint{2.345450in}{2.111547in}}%
\pgfpathlineto{\pgfqpoint{2.346724in}{2.109578in}}%
\pgfpathlineto{\pgfqpoint{2.347149in}{2.109815in}}%
\pgfpathlineto{\pgfqpoint{2.349272in}{2.112564in}}%
\pgfpathlineto{\pgfqpoint{2.350971in}{2.120393in}}%
\pgfpathlineto{\pgfqpoint{2.354794in}{2.147828in}}%
\pgfpathlineto{\pgfqpoint{2.355219in}{2.147728in}}%
\pgfpathlineto{\pgfqpoint{2.356493in}{2.146118in}}%
\pgfpathlineto{\pgfqpoint{2.356918in}{2.146542in}}%
\pgfpathlineto{\pgfqpoint{2.357767in}{2.151047in}}%
\pgfpathlineto{\pgfqpoint{2.359466in}{2.176940in}}%
\pgfpathlineto{\pgfqpoint{2.362864in}{2.227124in}}%
\pgfpathlineto{\pgfqpoint{2.363713in}{2.226302in}}%
\pgfpathlineto{\pgfqpoint{2.365412in}{2.220521in}}%
\pgfpathlineto{\pgfqpoint{2.365837in}{2.220903in}}%
\pgfpathlineto{\pgfqpoint{2.367111in}{2.231028in}}%
\pgfpathlineto{\pgfqpoint{2.370934in}{2.285714in}}%
\pgfpathlineto{\pgfqpoint{2.371358in}{2.284433in}}%
\pgfpathlineto{\pgfqpoint{2.373057in}{2.263729in}}%
\pgfpathlineto{\pgfqpoint{2.376030in}{2.204024in}}%
\pgfpathlineto{\pgfqpoint{2.378154in}{2.080077in}}%
\pgfpathlineto{\pgfqpoint{2.381552in}{1.878629in}}%
\pgfpathlineto{\pgfqpoint{2.381977in}{1.874883in}}%
\pgfpathlineto{\pgfqpoint{2.382401in}{1.876352in}}%
\pgfpathlineto{\pgfqpoint{2.383675in}{1.906548in}}%
\pgfpathlineto{\pgfqpoint{2.386224in}{2.053250in}}%
\pgfpathlineto{\pgfqpoint{2.389622in}{2.254726in}}%
\pgfpathlineto{\pgfqpoint{2.390046in}{2.253361in}}%
\pgfpathlineto{\pgfqpoint{2.391321in}{2.201578in}}%
\pgfpathlineto{\pgfqpoint{2.394294in}{2.029763in}}%
\pgfpathlineto{\pgfqpoint{2.394718in}{2.030638in}}%
\pgfpathlineto{\pgfqpoint{2.395993in}{2.092128in}}%
\pgfpathlineto{\pgfqpoint{2.400240in}{2.418730in}}%
\pgfpathlineto{\pgfqpoint{2.400665in}{2.415478in}}%
\pgfpathlineto{\pgfqpoint{2.402364in}{2.350901in}}%
\pgfpathlineto{\pgfqpoint{2.404487in}{2.298502in}}%
\pgfpathlineto{\pgfqpoint{2.405337in}{2.308440in}}%
\pgfpathlineto{\pgfqpoint{2.407460in}{2.338874in}}%
\pgfpathlineto{\pgfqpoint{2.408310in}{2.324334in}}%
\pgfpathlineto{\pgfqpoint{2.410009in}{2.230905in}}%
\pgfpathlineto{\pgfqpoint{2.415105in}{1.881624in}}%
\pgfpathlineto{\pgfqpoint{2.415530in}{1.882321in}}%
\pgfpathlineto{\pgfqpoint{2.416804in}{1.924403in}}%
\pgfpathlineto{\pgfqpoint{2.420627in}{2.091399in}}%
\pgfpathlineto{\pgfqpoint{2.421476in}{2.082475in}}%
\pgfpathlineto{\pgfqpoint{2.423175in}{2.060724in}}%
\pgfpathlineto{\pgfqpoint{2.424025in}{2.075998in}}%
\pgfpathlineto{\pgfqpoint{2.425724in}{2.184727in}}%
\pgfpathlineto{\pgfqpoint{2.428697in}{2.389914in}}%
\pgfpathlineto{\pgfqpoint{2.429122in}{2.390205in}}%
\pgfpathlineto{\pgfqpoint{2.429971in}{2.361415in}}%
\pgfpathlineto{\pgfqpoint{2.435068in}{2.043252in}}%
\pgfpathlineto{\pgfqpoint{2.435492in}{2.047494in}}%
\pgfpathlineto{\pgfqpoint{2.436767in}{2.091992in}}%
\pgfpathlineto{\pgfqpoint{2.441439in}{2.330138in}}%
\pgfpathlineto{\pgfqpoint{2.441863in}{2.328878in}}%
\pgfpathlineto{\pgfqpoint{2.443138in}{2.291245in}}%
\pgfpathlineto{\pgfqpoint{2.447385in}{2.088928in}}%
\pgfpathlineto{\pgfqpoint{2.447810in}{2.092930in}}%
\pgfpathlineto{\pgfqpoint{2.449509in}{2.162442in}}%
\pgfpathlineto{\pgfqpoint{2.451207in}{2.214877in}}%
\pgfpathlineto{\pgfqpoint{2.452057in}{2.193331in}}%
\pgfpathlineto{\pgfqpoint{2.455879in}{1.970128in}}%
\pgfpathlineto{\pgfqpoint{2.456729in}{1.988506in}}%
\pgfpathlineto{\pgfqpoint{2.460976in}{2.177514in}}%
\pgfpathlineto{\pgfqpoint{2.461401in}{2.176181in}}%
\pgfpathlineto{\pgfqpoint{2.462675in}{2.151863in}}%
\pgfpathlineto{\pgfqpoint{2.465224in}{2.096591in}}%
\pgfpathlineto{\pgfqpoint{2.465648in}{2.097119in}}%
\pgfpathlineto{\pgfqpoint{2.466498in}{2.111457in}}%
\pgfpathlineto{\pgfqpoint{2.468621in}{2.212071in}}%
\pgfpathlineto{\pgfqpoint{2.471170in}{2.313112in}}%
\pgfpathlineto{\pgfqpoint{2.471594in}{2.311722in}}%
\pgfpathlineto{\pgfqpoint{2.472869in}{2.268612in}}%
\pgfpathlineto{\pgfqpoint{2.476266in}{2.096421in}}%
\pgfpathlineto{\pgfqpoint{2.476691in}{2.096517in}}%
\pgfpathlineto{\pgfqpoint{2.477965in}{2.134988in}}%
\pgfpathlineto{\pgfqpoint{2.481363in}{2.282842in}}%
\pgfpathlineto{\pgfqpoint{2.481788in}{2.281012in}}%
\pgfpathlineto{\pgfqpoint{2.483062in}{2.236423in}}%
\pgfpathlineto{\pgfqpoint{2.487309in}{2.019323in}}%
\pgfpathlineto{\pgfqpoint{2.488159in}{2.042808in}}%
\pgfpathlineto{\pgfqpoint{2.490707in}{2.231789in}}%
\pgfpathlineto{\pgfqpoint{2.492831in}{2.332111in}}%
\pgfpathlineto{\pgfqpoint{2.493256in}{2.331699in}}%
\pgfpathlineto{\pgfqpoint{2.494530in}{2.291539in}}%
\pgfpathlineto{\pgfqpoint{2.500051in}{2.041579in}}%
\pgfpathlineto{\pgfqpoint{2.500901in}{2.036861in}}%
\pgfpathlineto{\pgfqpoint{2.501326in}{2.038104in}}%
\pgfpathlineto{\pgfqpoint{2.502600in}{2.054319in}}%
\pgfpathlineto{\pgfqpoint{2.505998in}{2.105407in}}%
\pgfpathlineto{\pgfqpoint{2.506422in}{2.105403in}}%
\pgfpathlineto{\pgfqpoint{2.507696in}{2.096108in}}%
\pgfpathlineto{\pgfqpoint{2.510670in}{2.064479in}}%
\pgfpathlineto{\pgfqpoint{2.511094in}{2.065952in}}%
\pgfpathlineto{\pgfqpoint{2.512368in}{2.087696in}}%
\pgfpathlineto{\pgfqpoint{2.522137in}{2.362282in}}%
\pgfpathlineto{\pgfqpoint{2.522562in}{2.361354in}}%
\pgfpathlineto{\pgfqpoint{2.523836in}{2.338385in}}%
\pgfpathlineto{\pgfqpoint{2.526385in}{2.229418in}}%
\pgfpathlineto{\pgfqpoint{2.531057in}{2.016220in}}%
\pgfpathlineto{\pgfqpoint{2.531481in}{2.016925in}}%
\pgfpathlineto{\pgfqpoint{2.532755in}{2.048223in}}%
\pgfpathlineto{\pgfqpoint{2.537003in}{2.190609in}}%
\pgfpathlineto{\pgfqpoint{2.537852in}{2.183990in}}%
\pgfpathlineto{\pgfqpoint{2.539976in}{2.130632in}}%
\pgfpathlineto{\pgfqpoint{2.543374in}{2.057428in}}%
\pgfpathlineto{\pgfqpoint{2.546347in}{2.043698in}}%
\pgfpathlineto{\pgfqpoint{2.547621in}{2.040875in}}%
\pgfpathlineto{\pgfqpoint{2.548470in}{2.047496in}}%
\pgfpathlineto{\pgfqpoint{2.549745in}{2.082240in}}%
\pgfpathlineto{\pgfqpoint{2.554841in}{2.328268in}}%
\pgfpathlineto{\pgfqpoint{2.555691in}{2.317762in}}%
\pgfpathlineto{\pgfqpoint{2.560363in}{2.183123in}}%
\pgfpathlineto{\pgfqpoint{2.561212in}{2.188379in}}%
\pgfpathlineto{\pgfqpoint{2.565035in}{2.238805in}}%
\pgfpathlineto{\pgfqpoint{2.565460in}{2.238163in}}%
\pgfpathlineto{\pgfqpoint{2.567159in}{2.233446in}}%
\pgfpathlineto{\pgfqpoint{2.568008in}{2.237942in}}%
\pgfpathlineto{\pgfqpoint{2.570132in}{2.261693in}}%
\pgfpathlineto{\pgfqpoint{2.570556in}{2.260908in}}%
\pgfpathlineto{\pgfqpoint{2.571406in}{2.245206in}}%
\pgfpathlineto{\pgfqpoint{2.573105in}{2.151520in}}%
\pgfpathlineto{\pgfqpoint{2.576927in}{1.914424in}}%
\pgfpathlineto{\pgfqpoint{2.577777in}{1.909117in}}%
\pgfpathlineto{\pgfqpoint{2.579051in}{1.934354in}}%
\pgfpathlineto{\pgfqpoint{2.584997in}{2.117252in}}%
\pgfpathlineto{\pgfqpoint{2.585847in}{2.113011in}}%
\pgfpathlineto{\pgfqpoint{2.587970in}{2.088247in}}%
\pgfpathlineto{\pgfqpoint{2.588395in}{2.088607in}}%
\pgfpathlineto{\pgfqpoint{2.589244in}{2.102758in}}%
\pgfpathlineto{\pgfqpoint{2.590943in}{2.193574in}}%
\pgfpathlineto{\pgfqpoint{2.595615in}{2.472573in}}%
\pgfpathlineto{\pgfqpoint{2.596465in}{2.462084in}}%
\pgfpathlineto{\pgfqpoint{2.598164in}{2.379942in}}%
\pgfpathlineto{\pgfqpoint{2.601986in}{2.197659in}}%
\pgfpathlineto{\pgfqpoint{2.602411in}{2.197797in}}%
\pgfpathlineto{\pgfqpoint{2.604110in}{2.222219in}}%
\pgfpathlineto{\pgfqpoint{2.605384in}{2.231914in}}%
\pgfpathlineto{\pgfqpoint{2.606234in}{2.221729in}}%
\pgfpathlineto{\pgfqpoint{2.608357in}{2.143289in}}%
\pgfpathlineto{\pgfqpoint{2.614304in}{1.844737in}}%
\pgfpathlineto{\pgfqpoint{2.615153in}{1.854019in}}%
\pgfpathlineto{\pgfqpoint{2.616852in}{1.935665in}}%
\pgfpathlineto{\pgfqpoint{2.620674in}{2.137010in}}%
\pgfpathlineto{\pgfqpoint{2.621524in}{2.145649in}}%
\pgfpathlineto{\pgfqpoint{2.621949in}{2.145156in}}%
\pgfpathlineto{\pgfqpoint{2.624497in}{2.126244in}}%
\pgfpathlineto{\pgfqpoint{2.624922in}{2.129154in}}%
\pgfpathlineto{\pgfqpoint{2.626196in}{2.161875in}}%
\pgfpathlineto{\pgfqpoint{2.629169in}{2.342953in}}%
\pgfpathlineto{\pgfqpoint{2.631293in}{2.415362in}}%
\pgfpathlineto{\pgfqpoint{2.632142in}{2.405367in}}%
\pgfpathlineto{\pgfqpoint{2.634266in}{2.307682in}}%
\pgfpathlineto{\pgfqpoint{2.637664in}{2.166075in}}%
\pgfpathlineto{\pgfqpoint{2.638938in}{2.157396in}}%
\pgfpathlineto{\pgfqpoint{2.639363in}{2.158083in}}%
\pgfpathlineto{\pgfqpoint{2.640637in}{2.160845in}}%
\pgfpathlineto{\pgfqpoint{2.641486in}{2.156502in}}%
\pgfpathlineto{\pgfqpoint{2.643610in}{2.118311in}}%
\pgfpathlineto{\pgfqpoint{2.644884in}{2.101878in}}%
\pgfpathlineto{\pgfqpoint{2.645309in}{2.102059in}}%
\pgfpathlineto{\pgfqpoint{2.646158in}{2.112758in}}%
\pgfpathlineto{\pgfqpoint{2.648707in}{2.197216in}}%
\pgfpathlineto{\pgfqpoint{2.650830in}{2.244431in}}%
\pgfpathlineto{\pgfqpoint{2.651255in}{2.244836in}}%
\pgfpathlineto{\pgfqpoint{2.652104in}{2.234268in}}%
\pgfpathlineto{\pgfqpoint{2.653803in}{2.167326in}}%
\pgfpathlineto{\pgfqpoint{2.657201in}{2.019492in}}%
\pgfpathlineto{\pgfqpoint{2.657626in}{2.021891in}}%
\pgfpathlineto{\pgfqpoint{2.658900in}{2.064331in}}%
\pgfpathlineto{\pgfqpoint{2.662298in}{2.193213in}}%
\pgfpathlineto{\pgfqpoint{2.663147in}{2.177551in}}%
\pgfpathlineto{\pgfqpoint{2.666545in}{2.078124in}}%
\pgfpathlineto{\pgfqpoint{2.666970in}{2.080350in}}%
\pgfpathlineto{\pgfqpoint{2.668669in}{2.118493in}}%
\pgfpathlineto{\pgfqpoint{2.670368in}{2.149172in}}%
\pgfpathlineto{\pgfqpoint{2.670793in}{2.148920in}}%
\pgfpathlineto{\pgfqpoint{2.672067in}{2.127185in}}%
\pgfpathlineto{\pgfqpoint{2.676314in}{2.005232in}}%
\pgfpathlineto{\pgfqpoint{2.676739in}{2.007693in}}%
\pgfpathlineto{\pgfqpoint{2.678013in}{2.049588in}}%
\pgfpathlineto{\pgfqpoint{2.680561in}{2.266647in}}%
\pgfpathlineto{\pgfqpoint{2.683110in}{2.432627in}}%
\pgfpathlineto{\pgfqpoint{2.683534in}{2.434167in}}%
\pgfpathlineto{\pgfqpoint{2.684384in}{2.414431in}}%
\pgfpathlineto{\pgfqpoint{2.689056in}{2.219116in}}%
\pgfpathlineto{\pgfqpoint{2.689481in}{2.219199in}}%
\pgfpathlineto{\pgfqpoint{2.692029in}{2.241117in}}%
\pgfpathlineto{\pgfqpoint{2.692454in}{2.238722in}}%
\pgfpathlineto{\pgfqpoint{2.693728in}{2.205579in}}%
\pgfpathlineto{\pgfqpoint{2.698400in}{1.979340in}}%
\pgfpathlineto{\pgfqpoint{2.699249in}{1.987259in}}%
\pgfpathlineto{\pgfqpoint{2.703497in}{2.086884in}}%
\pgfpathlineto{\pgfqpoint{2.704346in}{2.083805in}}%
\pgfpathlineto{\pgfqpoint{2.706470in}{2.069697in}}%
\pgfpathlineto{\pgfqpoint{2.707319in}{2.077581in}}%
\pgfpathlineto{\pgfqpoint{2.709018in}{2.134994in}}%
\pgfpathlineto{\pgfqpoint{2.713265in}{2.298738in}}%
\pgfpathlineto{\pgfqpoint{2.714115in}{2.301672in}}%
\pgfpathlineto{\pgfqpoint{2.714540in}{2.300531in}}%
\pgfpathlineto{\pgfqpoint{2.718362in}{2.276040in}}%
\pgfpathlineto{\pgfqpoint{2.720061in}{2.255094in}}%
\pgfpathlineto{\pgfqpoint{2.725158in}{2.161337in}}%
\pgfpathlineto{\pgfqpoint{2.726007in}{2.162348in}}%
\pgfpathlineto{\pgfqpoint{2.726857in}{2.163980in}}%
\pgfpathlineto{\pgfqpoint{2.727282in}{2.163521in}}%
\pgfpathlineto{\pgfqpoint{2.728131in}{2.157597in}}%
\pgfpathlineto{\pgfqpoint{2.729830in}{2.119468in}}%
\pgfpathlineto{\pgfqpoint{2.734077in}{1.994139in}}%
\pgfpathlineto{\pgfqpoint{2.734927in}{2.000739in}}%
\pgfpathlineto{\pgfqpoint{2.737900in}{2.045318in}}%
\pgfpathlineto{\pgfqpoint{2.738324in}{2.042441in}}%
\pgfpathlineto{\pgfqpoint{2.740023in}{2.003178in}}%
\pgfpathlineto{\pgfqpoint{2.741722in}{1.970044in}}%
\pgfpathlineto{\pgfqpoint{2.742147in}{1.971607in}}%
\pgfpathlineto{\pgfqpoint{2.743421in}{2.007368in}}%
\pgfpathlineto{\pgfqpoint{2.748943in}{2.238695in}}%
\pgfpathlineto{\pgfqpoint{2.751066in}{2.284608in}}%
\pgfpathlineto{\pgfqpoint{2.756588in}{2.459207in}}%
\pgfpathlineto{\pgfqpoint{2.757013in}{2.457693in}}%
\pgfpathlineto{\pgfqpoint{2.758287in}{2.434512in}}%
\pgfpathlineto{\pgfqpoint{2.765932in}{2.184986in}}%
\pgfpathlineto{\pgfqpoint{2.771029in}{1.902684in}}%
\pgfpathlineto{\pgfqpoint{2.774002in}{1.870898in}}%
\pgfpathlineto{\pgfqpoint{2.775276in}{1.864648in}}%
\pgfpathlineto{\pgfqpoint{2.776125in}{1.870904in}}%
\pgfpathlineto{\pgfqpoint{2.777824in}{1.920136in}}%
\pgfpathlineto{\pgfqpoint{2.782921in}{2.096187in}}%
\pgfpathlineto{\pgfqpoint{2.785894in}{2.201177in}}%
\pgfpathlineto{\pgfqpoint{2.790566in}{2.368832in}}%
\pgfpathlineto{\pgfqpoint{2.791416in}{2.374166in}}%
\pgfpathlineto{\pgfqpoint{2.791840in}{2.372988in}}%
\pgfpathlineto{\pgfqpoint{2.793539in}{2.351705in}}%
\pgfpathlineto{\pgfqpoint{2.796937in}{2.315090in}}%
\pgfpathlineto{\pgfqpoint{2.799061in}{2.283268in}}%
\pgfpathlineto{\pgfqpoint{2.802459in}{2.225745in}}%
\pgfpathlineto{\pgfqpoint{2.802883in}{2.225655in}}%
\pgfpathlineto{\pgfqpoint{2.804158in}{2.236328in}}%
\pgfpathlineto{\pgfqpoint{2.806706in}{2.260325in}}%
\pgfpathlineto{\pgfqpoint{2.807555in}{2.252463in}}%
\pgfpathlineto{\pgfqpoint{2.809254in}{2.202230in}}%
\pgfpathlineto{\pgfqpoint{2.812652in}{2.010420in}}%
\pgfpathlineto{\pgfqpoint{2.816475in}{1.832663in}}%
\pgfpathlineto{\pgfqpoint{2.817324in}{1.828956in}}%
\pgfpathlineto{\pgfqpoint{2.818598in}{1.843646in}}%
\pgfpathlineto{\pgfqpoint{2.823695in}{1.946949in}}%
\pgfpathlineto{\pgfqpoint{2.827093in}{2.120706in}}%
\pgfpathlineto{\pgfqpoint{2.836437in}{2.645848in}}%
\pgfpathlineto{\pgfqpoint{2.837286in}{2.636793in}}%
\pgfpathlineto{\pgfqpoint{2.838985in}{2.562137in}}%
\pgfpathlineto{\pgfqpoint{2.849604in}{1.882513in}}%
\pgfpathlineto{\pgfqpoint{2.853426in}{1.792961in}}%
\pgfpathlineto{\pgfqpoint{2.855975in}{1.758211in}}%
\pgfpathlineto{\pgfqpoint{2.856399in}{1.758917in}}%
\pgfpathlineto{\pgfqpoint{2.857249in}{1.769110in}}%
\pgfpathlineto{\pgfqpoint{2.858948in}{1.830038in}}%
\pgfpathlineto{\pgfqpoint{2.861921in}{2.052195in}}%
\pgfpathlineto{\pgfqpoint{2.867017in}{2.401772in}}%
\pgfpathlineto{\pgfqpoint{2.870415in}{2.507116in}}%
\pgfpathlineto{\pgfqpoint{2.870840in}{2.509359in}}%
\pgfpathlineto{\pgfqpoint{2.871265in}{2.508699in}}%
\pgfpathlineto{\pgfqpoint{2.872539in}{2.492270in}}%
\pgfpathlineto{\pgfqpoint{2.876786in}{2.426880in}}%
\pgfpathlineto{\pgfqpoint{2.878485in}{2.406623in}}%
\pgfpathlineto{\pgfqpoint{2.880184in}{2.351562in}}%
\pgfpathlineto{\pgfqpoint{2.882732in}{2.177899in}}%
\pgfpathlineto{\pgfqpoint{2.886980in}{1.906341in}}%
\pgfpathlineto{\pgfqpoint{2.888254in}{1.894018in}}%
\pgfpathlineto{\pgfqpoint{2.888679in}{1.895533in}}%
\pgfpathlineto{\pgfqpoint{2.890802in}{1.908986in}}%
\pgfpathlineto{\pgfqpoint{2.891227in}{1.908271in}}%
\pgfpathlineto{\pgfqpoint{2.892501in}{1.894379in}}%
\pgfpathlineto{\pgfqpoint{2.895899in}{1.828407in}}%
\pgfpathlineto{\pgfqpoint{2.896749in}{1.833873in}}%
\pgfpathlineto{\pgfqpoint{2.898023in}{1.879944in}}%
\pgfpathlineto{\pgfqpoint{2.900571in}{2.090522in}}%
\pgfpathlineto{\pgfqpoint{2.904394in}{2.371307in}}%
\pgfpathlineto{\pgfqpoint{2.907791in}{2.463205in}}%
\pgfpathlineto{\pgfqpoint{2.909490in}{2.476840in}}%
\pgfpathlineto{\pgfqpoint{2.910340in}{2.473413in}}%
\pgfpathlineto{\pgfqpoint{2.911614in}{2.453028in}}%
\pgfpathlineto{\pgfqpoint{2.914162in}{2.362942in}}%
\pgfpathlineto{\pgfqpoint{2.922232in}{1.987880in}}%
\pgfpathlineto{\pgfqpoint{2.925205in}{1.846981in}}%
\pgfpathlineto{\pgfqpoint{2.925630in}{1.846884in}}%
\pgfpathlineto{\pgfqpoint{2.926480in}{1.862221in}}%
\pgfpathlineto{\pgfqpoint{2.929028in}{1.979752in}}%
\pgfpathlineto{\pgfqpoint{2.932426in}{2.098657in}}%
\pgfpathlineto{\pgfqpoint{2.936248in}{2.200458in}}%
\pgfpathlineto{\pgfqpoint{2.938797in}{2.253798in}}%
\pgfpathlineto{\pgfqpoint{2.939221in}{2.253983in}}%
\pgfpathlineto{\pgfqpoint{2.940496in}{2.242875in}}%
\pgfpathlineto{\pgfqpoint{2.942619in}{2.224123in}}%
\pgfpathlineto{\pgfqpoint{2.943469in}{2.226663in}}%
\pgfpathlineto{\pgfqpoint{2.951114in}{2.279840in}}%
\pgfpathlineto{\pgfqpoint{2.951963in}{2.281250in}}%
\pgfpathlineto{\pgfqpoint{2.952388in}{2.280894in}}%
\pgfpathlineto{\pgfqpoint{2.953238in}{2.276824in}}%
\pgfpathlineto{\pgfqpoint{2.954936in}{2.249996in}}%
\pgfpathlineto{\pgfqpoint{2.957910in}{2.144747in}}%
\pgfpathlineto{\pgfqpoint{2.962157in}{2.003672in}}%
\pgfpathlineto{\pgfqpoint{2.964705in}{1.972280in}}%
\pgfpathlineto{\pgfqpoint{2.965555in}{1.969937in}}%
\pgfpathlineto{\pgfqpoint{2.965979in}{1.970236in}}%
\pgfpathlineto{\pgfqpoint{2.967254in}{1.976341in}}%
\pgfpathlineto{\pgfqpoint{2.969377in}{2.003380in}}%
\pgfpathlineto{\pgfqpoint{2.971501in}{2.062950in}}%
\pgfpathlineto{\pgfqpoint{2.977872in}{2.272924in}}%
\pgfpathlineto{\pgfqpoint{2.982544in}{2.386370in}}%
\pgfpathlineto{\pgfqpoint{2.983393in}{2.382614in}}%
\pgfpathlineto{\pgfqpoint{2.985092in}{2.342991in}}%
\pgfpathlineto{\pgfqpoint{2.989764in}{2.143495in}}%
\pgfpathlineto{\pgfqpoint{2.992737in}{2.060170in}}%
\pgfpathlineto{\pgfqpoint{2.994861in}{2.048043in}}%
\pgfpathlineto{\pgfqpoint{2.995710in}{2.048223in}}%
\pgfpathlineto{\pgfqpoint{2.996985in}{2.051557in}}%
\pgfpathlineto{\pgfqpoint{2.999533in}{2.067917in}}%
\pgfpathlineto{\pgfqpoint{3.002081in}{2.100138in}}%
\pgfpathlineto{\pgfqpoint{3.006329in}{2.202254in}}%
\pgfpathlineto{\pgfqpoint{3.010151in}{2.269756in}}%
\pgfpathlineto{\pgfqpoint{3.010576in}{2.271241in}}%
\pgfpathlineto{\pgfqpoint{3.011001in}{2.270788in}}%
\pgfpathlineto{\pgfqpoint{3.012275in}{2.256741in}}%
\pgfpathlineto{\pgfqpoint{3.014823in}{2.180219in}}%
\pgfpathlineto{\pgfqpoint{3.018221in}{2.087418in}}%
\pgfpathlineto{\pgfqpoint{3.019071in}{2.082848in}}%
\pgfpathlineto{\pgfqpoint{3.019495in}{2.083293in}}%
\pgfpathlineto{\pgfqpoint{3.022044in}{2.096434in}}%
\pgfpathlineto{\pgfqpoint{3.022893in}{2.093906in}}%
\pgfpathlineto{\pgfqpoint{3.024592in}{2.070862in}}%
\pgfpathlineto{\pgfqpoint{3.027990in}{2.018683in}}%
\pgfpathlineto{\pgfqpoint{3.028415in}{2.018487in}}%
\pgfpathlineto{\pgfqpoint{3.029264in}{2.024428in}}%
\pgfpathlineto{\pgfqpoint{3.031388in}{2.068758in}}%
\pgfpathlineto{\pgfqpoint{3.035635in}{2.214237in}}%
\pgfpathlineto{\pgfqpoint{3.041581in}{2.438854in}}%
\pgfpathlineto{\pgfqpoint{3.043280in}{2.457497in}}%
\pgfpathlineto{\pgfqpoint{3.044130in}{2.448670in}}%
\pgfpathlineto{\pgfqpoint{3.045829in}{2.393619in}}%
\pgfpathlineto{\pgfqpoint{3.059420in}{1.829806in}}%
\pgfpathlineto{\pgfqpoint{3.061544in}{1.793672in}}%
\pgfpathlineto{\pgfqpoint{3.061968in}{1.793099in}}%
\pgfpathlineto{\pgfqpoint{3.062818in}{1.799544in}}%
\pgfpathlineto{\pgfqpoint{3.064517in}{1.844721in}}%
\pgfpathlineto{\pgfqpoint{3.067914in}{2.031469in}}%
\pgfpathlineto{\pgfqpoint{3.078957in}{2.658680in}}%
\pgfpathlineto{\pgfqpoint{3.079382in}{2.658408in}}%
\pgfpathlineto{\pgfqpoint{3.080232in}{2.645273in}}%
\pgfpathlineto{\pgfqpoint{3.081931in}{2.568421in}}%
\pgfpathlineto{\pgfqpoint{3.085328in}{2.266139in}}%
\pgfpathlineto{\pgfqpoint{3.090425in}{1.836793in}}%
\pgfpathlineto{\pgfqpoint{3.092549in}{1.796916in}}%
\pgfpathlineto{\pgfqpoint{3.093398in}{1.803284in}}%
\pgfpathlineto{\pgfqpoint{3.096371in}{1.862683in}}%
\pgfpathlineto{\pgfqpoint{3.102742in}{2.012323in}}%
\pgfpathlineto{\pgfqpoint{3.111237in}{2.406724in}}%
\pgfpathlineto{\pgfqpoint{3.112511in}{2.408200in}}%
\pgfpathlineto{\pgfqpoint{3.113785in}{2.415781in}}%
\pgfpathlineto{\pgfqpoint{3.115909in}{2.430236in}}%
\pgfpathlineto{\pgfqpoint{3.116334in}{2.429353in}}%
\pgfpathlineto{\pgfqpoint{3.117608in}{2.413574in}}%
\pgfpathlineto{\pgfqpoint{3.119731in}{2.347064in}}%
\pgfpathlineto{\pgfqpoint{3.130350in}{1.942240in}}%
\pgfpathlineto{\pgfqpoint{3.132473in}{1.922564in}}%
\pgfpathlineto{\pgfqpoint{3.133323in}{1.925984in}}%
\pgfpathlineto{\pgfqpoint{3.135022in}{1.951107in}}%
\pgfpathlineto{\pgfqpoint{3.138844in}{2.055128in}}%
\pgfpathlineto{\pgfqpoint{3.144366in}{2.206365in}}%
\pgfpathlineto{\pgfqpoint{3.149038in}{2.280464in}}%
\pgfpathlineto{\pgfqpoint{3.154135in}{2.460685in}}%
\pgfpathlineto{\pgfqpoint{3.154984in}{2.446117in}}%
\pgfpathlineto{\pgfqpoint{3.157108in}{2.329514in}}%
\pgfpathlineto{\pgfqpoint{3.161780in}{2.063369in}}%
\pgfpathlineto{\pgfqpoint{3.164328in}{2.025440in}}%
\pgfpathlineto{\pgfqpoint{3.169000in}{1.997529in}}%
\pgfpathlineto{\pgfqpoint{3.170274in}{1.997139in}}%
\pgfpathlineto{\pgfqpoint{3.171548in}{1.999681in}}%
\pgfpathlineto{\pgfqpoint{3.172823in}{2.007207in}}%
\pgfpathlineto{\pgfqpoint{3.174522in}{2.034108in}}%
\pgfpathlineto{\pgfqpoint{3.178344in}{2.154472in}}%
\pgfpathlineto{\pgfqpoint{3.182167in}{2.242277in}}%
\pgfpathlineto{\pgfqpoint{3.185140in}{2.299974in}}%
\pgfpathlineto{\pgfqpoint{3.188962in}{2.387532in}}%
\pgfpathlineto{\pgfqpoint{3.189812in}{2.381349in}}%
\pgfpathlineto{\pgfqpoint{3.191511in}{2.334494in}}%
\pgfpathlineto{\pgfqpoint{3.203403in}{1.901900in}}%
\pgfpathlineto{\pgfqpoint{3.204253in}{1.892767in}}%
\pgfpathlineto{\pgfqpoint{3.204677in}{1.893330in}}%
\pgfpathlineto{\pgfqpoint{3.205951in}{1.914084in}}%
\pgfpathlineto{\pgfqpoint{3.209774in}{2.054024in}}%
\pgfpathlineto{\pgfqpoint{3.222516in}{2.473337in}}%
\pgfpathlineto{\pgfqpoint{3.223365in}{2.475949in}}%
\pgfpathlineto{\pgfqpoint{3.224215in}{2.468668in}}%
\pgfpathlineto{\pgfqpoint{3.225914in}{2.421092in}}%
\pgfpathlineto{\pgfqpoint{3.228462in}{2.267249in}}%
\pgfpathlineto{\pgfqpoint{3.234408in}{1.897462in}}%
\pgfpathlineto{\pgfqpoint{3.236957in}{1.855012in}}%
\pgfpathlineto{\pgfqpoint{3.238231in}{1.849230in}}%
\pgfpathlineto{\pgfqpoint{3.238656in}{1.849306in}}%
\pgfpathlineto{\pgfqpoint{3.239505in}{1.853335in}}%
\pgfpathlineto{\pgfqpoint{3.241204in}{1.881501in}}%
\pgfpathlineto{\pgfqpoint{3.244177in}{1.993353in}}%
\pgfpathlineto{\pgfqpoint{3.248424in}{2.243478in}}%
\pgfpathlineto{\pgfqpoint{3.252672in}{2.494865in}}%
\pgfpathlineto{\pgfqpoint{3.253946in}{2.510386in}}%
\pgfpathlineto{\pgfqpoint{3.254371in}{2.508286in}}%
\pgfpathlineto{\pgfqpoint{3.255645in}{2.483125in}}%
\pgfpathlineto{\pgfqpoint{3.259043in}{2.341287in}}%
\pgfpathlineto{\pgfqpoint{3.269661in}{1.910919in}}%
\pgfpathlineto{\pgfqpoint{3.270086in}{1.908822in}}%
\pgfpathlineto{\pgfqpoint{3.270510in}{1.909221in}}%
\pgfpathlineto{\pgfqpoint{3.271785in}{1.924421in}}%
\pgfpathlineto{\pgfqpoint{3.275607in}{2.028930in}}%
\pgfpathlineto{\pgfqpoint{3.280279in}{2.171414in}}%
\pgfpathlineto{\pgfqpoint{3.284951in}{2.329315in}}%
\pgfpathlineto{\pgfqpoint{3.286225in}{2.335135in}}%
\pgfpathlineto{\pgfqpoint{3.287075in}{2.330607in}}%
\pgfpathlineto{\pgfqpoint{3.289198in}{2.297704in}}%
\pgfpathlineto{\pgfqpoint{3.293446in}{2.229665in}}%
\pgfpathlineto{\pgfqpoint{3.295994in}{2.214812in}}%
\pgfpathlineto{\pgfqpoint{3.298118in}{2.174896in}}%
\pgfpathlineto{\pgfqpoint{3.305338in}{2.014532in}}%
\pgfpathlineto{\pgfqpoint{3.306188in}{2.013958in}}%
\pgfpathlineto{\pgfqpoint{3.307887in}{2.019052in}}%
\pgfpathlineto{\pgfqpoint{3.309585in}{2.030334in}}%
\pgfpathlineto{\pgfqpoint{3.311709in}{2.065347in}}%
\pgfpathlineto{\pgfqpoint{3.323602in}{2.320051in}}%
\pgfpathlineto{\pgfqpoint{3.326150in}{2.334127in}}%
\pgfpathlineto{\pgfqpoint{3.326999in}{2.332944in}}%
\pgfpathlineto{\pgfqpoint{3.328274in}{2.322484in}}%
\pgfpathlineto{\pgfqpoint{3.330822in}{2.270968in}}%
\pgfpathlineto{\pgfqpoint{3.335494in}{2.124080in}}%
\pgfpathlineto{\pgfqpoint{3.341440in}{1.912802in}}%
\pgfpathlineto{\pgfqpoint{3.342714in}{1.903573in}}%
\pgfpathlineto{\pgfqpoint{3.343139in}{1.904705in}}%
\pgfpathlineto{\pgfqpoint{3.344413in}{1.921477in}}%
\pgfpathlineto{\pgfqpoint{3.346962in}{2.003693in}}%
\pgfpathlineto{\pgfqpoint{3.357155in}{2.368050in}}%
\pgfpathlineto{\pgfqpoint{3.359704in}{2.397569in}}%
\pgfpathlineto{\pgfqpoint{3.361827in}{2.404526in}}%
\pgfpathlineto{\pgfqpoint{3.362677in}{2.400929in}}%
\pgfpathlineto{\pgfqpoint{3.363951in}{2.383127in}}%
\pgfpathlineto{\pgfqpoint{3.366074in}{2.311918in}}%
\pgfpathlineto{\pgfqpoint{3.374994in}{1.922676in}}%
\pgfpathlineto{\pgfqpoint{3.375419in}{1.921851in}}%
\pgfpathlineto{\pgfqpoint{3.375843in}{1.922619in}}%
\pgfpathlineto{\pgfqpoint{3.377117in}{1.932696in}}%
\pgfpathlineto{\pgfqpoint{3.380091in}{1.982014in}}%
\pgfpathlineto{\pgfqpoint{3.383488in}{2.073062in}}%
\pgfpathlineto{\pgfqpoint{3.388160in}{2.209759in}}%
\pgfpathlineto{\pgfqpoint{3.393682in}{2.289482in}}%
\pgfpathlineto{\pgfqpoint{3.397080in}{2.343387in}}%
\pgfpathlineto{\pgfqpoint{3.397504in}{2.343861in}}%
\pgfpathlineto{\pgfqpoint{3.397929in}{2.342672in}}%
\pgfpathlineto{\pgfqpoint{3.399628in}{2.324700in}}%
\pgfpathlineto{\pgfqpoint{3.405574in}{2.219751in}}%
\pgfpathlineto{\pgfqpoint{3.409822in}{2.069614in}}%
\pgfpathlineto{\pgfqpoint{3.413219in}{1.978883in}}%
\pgfpathlineto{\pgfqpoint{3.414494in}{1.971032in}}%
\pgfpathlineto{\pgfqpoint{3.414918in}{1.971443in}}%
\pgfpathlineto{\pgfqpoint{3.416193in}{1.978909in}}%
\pgfpathlineto{\pgfqpoint{3.418741in}{2.009966in}}%
\pgfpathlineto{\pgfqpoint{3.421714in}{2.080777in}}%
\pgfpathlineto{\pgfqpoint{3.426386in}{2.191540in}}%
\pgfpathlineto{\pgfqpoint{3.435305in}{2.305356in}}%
\pgfpathlineto{\pgfqpoint{3.436155in}{2.301473in}}%
\pgfpathlineto{\pgfqpoint{3.437854in}{2.277258in}}%
\pgfpathlineto{\pgfqpoint{3.445924in}{2.128174in}}%
\pgfpathlineto{\pgfqpoint{3.447622in}{2.122870in}}%
\pgfpathlineto{\pgfqpoint{3.448897in}{2.123832in}}%
\pgfpathlineto{\pgfqpoint{3.450171in}{2.124377in}}%
\pgfpathlineto{\pgfqpoint{3.451020in}{2.121695in}}%
\pgfpathlineto{\pgfqpoint{3.452719in}{2.104947in}}%
\pgfpathlineto{\pgfqpoint{3.456117in}{2.067338in}}%
\pgfpathlineto{\pgfqpoint{3.456967in}{2.070262in}}%
\pgfpathlineto{\pgfqpoint{3.459090in}{2.098675in}}%
\pgfpathlineto{\pgfqpoint{3.462063in}{2.131316in}}%
\pgfpathlineto{\pgfqpoint{3.463762in}{2.133294in}}%
\pgfpathlineto{\pgfqpoint{3.465036in}{2.134415in}}%
\pgfpathlineto{\pgfqpoint{3.466311in}{2.140065in}}%
\pgfpathlineto{\pgfqpoint{3.468434in}{2.163806in}}%
\pgfpathlineto{\pgfqpoint{3.473106in}{2.221369in}}%
\pgfpathlineto{\pgfqpoint{3.474380in}{2.213949in}}%
\pgfpathlineto{\pgfqpoint{3.476929in}{2.194868in}}%
\pgfpathlineto{\pgfqpoint{3.477354in}{2.195697in}}%
\pgfpathlineto{\pgfqpoint{3.478628in}{2.208626in}}%
\pgfpathlineto{\pgfqpoint{3.482875in}{2.269689in}}%
\pgfpathlineto{\pgfqpoint{3.483724in}{2.264039in}}%
\pgfpathlineto{\pgfqpoint{3.485848in}{2.224687in}}%
\pgfpathlineto{\pgfqpoint{3.492644in}{2.092667in}}%
\pgfpathlineto{\pgfqpoint{3.496042in}{2.045954in}}%
\pgfpathlineto{\pgfqpoint{3.496891in}{2.044783in}}%
\pgfpathlineto{\pgfqpoint{3.498165in}{2.050005in}}%
\pgfpathlineto{\pgfqpoint{3.501563in}{2.065699in}}%
\pgfpathlineto{\pgfqpoint{3.503262in}{2.071753in}}%
\pgfpathlineto{\pgfqpoint{3.504961in}{2.090379in}}%
\pgfpathlineto{\pgfqpoint{3.507509in}{2.154260in}}%
\pgfpathlineto{\pgfqpoint{3.512606in}{2.285740in}}%
\pgfpathlineto{\pgfqpoint{3.513880in}{2.290657in}}%
\pgfpathlineto{\pgfqpoint{3.514730in}{2.287080in}}%
\pgfpathlineto{\pgfqpoint{3.516853in}{2.261232in}}%
\pgfpathlineto{\pgfqpoint{3.524074in}{2.151002in}}%
\pgfpathlineto{\pgfqpoint{3.524923in}{2.152942in}}%
\pgfpathlineto{\pgfqpoint{3.528746in}{2.172433in}}%
\pgfpathlineto{\pgfqpoint{3.529171in}{2.171710in}}%
\pgfpathlineto{\pgfqpoint{3.530869in}{2.161085in}}%
\pgfpathlineto{\pgfqpoint{3.535966in}{2.123546in}}%
\pgfpathlineto{\pgfqpoint{3.538515in}{2.118490in}}%
\pgfpathlineto{\pgfqpoint{3.542337in}{2.115133in}}%
\pgfpathlineto{\pgfqpoint{3.544461in}{2.105935in}}%
\pgfpathlineto{\pgfqpoint{3.547434in}{2.092080in}}%
\pgfpathlineto{\pgfqpoint{3.548283in}{2.094115in}}%
\pgfpathlineto{\pgfqpoint{3.549982in}{2.109171in}}%
\pgfpathlineto{\pgfqpoint{3.556353in}{2.202768in}}%
\pgfpathlineto{\pgfqpoint{3.560176in}{2.263849in}}%
\pgfpathlineto{\pgfqpoint{3.560601in}{2.265094in}}%
\pgfpathlineto{\pgfqpoint{3.561025in}{2.264850in}}%
\pgfpathlineto{\pgfqpoint{3.562299in}{2.256540in}}%
\pgfpathlineto{\pgfqpoint{3.573767in}{2.133925in}}%
\pgfpathlineto{\pgfqpoint{3.574617in}{2.135092in}}%
\pgfpathlineto{\pgfqpoint{3.576315in}{2.137751in}}%
\pgfpathlineto{\pgfqpoint{3.577165in}{2.135342in}}%
\pgfpathlineto{\pgfqpoint{3.578864in}{2.118212in}}%
\pgfpathlineto{\pgfqpoint{3.582686in}{2.075747in}}%
\pgfpathlineto{\pgfqpoint{3.583536in}{2.080885in}}%
\pgfpathlineto{\pgfqpoint{3.586084in}{2.121791in}}%
\pgfpathlineto{\pgfqpoint{3.590332in}{2.178134in}}%
\pgfpathlineto{\pgfqpoint{3.593305in}{2.194160in}}%
\pgfpathlineto{\pgfqpoint{3.595004in}{2.194828in}}%
\pgfpathlineto{\pgfqpoint{3.596702in}{2.192298in}}%
\pgfpathlineto{\pgfqpoint{3.598826in}{2.184280in}}%
\pgfpathlineto{\pgfqpoint{3.603073in}{2.165523in}}%
\pgfpathlineto{\pgfqpoint{3.607745in}{2.160784in}}%
\pgfpathlineto{\pgfqpoint{3.609020in}{2.162227in}}%
\pgfpathlineto{\pgfqpoint{3.614966in}{2.178738in}}%
\pgfpathlineto{\pgfqpoint{3.618788in}{2.191029in}}%
\pgfpathlineto{\pgfqpoint{3.619638in}{2.189415in}}%
\pgfpathlineto{\pgfqpoint{3.621337in}{2.175740in}}%
\pgfpathlineto{\pgfqpoint{3.626858in}{2.123773in}}%
\pgfpathlineto{\pgfqpoint{3.628982in}{2.118607in}}%
\pgfpathlineto{\pgfqpoint{3.630681in}{2.119990in}}%
\pgfpathlineto{\pgfqpoint{3.631955in}{2.119902in}}%
\pgfpathlineto{\pgfqpoint{3.633654in}{2.114185in}}%
\pgfpathlineto{\pgfqpoint{3.637052in}{2.101997in}}%
\pgfpathlineto{\pgfqpoint{3.639175in}{2.100662in}}%
\pgfpathlineto{\pgfqpoint{3.640025in}{2.102063in}}%
\pgfpathlineto{\pgfqpoint{3.641299in}{2.109626in}}%
\pgfpathlineto{\pgfqpoint{3.642998in}{2.137202in}}%
\pgfpathlineto{\pgfqpoint{3.650218in}{2.309119in}}%
\pgfpathlineto{\pgfqpoint{3.650643in}{2.308217in}}%
\pgfpathlineto{\pgfqpoint{3.651917in}{2.295958in}}%
\pgfpathlineto{\pgfqpoint{3.659562in}{2.189035in}}%
\pgfpathlineto{\pgfqpoint{3.661261in}{2.159913in}}%
\pgfpathlineto{\pgfqpoint{3.667208in}{2.021572in}}%
\pgfpathlineto{\pgfqpoint{3.667632in}{2.022731in}}%
\pgfpathlineto{\pgfqpoint{3.669331in}{2.039469in}}%
\pgfpathlineto{\pgfqpoint{3.673154in}{2.076973in}}%
\pgfpathlineto{\pgfqpoint{3.675702in}{2.085020in}}%
\pgfpathlineto{\pgfqpoint{3.677401in}{2.109562in}}%
\pgfpathlineto{\pgfqpoint{3.685046in}{2.275582in}}%
\pgfpathlineto{\pgfqpoint{3.685896in}{2.273811in}}%
\pgfpathlineto{\pgfqpoint{3.687595in}{2.261053in}}%
\pgfpathlineto{\pgfqpoint{3.691417in}{2.231500in}}%
\pgfpathlineto{\pgfqpoint{3.693541in}{2.216624in}}%
\pgfpathlineto{\pgfqpoint{3.696939in}{2.168520in}}%
\pgfpathlineto{\pgfqpoint{3.702885in}{2.086979in}}%
\pgfpathlineto{\pgfqpoint{3.704584in}{2.083069in}}%
\pgfpathlineto{\pgfqpoint{3.707132in}{2.084130in}}%
\pgfpathlineto{\pgfqpoint{3.708406in}{2.087311in}}%
\pgfpathlineto{\pgfqpoint{3.710105in}{2.100877in}}%
\pgfpathlineto{\pgfqpoint{3.713503in}{2.158150in}}%
\pgfpathlineto{\pgfqpoint{3.715627in}{2.178588in}}%
\pgfpathlineto{\pgfqpoint{3.716051in}{2.178358in}}%
\pgfpathlineto{\pgfqpoint{3.717326in}{2.169951in}}%
\pgfpathlineto{\pgfqpoint{3.720723in}{2.141684in}}%
\pgfpathlineto{\pgfqpoint{3.721573in}{2.145097in}}%
\pgfpathlineto{\pgfqpoint{3.723272in}{2.167990in}}%
\pgfpathlineto{\pgfqpoint{3.727519in}{2.230157in}}%
\pgfpathlineto{\pgfqpoint{3.728369in}{2.230863in}}%
\pgfpathlineto{\pgfqpoint{3.730068in}{2.224965in}}%
\pgfpathlineto{\pgfqpoint{3.731766in}{2.221046in}}%
\pgfpathlineto{\pgfqpoint{3.732616in}{2.222407in}}%
\pgfpathlineto{\pgfqpoint{3.735589in}{2.232011in}}%
\pgfpathlineto{\pgfqpoint{3.736014in}{2.231225in}}%
\pgfpathlineto{\pgfqpoint{3.737288in}{2.222215in}}%
\pgfpathlineto{\pgfqpoint{3.740686in}{2.167406in}}%
\pgfpathlineto{\pgfqpoint{3.747057in}{2.079960in}}%
\pgfpathlineto{\pgfqpoint{3.752578in}{2.048048in}}%
\pgfpathlineto{\pgfqpoint{3.753428in}{2.047180in}}%
\pgfpathlineto{\pgfqpoint{3.754277in}{2.050338in}}%
\pgfpathlineto{\pgfqpoint{3.755976in}{2.069473in}}%
\pgfpathlineto{\pgfqpoint{3.768293in}{2.261739in}}%
\pgfpathlineto{\pgfqpoint{3.770842in}{2.270650in}}%
\pgfpathlineto{\pgfqpoint{3.771691in}{2.269547in}}%
\pgfpathlineto{\pgfqpoint{3.773390in}{2.259823in}}%
\pgfpathlineto{\pgfqpoint{3.780186in}{2.188603in}}%
\pgfpathlineto{\pgfqpoint{3.783159in}{2.124224in}}%
\pgfpathlineto{\pgfqpoint{3.788255in}{2.017950in}}%
\pgfpathlineto{\pgfqpoint{3.789105in}{2.014746in}}%
\pgfpathlineto{\pgfqpoint{3.789530in}{2.015626in}}%
\pgfpathlineto{\pgfqpoint{3.790804in}{2.027684in}}%
\pgfpathlineto{\pgfqpoint{3.794202in}{2.098411in}}%
\pgfpathlineto{\pgfqpoint{3.803121in}{2.277533in}}%
\pgfpathlineto{\pgfqpoint{3.805245in}{2.287554in}}%
\pgfpathlineto{\pgfqpoint{3.806094in}{2.284153in}}%
\pgfpathlineto{\pgfqpoint{3.807793in}{2.261229in}}%
\pgfpathlineto{\pgfqpoint{3.817986in}{2.062736in}}%
\pgfpathlineto{\pgfqpoint{3.818411in}{2.062959in}}%
\pgfpathlineto{\pgfqpoint{3.821384in}{2.066898in}}%
\pgfpathlineto{\pgfqpoint{3.823508in}{2.064631in}}%
\pgfpathlineto{\pgfqpoint{3.823933in}{2.065374in}}%
\pgfpathlineto{\pgfqpoint{3.825207in}{2.073978in}}%
\pgfpathlineto{\pgfqpoint{3.826906in}{2.105078in}}%
\pgfpathlineto{\pgfqpoint{3.833277in}{2.255511in}}%
\pgfpathlineto{\pgfqpoint{3.834976in}{2.256928in}}%
\pgfpathlineto{\pgfqpoint{3.836250in}{2.258405in}}%
\pgfpathlineto{\pgfqpoint{3.839648in}{2.270498in}}%
\pgfpathlineto{\pgfqpoint{3.840497in}{2.269047in}}%
\pgfpathlineto{\pgfqpoint{3.841771in}{2.259454in}}%
\pgfpathlineto{\pgfqpoint{3.844320in}{2.215166in}}%
\pgfpathlineto{\pgfqpoint{3.852390in}{2.049232in}}%
\pgfpathlineto{\pgfqpoint{3.853664in}{2.044224in}}%
\pgfpathlineto{\pgfqpoint{3.854088in}{2.044299in}}%
\pgfpathlineto{\pgfqpoint{3.855363in}{2.048449in}}%
\pgfpathlineto{\pgfqpoint{3.859610in}{2.077026in}}%
\pgfpathlineto{\pgfqpoint{3.862158in}{2.106882in}}%
\pgfpathlineto{\pgfqpoint{3.865131in}{2.178164in}}%
\pgfpathlineto{\pgfqpoint{3.868529in}{2.244878in}}%
\pgfpathlineto{\pgfqpoint{3.869379in}{2.247696in}}%
\pgfpathlineto{\pgfqpoint{3.869803in}{2.247379in}}%
\pgfpathlineto{\pgfqpoint{3.873201in}{2.238949in}}%
\pgfpathlineto{\pgfqpoint{3.873626in}{2.239321in}}%
\pgfpathlineto{\pgfqpoint{3.875325in}{2.240468in}}%
\pgfpathlineto{\pgfqpoint{3.876599in}{2.236649in}}%
\pgfpathlineto{\pgfqpoint{3.879148in}{2.215974in}}%
\pgfpathlineto{\pgfqpoint{3.884244in}{2.155753in}}%
\pgfpathlineto{\pgfqpoint{3.890615in}{2.076410in}}%
\pgfpathlineto{\pgfqpoint{3.891465in}{2.078791in}}%
\pgfpathlineto{\pgfqpoint{3.893588in}{2.099416in}}%
\pgfpathlineto{\pgfqpoint{3.896137in}{2.118116in}}%
\pgfpathlineto{\pgfqpoint{3.896986in}{2.118244in}}%
\pgfpathlineto{\pgfqpoint{3.899959in}{2.110407in}}%
\pgfpathlineto{\pgfqpoint{3.900384in}{2.111369in}}%
\pgfpathlineto{\pgfqpoint{3.901658in}{2.122206in}}%
\pgfpathlineto{\pgfqpoint{3.904207in}{2.174193in}}%
\pgfpathlineto{\pgfqpoint{3.908029in}{2.239002in}}%
\pgfpathlineto{\pgfqpoint{3.908879in}{2.241993in}}%
\pgfpathlineto{\pgfqpoint{3.909303in}{2.241754in}}%
\pgfpathlineto{\pgfqpoint{3.910577in}{2.235206in}}%
\pgfpathlineto{\pgfqpoint{3.913975in}{2.211336in}}%
\pgfpathlineto{\pgfqpoint{3.914400in}{2.211390in}}%
\pgfpathlineto{\pgfqpoint{3.916948in}{2.218325in}}%
\pgfpathlineto{\pgfqpoint{3.917798in}{2.215817in}}%
\pgfpathlineto{\pgfqpoint{3.919497in}{2.194755in}}%
\pgfpathlineto{\pgfqpoint{3.925018in}{2.103718in}}%
\pgfpathlineto{\pgfqpoint{3.927142in}{2.098829in}}%
\pgfpathlineto{\pgfqpoint{3.928841in}{2.095961in}}%
\pgfpathlineto{\pgfqpoint{3.932239in}{2.085443in}}%
\pgfpathlineto{\pgfqpoint{3.932663in}{2.086061in}}%
\pgfpathlineto{\pgfqpoint{3.933938in}{2.095165in}}%
\pgfpathlineto{\pgfqpoint{3.936486in}{2.144057in}}%
\pgfpathlineto{\pgfqpoint{3.939884in}{2.196188in}}%
\pgfpathlineto{\pgfqpoint{3.944981in}{2.226373in}}%
\pgfpathlineto{\pgfqpoint{3.946255in}{2.227384in}}%
\pgfpathlineto{\pgfqpoint{3.947529in}{2.223655in}}%
\pgfpathlineto{\pgfqpoint{3.949653in}{2.206069in}}%
\pgfpathlineto{\pgfqpoint{3.955599in}{2.144576in}}%
\pgfpathlineto{\pgfqpoint{3.958572in}{2.134624in}}%
\pgfpathlineto{\pgfqpoint{3.962394in}{2.119290in}}%
\pgfpathlineto{\pgfqpoint{3.964943in}{2.117137in}}%
\pgfpathlineto{\pgfqpoint{3.966217in}{2.118073in}}%
\pgfpathlineto{\pgfqpoint{3.967916in}{2.123136in}}%
\pgfpathlineto{\pgfqpoint{3.970889in}{2.140175in}}%
\pgfpathlineto{\pgfqpoint{3.973862in}{2.173083in}}%
\pgfpathlineto{\pgfqpoint{3.976835in}{2.197512in}}%
\pgfpathlineto{\pgfqpoint{3.977685in}{2.197771in}}%
\pgfpathlineto{\pgfqpoint{3.981083in}{2.191627in}}%
\pgfpathlineto{\pgfqpoint{3.981932in}{2.192530in}}%
\pgfpathlineto{\pgfqpoint{3.984056in}{2.199182in}}%
\pgfpathlineto{\pgfqpoint{3.986604in}{2.205302in}}%
\pgfpathlineto{\pgfqpoint{3.987453in}{2.203462in}}%
\pgfpathlineto{\pgfqpoint{3.989152in}{2.192180in}}%
\pgfpathlineto{\pgfqpoint{3.997222in}{2.125000in}}%
\pgfpathlineto{\pgfqpoint{3.998072in}{2.126274in}}%
\pgfpathlineto{\pgfqpoint{4.001894in}{2.140945in}}%
\pgfpathlineto{\pgfqpoint{4.002319in}{2.140296in}}%
\pgfpathlineto{\pgfqpoint{4.004018in}{2.131957in}}%
\pgfpathlineto{\pgfqpoint{4.008265in}{2.109045in}}%
\pgfpathlineto{\pgfqpoint{4.009115in}{2.110815in}}%
\pgfpathlineto{\pgfqpoint{4.010814in}{2.122003in}}%
\pgfpathlineto{\pgfqpoint{4.021857in}{2.215044in}}%
\pgfpathlineto{\pgfqpoint{4.023131in}{2.213606in}}%
\pgfpathlineto{\pgfqpoint{4.029077in}{2.196524in}}%
\pgfpathlineto{\pgfqpoint{4.031625in}{2.185214in}}%
\pgfpathlineto{\pgfqpoint{4.037572in}{2.138084in}}%
\pgfpathlineto{\pgfqpoint{4.040120in}{2.111295in}}%
\pgfpathlineto{\pgfqpoint{4.043942in}{2.068110in}}%
\pgfpathlineto{\pgfqpoint{4.044367in}{2.068165in}}%
\pgfpathlineto{\pgfqpoint{4.045217in}{2.072525in}}%
\pgfpathlineto{\pgfqpoint{4.047340in}{2.103972in}}%
\pgfpathlineto{\pgfqpoint{4.053287in}{2.203546in}}%
\pgfpathlineto{\pgfqpoint{4.057109in}{2.222828in}}%
\pgfpathlineto{\pgfqpoint{4.060082in}{2.229332in}}%
\pgfpathlineto{\pgfqpoint{4.061781in}{2.229626in}}%
\pgfpathlineto{\pgfqpoint{4.063055in}{2.226631in}}%
\pgfpathlineto{\pgfqpoint{4.064754in}{2.212789in}}%
\pgfpathlineto{\pgfqpoint{4.067727in}{2.159299in}}%
\pgfpathlineto{\pgfqpoint{4.071975in}{2.095080in}}%
\pgfpathlineto{\pgfqpoint{4.075372in}{2.070510in}}%
\pgfpathlineto{\pgfqpoint{4.076222in}{2.069574in}}%
\pgfpathlineto{\pgfqpoint{4.076647in}{2.070177in}}%
\pgfpathlineto{\pgfqpoint{4.077921in}{2.076645in}}%
\pgfpathlineto{\pgfqpoint{4.080469in}{2.107377in}}%
\pgfpathlineto{\pgfqpoint{4.091937in}{2.260830in}}%
\pgfpathlineto{\pgfqpoint{4.094485in}{2.272347in}}%
\pgfpathlineto{\pgfqpoint{4.095335in}{2.271304in}}%
\pgfpathlineto{\pgfqpoint{4.096609in}{2.262148in}}%
\pgfpathlineto{\pgfqpoint{4.098733in}{2.221593in}}%
\pgfpathlineto{\pgfqpoint{4.104254in}{2.097507in}}%
\pgfpathlineto{\pgfqpoint{4.107652in}{2.076133in}}%
\pgfpathlineto{\pgfqpoint{4.111050in}{2.057556in}}%
\pgfpathlineto{\pgfqpoint{4.111899in}{2.060611in}}%
\pgfpathlineto{\pgfqpoint{4.113598in}{2.083601in}}%
\pgfpathlineto{\pgfqpoint{4.119969in}{2.192230in}}%
\pgfpathlineto{\pgfqpoint{4.127614in}{2.225410in}}%
\pgfpathlineto{\pgfqpoint{4.128464in}{2.224532in}}%
\pgfpathlineto{\pgfqpoint{4.129738in}{2.218389in}}%
\pgfpathlineto{\pgfqpoint{4.131861in}{2.194880in}}%
\pgfpathlineto{\pgfqpoint{4.136533in}{2.136941in}}%
\pgfpathlineto{\pgfqpoint{4.136958in}{2.136847in}}%
\pgfpathlineto{\pgfqpoint{4.138232in}{2.142651in}}%
\pgfpathlineto{\pgfqpoint{4.141206in}{2.158310in}}%
\pgfpathlineto{\pgfqpoint{4.142055in}{2.155949in}}%
\pgfpathlineto{\pgfqpoint{4.144179in}{2.138002in}}%
\pgfpathlineto{\pgfqpoint{4.147152in}{2.118450in}}%
\pgfpathlineto{\pgfqpoint{4.148001in}{2.120469in}}%
\pgfpathlineto{\pgfqpoint{4.149700in}{2.135032in}}%
\pgfpathlineto{\pgfqpoint{4.153098in}{2.163020in}}%
\pgfpathlineto{\pgfqpoint{4.154372in}{2.163531in}}%
\pgfpathlineto{\pgfqpoint{4.155646in}{2.163549in}}%
\pgfpathlineto{\pgfqpoint{4.156921in}{2.166460in}}%
\pgfpathlineto{\pgfqpoint{4.162442in}{2.186795in}}%
\pgfpathlineto{\pgfqpoint{4.162867in}{2.186533in}}%
\pgfpathlineto{\pgfqpoint{4.164141in}{2.182874in}}%
\pgfpathlineto{\pgfqpoint{4.173485in}{2.139787in}}%
\pgfpathlineto{\pgfqpoint{4.174759in}{2.141275in}}%
\pgfpathlineto{\pgfqpoint{4.177308in}{2.149396in}}%
\pgfpathlineto{\pgfqpoint{4.185802in}{2.183024in}}%
\pgfpathlineto{\pgfqpoint{4.190899in}{2.195300in}}%
\pgfpathlineto{\pgfqpoint{4.191324in}{2.194518in}}%
\pgfpathlineto{\pgfqpoint{4.192598in}{2.186681in}}%
\pgfpathlineto{\pgfqpoint{4.198544in}{2.137132in}}%
\pgfpathlineto{\pgfqpoint{4.201942in}{2.114632in}}%
\pgfpathlineto{\pgfqpoint{4.204490in}{2.104242in}}%
\pgfpathlineto{\pgfqpoint{4.205764in}{2.104041in}}%
\pgfpathlineto{\pgfqpoint{4.207463in}{2.107931in}}%
\pgfpathlineto{\pgfqpoint{4.209587in}{2.120180in}}%
\pgfpathlineto{\pgfqpoint{4.212560in}{2.153456in}}%
\pgfpathlineto{\pgfqpoint{4.219780in}{2.241913in}}%
\pgfpathlineto{\pgfqpoint{4.221055in}{2.243540in}}%
\pgfpathlineto{\pgfqpoint{4.222754in}{2.240596in}}%
\pgfpathlineto{\pgfqpoint{4.225302in}{2.232684in}}%
\pgfpathlineto{\pgfqpoint{4.227426in}{2.215452in}}%
\pgfpathlineto{\pgfqpoint{4.230399in}{2.167508in}}%
\pgfpathlineto{\pgfqpoint{4.236345in}{2.069252in}}%
\pgfpathlineto{\pgfqpoint{4.236770in}{2.068575in}}%
\pgfpathlineto{\pgfqpoint{4.237194in}{2.068888in}}%
\pgfpathlineto{\pgfqpoint{4.238893in}{2.075960in}}%
\pgfpathlineto{\pgfqpoint{4.241442in}{2.084158in}}%
\pgfpathlineto{\pgfqpoint{4.243141in}{2.086948in}}%
\pgfpathlineto{\pgfqpoint{4.244839in}{2.098162in}}%
\pgfpathlineto{\pgfqpoint{4.247388in}{2.138341in}}%
\pgfpathlineto{\pgfqpoint{4.252485in}{2.222776in}}%
\pgfpathlineto{\pgfqpoint{4.254608in}{2.229807in}}%
\pgfpathlineto{\pgfqpoint{4.257157in}{2.229962in}}%
\pgfpathlineto{\pgfqpoint{4.258856in}{2.228281in}}%
\pgfpathlineto{\pgfqpoint{4.260554in}{2.222164in}}%
\pgfpathlineto{\pgfqpoint{4.263103in}{2.200827in}}%
\pgfpathlineto{\pgfqpoint{4.267350in}{2.165460in}}%
\pgfpathlineto{\pgfqpoint{4.273296in}{2.131520in}}%
\pgfpathlineto{\pgfqpoint{4.279243in}{2.092795in}}%
\pgfpathlineto{\pgfqpoint{4.280092in}{2.094239in}}%
\pgfpathlineto{\pgfqpoint{4.281791in}{2.103212in}}%
\pgfpathlineto{\pgfqpoint{4.286038in}{2.143466in}}%
\pgfpathlineto{\pgfqpoint{4.291560in}{2.189802in}}%
\pgfpathlineto{\pgfqpoint{4.297506in}{2.222440in}}%
\pgfpathlineto{\pgfqpoint{4.297931in}{2.222285in}}%
\pgfpathlineto{\pgfqpoint{4.299630in}{2.217849in}}%
\pgfpathlineto{\pgfqpoint{4.303027in}{2.201279in}}%
\pgfpathlineto{\pgfqpoint{4.305576in}{2.174943in}}%
\pgfpathlineto{\pgfqpoint{4.312796in}{2.091218in}}%
\pgfpathlineto{\pgfqpoint{4.314070in}{2.089703in}}%
\pgfpathlineto{\pgfqpoint{4.315345in}{2.093777in}}%
\pgfpathlineto{\pgfqpoint{4.318318in}{2.116690in}}%
\pgfpathlineto{\pgfqpoint{4.327237in}{2.200485in}}%
\pgfpathlineto{\pgfqpoint{4.330210in}{2.219034in}}%
\pgfpathlineto{\pgfqpoint{4.331060in}{2.220019in}}%
\pgfpathlineto{\pgfqpoint{4.331484in}{2.219751in}}%
\pgfpathlineto{\pgfqpoint{4.333183in}{2.214953in}}%
\pgfpathlineto{\pgfqpoint{4.336156in}{2.206981in}}%
\pgfpathlineto{\pgfqpoint{4.337855in}{2.203917in}}%
\pgfpathlineto{\pgfqpoint{4.339554in}{2.193558in}}%
\pgfpathlineto{\pgfqpoint{4.342952in}{2.149241in}}%
\pgfpathlineto{\pgfqpoint{4.347199in}{2.104246in}}%
\pgfpathlineto{\pgfqpoint{4.349748in}{2.095965in}}%
\pgfpathlineto{\pgfqpoint{4.350597in}{2.095953in}}%
\pgfpathlineto{\pgfqpoint{4.351871in}{2.099888in}}%
\pgfpathlineto{\pgfqpoint{4.353995in}{2.119077in}}%
\pgfpathlineto{\pgfqpoint{4.360366in}{2.186360in}}%
\pgfpathlineto{\pgfqpoint{4.362914in}{2.194808in}}%
\pgfpathlineto{\pgfqpoint{4.364188in}{2.194279in}}%
\pgfpathlineto{\pgfqpoint{4.366737in}{2.187781in}}%
\pgfpathlineto{\pgfqpoint{4.371834in}{2.175033in}}%
\pgfpathlineto{\pgfqpoint{4.373957in}{2.172655in}}%
\pgfpathlineto{\pgfqpoint{4.375656in}{2.164243in}}%
\pgfpathlineto{\pgfqpoint{4.380328in}{2.132088in}}%
\pgfpathlineto{\pgfqpoint{4.380753in}{2.132415in}}%
\pgfpathlineto{\pgfqpoint{4.382027in}{2.137549in}}%
\pgfpathlineto{\pgfqpoint{4.387549in}{2.167828in}}%
\pgfpathlineto{\pgfqpoint{4.389247in}{2.166681in}}%
\pgfpathlineto{\pgfqpoint{4.400290in}{2.151739in}}%
\pgfpathlineto{\pgfqpoint{4.401989in}{2.150821in}}%
\pgfpathlineto{\pgfqpoint{4.404113in}{2.153473in}}%
\pgfpathlineto{\pgfqpoint{4.408360in}{2.159401in}}%
\pgfpathlineto{\pgfqpoint{4.409634in}{2.158047in}}%
\pgfpathlineto{\pgfqpoint{4.412183in}{2.153667in}}%
\pgfpathlineto{\pgfqpoint{4.412608in}{2.154094in}}%
\pgfpathlineto{\pgfqpoint{4.413882in}{2.159273in}}%
\pgfpathlineto{\pgfqpoint{4.418129in}{2.185179in}}%
\pgfpathlineto{\pgfqpoint{4.418554in}{2.184972in}}%
\pgfpathlineto{\pgfqpoint{4.420253in}{2.180174in}}%
\pgfpathlineto{\pgfqpoint{4.423226in}{2.172970in}}%
\pgfpathlineto{\pgfqpoint{4.427473in}{2.166377in}}%
\pgfpathlineto{\pgfqpoint{4.430021in}{2.155515in}}%
\pgfpathlineto{\pgfqpoint{4.433419in}{2.140976in}}%
\pgfpathlineto{\pgfqpoint{4.434693in}{2.141220in}}%
\pgfpathlineto{\pgfqpoint{4.438516in}{2.148363in}}%
\pgfpathlineto{\pgfqpoint{4.441064in}{2.158230in}}%
\pgfpathlineto{\pgfqpoint{4.444462in}{2.169444in}}%
\pgfpathlineto{\pgfqpoint{4.445736in}{2.169333in}}%
\pgfpathlineto{\pgfqpoint{4.447860in}{2.166021in}}%
\pgfpathlineto{\pgfqpoint{4.451258in}{2.159597in}}%
\pgfpathlineto{\pgfqpoint{4.451683in}{2.159806in}}%
\pgfpathlineto{\pgfqpoint{4.452957in}{2.163309in}}%
\pgfpathlineto{\pgfqpoint{4.457204in}{2.182474in}}%
\pgfpathlineto{\pgfqpoint{4.457629in}{2.181552in}}%
\pgfpathlineto{\pgfqpoint{4.459328in}{2.169852in}}%
\pgfpathlineto{\pgfqpoint{4.464000in}{2.128137in}}%
\pgfpathlineto{\pgfqpoint{4.464849in}{2.129033in}}%
\pgfpathlineto{\pgfqpoint{4.466548in}{2.139588in}}%
\pgfpathlineto{\pgfqpoint{4.469521in}{2.156058in}}%
\pgfpathlineto{\pgfqpoint{4.470371in}{2.154888in}}%
\pgfpathlineto{\pgfqpoint{4.473344in}{2.144013in}}%
\pgfpathlineto{\pgfqpoint{4.473769in}{2.144393in}}%
\pgfpathlineto{\pgfqpoint{4.475043in}{2.151244in}}%
\pgfpathlineto{\pgfqpoint{4.480564in}{2.200377in}}%
\pgfpathlineto{\pgfqpoint{4.480989in}{2.199767in}}%
\pgfpathlineto{\pgfqpoint{4.482688in}{2.190374in}}%
\pgfpathlineto{\pgfqpoint{4.486510in}{2.165563in}}%
\pgfpathlineto{\pgfqpoint{4.487785in}{2.165569in}}%
\pgfpathlineto{\pgfqpoint{4.489484in}{2.167211in}}%
\pgfpathlineto{\pgfqpoint{4.489908in}{2.166826in}}%
\pgfpathlineto{\pgfqpoint{4.491182in}{2.162070in}}%
\pgfpathlineto{\pgfqpoint{4.497129in}{2.122443in}}%
\pgfpathlineto{\pgfqpoint{4.497978in}{2.123599in}}%
\pgfpathlineto{\pgfqpoint{4.500102in}{2.132306in}}%
\pgfpathlineto{\pgfqpoint{4.504774in}{2.160492in}}%
\pgfpathlineto{\pgfqpoint{4.508172in}{2.208304in}}%
\pgfpathlineto{\pgfqpoint{4.510295in}{2.226113in}}%
\pgfpathlineto{\pgfqpoint{4.511145in}{2.223547in}}%
\pgfpathlineto{\pgfqpoint{4.512844in}{2.203386in}}%
\pgfpathlineto{\pgfqpoint{4.517091in}{2.151416in}}%
\pgfpathlineto{\pgfqpoint{4.518790in}{2.149129in}}%
\pgfpathlineto{\pgfqpoint{4.520064in}{2.148760in}}%
\pgfpathlineto{\pgfqpoint{4.521338in}{2.145378in}}%
\pgfpathlineto{\pgfqpoint{4.523462in}{2.130894in}}%
\pgfpathlineto{\pgfqpoint{4.526860in}{2.109408in}}%
\pgfpathlineto{\pgfqpoint{4.527709in}{2.109592in}}%
\pgfpathlineto{\pgfqpoint{4.528983in}{2.115735in}}%
\pgfpathlineto{\pgfqpoint{4.531532in}{2.146567in}}%
\pgfpathlineto{\pgfqpoint{4.534930in}{2.180500in}}%
\pgfpathlineto{\pgfqpoint{4.537053in}{2.184170in}}%
\pgfpathlineto{\pgfqpoint{4.540026in}{2.187424in}}%
\pgfpathlineto{\pgfqpoint{4.542150in}{2.193073in}}%
\pgfpathlineto{\pgfqpoint{4.545123in}{2.205422in}}%
\pgfpathlineto{\pgfqpoint{4.545548in}{2.205250in}}%
\pgfpathlineto{\pgfqpoint{4.546822in}{2.198287in}}%
\pgfpathlineto{\pgfqpoint{4.553618in}{2.138052in}}%
\pgfpathlineto{\pgfqpoint{4.555317in}{2.137573in}}%
\pgfpathlineto{\pgfqpoint{4.557016in}{2.133088in}}%
\pgfpathlineto{\pgfqpoint{4.562112in}{2.112508in}}%
\pgfpathlineto{\pgfqpoint{4.562537in}{2.113146in}}%
\pgfpathlineto{\pgfqpoint{4.563811in}{2.119805in}}%
\pgfpathlineto{\pgfqpoint{4.566784in}{2.155802in}}%
\pgfpathlineto{\pgfqpoint{4.570607in}{2.191897in}}%
\pgfpathlineto{\pgfqpoint{4.573155in}{2.197991in}}%
\pgfpathlineto{\pgfqpoint{4.575704in}{2.200952in}}%
\pgfpathlineto{\pgfqpoint{4.579101in}{2.210443in}}%
\pgfpathlineto{\pgfqpoint{4.579951in}{2.209127in}}%
\pgfpathlineto{\pgfqpoint{4.581225in}{2.200301in}}%
\pgfpathlineto{\pgfqpoint{4.589720in}{2.116068in}}%
\pgfpathlineto{\pgfqpoint{4.592268in}{2.111405in}}%
\pgfpathlineto{\pgfqpoint{4.593542in}{2.112008in}}%
\pgfpathlineto{\pgfqpoint{4.594816in}{2.115940in}}%
\pgfpathlineto{\pgfqpoint{4.596940in}{2.133092in}}%
\pgfpathlineto{\pgfqpoint{4.602037in}{2.180854in}}%
\pgfpathlineto{\pgfqpoint{4.605010in}{2.188092in}}%
\pgfpathlineto{\pgfqpoint{4.608833in}{2.191812in}}%
\pgfpathlineto{\pgfqpoint{4.610107in}{2.190835in}}%
\pgfpathlineto{\pgfqpoint{4.611806in}{2.185691in}}%
\pgfpathlineto{\pgfqpoint{4.619451in}{2.155398in}}%
\pgfpathlineto{\pgfqpoint{4.621150in}{2.150369in}}%
\pgfpathlineto{\pgfqpoint{4.623698in}{2.132457in}}%
\pgfpathlineto{\pgfqpoint{4.626246in}{2.120030in}}%
\pgfpathlineto{\pgfqpoint{4.627096in}{2.121289in}}%
\pgfpathlineto{\pgfqpoint{4.629644in}{2.134728in}}%
\pgfpathlineto{\pgfqpoint{4.638139in}{2.190397in}}%
\pgfpathlineto{\pgfqpoint{4.641112in}{2.204044in}}%
\pgfpathlineto{\pgfqpoint{4.641961in}{2.203489in}}%
\pgfpathlineto{\pgfqpoint{4.643660in}{2.196589in}}%
\pgfpathlineto{\pgfqpoint{4.648757in}{2.172247in}}%
\pgfpathlineto{\pgfqpoint{4.653429in}{2.162440in}}%
\pgfpathlineto{\pgfqpoint{4.659800in}{2.136541in}}%
\pgfpathlineto{\pgfqpoint{4.661499in}{2.131666in}}%
\pgfpathlineto{\pgfqpoint{4.666596in}{2.107945in}}%
\pgfpathlineto{\pgfqpoint{4.667020in}{2.108207in}}%
\pgfpathlineto{\pgfqpoint{4.668295in}{2.111442in}}%
\pgfpathlineto{\pgfqpoint{4.670843in}{2.126697in}}%
\pgfpathlineto{\pgfqpoint{4.675515in}{2.173686in}}%
\pgfpathlineto{\pgfqpoint{4.680187in}{2.213746in}}%
\pgfpathlineto{\pgfqpoint{4.683585in}{2.225220in}}%
\pgfpathlineto{\pgfqpoint{4.687407in}{2.233605in}}%
\pgfpathlineto{\pgfqpoint{4.687832in}{2.233113in}}%
\pgfpathlineto{\pgfqpoint{4.689106in}{2.226431in}}%
\pgfpathlineto{\pgfqpoint{4.691230in}{2.196832in}}%
\pgfpathlineto{\pgfqpoint{4.697176in}{2.104947in}}%
\pgfpathlineto{\pgfqpoint{4.700149in}{2.091109in}}%
\pgfpathlineto{\pgfqpoint{4.703122in}{2.084862in}}%
\pgfpathlineto{\pgfqpoint{4.703972in}{2.085301in}}%
\pgfpathlineto{\pgfqpoint{4.705246in}{2.089440in}}%
\pgfpathlineto{\pgfqpoint{4.707370in}{2.107138in}}%
\pgfpathlineto{\pgfqpoint{4.713741in}{2.169017in}}%
\pgfpathlineto{\pgfqpoint{4.717139in}{2.198106in}}%
\pgfpathlineto{\pgfqpoint{4.721811in}{2.237576in}}%
\pgfpathlineto{\pgfqpoint{4.722660in}{2.237259in}}%
\pgfpathlineto{\pgfqpoint{4.723934in}{2.230963in}}%
\pgfpathlineto{\pgfqpoint{4.729456in}{2.195213in}}%
\pgfpathlineto{\pgfqpoint{4.731155in}{2.187793in}}%
\pgfpathlineto{\pgfqpoint{4.733278in}{2.161624in}}%
\pgfpathlineto{\pgfqpoint{4.738375in}{2.086357in}}%
\pgfpathlineto{\pgfqpoint{4.739224in}{2.085113in}}%
\pgfpathlineto{\pgfqpoint{4.739649in}{2.085926in}}%
\pgfpathlineto{\pgfqpoint{4.741348in}{2.096835in}}%
\pgfpathlineto{\pgfqpoint{4.746870in}{2.140408in}}%
\pgfpathlineto{\pgfqpoint{4.749843in}{2.155392in}}%
\pgfpathlineto{\pgfqpoint{4.757488in}{2.209046in}}%
\pgfpathlineto{\pgfqpoint{4.758337in}{2.208145in}}%
\pgfpathlineto{\pgfqpoint{4.759611in}{2.202448in}}%
\pgfpathlineto{\pgfqpoint{4.762160in}{2.175584in}}%
\pgfpathlineto{\pgfqpoint{4.766407in}{2.133663in}}%
\pgfpathlineto{\pgfqpoint{4.768106in}{2.130879in}}%
\pgfpathlineto{\pgfqpoint{4.769380in}{2.133679in}}%
\pgfpathlineto{\pgfqpoint{4.776601in}{2.157145in}}%
\pgfpathlineto{\pgfqpoint{4.779998in}{2.164542in}}%
\pgfpathlineto{\pgfqpoint{4.782547in}{2.180111in}}%
\pgfpathlineto{\pgfqpoint{4.785520in}{2.194595in}}%
\pgfpathlineto{\pgfqpoint{4.786369in}{2.192900in}}%
\pgfpathlineto{\pgfqpoint{4.791466in}{2.170198in}}%
\pgfpathlineto{\pgfqpoint{4.792316in}{2.170878in}}%
\pgfpathlineto{\pgfqpoint{4.793590in}{2.171053in}}%
\pgfpathlineto{\pgfqpoint{4.794864in}{2.166539in}}%
\pgfpathlineto{\pgfqpoint{4.797837in}{2.139216in}}%
\pgfpathlineto{\pgfqpoint{4.800385in}{2.126052in}}%
\pgfpathlineto{\pgfqpoint{4.801660in}{2.126041in}}%
\pgfpathlineto{\pgfqpoint{4.805057in}{2.131114in}}%
\pgfpathlineto{\pgfqpoint{4.807606in}{2.137011in}}%
\pgfpathlineto{\pgfqpoint{4.809730in}{2.148449in}}%
\pgfpathlineto{\pgfqpoint{4.816950in}{2.196904in}}%
\pgfpathlineto{\pgfqpoint{4.818224in}{2.195698in}}%
\pgfpathlineto{\pgfqpoint{4.820348in}{2.187507in}}%
\pgfpathlineto{\pgfqpoint{4.823321in}{2.178672in}}%
\pgfpathlineto{\pgfqpoint{4.824595in}{2.180342in}}%
\pgfpathlineto{\pgfqpoint{4.826294in}{2.182902in}}%
\pgfpathlineto{\pgfqpoint{4.826719in}{2.182555in}}%
\pgfpathlineto{\pgfqpoint{4.827993in}{2.177025in}}%
\pgfpathlineto{\pgfqpoint{4.830117in}{2.152419in}}%
\pgfpathlineto{\pgfqpoint{4.833514in}{2.113703in}}%
\pgfpathlineto{\pgfqpoint{4.834364in}{2.112578in}}%
\pgfpathlineto{\pgfqpoint{4.835638in}{2.118136in}}%
\pgfpathlineto{\pgfqpoint{4.840735in}{2.149559in}}%
\pgfpathlineto{\pgfqpoint{4.842434in}{2.147958in}}%
\pgfpathlineto{\pgfqpoint{4.844133in}{2.147833in}}%
\pgfpathlineto{\pgfqpoint{4.845832in}{2.152856in}}%
\pgfpathlineto{\pgfqpoint{4.852202in}{2.179468in}}%
\pgfpathlineto{\pgfqpoint{4.855600in}{2.179306in}}%
\pgfpathlineto{\pgfqpoint{4.858573in}{2.185188in}}%
\pgfpathlineto{\pgfqpoint{4.861971in}{2.190147in}}%
\pgfpathlineto{\pgfqpoint{4.862821in}{2.189246in}}%
\pgfpathlineto{\pgfqpoint{4.864095in}{2.184122in}}%
\pgfpathlineto{\pgfqpoint{4.866643in}{2.159893in}}%
\pgfpathlineto{\pgfqpoint{4.869616in}{2.139026in}}%
\pgfpathlineto{\pgfqpoint{4.870466in}{2.139995in}}%
\pgfpathlineto{\pgfqpoint{4.872589in}{2.151348in}}%
\pgfpathlineto{\pgfqpoint{4.874713in}{2.158752in}}%
\pgfpathlineto{\pgfqpoint{4.875563in}{2.157143in}}%
\pgfpathlineto{\pgfqpoint{4.877686in}{2.143064in}}%
\pgfpathlineto{\pgfqpoint{4.880235in}{2.129742in}}%
\pgfpathlineto{\pgfqpoint{4.881084in}{2.130893in}}%
\pgfpathlineto{\pgfqpoint{4.883208in}{2.143291in}}%
\pgfpathlineto{\pgfqpoint{4.886181in}{2.157487in}}%
\pgfpathlineto{\pgfqpoint{4.887030in}{2.157014in}}%
\pgfpathlineto{\pgfqpoint{4.889154in}{2.148940in}}%
\pgfpathlineto{\pgfqpoint{4.890853in}{2.145085in}}%
\pgfpathlineto{\pgfqpoint{4.891702in}{2.147068in}}%
\pgfpathlineto{\pgfqpoint{4.893826in}{2.162450in}}%
\pgfpathlineto{\pgfqpoint{4.897224in}{2.183666in}}%
\pgfpathlineto{\pgfqpoint{4.898923in}{2.184532in}}%
\pgfpathlineto{\pgfqpoint{4.901471in}{2.184379in}}%
\pgfpathlineto{\pgfqpoint{4.905294in}{2.188635in}}%
\pgfpathlineto{\pgfqpoint{4.905718in}{2.188230in}}%
\pgfpathlineto{\pgfqpoint{4.906993in}{2.184787in}}%
\pgfpathlineto{\pgfqpoint{4.909541in}{2.168511in}}%
\pgfpathlineto{\pgfqpoint{4.913788in}{2.143544in}}%
\pgfpathlineto{\pgfqpoint{4.916761in}{2.139308in}}%
\pgfpathlineto{\pgfqpoint{4.918885in}{2.139234in}}%
\pgfpathlineto{\pgfqpoint{4.921858in}{2.140879in}}%
\pgfpathlineto{\pgfqpoint{4.924406in}{2.138600in}}%
\pgfpathlineto{\pgfqpoint{4.924831in}{2.139378in}}%
\pgfpathlineto{\pgfqpoint{4.926105in}{2.146320in}}%
\pgfpathlineto{\pgfqpoint{4.931627in}{2.194255in}}%
\pgfpathlineto{\pgfqpoint{4.932052in}{2.193793in}}%
\pgfpathlineto{\pgfqpoint{4.933326in}{2.188650in}}%
\pgfpathlineto{\pgfqpoint{4.938422in}{2.161237in}}%
\pgfpathlineto{\pgfqpoint{4.940121in}{2.164098in}}%
\pgfpathlineto{\pgfqpoint{4.941396in}{2.165570in}}%
\pgfpathlineto{\pgfqpoint{4.941820in}{2.165303in}}%
\pgfpathlineto{\pgfqpoint{4.943095in}{2.161749in}}%
\pgfpathlineto{\pgfqpoint{4.946068in}{2.142690in}}%
\pgfpathlineto{\pgfqpoint{4.949041in}{2.129238in}}%
\pgfpathlineto{\pgfqpoint{4.949890in}{2.130201in}}%
\pgfpathlineto{\pgfqpoint{4.951589in}{2.139107in}}%
\pgfpathlineto{\pgfqpoint{4.955412in}{2.160885in}}%
\pgfpathlineto{\pgfqpoint{4.956686in}{2.161039in}}%
\pgfpathlineto{\pgfqpoint{4.959234in}{2.159746in}}%
\pgfpathlineto{\pgfqpoint{4.960508in}{2.163821in}}%
\pgfpathlineto{\pgfqpoint{4.963057in}{2.184905in}}%
\pgfpathlineto{\pgfqpoint{4.965605in}{2.200405in}}%
\pgfpathlineto{\pgfqpoint{4.966455in}{2.197999in}}%
\pgfpathlineto{\pgfqpoint{4.968578in}{2.178151in}}%
\pgfpathlineto{\pgfqpoint{4.971127in}{2.160872in}}%
\pgfpathlineto{\pgfqpoint{4.971976in}{2.161044in}}%
\pgfpathlineto{\pgfqpoint{4.974949in}{2.166781in}}%
\pgfpathlineto{\pgfqpoint{4.975374in}{2.166308in}}%
\pgfpathlineto{\pgfqpoint{4.976648in}{2.161447in}}%
\pgfpathlineto{\pgfqpoint{4.981320in}{2.133257in}}%
\pgfpathlineto{\pgfqpoint{4.981745in}{2.133675in}}%
\pgfpathlineto{\pgfqpoint{4.983869in}{2.141669in}}%
\pgfpathlineto{\pgfqpoint{4.985992in}{2.145876in}}%
\pgfpathlineto{\pgfqpoint{4.989390in}{2.143272in}}%
\pgfpathlineto{\pgfqpoint{4.989815in}{2.143980in}}%
\pgfpathlineto{\pgfqpoint{4.991514in}{2.150586in}}%
\pgfpathlineto{\pgfqpoint{4.995761in}{2.180964in}}%
\pgfpathlineto{\pgfqpoint{4.998734in}{2.195043in}}%
\pgfpathlineto{\pgfqpoint{4.999584in}{2.195396in}}%
\pgfpathlineto{\pgfqpoint{5.000008in}{2.194936in}}%
\pgfpathlineto{\pgfqpoint{5.001707in}{2.189480in}}%
\pgfpathlineto{\pgfqpoint{5.007229in}{2.166826in}}%
\pgfpathlineto{\pgfqpoint{5.009352in}{2.162588in}}%
\pgfpathlineto{\pgfqpoint{5.011476in}{2.150309in}}%
\pgfpathlineto{\pgfqpoint{5.015299in}{2.128487in}}%
\pgfpathlineto{\pgfqpoint{5.016997in}{2.127468in}}%
\pgfpathlineto{\pgfqpoint{5.020820in}{2.130862in}}%
\pgfpathlineto{\pgfqpoint{5.022519in}{2.135292in}}%
\pgfpathlineto{\pgfqpoint{5.025492in}{2.152025in}}%
\pgfpathlineto{\pgfqpoint{5.034411in}{2.203411in}}%
\pgfpathlineto{\pgfqpoint{5.035686in}{2.200366in}}%
\pgfpathlineto{\pgfqpoint{5.050126in}{2.140652in}}%
\pgfpathlineto{\pgfqpoint{5.051401in}{2.141216in}}%
\pgfpathlineto{\pgfqpoint{5.053524in}{2.142178in}}%
\pgfpathlineto{\pgfqpoint{5.054798in}{2.137950in}}%
\pgfpathlineto{\pgfqpoint{5.059046in}{2.111529in}}%
\pgfpathlineto{\pgfqpoint{5.059895in}{2.113883in}}%
\pgfpathlineto{\pgfqpoint{5.061594in}{2.130520in}}%
\pgfpathlineto{\pgfqpoint{5.066266in}{2.183446in}}%
\pgfpathlineto{\pgfqpoint{5.068390in}{2.188363in}}%
\pgfpathlineto{\pgfqpoint{5.071788in}{2.191422in}}%
\pgfpathlineto{\pgfqpoint{5.076035in}{2.200793in}}%
\pgfpathlineto{\pgfqpoint{5.076460in}{2.200416in}}%
\pgfpathlineto{\pgfqpoint{5.078158in}{2.194787in}}%
\pgfpathlineto{\pgfqpoint{5.084529in}{2.156573in}}%
\pgfpathlineto{\pgfqpoint{5.088352in}{2.137029in}}%
\pgfpathlineto{\pgfqpoint{5.095572in}{2.118446in}}%
\pgfpathlineto{\pgfqpoint{5.096847in}{2.118677in}}%
\pgfpathlineto{\pgfqpoint{5.098545in}{2.123084in}}%
\pgfpathlineto{\pgfqpoint{5.101519in}{2.140993in}}%
\pgfpathlineto{\pgfqpoint{5.107465in}{2.191989in}}%
\pgfpathlineto{\pgfqpoint{5.110438in}{2.209432in}}%
\pgfpathlineto{\pgfqpoint{5.112562in}{2.212805in}}%
\pgfpathlineto{\pgfqpoint{5.113836in}{2.211508in}}%
\pgfpathlineto{\pgfqpoint{5.115535in}{2.204377in}}%
\pgfpathlineto{\pgfqpoint{5.118508in}{2.177728in}}%
\pgfpathlineto{\pgfqpoint{5.123180in}{2.140163in}}%
\pgfpathlineto{\pgfqpoint{5.126153in}{2.132063in}}%
\pgfpathlineto{\pgfqpoint{5.127852in}{2.132033in}}%
\pgfpathlineto{\pgfqpoint{5.130825in}{2.133058in}}%
\pgfpathlineto{\pgfqpoint{5.134223in}{2.131649in}}%
\pgfpathlineto{\pgfqpoint{5.135922in}{2.135120in}}%
\pgfpathlineto{\pgfqpoint{5.138470in}{2.147103in}}%
\pgfpathlineto{\pgfqpoint{5.145690in}{2.185709in}}%
\pgfpathlineto{\pgfqpoint{5.149513in}{2.189982in}}%
\pgfpathlineto{\pgfqpoint{5.150787in}{2.188397in}}%
\pgfpathlineto{\pgfqpoint{5.156733in}{2.175658in}}%
\pgfpathlineto{\pgfqpoint{5.158008in}{2.174961in}}%
\pgfpathlineto{\pgfqpoint{5.159282in}{2.171215in}}%
\pgfpathlineto{\pgfqpoint{5.161830in}{2.153175in}}%
\pgfpathlineto{\pgfqpoint{5.165653in}{2.128700in}}%
\pgfpathlineto{\pgfqpoint{5.166502in}{2.128401in}}%
\pgfpathlineto{\pgfqpoint{5.167776in}{2.131749in}}%
\pgfpathlineto{\pgfqpoint{5.172448in}{2.147630in}}%
\pgfpathlineto{\pgfqpoint{5.176696in}{2.149880in}}%
\pgfpathlineto{\pgfqpoint{5.178395in}{2.156484in}}%
\pgfpathlineto{\pgfqpoint{5.183491in}{2.184892in}}%
\pgfpathlineto{\pgfqpoint{5.183916in}{2.184552in}}%
\pgfpathlineto{\pgfqpoint{5.185615in}{2.178678in}}%
\pgfpathlineto{\pgfqpoint{5.189862in}{2.162705in}}%
\pgfpathlineto{\pgfqpoint{5.191561in}{2.161525in}}%
\pgfpathlineto{\pgfqpoint{5.193685in}{2.163427in}}%
\pgfpathlineto{\pgfqpoint{5.195384in}{2.164296in}}%
\pgfpathlineto{\pgfqpoint{5.197083in}{2.161923in}}%
\pgfpathlineto{\pgfqpoint{5.200905in}{2.155460in}}%
\pgfpathlineto{\pgfqpoint{5.202179in}{2.156870in}}%
\pgfpathlineto{\pgfqpoint{5.205153in}{2.166543in}}%
\pgfpathlineto{\pgfqpoint{5.207276in}{2.170724in}}%
\pgfpathlineto{\pgfqpoint{5.208550in}{2.170267in}}%
\pgfpathlineto{\pgfqpoint{5.212798in}{2.163894in}}%
\pgfpathlineto{\pgfqpoint{5.216195in}{2.151240in}}%
\pgfpathlineto{\pgfqpoint{5.218319in}{2.147348in}}%
\pgfpathlineto{\pgfqpoint{5.224265in}{2.145202in}}%
\pgfpathlineto{\pgfqpoint{5.228937in}{2.151180in}}%
\pgfpathlineto{\pgfqpoint{5.231910in}{2.163587in}}%
\pgfpathlineto{\pgfqpoint{5.238281in}{2.188225in}}%
\pgfpathlineto{\pgfqpoint{5.239556in}{2.188474in}}%
\pgfpathlineto{\pgfqpoint{5.241679in}{2.184728in}}%
\pgfpathlineto{\pgfqpoint{5.245077in}{2.174927in}}%
\pgfpathlineto{\pgfqpoint{5.248475in}{2.156869in}}%
\pgfpathlineto{\pgfqpoint{5.252297in}{2.137942in}}%
\pgfpathlineto{\pgfqpoint{5.253572in}{2.137460in}}%
\pgfpathlineto{\pgfqpoint{5.258244in}{2.142257in}}%
\pgfpathlineto{\pgfqpoint{5.258668in}{2.141595in}}%
\pgfpathlineto{\pgfqpoint{5.261217in}{2.133677in}}%
\pgfpathlineto{\pgfqpoint{5.262916in}{2.130718in}}%
\pgfpathlineto{\pgfqpoint{5.263765in}{2.132349in}}%
\pgfpathlineto{\pgfqpoint{5.265464in}{2.142600in}}%
\pgfpathlineto{\pgfqpoint{5.273109in}{2.198972in}}%
\pgfpathlineto{\pgfqpoint{5.275233in}{2.203466in}}%
\pgfpathlineto{\pgfqpoint{5.276507in}{2.201607in}}%
\pgfpathlineto{\pgfqpoint{5.280330in}{2.187484in}}%
\pgfpathlineto{\pgfqpoint{5.284577in}{2.171539in}}%
\pgfpathlineto{\pgfqpoint{5.291373in}{2.124042in}}%
\pgfpathlineto{\pgfqpoint{5.291797in}{2.124161in}}%
\pgfpathlineto{\pgfqpoint{5.297319in}{2.127871in}}%
\pgfpathlineto{\pgfqpoint{5.299018in}{2.128362in}}%
\pgfpathlineto{\pgfqpoint{5.300292in}{2.132201in}}%
\pgfpathlineto{\pgfqpoint{5.302840in}{2.150646in}}%
\pgfpathlineto{\pgfqpoint{5.307088in}{2.180359in}}%
\pgfpathlineto{\pgfqpoint{5.309636in}{2.184577in}}%
\pgfpathlineto{\pgfqpoint{5.313883in}{2.188138in}}%
\pgfpathlineto{\pgfqpoint{5.316856in}{2.190584in}}%
\pgfpathlineto{\pgfqpoint{5.318555in}{2.189377in}}%
\pgfpathlineto{\pgfqpoint{5.320254in}{2.183851in}}%
\pgfpathlineto{\pgfqpoint{5.322803in}{2.164371in}}%
\pgfpathlineto{\pgfqpoint{5.326625in}{2.138192in}}%
\pgfpathlineto{\pgfqpoint{5.327475in}{2.137919in}}%
\pgfpathlineto{\pgfqpoint{5.329173in}{2.142230in}}%
\pgfpathlineto{\pgfqpoint{5.331722in}{2.147668in}}%
\pgfpathlineto{\pgfqpoint{5.332996in}{2.144475in}}%
\pgfpathlineto{\pgfqpoint{5.336394in}{2.130413in}}%
\pgfpathlineto{\pgfqpoint{5.336819in}{2.130560in}}%
\pgfpathlineto{\pgfqpoint{5.338093in}{2.134840in}}%
\pgfpathlineto{\pgfqpoint{5.341915in}{2.164217in}}%
\pgfpathlineto{\pgfqpoint{5.344464in}{2.175299in}}%
\pgfpathlineto{\pgfqpoint{5.346163in}{2.175814in}}%
\pgfpathlineto{\pgfqpoint{5.348286in}{2.175928in}}%
\pgfpathlineto{\pgfqpoint{5.351684in}{2.178864in}}%
\pgfpathlineto{\pgfqpoint{5.353383in}{2.175890in}}%
\pgfpathlineto{\pgfqpoint{5.356781in}{2.170595in}}%
\pgfpathlineto{\pgfqpoint{5.359329in}{2.167811in}}%
\pgfpathlineto{\pgfqpoint{5.361878in}{2.159562in}}%
\pgfpathlineto{\pgfqpoint{5.367399in}{2.137037in}}%
\pgfpathlineto{\pgfqpoint{5.367824in}{2.137282in}}%
\pgfpathlineto{\pgfqpoint{5.369098in}{2.141008in}}%
\pgfpathlineto{\pgfqpoint{5.374620in}{2.163486in}}%
\pgfpathlineto{\pgfqpoint{5.377593in}{2.167054in}}%
\pgfpathlineto{\pgfqpoint{5.382689in}{2.180683in}}%
\pgfpathlineto{\pgfqpoint{5.383114in}{2.180209in}}%
\pgfpathlineto{\pgfqpoint{5.384813in}{2.173900in}}%
\pgfpathlineto{\pgfqpoint{5.389910in}{2.149477in}}%
\pgfpathlineto{\pgfqpoint{5.392033in}{2.148243in}}%
\pgfpathlineto{\pgfqpoint{5.395856in}{2.147490in}}%
\pgfpathlineto{\pgfqpoint{5.398829in}{2.146112in}}%
\pgfpathlineto{\pgfqpoint{5.400103in}{2.148190in}}%
\pgfpathlineto{\pgfqpoint{5.405625in}{2.161288in}}%
\pgfpathlineto{\pgfqpoint{5.407324in}{2.161947in}}%
\pgfpathlineto{\pgfqpoint{5.409023in}{2.166359in}}%
\pgfpathlineto{\pgfqpoint{5.415394in}{2.186771in}}%
\pgfpathlineto{\pgfqpoint{5.417517in}{2.189310in}}%
\pgfpathlineto{\pgfqpoint{5.418791in}{2.188101in}}%
\pgfpathlineto{\pgfqpoint{5.420490in}{2.182175in}}%
\pgfpathlineto{\pgfqpoint{5.423039in}{2.161427in}}%
\pgfpathlineto{\pgfqpoint{5.426437in}{2.134610in}}%
\pgfpathlineto{\pgfqpoint{5.427286in}{2.134059in}}%
\pgfpathlineto{\pgfqpoint{5.428560in}{2.137265in}}%
\pgfpathlineto{\pgfqpoint{5.431109in}{2.143700in}}%
\pgfpathlineto{\pgfqpoint{5.432383in}{2.142201in}}%
\pgfpathlineto{\pgfqpoint{5.435356in}{2.136245in}}%
\pgfpathlineto{\pgfqpoint{5.435781in}{2.136604in}}%
\pgfpathlineto{\pgfqpoint{5.437479in}{2.142291in}}%
\pgfpathlineto{\pgfqpoint{5.441727in}{2.159549in}}%
\pgfpathlineto{\pgfqpoint{5.445125in}{2.163826in}}%
\pgfpathlineto{\pgfqpoint{5.447248in}{2.174895in}}%
\pgfpathlineto{\pgfqpoint{5.450646in}{2.192019in}}%
\pgfpathlineto{\pgfqpoint{5.451496in}{2.191447in}}%
\pgfpathlineto{\pgfqpoint{5.453619in}{2.183007in}}%
\pgfpathlineto{\pgfqpoint{5.456592in}{2.173405in}}%
\pgfpathlineto{\pgfqpoint{5.459565in}{2.167493in}}%
\pgfpathlineto{\pgfqpoint{5.462539in}{2.151383in}}%
\pgfpathlineto{\pgfqpoint{5.465512in}{2.139901in}}%
\pgfpathlineto{\pgfqpoint{5.466786in}{2.140497in}}%
\pgfpathlineto{\pgfqpoint{5.470608in}{2.146082in}}%
\pgfpathlineto{\pgfqpoint{5.474006in}{2.142855in}}%
\pgfpathlineto{\pgfqpoint{5.474431in}{2.143416in}}%
\pgfpathlineto{\pgfqpoint{5.476130in}{2.149927in}}%
\pgfpathlineto{\pgfqpoint{5.481651in}{2.178132in}}%
\pgfpathlineto{\pgfqpoint{5.482501in}{2.177388in}}%
\pgfpathlineto{\pgfqpoint{5.486323in}{2.167620in}}%
\pgfpathlineto{\pgfqpoint{5.487173in}{2.168773in}}%
\pgfpathlineto{\pgfqpoint{5.489296in}{2.178763in}}%
\pgfpathlineto{\pgfqpoint{5.491845in}{2.188468in}}%
\pgfpathlineto{\pgfqpoint{5.492694in}{2.187722in}}%
\pgfpathlineto{\pgfqpoint{5.494393in}{2.179488in}}%
\pgfpathlineto{\pgfqpoint{5.500764in}{2.139086in}}%
\pgfpathlineto{\pgfqpoint{5.504162in}{2.132895in}}%
\pgfpathlineto{\pgfqpoint{5.505436in}{2.133430in}}%
\pgfpathlineto{\pgfqpoint{5.507560in}{2.137794in}}%
\pgfpathlineto{\pgfqpoint{5.517329in}{2.166534in}}%
\pgfpathlineto{\pgfqpoint{5.522850in}{2.195002in}}%
\pgfpathlineto{\pgfqpoint{5.524974in}{2.197842in}}%
\pgfpathlineto{\pgfqpoint{5.526248in}{2.195440in}}%
\pgfpathlineto{\pgfqpoint{5.528796in}{2.182801in}}%
\pgfpathlineto{\pgfqpoint{5.536017in}{2.142937in}}%
\pgfpathlineto{\pgfqpoint{5.538565in}{2.143096in}}%
\pgfpathlineto{\pgfqpoint{5.540264in}{2.141839in}}%
\pgfpathlineto{\pgfqpoint{5.542388in}{2.136137in}}%
\pgfpathlineto{\pgfqpoint{5.545785in}{2.124198in}}%
\pgfpathlineto{\pgfqpoint{5.546210in}{2.124505in}}%
\pgfpathlineto{\pgfqpoint{5.547484in}{2.129750in}}%
\pgfpathlineto{\pgfqpoint{5.551732in}{2.168188in}}%
\pgfpathlineto{\pgfqpoint{5.555979in}{2.195018in}}%
\pgfpathlineto{\pgfqpoint{5.557678in}{2.196807in}}%
\pgfpathlineto{\pgfqpoint{5.559377in}{2.193928in}}%
\pgfpathlineto{\pgfqpoint{5.562775in}{2.183373in}}%
\pgfpathlineto{\pgfqpoint{5.573818in}{2.136200in}}%
\pgfpathlineto{\pgfqpoint{5.577215in}{2.130207in}}%
\pgfpathlineto{\pgfqpoint{5.578490in}{2.130120in}}%
\pgfpathlineto{\pgfqpoint{5.579764in}{2.133019in}}%
\pgfpathlineto{\pgfqpoint{5.581887in}{2.145629in}}%
\pgfpathlineto{\pgfqpoint{5.586559in}{2.175646in}}%
\pgfpathlineto{\pgfqpoint{5.587834in}{2.175750in}}%
\pgfpathlineto{\pgfqpoint{5.591656in}{2.170322in}}%
\pgfpathlineto{\pgfqpoint{5.592081in}{2.170821in}}%
\pgfpathlineto{\pgfqpoint{5.594629in}{2.177870in}}%
\pgfpathlineto{\pgfqpoint{5.596753in}{2.181408in}}%
\pgfpathlineto{\pgfqpoint{5.598027in}{2.180252in}}%
\pgfpathlineto{\pgfqpoint{5.600576in}{2.173057in}}%
\pgfpathlineto{\pgfqpoint{5.606946in}{2.153407in}}%
\pgfpathlineto{\pgfqpoint{5.609070in}{2.152839in}}%
\pgfpathlineto{\pgfqpoint{5.611194in}{2.152229in}}%
\pgfpathlineto{\pgfqpoint{5.612893in}{2.148839in}}%
\pgfpathlineto{\pgfqpoint{5.616291in}{2.140825in}}%
\pgfpathlineto{\pgfqpoint{5.617565in}{2.144021in}}%
\pgfpathlineto{\pgfqpoint{5.622661in}{2.167350in}}%
\pgfpathlineto{\pgfqpoint{5.623086in}{2.166969in}}%
\pgfpathlineto{\pgfqpoint{5.625210in}{2.161178in}}%
\pgfpathlineto{\pgfqpoint{5.626909in}{2.158290in}}%
\pgfpathlineto{\pgfqpoint{5.627333in}{2.158559in}}%
\pgfpathlineto{\pgfqpoint{5.628608in}{2.162028in}}%
\pgfpathlineto{\pgfqpoint{5.632006in}{2.173674in}}%
\pgfpathlineto{\pgfqpoint{5.632855in}{2.172028in}}%
\pgfpathlineto{\pgfqpoint{5.634979in}{2.158832in}}%
\pgfpathlineto{\pgfqpoint{5.637952in}{2.143327in}}%
\pgfpathlineto{\pgfqpoint{5.638801in}{2.144151in}}%
\pgfpathlineto{\pgfqpoint{5.640500in}{2.152488in}}%
\pgfpathlineto{\pgfqpoint{5.644323in}{2.171542in}}%
\pgfpathlineto{\pgfqpoint{5.648570in}{2.181243in}}%
\pgfpathlineto{\pgfqpoint{5.650694in}{2.184778in}}%
\pgfpathlineto{\pgfqpoint{5.651968in}{2.184101in}}%
\pgfpathlineto{\pgfqpoint{5.653667in}{2.180169in}}%
\pgfpathlineto{\pgfqpoint{5.656215in}{2.167387in}}%
\pgfpathlineto{\pgfqpoint{5.662161in}{2.136265in}}%
\pgfpathlineto{\pgfqpoint{5.664710in}{2.129194in}}%
\pgfpathlineto{\pgfqpoint{5.665559in}{2.129908in}}%
\pgfpathlineto{\pgfqpoint{5.667683in}{2.136945in}}%
\pgfpathlineto{\pgfqpoint{5.678301in}{2.174927in}}%
\pgfpathlineto{\pgfqpoint{5.680849in}{2.177053in}}%
\pgfpathlineto{\pgfqpoint{5.682973in}{2.175212in}}%
\pgfpathlineto{\pgfqpoint{5.685521in}{2.173937in}}%
\pgfpathlineto{\pgfqpoint{5.688070in}{2.174395in}}%
\pgfpathlineto{\pgfqpoint{5.689769in}{2.170992in}}%
\pgfpathlineto{\pgfqpoint{5.693167in}{2.155958in}}%
\pgfpathlineto{\pgfqpoint{5.696140in}{2.145687in}}%
\pgfpathlineto{\pgfqpoint{5.696989in}{2.145651in}}%
\pgfpathlineto{\pgfqpoint{5.698263in}{2.148657in}}%
\pgfpathlineto{\pgfqpoint{5.703360in}{2.164765in}}%
\pgfpathlineto{\pgfqpoint{5.708882in}{2.168425in}}%
\pgfpathlineto{\pgfqpoint{5.711430in}{2.169979in}}%
\pgfpathlineto{\pgfqpoint{5.712704in}{2.168747in}}%
\pgfpathlineto{\pgfqpoint{5.714828in}{2.162063in}}%
\pgfpathlineto{\pgfqpoint{5.719075in}{2.145047in}}%
\pgfpathlineto{\pgfqpoint{5.719500in}{2.145205in}}%
\pgfpathlineto{\pgfqpoint{5.720774in}{2.148690in}}%
\pgfpathlineto{\pgfqpoint{5.724597in}{2.161613in}}%
\pgfpathlineto{\pgfqpoint{5.725871in}{2.158146in}}%
\pgfpathlineto{\pgfqpoint{5.730118in}{2.140062in}}%
\pgfpathlineto{\pgfqpoint{5.730543in}{2.140394in}}%
\pgfpathlineto{\pgfqpoint{5.731817in}{2.144809in}}%
\pgfpathlineto{\pgfqpoint{5.737338in}{2.170033in}}%
\pgfpathlineto{\pgfqpoint{5.739462in}{2.169961in}}%
\pgfpathlineto{\pgfqpoint{5.742860in}{2.169685in}}%
\pgfpathlineto{\pgfqpoint{5.744559in}{2.172387in}}%
\pgfpathlineto{\pgfqpoint{5.746258in}{2.177999in}}%
\pgfpathlineto{\pgfqpoint{5.746258in}{2.177999in}}%
\pgfusepath{stroke}%
\end{pgfscope}%
\begin{pgfscope}%
\pgfsetrectcap%
\pgfsetmiterjoin%
\pgfsetlinewidth{0.803000pt}%
\definecolor{currentstroke}{rgb}{0.737255,0.737255,0.737255}%
\pgfsetstrokecolor{currentstroke}%
\pgfsetdash{}{0pt}%
\pgfpathmoveto{\pgfqpoint{0.649081in}{1.713187in}}%
\pgfpathlineto{\pgfqpoint{0.649081in}{2.703703in}}%
\pgfusepath{stroke}%
\end{pgfscope}%
\begin{pgfscope}%
\pgfsetrectcap%
\pgfsetmiterjoin%
\pgfsetlinewidth{0.803000pt}%
\definecolor{currentstroke}{rgb}{0.737255,0.737255,0.737255}%
\pgfsetstrokecolor{currentstroke}%
\pgfsetdash{}{0pt}%
\pgfpathmoveto{\pgfqpoint{5.745833in}{1.713187in}}%
\pgfpathlineto{\pgfqpoint{5.745833in}{2.703703in}}%
\pgfusepath{stroke}%
\end{pgfscope}%
\begin{pgfscope}%
\pgfsetrectcap%
\pgfsetmiterjoin%
\pgfsetlinewidth{0.803000pt}%
\definecolor{currentstroke}{rgb}{0.737255,0.737255,0.737255}%
\pgfsetstrokecolor{currentstroke}%
\pgfsetdash{}{0pt}%
\pgfpathmoveto{\pgfqpoint{0.649081in}{1.713187in}}%
\pgfpathlineto{\pgfqpoint{5.745833in}{1.713187in}}%
\pgfusepath{stroke}%
\end{pgfscope}%
\begin{pgfscope}%
\pgfsetrectcap%
\pgfsetmiterjoin%
\pgfsetlinewidth{0.803000pt}%
\definecolor{currentstroke}{rgb}{0.737255,0.737255,0.737255}%
\pgfsetstrokecolor{currentstroke}%
\pgfsetdash{}{0pt}%
\pgfpathmoveto{\pgfqpoint{0.649081in}{2.703703in}}%
\pgfpathlineto{\pgfqpoint{5.745833in}{2.703703in}}%
\pgfusepath{stroke}%
\end{pgfscope}%
\begin{pgfscope}%
\pgfsetbuttcap%
\pgfsetmiterjoin%
\definecolor{currentfill}{rgb}{0.933333,0.933333,0.933333}%
\pgfsetfillcolor{currentfill}%
\pgfsetlinewidth{0.000000pt}%
\definecolor{currentstroke}{rgb}{0.000000,0.000000,0.000000}%
\pgfsetstrokecolor{currentstroke}%
\pgfsetstrokeopacity{0.000000}%
\pgfsetdash{}{0pt}%
\pgfpathmoveto{\pgfqpoint{0.649081in}{0.544166in}}%
\pgfpathlineto{\pgfqpoint{5.745833in}{0.544166in}}%
\pgfpathlineto{\pgfqpoint{5.745833in}{1.534682in}}%
\pgfpathlineto{\pgfqpoint{0.649081in}{1.534682in}}%
\pgfpathlineto{\pgfqpoint{0.649081in}{0.544166in}}%
\pgfpathclose%
\pgfusepath{fill}%
\end{pgfscope}%
\begin{pgfscope}%
\pgfpathrectangle{\pgfqpoint{0.649081in}{0.544166in}}{\pgfqpoint{5.096752in}{0.990516in}}%
\pgfusepath{clip}%
\pgfsetbuttcap%
\pgfsetroundjoin%
\pgfsetlinewidth{0.501875pt}%
\definecolor{currentstroke}{rgb}{0.698039,0.698039,0.698039}%
\pgfsetstrokecolor{currentstroke}%
\pgfsetdash{{1.850000pt}{0.800000pt}}{0.000000pt}%
\pgfpathmoveto{\pgfqpoint{0.649081in}{0.544166in}}%
\pgfpathlineto{\pgfqpoint{0.649081in}{1.534682in}}%
\pgfusepath{stroke}%
\end{pgfscope}%
\begin{pgfscope}%
\pgfsetbuttcap%
\pgfsetroundjoin%
\definecolor{currentfill}{rgb}{0.000000,0.000000,0.000000}%
\pgfsetfillcolor{currentfill}%
\pgfsetlinewidth{0.803000pt}%
\definecolor{currentstroke}{rgb}{0.000000,0.000000,0.000000}%
\pgfsetstrokecolor{currentstroke}%
\pgfsetdash{}{0pt}%
\pgfsys@defobject{currentmarker}{\pgfqpoint{0.000000in}{0.000000in}}{\pgfqpoint{0.000000in}{0.048611in}}{%
\pgfpathmoveto{\pgfqpoint{0.000000in}{0.000000in}}%
\pgfpathlineto{\pgfqpoint{0.000000in}{0.048611in}}%
\pgfusepath{stroke,fill}%
}%
\begin{pgfscope}%
\pgfsys@transformshift{0.649081in}{0.544166in}%
\pgfsys@useobject{currentmarker}{}%
\end{pgfscope}%
\end{pgfscope}%
\begin{pgfscope}%
\definecolor{textcolor}{rgb}{0.000000,0.000000,0.000000}%
\pgfsetstrokecolor{textcolor}%
\pgfsetfillcolor{textcolor}%
\pgftext[x=0.649081in,y=0.495555in,,top]{\color{textcolor}\rmfamily\fontsize{10.000000}{12.000000}\selectfont \(\displaystyle {40}\)}%
\end{pgfscope}%
\begin{pgfscope}%
\pgfpathrectangle{\pgfqpoint{0.649081in}{0.544166in}}{\pgfqpoint{5.096752in}{0.990516in}}%
\pgfusepath{clip}%
\pgfsetbuttcap%
\pgfsetroundjoin%
\pgfsetlinewidth{0.501875pt}%
\definecolor{currentstroke}{rgb}{0.698039,0.698039,0.698039}%
\pgfsetstrokecolor{currentstroke}%
\pgfsetdash{{1.850000pt}{0.800000pt}}{0.000000pt}%
\pgfpathmoveto{\pgfqpoint{1.498540in}{0.544166in}}%
\pgfpathlineto{\pgfqpoint{1.498540in}{1.534682in}}%
\pgfusepath{stroke}%
\end{pgfscope}%
\begin{pgfscope}%
\pgfsetbuttcap%
\pgfsetroundjoin%
\definecolor{currentfill}{rgb}{0.000000,0.000000,0.000000}%
\pgfsetfillcolor{currentfill}%
\pgfsetlinewidth{0.803000pt}%
\definecolor{currentstroke}{rgb}{0.000000,0.000000,0.000000}%
\pgfsetstrokecolor{currentstroke}%
\pgfsetdash{}{0pt}%
\pgfsys@defobject{currentmarker}{\pgfqpoint{0.000000in}{0.000000in}}{\pgfqpoint{0.000000in}{0.048611in}}{%
\pgfpathmoveto{\pgfqpoint{0.000000in}{0.000000in}}%
\pgfpathlineto{\pgfqpoint{0.000000in}{0.048611in}}%
\pgfusepath{stroke,fill}%
}%
\begin{pgfscope}%
\pgfsys@transformshift{1.498540in}{0.544166in}%
\pgfsys@useobject{currentmarker}{}%
\end{pgfscope}%
\end{pgfscope}%
\begin{pgfscope}%
\definecolor{textcolor}{rgb}{0.000000,0.000000,0.000000}%
\pgfsetstrokecolor{textcolor}%
\pgfsetfillcolor{textcolor}%
\pgftext[x=1.498540in,y=0.495555in,,top]{\color{textcolor}\rmfamily\fontsize{10.000000}{12.000000}\selectfont \(\displaystyle {50}\)}%
\end{pgfscope}%
\begin{pgfscope}%
\pgfpathrectangle{\pgfqpoint{0.649081in}{0.544166in}}{\pgfqpoint{5.096752in}{0.990516in}}%
\pgfusepath{clip}%
\pgfsetbuttcap%
\pgfsetroundjoin%
\pgfsetlinewidth{0.501875pt}%
\definecolor{currentstroke}{rgb}{0.698039,0.698039,0.698039}%
\pgfsetstrokecolor{currentstroke}%
\pgfsetdash{{1.850000pt}{0.800000pt}}{0.000000pt}%
\pgfpathmoveto{\pgfqpoint{2.347998in}{0.544166in}}%
\pgfpathlineto{\pgfqpoint{2.347998in}{1.534682in}}%
\pgfusepath{stroke}%
\end{pgfscope}%
\begin{pgfscope}%
\pgfsetbuttcap%
\pgfsetroundjoin%
\definecolor{currentfill}{rgb}{0.000000,0.000000,0.000000}%
\pgfsetfillcolor{currentfill}%
\pgfsetlinewidth{0.803000pt}%
\definecolor{currentstroke}{rgb}{0.000000,0.000000,0.000000}%
\pgfsetstrokecolor{currentstroke}%
\pgfsetdash{}{0pt}%
\pgfsys@defobject{currentmarker}{\pgfqpoint{0.000000in}{0.000000in}}{\pgfqpoint{0.000000in}{0.048611in}}{%
\pgfpathmoveto{\pgfqpoint{0.000000in}{0.000000in}}%
\pgfpathlineto{\pgfqpoint{0.000000in}{0.048611in}}%
\pgfusepath{stroke,fill}%
}%
\begin{pgfscope}%
\pgfsys@transformshift{2.347998in}{0.544166in}%
\pgfsys@useobject{currentmarker}{}%
\end{pgfscope}%
\end{pgfscope}%
\begin{pgfscope}%
\definecolor{textcolor}{rgb}{0.000000,0.000000,0.000000}%
\pgfsetstrokecolor{textcolor}%
\pgfsetfillcolor{textcolor}%
\pgftext[x=2.347998in,y=0.495555in,,top]{\color{textcolor}\rmfamily\fontsize{10.000000}{12.000000}\selectfont \(\displaystyle {60}\)}%
\end{pgfscope}%
\begin{pgfscope}%
\pgfpathrectangle{\pgfqpoint{0.649081in}{0.544166in}}{\pgfqpoint{5.096752in}{0.990516in}}%
\pgfusepath{clip}%
\pgfsetbuttcap%
\pgfsetroundjoin%
\pgfsetlinewidth{0.501875pt}%
\definecolor{currentstroke}{rgb}{0.698039,0.698039,0.698039}%
\pgfsetstrokecolor{currentstroke}%
\pgfsetdash{{1.850000pt}{0.800000pt}}{0.000000pt}%
\pgfpathmoveto{\pgfqpoint{3.197457in}{0.544166in}}%
\pgfpathlineto{\pgfqpoint{3.197457in}{1.534682in}}%
\pgfusepath{stroke}%
\end{pgfscope}%
\begin{pgfscope}%
\pgfsetbuttcap%
\pgfsetroundjoin%
\definecolor{currentfill}{rgb}{0.000000,0.000000,0.000000}%
\pgfsetfillcolor{currentfill}%
\pgfsetlinewidth{0.803000pt}%
\definecolor{currentstroke}{rgb}{0.000000,0.000000,0.000000}%
\pgfsetstrokecolor{currentstroke}%
\pgfsetdash{}{0pt}%
\pgfsys@defobject{currentmarker}{\pgfqpoint{0.000000in}{0.000000in}}{\pgfqpoint{0.000000in}{0.048611in}}{%
\pgfpathmoveto{\pgfqpoint{0.000000in}{0.000000in}}%
\pgfpathlineto{\pgfqpoint{0.000000in}{0.048611in}}%
\pgfusepath{stroke,fill}%
}%
\begin{pgfscope}%
\pgfsys@transformshift{3.197457in}{0.544166in}%
\pgfsys@useobject{currentmarker}{}%
\end{pgfscope}%
\end{pgfscope}%
\begin{pgfscope}%
\definecolor{textcolor}{rgb}{0.000000,0.000000,0.000000}%
\pgfsetstrokecolor{textcolor}%
\pgfsetfillcolor{textcolor}%
\pgftext[x=3.197457in,y=0.495555in,,top]{\color{textcolor}\rmfamily\fontsize{10.000000}{12.000000}\selectfont \(\displaystyle {70}\)}%
\end{pgfscope}%
\begin{pgfscope}%
\pgfpathrectangle{\pgfqpoint{0.649081in}{0.544166in}}{\pgfqpoint{5.096752in}{0.990516in}}%
\pgfusepath{clip}%
\pgfsetbuttcap%
\pgfsetroundjoin%
\pgfsetlinewidth{0.501875pt}%
\definecolor{currentstroke}{rgb}{0.698039,0.698039,0.698039}%
\pgfsetstrokecolor{currentstroke}%
\pgfsetdash{{1.850000pt}{0.800000pt}}{0.000000pt}%
\pgfpathmoveto{\pgfqpoint{4.046916in}{0.544166in}}%
\pgfpathlineto{\pgfqpoint{4.046916in}{1.534682in}}%
\pgfusepath{stroke}%
\end{pgfscope}%
\begin{pgfscope}%
\pgfsetbuttcap%
\pgfsetroundjoin%
\definecolor{currentfill}{rgb}{0.000000,0.000000,0.000000}%
\pgfsetfillcolor{currentfill}%
\pgfsetlinewidth{0.803000pt}%
\definecolor{currentstroke}{rgb}{0.000000,0.000000,0.000000}%
\pgfsetstrokecolor{currentstroke}%
\pgfsetdash{}{0pt}%
\pgfsys@defobject{currentmarker}{\pgfqpoint{0.000000in}{0.000000in}}{\pgfqpoint{0.000000in}{0.048611in}}{%
\pgfpathmoveto{\pgfqpoint{0.000000in}{0.000000in}}%
\pgfpathlineto{\pgfqpoint{0.000000in}{0.048611in}}%
\pgfusepath{stroke,fill}%
}%
\begin{pgfscope}%
\pgfsys@transformshift{4.046916in}{0.544166in}%
\pgfsys@useobject{currentmarker}{}%
\end{pgfscope}%
\end{pgfscope}%
\begin{pgfscope}%
\definecolor{textcolor}{rgb}{0.000000,0.000000,0.000000}%
\pgfsetstrokecolor{textcolor}%
\pgfsetfillcolor{textcolor}%
\pgftext[x=4.046916in,y=0.495555in,,top]{\color{textcolor}\rmfamily\fontsize{10.000000}{12.000000}\selectfont \(\displaystyle {80}\)}%
\end{pgfscope}%
\begin{pgfscope}%
\pgfpathrectangle{\pgfqpoint{0.649081in}{0.544166in}}{\pgfqpoint{5.096752in}{0.990516in}}%
\pgfusepath{clip}%
\pgfsetbuttcap%
\pgfsetroundjoin%
\pgfsetlinewidth{0.501875pt}%
\definecolor{currentstroke}{rgb}{0.698039,0.698039,0.698039}%
\pgfsetstrokecolor{currentstroke}%
\pgfsetdash{{1.850000pt}{0.800000pt}}{0.000000pt}%
\pgfpathmoveto{\pgfqpoint{4.896374in}{0.544166in}}%
\pgfpathlineto{\pgfqpoint{4.896374in}{1.534682in}}%
\pgfusepath{stroke}%
\end{pgfscope}%
\begin{pgfscope}%
\pgfsetbuttcap%
\pgfsetroundjoin%
\definecolor{currentfill}{rgb}{0.000000,0.000000,0.000000}%
\pgfsetfillcolor{currentfill}%
\pgfsetlinewidth{0.803000pt}%
\definecolor{currentstroke}{rgb}{0.000000,0.000000,0.000000}%
\pgfsetstrokecolor{currentstroke}%
\pgfsetdash{}{0pt}%
\pgfsys@defobject{currentmarker}{\pgfqpoint{0.000000in}{0.000000in}}{\pgfqpoint{0.000000in}{0.048611in}}{%
\pgfpathmoveto{\pgfqpoint{0.000000in}{0.000000in}}%
\pgfpathlineto{\pgfqpoint{0.000000in}{0.048611in}}%
\pgfusepath{stroke,fill}%
}%
\begin{pgfscope}%
\pgfsys@transformshift{4.896374in}{0.544166in}%
\pgfsys@useobject{currentmarker}{}%
\end{pgfscope}%
\end{pgfscope}%
\begin{pgfscope}%
\definecolor{textcolor}{rgb}{0.000000,0.000000,0.000000}%
\pgfsetstrokecolor{textcolor}%
\pgfsetfillcolor{textcolor}%
\pgftext[x=4.896374in,y=0.495555in,,top]{\color{textcolor}\rmfamily\fontsize{10.000000}{12.000000}\selectfont \(\displaystyle {90}\)}%
\end{pgfscope}%
\begin{pgfscope}%
\pgfpathrectangle{\pgfqpoint{0.649081in}{0.544166in}}{\pgfqpoint{5.096752in}{0.990516in}}%
\pgfusepath{clip}%
\pgfsetbuttcap%
\pgfsetroundjoin%
\pgfsetlinewidth{0.501875pt}%
\definecolor{currentstroke}{rgb}{0.698039,0.698039,0.698039}%
\pgfsetstrokecolor{currentstroke}%
\pgfsetdash{{1.850000pt}{0.800000pt}}{0.000000pt}%
\pgfpathmoveto{\pgfqpoint{5.745833in}{0.544166in}}%
\pgfpathlineto{\pgfqpoint{5.745833in}{1.534682in}}%
\pgfusepath{stroke}%
\end{pgfscope}%
\begin{pgfscope}%
\pgfsetbuttcap%
\pgfsetroundjoin%
\definecolor{currentfill}{rgb}{0.000000,0.000000,0.000000}%
\pgfsetfillcolor{currentfill}%
\pgfsetlinewidth{0.803000pt}%
\definecolor{currentstroke}{rgb}{0.000000,0.000000,0.000000}%
\pgfsetstrokecolor{currentstroke}%
\pgfsetdash{}{0pt}%
\pgfsys@defobject{currentmarker}{\pgfqpoint{0.000000in}{0.000000in}}{\pgfqpoint{0.000000in}{0.048611in}}{%
\pgfpathmoveto{\pgfqpoint{0.000000in}{0.000000in}}%
\pgfpathlineto{\pgfqpoint{0.000000in}{0.048611in}}%
\pgfusepath{stroke,fill}%
}%
\begin{pgfscope}%
\pgfsys@transformshift{5.745833in}{0.544166in}%
\pgfsys@useobject{currentmarker}{}%
\end{pgfscope}%
\end{pgfscope}%
\begin{pgfscope}%
\definecolor{textcolor}{rgb}{0.000000,0.000000,0.000000}%
\pgfsetstrokecolor{textcolor}%
\pgfsetfillcolor{textcolor}%
\pgftext[x=5.745833in,y=0.495555in,,top]{\color{textcolor}\rmfamily\fontsize{10.000000}{12.000000}\selectfont \(\displaystyle {100}\)}%
\end{pgfscope}%
\begin{pgfscope}%
\definecolor{textcolor}{rgb}{0.000000,0.000000,0.000000}%
\pgfsetstrokecolor{textcolor}%
\pgfsetfillcolor{textcolor}%
\pgftext[x=3.197457in,y=0.316666in,,top]{\color{textcolor}\rmfamily\fontsize{12.000000}{14.400000}\selectfont Temps [s]}%
\end{pgfscope}%
\begin{pgfscope}%
\pgfpathrectangle{\pgfqpoint{0.649081in}{0.544166in}}{\pgfqpoint{5.096752in}{0.990516in}}%
\pgfusepath{clip}%
\pgfsetbuttcap%
\pgfsetroundjoin%
\pgfsetlinewidth{0.501875pt}%
\definecolor{currentstroke}{rgb}{0.698039,0.698039,0.698039}%
\pgfsetstrokecolor{currentstroke}%
\pgfsetdash{{1.850000pt}{0.800000pt}}{0.000000pt}%
\pgfpathmoveto{\pgfqpoint{0.649081in}{0.589190in}}%
\pgfpathlineto{\pgfqpoint{5.745833in}{0.589190in}}%
\pgfusepath{stroke}%
\end{pgfscope}%
\begin{pgfscope}%
\pgfsetbuttcap%
\pgfsetroundjoin%
\definecolor{currentfill}{rgb}{0.000000,0.000000,0.000000}%
\pgfsetfillcolor{currentfill}%
\pgfsetlinewidth{0.803000pt}%
\definecolor{currentstroke}{rgb}{0.000000,0.000000,0.000000}%
\pgfsetstrokecolor{currentstroke}%
\pgfsetdash{}{0pt}%
\pgfsys@defobject{currentmarker}{\pgfqpoint{0.000000in}{0.000000in}}{\pgfqpoint{0.048611in}{0.000000in}}{%
\pgfpathmoveto{\pgfqpoint{0.000000in}{0.000000in}}%
\pgfpathlineto{\pgfqpoint{0.048611in}{0.000000in}}%
\pgfusepath{stroke,fill}%
}%
\begin{pgfscope}%
\pgfsys@transformshift{0.649081in}{0.589190in}%
\pgfsys@useobject{currentmarker}{}%
\end{pgfscope}%
\end{pgfscope}%
\begin{pgfscope}%
\definecolor{textcolor}{rgb}{0.000000,0.000000,0.000000}%
\pgfsetstrokecolor{textcolor}%
\pgfsetfillcolor{textcolor}%
\pgftext[x=0.353555in, y=0.540995in, left, base]{\color{textcolor}\rmfamily\fontsize{10.000000}{12.000000}\selectfont \(\displaystyle {0.00}\)}%
\end{pgfscope}%
\begin{pgfscope}%
\pgfpathrectangle{\pgfqpoint{0.649081in}{0.544166in}}{\pgfqpoint{5.096752in}{0.990516in}}%
\pgfusepath{clip}%
\pgfsetbuttcap%
\pgfsetroundjoin%
\pgfsetlinewidth{0.501875pt}%
\definecolor{currentstroke}{rgb}{0.698039,0.698039,0.698039}%
\pgfsetstrokecolor{currentstroke}%
\pgfsetdash{{1.850000pt}{0.800000pt}}{0.000000pt}%
\pgfpathmoveto{\pgfqpoint{0.649081in}{1.049029in}}%
\pgfpathlineto{\pgfqpoint{5.745833in}{1.049029in}}%
\pgfusepath{stroke}%
\end{pgfscope}%
\begin{pgfscope}%
\pgfsetbuttcap%
\pgfsetroundjoin%
\definecolor{currentfill}{rgb}{0.000000,0.000000,0.000000}%
\pgfsetfillcolor{currentfill}%
\pgfsetlinewidth{0.803000pt}%
\definecolor{currentstroke}{rgb}{0.000000,0.000000,0.000000}%
\pgfsetstrokecolor{currentstroke}%
\pgfsetdash{}{0pt}%
\pgfsys@defobject{currentmarker}{\pgfqpoint{0.000000in}{0.000000in}}{\pgfqpoint{0.048611in}{0.000000in}}{%
\pgfpathmoveto{\pgfqpoint{0.000000in}{0.000000in}}%
\pgfpathlineto{\pgfqpoint{0.048611in}{0.000000in}}%
\pgfusepath{stroke,fill}%
}%
\begin{pgfscope}%
\pgfsys@transformshift{0.649081in}{1.049029in}%
\pgfsys@useobject{currentmarker}{}%
\end{pgfscope}%
\end{pgfscope}%
\begin{pgfscope}%
\definecolor{textcolor}{rgb}{0.000000,0.000000,0.000000}%
\pgfsetstrokecolor{textcolor}%
\pgfsetfillcolor{textcolor}%
\pgftext[x=0.353555in, y=1.000835in, left, base]{\color{textcolor}\rmfamily\fontsize{10.000000}{12.000000}\selectfont \(\displaystyle {0.01}\)}%
\end{pgfscope}%
\begin{pgfscope}%
\pgfpathrectangle{\pgfqpoint{0.649081in}{0.544166in}}{\pgfqpoint{5.096752in}{0.990516in}}%
\pgfusepath{clip}%
\pgfsetbuttcap%
\pgfsetroundjoin%
\pgfsetlinewidth{0.501875pt}%
\definecolor{currentstroke}{rgb}{0.698039,0.698039,0.698039}%
\pgfsetstrokecolor{currentstroke}%
\pgfsetdash{{1.850000pt}{0.800000pt}}{0.000000pt}%
\pgfpathmoveto{\pgfqpoint{0.649081in}{1.508869in}}%
\pgfpathlineto{\pgfqpoint{5.745833in}{1.508869in}}%
\pgfusepath{stroke}%
\end{pgfscope}%
\begin{pgfscope}%
\pgfsetbuttcap%
\pgfsetroundjoin%
\definecolor{currentfill}{rgb}{0.000000,0.000000,0.000000}%
\pgfsetfillcolor{currentfill}%
\pgfsetlinewidth{0.803000pt}%
\definecolor{currentstroke}{rgb}{0.000000,0.000000,0.000000}%
\pgfsetstrokecolor{currentstroke}%
\pgfsetdash{}{0pt}%
\pgfsys@defobject{currentmarker}{\pgfqpoint{0.000000in}{0.000000in}}{\pgfqpoint{0.048611in}{0.000000in}}{%
\pgfpathmoveto{\pgfqpoint{0.000000in}{0.000000in}}%
\pgfpathlineto{\pgfqpoint{0.048611in}{0.000000in}}%
\pgfusepath{stroke,fill}%
}%
\begin{pgfscope}%
\pgfsys@transformshift{0.649081in}{1.508869in}%
\pgfsys@useobject{currentmarker}{}%
\end{pgfscope}%
\end{pgfscope}%
\begin{pgfscope}%
\definecolor{textcolor}{rgb}{0.000000,0.000000,0.000000}%
\pgfsetstrokecolor{textcolor}%
\pgfsetfillcolor{textcolor}%
\pgftext[x=0.353555in, y=1.460675in, left, base]{\color{textcolor}\rmfamily\fontsize{10.000000}{12.000000}\selectfont \(\displaystyle {0.02}\)}%
\end{pgfscope}%
\begin{pgfscope}%
\definecolor{textcolor}{rgb}{0.000000,0.000000,0.000000}%
\pgfsetstrokecolor{textcolor}%
\pgfsetfillcolor{textcolor}%
\pgftext[x=0.298000in,y=1.039424in,,bottom,rotate=90.000000]{\color{textcolor}\rmfamily\fontsize{12.000000}{14.400000}\selectfont Enveloppe}%
\end{pgfscope}%
\begin{pgfscope}%
\pgfpathrectangle{\pgfqpoint{0.649081in}{0.544166in}}{\pgfqpoint{5.096752in}{0.990516in}}%
\pgfusepath{clip}%
\pgfsetrectcap%
\pgfsetroundjoin%
\pgfsetlinewidth{1.505625pt}%
\definecolor{currentstroke}{rgb}{0.121569,0.466667,0.705882}%
\pgfsetstrokecolor{currentstroke}%
\pgfsetdash{}{0pt}%
\pgfpathmoveto{\pgfqpoint{0.648656in}{0.589191in}}%
\pgfpathlineto{\pgfqpoint{0.820247in}{0.589328in}}%
\pgfpathlineto{\pgfqpoint{0.824069in}{0.589214in}}%
\pgfpathlineto{\pgfqpoint{1.013074in}{0.589302in}}%
\pgfpathlineto{\pgfqpoint{1.017746in}{0.589199in}}%
\pgfpathlineto{\pgfqpoint{1.103117in}{0.589972in}}%
\pgfpathlineto{\pgfqpoint{1.104391in}{0.589538in}}%
\pgfpathlineto{\pgfqpoint{1.109063in}{0.590409in}}%
\pgfpathlineto{\pgfqpoint{1.109487in}{0.589200in}}%
\pgfpathlineto{\pgfqpoint{1.110337in}{0.589543in}}%
\pgfpathlineto{\pgfqpoint{1.115009in}{0.591202in}}%
\pgfpathlineto{\pgfqpoint{1.115434in}{0.589322in}}%
\pgfpathlineto{\pgfqpoint{1.116283in}{0.589814in}}%
\pgfpathlineto{\pgfqpoint{1.119256in}{0.590104in}}%
\pgfpathlineto{\pgfqpoint{1.120106in}{0.590648in}}%
\pgfpathlineto{\pgfqpoint{1.121805in}{0.591863in}}%
\pgfpathlineto{\pgfqpoint{1.122229in}{0.589331in}}%
\pgfpathlineto{\pgfqpoint{1.123079in}{0.589806in}}%
\pgfpathlineto{\pgfqpoint{1.126052in}{0.590476in}}%
\pgfpathlineto{\pgfqpoint{1.126901in}{0.591004in}}%
\pgfpathlineto{\pgfqpoint{1.127326in}{0.591368in}}%
\pgfpathlineto{\pgfqpoint{1.127751in}{0.589268in}}%
\pgfpathlineto{\pgfqpoint{1.128600in}{0.590187in}}%
\pgfpathlineto{\pgfqpoint{1.130724in}{0.591096in}}%
\pgfpathlineto{\pgfqpoint{1.134546in}{0.590495in}}%
\pgfpathlineto{\pgfqpoint{1.137095in}{0.591160in}}%
\pgfpathlineto{\pgfqpoint{1.137520in}{0.591012in}}%
\pgfpathlineto{\pgfqpoint{1.139643in}{0.595314in}}%
\pgfpathlineto{\pgfqpoint{1.140068in}{0.589672in}}%
\pgfpathlineto{\pgfqpoint{1.140917in}{0.590624in}}%
\pgfpathlineto{\pgfqpoint{1.143041in}{0.591518in}}%
\pgfpathlineto{\pgfqpoint{1.146014in}{0.590517in}}%
\pgfpathlineto{\pgfqpoint{1.148138in}{0.592147in}}%
\pgfpathlineto{\pgfqpoint{1.149412in}{0.591092in}}%
\pgfpathlineto{\pgfqpoint{1.149837in}{0.592355in}}%
\pgfpathlineto{\pgfqpoint{1.150261in}{0.592908in}}%
\pgfpathlineto{\pgfqpoint{1.150686in}{0.589362in}}%
\pgfpathlineto{\pgfqpoint{1.151536in}{0.590678in}}%
\pgfpathlineto{\pgfqpoint{1.154084in}{0.592374in}}%
\pgfpathlineto{\pgfqpoint{1.154933in}{0.592296in}}%
\pgfpathlineto{\pgfqpoint{1.157482in}{0.604187in}}%
\pgfpathlineto{\pgfqpoint{1.157907in}{0.589392in}}%
\pgfpathlineto{\pgfqpoint{1.158756in}{0.591198in}}%
\pgfpathlineto{\pgfqpoint{1.160880in}{0.593834in}}%
\pgfpathlineto{\pgfqpoint{1.162154in}{0.592083in}}%
\pgfpathlineto{\pgfqpoint{1.162579in}{0.593886in}}%
\pgfpathlineto{\pgfqpoint{1.163003in}{0.594478in}}%
\pgfpathlineto{\pgfqpoint{1.163428in}{0.589826in}}%
\pgfpathlineto{\pgfqpoint{1.164278in}{0.591916in}}%
\pgfpathlineto{\pgfqpoint{1.165976in}{0.592884in}}%
\pgfpathlineto{\pgfqpoint{1.168525in}{0.589781in}}%
\pgfpathlineto{\pgfqpoint{1.168950in}{0.590483in}}%
\pgfpathlineto{\pgfqpoint{1.171498in}{0.592398in}}%
\pgfpathlineto{\pgfqpoint{1.172772in}{0.591927in}}%
\pgfpathlineto{\pgfqpoint{1.175320in}{0.600859in}}%
\pgfpathlineto{\pgfqpoint{1.175745in}{0.589357in}}%
\pgfpathlineto{\pgfqpoint{1.176595in}{0.590590in}}%
\pgfpathlineto{\pgfqpoint{1.179143in}{0.592802in}}%
\pgfpathlineto{\pgfqpoint{1.179993in}{0.592542in}}%
\pgfpathlineto{\pgfqpoint{1.182541in}{0.605210in}}%
\pgfpathlineto{\pgfqpoint{1.182966in}{0.605608in}}%
\pgfpathlineto{\pgfqpoint{1.183390in}{0.589383in}}%
\pgfpathlineto{\pgfqpoint{1.184240in}{0.590476in}}%
\pgfpathlineto{\pgfqpoint{1.186363in}{0.591339in}}%
\pgfpathlineto{\pgfqpoint{1.188062in}{0.591758in}}%
\pgfpathlineto{\pgfqpoint{1.188912in}{0.592425in}}%
\pgfpathlineto{\pgfqpoint{1.189337in}{0.589856in}}%
\pgfpathlineto{\pgfqpoint{1.190186in}{0.591487in}}%
\pgfpathlineto{\pgfqpoint{1.192310in}{0.593233in}}%
\pgfpathlineto{\pgfqpoint{1.194433in}{0.590115in}}%
\pgfpathlineto{\pgfqpoint{1.194858in}{0.591551in}}%
\pgfpathlineto{\pgfqpoint{1.197406in}{0.597131in}}%
\pgfpathlineto{\pgfqpoint{1.198681in}{0.595093in}}%
\pgfpathlineto{\pgfqpoint{1.199955in}{0.603359in}}%
\pgfpathlineto{\pgfqpoint{1.200380in}{0.590009in}}%
\pgfpathlineto{\pgfqpoint{1.201229in}{0.593301in}}%
\pgfpathlineto{\pgfqpoint{1.202928in}{0.596599in}}%
\pgfpathlineto{\pgfqpoint{1.203353in}{0.596427in}}%
\pgfpathlineto{\pgfqpoint{1.204627in}{0.593090in}}%
\pgfpathlineto{\pgfqpoint{1.205052in}{0.595067in}}%
\pgfpathlineto{\pgfqpoint{1.205476in}{0.589371in}}%
\pgfpathlineto{\pgfqpoint{1.205901in}{0.591605in}}%
\pgfpathlineto{\pgfqpoint{1.208025in}{0.597994in}}%
\pgfpathlineto{\pgfqpoint{1.208874in}{0.596374in}}%
\pgfpathlineto{\pgfqpoint{1.210573in}{0.590551in}}%
\pgfpathlineto{\pgfqpoint{1.212697in}{0.597815in}}%
\pgfpathlineto{\pgfqpoint{1.213546in}{0.596975in}}%
\pgfpathlineto{\pgfqpoint{1.214820in}{0.593226in}}%
\pgfpathlineto{\pgfqpoint{1.216095in}{0.597560in}}%
\pgfpathlineto{\pgfqpoint{1.216519in}{0.589790in}}%
\pgfpathlineto{\pgfqpoint{1.217369in}{0.591039in}}%
\pgfpathlineto{\pgfqpoint{1.219917in}{0.592139in}}%
\pgfpathlineto{\pgfqpoint{1.220767in}{0.591662in}}%
\pgfpathlineto{\pgfqpoint{1.222041in}{0.595732in}}%
\pgfpathlineto{\pgfqpoint{1.222465in}{0.589494in}}%
\pgfpathlineto{\pgfqpoint{1.223315in}{0.591605in}}%
\pgfpathlineto{\pgfqpoint{1.225014in}{0.593797in}}%
\pgfpathlineto{\pgfqpoint{1.226288in}{0.591385in}}%
\pgfpathlineto{\pgfqpoint{1.227137in}{0.589982in}}%
\pgfpathlineto{\pgfqpoint{1.229261in}{0.594740in}}%
\pgfpathlineto{\pgfqpoint{1.230535in}{0.592594in}}%
\pgfpathlineto{\pgfqpoint{1.231809in}{0.589566in}}%
\pgfpathlineto{\pgfqpoint{1.232234in}{0.590561in}}%
\pgfpathlineto{\pgfqpoint{1.233933in}{0.591452in}}%
\pgfpathlineto{\pgfqpoint{1.236482in}{0.589994in}}%
\pgfpathlineto{\pgfqpoint{1.240304in}{0.592028in}}%
\pgfpathlineto{\pgfqpoint{1.242852in}{0.607128in}}%
\pgfpathlineto{\pgfqpoint{1.243277in}{0.589318in}}%
\pgfpathlineto{\pgfqpoint{1.244127in}{0.591651in}}%
\pgfpathlineto{\pgfqpoint{1.245826in}{0.593714in}}%
\pgfpathlineto{\pgfqpoint{1.247524in}{0.592312in}}%
\pgfpathlineto{\pgfqpoint{1.249648in}{0.599375in}}%
\pgfpathlineto{\pgfqpoint{1.250073in}{0.589882in}}%
\pgfpathlineto{\pgfqpoint{1.250922in}{0.591384in}}%
\pgfpathlineto{\pgfqpoint{1.253046in}{0.592833in}}%
\pgfpathlineto{\pgfqpoint{1.254320in}{0.591026in}}%
\pgfpathlineto{\pgfqpoint{1.254745in}{0.591840in}}%
\pgfpathlineto{\pgfqpoint{1.255170in}{0.589524in}}%
\pgfpathlineto{\pgfqpoint{1.255594in}{0.590715in}}%
\pgfpathlineto{\pgfqpoint{1.257718in}{0.594153in}}%
\pgfpathlineto{\pgfqpoint{1.258992in}{0.592969in}}%
\pgfpathlineto{\pgfqpoint{1.259417in}{0.592264in}}%
\pgfpathlineto{\pgfqpoint{1.261116in}{0.597192in}}%
\pgfpathlineto{\pgfqpoint{1.261541in}{0.589777in}}%
\pgfpathlineto{\pgfqpoint{1.262390in}{0.591450in}}%
\pgfpathlineto{\pgfqpoint{1.264938in}{0.595883in}}%
\pgfpathlineto{\pgfqpoint{1.265363in}{0.595483in}}%
\pgfpathlineto{\pgfqpoint{1.265788in}{0.594523in}}%
\pgfpathlineto{\pgfqpoint{1.266637in}{0.600287in}}%
\pgfpathlineto{\pgfqpoint{1.267062in}{0.589462in}}%
\pgfpathlineto{\pgfqpoint{1.267911in}{0.593834in}}%
\pgfpathlineto{\pgfqpoint{1.269186in}{0.597355in}}%
\pgfpathlineto{\pgfqpoint{1.269610in}{0.597217in}}%
\pgfpathlineto{\pgfqpoint{1.270885in}{0.593223in}}%
\pgfpathlineto{\pgfqpoint{1.271734in}{0.589347in}}%
\pgfpathlineto{\pgfqpoint{1.272159in}{0.591388in}}%
\pgfpathlineto{\pgfqpoint{1.274282in}{0.596889in}}%
\pgfpathlineto{\pgfqpoint{1.275557in}{0.596057in}}%
\pgfpathlineto{\pgfqpoint{1.275981in}{0.595283in}}%
\pgfpathlineto{\pgfqpoint{1.277680in}{0.607357in}}%
\pgfpathlineto{\pgfqpoint{1.278105in}{0.590381in}}%
\pgfpathlineto{\pgfqpoint{1.278954in}{0.594364in}}%
\pgfpathlineto{\pgfqpoint{1.280653in}{0.598398in}}%
\pgfpathlineto{\pgfqpoint{1.281928in}{0.594605in}}%
\pgfpathlineto{\pgfqpoint{1.283202in}{0.590486in}}%
\pgfpathlineto{\pgfqpoint{1.285325in}{0.595397in}}%
\pgfpathlineto{\pgfqpoint{1.287024in}{0.591163in}}%
\pgfpathlineto{\pgfqpoint{1.287874in}{0.589846in}}%
\pgfpathlineto{\pgfqpoint{1.290422in}{0.593397in}}%
\pgfpathlineto{\pgfqpoint{1.292121in}{0.593547in}}%
\pgfpathlineto{\pgfqpoint{1.294245in}{0.605914in}}%
\pgfpathlineto{\pgfqpoint{1.294669in}{0.589680in}}%
\pgfpathlineto{\pgfqpoint{1.295519in}{0.591605in}}%
\pgfpathlineto{\pgfqpoint{1.297218in}{0.592679in}}%
\pgfpathlineto{\pgfqpoint{1.302739in}{0.589545in}}%
\pgfpathlineto{\pgfqpoint{1.306987in}{0.592196in}}%
\pgfpathlineto{\pgfqpoint{1.309960in}{0.599903in}}%
\pgfpathlineto{\pgfqpoint{1.310384in}{0.589380in}}%
\pgfpathlineto{\pgfqpoint{1.311234in}{0.589953in}}%
\pgfpathlineto{\pgfqpoint{1.313358in}{0.589976in}}%
\pgfpathlineto{\pgfqpoint{1.314207in}{0.589654in}}%
\pgfpathlineto{\pgfqpoint{1.317180in}{0.594410in}}%
\pgfpathlineto{\pgfqpoint{1.319304in}{0.590827in}}%
\pgfpathlineto{\pgfqpoint{1.321427in}{0.596218in}}%
\pgfpathlineto{\pgfqpoint{1.322702in}{0.593466in}}%
\pgfpathlineto{\pgfqpoint{1.323976in}{0.589347in}}%
\pgfpathlineto{\pgfqpoint{1.324400in}{0.590327in}}%
\pgfpathlineto{\pgfqpoint{1.325250in}{0.590816in}}%
\pgfpathlineto{\pgfqpoint{1.325675in}{0.590328in}}%
\pgfpathlineto{\pgfqpoint{1.326099in}{0.589465in}}%
\pgfpathlineto{\pgfqpoint{1.326524in}{0.590011in}}%
\pgfpathlineto{\pgfqpoint{1.329073in}{0.593902in}}%
\pgfpathlineto{\pgfqpoint{1.330771in}{0.591608in}}%
\pgfpathlineto{\pgfqpoint{1.331196in}{0.593446in}}%
\pgfpathlineto{\pgfqpoint{1.333745in}{0.598087in}}%
\pgfpathlineto{\pgfqpoint{1.334169in}{0.589572in}}%
\pgfpathlineto{\pgfqpoint{1.335019in}{0.591145in}}%
\pgfpathlineto{\pgfqpoint{1.337142in}{0.593646in}}%
\pgfpathlineto{\pgfqpoint{1.339266in}{0.589936in}}%
\pgfpathlineto{\pgfqpoint{1.340115in}{0.591710in}}%
\pgfpathlineto{\pgfqpoint{1.340965in}{0.591588in}}%
\pgfpathlineto{\pgfqpoint{1.341814in}{0.589424in}}%
\pgfpathlineto{\pgfqpoint{1.342239in}{0.590562in}}%
\pgfpathlineto{\pgfqpoint{1.344788in}{0.597136in}}%
\pgfpathlineto{\pgfqpoint{1.346062in}{0.594362in}}%
\pgfpathlineto{\pgfqpoint{1.346486in}{0.592888in}}%
\pgfpathlineto{\pgfqpoint{1.346911in}{0.595115in}}%
\pgfpathlineto{\pgfqpoint{1.347336in}{0.596026in}}%
\pgfpathlineto{\pgfqpoint{1.347761in}{0.589328in}}%
\pgfpathlineto{\pgfqpoint{1.348610in}{0.590452in}}%
\pgfpathlineto{\pgfqpoint{1.350309in}{0.590073in}}%
\pgfpathlineto{\pgfqpoint{1.352433in}{0.590135in}}%
\pgfpathlineto{\pgfqpoint{1.356680in}{0.595930in}}%
\pgfpathlineto{\pgfqpoint{1.358379in}{0.605953in}}%
\pgfpathlineto{\pgfqpoint{1.362201in}{0.637888in}}%
\pgfpathlineto{\pgfqpoint{1.362626in}{0.589196in}}%
\pgfpathlineto{\pgfqpoint{1.363476in}{0.590055in}}%
\pgfpathlineto{\pgfqpoint{1.366024in}{0.591067in}}%
\pgfpathlineto{\pgfqpoint{1.367723in}{0.592383in}}%
\pgfpathlineto{\pgfqpoint{1.369847in}{0.589679in}}%
\pgfpathlineto{\pgfqpoint{1.370696in}{0.591228in}}%
\pgfpathlineto{\pgfqpoint{1.372395in}{0.591537in}}%
\pgfpathlineto{\pgfqpoint{1.374519in}{0.589480in}}%
\pgfpathlineto{\pgfqpoint{1.374943in}{0.589954in}}%
\pgfpathlineto{\pgfqpoint{1.376642in}{0.590306in}}%
\pgfpathlineto{\pgfqpoint{1.377916in}{0.589662in}}%
\pgfpathlineto{\pgfqpoint{1.382164in}{0.594312in}}%
\pgfpathlineto{\pgfqpoint{1.386411in}{0.645299in}}%
\pgfpathlineto{\pgfqpoint{1.386836in}{0.589879in}}%
\pgfpathlineto{\pgfqpoint{1.387685in}{0.592754in}}%
\pgfpathlineto{\pgfqpoint{1.388959in}{0.593962in}}%
\pgfpathlineto{\pgfqpoint{1.391932in}{0.589929in}}%
\pgfpathlineto{\pgfqpoint{1.392357in}{0.590945in}}%
\pgfpathlineto{\pgfqpoint{1.393631in}{0.601719in}}%
\pgfpathlineto{\pgfqpoint{1.398303in}{0.656917in}}%
\pgfpathlineto{\pgfqpoint{1.400852in}{0.658413in}}%
\pgfpathlineto{\pgfqpoint{1.401277in}{0.589319in}}%
\pgfpathlineto{\pgfqpoint{1.402126in}{0.590225in}}%
\pgfpathlineto{\pgfqpoint{1.403825in}{0.591311in}}%
\pgfpathlineto{\pgfqpoint{1.407223in}{0.589456in}}%
\pgfpathlineto{\pgfqpoint{1.410621in}{0.598318in}}%
\pgfpathlineto{\pgfqpoint{1.411045in}{0.597670in}}%
\pgfpathlineto{\pgfqpoint{1.411470in}{0.596479in}}%
\pgfpathlineto{\pgfqpoint{1.412744in}{0.607745in}}%
\pgfpathlineto{\pgfqpoint{1.413169in}{0.589282in}}%
\pgfpathlineto{\pgfqpoint{1.414018in}{0.592265in}}%
\pgfpathlineto{\pgfqpoint{1.415293in}{0.593607in}}%
\pgfpathlineto{\pgfqpoint{1.416567in}{0.591639in}}%
\pgfpathlineto{\pgfqpoint{1.417416in}{0.589467in}}%
\pgfpathlineto{\pgfqpoint{1.417841in}{0.590038in}}%
\pgfpathlineto{\pgfqpoint{1.419965in}{0.593401in}}%
\pgfpathlineto{\pgfqpoint{1.421239in}{0.591368in}}%
\pgfpathlineto{\pgfqpoint{1.421664in}{0.589941in}}%
\pgfpathlineto{\pgfqpoint{1.422088in}{0.590115in}}%
\pgfpathlineto{\pgfqpoint{1.425061in}{0.599078in}}%
\pgfpathlineto{\pgfqpoint{1.425911in}{0.598235in}}%
\pgfpathlineto{\pgfqpoint{1.426336in}{0.597167in}}%
\pgfpathlineto{\pgfqpoint{1.428459in}{0.613753in}}%
\pgfpathlineto{\pgfqpoint{1.428884in}{0.590394in}}%
\pgfpathlineto{\pgfqpoint{1.429733in}{0.592264in}}%
\pgfpathlineto{\pgfqpoint{1.431432in}{0.593324in}}%
\pgfpathlineto{\pgfqpoint{1.433131in}{0.592110in}}%
\pgfpathlineto{\pgfqpoint{1.434405in}{0.596440in}}%
\pgfpathlineto{\pgfqpoint{1.434830in}{0.589493in}}%
\pgfpathlineto{\pgfqpoint{1.435680in}{0.590912in}}%
\pgfpathlineto{\pgfqpoint{1.436954in}{0.590946in}}%
\pgfpathlineto{\pgfqpoint{1.437803in}{0.589512in}}%
\pgfpathlineto{\pgfqpoint{1.438228in}{0.589819in}}%
\pgfpathlineto{\pgfqpoint{1.440352in}{0.592568in}}%
\pgfpathlineto{\pgfqpoint{1.441626in}{0.590521in}}%
\pgfpathlineto{\pgfqpoint{1.442051in}{0.589472in}}%
\pgfpathlineto{\pgfqpoint{1.442475in}{0.589928in}}%
\pgfpathlineto{\pgfqpoint{1.443749in}{0.591546in}}%
\pgfpathlineto{\pgfqpoint{1.444174in}{0.591318in}}%
\pgfpathlineto{\pgfqpoint{1.445024in}{0.589649in}}%
\pgfpathlineto{\pgfqpoint{1.445448in}{0.590026in}}%
\pgfpathlineto{\pgfqpoint{1.447997in}{0.596263in}}%
\pgfpathlineto{\pgfqpoint{1.448846in}{0.595314in}}%
\pgfpathlineto{\pgfqpoint{1.450545in}{0.589784in}}%
\pgfpathlineto{\pgfqpoint{1.450970in}{0.591929in}}%
\pgfpathlineto{\pgfqpoint{1.453518in}{0.599325in}}%
\pgfpathlineto{\pgfqpoint{1.454368in}{0.597707in}}%
\pgfpathlineto{\pgfqpoint{1.454792in}{0.596031in}}%
\pgfpathlineto{\pgfqpoint{1.455642in}{0.602858in}}%
\pgfpathlineto{\pgfqpoint{1.456067in}{0.589757in}}%
\pgfpathlineto{\pgfqpoint{1.456916in}{0.594946in}}%
\pgfpathlineto{\pgfqpoint{1.458615in}{0.599559in}}%
\pgfpathlineto{\pgfqpoint{1.459464in}{0.597841in}}%
\pgfpathlineto{\pgfqpoint{1.461163in}{0.589538in}}%
\pgfpathlineto{\pgfqpoint{1.461588in}{0.592136in}}%
\pgfpathlineto{\pgfqpoint{1.463712in}{0.600171in}}%
\pgfpathlineto{\pgfqpoint{1.464561in}{0.599671in}}%
\pgfpathlineto{\pgfqpoint{1.465411in}{0.597146in}}%
\pgfpathlineto{\pgfqpoint{1.466685in}{0.609123in}}%
\pgfpathlineto{\pgfqpoint{1.467110in}{0.589737in}}%
\pgfpathlineto{\pgfqpoint{1.467959in}{0.594468in}}%
\pgfpathlineto{\pgfqpoint{1.470507in}{0.602331in}}%
\pgfpathlineto{\pgfqpoint{1.471357in}{0.600191in}}%
\pgfpathlineto{\pgfqpoint{1.473056in}{0.619261in}}%
\pgfpathlineto{\pgfqpoint{1.473480in}{0.591851in}}%
\pgfpathlineto{\pgfqpoint{1.474330in}{0.596941in}}%
\pgfpathlineto{\pgfqpoint{1.475604in}{0.600414in}}%
\pgfpathlineto{\pgfqpoint{1.476029in}{0.600295in}}%
\pgfpathlineto{\pgfqpoint{1.477303in}{0.596814in}}%
\pgfpathlineto{\pgfqpoint{1.477728in}{0.594971in}}%
\pgfpathlineto{\pgfqpoint{1.478577in}{0.600593in}}%
\pgfpathlineto{\pgfqpoint{1.479002in}{0.589273in}}%
\pgfpathlineto{\pgfqpoint{1.479851in}{0.592498in}}%
\pgfpathlineto{\pgfqpoint{1.481550in}{0.595391in}}%
\pgfpathlineto{\pgfqpoint{1.482400in}{0.594562in}}%
\pgfpathlineto{\pgfqpoint{1.484099in}{0.590119in}}%
\pgfpathlineto{\pgfqpoint{1.484523in}{0.592183in}}%
\pgfpathlineto{\pgfqpoint{1.486647in}{0.597848in}}%
\pgfpathlineto{\pgfqpoint{1.487921in}{0.595274in}}%
\pgfpathlineto{\pgfqpoint{1.488346in}{0.593743in}}%
\pgfpathlineto{\pgfqpoint{1.489195in}{0.597725in}}%
\pgfpathlineto{\pgfqpoint{1.489620in}{0.589830in}}%
\pgfpathlineto{\pgfqpoint{1.490470in}{0.593223in}}%
\pgfpathlineto{\pgfqpoint{1.493443in}{0.601700in}}%
\pgfpathlineto{\pgfqpoint{1.493868in}{0.601520in}}%
\pgfpathlineto{\pgfqpoint{1.495991in}{0.637656in}}%
\pgfpathlineto{\pgfqpoint{1.496416in}{0.590543in}}%
\pgfpathlineto{\pgfqpoint{1.497265in}{0.596506in}}%
\pgfpathlineto{\pgfqpoint{1.498964in}{0.601075in}}%
\pgfpathlineto{\pgfqpoint{1.500238in}{0.596973in}}%
\pgfpathlineto{\pgfqpoint{1.500663in}{0.594675in}}%
\pgfpathlineto{\pgfqpoint{1.501088in}{0.597645in}}%
\pgfpathlineto{\pgfqpoint{1.501513in}{0.598026in}}%
\pgfpathlineto{\pgfqpoint{1.501937in}{0.591341in}}%
\pgfpathlineto{\pgfqpoint{1.502362in}{0.593709in}}%
\pgfpathlineto{\pgfqpoint{1.504910in}{0.600890in}}%
\pgfpathlineto{\pgfqpoint{1.505760in}{0.599576in}}%
\pgfpathlineto{\pgfqpoint{1.506185in}{0.598066in}}%
\pgfpathlineto{\pgfqpoint{1.507459in}{0.610266in}}%
\pgfpathlineto{\pgfqpoint{1.507884in}{0.591277in}}%
\pgfpathlineto{\pgfqpoint{1.508733in}{0.597887in}}%
\pgfpathlineto{\pgfqpoint{1.510857in}{0.605817in}}%
\pgfpathlineto{\pgfqpoint{1.512131in}{0.601404in}}%
\pgfpathlineto{\pgfqpoint{1.513830in}{0.622376in}}%
\pgfpathlineto{\pgfqpoint{1.514255in}{0.590913in}}%
\pgfpathlineto{\pgfqpoint{1.515104in}{0.594908in}}%
\pgfpathlineto{\pgfqpoint{1.516803in}{0.597240in}}%
\pgfpathlineto{\pgfqpoint{1.518502in}{0.594112in}}%
\pgfpathlineto{\pgfqpoint{1.519776in}{0.601041in}}%
\pgfpathlineto{\pgfqpoint{1.520201in}{0.589634in}}%
\pgfpathlineto{\pgfqpoint{1.521050in}{0.592027in}}%
\pgfpathlineto{\pgfqpoint{1.522324in}{0.593513in}}%
\pgfpathlineto{\pgfqpoint{1.522749in}{0.593215in}}%
\pgfpathlineto{\pgfqpoint{1.524448in}{0.590015in}}%
\pgfpathlineto{\pgfqpoint{1.524873in}{0.591736in}}%
\pgfpathlineto{\pgfqpoint{1.526996in}{0.597265in}}%
\pgfpathlineto{\pgfqpoint{1.527846in}{0.596536in}}%
\pgfpathlineto{\pgfqpoint{1.528695in}{0.594378in}}%
\pgfpathlineto{\pgfqpoint{1.529969in}{0.601574in}}%
\pgfpathlineto{\pgfqpoint{1.530394in}{0.589620in}}%
\pgfpathlineto{\pgfqpoint{1.531244in}{0.592322in}}%
\pgfpathlineto{\pgfqpoint{1.533792in}{0.597514in}}%
\pgfpathlineto{\pgfqpoint{1.534642in}{0.596345in}}%
\pgfpathlineto{\pgfqpoint{1.535916in}{0.607514in}}%
\pgfpathlineto{\pgfqpoint{1.536340in}{0.590291in}}%
\pgfpathlineto{\pgfqpoint{1.537190in}{0.595858in}}%
\pgfpathlineto{\pgfqpoint{1.539314in}{0.602939in}}%
\pgfpathlineto{\pgfqpoint{1.540588in}{0.598391in}}%
\pgfpathlineto{\pgfqpoint{1.541862in}{0.608211in}}%
\pgfpathlineto{\pgfqpoint{1.542287in}{0.592495in}}%
\pgfpathlineto{\pgfqpoint{1.543136in}{0.598659in}}%
\pgfpathlineto{\pgfqpoint{1.545260in}{0.605849in}}%
\pgfpathlineto{\pgfqpoint{1.546109in}{0.604094in}}%
\pgfpathlineto{\pgfqpoint{1.546534in}{0.602301in}}%
\pgfpathlineto{\pgfqpoint{1.548658in}{0.629844in}}%
\pgfpathlineto{\pgfqpoint{1.549082in}{0.591419in}}%
\pgfpathlineto{\pgfqpoint{1.549932in}{0.594945in}}%
\pgfpathlineto{\pgfqpoint{1.551206in}{0.596388in}}%
\pgfpathlineto{\pgfqpoint{1.552480in}{0.593551in}}%
\pgfpathlineto{\pgfqpoint{1.553754in}{0.589532in}}%
\pgfpathlineto{\pgfqpoint{1.554179in}{0.590947in}}%
\pgfpathlineto{\pgfqpoint{1.555453in}{0.593098in}}%
\pgfpathlineto{\pgfqpoint{1.555878in}{0.592860in}}%
\pgfpathlineto{\pgfqpoint{1.557152in}{0.589204in}}%
\pgfpathlineto{\pgfqpoint{1.557577in}{0.591166in}}%
\pgfpathlineto{\pgfqpoint{1.560975in}{0.604600in}}%
\pgfpathlineto{\pgfqpoint{1.561824in}{0.603306in}}%
\pgfpathlineto{\pgfqpoint{1.563948in}{0.639961in}}%
\pgfpathlineto{\pgfqpoint{1.564373in}{0.590886in}}%
\pgfpathlineto{\pgfqpoint{1.565222in}{0.596590in}}%
\pgfpathlineto{\pgfqpoint{1.567346in}{0.602426in}}%
\pgfpathlineto{\pgfqpoint{1.568620in}{0.600330in}}%
\pgfpathlineto{\pgfqpoint{1.571168in}{0.636023in}}%
\pgfpathlineto{\pgfqpoint{1.571593in}{0.636114in}}%
\pgfpathlineto{\pgfqpoint{1.572018in}{0.590663in}}%
\pgfpathlineto{\pgfqpoint{1.572867in}{0.593129in}}%
\pgfpathlineto{\pgfqpoint{1.574141in}{0.594331in}}%
\pgfpathlineto{\pgfqpoint{1.574566in}{0.594016in}}%
\pgfpathlineto{\pgfqpoint{1.577114in}{0.589840in}}%
\pgfpathlineto{\pgfqpoint{1.577539in}{0.590736in}}%
\pgfpathlineto{\pgfqpoint{1.580937in}{0.595131in}}%
\pgfpathlineto{\pgfqpoint{1.581362in}{0.595464in}}%
\pgfpathlineto{\pgfqpoint{1.584335in}{0.627432in}}%
\pgfpathlineto{\pgfqpoint{1.584760in}{0.590538in}}%
\pgfpathlineto{\pgfqpoint{1.585609in}{0.595151in}}%
\pgfpathlineto{\pgfqpoint{1.588157in}{0.602734in}}%
\pgfpathlineto{\pgfqpoint{1.589007in}{0.601851in}}%
\pgfpathlineto{\pgfqpoint{1.591980in}{0.653699in}}%
\pgfpathlineto{\pgfqpoint{1.592405in}{0.654254in}}%
\pgfpathlineto{\pgfqpoint{1.592829in}{0.590525in}}%
\pgfpathlineto{\pgfqpoint{1.593679in}{0.594075in}}%
\pgfpathlineto{\pgfqpoint{1.595803in}{0.598328in}}%
\pgfpathlineto{\pgfqpoint{1.597077in}{0.596542in}}%
\pgfpathlineto{\pgfqpoint{1.599625in}{0.618106in}}%
\pgfpathlineto{\pgfqpoint{1.601749in}{0.625270in}}%
\pgfpathlineto{\pgfqpoint{1.606421in}{0.658385in}}%
\pgfpathlineto{\pgfqpoint{1.606846in}{0.589537in}}%
\pgfpathlineto{\pgfqpoint{1.607695in}{0.590889in}}%
\pgfpathlineto{\pgfqpoint{1.609394in}{0.591255in}}%
\pgfpathlineto{\pgfqpoint{1.611093in}{0.589539in}}%
\pgfpathlineto{\pgfqpoint{1.611518in}{0.590058in}}%
\pgfpathlineto{\pgfqpoint{1.612792in}{0.590026in}}%
\pgfpathlineto{\pgfqpoint{1.613216in}{0.589356in}}%
\pgfpathlineto{\pgfqpoint{1.613641in}{0.590007in}}%
\pgfpathlineto{\pgfqpoint{1.617464in}{0.599046in}}%
\pgfpathlineto{\pgfqpoint{1.617888in}{0.598728in}}%
\pgfpathlineto{\pgfqpoint{1.620862in}{0.638337in}}%
\pgfpathlineto{\pgfqpoint{1.621286in}{0.639343in}}%
\pgfpathlineto{\pgfqpoint{1.621711in}{0.589207in}}%
\pgfpathlineto{\pgfqpoint{1.622560in}{0.590993in}}%
\pgfpathlineto{\pgfqpoint{1.624684in}{0.593221in}}%
\pgfpathlineto{\pgfqpoint{1.625958in}{0.592120in}}%
\pgfpathlineto{\pgfqpoint{1.628082in}{0.597424in}}%
\pgfpathlineto{\pgfqpoint{1.628931in}{0.598352in}}%
\pgfpathlineto{\pgfqpoint{1.630206in}{0.604799in}}%
\pgfpathlineto{\pgfqpoint{1.632329in}{0.633912in}}%
\pgfpathlineto{\pgfqpoint{1.637001in}{0.701916in}}%
\pgfpathlineto{\pgfqpoint{1.642523in}{0.740968in}}%
\pgfpathlineto{\pgfqpoint{1.642948in}{0.589355in}}%
\pgfpathlineto{\pgfqpoint{1.643797in}{0.593164in}}%
\pgfpathlineto{\pgfqpoint{1.646770in}{0.604743in}}%
\pgfpathlineto{\pgfqpoint{1.647195in}{0.604675in}}%
\pgfpathlineto{\pgfqpoint{1.650168in}{0.674389in}}%
\pgfpathlineto{\pgfqpoint{1.651017in}{0.677197in}}%
\pgfpathlineto{\pgfqpoint{1.651442in}{0.589930in}}%
\pgfpathlineto{\pgfqpoint{1.652292in}{0.591381in}}%
\pgfpathlineto{\pgfqpoint{1.653566in}{0.590634in}}%
\pgfpathlineto{\pgfqpoint{1.654415in}{0.589607in}}%
\pgfpathlineto{\pgfqpoint{1.657388in}{0.594369in}}%
\pgfpathlineto{\pgfqpoint{1.658662in}{0.591701in}}%
\pgfpathlineto{\pgfqpoint{1.659087in}{0.592614in}}%
\pgfpathlineto{\pgfqpoint{1.659512in}{0.590136in}}%
\pgfpathlineto{\pgfqpoint{1.659937in}{0.592175in}}%
\pgfpathlineto{\pgfqpoint{1.662910in}{0.603105in}}%
\pgfpathlineto{\pgfqpoint{1.663759in}{0.603039in}}%
\pgfpathlineto{\pgfqpoint{1.666732in}{0.666635in}}%
\pgfpathlineto{\pgfqpoint{1.668007in}{0.670721in}}%
\pgfpathlineto{\pgfqpoint{1.668431in}{0.589948in}}%
\pgfpathlineto{\pgfqpoint{1.669281in}{0.590952in}}%
\pgfpathlineto{\pgfqpoint{1.670980in}{0.590827in}}%
\pgfpathlineto{\pgfqpoint{1.673103in}{0.590841in}}%
\pgfpathlineto{\pgfqpoint{1.674802in}{0.595265in}}%
\pgfpathlineto{\pgfqpoint{1.676501in}{0.606977in}}%
\pgfpathlineto{\pgfqpoint{1.684571in}{0.682488in}}%
\pgfpathlineto{\pgfqpoint{1.685420in}{0.682975in}}%
\pgfpathlineto{\pgfqpoint{1.685845in}{0.589659in}}%
\pgfpathlineto{\pgfqpoint{1.686695in}{0.590817in}}%
\pgfpathlineto{\pgfqpoint{1.689668in}{0.593334in}}%
\pgfpathlineto{\pgfqpoint{1.690092in}{0.593233in}}%
\pgfpathlineto{\pgfqpoint{1.693490in}{0.615197in}}%
\pgfpathlineto{\pgfqpoint{1.694340in}{0.616318in}}%
\pgfpathlineto{\pgfqpoint{1.694764in}{0.589426in}}%
\pgfpathlineto{\pgfqpoint{1.695614in}{0.590303in}}%
\pgfpathlineto{\pgfqpoint{1.697738in}{0.590818in}}%
\pgfpathlineto{\pgfqpoint{1.699012in}{0.590539in}}%
\pgfpathlineto{\pgfqpoint{1.701985in}{0.603486in}}%
\pgfpathlineto{\pgfqpoint{1.704533in}{0.617823in}}%
\pgfpathlineto{\pgfqpoint{1.704958in}{0.589304in}}%
\pgfpathlineto{\pgfqpoint{1.705807in}{0.591783in}}%
\pgfpathlineto{\pgfqpoint{1.707931in}{0.594677in}}%
\pgfpathlineto{\pgfqpoint{1.709205in}{0.592807in}}%
\pgfpathlineto{\pgfqpoint{1.710479in}{0.597537in}}%
\pgfpathlineto{\pgfqpoint{1.710904in}{0.589734in}}%
\pgfpathlineto{\pgfqpoint{1.711754in}{0.591705in}}%
\pgfpathlineto{\pgfqpoint{1.714302in}{0.594623in}}%
\pgfpathlineto{\pgfqpoint{1.715151in}{0.594175in}}%
\pgfpathlineto{\pgfqpoint{1.717275in}{0.609088in}}%
\pgfpathlineto{\pgfqpoint{1.717700in}{0.589231in}}%
\pgfpathlineto{\pgfqpoint{1.718549in}{0.591785in}}%
\pgfpathlineto{\pgfqpoint{1.720673in}{0.595830in}}%
\pgfpathlineto{\pgfqpoint{1.721947in}{0.594519in}}%
\pgfpathlineto{\pgfqpoint{1.723646in}{0.604434in}}%
\pgfpathlineto{\pgfqpoint{1.724071in}{0.589832in}}%
\pgfpathlineto{\pgfqpoint{1.724920in}{0.591884in}}%
\pgfpathlineto{\pgfqpoint{1.727044in}{0.593607in}}%
\pgfpathlineto{\pgfqpoint{1.728318in}{0.592433in}}%
\pgfpathlineto{\pgfqpoint{1.729592in}{0.596880in}}%
\pgfpathlineto{\pgfqpoint{1.730017in}{0.590103in}}%
\pgfpathlineto{\pgfqpoint{1.730866in}{0.593129in}}%
\pgfpathlineto{\pgfqpoint{1.733415in}{0.598672in}}%
\pgfpathlineto{\pgfqpoint{1.734264in}{0.596825in}}%
\pgfpathlineto{\pgfqpoint{1.735539in}{0.608409in}}%
\pgfpathlineto{\pgfqpoint{1.735963in}{0.589799in}}%
\pgfpathlineto{\pgfqpoint{1.736813in}{0.593959in}}%
\pgfpathlineto{\pgfqpoint{1.738512in}{0.597667in}}%
\pgfpathlineto{\pgfqpoint{1.739786in}{0.595696in}}%
\pgfpathlineto{\pgfqpoint{1.740211in}{0.594436in}}%
\pgfpathlineto{\pgfqpoint{1.741485in}{0.601777in}}%
\pgfpathlineto{\pgfqpoint{1.741909in}{0.589385in}}%
\pgfpathlineto{\pgfqpoint{1.742759in}{0.591357in}}%
\pgfpathlineto{\pgfqpoint{1.744458in}{0.592785in}}%
\pgfpathlineto{\pgfqpoint{1.745732in}{0.591617in}}%
\pgfpathlineto{\pgfqpoint{1.747006in}{0.589900in}}%
\pgfpathlineto{\pgfqpoint{1.751253in}{0.600584in}}%
\pgfpathlineto{\pgfqpoint{1.753802in}{0.643433in}}%
\pgfpathlineto{\pgfqpoint{1.754227in}{0.643932in}}%
\pgfpathlineto{\pgfqpoint{1.754651in}{0.590676in}}%
\pgfpathlineto{\pgfqpoint{1.755501in}{0.594127in}}%
\pgfpathlineto{\pgfqpoint{1.758899in}{0.602139in}}%
\pgfpathlineto{\pgfqpoint{1.762721in}{0.695893in}}%
\pgfpathlineto{\pgfqpoint{1.763146in}{0.696563in}}%
\pgfpathlineto{\pgfqpoint{1.763571in}{0.591243in}}%
\pgfpathlineto{\pgfqpoint{1.764420in}{0.595924in}}%
\pgfpathlineto{\pgfqpoint{1.766119in}{0.599247in}}%
\pgfpathlineto{\pgfqpoint{1.767393in}{0.596486in}}%
\pgfpathlineto{\pgfqpoint{1.767818in}{0.594987in}}%
\pgfpathlineto{\pgfqpoint{1.769092in}{0.602401in}}%
\pgfpathlineto{\pgfqpoint{1.769517in}{0.590068in}}%
\pgfpathlineto{\pgfqpoint{1.770366in}{0.593126in}}%
\pgfpathlineto{\pgfqpoint{1.772490in}{0.597008in}}%
\pgfpathlineto{\pgfqpoint{1.773764in}{0.594417in}}%
\pgfpathlineto{\pgfqpoint{1.774614in}{0.599358in}}%
\pgfpathlineto{\pgfqpoint{1.775038in}{0.589918in}}%
\pgfpathlineto{\pgfqpoint{1.775888in}{0.594081in}}%
\pgfpathlineto{\pgfqpoint{1.777587in}{0.597901in}}%
\pgfpathlineto{\pgfqpoint{1.778861in}{0.595897in}}%
\pgfpathlineto{\pgfqpoint{1.779286in}{0.594687in}}%
\pgfpathlineto{\pgfqpoint{1.781409in}{0.605066in}}%
\pgfpathlineto{\pgfqpoint{1.781834in}{0.589283in}}%
\pgfpathlineto{\pgfqpoint{1.782683in}{0.589578in}}%
\pgfpathlineto{\pgfqpoint{1.786081in}{0.590695in}}%
\pgfpathlineto{\pgfqpoint{1.788205in}{0.595931in}}%
\pgfpathlineto{\pgfqpoint{1.788630in}{0.589422in}}%
\pgfpathlineto{\pgfqpoint{1.789479in}{0.590098in}}%
\pgfpathlineto{\pgfqpoint{1.791178in}{0.589679in}}%
\pgfpathlineto{\pgfqpoint{1.793302in}{0.589338in}}%
\pgfpathlineto{\pgfqpoint{1.795850in}{0.589971in}}%
\pgfpathlineto{\pgfqpoint{1.796700in}{0.589433in}}%
\pgfpathlineto{\pgfqpoint{1.797124in}{0.590262in}}%
\pgfpathlineto{\pgfqpoint{1.800522in}{0.596165in}}%
\pgfpathlineto{\pgfqpoint{1.800947in}{0.595645in}}%
\pgfpathlineto{\pgfqpoint{1.802646in}{0.608060in}}%
\pgfpathlineto{\pgfqpoint{1.803070in}{0.590908in}}%
\pgfpathlineto{\pgfqpoint{1.803920in}{0.594937in}}%
\pgfpathlineto{\pgfqpoint{1.805619in}{0.598474in}}%
\pgfpathlineto{\pgfqpoint{1.806893in}{0.595493in}}%
\pgfpathlineto{\pgfqpoint{1.807318in}{0.593671in}}%
\pgfpathlineto{\pgfqpoint{1.807742in}{0.596112in}}%
\pgfpathlineto{\pgfqpoint{1.808167in}{0.596437in}}%
\pgfpathlineto{\pgfqpoint{1.808592in}{0.590924in}}%
\pgfpathlineto{\pgfqpoint{1.809017in}{0.592798in}}%
\pgfpathlineto{\pgfqpoint{1.811140in}{0.596855in}}%
\pgfpathlineto{\pgfqpoint{1.812415in}{0.594935in}}%
\pgfpathlineto{\pgfqpoint{1.812839in}{0.593875in}}%
\pgfpathlineto{\pgfqpoint{1.814538in}{0.601388in}}%
\pgfpathlineto{\pgfqpoint{1.814963in}{0.589777in}}%
\pgfpathlineto{\pgfqpoint{1.815812in}{0.591085in}}%
\pgfpathlineto{\pgfqpoint{1.818361in}{0.592383in}}%
\pgfpathlineto{\pgfqpoint{1.819210in}{0.592650in}}%
\pgfpathlineto{\pgfqpoint{1.825156in}{0.637253in}}%
\pgfpathlineto{\pgfqpoint{1.825581in}{0.589610in}}%
\pgfpathlineto{\pgfqpoint{1.826431in}{0.591710in}}%
\pgfpathlineto{\pgfqpoint{1.829404in}{0.597234in}}%
\pgfpathlineto{\pgfqpoint{1.829828in}{0.596727in}}%
\pgfpathlineto{\pgfqpoint{1.831952in}{0.615147in}}%
\pgfpathlineto{\pgfqpoint{1.832377in}{0.590414in}}%
\pgfpathlineto{\pgfqpoint{1.833226in}{0.593267in}}%
\pgfpathlineto{\pgfqpoint{1.835350in}{0.595587in}}%
\pgfpathlineto{\pgfqpoint{1.836624in}{0.594381in}}%
\pgfpathlineto{\pgfqpoint{1.839172in}{0.611977in}}%
\pgfpathlineto{\pgfqpoint{1.839597in}{0.612496in}}%
\pgfpathlineto{\pgfqpoint{1.840022in}{0.589273in}}%
\pgfpathlineto{\pgfqpoint{1.840871in}{0.590161in}}%
\pgfpathlineto{\pgfqpoint{1.844694in}{0.593423in}}%
\pgfpathlineto{\pgfqpoint{1.846818in}{0.612328in}}%
\pgfpathlineto{\pgfqpoint{1.850215in}{0.645697in}}%
\pgfpathlineto{\pgfqpoint{1.850640in}{0.589959in}}%
\pgfpathlineto{\pgfqpoint{1.851490in}{0.593076in}}%
\pgfpathlineto{\pgfqpoint{1.853189in}{0.595416in}}%
\pgfpathlineto{\pgfqpoint{1.854463in}{0.592893in}}%
\pgfpathlineto{\pgfqpoint{1.855737in}{0.589797in}}%
\pgfpathlineto{\pgfqpoint{1.858285in}{0.594675in}}%
\pgfpathlineto{\pgfqpoint{1.859559in}{0.592858in}}%
\pgfpathlineto{\pgfqpoint{1.860834in}{0.589391in}}%
\pgfpathlineto{\pgfqpoint{1.861258in}{0.590759in}}%
\pgfpathlineto{\pgfqpoint{1.863382in}{0.593950in}}%
\pgfpathlineto{\pgfqpoint{1.864656in}{0.591533in}}%
\pgfpathlineto{\pgfqpoint{1.865506in}{0.589461in}}%
\pgfpathlineto{\pgfqpoint{1.865930in}{0.590720in}}%
\pgfpathlineto{\pgfqpoint{1.867629in}{0.592839in}}%
\pgfpathlineto{\pgfqpoint{1.869328in}{0.589895in}}%
\pgfpathlineto{\pgfqpoint{1.869753in}{0.591586in}}%
\pgfpathlineto{\pgfqpoint{1.872301in}{0.597402in}}%
\pgfpathlineto{\pgfqpoint{1.873576in}{0.595301in}}%
\pgfpathlineto{\pgfqpoint{1.875274in}{0.606617in}}%
\pgfpathlineto{\pgfqpoint{1.875699in}{0.589832in}}%
\pgfpathlineto{\pgfqpoint{1.876549in}{0.592285in}}%
\pgfpathlineto{\pgfqpoint{1.878672in}{0.595352in}}%
\pgfpathlineto{\pgfqpoint{1.879946in}{0.594763in}}%
\pgfpathlineto{\pgfqpoint{1.882920in}{0.617677in}}%
\pgfpathlineto{\pgfqpoint{1.887167in}{0.627544in}}%
\pgfpathlineto{\pgfqpoint{1.887592in}{0.627556in}}%
\pgfpathlineto{\pgfqpoint{1.888016in}{0.589527in}}%
\pgfpathlineto{\pgfqpoint{1.888866in}{0.590006in}}%
\pgfpathlineto{\pgfqpoint{1.894387in}{0.590070in}}%
\pgfpathlineto{\pgfqpoint{1.894812in}{0.589379in}}%
\pgfpathlineto{\pgfqpoint{1.895237in}{0.589865in}}%
\pgfpathlineto{\pgfqpoint{1.898635in}{0.596270in}}%
\pgfpathlineto{\pgfqpoint{1.899059in}{0.596050in}}%
\pgfpathlineto{\pgfqpoint{1.899484in}{0.595508in}}%
\pgfpathlineto{\pgfqpoint{1.901608in}{0.610248in}}%
\pgfpathlineto{\pgfqpoint{1.902032in}{0.590543in}}%
\pgfpathlineto{\pgfqpoint{1.902882in}{0.593107in}}%
\pgfpathlineto{\pgfqpoint{1.904581in}{0.594792in}}%
\pgfpathlineto{\pgfqpoint{1.905855in}{0.592673in}}%
\pgfpathlineto{\pgfqpoint{1.907129in}{0.589669in}}%
\pgfpathlineto{\pgfqpoint{1.907554in}{0.591114in}}%
\pgfpathlineto{\pgfqpoint{1.910102in}{0.596108in}}%
\pgfpathlineto{\pgfqpoint{1.911376in}{0.594056in}}%
\pgfpathlineto{\pgfqpoint{1.912651in}{0.599382in}}%
\pgfpathlineto{\pgfqpoint{1.913075in}{0.591046in}}%
\pgfpathlineto{\pgfqpoint{1.913925in}{0.594561in}}%
\pgfpathlineto{\pgfqpoint{1.916048in}{0.598238in}}%
\pgfpathlineto{\pgfqpoint{1.917323in}{0.596041in}}%
\pgfpathlineto{\pgfqpoint{1.919022in}{0.609608in}}%
\pgfpathlineto{\pgfqpoint{1.919446in}{0.589511in}}%
\pgfpathlineto{\pgfqpoint{1.920296in}{0.592099in}}%
\pgfpathlineto{\pgfqpoint{1.922419in}{0.595048in}}%
\pgfpathlineto{\pgfqpoint{1.923694in}{0.593880in}}%
\pgfpathlineto{\pgfqpoint{1.925393in}{0.602007in}}%
\pgfpathlineto{\pgfqpoint{1.925817in}{0.590579in}}%
\pgfpathlineto{\pgfqpoint{1.926667in}{0.593480in}}%
\pgfpathlineto{\pgfqpoint{1.928366in}{0.596694in}}%
\pgfpathlineto{\pgfqpoint{1.928790in}{0.596522in}}%
\pgfpathlineto{\pgfqpoint{1.930065in}{0.593471in}}%
\pgfpathlineto{\pgfqpoint{1.930914in}{0.596882in}}%
\pgfpathlineto{\pgfqpoint{1.931339in}{0.590268in}}%
\pgfpathlineto{\pgfqpoint{1.932188in}{0.593712in}}%
\pgfpathlineto{\pgfqpoint{1.934312in}{0.597165in}}%
\pgfpathlineto{\pgfqpoint{1.935586in}{0.595240in}}%
\pgfpathlineto{\pgfqpoint{1.937285in}{0.607549in}}%
\pgfpathlineto{\pgfqpoint{1.937710in}{0.589346in}}%
\pgfpathlineto{\pgfqpoint{1.938559in}{0.591610in}}%
\pgfpathlineto{\pgfqpoint{1.940258in}{0.593375in}}%
\pgfpathlineto{\pgfqpoint{1.941532in}{0.591511in}}%
\pgfpathlineto{\pgfqpoint{1.942806in}{0.590165in}}%
\pgfpathlineto{\pgfqpoint{1.944930in}{0.593032in}}%
\pgfpathlineto{\pgfqpoint{1.946629in}{0.591934in}}%
\pgfpathlineto{\pgfqpoint{1.947054in}{0.591480in}}%
\pgfpathlineto{\pgfqpoint{1.949177in}{0.596155in}}%
\pgfpathlineto{\pgfqpoint{1.949602in}{0.589637in}}%
\pgfpathlineto{\pgfqpoint{1.950452in}{0.590762in}}%
\pgfpathlineto{\pgfqpoint{1.953849in}{0.595736in}}%
\pgfpathlineto{\pgfqpoint{1.958097in}{0.649656in}}%
\pgfpathlineto{\pgfqpoint{1.958946in}{0.650953in}}%
\pgfpathlineto{\pgfqpoint{1.959371in}{0.589977in}}%
\pgfpathlineto{\pgfqpoint{1.960220in}{0.591907in}}%
\pgfpathlineto{\pgfqpoint{1.963618in}{0.598889in}}%
\pgfpathlineto{\pgfqpoint{1.966591in}{0.643041in}}%
\pgfpathlineto{\pgfqpoint{1.967016in}{0.589629in}}%
\pgfpathlineto{\pgfqpoint{1.967865in}{0.593130in}}%
\pgfpathlineto{\pgfqpoint{1.969989in}{0.597323in}}%
\pgfpathlineto{\pgfqpoint{1.971263in}{0.596542in}}%
\pgfpathlineto{\pgfqpoint{1.973387in}{0.616956in}}%
\pgfpathlineto{\pgfqpoint{1.973812in}{0.589401in}}%
\pgfpathlineto{\pgfqpoint{1.974661in}{0.591936in}}%
\pgfpathlineto{\pgfqpoint{1.975935in}{0.593331in}}%
\pgfpathlineto{\pgfqpoint{1.976360in}{0.593040in}}%
\pgfpathlineto{\pgfqpoint{1.979333in}{0.589216in}}%
\pgfpathlineto{\pgfqpoint{1.979758in}{0.589713in}}%
\pgfpathlineto{\pgfqpoint{1.983580in}{0.596457in}}%
\pgfpathlineto{\pgfqpoint{1.986129in}{0.623991in}}%
\pgfpathlineto{\pgfqpoint{1.986554in}{0.624951in}}%
\pgfpathlineto{\pgfqpoint{1.986978in}{0.589396in}}%
\pgfpathlineto{\pgfqpoint{1.987828in}{0.591883in}}%
\pgfpathlineto{\pgfqpoint{1.990801in}{0.598728in}}%
\pgfpathlineto{\pgfqpoint{1.991226in}{0.598652in}}%
\pgfpathlineto{\pgfqpoint{1.994199in}{0.638768in}}%
\pgfpathlineto{\pgfqpoint{1.995048in}{0.640021in}}%
\pgfpathlineto{\pgfqpoint{1.995473in}{0.589468in}}%
\pgfpathlineto{\pgfqpoint{1.996322in}{0.589759in}}%
\pgfpathlineto{\pgfqpoint{1.999720in}{0.589227in}}%
\pgfpathlineto{\pgfqpoint{2.003118in}{0.591833in}}%
\pgfpathlineto{\pgfqpoint{2.003967in}{0.591108in}}%
\pgfpathlineto{\pgfqpoint{2.005242in}{0.593400in}}%
\pgfpathlineto{\pgfqpoint{2.005666in}{0.589588in}}%
\pgfpathlineto{\pgfqpoint{2.006516in}{0.590274in}}%
\pgfpathlineto{\pgfqpoint{2.009064in}{0.591213in}}%
\pgfpathlineto{\pgfqpoint{2.012037in}{0.595589in}}%
\pgfpathlineto{\pgfqpoint{2.012462in}{0.595443in}}%
\pgfpathlineto{\pgfqpoint{2.015010in}{0.617068in}}%
\pgfpathlineto{\pgfqpoint{2.015435in}{0.589649in}}%
\pgfpathlineto{\pgfqpoint{2.016285in}{0.591834in}}%
\pgfpathlineto{\pgfqpoint{2.017984in}{0.593064in}}%
\pgfpathlineto{\pgfqpoint{2.020107in}{0.589931in}}%
\pgfpathlineto{\pgfqpoint{2.020532in}{0.591068in}}%
\pgfpathlineto{\pgfqpoint{2.022231in}{0.592959in}}%
\pgfpathlineto{\pgfqpoint{2.023505in}{0.590708in}}%
\pgfpathlineto{\pgfqpoint{2.023930in}{0.589457in}}%
\pgfpathlineto{\pgfqpoint{2.024354in}{0.590301in}}%
\pgfpathlineto{\pgfqpoint{2.027752in}{0.598289in}}%
\pgfpathlineto{\pgfqpoint{2.028602in}{0.597808in}}%
\pgfpathlineto{\pgfqpoint{2.031150in}{0.625678in}}%
\pgfpathlineto{\pgfqpoint{2.031575in}{0.589908in}}%
\pgfpathlineto{\pgfqpoint{2.032424in}{0.592670in}}%
\pgfpathlineto{\pgfqpoint{2.034123in}{0.594738in}}%
\pgfpathlineto{\pgfqpoint{2.035822in}{0.592218in}}%
\pgfpathlineto{\pgfqpoint{2.036247in}{0.594394in}}%
\pgfpathlineto{\pgfqpoint{2.037521in}{0.596838in}}%
\pgfpathlineto{\pgfqpoint{2.037946in}{0.589265in}}%
\pgfpathlineto{\pgfqpoint{2.038795in}{0.589710in}}%
\pgfpathlineto{\pgfqpoint{2.043043in}{0.591770in}}%
\pgfpathlineto{\pgfqpoint{2.043467in}{0.589729in}}%
\pgfpathlineto{\pgfqpoint{2.043892in}{0.590675in}}%
\pgfpathlineto{\pgfqpoint{2.046440in}{0.594234in}}%
\pgfpathlineto{\pgfqpoint{2.047715in}{0.592574in}}%
\pgfpathlineto{\pgfqpoint{2.049413in}{0.598206in}}%
\pgfpathlineto{\pgfqpoint{2.049838in}{0.589195in}}%
\pgfpathlineto{\pgfqpoint{2.050688in}{0.589500in}}%
\pgfpathlineto{\pgfqpoint{2.054086in}{0.590195in}}%
\pgfpathlineto{\pgfqpoint{2.059607in}{0.610186in}}%
\pgfpathlineto{\pgfqpoint{2.063005in}{0.628762in}}%
\pgfpathlineto{\pgfqpoint{2.063430in}{0.589614in}}%
\pgfpathlineto{\pgfqpoint{2.064279in}{0.591760in}}%
\pgfpathlineto{\pgfqpoint{2.066403in}{0.594246in}}%
\pgfpathlineto{\pgfqpoint{2.067677in}{0.592589in}}%
\pgfpathlineto{\pgfqpoint{2.068951in}{0.596787in}}%
\pgfpathlineto{\pgfqpoint{2.069376in}{0.589834in}}%
\pgfpathlineto{\pgfqpoint{2.070225in}{0.591387in}}%
\pgfpathlineto{\pgfqpoint{2.072349in}{0.592166in}}%
\pgfpathlineto{\pgfqpoint{2.073623in}{0.591378in}}%
\pgfpathlineto{\pgfqpoint{2.076171in}{0.597683in}}%
\pgfpathlineto{\pgfqpoint{2.078295in}{0.599504in}}%
\pgfpathlineto{\pgfqpoint{2.079569in}{0.604744in}}%
\pgfpathlineto{\pgfqpoint{2.082118in}{0.629114in}}%
\pgfpathlineto{\pgfqpoint{2.084241in}{0.642226in}}%
\pgfpathlineto{\pgfqpoint{2.084666in}{0.590086in}}%
\pgfpathlineto{\pgfqpoint{2.085515in}{0.592795in}}%
\pgfpathlineto{\pgfqpoint{2.087639in}{0.595974in}}%
\pgfpathlineto{\pgfqpoint{2.088913in}{0.594258in}}%
\pgfpathlineto{\pgfqpoint{2.090612in}{0.603110in}}%
\pgfpathlineto{\pgfqpoint{2.091037in}{0.589770in}}%
\pgfpathlineto{\pgfqpoint{2.091886in}{0.590969in}}%
\pgfpathlineto{\pgfqpoint{2.093161in}{0.590180in}}%
\pgfpathlineto{\pgfqpoint{2.093585in}{0.589318in}}%
\pgfpathlineto{\pgfqpoint{2.094010in}{0.590088in}}%
\pgfpathlineto{\pgfqpoint{2.096558in}{0.593786in}}%
\pgfpathlineto{\pgfqpoint{2.098257in}{0.590779in}}%
\pgfpathlineto{\pgfqpoint{2.098682in}{0.591270in}}%
\pgfpathlineto{\pgfqpoint{2.099107in}{0.589770in}}%
\pgfpathlineto{\pgfqpoint{2.099532in}{0.590736in}}%
\pgfpathlineto{\pgfqpoint{2.101230in}{0.592477in}}%
\pgfpathlineto{\pgfqpoint{2.102929in}{0.590912in}}%
\pgfpathlineto{\pgfqpoint{2.103354in}{0.590396in}}%
\pgfpathlineto{\pgfqpoint{2.103779in}{0.591167in}}%
\pgfpathlineto{\pgfqpoint{2.106327in}{0.592470in}}%
\pgfpathlineto{\pgfqpoint{2.106752in}{0.592495in}}%
\pgfpathlineto{\pgfqpoint{2.107177in}{0.589297in}}%
\pgfpathlineto{\pgfqpoint{2.108026in}{0.589601in}}%
\pgfpathlineto{\pgfqpoint{2.110575in}{0.590901in}}%
\pgfpathlineto{\pgfqpoint{2.113548in}{0.595634in}}%
\pgfpathlineto{\pgfqpoint{2.113972in}{0.595141in}}%
\pgfpathlineto{\pgfqpoint{2.115671in}{0.606891in}}%
\pgfpathlineto{\pgfqpoint{2.116096in}{0.590186in}}%
\pgfpathlineto{\pgfqpoint{2.116945in}{0.592871in}}%
\pgfpathlineto{\pgfqpoint{2.118220in}{0.594301in}}%
\pgfpathlineto{\pgfqpoint{2.118644in}{0.594052in}}%
\pgfpathlineto{\pgfqpoint{2.121617in}{0.589765in}}%
\pgfpathlineto{\pgfqpoint{2.122467in}{0.590887in}}%
\pgfpathlineto{\pgfqpoint{2.124166in}{0.591289in}}%
\pgfpathlineto{\pgfqpoint{2.125865in}{0.589195in}}%
\pgfpathlineto{\pgfqpoint{2.126289in}{0.589984in}}%
\pgfpathlineto{\pgfqpoint{2.128838in}{0.592986in}}%
\pgfpathlineto{\pgfqpoint{2.130112in}{0.591527in}}%
\pgfpathlineto{\pgfqpoint{2.130537in}{0.593085in}}%
\pgfpathlineto{\pgfqpoint{2.130962in}{0.593842in}}%
\pgfpathlineto{\pgfqpoint{2.131386in}{0.589193in}}%
\pgfpathlineto{\pgfqpoint{2.132236in}{0.590433in}}%
\pgfpathlineto{\pgfqpoint{2.134784in}{0.591719in}}%
\pgfpathlineto{\pgfqpoint{2.135634in}{0.591356in}}%
\pgfpathlineto{\pgfqpoint{2.136908in}{0.594989in}}%
\pgfpathlineto{\pgfqpoint{2.137332in}{0.589403in}}%
\pgfpathlineto{\pgfqpoint{2.138182in}{0.591233in}}%
\pgfpathlineto{\pgfqpoint{2.140730in}{0.594138in}}%
\pgfpathlineto{\pgfqpoint{2.141580in}{0.593154in}}%
\pgfpathlineto{\pgfqpoint{2.143279in}{0.600067in}}%
\pgfpathlineto{\pgfqpoint{2.143703in}{0.589880in}}%
\pgfpathlineto{\pgfqpoint{2.144553in}{0.591528in}}%
\pgfpathlineto{\pgfqpoint{2.147101in}{0.594140in}}%
\pgfpathlineto{\pgfqpoint{2.147951in}{0.593973in}}%
\pgfpathlineto{\pgfqpoint{2.150499in}{0.610566in}}%
\pgfpathlineto{\pgfqpoint{2.150924in}{0.610676in}}%
\pgfpathlineto{\pgfqpoint{2.151349in}{0.589712in}}%
\pgfpathlineto{\pgfqpoint{2.152198in}{0.590577in}}%
\pgfpathlineto{\pgfqpoint{2.156021in}{0.593315in}}%
\pgfpathlineto{\pgfqpoint{2.157719in}{0.597813in}}%
\pgfpathlineto{\pgfqpoint{2.158144in}{0.589494in}}%
\pgfpathlineto{\pgfqpoint{2.158994in}{0.590653in}}%
\pgfpathlineto{\pgfqpoint{2.160693in}{0.591030in}}%
\pgfpathlineto{\pgfqpoint{2.163241in}{0.590200in}}%
\pgfpathlineto{\pgfqpoint{2.164940in}{0.589911in}}%
\pgfpathlineto{\pgfqpoint{2.166639in}{0.589767in}}%
\pgfpathlineto{\pgfqpoint{2.170037in}{0.591263in}}%
\pgfpathlineto{\pgfqpoint{2.173010in}{0.595149in}}%
\pgfpathlineto{\pgfqpoint{2.176408in}{0.612652in}}%
\pgfpathlineto{\pgfqpoint{2.176832in}{0.613169in}}%
\pgfpathlineto{\pgfqpoint{2.177257in}{0.589412in}}%
\pgfpathlineto{\pgfqpoint{2.178106in}{0.591184in}}%
\pgfpathlineto{\pgfqpoint{2.180655in}{0.594874in}}%
\pgfpathlineto{\pgfqpoint{2.181504in}{0.594285in}}%
\pgfpathlineto{\pgfqpoint{2.184053in}{0.609769in}}%
\pgfpathlineto{\pgfqpoint{2.184477in}{0.609895in}}%
\pgfpathlineto{\pgfqpoint{2.184902in}{0.589543in}}%
\pgfpathlineto{\pgfqpoint{2.185752in}{0.590176in}}%
\pgfpathlineto{\pgfqpoint{2.189574in}{0.592440in}}%
\pgfpathlineto{\pgfqpoint{2.190848in}{0.595608in}}%
\pgfpathlineto{\pgfqpoint{2.191273in}{0.589236in}}%
\pgfpathlineto{\pgfqpoint{2.192123in}{0.590985in}}%
\pgfpathlineto{\pgfqpoint{2.194671in}{0.594426in}}%
\pgfpathlineto{\pgfqpoint{2.195520in}{0.593646in}}%
\pgfpathlineto{\pgfqpoint{2.197219in}{0.602736in}}%
\pgfpathlineto{\pgfqpoint{2.197644in}{0.589497in}}%
\pgfpathlineto{\pgfqpoint{2.198493in}{0.591382in}}%
\pgfpathlineto{\pgfqpoint{2.201042in}{0.594082in}}%
\pgfpathlineto{\pgfqpoint{2.201891in}{0.594072in}}%
\pgfpathlineto{\pgfqpoint{2.204864in}{0.617502in}}%
\pgfpathlineto{\pgfqpoint{2.208262in}{0.632269in}}%
\pgfpathlineto{\pgfqpoint{2.212934in}{0.675687in}}%
\pgfpathlineto{\pgfqpoint{2.213359in}{0.589912in}}%
\pgfpathlineto{\pgfqpoint{2.214208in}{0.592756in}}%
\pgfpathlineto{\pgfqpoint{2.216757in}{0.597113in}}%
\pgfpathlineto{\pgfqpoint{2.217606in}{0.596100in}}%
\pgfpathlineto{\pgfqpoint{2.220155in}{0.619523in}}%
\pgfpathlineto{\pgfqpoint{2.222703in}{0.626580in}}%
\pgfpathlineto{\pgfqpoint{2.223128in}{0.626808in}}%
\pgfpathlineto{\pgfqpoint{2.223553in}{0.589416in}}%
\pgfpathlineto{\pgfqpoint{2.224402in}{0.590445in}}%
\pgfpathlineto{\pgfqpoint{2.226526in}{0.591524in}}%
\pgfpathlineto{\pgfqpoint{2.228649in}{0.589514in}}%
\pgfpathlineto{\pgfqpoint{2.229499in}{0.590709in}}%
\pgfpathlineto{\pgfqpoint{2.232897in}{0.593077in}}%
\pgfpathlineto{\pgfqpoint{2.236294in}{0.618106in}}%
\pgfpathlineto{\pgfqpoint{2.236719in}{0.589678in}}%
\pgfpathlineto{\pgfqpoint{2.237569in}{0.592054in}}%
\pgfpathlineto{\pgfqpoint{2.239692in}{0.594162in}}%
\pgfpathlineto{\pgfqpoint{2.240966in}{0.592509in}}%
\pgfpathlineto{\pgfqpoint{2.246488in}{0.633909in}}%
\pgfpathlineto{\pgfqpoint{2.249461in}{0.654955in}}%
\pgfpathlineto{\pgfqpoint{2.250310in}{0.655937in}}%
\pgfpathlineto{\pgfqpoint{2.250735in}{0.589220in}}%
\pgfpathlineto{\pgfqpoint{2.251585in}{0.589658in}}%
\pgfpathlineto{\pgfqpoint{2.256681in}{0.591802in}}%
\pgfpathlineto{\pgfqpoint{2.258380in}{0.598133in}}%
\pgfpathlineto{\pgfqpoint{2.260929in}{0.621986in}}%
\pgfpathlineto{\pgfqpoint{2.265176in}{0.661257in}}%
\pgfpathlineto{\pgfqpoint{2.267724in}{0.669086in}}%
\pgfpathlineto{\pgfqpoint{2.269848in}{0.670409in}}%
\pgfpathlineto{\pgfqpoint{2.270273in}{0.589279in}}%
\pgfpathlineto{\pgfqpoint{2.271122in}{0.589615in}}%
\pgfpathlineto{\pgfqpoint{2.275794in}{0.592764in}}%
\pgfpathlineto{\pgfqpoint{2.277493in}{0.602366in}}%
\pgfpathlineto{\pgfqpoint{2.280891in}{0.622710in}}%
\pgfpathlineto{\pgfqpoint{2.281316in}{0.589822in}}%
\pgfpathlineto{\pgfqpoint{2.282165in}{0.591062in}}%
\pgfpathlineto{\pgfqpoint{2.283439in}{0.590506in}}%
\pgfpathlineto{\pgfqpoint{2.283864in}{0.589608in}}%
\pgfpathlineto{\pgfqpoint{2.284289in}{0.589938in}}%
\pgfpathlineto{\pgfqpoint{2.287262in}{0.596523in}}%
\pgfpathlineto{\pgfqpoint{2.288536in}{0.595543in}}%
\pgfpathlineto{\pgfqpoint{2.291084in}{0.615990in}}%
\pgfpathlineto{\pgfqpoint{2.291509in}{0.589267in}}%
\pgfpathlineto{\pgfqpoint{2.292359in}{0.591152in}}%
\pgfpathlineto{\pgfqpoint{2.295332in}{0.594482in}}%
\pgfpathlineto{\pgfqpoint{2.295757in}{0.594470in}}%
\pgfpathlineto{\pgfqpoint{2.300004in}{0.632955in}}%
\pgfpathlineto{\pgfqpoint{2.305101in}{0.666764in}}%
\pgfpathlineto{\pgfqpoint{2.307224in}{0.667774in}}%
\pgfpathlineto{\pgfqpoint{2.308923in}{0.669473in}}%
\pgfpathlineto{\pgfqpoint{2.311047in}{0.676157in}}%
\pgfpathlineto{\pgfqpoint{2.314445in}{0.692480in}}%
\pgfpathlineto{\pgfqpoint{2.314869in}{0.590211in}}%
\pgfpathlineto{\pgfqpoint{2.315719in}{0.594154in}}%
\pgfpathlineto{\pgfqpoint{2.318692in}{0.607102in}}%
\pgfpathlineto{\pgfqpoint{2.319117in}{0.606720in}}%
\pgfpathlineto{\pgfqpoint{2.321665in}{0.662813in}}%
\pgfpathlineto{\pgfqpoint{2.322090in}{0.591464in}}%
\pgfpathlineto{\pgfqpoint{2.322939in}{0.597023in}}%
\pgfpathlineto{\pgfqpoint{2.324638in}{0.600851in}}%
\pgfpathlineto{\pgfqpoint{2.325912in}{0.598168in}}%
\pgfpathlineto{\pgfqpoint{2.326337in}{0.596644in}}%
\pgfpathlineto{\pgfqpoint{2.328036in}{0.609064in}}%
\pgfpathlineto{\pgfqpoint{2.328461in}{0.590276in}}%
\pgfpathlineto{\pgfqpoint{2.329310in}{0.592629in}}%
\pgfpathlineto{\pgfqpoint{2.330160in}{0.593410in}}%
\pgfpathlineto{\pgfqpoint{2.330584in}{0.593060in}}%
\pgfpathlineto{\pgfqpoint{2.331859in}{0.589321in}}%
\pgfpathlineto{\pgfqpoint{2.332283in}{0.591377in}}%
\pgfpathlineto{\pgfqpoint{2.334832in}{0.600750in}}%
\pgfpathlineto{\pgfqpoint{2.335681in}{0.600195in}}%
\pgfpathlineto{\pgfqpoint{2.336106in}{0.599182in}}%
\pgfpathlineto{\pgfqpoint{2.338229in}{0.622737in}}%
\pgfpathlineto{\pgfqpoint{2.338654in}{0.590908in}}%
\pgfpathlineto{\pgfqpoint{2.339504in}{0.595172in}}%
\pgfpathlineto{\pgfqpoint{2.342052in}{0.602123in}}%
\pgfpathlineto{\pgfqpoint{2.342901in}{0.601212in}}%
\pgfpathlineto{\pgfqpoint{2.345450in}{0.642726in}}%
\pgfpathlineto{\pgfqpoint{2.346724in}{0.645085in}}%
\pgfpathlineto{\pgfqpoint{2.347149in}{0.589474in}}%
\pgfpathlineto{\pgfqpoint{2.347998in}{0.589708in}}%
\pgfpathlineto{\pgfqpoint{2.350122in}{0.591162in}}%
\pgfpathlineto{\pgfqpoint{2.351396in}{0.593228in}}%
\pgfpathlineto{\pgfqpoint{2.354794in}{0.622049in}}%
\pgfpathlineto{\pgfqpoint{2.355219in}{0.589310in}}%
\pgfpathlineto{\pgfqpoint{2.356068in}{0.590007in}}%
\pgfpathlineto{\pgfqpoint{2.356918in}{0.589697in}}%
\pgfpathlineto{\pgfqpoint{2.360315in}{0.599899in}}%
\pgfpathlineto{\pgfqpoint{2.360740in}{0.599437in}}%
\pgfpathlineto{\pgfqpoint{2.361165in}{0.598409in}}%
\pgfpathlineto{\pgfqpoint{2.363288in}{0.617939in}}%
\pgfpathlineto{\pgfqpoint{2.363713in}{0.590349in}}%
\pgfpathlineto{\pgfqpoint{2.364563in}{0.591390in}}%
\pgfpathlineto{\pgfqpoint{2.365837in}{0.589647in}}%
\pgfpathlineto{\pgfqpoint{2.366262in}{0.591351in}}%
\pgfpathlineto{\pgfqpoint{2.368810in}{0.599201in}}%
\pgfpathlineto{\pgfqpoint{2.369659in}{0.597415in}}%
\pgfpathlineto{\pgfqpoint{2.370084in}{0.595494in}}%
\pgfpathlineto{\pgfqpoint{2.370934in}{0.600541in}}%
\pgfpathlineto{\pgfqpoint{2.371358in}{0.590724in}}%
\pgfpathlineto{\pgfqpoint{2.372208in}{0.595067in}}%
\pgfpathlineto{\pgfqpoint{2.373907in}{0.597134in}}%
\pgfpathlineto{\pgfqpoint{2.374756in}{0.597891in}}%
\pgfpathlineto{\pgfqpoint{2.375606in}{0.601793in}}%
\pgfpathlineto{\pgfqpoint{2.377305in}{0.693589in}}%
\pgfpathlineto{\pgfqpoint{2.381977in}{1.012082in}}%
\pgfpathlineto{\pgfqpoint{2.382401in}{0.590950in}}%
\pgfpathlineto{\pgfqpoint{2.383251in}{0.601435in}}%
\pgfpathlineto{\pgfqpoint{2.386649in}{0.628567in}}%
\pgfpathlineto{\pgfqpoint{2.389622in}{0.830503in}}%
\pgfpathlineto{\pgfqpoint{2.390046in}{0.590825in}}%
\pgfpathlineto{\pgfqpoint{2.390896in}{0.610367in}}%
\pgfpathlineto{\pgfqpoint{2.392595in}{0.627071in}}%
\pgfpathlineto{\pgfqpoint{2.393444in}{0.618299in}}%
\pgfpathlineto{\pgfqpoint{2.394718in}{0.590237in}}%
\pgfpathlineto{\pgfqpoint{2.395143in}{0.602316in}}%
\pgfpathlineto{\pgfqpoint{2.397692in}{0.643779in}}%
\pgfpathlineto{\pgfqpoint{2.398541in}{0.635219in}}%
\pgfpathlineto{\pgfqpoint{2.398966in}{0.627075in}}%
\pgfpathlineto{\pgfqpoint{2.400240in}{0.678431in}}%
\pgfpathlineto{\pgfqpoint{2.400665in}{0.593085in}}%
\pgfpathlineto{\pgfqpoint{2.401514in}{0.607967in}}%
\pgfpathlineto{\pgfqpoint{2.402364in}{0.612786in}}%
\pgfpathlineto{\pgfqpoint{2.402788in}{0.611303in}}%
\pgfpathlineto{\pgfqpoint{2.404487in}{0.590592in}}%
\pgfpathlineto{\pgfqpoint{2.405337in}{0.597169in}}%
\pgfpathlineto{\pgfqpoint{2.406186in}{0.599896in}}%
\pgfpathlineto{\pgfqpoint{2.406611in}{0.598285in}}%
\pgfpathlineto{\pgfqpoint{2.407460in}{0.589871in}}%
\pgfpathlineto{\pgfqpoint{2.407885in}{0.594564in}}%
\pgfpathlineto{\pgfqpoint{2.411283in}{0.634263in}}%
\pgfpathlineto{\pgfqpoint{2.411708in}{0.634871in}}%
\pgfpathlineto{\pgfqpoint{2.412132in}{0.634311in}}%
\pgfpathlineto{\pgfqpoint{2.415105in}{0.833081in}}%
\pgfpathlineto{\pgfqpoint{2.415530in}{0.590025in}}%
\pgfpathlineto{\pgfqpoint{2.416380in}{0.606400in}}%
\pgfpathlineto{\pgfqpoint{2.418079in}{0.621563in}}%
\pgfpathlineto{\pgfqpoint{2.418928in}{0.616519in}}%
\pgfpathlineto{\pgfqpoint{2.419777in}{0.604929in}}%
\pgfpathlineto{\pgfqpoint{2.420202in}{0.613823in}}%
\pgfpathlineto{\pgfqpoint{2.420627in}{0.616124in}}%
\pgfpathlineto{\pgfqpoint{2.421052in}{0.592516in}}%
\pgfpathlineto{\pgfqpoint{2.421901in}{0.598530in}}%
\pgfpathlineto{\pgfqpoint{2.422326in}{0.598186in}}%
\pgfpathlineto{\pgfqpoint{2.423175in}{0.590612in}}%
\pgfpathlineto{\pgfqpoint{2.423600in}{0.594412in}}%
\pgfpathlineto{\pgfqpoint{2.426573in}{0.635860in}}%
\pgfpathlineto{\pgfqpoint{2.427423in}{0.629268in}}%
\pgfpathlineto{\pgfqpoint{2.427847in}{0.621996in}}%
\pgfpathlineto{\pgfqpoint{2.429122in}{0.657974in}}%
\pgfpathlineto{\pgfqpoint{2.429546in}{0.600829in}}%
\pgfpathlineto{\pgfqpoint{2.430396in}{0.621688in}}%
\pgfpathlineto{\pgfqpoint{2.431670in}{0.636244in}}%
\pgfpathlineto{\pgfqpoint{2.432095in}{0.635736in}}%
\pgfpathlineto{\pgfqpoint{2.433369in}{0.621737in}}%
\pgfpathlineto{\pgfqpoint{2.433794in}{0.614533in}}%
\pgfpathlineto{\pgfqpoint{2.435068in}{0.644262in}}%
\pgfpathlineto{\pgfqpoint{2.435492in}{0.594270in}}%
\pgfpathlineto{\pgfqpoint{2.436342in}{0.607210in}}%
\pgfpathlineto{\pgfqpoint{2.438466in}{0.623794in}}%
\pgfpathlineto{\pgfqpoint{2.439315in}{0.621546in}}%
\pgfpathlineto{\pgfqpoint{2.439740in}{0.618329in}}%
\pgfpathlineto{\pgfqpoint{2.441439in}{0.680383in}}%
\pgfpathlineto{\pgfqpoint{2.441863in}{0.590699in}}%
\pgfpathlineto{\pgfqpoint{2.442713in}{0.604454in}}%
\pgfpathlineto{\pgfqpoint{2.444837in}{0.622160in}}%
\pgfpathlineto{\pgfqpoint{2.445686in}{0.618571in}}%
\pgfpathlineto{\pgfqpoint{2.446111in}{0.614214in}}%
\pgfpathlineto{\pgfqpoint{2.447385in}{0.648824in}}%
\pgfpathlineto{\pgfqpoint{2.447810in}{0.593983in}}%
\pgfpathlineto{\pgfqpoint{2.448659in}{0.608839in}}%
\pgfpathlineto{\pgfqpoint{2.449509in}{0.615566in}}%
\pgfpathlineto{\pgfqpoint{2.449933in}{0.614372in}}%
\pgfpathlineto{\pgfqpoint{2.451207in}{0.592793in}}%
\pgfpathlineto{\pgfqpoint{2.451632in}{0.596558in}}%
\pgfpathlineto{\pgfqpoint{2.453756in}{0.631331in}}%
\pgfpathlineto{\pgfqpoint{2.454605in}{0.623277in}}%
\pgfpathlineto{\pgfqpoint{2.455879in}{0.593773in}}%
\pgfpathlineto{\pgfqpoint{2.456304in}{0.595372in}}%
\pgfpathlineto{\pgfqpoint{2.458428in}{0.621650in}}%
\pgfpathlineto{\pgfqpoint{2.459277in}{0.616092in}}%
\pgfpathlineto{\pgfqpoint{2.461401in}{0.590787in}}%
\pgfpathlineto{\pgfqpoint{2.461826in}{0.595502in}}%
\pgfpathlineto{\pgfqpoint{2.463525in}{0.603702in}}%
\pgfpathlineto{\pgfqpoint{2.464374in}{0.600641in}}%
\pgfpathlineto{\pgfqpoint{2.465648in}{0.589822in}}%
\pgfpathlineto{\pgfqpoint{2.466073in}{0.595072in}}%
\pgfpathlineto{\pgfqpoint{2.468621in}{0.618660in}}%
\pgfpathlineto{\pgfqpoint{2.469046in}{0.618596in}}%
\pgfpathlineto{\pgfqpoint{2.469896in}{0.613652in}}%
\pgfpathlineto{\pgfqpoint{2.471170in}{0.653065in}}%
\pgfpathlineto{\pgfqpoint{2.471594in}{0.590855in}}%
\pgfpathlineto{\pgfqpoint{2.472444in}{0.606722in}}%
\pgfpathlineto{\pgfqpoint{2.474143in}{0.623112in}}%
\pgfpathlineto{\pgfqpoint{2.474992in}{0.618218in}}%
\pgfpathlineto{\pgfqpoint{2.476691in}{0.589305in}}%
\pgfpathlineto{\pgfqpoint{2.477116in}{0.597534in}}%
\pgfpathlineto{\pgfqpoint{2.479240in}{0.618826in}}%
\pgfpathlineto{\pgfqpoint{2.480089in}{0.614012in}}%
\pgfpathlineto{\pgfqpoint{2.481788in}{0.591381in}}%
\pgfpathlineto{\pgfqpoint{2.482213in}{0.599519in}}%
\pgfpathlineto{\pgfqpoint{2.484761in}{0.626376in}}%
\pgfpathlineto{\pgfqpoint{2.485611in}{0.620631in}}%
\pgfpathlineto{\pgfqpoint{2.486035in}{0.614848in}}%
\pgfpathlineto{\pgfqpoint{2.487309in}{0.642913in}}%
\pgfpathlineto{\pgfqpoint{2.487734in}{0.598617in}}%
\pgfpathlineto{\pgfqpoint{2.488584in}{0.616250in}}%
\pgfpathlineto{\pgfqpoint{2.490283in}{0.632006in}}%
\pgfpathlineto{\pgfqpoint{2.491132in}{0.626823in}}%
\pgfpathlineto{\pgfqpoint{2.491981in}{0.614289in}}%
\pgfpathlineto{\pgfqpoint{2.492831in}{0.639595in}}%
\pgfpathlineto{\pgfqpoint{2.493256in}{0.589683in}}%
\pgfpathlineto{\pgfqpoint{2.494105in}{0.605632in}}%
\pgfpathlineto{\pgfqpoint{2.496229in}{0.621635in}}%
\pgfpathlineto{\pgfqpoint{2.497503in}{0.615429in}}%
\pgfpathlineto{\pgfqpoint{2.500051in}{0.705039in}}%
\pgfpathlineto{\pgfqpoint{2.500901in}{0.710690in}}%
\pgfpathlineto{\pgfqpoint{2.501326in}{0.590678in}}%
\pgfpathlineto{\pgfqpoint{2.502175in}{0.595821in}}%
\pgfpathlineto{\pgfqpoint{2.503449in}{0.599692in}}%
\pgfpathlineto{\pgfqpoint{2.503874in}{0.599548in}}%
\pgfpathlineto{\pgfqpoint{2.505148in}{0.595528in}}%
\pgfpathlineto{\pgfqpoint{2.506422in}{0.589194in}}%
\pgfpathlineto{\pgfqpoint{2.506847in}{0.591202in}}%
\pgfpathlineto{\pgfqpoint{2.508971in}{0.596402in}}%
\pgfpathlineto{\pgfqpoint{2.509820in}{0.594716in}}%
\pgfpathlineto{\pgfqpoint{2.510670in}{0.590411in}}%
\pgfpathlineto{\pgfqpoint{2.511094in}{0.590954in}}%
\pgfpathlineto{\pgfqpoint{2.514492in}{0.610856in}}%
\pgfpathlineto{\pgfqpoint{2.515342in}{0.608602in}}%
\pgfpathlineto{\pgfqpoint{2.519589in}{0.739451in}}%
\pgfpathlineto{\pgfqpoint{2.522137in}{0.799918in}}%
\pgfpathlineto{\pgfqpoint{2.522562in}{0.590301in}}%
\pgfpathlineto{\pgfqpoint{2.523411in}{0.598470in}}%
\pgfpathlineto{\pgfqpoint{2.526385in}{0.615388in}}%
\pgfpathlineto{\pgfqpoint{2.526809in}{0.616655in}}%
\pgfpathlineto{\pgfqpoint{2.531057in}{0.844543in}}%
\pgfpathlineto{\pgfqpoint{2.531481in}{0.590034in}}%
\pgfpathlineto{\pgfqpoint{2.532331in}{0.601993in}}%
\pgfpathlineto{\pgfqpoint{2.534030in}{0.613505in}}%
\pgfpathlineto{\pgfqpoint{2.534879in}{0.611334in}}%
\pgfpathlineto{\pgfqpoint{2.535729in}{0.604882in}}%
\pgfpathlineto{\pgfqpoint{2.537003in}{0.625488in}}%
\pgfpathlineto{\pgfqpoint{2.537428in}{0.591225in}}%
\pgfpathlineto{\pgfqpoint{2.538277in}{0.598312in}}%
\pgfpathlineto{\pgfqpoint{2.540401in}{0.604270in}}%
\pgfpathlineto{\pgfqpoint{2.541675in}{0.601544in}}%
\pgfpathlineto{\pgfqpoint{2.544223in}{0.639940in}}%
\pgfpathlineto{\pgfqpoint{2.547196in}{0.653485in}}%
\pgfpathlineto{\pgfqpoint{2.547621in}{0.589295in}}%
\pgfpathlineto{\pgfqpoint{2.548470in}{0.594716in}}%
\pgfpathlineto{\pgfqpoint{2.551868in}{0.623657in}}%
\pgfpathlineto{\pgfqpoint{2.554417in}{0.765853in}}%
\pgfpathlineto{\pgfqpoint{2.554841in}{0.768861in}}%
\pgfpathlineto{\pgfqpoint{2.555266in}{0.592616in}}%
\pgfpathlineto{\pgfqpoint{2.556116in}{0.603099in}}%
\pgfpathlineto{\pgfqpoint{2.557390in}{0.610040in}}%
\pgfpathlineto{\pgfqpoint{2.557815in}{0.609827in}}%
\pgfpathlineto{\pgfqpoint{2.559089in}{0.603004in}}%
\pgfpathlineto{\pgfqpoint{2.559513in}{0.599330in}}%
\pgfpathlineto{\pgfqpoint{2.560363in}{0.607704in}}%
\pgfpathlineto{\pgfqpoint{2.560788in}{0.590722in}}%
\pgfpathlineto{\pgfqpoint{2.561637in}{0.596483in}}%
\pgfpathlineto{\pgfqpoint{2.562487in}{0.598937in}}%
\pgfpathlineto{\pgfqpoint{2.562911in}{0.598809in}}%
\pgfpathlineto{\pgfqpoint{2.564185in}{0.594303in}}%
\pgfpathlineto{\pgfqpoint{2.565460in}{0.589959in}}%
\pgfpathlineto{\pgfqpoint{2.565884in}{0.591009in}}%
\pgfpathlineto{\pgfqpoint{2.566734in}{0.590719in}}%
\pgfpathlineto{\pgfqpoint{2.567159in}{0.589399in}}%
\pgfpathlineto{\pgfqpoint{2.567583in}{0.590845in}}%
\pgfpathlineto{\pgfqpoint{2.569282in}{0.596195in}}%
\pgfpathlineto{\pgfqpoint{2.570556in}{0.590130in}}%
\pgfpathlineto{\pgfqpoint{2.573954in}{0.630171in}}%
\pgfpathlineto{\pgfqpoint{2.574379in}{0.630023in}}%
\pgfpathlineto{\pgfqpoint{2.574804in}{0.628177in}}%
\pgfpathlineto{\pgfqpoint{2.577352in}{0.758211in}}%
\pgfpathlineto{\pgfqpoint{2.577777in}{0.758327in}}%
\pgfpathlineto{\pgfqpoint{2.578202in}{0.594657in}}%
\pgfpathlineto{\pgfqpoint{2.579051in}{0.603603in}}%
\pgfpathlineto{\pgfqpoint{2.581175in}{0.611255in}}%
\pgfpathlineto{\pgfqpoint{2.582449in}{0.607604in}}%
\pgfpathlineto{\pgfqpoint{2.584997in}{0.663581in}}%
\pgfpathlineto{\pgfqpoint{2.585422in}{0.590404in}}%
\pgfpathlineto{\pgfqpoint{2.586271in}{0.595103in}}%
\pgfpathlineto{\pgfqpoint{2.587121in}{0.596379in}}%
\pgfpathlineto{\pgfqpoint{2.588395in}{0.589620in}}%
\pgfpathlineto{\pgfqpoint{2.588820in}{0.594683in}}%
\pgfpathlineto{\pgfqpoint{2.592218in}{0.632209in}}%
\pgfpathlineto{\pgfqpoint{2.592642in}{0.631350in}}%
\pgfpathlineto{\pgfqpoint{2.595191in}{0.793797in}}%
\pgfpathlineto{\pgfqpoint{2.595615in}{0.798336in}}%
\pgfpathlineto{\pgfqpoint{2.596040in}{0.591918in}}%
\pgfpathlineto{\pgfqpoint{2.596890in}{0.605732in}}%
\pgfpathlineto{\pgfqpoint{2.599013in}{0.624044in}}%
\pgfpathlineto{\pgfqpoint{2.599863in}{0.620817in}}%
\pgfpathlineto{\pgfqpoint{2.600287in}{0.616977in}}%
\pgfpathlineto{\pgfqpoint{2.601986in}{0.672981in}}%
\pgfpathlineto{\pgfqpoint{2.602411in}{0.589355in}}%
\pgfpathlineto{\pgfqpoint{2.603261in}{0.596561in}}%
\pgfpathlineto{\pgfqpoint{2.603685in}{0.597985in}}%
\pgfpathlineto{\pgfqpoint{2.604110in}{0.597843in}}%
\pgfpathlineto{\pgfqpoint{2.605384in}{0.589589in}}%
\pgfpathlineto{\pgfqpoint{2.605809in}{0.593083in}}%
\pgfpathlineto{\pgfqpoint{2.608782in}{0.614934in}}%
\pgfpathlineto{\pgfqpoint{2.610056in}{0.619257in}}%
\pgfpathlineto{\pgfqpoint{2.614304in}{0.865007in}}%
\pgfpathlineto{\pgfqpoint{2.614728in}{0.591047in}}%
\pgfpathlineto{\pgfqpoint{2.615578in}{0.605458in}}%
\pgfpathlineto{\pgfqpoint{2.617701in}{0.623880in}}%
\pgfpathlineto{\pgfqpoint{2.618551in}{0.621661in}}%
\pgfpathlineto{\pgfqpoint{2.618976in}{0.618830in}}%
\pgfpathlineto{\pgfqpoint{2.621099in}{0.702506in}}%
\pgfpathlineto{\pgfqpoint{2.621524in}{0.705540in}}%
\pgfpathlineto{\pgfqpoint{2.621949in}{0.589781in}}%
\pgfpathlineto{\pgfqpoint{2.622798in}{0.594264in}}%
\pgfpathlineto{\pgfqpoint{2.623223in}{0.594838in}}%
\pgfpathlineto{\pgfqpoint{2.623648in}{0.594213in}}%
\pgfpathlineto{\pgfqpoint{2.624497in}{0.589522in}}%
\pgfpathlineto{\pgfqpoint{2.624922in}{0.592675in}}%
\pgfpathlineto{\pgfqpoint{2.628320in}{0.623923in}}%
\pgfpathlineto{\pgfqpoint{2.628744in}{0.623477in}}%
\pgfpathlineto{\pgfqpoint{2.629169in}{0.621469in}}%
\pgfpathlineto{\pgfqpoint{2.631293in}{0.708196in}}%
\pgfpathlineto{\pgfqpoint{2.631717in}{0.591811in}}%
\pgfpathlineto{\pgfqpoint{2.632567in}{0.604619in}}%
\pgfpathlineto{\pgfqpoint{2.634691in}{0.618430in}}%
\pgfpathlineto{\pgfqpoint{2.635540in}{0.615793in}}%
\pgfpathlineto{\pgfqpoint{2.635965in}{0.613194in}}%
\pgfpathlineto{\pgfqpoint{2.638513in}{0.684109in}}%
\pgfpathlineto{\pgfqpoint{2.638938in}{0.684937in}}%
\pgfpathlineto{\pgfqpoint{2.639363in}{0.590012in}}%
\pgfpathlineto{\pgfqpoint{2.640212in}{0.590588in}}%
\pgfpathlineto{\pgfqpoint{2.640637in}{0.589521in}}%
\pgfpathlineto{\pgfqpoint{2.641061in}{0.590656in}}%
\pgfpathlineto{\pgfqpoint{2.643185in}{0.599949in}}%
\pgfpathlineto{\pgfqpoint{2.643610in}{0.599716in}}%
\pgfpathlineto{\pgfqpoint{2.644884in}{0.592815in}}%
\pgfpathlineto{\pgfqpoint{2.645309in}{0.589407in}}%
\pgfpathlineto{\pgfqpoint{2.645733in}{0.593548in}}%
\pgfpathlineto{\pgfqpoint{2.648282in}{0.608247in}}%
\pgfpathlineto{\pgfqpoint{2.649131in}{0.606109in}}%
\pgfpathlineto{\pgfqpoint{2.649556in}{0.603876in}}%
\pgfpathlineto{\pgfqpoint{2.651255in}{0.629307in}}%
\pgfpathlineto{\pgfqpoint{2.651680in}{0.593174in}}%
\pgfpathlineto{\pgfqpoint{2.652529in}{0.602645in}}%
\pgfpathlineto{\pgfqpoint{2.654653in}{0.619276in}}%
\pgfpathlineto{\pgfqpoint{2.655078in}{0.619029in}}%
\pgfpathlineto{\pgfqpoint{2.655927in}{0.613394in}}%
\pgfpathlineto{\pgfqpoint{2.657201in}{0.649684in}}%
\pgfpathlineto{\pgfqpoint{2.657626in}{0.592063in}}%
\pgfpathlineto{\pgfqpoint{2.658475in}{0.606566in}}%
\pgfpathlineto{\pgfqpoint{2.659750in}{0.617757in}}%
\pgfpathlineto{\pgfqpoint{2.660174in}{0.617361in}}%
\pgfpathlineto{\pgfqpoint{2.661448in}{0.604340in}}%
\pgfpathlineto{\pgfqpoint{2.662723in}{0.595496in}}%
\pgfpathlineto{\pgfqpoint{2.664422in}{0.610146in}}%
\pgfpathlineto{\pgfqpoint{2.665271in}{0.606394in}}%
\pgfpathlineto{\pgfqpoint{2.666970in}{0.591856in}}%
\pgfpathlineto{\pgfqpoint{2.667395in}{0.596536in}}%
\pgfpathlineto{\pgfqpoint{2.668669in}{0.603261in}}%
\pgfpathlineto{\pgfqpoint{2.669094in}{0.602541in}}%
\pgfpathlineto{\pgfqpoint{2.670793in}{0.589491in}}%
\pgfpathlineto{\pgfqpoint{2.671217in}{0.593907in}}%
\pgfpathlineto{\pgfqpoint{2.673766in}{0.608442in}}%
\pgfpathlineto{\pgfqpoint{2.674615in}{0.606769in}}%
\pgfpathlineto{\pgfqpoint{2.675040in}{0.604575in}}%
\pgfpathlineto{\pgfqpoint{2.676314in}{0.627894in}}%
\pgfpathlineto{\pgfqpoint{2.676739in}{0.592138in}}%
\pgfpathlineto{\pgfqpoint{2.677588in}{0.605862in}}%
\pgfpathlineto{\pgfqpoint{2.680561in}{0.639907in}}%
\pgfpathlineto{\pgfqpoint{2.680986in}{0.638311in}}%
\pgfpathlineto{\pgfqpoint{2.683110in}{0.787989in}}%
\pgfpathlineto{\pgfqpoint{2.683534in}{0.789833in}}%
\pgfpathlineto{\pgfqpoint{2.683959in}{0.596856in}}%
\pgfpathlineto{\pgfqpoint{2.684809in}{0.611796in}}%
\pgfpathlineto{\pgfqpoint{2.686083in}{0.619721in}}%
\pgfpathlineto{\pgfqpoint{2.686508in}{0.618567in}}%
\pgfpathlineto{\pgfqpoint{2.688206in}{0.602725in}}%
\pgfpathlineto{\pgfqpoint{2.688631in}{0.611316in}}%
\pgfpathlineto{\pgfqpoint{2.689056in}{0.615263in}}%
\pgfpathlineto{\pgfqpoint{2.689481in}{0.589289in}}%
\pgfpathlineto{\pgfqpoint{2.690330in}{0.594597in}}%
\pgfpathlineto{\pgfqpoint{2.690755in}{0.595494in}}%
\pgfpathlineto{\pgfqpoint{2.691180in}{0.595084in}}%
\pgfpathlineto{\pgfqpoint{2.692029in}{0.590382in}}%
\pgfpathlineto{\pgfqpoint{2.692454in}{0.592058in}}%
\pgfpathlineto{\pgfqpoint{2.695852in}{0.623073in}}%
\pgfpathlineto{\pgfqpoint{2.696701in}{0.618053in}}%
\pgfpathlineto{\pgfqpoint{2.698400in}{0.675973in}}%
\pgfpathlineto{\pgfqpoint{2.698825in}{0.591096in}}%
\pgfpathlineto{\pgfqpoint{2.699674in}{0.601300in}}%
\pgfpathlineto{\pgfqpoint{2.700948in}{0.606464in}}%
\pgfpathlineto{\pgfqpoint{2.701798in}{0.603495in}}%
\pgfpathlineto{\pgfqpoint{2.703921in}{0.589959in}}%
\pgfpathlineto{\pgfqpoint{2.704346in}{0.592108in}}%
\pgfpathlineto{\pgfqpoint{2.705196in}{0.594004in}}%
\pgfpathlineto{\pgfqpoint{2.705620in}{0.593549in}}%
\pgfpathlineto{\pgfqpoint{2.706470in}{0.589693in}}%
\pgfpathlineto{\pgfqpoint{2.706895in}{0.591949in}}%
\pgfpathlineto{\pgfqpoint{2.710292in}{0.616154in}}%
\pgfpathlineto{\pgfqpoint{2.711142in}{0.612964in}}%
\pgfpathlineto{\pgfqpoint{2.713265in}{0.680190in}}%
\pgfpathlineto{\pgfqpoint{2.714115in}{0.683704in}}%
\pgfpathlineto{\pgfqpoint{2.714540in}{0.590556in}}%
\pgfpathlineto{\pgfqpoint{2.715389in}{0.592480in}}%
\pgfpathlineto{\pgfqpoint{2.717088in}{0.592304in}}%
\pgfpathlineto{\pgfqpoint{2.718362in}{0.592891in}}%
\pgfpathlineto{\pgfqpoint{2.718787in}{0.593658in}}%
\pgfpathlineto{\pgfqpoint{2.720911in}{0.635171in}}%
\pgfpathlineto{\pgfqpoint{2.725158in}{0.726573in}}%
\pgfpathlineto{\pgfqpoint{2.725583in}{0.589337in}}%
\pgfpathlineto{\pgfqpoint{2.726432in}{0.590444in}}%
\pgfpathlineto{\pgfqpoint{2.727282in}{0.589740in}}%
\pgfpathlineto{\pgfqpoint{2.731529in}{0.609661in}}%
\pgfpathlineto{\pgfqpoint{2.734077in}{0.681898in}}%
\pgfpathlineto{\pgfqpoint{2.734502in}{0.591244in}}%
\pgfpathlineto{\pgfqpoint{2.735351in}{0.597940in}}%
\pgfpathlineto{\pgfqpoint{2.736201in}{0.600101in}}%
\pgfpathlineto{\pgfqpoint{2.736626in}{0.599183in}}%
\pgfpathlineto{\pgfqpoint{2.737900in}{0.589952in}}%
\pgfpathlineto{\pgfqpoint{2.738324in}{0.592636in}}%
\pgfpathlineto{\pgfqpoint{2.740023in}{0.604003in}}%
\pgfpathlineto{\pgfqpoint{2.740448in}{0.603637in}}%
\pgfpathlineto{\pgfqpoint{2.742147in}{0.591062in}}%
\pgfpathlineto{\pgfqpoint{2.742572in}{0.597361in}}%
\pgfpathlineto{\pgfqpoint{2.745120in}{0.620339in}}%
\pgfpathlineto{\pgfqpoint{2.745970in}{0.617045in}}%
\pgfpathlineto{\pgfqpoint{2.746394in}{0.613822in}}%
\pgfpathlineto{\pgfqpoint{2.748943in}{0.692806in}}%
\pgfpathlineto{\pgfqpoint{2.751066in}{0.747798in}}%
\pgfpathlineto{\pgfqpoint{2.756588in}{0.956920in}}%
\pgfpathlineto{\pgfqpoint{2.757013in}{0.591003in}}%
\pgfpathlineto{\pgfqpoint{2.757862in}{0.598597in}}%
\pgfpathlineto{\pgfqpoint{2.759986in}{0.607416in}}%
\pgfpathlineto{\pgfqpoint{2.761260in}{0.603822in}}%
\pgfpathlineto{\pgfqpoint{2.765507in}{0.755268in}}%
\pgfpathlineto{\pgfqpoint{2.771878in}{1.138181in}}%
\pgfpathlineto{\pgfqpoint{2.775276in}{1.169302in}}%
\pgfpathlineto{\pgfqpoint{2.775701in}{0.591234in}}%
\pgfpathlineto{\pgfqpoint{2.776550in}{0.598431in}}%
\pgfpathlineto{\pgfqpoint{2.779099in}{0.612618in}}%
\pgfpathlineto{\pgfqpoint{2.779948in}{0.610121in}}%
\pgfpathlineto{\pgfqpoint{2.790566in}{1.035248in}}%
\pgfpathlineto{\pgfqpoint{2.791416in}{1.041637in}}%
\pgfpathlineto{\pgfqpoint{2.791840in}{0.590601in}}%
\pgfpathlineto{\pgfqpoint{2.792690in}{0.595233in}}%
\pgfpathlineto{\pgfqpoint{2.793964in}{0.597200in}}%
\pgfpathlineto{\pgfqpoint{2.796088in}{0.593121in}}%
\pgfpathlineto{\pgfqpoint{2.798211in}{0.619923in}}%
\pgfpathlineto{\pgfqpoint{2.802883in}{0.708178in}}%
\pgfpathlineto{\pgfqpoint{2.803308in}{0.591522in}}%
\pgfpathlineto{\pgfqpoint{2.804158in}{0.595207in}}%
\pgfpathlineto{\pgfqpoint{2.805007in}{0.596295in}}%
\pgfpathlineto{\pgfqpoint{2.805432in}{0.595668in}}%
\pgfpathlineto{\pgfqpoint{2.806706in}{0.589399in}}%
\pgfpathlineto{\pgfqpoint{2.807131in}{0.592166in}}%
\pgfpathlineto{\pgfqpoint{2.811378in}{0.619404in}}%
\pgfpathlineto{\pgfqpoint{2.816475in}{0.928525in}}%
\pgfpathlineto{\pgfqpoint{2.817324in}{0.932965in}}%
\pgfpathlineto{\pgfqpoint{2.817749in}{0.592592in}}%
\pgfpathlineto{\pgfqpoint{2.818598in}{0.597229in}}%
\pgfpathlineto{\pgfqpoint{2.819448in}{0.598571in}}%
\pgfpathlineto{\pgfqpoint{2.819873in}{0.598342in}}%
\pgfpathlineto{\pgfqpoint{2.821571in}{0.597276in}}%
\pgfpathlineto{\pgfqpoint{2.821996in}{0.598072in}}%
\pgfpathlineto{\pgfqpoint{2.824120in}{0.672442in}}%
\pgfpathlineto{\pgfqpoint{2.827942in}{0.924885in}}%
\pgfpathlineto{\pgfqpoint{2.836012in}{1.487458in}}%
\pgfpathlineto{\pgfqpoint{2.836437in}{1.489659in}}%
\pgfpathlineto{\pgfqpoint{2.836862in}{0.591934in}}%
\pgfpathlineto{\pgfqpoint{2.837711in}{0.602963in}}%
\pgfpathlineto{\pgfqpoint{2.840684in}{0.633291in}}%
\pgfpathlineto{\pgfqpoint{2.841109in}{0.633914in}}%
\pgfpathlineto{\pgfqpoint{2.845781in}{1.021204in}}%
\pgfpathlineto{\pgfqpoint{2.850453in}{1.272158in}}%
\pgfpathlineto{\pgfqpoint{2.855125in}{1.383815in}}%
\pgfpathlineto{\pgfqpoint{2.855975in}{1.390691in}}%
\pgfpathlineto{\pgfqpoint{2.856399in}{0.590036in}}%
\pgfpathlineto{\pgfqpoint{2.857249in}{0.597156in}}%
\pgfpathlineto{\pgfqpoint{2.860647in}{0.628414in}}%
\pgfpathlineto{\pgfqpoint{2.867442in}{1.199626in}}%
\pgfpathlineto{\pgfqpoint{2.870415in}{1.302707in}}%
\pgfpathlineto{\pgfqpoint{2.870840in}{1.305393in}}%
\pgfpathlineto{\pgfqpoint{2.871265in}{0.589980in}}%
\pgfpathlineto{\pgfqpoint{2.872114in}{0.595959in}}%
\pgfpathlineto{\pgfqpoint{2.873388in}{0.600091in}}%
\pgfpathlineto{\pgfqpoint{2.873813in}{0.599983in}}%
\pgfpathlineto{\pgfqpoint{2.875512in}{0.595471in}}%
\pgfpathlineto{\pgfqpoint{2.878910in}{0.644404in}}%
\pgfpathlineto{\pgfqpoint{2.880609in}{0.725106in}}%
\pgfpathlineto{\pgfqpoint{2.884007in}{1.034161in}}%
\pgfpathlineto{\pgfqpoint{2.886980in}{1.232313in}}%
\pgfpathlineto{\pgfqpoint{2.888254in}{1.247073in}}%
\pgfpathlineto{\pgfqpoint{2.888679in}{0.591005in}}%
\pgfpathlineto{\pgfqpoint{2.889528in}{0.593465in}}%
\pgfpathlineto{\pgfqpoint{2.890378in}{0.592160in}}%
\pgfpathlineto{\pgfqpoint{2.891227in}{0.590045in}}%
\pgfpathlineto{\pgfqpoint{2.894200in}{0.601566in}}%
\pgfpathlineto{\pgfqpoint{2.895050in}{0.599337in}}%
\pgfpathlineto{\pgfqpoint{2.896324in}{0.589774in}}%
\pgfpathlineto{\pgfqpoint{2.895899in}{0.600717in}}%
\pgfpathlineto{\pgfqpoint{2.896749in}{0.595152in}}%
\pgfpathlineto{\pgfqpoint{2.900571in}{0.638442in}}%
\pgfpathlineto{\pgfqpoint{2.904394in}{0.974745in}}%
\pgfpathlineto{\pgfqpoint{2.907791in}{1.084815in}}%
\pgfpathlineto{\pgfqpoint{2.909490in}{1.101145in}}%
\pgfpathlineto{\pgfqpoint{2.909915in}{0.590106in}}%
\pgfpathlineto{\pgfqpoint{2.910765in}{0.594785in}}%
\pgfpathlineto{\pgfqpoint{2.914162in}{0.610383in}}%
\pgfpathlineto{\pgfqpoint{2.922232in}{1.059605in}}%
\pgfpathlineto{\pgfqpoint{2.925205in}{1.228365in}}%
\pgfpathlineto{\pgfqpoint{2.925630in}{1.228480in}}%
\pgfpathlineto{\pgfqpoint{2.926055in}{0.595470in}}%
\pgfpathlineto{\pgfqpoint{2.926904in}{0.606279in}}%
\pgfpathlineto{\pgfqpoint{2.928603in}{0.615702in}}%
\pgfpathlineto{\pgfqpoint{2.929453in}{0.614220in}}%
\pgfpathlineto{\pgfqpoint{2.930302in}{0.610220in}}%
\pgfpathlineto{\pgfqpoint{2.933700in}{0.716109in}}%
\pgfpathlineto{\pgfqpoint{2.939221in}{0.869380in}}%
\pgfpathlineto{\pgfqpoint{2.939646in}{0.591760in}}%
\pgfpathlineto{\pgfqpoint{2.940496in}{0.595218in}}%
\pgfpathlineto{\pgfqpoint{2.941345in}{0.595299in}}%
\pgfpathlineto{\pgfqpoint{2.943044in}{0.589761in}}%
\pgfpathlineto{\pgfqpoint{2.943469in}{0.591661in}}%
\pgfpathlineto{\pgfqpoint{2.945168in}{0.595408in}}%
\pgfpathlineto{\pgfqpoint{2.946442in}{0.594155in}}%
\pgfpathlineto{\pgfqpoint{2.947291in}{0.592707in}}%
\pgfpathlineto{\pgfqpoint{2.950689in}{0.608925in}}%
\pgfpathlineto{\pgfqpoint{2.951963in}{0.612177in}}%
\pgfpathlineto{\pgfqpoint{2.952388in}{0.589616in}}%
\pgfpathlineto{\pgfqpoint{2.953238in}{0.592393in}}%
\pgfpathlineto{\pgfqpoint{2.956635in}{0.607793in}}%
\pgfpathlineto{\pgfqpoint{2.963006in}{0.856052in}}%
\pgfpathlineto{\pgfqpoint{2.965130in}{0.878265in}}%
\pgfpathlineto{\pgfqpoint{2.965555in}{0.879065in}}%
\pgfpathlineto{\pgfqpoint{2.965979in}{0.589548in}}%
\pgfpathlineto{\pgfqpoint{2.966829in}{0.591647in}}%
\pgfpathlineto{\pgfqpoint{2.970227in}{0.601451in}}%
\pgfpathlineto{\pgfqpoint{2.973200in}{0.736161in}}%
\pgfpathlineto{\pgfqpoint{2.977022in}{0.880684in}}%
\pgfpathlineto{\pgfqpoint{2.979995in}{0.972739in}}%
\pgfpathlineto{\pgfqpoint{2.982544in}{1.037432in}}%
\pgfpathlineto{\pgfqpoint{2.982969in}{0.589570in}}%
\pgfpathlineto{\pgfqpoint{2.983818in}{0.596797in}}%
\pgfpathlineto{\pgfqpoint{2.986791in}{0.610628in}}%
\pgfpathlineto{\pgfqpoint{2.987216in}{0.611485in}}%
\pgfpathlineto{\pgfqpoint{2.992313in}{0.841930in}}%
\pgfpathlineto{\pgfqpoint{2.994436in}{0.863155in}}%
\pgfpathlineto{\pgfqpoint{2.995286in}{0.864108in}}%
\pgfpathlineto{\pgfqpoint{2.995710in}{0.589564in}}%
\pgfpathlineto{\pgfqpoint{2.996560in}{0.590521in}}%
\pgfpathlineto{\pgfqpoint{2.999958in}{0.593799in}}%
\pgfpathlineto{\pgfqpoint{3.002506in}{0.637524in}}%
\pgfpathlineto{\pgfqpoint{3.010576in}{0.832716in}}%
\pgfpathlineto{\pgfqpoint{3.011001in}{0.589732in}}%
\pgfpathlineto{\pgfqpoint{3.011850in}{0.594795in}}%
\pgfpathlineto{\pgfqpoint{3.014823in}{0.607688in}}%
\pgfpathlineto{\pgfqpoint{3.015248in}{0.607821in}}%
\pgfpathlineto{\pgfqpoint{3.018221in}{0.700341in}}%
\pgfpathlineto{\pgfqpoint{3.019071in}{0.705814in}}%
\pgfpathlineto{\pgfqpoint{3.019495in}{0.589723in}}%
\pgfpathlineto{\pgfqpoint{3.020345in}{0.592344in}}%
\pgfpathlineto{\pgfqpoint{3.021194in}{0.592493in}}%
\pgfpathlineto{\pgfqpoint{3.022468in}{0.589751in}}%
\pgfpathlineto{\pgfqpoint{3.025442in}{0.599434in}}%
\pgfpathlineto{\pgfqpoint{3.026291in}{0.598659in}}%
\pgfpathlineto{\pgfqpoint{3.026716in}{0.597558in}}%
\pgfpathlineto{\pgfqpoint{3.028415in}{0.612207in}}%
\pgfpathlineto{\pgfqpoint{3.028839in}{0.591494in}}%
\pgfpathlineto{\pgfqpoint{3.029689in}{0.596350in}}%
\pgfpathlineto{\pgfqpoint{3.032237in}{0.603882in}}%
\pgfpathlineto{\pgfqpoint{3.033087in}{0.604973in}}%
\pgfpathlineto{\pgfqpoint{3.037334in}{0.812230in}}%
\pgfpathlineto{\pgfqpoint{3.041581in}{0.988443in}}%
\pgfpathlineto{\pgfqpoint{3.043280in}{1.010773in}}%
\pgfpathlineto{\pgfqpoint{3.043705in}{0.592416in}}%
\pgfpathlineto{\pgfqpoint{3.044554in}{0.600594in}}%
\pgfpathlineto{\pgfqpoint{3.046678in}{0.612927in}}%
\pgfpathlineto{\pgfqpoint{3.047527in}{0.612774in}}%
\pgfpathlineto{\pgfqpoint{3.047952in}{0.611923in}}%
\pgfpathlineto{\pgfqpoint{3.060269in}{1.195127in}}%
\pgfpathlineto{\pgfqpoint{3.061968in}{1.214401in}}%
\pgfpathlineto{\pgfqpoint{3.062393in}{0.591491in}}%
\pgfpathlineto{\pgfqpoint{3.063242in}{0.597828in}}%
\pgfpathlineto{\pgfqpoint{3.066640in}{0.619082in}}%
\pgfpathlineto{\pgfqpoint{3.078957in}{1.463676in}}%
\pgfpathlineto{\pgfqpoint{3.079382in}{0.589515in}}%
\pgfpathlineto{\pgfqpoint{3.080232in}{0.599607in}}%
\pgfpathlineto{\pgfqpoint{3.083629in}{0.634459in}}%
\pgfpathlineto{\pgfqpoint{3.091275in}{1.382728in}}%
\pgfpathlineto{\pgfqpoint{3.092549in}{1.397726in}}%
\pgfpathlineto{\pgfqpoint{3.092973in}{0.591607in}}%
\pgfpathlineto{\pgfqpoint{3.093823in}{0.596616in}}%
\pgfpathlineto{\pgfqpoint{3.095522in}{0.600451in}}%
\pgfpathlineto{\pgfqpoint{3.096796in}{0.599601in}}%
\pgfpathlineto{\pgfqpoint{3.097221in}{0.598957in}}%
\pgfpathlineto{\pgfqpoint{3.101893in}{0.713131in}}%
\pgfpathlineto{\pgfqpoint{3.104441in}{0.880809in}}%
\pgfpathlineto{\pgfqpoint{3.109538in}{1.215299in}}%
\pgfpathlineto{\pgfqpoint{3.111237in}{1.230391in}}%
\pgfpathlineto{\pgfqpoint{3.112511in}{1.232159in}}%
\pgfpathlineto{\pgfqpoint{3.113785in}{1.241239in}}%
\pgfpathlineto{\pgfqpoint{3.115909in}{1.258552in}}%
\pgfpathlineto{\pgfqpoint{3.116334in}{0.590248in}}%
\pgfpathlineto{\pgfqpoint{3.117183in}{0.595487in}}%
\pgfpathlineto{\pgfqpoint{3.120581in}{0.612129in}}%
\pgfpathlineto{\pgfqpoint{3.128651in}{1.015096in}}%
\pgfpathlineto{\pgfqpoint{3.132049in}{1.075043in}}%
\pgfpathlineto{\pgfqpoint{3.132473in}{1.076176in}}%
\pgfpathlineto{\pgfqpoint{3.132898in}{0.590184in}}%
\pgfpathlineto{\pgfqpoint{3.133748in}{0.594276in}}%
\pgfpathlineto{\pgfqpoint{3.136721in}{0.602398in}}%
\pgfpathlineto{\pgfqpoint{3.137145in}{0.603111in}}%
\pgfpathlineto{\pgfqpoint{3.145640in}{0.865358in}}%
\pgfpathlineto{\pgfqpoint{3.148188in}{0.908138in}}%
\pgfpathlineto{\pgfqpoint{3.150312in}{0.994774in}}%
\pgfpathlineto{\pgfqpoint{3.154135in}{1.151360in}}%
\pgfpathlineto{\pgfqpoint{3.154559in}{0.594596in}}%
\pgfpathlineto{\pgfqpoint{3.155409in}{0.607567in}}%
\pgfpathlineto{\pgfqpoint{3.157532in}{0.626148in}}%
\pgfpathlineto{\pgfqpoint{3.157957in}{0.626567in}}%
\pgfpathlineto{\pgfqpoint{3.158382in}{0.626011in}}%
\pgfpathlineto{\pgfqpoint{3.158807in}{0.624627in}}%
\pgfpathlineto{\pgfqpoint{3.162204in}{0.809259in}}%
\pgfpathlineto{\pgfqpoint{3.164753in}{0.845922in}}%
\pgfpathlineto{\pgfqpoint{3.169425in}{0.876277in}}%
\pgfpathlineto{\pgfqpoint{3.169850in}{0.876404in}}%
\pgfpathlineto{\pgfqpoint{3.170274in}{0.589465in}}%
\pgfpathlineto{\pgfqpoint{3.171124in}{0.590174in}}%
\pgfpathlineto{\pgfqpoint{3.172823in}{0.593235in}}%
\pgfpathlineto{\pgfqpoint{3.174522in}{0.599952in}}%
\pgfpathlineto{\pgfqpoint{3.178344in}{0.744115in}}%
\pgfpathlineto{\pgfqpoint{3.182167in}{0.849281in}}%
\pgfpathlineto{\pgfqpoint{3.185140in}{0.918386in}}%
\pgfpathlineto{\pgfqpoint{3.188962in}{1.023257in}}%
\pgfpathlineto{\pgfqpoint{3.189387in}{0.590931in}}%
\pgfpathlineto{\pgfqpoint{3.190237in}{0.598589in}}%
\pgfpathlineto{\pgfqpoint{3.192785in}{0.611483in}}%
\pgfpathlineto{\pgfqpoint{3.193634in}{0.611830in}}%
\pgfpathlineto{\pgfqpoint{3.198731in}{0.823412in}}%
\pgfpathlineto{\pgfqpoint{3.204253in}{1.031768in}}%
\pgfpathlineto{\pgfqpoint{3.204677in}{0.589864in}}%
\pgfpathlineto{\pgfqpoint{3.205527in}{0.597614in}}%
\pgfpathlineto{\pgfqpoint{3.208075in}{0.609220in}}%
\pgfpathlineto{\pgfqpoint{3.208925in}{0.609075in}}%
\pgfpathlineto{\pgfqpoint{3.221242in}{1.124999in}}%
\pgfpathlineto{\pgfqpoint{3.223365in}{1.152735in}}%
\pgfpathlineto{\pgfqpoint{3.223790in}{0.592007in}}%
\pgfpathlineto{\pgfqpoint{3.224640in}{0.598311in}}%
\pgfpathlineto{\pgfqpoint{3.228037in}{0.624390in}}%
\pgfpathlineto{\pgfqpoint{3.234408in}{1.104543in}}%
\pgfpathlineto{\pgfqpoint{3.236957in}{1.155385in}}%
\pgfpathlineto{\pgfqpoint{3.238231in}{1.162311in}}%
\pgfpathlineto{\pgfqpoint{3.238656in}{0.589281in}}%
\pgfpathlineto{\pgfqpoint{3.239505in}{0.592450in}}%
\pgfpathlineto{\pgfqpoint{3.242903in}{0.608847in}}%
\pgfpathlineto{\pgfqpoint{3.247150in}{0.867302in}}%
\pgfpathlineto{\pgfqpoint{3.253521in}{1.292503in}}%
\pgfpathlineto{\pgfqpoint{3.253946in}{1.294241in}}%
\pgfpathlineto{\pgfqpoint{3.254371in}{0.591705in}}%
\pgfpathlineto{\pgfqpoint{3.255220in}{0.599367in}}%
\pgfpathlineto{\pgfqpoint{3.257768in}{0.611957in}}%
\pgfpathlineto{\pgfqpoint{3.258618in}{0.612595in}}%
\pgfpathlineto{\pgfqpoint{3.269236in}{1.145807in}}%
\pgfpathlineto{\pgfqpoint{3.270086in}{1.153816in}}%
\pgfpathlineto{\pgfqpoint{3.270510in}{0.589668in}}%
\pgfpathlineto{\pgfqpoint{3.271360in}{0.595341in}}%
\pgfpathlineto{\pgfqpoint{3.273908in}{0.604377in}}%
\pgfpathlineto{\pgfqpoint{3.274758in}{0.604001in}}%
\pgfpathlineto{\pgfqpoint{3.281129in}{0.847564in}}%
\pgfpathlineto{\pgfqpoint{3.284526in}{0.984640in}}%
\pgfpathlineto{\pgfqpoint{3.286225in}{0.998575in}}%
\pgfpathlineto{\pgfqpoint{3.286650in}{0.591000in}}%
\pgfpathlineto{\pgfqpoint{3.287500in}{0.594449in}}%
\pgfpathlineto{\pgfqpoint{3.290048in}{0.600200in}}%
\pgfpathlineto{\pgfqpoint{3.290897in}{0.599646in}}%
\pgfpathlineto{\pgfqpoint{3.293446in}{0.638020in}}%
\pgfpathlineto{\pgfqpoint{3.295994in}{0.655809in}}%
\pgfpathlineto{\pgfqpoint{3.298118in}{0.703618in}}%
\pgfpathlineto{\pgfqpoint{3.305338in}{0.895690in}}%
\pgfpathlineto{\pgfqpoint{3.305763in}{0.896544in}}%
\pgfpathlineto{\pgfqpoint{3.306188in}{0.589357in}}%
\pgfpathlineto{\pgfqpoint{3.307037in}{0.590592in}}%
\pgfpathlineto{\pgfqpoint{3.309585in}{0.593768in}}%
\pgfpathlineto{\pgfqpoint{3.310435in}{0.596255in}}%
\pgfpathlineto{\pgfqpoint{3.313408in}{0.674846in}}%
\pgfpathlineto{\pgfqpoint{3.323177in}{0.925081in}}%
\pgfpathlineto{\pgfqpoint{3.325725in}{0.946401in}}%
\pgfpathlineto{\pgfqpoint{3.326150in}{0.947305in}}%
\pgfpathlineto{\pgfqpoint{3.326575in}{0.589304in}}%
\pgfpathlineto{\pgfqpoint{3.327424in}{0.591849in}}%
\pgfpathlineto{\pgfqpoint{3.330822in}{0.601724in}}%
\pgfpathlineto{\pgfqpoint{3.335494in}{0.777656in}}%
\pgfpathlineto{\pgfqpoint{3.341440in}{1.030710in}}%
\pgfpathlineto{\pgfqpoint{3.342714in}{1.041763in}}%
\pgfpathlineto{\pgfqpoint{3.343139in}{0.590545in}}%
\pgfpathlineto{\pgfqpoint{3.343989in}{0.595904in}}%
\pgfpathlineto{\pgfqpoint{3.346962in}{0.608949in}}%
\pgfpathlineto{\pgfqpoint{3.347386in}{0.609385in}}%
\pgfpathlineto{\pgfqpoint{3.357155in}{1.025590in}}%
\pgfpathlineto{\pgfqpoint{3.359704in}{1.060945in}}%
\pgfpathlineto{\pgfqpoint{3.361402in}{1.069293in}}%
\pgfpathlineto{\pgfqpoint{3.361827in}{0.589204in}}%
\pgfpathlineto{\pgfqpoint{3.362677in}{0.592157in}}%
\pgfpathlineto{\pgfqpoint{3.366074in}{0.611343in}}%
\pgfpathlineto{\pgfqpoint{3.374994in}{1.077549in}}%
\pgfpathlineto{\pgfqpoint{3.375419in}{1.078537in}}%
\pgfpathlineto{\pgfqpoint{3.375843in}{0.590109in}}%
\pgfpathlineto{\pgfqpoint{3.376693in}{0.593282in}}%
\pgfpathlineto{\pgfqpoint{3.379666in}{0.598717in}}%
\pgfpathlineto{\pgfqpoint{3.380091in}{0.599129in}}%
\pgfpathlineto{\pgfqpoint{3.383488in}{0.708179in}}%
\pgfpathlineto{\pgfqpoint{3.388160in}{0.871906in}}%
\pgfpathlineto{\pgfqpoint{3.393682in}{0.967392in}}%
\pgfpathlineto{\pgfqpoint{3.397080in}{1.031955in}}%
\pgfpathlineto{\pgfqpoint{3.397504in}{1.032523in}}%
\pgfpathlineto{\pgfqpoint{3.397929in}{0.590614in}}%
\pgfpathlineto{\pgfqpoint{3.398779in}{0.594062in}}%
\pgfpathlineto{\pgfqpoint{3.400902in}{0.597528in}}%
\pgfpathlineto{\pgfqpoint{3.402176in}{0.597036in}}%
\pgfpathlineto{\pgfqpoint{3.405574in}{0.674147in}}%
\pgfpathlineto{\pgfqpoint{3.409822in}{0.853970in}}%
\pgfpathlineto{\pgfqpoint{3.413219in}{0.962641in}}%
\pgfpathlineto{\pgfqpoint{3.414494in}{0.972044in}}%
\pgfpathlineto{\pgfqpoint{3.414918in}{0.589681in}}%
\pgfpathlineto{\pgfqpoint{3.415768in}{0.592253in}}%
\pgfpathlineto{\pgfqpoint{3.419166in}{0.598102in}}%
\pgfpathlineto{\pgfqpoint{3.422139in}{0.689211in}}%
\pgfpathlineto{\pgfqpoint{3.426386in}{0.806665in}}%
\pgfpathlineto{\pgfqpoint{3.435305in}{0.942986in}}%
\pgfpathlineto{\pgfqpoint{3.435730in}{0.590654in}}%
\pgfpathlineto{\pgfqpoint{3.436580in}{0.594092in}}%
\pgfpathlineto{\pgfqpoint{3.439553in}{0.601580in}}%
\pgfpathlineto{\pgfqpoint{3.439977in}{0.601658in}}%
\pgfpathlineto{\pgfqpoint{3.444649in}{0.707831in}}%
\pgfpathlineto{\pgfqpoint{3.447198in}{0.727106in}}%
\pgfpathlineto{\pgfqpoint{3.447622in}{0.727608in}}%
\pgfpathlineto{\pgfqpoint{3.448047in}{0.589242in}}%
\pgfpathlineto{\pgfqpoint{3.448897in}{0.589839in}}%
\pgfpathlineto{\pgfqpoint{3.450596in}{0.590281in}}%
\pgfpathlineto{\pgfqpoint{3.453993in}{0.596823in}}%
\pgfpathlineto{\pgfqpoint{3.454418in}{0.596302in}}%
\pgfpathlineto{\pgfqpoint{3.456117in}{0.611553in}}%
\pgfpathlineto{\pgfqpoint{3.456542in}{0.589965in}}%
\pgfpathlineto{\pgfqpoint{3.457391in}{0.593708in}}%
\pgfpathlineto{\pgfqpoint{3.459515in}{0.597376in}}%
\pgfpathlineto{\pgfqpoint{3.460789in}{0.595025in}}%
\pgfpathlineto{\pgfqpoint{3.462913in}{0.607636in}}%
\pgfpathlineto{\pgfqpoint{3.465036in}{0.609239in}}%
\pgfpathlineto{\pgfqpoint{3.466311in}{0.616007in}}%
\pgfpathlineto{\pgfqpoint{3.468434in}{0.644442in}}%
\pgfpathlineto{\pgfqpoint{3.473106in}{0.713387in}}%
\pgfpathlineto{\pgfqpoint{3.473531in}{0.590750in}}%
\pgfpathlineto{\pgfqpoint{3.474380in}{0.593440in}}%
\pgfpathlineto{\pgfqpoint{3.475230in}{0.594395in}}%
\pgfpathlineto{\pgfqpoint{3.475655in}{0.594081in}}%
\pgfpathlineto{\pgfqpoint{3.477354in}{0.590183in}}%
\pgfpathlineto{\pgfqpoint{3.477778in}{0.592311in}}%
\pgfpathlineto{\pgfqpoint{3.480327in}{0.599658in}}%
\pgfpathlineto{\pgfqpoint{3.481601in}{0.596257in}}%
\pgfpathlineto{\pgfqpoint{3.482875in}{0.604206in}}%
\pgfpathlineto{\pgfqpoint{3.483300in}{0.591409in}}%
\pgfpathlineto{\pgfqpoint{3.484149in}{0.595828in}}%
\pgfpathlineto{\pgfqpoint{3.486273in}{0.600926in}}%
\pgfpathlineto{\pgfqpoint{3.487547in}{0.600302in}}%
\pgfpathlineto{\pgfqpoint{3.494767in}{0.755161in}}%
\pgfpathlineto{\pgfqpoint{3.496466in}{0.769702in}}%
\pgfpathlineto{\pgfqpoint{3.496891in}{0.589191in}}%
\pgfpathlineto{\pgfqpoint{3.497741in}{0.591376in}}%
\pgfpathlineto{\pgfqpoint{3.499439in}{0.592235in}}%
\pgfpathlineto{\pgfqpoint{3.501138in}{0.590611in}}%
\pgfpathlineto{\pgfqpoint{3.501563in}{0.591828in}}%
\pgfpathlineto{\pgfqpoint{3.503262in}{0.599079in}}%
\pgfpathlineto{\pgfqpoint{3.504961in}{0.621388in}}%
\pgfpathlineto{\pgfqpoint{3.507509in}{0.697900in}}%
\pgfpathlineto{\pgfqpoint{3.512606in}{0.855378in}}%
\pgfpathlineto{\pgfqpoint{3.513880in}{0.861266in}}%
\pgfpathlineto{\pgfqpoint{3.514305in}{0.590600in}}%
\pgfpathlineto{\pgfqpoint{3.515154in}{0.593372in}}%
\pgfpathlineto{\pgfqpoint{3.518128in}{0.598886in}}%
\pgfpathlineto{\pgfqpoint{3.518552in}{0.599323in}}%
\pgfpathlineto{\pgfqpoint{3.523224in}{0.691061in}}%
\pgfpathlineto{\pgfqpoint{3.524074in}{0.693748in}}%
\pgfpathlineto{\pgfqpoint{3.524499in}{0.589794in}}%
\pgfpathlineto{\pgfqpoint{3.525348in}{0.591822in}}%
\pgfpathlineto{\pgfqpoint{3.527047in}{0.592659in}}%
\pgfpathlineto{\pgfqpoint{3.528746in}{0.589398in}}%
\pgfpathlineto{\pgfqpoint{3.529595in}{0.591096in}}%
\pgfpathlineto{\pgfqpoint{3.532144in}{0.594199in}}%
\pgfpathlineto{\pgfqpoint{3.533418in}{0.593434in}}%
\pgfpathlineto{\pgfqpoint{3.536391in}{0.611493in}}%
\pgfpathlineto{\pgfqpoint{3.538939in}{0.616374in}}%
\pgfpathlineto{\pgfqpoint{3.541912in}{0.618948in}}%
\pgfpathlineto{\pgfqpoint{3.543611in}{0.625473in}}%
\pgfpathlineto{\pgfqpoint{3.547434in}{0.647632in}}%
\pgfpathlineto{\pgfqpoint{3.547859in}{0.589821in}}%
\pgfpathlineto{\pgfqpoint{3.548708in}{0.592168in}}%
\pgfpathlineto{\pgfqpoint{3.551256in}{0.596326in}}%
\pgfpathlineto{\pgfqpoint{3.552106in}{0.596269in}}%
\pgfpathlineto{\pgfqpoint{3.557203in}{0.694240in}}%
\pgfpathlineto{\pgfqpoint{3.560176in}{0.746703in}}%
\pgfpathlineto{\pgfqpoint{3.560601in}{0.748196in}}%
\pgfpathlineto{\pgfqpoint{3.561025in}{0.589483in}}%
\pgfpathlineto{\pgfqpoint{3.561875in}{0.592599in}}%
\pgfpathlineto{\pgfqpoint{3.563574in}{0.595312in}}%
\pgfpathlineto{\pgfqpoint{3.565273in}{0.593937in}}%
\pgfpathlineto{\pgfqpoint{3.569095in}{0.642596in}}%
\pgfpathlineto{\pgfqpoint{3.572918in}{0.699843in}}%
\pgfpathlineto{\pgfqpoint{3.573767in}{0.701656in}}%
\pgfpathlineto{\pgfqpoint{3.574192in}{0.589615in}}%
\pgfpathlineto{\pgfqpoint{3.575041in}{0.590408in}}%
\pgfpathlineto{\pgfqpoint{3.576740in}{0.590064in}}%
\pgfpathlineto{\pgfqpoint{3.580138in}{0.597433in}}%
\pgfpathlineto{\pgfqpoint{3.580988in}{0.596191in}}%
\pgfpathlineto{\pgfqpoint{3.582686in}{0.608076in}}%
\pgfpathlineto{\pgfqpoint{3.583111in}{0.591273in}}%
\pgfpathlineto{\pgfqpoint{3.583961in}{0.595015in}}%
\pgfpathlineto{\pgfqpoint{3.586084in}{0.598478in}}%
\pgfpathlineto{\pgfqpoint{3.587358in}{0.597235in}}%
\pgfpathlineto{\pgfqpoint{3.591181in}{0.647058in}}%
\pgfpathlineto{\pgfqpoint{3.593729in}{0.658992in}}%
\pgfpathlineto{\pgfqpoint{3.594579in}{0.659323in}}%
\pgfpathlineto{\pgfqpoint{3.595004in}{0.589453in}}%
\pgfpathlineto{\pgfqpoint{3.595853in}{0.589850in}}%
\pgfpathlineto{\pgfqpoint{3.599251in}{0.592013in}}%
\pgfpathlineto{\pgfqpoint{3.603073in}{0.611655in}}%
\pgfpathlineto{\pgfqpoint{3.607745in}{0.617331in}}%
\pgfpathlineto{\pgfqpoint{3.608170in}{0.589445in}}%
\pgfpathlineto{\pgfqpoint{3.609020in}{0.590083in}}%
\pgfpathlineto{\pgfqpoint{3.611993in}{0.590763in}}%
\pgfpathlineto{\pgfqpoint{3.612417in}{0.590647in}}%
\pgfpathlineto{\pgfqpoint{3.618788in}{0.612913in}}%
\pgfpathlineto{\pgfqpoint{3.619213in}{0.589632in}}%
\pgfpathlineto{\pgfqpoint{3.620063in}{0.591789in}}%
\pgfpathlineto{\pgfqpoint{3.622611in}{0.595514in}}%
\pgfpathlineto{\pgfqpoint{3.623460in}{0.595016in}}%
\pgfpathlineto{\pgfqpoint{3.626858in}{0.626677in}}%
\pgfpathlineto{\pgfqpoint{3.628982in}{0.632866in}}%
\pgfpathlineto{\pgfqpoint{3.629407in}{0.589308in}}%
\pgfpathlineto{\pgfqpoint{3.630256in}{0.589759in}}%
\pgfpathlineto{\pgfqpoint{3.632804in}{0.590726in}}%
\pgfpathlineto{\pgfqpoint{3.634928in}{0.591521in}}%
\pgfpathlineto{\pgfqpoint{3.636202in}{0.590629in}}%
\pgfpathlineto{\pgfqpoint{3.636627in}{0.591770in}}%
\pgfpathlineto{\pgfqpoint{3.638751in}{0.594386in}}%
\pgfpathlineto{\pgfqpoint{3.639175in}{0.589318in}}%
\pgfpathlineto{\pgfqpoint{3.640025in}{0.590322in}}%
\pgfpathlineto{\pgfqpoint{3.642149in}{0.596513in}}%
\pgfpathlineto{\pgfqpoint{3.643423in}{0.601892in}}%
\pgfpathlineto{\pgfqpoint{3.650218in}{0.795099in}}%
\pgfpathlineto{\pgfqpoint{3.650643in}{0.590271in}}%
\pgfpathlineto{\pgfqpoint{3.651493in}{0.594133in}}%
\pgfpathlineto{\pgfqpoint{3.654041in}{0.600383in}}%
\pgfpathlineto{\pgfqpoint{3.654890in}{0.599584in}}%
\pgfpathlineto{\pgfqpoint{3.657864in}{0.641694in}}%
\pgfpathlineto{\pgfqpoint{3.659987in}{0.662081in}}%
\pgfpathlineto{\pgfqpoint{3.662111in}{0.718833in}}%
\pgfpathlineto{\pgfqpoint{3.666783in}{0.855823in}}%
\pgfpathlineto{\pgfqpoint{3.667208in}{0.856544in}}%
\pgfpathlineto{\pgfqpoint{3.667632in}{0.590578in}}%
\pgfpathlineto{\pgfqpoint{3.668482in}{0.593840in}}%
\pgfpathlineto{\pgfqpoint{3.670181in}{0.596068in}}%
\pgfpathlineto{\pgfqpoint{3.671455in}{0.594616in}}%
\pgfpathlineto{\pgfqpoint{3.671880in}{0.593775in}}%
\pgfpathlineto{\pgfqpoint{3.674003in}{0.604572in}}%
\pgfpathlineto{\pgfqpoint{3.675277in}{0.608771in}}%
\pgfpathlineto{\pgfqpoint{3.676552in}{0.622696in}}%
\pgfpathlineto{\pgfqpoint{3.678675in}{0.683050in}}%
\pgfpathlineto{\pgfqpoint{3.683347in}{0.829450in}}%
\pgfpathlineto{\pgfqpoint{3.685046in}{0.840095in}}%
\pgfpathlineto{\pgfqpoint{3.685471in}{0.589699in}}%
\pgfpathlineto{\pgfqpoint{3.686320in}{0.591817in}}%
\pgfpathlineto{\pgfqpoint{3.688444in}{0.594401in}}%
\pgfpathlineto{\pgfqpoint{3.689718in}{0.593244in}}%
\pgfpathlineto{\pgfqpoint{3.694815in}{0.641173in}}%
\pgfpathlineto{\pgfqpoint{3.703734in}{0.781354in}}%
\pgfpathlineto{\pgfqpoint{3.704584in}{0.782379in}}%
\pgfpathlineto{\pgfqpoint{3.705008in}{0.589200in}}%
\pgfpathlineto{\pgfqpoint{3.705858in}{0.589382in}}%
\pgfpathlineto{\pgfqpoint{3.707982in}{0.590404in}}%
\pgfpathlineto{\pgfqpoint{3.709256in}{0.592732in}}%
\pgfpathlineto{\pgfqpoint{3.711379in}{0.625190in}}%
\pgfpathlineto{\pgfqpoint{3.715627in}{0.695875in}}%
\pgfpathlineto{\pgfqpoint{3.716051in}{0.589465in}}%
\pgfpathlineto{\pgfqpoint{3.716901in}{0.592646in}}%
\pgfpathlineto{\pgfqpoint{3.718600in}{0.595086in}}%
\pgfpathlineto{\pgfqpoint{3.719874in}{0.592792in}}%
\pgfpathlineto{\pgfqpoint{3.721148in}{0.590342in}}%
\pgfpathlineto{\pgfqpoint{3.724121in}{0.599191in}}%
\pgfpathlineto{\pgfqpoint{3.724971in}{0.598884in}}%
\pgfpathlineto{\pgfqpoint{3.725395in}{0.598180in}}%
\pgfpathlineto{\pgfqpoint{3.727944in}{0.625339in}}%
\pgfpathlineto{\pgfqpoint{3.728369in}{0.589349in}}%
\pgfpathlineto{\pgfqpoint{3.729218in}{0.590924in}}%
\pgfpathlineto{\pgfqpoint{3.730492in}{0.591148in}}%
\pgfpathlineto{\pgfqpoint{3.732191in}{0.589663in}}%
\pgfpathlineto{\pgfqpoint{3.732616in}{0.590347in}}%
\pgfpathlineto{\pgfqpoint{3.734315in}{0.591386in}}%
\pgfpathlineto{\pgfqpoint{3.735589in}{0.589396in}}%
\pgfpathlineto{\pgfqpoint{3.736014in}{0.590131in}}%
\pgfpathlineto{\pgfqpoint{3.739412in}{0.598164in}}%
\pgfpathlineto{\pgfqpoint{3.740261in}{0.598039in}}%
\pgfpathlineto{\pgfqpoint{3.746632in}{0.707002in}}%
\pgfpathlineto{\pgfqpoint{3.750455in}{0.734829in}}%
\pgfpathlineto{\pgfqpoint{3.753003in}{0.750612in}}%
\pgfpathlineto{\pgfqpoint{3.753428in}{0.589219in}}%
\pgfpathlineto{\pgfqpoint{3.754277in}{0.591724in}}%
\pgfpathlineto{\pgfqpoint{3.757675in}{0.600155in}}%
\pgfpathlineto{\pgfqpoint{3.763621in}{0.731425in}}%
\pgfpathlineto{\pgfqpoint{3.768293in}{0.790926in}}%
\pgfpathlineto{\pgfqpoint{3.770842in}{0.801599in}}%
\pgfpathlineto{\pgfqpoint{3.771266in}{0.589490in}}%
\pgfpathlineto{\pgfqpoint{3.772116in}{0.590966in}}%
\pgfpathlineto{\pgfqpoint{3.775514in}{0.594962in}}%
\pgfpathlineto{\pgfqpoint{3.781884in}{0.692980in}}%
\pgfpathlineto{\pgfqpoint{3.788680in}{0.860667in}}%
\pgfpathlineto{\pgfqpoint{3.789105in}{0.861605in}}%
\pgfpathlineto{\pgfqpoint{3.789530in}{0.590244in}}%
\pgfpathlineto{\pgfqpoint{3.790379in}{0.594058in}}%
\pgfpathlineto{\pgfqpoint{3.792927in}{0.600567in}}%
\pgfpathlineto{\pgfqpoint{3.793777in}{0.600846in}}%
\pgfpathlineto{\pgfqpoint{3.803121in}{0.826974in}}%
\pgfpathlineto{\pgfqpoint{3.805245in}{0.838977in}}%
\pgfpathlineto{\pgfqpoint{3.805669in}{0.590467in}}%
\pgfpathlineto{\pgfqpoint{3.806519in}{0.593606in}}%
\pgfpathlineto{\pgfqpoint{3.809492in}{0.602117in}}%
\pgfpathlineto{\pgfqpoint{3.809917in}{0.602367in}}%
\pgfpathlineto{\pgfqpoint{3.816288in}{0.769596in}}%
\pgfpathlineto{\pgfqpoint{3.817986in}{0.779369in}}%
\pgfpathlineto{\pgfqpoint{3.818411in}{0.589457in}}%
\pgfpathlineto{\pgfqpoint{3.819261in}{0.590346in}}%
\pgfpathlineto{\pgfqpoint{3.823083in}{0.589709in}}%
\pgfpathlineto{\pgfqpoint{3.823508in}{0.589232in}}%
\pgfpathlineto{\pgfqpoint{3.823933in}{0.590079in}}%
\pgfpathlineto{\pgfqpoint{3.826481in}{0.599415in}}%
\pgfpathlineto{\pgfqpoint{3.827755in}{0.604132in}}%
\pgfpathlineto{\pgfqpoint{3.832427in}{0.749565in}}%
\pgfpathlineto{\pgfqpoint{3.834126in}{0.757467in}}%
\pgfpathlineto{\pgfqpoint{3.834551in}{0.757517in}}%
\pgfpathlineto{\pgfqpoint{3.834976in}{0.589210in}}%
\pgfpathlineto{\pgfqpoint{3.835825in}{0.589746in}}%
\pgfpathlineto{\pgfqpoint{3.838373in}{0.591492in}}%
\pgfpathlineto{\pgfqpoint{3.840072in}{0.589514in}}%
\pgfpathlineto{\pgfqpoint{3.840497in}{0.590604in}}%
\pgfpathlineto{\pgfqpoint{3.844320in}{0.600236in}}%
\pgfpathlineto{\pgfqpoint{3.852390in}{0.798980in}}%
\pgfpathlineto{\pgfqpoint{3.853664in}{0.804977in}}%
\pgfpathlineto{\pgfqpoint{3.854088in}{0.589280in}}%
\pgfpathlineto{\pgfqpoint{3.854938in}{0.590891in}}%
\pgfpathlineto{\pgfqpoint{3.857486in}{0.592702in}}%
\pgfpathlineto{\pgfqpoint{3.858336in}{0.592694in}}%
\pgfpathlineto{\pgfqpoint{3.861309in}{0.624065in}}%
\pgfpathlineto{\pgfqpoint{3.863857in}{0.683412in}}%
\pgfpathlineto{\pgfqpoint{3.868529in}{0.804717in}}%
\pgfpathlineto{\pgfqpoint{3.869379in}{0.808091in}}%
\pgfpathlineto{\pgfqpoint{3.869803in}{0.589569in}}%
\pgfpathlineto{\pgfqpoint{3.870653in}{0.591052in}}%
\pgfpathlineto{\pgfqpoint{3.871927in}{0.590735in}}%
\pgfpathlineto{\pgfqpoint{3.874051in}{0.589820in}}%
\pgfpathlineto{\pgfqpoint{3.876174in}{0.590706in}}%
\pgfpathlineto{\pgfqpoint{3.879148in}{0.594063in}}%
\pgfpathlineto{\pgfqpoint{3.879572in}{0.594156in}}%
\pgfpathlineto{\pgfqpoint{3.884244in}{0.661319in}}%
\pgfpathlineto{\pgfqpoint{3.890615in}{0.756349in}}%
\pgfpathlineto{\pgfqpoint{3.891040in}{0.589903in}}%
\pgfpathlineto{\pgfqpoint{3.891889in}{0.592611in}}%
\pgfpathlineto{\pgfqpoint{3.893588in}{0.595012in}}%
\pgfpathlineto{\pgfqpoint{3.894862in}{0.593534in}}%
\pgfpathlineto{\pgfqpoint{3.895287in}{0.592634in}}%
\pgfpathlineto{\pgfqpoint{3.896561in}{0.597064in}}%
\pgfpathlineto{\pgfqpoint{3.896986in}{0.589543in}}%
\pgfpathlineto{\pgfqpoint{3.897836in}{0.590800in}}%
\pgfpathlineto{\pgfqpoint{3.899110in}{0.590861in}}%
\pgfpathlineto{\pgfqpoint{3.899959in}{0.589296in}}%
\pgfpathlineto{\pgfqpoint{3.900384in}{0.590341in}}%
\pgfpathlineto{\pgfqpoint{3.904207in}{0.601246in}}%
\pgfpathlineto{\pgfqpoint{3.904631in}{0.601152in}}%
\pgfpathlineto{\pgfqpoint{3.908029in}{0.666814in}}%
\pgfpathlineto{\pgfqpoint{3.908879in}{0.670396in}}%
\pgfpathlineto{\pgfqpoint{3.909303in}{0.589477in}}%
\pgfpathlineto{\pgfqpoint{3.910153in}{0.591868in}}%
\pgfpathlineto{\pgfqpoint{3.911852in}{0.593972in}}%
\pgfpathlineto{\pgfqpoint{3.913126in}{0.592341in}}%
\pgfpathlineto{\pgfqpoint{3.914400in}{0.589253in}}%
\pgfpathlineto{\pgfqpoint{3.914825in}{0.590179in}}%
\pgfpathlineto{\pgfqpoint{3.916099in}{0.591060in}}%
\pgfpathlineto{\pgfqpoint{3.917373in}{0.589950in}}%
\pgfpathlineto{\pgfqpoint{3.921196in}{0.600565in}}%
\pgfpathlineto{\pgfqpoint{3.921620in}{0.600340in}}%
\pgfpathlineto{\pgfqpoint{3.924594in}{0.652154in}}%
\pgfpathlineto{\pgfqpoint{3.926717in}{0.660374in}}%
\pgfpathlineto{\pgfqpoint{3.928841in}{0.664383in}}%
\pgfpathlineto{\pgfqpoint{3.932239in}{0.676980in}}%
\pgfpathlineto{\pgfqpoint{3.932663in}{0.589929in}}%
\pgfpathlineto{\pgfqpoint{3.933513in}{0.592787in}}%
\pgfpathlineto{\pgfqpoint{3.936486in}{0.600649in}}%
\pgfpathlineto{\pgfqpoint{3.936911in}{0.600362in}}%
\pgfpathlineto{\pgfqpoint{3.939884in}{0.651628in}}%
\pgfpathlineto{\pgfqpoint{3.944981in}{0.687781in}}%
\pgfpathlineto{\pgfqpoint{3.945830in}{0.689219in}}%
\pgfpathlineto{\pgfqpoint{3.946255in}{0.589416in}}%
\pgfpathlineto{\pgfqpoint{3.947104in}{0.590676in}}%
\pgfpathlineto{\pgfqpoint{3.950502in}{0.596140in}}%
\pgfpathlineto{\pgfqpoint{3.954749in}{0.653142in}}%
\pgfpathlineto{\pgfqpoint{3.963669in}{0.688601in}}%
\pgfpathlineto{\pgfqpoint{3.964943in}{0.689336in}}%
\pgfpathlineto{\pgfqpoint{3.965368in}{0.589286in}}%
\pgfpathlineto{\pgfqpoint{3.966217in}{0.589858in}}%
\pgfpathlineto{\pgfqpoint{3.969615in}{0.592023in}}%
\pgfpathlineto{\pgfqpoint{3.972163in}{0.616676in}}%
\pgfpathlineto{\pgfqpoint{3.977260in}{0.671514in}}%
\pgfpathlineto{\pgfqpoint{3.977685in}{0.589421in}}%
\pgfpathlineto{\pgfqpoint{3.978534in}{0.590392in}}%
\pgfpathlineto{\pgfqpoint{3.980233in}{0.590059in}}%
\pgfpathlineto{\pgfqpoint{3.981932in}{0.589929in}}%
\pgfpathlineto{\pgfqpoint{3.984905in}{0.591029in}}%
\pgfpathlineto{\pgfqpoint{3.987029in}{0.589890in}}%
\pgfpathlineto{\pgfqpoint{3.987453in}{0.590692in}}%
\pgfpathlineto{\pgfqpoint{3.990427in}{0.594236in}}%
\pgfpathlineto{\pgfqpoint{3.991276in}{0.594144in}}%
\pgfpathlineto{\pgfqpoint{3.997222in}{0.649898in}}%
\pgfpathlineto{\pgfqpoint{3.997647in}{0.589508in}}%
\pgfpathlineto{\pgfqpoint{3.998496in}{0.591140in}}%
\pgfpathlineto{\pgfqpoint{4.000195in}{0.591854in}}%
\pgfpathlineto{\pgfqpoint{4.001894in}{0.589246in}}%
\pgfpathlineto{\pgfqpoint{4.002744in}{0.590754in}}%
\pgfpathlineto{\pgfqpoint{4.005292in}{0.593036in}}%
\pgfpathlineto{\pgfqpoint{4.006566in}{0.592288in}}%
\pgfpathlineto{\pgfqpoint{4.008265in}{0.598114in}}%
\pgfpathlineto{\pgfqpoint{4.008690in}{0.589813in}}%
\pgfpathlineto{\pgfqpoint{4.009539in}{0.591512in}}%
\pgfpathlineto{\pgfqpoint{4.012513in}{0.594749in}}%
\pgfpathlineto{\pgfqpoint{4.012937in}{0.594833in}}%
\pgfpathlineto{\pgfqpoint{4.019733in}{0.672310in}}%
\pgfpathlineto{\pgfqpoint{4.021857in}{0.679833in}}%
\pgfpathlineto{\pgfqpoint{4.022281in}{0.589360in}}%
\pgfpathlineto{\pgfqpoint{4.023131in}{0.590143in}}%
\pgfpathlineto{\pgfqpoint{4.025679in}{0.590746in}}%
\pgfpathlineto{\pgfqpoint{4.026529in}{0.590636in}}%
\pgfpathlineto{\pgfqpoint{4.031201in}{0.609760in}}%
\pgfpathlineto{\pgfqpoint{4.034174in}{0.637616in}}%
\pgfpathlineto{\pgfqpoint{4.041819in}{0.730998in}}%
\pgfpathlineto{\pgfqpoint{4.043942in}{0.753032in}}%
\pgfpathlineto{\pgfqpoint{4.044367in}{0.589255in}}%
\pgfpathlineto{\pgfqpoint{4.045217in}{0.592634in}}%
\pgfpathlineto{\pgfqpoint{4.048190in}{0.599949in}}%
\pgfpathlineto{\pgfqpoint{4.048615in}{0.600153in}}%
\pgfpathlineto{\pgfqpoint{4.052862in}{0.683259in}}%
\pgfpathlineto{\pgfqpoint{4.056260in}{0.706396in}}%
\pgfpathlineto{\pgfqpoint{4.059657in}{0.717711in}}%
\pgfpathlineto{\pgfqpoint{4.061356in}{0.718867in}}%
\pgfpathlineto{\pgfqpoint{4.061781in}{0.589456in}}%
\pgfpathlineto{\pgfqpoint{4.062631in}{0.590337in}}%
\pgfpathlineto{\pgfqpoint{4.065179in}{0.596124in}}%
\pgfpathlineto{\pgfqpoint{4.066028in}{0.598123in}}%
\pgfpathlineto{\pgfqpoint{4.072399in}{0.720529in}}%
\pgfpathlineto{\pgfqpoint{4.075372in}{0.744658in}}%
\pgfpathlineto{\pgfqpoint{4.076222in}{0.745779in}}%
\pgfpathlineto{\pgfqpoint{4.076647in}{0.589912in}}%
\pgfpathlineto{\pgfqpoint{4.077496in}{0.591773in}}%
\pgfpathlineto{\pgfqpoint{4.080894in}{0.596692in}}%
\pgfpathlineto{\pgfqpoint{4.091937in}{0.772985in}}%
\pgfpathlineto{\pgfqpoint{4.094485in}{0.786779in}}%
\pgfpathlineto{\pgfqpoint{4.094910in}{0.589325in}}%
\pgfpathlineto{\pgfqpoint{4.095759in}{0.591461in}}%
\pgfpathlineto{\pgfqpoint{4.099157in}{0.602666in}}%
\pgfpathlineto{\pgfqpoint{4.104254in}{0.737812in}}%
\pgfpathlineto{\pgfqpoint{4.107652in}{0.763412in}}%
\pgfpathlineto{\pgfqpoint{4.111050in}{0.785662in}}%
\pgfpathlineto{\pgfqpoint{4.111474in}{0.590202in}}%
\pgfpathlineto{\pgfqpoint{4.112324in}{0.593567in}}%
\pgfpathlineto{\pgfqpoint{4.115297in}{0.601388in}}%
\pgfpathlineto{\pgfqpoint{4.115722in}{0.601229in}}%
\pgfpathlineto{\pgfqpoint{4.119120in}{0.666472in}}%
\pgfpathlineto{\pgfqpoint{4.122093in}{0.684009in}}%
\pgfpathlineto{\pgfqpoint{4.127614in}{0.712905in}}%
\pgfpathlineto{\pgfqpoint{4.128039in}{0.589350in}}%
\pgfpathlineto{\pgfqpoint{4.128888in}{0.590847in}}%
\pgfpathlineto{\pgfqpoint{4.132286in}{0.596864in}}%
\pgfpathlineto{\pgfqpoint{4.136533in}{0.658586in}}%
\pgfpathlineto{\pgfqpoint{4.136958in}{0.658697in}}%
\pgfpathlineto{\pgfqpoint{4.137383in}{0.590442in}}%
\pgfpathlineto{\pgfqpoint{4.138232in}{0.592475in}}%
\pgfpathlineto{\pgfqpoint{4.139507in}{0.592931in}}%
\pgfpathlineto{\pgfqpoint{4.141206in}{0.589468in}}%
\pgfpathlineto{\pgfqpoint{4.142055in}{0.591155in}}%
\pgfpathlineto{\pgfqpoint{4.144603in}{0.594374in}}%
\pgfpathlineto{\pgfqpoint{4.145878in}{0.592898in}}%
\pgfpathlineto{\pgfqpoint{4.147152in}{0.598082in}}%
\pgfpathlineto{\pgfqpoint{4.147576in}{0.589790in}}%
\pgfpathlineto{\pgfqpoint{4.148426in}{0.592172in}}%
\pgfpathlineto{\pgfqpoint{4.150550in}{0.595081in}}%
\pgfpathlineto{\pgfqpoint{4.151824in}{0.593393in}}%
\pgfpathlineto{\pgfqpoint{4.153947in}{0.601123in}}%
\pgfpathlineto{\pgfqpoint{4.155222in}{0.589236in}}%
\pgfpathlineto{\pgfqpoint{4.159894in}{0.591542in}}%
\pgfpathlineto{\pgfqpoint{4.162442in}{0.598936in}}%
\pgfpathlineto{\pgfqpoint{4.162867in}{0.589503in}}%
\pgfpathlineto{\pgfqpoint{4.163716in}{0.590652in}}%
\pgfpathlineto{\pgfqpoint{4.166265in}{0.592836in}}%
\pgfpathlineto{\pgfqpoint{4.167114in}{0.592618in}}%
\pgfpathlineto{\pgfqpoint{4.171361in}{0.616894in}}%
\pgfpathlineto{\pgfqpoint{4.173485in}{0.620965in}}%
\pgfpathlineto{\pgfqpoint{4.173910in}{0.589473in}}%
\pgfpathlineto{\pgfqpoint{4.174759in}{0.590081in}}%
\pgfpathlineto{\pgfqpoint{4.178157in}{0.591650in}}%
\pgfpathlineto{\pgfqpoint{4.183678in}{0.619027in}}%
\pgfpathlineto{\pgfqpoint{4.190049in}{0.641151in}}%
\pgfpathlineto{\pgfqpoint{4.190474in}{0.641883in}}%
\pgfpathlineto{\pgfqpoint{4.190899in}{0.589192in}}%
\pgfpathlineto{\pgfqpoint{4.191748in}{0.591202in}}%
\pgfpathlineto{\pgfqpoint{4.194297in}{0.595518in}}%
\pgfpathlineto{\pgfqpoint{4.195146in}{0.594756in}}%
\pgfpathlineto{\pgfqpoint{4.198119in}{0.616554in}}%
\pgfpathlineto{\pgfqpoint{4.200668in}{0.634762in}}%
\pgfpathlineto{\pgfqpoint{4.204065in}{0.657317in}}%
\pgfpathlineto{\pgfqpoint{4.205340in}{0.658764in}}%
\pgfpathlineto{\pgfqpoint{4.205764in}{0.589490in}}%
\pgfpathlineto{\pgfqpoint{4.206614in}{0.590179in}}%
\pgfpathlineto{\pgfqpoint{4.210012in}{0.593514in}}%
\pgfpathlineto{\pgfqpoint{4.213410in}{0.643900in}}%
\pgfpathlineto{\pgfqpoint{4.219356in}{0.732901in}}%
\pgfpathlineto{\pgfqpoint{4.221055in}{0.736941in}}%
\pgfpathlineto{\pgfqpoint{4.221479in}{0.589666in}}%
\pgfpathlineto{\pgfqpoint{4.222329in}{0.590240in}}%
\pgfpathlineto{\pgfqpoint{4.225727in}{0.591940in}}%
\pgfpathlineto{\pgfqpoint{4.227850in}{0.616175in}}%
\pgfpathlineto{\pgfqpoint{4.231248in}{0.687838in}}%
\pgfpathlineto{\pgfqpoint{4.235920in}{0.782776in}}%
\pgfpathlineto{\pgfqpoint{4.236770in}{0.785747in}}%
\pgfpathlineto{\pgfqpoint{4.237194in}{0.589564in}}%
\pgfpathlineto{\pgfqpoint{4.238044in}{0.591229in}}%
\pgfpathlineto{\pgfqpoint{4.239743in}{0.591462in}}%
\pgfpathlineto{\pgfqpoint{4.241442in}{0.589884in}}%
\pgfpathlineto{\pgfqpoint{4.241866in}{0.590428in}}%
\pgfpathlineto{\pgfqpoint{4.243141in}{0.593227in}}%
\pgfpathlineto{\pgfqpoint{4.244839in}{0.606657in}}%
\pgfpathlineto{\pgfqpoint{4.247388in}{0.654782in}}%
\pgfpathlineto{\pgfqpoint{4.252485in}{0.755911in}}%
\pgfpathlineto{\pgfqpoint{4.254608in}{0.764333in}}%
\pgfpathlineto{\pgfqpoint{4.255882in}{0.764710in}}%
\pgfpathlineto{\pgfqpoint{4.256307in}{0.589211in}}%
\pgfpathlineto{\pgfqpoint{4.257157in}{0.589304in}}%
\pgfpathlineto{\pgfqpoint{4.259705in}{0.590771in}}%
\pgfpathlineto{\pgfqpoint{4.260554in}{0.591718in}}%
\pgfpathlineto{\pgfqpoint{4.263103in}{0.617273in}}%
\pgfpathlineto{\pgfqpoint{4.267350in}{0.659634in}}%
\pgfpathlineto{\pgfqpoint{4.273296in}{0.700284in}}%
\pgfpathlineto{\pgfqpoint{4.279243in}{0.746666in}}%
\pgfpathlineto{\pgfqpoint{4.279667in}{0.589703in}}%
\pgfpathlineto{\pgfqpoint{4.280517in}{0.591056in}}%
\pgfpathlineto{\pgfqpoint{4.283915in}{0.594128in}}%
\pgfpathlineto{\pgfqpoint{4.291984in}{0.678648in}}%
\pgfpathlineto{\pgfqpoint{4.297506in}{0.714702in}}%
\pgfpathlineto{\pgfqpoint{4.297931in}{0.589376in}}%
\pgfpathlineto{\pgfqpoint{4.298780in}{0.590398in}}%
\pgfpathlineto{\pgfqpoint{4.302603in}{0.594821in}}%
\pgfpathlineto{\pgfqpoint{4.305151in}{0.623218in}}%
\pgfpathlineto{\pgfqpoint{4.313221in}{0.731408in}}%
\pgfpathlineto{\pgfqpoint{4.313646in}{0.732015in}}%
\pgfpathlineto{\pgfqpoint{4.314070in}{0.589347in}}%
\pgfpathlineto{\pgfqpoint{4.314920in}{0.590832in}}%
\pgfpathlineto{\pgfqpoint{4.317893in}{0.593568in}}%
\pgfpathlineto{\pgfqpoint{4.318318in}{0.593540in}}%
\pgfpathlineto{\pgfqpoint{4.327237in}{0.693903in}}%
\pgfpathlineto{\pgfqpoint{4.330210in}{0.716119in}}%
\pgfpathlineto{\pgfqpoint{4.331060in}{0.717300in}}%
\pgfpathlineto{\pgfqpoint{4.331484in}{0.589510in}}%
\pgfpathlineto{\pgfqpoint{4.332334in}{0.590517in}}%
\pgfpathlineto{\pgfqpoint{4.334457in}{0.590872in}}%
\pgfpathlineto{\pgfqpoint{4.336156in}{0.590703in}}%
\pgfpathlineto{\pgfqpoint{4.337855in}{0.594374in}}%
\pgfpathlineto{\pgfqpoint{4.339554in}{0.606780in}}%
\pgfpathlineto{\pgfqpoint{4.342952in}{0.659861in}}%
\pgfpathlineto{\pgfqpoint{4.347199in}{0.713752in}}%
\pgfpathlineto{\pgfqpoint{4.349748in}{0.723670in}}%
\pgfpathlineto{\pgfqpoint{4.350172in}{0.723940in}}%
\pgfpathlineto{\pgfqpoint{4.350597in}{0.589445in}}%
\pgfpathlineto{\pgfqpoint{4.351447in}{0.590739in}}%
\pgfpathlineto{\pgfqpoint{4.354844in}{0.595916in}}%
\pgfpathlineto{\pgfqpoint{4.359516in}{0.656859in}}%
\pgfpathlineto{\pgfqpoint{4.362490in}{0.672864in}}%
\pgfpathlineto{\pgfqpoint{4.363339in}{0.673648in}}%
\pgfpathlineto{\pgfqpoint{4.363764in}{0.589416in}}%
\pgfpathlineto{\pgfqpoint{4.364613in}{0.590072in}}%
\pgfpathlineto{\pgfqpoint{4.368436in}{0.592488in}}%
\pgfpathlineto{\pgfqpoint{4.371834in}{0.601216in}}%
\pgfpathlineto{\pgfqpoint{4.373957in}{0.604064in}}%
\pgfpathlineto{\pgfqpoint{4.375656in}{0.614139in}}%
\pgfpathlineto{\pgfqpoint{4.380328in}{0.652652in}}%
\pgfpathlineto{\pgfqpoint{4.380753in}{0.589581in}}%
\pgfpathlineto{\pgfqpoint{4.381602in}{0.591274in}}%
\pgfpathlineto{\pgfqpoint{4.383726in}{0.593316in}}%
\pgfpathlineto{\pgfqpoint{4.385000in}{0.592603in}}%
\pgfpathlineto{\pgfqpoint{4.387124in}{0.601944in}}%
\pgfpathlineto{\pgfqpoint{4.387549in}{0.602297in}}%
\pgfpathlineto{\pgfqpoint{4.387973in}{0.589207in}}%
\pgfpathlineto{\pgfqpoint{4.388823in}{0.589670in}}%
\pgfpathlineto{\pgfqpoint{4.394344in}{0.592078in}}%
\pgfpathlineto{\pgfqpoint{4.397742in}{0.597081in}}%
\pgfpathlineto{\pgfqpoint{4.401565in}{0.605195in}}%
\pgfpathlineto{\pgfqpoint{4.401989in}{0.589316in}}%
\pgfpathlineto{\pgfqpoint{4.402839in}{0.589725in}}%
\pgfpathlineto{\pgfqpoint{4.408360in}{0.592609in}}%
\pgfpathlineto{\pgfqpoint{4.408785in}{0.589429in}}%
\pgfpathlineto{\pgfqpoint{4.409634in}{0.590023in}}%
\pgfpathlineto{\pgfqpoint{4.411333in}{0.590199in}}%
\pgfpathlineto{\pgfqpoint{4.412608in}{0.589701in}}%
\pgfpathlineto{\pgfqpoint{4.415581in}{0.593520in}}%
\pgfpathlineto{\pgfqpoint{4.416855in}{0.592191in}}%
\pgfpathlineto{\pgfqpoint{4.418129in}{0.596293in}}%
\pgfpathlineto{\pgfqpoint{4.418554in}{0.589438in}}%
\pgfpathlineto{\pgfqpoint{4.419403in}{0.590562in}}%
\pgfpathlineto{\pgfqpoint{4.421102in}{0.590847in}}%
\pgfpathlineto{\pgfqpoint{4.423226in}{0.590696in}}%
\pgfpathlineto{\pgfqpoint{4.427473in}{0.598594in}}%
\pgfpathlineto{\pgfqpoint{4.430021in}{0.611603in}}%
\pgfpathlineto{\pgfqpoint{4.433419in}{0.629017in}}%
\pgfpathlineto{\pgfqpoint{4.433844in}{0.629323in}}%
\pgfpathlineto{\pgfqpoint{4.434269in}{0.589315in}}%
\pgfpathlineto{\pgfqpoint{4.435118in}{0.589912in}}%
\pgfpathlineto{\pgfqpoint{4.438941in}{0.591801in}}%
\pgfpathlineto{\pgfqpoint{4.442339in}{0.609242in}}%
\pgfpathlineto{\pgfqpoint{4.444887in}{0.615872in}}%
\pgfpathlineto{\pgfqpoint{4.445312in}{0.589250in}}%
\pgfpathlineto{\pgfqpoint{4.446161in}{0.589693in}}%
\pgfpathlineto{\pgfqpoint{4.450408in}{0.592427in}}%
\pgfpathlineto{\pgfqpoint{4.451258in}{0.593284in}}%
\pgfpathlineto{\pgfqpoint{4.451683in}{0.589440in}}%
\pgfpathlineto{\pgfqpoint{4.452532in}{0.590590in}}%
\pgfpathlineto{\pgfqpoint{4.455081in}{0.592402in}}%
\pgfpathlineto{\pgfqpoint{4.455930in}{0.591567in}}%
\pgfpathlineto{\pgfqpoint{4.456779in}{0.594100in}}%
\pgfpathlineto{\pgfqpoint{4.457204in}{0.589287in}}%
\pgfpathlineto{\pgfqpoint{4.458054in}{0.591316in}}%
\pgfpathlineto{\pgfqpoint{4.460602in}{0.595189in}}%
\pgfpathlineto{\pgfqpoint{4.461451in}{0.595007in}}%
\pgfpathlineto{\pgfqpoint{4.464000in}{0.615986in}}%
\pgfpathlineto{\pgfqpoint{4.464425in}{0.615994in}}%
\pgfpathlineto{\pgfqpoint{4.464849in}{0.590271in}}%
\pgfpathlineto{\pgfqpoint{4.465699in}{0.592129in}}%
\pgfpathlineto{\pgfqpoint{4.467398in}{0.593276in}}%
\pgfpathlineto{\pgfqpoint{4.469097in}{0.590728in}}%
\pgfpathlineto{\pgfqpoint{4.469521in}{0.591345in}}%
\pgfpathlineto{\pgfqpoint{4.469946in}{0.589479in}}%
\pgfpathlineto{\pgfqpoint{4.470371in}{0.590302in}}%
\pgfpathlineto{\pgfqpoint{4.472070in}{0.591605in}}%
\pgfpathlineto{\pgfqpoint{4.473769in}{0.589645in}}%
\pgfpathlineto{\pgfqpoint{4.474193in}{0.590785in}}%
\pgfpathlineto{\pgfqpoint{4.477166in}{0.595676in}}%
\pgfpathlineto{\pgfqpoint{4.478016in}{0.595041in}}%
\pgfpathlineto{\pgfqpoint{4.480140in}{0.611963in}}%
\pgfpathlineto{\pgfqpoint{4.480564in}{0.612266in}}%
\pgfpathlineto{\pgfqpoint{4.480989in}{0.589920in}}%
\pgfpathlineto{\pgfqpoint{4.481838in}{0.591721in}}%
\pgfpathlineto{\pgfqpoint{4.483962in}{0.593542in}}%
\pgfpathlineto{\pgfqpoint{4.485236in}{0.592433in}}%
\pgfpathlineto{\pgfqpoint{4.486935in}{0.598606in}}%
\pgfpathlineto{\pgfqpoint{4.487360in}{0.589243in}}%
\pgfpathlineto{\pgfqpoint{4.488209in}{0.589871in}}%
\pgfpathlineto{\pgfqpoint{4.490333in}{0.590332in}}%
\pgfpathlineto{\pgfqpoint{4.493306in}{0.594143in}}%
\pgfpathlineto{\pgfqpoint{4.494156in}{0.593841in}}%
\pgfpathlineto{\pgfqpoint{4.496704in}{0.609852in}}%
\pgfpathlineto{\pgfqpoint{4.497129in}{0.610090in}}%
\pgfpathlineto{\pgfqpoint{4.497553in}{0.589595in}}%
\pgfpathlineto{\pgfqpoint{4.498403in}{0.590661in}}%
\pgfpathlineto{\pgfqpoint{4.500951in}{0.591774in}}%
\pgfpathlineto{\pgfqpoint{4.501801in}{0.591752in}}%
\pgfpathlineto{\pgfqpoint{4.504774in}{0.615219in}}%
\pgfpathlineto{\pgfqpoint{4.508172in}{0.672484in}}%
\pgfpathlineto{\pgfqpoint{4.510295in}{0.693814in}}%
\pgfpathlineto{\pgfqpoint{4.510720in}{0.589830in}}%
\pgfpathlineto{\pgfqpoint{4.511570in}{0.593296in}}%
\pgfpathlineto{\pgfqpoint{4.513693in}{0.597638in}}%
\pgfpathlineto{\pgfqpoint{4.514967in}{0.596466in}}%
\pgfpathlineto{\pgfqpoint{4.517516in}{0.620011in}}%
\pgfpathlineto{\pgfqpoint{4.518790in}{0.621225in}}%
\pgfpathlineto{\pgfqpoint{4.519215in}{0.589218in}}%
\pgfpathlineto{\pgfqpoint{4.520064in}{0.589570in}}%
\pgfpathlineto{\pgfqpoint{4.523887in}{0.593556in}}%
\pgfpathlineto{\pgfqpoint{4.526860in}{0.614924in}}%
\pgfpathlineto{\pgfqpoint{4.527284in}{0.615261in}}%
\pgfpathlineto{\pgfqpoint{4.527709in}{0.589747in}}%
\pgfpathlineto{\pgfqpoint{4.528559in}{0.591639in}}%
\pgfpathlineto{\pgfqpoint{4.531532in}{0.596553in}}%
\pgfpathlineto{\pgfqpoint{4.531957in}{0.596500in}}%
\pgfpathlineto{\pgfqpoint{4.534930in}{0.629832in}}%
\pgfpathlineto{\pgfqpoint{4.537053in}{0.634228in}}%
\pgfpathlineto{\pgfqpoint{4.540026in}{0.638125in}}%
\pgfpathlineto{\pgfqpoint{4.542150in}{0.644891in}}%
\pgfpathlineto{\pgfqpoint{4.545123in}{0.659682in}}%
\pgfpathlineto{\pgfqpoint{4.545548in}{0.589396in}}%
\pgfpathlineto{\pgfqpoint{4.546397in}{0.591977in}}%
\pgfpathlineto{\pgfqpoint{4.548946in}{0.596657in}}%
\pgfpathlineto{\pgfqpoint{4.549795in}{0.595994in}}%
\pgfpathlineto{\pgfqpoint{4.552344in}{0.619289in}}%
\pgfpathlineto{\pgfqpoint{4.553618in}{0.620719in}}%
\pgfpathlineto{\pgfqpoint{4.554467in}{0.589200in}}%
\pgfpathlineto{\pgfqpoint{4.554892in}{0.589364in}}%
\pgfpathlineto{\pgfqpoint{4.559139in}{0.591929in}}%
\pgfpathlineto{\pgfqpoint{4.562112in}{0.604110in}}%
\pgfpathlineto{\pgfqpoint{4.562537in}{0.589954in}}%
\pgfpathlineto{\pgfqpoint{4.563386in}{0.591852in}}%
\pgfpathlineto{\pgfqpoint{4.566360in}{0.596170in}}%
\pgfpathlineto{\pgfqpoint{4.566784in}{0.596124in}}%
\pgfpathlineto{\pgfqpoint{4.570607in}{0.639355in}}%
\pgfpathlineto{\pgfqpoint{4.573155in}{0.646654in}}%
\pgfpathlineto{\pgfqpoint{4.575704in}{0.650201in}}%
\pgfpathlineto{\pgfqpoint{4.579101in}{0.661569in}}%
\pgfpathlineto{\pgfqpoint{4.579526in}{0.589484in}}%
\pgfpathlineto{\pgfqpoint{4.580376in}{0.591573in}}%
\pgfpathlineto{\pgfqpoint{4.583349in}{0.597022in}}%
\pgfpathlineto{\pgfqpoint{4.583773in}{0.596938in}}%
\pgfpathlineto{\pgfqpoint{4.587171in}{0.640488in}}%
\pgfpathlineto{\pgfqpoint{4.590569in}{0.657884in}}%
\pgfpathlineto{\pgfqpoint{4.592693in}{0.660750in}}%
\pgfpathlineto{\pgfqpoint{4.593118in}{0.589410in}}%
\pgfpathlineto{\pgfqpoint{4.593967in}{0.590217in}}%
\pgfpathlineto{\pgfqpoint{4.597365in}{0.594985in}}%
\pgfpathlineto{\pgfqpoint{4.602037in}{0.646395in}}%
\pgfpathlineto{\pgfqpoint{4.605010in}{0.655064in}}%
\pgfpathlineto{\pgfqpoint{4.608833in}{0.659519in}}%
\pgfpathlineto{\pgfqpoint{4.609257in}{0.589294in}}%
\pgfpathlineto{\pgfqpoint{4.610107in}{0.589878in}}%
\pgfpathlineto{\pgfqpoint{4.613505in}{0.592033in}}%
\pgfpathlineto{\pgfqpoint{4.617752in}{0.614192in}}%
\pgfpathlineto{\pgfqpoint{4.621150in}{0.623878in}}%
\pgfpathlineto{\pgfqpoint{4.623698in}{0.645332in}}%
\pgfpathlineto{\pgfqpoint{4.626246in}{0.660217in}}%
\pgfpathlineto{\pgfqpoint{4.626671in}{0.589538in}}%
\pgfpathlineto{\pgfqpoint{4.627521in}{0.591040in}}%
\pgfpathlineto{\pgfqpoint{4.629644in}{0.592236in}}%
\pgfpathlineto{\pgfqpoint{4.630918in}{0.591986in}}%
\pgfpathlineto{\pgfqpoint{4.636440in}{0.634316in}}%
\pgfpathlineto{\pgfqpoint{4.640687in}{0.665690in}}%
\pgfpathlineto{\pgfqpoint{4.641537in}{0.666364in}}%
\pgfpathlineto{\pgfqpoint{4.641961in}{0.589855in}}%
\pgfpathlineto{\pgfqpoint{4.642811in}{0.591058in}}%
\pgfpathlineto{\pgfqpoint{4.645359in}{0.592351in}}%
\pgfpathlineto{\pgfqpoint{4.646209in}{0.591983in}}%
\pgfpathlineto{\pgfqpoint{4.649182in}{0.603668in}}%
\pgfpathlineto{\pgfqpoint{4.653004in}{0.612402in}}%
\pgfpathlineto{\pgfqpoint{4.656402in}{0.634524in}}%
\pgfpathlineto{\pgfqpoint{4.658951in}{0.644101in}}%
\pgfpathlineto{\pgfqpoint{4.661074in}{0.649292in}}%
\pgfpathlineto{\pgfqpoint{4.663623in}{0.666658in}}%
\pgfpathlineto{\pgfqpoint{4.666171in}{0.679529in}}%
\pgfpathlineto{\pgfqpoint{4.666596in}{0.679772in}}%
\pgfpathlineto{\pgfqpoint{4.667020in}{0.589504in}}%
\pgfpathlineto{\pgfqpoint{4.667870in}{0.590493in}}%
\pgfpathlineto{\pgfqpoint{4.671268in}{0.593340in}}%
\pgfpathlineto{\pgfqpoint{4.676789in}{0.661672in}}%
\pgfpathlineto{\pgfqpoint{4.680612in}{0.696018in}}%
\pgfpathlineto{\pgfqpoint{4.684010in}{0.708364in}}%
\pgfpathlineto{\pgfqpoint{4.687407in}{0.717237in}}%
\pgfpathlineto{\pgfqpoint{4.687832in}{0.589780in}}%
\pgfpathlineto{\pgfqpoint{4.688682in}{0.591830in}}%
\pgfpathlineto{\pgfqpoint{4.692079in}{0.598992in}}%
\pgfpathlineto{\pgfqpoint{4.696752in}{0.685591in}}%
\pgfpathlineto{\pgfqpoint{4.699725in}{0.704801in}}%
\pgfpathlineto{\pgfqpoint{4.703122in}{0.713842in}}%
\pgfpathlineto{\pgfqpoint{4.703547in}{0.589212in}}%
\pgfpathlineto{\pgfqpoint{4.704397in}{0.590234in}}%
\pgfpathlineto{\pgfqpoint{4.707794in}{0.595066in}}%
\pgfpathlineto{\pgfqpoint{4.713316in}{0.660113in}}%
\pgfpathlineto{\pgfqpoint{4.716714in}{0.692744in}}%
\pgfpathlineto{\pgfqpoint{4.721811in}{0.745418in}}%
\pgfpathlineto{\pgfqpoint{4.722235in}{0.745725in}}%
\pgfpathlineto{\pgfqpoint{4.722660in}{0.589876in}}%
\pgfpathlineto{\pgfqpoint{4.723509in}{0.591740in}}%
\pgfpathlineto{\pgfqpoint{4.725633in}{0.593939in}}%
\pgfpathlineto{\pgfqpoint{4.726907in}{0.592983in}}%
\pgfpathlineto{\pgfqpoint{4.729456in}{0.605283in}}%
\pgfpathlineto{\pgfqpoint{4.731155in}{0.614169in}}%
\pgfpathlineto{\pgfqpoint{4.733278in}{0.645513in}}%
\pgfpathlineto{\pgfqpoint{4.738375in}{0.735662in}}%
\pgfpathlineto{\pgfqpoint{4.739224in}{0.737152in}}%
\pgfpathlineto{\pgfqpoint{4.739649in}{0.590163in}}%
\pgfpathlineto{\pgfqpoint{4.740499in}{0.592126in}}%
\pgfpathlineto{\pgfqpoint{4.742622in}{0.594521in}}%
\pgfpathlineto{\pgfqpoint{4.743896in}{0.593995in}}%
\pgfpathlineto{\pgfqpoint{4.746870in}{0.615560in}}%
\pgfpathlineto{\pgfqpoint{4.749843in}{0.633507in}}%
\pgfpathlineto{\pgfqpoint{4.757488in}{0.697770in}}%
\pgfpathlineto{\pgfqpoint{4.757913in}{0.589413in}}%
\pgfpathlineto{\pgfqpoint{4.758762in}{0.590727in}}%
\pgfpathlineto{\pgfqpoint{4.762160in}{0.595760in}}%
\pgfpathlineto{\pgfqpoint{4.766407in}{0.645971in}}%
\pgfpathlineto{\pgfqpoint{4.767681in}{0.649384in}}%
\pgfpathlineto{\pgfqpoint{4.768106in}{0.589269in}}%
\pgfpathlineto{\pgfqpoint{4.768955in}{0.590330in}}%
\pgfpathlineto{\pgfqpoint{4.771504in}{0.591525in}}%
\pgfpathlineto{\pgfqpoint{4.772353in}{0.591257in}}%
\pgfpathlineto{\pgfqpoint{4.775751in}{0.602174in}}%
\pgfpathlineto{\pgfqpoint{4.779998in}{0.612783in}}%
\pgfpathlineto{\pgfqpoint{4.782547in}{0.631431in}}%
\pgfpathlineto{\pgfqpoint{4.785520in}{0.648778in}}%
\pgfpathlineto{\pgfqpoint{4.785945in}{0.589764in}}%
\pgfpathlineto{\pgfqpoint{4.786794in}{0.591421in}}%
\pgfpathlineto{\pgfqpoint{4.788493in}{0.592669in}}%
\pgfpathlineto{\pgfqpoint{4.791041in}{0.592548in}}%
\pgfpathlineto{\pgfqpoint{4.791466in}{0.592563in}}%
\pgfpathlineto{\pgfqpoint{4.792740in}{0.589611in}}%
\pgfpathlineto{\pgfqpoint{4.793590in}{0.589539in}}%
\pgfpathlineto{\pgfqpoint{4.797412in}{0.594352in}}%
\pgfpathlineto{\pgfqpoint{4.797837in}{0.594034in}}%
\pgfpathlineto{\pgfqpoint{4.800385in}{0.609801in}}%
\pgfpathlineto{\pgfqpoint{4.800810in}{0.610191in}}%
\pgfpathlineto{\pgfqpoint{4.801235in}{0.589222in}}%
\pgfpathlineto{\pgfqpoint{4.802084in}{0.589745in}}%
\pgfpathlineto{\pgfqpoint{4.806332in}{0.592082in}}%
\pgfpathlineto{\pgfqpoint{4.808455in}{0.600481in}}%
\pgfpathlineto{\pgfqpoint{4.811004in}{0.623011in}}%
\pgfpathlineto{\pgfqpoint{4.815676in}{0.665391in}}%
\pgfpathlineto{\pgfqpoint{4.816950in}{0.667987in}}%
\pgfpathlineto{\pgfqpoint{4.817375in}{0.589233in}}%
\pgfpathlineto{\pgfqpoint{4.818224in}{0.590106in}}%
\pgfpathlineto{\pgfqpoint{4.821197in}{0.591507in}}%
\pgfpathlineto{\pgfqpoint{4.821622in}{0.591249in}}%
\pgfpathlineto{\pgfqpoint{4.823321in}{0.595012in}}%
\pgfpathlineto{\pgfqpoint{4.823746in}{0.589502in}}%
\pgfpathlineto{\pgfqpoint{4.824595in}{0.590168in}}%
\pgfpathlineto{\pgfqpoint{4.825869in}{0.589917in}}%
\pgfpathlineto{\pgfqpoint{4.826719in}{0.589606in}}%
\pgfpathlineto{\pgfqpoint{4.830966in}{0.596690in}}%
\pgfpathlineto{\pgfqpoint{4.833939in}{0.629376in}}%
\pgfpathlineto{\pgfqpoint{4.834364in}{0.629406in}}%
\pgfpathlineto{\pgfqpoint{4.834789in}{0.590376in}}%
\pgfpathlineto{\pgfqpoint{4.835638in}{0.592383in}}%
\pgfpathlineto{\pgfqpoint{4.837337in}{0.593863in}}%
\pgfpathlineto{\pgfqpoint{4.839036in}{0.592128in}}%
\pgfpathlineto{\pgfqpoint{4.840735in}{0.597036in}}%
\pgfpathlineto{\pgfqpoint{4.841159in}{0.589363in}}%
\pgfpathlineto{\pgfqpoint{4.842009in}{0.589823in}}%
\pgfpathlineto{\pgfqpoint{4.845407in}{0.590905in}}%
\pgfpathlineto{\pgfqpoint{4.847955in}{0.592332in}}%
\pgfpathlineto{\pgfqpoint{4.851353in}{0.609380in}}%
\pgfpathlineto{\pgfqpoint{4.852202in}{0.609889in}}%
\pgfpathlineto{\pgfqpoint{4.852627in}{0.589326in}}%
\pgfpathlineto{\pgfqpoint{4.853477in}{0.589488in}}%
\pgfpathlineto{\pgfqpoint{4.856450in}{0.590048in}}%
\pgfpathlineto{\pgfqpoint{4.859423in}{0.591365in}}%
\pgfpathlineto{\pgfqpoint{4.861971in}{0.595130in}}%
\pgfpathlineto{\pgfqpoint{4.862396in}{0.589487in}}%
\pgfpathlineto{\pgfqpoint{4.863245in}{0.590554in}}%
\pgfpathlineto{\pgfqpoint{4.866643in}{0.594876in}}%
\pgfpathlineto{\pgfqpoint{4.869616in}{0.619869in}}%
\pgfpathlineto{\pgfqpoint{4.870041in}{0.589302in}}%
\pgfpathlineto{\pgfqpoint{4.870891in}{0.591048in}}%
\pgfpathlineto{\pgfqpoint{4.872589in}{0.592339in}}%
\pgfpathlineto{\pgfqpoint{4.874713in}{0.590721in}}%
\pgfpathlineto{\pgfqpoint{4.875138in}{0.589713in}}%
\pgfpathlineto{\pgfqpoint{4.875563in}{0.590594in}}%
\pgfpathlineto{\pgfqpoint{4.878111in}{0.593257in}}%
\pgfpathlineto{\pgfqpoint{4.879385in}{0.591705in}}%
\pgfpathlineto{\pgfqpoint{4.879810in}{0.593363in}}%
\pgfpathlineto{\pgfqpoint{4.880235in}{0.594085in}}%
\pgfpathlineto{\pgfqpoint{4.880659in}{0.589423in}}%
\pgfpathlineto{\pgfqpoint{4.881509in}{0.591150in}}%
\pgfpathlineto{\pgfqpoint{4.883632in}{0.592791in}}%
\pgfpathlineto{\pgfqpoint{4.884907in}{0.591820in}}%
\pgfpathlineto{\pgfqpoint{4.886606in}{0.596149in}}%
\pgfpathlineto{\pgfqpoint{4.887030in}{0.589813in}}%
\pgfpathlineto{\pgfqpoint{4.887880in}{0.590943in}}%
\pgfpathlineto{\pgfqpoint{4.889579in}{0.591206in}}%
\pgfpathlineto{\pgfqpoint{4.890853in}{0.589320in}}%
\pgfpathlineto{\pgfqpoint{4.891278in}{0.589930in}}%
\pgfpathlineto{\pgfqpoint{4.894251in}{0.593749in}}%
\pgfpathlineto{\pgfqpoint{4.895525in}{0.592815in}}%
\pgfpathlineto{\pgfqpoint{4.897648in}{0.602353in}}%
\pgfpathlineto{\pgfqpoint{4.898498in}{0.602763in}}%
\pgfpathlineto{\pgfqpoint{4.898923in}{0.589353in}}%
\pgfpathlineto{\pgfqpoint{4.899772in}{0.589451in}}%
\pgfpathlineto{\pgfqpoint{4.902321in}{0.589913in}}%
\pgfpathlineto{\pgfqpoint{4.904869in}{0.589417in}}%
\pgfpathlineto{\pgfqpoint{4.906143in}{0.590101in}}%
\pgfpathlineto{\pgfqpoint{4.909541in}{0.593149in}}%
\pgfpathlineto{\pgfqpoint{4.913788in}{0.623052in}}%
\pgfpathlineto{\pgfqpoint{4.916761in}{0.628125in}}%
\pgfpathlineto{\pgfqpoint{4.918035in}{0.628688in}}%
\pgfpathlineto{\pgfqpoint{4.918460in}{0.589340in}}%
\pgfpathlineto{\pgfqpoint{4.919310in}{0.589651in}}%
\pgfpathlineto{\pgfqpoint{4.923982in}{0.589509in}}%
\pgfpathlineto{\pgfqpoint{4.924831in}{0.590121in}}%
\pgfpathlineto{\pgfqpoint{4.928229in}{0.595610in}}%
\pgfpathlineto{\pgfqpoint{4.928654in}{0.595397in}}%
\pgfpathlineto{\pgfqpoint{4.931202in}{0.617341in}}%
\pgfpathlineto{\pgfqpoint{4.931627in}{0.617685in}}%
\pgfpathlineto{\pgfqpoint{4.932052in}{0.589744in}}%
\pgfpathlineto{\pgfqpoint{4.932901in}{0.591278in}}%
\pgfpathlineto{\pgfqpoint{4.935449in}{0.593083in}}%
\pgfpathlineto{\pgfqpoint{4.936299in}{0.592506in}}%
\pgfpathlineto{\pgfqpoint{4.938422in}{0.599876in}}%
\pgfpathlineto{\pgfqpoint{4.938847in}{0.589623in}}%
\pgfpathlineto{\pgfqpoint{4.939697in}{0.590251in}}%
\pgfpathlineto{\pgfqpoint{4.943519in}{0.591622in}}%
\pgfpathlineto{\pgfqpoint{4.946068in}{0.592844in}}%
\pgfpathlineto{\pgfqpoint{4.949041in}{0.608956in}}%
\pgfpathlineto{\pgfqpoint{4.949465in}{0.589376in}}%
\pgfpathlineto{\pgfqpoint{4.950315in}{0.590916in}}%
\pgfpathlineto{\pgfqpoint{4.952863in}{0.593061in}}%
\pgfpathlineto{\pgfqpoint{4.953713in}{0.592454in}}%
\pgfpathlineto{\pgfqpoint{4.955836in}{0.600513in}}%
\pgfpathlineto{\pgfqpoint{4.956261in}{0.600538in}}%
\pgfpathlineto{\pgfqpoint{4.956686in}{0.589508in}}%
\pgfpathlineto{\pgfqpoint{4.957535in}{0.589799in}}%
\pgfpathlineto{\pgfqpoint{4.959659in}{0.590192in}}%
\pgfpathlineto{\pgfqpoint{4.963057in}{0.594231in}}%
\pgfpathlineto{\pgfqpoint{4.965605in}{0.612796in}}%
\pgfpathlineto{\pgfqpoint{4.966030in}{0.589974in}}%
\pgfpathlineto{\pgfqpoint{4.966879in}{0.592493in}}%
\pgfpathlineto{\pgfqpoint{4.968578in}{0.594783in}}%
\pgfpathlineto{\pgfqpoint{4.969852in}{0.593255in}}%
\pgfpathlineto{\pgfqpoint{4.970277in}{0.592330in}}%
\pgfpathlineto{\pgfqpoint{4.971551in}{0.595897in}}%
\pgfpathlineto{\pgfqpoint{4.971976in}{0.589662in}}%
\pgfpathlineto{\pgfqpoint{4.972826in}{0.590527in}}%
\pgfpathlineto{\pgfqpoint{4.974524in}{0.589809in}}%
\pgfpathlineto{\pgfqpoint{4.974949in}{0.589258in}}%
\pgfpathlineto{\pgfqpoint{4.975374in}{0.589756in}}%
\pgfpathlineto{\pgfqpoint{4.978772in}{0.593445in}}%
\pgfpathlineto{\pgfqpoint{4.979621in}{0.592736in}}%
\pgfpathlineto{\pgfqpoint{4.981320in}{0.599318in}}%
\pgfpathlineto{\pgfqpoint{4.981745in}{0.589690in}}%
\pgfpathlineto{\pgfqpoint{4.982594in}{0.590970in}}%
\pgfpathlineto{\pgfqpoint{4.984293in}{0.591104in}}%
\pgfpathlineto{\pgfqpoint{4.986842in}{0.589801in}}%
\pgfpathlineto{\pgfqpoint{4.988541in}{0.589571in}}%
\pgfpathlineto{\pgfqpoint{4.989390in}{0.589576in}}%
\pgfpathlineto{\pgfqpoint{4.993637in}{0.592955in}}%
\pgfpathlineto{\pgfqpoint{4.998309in}{0.628278in}}%
\pgfpathlineto{\pgfqpoint{4.999159in}{0.629750in}}%
\pgfpathlineto{\pgfqpoint{4.999584in}{0.589239in}}%
\pgfpathlineto{\pgfqpoint{5.000433in}{0.590214in}}%
\pgfpathlineto{\pgfqpoint{5.003406in}{0.592045in}}%
\pgfpathlineto{\pgfqpoint{5.003831in}{0.592004in}}%
\pgfpathlineto{\pgfqpoint{5.006804in}{0.604794in}}%
\pgfpathlineto{\pgfqpoint{5.009352in}{0.610584in}}%
\pgfpathlineto{\pgfqpoint{5.011476in}{0.625291in}}%
\pgfpathlineto{\pgfqpoint{5.015299in}{0.651427in}}%
\pgfpathlineto{\pgfqpoint{5.016573in}{0.652789in}}%
\pgfpathlineto{\pgfqpoint{5.016997in}{0.589331in}}%
\pgfpathlineto{\pgfqpoint{5.017847in}{0.589592in}}%
\pgfpathlineto{\pgfqpoint{5.021245in}{0.590072in}}%
\pgfpathlineto{\pgfqpoint{5.022944in}{0.596652in}}%
\pgfpathlineto{\pgfqpoint{5.027616in}{0.630339in}}%
\pgfpathlineto{\pgfqpoint{5.034411in}{0.676084in}}%
\pgfpathlineto{\pgfqpoint{5.034836in}{0.589810in}}%
\pgfpathlineto{\pgfqpoint{5.035686in}{0.590975in}}%
\pgfpathlineto{\pgfqpoint{5.037809in}{0.591988in}}%
\pgfpathlineto{\pgfqpoint{5.039083in}{0.591538in}}%
\pgfpathlineto{\pgfqpoint{5.050126in}{0.642349in}}%
\pgfpathlineto{\pgfqpoint{5.050551in}{0.642400in}}%
\pgfpathlineto{\pgfqpoint{5.050976in}{0.589443in}}%
\pgfpathlineto{\pgfqpoint{5.051825in}{0.589769in}}%
\pgfpathlineto{\pgfqpoint{5.053949in}{0.590194in}}%
\pgfpathlineto{\pgfqpoint{5.056922in}{0.593462in}}%
\pgfpathlineto{\pgfqpoint{5.057771in}{0.592644in}}%
\pgfpathlineto{\pgfqpoint{5.059046in}{0.597508in}}%
\pgfpathlineto{\pgfqpoint{5.059470in}{0.589943in}}%
\pgfpathlineto{\pgfqpoint{5.060320in}{0.592544in}}%
\pgfpathlineto{\pgfqpoint{5.062868in}{0.596618in}}%
\pgfpathlineto{\pgfqpoint{5.063718in}{0.596074in}}%
\pgfpathlineto{\pgfqpoint{5.066691in}{0.625625in}}%
\pgfpathlineto{\pgfqpoint{5.068814in}{0.629689in}}%
\pgfpathlineto{\pgfqpoint{5.071363in}{0.632118in}}%
\pgfpathlineto{\pgfqpoint{5.073911in}{0.640093in}}%
\pgfpathlineto{\pgfqpoint{5.076035in}{0.644291in}}%
\pgfpathlineto{\pgfqpoint{5.076460in}{0.589642in}}%
\pgfpathlineto{\pgfqpoint{5.077309in}{0.590677in}}%
\pgfpathlineto{\pgfqpoint{5.079857in}{0.592137in}}%
\pgfpathlineto{\pgfqpoint{5.080707in}{0.592064in}}%
\pgfpathlineto{\pgfqpoint{5.090900in}{0.653303in}}%
\pgfpathlineto{\pgfqpoint{5.095148in}{0.665955in}}%
\pgfpathlineto{\pgfqpoint{5.095997in}{0.666594in}}%
\pgfpathlineto{\pgfqpoint{5.096422in}{0.589270in}}%
\pgfpathlineto{\pgfqpoint{5.097271in}{0.589933in}}%
\pgfpathlineto{\pgfqpoint{5.100669in}{0.592544in}}%
\pgfpathlineto{\pgfqpoint{5.106191in}{0.646755in}}%
\pgfpathlineto{\pgfqpoint{5.110013in}{0.679836in}}%
\pgfpathlineto{\pgfqpoint{5.112137in}{0.685447in}}%
\pgfpathlineto{\pgfqpoint{5.112562in}{0.685648in}}%
\pgfpathlineto{\pgfqpoint{5.112986in}{0.589322in}}%
\pgfpathlineto{\pgfqpoint{5.113836in}{0.590107in}}%
\pgfpathlineto{\pgfqpoint{5.117234in}{0.593909in}}%
\pgfpathlineto{\pgfqpoint{5.123604in}{0.656651in}}%
\pgfpathlineto{\pgfqpoint{5.126153in}{0.664110in}}%
\pgfpathlineto{\pgfqpoint{5.127002in}{0.664503in}}%
\pgfpathlineto{\pgfqpoint{5.127427in}{0.589296in}}%
\pgfpathlineto{\pgfqpoint{5.128277in}{0.589533in}}%
\pgfpathlineto{\pgfqpoint{5.136771in}{0.591220in}}%
\pgfpathlineto{\pgfqpoint{5.138045in}{0.591950in}}%
\pgfpathlineto{\pgfqpoint{5.146115in}{0.641910in}}%
\pgfpathlineto{\pgfqpoint{5.149513in}{0.646293in}}%
\pgfpathlineto{\pgfqpoint{5.149938in}{0.589470in}}%
\pgfpathlineto{\pgfqpoint{5.150787in}{0.590175in}}%
\pgfpathlineto{\pgfqpoint{5.153336in}{0.590968in}}%
\pgfpathlineto{\pgfqpoint{5.154185in}{0.590513in}}%
\pgfpathlineto{\pgfqpoint{5.154610in}{0.591524in}}%
\pgfpathlineto{\pgfqpoint{5.155884in}{0.592726in}}%
\pgfpathlineto{\pgfqpoint{5.157158in}{0.589217in}}%
\pgfpathlineto{\pgfqpoint{5.159282in}{0.591202in}}%
\pgfpathlineto{\pgfqpoint{5.161405in}{0.593385in}}%
\pgfpathlineto{\pgfqpoint{5.165653in}{0.627082in}}%
\pgfpathlineto{\pgfqpoint{5.166077in}{0.627599in}}%
\pgfpathlineto{\pgfqpoint{5.166502in}{0.589349in}}%
\pgfpathlineto{\pgfqpoint{5.167352in}{0.590557in}}%
\pgfpathlineto{\pgfqpoint{5.169475in}{0.591647in}}%
\pgfpathlineto{\pgfqpoint{5.170749in}{0.590880in}}%
\pgfpathlineto{\pgfqpoint{5.172873in}{0.594524in}}%
\pgfpathlineto{\pgfqpoint{5.173298in}{0.594596in}}%
\pgfpathlineto{\pgfqpoint{5.174572in}{0.589191in}}%
\pgfpathlineto{\pgfqpoint{5.177120in}{0.590550in}}%
\pgfpathlineto{\pgfqpoint{5.178819in}{0.592213in}}%
\pgfpathlineto{\pgfqpoint{5.183491in}{0.623214in}}%
\pgfpathlineto{\pgfqpoint{5.183916in}{0.589597in}}%
\pgfpathlineto{\pgfqpoint{5.184766in}{0.590801in}}%
\pgfpathlineto{\pgfqpoint{5.186889in}{0.591658in}}%
\pgfpathlineto{\pgfqpoint{5.188163in}{0.591112in}}%
\pgfpathlineto{\pgfqpoint{5.190712in}{0.597535in}}%
\pgfpathlineto{\pgfqpoint{5.191561in}{0.597777in}}%
\pgfpathlineto{\pgfqpoint{5.191986in}{0.589389in}}%
\pgfpathlineto{\pgfqpoint{5.192835in}{0.589703in}}%
\pgfpathlineto{\pgfqpoint{5.200905in}{0.591104in}}%
\pgfpathlineto{\pgfqpoint{5.201330in}{0.589429in}}%
\pgfpathlineto{\pgfqpoint{5.202179in}{0.590074in}}%
\pgfpathlineto{\pgfqpoint{5.205153in}{0.590940in}}%
\pgfpathlineto{\pgfqpoint{5.205577in}{0.590759in}}%
\pgfpathlineto{\pgfqpoint{5.207701in}{0.594267in}}%
\pgfpathlineto{\pgfqpoint{5.208126in}{0.589390in}}%
\pgfpathlineto{\pgfqpoint{5.208975in}{0.589761in}}%
\pgfpathlineto{\pgfqpoint{5.212798in}{0.591316in}}%
\pgfpathlineto{\pgfqpoint{5.215771in}{0.604650in}}%
\pgfpathlineto{\pgfqpoint{5.218319in}{0.611132in}}%
\pgfpathlineto{\pgfqpoint{5.222991in}{0.614490in}}%
\pgfpathlineto{\pgfqpoint{5.223416in}{0.589290in}}%
\pgfpathlineto{\pgfqpoint{5.224265in}{0.589604in}}%
\pgfpathlineto{\pgfqpoint{5.228513in}{0.591520in}}%
\pgfpathlineto{\pgfqpoint{5.230636in}{0.600502in}}%
\pgfpathlineto{\pgfqpoint{5.238281in}{0.637149in}}%
\pgfpathlineto{\pgfqpoint{5.239131in}{0.637717in}}%
\pgfpathlineto{\pgfqpoint{5.239556in}{0.589460in}}%
\pgfpathlineto{\pgfqpoint{5.240405in}{0.589983in}}%
\pgfpathlineto{\pgfqpoint{5.244228in}{0.592258in}}%
\pgfpathlineto{\pgfqpoint{5.247201in}{0.608219in}}%
\pgfpathlineto{\pgfqpoint{5.252722in}{0.640864in}}%
\pgfpathlineto{\pgfqpoint{5.253147in}{0.641040in}}%
\pgfpathlineto{\pgfqpoint{5.253572in}{0.589409in}}%
\pgfpathlineto{\pgfqpoint{5.254421in}{0.589983in}}%
\pgfpathlineto{\pgfqpoint{5.256545in}{0.589859in}}%
\pgfpathlineto{\pgfqpoint{5.258244in}{0.589661in}}%
\pgfpathlineto{\pgfqpoint{5.261217in}{0.590951in}}%
\pgfpathlineto{\pgfqpoint{5.263765in}{0.590540in}}%
\pgfpathlineto{\pgfqpoint{5.266738in}{0.593825in}}%
\pgfpathlineto{\pgfqpoint{5.267588in}{0.593691in}}%
\pgfpathlineto{\pgfqpoint{5.272684in}{0.636208in}}%
\pgfpathlineto{\pgfqpoint{5.274808in}{0.643869in}}%
\pgfpathlineto{\pgfqpoint{5.275233in}{0.643989in}}%
\pgfpathlineto{\pgfqpoint{5.275658in}{0.589527in}}%
\pgfpathlineto{\pgfqpoint{5.276507in}{0.590321in}}%
\pgfpathlineto{\pgfqpoint{5.279480in}{0.591200in}}%
\pgfpathlineto{\pgfqpoint{5.279905in}{0.591095in}}%
\pgfpathlineto{\pgfqpoint{5.285002in}{0.615763in}}%
\pgfpathlineto{\pgfqpoint{5.290948in}{0.668612in}}%
\pgfpathlineto{\pgfqpoint{5.291373in}{0.668853in}}%
\pgfpathlineto{\pgfqpoint{5.291797in}{0.589332in}}%
\pgfpathlineto{\pgfqpoint{5.292647in}{0.589823in}}%
\pgfpathlineto{\pgfqpoint{5.295195in}{0.589546in}}%
\pgfpathlineto{\pgfqpoint{5.298168in}{0.589221in}}%
\pgfpathlineto{\pgfqpoint{5.300292in}{0.591247in}}%
\pgfpathlineto{\pgfqpoint{5.302840in}{0.593687in}}%
\pgfpathlineto{\pgfqpoint{5.307088in}{0.629275in}}%
\pgfpathlineto{\pgfqpoint{5.309636in}{0.634327in}}%
\pgfpathlineto{\pgfqpoint{5.313883in}{0.638593in}}%
\pgfpathlineto{\pgfqpoint{5.316856in}{0.641522in}}%
\pgfpathlineto{\pgfqpoint{5.317281in}{0.589244in}}%
\pgfpathlineto{\pgfqpoint{5.318131in}{0.589637in}}%
\pgfpathlineto{\pgfqpoint{5.321528in}{0.592964in}}%
\pgfpathlineto{\pgfqpoint{5.327050in}{0.638238in}}%
\pgfpathlineto{\pgfqpoint{5.327475in}{0.589351in}}%
\pgfpathlineto{\pgfqpoint{5.328324in}{0.590378in}}%
\pgfpathlineto{\pgfqpoint{5.330023in}{0.590873in}}%
\pgfpathlineto{\pgfqpoint{5.332571in}{0.590483in}}%
\pgfpathlineto{\pgfqpoint{5.334695in}{0.591902in}}%
\pgfpathlineto{\pgfqpoint{5.336819in}{0.589365in}}%
\pgfpathlineto{\pgfqpoint{5.337668in}{0.590920in}}%
\pgfpathlineto{\pgfqpoint{5.340216in}{0.593480in}}%
\pgfpathlineto{\pgfqpoint{5.341066in}{0.593402in}}%
\pgfpathlineto{\pgfqpoint{5.344039in}{0.613415in}}%
\pgfpathlineto{\pgfqpoint{5.345313in}{0.615249in}}%
\pgfpathlineto{\pgfqpoint{5.345738in}{0.589215in}}%
\pgfpathlineto{\pgfqpoint{5.346587in}{0.589414in}}%
\pgfpathlineto{\pgfqpoint{5.352109in}{0.589745in}}%
\pgfpathlineto{\pgfqpoint{5.354657in}{0.590232in}}%
\pgfpathlineto{\pgfqpoint{5.356781in}{0.590530in}}%
\pgfpathlineto{\pgfqpoint{5.358905in}{0.592920in}}%
\pgfpathlineto{\pgfqpoint{5.361028in}{0.599726in}}%
\pgfpathlineto{\pgfqpoint{5.365275in}{0.623939in}}%
\pgfpathlineto{\pgfqpoint{5.367399in}{0.630724in}}%
\pgfpathlineto{\pgfqpoint{5.367824in}{0.589483in}}%
\pgfpathlineto{\pgfqpoint{5.368673in}{0.590698in}}%
\pgfpathlineto{\pgfqpoint{5.371222in}{0.592141in}}%
\pgfpathlineto{\pgfqpoint{5.372071in}{0.591658in}}%
\pgfpathlineto{\pgfqpoint{5.374620in}{0.599131in}}%
\pgfpathlineto{\pgfqpoint{5.377168in}{0.602294in}}%
\pgfpathlineto{\pgfqpoint{5.379716in}{0.611818in}}%
\pgfpathlineto{\pgfqpoint{5.382265in}{0.619756in}}%
\pgfpathlineto{\pgfqpoint{5.382689in}{0.589217in}}%
\pgfpathlineto{\pgfqpoint{5.383539in}{0.590313in}}%
\pgfpathlineto{\pgfqpoint{5.386512in}{0.592448in}}%
\pgfpathlineto{\pgfqpoint{5.386937in}{0.592322in}}%
\pgfpathlineto{\pgfqpoint{5.389910in}{0.605806in}}%
\pgfpathlineto{\pgfqpoint{5.392033in}{0.607284in}}%
\pgfpathlineto{\pgfqpoint{5.393308in}{0.589222in}}%
\pgfpathlineto{\pgfqpoint{5.399254in}{0.589682in}}%
\pgfpathlineto{\pgfqpoint{5.402652in}{0.591020in}}%
\pgfpathlineto{\pgfqpoint{5.403076in}{0.590776in}}%
\pgfpathlineto{\pgfqpoint{5.405200in}{0.593918in}}%
\pgfpathlineto{\pgfqpoint{5.406474in}{0.589241in}}%
\pgfpathlineto{\pgfqpoint{5.410722in}{0.591193in}}%
\pgfpathlineto{\pgfqpoint{5.415394in}{0.607654in}}%
\pgfpathlineto{\pgfqpoint{5.417517in}{0.610695in}}%
\pgfpathlineto{\pgfqpoint{5.417942in}{0.589358in}}%
\pgfpathlineto{\pgfqpoint{5.418791in}{0.589993in}}%
\pgfpathlineto{\pgfqpoint{5.422189in}{0.593644in}}%
\pgfpathlineto{\pgfqpoint{5.426861in}{0.636538in}}%
\pgfpathlineto{\pgfqpoint{5.427286in}{0.589275in}}%
\pgfpathlineto{\pgfqpoint{5.428135in}{0.590533in}}%
\pgfpathlineto{\pgfqpoint{5.429834in}{0.590800in}}%
\pgfpathlineto{\pgfqpoint{5.431958in}{0.589813in}}%
\pgfpathlineto{\pgfqpoint{5.434082in}{0.590473in}}%
\pgfpathlineto{\pgfqpoint{5.436205in}{0.590172in}}%
\pgfpathlineto{\pgfqpoint{5.438754in}{0.591949in}}%
\pgfpathlineto{\pgfqpoint{5.440028in}{0.591365in}}%
\pgfpathlineto{\pgfqpoint{5.442576in}{0.597556in}}%
\pgfpathlineto{\pgfqpoint{5.444700in}{0.600345in}}%
\pgfpathlineto{\pgfqpoint{5.446399in}{0.608591in}}%
\pgfpathlineto{\pgfqpoint{5.451071in}{0.635527in}}%
\pgfpathlineto{\pgfqpoint{5.451496in}{0.589877in}}%
\pgfpathlineto{\pgfqpoint{5.452345in}{0.590992in}}%
\pgfpathlineto{\pgfqpoint{5.454469in}{0.591401in}}%
\pgfpathlineto{\pgfqpoint{5.455743in}{0.590543in}}%
\pgfpathlineto{\pgfqpoint{5.456168in}{0.591620in}}%
\pgfpathlineto{\pgfqpoint{5.459990in}{0.601601in}}%
\pgfpathlineto{\pgfqpoint{5.465936in}{0.632793in}}%
\pgfpathlineto{\pgfqpoint{5.466361in}{0.589449in}}%
\pgfpathlineto{\pgfqpoint{5.467211in}{0.590122in}}%
\pgfpathlineto{\pgfqpoint{5.469334in}{0.589963in}}%
\pgfpathlineto{\pgfqpoint{5.471458in}{0.589819in}}%
\pgfpathlineto{\pgfqpoint{5.473581in}{0.589403in}}%
\pgfpathlineto{\pgfqpoint{5.474856in}{0.590381in}}%
\pgfpathlineto{\pgfqpoint{5.477829in}{0.592754in}}%
\pgfpathlineto{\pgfqpoint{5.478253in}{0.592728in}}%
\pgfpathlineto{\pgfqpoint{5.481227in}{0.609199in}}%
\pgfpathlineto{\pgfqpoint{5.481651in}{0.609570in}}%
\pgfpathlineto{\pgfqpoint{5.482076in}{0.589378in}}%
\pgfpathlineto{\pgfqpoint{5.482926in}{0.590337in}}%
\pgfpathlineto{\pgfqpoint{5.485049in}{0.590748in}}%
\pgfpathlineto{\pgfqpoint{5.486748in}{0.589571in}}%
\pgfpathlineto{\pgfqpoint{5.487173in}{0.590188in}}%
\pgfpathlineto{\pgfqpoint{5.489721in}{0.592078in}}%
\pgfpathlineto{\pgfqpoint{5.490995in}{0.591037in}}%
\pgfpathlineto{\pgfqpoint{5.491420in}{0.592302in}}%
\pgfpathlineto{\pgfqpoint{5.491845in}{0.592913in}}%
\pgfpathlineto{\pgfqpoint{5.492270in}{0.589278in}}%
\pgfpathlineto{\pgfqpoint{5.493119in}{0.590703in}}%
\pgfpathlineto{\pgfqpoint{5.496092in}{0.593581in}}%
\pgfpathlineto{\pgfqpoint{5.496517in}{0.593525in}}%
\pgfpathlineto{\pgfqpoint{5.499915in}{0.617725in}}%
\pgfpathlineto{\pgfqpoint{5.503313in}{0.627553in}}%
\pgfpathlineto{\pgfqpoint{5.504587in}{0.628406in}}%
\pgfpathlineto{\pgfqpoint{5.505011in}{0.589406in}}%
\pgfpathlineto{\pgfqpoint{5.505861in}{0.589887in}}%
\pgfpathlineto{\pgfqpoint{5.509683in}{0.592019in}}%
\pgfpathlineto{\pgfqpoint{5.517329in}{0.619320in}}%
\pgfpathlineto{\pgfqpoint{5.522850in}{0.653416in}}%
\pgfpathlineto{\pgfqpoint{5.524549in}{0.656888in}}%
\pgfpathlineto{\pgfqpoint{5.524974in}{0.589261in}}%
\pgfpathlineto{\pgfqpoint{5.525823in}{0.590153in}}%
\pgfpathlineto{\pgfqpoint{5.529646in}{0.595665in}}%
\pgfpathlineto{\pgfqpoint{5.535592in}{0.636427in}}%
\pgfpathlineto{\pgfqpoint{5.536441in}{0.637141in}}%
\pgfpathlineto{\pgfqpoint{5.536866in}{0.589194in}}%
\pgfpathlineto{\pgfqpoint{5.537716in}{0.589340in}}%
\pgfpathlineto{\pgfqpoint{5.540689in}{0.590058in}}%
\pgfpathlineto{\pgfqpoint{5.543237in}{0.591458in}}%
\pgfpathlineto{\pgfqpoint{5.545785in}{0.601386in}}%
\pgfpathlineto{\pgfqpoint{5.546210in}{0.589557in}}%
\pgfpathlineto{\pgfqpoint{5.547060in}{0.591300in}}%
\pgfpathlineto{\pgfqpoint{5.549608in}{0.594246in}}%
\pgfpathlineto{\pgfqpoint{5.550457in}{0.594089in}}%
\pgfpathlineto{\pgfqpoint{5.555130in}{0.635622in}}%
\pgfpathlineto{\pgfqpoint{5.557253in}{0.642028in}}%
\pgfpathlineto{\pgfqpoint{5.557678in}{0.589360in}}%
\pgfpathlineto{\pgfqpoint{5.558527in}{0.589992in}}%
\pgfpathlineto{\pgfqpoint{5.562350in}{0.592864in}}%
\pgfpathlineto{\pgfqpoint{5.575941in}{0.656674in}}%
\pgfpathlineto{\pgfqpoint{5.578065in}{0.659094in}}%
\pgfpathlineto{\pgfqpoint{5.578490in}{0.589478in}}%
\pgfpathlineto{\pgfqpoint{5.579339in}{0.590337in}}%
\pgfpathlineto{\pgfqpoint{5.582737in}{0.593441in}}%
\pgfpathlineto{\pgfqpoint{5.586559in}{0.621054in}}%
\pgfpathlineto{\pgfqpoint{5.587409in}{0.621611in}}%
\pgfpathlineto{\pgfqpoint{5.587834in}{0.589622in}}%
\pgfpathlineto{\pgfqpoint{5.588683in}{0.590223in}}%
\pgfpathlineto{\pgfqpoint{5.590807in}{0.589718in}}%
\pgfpathlineto{\pgfqpoint{5.592081in}{0.589787in}}%
\pgfpathlineto{\pgfqpoint{5.594629in}{0.590757in}}%
\pgfpathlineto{\pgfqpoint{5.598877in}{0.590406in}}%
\pgfpathlineto{\pgfqpoint{5.601425in}{0.591171in}}%
\pgfpathlineto{\pgfqpoint{5.606522in}{0.610257in}}%
\pgfpathlineto{\pgfqpoint{5.608645in}{0.611562in}}%
\pgfpathlineto{\pgfqpoint{5.609920in}{0.589190in}}%
\pgfpathlineto{\pgfqpoint{5.614592in}{0.590811in}}%
\pgfpathlineto{\pgfqpoint{5.615866in}{0.593830in}}%
\pgfpathlineto{\pgfqpoint{5.616291in}{0.589190in}}%
\pgfpathlineto{\pgfqpoint{5.617140in}{0.590475in}}%
\pgfpathlineto{\pgfqpoint{5.619688in}{0.592561in}}%
\pgfpathlineto{\pgfqpoint{5.620538in}{0.592086in}}%
\pgfpathlineto{\pgfqpoint{5.622661in}{0.598591in}}%
\pgfpathlineto{\pgfqpoint{5.623086in}{0.589646in}}%
\pgfpathlineto{\pgfqpoint{5.623936in}{0.590475in}}%
\pgfpathlineto{\pgfqpoint{5.625635in}{0.590602in}}%
\pgfpathlineto{\pgfqpoint{5.627758in}{0.590069in}}%
\pgfpathlineto{\pgfqpoint{5.630307in}{0.591498in}}%
\pgfpathlineto{\pgfqpoint{5.632430in}{0.589786in}}%
\pgfpathlineto{\pgfqpoint{5.632855in}{0.590565in}}%
\pgfpathlineto{\pgfqpoint{5.635403in}{0.593121in}}%
\pgfpathlineto{\pgfqpoint{5.636678in}{0.592103in}}%
\pgfpathlineto{\pgfqpoint{5.637952in}{0.596643in}}%
\pgfpathlineto{\pgfqpoint{5.638376in}{0.589280in}}%
\pgfpathlineto{\pgfqpoint{5.639226in}{0.590835in}}%
\pgfpathlineto{\pgfqpoint{5.641350in}{0.592656in}}%
\pgfpathlineto{\pgfqpoint{5.642624in}{0.591916in}}%
\pgfpathlineto{\pgfqpoint{5.645172in}{0.600406in}}%
\pgfpathlineto{\pgfqpoint{5.647721in}{0.607303in}}%
\pgfpathlineto{\pgfqpoint{5.650694in}{0.614536in}}%
\pgfpathlineto{\pgfqpoint{5.651118in}{0.614537in}}%
\pgfpathlineto{\pgfqpoint{5.651543in}{0.589465in}}%
\pgfpathlineto{\pgfqpoint{5.652393in}{0.589978in}}%
\pgfpathlineto{\pgfqpoint{5.655790in}{0.592143in}}%
\pgfpathlineto{\pgfqpoint{5.662586in}{0.634718in}}%
\pgfpathlineto{\pgfqpoint{5.664710in}{0.640998in}}%
\pgfpathlineto{\pgfqpoint{5.665134in}{0.589361in}}%
\pgfpathlineto{\pgfqpoint{5.665984in}{0.590331in}}%
\pgfpathlineto{\pgfqpoint{5.668532in}{0.591133in}}%
\pgfpathlineto{\pgfqpoint{5.669382in}{0.590933in}}%
\pgfpathlineto{\pgfqpoint{5.674478in}{0.615221in}}%
\pgfpathlineto{\pgfqpoint{5.677876in}{0.627960in}}%
\pgfpathlineto{\pgfqpoint{5.680425in}{0.631433in}}%
\pgfpathlineto{\pgfqpoint{5.680849in}{0.589196in}}%
\pgfpathlineto{\pgfqpoint{5.681699in}{0.589545in}}%
\pgfpathlineto{\pgfqpoint{5.684672in}{0.589387in}}%
\pgfpathlineto{\pgfqpoint{5.687645in}{0.589232in}}%
\pgfpathlineto{\pgfqpoint{5.691892in}{0.591594in}}%
\pgfpathlineto{\pgfqpoint{5.696140in}{0.611347in}}%
\pgfpathlineto{\pgfqpoint{5.696564in}{0.611624in}}%
\pgfpathlineto{\pgfqpoint{5.696989in}{0.589424in}}%
\pgfpathlineto{\pgfqpoint{5.697839in}{0.590410in}}%
\pgfpathlineto{\pgfqpoint{5.700387in}{0.591396in}}%
\pgfpathlineto{\pgfqpoint{5.701236in}{0.590956in}}%
\pgfpathlineto{\pgfqpoint{5.703360in}{0.595487in}}%
\pgfpathlineto{\pgfqpoint{5.708457in}{0.599291in}}%
\pgfpathlineto{\pgfqpoint{5.711005in}{0.601763in}}%
\pgfpathlineto{\pgfqpoint{5.711430in}{0.589221in}}%
\pgfpathlineto{\pgfqpoint{5.712279in}{0.589673in}}%
\pgfpathlineto{\pgfqpoint{5.715677in}{0.591741in}}%
\pgfpathlineto{\pgfqpoint{5.719075in}{0.607192in}}%
\pgfpathlineto{\pgfqpoint{5.719500in}{0.589378in}}%
\pgfpathlineto{\pgfqpoint{5.720349in}{0.590611in}}%
\pgfpathlineto{\pgfqpoint{5.722473in}{0.591649in}}%
\pgfpathlineto{\pgfqpoint{5.724597in}{0.589284in}}%
\pgfpathlineto{\pgfqpoint{5.725446in}{0.590599in}}%
\pgfpathlineto{\pgfqpoint{5.727570in}{0.592160in}}%
\pgfpathlineto{\pgfqpoint{5.728844in}{0.591355in}}%
\pgfpathlineto{\pgfqpoint{5.730118in}{0.594329in}}%
\pgfpathlineto{\pgfqpoint{5.730543in}{0.589587in}}%
\pgfpathlineto{\pgfqpoint{5.731392in}{0.590979in}}%
\pgfpathlineto{\pgfqpoint{5.733516in}{0.592625in}}%
\pgfpathlineto{\pgfqpoint{5.734790in}{0.591939in}}%
\pgfpathlineto{\pgfqpoint{5.737338in}{0.600082in}}%
\pgfpathlineto{\pgfqpoint{5.738188in}{0.600387in}}%
\pgfpathlineto{\pgfqpoint{5.738613in}{0.589251in}}%
\pgfpathlineto{\pgfqpoint{5.739462in}{0.589377in}}%
\pgfpathlineto{\pgfqpoint{5.744984in}{0.590588in}}%
\pgfpathlineto{\pgfqpoint{5.746258in}{0.591105in}}%
\pgfpathlineto{\pgfqpoint{5.746258in}{0.591105in}}%
\pgfusepath{stroke}%
\end{pgfscope}%
\begin{pgfscope}%
\pgfsetrectcap%
\pgfsetmiterjoin%
\pgfsetlinewidth{0.803000pt}%
\definecolor{currentstroke}{rgb}{0.737255,0.737255,0.737255}%
\pgfsetstrokecolor{currentstroke}%
\pgfsetdash{}{0pt}%
\pgfpathmoveto{\pgfqpoint{0.649081in}{0.544166in}}%
\pgfpathlineto{\pgfqpoint{0.649081in}{1.534682in}}%
\pgfusepath{stroke}%
\end{pgfscope}%
\begin{pgfscope}%
\pgfsetrectcap%
\pgfsetmiterjoin%
\pgfsetlinewidth{0.803000pt}%
\definecolor{currentstroke}{rgb}{0.737255,0.737255,0.737255}%
\pgfsetstrokecolor{currentstroke}%
\pgfsetdash{}{0pt}%
\pgfpathmoveto{\pgfqpoint{5.745833in}{0.544166in}}%
\pgfpathlineto{\pgfqpoint{5.745833in}{1.534682in}}%
\pgfusepath{stroke}%
\end{pgfscope}%
\begin{pgfscope}%
\pgfsetrectcap%
\pgfsetmiterjoin%
\pgfsetlinewidth{0.803000pt}%
\definecolor{currentstroke}{rgb}{0.737255,0.737255,0.737255}%
\pgfsetstrokecolor{currentstroke}%
\pgfsetdash{}{0pt}%
\pgfpathmoveto{\pgfqpoint{0.649081in}{0.544166in}}%
\pgfpathlineto{\pgfqpoint{5.745833in}{0.544166in}}%
\pgfusepath{stroke}%
\end{pgfscope}%
\begin{pgfscope}%
\pgfsetrectcap%
\pgfsetmiterjoin%
\pgfsetlinewidth{0.803000pt}%
\definecolor{currentstroke}{rgb}{0.737255,0.737255,0.737255}%
\pgfsetstrokecolor{currentstroke}%
\pgfsetdash{}{0pt}%
\pgfpathmoveto{\pgfqpoint{0.649081in}{1.534682in}}%
\pgfpathlineto{\pgfqpoint{5.745833in}{1.534682in}}%
\pgfusepath{stroke}%
\end{pgfscope}%
\end{pgfpicture}%
\makeatother%
\endgroup%
}
    \caption{Comparaison de l'enveloppe de Stewart (bleu) au signal d'entrée (gris).}
    \label{fig:env-stewart}
\end{figure}

\cite{stewart1977} a utilisé une enveloppe modifiée $mdx$ basée sur la dérivée des données mettant en évidence les changements de pente. La valeur de $mdx$ est calculée à partir d'une estimation de la dérivée $dx$ en chaque point [d'après \cite{wither1998}] :
\begin{equation}
   dx_i=x_i-x_{i-1}
\end{equation}
Si le signe de $dx$ a été constant pour moins de 8 valeurs consécutives, alors 
\begin{equation}
   mdx_i = mdx_i + dx_i
\end{equation}
Sinon,
\begin{equation}
   mdx_i = dx_i
\end{equation}

Cette transformation permet de mettre en avant les variations . Il agit en quelque sorte comme un filtre passe haut, et est particulièrement utile pour des signaux bruts, donc l'intérêt est limité étant donné que notre signal a déjà été filtré.

\subsubsection{Enveloppe supérieure et approximation géométrique}

\begin{figure}[!ht]
    \centering
    \scalebox{1}{%% Creator: Matplotlib, PGF backend
%%
%% To include the figure in your LaTeX document, write
%%   \input{<filename>.pgf}
%%
%% Make sure the required packages are loaded in your preamble
%%   \usepackage{pgf}
%%
%% Also ensure that all the required font packages are loaded; for instance,
%% the lmodern package is sometimes necessary when using math font.
%%   \usepackage{lmodern}
%%
%% Figures using additional raster images can only be included by \input if
%% they are in the same directory as the main LaTeX file. For loading figures
%% from other directories you can use the `import` package
%%   \usepackage{import}
%%
%% and then include the figures with
%%   \import{<path to file>}{<filename>.pgf}
%%
%% Matplotlib used the following preamble
%%   \usepackage{fontspec}
%%
\begingroup%
\makeatletter%
\begin{pgfpicture}%
\pgfpathrectangle{\pgfpointorigin}{\pgfqpoint{6.000000in}{3.000000in}}%
\pgfusepath{use as bounding box, clip}%
\begin{pgfscope}%
\pgfsetbuttcap%
\pgfsetmiterjoin%
\definecolor{currentfill}{rgb}{1.000000,1.000000,1.000000}%
\pgfsetfillcolor{currentfill}%
\pgfsetlinewidth{0.000000pt}%
\definecolor{currentstroke}{rgb}{1.000000,1.000000,1.000000}%
\pgfsetstrokecolor{currentstroke}%
\pgfsetdash{}{0pt}%
\pgfpathmoveto{\pgfqpoint{0.000000in}{0.000000in}}%
\pgfpathlineto{\pgfqpoint{6.000000in}{0.000000in}}%
\pgfpathlineto{\pgfqpoint{6.000000in}{3.000000in}}%
\pgfpathlineto{\pgfqpoint{0.000000in}{3.000000in}}%
\pgfpathlineto{\pgfqpoint{0.000000in}{0.000000in}}%
\pgfpathclose%
\pgfusepath{fill}%
\end{pgfscope}%
\begin{pgfscope}%
\pgfsetbuttcap%
\pgfsetmiterjoin%
\definecolor{currentfill}{rgb}{0.933333,0.933333,0.933333}%
\pgfsetfillcolor{currentfill}%
\pgfsetlinewidth{0.000000pt}%
\definecolor{currentstroke}{rgb}{0.000000,0.000000,0.000000}%
\pgfsetstrokecolor{currentstroke}%
\pgfsetstrokeopacity{0.000000}%
\pgfsetdash{}{0pt}%
\pgfpathmoveto{\pgfqpoint{0.750000in}{0.330000in}}%
\pgfpathlineto{\pgfqpoint{5.400000in}{0.330000in}}%
\pgfpathlineto{\pgfqpoint{5.400000in}{2.640000in}}%
\pgfpathlineto{\pgfqpoint{0.750000in}{2.640000in}}%
\pgfpathlineto{\pgfqpoint{0.750000in}{0.330000in}}%
\pgfpathclose%
\pgfusepath{fill}%
\end{pgfscope}%
\begin{pgfscope}%
\pgfpathrectangle{\pgfqpoint{0.750000in}{0.330000in}}{\pgfqpoint{4.650000in}{2.310000in}}%
\pgfusepath{clip}%
\pgfsetbuttcap%
\pgfsetroundjoin%
\pgfsetlinewidth{0.501875pt}%
\definecolor{currentstroke}{rgb}{0.698039,0.698039,0.698039}%
\pgfsetstrokecolor{currentstroke}%
\pgfsetdash{{1.850000pt}{0.800000pt}}{0.000000pt}%
\pgfpathmoveto{\pgfqpoint{0.750000in}{0.330000in}}%
\pgfpathlineto{\pgfqpoint{0.750000in}{2.640000in}}%
\pgfusepath{stroke}%
\end{pgfscope}%
\begin{pgfscope}%
\pgfsetbuttcap%
\pgfsetroundjoin%
\definecolor{currentfill}{rgb}{0.000000,0.000000,0.000000}%
\pgfsetfillcolor{currentfill}%
\pgfsetlinewidth{0.803000pt}%
\definecolor{currentstroke}{rgb}{0.000000,0.000000,0.000000}%
\pgfsetstrokecolor{currentstroke}%
\pgfsetdash{}{0pt}%
\pgfsys@defobject{currentmarker}{\pgfqpoint{0.000000in}{0.000000in}}{\pgfqpoint{0.000000in}{0.048611in}}{%
\pgfpathmoveto{\pgfqpoint{0.000000in}{0.000000in}}%
\pgfpathlineto{\pgfqpoint{0.000000in}{0.048611in}}%
\pgfusepath{stroke,fill}%
}%
\begin{pgfscope}%
\pgfsys@transformshift{0.750000in}{0.330000in}%
\pgfsys@useobject{currentmarker}{}%
\end{pgfscope}%
\end{pgfscope}%
\begin{pgfscope}%
\definecolor{textcolor}{rgb}{0.000000,0.000000,0.000000}%
\pgfsetstrokecolor{textcolor}%
\pgfsetfillcolor{textcolor}%
\pgftext[x=0.750000in,y=0.281389in,,top]{\color{textcolor}\rmfamily\fontsize{10.000000}{12.000000}\selectfont \(\displaystyle {44}\)}%
\end{pgfscope}%
\begin{pgfscope}%
\pgfpathrectangle{\pgfqpoint{0.750000in}{0.330000in}}{\pgfqpoint{4.650000in}{2.310000in}}%
\pgfusepath{clip}%
\pgfsetbuttcap%
\pgfsetroundjoin%
\pgfsetlinewidth{0.501875pt}%
\definecolor{currentstroke}{rgb}{0.698039,0.698039,0.698039}%
\pgfsetstrokecolor{currentstroke}%
\pgfsetdash{{1.850000pt}{0.800000pt}}{0.000000pt}%
\pgfpathmoveto{\pgfqpoint{1.266667in}{0.330000in}}%
\pgfpathlineto{\pgfqpoint{1.266667in}{2.640000in}}%
\pgfusepath{stroke}%
\end{pgfscope}%
\begin{pgfscope}%
\pgfsetbuttcap%
\pgfsetroundjoin%
\definecolor{currentfill}{rgb}{0.000000,0.000000,0.000000}%
\pgfsetfillcolor{currentfill}%
\pgfsetlinewidth{0.803000pt}%
\definecolor{currentstroke}{rgb}{0.000000,0.000000,0.000000}%
\pgfsetstrokecolor{currentstroke}%
\pgfsetdash{}{0pt}%
\pgfsys@defobject{currentmarker}{\pgfqpoint{0.000000in}{0.000000in}}{\pgfqpoint{0.000000in}{0.048611in}}{%
\pgfpathmoveto{\pgfqpoint{0.000000in}{0.000000in}}%
\pgfpathlineto{\pgfqpoint{0.000000in}{0.048611in}}%
\pgfusepath{stroke,fill}%
}%
\begin{pgfscope}%
\pgfsys@transformshift{1.266667in}{0.330000in}%
\pgfsys@useobject{currentmarker}{}%
\end{pgfscope}%
\end{pgfscope}%
\begin{pgfscope}%
\definecolor{textcolor}{rgb}{0.000000,0.000000,0.000000}%
\pgfsetstrokecolor{textcolor}%
\pgfsetfillcolor{textcolor}%
\pgftext[x=1.266667in,y=0.281389in,,top]{\color{textcolor}\rmfamily\fontsize{10.000000}{12.000000}\selectfont \(\displaystyle {45}\)}%
\end{pgfscope}%
\begin{pgfscope}%
\pgfpathrectangle{\pgfqpoint{0.750000in}{0.330000in}}{\pgfqpoint{4.650000in}{2.310000in}}%
\pgfusepath{clip}%
\pgfsetbuttcap%
\pgfsetroundjoin%
\pgfsetlinewidth{0.501875pt}%
\definecolor{currentstroke}{rgb}{0.698039,0.698039,0.698039}%
\pgfsetstrokecolor{currentstroke}%
\pgfsetdash{{1.850000pt}{0.800000pt}}{0.000000pt}%
\pgfpathmoveto{\pgfqpoint{1.783333in}{0.330000in}}%
\pgfpathlineto{\pgfqpoint{1.783333in}{2.640000in}}%
\pgfusepath{stroke}%
\end{pgfscope}%
\begin{pgfscope}%
\pgfsetbuttcap%
\pgfsetroundjoin%
\definecolor{currentfill}{rgb}{0.000000,0.000000,0.000000}%
\pgfsetfillcolor{currentfill}%
\pgfsetlinewidth{0.803000pt}%
\definecolor{currentstroke}{rgb}{0.000000,0.000000,0.000000}%
\pgfsetstrokecolor{currentstroke}%
\pgfsetdash{}{0pt}%
\pgfsys@defobject{currentmarker}{\pgfqpoint{0.000000in}{0.000000in}}{\pgfqpoint{0.000000in}{0.048611in}}{%
\pgfpathmoveto{\pgfqpoint{0.000000in}{0.000000in}}%
\pgfpathlineto{\pgfqpoint{0.000000in}{0.048611in}}%
\pgfusepath{stroke,fill}%
}%
\begin{pgfscope}%
\pgfsys@transformshift{1.783333in}{0.330000in}%
\pgfsys@useobject{currentmarker}{}%
\end{pgfscope}%
\end{pgfscope}%
\begin{pgfscope}%
\definecolor{textcolor}{rgb}{0.000000,0.000000,0.000000}%
\pgfsetstrokecolor{textcolor}%
\pgfsetfillcolor{textcolor}%
\pgftext[x=1.783333in,y=0.281389in,,top]{\color{textcolor}\rmfamily\fontsize{10.000000}{12.000000}\selectfont \(\displaystyle {46}\)}%
\end{pgfscope}%
\begin{pgfscope}%
\pgfpathrectangle{\pgfqpoint{0.750000in}{0.330000in}}{\pgfqpoint{4.650000in}{2.310000in}}%
\pgfusepath{clip}%
\pgfsetbuttcap%
\pgfsetroundjoin%
\pgfsetlinewidth{0.501875pt}%
\definecolor{currentstroke}{rgb}{0.698039,0.698039,0.698039}%
\pgfsetstrokecolor{currentstroke}%
\pgfsetdash{{1.850000pt}{0.800000pt}}{0.000000pt}%
\pgfpathmoveto{\pgfqpoint{2.300000in}{0.330000in}}%
\pgfpathlineto{\pgfqpoint{2.300000in}{2.640000in}}%
\pgfusepath{stroke}%
\end{pgfscope}%
\begin{pgfscope}%
\pgfsetbuttcap%
\pgfsetroundjoin%
\definecolor{currentfill}{rgb}{0.000000,0.000000,0.000000}%
\pgfsetfillcolor{currentfill}%
\pgfsetlinewidth{0.803000pt}%
\definecolor{currentstroke}{rgb}{0.000000,0.000000,0.000000}%
\pgfsetstrokecolor{currentstroke}%
\pgfsetdash{}{0pt}%
\pgfsys@defobject{currentmarker}{\pgfqpoint{0.000000in}{0.000000in}}{\pgfqpoint{0.000000in}{0.048611in}}{%
\pgfpathmoveto{\pgfqpoint{0.000000in}{0.000000in}}%
\pgfpathlineto{\pgfqpoint{0.000000in}{0.048611in}}%
\pgfusepath{stroke,fill}%
}%
\begin{pgfscope}%
\pgfsys@transformshift{2.300000in}{0.330000in}%
\pgfsys@useobject{currentmarker}{}%
\end{pgfscope}%
\end{pgfscope}%
\begin{pgfscope}%
\definecolor{textcolor}{rgb}{0.000000,0.000000,0.000000}%
\pgfsetstrokecolor{textcolor}%
\pgfsetfillcolor{textcolor}%
\pgftext[x=2.300000in,y=0.281389in,,top]{\color{textcolor}\rmfamily\fontsize{10.000000}{12.000000}\selectfont \(\displaystyle {47}\)}%
\end{pgfscope}%
\begin{pgfscope}%
\pgfpathrectangle{\pgfqpoint{0.750000in}{0.330000in}}{\pgfqpoint{4.650000in}{2.310000in}}%
\pgfusepath{clip}%
\pgfsetbuttcap%
\pgfsetroundjoin%
\pgfsetlinewidth{0.501875pt}%
\definecolor{currentstroke}{rgb}{0.698039,0.698039,0.698039}%
\pgfsetstrokecolor{currentstroke}%
\pgfsetdash{{1.850000pt}{0.800000pt}}{0.000000pt}%
\pgfpathmoveto{\pgfqpoint{2.816667in}{0.330000in}}%
\pgfpathlineto{\pgfqpoint{2.816667in}{2.640000in}}%
\pgfusepath{stroke}%
\end{pgfscope}%
\begin{pgfscope}%
\pgfsetbuttcap%
\pgfsetroundjoin%
\definecolor{currentfill}{rgb}{0.000000,0.000000,0.000000}%
\pgfsetfillcolor{currentfill}%
\pgfsetlinewidth{0.803000pt}%
\definecolor{currentstroke}{rgb}{0.000000,0.000000,0.000000}%
\pgfsetstrokecolor{currentstroke}%
\pgfsetdash{}{0pt}%
\pgfsys@defobject{currentmarker}{\pgfqpoint{0.000000in}{0.000000in}}{\pgfqpoint{0.000000in}{0.048611in}}{%
\pgfpathmoveto{\pgfqpoint{0.000000in}{0.000000in}}%
\pgfpathlineto{\pgfqpoint{0.000000in}{0.048611in}}%
\pgfusepath{stroke,fill}%
}%
\begin{pgfscope}%
\pgfsys@transformshift{2.816667in}{0.330000in}%
\pgfsys@useobject{currentmarker}{}%
\end{pgfscope}%
\end{pgfscope}%
\begin{pgfscope}%
\definecolor{textcolor}{rgb}{0.000000,0.000000,0.000000}%
\pgfsetstrokecolor{textcolor}%
\pgfsetfillcolor{textcolor}%
\pgftext[x=2.816667in,y=0.281389in,,top]{\color{textcolor}\rmfamily\fontsize{10.000000}{12.000000}\selectfont \(\displaystyle {48}\)}%
\end{pgfscope}%
\begin{pgfscope}%
\pgfpathrectangle{\pgfqpoint{0.750000in}{0.330000in}}{\pgfqpoint{4.650000in}{2.310000in}}%
\pgfusepath{clip}%
\pgfsetbuttcap%
\pgfsetroundjoin%
\pgfsetlinewidth{0.501875pt}%
\definecolor{currentstroke}{rgb}{0.698039,0.698039,0.698039}%
\pgfsetstrokecolor{currentstroke}%
\pgfsetdash{{1.850000pt}{0.800000pt}}{0.000000pt}%
\pgfpathmoveto{\pgfqpoint{3.333333in}{0.330000in}}%
\pgfpathlineto{\pgfqpoint{3.333333in}{2.640000in}}%
\pgfusepath{stroke}%
\end{pgfscope}%
\begin{pgfscope}%
\pgfsetbuttcap%
\pgfsetroundjoin%
\definecolor{currentfill}{rgb}{0.000000,0.000000,0.000000}%
\pgfsetfillcolor{currentfill}%
\pgfsetlinewidth{0.803000pt}%
\definecolor{currentstroke}{rgb}{0.000000,0.000000,0.000000}%
\pgfsetstrokecolor{currentstroke}%
\pgfsetdash{}{0pt}%
\pgfsys@defobject{currentmarker}{\pgfqpoint{0.000000in}{0.000000in}}{\pgfqpoint{0.000000in}{0.048611in}}{%
\pgfpathmoveto{\pgfqpoint{0.000000in}{0.000000in}}%
\pgfpathlineto{\pgfqpoint{0.000000in}{0.048611in}}%
\pgfusepath{stroke,fill}%
}%
\begin{pgfscope}%
\pgfsys@transformshift{3.333333in}{0.330000in}%
\pgfsys@useobject{currentmarker}{}%
\end{pgfscope}%
\end{pgfscope}%
\begin{pgfscope}%
\definecolor{textcolor}{rgb}{0.000000,0.000000,0.000000}%
\pgfsetstrokecolor{textcolor}%
\pgfsetfillcolor{textcolor}%
\pgftext[x=3.333333in,y=0.281389in,,top]{\color{textcolor}\rmfamily\fontsize{10.000000}{12.000000}\selectfont \(\displaystyle {49}\)}%
\end{pgfscope}%
\begin{pgfscope}%
\pgfpathrectangle{\pgfqpoint{0.750000in}{0.330000in}}{\pgfqpoint{4.650000in}{2.310000in}}%
\pgfusepath{clip}%
\pgfsetbuttcap%
\pgfsetroundjoin%
\pgfsetlinewidth{0.501875pt}%
\definecolor{currentstroke}{rgb}{0.698039,0.698039,0.698039}%
\pgfsetstrokecolor{currentstroke}%
\pgfsetdash{{1.850000pt}{0.800000pt}}{0.000000pt}%
\pgfpathmoveto{\pgfqpoint{3.850000in}{0.330000in}}%
\pgfpathlineto{\pgfqpoint{3.850000in}{2.640000in}}%
\pgfusepath{stroke}%
\end{pgfscope}%
\begin{pgfscope}%
\pgfsetbuttcap%
\pgfsetroundjoin%
\definecolor{currentfill}{rgb}{0.000000,0.000000,0.000000}%
\pgfsetfillcolor{currentfill}%
\pgfsetlinewidth{0.803000pt}%
\definecolor{currentstroke}{rgb}{0.000000,0.000000,0.000000}%
\pgfsetstrokecolor{currentstroke}%
\pgfsetdash{}{0pt}%
\pgfsys@defobject{currentmarker}{\pgfqpoint{0.000000in}{0.000000in}}{\pgfqpoint{0.000000in}{0.048611in}}{%
\pgfpathmoveto{\pgfqpoint{0.000000in}{0.000000in}}%
\pgfpathlineto{\pgfqpoint{0.000000in}{0.048611in}}%
\pgfusepath{stroke,fill}%
}%
\begin{pgfscope}%
\pgfsys@transformshift{3.850000in}{0.330000in}%
\pgfsys@useobject{currentmarker}{}%
\end{pgfscope}%
\end{pgfscope}%
\begin{pgfscope}%
\definecolor{textcolor}{rgb}{0.000000,0.000000,0.000000}%
\pgfsetstrokecolor{textcolor}%
\pgfsetfillcolor{textcolor}%
\pgftext[x=3.850000in,y=0.281389in,,top]{\color{textcolor}\rmfamily\fontsize{10.000000}{12.000000}\selectfont \(\displaystyle {50}\)}%
\end{pgfscope}%
\begin{pgfscope}%
\pgfpathrectangle{\pgfqpoint{0.750000in}{0.330000in}}{\pgfqpoint{4.650000in}{2.310000in}}%
\pgfusepath{clip}%
\pgfsetbuttcap%
\pgfsetroundjoin%
\pgfsetlinewidth{0.501875pt}%
\definecolor{currentstroke}{rgb}{0.698039,0.698039,0.698039}%
\pgfsetstrokecolor{currentstroke}%
\pgfsetdash{{1.850000pt}{0.800000pt}}{0.000000pt}%
\pgfpathmoveto{\pgfqpoint{4.366667in}{0.330000in}}%
\pgfpathlineto{\pgfqpoint{4.366667in}{2.640000in}}%
\pgfusepath{stroke}%
\end{pgfscope}%
\begin{pgfscope}%
\pgfsetbuttcap%
\pgfsetroundjoin%
\definecolor{currentfill}{rgb}{0.000000,0.000000,0.000000}%
\pgfsetfillcolor{currentfill}%
\pgfsetlinewidth{0.803000pt}%
\definecolor{currentstroke}{rgb}{0.000000,0.000000,0.000000}%
\pgfsetstrokecolor{currentstroke}%
\pgfsetdash{}{0pt}%
\pgfsys@defobject{currentmarker}{\pgfqpoint{0.000000in}{0.000000in}}{\pgfqpoint{0.000000in}{0.048611in}}{%
\pgfpathmoveto{\pgfqpoint{0.000000in}{0.000000in}}%
\pgfpathlineto{\pgfqpoint{0.000000in}{0.048611in}}%
\pgfusepath{stroke,fill}%
}%
\begin{pgfscope}%
\pgfsys@transformshift{4.366667in}{0.330000in}%
\pgfsys@useobject{currentmarker}{}%
\end{pgfscope}%
\end{pgfscope}%
\begin{pgfscope}%
\definecolor{textcolor}{rgb}{0.000000,0.000000,0.000000}%
\pgfsetstrokecolor{textcolor}%
\pgfsetfillcolor{textcolor}%
\pgftext[x=4.366667in,y=0.281389in,,top]{\color{textcolor}\rmfamily\fontsize{10.000000}{12.000000}\selectfont \(\displaystyle {51}\)}%
\end{pgfscope}%
\begin{pgfscope}%
\pgfpathrectangle{\pgfqpoint{0.750000in}{0.330000in}}{\pgfqpoint{4.650000in}{2.310000in}}%
\pgfusepath{clip}%
\pgfsetbuttcap%
\pgfsetroundjoin%
\pgfsetlinewidth{0.501875pt}%
\definecolor{currentstroke}{rgb}{0.698039,0.698039,0.698039}%
\pgfsetstrokecolor{currentstroke}%
\pgfsetdash{{1.850000pt}{0.800000pt}}{0.000000pt}%
\pgfpathmoveto{\pgfqpoint{4.883333in}{0.330000in}}%
\pgfpathlineto{\pgfqpoint{4.883333in}{2.640000in}}%
\pgfusepath{stroke}%
\end{pgfscope}%
\begin{pgfscope}%
\pgfsetbuttcap%
\pgfsetroundjoin%
\definecolor{currentfill}{rgb}{0.000000,0.000000,0.000000}%
\pgfsetfillcolor{currentfill}%
\pgfsetlinewidth{0.803000pt}%
\definecolor{currentstroke}{rgb}{0.000000,0.000000,0.000000}%
\pgfsetstrokecolor{currentstroke}%
\pgfsetdash{}{0pt}%
\pgfsys@defobject{currentmarker}{\pgfqpoint{0.000000in}{0.000000in}}{\pgfqpoint{0.000000in}{0.048611in}}{%
\pgfpathmoveto{\pgfqpoint{0.000000in}{0.000000in}}%
\pgfpathlineto{\pgfqpoint{0.000000in}{0.048611in}}%
\pgfusepath{stroke,fill}%
}%
\begin{pgfscope}%
\pgfsys@transformshift{4.883333in}{0.330000in}%
\pgfsys@useobject{currentmarker}{}%
\end{pgfscope}%
\end{pgfscope}%
\begin{pgfscope}%
\definecolor{textcolor}{rgb}{0.000000,0.000000,0.000000}%
\pgfsetstrokecolor{textcolor}%
\pgfsetfillcolor{textcolor}%
\pgftext[x=4.883333in,y=0.281389in,,top]{\color{textcolor}\rmfamily\fontsize{10.000000}{12.000000}\selectfont \(\displaystyle {52}\)}%
\end{pgfscope}%
\begin{pgfscope}%
\pgfpathrectangle{\pgfqpoint{0.750000in}{0.330000in}}{\pgfqpoint{4.650000in}{2.310000in}}%
\pgfusepath{clip}%
\pgfsetbuttcap%
\pgfsetroundjoin%
\pgfsetlinewidth{0.501875pt}%
\definecolor{currentstroke}{rgb}{0.698039,0.698039,0.698039}%
\pgfsetstrokecolor{currentstroke}%
\pgfsetdash{{1.850000pt}{0.800000pt}}{0.000000pt}%
\pgfpathmoveto{\pgfqpoint{5.400000in}{0.330000in}}%
\pgfpathlineto{\pgfqpoint{5.400000in}{2.640000in}}%
\pgfusepath{stroke}%
\end{pgfscope}%
\begin{pgfscope}%
\pgfsetbuttcap%
\pgfsetroundjoin%
\definecolor{currentfill}{rgb}{0.000000,0.000000,0.000000}%
\pgfsetfillcolor{currentfill}%
\pgfsetlinewidth{0.803000pt}%
\definecolor{currentstroke}{rgb}{0.000000,0.000000,0.000000}%
\pgfsetstrokecolor{currentstroke}%
\pgfsetdash{}{0pt}%
\pgfsys@defobject{currentmarker}{\pgfqpoint{0.000000in}{0.000000in}}{\pgfqpoint{0.000000in}{0.048611in}}{%
\pgfpathmoveto{\pgfqpoint{0.000000in}{0.000000in}}%
\pgfpathlineto{\pgfqpoint{0.000000in}{0.048611in}}%
\pgfusepath{stroke,fill}%
}%
\begin{pgfscope}%
\pgfsys@transformshift{5.400000in}{0.330000in}%
\pgfsys@useobject{currentmarker}{}%
\end{pgfscope}%
\end{pgfscope}%
\begin{pgfscope}%
\definecolor{textcolor}{rgb}{0.000000,0.000000,0.000000}%
\pgfsetstrokecolor{textcolor}%
\pgfsetfillcolor{textcolor}%
\pgftext[x=5.400000in,y=0.281389in,,top]{\color{textcolor}\rmfamily\fontsize{10.000000}{12.000000}\selectfont \(\displaystyle {53}\)}%
\end{pgfscope}%
\begin{pgfscope}%
\definecolor{textcolor}{rgb}{0.000000,0.000000,0.000000}%
\pgfsetstrokecolor{textcolor}%
\pgfsetfillcolor{textcolor}%
\pgftext[x=3.075000in,y=0.102500in,,top]{\color{textcolor}\rmfamily\fontsize{12.000000}{14.400000}\selectfont Temps [s]}%
\end{pgfscope}%
\begin{pgfscope}%
\pgfpathrectangle{\pgfqpoint{0.750000in}{0.330000in}}{\pgfqpoint{4.650000in}{2.310000in}}%
\pgfusepath{clip}%
\pgfsetbuttcap%
\pgfsetroundjoin%
\pgfsetlinewidth{0.501875pt}%
\definecolor{currentstroke}{rgb}{0.698039,0.698039,0.698039}%
\pgfsetstrokecolor{currentstroke}%
\pgfsetdash{{1.850000pt}{0.800000pt}}{0.000000pt}%
\pgfpathmoveto{\pgfqpoint{0.750000in}{0.330000in}}%
\pgfpathlineto{\pgfqpoint{5.400000in}{0.330000in}}%
\pgfusepath{stroke}%
\end{pgfscope}%
\begin{pgfscope}%
\pgfsetbuttcap%
\pgfsetroundjoin%
\definecolor{currentfill}{rgb}{0.000000,0.000000,0.000000}%
\pgfsetfillcolor{currentfill}%
\pgfsetlinewidth{0.803000pt}%
\definecolor{currentstroke}{rgb}{0.000000,0.000000,0.000000}%
\pgfsetstrokecolor{currentstroke}%
\pgfsetdash{}{0pt}%
\pgfsys@defobject{currentmarker}{\pgfqpoint{0.000000in}{0.000000in}}{\pgfqpoint{0.048611in}{0.000000in}}{%
\pgfpathmoveto{\pgfqpoint{0.000000in}{0.000000in}}%
\pgfpathlineto{\pgfqpoint{0.048611in}{0.000000in}}%
\pgfusepath{stroke,fill}%
}%
\begin{pgfscope}%
\pgfsys@transformshift{0.750000in}{0.330000in}%
\pgfsys@useobject{currentmarker}{}%
\end{pgfscope}%
\end{pgfscope}%
\begin{pgfscope}%
\definecolor{textcolor}{rgb}{0.000000,0.000000,0.000000}%
\pgfsetstrokecolor{textcolor}%
\pgfsetfillcolor{textcolor}%
\pgftext[x=0.415894in, y=0.281806in, left, base]{\color{textcolor}\rmfamily\fontsize{10.000000}{12.000000}\selectfont \(\displaystyle {\ensuremath{-}1.0}\)}%
\end{pgfscope}%
\begin{pgfscope}%
\pgfpathrectangle{\pgfqpoint{0.750000in}{0.330000in}}{\pgfqpoint{4.650000in}{2.310000in}}%
\pgfusepath{clip}%
\pgfsetbuttcap%
\pgfsetroundjoin%
\pgfsetlinewidth{0.501875pt}%
\definecolor{currentstroke}{rgb}{0.698039,0.698039,0.698039}%
\pgfsetstrokecolor{currentstroke}%
\pgfsetdash{{1.850000pt}{0.800000pt}}{0.000000pt}%
\pgfpathmoveto{\pgfqpoint{0.750000in}{0.715000in}}%
\pgfpathlineto{\pgfqpoint{5.400000in}{0.715000in}}%
\pgfusepath{stroke}%
\end{pgfscope}%
\begin{pgfscope}%
\pgfsetbuttcap%
\pgfsetroundjoin%
\definecolor{currentfill}{rgb}{0.000000,0.000000,0.000000}%
\pgfsetfillcolor{currentfill}%
\pgfsetlinewidth{0.803000pt}%
\definecolor{currentstroke}{rgb}{0.000000,0.000000,0.000000}%
\pgfsetstrokecolor{currentstroke}%
\pgfsetdash{}{0pt}%
\pgfsys@defobject{currentmarker}{\pgfqpoint{0.000000in}{0.000000in}}{\pgfqpoint{0.048611in}{0.000000in}}{%
\pgfpathmoveto{\pgfqpoint{0.000000in}{0.000000in}}%
\pgfpathlineto{\pgfqpoint{0.048611in}{0.000000in}}%
\pgfusepath{stroke,fill}%
}%
\begin{pgfscope}%
\pgfsys@transformshift{0.750000in}{0.715000in}%
\pgfsys@useobject{currentmarker}{}%
\end{pgfscope}%
\end{pgfscope}%
\begin{pgfscope}%
\definecolor{textcolor}{rgb}{0.000000,0.000000,0.000000}%
\pgfsetstrokecolor{textcolor}%
\pgfsetfillcolor{textcolor}%
\pgftext[x=0.415894in, y=0.666806in, left, base]{\color{textcolor}\rmfamily\fontsize{10.000000}{12.000000}\selectfont \(\displaystyle {\ensuremath{-}0.5}\)}%
\end{pgfscope}%
\begin{pgfscope}%
\pgfpathrectangle{\pgfqpoint{0.750000in}{0.330000in}}{\pgfqpoint{4.650000in}{2.310000in}}%
\pgfusepath{clip}%
\pgfsetbuttcap%
\pgfsetroundjoin%
\pgfsetlinewidth{0.501875pt}%
\definecolor{currentstroke}{rgb}{0.698039,0.698039,0.698039}%
\pgfsetstrokecolor{currentstroke}%
\pgfsetdash{{1.850000pt}{0.800000pt}}{0.000000pt}%
\pgfpathmoveto{\pgfqpoint{0.750000in}{1.100000in}}%
\pgfpathlineto{\pgfqpoint{5.400000in}{1.100000in}}%
\pgfusepath{stroke}%
\end{pgfscope}%
\begin{pgfscope}%
\pgfsetbuttcap%
\pgfsetroundjoin%
\definecolor{currentfill}{rgb}{0.000000,0.000000,0.000000}%
\pgfsetfillcolor{currentfill}%
\pgfsetlinewidth{0.803000pt}%
\definecolor{currentstroke}{rgb}{0.000000,0.000000,0.000000}%
\pgfsetstrokecolor{currentstroke}%
\pgfsetdash{}{0pt}%
\pgfsys@defobject{currentmarker}{\pgfqpoint{0.000000in}{0.000000in}}{\pgfqpoint{0.048611in}{0.000000in}}{%
\pgfpathmoveto{\pgfqpoint{0.000000in}{0.000000in}}%
\pgfpathlineto{\pgfqpoint{0.048611in}{0.000000in}}%
\pgfusepath{stroke,fill}%
}%
\begin{pgfscope}%
\pgfsys@transformshift{0.750000in}{1.100000in}%
\pgfsys@useobject{currentmarker}{}%
\end{pgfscope}%
\end{pgfscope}%
\begin{pgfscope}%
\definecolor{textcolor}{rgb}{0.000000,0.000000,0.000000}%
\pgfsetstrokecolor{textcolor}%
\pgfsetfillcolor{textcolor}%
\pgftext[x=0.523919in, y=1.051806in, left, base]{\color{textcolor}\rmfamily\fontsize{10.000000}{12.000000}\selectfont \(\displaystyle {0.0}\)}%
\end{pgfscope}%
\begin{pgfscope}%
\pgfpathrectangle{\pgfqpoint{0.750000in}{0.330000in}}{\pgfqpoint{4.650000in}{2.310000in}}%
\pgfusepath{clip}%
\pgfsetbuttcap%
\pgfsetroundjoin%
\pgfsetlinewidth{0.501875pt}%
\definecolor{currentstroke}{rgb}{0.698039,0.698039,0.698039}%
\pgfsetstrokecolor{currentstroke}%
\pgfsetdash{{1.850000pt}{0.800000pt}}{0.000000pt}%
\pgfpathmoveto{\pgfqpoint{0.750000in}{1.485000in}}%
\pgfpathlineto{\pgfqpoint{5.400000in}{1.485000in}}%
\pgfusepath{stroke}%
\end{pgfscope}%
\begin{pgfscope}%
\pgfsetbuttcap%
\pgfsetroundjoin%
\definecolor{currentfill}{rgb}{0.000000,0.000000,0.000000}%
\pgfsetfillcolor{currentfill}%
\pgfsetlinewidth{0.803000pt}%
\definecolor{currentstroke}{rgb}{0.000000,0.000000,0.000000}%
\pgfsetstrokecolor{currentstroke}%
\pgfsetdash{}{0pt}%
\pgfsys@defobject{currentmarker}{\pgfqpoint{0.000000in}{0.000000in}}{\pgfqpoint{0.048611in}{0.000000in}}{%
\pgfpathmoveto{\pgfqpoint{0.000000in}{0.000000in}}%
\pgfpathlineto{\pgfqpoint{0.048611in}{0.000000in}}%
\pgfusepath{stroke,fill}%
}%
\begin{pgfscope}%
\pgfsys@transformshift{0.750000in}{1.485000in}%
\pgfsys@useobject{currentmarker}{}%
\end{pgfscope}%
\end{pgfscope}%
\begin{pgfscope}%
\definecolor{textcolor}{rgb}{0.000000,0.000000,0.000000}%
\pgfsetstrokecolor{textcolor}%
\pgfsetfillcolor{textcolor}%
\pgftext[x=0.523919in, y=1.436806in, left, base]{\color{textcolor}\rmfamily\fontsize{10.000000}{12.000000}\selectfont \(\displaystyle {0.5}\)}%
\end{pgfscope}%
\begin{pgfscope}%
\pgfpathrectangle{\pgfqpoint{0.750000in}{0.330000in}}{\pgfqpoint{4.650000in}{2.310000in}}%
\pgfusepath{clip}%
\pgfsetbuttcap%
\pgfsetroundjoin%
\pgfsetlinewidth{0.501875pt}%
\definecolor{currentstroke}{rgb}{0.698039,0.698039,0.698039}%
\pgfsetstrokecolor{currentstroke}%
\pgfsetdash{{1.850000pt}{0.800000pt}}{0.000000pt}%
\pgfpathmoveto{\pgfqpoint{0.750000in}{1.870000in}}%
\pgfpathlineto{\pgfqpoint{5.400000in}{1.870000in}}%
\pgfusepath{stroke}%
\end{pgfscope}%
\begin{pgfscope}%
\pgfsetbuttcap%
\pgfsetroundjoin%
\definecolor{currentfill}{rgb}{0.000000,0.000000,0.000000}%
\pgfsetfillcolor{currentfill}%
\pgfsetlinewidth{0.803000pt}%
\definecolor{currentstroke}{rgb}{0.000000,0.000000,0.000000}%
\pgfsetstrokecolor{currentstroke}%
\pgfsetdash{}{0pt}%
\pgfsys@defobject{currentmarker}{\pgfqpoint{0.000000in}{0.000000in}}{\pgfqpoint{0.048611in}{0.000000in}}{%
\pgfpathmoveto{\pgfqpoint{0.000000in}{0.000000in}}%
\pgfpathlineto{\pgfqpoint{0.048611in}{0.000000in}}%
\pgfusepath{stroke,fill}%
}%
\begin{pgfscope}%
\pgfsys@transformshift{0.750000in}{1.870000in}%
\pgfsys@useobject{currentmarker}{}%
\end{pgfscope}%
\end{pgfscope}%
\begin{pgfscope}%
\definecolor{textcolor}{rgb}{0.000000,0.000000,0.000000}%
\pgfsetstrokecolor{textcolor}%
\pgfsetfillcolor{textcolor}%
\pgftext[x=0.523919in, y=1.821806in, left, base]{\color{textcolor}\rmfamily\fontsize{10.000000}{12.000000}\selectfont \(\displaystyle {1.0}\)}%
\end{pgfscope}%
\begin{pgfscope}%
\pgfpathrectangle{\pgfqpoint{0.750000in}{0.330000in}}{\pgfqpoint{4.650000in}{2.310000in}}%
\pgfusepath{clip}%
\pgfsetbuttcap%
\pgfsetroundjoin%
\pgfsetlinewidth{0.501875pt}%
\definecolor{currentstroke}{rgb}{0.698039,0.698039,0.698039}%
\pgfsetstrokecolor{currentstroke}%
\pgfsetdash{{1.850000pt}{0.800000pt}}{0.000000pt}%
\pgfpathmoveto{\pgfqpoint{0.750000in}{2.255000in}}%
\pgfpathlineto{\pgfqpoint{5.400000in}{2.255000in}}%
\pgfusepath{stroke}%
\end{pgfscope}%
\begin{pgfscope}%
\pgfsetbuttcap%
\pgfsetroundjoin%
\definecolor{currentfill}{rgb}{0.000000,0.000000,0.000000}%
\pgfsetfillcolor{currentfill}%
\pgfsetlinewidth{0.803000pt}%
\definecolor{currentstroke}{rgb}{0.000000,0.000000,0.000000}%
\pgfsetstrokecolor{currentstroke}%
\pgfsetdash{}{0pt}%
\pgfsys@defobject{currentmarker}{\pgfqpoint{0.000000in}{0.000000in}}{\pgfqpoint{0.048611in}{0.000000in}}{%
\pgfpathmoveto{\pgfqpoint{0.000000in}{0.000000in}}%
\pgfpathlineto{\pgfqpoint{0.048611in}{0.000000in}}%
\pgfusepath{stroke,fill}%
}%
\begin{pgfscope}%
\pgfsys@transformshift{0.750000in}{2.255000in}%
\pgfsys@useobject{currentmarker}{}%
\end{pgfscope}%
\end{pgfscope}%
\begin{pgfscope}%
\definecolor{textcolor}{rgb}{0.000000,0.000000,0.000000}%
\pgfsetstrokecolor{textcolor}%
\pgfsetfillcolor{textcolor}%
\pgftext[x=0.523919in, y=2.206806in, left, base]{\color{textcolor}\rmfamily\fontsize{10.000000}{12.000000}\selectfont \(\displaystyle {1.5}\)}%
\end{pgfscope}%
\begin{pgfscope}%
\pgfpathrectangle{\pgfqpoint{0.750000in}{0.330000in}}{\pgfqpoint{4.650000in}{2.310000in}}%
\pgfusepath{clip}%
\pgfsetbuttcap%
\pgfsetroundjoin%
\pgfsetlinewidth{0.501875pt}%
\definecolor{currentstroke}{rgb}{0.698039,0.698039,0.698039}%
\pgfsetstrokecolor{currentstroke}%
\pgfsetdash{{1.850000pt}{0.800000pt}}{0.000000pt}%
\pgfpathmoveto{\pgfqpoint{0.750000in}{2.640000in}}%
\pgfpathlineto{\pgfqpoint{5.400000in}{2.640000in}}%
\pgfusepath{stroke}%
\end{pgfscope}%
\begin{pgfscope}%
\pgfsetbuttcap%
\pgfsetroundjoin%
\definecolor{currentfill}{rgb}{0.000000,0.000000,0.000000}%
\pgfsetfillcolor{currentfill}%
\pgfsetlinewidth{0.803000pt}%
\definecolor{currentstroke}{rgb}{0.000000,0.000000,0.000000}%
\pgfsetstrokecolor{currentstroke}%
\pgfsetdash{}{0pt}%
\pgfsys@defobject{currentmarker}{\pgfqpoint{0.000000in}{0.000000in}}{\pgfqpoint{0.048611in}{0.000000in}}{%
\pgfpathmoveto{\pgfqpoint{0.000000in}{0.000000in}}%
\pgfpathlineto{\pgfqpoint{0.048611in}{0.000000in}}%
\pgfusepath{stroke,fill}%
}%
\begin{pgfscope}%
\pgfsys@transformshift{0.750000in}{2.640000in}%
\pgfsys@useobject{currentmarker}{}%
\end{pgfscope}%
\end{pgfscope}%
\begin{pgfscope}%
\definecolor{textcolor}{rgb}{0.000000,0.000000,0.000000}%
\pgfsetstrokecolor{textcolor}%
\pgfsetfillcolor{textcolor}%
\pgftext[x=0.523919in, y=2.591806in, left, base]{\color{textcolor}\rmfamily\fontsize{10.000000}{12.000000}\selectfont \(\displaystyle {2.0}\)}%
\end{pgfscope}%
\begin{pgfscope}%
\definecolor{textcolor}{rgb}{0.000000,0.000000,0.000000}%
\pgfsetstrokecolor{textcolor}%
\pgfsetfillcolor{textcolor}%
\pgftext[x=0.360339in,y=1.485000in,,bottom,rotate=90.000000]{\color{textcolor}\rmfamily\fontsize{12.000000}{14.400000}\selectfont Amplitude [m/s]}%
\end{pgfscope}%
\begin{pgfscope}%
\definecolor{textcolor}{rgb}{0.000000,0.000000,0.000000}%
\pgfsetstrokecolor{textcolor}%
\pgfsetfillcolor{textcolor}%
\pgftext[x=0.750000in,y=2.681667in,left,base]{\color{textcolor}\rmfamily\fontsize{10.000000}{12.000000}\selectfont \(\displaystyle \times{10^{\ensuremath{-}5}}{}\)}%
\end{pgfscope}%
\begin{pgfscope}%
\pgfpathrectangle{\pgfqpoint{0.750000in}{0.330000in}}{\pgfqpoint{4.650000in}{2.310000in}}%
\pgfusepath{clip}%
\pgfsetrectcap%
\pgfsetroundjoin%
\pgfsetlinewidth{1.505625pt}%
\definecolor{currentstroke}{rgb}{0.501961,0.501961,0.501961}%
\pgfsetstrokecolor{currentstroke}%
\pgfsetdash{}{0pt}%
\pgfpathmoveto{\pgfqpoint{0.747417in}{1.099242in}}%
\pgfpathlineto{\pgfqpoint{0.770667in}{1.100579in}}%
\pgfpathlineto{\pgfqpoint{0.793917in}{1.101292in}}%
\pgfpathlineto{\pgfqpoint{0.840417in}{1.100341in}}%
\pgfpathlineto{\pgfqpoint{0.861083in}{1.100273in}}%
\pgfpathlineto{\pgfqpoint{0.907583in}{1.098500in}}%
\pgfpathlineto{\pgfqpoint{0.930833in}{1.098473in}}%
\pgfpathlineto{\pgfqpoint{0.951500in}{1.097949in}}%
\pgfpathlineto{\pgfqpoint{0.967000in}{1.100880in}}%
\pgfpathlineto{\pgfqpoint{0.990250in}{1.104878in}}%
\pgfpathlineto{\pgfqpoint{1.005750in}{1.106632in}}%
\pgfpathlineto{\pgfqpoint{1.013500in}{1.105075in}}%
\pgfpathlineto{\pgfqpoint{1.031583in}{1.098518in}}%
\pgfpathlineto{\pgfqpoint{1.036750in}{1.100235in}}%
\pgfpathlineto{\pgfqpoint{1.044500in}{1.106724in}}%
\pgfpathlineto{\pgfqpoint{1.052250in}{1.113883in}}%
\pgfpathlineto{\pgfqpoint{1.057417in}{1.116076in}}%
\pgfpathlineto{\pgfqpoint{1.062583in}{1.114576in}}%
\pgfpathlineto{\pgfqpoint{1.067750in}{1.109059in}}%
\pgfpathlineto{\pgfqpoint{1.075500in}{1.095484in}}%
\pgfpathlineto{\pgfqpoint{1.083250in}{1.081688in}}%
\pgfpathlineto{\pgfqpoint{1.088417in}{1.076072in}}%
\pgfpathlineto{\pgfqpoint{1.093583in}{1.074880in}}%
\pgfpathlineto{\pgfqpoint{1.098750in}{1.077917in}}%
\pgfpathlineto{\pgfqpoint{1.106500in}{1.087298in}}%
\pgfpathlineto{\pgfqpoint{1.116833in}{1.099961in}}%
\pgfpathlineto{\pgfqpoint{1.122000in}{1.103566in}}%
\pgfpathlineto{\pgfqpoint{1.127167in}{1.104704in}}%
\pgfpathlineto{\pgfqpoint{1.132333in}{1.103618in}}%
\pgfpathlineto{\pgfqpoint{1.140083in}{1.098941in}}%
\pgfpathlineto{\pgfqpoint{1.150417in}{1.089260in}}%
\pgfpathlineto{\pgfqpoint{1.160750in}{1.080114in}}%
\pgfpathlineto{\pgfqpoint{1.165917in}{1.078999in}}%
\pgfpathlineto{\pgfqpoint{1.171083in}{1.082090in}}%
\pgfpathlineto{\pgfqpoint{1.176250in}{1.089600in}}%
\pgfpathlineto{\pgfqpoint{1.196917in}{1.128105in}}%
\pgfpathlineto{\pgfqpoint{1.199500in}{1.129162in}}%
\pgfpathlineto{\pgfqpoint{1.202083in}{1.128881in}}%
\pgfpathlineto{\pgfqpoint{1.207250in}{1.124825in}}%
\pgfpathlineto{\pgfqpoint{1.225333in}{1.101921in}}%
\pgfpathlineto{\pgfqpoint{1.230500in}{1.099225in}}%
\pgfpathlineto{\pgfqpoint{1.238250in}{1.098308in}}%
\pgfpathlineto{\pgfqpoint{1.246000in}{1.100113in}}%
\pgfpathlineto{\pgfqpoint{1.253750in}{1.104875in}}%
\pgfpathlineto{\pgfqpoint{1.264083in}{1.115295in}}%
\pgfpathlineto{\pgfqpoint{1.271833in}{1.122492in}}%
\pgfpathlineto{\pgfqpoint{1.277000in}{1.123542in}}%
\pgfpathlineto{\pgfqpoint{1.279583in}{1.122284in}}%
\pgfpathlineto{\pgfqpoint{1.284750in}{1.115923in}}%
\pgfpathlineto{\pgfqpoint{1.292500in}{1.098920in}}%
\pgfpathlineto{\pgfqpoint{1.302833in}{1.074549in}}%
\pgfpathlineto{\pgfqpoint{1.308000in}{1.067416in}}%
\pgfpathlineto{\pgfqpoint{1.310583in}{1.065987in}}%
\pgfpathlineto{\pgfqpoint{1.313167in}{1.066153in}}%
\pgfpathlineto{\pgfqpoint{1.315750in}{1.067937in}}%
\pgfpathlineto{\pgfqpoint{1.320917in}{1.075961in}}%
\pgfpathlineto{\pgfqpoint{1.339000in}{1.114201in}}%
\pgfpathlineto{\pgfqpoint{1.341583in}{1.115561in}}%
\pgfpathlineto{\pgfqpoint{1.344167in}{1.115263in}}%
\pgfpathlineto{\pgfqpoint{1.346750in}{1.113378in}}%
\pgfpathlineto{\pgfqpoint{1.351917in}{1.105601in}}%
\pgfpathlineto{\pgfqpoint{1.367417in}{1.075022in}}%
\pgfpathlineto{\pgfqpoint{1.372583in}{1.071913in}}%
\pgfpathlineto{\pgfqpoint{1.375167in}{1.072566in}}%
\pgfpathlineto{\pgfqpoint{1.380333in}{1.077881in}}%
\pgfpathlineto{\pgfqpoint{1.388083in}{1.092085in}}%
\pgfpathlineto{\pgfqpoint{1.398417in}{1.110857in}}%
\pgfpathlineto{\pgfqpoint{1.403583in}{1.117089in}}%
\pgfpathlineto{\pgfqpoint{1.408750in}{1.120466in}}%
\pgfpathlineto{\pgfqpoint{1.413917in}{1.120610in}}%
\pgfpathlineto{\pgfqpoint{1.419083in}{1.117255in}}%
\pgfpathlineto{\pgfqpoint{1.424250in}{1.110622in}}%
\pgfpathlineto{\pgfqpoint{1.442333in}{1.082956in}}%
\pgfpathlineto{\pgfqpoint{1.444917in}{1.081935in}}%
\pgfpathlineto{\pgfqpoint{1.447500in}{1.082260in}}%
\pgfpathlineto{\pgfqpoint{1.452667in}{1.086856in}}%
\pgfpathlineto{\pgfqpoint{1.460417in}{1.100621in}}%
\pgfpathlineto{\pgfqpoint{1.470750in}{1.120377in}}%
\pgfpathlineto{\pgfqpoint{1.475917in}{1.127271in}}%
\pgfpathlineto{\pgfqpoint{1.481083in}{1.130590in}}%
\pgfpathlineto{\pgfqpoint{1.483667in}{1.130505in}}%
\pgfpathlineto{\pgfqpoint{1.486250in}{1.129045in}}%
\pgfpathlineto{\pgfqpoint{1.491417in}{1.121588in}}%
\pgfpathlineto{\pgfqpoint{1.496583in}{1.108341in}}%
\pgfpathlineto{\pgfqpoint{1.512083in}{1.062187in}}%
\pgfpathlineto{\pgfqpoint{1.514667in}{1.058829in}}%
\pgfpathlineto{\pgfqpoint{1.517250in}{1.057675in}}%
\pgfpathlineto{\pgfqpoint{1.519833in}{1.058782in}}%
\pgfpathlineto{\pgfqpoint{1.522417in}{1.062058in}}%
\pgfpathlineto{\pgfqpoint{1.527583in}{1.074150in}}%
\pgfpathlineto{\pgfqpoint{1.548250in}{1.137697in}}%
\pgfpathlineto{\pgfqpoint{1.553417in}{1.145387in}}%
\pgfpathlineto{\pgfqpoint{1.556000in}{1.147362in}}%
\pgfpathlineto{\pgfqpoint{1.558583in}{1.147874in}}%
\pgfpathlineto{\pgfqpoint{1.561167in}{1.146694in}}%
\pgfpathlineto{\pgfqpoint{1.563750in}{1.143599in}}%
\pgfpathlineto{\pgfqpoint{1.568917in}{1.131218in}}%
\pgfpathlineto{\pgfqpoint{1.574083in}{1.111409in}}%
\pgfpathlineto{\pgfqpoint{1.587000in}{1.056001in}}%
\pgfpathlineto{\pgfqpoint{1.589583in}{1.049712in}}%
\pgfpathlineto{\pgfqpoint{1.592167in}{1.046667in}}%
\pgfpathlineto{\pgfqpoint{1.594750in}{1.047320in}}%
\pgfpathlineto{\pgfqpoint{1.597333in}{1.051872in}}%
\pgfpathlineto{\pgfqpoint{1.602500in}{1.071929in}}%
\pgfpathlineto{\pgfqpoint{1.620583in}{1.169962in}}%
\pgfpathlineto{\pgfqpoint{1.623167in}{1.175428in}}%
\pgfpathlineto{\pgfqpoint{1.625750in}{1.177300in}}%
\pgfpathlineto{\pgfqpoint{1.628333in}{1.175603in}}%
\pgfpathlineto{\pgfqpoint{1.630917in}{1.170513in}}%
\pgfpathlineto{\pgfqpoint{1.636083in}{1.151394in}}%
\pgfpathlineto{\pgfqpoint{1.643833in}{1.106827in}}%
\pgfpathlineto{\pgfqpoint{1.656750in}{1.027655in}}%
\pgfpathlineto{\pgfqpoint{1.661917in}{1.009158in}}%
\pgfpathlineto{\pgfqpoint{1.664500in}{1.005088in}}%
\pgfpathlineto{\pgfqpoint{1.667083in}{1.004985in}}%
\pgfpathlineto{\pgfqpoint{1.669667in}{1.009023in}}%
\pgfpathlineto{\pgfqpoint{1.672250in}{1.017174in}}%
\pgfpathlineto{\pgfqpoint{1.677417in}{1.044564in}}%
\pgfpathlineto{\pgfqpoint{1.695500in}{1.163805in}}%
\pgfpathlineto{\pgfqpoint{1.698083in}{1.168930in}}%
\pgfpathlineto{\pgfqpoint{1.700667in}{1.168671in}}%
\pgfpathlineto{\pgfqpoint{1.703250in}{1.162896in}}%
\pgfpathlineto{\pgfqpoint{1.708417in}{1.135833in}}%
\pgfpathlineto{\pgfqpoint{1.716167in}{1.068209in}}%
\pgfpathlineto{\pgfqpoint{1.723917in}{0.999863in}}%
\pgfpathlineto{\pgfqpoint{1.729083in}{0.973366in}}%
\pgfpathlineto{\pgfqpoint{1.731667in}{0.968736in}}%
\pgfpathlineto{\pgfqpoint{1.734250in}{0.970176in}}%
\pgfpathlineto{\pgfqpoint{1.736833in}{0.977419in}}%
\pgfpathlineto{\pgfqpoint{1.742000in}{1.006789in}}%
\pgfpathlineto{\pgfqpoint{1.749750in}{1.075419in}}%
\pgfpathlineto{\pgfqpoint{1.767833in}{1.248230in}}%
\pgfpathlineto{\pgfqpoint{1.773000in}{1.275287in}}%
\pgfpathlineto{\pgfqpoint{1.775583in}{1.280416in}}%
\pgfpathlineto{\pgfqpoint{1.778167in}{1.278720in}}%
\pgfpathlineto{\pgfqpoint{1.780750in}{1.269576in}}%
\pgfpathlineto{\pgfqpoint{1.783333in}{1.252763in}}%
\pgfpathlineto{\pgfqpoint{1.788500in}{1.197913in}}%
\pgfpathlineto{\pgfqpoint{1.804000in}{0.992551in}}%
\pgfpathlineto{\pgfqpoint{1.806583in}{0.977453in}}%
\pgfpathlineto{\pgfqpoint{1.809167in}{0.972497in}}%
\pgfpathlineto{\pgfqpoint{1.811750in}{0.977821in}}%
\pgfpathlineto{\pgfqpoint{1.814333in}{0.992672in}}%
\pgfpathlineto{\pgfqpoint{1.819500in}{1.044153in}}%
\pgfpathlineto{\pgfqpoint{1.829833in}{1.166837in}}%
\pgfpathlineto{\pgfqpoint{1.835000in}{1.203997in}}%
\pgfpathlineto{\pgfqpoint{1.837583in}{1.212491in}}%
\pgfpathlineto{\pgfqpoint{1.840167in}{1.214074in}}%
\pgfpathlineto{\pgfqpoint{1.842750in}{1.209120in}}%
\pgfpathlineto{\pgfqpoint{1.845333in}{1.198291in}}%
\pgfpathlineto{\pgfqpoint{1.850500in}{1.162461in}}%
\pgfpathlineto{\pgfqpoint{1.860833in}{1.060615in}}%
\pgfpathlineto{\pgfqpoint{1.871167in}{0.966051in}}%
\pgfpathlineto{\pgfqpoint{1.876333in}{0.933767in}}%
\pgfpathlineto{\pgfqpoint{1.881500in}{0.915011in}}%
\pgfpathlineto{\pgfqpoint{1.884083in}{0.911721in}}%
\pgfpathlineto{\pgfqpoint{1.886667in}{0.913126in}}%
\pgfpathlineto{\pgfqpoint{1.889250in}{0.919607in}}%
\pgfpathlineto{\pgfqpoint{1.894417in}{0.948176in}}%
\pgfpathlineto{\pgfqpoint{1.899583in}{0.995011in}}%
\pgfpathlineto{\pgfqpoint{1.920250in}{1.209875in}}%
\pgfpathlineto{\pgfqpoint{1.925417in}{1.238847in}}%
\pgfpathlineto{\pgfqpoint{1.928000in}{1.246947in}}%
\pgfpathlineto{\pgfqpoint{1.930583in}{1.250277in}}%
\pgfpathlineto{\pgfqpoint{1.933167in}{1.248661in}}%
\pgfpathlineto{\pgfqpoint{1.935750in}{1.242228in}}%
\pgfpathlineto{\pgfqpoint{1.940917in}{1.217164in}}%
\pgfpathlineto{\pgfqpoint{1.959000in}{1.101827in}}%
\pgfpathlineto{\pgfqpoint{1.964167in}{1.087382in}}%
\pgfpathlineto{\pgfqpoint{1.966750in}{1.087047in}}%
\pgfpathlineto{\pgfqpoint{1.969333in}{1.092628in}}%
\pgfpathlineto{\pgfqpoint{1.971917in}{1.104896in}}%
\pgfpathlineto{\pgfqpoint{1.977083in}{1.149921in}}%
\pgfpathlineto{\pgfqpoint{1.992583in}{1.337320in}}%
\pgfpathlineto{\pgfqpoint{1.995167in}{1.351380in}}%
\pgfpathlineto{\pgfqpoint{1.997750in}{1.355094in}}%
\pgfpathlineto{\pgfqpoint{2.000333in}{1.347346in}}%
\pgfpathlineto{\pgfqpoint{2.002917in}{1.327575in}}%
\pgfpathlineto{\pgfqpoint{2.008083in}{1.252898in}}%
\pgfpathlineto{\pgfqpoint{2.015833in}{1.074939in}}%
\pgfpathlineto{\pgfqpoint{2.026167in}{0.836082in}}%
\pgfpathlineto{\pgfqpoint{2.031333in}{0.775217in}}%
\pgfpathlineto{\pgfqpoint{2.033917in}{0.766869in}}%
\pgfpathlineto{\pgfqpoint{2.036500in}{0.773732in}}%
\pgfpathlineto{\pgfqpoint{2.039083in}{0.795086in}}%
\pgfpathlineto{\pgfqpoint{2.044250in}{0.874995in}}%
\pgfpathlineto{\pgfqpoint{2.062333in}{1.245026in}}%
\pgfpathlineto{\pgfqpoint{2.064917in}{1.261578in}}%
\pgfpathlineto{\pgfqpoint{2.067500in}{1.260059in}}%
\pgfpathlineto{\pgfqpoint{2.070083in}{1.239833in}}%
\pgfpathlineto{\pgfqpoint{2.075250in}{1.148223in}}%
\pgfpathlineto{\pgfqpoint{2.090750in}{0.761150in}}%
\pgfpathlineto{\pgfqpoint{2.093333in}{0.732582in}}%
\pgfpathlineto{\pgfqpoint{2.095917in}{0.723697in}}%
\pgfpathlineto{\pgfqpoint{2.098500in}{0.735091in}}%
\pgfpathlineto{\pgfqpoint{2.101083in}{0.765763in}}%
\pgfpathlineto{\pgfqpoint{2.106250in}{0.873652in}}%
\pgfpathlineto{\pgfqpoint{2.119167in}{1.207041in}}%
\pgfpathlineto{\pgfqpoint{2.124333in}{1.286352in}}%
\pgfpathlineto{\pgfqpoint{2.126917in}{1.308571in}}%
\pgfpathlineto{\pgfqpoint{2.129500in}{1.320180in}}%
\pgfpathlineto{\pgfqpoint{2.132083in}{1.322642in}}%
\pgfpathlineto{\pgfqpoint{2.134667in}{1.317619in}}%
\pgfpathlineto{\pgfqpoint{2.139833in}{1.291214in}}%
\pgfpathlineto{\pgfqpoint{2.147583in}{1.226967in}}%
\pgfpathlineto{\pgfqpoint{2.160500in}{1.108368in}}%
\pgfpathlineto{\pgfqpoint{2.165667in}{1.079587in}}%
\pgfpathlineto{\pgfqpoint{2.168250in}{1.074290in}}%
\pgfpathlineto{\pgfqpoint{2.170833in}{1.076836in}}%
\pgfpathlineto{\pgfqpoint{2.173417in}{1.088125in}}%
\pgfpathlineto{\pgfqpoint{2.178583in}{1.136735in}}%
\pgfpathlineto{\pgfqpoint{2.191500in}{1.311792in}}%
\pgfpathlineto{\pgfqpoint{2.194083in}{1.330171in}}%
\pgfpathlineto{\pgfqpoint{2.196667in}{1.336569in}}%
\pgfpathlineto{\pgfqpoint{2.199250in}{1.329939in}}%
\pgfpathlineto{\pgfqpoint{2.201833in}{1.310453in}}%
\pgfpathlineto{\pgfqpoint{2.207000in}{1.239783in}}%
\pgfpathlineto{\pgfqpoint{2.217333in}{1.066911in}}%
\pgfpathlineto{\pgfqpoint{2.222500in}{1.020499in}}%
\pgfpathlineto{\pgfqpoint{2.225083in}{1.013531in}}%
\pgfpathlineto{\pgfqpoint{2.227667in}{1.016684in}}%
\pgfpathlineto{\pgfqpoint{2.230250in}{1.028168in}}%
\pgfpathlineto{\pgfqpoint{2.238000in}{1.086876in}}%
\pgfpathlineto{\pgfqpoint{2.243167in}{1.121063in}}%
\pgfpathlineto{\pgfqpoint{2.245750in}{1.131577in}}%
\pgfpathlineto{\pgfqpoint{2.248333in}{1.137012in}}%
\pgfpathlineto{\pgfqpoint{2.250917in}{1.137629in}}%
\pgfpathlineto{\pgfqpoint{2.253500in}{1.134120in}}%
\pgfpathlineto{\pgfqpoint{2.258667in}{1.118341in}}%
\pgfpathlineto{\pgfqpoint{2.266417in}{1.083007in}}%
\pgfpathlineto{\pgfqpoint{2.274167in}{1.034586in}}%
\pgfpathlineto{\pgfqpoint{2.281917in}{0.963847in}}%
\pgfpathlineto{\pgfqpoint{2.289667in}{0.888500in}}%
\pgfpathlineto{\pgfqpoint{2.292250in}{0.871897in}}%
\pgfpathlineto{\pgfqpoint{2.294833in}{0.863394in}}%
\pgfpathlineto{\pgfqpoint{2.297417in}{0.864465in}}%
\pgfpathlineto{\pgfqpoint{2.300000in}{0.875598in}}%
\pgfpathlineto{\pgfqpoint{2.305167in}{0.924721in}}%
\pgfpathlineto{\pgfqpoint{2.323250in}{1.160453in}}%
\pgfpathlineto{\pgfqpoint{2.328417in}{1.199753in}}%
\pgfpathlineto{\pgfqpoint{2.331000in}{1.212127in}}%
\pgfpathlineto{\pgfqpoint{2.333583in}{1.218920in}}%
\pgfpathlineto{\pgfqpoint{2.336167in}{1.219589in}}%
\pgfpathlineto{\pgfqpoint{2.338750in}{1.213792in}}%
\pgfpathlineto{\pgfqpoint{2.341333in}{1.201517in}}%
\pgfpathlineto{\pgfqpoint{2.346500in}{1.159444in}}%
\pgfpathlineto{\pgfqpoint{2.364583in}{0.973810in}}%
\pgfpathlineto{\pgfqpoint{2.367167in}{0.966996in}}%
\pgfpathlineto{\pgfqpoint{2.369750in}{0.969798in}}%
\pgfpathlineto{\pgfqpoint{2.372333in}{0.982565in}}%
\pgfpathlineto{\pgfqpoint{2.377500in}{1.035303in}}%
\pgfpathlineto{\pgfqpoint{2.395583in}{1.288198in}}%
\pgfpathlineto{\pgfqpoint{2.400750in}{1.320980in}}%
\pgfpathlineto{\pgfqpoint{2.403333in}{1.328212in}}%
\pgfpathlineto{\pgfqpoint{2.405917in}{1.329459in}}%
\pgfpathlineto{\pgfqpoint{2.408500in}{1.324541in}}%
\pgfpathlineto{\pgfqpoint{2.411083in}{1.312972in}}%
\pgfpathlineto{\pgfqpoint{2.416250in}{1.267050in}}%
\pgfpathlineto{\pgfqpoint{2.421417in}{1.187953in}}%
\pgfpathlineto{\pgfqpoint{2.436917in}{0.896793in}}%
\pgfpathlineto{\pgfqpoint{2.439500in}{0.880711in}}%
\pgfpathlineto{\pgfqpoint{2.442083in}{0.882992in}}%
\pgfpathlineto{\pgfqpoint{2.444667in}{0.904211in}}%
\pgfpathlineto{\pgfqpoint{2.449833in}{0.996623in}}%
\pgfpathlineto{\pgfqpoint{2.462750in}{1.305219in}}%
\pgfpathlineto{\pgfqpoint{2.465333in}{1.338988in}}%
\pgfpathlineto{\pgfqpoint{2.467917in}{1.355520in}}%
\pgfpathlineto{\pgfqpoint{2.470500in}{1.354204in}}%
\pgfpathlineto{\pgfqpoint{2.473083in}{1.335800in}}%
\pgfpathlineto{\pgfqpoint{2.478250in}{1.256073in}}%
\pgfpathlineto{\pgfqpoint{2.486000in}{1.074574in}}%
\pgfpathlineto{\pgfqpoint{2.496333in}{0.839492in}}%
\pgfpathlineto{\pgfqpoint{2.501500in}{0.764423in}}%
\pgfpathlineto{\pgfqpoint{2.504083in}{0.744294in}}%
\pgfpathlineto{\pgfqpoint{2.506667in}{0.738399in}}%
\pgfpathlineto{\pgfqpoint{2.509250in}{0.748376in}}%
\pgfpathlineto{\pgfqpoint{2.511833in}{0.775142in}}%
\pgfpathlineto{\pgfqpoint{2.517000in}{0.876616in}}%
\pgfpathlineto{\pgfqpoint{2.532500in}{1.277528in}}%
\pgfpathlineto{\pgfqpoint{2.535083in}{1.304566in}}%
\pgfpathlineto{\pgfqpoint{2.537667in}{1.312288in}}%
\pgfpathlineto{\pgfqpoint{2.540250in}{1.301434in}}%
\pgfpathlineto{\pgfqpoint{2.542833in}{1.274260in}}%
\pgfpathlineto{\pgfqpoint{2.548000in}{1.185508in}}%
\pgfpathlineto{\pgfqpoint{2.558333in}{0.995063in}}%
\pgfpathlineto{\pgfqpoint{2.563500in}{0.950379in}}%
\pgfpathlineto{\pgfqpoint{2.566083in}{0.945391in}}%
\pgfpathlineto{\pgfqpoint{2.568667in}{0.950886in}}%
\pgfpathlineto{\pgfqpoint{2.571250in}{0.965216in}}%
\pgfpathlineto{\pgfqpoint{2.576417in}{1.013011in}}%
\pgfpathlineto{\pgfqpoint{2.584167in}{1.111624in}}%
\pgfpathlineto{\pgfqpoint{2.599667in}{1.322732in}}%
\pgfpathlineto{\pgfqpoint{2.604833in}{1.358944in}}%
\pgfpathlineto{\pgfqpoint{2.607417in}{1.364050in}}%
\pgfpathlineto{\pgfqpoint{2.610000in}{1.359944in}}%
\pgfpathlineto{\pgfqpoint{2.612583in}{1.347152in}}%
\pgfpathlineto{\pgfqpoint{2.617750in}{1.301112in}}%
\pgfpathlineto{\pgfqpoint{2.630667in}{1.163948in}}%
\pgfpathlineto{\pgfqpoint{2.635833in}{1.135044in}}%
\pgfpathlineto{\pgfqpoint{2.638417in}{1.128306in}}%
\pgfpathlineto{\pgfqpoint{2.641000in}{1.125978in}}%
\pgfpathlineto{\pgfqpoint{2.643583in}{1.127021in}}%
\pgfpathlineto{\pgfqpoint{2.653917in}{1.138333in}}%
\pgfpathlineto{\pgfqpoint{2.656500in}{1.137554in}}%
\pgfpathlineto{\pgfqpoint{2.659083in}{1.134583in}}%
\pgfpathlineto{\pgfqpoint{2.664250in}{1.123818in}}%
\pgfpathlineto{\pgfqpoint{2.674583in}{1.094629in}}%
\pgfpathlineto{\pgfqpoint{2.679750in}{1.075061in}}%
\pgfpathlineto{\pgfqpoint{2.690083in}{1.022206in}}%
\pgfpathlineto{\pgfqpoint{2.695250in}{0.999045in}}%
\pgfpathlineto{\pgfqpoint{2.700417in}{0.986733in}}%
\pgfpathlineto{\pgfqpoint{2.703000in}{0.985360in}}%
\pgfpathlineto{\pgfqpoint{2.705583in}{0.986954in}}%
\pgfpathlineto{\pgfqpoint{2.708167in}{0.991150in}}%
\pgfpathlineto{\pgfqpoint{2.713333in}{1.005685in}}%
\pgfpathlineto{\pgfqpoint{2.723667in}{1.040341in}}%
\pgfpathlineto{\pgfqpoint{2.726250in}{1.042652in}}%
\pgfpathlineto{\pgfqpoint{2.728833in}{1.038764in}}%
\pgfpathlineto{\pgfqpoint{2.731417in}{1.027063in}}%
\pgfpathlineto{\pgfqpoint{2.736583in}{0.977366in}}%
\pgfpathlineto{\pgfqpoint{2.744333in}{0.852950in}}%
\pgfpathlineto{\pgfqpoint{2.749500in}{0.770837in}}%
\pgfpathlineto{\pgfqpoint{2.754667in}{0.724873in}}%
\pgfpathlineto{\pgfqpoint{2.757250in}{0.722873in}}%
\pgfpathlineto{\pgfqpoint{2.759833in}{0.736579in}}%
\pgfpathlineto{\pgfqpoint{2.762417in}{0.765488in}}%
\pgfpathlineto{\pgfqpoint{2.767583in}{0.859757in}}%
\pgfpathlineto{\pgfqpoint{2.777917in}{1.078534in}}%
\pgfpathlineto{\pgfqpoint{2.783083in}{1.137226in}}%
\pgfpathlineto{\pgfqpoint{2.785667in}{1.147177in}}%
\pgfpathlineto{\pgfqpoint{2.788250in}{1.145856in}}%
\pgfpathlineto{\pgfqpoint{2.793417in}{1.122618in}}%
\pgfpathlineto{\pgfqpoint{2.798583in}{1.099470in}}%
\pgfpathlineto{\pgfqpoint{2.801167in}{1.097163in}}%
\pgfpathlineto{\pgfqpoint{2.803750in}{1.104037in}}%
\pgfpathlineto{\pgfqpoint{2.806333in}{1.120722in}}%
\pgfpathlineto{\pgfqpoint{2.811500in}{1.179832in}}%
\pgfpathlineto{\pgfqpoint{2.824417in}{1.365459in}}%
\pgfpathlineto{\pgfqpoint{2.829583in}{1.411540in}}%
\pgfpathlineto{\pgfqpoint{2.834750in}{1.438533in}}%
\pgfpathlineto{\pgfqpoint{2.839917in}{1.454757in}}%
\pgfpathlineto{\pgfqpoint{2.845083in}{1.464604in}}%
\pgfpathlineto{\pgfqpoint{2.847667in}{1.465780in}}%
\pgfpathlineto{\pgfqpoint{2.850250in}{1.462578in}}%
\pgfpathlineto{\pgfqpoint{2.852833in}{1.453398in}}%
\pgfpathlineto{\pgfqpoint{2.855417in}{1.437029in}}%
\pgfpathlineto{\pgfqpoint{2.860583in}{1.382225in}}%
\pgfpathlineto{\pgfqpoint{2.873500in}{1.212465in}}%
\pgfpathlineto{\pgfqpoint{2.876083in}{1.196969in}}%
\pgfpathlineto{\pgfqpoint{2.878667in}{1.192475in}}%
\pgfpathlineto{\pgfqpoint{2.881250in}{1.198721in}}%
\pgfpathlineto{\pgfqpoint{2.886417in}{1.235048in}}%
\pgfpathlineto{\pgfqpoint{2.891583in}{1.277993in}}%
\pgfpathlineto{\pgfqpoint{2.894167in}{1.290863in}}%
\pgfpathlineto{\pgfqpoint{2.896750in}{1.292825in}}%
\pgfpathlineto{\pgfqpoint{2.899333in}{1.281334in}}%
\pgfpathlineto{\pgfqpoint{2.901917in}{1.255279in}}%
\pgfpathlineto{\pgfqpoint{2.907083in}{1.162729in}}%
\pgfpathlineto{\pgfqpoint{2.922583in}{0.808747in}}%
\pgfpathlineto{\pgfqpoint{2.927750in}{0.759137in}}%
\pgfpathlineto{\pgfqpoint{2.930333in}{0.751511in}}%
\pgfpathlineto{\pgfqpoint{2.932917in}{0.752668in}}%
\pgfpathlineto{\pgfqpoint{2.935500in}{0.759895in}}%
\pgfpathlineto{\pgfqpoint{2.948417in}{0.809628in}}%
\pgfpathlineto{\pgfqpoint{2.953583in}{0.820034in}}%
\pgfpathlineto{\pgfqpoint{2.982000in}{0.865399in}}%
\pgfpathlineto{\pgfqpoint{2.987167in}{0.880525in}}%
\pgfpathlineto{\pgfqpoint{2.992333in}{0.911249in}}%
\pgfpathlineto{\pgfqpoint{2.997500in}{0.964444in}}%
\pgfpathlineto{\pgfqpoint{3.005250in}{1.079034in}}%
\pgfpathlineto{\pgfqpoint{3.013000in}{1.186887in}}%
\pgfpathlineto{\pgfqpoint{3.018167in}{1.225058in}}%
\pgfpathlineto{\pgfqpoint{3.020750in}{1.231819in}}%
\pgfpathlineto{\pgfqpoint{3.023333in}{1.231767in}}%
\pgfpathlineto{\pgfqpoint{3.025917in}{1.226926in}}%
\pgfpathlineto{\pgfqpoint{3.033667in}{1.207345in}}%
\pgfpathlineto{\pgfqpoint{3.036250in}{1.206160in}}%
\pgfpathlineto{\pgfqpoint{3.038833in}{1.210129in}}%
\pgfpathlineto{\pgfqpoint{3.041417in}{1.219939in}}%
\pgfpathlineto{\pgfqpoint{3.046583in}{1.256661in}}%
\pgfpathlineto{\pgfqpoint{3.059500in}{1.378148in}}%
\pgfpathlineto{\pgfqpoint{3.062083in}{1.390021in}}%
\pgfpathlineto{\pgfqpoint{3.064667in}{1.393964in}}%
\pgfpathlineto{\pgfqpoint{3.067250in}{1.389871in}}%
\pgfpathlineto{\pgfqpoint{3.069833in}{1.378528in}}%
\pgfpathlineto{\pgfqpoint{3.075000in}{1.340663in}}%
\pgfpathlineto{\pgfqpoint{3.085333in}{1.260634in}}%
\pgfpathlineto{\pgfqpoint{3.090500in}{1.241787in}}%
\pgfpathlineto{\pgfqpoint{3.093083in}{1.239624in}}%
\pgfpathlineto{\pgfqpoint{3.095667in}{1.242054in}}%
\pgfpathlineto{\pgfqpoint{3.098250in}{1.248449in}}%
\pgfpathlineto{\pgfqpoint{3.111167in}{1.295225in}}%
\pgfpathlineto{\pgfqpoint{3.113750in}{1.296621in}}%
\pgfpathlineto{\pgfqpoint{3.116333in}{1.292668in}}%
\pgfpathlineto{\pgfqpoint{3.118917in}{1.283248in}}%
\pgfpathlineto{\pgfqpoint{3.124083in}{1.249956in}}%
\pgfpathlineto{\pgfqpoint{3.131833in}{1.173876in}}%
\pgfpathlineto{\pgfqpoint{3.139583in}{1.069680in}}%
\pgfpathlineto{\pgfqpoint{3.147333in}{0.927598in}}%
\pgfpathlineto{\pgfqpoint{3.160250in}{0.664992in}}%
\pgfpathlineto{\pgfqpoint{3.165417in}{0.608281in}}%
\pgfpathlineto{\pgfqpoint{3.168000in}{0.599908in}}%
\pgfpathlineto{\pgfqpoint{3.170583in}{0.605677in}}%
\pgfpathlineto{\pgfqpoint{3.173167in}{0.624612in}}%
\pgfpathlineto{\pgfqpoint{3.178333in}{0.691855in}}%
\pgfpathlineto{\pgfqpoint{3.186083in}{0.807778in}}%
\pgfpathlineto{\pgfqpoint{3.191250in}{0.853339in}}%
\pgfpathlineto{\pgfqpoint{3.193833in}{0.862913in}}%
\pgfpathlineto{\pgfqpoint{3.196417in}{0.866143in}}%
\pgfpathlineto{\pgfqpoint{3.199000in}{0.866729in}}%
\pgfpathlineto{\pgfqpoint{3.201583in}{0.869100in}}%
\pgfpathlineto{\pgfqpoint{3.204167in}{0.877611in}}%
\pgfpathlineto{\pgfqpoint{3.206750in}{0.895787in}}%
\pgfpathlineto{\pgfqpoint{3.211917in}{0.967814in}}%
\pgfpathlineto{\pgfqpoint{3.219667in}{1.145453in}}%
\pgfpathlineto{\pgfqpoint{3.227417in}{1.319005in}}%
\pgfpathlineto{\pgfqpoint{3.232583in}{1.390314in}}%
\pgfpathlineto{\pgfqpoint{3.237750in}{1.423286in}}%
\pgfpathlineto{\pgfqpoint{3.240333in}{1.429956in}}%
\pgfpathlineto{\pgfqpoint{3.245500in}{1.435150in}}%
\pgfpathlineto{\pgfqpoint{3.255833in}{1.442484in}}%
\pgfpathlineto{\pgfqpoint{3.258417in}{1.441400in}}%
\pgfpathlineto{\pgfqpoint{3.261000in}{1.436893in}}%
\pgfpathlineto{\pgfqpoint{3.263583in}{1.428224in}}%
\pgfpathlineto{\pgfqpoint{3.268750in}{1.399082in}}%
\pgfpathlineto{\pgfqpoint{3.276500in}{1.348052in}}%
\pgfpathlineto{\pgfqpoint{3.279083in}{1.337334in}}%
\pgfpathlineto{\pgfqpoint{3.281667in}{1.332120in}}%
\pgfpathlineto{\pgfqpoint{3.284250in}{1.332278in}}%
\pgfpathlineto{\pgfqpoint{3.292000in}{1.345458in}}%
\pgfpathlineto{\pgfqpoint{3.294583in}{1.343225in}}%
\pgfpathlineto{\pgfqpoint{3.297167in}{1.331659in}}%
\pgfpathlineto{\pgfqpoint{3.299750in}{1.307949in}}%
\pgfpathlineto{\pgfqpoint{3.304917in}{1.219358in}}%
\pgfpathlineto{\pgfqpoint{3.312667in}{1.007105in}}%
\pgfpathlineto{\pgfqpoint{3.320417in}{0.798645in}}%
\pgfpathlineto{\pgfqpoint{3.325583in}{0.719455in}}%
\pgfpathlineto{\pgfqpoint{3.328167in}{0.704323in}}%
\pgfpathlineto{\pgfqpoint{3.330750in}{0.705092in}}%
\pgfpathlineto{\pgfqpoint{3.333333in}{0.719816in}}%
\pgfpathlineto{\pgfqpoint{3.338500in}{0.778781in}}%
\pgfpathlineto{\pgfqpoint{3.348833in}{0.914640in}}%
\pgfpathlineto{\pgfqpoint{3.354000in}{0.946856in}}%
\pgfpathlineto{\pgfqpoint{3.356583in}{0.949175in}}%
\pgfpathlineto{\pgfqpoint{3.359167in}{0.942072in}}%
\pgfpathlineto{\pgfqpoint{3.361750in}{0.926034in}}%
\pgfpathlineto{\pgfqpoint{3.366917in}{0.871908in}}%
\pgfpathlineto{\pgfqpoint{3.377250in}{0.742139in}}%
\pgfpathlineto{\pgfqpoint{3.379833in}{0.723903in}}%
\pgfpathlineto{\pgfqpoint{3.382417in}{0.717614in}}%
\pgfpathlineto{\pgfqpoint{3.385000in}{0.725362in}}%
\pgfpathlineto{\pgfqpoint{3.387583in}{0.748258in}}%
\pgfpathlineto{\pgfqpoint{3.392750in}{0.838650in}}%
\pgfpathlineto{\pgfqpoint{3.400500in}{1.057764in}}%
\pgfpathlineto{\pgfqpoint{3.410833in}{1.364292in}}%
\pgfpathlineto{\pgfqpoint{3.416000in}{1.461084in}}%
\pgfpathlineto{\pgfqpoint{3.418583in}{1.488599in}}%
\pgfpathlineto{\pgfqpoint{3.421167in}{1.502245in}}%
\pgfpathlineto{\pgfqpoint{3.423750in}{1.503155in}}%
\pgfpathlineto{\pgfqpoint{3.426333in}{1.493068in}}%
\pgfpathlineto{\pgfqpoint{3.431500in}{1.448325in}}%
\pgfpathlineto{\pgfqpoint{3.441833in}{1.315535in}}%
\pgfpathlineto{\pgfqpoint{3.449583in}{1.221526in}}%
\pgfpathlineto{\pgfqpoint{3.454750in}{1.178108in}}%
\pgfpathlineto{\pgfqpoint{3.457333in}{1.165771in}}%
\pgfpathlineto{\pgfqpoint{3.459917in}{1.160824in}}%
\pgfpathlineto{\pgfqpoint{3.462500in}{1.163367in}}%
\pgfpathlineto{\pgfqpoint{3.465083in}{1.172652in}}%
\pgfpathlineto{\pgfqpoint{3.478000in}{1.245992in}}%
\pgfpathlineto{\pgfqpoint{3.480583in}{1.248687in}}%
\pgfpathlineto{\pgfqpoint{3.483167in}{1.243420in}}%
\pgfpathlineto{\pgfqpoint{3.485750in}{1.230102in}}%
\pgfpathlineto{\pgfqpoint{3.490917in}{1.183853in}}%
\pgfpathlineto{\pgfqpoint{3.498667in}{1.102049in}}%
\pgfpathlineto{\pgfqpoint{3.503833in}{1.071872in}}%
\pgfpathlineto{\pgfqpoint{3.506417in}{1.069511in}}%
\pgfpathlineto{\pgfqpoint{3.509000in}{1.075690in}}%
\pgfpathlineto{\pgfqpoint{3.514167in}{1.107123in}}%
\pgfpathlineto{\pgfqpoint{3.519333in}{1.144671in}}%
\pgfpathlineto{\pgfqpoint{3.521917in}{1.157054in}}%
\pgfpathlineto{\pgfqpoint{3.524500in}{1.160902in}}%
\pgfpathlineto{\pgfqpoint{3.527083in}{1.153904in}}%
\pgfpathlineto{\pgfqpoint{3.529667in}{1.134800in}}%
\pgfpathlineto{\pgfqpoint{3.534833in}{1.061132in}}%
\pgfpathlineto{\pgfqpoint{3.552917in}{0.715347in}}%
\pgfpathlineto{\pgfqpoint{3.555500in}{0.702915in}}%
\pgfpathlineto{\pgfqpoint{3.558083in}{0.707888in}}%
\pgfpathlineto{\pgfqpoint{3.560667in}{0.730822in}}%
\pgfpathlineto{\pgfqpoint{3.565833in}{0.827224in}}%
\pgfpathlineto{\pgfqpoint{3.573583in}{1.059271in}}%
\pgfpathlineto{\pgfqpoint{3.581333in}{1.296034in}}%
\pgfpathlineto{\pgfqpoint{3.586500in}{1.392473in}}%
\pgfpathlineto{\pgfqpoint{3.589083in}{1.410474in}}%
\pgfpathlineto{\pgfqpoint{3.591667in}{1.405728in}}%
\pgfpathlineto{\pgfqpoint{3.594250in}{1.378338in}}%
\pgfpathlineto{\pgfqpoint{3.599417in}{1.264579in}}%
\pgfpathlineto{\pgfqpoint{3.614917in}{0.801378in}}%
\pgfpathlineto{\pgfqpoint{3.617500in}{0.761707in}}%
\pgfpathlineto{\pgfqpoint{3.620083in}{0.742732in}}%
\pgfpathlineto{\pgfqpoint{3.622667in}{0.745647in}}%
\pgfpathlineto{\pgfqpoint{3.625250in}{0.770317in}}%
\pgfpathlineto{\pgfqpoint{3.630417in}{0.878064in}}%
\pgfpathlineto{\pgfqpoint{3.638167in}{1.133718in}}%
\pgfpathlineto{\pgfqpoint{3.645917in}{1.392684in}}%
\pgfpathlineto{\pgfqpoint{3.651083in}{1.510610in}}%
\pgfpathlineto{\pgfqpoint{3.653667in}{1.544511in}}%
\pgfpathlineto{\pgfqpoint{3.656250in}{1.559578in}}%
\pgfpathlineto{\pgfqpoint{3.658833in}{1.554994in}}%
\pgfpathlineto{\pgfqpoint{3.661417in}{1.530471in}}%
\pgfpathlineto{\pgfqpoint{3.666583in}{1.423358in}}%
\pgfpathlineto{\pgfqpoint{3.671750in}{1.249400in}}%
\pgfpathlineto{\pgfqpoint{3.687250in}{0.653925in}}%
\pgfpathlineto{\pgfqpoint{3.689833in}{0.600721in}}%
\pgfpathlineto{\pgfqpoint{3.692417in}{0.572030in}}%
\pgfpathlineto{\pgfqpoint{3.695000in}{0.569080in}}%
\pgfpathlineto{\pgfqpoint{3.697583in}{0.591358in}}%
\pgfpathlineto{\pgfqpoint{3.700167in}{0.636715in}}%
\pgfpathlineto{\pgfqpoint{3.705333in}{0.781492in}}%
\pgfpathlineto{\pgfqpoint{3.720833in}{1.286113in}}%
\pgfpathlineto{\pgfqpoint{3.726000in}{1.366379in}}%
\pgfpathlineto{\pgfqpoint{3.728583in}{1.381587in}}%
\pgfpathlineto{\pgfqpoint{3.731167in}{1.380890in}}%
\pgfpathlineto{\pgfqpoint{3.733750in}{1.365718in}}%
\pgfpathlineto{\pgfqpoint{3.738917in}{1.300109in}}%
\pgfpathlineto{\pgfqpoint{3.757000in}{0.997096in}}%
\pgfpathlineto{\pgfqpoint{3.759583in}{0.988109in}}%
\pgfpathlineto{\pgfqpoint{3.762167in}{0.995892in}}%
\pgfpathlineto{\pgfqpoint{3.764750in}{1.020957in}}%
\pgfpathlineto{\pgfqpoint{3.769917in}{1.117628in}}%
\pgfpathlineto{\pgfqpoint{3.785417in}{1.508603in}}%
\pgfpathlineto{\pgfqpoint{3.790583in}{1.570733in}}%
\pgfpathlineto{\pgfqpoint{3.793167in}{1.580063in}}%
\pgfpathlineto{\pgfqpoint{3.795750in}{1.574705in}}%
\pgfpathlineto{\pgfqpoint{3.798333in}{1.554945in}}%
\pgfpathlineto{\pgfqpoint{3.803500in}{1.473803in}}%
\pgfpathlineto{\pgfqpoint{3.808667in}{1.340130in}}%
\pgfpathlineto{\pgfqpoint{3.816417in}{1.059589in}}%
\pgfpathlineto{\pgfqpoint{3.826750in}{0.672047in}}%
\pgfpathlineto{\pgfqpoint{3.831917in}{0.565016in}}%
\pgfpathlineto{\pgfqpoint{3.834500in}{0.549060in}}%
\pgfpathlineto{\pgfqpoint{3.837083in}{0.560392in}}%
\pgfpathlineto{\pgfqpoint{3.839667in}{0.598134in}}%
\pgfpathlineto{\pgfqpoint{3.844833in}{0.739533in}}%
\pgfpathlineto{\pgfqpoint{3.860333in}{1.271496in}}%
\pgfpathlineto{\pgfqpoint{3.862917in}{1.317424in}}%
\pgfpathlineto{\pgfqpoint{3.865500in}{1.342287in}}%
\pgfpathlineto{\pgfqpoint{3.868083in}{1.345476in}}%
\pgfpathlineto{\pgfqpoint{3.870667in}{1.327465in}}%
\pgfpathlineto{\pgfqpoint{3.873250in}{1.289629in}}%
\pgfpathlineto{\pgfqpoint{3.878417in}{1.163333in}}%
\pgfpathlineto{\pgfqpoint{3.886167in}{0.892047in}}%
\pgfpathlineto{\pgfqpoint{3.896500in}{0.538021in}}%
\pgfpathlineto{\pgfqpoint{3.901667in}{0.445883in}}%
\pgfpathlineto{\pgfqpoint{3.904250in}{0.435880in}}%
\pgfpathlineto{\pgfqpoint{3.906833in}{0.453355in}}%
\pgfpathlineto{\pgfqpoint{3.909417in}{0.499023in}}%
\pgfpathlineto{\pgfqpoint{3.914583in}{0.668998in}}%
\pgfpathlineto{\pgfqpoint{3.922333in}{1.055911in}}%
\pgfpathlineto{\pgfqpoint{3.930083in}{1.448137in}}%
\pgfpathlineto{\pgfqpoint{3.935250in}{1.630851in}}%
\pgfpathlineto{\pgfqpoint{3.937833in}{1.687145in}}%
\pgfpathlineto{\pgfqpoint{3.940417in}{1.718578in}}%
\pgfpathlineto{\pgfqpoint{3.943000in}{1.725991in}}%
\pgfpathlineto{\pgfqpoint{3.945583in}{1.711561in}}%
\pgfpathlineto{\pgfqpoint{3.948167in}{1.678484in}}%
\pgfpathlineto{\pgfqpoint{3.953333in}{1.572057in}}%
\pgfpathlineto{\pgfqpoint{3.968833in}{1.198872in}}%
\pgfpathlineto{\pgfqpoint{3.974000in}{1.126918in}}%
\pgfpathlineto{\pgfqpoint{3.976583in}{1.107452in}}%
\pgfpathlineto{\pgfqpoint{3.979167in}{1.099651in}}%
\pgfpathlineto{\pgfqpoint{3.981750in}{1.103370in}}%
\pgfpathlineto{\pgfqpoint{3.984333in}{1.117840in}}%
\pgfpathlineto{\pgfqpoint{3.989500in}{1.172457in}}%
\pgfpathlineto{\pgfqpoint{3.999833in}{1.305218in}}%
\pgfpathlineto{\pgfqpoint{4.002417in}{1.323730in}}%
\pgfpathlineto{\pgfqpoint{4.005000in}{1.330432in}}%
\pgfpathlineto{\pgfqpoint{4.007583in}{1.323520in}}%
\pgfpathlineto{\pgfqpoint{4.010167in}{1.302198in}}%
\pgfpathlineto{\pgfqpoint{4.015333in}{1.218770in}}%
\pgfpathlineto{\pgfqpoint{4.025667in}{0.961849in}}%
\pgfpathlineto{\pgfqpoint{4.033417in}{0.803333in}}%
\pgfpathlineto{\pgfqpoint{4.038583in}{0.751216in}}%
\pgfpathlineto{\pgfqpoint{4.041167in}{0.743085in}}%
\pgfpathlineto{\pgfqpoint{4.043750in}{0.746685in}}%
\pgfpathlineto{\pgfqpoint{4.046333in}{0.761736in}}%
\pgfpathlineto{\pgfqpoint{4.051500in}{0.824995in}}%
\pgfpathlineto{\pgfqpoint{4.056667in}{0.928550in}}%
\pgfpathlineto{\pgfqpoint{4.072167in}{1.305410in}}%
\pgfpathlineto{\pgfqpoint{4.074750in}{1.337782in}}%
\pgfpathlineto{\pgfqpoint{4.077333in}{1.350661in}}%
\pgfpathlineto{\pgfqpoint{4.079917in}{1.341438in}}%
\pgfpathlineto{\pgfqpoint{4.082500in}{1.308803in}}%
\pgfpathlineto{\pgfqpoint{4.087667in}{1.175818in}}%
\pgfpathlineto{\pgfqpoint{4.095417in}{0.858032in}}%
\pgfpathlineto{\pgfqpoint{4.103167in}{0.538711in}}%
\pgfpathlineto{\pgfqpoint{4.108333in}{0.407652in}}%
\pgfpathlineto{\pgfqpoint{4.110917in}{0.379708in}}%
\pgfpathlineto{\pgfqpoint{4.113500in}{0.379458in}}%
\pgfpathlineto{\pgfqpoint{4.116083in}{0.407131in}}%
\pgfpathlineto{\pgfqpoint{4.118667in}{0.461690in}}%
\pgfpathlineto{\pgfqpoint{4.123833in}{0.641805in}}%
\pgfpathlineto{\pgfqpoint{4.131583in}{1.029256in}}%
\pgfpathlineto{\pgfqpoint{4.141917in}{1.538216in}}%
\pgfpathlineto{\pgfqpoint{4.147083in}{1.698344in}}%
\pgfpathlineto{\pgfqpoint{4.149667in}{1.744584in}}%
\pgfpathlineto{\pgfqpoint{4.152250in}{1.767702in}}%
\pgfpathlineto{\pgfqpoint{4.154833in}{1.768821in}}%
\pgfpathlineto{\pgfqpoint{4.157417in}{1.750155in}}%
\pgfpathlineto{\pgfqpoint{4.162583in}{1.666633in}}%
\pgfpathlineto{\pgfqpoint{4.178083in}{1.349306in}}%
\pgfpathlineto{\pgfqpoint{4.180667in}{1.325507in}}%
\pgfpathlineto{\pgfqpoint{4.183250in}{1.315186in}}%
\pgfpathlineto{\pgfqpoint{4.185833in}{1.318056in}}%
\pgfpathlineto{\pgfqpoint{4.188417in}{1.332767in}}%
\pgfpathlineto{\pgfqpoint{4.193583in}{1.387376in}}%
\pgfpathlineto{\pgfqpoint{4.201333in}{1.475225in}}%
\pgfpathlineto{\pgfqpoint{4.203917in}{1.489241in}}%
\pgfpathlineto{\pgfqpoint{4.206500in}{1.489359in}}%
\pgfpathlineto{\pgfqpoint{4.209083in}{1.472809in}}%
\pgfpathlineto{\pgfqpoint{4.211667in}{1.437714in}}%
\pgfpathlineto{\pgfqpoint{4.216833in}{1.309514in}}%
\pgfpathlineto{\pgfqpoint{4.222000in}{1.111125in}}%
\pgfpathlineto{\pgfqpoint{4.239524in}{0.320000in}}%
\pgfpathmoveto{\pgfqpoint{4.256187in}{0.320000in}}%
\pgfpathlineto{\pgfqpoint{4.260750in}{0.477018in}}%
\pgfpathlineto{\pgfqpoint{4.278833in}{1.188424in}}%
\pgfpathlineto{\pgfqpoint{4.284000in}{1.316708in}}%
\pgfpathlineto{\pgfqpoint{4.289167in}{1.389692in}}%
\pgfpathlineto{\pgfqpoint{4.291750in}{1.404533in}}%
\pgfpathlineto{\pgfqpoint{4.294333in}{1.405296in}}%
\pgfpathlineto{\pgfqpoint{4.296917in}{1.392965in}}%
\pgfpathlineto{\pgfqpoint{4.302083in}{1.336189in}}%
\pgfpathlineto{\pgfqpoint{4.317583in}{1.106847in}}%
\pgfpathlineto{\pgfqpoint{4.322750in}{1.075147in}}%
\pgfpathlineto{\pgfqpoint{4.325333in}{1.072347in}}%
\pgfpathlineto{\pgfqpoint{4.327917in}{1.077788in}}%
\pgfpathlineto{\pgfqpoint{4.330500in}{1.090731in}}%
\pgfpathlineto{\pgfqpoint{4.335667in}{1.135610in}}%
\pgfpathlineto{\pgfqpoint{4.343417in}{1.239195in}}%
\pgfpathlineto{\pgfqpoint{4.351167in}{1.377762in}}%
\pgfpathlineto{\pgfqpoint{4.364083in}{1.633510in}}%
\pgfpathlineto{\pgfqpoint{4.369250in}{1.691239in}}%
\pgfpathlineto{\pgfqpoint{4.371833in}{1.697946in}}%
\pgfpathlineto{\pgfqpoint{4.374417in}{1.686655in}}%
\pgfpathlineto{\pgfqpoint{4.377000in}{1.656110in}}%
\pgfpathlineto{\pgfqpoint{4.382167in}{1.538020in}}%
\pgfpathlineto{\pgfqpoint{4.389917in}{1.249933in}}%
\pgfpathlineto{\pgfqpoint{4.405417in}{0.620846in}}%
\pgfpathlineto{\pgfqpoint{4.410583in}{0.479813in}}%
\pgfpathlineto{\pgfqpoint{4.415750in}{0.393658in}}%
\pgfpathlineto{\pgfqpoint{4.418333in}{0.373243in}}%
\pgfpathlineto{\pgfqpoint{4.420917in}{0.368596in}}%
\pgfpathlineto{\pgfqpoint{4.423500in}{0.379774in}}%
\pgfpathlineto{\pgfqpoint{4.426083in}{0.406334in}}%
\pgfpathlineto{\pgfqpoint{4.431250in}{0.500789in}}%
\pgfpathlineto{\pgfqpoint{4.439000in}{0.711601in}}%
\pgfpathlineto{\pgfqpoint{4.449333in}{0.992881in}}%
\pgfpathlineto{\pgfqpoint{4.454500in}{1.088152in}}%
\pgfpathlineto{\pgfqpoint{4.459667in}{1.145255in}}%
\pgfpathlineto{\pgfqpoint{4.464833in}{1.173426in}}%
\pgfpathlineto{\pgfqpoint{4.472583in}{1.200436in}}%
\pgfpathlineto{\pgfqpoint{4.477750in}{1.233402in}}%
\pgfpathlineto{\pgfqpoint{4.482917in}{1.286852in}}%
\pgfpathlineto{\pgfqpoint{4.498417in}{1.477251in}}%
\pgfpathlineto{\pgfqpoint{4.503583in}{1.506637in}}%
\pgfpathlineto{\pgfqpoint{4.506167in}{1.510657in}}%
\pgfpathlineto{\pgfqpoint{4.508750in}{1.507750in}}%
\pgfpathlineto{\pgfqpoint{4.511333in}{1.498621in}}%
\pgfpathlineto{\pgfqpoint{4.516500in}{1.466563in}}%
\pgfpathlineto{\pgfqpoint{4.524250in}{1.410198in}}%
\pgfpathlineto{\pgfqpoint{4.529417in}{1.388560in}}%
\pgfpathlineto{\pgfqpoint{4.532000in}{1.386034in}}%
\pgfpathlineto{\pgfqpoint{4.534583in}{1.388960in}}%
\pgfpathlineto{\pgfqpoint{4.539750in}{1.405969in}}%
\pgfpathlineto{\pgfqpoint{4.544917in}{1.422739in}}%
\pgfpathlineto{\pgfqpoint{4.547500in}{1.424130in}}%
\pgfpathlineto{\pgfqpoint{4.550083in}{1.417285in}}%
\pgfpathlineto{\pgfqpoint{4.552667in}{1.400074in}}%
\pgfpathlineto{\pgfqpoint{4.557833in}{1.329733in}}%
\pgfpathlineto{\pgfqpoint{4.563000in}{1.212306in}}%
\pgfpathlineto{\pgfqpoint{4.573333in}{0.895645in}}%
\pgfpathlineto{\pgfqpoint{4.581083in}{0.673955in}}%
\pgfpathlineto{\pgfqpoint{4.586250in}{0.567397in}}%
\pgfpathlineto{\pgfqpoint{4.591417in}{0.501860in}}%
\pgfpathlineto{\pgfqpoint{4.594000in}{0.484157in}}%
\pgfpathlineto{\pgfqpoint{4.596583in}{0.475735in}}%
\pgfpathlineto{\pgfqpoint{4.599167in}{0.475883in}}%
\pgfpathlineto{\pgfqpoint{4.601750in}{0.483875in}}%
\pgfpathlineto{\pgfqpoint{4.606917in}{0.520355in}}%
\pgfpathlineto{\pgfqpoint{4.614667in}{0.610808in}}%
\pgfpathlineto{\pgfqpoint{4.622417in}{0.706267in}}%
\pgfpathlineto{\pgfqpoint{4.627583in}{0.749601in}}%
\pgfpathlineto{\pgfqpoint{4.632750in}{0.769910in}}%
\pgfpathlineto{\pgfqpoint{4.637917in}{0.775211in}}%
\pgfpathlineto{\pgfqpoint{4.640500in}{0.777557in}}%
\pgfpathlineto{\pgfqpoint{4.643083in}{0.782982in}}%
\pgfpathlineto{\pgfqpoint{4.645667in}{0.793673in}}%
\pgfpathlineto{\pgfqpoint{4.648250in}{0.811310in}}%
\pgfpathlineto{\pgfqpoint{4.653417in}{0.871025in}}%
\pgfpathlineto{\pgfqpoint{4.658583in}{0.963224in}}%
\pgfpathlineto{\pgfqpoint{4.668917in}{1.210865in}}%
\pgfpathlineto{\pgfqpoint{4.679250in}{1.454566in}}%
\pgfpathlineto{\pgfqpoint{4.687000in}{1.587762in}}%
\pgfpathlineto{\pgfqpoint{4.694750in}{1.675468in}}%
\pgfpathlineto{\pgfqpoint{4.707667in}{1.797043in}}%
\pgfpathlineto{\pgfqpoint{4.723167in}{1.966193in}}%
\pgfpathlineto{\pgfqpoint{4.725750in}{1.977026in}}%
\pgfpathlineto{\pgfqpoint{4.728333in}{1.975643in}}%
\pgfpathlineto{\pgfqpoint{4.730917in}{1.959384in}}%
\pgfpathlineto{\pgfqpoint{4.733500in}{1.926108in}}%
\pgfpathlineto{\pgfqpoint{4.738667in}{1.804079in}}%
\pgfpathlineto{\pgfqpoint{4.743833in}{1.611524in}}%
\pgfpathlineto{\pgfqpoint{4.764500in}{0.680765in}}%
\pgfpathlineto{\pgfqpoint{4.769667in}{0.559504in}}%
\pgfpathlineto{\pgfqpoint{4.772250in}{0.524916in}}%
\pgfpathlineto{\pgfqpoint{4.774833in}{0.506253in}}%
\pgfpathlineto{\pgfqpoint{4.777417in}{0.501407in}}%
\pgfpathlineto{\pgfqpoint{4.780000in}{0.507604in}}%
\pgfpathlineto{\pgfqpoint{4.785167in}{0.539951in}}%
\pgfpathlineto{\pgfqpoint{4.790333in}{0.576246in}}%
\pgfpathlineto{\pgfqpoint{4.792917in}{0.588334in}}%
\pgfpathlineto{\pgfqpoint{4.795500in}{0.593361in}}%
\pgfpathlineto{\pgfqpoint{4.798083in}{0.589871in}}%
\pgfpathlineto{\pgfqpoint{4.800667in}{0.577130in}}%
\pgfpathlineto{\pgfqpoint{4.805833in}{0.524742in}}%
\pgfpathlineto{\pgfqpoint{4.813583in}{0.400932in}}%
\pgfpathlineto{\pgfqpoint{4.818645in}{0.320000in}}%
\pgfpathmoveto{\pgfqpoint{4.833419in}{0.320000in}}%
\pgfpathlineto{\pgfqpoint{4.836833in}{0.394023in}}%
\pgfpathlineto{\pgfqpoint{4.842000in}{0.561831in}}%
\pgfpathlineto{\pgfqpoint{4.852333in}{1.010628in}}%
\pgfpathlineto{\pgfqpoint{4.862667in}{1.430715in}}%
\pgfpathlineto{\pgfqpoint{4.867833in}{1.573574in}}%
\pgfpathlineto{\pgfqpoint{4.873000in}{1.659039in}}%
\pgfpathlineto{\pgfqpoint{4.875583in}{1.681423in}}%
\pgfpathlineto{\pgfqpoint{4.878167in}{1.692160in}}%
\pgfpathlineto{\pgfqpoint{4.880750in}{1.693248in}}%
\pgfpathlineto{\pgfqpoint{4.883333in}{1.686895in}}%
\pgfpathlineto{\pgfqpoint{4.888500in}{1.660566in}}%
\pgfpathlineto{\pgfqpoint{4.898833in}{1.599185in}}%
\pgfpathlineto{\pgfqpoint{4.904000in}{1.577324in}}%
\pgfpathlineto{\pgfqpoint{4.911750in}{1.555102in}}%
\pgfpathlineto{\pgfqpoint{4.916917in}{1.540584in}}%
\pgfpathlineto{\pgfqpoint{4.922083in}{1.518064in}}%
\pgfpathlineto{\pgfqpoint{4.927250in}{1.479650in}}%
\pgfpathlineto{\pgfqpoint{4.932417in}{1.420008in}}%
\pgfpathlineto{\pgfqpoint{4.940167in}{1.293170in}}%
\pgfpathlineto{\pgfqpoint{4.958250in}{0.968912in}}%
\pgfpathlineto{\pgfqpoint{4.966000in}{0.875819in}}%
\pgfpathlineto{\pgfqpoint{4.971167in}{0.833590in}}%
\pgfpathlineto{\pgfqpoint{4.976333in}{0.804789in}}%
\pgfpathlineto{\pgfqpoint{4.981500in}{0.787792in}}%
\pgfpathlineto{\pgfqpoint{4.984083in}{0.783944in}}%
\pgfpathlineto{\pgfqpoint{4.986667in}{0.783716in}}%
\pgfpathlineto{\pgfqpoint{4.989250in}{0.787643in}}%
\pgfpathlineto{\pgfqpoint{4.991833in}{0.796240in}}%
\pgfpathlineto{\pgfqpoint{4.997000in}{0.828591in}}%
\pgfpathlineto{\pgfqpoint{5.002167in}{0.879993in}}%
\pgfpathlineto{\pgfqpoint{5.028000in}{1.185870in}}%
\pgfpathlineto{\pgfqpoint{5.033167in}{1.220370in}}%
\pgfpathlineto{\pgfqpoint{5.038333in}{1.238682in}}%
\pgfpathlineto{\pgfqpoint{5.040917in}{1.241110in}}%
\pgfpathlineto{\pgfqpoint{5.043500in}{1.239129in}}%
\pgfpathlineto{\pgfqpoint{5.046083in}{1.233149in}}%
\pgfpathlineto{\pgfqpoint{5.051250in}{1.212002in}}%
\pgfpathlineto{\pgfqpoint{5.071917in}{1.109420in}}%
\pgfpathlineto{\pgfqpoint{5.079667in}{1.070734in}}%
\pgfpathlineto{\pgfqpoint{5.084833in}{1.034359in}}%
\pgfpathlineto{\pgfqpoint{5.100333in}{0.900546in}}%
\pgfpathlineto{\pgfqpoint{5.102917in}{0.892111in}}%
\pgfpathlineto{\pgfqpoint{5.105500in}{0.893068in}}%
\pgfpathlineto{\pgfqpoint{5.108083in}{0.904389in}}%
\pgfpathlineto{\pgfqpoint{5.110667in}{0.926103in}}%
\pgfpathlineto{\pgfqpoint{5.115833in}{0.996076in}}%
\pgfpathlineto{\pgfqpoint{5.131333in}{1.242927in}}%
\pgfpathlineto{\pgfqpoint{5.136500in}{1.278242in}}%
\pgfpathlineto{\pgfqpoint{5.139083in}{1.282526in}}%
\pgfpathlineto{\pgfqpoint{5.141667in}{1.277971in}}%
\pgfpathlineto{\pgfqpoint{5.144250in}{1.264907in}}%
\pgfpathlineto{\pgfqpoint{5.149417in}{1.215516in}}%
\pgfpathlineto{\pgfqpoint{5.157167in}{1.098063in}}%
\pgfpathlineto{\pgfqpoint{5.170083in}{0.883529in}}%
\pgfpathlineto{\pgfqpoint{5.175250in}{0.824616in}}%
\pgfpathlineto{\pgfqpoint{5.177833in}{0.806097in}}%
\pgfpathlineto{\pgfqpoint{5.180417in}{0.796505in}}%
\pgfpathlineto{\pgfqpoint{5.183000in}{0.796854in}}%
\pgfpathlineto{\pgfqpoint{5.185583in}{0.807810in}}%
\pgfpathlineto{\pgfqpoint{5.188167in}{0.829539in}}%
\pgfpathlineto{\pgfqpoint{5.193333in}{0.902789in}}%
\pgfpathlineto{\pgfqpoint{5.214000in}{1.273219in}}%
\pgfpathlineto{\pgfqpoint{5.216583in}{1.288864in}}%
\pgfpathlineto{\pgfqpoint{5.219167in}{1.293749in}}%
\pgfpathlineto{\pgfqpoint{5.221750in}{1.288369in}}%
\pgfpathlineto{\pgfqpoint{5.224333in}{1.273714in}}%
\pgfpathlineto{\pgfqpoint{5.229500in}{1.222320in}}%
\pgfpathlineto{\pgfqpoint{5.250167in}{0.963326in}}%
\pgfpathlineto{\pgfqpoint{5.252750in}{0.950374in}}%
\pgfpathlineto{\pgfqpoint{5.255333in}{0.946961in}}%
\pgfpathlineto{\pgfqpoint{5.257917in}{0.954604in}}%
\pgfpathlineto{\pgfqpoint{5.260500in}{0.974540in}}%
\pgfpathlineto{\pgfqpoint{5.265667in}{1.053613in}}%
\pgfpathlineto{\pgfqpoint{5.270833in}{1.180781in}}%
\pgfpathlineto{\pgfqpoint{5.283750in}{1.554499in}}%
\pgfpathlineto{\pgfqpoint{5.288917in}{1.637347in}}%
\pgfpathlineto{\pgfqpoint{5.291500in}{1.651488in}}%
\pgfpathlineto{\pgfqpoint{5.294083in}{1.646389in}}%
\pgfpathlineto{\pgfqpoint{5.296667in}{1.622884in}}%
\pgfpathlineto{\pgfqpoint{5.301833in}{1.529522in}}%
\pgfpathlineto{\pgfqpoint{5.319917in}{1.095789in}}%
\pgfpathlineto{\pgfqpoint{5.325083in}{1.043178in}}%
\pgfpathlineto{\pgfqpoint{5.327667in}{1.034329in}}%
\pgfpathlineto{\pgfqpoint{5.330250in}{1.035965in}}%
\pgfpathlineto{\pgfqpoint{5.332833in}{1.046681in}}%
\pgfpathlineto{\pgfqpoint{5.338000in}{1.088640in}}%
\pgfpathlineto{\pgfqpoint{5.353500in}{1.250585in}}%
\pgfpathlineto{\pgfqpoint{5.356083in}{1.263936in}}%
\pgfpathlineto{\pgfqpoint{5.358667in}{1.268480in}}%
\pgfpathlineto{\pgfqpoint{5.361250in}{1.262533in}}%
\pgfpathlineto{\pgfqpoint{5.363833in}{1.244708in}}%
\pgfpathlineto{\pgfqpoint{5.369000in}{1.170220in}}%
\pgfpathlineto{\pgfqpoint{5.374167in}{1.044522in}}%
\pgfpathlineto{\pgfqpoint{5.381917in}{0.783263in}}%
\pgfpathlineto{\pgfqpoint{5.392250in}{0.418850in}}%
\pgfpathlineto{\pgfqpoint{5.396172in}{0.320000in}}%
\pgfpathlineto{\pgfqpoint{5.396172in}{0.320000in}}%
\pgfusepath{stroke}%
\end{pgfscope}%
\begin{pgfscope}%
\pgfpathrectangle{\pgfqpoint{0.750000in}{0.330000in}}{\pgfqpoint{4.650000in}{2.310000in}}%
\pgfusepath{clip}%
\pgfsetbuttcap%
\pgfsetroundjoin%
\pgfsetlinewidth{1.003750pt}%
\definecolor{currentstroke}{rgb}{0.000000,0.000000,0.000000}%
\pgfsetstrokecolor{currentstroke}%
\pgfsetdash{{1.000000pt}{1.650000pt}}{0.000000pt}%
\pgfpathmoveto{\pgfqpoint{0.747417in}{1.100758in}}%
\pgfpathlineto{\pgfqpoint{0.783583in}{1.101323in}}%
\pgfpathlineto{\pgfqpoint{0.801667in}{1.100842in}}%
\pgfpathlineto{\pgfqpoint{0.819750in}{1.100421in}}%
\pgfpathlineto{\pgfqpoint{0.956667in}{1.101324in}}%
\pgfpathlineto{\pgfqpoint{0.964417in}{1.100296in}}%
\pgfpathlineto{\pgfqpoint{0.987667in}{1.104451in}}%
\pgfpathlineto{\pgfqpoint{1.005750in}{1.106632in}}%
\pgfpathlineto{\pgfqpoint{1.013500in}{1.105075in}}%
\pgfpathlineto{\pgfqpoint{1.026417in}{1.100797in}}%
\pgfpathlineto{\pgfqpoint{1.031583in}{1.101482in}}%
\pgfpathlineto{\pgfqpoint{1.036750in}{1.100235in}}%
\pgfpathlineto{\pgfqpoint{1.044500in}{1.106724in}}%
\pgfpathlineto{\pgfqpoint{1.052250in}{1.113883in}}%
\pgfpathlineto{\pgfqpoint{1.057417in}{1.116076in}}%
\pgfpathlineto{\pgfqpoint{1.062583in}{1.114576in}}%
\pgfpathlineto{\pgfqpoint{1.067750in}{1.109059in}}%
\pgfpathlineto{\pgfqpoint{1.072917in}{1.100418in}}%
\pgfpathlineto{\pgfqpoint{1.088417in}{1.123928in}}%
\pgfpathlineto{\pgfqpoint{1.093583in}{1.125120in}}%
\pgfpathlineto{\pgfqpoint{1.098750in}{1.122083in}}%
\pgfpathlineto{\pgfqpoint{1.106500in}{1.112702in}}%
\pgfpathlineto{\pgfqpoint{1.116833in}{1.100039in}}%
\pgfpathlineto{\pgfqpoint{1.122000in}{1.103566in}}%
\pgfpathlineto{\pgfqpoint{1.127167in}{1.104704in}}%
\pgfpathlineto{\pgfqpoint{1.132333in}{1.103618in}}%
\pgfpathlineto{\pgfqpoint{1.140083in}{1.101059in}}%
\pgfpathlineto{\pgfqpoint{1.150417in}{1.110740in}}%
\pgfpathlineto{\pgfqpoint{1.160750in}{1.119886in}}%
\pgfpathlineto{\pgfqpoint{1.165917in}{1.121001in}}%
\pgfpathlineto{\pgfqpoint{1.171083in}{1.117910in}}%
\pgfpathlineto{\pgfqpoint{1.176250in}{1.110400in}}%
\pgfpathlineto{\pgfqpoint{1.181417in}{1.100327in}}%
\pgfpathlineto{\pgfqpoint{1.191750in}{1.122087in}}%
\pgfpathlineto{\pgfqpoint{1.196917in}{1.128105in}}%
\pgfpathlineto{\pgfqpoint{1.199500in}{1.129162in}}%
\pgfpathlineto{\pgfqpoint{1.202083in}{1.128881in}}%
\pgfpathlineto{\pgfqpoint{1.207250in}{1.124825in}}%
\pgfpathlineto{\pgfqpoint{1.225333in}{1.101921in}}%
\pgfpathlineto{\pgfqpoint{1.227917in}{1.100312in}}%
\pgfpathlineto{\pgfqpoint{1.238250in}{1.101692in}}%
\pgfpathlineto{\pgfqpoint{1.248583in}{1.101354in}}%
\pgfpathlineto{\pgfqpoint{1.256333in}{1.107133in}}%
\pgfpathlineto{\pgfqpoint{1.274417in}{1.123559in}}%
\pgfpathlineto{\pgfqpoint{1.277000in}{1.123542in}}%
\pgfpathlineto{\pgfqpoint{1.279583in}{1.122284in}}%
\pgfpathlineto{\pgfqpoint{1.284750in}{1.115923in}}%
\pgfpathlineto{\pgfqpoint{1.292500in}{1.101080in}}%
\pgfpathlineto{\pgfqpoint{1.302833in}{1.125451in}}%
\pgfpathlineto{\pgfqpoint{1.308000in}{1.132584in}}%
\pgfpathlineto{\pgfqpoint{1.310583in}{1.134013in}}%
\pgfpathlineto{\pgfqpoint{1.313167in}{1.133847in}}%
\pgfpathlineto{\pgfqpoint{1.315750in}{1.132063in}}%
\pgfpathlineto{\pgfqpoint{1.320917in}{1.124039in}}%
\pgfpathlineto{\pgfqpoint{1.331250in}{1.101189in}}%
\pgfpathlineto{\pgfqpoint{1.336417in}{1.111222in}}%
\pgfpathlineto{\pgfqpoint{1.341583in}{1.115561in}}%
\pgfpathlineto{\pgfqpoint{1.344167in}{1.115263in}}%
\pgfpathlineto{\pgfqpoint{1.346750in}{1.113378in}}%
\pgfpathlineto{\pgfqpoint{1.351917in}{1.105601in}}%
\pgfpathlineto{\pgfqpoint{1.354500in}{1.100290in}}%
\pgfpathlineto{\pgfqpoint{1.367417in}{1.124978in}}%
\pgfpathlineto{\pgfqpoint{1.372583in}{1.128087in}}%
\pgfpathlineto{\pgfqpoint{1.375167in}{1.127434in}}%
\pgfpathlineto{\pgfqpoint{1.380333in}{1.122119in}}%
\pgfpathlineto{\pgfqpoint{1.388083in}{1.107915in}}%
\pgfpathlineto{\pgfqpoint{1.390667in}{1.102736in}}%
\pgfpathlineto{\pgfqpoint{1.393250in}{1.102230in}}%
\pgfpathlineto{\pgfqpoint{1.401000in}{1.114306in}}%
\pgfpathlineto{\pgfqpoint{1.406167in}{1.119158in}}%
\pgfpathlineto{\pgfqpoint{1.411333in}{1.120964in}}%
\pgfpathlineto{\pgfqpoint{1.416500in}{1.119373in}}%
\pgfpathlineto{\pgfqpoint{1.421667in}{1.114305in}}%
\pgfpathlineto{\pgfqpoint{1.429417in}{1.101756in}}%
\pgfpathlineto{\pgfqpoint{1.432000in}{1.102953in}}%
\pgfpathlineto{\pgfqpoint{1.439750in}{1.114794in}}%
\pgfpathlineto{\pgfqpoint{1.444917in}{1.118065in}}%
\pgfpathlineto{\pgfqpoint{1.447500in}{1.117740in}}%
\pgfpathlineto{\pgfqpoint{1.452667in}{1.113144in}}%
\pgfpathlineto{\pgfqpoint{1.460417in}{1.100621in}}%
\pgfpathlineto{\pgfqpoint{1.470750in}{1.120377in}}%
\pgfpathlineto{\pgfqpoint{1.475917in}{1.127271in}}%
\pgfpathlineto{\pgfqpoint{1.481083in}{1.130590in}}%
\pgfpathlineto{\pgfqpoint{1.483667in}{1.130505in}}%
\pgfpathlineto{\pgfqpoint{1.486250in}{1.129045in}}%
\pgfpathlineto{\pgfqpoint{1.491417in}{1.121588in}}%
\pgfpathlineto{\pgfqpoint{1.496583in}{1.108341in}}%
\pgfpathlineto{\pgfqpoint{1.499167in}{1.100114in}}%
\pgfpathlineto{\pgfqpoint{1.509500in}{1.132460in}}%
\pgfpathlineto{\pgfqpoint{1.514667in}{1.141171in}}%
\pgfpathlineto{\pgfqpoint{1.517250in}{1.142325in}}%
\pgfpathlineto{\pgfqpoint{1.519833in}{1.141218in}}%
\pgfpathlineto{\pgfqpoint{1.522417in}{1.137942in}}%
\pgfpathlineto{\pgfqpoint{1.527583in}{1.125850in}}%
\pgfpathlineto{\pgfqpoint{1.535333in}{1.100311in}}%
\pgfpathlineto{\pgfqpoint{1.543083in}{1.125488in}}%
\pgfpathlineto{\pgfqpoint{1.548250in}{1.137697in}}%
\pgfpathlineto{\pgfqpoint{1.553417in}{1.145387in}}%
\pgfpathlineto{\pgfqpoint{1.556000in}{1.147362in}}%
\pgfpathlineto{\pgfqpoint{1.558583in}{1.147874in}}%
\pgfpathlineto{\pgfqpoint{1.561167in}{1.146694in}}%
\pgfpathlineto{\pgfqpoint{1.563750in}{1.143599in}}%
\pgfpathlineto{\pgfqpoint{1.568917in}{1.131218in}}%
\pgfpathlineto{\pgfqpoint{1.576667in}{1.100325in}}%
\pgfpathlineto{\pgfqpoint{1.587000in}{1.143999in}}%
\pgfpathlineto{\pgfqpoint{1.589583in}{1.150288in}}%
\pgfpathlineto{\pgfqpoint{1.592167in}{1.153333in}}%
\pgfpathlineto{\pgfqpoint{1.594750in}{1.152680in}}%
\pgfpathlineto{\pgfqpoint{1.597333in}{1.148128in}}%
\pgfpathlineto{\pgfqpoint{1.602500in}{1.128071in}}%
\pgfpathlineto{\pgfqpoint{1.607667in}{1.102249in}}%
\pgfpathlineto{\pgfqpoint{1.615417in}{1.149119in}}%
\pgfpathlineto{\pgfqpoint{1.620583in}{1.169962in}}%
\pgfpathlineto{\pgfqpoint{1.623167in}{1.175428in}}%
\pgfpathlineto{\pgfqpoint{1.625750in}{1.177300in}}%
\pgfpathlineto{\pgfqpoint{1.628333in}{1.175603in}}%
\pgfpathlineto{\pgfqpoint{1.630917in}{1.170513in}}%
\pgfpathlineto{\pgfqpoint{1.636083in}{1.151394in}}%
\pgfpathlineto{\pgfqpoint{1.643833in}{1.106827in}}%
\pgfpathlineto{\pgfqpoint{1.646417in}{1.110172in}}%
\pgfpathlineto{\pgfqpoint{1.656750in}{1.172345in}}%
\pgfpathlineto{\pgfqpoint{1.661917in}{1.190842in}}%
\pgfpathlineto{\pgfqpoint{1.664500in}{1.194912in}}%
\pgfpathlineto{\pgfqpoint{1.667083in}{1.195015in}}%
\pgfpathlineto{\pgfqpoint{1.669667in}{1.190977in}}%
\pgfpathlineto{\pgfqpoint{1.672250in}{1.182826in}}%
\pgfpathlineto{\pgfqpoint{1.677417in}{1.155436in}}%
\pgfpathlineto{\pgfqpoint{1.685167in}{1.102503in}}%
\pgfpathlineto{\pgfqpoint{1.692917in}{1.153737in}}%
\pgfpathlineto{\pgfqpoint{1.695500in}{1.163805in}}%
\pgfpathlineto{\pgfqpoint{1.698083in}{1.168930in}}%
\pgfpathlineto{\pgfqpoint{1.700667in}{1.168671in}}%
\pgfpathlineto{\pgfqpoint{1.703250in}{1.162896in}}%
\pgfpathlineto{\pgfqpoint{1.708417in}{1.135833in}}%
\pgfpathlineto{\pgfqpoint{1.711000in}{1.115825in}}%
\pgfpathlineto{\pgfqpoint{1.713583in}{1.107159in}}%
\pgfpathlineto{\pgfqpoint{1.723917in}{1.200137in}}%
\pgfpathlineto{\pgfqpoint{1.729083in}{1.226634in}}%
\pgfpathlineto{\pgfqpoint{1.731667in}{1.231264in}}%
\pgfpathlineto{\pgfqpoint{1.734250in}{1.229824in}}%
\pgfpathlineto{\pgfqpoint{1.736833in}{1.222581in}}%
\pgfpathlineto{\pgfqpoint{1.742000in}{1.193211in}}%
\pgfpathlineto{\pgfqpoint{1.749750in}{1.124581in}}%
\pgfpathlineto{\pgfqpoint{1.752333in}{1.101566in}}%
\pgfpathlineto{\pgfqpoint{1.765250in}{1.228365in}}%
\pgfpathlineto{\pgfqpoint{1.770417in}{1.264219in}}%
\pgfpathlineto{\pgfqpoint{1.773000in}{1.275287in}}%
\pgfpathlineto{\pgfqpoint{1.775583in}{1.280416in}}%
\pgfpathlineto{\pgfqpoint{1.778167in}{1.278720in}}%
\pgfpathlineto{\pgfqpoint{1.780750in}{1.269576in}}%
\pgfpathlineto{\pgfqpoint{1.783333in}{1.252763in}}%
\pgfpathlineto{\pgfqpoint{1.788500in}{1.197913in}}%
\pgfpathlineto{\pgfqpoint{1.793667in}{1.123790in}}%
\pgfpathlineto{\pgfqpoint{1.796250in}{1.115096in}}%
\pgfpathlineto{\pgfqpoint{1.801417in}{1.183224in}}%
\pgfpathlineto{\pgfqpoint{1.806583in}{1.222547in}}%
\pgfpathlineto{\pgfqpoint{1.809167in}{1.227503in}}%
\pgfpathlineto{\pgfqpoint{1.811750in}{1.222179in}}%
\pgfpathlineto{\pgfqpoint{1.814333in}{1.207328in}}%
\pgfpathlineto{\pgfqpoint{1.819500in}{1.155847in}}%
\pgfpathlineto{\pgfqpoint{1.822083in}{1.123891in}}%
\pgfpathlineto{\pgfqpoint{1.824667in}{1.108768in}}%
\pgfpathlineto{\pgfqpoint{1.832417in}{1.188614in}}%
\pgfpathlineto{\pgfqpoint{1.835000in}{1.203997in}}%
\pgfpathlineto{\pgfqpoint{1.837583in}{1.212491in}}%
\pgfpathlineto{\pgfqpoint{1.840167in}{1.214074in}}%
\pgfpathlineto{\pgfqpoint{1.842750in}{1.209120in}}%
\pgfpathlineto{\pgfqpoint{1.845333in}{1.198291in}}%
\pgfpathlineto{\pgfqpoint{1.850500in}{1.162461in}}%
\pgfpathlineto{\pgfqpoint{1.855667in}{1.114031in}}%
\pgfpathlineto{\pgfqpoint{1.858250in}{1.112525in}}%
\pgfpathlineto{\pgfqpoint{1.868583in}{1.213491in}}%
\pgfpathlineto{\pgfqpoint{1.876333in}{1.266233in}}%
\pgfpathlineto{\pgfqpoint{1.881500in}{1.284989in}}%
\pgfpathlineto{\pgfqpoint{1.884083in}{1.288279in}}%
\pgfpathlineto{\pgfqpoint{1.886667in}{1.286874in}}%
\pgfpathlineto{\pgfqpoint{1.889250in}{1.280393in}}%
\pgfpathlineto{\pgfqpoint{1.894417in}{1.251824in}}%
\pgfpathlineto{\pgfqpoint{1.899583in}{1.204989in}}%
\pgfpathlineto{\pgfqpoint{1.907333in}{1.116782in}}%
\pgfpathlineto{\pgfqpoint{1.909917in}{1.112816in}}%
\pgfpathlineto{\pgfqpoint{1.917667in}{1.189874in}}%
\pgfpathlineto{\pgfqpoint{1.922833in}{1.226342in}}%
\pgfpathlineto{\pgfqpoint{1.928000in}{1.246947in}}%
\pgfpathlineto{\pgfqpoint{1.930583in}{1.250277in}}%
\pgfpathlineto{\pgfqpoint{1.933167in}{1.248661in}}%
\pgfpathlineto{\pgfqpoint{1.935750in}{1.242228in}}%
\pgfpathlineto{\pgfqpoint{1.940917in}{1.217164in}}%
\pgfpathlineto{\pgfqpoint{1.959000in}{1.101827in}}%
\pgfpathlineto{\pgfqpoint{1.964167in}{1.112618in}}%
\pgfpathlineto{\pgfqpoint{1.966750in}{1.112953in}}%
\pgfpathlineto{\pgfqpoint{1.969333in}{1.107372in}}%
\pgfpathlineto{\pgfqpoint{1.971917in}{1.104896in}}%
\pgfpathlineto{\pgfqpoint{1.977083in}{1.149921in}}%
\pgfpathlineto{\pgfqpoint{1.992583in}{1.337320in}}%
\pgfpathlineto{\pgfqpoint{1.995167in}{1.351380in}}%
\pgfpathlineto{\pgfqpoint{1.997750in}{1.355094in}}%
\pgfpathlineto{\pgfqpoint{2.000333in}{1.347346in}}%
\pgfpathlineto{\pgfqpoint{2.002917in}{1.327575in}}%
\pgfpathlineto{\pgfqpoint{2.008083in}{1.252898in}}%
\pgfpathlineto{\pgfqpoint{2.013250in}{1.139960in}}%
\pgfpathlineto{\pgfqpoint{2.015833in}{1.125061in}}%
\pgfpathlineto{\pgfqpoint{2.026167in}{1.363918in}}%
\pgfpathlineto{\pgfqpoint{2.031333in}{1.424783in}}%
\pgfpathlineto{\pgfqpoint{2.033917in}{1.433131in}}%
\pgfpathlineto{\pgfqpoint{2.036500in}{1.426268in}}%
\pgfpathlineto{\pgfqpoint{2.039083in}{1.404914in}}%
\pgfpathlineto{\pgfqpoint{2.044250in}{1.325005in}}%
\pgfpathlineto{\pgfqpoint{2.054583in}{1.111441in}}%
\pgfpathlineto{\pgfqpoint{2.059750in}{1.212369in}}%
\pgfpathlineto{\pgfqpoint{2.062333in}{1.245026in}}%
\pgfpathlineto{\pgfqpoint{2.064917in}{1.261578in}}%
\pgfpathlineto{\pgfqpoint{2.067500in}{1.260059in}}%
\pgfpathlineto{\pgfqpoint{2.070083in}{1.239833in}}%
\pgfpathlineto{\pgfqpoint{2.077833in}{1.117052in}}%
\pgfpathlineto{\pgfqpoint{2.088167in}{1.392738in}}%
\pgfpathlineto{\pgfqpoint{2.093333in}{1.467418in}}%
\pgfpathlineto{\pgfqpoint{2.095917in}{1.476303in}}%
\pgfpathlineto{\pgfqpoint{2.098500in}{1.464909in}}%
\pgfpathlineto{\pgfqpoint{2.101083in}{1.434237in}}%
\pgfpathlineto{\pgfqpoint{2.106250in}{1.326348in}}%
\pgfpathlineto{\pgfqpoint{2.114000in}{1.114515in}}%
\pgfpathlineto{\pgfqpoint{2.116583in}{1.150667in}}%
\pgfpathlineto{\pgfqpoint{2.121750in}{1.252561in}}%
\pgfpathlineto{\pgfqpoint{2.126917in}{1.308571in}}%
\pgfpathlineto{\pgfqpoint{2.129500in}{1.320180in}}%
\pgfpathlineto{\pgfqpoint{2.132083in}{1.322642in}}%
\pgfpathlineto{\pgfqpoint{2.134667in}{1.317619in}}%
\pgfpathlineto{\pgfqpoint{2.139833in}{1.291214in}}%
\pgfpathlineto{\pgfqpoint{2.147583in}{1.226967in}}%
\pgfpathlineto{\pgfqpoint{2.160500in}{1.108368in}}%
\pgfpathlineto{\pgfqpoint{2.163083in}{1.108579in}}%
\pgfpathlineto{\pgfqpoint{2.165667in}{1.120413in}}%
\pgfpathlineto{\pgfqpoint{2.168250in}{1.125710in}}%
\pgfpathlineto{\pgfqpoint{2.170833in}{1.123164in}}%
\pgfpathlineto{\pgfqpoint{2.173417in}{1.111875in}}%
\pgfpathlineto{\pgfqpoint{2.176000in}{1.108348in}}%
\pgfpathlineto{\pgfqpoint{2.181167in}{1.171475in}}%
\pgfpathlineto{\pgfqpoint{2.191500in}{1.311792in}}%
\pgfpathlineto{\pgfqpoint{2.194083in}{1.330171in}}%
\pgfpathlineto{\pgfqpoint{2.196667in}{1.336569in}}%
\pgfpathlineto{\pgfqpoint{2.199250in}{1.329939in}}%
\pgfpathlineto{\pgfqpoint{2.201833in}{1.310453in}}%
\pgfpathlineto{\pgfqpoint{2.207000in}{1.239783in}}%
\pgfpathlineto{\pgfqpoint{2.214750in}{1.104418in}}%
\pgfpathlineto{\pgfqpoint{2.222500in}{1.179501in}}%
\pgfpathlineto{\pgfqpoint{2.225083in}{1.186469in}}%
\pgfpathlineto{\pgfqpoint{2.227667in}{1.183316in}}%
\pgfpathlineto{\pgfqpoint{2.230250in}{1.171832in}}%
\pgfpathlineto{\pgfqpoint{2.240583in}{1.105814in}}%
\pgfpathlineto{\pgfqpoint{2.245750in}{1.131577in}}%
\pgfpathlineto{\pgfqpoint{2.248333in}{1.137012in}}%
\pgfpathlineto{\pgfqpoint{2.250917in}{1.137629in}}%
\pgfpathlineto{\pgfqpoint{2.253500in}{1.134120in}}%
\pgfpathlineto{\pgfqpoint{2.258667in}{1.118341in}}%
\pgfpathlineto{\pgfqpoint{2.261250in}{1.107697in}}%
\pgfpathlineto{\pgfqpoint{2.263833in}{1.104110in}}%
\pgfpathlineto{\pgfqpoint{2.271583in}{1.147140in}}%
\pgfpathlineto{\pgfqpoint{2.276750in}{1.186415in}}%
\pgfpathlineto{\pgfqpoint{2.292250in}{1.328103in}}%
\pgfpathlineto{\pgfqpoint{2.294833in}{1.336606in}}%
\pgfpathlineto{\pgfqpoint{2.297417in}{1.335535in}}%
\pgfpathlineto{\pgfqpoint{2.300000in}{1.324402in}}%
\pgfpathlineto{\pgfqpoint{2.305167in}{1.275279in}}%
\pgfpathlineto{\pgfqpoint{2.318083in}{1.104198in}}%
\pgfpathlineto{\pgfqpoint{2.325833in}{1.182371in}}%
\pgfpathlineto{\pgfqpoint{2.331000in}{1.212127in}}%
\pgfpathlineto{\pgfqpoint{2.333583in}{1.218920in}}%
\pgfpathlineto{\pgfqpoint{2.336167in}{1.219589in}}%
\pgfpathlineto{\pgfqpoint{2.338750in}{1.213792in}}%
\pgfpathlineto{\pgfqpoint{2.341333in}{1.201517in}}%
\pgfpathlineto{\pgfqpoint{2.346500in}{1.159444in}}%
\pgfpathlineto{\pgfqpoint{2.351667in}{1.101021in}}%
\pgfpathlineto{\pgfqpoint{2.362000in}{1.210813in}}%
\pgfpathlineto{\pgfqpoint{2.364583in}{1.226190in}}%
\pgfpathlineto{\pgfqpoint{2.367167in}{1.233004in}}%
\pgfpathlineto{\pgfqpoint{2.369750in}{1.230202in}}%
\pgfpathlineto{\pgfqpoint{2.372333in}{1.217435in}}%
\pgfpathlineto{\pgfqpoint{2.377500in}{1.164697in}}%
\pgfpathlineto{\pgfqpoint{2.380083in}{1.128007in}}%
\pgfpathlineto{\pgfqpoint{2.382667in}{1.112437in}}%
\pgfpathlineto{\pgfqpoint{2.393000in}{1.262459in}}%
\pgfpathlineto{\pgfqpoint{2.398167in}{1.307696in}}%
\pgfpathlineto{\pgfqpoint{2.400750in}{1.320980in}}%
\pgfpathlineto{\pgfqpoint{2.403333in}{1.328212in}}%
\pgfpathlineto{\pgfqpoint{2.405917in}{1.329459in}}%
\pgfpathlineto{\pgfqpoint{2.408500in}{1.324541in}}%
\pgfpathlineto{\pgfqpoint{2.411083in}{1.312972in}}%
\pgfpathlineto{\pgfqpoint{2.416250in}{1.267050in}}%
\pgfpathlineto{\pgfqpoint{2.421417in}{1.187953in}}%
\pgfpathlineto{\pgfqpoint{2.424000in}{1.137401in}}%
\pgfpathlineto{\pgfqpoint{2.426583in}{1.117652in}}%
\pgfpathlineto{\pgfqpoint{2.434333in}{1.271032in}}%
\pgfpathlineto{\pgfqpoint{2.436917in}{1.303207in}}%
\pgfpathlineto{\pgfqpoint{2.439500in}{1.319289in}}%
\pgfpathlineto{\pgfqpoint{2.442083in}{1.317008in}}%
\pgfpathlineto{\pgfqpoint{2.444667in}{1.295789in}}%
\pgfpathlineto{\pgfqpoint{2.449833in}{1.203377in}}%
\pgfpathlineto{\pgfqpoint{2.452417in}{1.139702in}}%
\pgfpathlineto{\pgfqpoint{2.455000in}{1.128661in}}%
\pgfpathlineto{\pgfqpoint{2.462750in}{1.305219in}}%
\pgfpathlineto{\pgfqpoint{2.465333in}{1.338988in}}%
\pgfpathlineto{\pgfqpoint{2.467917in}{1.355520in}}%
\pgfpathlineto{\pgfqpoint{2.470500in}{1.354204in}}%
\pgfpathlineto{\pgfqpoint{2.473083in}{1.335800in}}%
\pgfpathlineto{\pgfqpoint{2.478250in}{1.256073in}}%
\pgfpathlineto{\pgfqpoint{2.483417in}{1.139033in}}%
\pgfpathlineto{\pgfqpoint{2.486000in}{1.125426in}}%
\pgfpathlineto{\pgfqpoint{2.496333in}{1.360508in}}%
\pgfpathlineto{\pgfqpoint{2.501500in}{1.435577in}}%
\pgfpathlineto{\pgfqpoint{2.504083in}{1.455706in}}%
\pgfpathlineto{\pgfqpoint{2.506667in}{1.461601in}}%
\pgfpathlineto{\pgfqpoint{2.509250in}{1.451624in}}%
\pgfpathlineto{\pgfqpoint{2.511833in}{1.424858in}}%
\pgfpathlineto{\pgfqpoint{2.517000in}{1.323384in}}%
\pgfpathlineto{\pgfqpoint{2.524750in}{1.100384in}}%
\pgfpathlineto{\pgfqpoint{2.529917in}{1.232190in}}%
\pgfpathlineto{\pgfqpoint{2.535083in}{1.304566in}}%
\pgfpathlineto{\pgfqpoint{2.537667in}{1.312288in}}%
\pgfpathlineto{\pgfqpoint{2.540250in}{1.301434in}}%
\pgfpathlineto{\pgfqpoint{2.542833in}{1.274260in}}%
\pgfpathlineto{\pgfqpoint{2.548000in}{1.185508in}}%
\pgfpathlineto{\pgfqpoint{2.550583in}{1.132846in}}%
\pgfpathlineto{\pgfqpoint{2.553167in}{1.119125in}}%
\pgfpathlineto{\pgfqpoint{2.558333in}{1.204937in}}%
\pgfpathlineto{\pgfqpoint{2.563500in}{1.249621in}}%
\pgfpathlineto{\pgfqpoint{2.566083in}{1.254609in}}%
\pgfpathlineto{\pgfqpoint{2.568667in}{1.249114in}}%
\pgfpathlineto{\pgfqpoint{2.571250in}{1.234784in}}%
\pgfpathlineto{\pgfqpoint{2.576417in}{1.186989in}}%
\pgfpathlineto{\pgfqpoint{2.581583in}{1.123602in}}%
\pgfpathlineto{\pgfqpoint{2.584167in}{1.111624in}}%
\pgfpathlineto{\pgfqpoint{2.599667in}{1.322732in}}%
\pgfpathlineto{\pgfqpoint{2.604833in}{1.358944in}}%
\pgfpathlineto{\pgfqpoint{2.607417in}{1.364050in}}%
\pgfpathlineto{\pgfqpoint{2.610000in}{1.359944in}}%
\pgfpathlineto{\pgfqpoint{2.612583in}{1.347152in}}%
\pgfpathlineto{\pgfqpoint{2.617750in}{1.301112in}}%
\pgfpathlineto{\pgfqpoint{2.630667in}{1.163948in}}%
\pgfpathlineto{\pgfqpoint{2.635833in}{1.135044in}}%
\pgfpathlineto{\pgfqpoint{2.638417in}{1.128306in}}%
\pgfpathlineto{\pgfqpoint{2.641000in}{1.125978in}}%
\pgfpathlineto{\pgfqpoint{2.643583in}{1.127021in}}%
\pgfpathlineto{\pgfqpoint{2.653917in}{1.138333in}}%
\pgfpathlineto{\pgfqpoint{2.656500in}{1.137554in}}%
\pgfpathlineto{\pgfqpoint{2.659083in}{1.134583in}}%
\pgfpathlineto{\pgfqpoint{2.664250in}{1.123818in}}%
\pgfpathlineto{\pgfqpoint{2.672000in}{1.102689in}}%
\pgfpathlineto{\pgfqpoint{2.674583in}{1.105371in}}%
\pgfpathlineto{\pgfqpoint{2.679750in}{1.124939in}}%
\pgfpathlineto{\pgfqpoint{2.690083in}{1.177794in}}%
\pgfpathlineto{\pgfqpoint{2.695250in}{1.200955in}}%
\pgfpathlineto{\pgfqpoint{2.700417in}{1.213267in}}%
\pgfpathlineto{\pgfqpoint{2.703000in}{1.214640in}}%
\pgfpathlineto{\pgfqpoint{2.705583in}{1.213046in}}%
\pgfpathlineto{\pgfqpoint{2.708167in}{1.208850in}}%
\pgfpathlineto{\pgfqpoint{2.713333in}{1.194315in}}%
\pgfpathlineto{\pgfqpoint{2.723667in}{1.159659in}}%
\pgfpathlineto{\pgfqpoint{2.726250in}{1.157348in}}%
\pgfpathlineto{\pgfqpoint{2.728833in}{1.161236in}}%
\pgfpathlineto{\pgfqpoint{2.731417in}{1.172937in}}%
\pgfpathlineto{\pgfqpoint{2.736583in}{1.222634in}}%
\pgfpathlineto{\pgfqpoint{2.744333in}{1.347050in}}%
\pgfpathlineto{\pgfqpoint{2.749500in}{1.429163in}}%
\pgfpathlineto{\pgfqpoint{2.754667in}{1.475127in}}%
\pgfpathlineto{\pgfqpoint{2.757250in}{1.477127in}}%
\pgfpathlineto{\pgfqpoint{2.759833in}{1.463421in}}%
\pgfpathlineto{\pgfqpoint{2.762417in}{1.434512in}}%
\pgfpathlineto{\pgfqpoint{2.767583in}{1.340243in}}%
\pgfpathlineto{\pgfqpoint{2.777917in}{1.121466in}}%
\pgfpathlineto{\pgfqpoint{2.780500in}{1.114339in}}%
\pgfpathlineto{\pgfqpoint{2.783083in}{1.137226in}}%
\pgfpathlineto{\pgfqpoint{2.785667in}{1.147177in}}%
\pgfpathlineto{\pgfqpoint{2.788250in}{1.145856in}}%
\pgfpathlineto{\pgfqpoint{2.793417in}{1.122618in}}%
\pgfpathlineto{\pgfqpoint{2.798583in}{1.100530in}}%
\pgfpathlineto{\pgfqpoint{2.803750in}{1.104037in}}%
\pgfpathlineto{\pgfqpoint{2.806333in}{1.120722in}}%
\pgfpathlineto{\pgfqpoint{2.811500in}{1.179832in}}%
\pgfpathlineto{\pgfqpoint{2.824417in}{1.365459in}}%
\pgfpathlineto{\pgfqpoint{2.829583in}{1.411540in}}%
\pgfpathlineto{\pgfqpoint{2.834750in}{1.438533in}}%
\pgfpathlineto{\pgfqpoint{2.839917in}{1.454757in}}%
\pgfpathlineto{\pgfqpoint{2.845083in}{1.464604in}}%
\pgfpathlineto{\pgfqpoint{2.847667in}{1.465780in}}%
\pgfpathlineto{\pgfqpoint{2.850250in}{1.462578in}}%
\pgfpathlineto{\pgfqpoint{2.852833in}{1.453398in}}%
\pgfpathlineto{\pgfqpoint{2.855417in}{1.437029in}}%
\pgfpathlineto{\pgfqpoint{2.860583in}{1.382225in}}%
\pgfpathlineto{\pgfqpoint{2.873500in}{1.212465in}}%
\pgfpathlineto{\pgfqpoint{2.876083in}{1.196969in}}%
\pgfpathlineto{\pgfqpoint{2.878667in}{1.192475in}}%
\pgfpathlineto{\pgfqpoint{2.881250in}{1.198721in}}%
\pgfpathlineto{\pgfqpoint{2.886417in}{1.235048in}}%
\pgfpathlineto{\pgfqpoint{2.891583in}{1.277993in}}%
\pgfpathlineto{\pgfqpoint{2.894167in}{1.290863in}}%
\pgfpathlineto{\pgfqpoint{2.896750in}{1.292825in}}%
\pgfpathlineto{\pgfqpoint{2.899333in}{1.281334in}}%
\pgfpathlineto{\pgfqpoint{2.901917in}{1.255279in}}%
\pgfpathlineto{\pgfqpoint{2.907083in}{1.162729in}}%
\pgfpathlineto{\pgfqpoint{2.909667in}{1.101362in}}%
\pgfpathlineto{\pgfqpoint{2.920000in}{1.347948in}}%
\pgfpathlineto{\pgfqpoint{2.925167in}{1.422218in}}%
\pgfpathlineto{\pgfqpoint{2.927750in}{1.440863in}}%
\pgfpathlineto{\pgfqpoint{2.930333in}{1.448489in}}%
\pgfpathlineto{\pgfqpoint{2.932917in}{1.447332in}}%
\pgfpathlineto{\pgfqpoint{2.935500in}{1.440105in}}%
\pgfpathlineto{\pgfqpoint{2.948417in}{1.390372in}}%
\pgfpathlineto{\pgfqpoint{2.953583in}{1.379966in}}%
\pgfpathlineto{\pgfqpoint{2.982000in}{1.334601in}}%
\pgfpathlineto{\pgfqpoint{2.987167in}{1.319475in}}%
\pgfpathlineto{\pgfqpoint{2.992333in}{1.288751in}}%
\pgfpathlineto{\pgfqpoint{2.997500in}{1.235556in}}%
\pgfpathlineto{\pgfqpoint{3.005250in}{1.120966in}}%
\pgfpathlineto{\pgfqpoint{3.007833in}{1.119190in}}%
\pgfpathlineto{\pgfqpoint{3.013000in}{1.186887in}}%
\pgfpathlineto{\pgfqpoint{3.018167in}{1.225058in}}%
\pgfpathlineto{\pgfqpoint{3.020750in}{1.231819in}}%
\pgfpathlineto{\pgfqpoint{3.023333in}{1.231767in}}%
\pgfpathlineto{\pgfqpoint{3.025917in}{1.226926in}}%
\pgfpathlineto{\pgfqpoint{3.033667in}{1.207345in}}%
\pgfpathlineto{\pgfqpoint{3.036250in}{1.206160in}}%
\pgfpathlineto{\pgfqpoint{3.038833in}{1.210129in}}%
\pgfpathlineto{\pgfqpoint{3.041417in}{1.219939in}}%
\pgfpathlineto{\pgfqpoint{3.046583in}{1.256661in}}%
\pgfpathlineto{\pgfqpoint{3.059500in}{1.378148in}}%
\pgfpathlineto{\pgfqpoint{3.062083in}{1.390021in}}%
\pgfpathlineto{\pgfqpoint{3.064667in}{1.393964in}}%
\pgfpathlineto{\pgfqpoint{3.067250in}{1.389871in}}%
\pgfpathlineto{\pgfqpoint{3.069833in}{1.378528in}}%
\pgfpathlineto{\pgfqpoint{3.075000in}{1.340663in}}%
\pgfpathlineto{\pgfqpoint{3.085333in}{1.260634in}}%
\pgfpathlineto{\pgfqpoint{3.090500in}{1.241787in}}%
\pgfpathlineto{\pgfqpoint{3.093083in}{1.239624in}}%
\pgfpathlineto{\pgfqpoint{3.095667in}{1.242054in}}%
\pgfpathlineto{\pgfqpoint{3.098250in}{1.248449in}}%
\pgfpathlineto{\pgfqpoint{3.111167in}{1.295225in}}%
\pgfpathlineto{\pgfqpoint{3.113750in}{1.296621in}}%
\pgfpathlineto{\pgfqpoint{3.116333in}{1.292668in}}%
\pgfpathlineto{\pgfqpoint{3.118917in}{1.283248in}}%
\pgfpathlineto{\pgfqpoint{3.124083in}{1.249956in}}%
\pgfpathlineto{\pgfqpoint{3.131833in}{1.173876in}}%
\pgfpathlineto{\pgfqpoint{3.137000in}{1.108153in}}%
\pgfpathlineto{\pgfqpoint{3.139583in}{1.130320in}}%
\pgfpathlineto{\pgfqpoint{3.147333in}{1.272402in}}%
\pgfpathlineto{\pgfqpoint{3.160250in}{1.535008in}}%
\pgfpathlineto{\pgfqpoint{3.165417in}{1.591719in}}%
\pgfpathlineto{\pgfqpoint{3.168000in}{1.600092in}}%
\pgfpathlineto{\pgfqpoint{3.170583in}{1.594323in}}%
\pgfpathlineto{\pgfqpoint{3.173167in}{1.575388in}}%
\pgfpathlineto{\pgfqpoint{3.178333in}{1.508145in}}%
\pgfpathlineto{\pgfqpoint{3.186083in}{1.392222in}}%
\pgfpathlineto{\pgfqpoint{3.191250in}{1.346661in}}%
\pgfpathlineto{\pgfqpoint{3.193833in}{1.337087in}}%
\pgfpathlineto{\pgfqpoint{3.196417in}{1.333857in}}%
\pgfpathlineto{\pgfqpoint{3.199000in}{1.333271in}}%
\pgfpathlineto{\pgfqpoint{3.201583in}{1.330900in}}%
\pgfpathlineto{\pgfqpoint{3.204167in}{1.322389in}}%
\pgfpathlineto{\pgfqpoint{3.206750in}{1.304213in}}%
\pgfpathlineto{\pgfqpoint{3.211917in}{1.232186in}}%
\pgfpathlineto{\pgfqpoint{3.217083in}{1.118856in}}%
\pgfpathlineto{\pgfqpoint{3.219667in}{1.145453in}}%
\pgfpathlineto{\pgfqpoint{3.227417in}{1.319005in}}%
\pgfpathlineto{\pgfqpoint{3.232583in}{1.390314in}}%
\pgfpathlineto{\pgfqpoint{3.237750in}{1.423286in}}%
\pgfpathlineto{\pgfqpoint{3.240333in}{1.429956in}}%
\pgfpathlineto{\pgfqpoint{3.245500in}{1.435150in}}%
\pgfpathlineto{\pgfqpoint{3.255833in}{1.442484in}}%
\pgfpathlineto{\pgfqpoint{3.258417in}{1.441400in}}%
\pgfpathlineto{\pgfqpoint{3.261000in}{1.436893in}}%
\pgfpathlineto{\pgfqpoint{3.263583in}{1.428224in}}%
\pgfpathlineto{\pgfqpoint{3.268750in}{1.399082in}}%
\pgfpathlineto{\pgfqpoint{3.276500in}{1.348052in}}%
\pgfpathlineto{\pgfqpoint{3.279083in}{1.337334in}}%
\pgfpathlineto{\pgfqpoint{3.281667in}{1.332120in}}%
\pgfpathlineto{\pgfqpoint{3.284250in}{1.332278in}}%
\pgfpathlineto{\pgfqpoint{3.292000in}{1.345458in}}%
\pgfpathlineto{\pgfqpoint{3.294583in}{1.343225in}}%
\pgfpathlineto{\pgfqpoint{3.297167in}{1.331659in}}%
\pgfpathlineto{\pgfqpoint{3.299750in}{1.307949in}}%
\pgfpathlineto{\pgfqpoint{3.304917in}{1.219358in}}%
\pgfpathlineto{\pgfqpoint{3.310083in}{1.116093in}}%
\pgfpathlineto{\pgfqpoint{3.320417in}{1.401355in}}%
\pgfpathlineto{\pgfqpoint{3.325583in}{1.480545in}}%
\pgfpathlineto{\pgfqpoint{3.328167in}{1.495677in}}%
\pgfpathlineto{\pgfqpoint{3.330750in}{1.494908in}}%
\pgfpathlineto{\pgfqpoint{3.333333in}{1.480184in}}%
\pgfpathlineto{\pgfqpoint{3.338500in}{1.421219in}}%
\pgfpathlineto{\pgfqpoint{3.348833in}{1.285360in}}%
\pgfpathlineto{\pgfqpoint{3.354000in}{1.253144in}}%
\pgfpathlineto{\pgfqpoint{3.356583in}{1.250825in}}%
\pgfpathlineto{\pgfqpoint{3.359167in}{1.257928in}}%
\pgfpathlineto{\pgfqpoint{3.361750in}{1.273966in}}%
\pgfpathlineto{\pgfqpoint{3.366917in}{1.328092in}}%
\pgfpathlineto{\pgfqpoint{3.377250in}{1.457861in}}%
\pgfpathlineto{\pgfqpoint{3.379833in}{1.476097in}}%
\pgfpathlineto{\pgfqpoint{3.382417in}{1.482386in}}%
\pgfpathlineto{\pgfqpoint{3.385000in}{1.474638in}}%
\pgfpathlineto{\pgfqpoint{3.387583in}{1.451742in}}%
\pgfpathlineto{\pgfqpoint{3.392750in}{1.361350in}}%
\pgfpathlineto{\pgfqpoint{3.400500in}{1.142236in}}%
\pgfpathlineto{\pgfqpoint{3.403083in}{1.140554in}}%
\pgfpathlineto{\pgfqpoint{3.410833in}{1.364292in}}%
\pgfpathlineto{\pgfqpoint{3.416000in}{1.461084in}}%
\pgfpathlineto{\pgfqpoint{3.418583in}{1.488599in}}%
\pgfpathlineto{\pgfqpoint{3.421167in}{1.502245in}}%
\pgfpathlineto{\pgfqpoint{3.423750in}{1.503155in}}%
\pgfpathlineto{\pgfqpoint{3.426333in}{1.493068in}}%
\pgfpathlineto{\pgfqpoint{3.431500in}{1.448325in}}%
\pgfpathlineto{\pgfqpoint{3.441833in}{1.315535in}}%
\pgfpathlineto{\pgfqpoint{3.449583in}{1.221526in}}%
\pgfpathlineto{\pgfqpoint{3.454750in}{1.178108in}}%
\pgfpathlineto{\pgfqpoint{3.457333in}{1.165771in}}%
\pgfpathlineto{\pgfqpoint{3.459917in}{1.160824in}}%
\pgfpathlineto{\pgfqpoint{3.462500in}{1.163367in}}%
\pgfpathlineto{\pgfqpoint{3.465083in}{1.172652in}}%
\pgfpathlineto{\pgfqpoint{3.478000in}{1.245992in}}%
\pgfpathlineto{\pgfqpoint{3.480583in}{1.248687in}}%
\pgfpathlineto{\pgfqpoint{3.483167in}{1.243420in}}%
\pgfpathlineto{\pgfqpoint{3.485750in}{1.230102in}}%
\pgfpathlineto{\pgfqpoint{3.490917in}{1.183853in}}%
\pgfpathlineto{\pgfqpoint{3.498667in}{1.102049in}}%
\pgfpathlineto{\pgfqpoint{3.503833in}{1.128128in}}%
\pgfpathlineto{\pgfqpoint{3.506417in}{1.130489in}}%
\pgfpathlineto{\pgfqpoint{3.509000in}{1.124310in}}%
\pgfpathlineto{\pgfqpoint{3.511583in}{1.110976in}}%
\pgfpathlineto{\pgfqpoint{3.514167in}{1.107123in}}%
\pgfpathlineto{\pgfqpoint{3.519333in}{1.144671in}}%
\pgfpathlineto{\pgfqpoint{3.521917in}{1.157054in}}%
\pgfpathlineto{\pgfqpoint{3.524500in}{1.160902in}}%
\pgfpathlineto{\pgfqpoint{3.527083in}{1.153904in}}%
\pgfpathlineto{\pgfqpoint{3.529667in}{1.134800in}}%
\pgfpathlineto{\pgfqpoint{3.532250in}{1.103507in}}%
\pgfpathlineto{\pgfqpoint{3.534833in}{1.138868in}}%
\pgfpathlineto{\pgfqpoint{3.552917in}{1.484653in}}%
\pgfpathlineto{\pgfqpoint{3.555500in}{1.497085in}}%
\pgfpathlineto{\pgfqpoint{3.558083in}{1.492112in}}%
\pgfpathlineto{\pgfqpoint{3.560667in}{1.469178in}}%
\pgfpathlineto{\pgfqpoint{3.565833in}{1.372776in}}%
\pgfpathlineto{\pgfqpoint{3.573583in}{1.140729in}}%
\pgfpathlineto{\pgfqpoint{3.576167in}{1.144129in}}%
\pgfpathlineto{\pgfqpoint{3.583917in}{1.353316in}}%
\pgfpathlineto{\pgfqpoint{3.586500in}{1.392473in}}%
\pgfpathlineto{\pgfqpoint{3.589083in}{1.410474in}}%
\pgfpathlineto{\pgfqpoint{3.591667in}{1.405728in}}%
\pgfpathlineto{\pgfqpoint{3.594250in}{1.378338in}}%
\pgfpathlineto{\pgfqpoint{3.599417in}{1.264579in}}%
\pgfpathlineto{\pgfqpoint{3.604583in}{1.100828in}}%
\pgfpathlineto{\pgfqpoint{3.612333in}{1.340739in}}%
\pgfpathlineto{\pgfqpoint{3.617500in}{1.438293in}}%
\pgfpathlineto{\pgfqpoint{3.620083in}{1.457268in}}%
\pgfpathlineto{\pgfqpoint{3.622667in}{1.454353in}}%
\pgfpathlineto{\pgfqpoint{3.625250in}{1.429683in}}%
\pgfpathlineto{\pgfqpoint{3.630417in}{1.321936in}}%
\pgfpathlineto{\pgfqpoint{3.635583in}{1.158220in}}%
\pgfpathlineto{\pgfqpoint{3.638167in}{1.133718in}}%
\pgfpathlineto{\pgfqpoint{3.645917in}{1.392684in}}%
\pgfpathlineto{\pgfqpoint{3.651083in}{1.510610in}}%
\pgfpathlineto{\pgfqpoint{3.653667in}{1.544511in}}%
\pgfpathlineto{\pgfqpoint{3.656250in}{1.559578in}}%
\pgfpathlineto{\pgfqpoint{3.658833in}{1.554994in}}%
\pgfpathlineto{\pgfqpoint{3.661417in}{1.530471in}}%
\pgfpathlineto{\pgfqpoint{3.666583in}{1.423358in}}%
\pgfpathlineto{\pgfqpoint{3.671750in}{1.249400in}}%
\pgfpathlineto{\pgfqpoint{3.674333in}{1.144943in}}%
\pgfpathlineto{\pgfqpoint{3.676917in}{1.165087in}}%
\pgfpathlineto{\pgfqpoint{3.684667in}{1.471256in}}%
\pgfpathlineto{\pgfqpoint{3.689833in}{1.599279in}}%
\pgfpathlineto{\pgfqpoint{3.692417in}{1.627970in}}%
\pgfpathlineto{\pgfqpoint{3.695000in}{1.630920in}}%
\pgfpathlineto{\pgfqpoint{3.697583in}{1.608642in}}%
\pgfpathlineto{\pgfqpoint{3.700167in}{1.563285in}}%
\pgfpathlineto{\pgfqpoint{3.705333in}{1.418508in}}%
\pgfpathlineto{\pgfqpoint{3.713083in}{1.141906in}}%
\pgfpathlineto{\pgfqpoint{3.715667in}{1.145199in}}%
\pgfpathlineto{\pgfqpoint{3.720833in}{1.286113in}}%
\pgfpathlineto{\pgfqpoint{3.726000in}{1.366379in}}%
\pgfpathlineto{\pgfqpoint{3.728583in}{1.381587in}}%
\pgfpathlineto{\pgfqpoint{3.731167in}{1.380890in}}%
\pgfpathlineto{\pgfqpoint{3.733750in}{1.365718in}}%
\pgfpathlineto{\pgfqpoint{3.738917in}{1.300109in}}%
\pgfpathlineto{\pgfqpoint{3.749250in}{1.102172in}}%
\pgfpathlineto{\pgfqpoint{3.754417in}{1.178950in}}%
\pgfpathlineto{\pgfqpoint{3.757000in}{1.202904in}}%
\pgfpathlineto{\pgfqpoint{3.759583in}{1.211891in}}%
\pgfpathlineto{\pgfqpoint{3.762167in}{1.204108in}}%
\pgfpathlineto{\pgfqpoint{3.764750in}{1.179043in}}%
\pgfpathlineto{\pgfqpoint{3.767333in}{1.137660in}}%
\pgfpathlineto{\pgfqpoint{3.769917in}{1.117628in}}%
\pgfpathlineto{\pgfqpoint{3.785417in}{1.508603in}}%
\pgfpathlineto{\pgfqpoint{3.790583in}{1.570733in}}%
\pgfpathlineto{\pgfqpoint{3.793167in}{1.580063in}}%
\pgfpathlineto{\pgfqpoint{3.795750in}{1.574705in}}%
\pgfpathlineto{\pgfqpoint{3.798333in}{1.554945in}}%
\pgfpathlineto{\pgfqpoint{3.803500in}{1.473803in}}%
\pgfpathlineto{\pgfqpoint{3.808667in}{1.340130in}}%
\pgfpathlineto{\pgfqpoint{3.813833in}{1.161063in}}%
\pgfpathlineto{\pgfqpoint{3.816417in}{1.140411in}}%
\pgfpathlineto{\pgfqpoint{3.826750in}{1.527953in}}%
\pgfpathlineto{\pgfqpoint{3.831917in}{1.634984in}}%
\pgfpathlineto{\pgfqpoint{3.834500in}{1.650940in}}%
\pgfpathlineto{\pgfqpoint{3.837083in}{1.639608in}}%
\pgfpathlineto{\pgfqpoint{3.839667in}{1.601866in}}%
\pgfpathlineto{\pgfqpoint{3.844833in}{1.460467in}}%
\pgfpathlineto{\pgfqpoint{3.855167in}{1.125012in}}%
\pgfpathlineto{\pgfqpoint{3.860333in}{1.271496in}}%
\pgfpathlineto{\pgfqpoint{3.862917in}{1.317424in}}%
\pgfpathlineto{\pgfqpoint{3.865500in}{1.342287in}}%
\pgfpathlineto{\pgfqpoint{3.868083in}{1.345476in}}%
\pgfpathlineto{\pgfqpoint{3.870667in}{1.327465in}}%
\pgfpathlineto{\pgfqpoint{3.873250in}{1.289629in}}%
\pgfpathlineto{\pgfqpoint{3.881000in}{1.119513in}}%
\pgfpathlineto{\pgfqpoint{3.896500in}{1.661979in}}%
\pgfpathlineto{\pgfqpoint{3.901667in}{1.754117in}}%
\pgfpathlineto{\pgfqpoint{3.904250in}{1.764120in}}%
\pgfpathlineto{\pgfqpoint{3.906833in}{1.746645in}}%
\pgfpathlineto{\pgfqpoint{3.909417in}{1.700977in}}%
\pgfpathlineto{\pgfqpoint{3.914583in}{1.531002in}}%
\pgfpathlineto{\pgfqpoint{3.922333in}{1.144089in}}%
\pgfpathlineto{\pgfqpoint{3.924917in}{1.195119in}}%
\pgfpathlineto{\pgfqpoint{3.932667in}{1.550402in}}%
\pgfpathlineto{\pgfqpoint{3.937833in}{1.687145in}}%
\pgfpathlineto{\pgfqpoint{3.940417in}{1.718578in}}%
\pgfpathlineto{\pgfqpoint{3.943000in}{1.725991in}}%
\pgfpathlineto{\pgfqpoint{3.945583in}{1.711561in}}%
\pgfpathlineto{\pgfqpoint{3.948167in}{1.678484in}}%
\pgfpathlineto{\pgfqpoint{3.953333in}{1.572057in}}%
\pgfpathlineto{\pgfqpoint{3.968833in}{1.198872in}}%
\pgfpathlineto{\pgfqpoint{3.974000in}{1.126918in}}%
\pgfpathlineto{\pgfqpoint{3.976583in}{1.107452in}}%
\pgfpathlineto{\pgfqpoint{3.979167in}{1.100349in}}%
\pgfpathlineto{\pgfqpoint{3.981750in}{1.103370in}}%
\pgfpathlineto{\pgfqpoint{3.984333in}{1.117840in}}%
\pgfpathlineto{\pgfqpoint{3.989500in}{1.172457in}}%
\pgfpathlineto{\pgfqpoint{3.999833in}{1.305218in}}%
\pgfpathlineto{\pgfqpoint{4.002417in}{1.323730in}}%
\pgfpathlineto{\pgfqpoint{4.005000in}{1.330432in}}%
\pgfpathlineto{\pgfqpoint{4.007583in}{1.323520in}}%
\pgfpathlineto{\pgfqpoint{4.010167in}{1.302198in}}%
\pgfpathlineto{\pgfqpoint{4.015333in}{1.218770in}}%
\pgfpathlineto{\pgfqpoint{4.020500in}{1.104136in}}%
\pgfpathlineto{\pgfqpoint{4.030833in}{1.353227in}}%
\pgfpathlineto{\pgfqpoint{4.036000in}{1.428695in}}%
\pgfpathlineto{\pgfqpoint{4.038583in}{1.448784in}}%
\pgfpathlineto{\pgfqpoint{4.041167in}{1.456915in}}%
\pgfpathlineto{\pgfqpoint{4.043750in}{1.453315in}}%
\pgfpathlineto{\pgfqpoint{4.046333in}{1.438264in}}%
\pgfpathlineto{\pgfqpoint{4.051500in}{1.375005in}}%
\pgfpathlineto{\pgfqpoint{4.056667in}{1.271450in}}%
\pgfpathlineto{\pgfqpoint{4.061833in}{1.139476in}}%
\pgfpathlineto{\pgfqpoint{4.064417in}{1.130223in}}%
\pgfpathlineto{\pgfqpoint{4.072167in}{1.305410in}}%
\pgfpathlineto{\pgfqpoint{4.074750in}{1.337782in}}%
\pgfpathlineto{\pgfqpoint{4.077333in}{1.350661in}}%
\pgfpathlineto{\pgfqpoint{4.079917in}{1.341438in}}%
\pgfpathlineto{\pgfqpoint{4.082500in}{1.308803in}}%
\pgfpathlineto{\pgfqpoint{4.090250in}{1.119183in}}%
\pgfpathlineto{\pgfqpoint{4.105750in}{1.738328in}}%
\pgfpathlineto{\pgfqpoint{4.108333in}{1.792348in}}%
\pgfpathlineto{\pgfqpoint{4.110917in}{1.820292in}}%
\pgfpathlineto{\pgfqpoint{4.113500in}{1.820542in}}%
\pgfpathlineto{\pgfqpoint{4.116083in}{1.792869in}}%
\pgfpathlineto{\pgfqpoint{4.118667in}{1.738310in}}%
\pgfpathlineto{\pgfqpoint{4.123833in}{1.558195in}}%
\pgfpathlineto{\pgfqpoint{4.131583in}{1.170744in}}%
\pgfpathlineto{\pgfqpoint{4.134167in}{1.168738in}}%
\pgfpathlineto{\pgfqpoint{4.141917in}{1.538216in}}%
\pgfpathlineto{\pgfqpoint{4.147083in}{1.698344in}}%
\pgfpathlineto{\pgfqpoint{4.149667in}{1.744584in}}%
\pgfpathlineto{\pgfqpoint{4.152250in}{1.767702in}}%
\pgfpathlineto{\pgfqpoint{4.154833in}{1.768821in}}%
\pgfpathlineto{\pgfqpoint{4.157417in}{1.750155in}}%
\pgfpathlineto{\pgfqpoint{4.162583in}{1.666633in}}%
\pgfpathlineto{\pgfqpoint{4.178083in}{1.349306in}}%
\pgfpathlineto{\pgfqpoint{4.180667in}{1.325507in}}%
\pgfpathlineto{\pgfqpoint{4.183250in}{1.315186in}}%
\pgfpathlineto{\pgfqpoint{4.185833in}{1.318056in}}%
\pgfpathlineto{\pgfqpoint{4.188417in}{1.332767in}}%
\pgfpathlineto{\pgfqpoint{4.193583in}{1.387376in}}%
\pgfpathlineto{\pgfqpoint{4.201333in}{1.475225in}}%
\pgfpathlineto{\pgfqpoint{4.203917in}{1.489241in}}%
\pgfpathlineto{\pgfqpoint{4.206500in}{1.489359in}}%
\pgfpathlineto{\pgfqpoint{4.209083in}{1.472809in}}%
\pgfpathlineto{\pgfqpoint{4.211667in}{1.437714in}}%
\pgfpathlineto{\pgfqpoint{4.216833in}{1.309514in}}%
\pgfpathlineto{\pgfqpoint{4.222000in}{1.111125in}}%
\pgfpathlineto{\pgfqpoint{4.227167in}{1.333996in}}%
\pgfpathlineto{\pgfqpoint{4.237500in}{1.812279in}}%
\pgfpathlineto{\pgfqpoint{4.242667in}{1.962914in}}%
\pgfpathlineto{\pgfqpoint{4.245250in}{2.001987in}}%
\pgfpathlineto{\pgfqpoint{4.247833in}{2.014443in}}%
\pgfpathlineto{\pgfqpoint{4.250417in}{2.000244in}}%
\pgfpathlineto{\pgfqpoint{4.253000in}{1.960809in}}%
\pgfpathlineto{\pgfqpoint{4.258167in}{1.818113in}}%
\pgfpathlineto{\pgfqpoint{4.268500in}{1.397342in}}%
\pgfpathlineto{\pgfqpoint{4.276250in}{1.105696in}}%
\pgfpathlineto{\pgfqpoint{4.281417in}{1.259127in}}%
\pgfpathlineto{\pgfqpoint{4.286583in}{1.360385in}}%
\pgfpathlineto{\pgfqpoint{4.289167in}{1.389692in}}%
\pgfpathlineto{\pgfqpoint{4.291750in}{1.404533in}}%
\pgfpathlineto{\pgfqpoint{4.294333in}{1.405296in}}%
\pgfpathlineto{\pgfqpoint{4.296917in}{1.392965in}}%
\pgfpathlineto{\pgfqpoint{4.302083in}{1.336189in}}%
\pgfpathlineto{\pgfqpoint{4.317583in}{1.106847in}}%
\pgfpathlineto{\pgfqpoint{4.320167in}{1.113360in}}%
\pgfpathlineto{\pgfqpoint{4.322750in}{1.124853in}}%
\pgfpathlineto{\pgfqpoint{4.325333in}{1.127653in}}%
\pgfpathlineto{\pgfqpoint{4.327917in}{1.122212in}}%
\pgfpathlineto{\pgfqpoint{4.330500in}{1.109269in}}%
\pgfpathlineto{\pgfqpoint{4.333083in}{1.110300in}}%
\pgfpathlineto{\pgfqpoint{4.338250in}{1.165879in}}%
\pgfpathlineto{\pgfqpoint{4.346000in}{1.281733in}}%
\pgfpathlineto{\pgfqpoint{4.356333in}{1.484359in}}%
\pgfpathlineto{\pgfqpoint{4.364083in}{1.633510in}}%
\pgfpathlineto{\pgfqpoint{4.369250in}{1.691239in}}%
\pgfpathlineto{\pgfqpoint{4.371833in}{1.697946in}}%
\pgfpathlineto{\pgfqpoint{4.374417in}{1.686655in}}%
\pgfpathlineto{\pgfqpoint{4.377000in}{1.656110in}}%
\pgfpathlineto{\pgfqpoint{4.382167in}{1.538020in}}%
\pgfpathlineto{\pgfqpoint{4.389917in}{1.249933in}}%
\pgfpathlineto{\pgfqpoint{4.392500in}{1.138004in}}%
\pgfpathlineto{\pgfqpoint{4.395083in}{1.175392in}}%
\pgfpathlineto{\pgfqpoint{4.405417in}{1.579154in}}%
\pgfpathlineto{\pgfqpoint{4.410583in}{1.720187in}}%
\pgfpathlineto{\pgfqpoint{4.415750in}{1.806342in}}%
\pgfpathlineto{\pgfqpoint{4.418333in}{1.826757in}}%
\pgfpathlineto{\pgfqpoint{4.420917in}{1.831404in}}%
\pgfpathlineto{\pgfqpoint{4.423500in}{1.820226in}}%
\pgfpathlineto{\pgfqpoint{4.426083in}{1.793666in}}%
\pgfpathlineto{\pgfqpoint{4.431250in}{1.699211in}}%
\pgfpathlineto{\pgfqpoint{4.439000in}{1.488399in}}%
\pgfpathlineto{\pgfqpoint{4.449333in}{1.207119in}}%
\pgfpathlineto{\pgfqpoint{4.454500in}{1.111848in}}%
\pgfpathlineto{\pgfqpoint{4.457083in}{1.121155in}}%
\pgfpathlineto{\pgfqpoint{4.462250in}{1.161984in}}%
\pgfpathlineto{\pgfqpoint{4.467417in}{1.182008in}}%
\pgfpathlineto{\pgfqpoint{4.472583in}{1.200436in}}%
\pgfpathlineto{\pgfqpoint{4.477750in}{1.233402in}}%
\pgfpathlineto{\pgfqpoint{4.482917in}{1.286852in}}%
\pgfpathlineto{\pgfqpoint{4.498417in}{1.477251in}}%
\pgfpathlineto{\pgfqpoint{4.503583in}{1.506637in}}%
\pgfpathlineto{\pgfqpoint{4.506167in}{1.510657in}}%
\pgfpathlineto{\pgfqpoint{4.508750in}{1.507750in}}%
\pgfpathlineto{\pgfqpoint{4.511333in}{1.498621in}}%
\pgfpathlineto{\pgfqpoint{4.516500in}{1.466563in}}%
\pgfpathlineto{\pgfqpoint{4.524250in}{1.410198in}}%
\pgfpathlineto{\pgfqpoint{4.529417in}{1.388560in}}%
\pgfpathlineto{\pgfqpoint{4.532000in}{1.386034in}}%
\pgfpathlineto{\pgfqpoint{4.534583in}{1.388960in}}%
\pgfpathlineto{\pgfqpoint{4.539750in}{1.405969in}}%
\pgfpathlineto{\pgfqpoint{4.544917in}{1.422739in}}%
\pgfpathlineto{\pgfqpoint{4.547500in}{1.424130in}}%
\pgfpathlineto{\pgfqpoint{4.550083in}{1.417285in}}%
\pgfpathlineto{\pgfqpoint{4.552667in}{1.400074in}}%
\pgfpathlineto{\pgfqpoint{4.557833in}{1.329733in}}%
\pgfpathlineto{\pgfqpoint{4.563000in}{1.212306in}}%
\pgfpathlineto{\pgfqpoint{4.565583in}{1.139464in}}%
\pgfpathlineto{\pgfqpoint{4.568167in}{1.139554in}}%
\pgfpathlineto{\pgfqpoint{4.581083in}{1.526045in}}%
\pgfpathlineto{\pgfqpoint{4.586250in}{1.632603in}}%
\pgfpathlineto{\pgfqpoint{4.591417in}{1.698140in}}%
\pgfpathlineto{\pgfqpoint{4.594000in}{1.715843in}}%
\pgfpathlineto{\pgfqpoint{4.596583in}{1.724265in}}%
\pgfpathlineto{\pgfqpoint{4.599167in}{1.724117in}}%
\pgfpathlineto{\pgfqpoint{4.601750in}{1.716125in}}%
\pgfpathlineto{\pgfqpoint{4.606917in}{1.679645in}}%
\pgfpathlineto{\pgfqpoint{4.614667in}{1.589192in}}%
\pgfpathlineto{\pgfqpoint{4.622417in}{1.493733in}}%
\pgfpathlineto{\pgfqpoint{4.627583in}{1.450399in}}%
\pgfpathlineto{\pgfqpoint{4.632750in}{1.430090in}}%
\pgfpathlineto{\pgfqpoint{4.637917in}{1.424789in}}%
\pgfpathlineto{\pgfqpoint{4.640500in}{1.422443in}}%
\pgfpathlineto{\pgfqpoint{4.643083in}{1.417018in}}%
\pgfpathlineto{\pgfqpoint{4.645667in}{1.406327in}}%
\pgfpathlineto{\pgfqpoint{4.648250in}{1.388690in}}%
\pgfpathlineto{\pgfqpoint{4.653417in}{1.328975in}}%
\pgfpathlineto{\pgfqpoint{4.658583in}{1.236776in}}%
\pgfpathlineto{\pgfqpoint{4.663750in}{1.119292in}}%
\pgfpathlineto{\pgfqpoint{4.666333in}{1.145103in}}%
\pgfpathlineto{\pgfqpoint{4.679250in}{1.454566in}}%
\pgfpathlineto{\pgfqpoint{4.687000in}{1.587762in}}%
\pgfpathlineto{\pgfqpoint{4.694750in}{1.675468in}}%
\pgfpathlineto{\pgfqpoint{4.707667in}{1.797043in}}%
\pgfpathlineto{\pgfqpoint{4.723167in}{1.966193in}}%
\pgfpathlineto{\pgfqpoint{4.725750in}{1.977026in}}%
\pgfpathlineto{\pgfqpoint{4.728333in}{1.975643in}}%
\pgfpathlineto{\pgfqpoint{4.730917in}{1.959384in}}%
\pgfpathlineto{\pgfqpoint{4.733500in}{1.926108in}}%
\pgfpathlineto{\pgfqpoint{4.738667in}{1.804079in}}%
\pgfpathlineto{\pgfqpoint{4.743833in}{1.611524in}}%
\pgfpathlineto{\pgfqpoint{4.754167in}{1.108592in}}%
\pgfpathlineto{\pgfqpoint{4.764500in}{1.519235in}}%
\pgfpathlineto{\pgfqpoint{4.769667in}{1.640496in}}%
\pgfpathlineto{\pgfqpoint{4.772250in}{1.675084in}}%
\pgfpathlineto{\pgfqpoint{4.774833in}{1.693747in}}%
\pgfpathlineto{\pgfqpoint{4.777417in}{1.698593in}}%
\pgfpathlineto{\pgfqpoint{4.780000in}{1.692396in}}%
\pgfpathlineto{\pgfqpoint{4.785167in}{1.660049in}}%
\pgfpathlineto{\pgfqpoint{4.790333in}{1.623754in}}%
\pgfpathlineto{\pgfqpoint{4.792917in}{1.611666in}}%
\pgfpathlineto{\pgfqpoint{4.795500in}{1.606639in}}%
\pgfpathlineto{\pgfqpoint{4.798083in}{1.610129in}}%
\pgfpathlineto{\pgfqpoint{4.800667in}{1.622870in}}%
\pgfpathlineto{\pgfqpoint{4.805833in}{1.675258in}}%
\pgfpathlineto{\pgfqpoint{4.813583in}{1.799068in}}%
\pgfpathlineto{\pgfqpoint{4.821333in}{1.913241in}}%
\pgfpathlineto{\pgfqpoint{4.823917in}{1.934267in}}%
\pgfpathlineto{\pgfqpoint{4.826500in}{1.941914in}}%
\pgfpathlineto{\pgfqpoint{4.829083in}{1.933987in}}%
\pgfpathlineto{\pgfqpoint{4.831667in}{1.908992in}}%
\pgfpathlineto{\pgfqpoint{4.836833in}{1.805977in}}%
\pgfpathlineto{\pgfqpoint{4.842000in}{1.638169in}}%
\pgfpathlineto{\pgfqpoint{4.854917in}{1.126585in}}%
\pgfpathlineto{\pgfqpoint{4.862667in}{1.430715in}}%
\pgfpathlineto{\pgfqpoint{4.867833in}{1.573574in}}%
\pgfpathlineto{\pgfqpoint{4.873000in}{1.659039in}}%
\pgfpathlineto{\pgfqpoint{4.875583in}{1.681423in}}%
\pgfpathlineto{\pgfqpoint{4.878167in}{1.692160in}}%
\pgfpathlineto{\pgfqpoint{4.880750in}{1.693248in}}%
\pgfpathlineto{\pgfqpoint{4.883333in}{1.686895in}}%
\pgfpathlineto{\pgfqpoint{4.888500in}{1.660566in}}%
\pgfpathlineto{\pgfqpoint{4.898833in}{1.599185in}}%
\pgfpathlineto{\pgfqpoint{4.904000in}{1.577324in}}%
\pgfpathlineto{\pgfqpoint{4.911750in}{1.555102in}}%
\pgfpathlineto{\pgfqpoint{4.916917in}{1.540584in}}%
\pgfpathlineto{\pgfqpoint{4.922083in}{1.518064in}}%
\pgfpathlineto{\pgfqpoint{4.927250in}{1.479650in}}%
\pgfpathlineto{\pgfqpoint{4.932417in}{1.420008in}}%
\pgfpathlineto{\pgfqpoint{4.940167in}{1.293170in}}%
\pgfpathlineto{\pgfqpoint{4.950500in}{1.103244in}}%
\pgfpathlineto{\pgfqpoint{4.958250in}{1.231088in}}%
\pgfpathlineto{\pgfqpoint{4.966000in}{1.324181in}}%
\pgfpathlineto{\pgfqpoint{4.971167in}{1.366410in}}%
\pgfpathlineto{\pgfqpoint{4.976333in}{1.395211in}}%
\pgfpathlineto{\pgfqpoint{4.981500in}{1.412208in}}%
\pgfpathlineto{\pgfqpoint{4.984083in}{1.416056in}}%
\pgfpathlineto{\pgfqpoint{4.986667in}{1.416284in}}%
\pgfpathlineto{\pgfqpoint{4.989250in}{1.412357in}}%
\pgfpathlineto{\pgfqpoint{4.991833in}{1.403760in}}%
\pgfpathlineto{\pgfqpoint{4.997000in}{1.371409in}}%
\pgfpathlineto{\pgfqpoint{5.002167in}{1.320007in}}%
\pgfpathlineto{\pgfqpoint{5.020250in}{1.110403in}}%
\pgfpathlineto{\pgfqpoint{5.028000in}{1.185870in}}%
\pgfpathlineto{\pgfqpoint{5.033167in}{1.220370in}}%
\pgfpathlineto{\pgfqpoint{5.038333in}{1.238682in}}%
\pgfpathlineto{\pgfqpoint{5.040917in}{1.241110in}}%
\pgfpathlineto{\pgfqpoint{5.043500in}{1.239129in}}%
\pgfpathlineto{\pgfqpoint{5.046083in}{1.233149in}}%
\pgfpathlineto{\pgfqpoint{5.051250in}{1.212002in}}%
\pgfpathlineto{\pgfqpoint{5.071917in}{1.109420in}}%
\pgfpathlineto{\pgfqpoint{5.074500in}{1.102133in}}%
\pgfpathlineto{\pgfqpoint{5.079667in}{1.129266in}}%
\pgfpathlineto{\pgfqpoint{5.084833in}{1.165641in}}%
\pgfpathlineto{\pgfqpoint{5.100333in}{1.299454in}}%
\pgfpathlineto{\pgfqpoint{5.102917in}{1.307889in}}%
\pgfpathlineto{\pgfqpoint{5.105500in}{1.306932in}}%
\pgfpathlineto{\pgfqpoint{5.108083in}{1.295611in}}%
\pgfpathlineto{\pgfqpoint{5.110667in}{1.273897in}}%
\pgfpathlineto{\pgfqpoint{5.115833in}{1.203924in}}%
\pgfpathlineto{\pgfqpoint{5.121000in}{1.113496in}}%
\pgfpathlineto{\pgfqpoint{5.123583in}{1.132446in}}%
\pgfpathlineto{\pgfqpoint{5.131333in}{1.242927in}}%
\pgfpathlineto{\pgfqpoint{5.136500in}{1.278242in}}%
\pgfpathlineto{\pgfqpoint{5.139083in}{1.282526in}}%
\pgfpathlineto{\pgfqpoint{5.141667in}{1.277971in}}%
\pgfpathlineto{\pgfqpoint{5.144250in}{1.264907in}}%
\pgfpathlineto{\pgfqpoint{5.149417in}{1.215516in}}%
\pgfpathlineto{\pgfqpoint{5.157167in}{1.101937in}}%
\pgfpathlineto{\pgfqpoint{5.170083in}{1.316471in}}%
\pgfpathlineto{\pgfqpoint{5.175250in}{1.375384in}}%
\pgfpathlineto{\pgfqpoint{5.177833in}{1.393903in}}%
\pgfpathlineto{\pgfqpoint{5.180417in}{1.403495in}}%
\pgfpathlineto{\pgfqpoint{5.183000in}{1.403146in}}%
\pgfpathlineto{\pgfqpoint{5.185583in}{1.392190in}}%
\pgfpathlineto{\pgfqpoint{5.188167in}{1.370461in}}%
\pgfpathlineto{\pgfqpoint{5.193333in}{1.297211in}}%
\pgfpathlineto{\pgfqpoint{5.203667in}{1.115141in}}%
\pgfpathlineto{\pgfqpoint{5.211417in}{1.246920in}}%
\pgfpathlineto{\pgfqpoint{5.216583in}{1.288864in}}%
\pgfpathlineto{\pgfqpoint{5.219167in}{1.293749in}}%
\pgfpathlineto{\pgfqpoint{5.221750in}{1.288369in}}%
\pgfpathlineto{\pgfqpoint{5.224333in}{1.273714in}}%
\pgfpathlineto{\pgfqpoint{5.229500in}{1.222320in}}%
\pgfpathlineto{\pgfqpoint{5.237250in}{1.115393in}}%
\pgfpathlineto{\pgfqpoint{5.239833in}{1.121593in}}%
\pgfpathlineto{\pgfqpoint{5.247583in}{1.215809in}}%
\pgfpathlineto{\pgfqpoint{5.252750in}{1.249626in}}%
\pgfpathlineto{\pgfqpoint{5.255333in}{1.253039in}}%
\pgfpathlineto{\pgfqpoint{5.257917in}{1.245396in}}%
\pgfpathlineto{\pgfqpoint{5.260500in}{1.225460in}}%
\pgfpathlineto{\pgfqpoint{5.265667in}{1.146387in}}%
\pgfpathlineto{\pgfqpoint{5.268250in}{1.111981in}}%
\pgfpathlineto{\pgfqpoint{5.278583in}{1.416630in}}%
\pgfpathlineto{\pgfqpoint{5.283750in}{1.554499in}}%
\pgfpathlineto{\pgfqpoint{5.288917in}{1.637347in}}%
\pgfpathlineto{\pgfqpoint{5.291500in}{1.651488in}}%
\pgfpathlineto{\pgfqpoint{5.294083in}{1.646389in}}%
\pgfpathlineto{\pgfqpoint{5.296667in}{1.622884in}}%
\pgfpathlineto{\pgfqpoint{5.301833in}{1.529522in}}%
\pgfpathlineto{\pgfqpoint{5.319917in}{1.104211in}}%
\pgfpathlineto{\pgfqpoint{5.325083in}{1.156822in}}%
\pgfpathlineto{\pgfqpoint{5.327667in}{1.165671in}}%
\pgfpathlineto{\pgfqpoint{5.330250in}{1.164035in}}%
\pgfpathlineto{\pgfqpoint{5.332833in}{1.153319in}}%
\pgfpathlineto{\pgfqpoint{5.338000in}{1.111360in}}%
\pgfpathlineto{\pgfqpoint{5.340583in}{1.116321in}}%
\pgfpathlineto{\pgfqpoint{5.350917in}{1.230264in}}%
\pgfpathlineto{\pgfqpoint{5.356083in}{1.263936in}}%
\pgfpathlineto{\pgfqpoint{5.358667in}{1.268480in}}%
\pgfpathlineto{\pgfqpoint{5.361250in}{1.262533in}}%
\pgfpathlineto{\pgfqpoint{5.363833in}{1.244708in}}%
\pgfpathlineto{\pgfqpoint{5.369000in}{1.170220in}}%
\pgfpathlineto{\pgfqpoint{5.371583in}{1.113415in}}%
\pgfpathlineto{\pgfqpoint{5.374167in}{1.155478in}}%
\pgfpathlineto{\pgfqpoint{5.381917in}{1.416737in}}%
\pgfpathlineto{\pgfqpoint{5.392250in}{1.781150in}}%
\pgfpathlineto{\pgfqpoint{5.397417in}{1.906724in}}%
\pgfpathlineto{\pgfqpoint{5.400000in}{1.945708in}}%
\pgfpathlineto{\pgfqpoint{5.402583in}{1.967290in}}%
\pgfpathlineto{\pgfqpoint{5.402583in}{1.967290in}}%
\pgfusepath{stroke}%
\end{pgfscope}%
\begin{pgfscope}%
\pgfpathrectangle{\pgfqpoint{0.750000in}{0.330000in}}{\pgfqpoint{4.650000in}{2.310000in}}%
\pgfusepath{clip}%
\pgfsetrectcap%
\pgfsetroundjoin%
\pgfsetlinewidth{2.509375pt}%
\definecolor{currentstroke}{rgb}{1.000000,0.000000,0.000000}%
\pgfsetstrokecolor{currentstroke}%
\pgfsetdash{}{0pt}%
\pgfpathmoveto{\pgfqpoint{0.744833in}{1.100764in}}%
\pgfpathlineto{\pgfqpoint{0.788750in}{1.101390in}}%
\pgfpathlineto{\pgfqpoint{0.822333in}{1.100454in}}%
\pgfpathlineto{\pgfqpoint{0.850750in}{1.100611in}}%
\pgfpathlineto{\pgfqpoint{0.897250in}{1.101743in}}%
\pgfpathlineto{\pgfqpoint{0.920500in}{1.101125in}}%
\pgfpathlineto{\pgfqpoint{0.946333in}{1.102355in}}%
\pgfpathlineto{\pgfqpoint{1.003167in}{1.106643in}}%
\pgfpathlineto{\pgfqpoint{1.031583in}{1.101482in}}%
\pgfpathlineto{\pgfqpoint{1.057417in}{1.116076in}}%
\pgfpathlineto{\pgfqpoint{1.093583in}{1.125120in}}%
\pgfpathlineto{\pgfqpoint{1.127167in}{1.104704in}}%
\pgfpathlineto{\pgfqpoint{1.165917in}{1.121001in}}%
\pgfpathlineto{\pgfqpoint{1.199500in}{1.129162in}}%
\pgfpathlineto{\pgfqpoint{1.235667in}{1.101711in}}%
\pgfpathlineto{\pgfqpoint{1.274417in}{1.123559in}}%
\pgfpathlineto{\pgfqpoint{1.310583in}{1.134013in}}%
\pgfpathlineto{\pgfqpoint{1.341583in}{1.115561in}}%
\pgfpathlineto{\pgfqpoint{1.372583in}{1.128087in}}%
\pgfpathlineto{\pgfqpoint{1.411333in}{1.120964in}}%
\pgfpathlineto{\pgfqpoint{1.444917in}{1.118065in}}%
\pgfpathlineto{\pgfqpoint{1.481083in}{1.130590in}}%
\pgfpathlineto{\pgfqpoint{1.517250in}{1.142325in}}%
\pgfpathlineto{\pgfqpoint{1.558583in}{1.147874in}}%
\pgfpathlineto{\pgfqpoint{1.592167in}{1.153333in}}%
\pgfpathlineto{\pgfqpoint{1.625750in}{1.177300in}}%
\pgfpathlineto{\pgfqpoint{1.667083in}{1.195015in}}%
\pgfpathlineto{\pgfqpoint{1.698083in}{1.168930in}}%
\pgfpathlineto{\pgfqpoint{1.731667in}{1.231264in}}%
\pgfpathlineto{\pgfqpoint{1.775583in}{1.280416in}}%
\pgfpathlineto{\pgfqpoint{1.809167in}{1.227503in}}%
\pgfpathlineto{\pgfqpoint{1.840167in}{1.214074in}}%
\pgfpathlineto{\pgfqpoint{1.884083in}{1.288279in}}%
\pgfpathlineto{\pgfqpoint{1.930583in}{1.250277in}}%
\pgfpathlineto{\pgfqpoint{1.966750in}{1.112953in}}%
\pgfpathlineto{\pgfqpoint{1.997750in}{1.355094in}}%
\pgfpathlineto{\pgfqpoint{2.033917in}{1.433131in}}%
\pgfpathlineto{\pgfqpoint{2.064917in}{1.261578in}}%
\pgfpathlineto{\pgfqpoint{2.095917in}{1.476303in}}%
\pgfpathlineto{\pgfqpoint{2.132083in}{1.322642in}}%
\pgfpathlineto{\pgfqpoint{2.168250in}{1.125710in}}%
\pgfpathlineto{\pgfqpoint{2.196667in}{1.336569in}}%
\pgfpathlineto{\pgfqpoint{2.225083in}{1.186469in}}%
\pgfpathlineto{\pgfqpoint{2.250917in}{1.137629in}}%
\pgfpathlineto{\pgfqpoint{2.294833in}{1.336606in}}%
\pgfpathlineto{\pgfqpoint{2.336167in}{1.219589in}}%
\pgfpathlineto{\pgfqpoint{2.367167in}{1.233004in}}%
\pgfpathlineto{\pgfqpoint{2.405917in}{1.329459in}}%
\pgfpathlineto{\pgfqpoint{2.439500in}{1.319289in}}%
\pgfpathlineto{\pgfqpoint{2.467917in}{1.355520in}}%
\pgfpathlineto{\pgfqpoint{2.506667in}{1.461601in}}%
\pgfpathlineto{\pgfqpoint{2.537667in}{1.312288in}}%
\pgfpathlineto{\pgfqpoint{2.566083in}{1.254609in}}%
\pgfpathlineto{\pgfqpoint{2.607417in}{1.364050in}}%
\pgfpathlineto{\pgfqpoint{2.641000in}{1.125978in}}%
\pgfpathlineto{\pgfqpoint{2.653917in}{1.138333in}}%
\pgfpathlineto{\pgfqpoint{2.703000in}{1.214640in}}%
\pgfpathlineto{\pgfqpoint{2.726250in}{1.157348in}}%
\pgfpathlineto{\pgfqpoint{2.757250in}{1.477127in}}%
\pgfpathlineto{\pgfqpoint{2.785667in}{1.147177in}}%
\pgfpathlineto{\pgfqpoint{2.801167in}{1.102837in}}%
\pgfpathlineto{\pgfqpoint{2.847667in}{1.465780in}}%
\pgfpathlineto{\pgfqpoint{2.878667in}{1.192475in}}%
\pgfpathlineto{\pgfqpoint{2.896750in}{1.292825in}}%
\pgfpathlineto{\pgfqpoint{2.930333in}{1.448489in}}%
\pgfpathlineto{\pgfqpoint{3.020750in}{1.231819in}}%
\pgfpathlineto{\pgfqpoint{3.036250in}{1.206160in}}%
\pgfpathlineto{\pgfqpoint{3.064667in}{1.393964in}}%
\pgfpathlineto{\pgfqpoint{3.093083in}{1.239624in}}%
\pgfpathlineto{\pgfqpoint{3.113750in}{1.296621in}}%
\pgfpathlineto{\pgfqpoint{3.168000in}{1.600092in}}%
\pgfpathlineto{\pgfqpoint{3.255833in}{1.442484in}}%
\pgfpathlineto{\pgfqpoint{3.281667in}{1.332120in}}%
\pgfpathlineto{\pgfqpoint{3.292000in}{1.345458in}}%
\pgfpathlineto{\pgfqpoint{3.328167in}{1.495677in}}%
\pgfpathlineto{\pgfqpoint{3.356583in}{1.250825in}}%
\pgfpathlineto{\pgfqpoint{3.382417in}{1.482386in}}%
\pgfpathlineto{\pgfqpoint{3.423750in}{1.503155in}}%
\pgfpathlineto{\pgfqpoint{3.459917in}{1.160824in}}%
\pgfpathlineto{\pgfqpoint{3.480583in}{1.248687in}}%
\pgfpathlineto{\pgfqpoint{3.506417in}{1.130489in}}%
\pgfpathlineto{\pgfqpoint{3.524500in}{1.160902in}}%
\pgfpathlineto{\pgfqpoint{3.555500in}{1.497085in}}%
\pgfpathlineto{\pgfqpoint{3.589083in}{1.410474in}}%
\pgfpathlineto{\pgfqpoint{3.620083in}{1.457268in}}%
\pgfpathlineto{\pgfqpoint{3.656250in}{1.559578in}}%
\pgfpathlineto{\pgfqpoint{3.695000in}{1.630920in}}%
\pgfpathlineto{\pgfqpoint{3.728583in}{1.381587in}}%
\pgfpathlineto{\pgfqpoint{3.759583in}{1.211891in}}%
\pgfpathlineto{\pgfqpoint{3.793167in}{1.580063in}}%
\pgfpathlineto{\pgfqpoint{3.834500in}{1.650940in}}%
\pgfpathlineto{\pgfqpoint{3.868083in}{1.345476in}}%
\pgfpathlineto{\pgfqpoint{3.904250in}{1.764120in}}%
\pgfpathlineto{\pgfqpoint{3.943000in}{1.725991in}}%
\pgfpathlineto{\pgfqpoint{3.979167in}{1.100349in}}%
\pgfpathlineto{\pgfqpoint{4.005000in}{1.330432in}}%
\pgfpathlineto{\pgfqpoint{4.041167in}{1.456915in}}%
\pgfpathlineto{\pgfqpoint{4.077333in}{1.350661in}}%
\pgfpathlineto{\pgfqpoint{4.113500in}{1.820542in}}%
\pgfpathlineto{\pgfqpoint{4.154833in}{1.768821in}}%
\pgfpathlineto{\pgfqpoint{4.183250in}{1.315186in}}%
\pgfpathlineto{\pgfqpoint{4.206500in}{1.489359in}}%
\pgfpathlineto{\pgfqpoint{4.247833in}{2.014443in}}%
\pgfpathlineto{\pgfqpoint{4.294333in}{1.405296in}}%
\pgfpathlineto{\pgfqpoint{4.325333in}{1.127653in}}%
\pgfpathlineto{\pgfqpoint{4.371833in}{1.697946in}}%
\pgfpathlineto{\pgfqpoint{4.420917in}{1.831404in}}%
\pgfpathlineto{\pgfqpoint{4.506167in}{1.510657in}}%
\pgfpathlineto{\pgfqpoint{4.532000in}{1.386034in}}%
\pgfpathlineto{\pgfqpoint{4.547500in}{1.424130in}}%
\pgfpathlineto{\pgfqpoint{4.596583in}{1.724265in}}%
\pgfpathlineto{\pgfqpoint{4.725750in}{1.977026in}}%
\pgfpathlineto{\pgfqpoint{4.777417in}{1.698593in}}%
\pgfpathlineto{\pgfqpoint{4.795500in}{1.606639in}}%
\pgfpathlineto{\pgfqpoint{4.826500in}{1.941914in}}%
\pgfpathlineto{\pgfqpoint{4.880750in}{1.693248in}}%
\pgfpathlineto{\pgfqpoint{4.986667in}{1.416284in}}%
\pgfpathlineto{\pgfqpoint{5.040917in}{1.241110in}}%
\pgfpathlineto{\pgfqpoint{5.102917in}{1.307889in}}%
\pgfpathlineto{\pgfqpoint{5.139083in}{1.282526in}}%
\pgfpathlineto{\pgfqpoint{5.180417in}{1.403495in}}%
\pgfpathlineto{\pgfqpoint{5.219167in}{1.293749in}}%
\pgfpathlineto{\pgfqpoint{5.255333in}{1.253039in}}%
\pgfpathlineto{\pgfqpoint{5.291500in}{1.651488in}}%
\pgfpathlineto{\pgfqpoint{5.327667in}{1.165671in}}%
\pgfpathlineto{\pgfqpoint{5.358667in}{1.268480in}}%
\pgfpathlineto{\pgfqpoint{5.405167in}{1.971469in}}%
\pgfusepath{stroke}%
\end{pgfscope}%
\begin{pgfscope}%
\pgfpathrectangle{\pgfqpoint{0.750000in}{0.330000in}}{\pgfqpoint{4.650000in}{2.310000in}}%
\pgfusepath{clip}%
\pgfsetrectcap%
\pgfsetroundjoin%
\pgfsetlinewidth{2.509375pt}%
\definecolor{currentstroke}{rgb}{0.000000,0.501961,0.000000}%
\pgfsetstrokecolor{currentstroke}%
\pgfsetstrokeopacity{0.700000}%
\pgfsetdash{}{0pt}%
\pgfpathmoveto{\pgfqpoint{0.747417in}{1.100954in}}%
\pgfpathlineto{\pgfqpoint{0.799083in}{1.101301in}}%
\pgfpathlineto{\pgfqpoint{0.866250in}{1.101271in}}%
\pgfpathlineto{\pgfqpoint{0.915333in}{1.101378in}}%
\pgfpathlineto{\pgfqpoint{0.933417in}{1.101841in}}%
\pgfpathlineto{\pgfqpoint{0.977333in}{1.105130in}}%
\pgfpathlineto{\pgfqpoint{1.021250in}{1.104946in}}%
\pgfpathlineto{\pgfqpoint{1.034167in}{1.101926in}}%
\pgfpathlineto{\pgfqpoint{1.044500in}{1.108332in}}%
\pgfpathlineto{\pgfqpoint{1.065167in}{1.122170in}}%
\pgfpathlineto{\pgfqpoint{1.075500in}{1.126286in}}%
\pgfpathlineto{\pgfqpoint{1.083250in}{1.127229in}}%
\pgfpathlineto{\pgfqpoint{1.091000in}{1.126060in}}%
\pgfpathlineto{\pgfqpoint{1.101333in}{1.121617in}}%
\pgfpathlineto{\pgfqpoint{1.119417in}{1.110225in}}%
\pgfpathlineto{\pgfqpoint{1.129750in}{1.104468in}}%
\pgfpathlineto{\pgfqpoint{1.134917in}{1.103990in}}%
\pgfpathlineto{\pgfqpoint{1.142667in}{1.107349in}}%
\pgfpathlineto{\pgfqpoint{1.155583in}{1.116160in}}%
\pgfpathlineto{\pgfqpoint{1.173667in}{1.129135in}}%
\pgfpathlineto{\pgfqpoint{1.181417in}{1.132355in}}%
\pgfpathlineto{\pgfqpoint{1.189167in}{1.133272in}}%
\pgfpathlineto{\pgfqpoint{1.196917in}{1.131598in}}%
\pgfpathlineto{\pgfqpoint{1.204667in}{1.127428in}}%
\pgfpathlineto{\pgfqpoint{1.217583in}{1.116962in}}%
\pgfpathlineto{\pgfqpoint{1.230500in}{1.106819in}}%
\pgfpathlineto{\pgfqpoint{1.240833in}{1.101388in}}%
\pgfpathlineto{\pgfqpoint{1.246000in}{1.103031in}}%
\pgfpathlineto{\pgfqpoint{1.258917in}{1.111775in}}%
\pgfpathlineto{\pgfqpoint{1.287333in}{1.133815in}}%
\pgfpathlineto{\pgfqpoint{1.295083in}{1.136322in}}%
\pgfpathlineto{\pgfqpoint{1.302833in}{1.136662in}}%
\pgfpathlineto{\pgfqpoint{1.310583in}{1.134898in}}%
\pgfpathlineto{\pgfqpoint{1.318333in}{1.131137in}}%
\pgfpathlineto{\pgfqpoint{1.331250in}{1.121661in}}%
\pgfpathlineto{\pgfqpoint{1.339000in}{1.116705in}}%
\pgfpathlineto{\pgfqpoint{1.344167in}{1.115539in}}%
\pgfpathlineto{\pgfqpoint{1.349333in}{1.116585in}}%
\pgfpathlineto{\pgfqpoint{1.359667in}{1.122225in}}%
\pgfpathlineto{\pgfqpoint{1.370000in}{1.127412in}}%
\pgfpathlineto{\pgfqpoint{1.377750in}{1.129229in}}%
\pgfpathlineto{\pgfqpoint{1.385500in}{1.128977in}}%
\pgfpathlineto{\pgfqpoint{1.398417in}{1.125625in}}%
\pgfpathlineto{\pgfqpoint{1.429417in}{1.116502in}}%
\pgfpathlineto{\pgfqpoint{1.437167in}{1.116351in}}%
\pgfpathlineto{\pgfqpoint{1.447500in}{1.119037in}}%
\pgfpathlineto{\pgfqpoint{1.506917in}{1.139977in}}%
\pgfpathlineto{\pgfqpoint{1.525000in}{1.143782in}}%
\pgfpathlineto{\pgfqpoint{1.543083in}{1.145442in}}%
\pgfpathlineto{\pgfqpoint{1.553417in}{1.146795in}}%
\pgfpathlineto{\pgfqpoint{1.579250in}{1.152429in}}%
\pgfpathlineto{\pgfqpoint{1.594750in}{1.154036in}}%
\pgfpathlineto{\pgfqpoint{1.602500in}{1.157599in}}%
\pgfpathlineto{\pgfqpoint{1.612833in}{1.165879in}}%
\pgfpathlineto{\pgfqpoint{1.636083in}{1.186653in}}%
\pgfpathlineto{\pgfqpoint{1.646417in}{1.193309in}}%
\pgfpathlineto{\pgfqpoint{1.654167in}{1.196254in}}%
\pgfpathlineto{\pgfqpoint{1.661917in}{1.196636in}}%
\pgfpathlineto{\pgfqpoint{1.667083in}{1.195101in}}%
\pgfpathlineto{\pgfqpoint{1.674833in}{1.189937in}}%
\pgfpathlineto{\pgfqpoint{1.685167in}{1.178964in}}%
\pgfpathlineto{\pgfqpoint{1.692917in}{1.171278in}}%
\pgfpathlineto{\pgfqpoint{1.698083in}{1.169377in}}%
\pgfpathlineto{\pgfqpoint{1.700667in}{1.170012in}}%
\pgfpathlineto{\pgfqpoint{1.705833in}{1.174856in}}%
\pgfpathlineto{\pgfqpoint{1.711000in}{1.183962in}}%
\pgfpathlineto{\pgfqpoint{1.721333in}{1.208893in}}%
\pgfpathlineto{\pgfqpoint{1.731667in}{1.232274in}}%
\pgfpathlineto{\pgfqpoint{1.739417in}{1.244804in}}%
\pgfpathlineto{\pgfqpoint{1.749750in}{1.256258in}}%
\pgfpathlineto{\pgfqpoint{1.775583in}{1.280434in}}%
\pgfpathlineto{\pgfqpoint{1.780750in}{1.281884in}}%
\pgfpathlineto{\pgfqpoint{1.785917in}{1.280187in}}%
\pgfpathlineto{\pgfqpoint{1.791083in}{1.274706in}}%
\pgfpathlineto{\pgfqpoint{1.796250in}{1.265312in}}%
\pgfpathlineto{\pgfqpoint{1.804000in}{1.245250in}}%
\pgfpathlineto{\pgfqpoint{1.816917in}{1.210133in}}%
\pgfpathlineto{\pgfqpoint{1.822083in}{1.202502in}}%
\pgfpathlineto{\pgfqpoint{1.824667in}{1.200993in}}%
\pgfpathlineto{\pgfqpoint{1.827250in}{1.201026in}}%
\pgfpathlineto{\pgfqpoint{1.832417in}{1.205014in}}%
\pgfpathlineto{\pgfqpoint{1.840167in}{1.216985in}}%
\pgfpathlineto{\pgfqpoint{1.866000in}{1.261830in}}%
\pgfpathlineto{\pgfqpoint{1.894417in}{1.301575in}}%
\pgfpathlineto{\pgfqpoint{1.899583in}{1.303463in}}%
\pgfpathlineto{\pgfqpoint{1.904750in}{1.301873in}}%
\pgfpathlineto{\pgfqpoint{1.909917in}{1.297259in}}%
\pgfpathlineto{\pgfqpoint{1.917667in}{1.286520in}}%
\pgfpathlineto{\pgfqpoint{1.925417in}{1.272003in}}%
\pgfpathlineto{\pgfqpoint{1.933167in}{1.251141in}}%
\pgfpathlineto{\pgfqpoint{1.940917in}{1.221811in}}%
\pgfpathlineto{\pgfqpoint{1.953833in}{1.159868in}}%
\pgfpathlineto{\pgfqpoint{1.961583in}{1.119027in}}%
\pgfpathlineto{\pgfqpoint{1.964167in}{1.112618in}}%
\pgfpathlineto{\pgfqpoint{1.966750in}{1.123665in}}%
\pgfpathlineto{\pgfqpoint{1.974500in}{1.182402in}}%
\pgfpathlineto{\pgfqpoint{1.997750in}{1.375911in}}%
\pgfpathlineto{\pgfqpoint{2.005500in}{1.424041in}}%
\pgfpathlineto{\pgfqpoint{2.010667in}{1.447878in}}%
\pgfpathlineto{\pgfqpoint{2.015833in}{1.463418in}}%
\pgfpathlineto{\pgfqpoint{2.018417in}{1.467649in}}%
\pgfpathlineto{\pgfqpoint{2.021000in}{1.469338in}}%
\pgfpathlineto{\pgfqpoint{2.023583in}{1.468403in}}%
\pgfpathlineto{\pgfqpoint{2.026167in}{1.464824in}}%
\pgfpathlineto{\pgfqpoint{2.031333in}{1.450042in}}%
\pgfpathlineto{\pgfqpoint{2.036500in}{1.426277in}}%
\pgfpathlineto{\pgfqpoint{2.044250in}{1.378144in}}%
\pgfpathlineto{\pgfqpoint{2.059750in}{1.273640in}}%
\pgfpathlineto{\pgfqpoint{2.062333in}{1.264870in}}%
\pgfpathlineto{\pgfqpoint{2.064917in}{1.262438in}}%
\pgfpathlineto{\pgfqpoint{2.067500in}{1.267176in}}%
\pgfpathlineto{\pgfqpoint{2.070083in}{1.278749in}}%
\pgfpathlineto{\pgfqpoint{2.075250in}{1.316852in}}%
\pgfpathlineto{\pgfqpoint{2.090750in}{1.451590in}}%
\pgfpathlineto{\pgfqpoint{2.095917in}{1.481176in}}%
\pgfpathlineto{\pgfqpoint{2.101083in}{1.497387in}}%
\pgfpathlineto{\pgfqpoint{2.103667in}{1.500016in}}%
\pgfpathlineto{\pgfqpoint{2.106250in}{1.498942in}}%
\pgfpathlineto{\pgfqpoint{2.108833in}{1.494267in}}%
\pgfpathlineto{\pgfqpoint{2.114000in}{1.475052in}}%
\pgfpathlineto{\pgfqpoint{2.119167in}{1.445167in}}%
\pgfpathlineto{\pgfqpoint{2.132083in}{1.348579in}}%
\pgfpathlineto{\pgfqpoint{2.150167in}{1.219774in}}%
\pgfpathlineto{\pgfqpoint{2.163083in}{1.129495in}}%
\pgfpathlineto{\pgfqpoint{2.165667in}{1.121341in}}%
\pgfpathlineto{\pgfqpoint{2.168250in}{1.132117in}}%
\pgfpathlineto{\pgfqpoint{2.176000in}{1.197862in}}%
\pgfpathlineto{\pgfqpoint{2.186333in}{1.285220in}}%
\pgfpathlineto{\pgfqpoint{2.191500in}{1.318029in}}%
\pgfpathlineto{\pgfqpoint{2.196667in}{1.339601in}}%
\pgfpathlineto{\pgfqpoint{2.199250in}{1.345550in}}%
\pgfpathlineto{\pgfqpoint{2.201833in}{1.348044in}}%
\pgfpathlineto{\pgfqpoint{2.204417in}{1.346988in}}%
\pgfpathlineto{\pgfqpoint{2.207000in}{1.342365in}}%
\pgfpathlineto{\pgfqpoint{2.212167in}{1.322813in}}%
\pgfpathlineto{\pgfqpoint{2.217333in}{1.291126in}}%
\pgfpathlineto{\pgfqpoint{2.225083in}{1.227813in}}%
\pgfpathlineto{\pgfqpoint{2.238000in}{1.113753in}}%
\pgfpathlineto{\pgfqpoint{2.240583in}{1.109974in}}%
\pgfpathlineto{\pgfqpoint{2.250917in}{1.172459in}}%
\pgfpathlineto{\pgfqpoint{2.258667in}{1.202784in}}%
\pgfpathlineto{\pgfqpoint{2.274167in}{1.254109in}}%
\pgfpathlineto{\pgfqpoint{2.292250in}{1.330676in}}%
\pgfpathlineto{\pgfqpoint{2.297417in}{1.340184in}}%
\pgfpathlineto{\pgfqpoint{2.300000in}{1.341116in}}%
\pgfpathlineto{\pgfqpoint{2.302583in}{1.339483in}}%
\pgfpathlineto{\pgfqpoint{2.305167in}{1.335448in}}%
\pgfpathlineto{\pgfqpoint{2.310333in}{1.321437in}}%
\pgfpathlineto{\pgfqpoint{2.320667in}{1.281851in}}%
\pgfpathlineto{\pgfqpoint{2.341333in}{1.205077in}}%
\pgfpathlineto{\pgfqpoint{2.346500in}{1.191446in}}%
\pgfpathlineto{\pgfqpoint{2.349083in}{1.187846in}}%
\pgfpathlineto{\pgfqpoint{2.351667in}{1.187217in}}%
\pgfpathlineto{\pgfqpoint{2.354250in}{1.189854in}}%
\pgfpathlineto{\pgfqpoint{2.359417in}{1.204051in}}%
\pgfpathlineto{\pgfqpoint{2.367167in}{1.238706in}}%
\pgfpathlineto{\pgfqpoint{2.377500in}{1.286026in}}%
\pgfpathlineto{\pgfqpoint{2.382667in}{1.303462in}}%
\pgfpathlineto{\pgfqpoint{2.387833in}{1.315218in}}%
\pgfpathlineto{\pgfqpoint{2.393000in}{1.321982in}}%
\pgfpathlineto{\pgfqpoint{2.400750in}{1.327158in}}%
\pgfpathlineto{\pgfqpoint{2.408500in}{1.332816in}}%
\pgfpathlineto{\pgfqpoint{2.421417in}{1.344512in}}%
\pgfpathlineto{\pgfqpoint{2.424000in}{1.345034in}}%
\pgfpathlineto{\pgfqpoint{2.426583in}{1.344330in}}%
\pgfpathlineto{\pgfqpoint{2.431750in}{1.338953in}}%
\pgfpathlineto{\pgfqpoint{2.439500in}{1.323570in}}%
\pgfpathlineto{\pgfqpoint{2.444667in}{1.313486in}}%
\pgfpathlineto{\pgfqpoint{2.447250in}{1.310259in}}%
\pgfpathlineto{\pgfqpoint{2.449833in}{1.308962in}}%
\pgfpathlineto{\pgfqpoint{2.452417in}{1.309938in}}%
\pgfpathlineto{\pgfqpoint{2.455000in}{1.313298in}}%
\pgfpathlineto{\pgfqpoint{2.460167in}{1.326428in}}%
\pgfpathlineto{\pgfqpoint{2.470500in}{1.366251in}}%
\pgfpathlineto{\pgfqpoint{2.480833in}{1.403903in}}%
\pgfpathlineto{\pgfqpoint{2.491167in}{1.432045in}}%
\pgfpathlineto{\pgfqpoint{2.501500in}{1.454598in}}%
\pgfpathlineto{\pgfqpoint{2.506667in}{1.462298in}}%
\pgfpathlineto{\pgfqpoint{2.509250in}{1.464153in}}%
\pgfpathlineto{\pgfqpoint{2.511833in}{1.464104in}}%
\pgfpathlineto{\pgfqpoint{2.514417in}{1.461737in}}%
\pgfpathlineto{\pgfqpoint{2.517000in}{1.456703in}}%
\pgfpathlineto{\pgfqpoint{2.522167in}{1.437822in}}%
\pgfpathlineto{\pgfqpoint{2.527333in}{1.407366in}}%
\pgfpathlineto{\pgfqpoint{2.535083in}{1.345919in}}%
\pgfpathlineto{\pgfqpoint{2.545417in}{1.262137in}}%
\pgfpathlineto{\pgfqpoint{2.550583in}{1.235997in}}%
\pgfpathlineto{\pgfqpoint{2.553167in}{1.230281in}}%
\pgfpathlineto{\pgfqpoint{2.555750in}{1.229758in}}%
\pgfpathlineto{\pgfqpoint{2.558333in}{1.233664in}}%
\pgfpathlineto{\pgfqpoint{2.563500in}{1.249676in}}%
\pgfpathlineto{\pgfqpoint{2.581583in}{1.313902in}}%
\pgfpathlineto{\pgfqpoint{2.594500in}{1.353955in}}%
\pgfpathlineto{\pgfqpoint{2.599667in}{1.364106in}}%
\pgfpathlineto{\pgfqpoint{2.602250in}{1.366569in}}%
\pgfpathlineto{\pgfqpoint{2.604833in}{1.366925in}}%
\pgfpathlineto{\pgfqpoint{2.607417in}{1.364993in}}%
\pgfpathlineto{\pgfqpoint{2.610000in}{1.360679in}}%
\pgfpathlineto{\pgfqpoint{2.615167in}{1.344996in}}%
\pgfpathlineto{\pgfqpoint{2.620333in}{1.320977in}}%
\pgfpathlineto{\pgfqpoint{2.630667in}{1.258406in}}%
\pgfpathlineto{\pgfqpoint{2.638417in}{1.214046in}}%
\pgfpathlineto{\pgfqpoint{2.643583in}{1.196180in}}%
\pgfpathlineto{\pgfqpoint{2.646167in}{1.192916in}}%
\pgfpathlineto{\pgfqpoint{2.648750in}{1.193374in}}%
\pgfpathlineto{\pgfqpoint{2.653917in}{1.201508in}}%
\pgfpathlineto{\pgfqpoint{2.661667in}{1.216258in}}%
\pgfpathlineto{\pgfqpoint{2.687500in}{1.251250in}}%
\pgfpathlineto{\pgfqpoint{2.690083in}{1.252606in}}%
\pgfpathlineto{\pgfqpoint{2.692667in}{1.252395in}}%
\pgfpathlineto{\pgfqpoint{2.695250in}{1.250490in}}%
\pgfpathlineto{\pgfqpoint{2.700417in}{1.241785in}}%
\pgfpathlineto{\pgfqpoint{2.705583in}{1.227772in}}%
\pgfpathlineto{\pgfqpoint{2.721083in}{1.177690in}}%
\pgfpathlineto{\pgfqpoint{2.723667in}{1.179014in}}%
\pgfpathlineto{\pgfqpoint{2.726250in}{1.188576in}}%
\pgfpathlineto{\pgfqpoint{2.731417in}{1.229517in}}%
\pgfpathlineto{\pgfqpoint{2.741750in}{1.349675in}}%
\pgfpathlineto{\pgfqpoint{2.752083in}{1.458623in}}%
\pgfpathlineto{\pgfqpoint{2.757250in}{1.492988in}}%
\pgfpathlineto{\pgfqpoint{2.759833in}{1.503160in}}%
\pgfpathlineto{\pgfqpoint{2.762417in}{1.508172in}}%
\pgfpathlineto{\pgfqpoint{2.765000in}{1.507804in}}%
\pgfpathlineto{\pgfqpoint{2.767583in}{1.501967in}}%
\pgfpathlineto{\pgfqpoint{2.770167in}{1.490720in}}%
\pgfpathlineto{\pgfqpoint{2.775333in}{1.453129in}}%
\pgfpathlineto{\pgfqpoint{2.783083in}{1.369863in}}%
\pgfpathlineto{\pgfqpoint{2.788250in}{1.314275in}}%
\pgfpathlineto{\pgfqpoint{2.790833in}{1.294597in}}%
\pgfpathlineto{\pgfqpoint{2.793417in}{1.284779in}}%
\pgfpathlineto{\pgfqpoint{2.796000in}{1.286539in}}%
\pgfpathlineto{\pgfqpoint{2.798583in}{1.298776in}}%
\pgfpathlineto{\pgfqpoint{2.803750in}{1.341639in}}%
\pgfpathlineto{\pgfqpoint{2.811500in}{1.410044in}}%
\pgfpathlineto{\pgfqpoint{2.816667in}{1.441860in}}%
\pgfpathlineto{\pgfqpoint{2.821833in}{1.459139in}}%
\pgfpathlineto{\pgfqpoint{2.824417in}{1.462915in}}%
\pgfpathlineto{\pgfqpoint{2.827000in}{1.464218in}}%
\pgfpathlineto{\pgfqpoint{2.832167in}{1.462545in}}%
\pgfpathlineto{\pgfqpoint{2.837333in}{1.460491in}}%
\pgfpathlineto{\pgfqpoint{2.839917in}{1.460818in}}%
\pgfpathlineto{\pgfqpoint{2.845083in}{1.464616in}}%
\pgfpathlineto{\pgfqpoint{2.850250in}{1.468999in}}%
\pgfpathlineto{\pgfqpoint{2.852833in}{1.469132in}}%
\pgfpathlineto{\pgfqpoint{2.855417in}{1.466577in}}%
\pgfpathlineto{\pgfqpoint{2.858000in}{1.460523in}}%
\pgfpathlineto{\pgfqpoint{2.863167in}{1.436037in}}%
\pgfpathlineto{\pgfqpoint{2.868333in}{1.395201in}}%
\pgfpathlineto{\pgfqpoint{2.878667in}{1.297349in}}%
\pgfpathlineto{\pgfqpoint{2.881250in}{1.284933in}}%
\pgfpathlineto{\pgfqpoint{2.883833in}{1.284941in}}%
\pgfpathlineto{\pgfqpoint{2.886417in}{1.298366in}}%
\pgfpathlineto{\pgfqpoint{2.891583in}{1.354084in}}%
\pgfpathlineto{\pgfqpoint{2.904500in}{1.514933in}}%
\pgfpathlineto{\pgfqpoint{2.909667in}{1.553625in}}%
\pgfpathlineto{\pgfqpoint{2.912250in}{1.564696in}}%
\pgfpathlineto{\pgfqpoint{2.914833in}{1.569995in}}%
\pgfpathlineto{\pgfqpoint{2.917417in}{1.569595in}}%
\pgfpathlineto{\pgfqpoint{2.920000in}{1.563800in}}%
\pgfpathlineto{\pgfqpoint{2.925167in}{1.538410in}}%
\pgfpathlineto{\pgfqpoint{2.935500in}{1.459506in}}%
\pgfpathlineto{\pgfqpoint{2.943250in}{1.409833in}}%
\pgfpathlineto{\pgfqpoint{2.948417in}{1.390574in}}%
\pgfpathlineto{\pgfqpoint{2.953583in}{1.380409in}}%
\pgfpathlineto{\pgfqpoint{2.958750in}{1.375309in}}%
\pgfpathlineto{\pgfqpoint{2.963917in}{1.372878in}}%
\pgfpathlineto{\pgfqpoint{2.969083in}{1.372927in}}%
\pgfpathlineto{\pgfqpoint{2.974250in}{1.376583in}}%
\pgfpathlineto{\pgfqpoint{2.979417in}{1.385395in}}%
\pgfpathlineto{\pgfqpoint{2.984583in}{1.399843in}}%
\pgfpathlineto{\pgfqpoint{2.994917in}{1.433874in}}%
\pgfpathlineto{\pgfqpoint{3.000083in}{1.442674in}}%
\pgfpathlineto{\pgfqpoint{3.002667in}{1.442671in}}%
\pgfpathlineto{\pgfqpoint{3.005250in}{1.439113in}}%
\pgfpathlineto{\pgfqpoint{3.007833in}{1.431775in}}%
\pgfpathlineto{\pgfqpoint{3.013000in}{1.406118in}}%
\pgfpathlineto{\pgfqpoint{3.031083in}{1.290047in}}%
\pgfpathlineto{\pgfqpoint{3.033667in}{1.287411in}}%
\pgfpathlineto{\pgfqpoint{3.036250in}{1.290670in}}%
\pgfpathlineto{\pgfqpoint{3.038833in}{1.299038in}}%
\pgfpathlineto{\pgfqpoint{3.044000in}{1.325794in}}%
\pgfpathlineto{\pgfqpoint{3.054333in}{1.382110in}}%
\pgfpathlineto{\pgfqpoint{3.059500in}{1.396130in}}%
\pgfpathlineto{\pgfqpoint{3.062083in}{1.397753in}}%
\pgfpathlineto{\pgfqpoint{3.064667in}{1.395631in}}%
\pgfpathlineto{\pgfqpoint{3.067250in}{1.389899in}}%
\pgfpathlineto{\pgfqpoint{3.072417in}{1.369067in}}%
\pgfpathlineto{\pgfqpoint{3.090500in}{1.271707in}}%
\pgfpathlineto{\pgfqpoint{3.093083in}{1.266504in}}%
\pgfpathlineto{\pgfqpoint{3.095667in}{1.266092in}}%
\pgfpathlineto{\pgfqpoint{3.098250in}{1.270696in}}%
\pgfpathlineto{\pgfqpoint{3.103417in}{1.293057in}}%
\pgfpathlineto{\pgfqpoint{3.129250in}{1.444586in}}%
\pgfpathlineto{\pgfqpoint{3.139583in}{1.498457in}}%
\pgfpathlineto{\pgfqpoint{3.155083in}{1.590221in}}%
\pgfpathlineto{\pgfqpoint{3.160250in}{1.607254in}}%
\pgfpathlineto{\pgfqpoint{3.162833in}{1.610427in}}%
\pgfpathlineto{\pgfqpoint{3.165417in}{1.609470in}}%
\pgfpathlineto{\pgfqpoint{3.168000in}{1.604134in}}%
\pgfpathlineto{\pgfqpoint{3.173167in}{1.580233in}}%
\pgfpathlineto{\pgfqpoint{3.178333in}{1.540820in}}%
\pgfpathlineto{\pgfqpoint{3.188667in}{1.450175in}}%
\pgfpathlineto{\pgfqpoint{3.191250in}{1.436596in}}%
\pgfpathlineto{\pgfqpoint{3.193833in}{1.430971in}}%
\pgfpathlineto{\pgfqpoint{3.196417in}{1.434098in}}%
\pgfpathlineto{\pgfqpoint{3.199000in}{1.445282in}}%
\pgfpathlineto{\pgfqpoint{3.204167in}{1.483533in}}%
\pgfpathlineto{\pgfqpoint{3.211917in}{1.544956in}}%
\pgfpathlineto{\pgfqpoint{3.217083in}{1.569096in}}%
\pgfpathlineto{\pgfqpoint{3.219667in}{1.573752in}}%
\pgfpathlineto{\pgfqpoint{3.222250in}{1.573317in}}%
\pgfpathlineto{\pgfqpoint{3.224833in}{1.568102in}}%
\pgfpathlineto{\pgfqpoint{3.230000in}{1.545966in}}%
\pgfpathlineto{\pgfqpoint{3.245500in}{1.460697in}}%
\pgfpathlineto{\pgfqpoint{3.250667in}{1.448535in}}%
\pgfpathlineto{\pgfqpoint{3.255833in}{1.443486in}}%
\pgfpathlineto{\pgfqpoint{3.261000in}{1.438457in}}%
\pgfpathlineto{\pgfqpoint{3.266167in}{1.427611in}}%
\pgfpathlineto{\pgfqpoint{3.279083in}{1.385173in}}%
\pgfpathlineto{\pgfqpoint{3.281667in}{1.383914in}}%
\pgfpathlineto{\pgfqpoint{3.284250in}{1.388701in}}%
\pgfpathlineto{\pgfqpoint{3.286833in}{1.400057in}}%
\pgfpathlineto{\pgfqpoint{3.292000in}{1.439832in}}%
\pgfpathlineto{\pgfqpoint{3.307500in}{1.583870in}}%
\pgfpathlineto{\pgfqpoint{3.312667in}{1.605526in}}%
\pgfpathlineto{\pgfqpoint{3.315250in}{1.608319in}}%
\pgfpathlineto{\pgfqpoint{3.317833in}{1.605565in}}%
\pgfpathlineto{\pgfqpoint{3.320417in}{1.597350in}}%
\pgfpathlineto{\pgfqpoint{3.325583in}{1.565639in}}%
\pgfpathlineto{\pgfqpoint{3.330750in}{1.516580in}}%
\pgfpathlineto{\pgfqpoint{3.341083in}{1.387535in}}%
\pgfpathlineto{\pgfqpoint{3.348833in}{1.293076in}}%
\pgfpathlineto{\pgfqpoint{3.354000in}{1.254993in}}%
\pgfpathlineto{\pgfqpoint{3.356583in}{1.250905in}}%
\pgfpathlineto{\pgfqpoint{3.359167in}{1.258474in}}%
\pgfpathlineto{\pgfqpoint{3.364333in}{1.301244in}}%
\pgfpathlineto{\pgfqpoint{3.374667in}{1.431591in}}%
\pgfpathlineto{\pgfqpoint{3.385000in}{1.552405in}}%
\pgfpathlineto{\pgfqpoint{3.390167in}{1.596040in}}%
\pgfpathlineto{\pgfqpoint{3.395333in}{1.625307in}}%
\pgfpathlineto{\pgfqpoint{3.400500in}{1.639188in}}%
\pgfpathlineto{\pgfqpoint{3.403083in}{1.640182in}}%
\pgfpathlineto{\pgfqpoint{3.405667in}{1.637202in}}%
\pgfpathlineto{\pgfqpoint{3.408250in}{1.630329in}}%
\pgfpathlineto{\pgfqpoint{3.413417in}{1.605737in}}%
\pgfpathlineto{\pgfqpoint{3.421167in}{1.547302in}}%
\pgfpathlineto{\pgfqpoint{3.439250in}{1.374275in}}%
\pgfpathlineto{\pgfqpoint{3.459917in}{1.189184in}}%
\pgfpathlineto{\pgfqpoint{3.462500in}{1.179193in}}%
\pgfpathlineto{\pgfqpoint{3.465083in}{1.180024in}}%
\pgfpathlineto{\pgfqpoint{3.467667in}{1.190721in}}%
\pgfpathlineto{\pgfqpoint{3.483167in}{1.285404in}}%
\pgfpathlineto{\pgfqpoint{3.485750in}{1.292587in}}%
\pgfpathlineto{\pgfqpoint{3.488333in}{1.295852in}}%
\pgfpathlineto{\pgfqpoint{3.490917in}{1.295027in}}%
\pgfpathlineto{\pgfqpoint{3.493500in}{1.290064in}}%
\pgfpathlineto{\pgfqpoint{3.498667in}{1.268253in}}%
\pgfpathlineto{\pgfqpoint{3.506417in}{1.212977in}}%
\pgfpathlineto{\pgfqpoint{3.511583in}{1.177412in}}%
\pgfpathlineto{\pgfqpoint{3.514167in}{1.171788in}}%
\pgfpathlineto{\pgfqpoint{3.516750in}{1.180406in}}%
\pgfpathlineto{\pgfqpoint{3.521917in}{1.226887in}}%
\pgfpathlineto{\pgfqpoint{3.537417in}{1.395485in}}%
\pgfpathlineto{\pgfqpoint{3.545167in}{1.455273in}}%
\pgfpathlineto{\pgfqpoint{3.550333in}{1.481719in}}%
\pgfpathlineto{\pgfqpoint{3.555500in}{1.497124in}}%
\pgfpathlineto{\pgfqpoint{3.558083in}{1.500804in}}%
\pgfpathlineto{\pgfqpoint{3.560667in}{1.501947in}}%
\pgfpathlineto{\pgfqpoint{3.563250in}{1.500714in}}%
\pgfpathlineto{\pgfqpoint{3.565833in}{1.497303in}}%
\pgfpathlineto{\pgfqpoint{3.571000in}{1.484900in}}%
\pgfpathlineto{\pgfqpoint{3.578750in}{1.456557in}}%
\pgfpathlineto{\pgfqpoint{3.591667in}{1.407425in}}%
\pgfpathlineto{\pgfqpoint{3.596833in}{1.397632in}}%
\pgfpathlineto{\pgfqpoint{3.599417in}{1.396415in}}%
\pgfpathlineto{\pgfqpoint{3.602000in}{1.397801in}}%
\pgfpathlineto{\pgfqpoint{3.604583in}{1.401677in}}%
\pgfpathlineto{\pgfqpoint{3.609750in}{1.415723in}}%
\pgfpathlineto{\pgfqpoint{3.617500in}{1.446355in}}%
\pgfpathlineto{\pgfqpoint{3.630417in}{1.498577in}}%
\pgfpathlineto{\pgfqpoint{3.638167in}{1.522032in}}%
\pgfpathlineto{\pgfqpoint{3.645917in}{1.539286in}}%
\pgfpathlineto{\pgfqpoint{3.658833in}{1.565449in}}%
\pgfpathlineto{\pgfqpoint{3.669167in}{1.593916in}}%
\pgfpathlineto{\pgfqpoint{3.679500in}{1.622691in}}%
\pgfpathlineto{\pgfqpoint{3.684667in}{1.632113in}}%
\pgfpathlineto{\pgfqpoint{3.687250in}{1.634692in}}%
\pgfpathlineto{\pgfqpoint{3.689833in}{1.635576in}}%
\pgfpathlineto{\pgfqpoint{3.692417in}{1.634609in}}%
\pgfpathlineto{\pgfqpoint{3.695000in}{1.631675in}}%
\pgfpathlineto{\pgfqpoint{3.700167in}{1.619657in}}%
\pgfpathlineto{\pgfqpoint{3.705333in}{1.599388in}}%
\pgfpathlineto{\pgfqpoint{3.713083in}{1.554207in}}%
\pgfpathlineto{\pgfqpoint{3.720833in}{1.493287in}}%
\pgfpathlineto{\pgfqpoint{3.731167in}{1.394004in}}%
\pgfpathlineto{\pgfqpoint{3.749250in}{1.206553in}}%
\pgfpathlineto{\pgfqpoint{3.751833in}{1.191513in}}%
\pgfpathlineto{\pgfqpoint{3.754417in}{1.189744in}}%
\pgfpathlineto{\pgfqpoint{3.757000in}{1.203064in}}%
\pgfpathlineto{\pgfqpoint{3.762167in}{1.257886in}}%
\pgfpathlineto{\pgfqpoint{3.780250in}{1.482466in}}%
\pgfpathlineto{\pgfqpoint{3.788000in}{1.551720in}}%
\pgfpathlineto{\pgfqpoint{3.798333in}{1.620071in}}%
\pgfpathlineto{\pgfqpoint{3.808667in}{1.674887in}}%
\pgfpathlineto{\pgfqpoint{3.813833in}{1.696450in}}%
\pgfpathlineto{\pgfqpoint{3.819000in}{1.709706in}}%
\pgfpathlineto{\pgfqpoint{3.821583in}{1.711855in}}%
\pgfpathlineto{\pgfqpoint{3.824167in}{1.710354in}}%
\pgfpathlineto{\pgfqpoint{3.826750in}{1.704819in}}%
\pgfpathlineto{\pgfqpoint{3.829333in}{1.694993in}}%
\pgfpathlineto{\pgfqpoint{3.834500in}{1.662237in}}%
\pgfpathlineto{\pgfqpoint{3.839667in}{1.613299in}}%
\pgfpathlineto{\pgfqpoint{3.850000in}{1.484505in}}%
\pgfpathlineto{\pgfqpoint{3.857750in}{1.391354in}}%
\pgfpathlineto{\pgfqpoint{3.862917in}{1.353122in}}%
\pgfpathlineto{\pgfqpoint{3.865500in}{1.346400in}}%
\pgfpathlineto{\pgfqpoint{3.868083in}{1.349132in}}%
\pgfpathlineto{\pgfqpoint{3.870667in}{1.360843in}}%
\pgfpathlineto{\pgfqpoint{3.875833in}{1.405407in}}%
\pgfpathlineto{\pgfqpoint{3.883583in}{1.502551in}}%
\pgfpathlineto{\pgfqpoint{3.906833in}{1.814884in}}%
\pgfpathlineto{\pgfqpoint{3.912000in}{1.859742in}}%
\pgfpathlineto{\pgfqpoint{3.917167in}{1.887328in}}%
\pgfpathlineto{\pgfqpoint{3.919750in}{1.894022in}}%
\pgfpathlineto{\pgfqpoint{3.922333in}{1.895832in}}%
\pgfpathlineto{\pgfqpoint{3.924917in}{1.892746in}}%
\pgfpathlineto{\pgfqpoint{3.927500in}{1.884830in}}%
\pgfpathlineto{\pgfqpoint{3.932667in}{1.855151in}}%
\pgfpathlineto{\pgfqpoint{3.937833in}{1.808815in}}%
\pgfpathlineto{\pgfqpoint{3.945583in}{1.715042in}}%
\pgfpathlineto{\pgfqpoint{3.961083in}{1.486007in}}%
\pgfpathlineto{\pgfqpoint{3.984333in}{1.147271in}}%
\pgfpathlineto{\pgfqpoint{3.986917in}{1.145905in}}%
\pgfpathlineto{\pgfqpoint{3.992083in}{1.207633in}}%
\pgfpathlineto{\pgfqpoint{4.005000in}{1.379214in}}%
\pgfpathlineto{\pgfqpoint{4.012750in}{1.454119in}}%
\pgfpathlineto{\pgfqpoint{4.017917in}{1.486561in}}%
\pgfpathlineto{\pgfqpoint{4.023083in}{1.503587in}}%
\pgfpathlineto{\pgfqpoint{4.025667in}{1.506451in}}%
\pgfpathlineto{\pgfqpoint{4.028250in}{1.505864in}}%
\pgfpathlineto{\pgfqpoint{4.030833in}{1.502202in}}%
\pgfpathlineto{\pgfqpoint{4.036000in}{1.487515in}}%
\pgfpathlineto{\pgfqpoint{4.043750in}{1.453837in}}%
\pgfpathlineto{\pgfqpoint{4.054083in}{1.397565in}}%
\pgfpathlineto{\pgfqpoint{4.064417in}{1.339866in}}%
\pgfpathlineto{\pgfqpoint{4.067000in}{1.330625in}}%
\pgfpathlineto{\pgfqpoint{4.069583in}{1.326464in}}%
\pgfpathlineto{\pgfqpoint{4.072167in}{1.328994in}}%
\pgfpathlineto{\pgfqpoint{4.074750in}{1.339227in}}%
\pgfpathlineto{\pgfqpoint{4.079917in}{1.382360in}}%
\pgfpathlineto{\pgfqpoint{4.087667in}{1.487056in}}%
\pgfpathlineto{\pgfqpoint{4.108333in}{1.794573in}}%
\pgfpathlineto{\pgfqpoint{4.116083in}{1.874745in}}%
\pgfpathlineto{\pgfqpoint{4.121250in}{1.912751in}}%
\pgfpathlineto{\pgfqpoint{4.126417in}{1.937262in}}%
\pgfpathlineto{\pgfqpoint{4.129000in}{1.944007in}}%
\pgfpathlineto{\pgfqpoint{4.131583in}{1.946815in}}%
\pgfpathlineto{\pgfqpoint{4.134167in}{1.945525in}}%
\pgfpathlineto{\pgfqpoint{4.136750in}{1.940032in}}%
\pgfpathlineto{\pgfqpoint{4.141917in}{1.916376in}}%
\pgfpathlineto{\pgfqpoint{4.147083in}{1.876383in}}%
\pgfpathlineto{\pgfqpoint{4.154833in}{1.788920in}}%
\pgfpathlineto{\pgfqpoint{4.165167in}{1.632345in}}%
\pgfpathlineto{\pgfqpoint{4.180667in}{1.388829in}}%
\pgfpathlineto{\pgfqpoint{4.183250in}{1.366708in}}%
\pgfpathlineto{\pgfqpoint{4.185833in}{1.357494in}}%
\pgfpathlineto{\pgfqpoint{4.188417in}{1.363042in}}%
\pgfpathlineto{\pgfqpoint{4.191000in}{1.382642in}}%
\pgfpathlineto{\pgfqpoint{4.196167in}{1.452226in}}%
\pgfpathlineto{\pgfqpoint{4.224583in}{1.917436in}}%
\pgfpathlineto{\pgfqpoint{4.232333in}{1.989818in}}%
\pgfpathlineto{\pgfqpoint{4.237500in}{2.017084in}}%
\pgfpathlineto{\pgfqpoint{4.240083in}{2.023878in}}%
\pgfpathlineto{\pgfqpoint{4.242667in}{2.025975in}}%
\pgfpathlineto{\pgfqpoint{4.245250in}{2.023358in}}%
\pgfpathlineto{\pgfqpoint{4.247833in}{2.016092in}}%
\pgfpathlineto{\pgfqpoint{4.253000in}{1.988352in}}%
\pgfpathlineto{\pgfqpoint{4.258167in}{1.945162in}}%
\pgfpathlineto{\pgfqpoint{4.265917in}{1.859230in}}%
\pgfpathlineto{\pgfqpoint{4.281417in}{1.651657in}}%
\pgfpathlineto{\pgfqpoint{4.299500in}{1.379753in}}%
\pgfpathlineto{\pgfqpoint{4.315000in}{1.137518in}}%
\pgfpathlineto{\pgfqpoint{4.317583in}{1.110923in}}%
\pgfpathlineto{\pgfqpoint{4.325333in}{1.202754in}}%
\pgfpathlineto{\pgfqpoint{4.338250in}{1.340884in}}%
\pgfpathlineto{\pgfqpoint{4.351167in}{1.475895in}}%
\pgfpathlineto{\pgfqpoint{4.379583in}{1.805315in}}%
\pgfpathlineto{\pgfqpoint{4.384750in}{1.840561in}}%
\pgfpathlineto{\pgfqpoint{4.389917in}{1.863252in}}%
\pgfpathlineto{\pgfqpoint{4.395083in}{1.874634in}}%
\pgfpathlineto{\pgfqpoint{4.397667in}{1.876783in}}%
\pgfpathlineto{\pgfqpoint{4.400250in}{1.877011in}}%
\pgfpathlineto{\pgfqpoint{4.402833in}{1.875643in}}%
\pgfpathlineto{\pgfqpoint{4.408000in}{1.869146in}}%
\pgfpathlineto{\pgfqpoint{4.413167in}{1.858443in}}%
\pgfpathlineto{\pgfqpoint{4.418333in}{1.843143in}}%
\pgfpathlineto{\pgfqpoint{4.423500in}{1.821771in}}%
\pgfpathlineto{\pgfqpoint{4.428667in}{1.792570in}}%
\pgfpathlineto{\pgfqpoint{4.436417in}{1.731809in}}%
\pgfpathlineto{\pgfqpoint{4.446750in}{1.623292in}}%
\pgfpathlineto{\pgfqpoint{4.459667in}{1.483451in}}%
\pgfpathlineto{\pgfqpoint{4.464833in}{1.446588in}}%
\pgfpathlineto{\pgfqpoint{4.467417in}{1.435941in}}%
\pgfpathlineto{\pgfqpoint{4.470000in}{1.431074in}}%
\pgfpathlineto{\pgfqpoint{4.472583in}{1.431767in}}%
\pgfpathlineto{\pgfqpoint{4.475167in}{1.437254in}}%
\pgfpathlineto{\pgfqpoint{4.480333in}{1.457816in}}%
\pgfpathlineto{\pgfqpoint{4.490667in}{1.504014in}}%
\pgfpathlineto{\pgfqpoint{4.495833in}{1.517376in}}%
\pgfpathlineto{\pgfqpoint{4.498417in}{1.520439in}}%
\pgfpathlineto{\pgfqpoint{4.501000in}{1.520985in}}%
\pgfpathlineto{\pgfqpoint{4.503583in}{1.519040in}}%
\pgfpathlineto{\pgfqpoint{4.506167in}{1.514682in}}%
\pgfpathlineto{\pgfqpoint{4.511333in}{1.499324in}}%
\pgfpathlineto{\pgfqpoint{4.519083in}{1.464306in}}%
\pgfpathlineto{\pgfqpoint{4.526833in}{1.431706in}}%
\pgfpathlineto{\pgfqpoint{4.529417in}{1.426926in}}%
\pgfpathlineto{\pgfqpoint{4.532000in}{1.427403in}}%
\pgfpathlineto{\pgfqpoint{4.534583in}{1.433976in}}%
\pgfpathlineto{\pgfqpoint{4.537167in}{1.446849in}}%
\pgfpathlineto{\pgfqpoint{4.542333in}{1.489186in}}%
\pgfpathlineto{\pgfqpoint{4.552667in}{1.607988in}}%
\pgfpathlineto{\pgfqpoint{4.563000in}{1.715968in}}%
\pgfpathlineto{\pgfqpoint{4.568167in}{1.753030in}}%
\pgfpathlineto{\pgfqpoint{4.573333in}{1.776316in}}%
\pgfpathlineto{\pgfqpoint{4.575917in}{1.782825in}}%
\pgfpathlineto{\pgfqpoint{4.578500in}{1.786094in}}%
\pgfpathlineto{\pgfqpoint{4.581083in}{1.786351in}}%
\pgfpathlineto{\pgfqpoint{4.583667in}{1.783880in}}%
\pgfpathlineto{\pgfqpoint{4.588833in}{1.772082in}}%
\pgfpathlineto{\pgfqpoint{4.594000in}{1.753380in}}%
\pgfpathlineto{\pgfqpoint{4.601750in}{1.716519in}}%
\pgfpathlineto{\pgfqpoint{4.609500in}{1.669943in}}%
\pgfpathlineto{\pgfqpoint{4.627583in}{1.552667in}}%
\pgfpathlineto{\pgfqpoint{4.630167in}{1.546807in}}%
\pgfpathlineto{\pgfqpoint{4.632750in}{1.546854in}}%
\pgfpathlineto{\pgfqpoint{4.635333in}{1.553053in}}%
\pgfpathlineto{\pgfqpoint{4.640500in}{1.581703in}}%
\pgfpathlineto{\pgfqpoint{4.650833in}{1.670746in}}%
\pgfpathlineto{\pgfqpoint{4.658583in}{1.732607in}}%
\pgfpathlineto{\pgfqpoint{4.663750in}{1.763261in}}%
\pgfpathlineto{\pgfqpoint{4.668917in}{1.783734in}}%
\pgfpathlineto{\pgfqpoint{4.674083in}{1.794304in}}%
\pgfpathlineto{\pgfqpoint{4.676667in}{1.796323in}}%
\pgfpathlineto{\pgfqpoint{4.679250in}{1.796528in}}%
\pgfpathlineto{\pgfqpoint{4.684417in}{1.792872in}}%
\pgfpathlineto{\pgfqpoint{4.692167in}{1.784090in}}%
\pgfpathlineto{\pgfqpoint{4.694750in}{1.782804in}}%
\pgfpathlineto{\pgfqpoint{4.697333in}{1.783644in}}%
\pgfpathlineto{\pgfqpoint{4.699917in}{1.787381in}}%
\pgfpathlineto{\pgfqpoint{4.702500in}{1.794651in}}%
\pgfpathlineto{\pgfqpoint{4.707667in}{1.821259in}}%
\pgfpathlineto{\pgfqpoint{4.712833in}{1.863960in}}%
\pgfpathlineto{\pgfqpoint{4.720583in}{1.950959in}}%
\pgfpathlineto{\pgfqpoint{4.733500in}{2.105572in}}%
\pgfpathlineto{\pgfqpoint{4.738667in}{2.150174in}}%
\pgfpathlineto{\pgfqpoint{4.743833in}{2.175714in}}%
\pgfpathlineto{\pgfqpoint{4.746417in}{2.180091in}}%
\pgfpathlineto{\pgfqpoint{4.749000in}{2.178519in}}%
\pgfpathlineto{\pgfqpoint{4.751583in}{2.170943in}}%
\pgfpathlineto{\pgfqpoint{4.756750in}{2.138361in}}%
\pgfpathlineto{\pgfqpoint{4.761917in}{2.084850in}}%
\pgfpathlineto{\pgfqpoint{4.769667in}{1.974561in}}%
\pgfpathlineto{\pgfqpoint{4.787750in}{1.694084in}}%
\pgfpathlineto{\pgfqpoint{4.792917in}{1.653391in}}%
\pgfpathlineto{\pgfqpoint{4.795500in}{1.646198in}}%
\pgfpathlineto{\pgfqpoint{4.798083in}{1.648352in}}%
\pgfpathlineto{\pgfqpoint{4.800667in}{1.659480in}}%
\pgfpathlineto{\pgfqpoint{4.805833in}{1.704428in}}%
\pgfpathlineto{\pgfqpoint{4.813583in}{1.806594in}}%
\pgfpathlineto{\pgfqpoint{4.826500in}{1.984232in}}%
\pgfpathlineto{\pgfqpoint{4.834250in}{2.058735in}}%
\pgfpathlineto{\pgfqpoint{4.839417in}{2.087300in}}%
\pgfpathlineto{\pgfqpoint{4.842000in}{2.094640in}}%
\pgfpathlineto{\pgfqpoint{4.844583in}{2.097321in}}%
\pgfpathlineto{\pgfqpoint{4.847167in}{2.095437in}}%
\pgfpathlineto{\pgfqpoint{4.849750in}{2.089137in}}%
\pgfpathlineto{\pgfqpoint{4.854917in}{2.064045in}}%
\pgfpathlineto{\pgfqpoint{4.860083in}{2.023737in}}%
\pgfpathlineto{\pgfqpoint{4.867833in}{1.939941in}}%
\pgfpathlineto{\pgfqpoint{4.891083in}{1.655054in}}%
\pgfpathlineto{\pgfqpoint{4.896250in}{1.614602in}}%
\pgfpathlineto{\pgfqpoint{4.901417in}{1.587454in}}%
\pgfpathlineto{\pgfqpoint{4.906583in}{1.572640in}}%
\pgfpathlineto{\pgfqpoint{4.909167in}{1.569239in}}%
\pgfpathlineto{\pgfqpoint{4.911750in}{1.568093in}}%
\pgfpathlineto{\pgfqpoint{4.914333in}{1.568848in}}%
\pgfpathlineto{\pgfqpoint{4.919500in}{1.574447in}}%
\pgfpathlineto{\pgfqpoint{4.929833in}{1.588708in}}%
\pgfpathlineto{\pgfqpoint{4.932417in}{1.590186in}}%
\pgfpathlineto{\pgfqpoint{4.935000in}{1.590131in}}%
\pgfpathlineto{\pgfqpoint{4.937583in}{1.588373in}}%
\pgfpathlineto{\pgfqpoint{4.942750in}{1.579629in}}%
\pgfpathlineto{\pgfqpoint{4.947917in}{1.564652in}}%
\pgfpathlineto{\pgfqpoint{4.955667in}{1.534300in}}%
\pgfpathlineto{\pgfqpoint{4.976333in}{1.449092in}}%
\pgfpathlineto{\pgfqpoint{4.986667in}{1.419413in}}%
\pgfpathlineto{\pgfqpoint{4.997000in}{1.388950in}}%
\pgfpathlineto{\pgfqpoint{5.007333in}{1.349116in}}%
\pgfpathlineto{\pgfqpoint{5.022833in}{1.288801in}}%
\pgfpathlineto{\pgfqpoint{5.030583in}{1.266695in}}%
\pgfpathlineto{\pgfqpoint{5.040917in}{1.244086in}}%
\pgfpathlineto{\pgfqpoint{5.051250in}{1.226065in}}%
\pgfpathlineto{\pgfqpoint{5.059000in}{1.216190in}}%
\pgfpathlineto{\pgfqpoint{5.064167in}{1.212702in}}%
\pgfpathlineto{\pgfqpoint{5.066750in}{1.212351in}}%
\pgfpathlineto{\pgfqpoint{5.069333in}{1.213154in}}%
\pgfpathlineto{\pgfqpoint{5.074500in}{1.218729in}}%
\pgfpathlineto{\pgfqpoint{5.079667in}{1.229897in}}%
\pgfpathlineto{\pgfqpoint{5.087417in}{1.255475in}}%
\pgfpathlineto{\pgfqpoint{5.100333in}{1.301667in}}%
\pgfpathlineto{\pgfqpoint{5.105500in}{1.312848in}}%
\pgfpathlineto{\pgfqpoint{5.108083in}{1.315908in}}%
\pgfpathlineto{\pgfqpoint{5.110667in}{1.317227in}}%
\pgfpathlineto{\pgfqpoint{5.113250in}{1.316885in}}%
\pgfpathlineto{\pgfqpoint{5.118417in}{1.311949in}}%
\pgfpathlineto{\pgfqpoint{5.126167in}{1.298247in}}%
\pgfpathlineto{\pgfqpoint{5.133917in}{1.285658in}}%
\pgfpathlineto{\pgfqpoint{5.139083in}{1.282526in}}%
\pgfpathlineto{\pgfqpoint{5.141667in}{1.283226in}}%
\pgfpathlineto{\pgfqpoint{5.144250in}{1.285527in}}%
\pgfpathlineto{\pgfqpoint{5.149417in}{1.294630in}}%
\pgfpathlineto{\pgfqpoint{5.157167in}{1.316888in}}%
\pgfpathlineto{\pgfqpoint{5.170083in}{1.365240in}}%
\pgfpathlineto{\pgfqpoint{5.183000in}{1.412458in}}%
\pgfpathlineto{\pgfqpoint{5.188167in}{1.424645in}}%
\pgfpathlineto{\pgfqpoint{5.190750in}{1.428118in}}%
\pgfpathlineto{\pgfqpoint{5.193333in}{1.429501in}}%
\pgfpathlineto{\pgfqpoint{5.195917in}{1.428597in}}%
\pgfpathlineto{\pgfqpoint{5.198500in}{1.425272in}}%
\pgfpathlineto{\pgfqpoint{5.203667in}{1.411165in}}%
\pgfpathlineto{\pgfqpoint{5.208833in}{1.387474in}}%
\pgfpathlineto{\pgfqpoint{5.216583in}{1.336872in}}%
\pgfpathlineto{\pgfqpoint{5.237250in}{1.178542in}}%
\pgfpathlineto{\pgfqpoint{5.239833in}{1.173557in}}%
\pgfpathlineto{\pgfqpoint{5.242417in}{1.179194in}}%
\pgfpathlineto{\pgfqpoint{5.247583in}{1.215815in}}%
\pgfpathlineto{\pgfqpoint{5.255333in}{1.300623in}}%
\pgfpathlineto{\pgfqpoint{5.278583in}{1.583447in}}%
\pgfpathlineto{\pgfqpoint{5.283750in}{1.623597in}}%
\pgfpathlineto{\pgfqpoint{5.288917in}{1.646870in}}%
\pgfpathlineto{\pgfqpoint{5.291500in}{1.651633in}}%
\pgfpathlineto{\pgfqpoint{5.294083in}{1.651730in}}%
\pgfpathlineto{\pgfqpoint{5.296667in}{1.647207in}}%
\pgfpathlineto{\pgfqpoint{5.299250in}{1.638174in}}%
\pgfpathlineto{\pgfqpoint{5.304417in}{1.607281in}}%
\pgfpathlineto{\pgfqpoint{5.309583in}{1.560941in}}%
\pgfpathlineto{\pgfqpoint{5.317333in}{1.468818in}}%
\pgfpathlineto{\pgfqpoint{5.338000in}{1.200780in}}%
\pgfpathlineto{\pgfqpoint{5.340583in}{1.191543in}}%
\pgfpathlineto{\pgfqpoint{5.343167in}{1.200107in}}%
\pgfpathlineto{\pgfqpoint{5.348333in}{1.254665in}}%
\pgfpathlineto{\pgfqpoint{5.358667in}{1.413230in}}%
\pgfpathlineto{\pgfqpoint{5.381917in}{1.786231in}}%
\pgfpathlineto{\pgfqpoint{5.389667in}{1.880131in}}%
\pgfpathlineto{\pgfqpoint{5.394833in}{1.926113in}}%
\pgfpathlineto{\pgfqpoint{5.400000in}{1.957414in}}%
\pgfpathlineto{\pgfqpoint{5.402583in}{1.967841in}}%
\pgfpathlineto{\pgfqpoint{5.402583in}{1.967841in}}%
\pgfusepath{stroke}%
\end{pgfscope}%
\begin{pgfscope}%
\pgfsetrectcap%
\pgfsetmiterjoin%
\pgfsetlinewidth{0.803000pt}%
\definecolor{currentstroke}{rgb}{0.737255,0.737255,0.737255}%
\pgfsetstrokecolor{currentstroke}%
\pgfsetdash{}{0pt}%
\pgfpathmoveto{\pgfqpoint{0.750000in}{0.330000in}}%
\pgfpathlineto{\pgfqpoint{0.750000in}{2.640000in}}%
\pgfusepath{stroke}%
\end{pgfscope}%
\begin{pgfscope}%
\pgfsetrectcap%
\pgfsetmiterjoin%
\pgfsetlinewidth{0.803000pt}%
\definecolor{currentstroke}{rgb}{0.737255,0.737255,0.737255}%
\pgfsetstrokecolor{currentstroke}%
\pgfsetdash{}{0pt}%
\pgfpathmoveto{\pgfqpoint{5.400000in}{0.330000in}}%
\pgfpathlineto{\pgfqpoint{5.400000in}{2.640000in}}%
\pgfusepath{stroke}%
\end{pgfscope}%
\begin{pgfscope}%
\pgfsetrectcap%
\pgfsetmiterjoin%
\pgfsetlinewidth{0.803000pt}%
\definecolor{currentstroke}{rgb}{0.737255,0.737255,0.737255}%
\pgfsetstrokecolor{currentstroke}%
\pgfsetdash{}{0pt}%
\pgfpathmoveto{\pgfqpoint{0.750000in}{0.330000in}}%
\pgfpathlineto{\pgfqpoint{5.400000in}{0.330000in}}%
\pgfusepath{stroke}%
\end{pgfscope}%
\begin{pgfscope}%
\pgfsetrectcap%
\pgfsetmiterjoin%
\pgfsetlinewidth{0.803000pt}%
\definecolor{currentstroke}{rgb}{0.737255,0.737255,0.737255}%
\pgfsetstrokecolor{currentstroke}%
\pgfsetdash{}{0pt}%
\pgfpathmoveto{\pgfqpoint{0.750000in}{2.640000in}}%
\pgfpathlineto{\pgfqpoint{5.400000in}{2.640000in}}%
\pgfusepath{stroke}%
\end{pgfscope}%
\begin{pgfscope}%
\pgfsetbuttcap%
\pgfsetmiterjoin%
\definecolor{currentfill}{rgb}{0.933333,0.933333,0.933333}%
\pgfsetfillcolor{currentfill}%
\pgfsetfillopacity{0.800000}%
\pgfsetlinewidth{0.501875pt}%
\definecolor{currentstroke}{rgb}{0.800000,0.800000,0.800000}%
\pgfsetstrokecolor{currentstroke}%
\pgfsetstrokeopacity{0.800000}%
\pgfsetdash{}{0pt}%
\pgfpathmoveto{\pgfqpoint{0.847222in}{1.737362in}}%
\pgfpathlineto{\pgfqpoint{2.838334in}{1.737362in}}%
\pgfpathquadraticcurveto{\pgfqpoint{2.866111in}{1.737362in}}{\pgfqpoint{2.866111in}{1.765140in}}%
\pgfpathlineto{\pgfqpoint{2.866111in}{2.542778in}}%
\pgfpathquadraticcurveto{\pgfqpoint{2.866111in}{2.570556in}}{\pgfqpoint{2.838334in}{2.570556in}}%
\pgfpathlineto{\pgfqpoint{0.847222in}{2.570556in}}%
\pgfpathquadraticcurveto{\pgfqpoint{0.819444in}{2.570556in}}{\pgfqpoint{0.819444in}{2.542778in}}%
\pgfpathlineto{\pgfqpoint{0.819444in}{1.765140in}}%
\pgfpathquadraticcurveto{\pgfqpoint{0.819444in}{1.737362in}}{\pgfqpoint{0.847222in}{1.737362in}}%
\pgfpathlineto{\pgfqpoint{0.847222in}{1.737362in}}%
\pgfpathclose%
\pgfusepath{stroke,fill}%
\end{pgfscope}%
\begin{pgfscope}%
\pgfsetrectcap%
\pgfsetroundjoin%
\pgfsetlinewidth{1.505625pt}%
\definecolor{currentstroke}{rgb}{0.501961,0.501961,0.501961}%
\pgfsetstrokecolor{currentstroke}%
\pgfsetdash{}{0pt}%
\pgfpathmoveto{\pgfqpoint{0.875000in}{2.465694in}}%
\pgfpathlineto{\pgfqpoint{1.013889in}{2.465694in}}%
\pgfpathlineto{\pgfqpoint{1.152778in}{2.465694in}}%
\pgfusepath{stroke}%
\end{pgfscope}%
\begin{pgfscope}%
\definecolor{textcolor}{rgb}{0.000000,0.000000,0.000000}%
\pgfsetstrokecolor{textcolor}%
\pgfsetfillcolor{textcolor}%
\pgftext[x=1.263889in,y=2.417083in,left,base]{\color{textcolor}\rmfamily\fontsize{10.000000}{12.000000}\selectfont Signal}%
\end{pgfscope}%
\begin{pgfscope}%
\pgfsetbuttcap%
\pgfsetroundjoin%
\pgfsetlinewidth{1.003750pt}%
\definecolor{currentstroke}{rgb}{0.000000,0.000000,0.000000}%
\pgfsetstrokecolor{currentstroke}%
\pgfsetdash{{1.000000pt}{1.650000pt}}{0.000000pt}%
\pgfpathmoveto{\pgfqpoint{0.875000in}{2.270417in}}%
\pgfpathlineto{\pgfqpoint{1.013889in}{2.270417in}}%
\pgfpathlineto{\pgfqpoint{1.152778in}{2.270417in}}%
\pgfusepath{stroke}%
\end{pgfscope}%
\begin{pgfscope}%
\definecolor{textcolor}{rgb}{0.000000,0.000000,0.000000}%
\pgfsetstrokecolor{textcolor}%
\pgfsetfillcolor{textcolor}%
\pgftext[x=1.263889in,y=2.221806in,left,base]{\color{textcolor}\rmfamily\fontsize{10.000000}{12.000000}\selectfont Valeur absolue}%
\end{pgfscope}%
\begin{pgfscope}%
\pgfsetrectcap%
\pgfsetroundjoin%
\pgfsetlinewidth{2.509375pt}%
\definecolor{currentstroke}{rgb}{1.000000,0.000000,0.000000}%
\pgfsetstrokecolor{currentstroke}%
\pgfsetdash{}{0pt}%
\pgfpathmoveto{\pgfqpoint{0.875000in}{2.069862in}}%
\pgfpathlineto{\pgfqpoint{1.013889in}{2.069862in}}%
\pgfpathlineto{\pgfqpoint{1.152778in}{2.069862in}}%
\pgfusepath{stroke}%
\end{pgfscope}%
\begin{pgfscope}%
\definecolor{textcolor}{rgb}{0.000000,0.000000,0.000000}%
\pgfsetstrokecolor{textcolor}%
\pgfsetfillcolor{textcolor}%
\pgftext[x=1.263889in,y=2.021250in,left,base]{\color{textcolor}\rmfamily\fontsize{10.000000}{12.000000}\selectfont Enveloppe sup. (approx.)}%
\end{pgfscope}%
\begin{pgfscope}%
\pgfsetrectcap%
\pgfsetroundjoin%
\pgfsetlinewidth{2.509375pt}%
\definecolor{currentstroke}{rgb}{0.000000,0.501961,0.000000}%
\pgfsetstrokecolor{currentstroke}%
\pgfsetstrokeopacity{0.700000}%
\pgfsetdash{}{0pt}%
\pgfpathmoveto{\pgfqpoint{0.875000in}{1.868473in}}%
\pgfpathlineto{\pgfqpoint{1.013889in}{1.868473in}}%
\pgfpathlineto{\pgfqpoint{1.152778in}{1.868473in}}%
\pgfusepath{stroke}%
\end{pgfscope}%
\begin{pgfscope}%
\definecolor{textcolor}{rgb}{0.000000,0.000000,0.000000}%
\pgfsetstrokecolor{textcolor}%
\pgfsetfillcolor{textcolor}%
\pgftext[x=1.263889in,y=1.819862in,left,base]{\color{textcolor}\rmfamily\fontsize{10.000000}{12.000000}\selectfont Enveloppe sup.}%
\end{pgfscope}%
\end{pgfpicture}%
\makeatother%
\endgroup%
}
    \caption{Comparaison de l'enveloppe du signal (gris) obtenue à partir d'une transformée de Hilbert (vert) ou d'une méthode géométrique (rouge). La valeur absolue (noir pointillé) permet de distinguer la nuance avec l'enveloppe.}
    \label{fig:env-approx}
\end{figure}

D'autres auteurs utilisent l'enveloppe classique appelée enveloppe supérieure, une courbe lisse qui décrit les amplitudes extrêmes du signal. Elle est définie par rapport à la transformation de Hilbert. Mais elle peut être approximée par des méthodes géométriques, par exemple en gardant la valeur absolue d'extremums locaux du signal. 

En ne gardant que les points correspondant aux extremums locaux du signal initial et sans faire d'interpolation, nous réduisons grandement le nombre de points. Une autre conséquence est que les points ne sont plus espacés linéairement ce qui rajoute de la complexité et une source d'erreur pour la suite des algorithmes.

\subsubsection{Enveloppe de Allen}

\cite{allen1982} définit une nouvelle enveloppe [d'après \cite{kuperkoch2010}]
\begin{equation}
   E_i^2 = x_i^2 + C_i \times (x_i^2-x_{i-1}^2)
\end{equation}
avec $C_i$ un coefficient tel que
\begin{equation}
   C_i = \frac{\sum_{j=1}^{i}{|x_j|}}{\sum_{j=1}^{i}{|x_j-x_{j-1}|}}
\end{equation}

Cette enveloppe approxime l'enveloppe supérieure sans pour autant réduire le nombre de points ni même d'avoir à réaliser une transformée de Hilbert (figure \ref{fig:env-comp}).

\subsubsection{Enveloppe de Baer et Kradolfer}

\cite{baer1987} proposent une amélioration de l'enveloppe d'Allen définit ainsi (figure \ref{fig:env-comp}) [d'après \cite{kuperkoch2010}]
\begin{equation}
   E_i^2 = x_i^2 + C_i \times (\dot{x_i}^2)
\end{equation}
avec $C_i$ un coefficient tel que
\begin{equation}
   C_i = \frac{\sum_{j=1}^{i}{x_j^2}}{\sum_{j=1}^{i}{\dot{x_j}^2}}
\end{equation}

\begin{figure}[ht]
    \centering
    \scalebox{1}{%% Creator: Matplotlib, PGF backend
%%
%% To include the figure in your LaTeX document, write
%%   \input{<filename>.pgf}
%%
%% Make sure the required packages are loaded in your preamble
%%   \usepackage{pgf}
%%
%% Also ensure that all the required font packages are loaded; for instance,
%% the lmodern package is sometimes necessary when using math font.
%%   \usepackage{lmodern}
%%
%% Figures using additional raster images can only be included by \input if
%% they are in the same directory as the main LaTeX file. For loading figures
%% from other directories you can use the `import` package
%%   \usepackage{import}
%%
%% and then include the figures with
%%   \import{<path to file>}{<filename>.pgf}
%%
%% Matplotlib used the following preamble
%%   \usepackage{fontspec}
%%
\begingroup%
\makeatletter%
\begin{pgfpicture}%
\pgfpathrectangle{\pgfpointorigin}{\pgfqpoint{6.000000in}{3.000000in}}%
\pgfusepath{use as bounding box, clip}%
\begin{pgfscope}%
\pgfsetbuttcap%
\pgfsetmiterjoin%
\definecolor{currentfill}{rgb}{1.000000,1.000000,1.000000}%
\pgfsetfillcolor{currentfill}%
\pgfsetlinewidth{0.000000pt}%
\definecolor{currentstroke}{rgb}{1.000000,1.000000,1.000000}%
\pgfsetstrokecolor{currentstroke}%
\pgfsetdash{}{0pt}%
\pgfpathmoveto{\pgfqpoint{0.000000in}{0.000000in}}%
\pgfpathlineto{\pgfqpoint{6.000000in}{0.000000in}}%
\pgfpathlineto{\pgfqpoint{6.000000in}{3.000000in}}%
\pgfpathlineto{\pgfqpoint{0.000000in}{3.000000in}}%
\pgfpathlineto{\pgfqpoint{0.000000in}{0.000000in}}%
\pgfpathclose%
\pgfusepath{fill}%
\end{pgfscope}%
\begin{pgfscope}%
\pgfsetbuttcap%
\pgfsetmiterjoin%
\definecolor{currentfill}{rgb}{0.933333,0.933333,0.933333}%
\pgfsetfillcolor{currentfill}%
\pgfsetlinewidth{0.000000pt}%
\definecolor{currentstroke}{rgb}{0.000000,0.000000,0.000000}%
\pgfsetstrokecolor{currentstroke}%
\pgfsetstrokeopacity{0.000000}%
\pgfsetdash{}{0pt}%
\pgfpathmoveto{\pgfqpoint{0.750000in}{0.330000in}}%
\pgfpathlineto{\pgfqpoint{4.470000in}{0.330000in}}%
\pgfpathlineto{\pgfqpoint{4.470000in}{2.640000in}}%
\pgfpathlineto{\pgfqpoint{0.750000in}{2.640000in}}%
\pgfpathlineto{\pgfqpoint{0.750000in}{0.330000in}}%
\pgfpathclose%
\pgfusepath{fill}%
\end{pgfscope}%
\begin{pgfscope}%
\pgfpathrectangle{\pgfqpoint{0.750000in}{0.330000in}}{\pgfqpoint{3.720000in}{2.310000in}}%
\pgfusepath{clip}%
\pgfsetbuttcap%
\pgfsetroundjoin%
\pgfsetlinewidth{0.501875pt}%
\definecolor{currentstroke}{rgb}{0.698039,0.698039,0.698039}%
\pgfsetstrokecolor{currentstroke}%
\pgfsetdash{{1.850000pt}{0.800000pt}}{0.000000pt}%
\pgfpathmoveto{\pgfqpoint{0.750000in}{0.330000in}}%
\pgfpathlineto{\pgfqpoint{0.750000in}{2.640000in}}%
\pgfusepath{stroke}%
\end{pgfscope}%
\begin{pgfscope}%
\pgfsetbuttcap%
\pgfsetroundjoin%
\definecolor{currentfill}{rgb}{0.000000,0.000000,0.000000}%
\pgfsetfillcolor{currentfill}%
\pgfsetlinewidth{0.803000pt}%
\definecolor{currentstroke}{rgb}{0.000000,0.000000,0.000000}%
\pgfsetstrokecolor{currentstroke}%
\pgfsetdash{}{0pt}%
\pgfsys@defobject{currentmarker}{\pgfqpoint{0.000000in}{0.000000in}}{\pgfqpoint{0.000000in}{0.048611in}}{%
\pgfpathmoveto{\pgfqpoint{0.000000in}{0.000000in}}%
\pgfpathlineto{\pgfqpoint{0.000000in}{0.048611in}}%
\pgfusepath{stroke,fill}%
}%
\begin{pgfscope}%
\pgfsys@transformshift{0.750000in}{0.330000in}%
\pgfsys@useobject{currentmarker}{}%
\end{pgfscope}%
\end{pgfscope}%
\begin{pgfscope}%
\definecolor{textcolor}{rgb}{0.000000,0.000000,0.000000}%
\pgfsetstrokecolor{textcolor}%
\pgfsetfillcolor{textcolor}%
\pgftext[x=0.750000in,y=0.281389in,,top]{\color{textcolor}\rmfamily\fontsize{10.000000}{12.000000}\selectfont \(\displaystyle {46.0}\)}%
\end{pgfscope}%
\begin{pgfscope}%
\pgfpathrectangle{\pgfqpoint{0.750000in}{0.330000in}}{\pgfqpoint{3.720000in}{2.310000in}}%
\pgfusepath{clip}%
\pgfsetbuttcap%
\pgfsetroundjoin%
\pgfsetlinewidth{0.501875pt}%
\definecolor{currentstroke}{rgb}{0.698039,0.698039,0.698039}%
\pgfsetstrokecolor{currentstroke}%
\pgfsetdash{{1.850000pt}{0.800000pt}}{0.000000pt}%
\pgfpathmoveto{\pgfqpoint{1.494000in}{0.330000in}}%
\pgfpathlineto{\pgfqpoint{1.494000in}{2.640000in}}%
\pgfusepath{stroke}%
\end{pgfscope}%
\begin{pgfscope}%
\pgfsetbuttcap%
\pgfsetroundjoin%
\definecolor{currentfill}{rgb}{0.000000,0.000000,0.000000}%
\pgfsetfillcolor{currentfill}%
\pgfsetlinewidth{0.803000pt}%
\definecolor{currentstroke}{rgb}{0.000000,0.000000,0.000000}%
\pgfsetstrokecolor{currentstroke}%
\pgfsetdash{}{0pt}%
\pgfsys@defobject{currentmarker}{\pgfqpoint{0.000000in}{0.000000in}}{\pgfqpoint{0.000000in}{0.048611in}}{%
\pgfpathmoveto{\pgfqpoint{0.000000in}{0.000000in}}%
\pgfpathlineto{\pgfqpoint{0.000000in}{0.048611in}}%
\pgfusepath{stroke,fill}%
}%
\begin{pgfscope}%
\pgfsys@transformshift{1.494000in}{0.330000in}%
\pgfsys@useobject{currentmarker}{}%
\end{pgfscope}%
\end{pgfscope}%
\begin{pgfscope}%
\definecolor{textcolor}{rgb}{0.000000,0.000000,0.000000}%
\pgfsetstrokecolor{textcolor}%
\pgfsetfillcolor{textcolor}%
\pgftext[x=1.494000in,y=0.281389in,,top]{\color{textcolor}\rmfamily\fontsize{10.000000}{12.000000}\selectfont \(\displaystyle {46.2}\)}%
\end{pgfscope}%
\begin{pgfscope}%
\pgfpathrectangle{\pgfqpoint{0.750000in}{0.330000in}}{\pgfqpoint{3.720000in}{2.310000in}}%
\pgfusepath{clip}%
\pgfsetbuttcap%
\pgfsetroundjoin%
\pgfsetlinewidth{0.501875pt}%
\definecolor{currentstroke}{rgb}{0.698039,0.698039,0.698039}%
\pgfsetstrokecolor{currentstroke}%
\pgfsetdash{{1.850000pt}{0.800000pt}}{0.000000pt}%
\pgfpathmoveto{\pgfqpoint{2.238000in}{0.330000in}}%
\pgfpathlineto{\pgfqpoint{2.238000in}{2.640000in}}%
\pgfusepath{stroke}%
\end{pgfscope}%
\begin{pgfscope}%
\pgfsetbuttcap%
\pgfsetroundjoin%
\definecolor{currentfill}{rgb}{0.000000,0.000000,0.000000}%
\pgfsetfillcolor{currentfill}%
\pgfsetlinewidth{0.803000pt}%
\definecolor{currentstroke}{rgb}{0.000000,0.000000,0.000000}%
\pgfsetstrokecolor{currentstroke}%
\pgfsetdash{}{0pt}%
\pgfsys@defobject{currentmarker}{\pgfqpoint{0.000000in}{0.000000in}}{\pgfqpoint{0.000000in}{0.048611in}}{%
\pgfpathmoveto{\pgfqpoint{0.000000in}{0.000000in}}%
\pgfpathlineto{\pgfqpoint{0.000000in}{0.048611in}}%
\pgfusepath{stroke,fill}%
}%
\begin{pgfscope}%
\pgfsys@transformshift{2.238000in}{0.330000in}%
\pgfsys@useobject{currentmarker}{}%
\end{pgfscope}%
\end{pgfscope}%
\begin{pgfscope}%
\definecolor{textcolor}{rgb}{0.000000,0.000000,0.000000}%
\pgfsetstrokecolor{textcolor}%
\pgfsetfillcolor{textcolor}%
\pgftext[x=2.238000in,y=0.281389in,,top]{\color{textcolor}\rmfamily\fontsize{10.000000}{12.000000}\selectfont \(\displaystyle {46.4}\)}%
\end{pgfscope}%
\begin{pgfscope}%
\pgfpathrectangle{\pgfqpoint{0.750000in}{0.330000in}}{\pgfqpoint{3.720000in}{2.310000in}}%
\pgfusepath{clip}%
\pgfsetbuttcap%
\pgfsetroundjoin%
\pgfsetlinewidth{0.501875pt}%
\definecolor{currentstroke}{rgb}{0.698039,0.698039,0.698039}%
\pgfsetstrokecolor{currentstroke}%
\pgfsetdash{{1.850000pt}{0.800000pt}}{0.000000pt}%
\pgfpathmoveto{\pgfqpoint{2.982000in}{0.330000in}}%
\pgfpathlineto{\pgfqpoint{2.982000in}{2.640000in}}%
\pgfusepath{stroke}%
\end{pgfscope}%
\begin{pgfscope}%
\pgfsetbuttcap%
\pgfsetroundjoin%
\definecolor{currentfill}{rgb}{0.000000,0.000000,0.000000}%
\pgfsetfillcolor{currentfill}%
\pgfsetlinewidth{0.803000pt}%
\definecolor{currentstroke}{rgb}{0.000000,0.000000,0.000000}%
\pgfsetstrokecolor{currentstroke}%
\pgfsetdash{}{0pt}%
\pgfsys@defobject{currentmarker}{\pgfqpoint{0.000000in}{0.000000in}}{\pgfqpoint{0.000000in}{0.048611in}}{%
\pgfpathmoveto{\pgfqpoint{0.000000in}{0.000000in}}%
\pgfpathlineto{\pgfqpoint{0.000000in}{0.048611in}}%
\pgfusepath{stroke,fill}%
}%
\begin{pgfscope}%
\pgfsys@transformshift{2.982000in}{0.330000in}%
\pgfsys@useobject{currentmarker}{}%
\end{pgfscope}%
\end{pgfscope}%
\begin{pgfscope}%
\definecolor{textcolor}{rgb}{0.000000,0.000000,0.000000}%
\pgfsetstrokecolor{textcolor}%
\pgfsetfillcolor{textcolor}%
\pgftext[x=2.982000in,y=0.281389in,,top]{\color{textcolor}\rmfamily\fontsize{10.000000}{12.000000}\selectfont \(\displaystyle {46.6}\)}%
\end{pgfscope}%
\begin{pgfscope}%
\pgfpathrectangle{\pgfqpoint{0.750000in}{0.330000in}}{\pgfqpoint{3.720000in}{2.310000in}}%
\pgfusepath{clip}%
\pgfsetbuttcap%
\pgfsetroundjoin%
\pgfsetlinewidth{0.501875pt}%
\definecolor{currentstroke}{rgb}{0.698039,0.698039,0.698039}%
\pgfsetstrokecolor{currentstroke}%
\pgfsetdash{{1.850000pt}{0.800000pt}}{0.000000pt}%
\pgfpathmoveto{\pgfqpoint{3.726000in}{0.330000in}}%
\pgfpathlineto{\pgfqpoint{3.726000in}{2.640000in}}%
\pgfusepath{stroke}%
\end{pgfscope}%
\begin{pgfscope}%
\pgfsetbuttcap%
\pgfsetroundjoin%
\definecolor{currentfill}{rgb}{0.000000,0.000000,0.000000}%
\pgfsetfillcolor{currentfill}%
\pgfsetlinewidth{0.803000pt}%
\definecolor{currentstroke}{rgb}{0.000000,0.000000,0.000000}%
\pgfsetstrokecolor{currentstroke}%
\pgfsetdash{}{0pt}%
\pgfsys@defobject{currentmarker}{\pgfqpoint{0.000000in}{0.000000in}}{\pgfqpoint{0.000000in}{0.048611in}}{%
\pgfpathmoveto{\pgfqpoint{0.000000in}{0.000000in}}%
\pgfpathlineto{\pgfqpoint{0.000000in}{0.048611in}}%
\pgfusepath{stroke,fill}%
}%
\begin{pgfscope}%
\pgfsys@transformshift{3.726000in}{0.330000in}%
\pgfsys@useobject{currentmarker}{}%
\end{pgfscope}%
\end{pgfscope}%
\begin{pgfscope}%
\definecolor{textcolor}{rgb}{0.000000,0.000000,0.000000}%
\pgfsetstrokecolor{textcolor}%
\pgfsetfillcolor{textcolor}%
\pgftext[x=3.726000in,y=0.281389in,,top]{\color{textcolor}\rmfamily\fontsize{10.000000}{12.000000}\selectfont \(\displaystyle {46.8}\)}%
\end{pgfscope}%
\begin{pgfscope}%
\pgfpathrectangle{\pgfqpoint{0.750000in}{0.330000in}}{\pgfqpoint{3.720000in}{2.310000in}}%
\pgfusepath{clip}%
\pgfsetbuttcap%
\pgfsetroundjoin%
\pgfsetlinewidth{0.501875pt}%
\definecolor{currentstroke}{rgb}{0.698039,0.698039,0.698039}%
\pgfsetstrokecolor{currentstroke}%
\pgfsetdash{{1.850000pt}{0.800000pt}}{0.000000pt}%
\pgfpathmoveto{\pgfqpoint{4.470000in}{0.330000in}}%
\pgfpathlineto{\pgfqpoint{4.470000in}{2.640000in}}%
\pgfusepath{stroke}%
\end{pgfscope}%
\begin{pgfscope}%
\pgfsetbuttcap%
\pgfsetroundjoin%
\definecolor{currentfill}{rgb}{0.000000,0.000000,0.000000}%
\pgfsetfillcolor{currentfill}%
\pgfsetlinewidth{0.803000pt}%
\definecolor{currentstroke}{rgb}{0.000000,0.000000,0.000000}%
\pgfsetstrokecolor{currentstroke}%
\pgfsetdash{}{0pt}%
\pgfsys@defobject{currentmarker}{\pgfqpoint{0.000000in}{0.000000in}}{\pgfqpoint{0.000000in}{0.048611in}}{%
\pgfpathmoveto{\pgfqpoint{0.000000in}{0.000000in}}%
\pgfpathlineto{\pgfqpoint{0.000000in}{0.048611in}}%
\pgfusepath{stroke,fill}%
}%
\begin{pgfscope}%
\pgfsys@transformshift{4.470000in}{0.330000in}%
\pgfsys@useobject{currentmarker}{}%
\end{pgfscope}%
\end{pgfscope}%
\begin{pgfscope}%
\definecolor{textcolor}{rgb}{0.000000,0.000000,0.000000}%
\pgfsetstrokecolor{textcolor}%
\pgfsetfillcolor{textcolor}%
\pgftext[x=4.470000in,y=0.281389in,,top]{\color{textcolor}\rmfamily\fontsize{10.000000}{12.000000}\selectfont \(\displaystyle {47.0}\)}%
\end{pgfscope}%
\begin{pgfscope}%
\definecolor{textcolor}{rgb}{0.000000,0.000000,0.000000}%
\pgfsetstrokecolor{textcolor}%
\pgfsetfillcolor{textcolor}%
\pgftext[x=2.610000in,y=0.102500in,,top]{\color{textcolor}\rmfamily\fontsize{12.000000}{14.400000}\selectfont Temps [s]}%
\end{pgfscope}%
\begin{pgfscope}%
\pgfpathrectangle{\pgfqpoint{0.750000in}{0.330000in}}{\pgfqpoint{3.720000in}{2.310000in}}%
\pgfusepath{clip}%
\pgfsetbuttcap%
\pgfsetroundjoin%
\pgfsetlinewidth{0.501875pt}%
\definecolor{currentstroke}{rgb}{0.698039,0.698039,0.698039}%
\pgfsetstrokecolor{currentstroke}%
\pgfsetdash{{1.850000pt}{0.800000pt}}{0.000000pt}%
\pgfpathmoveto{\pgfqpoint{0.750000in}{0.330000in}}%
\pgfpathlineto{\pgfqpoint{4.470000in}{0.330000in}}%
\pgfusepath{stroke}%
\end{pgfscope}%
\begin{pgfscope}%
\pgfsetbuttcap%
\pgfsetroundjoin%
\definecolor{currentfill}{rgb}{0.000000,0.000000,0.000000}%
\pgfsetfillcolor{currentfill}%
\pgfsetlinewidth{0.803000pt}%
\definecolor{currentstroke}{rgb}{0.000000,0.000000,0.000000}%
\pgfsetstrokecolor{currentstroke}%
\pgfsetdash{}{0pt}%
\pgfsys@defobject{currentmarker}{\pgfqpoint{0.000000in}{0.000000in}}{\pgfqpoint{0.048611in}{0.000000in}}{%
\pgfpathmoveto{\pgfqpoint{0.000000in}{0.000000in}}%
\pgfpathlineto{\pgfqpoint{0.048611in}{0.000000in}}%
\pgfusepath{stroke,fill}%
}%
\begin{pgfscope}%
\pgfsys@transformshift{0.750000in}{0.330000in}%
\pgfsys@useobject{currentmarker}{}%
\end{pgfscope}%
\end{pgfscope}%
\begin{pgfscope}%
\definecolor{textcolor}{rgb}{0.000000,0.000000,0.000000}%
\pgfsetstrokecolor{textcolor}%
\pgfsetfillcolor{textcolor}%
\pgftext[x=0.415894in, y=0.281806in, left, base]{\color{textcolor}\rmfamily\fontsize{10.000000}{12.000000}\selectfont \(\displaystyle {\ensuremath{-}0.2}\)}%
\end{pgfscope}%
\begin{pgfscope}%
\pgfpathrectangle{\pgfqpoint{0.750000in}{0.330000in}}{\pgfqpoint{3.720000in}{2.310000in}}%
\pgfusepath{clip}%
\pgfsetbuttcap%
\pgfsetroundjoin%
\pgfsetlinewidth{0.501875pt}%
\definecolor{currentstroke}{rgb}{0.698039,0.698039,0.698039}%
\pgfsetstrokecolor{currentstroke}%
\pgfsetdash{{1.850000pt}{0.800000pt}}{0.000000pt}%
\pgfpathmoveto{\pgfqpoint{0.750000in}{0.715000in}}%
\pgfpathlineto{\pgfqpoint{4.470000in}{0.715000in}}%
\pgfusepath{stroke}%
\end{pgfscope}%
\begin{pgfscope}%
\pgfsetbuttcap%
\pgfsetroundjoin%
\definecolor{currentfill}{rgb}{0.000000,0.000000,0.000000}%
\pgfsetfillcolor{currentfill}%
\pgfsetlinewidth{0.803000pt}%
\definecolor{currentstroke}{rgb}{0.000000,0.000000,0.000000}%
\pgfsetstrokecolor{currentstroke}%
\pgfsetdash{}{0pt}%
\pgfsys@defobject{currentmarker}{\pgfqpoint{0.000000in}{0.000000in}}{\pgfqpoint{0.048611in}{0.000000in}}{%
\pgfpathmoveto{\pgfqpoint{0.000000in}{0.000000in}}%
\pgfpathlineto{\pgfqpoint{0.048611in}{0.000000in}}%
\pgfusepath{stroke,fill}%
}%
\begin{pgfscope}%
\pgfsys@transformshift{0.750000in}{0.715000in}%
\pgfsys@useobject{currentmarker}{}%
\end{pgfscope}%
\end{pgfscope}%
\begin{pgfscope}%
\definecolor{textcolor}{rgb}{0.000000,0.000000,0.000000}%
\pgfsetstrokecolor{textcolor}%
\pgfsetfillcolor{textcolor}%
\pgftext[x=0.523919in, y=0.666806in, left, base]{\color{textcolor}\rmfamily\fontsize{10.000000}{12.000000}\selectfont \(\displaystyle {0.0}\)}%
\end{pgfscope}%
\begin{pgfscope}%
\pgfpathrectangle{\pgfqpoint{0.750000in}{0.330000in}}{\pgfqpoint{3.720000in}{2.310000in}}%
\pgfusepath{clip}%
\pgfsetbuttcap%
\pgfsetroundjoin%
\pgfsetlinewidth{0.501875pt}%
\definecolor{currentstroke}{rgb}{0.698039,0.698039,0.698039}%
\pgfsetstrokecolor{currentstroke}%
\pgfsetdash{{1.850000pt}{0.800000pt}}{0.000000pt}%
\pgfpathmoveto{\pgfqpoint{0.750000in}{1.100000in}}%
\pgfpathlineto{\pgfqpoint{4.470000in}{1.100000in}}%
\pgfusepath{stroke}%
\end{pgfscope}%
\begin{pgfscope}%
\pgfsetbuttcap%
\pgfsetroundjoin%
\definecolor{currentfill}{rgb}{0.000000,0.000000,0.000000}%
\pgfsetfillcolor{currentfill}%
\pgfsetlinewidth{0.803000pt}%
\definecolor{currentstroke}{rgb}{0.000000,0.000000,0.000000}%
\pgfsetstrokecolor{currentstroke}%
\pgfsetdash{}{0pt}%
\pgfsys@defobject{currentmarker}{\pgfqpoint{0.000000in}{0.000000in}}{\pgfqpoint{0.048611in}{0.000000in}}{%
\pgfpathmoveto{\pgfqpoint{0.000000in}{0.000000in}}%
\pgfpathlineto{\pgfqpoint{0.048611in}{0.000000in}}%
\pgfusepath{stroke,fill}%
}%
\begin{pgfscope}%
\pgfsys@transformshift{0.750000in}{1.100000in}%
\pgfsys@useobject{currentmarker}{}%
\end{pgfscope}%
\end{pgfscope}%
\begin{pgfscope}%
\definecolor{textcolor}{rgb}{0.000000,0.000000,0.000000}%
\pgfsetstrokecolor{textcolor}%
\pgfsetfillcolor{textcolor}%
\pgftext[x=0.523919in, y=1.051806in, left, base]{\color{textcolor}\rmfamily\fontsize{10.000000}{12.000000}\selectfont \(\displaystyle {0.2}\)}%
\end{pgfscope}%
\begin{pgfscope}%
\pgfpathrectangle{\pgfqpoint{0.750000in}{0.330000in}}{\pgfqpoint{3.720000in}{2.310000in}}%
\pgfusepath{clip}%
\pgfsetbuttcap%
\pgfsetroundjoin%
\pgfsetlinewidth{0.501875pt}%
\definecolor{currentstroke}{rgb}{0.698039,0.698039,0.698039}%
\pgfsetstrokecolor{currentstroke}%
\pgfsetdash{{1.850000pt}{0.800000pt}}{0.000000pt}%
\pgfpathmoveto{\pgfqpoint{0.750000in}{1.485000in}}%
\pgfpathlineto{\pgfqpoint{4.470000in}{1.485000in}}%
\pgfusepath{stroke}%
\end{pgfscope}%
\begin{pgfscope}%
\pgfsetbuttcap%
\pgfsetroundjoin%
\definecolor{currentfill}{rgb}{0.000000,0.000000,0.000000}%
\pgfsetfillcolor{currentfill}%
\pgfsetlinewidth{0.803000pt}%
\definecolor{currentstroke}{rgb}{0.000000,0.000000,0.000000}%
\pgfsetstrokecolor{currentstroke}%
\pgfsetdash{}{0pt}%
\pgfsys@defobject{currentmarker}{\pgfqpoint{0.000000in}{0.000000in}}{\pgfqpoint{0.048611in}{0.000000in}}{%
\pgfpathmoveto{\pgfqpoint{0.000000in}{0.000000in}}%
\pgfpathlineto{\pgfqpoint{0.048611in}{0.000000in}}%
\pgfusepath{stroke,fill}%
}%
\begin{pgfscope}%
\pgfsys@transformshift{0.750000in}{1.485000in}%
\pgfsys@useobject{currentmarker}{}%
\end{pgfscope}%
\end{pgfscope}%
\begin{pgfscope}%
\definecolor{textcolor}{rgb}{0.000000,0.000000,0.000000}%
\pgfsetstrokecolor{textcolor}%
\pgfsetfillcolor{textcolor}%
\pgftext[x=0.523919in, y=1.436806in, left, base]{\color{textcolor}\rmfamily\fontsize{10.000000}{12.000000}\selectfont \(\displaystyle {0.4}\)}%
\end{pgfscope}%
\begin{pgfscope}%
\pgfpathrectangle{\pgfqpoint{0.750000in}{0.330000in}}{\pgfqpoint{3.720000in}{2.310000in}}%
\pgfusepath{clip}%
\pgfsetbuttcap%
\pgfsetroundjoin%
\pgfsetlinewidth{0.501875pt}%
\definecolor{currentstroke}{rgb}{0.698039,0.698039,0.698039}%
\pgfsetstrokecolor{currentstroke}%
\pgfsetdash{{1.850000pt}{0.800000pt}}{0.000000pt}%
\pgfpathmoveto{\pgfqpoint{0.750000in}{1.870000in}}%
\pgfpathlineto{\pgfqpoint{4.470000in}{1.870000in}}%
\pgfusepath{stroke}%
\end{pgfscope}%
\begin{pgfscope}%
\pgfsetbuttcap%
\pgfsetroundjoin%
\definecolor{currentfill}{rgb}{0.000000,0.000000,0.000000}%
\pgfsetfillcolor{currentfill}%
\pgfsetlinewidth{0.803000pt}%
\definecolor{currentstroke}{rgb}{0.000000,0.000000,0.000000}%
\pgfsetstrokecolor{currentstroke}%
\pgfsetdash{}{0pt}%
\pgfsys@defobject{currentmarker}{\pgfqpoint{0.000000in}{0.000000in}}{\pgfqpoint{0.048611in}{0.000000in}}{%
\pgfpathmoveto{\pgfqpoint{0.000000in}{0.000000in}}%
\pgfpathlineto{\pgfqpoint{0.048611in}{0.000000in}}%
\pgfusepath{stroke,fill}%
}%
\begin{pgfscope}%
\pgfsys@transformshift{0.750000in}{1.870000in}%
\pgfsys@useobject{currentmarker}{}%
\end{pgfscope}%
\end{pgfscope}%
\begin{pgfscope}%
\definecolor{textcolor}{rgb}{0.000000,0.000000,0.000000}%
\pgfsetstrokecolor{textcolor}%
\pgfsetfillcolor{textcolor}%
\pgftext[x=0.523919in, y=1.821806in, left, base]{\color{textcolor}\rmfamily\fontsize{10.000000}{12.000000}\selectfont \(\displaystyle {0.6}\)}%
\end{pgfscope}%
\begin{pgfscope}%
\pgfpathrectangle{\pgfqpoint{0.750000in}{0.330000in}}{\pgfqpoint{3.720000in}{2.310000in}}%
\pgfusepath{clip}%
\pgfsetbuttcap%
\pgfsetroundjoin%
\pgfsetlinewidth{0.501875pt}%
\definecolor{currentstroke}{rgb}{0.698039,0.698039,0.698039}%
\pgfsetstrokecolor{currentstroke}%
\pgfsetdash{{1.850000pt}{0.800000pt}}{0.000000pt}%
\pgfpathmoveto{\pgfqpoint{0.750000in}{2.255000in}}%
\pgfpathlineto{\pgfqpoint{4.470000in}{2.255000in}}%
\pgfusepath{stroke}%
\end{pgfscope}%
\begin{pgfscope}%
\pgfsetbuttcap%
\pgfsetroundjoin%
\definecolor{currentfill}{rgb}{0.000000,0.000000,0.000000}%
\pgfsetfillcolor{currentfill}%
\pgfsetlinewidth{0.803000pt}%
\definecolor{currentstroke}{rgb}{0.000000,0.000000,0.000000}%
\pgfsetstrokecolor{currentstroke}%
\pgfsetdash{}{0pt}%
\pgfsys@defobject{currentmarker}{\pgfqpoint{0.000000in}{0.000000in}}{\pgfqpoint{0.048611in}{0.000000in}}{%
\pgfpathmoveto{\pgfqpoint{0.000000in}{0.000000in}}%
\pgfpathlineto{\pgfqpoint{0.048611in}{0.000000in}}%
\pgfusepath{stroke,fill}%
}%
\begin{pgfscope}%
\pgfsys@transformshift{0.750000in}{2.255000in}%
\pgfsys@useobject{currentmarker}{}%
\end{pgfscope}%
\end{pgfscope}%
\begin{pgfscope}%
\definecolor{textcolor}{rgb}{0.000000,0.000000,0.000000}%
\pgfsetstrokecolor{textcolor}%
\pgfsetfillcolor{textcolor}%
\pgftext[x=0.523919in, y=2.206806in, left, base]{\color{textcolor}\rmfamily\fontsize{10.000000}{12.000000}\selectfont \(\displaystyle {0.8}\)}%
\end{pgfscope}%
\begin{pgfscope}%
\pgfpathrectangle{\pgfqpoint{0.750000in}{0.330000in}}{\pgfqpoint{3.720000in}{2.310000in}}%
\pgfusepath{clip}%
\pgfsetbuttcap%
\pgfsetroundjoin%
\pgfsetlinewidth{0.501875pt}%
\definecolor{currentstroke}{rgb}{0.698039,0.698039,0.698039}%
\pgfsetstrokecolor{currentstroke}%
\pgfsetdash{{1.850000pt}{0.800000pt}}{0.000000pt}%
\pgfpathmoveto{\pgfqpoint{0.750000in}{2.640000in}}%
\pgfpathlineto{\pgfqpoint{4.470000in}{2.640000in}}%
\pgfusepath{stroke}%
\end{pgfscope}%
\begin{pgfscope}%
\pgfsetbuttcap%
\pgfsetroundjoin%
\definecolor{currentfill}{rgb}{0.000000,0.000000,0.000000}%
\pgfsetfillcolor{currentfill}%
\pgfsetlinewidth{0.803000pt}%
\definecolor{currentstroke}{rgb}{0.000000,0.000000,0.000000}%
\pgfsetstrokecolor{currentstroke}%
\pgfsetdash{}{0pt}%
\pgfsys@defobject{currentmarker}{\pgfqpoint{0.000000in}{0.000000in}}{\pgfqpoint{0.048611in}{0.000000in}}{%
\pgfpathmoveto{\pgfqpoint{0.000000in}{0.000000in}}%
\pgfpathlineto{\pgfqpoint{0.048611in}{0.000000in}}%
\pgfusepath{stroke,fill}%
}%
\begin{pgfscope}%
\pgfsys@transformshift{0.750000in}{2.640000in}%
\pgfsys@useobject{currentmarker}{}%
\end{pgfscope}%
\end{pgfscope}%
\begin{pgfscope}%
\definecolor{textcolor}{rgb}{0.000000,0.000000,0.000000}%
\pgfsetstrokecolor{textcolor}%
\pgfsetfillcolor{textcolor}%
\pgftext[x=0.523919in, y=2.591806in, left, base]{\color{textcolor}\rmfamily\fontsize{10.000000}{12.000000}\selectfont \(\displaystyle {1.0}\)}%
\end{pgfscope}%
\begin{pgfscope}%
\definecolor{textcolor}{rgb}{0.000000,0.000000,0.000000}%
\pgfsetstrokecolor{textcolor}%
\pgfsetfillcolor{textcolor}%
\pgftext[x=0.360339in,y=1.485000in,,bottom,rotate=90.000000]{\color{textcolor}\rmfamily\fontsize{12.000000}{14.400000}\selectfont Amplitude [m/s]}%
\end{pgfscope}%
\begin{pgfscope}%
\definecolor{textcolor}{rgb}{0.000000,0.000000,0.000000}%
\pgfsetstrokecolor{textcolor}%
\pgfsetfillcolor{textcolor}%
\pgftext[x=0.750000in,y=2.681667in,left,base]{\color{textcolor}\rmfamily\fontsize{10.000000}{12.000000}\selectfont \(\displaystyle \times{10^{\ensuremath{-}5}}{}\)}%
\end{pgfscope}%
\begin{pgfscope}%
\pgfpathrectangle{\pgfqpoint{0.750000in}{0.330000in}}{\pgfqpoint{3.720000in}{2.310000in}}%
\pgfusepath{clip}%
\pgfsetrectcap%
\pgfsetroundjoin%
\pgfsetlinewidth{5.018750pt}%
\definecolor{currentstroke}{rgb}{0.501961,0.501961,0.501961}%
\pgfsetstrokecolor{currentstroke}%
\pgfsetdash{}{0pt}%
\pgfpathmoveto{\pgfqpoint{0.740000in}{1.119506in}}%
\pgfpathlineto{\pgfqpoint{0.750000in}{1.096907in}}%
\pgfpathlineto{\pgfqpoint{0.768600in}{1.036439in}}%
\pgfpathlineto{\pgfqpoint{0.787200in}{0.959783in}}%
\pgfpathlineto{\pgfqpoint{0.805800in}{0.870731in}}%
\pgfpathlineto{\pgfqpoint{0.861600in}{0.585859in}}%
\pgfpathlineto{\pgfqpoint{0.880200in}{0.506940in}}%
\pgfpathlineto{\pgfqpoint{0.898800in}{0.446379in}}%
\pgfpathlineto{\pgfqpoint{0.917400in}{0.408633in}}%
\pgfpathlineto{\pgfqpoint{0.936000in}{0.396244in}}%
\pgfpathlineto{\pgfqpoint{0.954600in}{0.409553in}}%
\pgfpathlineto{\pgfqpoint{0.973200in}{0.446680in}}%
\pgfpathlineto{\pgfqpoint{0.991800in}{0.503754in}}%
\pgfpathlineto{\pgfqpoint{1.010400in}{0.575382in}}%
\pgfpathlineto{\pgfqpoint{1.066200in}{0.814246in}}%
\pgfpathlineto{\pgfqpoint{1.084800in}{0.882094in}}%
\pgfpathlineto{\pgfqpoint{1.103400in}{0.936534in}}%
\pgfpathlineto{\pgfqpoint{1.122000in}{0.974993in}}%
\pgfpathlineto{\pgfqpoint{1.140600in}{0.996227in}}%
\pgfpathlineto{\pgfqpoint{1.159200in}{1.000184in}}%
\pgfpathlineto{\pgfqpoint{1.177800in}{0.987799in}}%
\pgfpathlineto{\pgfqpoint{1.196400in}{0.960727in}}%
\pgfpathlineto{\pgfqpoint{1.215000in}{0.921071in}}%
\pgfpathlineto{\pgfqpoint{1.233600in}{0.871154in}}%
\pgfpathlineto{\pgfqpoint{1.252200in}{0.813359in}}%
\pgfpathlineto{\pgfqpoint{1.289400in}{0.683688in}}%
\pgfpathlineto{\pgfqpoint{1.326600in}{0.550843in}}%
\pgfpathlineto{\pgfqpoint{1.345200in}{0.488564in}}%
\pgfpathlineto{\pgfqpoint{1.363800in}{0.431273in}}%
\pgfpathlineto{\pgfqpoint{1.382400in}{0.380128in}}%
\pgfpathlineto{\pgfqpoint{1.401000in}{0.335949in}}%
\pgfpathlineto{\pgfqpoint{1.409120in}{0.320000in}}%
\pgfpathmoveto{\pgfqpoint{1.542980in}{0.320000in}}%
\pgfpathlineto{\pgfqpoint{1.549800in}{0.335439in}}%
\pgfpathlineto{\pgfqpoint{1.568400in}{0.389161in}}%
\pgfpathlineto{\pgfqpoint{1.587000in}{0.452529in}}%
\pgfpathlineto{\pgfqpoint{1.624200in}{0.597443in}}%
\pgfpathlineto{\pgfqpoint{1.661400in}{0.747041in}}%
\pgfpathlineto{\pgfqpoint{1.680000in}{0.817207in}}%
\pgfpathlineto{\pgfqpoint{1.698600in}{0.881844in}}%
\pgfpathlineto{\pgfqpoint{1.717200in}{0.939684in}}%
\pgfpathlineto{\pgfqpoint{1.735800in}{0.989687in}}%
\pgfpathlineto{\pgfqpoint{1.754400in}{1.030856in}}%
\pgfpathlineto{\pgfqpoint{1.773000in}{1.062116in}}%
\pgfpathlineto{\pgfqpoint{1.791600in}{1.082369in}}%
\pgfpathlineto{\pgfqpoint{1.810200in}{1.090693in}}%
\pgfpathlineto{\pgfqpoint{1.828800in}{1.086652in}}%
\pgfpathlineto{\pgfqpoint{1.847400in}{1.070570in}}%
\pgfpathlineto{\pgfqpoint{1.866000in}{1.043653in}}%
\pgfpathlineto{\pgfqpoint{1.884600in}{1.007910in}}%
\pgfpathlineto{\pgfqpoint{1.903200in}{0.965895in}}%
\pgfpathlineto{\pgfqpoint{1.959000in}{0.828990in}}%
\pgfpathlineto{\pgfqpoint{1.977600in}{0.787280in}}%
\pgfpathlineto{\pgfqpoint{1.996200in}{0.750338in}}%
\pgfpathlineto{\pgfqpoint{2.014800in}{0.719568in}}%
\pgfpathlineto{\pgfqpoint{2.033400in}{0.696589in}}%
\pgfpathlineto{\pgfqpoint{2.052000in}{0.683456in}}%
\pgfpathlineto{\pgfqpoint{2.070600in}{0.682618in}}%
\pgfpathlineto{\pgfqpoint{2.089200in}{0.696569in}}%
\pgfpathlineto{\pgfqpoint{2.107800in}{0.727239in}}%
\pgfpathlineto{\pgfqpoint{2.126400in}{0.775333in}}%
\pgfpathlineto{\pgfqpoint{2.145000in}{0.839802in}}%
\pgfpathlineto{\pgfqpoint{2.163600in}{0.917627in}}%
\pgfpathlineto{\pgfqpoint{2.219400in}{1.177462in}}%
\pgfpathlineto{\pgfqpoint{2.238000in}{1.251278in}}%
\pgfpathlineto{\pgfqpoint{2.256600in}{1.308299in}}%
\pgfpathlineto{\pgfqpoint{2.275200in}{1.343451in}}%
\pgfpathlineto{\pgfqpoint{2.293800in}{1.352734in}}%
\pgfpathlineto{\pgfqpoint{2.312400in}{1.333365in}}%
\pgfpathlineto{\pgfqpoint{2.331000in}{1.283937in}}%
\pgfpathlineto{\pgfqpoint{2.349600in}{1.204604in}}%
\pgfpathlineto{\pgfqpoint{2.368200in}{1.097245in}}%
\pgfpathlineto{\pgfqpoint{2.386800in}{0.965534in}}%
\pgfpathlineto{\pgfqpoint{2.405400in}{0.814901in}}%
\pgfpathlineto{\pgfqpoint{2.461863in}{0.320000in}}%
\pgfpathmoveto{\pgfqpoint{2.651224in}{0.320000in}}%
\pgfpathlineto{\pgfqpoint{2.721600in}{0.881530in}}%
\pgfpathlineto{\pgfqpoint{2.740200in}{0.995922in}}%
\pgfpathlineto{\pgfqpoint{2.758800in}{1.077566in}}%
\pgfpathlineto{\pgfqpoint{2.777400in}{1.118946in}}%
\pgfpathlineto{\pgfqpoint{2.796000in}{1.115147in}}%
\pgfpathlineto{\pgfqpoint{2.814600in}{1.064582in}}%
\pgfpathlineto{\pgfqpoint{2.833200in}{0.969407in}}%
\pgfpathlineto{\pgfqpoint{2.851800in}{0.835557in}}%
\pgfpathlineto{\pgfqpoint{2.870400in}{0.672370in}}%
\pgfpathlineto{\pgfqpoint{2.906342in}{0.320000in}}%
\pgfpathmoveto{\pgfqpoint{3.093471in}{0.320000in}}%
\pgfpathlineto{\pgfqpoint{3.112200in}{0.501737in}}%
\pgfpathlineto{\pgfqpoint{3.130800in}{0.678713in}}%
\pgfpathlineto{\pgfqpoint{3.149400in}{0.841668in}}%
\pgfpathlineto{\pgfqpoint{3.168000in}{0.982602in}}%
\pgfpathlineto{\pgfqpoint{3.186600in}{1.096403in}}%
\pgfpathlineto{\pgfqpoint{3.205200in}{1.180880in}}%
\pgfpathlineto{\pgfqpoint{3.223800in}{1.236428in}}%
\pgfpathlineto{\pgfqpoint{3.242400in}{1.265449in}}%
\pgfpathlineto{\pgfqpoint{3.261000in}{1.271606in}}%
\pgfpathlineto{\pgfqpoint{3.279600in}{1.259047in}}%
\pgfpathlineto{\pgfqpoint{3.298200in}{1.231740in}}%
\pgfpathlineto{\pgfqpoint{3.316800in}{1.193035in}}%
\pgfpathlineto{\pgfqpoint{3.335400in}{1.145564in}}%
\pgfpathlineto{\pgfqpoint{3.354000in}{1.091413in}}%
\pgfpathlineto{\pgfqpoint{3.391200in}{0.970437in}}%
\pgfpathlineto{\pgfqpoint{3.428400in}{0.845751in}}%
\pgfpathlineto{\pgfqpoint{3.447000in}{0.787667in}}%
\pgfpathlineto{\pgfqpoint{3.465600in}{0.735919in}}%
\pgfpathlineto{\pgfqpoint{3.484200in}{0.693553in}}%
\pgfpathlineto{\pgfqpoint{3.502800in}{0.663967in}}%
\pgfpathlineto{\pgfqpoint{3.521400in}{0.650725in}}%
\pgfpathlineto{\pgfqpoint{3.540000in}{0.657089in}}%
\pgfpathlineto{\pgfqpoint{3.558600in}{0.685314in}}%
\pgfpathlineto{\pgfqpoint{3.577200in}{0.735870in}}%
\pgfpathlineto{\pgfqpoint{3.595800in}{0.806837in}}%
\pgfpathlineto{\pgfqpoint{3.614400in}{0.893688in}}%
\pgfpathlineto{\pgfqpoint{3.651600in}{1.085997in}}%
\pgfpathlineto{\pgfqpoint{3.670200in}{1.173874in}}%
\pgfpathlineto{\pgfqpoint{3.688800in}{1.244480in}}%
\pgfpathlineto{\pgfqpoint{3.707400in}{1.290427in}}%
\pgfpathlineto{\pgfqpoint{3.726000in}{1.306422in}}%
\pgfpathlineto{\pgfqpoint{3.744600in}{1.289847in}}%
\pgfpathlineto{\pgfqpoint{3.763200in}{1.241133in}}%
\pgfpathlineto{\pgfqpoint{3.781800in}{1.163850in}}%
\pgfpathlineto{\pgfqpoint{3.800400in}{1.064458in}}%
\pgfpathlineto{\pgfqpoint{3.856200in}{0.726044in}}%
\pgfpathlineto{\pgfqpoint{3.874800in}{0.632277in}}%
\pgfpathlineto{\pgfqpoint{3.893400in}{0.561021in}}%
\pgfpathlineto{\pgfqpoint{3.912000in}{0.516247in}}%
\pgfpathlineto{\pgfqpoint{3.930600in}{0.498828in}}%
\pgfpathlineto{\pgfqpoint{3.949200in}{0.506710in}}%
\pgfpathlineto{\pgfqpoint{3.967800in}{0.535420in}}%
\pgfpathlineto{\pgfqpoint{3.986400in}{0.578820in}}%
\pgfpathlineto{\pgfqpoint{4.042200in}{0.729534in}}%
\pgfpathlineto{\pgfqpoint{4.060800in}{0.767657in}}%
\pgfpathlineto{\pgfqpoint{4.079400in}{0.793942in}}%
\pgfpathlineto{\pgfqpoint{4.098000in}{0.807530in}}%
\pgfpathlineto{\pgfqpoint{4.116600in}{0.809073in}}%
\pgfpathlineto{\pgfqpoint{4.135200in}{0.800301in}}%
\pgfpathlineto{\pgfqpoint{4.153800in}{0.783475in}}%
\pgfpathlineto{\pgfqpoint{4.172400in}{0.760852in}}%
\pgfpathlineto{\pgfqpoint{4.191000in}{0.734243in}}%
\pgfpathlineto{\pgfqpoint{4.209600in}{0.704724in}}%
\pgfpathlineto{\pgfqpoint{4.228200in}{0.672518in}}%
\pgfpathlineto{\pgfqpoint{4.246800in}{0.637047in}}%
\pgfpathlineto{\pgfqpoint{4.265400in}{0.597150in}}%
\pgfpathlineto{\pgfqpoint{4.284000in}{0.551465in}}%
\pgfpathlineto{\pgfqpoint{4.302600in}{0.498963in}}%
\pgfpathlineto{\pgfqpoint{4.321200in}{0.439552in}}%
\pgfpathlineto{\pgfqpoint{4.354897in}{0.320000in}}%
\pgfpathlineto{\pgfqpoint{4.354897in}{0.320000in}}%
\pgfusepath{stroke}%
\end{pgfscope}%
\begin{pgfscope}%
\pgfpathrectangle{\pgfqpoint{0.750000in}{0.330000in}}{\pgfqpoint{3.720000in}{2.310000in}}%
\pgfusepath{clip}%
\pgfsetbuttcap%
\pgfsetroundjoin%
\pgfsetlinewidth{1.505625pt}%
\definecolor{currentstroke}{rgb}{0.000000,0.000000,0.000000}%
\pgfsetstrokecolor{currentstroke}%
\pgfsetdash{{1.500000pt}{2.475000pt}}{0.000000pt}%
\pgfpathmoveto{\pgfqpoint{0.740000in}{1.119506in}}%
\pgfpathlineto{\pgfqpoint{0.750000in}{1.096907in}}%
\pgfpathlineto{\pgfqpoint{0.768600in}{1.036439in}}%
\pgfpathlineto{\pgfqpoint{0.787200in}{0.959783in}}%
\pgfpathlineto{\pgfqpoint{0.805800in}{0.870731in}}%
\pgfpathlineto{\pgfqpoint{0.824400in}{0.774476in}}%
\pgfpathlineto{\pgfqpoint{0.843000in}{0.752740in}}%
\pgfpathlineto{\pgfqpoint{0.861600in}{0.844141in}}%
\pgfpathlineto{\pgfqpoint{0.880200in}{0.923060in}}%
\pgfpathlineto{\pgfqpoint{0.898800in}{0.983621in}}%
\pgfpathlineto{\pgfqpoint{0.917400in}{1.021367in}}%
\pgfpathlineto{\pgfqpoint{0.936000in}{1.033756in}}%
\pgfpathlineto{\pgfqpoint{0.954600in}{1.020447in}}%
\pgfpathlineto{\pgfqpoint{0.973200in}{0.983320in}}%
\pgfpathlineto{\pgfqpoint{0.991800in}{0.926246in}}%
\pgfpathlineto{\pgfqpoint{1.010400in}{0.854618in}}%
\pgfpathlineto{\pgfqpoint{1.029000in}{0.774728in}}%
\pgfpathlineto{\pgfqpoint{1.047600in}{0.736920in}}%
\pgfpathlineto{\pgfqpoint{1.066200in}{0.814246in}}%
\pgfpathlineto{\pgfqpoint{1.084800in}{0.882094in}}%
\pgfpathlineto{\pgfqpoint{1.103400in}{0.936534in}}%
\pgfpathlineto{\pgfqpoint{1.122000in}{0.974993in}}%
\pgfpathlineto{\pgfqpoint{1.140600in}{0.996227in}}%
\pgfpathlineto{\pgfqpoint{1.159200in}{1.000184in}}%
\pgfpathlineto{\pgfqpoint{1.177800in}{0.987799in}}%
\pgfpathlineto{\pgfqpoint{1.196400in}{0.960727in}}%
\pgfpathlineto{\pgfqpoint{1.215000in}{0.921071in}}%
\pgfpathlineto{\pgfqpoint{1.233600in}{0.871154in}}%
\pgfpathlineto{\pgfqpoint{1.252200in}{0.813359in}}%
\pgfpathlineto{\pgfqpoint{1.270800in}{0.750077in}}%
\pgfpathlineto{\pgfqpoint{1.289400in}{0.746312in}}%
\pgfpathlineto{\pgfqpoint{1.326600in}{0.879157in}}%
\pgfpathlineto{\pgfqpoint{1.345200in}{0.941436in}}%
\pgfpathlineto{\pgfqpoint{1.363800in}{0.998727in}}%
\pgfpathlineto{\pgfqpoint{1.382400in}{1.049872in}}%
\pgfpathlineto{\pgfqpoint{1.401000in}{1.094051in}}%
\pgfpathlineto{\pgfqpoint{1.419600in}{1.130583in}}%
\pgfpathlineto{\pgfqpoint{1.438200in}{1.158712in}}%
\pgfpathlineto{\pgfqpoint{1.456800in}{1.177473in}}%
\pgfpathlineto{\pgfqpoint{1.475400in}{1.185698in}}%
\pgfpathlineto{\pgfqpoint{1.494000in}{1.182186in}}%
\pgfpathlineto{\pgfqpoint{1.512600in}{1.165983in}}%
\pgfpathlineto{\pgfqpoint{1.531200in}{1.136665in}}%
\pgfpathlineto{\pgfqpoint{1.549800in}{1.094561in}}%
\pgfpathlineto{\pgfqpoint{1.568400in}{1.040839in}}%
\pgfpathlineto{\pgfqpoint{1.587000in}{0.977471in}}%
\pgfpathlineto{\pgfqpoint{1.624200in}{0.832557in}}%
\pgfpathlineto{\pgfqpoint{1.642800in}{0.756955in}}%
\pgfpathlineto{\pgfqpoint{1.661400in}{0.747041in}}%
\pgfpathlineto{\pgfqpoint{1.680000in}{0.817207in}}%
\pgfpathlineto{\pgfqpoint{1.698600in}{0.881844in}}%
\pgfpathlineto{\pgfqpoint{1.717200in}{0.939684in}}%
\pgfpathlineto{\pgfqpoint{1.735800in}{0.989687in}}%
\pgfpathlineto{\pgfqpoint{1.754400in}{1.030856in}}%
\pgfpathlineto{\pgfqpoint{1.773000in}{1.062116in}}%
\pgfpathlineto{\pgfqpoint{1.791600in}{1.082369in}}%
\pgfpathlineto{\pgfqpoint{1.810200in}{1.090693in}}%
\pgfpathlineto{\pgfqpoint{1.828800in}{1.086652in}}%
\pgfpathlineto{\pgfqpoint{1.847400in}{1.070570in}}%
\pgfpathlineto{\pgfqpoint{1.866000in}{1.043653in}}%
\pgfpathlineto{\pgfqpoint{1.884600in}{1.007910in}}%
\pgfpathlineto{\pgfqpoint{1.903200in}{0.965895in}}%
\pgfpathlineto{\pgfqpoint{1.959000in}{0.828990in}}%
\pgfpathlineto{\pgfqpoint{1.977600in}{0.787280in}}%
\pgfpathlineto{\pgfqpoint{1.996200in}{0.750338in}}%
\pgfpathlineto{\pgfqpoint{2.014800in}{0.719568in}}%
\pgfpathlineto{\pgfqpoint{2.052000in}{0.746544in}}%
\pgfpathlineto{\pgfqpoint{2.070600in}{0.747382in}}%
\pgfpathlineto{\pgfqpoint{2.089200in}{0.733431in}}%
\pgfpathlineto{\pgfqpoint{2.107800in}{0.727239in}}%
\pgfpathlineto{\pgfqpoint{2.126400in}{0.775333in}}%
\pgfpathlineto{\pgfqpoint{2.145000in}{0.839802in}}%
\pgfpathlineto{\pgfqpoint{2.163600in}{0.917627in}}%
\pgfpathlineto{\pgfqpoint{2.219400in}{1.177462in}}%
\pgfpathlineto{\pgfqpoint{2.238000in}{1.251278in}}%
\pgfpathlineto{\pgfqpoint{2.256600in}{1.308299in}}%
\pgfpathlineto{\pgfqpoint{2.275200in}{1.343451in}}%
\pgfpathlineto{\pgfqpoint{2.293800in}{1.352734in}}%
\pgfpathlineto{\pgfqpoint{2.312400in}{1.333365in}}%
\pgfpathlineto{\pgfqpoint{2.331000in}{1.283937in}}%
\pgfpathlineto{\pgfqpoint{2.349600in}{1.204604in}}%
\pgfpathlineto{\pgfqpoint{2.368200in}{1.097245in}}%
\pgfpathlineto{\pgfqpoint{2.386800in}{0.965534in}}%
\pgfpathlineto{\pgfqpoint{2.405400in}{0.814901in}}%
\pgfpathlineto{\pgfqpoint{2.424000in}{0.777652in}}%
\pgfpathlineto{\pgfqpoint{2.461200in}{1.104780in}}%
\pgfpathlineto{\pgfqpoint{2.479800in}{1.251241in}}%
\pgfpathlineto{\pgfqpoint{2.498400in}{1.374796in}}%
\pgfpathlineto{\pgfqpoint{2.517000in}{1.468384in}}%
\pgfpathlineto{\pgfqpoint{2.535600in}{1.526958in}}%
\pgfpathlineto{\pgfqpoint{2.554200in}{1.547828in}}%
\pgfpathlineto{\pgfqpoint{2.572800in}{1.530669in}}%
\pgfpathlineto{\pgfqpoint{2.591400in}{1.477285in}}%
\pgfpathlineto{\pgfqpoint{2.610000in}{1.391237in}}%
\pgfpathlineto{\pgfqpoint{2.628600in}{1.277512in}}%
\pgfpathlineto{\pgfqpoint{2.647200in}{1.142312in}}%
\pgfpathlineto{\pgfqpoint{2.684400in}{0.837881in}}%
\pgfpathlineto{\pgfqpoint{2.703000in}{0.743602in}}%
\pgfpathlineto{\pgfqpoint{2.721600in}{0.881530in}}%
\pgfpathlineto{\pgfqpoint{2.740200in}{0.995922in}}%
\pgfpathlineto{\pgfqpoint{2.758800in}{1.077566in}}%
\pgfpathlineto{\pgfqpoint{2.777400in}{1.118946in}}%
\pgfpathlineto{\pgfqpoint{2.796000in}{1.115147in}}%
\pgfpathlineto{\pgfqpoint{2.814600in}{1.064582in}}%
\pgfpathlineto{\pgfqpoint{2.833200in}{0.969407in}}%
\pgfpathlineto{\pgfqpoint{2.851800in}{0.835557in}}%
\pgfpathlineto{\pgfqpoint{2.870400in}{0.757630in}}%
\pgfpathlineto{\pgfqpoint{2.926200in}{1.296460in}}%
\pgfpathlineto{\pgfqpoint{2.944800in}{1.446844in}}%
\pgfpathlineto{\pgfqpoint{2.963400in}{1.562126in}}%
\pgfpathlineto{\pgfqpoint{2.982000in}{1.633544in}}%
\pgfpathlineto{\pgfqpoint{3.000600in}{1.655756in}}%
\pgfpathlineto{\pgfqpoint{3.019200in}{1.627273in}}%
\pgfpathlineto{\pgfqpoint{3.037800in}{1.550593in}}%
\pgfpathlineto{\pgfqpoint{3.056400in}{1.431987in}}%
\pgfpathlineto{\pgfqpoint{3.075000in}{1.280869in}}%
\pgfpathlineto{\pgfqpoint{3.112200in}{0.928263in}}%
\pgfpathlineto{\pgfqpoint{3.130800in}{0.751287in}}%
\pgfpathlineto{\pgfqpoint{3.149400in}{0.841668in}}%
\pgfpathlineto{\pgfqpoint{3.168000in}{0.982602in}}%
\pgfpathlineto{\pgfqpoint{3.186600in}{1.096403in}}%
\pgfpathlineto{\pgfqpoint{3.205200in}{1.180880in}}%
\pgfpathlineto{\pgfqpoint{3.223800in}{1.236428in}}%
\pgfpathlineto{\pgfqpoint{3.242400in}{1.265449in}}%
\pgfpathlineto{\pgfqpoint{3.261000in}{1.271606in}}%
\pgfpathlineto{\pgfqpoint{3.279600in}{1.259047in}}%
\pgfpathlineto{\pgfqpoint{3.298200in}{1.231740in}}%
\pgfpathlineto{\pgfqpoint{3.316800in}{1.193035in}}%
\pgfpathlineto{\pgfqpoint{3.335400in}{1.145564in}}%
\pgfpathlineto{\pgfqpoint{3.354000in}{1.091413in}}%
\pgfpathlineto{\pgfqpoint{3.391200in}{0.970437in}}%
\pgfpathlineto{\pgfqpoint{3.428400in}{0.845751in}}%
\pgfpathlineto{\pgfqpoint{3.447000in}{0.787667in}}%
\pgfpathlineto{\pgfqpoint{3.465600in}{0.735919in}}%
\pgfpathlineto{\pgfqpoint{3.484200in}{0.736447in}}%
\pgfpathlineto{\pgfqpoint{3.502800in}{0.766033in}}%
\pgfpathlineto{\pgfqpoint{3.521400in}{0.779275in}}%
\pgfpathlineto{\pgfqpoint{3.540000in}{0.772911in}}%
\pgfpathlineto{\pgfqpoint{3.558600in}{0.744686in}}%
\pgfpathlineto{\pgfqpoint{3.577200in}{0.735870in}}%
\pgfpathlineto{\pgfqpoint{3.595800in}{0.806837in}}%
\pgfpathlineto{\pgfqpoint{3.614400in}{0.893688in}}%
\pgfpathlineto{\pgfqpoint{3.651600in}{1.085997in}}%
\pgfpathlineto{\pgfqpoint{3.670200in}{1.173874in}}%
\pgfpathlineto{\pgfqpoint{3.688800in}{1.244480in}}%
\pgfpathlineto{\pgfqpoint{3.707400in}{1.290427in}}%
\pgfpathlineto{\pgfqpoint{3.726000in}{1.306422in}}%
\pgfpathlineto{\pgfqpoint{3.744600in}{1.289847in}}%
\pgfpathlineto{\pgfqpoint{3.763200in}{1.241133in}}%
\pgfpathlineto{\pgfqpoint{3.781800in}{1.163850in}}%
\pgfpathlineto{\pgfqpoint{3.800400in}{1.064458in}}%
\pgfpathlineto{\pgfqpoint{3.856200in}{0.726044in}}%
\pgfpathlineto{\pgfqpoint{3.893400in}{0.868979in}}%
\pgfpathlineto{\pgfqpoint{3.912000in}{0.913753in}}%
\pgfpathlineto{\pgfqpoint{3.930600in}{0.931172in}}%
\pgfpathlineto{\pgfqpoint{3.949200in}{0.923290in}}%
\pgfpathlineto{\pgfqpoint{3.967800in}{0.894580in}}%
\pgfpathlineto{\pgfqpoint{3.986400in}{0.851180in}}%
\pgfpathlineto{\pgfqpoint{4.023600in}{0.747810in}}%
\pgfpathlineto{\pgfqpoint{4.042200in}{0.729534in}}%
\pgfpathlineto{\pgfqpoint{4.060800in}{0.767657in}}%
\pgfpathlineto{\pgfqpoint{4.079400in}{0.793942in}}%
\pgfpathlineto{\pgfqpoint{4.098000in}{0.807530in}}%
\pgfpathlineto{\pgfqpoint{4.116600in}{0.809073in}}%
\pgfpathlineto{\pgfqpoint{4.135200in}{0.800301in}}%
\pgfpathlineto{\pgfqpoint{4.153800in}{0.783475in}}%
\pgfpathlineto{\pgfqpoint{4.172400in}{0.760852in}}%
\pgfpathlineto{\pgfqpoint{4.191000in}{0.734243in}}%
\pgfpathlineto{\pgfqpoint{4.209600in}{0.725276in}}%
\pgfpathlineto{\pgfqpoint{4.228200in}{0.757482in}}%
\pgfpathlineto{\pgfqpoint{4.246800in}{0.792953in}}%
\pgfpathlineto{\pgfqpoint{4.265400in}{0.832850in}}%
\pgfpathlineto{\pgfqpoint{4.284000in}{0.878535in}}%
\pgfpathlineto{\pgfqpoint{4.302600in}{0.931037in}}%
\pgfpathlineto{\pgfqpoint{4.321200in}{0.990448in}}%
\pgfpathlineto{\pgfqpoint{4.377000in}{1.187466in}}%
\pgfpathlineto{\pgfqpoint{4.395600in}{1.243751in}}%
\pgfpathlineto{\pgfqpoint{4.414200in}{1.285257in}}%
\pgfpathlineto{\pgfqpoint{4.432800in}{1.306515in}}%
\pgfpathlineto{\pgfqpoint{4.451400in}{1.303837in}}%
\pgfpathlineto{\pgfqpoint{4.470000in}{1.276004in}}%
\pgfpathlineto{\pgfqpoint{4.480000in}{1.248308in}}%
\pgfpathlineto{\pgfqpoint{4.480000in}{1.248308in}}%
\pgfusepath{stroke}%
\end{pgfscope}%
\begin{pgfscope}%
\pgfpathrectangle{\pgfqpoint{0.750000in}{0.330000in}}{\pgfqpoint{3.720000in}{2.310000in}}%
\pgfusepath{clip}%
\pgfsetrectcap%
\pgfsetroundjoin%
\pgfsetlinewidth{2.509375pt}%
\definecolor{currentstroke}{rgb}{0.203922,0.541176,0.741176}%
\pgfsetstrokecolor{currentstroke}%
\pgfsetdash{}{0pt}%
\pgfpathmoveto{\pgfqpoint{0.740000in}{1.169241in}}%
\pgfpathlineto{\pgfqpoint{0.750000in}{1.168695in}}%
\pgfpathlineto{\pgfqpoint{0.768600in}{1.165468in}}%
\pgfpathlineto{\pgfqpoint{0.787200in}{1.159855in}}%
\pgfpathlineto{\pgfqpoint{0.805800in}{1.151766in}}%
\pgfpathlineto{\pgfqpoint{0.824400in}{1.141205in}}%
\pgfpathlineto{\pgfqpoint{0.843000in}{1.128280in}}%
\pgfpathlineto{\pgfqpoint{0.861600in}{1.113213in}}%
\pgfpathlineto{\pgfqpoint{0.898800in}{1.078124in}}%
\pgfpathlineto{\pgfqpoint{0.954600in}{1.021676in}}%
\pgfpathlineto{\pgfqpoint{0.973200in}{1.004821in}}%
\pgfpathlineto{\pgfqpoint{0.991800in}{0.990332in}}%
\pgfpathlineto{\pgfqpoint{1.010400in}{0.978959in}}%
\pgfpathlineto{\pgfqpoint{1.029000in}{0.971255in}}%
\pgfpathlineto{\pgfqpoint{1.047600in}{0.967483in}}%
\pgfpathlineto{\pgfqpoint{1.066200in}{0.967565in}}%
\pgfpathlineto{\pgfqpoint{1.084800in}{0.971114in}}%
\pgfpathlineto{\pgfqpoint{1.103400in}{0.977534in}}%
\pgfpathlineto{\pgfqpoint{1.122000in}{0.986151in}}%
\pgfpathlineto{\pgfqpoint{1.159200in}{1.007463in}}%
\pgfpathlineto{\pgfqpoint{1.289400in}{1.089203in}}%
\pgfpathlineto{\pgfqpoint{1.326600in}{1.109873in}}%
\pgfpathlineto{\pgfqpoint{1.512600in}{1.206277in}}%
\pgfpathlineto{\pgfqpoint{1.531200in}{1.213480in}}%
\pgfpathlineto{\pgfqpoint{1.549800in}{1.218939in}}%
\pgfpathlineto{\pgfqpoint{1.568400in}{1.222384in}}%
\pgfpathlineto{\pgfqpoint{1.587000in}{1.223658in}}%
\pgfpathlineto{\pgfqpoint{1.605600in}{1.222726in}}%
\pgfpathlineto{\pgfqpoint{1.624200in}{1.219681in}}%
\pgfpathlineto{\pgfqpoint{1.642800in}{1.214729in}}%
\pgfpathlineto{\pgfqpoint{1.661400in}{1.208147in}}%
\pgfpathlineto{\pgfqpoint{1.698600in}{1.191238in}}%
\pgfpathlineto{\pgfqpoint{1.735800in}{1.170415in}}%
\pgfpathlineto{\pgfqpoint{1.754400in}{1.158416in}}%
\pgfpathlineto{\pgfqpoint{1.773000in}{1.145008in}}%
\pgfpathlineto{\pgfqpoint{1.791600in}{1.129828in}}%
\pgfpathlineto{\pgfqpoint{1.810200in}{1.112526in}}%
\pgfpathlineto{\pgfqpoint{1.828800in}{1.092854in}}%
\pgfpathlineto{\pgfqpoint{1.847400in}{1.070718in}}%
\pgfpathlineto{\pgfqpoint{1.866000in}{1.046200in}}%
\pgfpathlineto{\pgfqpoint{1.903200in}{0.991019in}}%
\pgfpathlineto{\pgfqpoint{1.940400in}{0.929844in}}%
\pgfpathlineto{\pgfqpoint{1.977600in}{0.864670in}}%
\pgfpathlineto{\pgfqpoint{2.033400in}{0.762567in}}%
\pgfpathlineto{\pgfqpoint{2.052000in}{0.746545in}}%
\pgfpathlineto{\pgfqpoint{2.070600in}{0.774162in}}%
\pgfpathlineto{\pgfqpoint{2.089200in}{0.818234in}}%
\pgfpathlineto{\pgfqpoint{2.107800in}{0.867888in}}%
\pgfpathlineto{\pgfqpoint{2.145000in}{0.976464in}}%
\pgfpathlineto{\pgfqpoint{2.219400in}{1.202912in}}%
\pgfpathlineto{\pgfqpoint{2.256600in}{1.308679in}}%
\pgfpathlineto{\pgfqpoint{2.293800in}{1.404777in}}%
\pgfpathlineto{\pgfqpoint{2.312400in}{1.448429in}}%
\pgfpathlineto{\pgfqpoint{2.331000in}{1.488676in}}%
\pgfpathlineto{\pgfqpoint{2.349600in}{1.525103in}}%
\pgfpathlineto{\pgfqpoint{2.368200in}{1.557264in}}%
\pgfpathlineto{\pgfqpoint{2.386800in}{1.584695in}}%
\pgfpathlineto{\pgfqpoint{2.405400in}{1.606936in}}%
\pgfpathlineto{\pgfqpoint{2.424000in}{1.623544in}}%
\pgfpathlineto{\pgfqpoint{2.442600in}{1.634122in}}%
\pgfpathlineto{\pgfqpoint{2.461200in}{1.638345in}}%
\pgfpathlineto{\pgfqpoint{2.479800in}{1.636007in}}%
\pgfpathlineto{\pgfqpoint{2.498400in}{1.627061in}}%
\pgfpathlineto{\pgfqpoint{2.517000in}{1.611649in}}%
\pgfpathlineto{\pgfqpoint{2.535600in}{1.590104in}}%
\pgfpathlineto{\pgfqpoint{2.554200in}{1.562918in}}%
\pgfpathlineto{\pgfqpoint{2.572800in}{1.530692in}}%
\pgfpathlineto{\pgfqpoint{2.591400in}{1.494080in}}%
\pgfpathlineto{\pgfqpoint{2.610000in}{1.453746in}}%
\pgfpathlineto{\pgfqpoint{2.647200in}{1.364655in}}%
\pgfpathlineto{\pgfqpoint{2.703000in}{1.224479in}}%
\pgfpathlineto{\pgfqpoint{2.721600in}{1.182940in}}%
\pgfpathlineto{\pgfqpoint{2.740200in}{1.149101in}}%
\pgfpathlineto{\pgfqpoint{2.758800in}{1.127175in}}%
\pgfpathlineto{\pgfqpoint{2.777400in}{1.121096in}}%
\pgfpathlineto{\pgfqpoint{2.796000in}{1.132939in}}%
\pgfpathlineto{\pgfqpoint{2.814600in}{1.161873in}}%
\pgfpathlineto{\pgfqpoint{2.833200in}{1.204667in}}%
\pgfpathlineto{\pgfqpoint{2.851800in}{1.257131in}}%
\pgfpathlineto{\pgfqpoint{2.889000in}{1.375675in}}%
\pgfpathlineto{\pgfqpoint{2.926200in}{1.493165in}}%
\pgfpathlineto{\pgfqpoint{2.944800in}{1.546387in}}%
\pgfpathlineto{\pgfqpoint{2.963400in}{1.593975in}}%
\pgfpathlineto{\pgfqpoint{2.982000in}{1.634800in}}%
\pgfpathlineto{\pgfqpoint{3.000600in}{1.667941in}}%
\pgfpathlineto{\pgfqpoint{3.019200in}{1.692671in}}%
\pgfpathlineto{\pgfqpoint{3.037800in}{1.708468in}}%
\pgfpathlineto{\pgfqpoint{3.056400in}{1.715040in}}%
\pgfpathlineto{\pgfqpoint{3.075000in}{1.712356in}}%
\pgfpathlineto{\pgfqpoint{3.093600in}{1.700667in}}%
\pgfpathlineto{\pgfqpoint{3.112200in}{1.680497in}}%
\pgfpathlineto{\pgfqpoint{3.130800in}{1.652630in}}%
\pgfpathlineto{\pgfqpoint{3.149400in}{1.618056in}}%
\pgfpathlineto{\pgfqpoint{3.168000in}{1.577919in}}%
\pgfpathlineto{\pgfqpoint{3.186600in}{1.533448in}}%
\pgfpathlineto{\pgfqpoint{3.223800in}{1.436500in}}%
\pgfpathlineto{\pgfqpoint{3.279600in}{1.287438in}}%
\pgfpathlineto{\pgfqpoint{3.316800in}{1.193315in}}%
\pgfpathlineto{\pgfqpoint{3.465600in}{0.831149in}}%
\pgfpathlineto{\pgfqpoint{3.484200in}{0.788738in}}%
\pgfpathlineto{\pgfqpoint{3.502800in}{0.768353in}}%
\pgfpathlineto{\pgfqpoint{3.521400in}{0.795292in}}%
\pgfpathlineto{\pgfqpoint{3.540000in}{0.845070in}}%
\pgfpathlineto{\pgfqpoint{3.577200in}{0.959655in}}%
\pgfpathlineto{\pgfqpoint{3.614400in}{1.074784in}}%
\pgfpathlineto{\pgfqpoint{3.633000in}{1.128549in}}%
\pgfpathlineto{\pgfqpoint{3.651600in}{1.178051in}}%
\pgfpathlineto{\pgfqpoint{3.670200in}{1.222200in}}%
\pgfpathlineto{\pgfqpoint{3.688800in}{1.260073in}}%
\pgfpathlineto{\pgfqpoint{3.707400in}{1.290890in}}%
\pgfpathlineto{\pgfqpoint{3.726000in}{1.314001in}}%
\pgfpathlineto{\pgfqpoint{3.744600in}{1.328874in}}%
\pgfpathlineto{\pgfqpoint{3.763200in}{1.335109in}}%
\pgfpathlineto{\pgfqpoint{3.781800in}{1.332470in}}%
\pgfpathlineto{\pgfqpoint{3.800400in}{1.320913in}}%
\pgfpathlineto{\pgfqpoint{3.819000in}{1.300625in}}%
\pgfpathlineto{\pgfqpoint{3.837600in}{1.272031in}}%
\pgfpathlineto{\pgfqpoint{3.856200in}{1.235800in}}%
\pgfpathlineto{\pgfqpoint{3.874800in}{1.192816in}}%
\pgfpathlineto{\pgfqpoint{3.893400in}{1.144143in}}%
\pgfpathlineto{\pgfqpoint{3.930600in}{1.034533in}}%
\pgfpathlineto{\pgfqpoint{4.005000in}{0.801860in}}%
\pgfpathlineto{\pgfqpoint{4.023600in}{0.749383in}}%
\pgfpathlineto{\pgfqpoint{4.042200in}{0.739935in}}%
\pgfpathlineto{\pgfqpoint{4.079400in}{0.825606in}}%
\pgfpathlineto{\pgfqpoint{4.098000in}{0.863147in}}%
\pgfpathlineto{\pgfqpoint{4.116600in}{0.896148in}}%
\pgfpathlineto{\pgfqpoint{4.135200in}{0.924870in}}%
\pgfpathlineto{\pgfqpoint{4.153800in}{0.949873in}}%
\pgfpathlineto{\pgfqpoint{4.172400in}{0.971960in}}%
\pgfpathlineto{\pgfqpoint{4.246800in}{1.051960in}}%
\pgfpathlineto{\pgfqpoint{4.265400in}{1.074897in}}%
\pgfpathlineto{\pgfqpoint{4.284000in}{1.100273in}}%
\pgfpathlineto{\pgfqpoint{4.321200in}{1.157606in}}%
\pgfpathlineto{\pgfqpoint{4.358400in}{1.218079in}}%
\pgfpathlineto{\pgfqpoint{4.377000in}{1.246276in}}%
\pgfpathlineto{\pgfqpoint{4.395600in}{1.271231in}}%
\pgfpathlineto{\pgfqpoint{4.414200in}{1.291691in}}%
\pgfpathlineto{\pgfqpoint{4.432800in}{1.306657in}}%
\pgfpathlineto{\pgfqpoint{4.451400in}{1.315460in}}%
\pgfpathlineto{\pgfqpoint{4.470000in}{1.317791in}}%
\pgfpathlineto{\pgfqpoint{4.480000in}{1.315596in}}%
\pgfpathlineto{\pgfqpoint{4.480000in}{1.315596in}}%
\pgfusepath{stroke}%
\end{pgfscope}%
\begin{pgfscope}%
\pgfpathrectangle{\pgfqpoint{0.750000in}{0.330000in}}{\pgfqpoint{3.720000in}{2.310000in}}%
\pgfusepath{clip}%
\pgfsetrectcap%
\pgfsetroundjoin%
\pgfsetlinewidth{2.509375pt}%
\definecolor{currentstroke}{rgb}{0.650980,0.023529,0.156863}%
\pgfsetstrokecolor{currentstroke}%
\pgfsetdash{}{0pt}%
\pgfpathmoveto{\pgfqpoint{0.740000in}{1.140984in}}%
\pgfpathlineto{\pgfqpoint{0.936000in}{1.033756in}}%
\pgfpathlineto{\pgfqpoint{1.159200in}{1.000184in}}%
\pgfpathlineto{\pgfqpoint{1.475400in}{1.185698in}}%
\pgfpathlineto{\pgfqpoint{1.810200in}{1.090693in}}%
\pgfpathlineto{\pgfqpoint{2.070600in}{0.747382in}}%
\pgfpathlineto{\pgfqpoint{2.293800in}{1.352734in}}%
\pgfpathlineto{\pgfqpoint{2.554200in}{1.547828in}}%
\pgfpathlineto{\pgfqpoint{2.777400in}{1.118946in}}%
\pgfpathlineto{\pgfqpoint{3.000600in}{1.655756in}}%
\pgfpathlineto{\pgfqpoint{3.261000in}{1.271606in}}%
\pgfpathlineto{\pgfqpoint{3.521400in}{0.779275in}}%
\pgfpathlineto{\pgfqpoint{3.726000in}{1.306422in}}%
\pgfpathlineto{\pgfqpoint{3.930600in}{0.931172in}}%
\pgfpathlineto{\pgfqpoint{4.116600in}{0.809073in}}%
\pgfpathlineto{\pgfqpoint{4.432800in}{1.306515in}}%
\pgfpathlineto{\pgfqpoint{4.480000in}{1.260117in}}%
\pgfusepath{stroke}%
\end{pgfscope}%
\begin{pgfscope}%
\pgfpathrectangle{\pgfqpoint{0.750000in}{0.330000in}}{\pgfqpoint{3.720000in}{2.310000in}}%
\pgfusepath{clip}%
\pgfsetrectcap%
\pgfsetroundjoin%
\pgfsetlinewidth{2.509375pt}%
\definecolor{currentstroke}{rgb}{0.478431,0.407843,0.650980}%
\pgfsetstrokecolor{currentstroke}%
\pgfsetdash{}{0pt}%
\pgfpathmoveto{\pgfqpoint{0.740000in}{1.133698in}}%
\pgfpathlineto{\pgfqpoint{0.750000in}{1.120216in}}%
\pgfpathlineto{\pgfqpoint{0.787200in}{1.062760in}}%
\pgfpathlineto{\pgfqpoint{0.805800in}{1.041497in}}%
\pgfpathlineto{\pgfqpoint{0.824400in}{1.030827in}}%
\pgfpathlineto{\pgfqpoint{0.843000in}{1.030535in}}%
\pgfpathlineto{\pgfqpoint{0.880200in}{1.043580in}}%
\pgfpathlineto{\pgfqpoint{0.898800in}{1.047027in}}%
\pgfpathlineto{\pgfqpoint{0.917400in}{1.044629in}}%
\pgfpathlineto{\pgfqpoint{0.936000in}{1.036247in}}%
\pgfpathlineto{\pgfqpoint{0.954600in}{1.023443in}}%
\pgfpathlineto{\pgfqpoint{0.991800in}{0.995091in}}%
\pgfpathlineto{\pgfqpoint{1.010400in}{0.984758in}}%
\pgfpathlineto{\pgfqpoint{1.029000in}{0.979277in}}%
\pgfpathlineto{\pgfqpoint{1.047600in}{0.979015in}}%
\pgfpathlineto{\pgfqpoint{1.066200in}{0.983216in}}%
\pgfpathlineto{\pgfqpoint{1.122000in}{1.003021in}}%
\pgfpathlineto{\pgfqpoint{1.140600in}{1.004431in}}%
\pgfpathlineto{\pgfqpoint{1.159200in}{1.000469in}}%
\pgfpathlineto{\pgfqpoint{1.177800in}{0.990703in}}%
\pgfpathlineto{\pgfqpoint{1.196400in}{0.975753in}}%
\pgfpathlineto{\pgfqpoint{1.233600in}{0.939185in}}%
\pgfpathlineto{\pgfqpoint{1.252200in}{0.925617in}}%
\pgfpathlineto{\pgfqpoint{1.270800in}{0.921917in}}%
\pgfpathlineto{\pgfqpoint{1.289400in}{0.931211in}}%
\pgfpathlineto{\pgfqpoint{1.308000in}{0.952737in}}%
\pgfpathlineto{\pgfqpoint{1.326600in}{0.982885in}}%
\pgfpathlineto{\pgfqpoint{1.401000in}{1.119904in}}%
\pgfpathlineto{\pgfqpoint{1.419600in}{1.146934in}}%
\pgfpathlineto{\pgfqpoint{1.438200in}{1.167876in}}%
\pgfpathlineto{\pgfqpoint{1.456800in}{1.181408in}}%
\pgfpathlineto{\pgfqpoint{1.475400in}{1.186443in}}%
\pgfpathlineto{\pgfqpoint{1.494000in}{1.182323in}}%
\pgfpathlineto{\pgfqpoint{1.512600in}{1.168995in}}%
\pgfpathlineto{\pgfqpoint{1.531200in}{1.147119in}}%
\pgfpathlineto{\pgfqpoint{1.549800in}{1.118082in}}%
\pgfpathlineto{\pgfqpoint{1.587000in}{1.047549in}}%
\pgfpathlineto{\pgfqpoint{1.605600in}{1.012273in}}%
\pgfpathlineto{\pgfqpoint{1.624200in}{0.982311in}}%
\pgfpathlineto{\pgfqpoint{1.642800in}{0.962208in}}%
\pgfpathlineto{\pgfqpoint{1.661400in}{0.955591in}}%
\pgfpathlineto{\pgfqpoint{1.680000in}{0.963131in}}%
\pgfpathlineto{\pgfqpoint{1.698600in}{0.981875in}}%
\pgfpathlineto{\pgfqpoint{1.754400in}{1.057584in}}%
\pgfpathlineto{\pgfqpoint{1.773000in}{1.076438in}}%
\pgfpathlineto{\pgfqpoint{1.791600in}{1.088121in}}%
\pgfpathlineto{\pgfqpoint{1.810200in}{1.091649in}}%
\pgfpathlineto{\pgfqpoint{1.828800in}{1.086880in}}%
\pgfpathlineto{\pgfqpoint{1.847400in}{1.074327in}}%
\pgfpathlineto{\pgfqpoint{1.866000in}{1.054906in}}%
\pgfpathlineto{\pgfqpoint{1.884600in}{1.029742in}}%
\pgfpathlineto{\pgfqpoint{1.903200in}{1.000094in}}%
\pgfpathlineto{\pgfqpoint{1.940400in}{0.933234in}}%
\pgfpathlineto{\pgfqpoint{1.977600in}{0.867612in}}%
\pgfpathlineto{\pgfqpoint{1.996200in}{0.839176in}}%
\pgfpathlineto{\pgfqpoint{2.014800in}{0.814260in}}%
\pgfpathlineto{\pgfqpoint{2.052000in}{0.767784in}}%
\pgfpathlineto{\pgfqpoint{2.070600in}{0.747494in}}%
\pgfpathlineto{\pgfqpoint{2.089200in}{0.763586in}}%
\pgfpathlineto{\pgfqpoint{2.107800in}{0.814588in}}%
\pgfpathlineto{\pgfqpoint{2.126400in}{0.881310in}}%
\pgfpathlineto{\pgfqpoint{2.163600in}{1.037416in}}%
\pgfpathlineto{\pgfqpoint{2.182200in}{1.116242in}}%
\pgfpathlineto{\pgfqpoint{2.200800in}{1.189042in}}%
\pgfpathlineto{\pgfqpoint{2.219400in}{1.251870in}}%
\pgfpathlineto{\pgfqpoint{2.238000in}{1.301663in}}%
\pgfpathlineto{\pgfqpoint{2.256600in}{1.336102in}}%
\pgfpathlineto{\pgfqpoint{2.275200in}{1.353577in}}%
\pgfpathlineto{\pgfqpoint{2.293800in}{1.353435in}}%
\pgfpathlineto{\pgfqpoint{2.312400in}{1.336507in}}%
\pgfpathlineto{\pgfqpoint{2.331000in}{1.305812in}}%
\pgfpathlineto{\pgfqpoint{2.368200in}{1.230555in}}%
\pgfpathlineto{\pgfqpoint{2.386800in}{1.207855in}}%
\pgfpathlineto{\pgfqpoint{2.405400in}{1.210578in}}%
\pgfpathlineto{\pgfqpoint{2.424000in}{1.242551in}}%
\pgfpathlineto{\pgfqpoint{2.442600in}{1.297506in}}%
\pgfpathlineto{\pgfqpoint{2.479800in}{1.429353in}}%
\pgfpathlineto{\pgfqpoint{2.498400in}{1.485619in}}%
\pgfpathlineto{\pgfqpoint{2.517000in}{1.526503in}}%
\pgfpathlineto{\pgfqpoint{2.535600in}{1.548608in}}%
\pgfpathlineto{\pgfqpoint{2.554200in}{1.550538in}}%
\pgfpathlineto{\pgfqpoint{2.572800in}{1.532541in}}%
\pgfpathlineto{\pgfqpoint{2.591400in}{1.496455in}}%
\pgfpathlineto{\pgfqpoint{2.610000in}{1.445879in}}%
\pgfpathlineto{\pgfqpoint{2.665800in}{1.270763in}}%
\pgfpathlineto{\pgfqpoint{2.684400in}{1.229639in}}%
\pgfpathlineto{\pgfqpoint{2.703000in}{1.203980in}}%
\pgfpathlineto{\pgfqpoint{2.721600in}{1.189638in}}%
\pgfpathlineto{\pgfqpoint{2.740200in}{1.178463in}}%
\pgfpathlineto{\pgfqpoint{2.758800in}{1.162963in}}%
\pgfpathlineto{\pgfqpoint{2.777400in}{1.140385in}}%
\pgfpathlineto{\pgfqpoint{2.796000in}{1.115334in}}%
\pgfpathlineto{\pgfqpoint{2.814600in}{1.100692in}}%
\pgfpathlineto{\pgfqpoint{2.833200in}{1.113478in}}%
\pgfpathlineto{\pgfqpoint{2.851800in}{1.162851in}}%
\pgfpathlineto{\pgfqpoint{2.870400in}{1.242585in}}%
\pgfpathlineto{\pgfqpoint{2.907600in}{1.435217in}}%
\pgfpathlineto{\pgfqpoint{2.926200in}{1.522754in}}%
\pgfpathlineto{\pgfqpoint{2.944800in}{1.592745in}}%
\pgfpathlineto{\pgfqpoint{2.963400in}{1.640000in}}%
\pgfpathlineto{\pgfqpoint{2.982000in}{1.661936in}}%
\pgfpathlineto{\pgfqpoint{3.000600in}{1.658475in}}%
\pgfpathlineto{\pgfqpoint{3.019200in}{1.631878in}}%
\pgfpathlineto{\pgfqpoint{3.037800in}{1.586362in}}%
\pgfpathlineto{\pgfqpoint{3.056400in}{1.527494in}}%
\pgfpathlineto{\pgfqpoint{3.093600in}{1.395082in}}%
\pgfpathlineto{\pgfqpoint{3.112200in}{1.334643in}}%
\pgfpathlineto{\pgfqpoint{3.130800in}{1.286445in}}%
\pgfpathlineto{\pgfqpoint{3.149400in}{1.255175in}}%
\pgfpathlineto{\pgfqpoint{3.168000in}{1.242126in}}%
\pgfpathlineto{\pgfqpoint{3.186600in}{1.244103in}}%
\pgfpathlineto{\pgfqpoint{3.223800in}{1.266297in}}%
\pgfpathlineto{\pgfqpoint{3.242400in}{1.273337in}}%
\pgfpathlineto{\pgfqpoint{3.261000in}{1.271959in}}%
\pgfpathlineto{\pgfqpoint{3.279600in}{1.260550in}}%
\pgfpathlineto{\pgfqpoint{3.298200in}{1.239179in}}%
\pgfpathlineto{\pgfqpoint{3.316800in}{1.209038in}}%
\pgfpathlineto{\pgfqpoint{3.335400in}{1.171931in}}%
\pgfpathlineto{\pgfqpoint{3.354000in}{1.129894in}}%
\pgfpathlineto{\pgfqpoint{3.428400in}{0.953043in}}%
\pgfpathlineto{\pgfqpoint{3.447000in}{0.915784in}}%
\pgfpathlineto{\pgfqpoint{3.465600in}{0.883060in}}%
\pgfpathlineto{\pgfqpoint{3.521400in}{0.792149in}}%
\pgfpathlineto{\pgfqpoint{3.540000in}{0.776435in}}%
\pgfpathlineto{\pgfqpoint{3.558600in}{0.810673in}}%
\pgfpathlineto{\pgfqpoint{3.577200in}{0.879245in}}%
\pgfpathlineto{\pgfqpoint{3.633000in}{1.128329in}}%
\pgfpathlineto{\pgfqpoint{3.651600in}{1.198944in}}%
\pgfpathlineto{\pgfqpoint{3.670200in}{1.254217in}}%
\pgfpathlineto{\pgfqpoint{3.688800in}{1.291295in}}%
\pgfpathlineto{\pgfqpoint{3.707400in}{1.309171in}}%
\pgfpathlineto{\pgfqpoint{3.726000in}{1.308664in}}%
\pgfpathlineto{\pgfqpoint{3.744600in}{1.292323in}}%
\pgfpathlineto{\pgfqpoint{3.763200in}{1.264052in}}%
\pgfpathlineto{\pgfqpoint{3.781800in}{1.228309in}}%
\pgfpathlineto{\pgfqpoint{3.837600in}{1.108271in}}%
\pgfpathlineto{\pgfqpoint{3.874800in}{1.028277in}}%
\pgfpathlineto{\pgfqpoint{3.893400in}{0.991467in}}%
\pgfpathlineto{\pgfqpoint{3.912000in}{0.960601in}}%
\pgfpathlineto{\pgfqpoint{3.930600in}{0.938340in}}%
\pgfpathlineto{\pgfqpoint{3.949200in}{0.924833in}}%
\pgfpathlineto{\pgfqpoint{4.005000in}{0.900543in}}%
\pgfpathlineto{\pgfqpoint{4.023600in}{0.886346in}}%
\pgfpathlineto{\pgfqpoint{4.079400in}{0.830785in}}%
\pgfpathlineto{\pgfqpoint{4.098000in}{0.817366in}}%
\pgfpathlineto{\pgfqpoint{4.116600in}{0.809204in}}%
\pgfpathlineto{\pgfqpoint{4.135200in}{0.804863in}}%
\pgfpathlineto{\pgfqpoint{4.153800in}{0.802342in}}%
\pgfpathlineto{\pgfqpoint{4.172400in}{0.801121in}}%
\pgfpathlineto{\pgfqpoint{4.191000in}{0.802878in}}%
\pgfpathlineto{\pgfqpoint{4.209600in}{0.810678in}}%
\pgfpathlineto{\pgfqpoint{4.228200in}{0.827141in}}%
\pgfpathlineto{\pgfqpoint{4.246800in}{0.853353in}}%
\pgfpathlineto{\pgfqpoint{4.265400in}{0.889407in}}%
\pgfpathlineto{\pgfqpoint{4.284000in}{0.935037in}}%
\pgfpathlineto{\pgfqpoint{4.302600in}{0.989400in}}%
\pgfpathlineto{\pgfqpoint{4.339800in}{1.114555in}}%
\pgfpathlineto{\pgfqpoint{4.358400in}{1.176754in}}%
\pgfpathlineto{\pgfqpoint{4.377000in}{1.231544in}}%
\pgfpathlineto{\pgfqpoint{4.395600in}{1.273994in}}%
\pgfpathlineto{\pgfqpoint{4.414200in}{1.300733in}}%
\pgfpathlineto{\pgfqpoint{4.432800in}{1.310468in}}%
\pgfpathlineto{\pgfqpoint{4.451400in}{1.303900in}}%
\pgfpathlineto{\pgfqpoint{4.470000in}{1.283128in}}%
\pgfpathlineto{\pgfqpoint{4.480000in}{1.265775in}}%
\pgfpathlineto{\pgfqpoint{4.480000in}{1.265775in}}%
\pgfusepath{stroke}%
\end{pgfscope}%
\begin{pgfscope}%
\pgfpathrectangle{\pgfqpoint{0.750000in}{0.330000in}}{\pgfqpoint{3.720000in}{2.310000in}}%
\pgfusepath{clip}%
\pgfsetrectcap%
\pgfsetroundjoin%
\pgfsetlinewidth{2.509375pt}%
\definecolor{currentstroke}{rgb}{0.274510,0.470588,0.129412}%
\pgfsetstrokecolor{currentstroke}%
\pgfsetdash{}{0pt}%
\pgfpathmoveto{\pgfqpoint{0.740000in}{1.245456in}}%
\pgfpathlineto{\pgfqpoint{0.750000in}{1.296625in}}%
\pgfpathlineto{\pgfqpoint{0.787200in}{1.551604in}}%
\pgfpathlineto{\pgfqpoint{0.805800in}{1.657312in}}%
\pgfpathlineto{\pgfqpoint{0.824400in}{1.721290in}}%
\pgfpathlineto{\pgfqpoint{0.843000in}{1.730251in}}%
\pgfpathlineto{\pgfqpoint{0.861600in}{1.677571in}}%
\pgfpathlineto{\pgfqpoint{0.880200in}{1.564489in}}%
\pgfpathlineto{\pgfqpoint{0.898800in}{1.401736in}}%
\pgfpathlineto{\pgfqpoint{0.917400in}{1.214030in}}%
\pgfpathlineto{\pgfqpoint{0.936000in}{1.058982in}}%
\pgfpathlineto{\pgfqpoint{0.954600in}{1.050545in}}%
\pgfpathlineto{\pgfqpoint{0.973200in}{1.186297in}}%
\pgfpathlineto{\pgfqpoint{0.991800in}{1.346986in}}%
\pgfpathlineto{\pgfqpoint{1.010400in}{1.475446in}}%
\pgfpathlineto{\pgfqpoint{1.029000in}{1.550874in}}%
\pgfpathlineto{\pgfqpoint{1.047600in}{1.567364in}}%
\pgfpathlineto{\pgfqpoint{1.066200in}{1.528067in}}%
\pgfpathlineto{\pgfqpoint{1.084800in}{1.442512in}}%
\pgfpathlineto{\pgfqpoint{1.103400in}{1.324810in}}%
\pgfpathlineto{\pgfqpoint{1.122000in}{1.193217in}}%
\pgfpathlineto{\pgfqpoint{1.140600in}{1.073039in}}%
\pgfpathlineto{\pgfqpoint{1.159200in}{1.003159in}}%
\pgfpathlineto{\pgfqpoint{1.177800in}{1.016870in}}%
\pgfpathlineto{\pgfqpoint{1.196400in}{1.089439in}}%
\pgfpathlineto{\pgfqpoint{1.215000in}{1.177314in}}%
\pgfpathlineto{\pgfqpoint{1.233600in}{1.258848in}}%
\pgfpathlineto{\pgfqpoint{1.252200in}{1.326120in}}%
\pgfpathlineto{\pgfqpoint{1.270800in}{1.376349in}}%
\pgfpathlineto{\pgfqpoint{1.289400in}{1.408543in}}%
\pgfpathlineto{\pgfqpoint{1.308000in}{1.422676in}}%
\pgfpathlineto{\pgfqpoint{1.326600in}{1.419972in}}%
\pgfpathlineto{\pgfqpoint{1.345200in}{1.403272in}}%
\pgfpathlineto{\pgfqpoint{1.363800in}{1.376791in}}%
\pgfpathlineto{\pgfqpoint{1.401000in}{1.311870in}}%
\pgfpathlineto{\pgfqpoint{1.438200in}{1.247033in}}%
\pgfpathlineto{\pgfqpoint{1.456800in}{1.217210in}}%
\pgfpathlineto{\pgfqpoint{1.475400in}{1.193459in}}%
\pgfpathlineto{\pgfqpoint{1.494000in}{1.183621in}}%
\pgfpathlineto{\pgfqpoint{1.512600in}{1.196643in}}%
\pgfpathlineto{\pgfqpoint{1.531200in}{1.235974in}}%
\pgfpathlineto{\pgfqpoint{1.549800in}{1.295641in}}%
\pgfpathlineto{\pgfqpoint{1.568400in}{1.363451in}}%
\pgfpathlineto{\pgfqpoint{1.587000in}{1.426498in}}%
\pgfpathlineto{\pgfqpoint{1.605600in}{1.474528in}}%
\pgfpathlineto{\pgfqpoint{1.624200in}{1.501335in}}%
\pgfpathlineto{\pgfqpoint{1.642800in}{1.505102in}}%
\pgfpathlineto{\pgfqpoint{1.661400in}{1.487896in}}%
\pgfpathlineto{\pgfqpoint{1.680000in}{1.454353in}}%
\pgfpathlineto{\pgfqpoint{1.698600in}{1.409894in}}%
\pgfpathlineto{\pgfqpoint{1.717200in}{1.359078in}}%
\pgfpathlineto{\pgfqpoint{1.754400in}{1.248247in}}%
\pgfpathlineto{\pgfqpoint{1.773000in}{1.191362in}}%
\pgfpathlineto{\pgfqpoint{1.791600in}{1.138829in}}%
\pgfpathlineto{\pgfqpoint{1.810200in}{1.100605in}}%
\pgfpathlineto{\pgfqpoint{1.828800in}{1.089037in}}%
\pgfpathlineto{\pgfqpoint{1.847400in}{1.108191in}}%
\pgfpathlineto{\pgfqpoint{1.866000in}{1.147346in}}%
\pgfpathlineto{\pgfqpoint{1.884600in}{1.189275in}}%
\pgfpathlineto{\pgfqpoint{1.903200in}{1.220177in}}%
\pgfpathlineto{\pgfqpoint{1.921800in}{1.232610in}}%
\pgfpathlineto{\pgfqpoint{1.940400in}{1.224643in}}%
\pgfpathlineto{\pgfqpoint{1.959000in}{1.198053in}}%
\pgfpathlineto{\pgfqpoint{1.977600in}{1.156258in}}%
\pgfpathlineto{\pgfqpoint{1.996200in}{1.102142in}}%
\pgfpathlineto{\pgfqpoint{2.014800in}{1.036153in}}%
\pgfpathlineto{\pgfqpoint{2.033400in}{0.955516in}}%
\pgfpathlineto{\pgfqpoint{2.052000in}{0.855644in}}%
\pgfpathlineto{\pgfqpoint{2.070600in}{0.748540in}}%
\pgfpathlineto{\pgfqpoint{2.089200in}{0.861750in}}%
\pgfpathlineto{\pgfqpoint{2.107800in}{1.035313in}}%
\pgfpathlineto{\pgfqpoint{2.145000in}{1.399272in}}%
\pgfpathlineto{\pgfqpoint{2.163600in}{1.552090in}}%
\pgfpathlineto{\pgfqpoint{2.182200in}{1.661621in}}%
\pgfpathlineto{\pgfqpoint{2.200800in}{1.716156in}}%
\pgfpathlineto{\pgfqpoint{2.219400in}{1.711882in}}%
\pgfpathlineto{\pgfqpoint{2.238000in}{1.653631in}}%
\pgfpathlineto{\pgfqpoint{2.256600in}{1.555315in}}%
\pgfpathlineto{\pgfqpoint{2.275200in}{1.442684in}}%
\pgfpathlineto{\pgfqpoint{2.293800in}{1.360052in}}%
\pgfpathlineto{\pgfqpoint{2.312400in}{1.365565in}}%
\pgfpathlineto{\pgfqpoint{2.331000in}{1.482974in}}%
\pgfpathlineto{\pgfqpoint{2.349600in}{1.676859in}}%
\pgfpathlineto{\pgfqpoint{2.386800in}{2.112190in}}%
\pgfpathlineto{\pgfqpoint{2.405400in}{2.290194in}}%
\pgfpathlineto{\pgfqpoint{2.424000in}{2.412581in}}%
\pgfpathlineto{\pgfqpoint{2.442600in}{2.464801in}}%
\pgfpathlineto{\pgfqpoint{2.461200in}{2.438816in}}%
\pgfpathlineto{\pgfqpoint{2.479800in}{2.334818in}}%
\pgfpathlineto{\pgfqpoint{2.498400in}{2.163440in}}%
\pgfpathlineto{\pgfqpoint{2.535600in}{1.731341in}}%
\pgfpathlineto{\pgfqpoint{2.554200in}{1.575835in}}%
\pgfpathlineto{\pgfqpoint{2.572800in}{1.550093in}}%
\pgfpathlineto{\pgfqpoint{2.591400in}{1.659173in}}%
\pgfpathlineto{\pgfqpoint{2.647200in}{2.189247in}}%
\pgfpathlineto{\pgfqpoint{2.665800in}{2.298179in}}%
\pgfpathlineto{\pgfqpoint{2.684400in}{2.338151in}}%
\pgfpathlineto{\pgfqpoint{2.703000in}{2.296149in}}%
\pgfpathlineto{\pgfqpoint{2.721600in}{2.164036in}}%
\pgfpathlineto{\pgfqpoint{2.740200in}{1.941414in}}%
\pgfpathlineto{\pgfqpoint{2.758800in}{1.640974in}}%
\pgfpathlineto{\pgfqpoint{2.777400in}{1.306322in}}%
\pgfpathlineto{\pgfqpoint{2.796000in}{1.117106in}}%
\pgfpathlineto{\pgfqpoint{2.814600in}{1.347993in}}%
\pgfpathlineto{\pgfqpoint{2.851800in}{2.117057in}}%
\pgfpathlineto{\pgfqpoint{2.870400in}{2.418572in}}%
\pgfpathlineto{\pgfqpoint{2.889000in}{2.612323in}}%
\pgfpathlineto{\pgfqpoint{2.899205in}{2.650000in}}%
\pgfpathmoveto{\pgfqpoint{2.917316in}{2.650000in}}%
\pgfpathlineto{\pgfqpoint{2.926200in}{2.621658in}}%
\pgfpathlineto{\pgfqpoint{2.944800in}{2.446673in}}%
\pgfpathlineto{\pgfqpoint{2.963400in}{2.186412in}}%
\pgfpathlineto{\pgfqpoint{2.982000in}{1.897895in}}%
\pgfpathlineto{\pgfqpoint{3.000600in}{1.683895in}}%
\pgfpathlineto{\pgfqpoint{3.019200in}{1.674482in}}%
\pgfpathlineto{\pgfqpoint{3.037800in}{1.871975in}}%
\pgfpathlineto{\pgfqpoint{3.056400in}{2.145448in}}%
\pgfpathlineto{\pgfqpoint{3.075000in}{2.390527in}}%
\pgfpathlineto{\pgfqpoint{3.093600in}{2.553347in}}%
\pgfpathlineto{\pgfqpoint{3.112200in}{2.611197in}}%
\pgfpathlineto{\pgfqpoint{3.130800in}{2.562293in}}%
\pgfpathlineto{\pgfqpoint{3.149400in}{2.420333in}}%
\pgfpathlineto{\pgfqpoint{3.168000in}{2.209943in}}%
\pgfpathlineto{\pgfqpoint{3.205200in}{1.712132in}}%
\pgfpathlineto{\pgfqpoint{3.223800in}{1.494712in}}%
\pgfpathlineto{\pgfqpoint{3.242400in}{1.343269in}}%
\pgfpathlineto{\pgfqpoint{3.261000in}{1.275302in}}%
\pgfpathlineto{\pgfqpoint{3.279600in}{1.274610in}}%
\pgfpathlineto{\pgfqpoint{3.298200in}{1.305116in}}%
\pgfpathlineto{\pgfqpoint{3.316800in}{1.340840in}}%
\pgfpathlineto{\pgfqpoint{3.335400in}{1.371364in}}%
\pgfpathlineto{\pgfqpoint{3.354000in}{1.394017in}}%
\pgfpathlineto{\pgfqpoint{3.372600in}{1.407681in}}%
\pgfpathlineto{\pgfqpoint{3.391200in}{1.410449in}}%
\pgfpathlineto{\pgfqpoint{3.409800in}{1.399637in}}%
\pgfpathlineto{\pgfqpoint{3.428400in}{1.372347in}}%
\pgfpathlineto{\pgfqpoint{3.447000in}{1.325514in}}%
\pgfpathlineto{\pgfqpoint{3.465600in}{1.255449in}}%
\pgfpathlineto{\pgfqpoint{3.484200in}{1.157664in}}%
\pgfpathlineto{\pgfqpoint{3.502800in}{1.027950in}}%
\pgfpathlineto{\pgfqpoint{3.521400in}{0.867406in}}%
\pgfpathlineto{\pgfqpoint{3.540000in}{0.803118in}}%
\pgfpathlineto{\pgfqpoint{3.558600in}{1.011045in}}%
\pgfpathlineto{\pgfqpoint{3.595800in}{1.461295in}}%
\pgfpathlineto{\pgfqpoint{3.614400in}{1.638832in}}%
\pgfpathlineto{\pgfqpoint{3.633000in}{1.752573in}}%
\pgfpathlineto{\pgfqpoint{3.651600in}{1.787580in}}%
\pgfpathlineto{\pgfqpoint{3.670200in}{1.740488in}}%
\pgfpathlineto{\pgfqpoint{3.688800in}{1.622359in}}%
\pgfpathlineto{\pgfqpoint{3.707400in}{1.464029in}}%
\pgfpathlineto{\pgfqpoint{3.726000in}{1.329526in}}%
\pgfpathlineto{\pgfqpoint{3.744600in}{1.315308in}}%
\pgfpathlineto{\pgfqpoint{3.763200in}{1.446625in}}%
\pgfpathlineto{\pgfqpoint{3.781800in}{1.638019in}}%
\pgfpathlineto{\pgfqpoint{3.800400in}{1.809549in}}%
\pgfpathlineto{\pgfqpoint{3.819000in}{1.915578in}}%
\pgfpathlineto{\pgfqpoint{3.837600in}{1.933353in}}%
\pgfpathlineto{\pgfqpoint{3.856200in}{1.857379in}}%
\pgfpathlineto{\pgfqpoint{3.874800in}{1.697053in}}%
\pgfpathlineto{\pgfqpoint{3.893400in}{1.474411in}}%
\pgfpathlineto{\pgfqpoint{3.912000in}{1.222780in}}%
\pgfpathlineto{\pgfqpoint{3.930600in}{0.997445in}}%
\pgfpathlineto{\pgfqpoint{3.949200in}{0.938944in}}%
\pgfpathlineto{\pgfqpoint{3.986400in}{1.187957in}}%
\pgfpathlineto{\pgfqpoint{4.005000in}{1.255855in}}%
\pgfpathlineto{\pgfqpoint{4.023600in}{1.260638in}}%
\pgfpathlineto{\pgfqpoint{4.042200in}{1.209299in}}%
\pgfpathlineto{\pgfqpoint{4.060800in}{1.116328in}}%
\pgfpathlineto{\pgfqpoint{4.098000in}{0.884318in}}%
\pgfpathlineto{\pgfqpoint{4.116600in}{0.810442in}}%
\pgfpathlineto{\pgfqpoint{4.135200in}{0.840129in}}%
\pgfpathlineto{\pgfqpoint{4.153800in}{0.903476in}}%
\pgfpathlineto{\pgfqpoint{4.172400in}{0.955503in}}%
\pgfpathlineto{\pgfqpoint{4.191000in}{0.993359in}}%
\pgfpathlineto{\pgfqpoint{4.228200in}{1.053782in}}%
\pgfpathlineto{\pgfqpoint{4.246800in}{1.093295in}}%
\pgfpathlineto{\pgfqpoint{4.265400in}{1.147726in}}%
\pgfpathlineto{\pgfqpoint{4.284000in}{1.219037in}}%
\pgfpathlineto{\pgfqpoint{4.339800in}{1.473342in}}%
\pgfpathlineto{\pgfqpoint{4.358400in}{1.527008in}}%
\pgfpathlineto{\pgfqpoint{4.377000in}{1.539893in}}%
\pgfpathlineto{\pgfqpoint{4.395600in}{1.505324in}}%
\pgfpathlineto{\pgfqpoint{4.414200in}{1.431119in}}%
\pgfpathlineto{\pgfqpoint{4.432800in}{1.346748in}}%
\pgfpathlineto{\pgfqpoint{4.451400in}{1.304500in}}%
\pgfpathlineto{\pgfqpoint{4.470000in}{1.346743in}}%
\pgfpathlineto{\pgfqpoint{4.480000in}{1.405313in}}%
\pgfpathlineto{\pgfqpoint{4.480000in}{1.405313in}}%
\pgfusepath{stroke}%
\end{pgfscope}%
\begin{pgfscope}%
\pgfsetrectcap%
\pgfsetmiterjoin%
\pgfsetlinewidth{0.803000pt}%
\definecolor{currentstroke}{rgb}{0.737255,0.737255,0.737255}%
\pgfsetstrokecolor{currentstroke}%
\pgfsetdash{}{0pt}%
\pgfpathmoveto{\pgfqpoint{0.750000in}{0.330000in}}%
\pgfpathlineto{\pgfqpoint{0.750000in}{2.640000in}}%
\pgfusepath{stroke}%
\end{pgfscope}%
\begin{pgfscope}%
\pgfsetrectcap%
\pgfsetmiterjoin%
\pgfsetlinewidth{0.803000pt}%
\definecolor{currentstroke}{rgb}{0.737255,0.737255,0.737255}%
\pgfsetstrokecolor{currentstroke}%
\pgfsetdash{}{0pt}%
\pgfpathmoveto{\pgfqpoint{4.470000in}{0.330000in}}%
\pgfpathlineto{\pgfqpoint{4.470000in}{2.640000in}}%
\pgfusepath{stroke}%
\end{pgfscope}%
\begin{pgfscope}%
\pgfsetrectcap%
\pgfsetmiterjoin%
\pgfsetlinewidth{0.803000pt}%
\definecolor{currentstroke}{rgb}{0.737255,0.737255,0.737255}%
\pgfsetstrokecolor{currentstroke}%
\pgfsetdash{}{0pt}%
\pgfpathmoveto{\pgfqpoint{0.750000in}{0.330000in}}%
\pgfpathlineto{\pgfqpoint{4.470000in}{0.330000in}}%
\pgfusepath{stroke}%
\end{pgfscope}%
\begin{pgfscope}%
\pgfsetrectcap%
\pgfsetmiterjoin%
\pgfsetlinewidth{0.803000pt}%
\definecolor{currentstroke}{rgb}{0.737255,0.737255,0.737255}%
\pgfsetstrokecolor{currentstroke}%
\pgfsetdash{}{0pt}%
\pgfpathmoveto{\pgfqpoint{0.750000in}{2.640000in}}%
\pgfpathlineto{\pgfqpoint{4.470000in}{2.640000in}}%
\pgfusepath{stroke}%
\end{pgfscope}%
\begin{pgfscope}%
\pgfsetbuttcap%
\pgfsetmiterjoin%
\definecolor{currentfill}{rgb}{0.933333,0.933333,0.933333}%
\pgfsetfillcolor{currentfill}%
\pgfsetfillopacity{0.800000}%
\pgfsetlinewidth{0.501875pt}%
\definecolor{currentstroke}{rgb}{0.800000,0.800000,0.800000}%
\pgfsetstrokecolor{currentstroke}%
\pgfsetstrokeopacity{0.800000}%
\pgfsetdash{}{0pt}%
\pgfpathmoveto{\pgfqpoint{4.528333in}{1.118376in}}%
\pgfpathlineto{\pgfqpoint{5.933333in}{1.118376in}}%
\pgfpathquadraticcurveto{\pgfqpoint{5.950000in}{1.118376in}}{\pgfqpoint{5.950000in}{1.135042in}}%
\pgfpathlineto{\pgfqpoint{5.950000in}{1.834958in}}%
\pgfpathquadraticcurveto{\pgfqpoint{5.950000in}{1.851624in}}{\pgfqpoint{5.933333in}{1.851624in}}%
\pgfpathlineto{\pgfqpoint{4.528333in}{1.851624in}}%
\pgfpathquadraticcurveto{\pgfqpoint{4.511667in}{1.851624in}}{\pgfqpoint{4.511667in}{1.834958in}}%
\pgfpathlineto{\pgfqpoint{4.511667in}{1.135042in}}%
\pgfpathquadraticcurveto{\pgfqpoint{4.511667in}{1.118376in}}{\pgfqpoint{4.528333in}{1.118376in}}%
\pgfpathlineto{\pgfqpoint{4.528333in}{1.118376in}}%
\pgfpathclose%
\pgfusepath{stroke,fill}%
\end{pgfscope}%
\begin{pgfscope}%
\pgfsetrectcap%
\pgfsetroundjoin%
\pgfsetlinewidth{5.018750pt}%
\definecolor{currentstroke}{rgb}{0.501961,0.501961,0.501961}%
\pgfsetstrokecolor{currentstroke}%
\pgfsetdash{}{0pt}%
\pgfpathmoveto{\pgfqpoint{4.545000in}{1.788791in}}%
\pgfpathlineto{\pgfqpoint{4.628333in}{1.788791in}}%
\pgfpathlineto{\pgfqpoint{4.711667in}{1.788791in}}%
\pgfusepath{stroke}%
\end{pgfscope}%
\begin{pgfscope}%
\definecolor{textcolor}{rgb}{0.000000,0.000000,0.000000}%
\pgfsetstrokecolor{textcolor}%
\pgfsetfillcolor{textcolor}%
\pgftext[x=4.778333in,y=1.759624in,left,base]{\color{textcolor}\rmfamily\fontsize{6.000000}{7.200000}\selectfont Signal}%
\end{pgfscope}%
\begin{pgfscope}%
\pgfsetbuttcap%
\pgfsetroundjoin%
\pgfsetlinewidth{1.505625pt}%
\definecolor{currentstroke}{rgb}{0.000000,0.000000,0.000000}%
\pgfsetstrokecolor{currentstroke}%
\pgfsetdash{{1.500000pt}{2.475000pt}}{0.000000pt}%
\pgfpathmoveto{\pgfqpoint{4.545000in}{1.671708in}}%
\pgfpathlineto{\pgfqpoint{4.628333in}{1.671708in}}%
\pgfpathlineto{\pgfqpoint{4.711667in}{1.671708in}}%
\pgfusepath{stroke}%
\end{pgfscope}%
\begin{pgfscope}%
\definecolor{textcolor}{rgb}{0.000000,0.000000,0.000000}%
\pgfsetstrokecolor{textcolor}%
\pgfsetfillcolor{textcolor}%
\pgftext[x=4.778333in,y=1.642541in,left,base]{\color{textcolor}\rmfamily\fontsize{6.000000}{7.200000}\selectfont Valeur absolue}%
\end{pgfscope}%
\begin{pgfscope}%
\pgfsetrectcap%
\pgfsetroundjoin%
\pgfsetlinewidth{2.509375pt}%
\definecolor{currentstroke}{rgb}{0.203922,0.541176,0.741176}%
\pgfsetstrokecolor{currentstroke}%
\pgfsetdash{}{0pt}%
\pgfpathmoveto{\pgfqpoint{4.545000in}{1.555542in}}%
\pgfpathlineto{\pgfqpoint{4.628333in}{1.555542in}}%
\pgfpathlineto{\pgfqpoint{4.711667in}{1.555542in}}%
\pgfusepath{stroke}%
\end{pgfscope}%
\begin{pgfscope}%
\definecolor{textcolor}{rgb}{0.000000,0.000000,0.000000}%
\pgfsetstrokecolor{textcolor}%
\pgfsetfillcolor{textcolor}%
\pgftext[x=4.778333in,y=1.526375in,left,base]{\color{textcolor}\rmfamily\fontsize{6.000000}{7.200000}\selectfont Enveloppe sup.}%
\end{pgfscope}%
\begin{pgfscope}%
\pgfsetrectcap%
\pgfsetroundjoin%
\pgfsetlinewidth{2.509375pt}%
\definecolor{currentstroke}{rgb}{0.650980,0.023529,0.156863}%
\pgfsetstrokecolor{currentstroke}%
\pgfsetdash{}{0pt}%
\pgfpathmoveto{\pgfqpoint{4.545000in}{1.435208in}}%
\pgfpathlineto{\pgfqpoint{4.628333in}{1.435208in}}%
\pgfpathlineto{\pgfqpoint{4.711667in}{1.435208in}}%
\pgfusepath{stroke}%
\end{pgfscope}%
\begin{pgfscope}%
\definecolor{textcolor}{rgb}{0.000000,0.000000,0.000000}%
\pgfsetstrokecolor{textcolor}%
\pgfsetfillcolor{textcolor}%
\pgftext[x=4.778333in,y=1.406042in,left,base]{\color{textcolor}\rmfamily\fontsize{6.000000}{7.200000}\selectfont Enveloppe sup. (approx.)}%
\end{pgfscope}%
\begin{pgfscope}%
\pgfsetrectcap%
\pgfsetroundjoin%
\pgfsetlinewidth{2.509375pt}%
\definecolor{currentstroke}{rgb}{0.478431,0.407843,0.650980}%
\pgfsetstrokecolor{currentstroke}%
\pgfsetdash{}{0pt}%
\pgfpathmoveto{\pgfqpoint{4.545000in}{1.313209in}}%
\pgfpathlineto{\pgfqpoint{4.628333in}{1.313209in}}%
\pgfpathlineto{\pgfqpoint{4.711667in}{1.313209in}}%
\pgfusepath{stroke}%
\end{pgfscope}%
\begin{pgfscope}%
\definecolor{textcolor}{rgb}{0.000000,0.000000,0.000000}%
\pgfsetstrokecolor{textcolor}%
\pgfsetfillcolor{textcolor}%
\pgftext[x=4.778333in,y=1.284042in,left,base]{\color{textcolor}\rmfamily\fontsize{6.000000}{7.200000}\selectfont Enveloppe de Allen}%
\end{pgfscope}%
\begin{pgfscope}%
\pgfsetrectcap%
\pgfsetroundjoin%
\pgfsetlinewidth{2.509375pt}%
\definecolor{currentstroke}{rgb}{0.274510,0.470588,0.129412}%
\pgfsetstrokecolor{currentstroke}%
\pgfsetdash{}{0pt}%
\pgfpathmoveto{\pgfqpoint{4.545000in}{1.197042in}}%
\pgfpathlineto{\pgfqpoint{4.628333in}{1.197042in}}%
\pgfpathlineto{\pgfqpoint{4.711667in}{1.197042in}}%
\pgfusepath{stroke}%
\end{pgfscope}%
\begin{pgfscope}%
\definecolor{textcolor}{rgb}{0.000000,0.000000,0.000000}%
\pgfsetstrokecolor{textcolor}%
\pgfsetfillcolor{textcolor}%
\pgftext[x=4.778333in,y=1.167875in,left,base]{\color{textcolor}\rmfamily\fontsize{6.000000}{7.200000}\selectfont Enveloppe de Baer}%
\end{pgfscope}%
\end{pgfpicture}%
\makeatother%
\endgroup%
}
    \caption{Comparaison de différentes enveloppes calculées à partir du même signal (gris). La valeur absolue (noir pointillé), l'enveloppe supérieure (bleu), l'approximation géométrique de l'enveloppe supérieure (rouge), l'enveloppe de Allen (violet) et de l'enveloppe de Baer et Kradolfer (vert).}
    \label{fig:env-comp}
\end{figure}

\subsection{Calcul de fonctions caractéristiques}

\subsubsection{Méthode STA/LTA}

\begin{figure}[!ht]
    \centering
    \scalebox{.9}{%% Creator: Matplotlib, PGF backend
%%
%% To include the figure in your LaTeX document, write
%%   \input{<filename>.pgf}
%%
%% Make sure the required packages are loaded in your preamble
%%   \usepackage{pgf}
%%
%% Also ensure that all the required font packages are loaded; for instance,
%% the lmodern package is sometimes necessary when using math font.
%%   \usepackage{lmodern}
%%
%% Figures using additional raster images can only be included by \input if
%% they are in the same directory as the main LaTeX file. For loading figures
%% from other directories you can use the `import` package
%%   \usepackage{import}
%%
%% and then include the figures with
%%   \import{<path to file>}{<filename>.pgf}
%%
%% Matplotlib used the following preamble
%%   \usepackage{fontspec}
%%
\begingroup%
\makeatletter%
\begin{pgfpicture}%
\pgfpathrectangle{\pgfpointorigin}{\pgfqpoint{6.000000in}{3.000000in}}%
\pgfusepath{use as bounding box, clip}%
\begin{pgfscope}%
\pgfsetbuttcap%
\pgfsetmiterjoin%
\definecolor{currentfill}{rgb}{1.000000,1.000000,1.000000}%
\pgfsetfillcolor{currentfill}%
\pgfsetlinewidth{0.000000pt}%
\definecolor{currentstroke}{rgb}{1.000000,1.000000,1.000000}%
\pgfsetstrokecolor{currentstroke}%
\pgfsetdash{}{0pt}%
\pgfpathmoveto{\pgfqpoint{0.000000in}{0.000000in}}%
\pgfpathlineto{\pgfqpoint{6.000000in}{0.000000in}}%
\pgfpathlineto{\pgfqpoint{6.000000in}{3.000000in}}%
\pgfpathlineto{\pgfqpoint{0.000000in}{3.000000in}}%
\pgfpathlineto{\pgfqpoint{0.000000in}{0.000000in}}%
\pgfpathclose%
\pgfusepath{fill}%
\end{pgfscope}%
\begin{pgfscope}%
\pgfsetbuttcap%
\pgfsetmiterjoin%
\definecolor{currentfill}{rgb}{0.933333,0.933333,0.933333}%
\pgfsetfillcolor{currentfill}%
\pgfsetlinewidth{0.000000pt}%
\definecolor{currentstroke}{rgb}{0.000000,0.000000,0.000000}%
\pgfsetstrokecolor{currentstroke}%
\pgfsetstrokeopacity{0.000000}%
\pgfsetdash{}{0pt}%
\pgfpathmoveto{\pgfqpoint{0.583136in}{1.796846in}}%
\pgfpathlineto{\pgfqpoint{5.745833in}{1.796846in}}%
\pgfpathlineto{\pgfqpoint{5.745833in}{2.703703in}}%
\pgfpathlineto{\pgfqpoint{0.583136in}{2.703703in}}%
\pgfpathlineto{\pgfqpoint{0.583136in}{1.796846in}}%
\pgfpathclose%
\pgfusepath{fill}%
\end{pgfscope}%
\begin{pgfscope}%
\pgfpathrectangle{\pgfqpoint{0.583136in}{1.796846in}}{\pgfqpoint{5.162697in}{0.906858in}}%
\pgfusepath{clip}%
\pgfsetbuttcap%
\pgfsetroundjoin%
\pgfsetlinewidth{0.501875pt}%
\definecolor{currentstroke}{rgb}{0.698039,0.698039,0.698039}%
\pgfsetstrokecolor{currentstroke}%
\pgfsetdash{{1.850000pt}{0.800000pt}}{0.000000pt}%
\pgfpathmoveto{\pgfqpoint{0.583136in}{1.796846in}}%
\pgfpathlineto{\pgfqpoint{0.583136in}{2.703703in}}%
\pgfusepath{stroke}%
\end{pgfscope}%
\begin{pgfscope}%
\pgfsetbuttcap%
\pgfsetroundjoin%
\definecolor{currentfill}{rgb}{0.000000,0.000000,0.000000}%
\pgfsetfillcolor{currentfill}%
\pgfsetlinewidth{0.803000pt}%
\definecolor{currentstroke}{rgb}{0.000000,0.000000,0.000000}%
\pgfsetstrokecolor{currentstroke}%
\pgfsetdash{}{0pt}%
\pgfsys@defobject{currentmarker}{\pgfqpoint{0.000000in}{0.000000in}}{\pgfqpoint{0.000000in}{0.048611in}}{%
\pgfpathmoveto{\pgfqpoint{0.000000in}{0.000000in}}%
\pgfpathlineto{\pgfqpoint{0.000000in}{0.048611in}}%
\pgfusepath{stroke,fill}%
}%
\begin{pgfscope}%
\pgfsys@transformshift{0.583136in}{1.796846in}%
\pgfsys@useobject{currentmarker}{}%
\end{pgfscope}%
\end{pgfscope}%
\begin{pgfscope}%
\definecolor{textcolor}{rgb}{0.000000,0.000000,0.000000}%
\pgfsetstrokecolor{textcolor}%
\pgfsetfillcolor{textcolor}%
\pgftext[x=0.583136in,y=1.748234in,,top]{\color{textcolor}\rmfamily\fontsize{10.000000}{12.000000}\selectfont \(\displaystyle {0}\)}%
\end{pgfscope}%
\begin{pgfscope}%
\pgfpathrectangle{\pgfqpoint{0.583136in}{1.796846in}}{\pgfqpoint{5.162697in}{0.906858in}}%
\pgfusepath{clip}%
\pgfsetbuttcap%
\pgfsetroundjoin%
\pgfsetlinewidth{0.501875pt}%
\definecolor{currentstroke}{rgb}{0.698039,0.698039,0.698039}%
\pgfsetstrokecolor{currentstroke}%
\pgfsetdash{{1.850000pt}{0.800000pt}}{0.000000pt}%
\pgfpathmoveto{\pgfqpoint{1.615676in}{1.796846in}}%
\pgfpathlineto{\pgfqpoint{1.615676in}{2.703703in}}%
\pgfusepath{stroke}%
\end{pgfscope}%
\begin{pgfscope}%
\pgfsetbuttcap%
\pgfsetroundjoin%
\definecolor{currentfill}{rgb}{0.000000,0.000000,0.000000}%
\pgfsetfillcolor{currentfill}%
\pgfsetlinewidth{0.803000pt}%
\definecolor{currentstroke}{rgb}{0.000000,0.000000,0.000000}%
\pgfsetstrokecolor{currentstroke}%
\pgfsetdash{}{0pt}%
\pgfsys@defobject{currentmarker}{\pgfqpoint{0.000000in}{0.000000in}}{\pgfqpoint{0.000000in}{0.048611in}}{%
\pgfpathmoveto{\pgfqpoint{0.000000in}{0.000000in}}%
\pgfpathlineto{\pgfqpoint{0.000000in}{0.048611in}}%
\pgfusepath{stroke,fill}%
}%
\begin{pgfscope}%
\pgfsys@transformshift{1.615676in}{1.796846in}%
\pgfsys@useobject{currentmarker}{}%
\end{pgfscope}%
\end{pgfscope}%
\begin{pgfscope}%
\definecolor{textcolor}{rgb}{0.000000,0.000000,0.000000}%
\pgfsetstrokecolor{textcolor}%
\pgfsetfillcolor{textcolor}%
\pgftext[x=1.615676in,y=1.748234in,,top]{\color{textcolor}\rmfamily\fontsize{10.000000}{12.000000}\selectfont \(\displaystyle {20}\)}%
\end{pgfscope}%
\begin{pgfscope}%
\pgfpathrectangle{\pgfqpoint{0.583136in}{1.796846in}}{\pgfqpoint{5.162697in}{0.906858in}}%
\pgfusepath{clip}%
\pgfsetbuttcap%
\pgfsetroundjoin%
\pgfsetlinewidth{0.501875pt}%
\definecolor{currentstroke}{rgb}{0.698039,0.698039,0.698039}%
\pgfsetstrokecolor{currentstroke}%
\pgfsetdash{{1.850000pt}{0.800000pt}}{0.000000pt}%
\pgfpathmoveto{\pgfqpoint{2.648215in}{1.796846in}}%
\pgfpathlineto{\pgfqpoint{2.648215in}{2.703703in}}%
\pgfusepath{stroke}%
\end{pgfscope}%
\begin{pgfscope}%
\pgfsetbuttcap%
\pgfsetroundjoin%
\definecolor{currentfill}{rgb}{0.000000,0.000000,0.000000}%
\pgfsetfillcolor{currentfill}%
\pgfsetlinewidth{0.803000pt}%
\definecolor{currentstroke}{rgb}{0.000000,0.000000,0.000000}%
\pgfsetstrokecolor{currentstroke}%
\pgfsetdash{}{0pt}%
\pgfsys@defobject{currentmarker}{\pgfqpoint{0.000000in}{0.000000in}}{\pgfqpoint{0.000000in}{0.048611in}}{%
\pgfpathmoveto{\pgfqpoint{0.000000in}{0.000000in}}%
\pgfpathlineto{\pgfqpoint{0.000000in}{0.048611in}}%
\pgfusepath{stroke,fill}%
}%
\begin{pgfscope}%
\pgfsys@transformshift{2.648215in}{1.796846in}%
\pgfsys@useobject{currentmarker}{}%
\end{pgfscope}%
\end{pgfscope}%
\begin{pgfscope}%
\definecolor{textcolor}{rgb}{0.000000,0.000000,0.000000}%
\pgfsetstrokecolor{textcolor}%
\pgfsetfillcolor{textcolor}%
\pgftext[x=2.648215in,y=1.748234in,,top]{\color{textcolor}\rmfamily\fontsize{10.000000}{12.000000}\selectfont \(\displaystyle {40}\)}%
\end{pgfscope}%
\begin{pgfscope}%
\pgfpathrectangle{\pgfqpoint{0.583136in}{1.796846in}}{\pgfqpoint{5.162697in}{0.906858in}}%
\pgfusepath{clip}%
\pgfsetbuttcap%
\pgfsetroundjoin%
\pgfsetlinewidth{0.501875pt}%
\definecolor{currentstroke}{rgb}{0.698039,0.698039,0.698039}%
\pgfsetstrokecolor{currentstroke}%
\pgfsetdash{{1.850000pt}{0.800000pt}}{0.000000pt}%
\pgfpathmoveto{\pgfqpoint{3.680754in}{1.796846in}}%
\pgfpathlineto{\pgfqpoint{3.680754in}{2.703703in}}%
\pgfusepath{stroke}%
\end{pgfscope}%
\begin{pgfscope}%
\pgfsetbuttcap%
\pgfsetroundjoin%
\definecolor{currentfill}{rgb}{0.000000,0.000000,0.000000}%
\pgfsetfillcolor{currentfill}%
\pgfsetlinewidth{0.803000pt}%
\definecolor{currentstroke}{rgb}{0.000000,0.000000,0.000000}%
\pgfsetstrokecolor{currentstroke}%
\pgfsetdash{}{0pt}%
\pgfsys@defobject{currentmarker}{\pgfqpoint{0.000000in}{0.000000in}}{\pgfqpoint{0.000000in}{0.048611in}}{%
\pgfpathmoveto{\pgfqpoint{0.000000in}{0.000000in}}%
\pgfpathlineto{\pgfqpoint{0.000000in}{0.048611in}}%
\pgfusepath{stroke,fill}%
}%
\begin{pgfscope}%
\pgfsys@transformshift{3.680754in}{1.796846in}%
\pgfsys@useobject{currentmarker}{}%
\end{pgfscope}%
\end{pgfscope}%
\begin{pgfscope}%
\definecolor{textcolor}{rgb}{0.000000,0.000000,0.000000}%
\pgfsetstrokecolor{textcolor}%
\pgfsetfillcolor{textcolor}%
\pgftext[x=3.680754in,y=1.748234in,,top]{\color{textcolor}\rmfamily\fontsize{10.000000}{12.000000}\selectfont \(\displaystyle {60}\)}%
\end{pgfscope}%
\begin{pgfscope}%
\pgfpathrectangle{\pgfqpoint{0.583136in}{1.796846in}}{\pgfqpoint{5.162697in}{0.906858in}}%
\pgfusepath{clip}%
\pgfsetbuttcap%
\pgfsetroundjoin%
\pgfsetlinewidth{0.501875pt}%
\definecolor{currentstroke}{rgb}{0.698039,0.698039,0.698039}%
\pgfsetstrokecolor{currentstroke}%
\pgfsetdash{{1.850000pt}{0.800000pt}}{0.000000pt}%
\pgfpathmoveto{\pgfqpoint{4.713294in}{1.796846in}}%
\pgfpathlineto{\pgfqpoint{4.713294in}{2.703703in}}%
\pgfusepath{stroke}%
\end{pgfscope}%
\begin{pgfscope}%
\pgfsetbuttcap%
\pgfsetroundjoin%
\definecolor{currentfill}{rgb}{0.000000,0.000000,0.000000}%
\pgfsetfillcolor{currentfill}%
\pgfsetlinewidth{0.803000pt}%
\definecolor{currentstroke}{rgb}{0.000000,0.000000,0.000000}%
\pgfsetstrokecolor{currentstroke}%
\pgfsetdash{}{0pt}%
\pgfsys@defobject{currentmarker}{\pgfqpoint{0.000000in}{0.000000in}}{\pgfqpoint{0.000000in}{0.048611in}}{%
\pgfpathmoveto{\pgfqpoint{0.000000in}{0.000000in}}%
\pgfpathlineto{\pgfqpoint{0.000000in}{0.048611in}}%
\pgfusepath{stroke,fill}%
}%
\begin{pgfscope}%
\pgfsys@transformshift{4.713294in}{1.796846in}%
\pgfsys@useobject{currentmarker}{}%
\end{pgfscope}%
\end{pgfscope}%
\begin{pgfscope}%
\definecolor{textcolor}{rgb}{0.000000,0.000000,0.000000}%
\pgfsetstrokecolor{textcolor}%
\pgfsetfillcolor{textcolor}%
\pgftext[x=4.713294in,y=1.748234in,,top]{\color{textcolor}\rmfamily\fontsize{10.000000}{12.000000}\selectfont \(\displaystyle {80}\)}%
\end{pgfscope}%
\begin{pgfscope}%
\pgfpathrectangle{\pgfqpoint{0.583136in}{1.796846in}}{\pgfqpoint{5.162697in}{0.906858in}}%
\pgfusepath{clip}%
\pgfsetbuttcap%
\pgfsetroundjoin%
\pgfsetlinewidth{0.501875pt}%
\definecolor{currentstroke}{rgb}{0.698039,0.698039,0.698039}%
\pgfsetstrokecolor{currentstroke}%
\pgfsetdash{{1.850000pt}{0.800000pt}}{0.000000pt}%
\pgfpathmoveto{\pgfqpoint{5.745833in}{1.796846in}}%
\pgfpathlineto{\pgfqpoint{5.745833in}{2.703703in}}%
\pgfusepath{stroke}%
\end{pgfscope}%
\begin{pgfscope}%
\pgfsetbuttcap%
\pgfsetroundjoin%
\definecolor{currentfill}{rgb}{0.000000,0.000000,0.000000}%
\pgfsetfillcolor{currentfill}%
\pgfsetlinewidth{0.803000pt}%
\definecolor{currentstroke}{rgb}{0.000000,0.000000,0.000000}%
\pgfsetstrokecolor{currentstroke}%
\pgfsetdash{}{0pt}%
\pgfsys@defobject{currentmarker}{\pgfqpoint{0.000000in}{0.000000in}}{\pgfqpoint{0.000000in}{0.048611in}}{%
\pgfpathmoveto{\pgfqpoint{0.000000in}{0.000000in}}%
\pgfpathlineto{\pgfqpoint{0.000000in}{0.048611in}}%
\pgfusepath{stroke,fill}%
}%
\begin{pgfscope}%
\pgfsys@transformshift{5.745833in}{1.796846in}%
\pgfsys@useobject{currentmarker}{}%
\end{pgfscope}%
\end{pgfscope}%
\begin{pgfscope}%
\definecolor{textcolor}{rgb}{0.000000,0.000000,0.000000}%
\pgfsetstrokecolor{textcolor}%
\pgfsetfillcolor{textcolor}%
\pgftext[x=5.745833in,y=1.748234in,,top]{\color{textcolor}\rmfamily\fontsize{10.000000}{12.000000}\selectfont \(\displaystyle {100}\)}%
\end{pgfscope}%
\begin{pgfscope}%
\pgfpathrectangle{\pgfqpoint{0.583136in}{1.796846in}}{\pgfqpoint{5.162697in}{0.906858in}}%
\pgfusepath{clip}%
\pgfsetbuttcap%
\pgfsetroundjoin%
\pgfsetlinewidth{0.501875pt}%
\definecolor{currentstroke}{rgb}{0.698039,0.698039,0.698039}%
\pgfsetstrokecolor{currentstroke}%
\pgfsetdash{{1.850000pt}{0.800000pt}}{0.000000pt}%
\pgfpathmoveto{\pgfqpoint{0.583136in}{1.855113in}}%
\pgfpathlineto{\pgfqpoint{5.745833in}{1.855113in}}%
\pgfusepath{stroke}%
\end{pgfscope}%
\begin{pgfscope}%
\pgfsetbuttcap%
\pgfsetroundjoin%
\definecolor{currentfill}{rgb}{0.000000,0.000000,0.000000}%
\pgfsetfillcolor{currentfill}%
\pgfsetlinewidth{0.803000pt}%
\definecolor{currentstroke}{rgb}{0.000000,0.000000,0.000000}%
\pgfsetstrokecolor{currentstroke}%
\pgfsetdash{}{0pt}%
\pgfsys@defobject{currentmarker}{\pgfqpoint{0.000000in}{0.000000in}}{\pgfqpoint{0.048611in}{0.000000in}}{%
\pgfpathmoveto{\pgfqpoint{0.000000in}{0.000000in}}%
\pgfpathlineto{\pgfqpoint{0.048611in}{0.000000in}}%
\pgfusepath{stroke,fill}%
}%
\begin{pgfscope}%
\pgfsys@transformshift{0.583136in}{1.855113in}%
\pgfsys@useobject{currentmarker}{}%
\end{pgfscope}%
\end{pgfscope}%
\begin{pgfscope}%
\definecolor{textcolor}{rgb}{0.000000,0.000000,0.000000}%
\pgfsetstrokecolor{textcolor}%
\pgfsetfillcolor{textcolor}%
\pgftext[x=0.357055in, y=1.806919in, left, base]{\color{textcolor}\rmfamily\fontsize{10.000000}{12.000000}\selectfont \(\displaystyle {\ensuremath{-}5}\)}%
\end{pgfscope}%
\begin{pgfscope}%
\pgfpathrectangle{\pgfqpoint{0.583136in}{1.796846in}}{\pgfqpoint{5.162697in}{0.906858in}}%
\pgfusepath{clip}%
\pgfsetbuttcap%
\pgfsetroundjoin%
\pgfsetlinewidth{0.501875pt}%
\definecolor{currentstroke}{rgb}{0.698039,0.698039,0.698039}%
\pgfsetstrokecolor{currentstroke}%
\pgfsetdash{{1.850000pt}{0.800000pt}}{0.000000pt}%
\pgfpathmoveto{\pgfqpoint{0.583136in}{2.206614in}}%
\pgfpathlineto{\pgfqpoint{5.745833in}{2.206614in}}%
\pgfusepath{stroke}%
\end{pgfscope}%
\begin{pgfscope}%
\pgfsetbuttcap%
\pgfsetroundjoin%
\definecolor{currentfill}{rgb}{0.000000,0.000000,0.000000}%
\pgfsetfillcolor{currentfill}%
\pgfsetlinewidth{0.803000pt}%
\definecolor{currentstroke}{rgb}{0.000000,0.000000,0.000000}%
\pgfsetstrokecolor{currentstroke}%
\pgfsetdash{}{0pt}%
\pgfsys@defobject{currentmarker}{\pgfqpoint{0.000000in}{0.000000in}}{\pgfqpoint{0.048611in}{0.000000in}}{%
\pgfpathmoveto{\pgfqpoint{0.000000in}{0.000000in}}%
\pgfpathlineto{\pgfqpoint{0.048611in}{0.000000in}}%
\pgfusepath{stroke,fill}%
}%
\begin{pgfscope}%
\pgfsys@transformshift{0.583136in}{2.206614in}%
\pgfsys@useobject{currentmarker}{}%
\end{pgfscope}%
\end{pgfscope}%
\begin{pgfscope}%
\definecolor{textcolor}{rgb}{0.000000,0.000000,0.000000}%
\pgfsetstrokecolor{textcolor}%
\pgfsetfillcolor{textcolor}%
\pgftext[x=0.465080in, y=2.158419in, left, base]{\color{textcolor}\rmfamily\fontsize{10.000000}{12.000000}\selectfont \(\displaystyle {0}\)}%
\end{pgfscope}%
\begin{pgfscope}%
\pgfpathrectangle{\pgfqpoint{0.583136in}{1.796846in}}{\pgfqpoint{5.162697in}{0.906858in}}%
\pgfusepath{clip}%
\pgfsetbuttcap%
\pgfsetroundjoin%
\pgfsetlinewidth{0.501875pt}%
\definecolor{currentstroke}{rgb}{0.698039,0.698039,0.698039}%
\pgfsetstrokecolor{currentstroke}%
\pgfsetdash{{1.850000pt}{0.800000pt}}{0.000000pt}%
\pgfpathmoveto{\pgfqpoint{0.583136in}{2.558114in}}%
\pgfpathlineto{\pgfqpoint{5.745833in}{2.558114in}}%
\pgfusepath{stroke}%
\end{pgfscope}%
\begin{pgfscope}%
\pgfsetbuttcap%
\pgfsetroundjoin%
\definecolor{currentfill}{rgb}{0.000000,0.000000,0.000000}%
\pgfsetfillcolor{currentfill}%
\pgfsetlinewidth{0.803000pt}%
\definecolor{currentstroke}{rgb}{0.000000,0.000000,0.000000}%
\pgfsetstrokecolor{currentstroke}%
\pgfsetdash{}{0pt}%
\pgfsys@defobject{currentmarker}{\pgfqpoint{0.000000in}{0.000000in}}{\pgfqpoint{0.048611in}{0.000000in}}{%
\pgfpathmoveto{\pgfqpoint{0.000000in}{0.000000in}}%
\pgfpathlineto{\pgfqpoint{0.048611in}{0.000000in}}%
\pgfusepath{stroke,fill}%
}%
\begin{pgfscope}%
\pgfsys@transformshift{0.583136in}{2.558114in}%
\pgfsys@useobject{currentmarker}{}%
\end{pgfscope}%
\end{pgfscope}%
\begin{pgfscope}%
\definecolor{textcolor}{rgb}{0.000000,0.000000,0.000000}%
\pgfsetstrokecolor{textcolor}%
\pgfsetfillcolor{textcolor}%
\pgftext[x=0.465080in, y=2.509920in, left, base]{\color{textcolor}\rmfamily\fontsize{10.000000}{12.000000}\selectfont \(\displaystyle {5}\)}%
\end{pgfscope}%
\begin{pgfscope}%
\definecolor{textcolor}{rgb}{0.000000,0.000000,0.000000}%
\pgfsetstrokecolor{textcolor}%
\pgfsetfillcolor{textcolor}%
\pgftext[x=0.301500in,y=2.250274in,,bottom,rotate=90.000000]{\color{textcolor}\rmfamily\fontsize{12.000000}{14.400000}\selectfont Signal}%
\end{pgfscope}%
\begin{pgfscope}%
\definecolor{textcolor}{rgb}{0.000000,0.000000,0.000000}%
\pgfsetstrokecolor{textcolor}%
\pgfsetfillcolor{textcolor}%
\pgftext[x=0.583136in,y=2.745370in,left,base]{\color{textcolor}\rmfamily\fontsize{10.000000}{12.000000}\selectfont \(\displaystyle \times{10^{\ensuremath{-}5}}{}\)}%
\end{pgfscope}%
\begin{pgfscope}%
\pgfpathrectangle{\pgfqpoint{0.583136in}{1.796846in}}{\pgfqpoint{5.162697in}{0.906858in}}%
\pgfusepath{clip}%
\pgfsetrectcap%
\pgfsetroundjoin%
\pgfsetlinewidth{1.505625pt}%
\definecolor{currentstroke}{rgb}{0.498039,0.498039,0.498039}%
\pgfsetstrokecolor{currentstroke}%
\pgfsetdash{}{0pt}%
\pgfpathmoveto{\pgfqpoint{0.583136in}{2.206614in}}%
\pgfpathlineto{\pgfqpoint{1.434981in}{2.206414in}}%
\pgfpathlineto{\pgfqpoint{1.447114in}{2.206768in}}%
\pgfpathlineto{\pgfqpoint{1.569728in}{2.206461in}}%
\pgfpathlineto{\pgfqpoint{1.581860in}{2.206699in}}%
\pgfpathlineto{\pgfqpoint{1.591669in}{2.206418in}}%
\pgfpathlineto{\pgfqpoint{1.687437in}{2.206691in}}%
\pgfpathlineto{\pgfqpoint{1.711960in}{2.206621in}}%
\pgfpathlineto{\pgfqpoint{1.742936in}{2.206726in}}%
\pgfpathlineto{\pgfqpoint{1.776235in}{2.206412in}}%
\pgfpathlineto{\pgfqpoint{1.849546in}{2.206734in}}%
\pgfpathlineto{\pgfqpoint{1.860904in}{2.206451in}}%
\pgfpathlineto{\pgfqpoint{1.872003in}{2.206627in}}%
\pgfpathlineto{\pgfqpoint{1.932923in}{2.206533in}}%
\pgfpathlineto{\pgfqpoint{1.940667in}{2.206576in}}%
\pgfpathlineto{\pgfqpoint{1.968804in}{2.206663in}}%
\pgfpathlineto{\pgfqpoint{1.978097in}{2.206576in}}%
\pgfpathlineto{\pgfqpoint{1.986615in}{2.206575in}}%
\pgfpathlineto{\pgfqpoint{1.999264in}{2.206553in}}%
\pgfpathlineto{\pgfqpoint{2.010106in}{2.206768in}}%
\pgfpathlineto{\pgfqpoint{2.045986in}{2.206616in}}%
\pgfpathlineto{\pgfqpoint{2.061991in}{2.206736in}}%
\pgfpathlineto{\pgfqpoint{2.071025in}{2.206537in}}%
\pgfpathlineto{\pgfqpoint{2.098130in}{2.206615in}}%
\pgfpathlineto{\pgfqpoint{2.127815in}{2.206649in}}%
\pgfpathlineto{\pgfqpoint{2.150015in}{2.206473in}}%
\pgfpathlineto{\pgfqpoint{2.190026in}{2.206779in}}%
\pgfpathlineto{\pgfqpoint{2.233650in}{2.206721in}}%
\pgfpathlineto{\pgfqpoint{2.245783in}{2.206521in}}%
\pgfpathlineto{\pgfqpoint{2.251204in}{2.206776in}}%
\pgfpathlineto{\pgfqpoint{2.262045in}{2.206646in}}%
\pgfpathlineto{\pgfqpoint{2.424412in}{2.206538in}}%
\pgfpathlineto{\pgfqpoint{2.446095in}{2.206614in}}%
\pgfpathlineto{\pgfqpoint{2.513469in}{2.206440in}}%
\pgfpathlineto{\pgfqpoint{2.531796in}{2.206640in}}%
\pgfpathlineto{\pgfqpoint{2.660347in}{2.206763in}}%
\pgfpathlineto{\pgfqpoint{2.695712in}{2.206546in}}%
\pgfpathlineto{\pgfqpoint{2.704230in}{2.206637in}}%
\pgfpathlineto{\pgfqpoint{2.884666in}{2.207690in}}%
\pgfpathlineto{\pgfqpoint{2.886215in}{2.207736in}}%
\pgfpathlineto{\pgfqpoint{2.889571in}{2.204598in}}%
\pgfpathlineto{\pgfqpoint{2.889829in}{2.204845in}}%
\pgfpathlineto{\pgfqpoint{2.892927in}{2.206944in}}%
\pgfpathlineto{\pgfqpoint{2.897573in}{2.206128in}}%
\pgfpathlineto{\pgfqpoint{2.899896in}{2.209250in}}%
\pgfpathlineto{\pgfqpoint{2.900155in}{2.209112in}}%
\pgfpathlineto{\pgfqpoint{2.904285in}{2.206624in}}%
\pgfpathlineto{\pgfqpoint{2.908415in}{2.207620in}}%
\pgfpathlineto{\pgfqpoint{2.910996in}{2.203523in}}%
\pgfpathlineto{\pgfqpoint{2.911254in}{2.203686in}}%
\pgfpathlineto{\pgfqpoint{2.914352in}{2.207835in}}%
\pgfpathlineto{\pgfqpoint{2.915126in}{2.206640in}}%
\pgfpathlineto{\pgfqpoint{2.917191in}{2.204109in}}%
\pgfpathlineto{\pgfqpoint{2.917450in}{2.204296in}}%
\pgfpathlineto{\pgfqpoint{2.921580in}{2.208189in}}%
\pgfpathlineto{\pgfqpoint{2.921838in}{2.207920in}}%
\pgfpathlineto{\pgfqpoint{2.924677in}{2.205147in}}%
\pgfpathlineto{\pgfqpoint{2.924935in}{2.205414in}}%
\pgfpathlineto{\pgfqpoint{2.928291in}{2.209265in}}%
\pgfpathlineto{\pgfqpoint{2.928549in}{2.208995in}}%
\pgfpathlineto{\pgfqpoint{2.930614in}{2.203650in}}%
\pgfpathlineto{\pgfqpoint{2.931647in}{2.202851in}}%
\pgfpathlineto{\pgfqpoint{2.931905in}{2.203150in}}%
\pgfpathlineto{\pgfqpoint{2.935777in}{2.210877in}}%
\pgfpathlineto{\pgfqpoint{2.936552in}{2.209464in}}%
\pgfpathlineto{\pgfqpoint{2.939133in}{2.201804in}}%
\pgfpathlineto{\pgfqpoint{2.939649in}{2.202982in}}%
\pgfpathlineto{\pgfqpoint{2.942231in}{2.213671in}}%
\pgfpathlineto{\pgfqpoint{2.943005in}{2.212303in}}%
\pgfpathlineto{\pgfqpoint{2.946361in}{2.197939in}}%
\pgfpathlineto{\pgfqpoint{2.947135in}{2.200148in}}%
\pgfpathlineto{\pgfqpoint{2.949716in}{2.212883in}}%
\pgfpathlineto{\pgfqpoint{2.950233in}{2.211342in}}%
\pgfpathlineto{\pgfqpoint{2.952814in}{2.194629in}}%
\pgfpathlineto{\pgfqpoint{2.953588in}{2.196560in}}%
\pgfpathlineto{\pgfqpoint{2.957202in}{2.223085in}}%
\pgfpathlineto{\pgfqpoint{2.958235in}{2.218352in}}%
\pgfpathlineto{\pgfqpoint{2.960558in}{2.194973in}}%
\pgfpathlineto{\pgfqpoint{2.961074in}{2.196815in}}%
\pgfpathlineto{\pgfqpoint{2.963656in}{2.217028in}}%
\pgfpathlineto{\pgfqpoint{2.964172in}{2.215587in}}%
\pgfpathlineto{\pgfqpoint{2.966495in}{2.196252in}}%
\pgfpathlineto{\pgfqpoint{2.968044in}{2.189424in}}%
\pgfpathlineto{\pgfqpoint{2.968302in}{2.189552in}}%
\pgfpathlineto{\pgfqpoint{2.969077in}{2.192752in}}%
\pgfpathlineto{\pgfqpoint{2.972690in}{2.220334in}}%
\pgfpathlineto{\pgfqpoint{2.973465in}{2.218616in}}%
\pgfpathlineto{\pgfqpoint{2.976304in}{2.205431in}}%
\pgfpathlineto{\pgfqpoint{2.976821in}{2.207061in}}%
\pgfpathlineto{\pgfqpoint{2.979402in}{2.229903in}}%
\pgfpathlineto{\pgfqpoint{2.980176in}{2.224494in}}%
\pgfpathlineto{\pgfqpoint{2.983016in}{2.176199in}}%
\pgfpathlineto{\pgfqpoint{2.983790in}{2.181918in}}%
\pgfpathlineto{\pgfqpoint{2.986113in}{2.221366in}}%
\pgfpathlineto{\pgfqpoint{2.986888in}{2.215904in}}%
\pgfpathlineto{\pgfqpoint{2.989211in}{2.172258in}}%
\pgfpathlineto{\pgfqpoint{2.989727in}{2.176098in}}%
\pgfpathlineto{\pgfqpoint{2.992825in}{2.226941in}}%
\pgfpathlineto{\pgfqpoint{2.993599in}{2.224071in}}%
\pgfpathlineto{\pgfqpoint{2.996439in}{2.204266in}}%
\pgfpathlineto{\pgfqpoint{2.996955in}{2.205530in}}%
\pgfpathlineto{\pgfqpoint{2.999278in}{2.228212in}}%
\pgfpathlineto{\pgfqpoint{3.000053in}{2.223005in}}%
\pgfpathlineto{\pgfqpoint{3.002118in}{2.198719in}}%
\pgfpathlineto{\pgfqpoint{3.002634in}{2.200055in}}%
\pgfpathlineto{\pgfqpoint{3.004699in}{2.210049in}}%
\pgfpathlineto{\pgfqpoint{3.005215in}{2.209114in}}%
\pgfpathlineto{\pgfqpoint{3.007022in}{2.200641in}}%
\pgfpathlineto{\pgfqpoint{3.009087in}{2.185012in}}%
\pgfpathlineto{\pgfqpoint{3.009604in}{2.186126in}}%
\pgfpathlineto{\pgfqpoint{3.011411in}{2.206997in}}%
\pgfpathlineto{\pgfqpoint{3.013218in}{2.217532in}}%
\pgfpathlineto{\pgfqpoint{3.013992in}{2.214204in}}%
\pgfpathlineto{\pgfqpoint{3.016315in}{2.194471in}}%
\pgfpathlineto{\pgfqpoint{3.016831in}{2.195892in}}%
\pgfpathlineto{\pgfqpoint{3.018638in}{2.218557in}}%
\pgfpathlineto{\pgfqpoint{3.020187in}{2.227563in}}%
\pgfpathlineto{\pgfqpoint{3.020445in}{2.227114in}}%
\pgfpathlineto{\pgfqpoint{3.021478in}{2.218628in}}%
\pgfpathlineto{\pgfqpoint{3.023543in}{2.186593in}}%
\pgfpathlineto{\pgfqpoint{3.024059in}{2.188738in}}%
\pgfpathlineto{\pgfqpoint{3.026382in}{2.229942in}}%
\pgfpathlineto{\pgfqpoint{3.027415in}{2.220863in}}%
\pgfpathlineto{\pgfqpoint{3.030255in}{2.173600in}}%
\pgfpathlineto{\pgfqpoint{3.030513in}{2.174511in}}%
\pgfpathlineto{\pgfqpoint{3.031545in}{2.192576in}}%
\pgfpathlineto{\pgfqpoint{3.033352in}{2.225995in}}%
\pgfpathlineto{\pgfqpoint{3.033610in}{2.225004in}}%
\pgfpathlineto{\pgfqpoint{3.036192in}{2.192498in}}%
\pgfpathlineto{\pgfqpoint{3.037224in}{2.198672in}}%
\pgfpathlineto{\pgfqpoint{3.040322in}{2.230721in}}%
\pgfpathlineto{\pgfqpoint{3.040838in}{2.229178in}}%
\pgfpathlineto{\pgfqpoint{3.043678in}{2.208985in}}%
\pgfpathlineto{\pgfqpoint{3.044452in}{2.209705in}}%
\pgfpathlineto{\pgfqpoint{3.045226in}{2.210042in}}%
\pgfpathlineto{\pgfqpoint{3.045484in}{2.209771in}}%
\pgfpathlineto{\pgfqpoint{3.047550in}{2.204337in}}%
\pgfpathlineto{\pgfqpoint{3.049873in}{2.196147in}}%
\pgfpathlineto{\pgfqpoint{3.050389in}{2.196676in}}%
\pgfpathlineto{\pgfqpoint{3.052196in}{2.201378in}}%
\pgfpathlineto{\pgfqpoint{3.052712in}{2.199955in}}%
\pgfpathlineto{\pgfqpoint{3.054003in}{2.184058in}}%
\pgfpathlineto{\pgfqpoint{3.055294in}{2.172182in}}%
\pgfpathlineto{\pgfqpoint{3.055552in}{2.173434in}}%
\pgfpathlineto{\pgfqpoint{3.057101in}{2.200362in}}%
\pgfpathlineto{\pgfqpoint{3.058133in}{2.210921in}}%
\pgfpathlineto{\pgfqpoint{3.058649in}{2.209931in}}%
\pgfpathlineto{\pgfqpoint{3.059682in}{2.206355in}}%
\pgfpathlineto{\pgfqpoint{3.059940in}{2.206982in}}%
\pgfpathlineto{\pgfqpoint{3.061231in}{2.221063in}}%
\pgfpathlineto{\pgfqpoint{3.063554in}{2.239003in}}%
\pgfpathlineto{\pgfqpoint{3.064328in}{2.240009in}}%
\pgfpathlineto{\pgfqpoint{3.064586in}{2.239717in}}%
\pgfpathlineto{\pgfqpoint{3.065361in}{2.235197in}}%
\pgfpathlineto{\pgfqpoint{3.067426in}{2.215057in}}%
\pgfpathlineto{\pgfqpoint{3.067942in}{2.217017in}}%
\pgfpathlineto{\pgfqpoint{3.069233in}{2.224218in}}%
\pgfpathlineto{\pgfqpoint{3.069491in}{2.223169in}}%
\pgfpathlineto{\pgfqpoint{3.070524in}{2.206738in}}%
\pgfpathlineto{\pgfqpoint{3.072589in}{2.174797in}}%
\pgfpathlineto{\pgfqpoint{3.072847in}{2.174903in}}%
\pgfpathlineto{\pgfqpoint{3.079300in}{2.194238in}}%
\pgfpathlineto{\pgfqpoint{3.081881in}{2.218644in}}%
\pgfpathlineto{\pgfqpoint{3.082140in}{2.218202in}}%
\pgfpathlineto{\pgfqpoint{3.083172in}{2.216306in}}%
\pgfpathlineto{\pgfqpoint{3.083430in}{2.216668in}}%
\pgfpathlineto{\pgfqpoint{3.084463in}{2.223198in}}%
\pgfpathlineto{\pgfqpoint{3.086012in}{2.233452in}}%
\pgfpathlineto{\pgfqpoint{3.086270in}{2.233079in}}%
\pgfpathlineto{\pgfqpoint{3.088851in}{2.219361in}}%
\pgfpathlineto{\pgfqpoint{3.089884in}{2.222027in}}%
\pgfpathlineto{\pgfqpoint{3.090916in}{2.224565in}}%
\pgfpathlineto{\pgfqpoint{3.091174in}{2.224204in}}%
\pgfpathlineto{\pgfqpoint{3.092207in}{2.218259in}}%
\pgfpathlineto{\pgfqpoint{3.094272in}{2.190874in}}%
\pgfpathlineto{\pgfqpoint{3.096337in}{2.160956in}}%
\pgfpathlineto{\pgfqpoint{3.096595in}{2.161483in}}%
\pgfpathlineto{\pgfqpoint{3.097886in}{2.176728in}}%
\pgfpathlineto{\pgfqpoint{3.099693in}{2.185533in}}%
\pgfpathlineto{\pgfqpoint{3.100467in}{2.190703in}}%
\pgfpathlineto{\pgfqpoint{3.104597in}{2.237578in}}%
\pgfpathlineto{\pgfqpoint{3.105372in}{2.237783in}}%
\pgfpathlineto{\pgfqpoint{3.105630in}{2.237372in}}%
\pgfpathlineto{\pgfqpoint{3.109760in}{2.222182in}}%
\pgfpathlineto{\pgfqpoint{3.112341in}{2.170489in}}%
\pgfpathlineto{\pgfqpoint{3.113116in}{2.174254in}}%
\pgfpathlineto{\pgfqpoint{3.115181in}{2.192844in}}%
\pgfpathlineto{\pgfqpoint{3.115439in}{2.192195in}}%
\pgfpathlineto{\pgfqpoint{3.116730in}{2.179451in}}%
\pgfpathlineto{\pgfqpoint{3.117762in}{2.171702in}}%
\pgfpathlineto{\pgfqpoint{3.118020in}{2.172410in}}%
\pgfpathlineto{\pgfqpoint{3.119053in}{2.188650in}}%
\pgfpathlineto{\pgfqpoint{3.121892in}{2.243421in}}%
\pgfpathlineto{\pgfqpoint{3.122151in}{2.242500in}}%
\pgfpathlineto{\pgfqpoint{3.123957in}{2.223220in}}%
\pgfpathlineto{\pgfqpoint{3.125506in}{2.212167in}}%
\pgfpathlineto{\pgfqpoint{3.125764in}{2.212399in}}%
\pgfpathlineto{\pgfqpoint{3.127571in}{2.220189in}}%
\pgfpathlineto{\pgfqpoint{3.128088in}{2.218492in}}%
\pgfpathlineto{\pgfqpoint{3.130153in}{2.203830in}}%
\pgfpathlineto{\pgfqpoint{3.130669in}{2.205612in}}%
\pgfpathlineto{\pgfqpoint{3.131960in}{2.212174in}}%
\pgfpathlineto{\pgfqpoint{3.132218in}{2.211535in}}%
\pgfpathlineto{\pgfqpoint{3.133250in}{2.198384in}}%
\pgfpathlineto{\pgfqpoint{3.135057in}{2.170360in}}%
\pgfpathlineto{\pgfqpoint{3.135574in}{2.172908in}}%
\pgfpathlineto{\pgfqpoint{3.137122in}{2.210643in}}%
\pgfpathlineto{\pgfqpoint{3.138413in}{2.234960in}}%
\pgfpathlineto{\pgfqpoint{3.138929in}{2.232026in}}%
\pgfpathlineto{\pgfqpoint{3.141511in}{2.173996in}}%
\pgfpathlineto{\pgfqpoint{3.142543in}{2.186351in}}%
\pgfpathlineto{\pgfqpoint{3.145125in}{2.248573in}}%
\pgfpathlineto{\pgfqpoint{3.145641in}{2.245915in}}%
\pgfpathlineto{\pgfqpoint{3.146931in}{2.210717in}}%
\pgfpathlineto{\pgfqpoint{3.148997in}{2.158141in}}%
\pgfpathlineto{\pgfqpoint{3.149255in}{2.160175in}}%
\pgfpathlineto{\pgfqpoint{3.150803in}{2.202788in}}%
\pgfpathlineto{\pgfqpoint{3.152352in}{2.232322in}}%
\pgfpathlineto{\pgfqpoint{3.152610in}{2.232259in}}%
\pgfpathlineto{\pgfqpoint{3.153643in}{2.220728in}}%
\pgfpathlineto{\pgfqpoint{3.155450in}{2.196398in}}%
\pgfpathlineto{\pgfqpoint{3.155708in}{2.197109in}}%
\pgfpathlineto{\pgfqpoint{3.156741in}{2.214209in}}%
\pgfpathlineto{\pgfqpoint{3.158806in}{2.250443in}}%
\pgfpathlineto{\pgfqpoint{3.159064in}{2.249954in}}%
\pgfpathlineto{\pgfqpoint{3.160096in}{2.235213in}}%
\pgfpathlineto{\pgfqpoint{3.162936in}{2.156313in}}%
\pgfpathlineto{\pgfqpoint{3.163710in}{2.166386in}}%
\pgfpathlineto{\pgfqpoint{3.166292in}{2.229025in}}%
\pgfpathlineto{\pgfqpoint{3.166808in}{2.223927in}}%
\pgfpathlineto{\pgfqpoint{3.168873in}{2.162091in}}%
\pgfpathlineto{\pgfqpoint{3.169905in}{2.145980in}}%
\pgfpathlineto{\pgfqpoint{3.170164in}{2.147576in}}%
\pgfpathlineto{\pgfqpoint{3.171196in}{2.177949in}}%
\pgfpathlineto{\pgfqpoint{3.173777in}{2.263766in}}%
\pgfpathlineto{\pgfqpoint{3.174036in}{2.262448in}}%
\pgfpathlineto{\pgfqpoint{3.175584in}{2.231406in}}%
\pgfpathlineto{\pgfqpoint{3.177391in}{2.206582in}}%
\pgfpathlineto{\pgfqpoint{3.177649in}{2.206921in}}%
\pgfpathlineto{\pgfqpoint{3.178682in}{2.216440in}}%
\pgfpathlineto{\pgfqpoint{3.179973in}{2.227652in}}%
\pgfpathlineto{\pgfqpoint{3.180231in}{2.227021in}}%
\pgfpathlineto{\pgfqpoint{3.181263in}{2.212153in}}%
\pgfpathlineto{\pgfqpoint{3.183587in}{2.174028in}}%
\pgfpathlineto{\pgfqpoint{3.183845in}{2.174356in}}%
\pgfpathlineto{\pgfqpoint{3.184877in}{2.185818in}}%
\pgfpathlineto{\pgfqpoint{3.187200in}{2.229499in}}%
\pgfpathlineto{\pgfqpoint{3.187717in}{2.225677in}}%
\pgfpathlineto{\pgfqpoint{3.189266in}{2.174012in}}%
\pgfpathlineto{\pgfqpoint{3.190814in}{2.140829in}}%
\pgfpathlineto{\pgfqpoint{3.191073in}{2.143356in}}%
\pgfpathlineto{\pgfqpoint{3.192363in}{2.187542in}}%
\pgfpathlineto{\pgfqpoint{3.194945in}{2.267676in}}%
\pgfpathlineto{\pgfqpoint{3.195719in}{2.258347in}}%
\pgfpathlineto{\pgfqpoint{3.197784in}{2.226260in}}%
\pgfpathlineto{\pgfqpoint{3.198300in}{2.227865in}}%
\pgfpathlineto{\pgfqpoint{3.200107in}{2.242162in}}%
\pgfpathlineto{\pgfqpoint{3.200365in}{2.240651in}}%
\pgfpathlineto{\pgfqpoint{3.201398in}{2.217387in}}%
\pgfpathlineto{\pgfqpoint{3.204237in}{2.123126in}}%
\pgfpathlineto{\pgfqpoint{3.204754in}{2.128023in}}%
\pgfpathlineto{\pgfqpoint{3.206561in}{2.189319in}}%
\pgfpathlineto{\pgfqpoint{3.208884in}{2.234487in}}%
\pgfpathlineto{\pgfqpoint{3.209658in}{2.228177in}}%
\pgfpathlineto{\pgfqpoint{3.211981in}{2.204089in}}%
\pgfpathlineto{\pgfqpoint{3.212240in}{2.204586in}}%
\pgfpathlineto{\pgfqpoint{3.213272in}{2.212628in}}%
\pgfpathlineto{\pgfqpoint{3.216628in}{2.261205in}}%
\pgfpathlineto{\pgfqpoint{3.217402in}{2.252829in}}%
\pgfpathlineto{\pgfqpoint{3.219725in}{2.170931in}}%
\pgfpathlineto{\pgfqpoint{3.221532in}{2.139837in}}%
\pgfpathlineto{\pgfqpoint{3.221791in}{2.140858in}}%
\pgfpathlineto{\pgfqpoint{3.222823in}{2.157738in}}%
\pgfpathlineto{\pgfqpoint{3.225921in}{2.213317in}}%
\pgfpathlineto{\pgfqpoint{3.227470in}{2.221011in}}%
\pgfpathlineto{\pgfqpoint{3.230051in}{2.244106in}}%
\pgfpathlineto{\pgfqpoint{3.230567in}{2.243007in}}%
\pgfpathlineto{\pgfqpoint{3.232632in}{2.232728in}}%
\pgfpathlineto{\pgfqpoint{3.233148in}{2.233659in}}%
\pgfpathlineto{\pgfqpoint{3.234181in}{2.236206in}}%
\pgfpathlineto{\pgfqpoint{3.234439in}{2.235581in}}%
\pgfpathlineto{\pgfqpoint{3.235472in}{2.222718in}}%
\pgfpathlineto{\pgfqpoint{3.239086in}{2.149619in}}%
\pgfpathlineto{\pgfqpoint{3.239602in}{2.150362in}}%
\pgfpathlineto{\pgfqpoint{3.240893in}{2.161951in}}%
\pgfpathlineto{\pgfqpoint{3.243216in}{2.176961in}}%
\pgfpathlineto{\pgfqpoint{3.243990in}{2.178646in}}%
\pgfpathlineto{\pgfqpoint{3.245023in}{2.189573in}}%
\pgfpathlineto{\pgfqpoint{3.251476in}{2.283858in}}%
\pgfpathlineto{\pgfqpoint{3.251992in}{2.286685in}}%
\pgfpathlineto{\pgfqpoint{3.252509in}{2.285074in}}%
\pgfpathlineto{\pgfqpoint{3.253541in}{2.262833in}}%
\pgfpathlineto{\pgfqpoint{3.257155in}{2.151963in}}%
\pgfpathlineto{\pgfqpoint{3.257413in}{2.152529in}}%
\pgfpathlineto{\pgfqpoint{3.258962in}{2.160358in}}%
\pgfpathlineto{\pgfqpoint{3.259478in}{2.158876in}}%
\pgfpathlineto{\pgfqpoint{3.261027in}{2.138831in}}%
\pgfpathlineto{\pgfqpoint{3.262060in}{2.129748in}}%
\pgfpathlineto{\pgfqpoint{3.262576in}{2.132754in}}%
\pgfpathlineto{\pgfqpoint{3.263867in}{2.166872in}}%
\pgfpathlineto{\pgfqpoint{3.267222in}{2.260677in}}%
\pgfpathlineto{\pgfqpoint{3.267739in}{2.260197in}}%
\pgfpathlineto{\pgfqpoint{3.272643in}{2.235830in}}%
\pgfpathlineto{\pgfqpoint{3.278064in}{2.177737in}}%
\pgfpathlineto{\pgfqpoint{3.278838in}{2.180124in}}%
\pgfpathlineto{\pgfqpoint{3.280645in}{2.198760in}}%
\pgfpathlineto{\pgfqpoint{3.283485in}{2.219497in}}%
\pgfpathlineto{\pgfqpoint{3.284259in}{2.217919in}}%
\pgfpathlineto{\pgfqpoint{3.288389in}{2.196373in}}%
\pgfpathlineto{\pgfqpoint{3.289680in}{2.187634in}}%
\pgfpathlineto{\pgfqpoint{3.290196in}{2.188755in}}%
\pgfpathlineto{\pgfqpoint{3.291487in}{2.205381in}}%
\pgfpathlineto{\pgfqpoint{3.293294in}{2.223278in}}%
\pgfpathlineto{\pgfqpoint{3.293552in}{2.222862in}}%
\pgfpathlineto{\pgfqpoint{3.294585in}{2.214000in}}%
\pgfpathlineto{\pgfqpoint{3.297424in}{2.178905in}}%
\pgfpathlineto{\pgfqpoint{3.297940in}{2.179937in}}%
\pgfpathlineto{\pgfqpoint{3.299231in}{2.197923in}}%
\pgfpathlineto{\pgfqpoint{3.301296in}{2.224303in}}%
\pgfpathlineto{\pgfqpoint{3.301554in}{2.223812in}}%
\pgfpathlineto{\pgfqpoint{3.302845in}{2.211430in}}%
\pgfpathlineto{\pgfqpoint{3.304910in}{2.192641in}}%
\pgfpathlineto{\pgfqpoint{3.305168in}{2.193339in}}%
\pgfpathlineto{\pgfqpoint{3.306201in}{2.207708in}}%
\pgfpathlineto{\pgfqpoint{3.308524in}{2.256964in}}%
\pgfpathlineto{\pgfqpoint{3.309040in}{2.254352in}}%
\pgfpathlineto{\pgfqpoint{3.312138in}{2.200618in}}%
\pgfpathlineto{\pgfqpoint{3.313428in}{2.208104in}}%
\pgfpathlineto{\pgfqpoint{3.315235in}{2.221996in}}%
\pgfpathlineto{\pgfqpoint{3.315493in}{2.221453in}}%
\pgfpathlineto{\pgfqpoint{3.316268in}{2.213025in}}%
\pgfpathlineto{\pgfqpoint{3.318591in}{2.144425in}}%
\pgfpathlineto{\pgfqpoint{3.319882in}{2.127050in}}%
\pgfpathlineto{\pgfqpoint{3.320140in}{2.128186in}}%
\pgfpathlineto{\pgfqpoint{3.321172in}{2.145163in}}%
\pgfpathlineto{\pgfqpoint{3.325303in}{2.267585in}}%
\pgfpathlineto{\pgfqpoint{3.326335in}{2.250938in}}%
\pgfpathlineto{\pgfqpoint{3.328917in}{2.198741in}}%
\pgfpathlineto{\pgfqpoint{3.329175in}{2.199412in}}%
\pgfpathlineto{\pgfqpoint{3.330207in}{2.213205in}}%
\pgfpathlineto{\pgfqpoint{3.332272in}{2.249094in}}%
\pgfpathlineto{\pgfqpoint{3.332789in}{2.246321in}}%
\pgfpathlineto{\pgfqpoint{3.336402in}{2.188849in}}%
\pgfpathlineto{\pgfqpoint{3.337693in}{2.190055in}}%
\pgfpathlineto{\pgfqpoint{3.338984in}{2.193239in}}%
\pgfpathlineto{\pgfqpoint{3.340533in}{2.198392in}}%
\pgfpathlineto{\pgfqpoint{3.341049in}{2.197723in}}%
\pgfpathlineto{\pgfqpoint{3.343372in}{2.194390in}}%
\pgfpathlineto{\pgfqpoint{3.344405in}{2.192900in}}%
\pgfpathlineto{\pgfqpoint{3.345437in}{2.190534in}}%
\pgfpathlineto{\pgfqpoint{3.345953in}{2.191540in}}%
\pgfpathlineto{\pgfqpoint{3.346986in}{2.202844in}}%
\pgfpathlineto{\pgfqpoint{3.349309in}{2.237902in}}%
\pgfpathlineto{\pgfqpoint{3.349567in}{2.236588in}}%
\pgfpathlineto{\pgfqpoint{3.350858in}{2.210092in}}%
\pgfpathlineto{\pgfqpoint{3.352665in}{2.179967in}}%
\pgfpathlineto{\pgfqpoint{3.352923in}{2.181293in}}%
\pgfpathlineto{\pgfqpoint{3.354214in}{2.204332in}}%
\pgfpathlineto{\pgfqpoint{3.356537in}{2.234575in}}%
\pgfpathlineto{\pgfqpoint{3.357311in}{2.231671in}}%
\pgfpathlineto{\pgfqpoint{3.360151in}{2.204577in}}%
\pgfpathlineto{\pgfqpoint{3.362990in}{2.179327in}}%
\pgfpathlineto{\pgfqpoint{3.363248in}{2.179649in}}%
\pgfpathlineto{\pgfqpoint{3.364023in}{2.185495in}}%
\pgfpathlineto{\pgfqpoint{3.367120in}{2.237456in}}%
\pgfpathlineto{\pgfqpoint{3.367895in}{2.231297in}}%
\pgfpathlineto{\pgfqpoint{3.371767in}{2.181348in}}%
\pgfpathlineto{\pgfqpoint{3.372025in}{2.181412in}}%
\pgfpathlineto{\pgfqpoint{3.373316in}{2.185710in}}%
\pgfpathlineto{\pgfqpoint{3.375381in}{2.198271in}}%
\pgfpathlineto{\pgfqpoint{3.378220in}{2.233694in}}%
\pgfpathlineto{\pgfqpoint{3.378995in}{2.228298in}}%
\pgfpathlineto{\pgfqpoint{3.381318in}{2.196935in}}%
\pgfpathlineto{\pgfqpoint{3.381834in}{2.198936in}}%
\pgfpathlineto{\pgfqpoint{3.384415in}{2.229969in}}%
\pgfpathlineto{\pgfqpoint{3.385448in}{2.223551in}}%
\pgfpathlineto{\pgfqpoint{3.387255in}{2.204245in}}%
\pgfpathlineto{\pgfqpoint{3.387771in}{2.205622in}}%
\pgfpathlineto{\pgfqpoint{3.389578in}{2.220505in}}%
\pgfpathlineto{\pgfqpoint{3.390094in}{2.218134in}}%
\pgfpathlineto{\pgfqpoint{3.391643in}{2.187106in}}%
\pgfpathlineto{\pgfqpoint{3.393450in}{2.162347in}}%
\pgfpathlineto{\pgfqpoint{3.393708in}{2.162838in}}%
\pgfpathlineto{\pgfqpoint{3.394741in}{2.173595in}}%
\pgfpathlineto{\pgfqpoint{3.398613in}{2.219996in}}%
\pgfpathlineto{\pgfqpoint{3.400936in}{2.225689in}}%
\pgfpathlineto{\pgfqpoint{3.401194in}{2.225431in}}%
\pgfpathlineto{\pgfqpoint{3.402743in}{2.222453in}}%
\pgfpathlineto{\pgfqpoint{3.403259in}{2.223088in}}%
\pgfpathlineto{\pgfqpoint{3.405324in}{2.230748in}}%
\pgfpathlineto{\pgfqpoint{3.405841in}{2.228950in}}%
\pgfpathlineto{\pgfqpoint{3.407131in}{2.210637in}}%
\pgfpathlineto{\pgfqpoint{3.409455in}{2.178556in}}%
\pgfpathlineto{\pgfqpoint{3.409713in}{2.179590in}}%
\pgfpathlineto{\pgfqpoint{3.411262in}{2.201156in}}%
\pgfpathlineto{\pgfqpoint{3.412552in}{2.213672in}}%
\pgfpathlineto{\pgfqpoint{3.413068in}{2.211835in}}%
\pgfpathlineto{\pgfqpoint{3.414617in}{2.186145in}}%
\pgfpathlineto{\pgfqpoint{3.416166in}{2.168761in}}%
\pgfpathlineto{\pgfqpoint{3.416424in}{2.170180in}}%
\pgfpathlineto{\pgfqpoint{3.417715in}{2.195048in}}%
\pgfpathlineto{\pgfqpoint{3.420038in}{2.237146in}}%
\pgfpathlineto{\pgfqpoint{3.420296in}{2.236900in}}%
\pgfpathlineto{\pgfqpoint{3.421329in}{2.226711in}}%
\pgfpathlineto{\pgfqpoint{3.423910in}{2.195484in}}%
\pgfpathlineto{\pgfqpoint{3.424168in}{2.196546in}}%
\pgfpathlineto{\pgfqpoint{3.425459in}{2.216808in}}%
\pgfpathlineto{\pgfqpoint{3.427266in}{2.243499in}}%
\pgfpathlineto{\pgfqpoint{3.427524in}{2.242675in}}%
\pgfpathlineto{\pgfqpoint{3.428557in}{2.227171in}}%
\pgfpathlineto{\pgfqpoint{3.431138in}{2.183925in}}%
\pgfpathlineto{\pgfqpoint{3.431396in}{2.184044in}}%
\pgfpathlineto{\pgfqpoint{3.432429in}{2.192411in}}%
\pgfpathlineto{\pgfqpoint{3.434236in}{2.207055in}}%
\pgfpathlineto{\pgfqpoint{3.434494in}{2.206309in}}%
\pgfpathlineto{\pgfqpoint{3.436042in}{2.191435in}}%
\pgfpathlineto{\pgfqpoint{3.438108in}{2.178933in}}%
\pgfpathlineto{\pgfqpoint{3.438624in}{2.179244in}}%
\pgfpathlineto{\pgfqpoint{3.439656in}{2.185096in}}%
\pgfpathlineto{\pgfqpoint{3.441463in}{2.214039in}}%
\pgfpathlineto{\pgfqpoint{3.444303in}{2.250999in}}%
\pgfpathlineto{\pgfqpoint{3.445077in}{2.246988in}}%
\pgfpathlineto{\pgfqpoint{3.446626in}{2.217900in}}%
\pgfpathlineto{\pgfqpoint{3.448949in}{2.169552in}}%
\pgfpathlineto{\pgfqpoint{3.449207in}{2.169888in}}%
\pgfpathlineto{\pgfqpoint{3.450240in}{2.183687in}}%
\pgfpathlineto{\pgfqpoint{3.453079in}{2.234971in}}%
\pgfpathlineto{\pgfqpoint{3.453596in}{2.233593in}}%
\pgfpathlineto{\pgfqpoint{3.459016in}{2.182228in}}%
\pgfpathlineto{\pgfqpoint{3.461082in}{2.154893in}}%
\pgfpathlineto{\pgfqpoint{3.461340in}{2.155051in}}%
\pgfpathlineto{\pgfqpoint{3.462114in}{2.161256in}}%
\pgfpathlineto{\pgfqpoint{3.463921in}{2.204366in}}%
\pgfpathlineto{\pgfqpoint{3.465986in}{2.235806in}}%
\pgfpathlineto{\pgfqpoint{3.466502in}{2.235775in}}%
\pgfpathlineto{\pgfqpoint{3.466761in}{2.235365in}}%
\pgfpathlineto{\pgfqpoint{3.467793in}{2.234539in}}%
\pgfpathlineto{\pgfqpoint{3.468051in}{2.234796in}}%
\pgfpathlineto{\pgfqpoint{3.469084in}{2.235706in}}%
\pgfpathlineto{\pgfqpoint{3.469342in}{2.235332in}}%
\pgfpathlineto{\pgfqpoint{3.470374in}{2.229935in}}%
\pgfpathlineto{\pgfqpoint{3.472439in}{2.218737in}}%
\pgfpathlineto{\pgfqpoint{3.472698in}{2.219041in}}%
\pgfpathlineto{\pgfqpoint{3.473988in}{2.222372in}}%
\pgfpathlineto{\pgfqpoint{3.474505in}{2.221353in}}%
\pgfpathlineto{\pgfqpoint{3.475537in}{2.210902in}}%
\pgfpathlineto{\pgfqpoint{3.478377in}{2.171148in}}%
\pgfpathlineto{\pgfqpoint{3.478893in}{2.172751in}}%
\pgfpathlineto{\pgfqpoint{3.481216in}{2.192583in}}%
\pgfpathlineto{\pgfqpoint{3.481990in}{2.188415in}}%
\pgfpathlineto{\pgfqpoint{3.483797in}{2.174053in}}%
\pgfpathlineto{\pgfqpoint{3.484056in}{2.174903in}}%
\pgfpathlineto{\pgfqpoint{3.485088in}{2.188891in}}%
\pgfpathlineto{\pgfqpoint{3.488186in}{2.249518in}}%
\pgfpathlineto{\pgfqpoint{3.488702in}{2.247289in}}%
\pgfpathlineto{\pgfqpoint{3.492058in}{2.212374in}}%
\pgfpathlineto{\pgfqpoint{3.492832in}{2.213078in}}%
\pgfpathlineto{\pgfqpoint{3.494381in}{2.218584in}}%
\pgfpathlineto{\pgfqpoint{3.495413in}{2.221768in}}%
\pgfpathlineto{\pgfqpoint{3.495672in}{2.221356in}}%
\pgfpathlineto{\pgfqpoint{3.496704in}{2.212504in}}%
\pgfpathlineto{\pgfqpoint{3.499544in}{2.187460in}}%
\pgfpathlineto{\pgfqpoint{3.502125in}{2.190480in}}%
\pgfpathlineto{\pgfqpoint{3.505739in}{2.207678in}}%
\pgfpathlineto{\pgfqpoint{3.507546in}{2.219960in}}%
\pgfpathlineto{\pgfqpoint{3.508062in}{2.218476in}}%
\pgfpathlineto{\pgfqpoint{3.509611in}{2.199145in}}%
\pgfpathlineto{\pgfqpoint{3.511160in}{2.186502in}}%
\pgfpathlineto{\pgfqpoint{3.511418in}{2.186994in}}%
\pgfpathlineto{\pgfqpoint{3.512967in}{2.198929in}}%
\pgfpathlineto{\pgfqpoint{3.515290in}{2.211449in}}%
\pgfpathlineto{\pgfqpoint{3.517097in}{2.214023in}}%
\pgfpathlineto{\pgfqpoint{3.518387in}{2.225224in}}%
\pgfpathlineto{\pgfqpoint{3.520453in}{2.245770in}}%
\pgfpathlineto{\pgfqpoint{3.520711in}{2.245085in}}%
\pgfpathlineto{\pgfqpoint{3.521743in}{2.232571in}}%
\pgfpathlineto{\pgfqpoint{3.524325in}{2.197217in}}%
\pgfpathlineto{\pgfqpoint{3.524583in}{2.197661in}}%
\pgfpathlineto{\pgfqpoint{3.526132in}{2.203542in}}%
\pgfpathlineto{\pgfqpoint{3.526648in}{2.201344in}}%
\pgfpathlineto{\pgfqpoint{3.529229in}{2.176178in}}%
\pgfpathlineto{\pgfqpoint{3.530004in}{2.179591in}}%
\pgfpathlineto{\pgfqpoint{3.532843in}{2.196383in}}%
\pgfpathlineto{\pgfqpoint{3.534908in}{2.196245in}}%
\pgfpathlineto{\pgfqpoint{3.535941in}{2.195552in}}%
\pgfpathlineto{\pgfqpoint{3.536199in}{2.196222in}}%
\pgfpathlineto{\pgfqpoint{3.537231in}{2.206005in}}%
\pgfpathlineto{\pgfqpoint{3.539555in}{2.238441in}}%
\pgfpathlineto{\pgfqpoint{3.540071in}{2.235790in}}%
\pgfpathlineto{\pgfqpoint{3.542910in}{2.207533in}}%
\pgfpathlineto{\pgfqpoint{3.543427in}{2.208892in}}%
\pgfpathlineto{\pgfqpoint{3.545750in}{2.219691in}}%
\pgfpathlineto{\pgfqpoint{3.546266in}{2.217857in}}%
\pgfpathlineto{\pgfqpoint{3.549106in}{2.196643in}}%
\pgfpathlineto{\pgfqpoint{3.549880in}{2.199410in}}%
\pgfpathlineto{\pgfqpoint{3.552461in}{2.214769in}}%
\pgfpathlineto{\pgfqpoint{3.552978in}{2.213760in}}%
\pgfpathlineto{\pgfqpoint{3.554268in}{2.199920in}}%
\pgfpathlineto{\pgfqpoint{3.556333in}{2.180464in}}%
\pgfpathlineto{\pgfqpoint{3.556591in}{2.180992in}}%
\pgfpathlineto{\pgfqpoint{3.557882in}{2.192182in}}%
\pgfpathlineto{\pgfqpoint{3.560980in}{2.223286in}}%
\pgfpathlineto{\pgfqpoint{3.561238in}{2.222887in}}%
\pgfpathlineto{\pgfqpoint{3.563045in}{2.214007in}}%
\pgfpathlineto{\pgfqpoint{3.565110in}{2.204220in}}%
\pgfpathlineto{\pgfqpoint{3.565368in}{2.204453in}}%
\pgfpathlineto{\pgfqpoint{3.566659in}{2.210831in}}%
\pgfpathlineto{\pgfqpoint{3.567691in}{2.214390in}}%
\pgfpathlineto{\pgfqpoint{3.568208in}{2.213761in}}%
\pgfpathlineto{\pgfqpoint{3.570014in}{2.209589in}}%
\pgfpathlineto{\pgfqpoint{3.570531in}{2.210303in}}%
\pgfpathlineto{\pgfqpoint{3.571563in}{2.211431in}}%
\pgfpathlineto{\pgfqpoint{3.571821in}{2.211015in}}%
\pgfpathlineto{\pgfqpoint{3.573112in}{2.203394in}}%
\pgfpathlineto{\pgfqpoint{3.576726in}{2.178152in}}%
\pgfpathlineto{\pgfqpoint{3.576984in}{2.178322in}}%
\pgfpathlineto{\pgfqpoint{3.577758in}{2.182912in}}%
\pgfpathlineto{\pgfqpoint{3.581372in}{2.222027in}}%
\pgfpathlineto{\pgfqpoint{3.582147in}{2.220451in}}%
\pgfpathlineto{\pgfqpoint{3.585503in}{2.207968in}}%
\pgfpathlineto{\pgfqpoint{3.586019in}{2.210005in}}%
\pgfpathlineto{\pgfqpoint{3.587568in}{2.230010in}}%
\pgfpathlineto{\pgfqpoint{3.589116in}{2.244167in}}%
\pgfpathlineto{\pgfqpoint{3.589375in}{2.243932in}}%
\pgfpathlineto{\pgfqpoint{3.590407in}{2.236250in}}%
\pgfpathlineto{\pgfqpoint{3.595054in}{2.191942in}}%
\pgfpathlineto{\pgfqpoint{3.596602in}{2.175224in}}%
\pgfpathlineto{\pgfqpoint{3.598667in}{2.152446in}}%
\pgfpathlineto{\pgfqpoint{3.598926in}{2.152998in}}%
\pgfpathlineto{\pgfqpoint{3.599958in}{2.165736in}}%
\pgfpathlineto{\pgfqpoint{3.603572in}{2.220456in}}%
\pgfpathlineto{\pgfqpoint{3.604863in}{2.223586in}}%
\pgfpathlineto{\pgfqpoint{3.605121in}{2.223413in}}%
\pgfpathlineto{\pgfqpoint{3.606153in}{2.218935in}}%
\pgfpathlineto{\pgfqpoint{3.607960in}{2.209692in}}%
\pgfpathlineto{\pgfqpoint{3.608218in}{2.209940in}}%
\pgfpathlineto{\pgfqpoint{3.609251in}{2.215178in}}%
\pgfpathlineto{\pgfqpoint{3.612865in}{2.249464in}}%
\pgfpathlineto{\pgfqpoint{3.613639in}{2.245589in}}%
\pgfpathlineto{\pgfqpoint{3.621383in}{2.170044in}}%
\pgfpathlineto{\pgfqpoint{3.622416in}{2.171104in}}%
\pgfpathlineto{\pgfqpoint{3.626288in}{2.181278in}}%
\pgfpathlineto{\pgfqpoint{3.628353in}{2.207142in}}%
\pgfpathlineto{\pgfqpoint{3.631192in}{2.233396in}}%
\pgfpathlineto{\pgfqpoint{3.633257in}{2.236526in}}%
\pgfpathlineto{\pgfqpoint{3.634290in}{2.235481in}}%
\pgfpathlineto{\pgfqpoint{3.637388in}{2.226141in}}%
\pgfpathlineto{\pgfqpoint{3.639969in}{2.206149in}}%
\pgfpathlineto{\pgfqpoint{3.640485in}{2.207682in}}%
\pgfpathlineto{\pgfqpoint{3.641776in}{2.213442in}}%
\pgfpathlineto{\pgfqpoint{3.642034in}{2.212869in}}%
\pgfpathlineto{\pgfqpoint{3.643067in}{2.200572in}}%
\pgfpathlineto{\pgfqpoint{3.646164in}{2.153146in}}%
\pgfpathlineto{\pgfqpoint{3.646422in}{2.153205in}}%
\pgfpathlineto{\pgfqpoint{3.647197in}{2.157631in}}%
\pgfpathlineto{\pgfqpoint{3.655973in}{2.238702in}}%
\pgfpathlineto{\pgfqpoint{3.657006in}{2.240001in}}%
\pgfpathlineto{\pgfqpoint{3.658555in}{2.246727in}}%
\pgfpathlineto{\pgfqpoint{3.660362in}{2.257588in}}%
\pgfpathlineto{\pgfqpoint{3.660620in}{2.256808in}}%
\pgfpathlineto{\pgfqpoint{3.661394in}{2.245243in}}%
\pgfpathlineto{\pgfqpoint{3.663201in}{2.164453in}}%
\pgfpathlineto{\pgfqpoint{3.664750in}{2.121576in}}%
\pgfpathlineto{\pgfqpoint{3.665008in}{2.123315in}}%
\pgfpathlineto{\pgfqpoint{3.666299in}{2.158113in}}%
\pgfpathlineto{\pgfqpoint{3.668622in}{2.205456in}}%
\pgfpathlineto{\pgfqpoint{3.669396in}{2.200146in}}%
\pgfpathlineto{\pgfqpoint{3.670687in}{2.187311in}}%
\pgfpathlineto{\pgfqpoint{3.671203in}{2.189083in}}%
\pgfpathlineto{\pgfqpoint{3.672494in}{2.219866in}}%
\pgfpathlineto{\pgfqpoint{3.674817in}{2.271583in}}%
\pgfpathlineto{\pgfqpoint{3.675075in}{2.270270in}}%
\pgfpathlineto{\pgfqpoint{3.676108in}{2.249368in}}%
\pgfpathlineto{\pgfqpoint{3.679722in}{2.159808in}}%
\pgfpathlineto{\pgfqpoint{3.680754in}{2.160713in}}%
\pgfpathlineto{\pgfqpoint{3.682045in}{2.165082in}}%
\pgfpathlineto{\pgfqpoint{3.683852in}{2.187369in}}%
\pgfpathlineto{\pgfqpoint{3.684884in}{2.194777in}}%
\pgfpathlineto{\pgfqpoint{3.685401in}{2.194150in}}%
\pgfpathlineto{\pgfqpoint{3.685917in}{2.193211in}}%
\pgfpathlineto{\pgfqpoint{3.686175in}{2.193599in}}%
\pgfpathlineto{\pgfqpoint{3.686950in}{2.201809in}}%
\pgfpathlineto{\pgfqpoint{3.690047in}{2.267509in}}%
\pgfpathlineto{\pgfqpoint{3.690822in}{2.263433in}}%
\pgfpathlineto{\pgfqpoint{3.691338in}{2.261330in}}%
\pgfpathlineto{\pgfqpoint{3.691854in}{2.263332in}}%
\pgfpathlineto{\pgfqpoint{3.693145in}{2.291149in}}%
\pgfpathlineto{\pgfqpoint{3.694694in}{2.321017in}}%
\pgfpathlineto{\pgfqpoint{3.694952in}{2.319844in}}%
\pgfpathlineto{\pgfqpoint{3.696501in}{2.288720in}}%
\pgfpathlineto{\pgfqpoint{3.698049in}{2.230648in}}%
\pgfpathlineto{\pgfqpoint{3.701405in}{1.944884in}}%
\pgfpathlineto{\pgfqpoint{3.702438in}{1.973875in}}%
\pgfpathlineto{\pgfqpoint{3.704245in}{2.138286in}}%
\pgfpathlineto{\pgfqpoint{3.706052in}{2.292646in}}%
\pgfpathlineto{\pgfqpoint{3.706310in}{2.291396in}}%
\pgfpathlineto{\pgfqpoint{3.707342in}{2.217415in}}%
\pgfpathlineto{\pgfqpoint{3.708891in}{2.086684in}}%
\pgfpathlineto{\pgfqpoint{3.709407in}{2.097518in}}%
\pgfpathlineto{\pgfqpoint{3.710698in}{2.258109in}}%
\pgfpathlineto{\pgfqpoint{3.712505in}{2.442798in}}%
\pgfpathlineto{\pgfqpoint{3.712763in}{2.439821in}}%
\pgfpathlineto{\pgfqpoint{3.715086in}{2.332725in}}%
\pgfpathlineto{\pgfqpoint{3.715861in}{2.349714in}}%
\pgfpathlineto{\pgfqpoint{3.716893in}{2.369687in}}%
\pgfpathlineto{\pgfqpoint{3.717151in}{2.365579in}}%
\pgfpathlineto{\pgfqpoint{3.718184in}{2.298631in}}%
\pgfpathlineto{\pgfqpoint{3.721540in}{1.951056in}}%
\pgfpathlineto{\pgfqpoint{3.722314in}{1.972018in}}%
\pgfpathlineto{\pgfqpoint{3.724895in}{2.143114in}}%
\pgfpathlineto{\pgfqpoint{3.725412in}{2.134943in}}%
\pgfpathlineto{\pgfqpoint{3.726444in}{2.115029in}}%
\pgfpathlineto{\pgfqpoint{3.726702in}{2.119021in}}%
\pgfpathlineto{\pgfqpoint{3.727735in}{2.196159in}}%
\pgfpathlineto{\pgfqpoint{3.730058in}{2.416683in}}%
\pgfpathlineto{\pgfqpoint{3.730316in}{2.407786in}}%
\pgfpathlineto{\pgfqpoint{3.731865in}{2.229056in}}%
\pgfpathlineto{\pgfqpoint{3.733672in}{2.099033in}}%
\pgfpathlineto{\pgfqpoint{3.733930in}{2.102916in}}%
\pgfpathlineto{\pgfqpoint{3.734963in}{2.164985in}}%
\pgfpathlineto{\pgfqpoint{3.737544in}{2.361689in}}%
\pgfpathlineto{\pgfqpoint{3.738060in}{2.353963in}}%
\pgfpathlineto{\pgfqpoint{3.739351in}{2.258587in}}%
\pgfpathlineto{\pgfqpoint{3.741158in}{2.140851in}}%
\pgfpathlineto{\pgfqpoint{3.741416in}{2.144515in}}%
\pgfpathlineto{\pgfqpoint{3.743481in}{2.256163in}}%
\pgfpathlineto{\pgfqpoint{3.744255in}{2.214728in}}%
\pgfpathlineto{\pgfqpoint{3.746321in}{2.032085in}}%
\pgfpathlineto{\pgfqpoint{3.746837in}{2.048911in}}%
\pgfpathlineto{\pgfqpoint{3.749418in}{2.221955in}}%
\pgfpathlineto{\pgfqpoint{3.750193in}{2.208211in}}%
\pgfpathlineto{\pgfqpoint{3.752000in}{2.147867in}}%
\pgfpathlineto{\pgfqpoint{3.752516in}{2.152847in}}%
\pgfpathlineto{\pgfqpoint{3.753806in}{2.231066in}}%
\pgfpathlineto{\pgfqpoint{3.755613in}{2.346101in}}%
\pgfpathlineto{\pgfqpoint{3.756130in}{2.337348in}}%
\pgfpathlineto{\pgfqpoint{3.757678in}{2.206759in}}%
\pgfpathlineto{\pgfqpoint{3.758711in}{2.147712in}}%
\pgfpathlineto{\pgfqpoint{3.759227in}{2.154178in}}%
\pgfpathlineto{\pgfqpoint{3.761034in}{2.288633in}}%
\pgfpathlineto{\pgfqpoint{3.761809in}{2.318388in}}%
\pgfpathlineto{\pgfqpoint{3.762325in}{2.308817in}}%
\pgfpathlineto{\pgfqpoint{3.763874in}{2.169234in}}%
\pgfpathlineto{\pgfqpoint{3.765423in}{2.077125in}}%
\pgfpathlineto{\pgfqpoint{3.765681in}{2.084331in}}%
\pgfpathlineto{\pgfqpoint{3.766971in}{2.207360in}}%
\pgfpathlineto{\pgfqpoint{3.768778in}{2.363495in}}%
\pgfpathlineto{\pgfqpoint{3.769036in}{2.363118in}}%
\pgfpathlineto{\pgfqpoint{3.769811in}{2.326350in}}%
\pgfpathlineto{\pgfqpoint{3.773683in}{2.093182in}}%
\pgfpathlineto{\pgfqpoint{3.774199in}{2.097536in}}%
\pgfpathlineto{\pgfqpoint{3.777039in}{2.155935in}}%
\pgfpathlineto{\pgfqpoint{3.777813in}{2.147425in}}%
\pgfpathlineto{\pgfqpoint{3.779620in}{2.118467in}}%
\pgfpathlineto{\pgfqpoint{3.780136in}{2.123737in}}%
\pgfpathlineto{\pgfqpoint{3.781427in}{2.181065in}}%
\pgfpathlineto{\pgfqpoint{3.785041in}{2.344896in}}%
\pgfpathlineto{\pgfqpoint{3.786590in}{2.391118in}}%
\pgfpathlineto{\pgfqpoint{3.787106in}{2.386236in}}%
\pgfpathlineto{\pgfqpoint{3.788397in}{2.326142in}}%
\pgfpathlineto{\pgfqpoint{3.792010in}{2.074284in}}%
\pgfpathlineto{\pgfqpoint{3.792785in}{2.090158in}}%
\pgfpathlineto{\pgfqpoint{3.795624in}{2.233944in}}%
\pgfpathlineto{\pgfqpoint{3.796399in}{2.220911in}}%
\pgfpathlineto{\pgfqpoint{3.800787in}{2.102828in}}%
\pgfpathlineto{\pgfqpoint{3.801820in}{2.096776in}}%
\pgfpathlineto{\pgfqpoint{3.802336in}{2.098695in}}%
\pgfpathlineto{\pgfqpoint{3.803110in}{2.120613in}}%
\pgfpathlineto{\pgfqpoint{3.804917in}{2.274887in}}%
\pgfpathlineto{\pgfqpoint{3.806466in}{2.359977in}}%
\pgfpathlineto{\pgfqpoint{3.806724in}{2.357358in}}%
\pgfpathlineto{\pgfqpoint{3.808015in}{2.295267in}}%
\pgfpathlineto{\pgfqpoint{3.809822in}{2.227091in}}%
\pgfpathlineto{\pgfqpoint{3.810080in}{2.228262in}}%
\pgfpathlineto{\pgfqpoint{3.811629in}{2.265784in}}%
\pgfpathlineto{\pgfqpoint{3.812661in}{2.278070in}}%
\pgfpathlineto{\pgfqpoint{3.813177in}{2.276091in}}%
\pgfpathlineto{\pgfqpoint{3.813952in}{2.273163in}}%
\pgfpathlineto{\pgfqpoint{3.814210in}{2.274428in}}%
\pgfpathlineto{\pgfqpoint{3.815759in}{2.299025in}}%
\pgfpathlineto{\pgfqpoint{3.816275in}{2.293495in}}%
\pgfpathlineto{\pgfqpoint{3.817308in}{2.225761in}}%
\pgfpathlineto{\pgfqpoint{3.820405in}{1.976227in}}%
\pgfpathlineto{\pgfqpoint{3.820663in}{1.980406in}}%
\pgfpathlineto{\pgfqpoint{3.822212in}{2.061317in}}%
\pgfpathlineto{\pgfqpoint{3.824794in}{2.166784in}}%
\pgfpathlineto{\pgfqpoint{3.825310in}{2.162901in}}%
\pgfpathlineto{\pgfqpoint{3.826600in}{2.140228in}}%
\pgfpathlineto{\pgfqpoint{3.827117in}{2.144756in}}%
\pgfpathlineto{\pgfqpoint{3.828149in}{2.209290in}}%
\pgfpathlineto{\pgfqpoint{3.831247in}{2.492094in}}%
\pgfpathlineto{\pgfqpoint{3.831763in}{2.482491in}}%
\pgfpathlineto{\pgfqpoint{3.833312in}{2.354646in}}%
\pgfpathlineto{\pgfqpoint{3.835119in}{2.240399in}}%
\pgfpathlineto{\pgfqpoint{3.835377in}{2.240526in}}%
\pgfpathlineto{\pgfqpoint{3.837184in}{2.271760in}}%
\pgfpathlineto{\pgfqpoint{3.837700in}{2.262436in}}%
\pgfpathlineto{\pgfqpoint{3.839249in}{2.170942in}}%
\pgfpathlineto{\pgfqpoint{3.842605in}{1.917284in}}%
\pgfpathlineto{\pgfqpoint{3.843121in}{1.925782in}}%
\pgfpathlineto{\pgfqpoint{3.844670in}{2.052870in}}%
\pgfpathlineto{\pgfqpoint{3.846993in}{2.192782in}}%
\pgfpathlineto{\pgfqpoint{3.847509in}{2.189769in}}%
\pgfpathlineto{\pgfqpoint{3.848800in}{2.175016in}}%
\pgfpathlineto{\pgfqpoint{3.849058in}{2.177680in}}%
\pgfpathlineto{\pgfqpoint{3.850091in}{2.225321in}}%
\pgfpathlineto{\pgfqpoint{3.852930in}{2.439715in}}%
\pgfpathlineto{\pgfqpoint{3.853446in}{2.430564in}}%
\pgfpathlineto{\pgfqpoint{3.854995in}{2.318778in}}%
\pgfpathlineto{\pgfqpoint{3.857319in}{2.204170in}}%
\pgfpathlineto{\pgfqpoint{3.857577in}{2.203537in}}%
\pgfpathlineto{\pgfqpoint{3.857835in}{2.204165in}}%
\pgfpathlineto{\pgfqpoint{3.858609in}{2.206694in}}%
\pgfpathlineto{\pgfqpoint{3.858867in}{2.205574in}}%
\pgfpathlineto{\pgfqpoint{3.859900in}{2.184023in}}%
\pgfpathlineto{\pgfqpoint{3.861191in}{2.152708in}}%
\pgfpathlineto{\pgfqpoint{3.861707in}{2.156205in}}%
\pgfpathlineto{\pgfqpoint{3.862997in}{2.211351in}}%
\pgfpathlineto{\pgfqpoint{3.865063in}{2.283592in}}%
\pgfpathlineto{\pgfqpoint{3.865321in}{2.280546in}}%
\pgfpathlineto{\pgfqpoint{3.866353in}{2.232593in}}%
\pgfpathlineto{\pgfqpoint{3.868676in}{2.077280in}}%
\pgfpathlineto{\pgfqpoint{3.869193in}{2.087509in}}%
\pgfpathlineto{\pgfqpoint{3.871774in}{2.236328in}}%
\pgfpathlineto{\pgfqpoint{3.872548in}{2.208882in}}%
\pgfpathlineto{\pgfqpoint{3.874355in}{2.130960in}}%
\pgfpathlineto{\pgfqpoint{3.874614in}{2.132998in}}%
\pgfpathlineto{\pgfqpoint{3.876679in}{2.196007in}}%
\pgfpathlineto{\pgfqpoint{3.877453in}{2.185403in}}%
\pgfpathlineto{\pgfqpoint{3.880293in}{2.064224in}}%
\pgfpathlineto{\pgfqpoint{3.881067in}{2.086549in}}%
\pgfpathlineto{\pgfqpoint{3.882616in}{2.264792in}}%
\pgfpathlineto{\pgfqpoint{3.884681in}{2.456932in}}%
\pgfpathlineto{\pgfqpoint{3.884939in}{2.451072in}}%
\pgfpathlineto{\pgfqpoint{3.886488in}{2.332075in}}%
\pgfpathlineto{\pgfqpoint{3.888037in}{2.260044in}}%
\pgfpathlineto{\pgfqpoint{3.888295in}{2.260119in}}%
\pgfpathlineto{\pgfqpoint{3.889844in}{2.280187in}}%
\pgfpathlineto{\pgfqpoint{3.890360in}{2.272055in}}%
\pgfpathlineto{\pgfqpoint{3.891650in}{2.183180in}}%
\pgfpathlineto{\pgfqpoint{3.893716in}{2.040519in}}%
\pgfpathlineto{\pgfqpoint{3.893974in}{2.041976in}}%
\pgfpathlineto{\pgfqpoint{3.895264in}{2.094889in}}%
\pgfpathlineto{\pgfqpoint{3.896813in}{2.138980in}}%
\pgfpathlineto{\pgfqpoint{3.897071in}{2.138392in}}%
\pgfpathlineto{\pgfqpoint{3.898620in}{2.123245in}}%
\pgfpathlineto{\pgfqpoint{3.898878in}{2.125354in}}%
\pgfpathlineto{\pgfqpoint{3.899911in}{2.165509in}}%
\pgfpathlineto{\pgfqpoint{3.903267in}{2.335627in}}%
\pgfpathlineto{\pgfqpoint{3.904299in}{2.327373in}}%
\pgfpathlineto{\pgfqpoint{3.906880in}{2.292983in}}%
\pgfpathlineto{\pgfqpoint{3.909978in}{2.207145in}}%
\pgfpathlineto{\pgfqpoint{3.910752in}{2.209030in}}%
\pgfpathlineto{\pgfqpoint{3.911011in}{2.209565in}}%
\pgfpathlineto{\pgfqpoint{3.911269in}{2.209144in}}%
\pgfpathlineto{\pgfqpoint{3.912043in}{2.198129in}}%
\pgfpathlineto{\pgfqpoint{3.913592in}{2.124935in}}%
\pgfpathlineto{\pgfqpoint{3.915399in}{2.054068in}}%
\pgfpathlineto{\pgfqpoint{3.915657in}{2.055639in}}%
\pgfpathlineto{\pgfqpoint{3.917722in}{2.100925in}}%
\pgfpathlineto{\pgfqpoint{3.918238in}{2.092528in}}%
\pgfpathlineto{\pgfqpoint{3.920045in}{2.032008in}}%
\pgfpathlineto{\pgfqpoint{3.920562in}{2.039686in}}%
\pgfpathlineto{\pgfqpoint{3.921852in}{2.130622in}}%
\pgfpathlineto{\pgfqpoint{3.924434in}{2.277969in}}%
\pgfpathlineto{\pgfqpoint{3.925982in}{2.333025in}}%
\pgfpathlineto{\pgfqpoint{3.929080in}{2.479857in}}%
\pgfpathlineto{\pgfqpoint{3.929596in}{2.474071in}}%
\pgfpathlineto{\pgfqpoint{3.931145in}{2.404881in}}%
\pgfpathlineto{\pgfqpoint{3.935792in}{2.118851in}}%
\pgfpathlineto{\pgfqpoint{3.938373in}{1.959303in}}%
\pgfpathlineto{\pgfqpoint{3.940438in}{1.935514in}}%
\pgfpathlineto{\pgfqpoint{3.940696in}{1.937077in}}%
\pgfpathlineto{\pgfqpoint{3.941729in}{1.971153in}}%
\pgfpathlineto{\pgfqpoint{3.949989in}{2.400784in}}%
\pgfpathlineto{\pgfqpoint{3.950247in}{2.401999in}}%
\pgfpathlineto{\pgfqpoint{3.950505in}{2.400920in}}%
\pgfpathlineto{\pgfqpoint{3.952054in}{2.369625in}}%
\pgfpathlineto{\pgfqpoint{3.957217in}{2.266031in}}%
\pgfpathlineto{\pgfqpoint{3.957991in}{2.275802in}}%
\pgfpathlineto{\pgfqpoint{3.959540in}{2.297772in}}%
\pgfpathlineto{\pgfqpoint{3.959798in}{2.295497in}}%
\pgfpathlineto{\pgfqpoint{3.960831in}{2.259758in}}%
\pgfpathlineto{\pgfqpoint{3.963412in}{2.044024in}}%
\pgfpathlineto{\pgfqpoint{3.965993in}{1.902836in}}%
\pgfpathlineto{\pgfqpoint{3.967026in}{1.923224in}}%
\pgfpathlineto{\pgfqpoint{3.970382in}{2.043095in}}%
\pgfpathlineto{\pgfqpoint{3.973737in}{2.360271in}}%
\pgfpathlineto{\pgfqpoint{3.977093in}{2.643930in}}%
\pgfpathlineto{\pgfqpoint{3.977609in}{2.650735in}}%
\pgfpathlineto{\pgfqpoint{3.978126in}{2.642444in}}%
\pgfpathlineto{\pgfqpoint{3.979416in}{2.546952in}}%
\pgfpathlineto{\pgfqpoint{3.985612in}{1.951870in}}%
\pgfpathlineto{\pgfqpoint{3.988967in}{1.843322in}}%
\pgfpathlineto{\pgfqpoint{3.989484in}{1.838066in}}%
\pgfpathlineto{\pgfqpoint{3.990000in}{1.841955in}}%
\pgfpathlineto{\pgfqpoint{3.991032in}{1.885042in}}%
\pgfpathlineto{\pgfqpoint{3.993097in}{2.107221in}}%
\pgfpathlineto{\pgfqpoint{3.996969in}{2.476519in}}%
\pgfpathlineto{\pgfqpoint{3.998518in}{2.525773in}}%
\pgfpathlineto{\pgfqpoint{3.998776in}{2.525169in}}%
\pgfpathlineto{\pgfqpoint{3.999809in}{2.502264in}}%
\pgfpathlineto{\pgfqpoint{4.002648in}{2.443242in}}%
\pgfpathlineto{\pgfqpoint{4.003939in}{2.398140in}}%
\pgfpathlineto{\pgfqpoint{4.005746in}{2.222308in}}%
\pgfpathlineto{\pgfqpoint{4.008844in}{1.963136in}}%
\pgfpathlineto{\pgfqpoint{4.009102in}{1.962403in}}%
\pgfpathlineto{\pgfqpoint{4.009360in}{1.963791in}}%
\pgfpathlineto{\pgfqpoint{4.010651in}{1.976107in}}%
\pgfpathlineto{\pgfqpoint{4.011167in}{1.973000in}}%
\pgfpathlineto{\pgfqpoint{4.012458in}{1.937331in}}%
\pgfpathlineto{\pgfqpoint{4.013748in}{1.902334in}}%
\pgfpathlineto{\pgfqpoint{4.014265in}{1.907338in}}%
\pgfpathlineto{\pgfqpoint{4.015297in}{1.973277in}}%
\pgfpathlineto{\pgfqpoint{4.020976in}{2.483518in}}%
\pgfpathlineto{\pgfqpoint{4.022009in}{2.496001in}}%
\pgfpathlineto{\pgfqpoint{4.022525in}{2.492864in}}%
\pgfpathlineto{\pgfqpoint{4.023557in}{2.464067in}}%
\pgfpathlineto{\pgfqpoint{4.028204in}{2.204061in}}%
\pgfpathlineto{\pgfqpoint{4.031818in}{1.919250in}}%
\pgfpathlineto{\pgfqpoint{4.032334in}{1.933291in}}%
\pgfpathlineto{\pgfqpoint{4.040078in}{2.291966in}}%
\pgfpathlineto{\pgfqpoint{4.040336in}{2.290001in}}%
\pgfpathlineto{\pgfqpoint{4.042143in}{2.264628in}}%
\pgfpathlineto{\pgfqpoint{4.042659in}{2.266954in}}%
\pgfpathlineto{\pgfqpoint{4.047822in}{2.316930in}}%
\pgfpathlineto{\pgfqpoint{4.048338in}{2.315326in}}%
\pgfpathlineto{\pgfqpoint{4.049371in}{2.296852in}}%
\pgfpathlineto{\pgfqpoint{4.051436in}{2.191956in}}%
\pgfpathlineto{\pgfqpoint{4.054792in}{2.044261in}}%
\pgfpathlineto{\pgfqpoint{4.056082in}{2.031910in}}%
\pgfpathlineto{\pgfqpoint{4.056340in}{2.032184in}}%
\pgfpathlineto{\pgfqpoint{4.057373in}{2.041090in}}%
\pgfpathlineto{\pgfqpoint{4.058922in}{2.079908in}}%
\pgfpathlineto{\pgfqpoint{4.061761in}{2.245096in}}%
\pgfpathlineto{\pgfqpoint{4.066408in}{2.413172in}}%
\pgfpathlineto{\pgfqpoint{4.066666in}{2.412881in}}%
\pgfpathlineto{\pgfqpoint{4.067440in}{2.395715in}}%
\pgfpathlineto{\pgfqpoint{4.069763in}{2.260975in}}%
\pgfpathlineto{\pgfqpoint{4.072861in}{2.110267in}}%
\pgfpathlineto{\pgfqpoint{4.074152in}{2.103298in}}%
\pgfpathlineto{\pgfqpoint{4.074410in}{2.103585in}}%
\pgfpathlineto{\pgfqpoint{4.075442in}{2.108370in}}%
\pgfpathlineto{\pgfqpoint{4.077508in}{2.133533in}}%
\pgfpathlineto{\pgfqpoint{4.079573in}{2.194832in}}%
\pgfpathlineto{\pgfqpoint{4.083445in}{2.307766in}}%
\pgfpathlineto{\pgfqpoint{4.083961in}{2.305037in}}%
\pgfpathlineto{\pgfqpoint{4.084993in}{2.276392in}}%
\pgfpathlineto{\pgfqpoint{4.088607in}{2.135285in}}%
\pgfpathlineto{\pgfqpoint{4.089124in}{2.137322in}}%
\pgfpathlineto{\pgfqpoint{4.090414in}{2.147724in}}%
\pgfpathlineto{\pgfqpoint{4.090931in}{2.145409in}}%
\pgfpathlineto{\pgfqpoint{4.092221in}{2.116745in}}%
\pgfpathlineto{\pgfqpoint{4.094286in}{2.076360in}}%
\pgfpathlineto{\pgfqpoint{4.094544in}{2.078121in}}%
\pgfpathlineto{\pgfqpoint{4.095835in}{2.112167in}}%
\pgfpathlineto{\pgfqpoint{4.098933in}{2.272965in}}%
\pgfpathlineto{\pgfqpoint{4.102805in}{2.474450in}}%
\pgfpathlineto{\pgfqpoint{4.103321in}{2.478292in}}%
\pgfpathlineto{\pgfqpoint{4.103579in}{2.475825in}}%
\pgfpathlineto{\pgfqpoint{4.104612in}{2.435835in}}%
\pgfpathlineto{\pgfqpoint{4.114679in}{1.870008in}}%
\pgfpathlineto{\pgfqpoint{4.115453in}{1.882511in}}%
\pgfpathlineto{\pgfqpoint{4.117002in}{1.972290in}}%
\pgfpathlineto{\pgfqpoint{4.125004in}{2.662482in}}%
\pgfpathlineto{\pgfqpoint{4.125779in}{2.650208in}}%
\pgfpathlineto{\pgfqpoint{4.127069in}{2.553354in}}%
\pgfpathlineto{\pgfqpoint{4.133265in}{1.873503in}}%
\pgfpathlineto{\pgfqpoint{4.133781in}{1.879332in}}%
\pgfpathlineto{\pgfqpoint{4.137911in}{1.993393in}}%
\pgfpathlineto{\pgfqpoint{4.139460in}{2.070716in}}%
\pgfpathlineto{\pgfqpoint{4.144881in}{2.432044in}}%
\pgfpathlineto{\pgfqpoint{4.145655in}{2.434698in}}%
\pgfpathlineto{\pgfqpoint{4.147462in}{2.453333in}}%
\pgfpathlineto{\pgfqpoint{4.147978in}{2.449774in}}%
\pgfpathlineto{\pgfqpoint{4.149269in}{2.406184in}}%
\pgfpathlineto{\pgfqpoint{4.152109in}{2.199715in}}%
\pgfpathlineto{\pgfqpoint{4.155206in}{2.035228in}}%
\pgfpathlineto{\pgfqpoint{4.157529in}{1.988539in}}%
\pgfpathlineto{\pgfqpoint{4.157787in}{1.989299in}}%
\pgfpathlineto{\pgfqpoint{4.158820in}{2.007246in}}%
\pgfpathlineto{\pgfqpoint{4.161401in}{2.109906in}}%
\pgfpathlineto{\pgfqpoint{4.165532in}{2.262590in}}%
\pgfpathlineto{\pgfqpoint{4.167338in}{2.304831in}}%
\pgfpathlineto{\pgfqpoint{4.170694in}{2.481210in}}%
\pgfpathlineto{\pgfqpoint{4.171469in}{2.453825in}}%
\pgfpathlineto{\pgfqpoint{4.177922in}{2.071711in}}%
\pgfpathlineto{\pgfqpoint{4.180245in}{2.056604in}}%
\pgfpathlineto{\pgfqpoint{4.181020in}{2.058045in}}%
\pgfpathlineto{\pgfqpoint{4.182052in}{2.066032in}}%
\pgfpathlineto{\pgfqpoint{4.183343in}{2.100255in}}%
\pgfpathlineto{\pgfqpoint{4.191861in}{2.414235in}}%
\pgfpathlineto{\pgfqpoint{4.192378in}{2.408575in}}%
\pgfpathlineto{\pgfqpoint{4.193668in}{2.350329in}}%
\pgfpathlineto{\pgfqpoint{4.200638in}{1.969619in}}%
\pgfpathlineto{\pgfqpoint{4.201154in}{1.961258in}}%
\pgfpathlineto{\pgfqpoint{4.201670in}{1.965358in}}%
\pgfpathlineto{\pgfqpoint{4.202961in}{2.018816in}}%
\pgfpathlineto{\pgfqpoint{4.212254in}{2.492793in}}%
\pgfpathlineto{\pgfqpoint{4.212770in}{2.495185in}}%
\pgfpathlineto{\pgfqpoint{4.213028in}{2.493031in}}%
\pgfpathlineto{\pgfqpoint{4.214061in}{2.459839in}}%
\pgfpathlineto{\pgfqpoint{4.215868in}{2.304112in}}%
\pgfpathlineto{\pgfqpoint{4.219998in}{1.946903in}}%
\pgfpathlineto{\pgfqpoint{4.221805in}{1.921398in}}%
\pgfpathlineto{\pgfqpoint{4.222063in}{1.921468in}}%
\pgfpathlineto{\pgfqpoint{4.222837in}{1.929111in}}%
\pgfpathlineto{\pgfqpoint{4.224128in}{1.973994in}}%
\pgfpathlineto{\pgfqpoint{4.226968in}{2.175735in}}%
\pgfpathlineto{\pgfqpoint{4.231356in}{2.526714in}}%
\pgfpathlineto{\pgfqpoint{4.231614in}{2.524791in}}%
\pgfpathlineto{\pgfqpoint{4.232647in}{2.489320in}}%
\pgfpathlineto{\pgfqpoint{4.237809in}{2.144635in}}%
\pgfpathlineto{\pgfqpoint{4.240907in}{1.977877in}}%
\pgfpathlineto{\pgfqpoint{4.241165in}{1.975957in}}%
\pgfpathlineto{\pgfqpoint{4.241681in}{1.978918in}}%
\pgfpathlineto{\pgfqpoint{4.242972in}{2.018819in}}%
\pgfpathlineto{\pgfqpoint{4.250974in}{2.366264in}}%
\pgfpathlineto{\pgfqpoint{4.251749in}{2.358098in}}%
\pgfpathlineto{\pgfqpoint{4.258718in}{2.198730in}}%
\pgfpathlineto{\pgfqpoint{4.262590in}{2.072739in}}%
\pgfpathlineto{\pgfqpoint{4.262848in}{2.072086in}}%
\pgfpathlineto{\pgfqpoint{4.263365in}{2.072904in}}%
\pgfpathlineto{\pgfqpoint{4.264913in}{2.083706in}}%
\pgfpathlineto{\pgfqpoint{4.266462in}{2.119262in}}%
\pgfpathlineto{\pgfqpoint{4.274464in}{2.361369in}}%
\pgfpathlineto{\pgfqpoint{4.275239in}{2.365341in}}%
\pgfpathlineto{\pgfqpoint{4.275755in}{2.364258in}}%
\pgfpathlineto{\pgfqpoint{4.276788in}{2.349114in}}%
\pgfpathlineto{\pgfqpoint{4.279369in}{2.255555in}}%
\pgfpathlineto{\pgfqpoint{4.282725in}{2.055100in}}%
\pgfpathlineto{\pgfqpoint{4.285306in}{1.971152in}}%
\pgfpathlineto{\pgfqpoint{4.285822in}{1.975265in}}%
\pgfpathlineto{\pgfqpoint{4.287113in}{2.019694in}}%
\pgfpathlineto{\pgfqpoint{4.295890in}{2.425733in}}%
\pgfpathlineto{\pgfqpoint{4.296922in}{2.429794in}}%
\pgfpathlineto{\pgfqpoint{4.297180in}{2.428769in}}%
\pgfpathlineto{\pgfqpoint{4.298213in}{2.410202in}}%
\pgfpathlineto{\pgfqpoint{4.299762in}{2.326290in}}%
\pgfpathlineto{\pgfqpoint{4.304924in}{1.988641in}}%
\pgfpathlineto{\pgfqpoint{4.305182in}{1.987886in}}%
\pgfpathlineto{\pgfqpoint{4.305441in}{1.988589in}}%
\pgfpathlineto{\pgfqpoint{4.306473in}{2.002740in}}%
\pgfpathlineto{\pgfqpoint{4.309054in}{2.078176in}}%
\pgfpathlineto{\pgfqpoint{4.315508in}{2.299690in}}%
\pgfpathlineto{\pgfqpoint{4.318605in}{2.374253in}}%
\pgfpathlineto{\pgfqpoint{4.318864in}{2.373164in}}%
\pgfpathlineto{\pgfqpoint{4.320154in}{2.350696in}}%
\pgfpathlineto{\pgfqpoint{4.324284in}{2.224585in}}%
\pgfpathlineto{\pgfqpoint{4.328931in}{2.032913in}}%
\pgfpathlineto{\pgfqpoint{4.329447in}{2.034749in}}%
\pgfpathlineto{\pgfqpoint{4.330996in}{2.057070in}}%
\pgfpathlineto{\pgfqpoint{4.333061in}{2.122031in}}%
\pgfpathlineto{\pgfqpoint{4.337191in}{2.256298in}}%
\pgfpathlineto{\pgfqpoint{4.341579in}{2.339000in}}%
\pgfpathlineto{\pgfqpoint{4.341838in}{2.337880in}}%
\pgfpathlineto{\pgfqpoint{4.342870in}{2.320499in}}%
\pgfpathlineto{\pgfqpoint{4.349065in}{2.171927in}}%
\pgfpathlineto{\pgfqpoint{4.349840in}{2.172807in}}%
\pgfpathlineto{\pgfqpoint{4.350356in}{2.173509in}}%
\pgfpathlineto{\pgfqpoint{4.350872in}{2.172472in}}%
\pgfpathlineto{\pgfqpoint{4.351905in}{2.160586in}}%
\pgfpathlineto{\pgfqpoint{4.354228in}{2.121085in}}%
\pgfpathlineto{\pgfqpoint{4.354744in}{2.123762in}}%
\pgfpathlineto{\pgfqpoint{4.356551in}{2.161912in}}%
\pgfpathlineto{\pgfqpoint{4.358616in}{2.181445in}}%
\pgfpathlineto{\pgfqpoint{4.359649in}{2.182496in}}%
\pgfpathlineto{\pgfqpoint{4.360681in}{2.190665in}}%
\pgfpathlineto{\pgfqpoint{4.362747in}{2.235663in}}%
\pgfpathlineto{\pgfqpoint{4.364553in}{2.262106in}}%
\pgfpathlineto{\pgfqpoint{4.364812in}{2.260914in}}%
\pgfpathlineto{\pgfqpoint{4.366877in}{2.237843in}}%
\pgfpathlineto{\pgfqpoint{4.367651in}{2.244982in}}%
\pgfpathlineto{\pgfqpoint{4.370491in}{2.306346in}}%
\pgfpathlineto{\pgfqpoint{4.371007in}{2.301173in}}%
\pgfpathlineto{\pgfqpoint{4.372814in}{2.247207in}}%
\pgfpathlineto{\pgfqpoint{4.377976in}{2.107098in}}%
\pgfpathlineto{\pgfqpoint{4.379009in}{2.100434in}}%
\pgfpathlineto{\pgfqpoint{4.379267in}{2.101368in}}%
\pgfpathlineto{\pgfqpoint{4.383655in}{2.136312in}}%
\pgfpathlineto{\pgfqpoint{4.385462in}{2.200665in}}%
\pgfpathlineto{\pgfqpoint{4.389076in}{2.325390in}}%
\pgfpathlineto{\pgfqpoint{4.389334in}{2.325542in}}%
\pgfpathlineto{\pgfqpoint{4.389593in}{2.324465in}}%
\pgfpathlineto{\pgfqpoint{4.390883in}{2.304667in}}%
\pgfpathlineto{\pgfqpoint{4.395530in}{2.197683in}}%
\pgfpathlineto{\pgfqpoint{4.396304in}{2.201471in}}%
\pgfpathlineto{\pgfqpoint{4.398369in}{2.217303in}}%
\pgfpathlineto{\pgfqpoint{4.398627in}{2.216642in}}%
\pgfpathlineto{\pgfqpoint{4.399918in}{2.203296in}}%
\pgfpathlineto{\pgfqpoint{4.403016in}{2.171362in}}%
\pgfpathlineto{\pgfqpoint{4.405081in}{2.167280in}}%
\pgfpathlineto{\pgfqpoint{4.406371in}{2.165664in}}%
\pgfpathlineto{\pgfqpoint{4.407662in}{2.158679in}}%
\pgfpathlineto{\pgfqpoint{4.409727in}{2.143738in}}%
\pgfpathlineto{\pgfqpoint{4.410243in}{2.145600in}}%
\pgfpathlineto{\pgfqpoint{4.411534in}{2.164365in}}%
\pgfpathlineto{\pgfqpoint{4.417729in}{2.302139in}}%
\pgfpathlineto{\pgfqpoint{4.418762in}{2.294307in}}%
\pgfpathlineto{\pgfqpoint{4.425731in}{2.182048in}}%
\pgfpathlineto{\pgfqpoint{4.426506in}{2.184048in}}%
\pgfpathlineto{\pgfqpoint{4.427280in}{2.185551in}}%
\pgfpathlineto{\pgfqpoint{4.427538in}{2.184883in}}%
\pgfpathlineto{\pgfqpoint{4.428571in}{2.172919in}}%
\pgfpathlineto{\pgfqpoint{4.431152in}{2.128784in}}%
\pgfpathlineto{\pgfqpoint{4.431410in}{2.130376in}}%
\pgfpathlineto{\pgfqpoint{4.432701in}{2.156728in}}%
\pgfpathlineto{\pgfqpoint{4.436315in}{2.228632in}}%
\pgfpathlineto{\pgfqpoint{4.438380in}{2.238008in}}%
\pgfpathlineto{\pgfqpoint{4.439413in}{2.236294in}}%
\pgfpathlineto{\pgfqpoint{4.440961in}{2.228150in}}%
\pgfpathlineto{\pgfqpoint{4.444059in}{2.210087in}}%
\pgfpathlineto{\pgfqpoint{4.446382in}{2.206639in}}%
\pgfpathlineto{\pgfqpoint{4.446898in}{2.207277in}}%
\pgfpathlineto{\pgfqpoint{4.448705in}{2.214993in}}%
\pgfpathlineto{\pgfqpoint{4.453094in}{2.234329in}}%
\pgfpathlineto{\pgfqpoint{4.453610in}{2.232851in}}%
\pgfpathlineto{\pgfqpoint{4.454901in}{2.215750in}}%
\pgfpathlineto{\pgfqpoint{4.458256in}{2.171085in}}%
\pgfpathlineto{\pgfqpoint{4.459289in}{2.168023in}}%
\pgfpathlineto{\pgfqpoint{4.459805in}{2.168432in}}%
\pgfpathlineto{\pgfqpoint{4.460838in}{2.169569in}}%
\pgfpathlineto{\pgfqpoint{4.461096in}{2.169209in}}%
\pgfpathlineto{\pgfqpoint{4.462387in}{2.162081in}}%
\pgfpathlineto{\pgfqpoint{4.464710in}{2.151891in}}%
\pgfpathlineto{\pgfqpoint{4.465484in}{2.151595in}}%
\pgfpathlineto{\pgfqpoint{4.465742in}{2.152011in}}%
\pgfpathlineto{\pgfqpoint{4.466517in}{2.156596in}}%
\pgfpathlineto{\pgfqpoint{4.467807in}{2.185048in}}%
\pgfpathlineto{\pgfqpoint{4.472196in}{2.342445in}}%
\pgfpathlineto{\pgfqpoint{4.472712in}{2.339278in}}%
\pgfpathlineto{\pgfqpoint{4.474261in}{2.300500in}}%
\pgfpathlineto{\pgfqpoint{4.477100in}{2.241249in}}%
\pgfpathlineto{\pgfqpoint{4.478391in}{2.221898in}}%
\pgfpathlineto{\pgfqpoint{4.480456in}{2.133494in}}%
\pgfpathlineto{\pgfqpoint{4.482521in}{2.079184in}}%
\pgfpathlineto{\pgfqpoint{4.483295in}{2.086241in}}%
\pgfpathlineto{\pgfqpoint{4.487168in}{2.133404in}}%
\pgfpathlineto{\pgfqpoint{4.488200in}{2.145561in}}%
\pgfpathlineto{\pgfqpoint{4.489749in}{2.204320in}}%
\pgfpathlineto{\pgfqpoint{4.492846in}{2.309753in}}%
\pgfpathlineto{\pgfqpoint{4.493363in}{2.311741in}}%
\pgfpathlineto{\pgfqpoint{4.493879in}{2.310119in}}%
\pgfpathlineto{\pgfqpoint{4.495428in}{2.290514in}}%
\pgfpathlineto{\pgfqpoint{4.502139in}{2.176253in}}%
\pgfpathlineto{\pgfqpoint{4.504721in}{2.136271in}}%
\pgfpathlineto{\pgfqpoint{4.505753in}{2.135609in}}%
\pgfpathlineto{\pgfqpoint{4.506011in}{2.135756in}}%
\pgfpathlineto{\pgfqpoint{4.507044in}{2.137026in}}%
\pgfpathlineto{\pgfqpoint{4.508076in}{2.144097in}}%
\pgfpathlineto{\pgfqpoint{4.509883in}{2.183011in}}%
\pgfpathlineto{\pgfqpoint{4.511948in}{2.222939in}}%
\pgfpathlineto{\pgfqpoint{4.512207in}{2.222728in}}%
\pgfpathlineto{\pgfqpoint{4.513239in}{2.210856in}}%
\pgfpathlineto{\pgfqpoint{4.515046in}{2.189151in}}%
\pgfpathlineto{\pgfqpoint{4.515304in}{2.190032in}}%
\pgfpathlineto{\pgfqpoint{4.516337in}{2.206515in}}%
\pgfpathlineto{\pgfqpoint{4.519434in}{2.270921in}}%
\pgfpathlineto{\pgfqpoint{4.519693in}{2.270799in}}%
\pgfpathlineto{\pgfqpoint{4.521758in}{2.261811in}}%
\pgfpathlineto{\pgfqpoint{4.522532in}{2.264379in}}%
\pgfpathlineto{\pgfqpoint{4.524081in}{2.271850in}}%
\pgfpathlineto{\pgfqpoint{4.524339in}{2.271130in}}%
\pgfpathlineto{\pgfqpoint{4.525371in}{2.258142in}}%
\pgfpathlineto{\pgfqpoint{4.533116in}{2.114671in}}%
\pgfpathlineto{\pgfqpoint{4.534922in}{2.102630in}}%
\pgfpathlineto{\pgfqpoint{4.535181in}{2.103584in}}%
\pgfpathlineto{\pgfqpoint{4.536213in}{2.117265in}}%
\pgfpathlineto{\pgfqpoint{4.539053in}{2.204567in}}%
\pgfpathlineto{\pgfqpoint{4.542925in}{2.286212in}}%
\pgfpathlineto{\pgfqpoint{4.545248in}{2.306941in}}%
\pgfpathlineto{\pgfqpoint{4.545764in}{2.306996in}}%
\pgfpathlineto{\pgfqpoint{4.546022in}{2.306215in}}%
\pgfpathlineto{\pgfqpoint{4.547313in}{2.293755in}}%
\pgfpathlineto{\pgfqpoint{4.552218in}{2.201831in}}%
\pgfpathlineto{\pgfqpoint{4.556606in}{2.072934in}}%
\pgfpathlineto{\pgfqpoint{4.556864in}{2.073741in}}%
\pgfpathlineto{\pgfqpoint{4.557896in}{2.090928in}}%
\pgfpathlineto{\pgfqpoint{4.563059in}{2.263087in}}%
\pgfpathlineto{\pgfqpoint{4.565899in}{2.321571in}}%
\pgfpathlineto{\pgfqpoint{4.566415in}{2.322702in}}%
\pgfpathlineto{\pgfqpoint{4.566673in}{2.321725in}}%
\pgfpathlineto{\pgfqpoint{4.567706in}{2.305664in}}%
\pgfpathlineto{\pgfqpoint{4.570803in}{2.192869in}}%
\pgfpathlineto{\pgfqpoint{4.573901in}{2.117333in}}%
\pgfpathlineto{\pgfqpoint{4.574159in}{2.116871in}}%
\pgfpathlineto{\pgfqpoint{4.574675in}{2.117731in}}%
\pgfpathlineto{\pgfqpoint{4.575966in}{2.120883in}}%
\pgfpathlineto{\pgfqpoint{4.576482in}{2.120207in}}%
\pgfpathlineto{\pgfqpoint{4.577515in}{2.118607in}}%
\pgfpathlineto{\pgfqpoint{4.577773in}{2.119287in}}%
\pgfpathlineto{\pgfqpoint{4.578547in}{2.127164in}}%
\pgfpathlineto{\pgfqpoint{4.580096in}{2.177430in}}%
\pgfpathlineto{\pgfqpoint{4.583452in}{2.293365in}}%
\pgfpathlineto{\pgfqpoint{4.587324in}{2.307087in}}%
\pgfpathlineto{\pgfqpoint{4.587840in}{2.305757in}}%
\pgfpathlineto{\pgfqpoint{4.588873in}{2.292177in}}%
\pgfpathlineto{\pgfqpoint{4.591196in}{2.219128in}}%
\pgfpathlineto{\pgfqpoint{4.595326in}{2.102118in}}%
\pgfpathlineto{\pgfqpoint{4.595842in}{2.099923in}}%
\pgfpathlineto{\pgfqpoint{4.596359in}{2.100725in}}%
\pgfpathlineto{\pgfqpoint{4.597907in}{2.113512in}}%
\pgfpathlineto{\pgfqpoint{4.601005in}{2.157289in}}%
\pgfpathlineto{\pgfqpoint{4.605393in}{2.286210in}}%
\pgfpathlineto{\pgfqpoint{4.606426in}{2.281927in}}%
\pgfpathlineto{\pgfqpoint{4.607458in}{2.278152in}}%
\pgfpathlineto{\pgfqpoint{4.607975in}{2.278543in}}%
\pgfpathlineto{\pgfqpoint{4.609007in}{2.279593in}}%
\pgfpathlineto{\pgfqpoint{4.609265in}{2.278988in}}%
\pgfpathlineto{\pgfqpoint{4.610298in}{2.271063in}}%
\pgfpathlineto{\pgfqpoint{4.613654in}{2.218265in}}%
\pgfpathlineto{\pgfqpoint{4.618300in}{2.129391in}}%
\pgfpathlineto{\pgfqpoint{4.618816in}{2.131571in}}%
\pgfpathlineto{\pgfqpoint{4.621914in}{2.167962in}}%
\pgfpathlineto{\pgfqpoint{4.622946in}{2.164175in}}%
\pgfpathlineto{\pgfqpoint{4.623979in}{2.160517in}}%
\pgfpathlineto{\pgfqpoint{4.624237in}{2.161397in}}%
\pgfpathlineto{\pgfqpoint{4.625270in}{2.177170in}}%
\pgfpathlineto{\pgfqpoint{4.629400in}{2.280989in}}%
\pgfpathlineto{\pgfqpoint{4.629916in}{2.279580in}}%
\pgfpathlineto{\pgfqpoint{4.632497in}{2.252921in}}%
\pgfpathlineto{\pgfqpoint{4.633530in}{2.256487in}}%
\pgfpathlineto{\pgfqpoint{4.634304in}{2.259320in}}%
\pgfpathlineto{\pgfqpoint{4.634563in}{2.258738in}}%
\pgfpathlineto{\pgfqpoint{4.635337in}{2.249813in}}%
\pgfpathlineto{\pgfqpoint{4.641016in}{2.148928in}}%
\pgfpathlineto{\pgfqpoint{4.642565in}{2.141640in}}%
\pgfpathlineto{\pgfqpoint{4.643597in}{2.137661in}}%
\pgfpathlineto{\pgfqpoint{4.643855in}{2.138226in}}%
\pgfpathlineto{\pgfqpoint{4.644630in}{2.146562in}}%
\pgfpathlineto{\pgfqpoint{4.647211in}{2.221629in}}%
\pgfpathlineto{\pgfqpoint{4.649792in}{2.254593in}}%
\pgfpathlineto{\pgfqpoint{4.651858in}{2.267787in}}%
\pgfpathlineto{\pgfqpoint{4.652116in}{2.267613in}}%
\pgfpathlineto{\pgfqpoint{4.653148in}{2.262075in}}%
\pgfpathlineto{\pgfqpoint{4.654955in}{2.232287in}}%
\pgfpathlineto{\pgfqpoint{4.657795in}{2.191799in}}%
\pgfpathlineto{\pgfqpoint{4.659860in}{2.180674in}}%
\pgfpathlineto{\pgfqpoint{4.662441in}{2.167606in}}%
\pgfpathlineto{\pgfqpoint{4.663732in}{2.166751in}}%
\pgfpathlineto{\pgfqpoint{4.663990in}{2.167024in}}%
\pgfpathlineto{\pgfqpoint{4.665022in}{2.170650in}}%
\pgfpathlineto{\pgfqpoint{4.667088in}{2.187770in}}%
\pgfpathlineto{\pgfqpoint{4.670960in}{2.240679in}}%
\pgfpathlineto{\pgfqpoint{4.671992in}{2.237904in}}%
\pgfpathlineto{\pgfqpoint{4.673283in}{2.234877in}}%
\pgfpathlineto{\pgfqpoint{4.673541in}{2.235138in}}%
\pgfpathlineto{\pgfqpoint{4.674832in}{2.240304in}}%
\pgfpathlineto{\pgfqpoint{4.676638in}{2.247396in}}%
\pgfpathlineto{\pgfqpoint{4.676897in}{2.246861in}}%
\pgfpathlineto{\pgfqpoint{4.677929in}{2.238691in}}%
\pgfpathlineto{\pgfqpoint{4.683092in}{2.173876in}}%
\pgfpathlineto{\pgfqpoint{4.683866in}{2.176534in}}%
\pgfpathlineto{\pgfqpoint{4.685931in}{2.188475in}}%
\pgfpathlineto{\pgfqpoint{4.686189in}{2.187881in}}%
\pgfpathlineto{\pgfqpoint{4.687480in}{2.177474in}}%
\pgfpathlineto{\pgfqpoint{4.689803in}{2.159269in}}%
\pgfpathlineto{\pgfqpoint{4.690062in}{2.159746in}}%
\pgfpathlineto{\pgfqpoint{4.691094in}{2.167859in}}%
\pgfpathlineto{\pgfqpoint{4.698064in}{2.256316in}}%
\pgfpathlineto{\pgfqpoint{4.698838in}{2.254999in}}%
\pgfpathlineto{\pgfqpoint{4.703226in}{2.235103in}}%
\pgfpathlineto{\pgfqpoint{4.705291in}{2.213527in}}%
\pgfpathlineto{\pgfqpoint{4.711487in}{2.121792in}}%
\pgfpathlineto{\pgfqpoint{4.712519in}{2.129668in}}%
\pgfpathlineto{\pgfqpoint{4.714584in}{2.187495in}}%
\pgfpathlineto{\pgfqpoint{4.717682in}{2.250873in}}%
\pgfpathlineto{\pgfqpoint{4.720780in}{2.268453in}}%
\pgfpathlineto{\pgfqpoint{4.722070in}{2.269870in}}%
\pgfpathlineto{\pgfqpoint{4.722328in}{2.269666in}}%
\pgfpathlineto{\pgfqpoint{4.723103in}{2.266924in}}%
\pgfpathlineto{\pgfqpoint{4.724393in}{2.248951in}}%
\pgfpathlineto{\pgfqpoint{4.730589in}{2.123989in}}%
\pgfpathlineto{\pgfqpoint{4.731105in}{2.123132in}}%
\pgfpathlineto{\pgfqpoint{4.731363in}{2.123684in}}%
\pgfpathlineto{\pgfqpoint{4.732396in}{2.132990in}}%
\pgfpathlineto{\pgfqpoint{4.737558in}{2.230187in}}%
\pgfpathlineto{\pgfqpoint{4.741947in}{2.308255in}}%
\pgfpathlineto{\pgfqpoint{4.742463in}{2.308676in}}%
\pgfpathlineto{\pgfqpoint{4.742721in}{2.307824in}}%
\pgfpathlineto{\pgfqpoint{4.743754in}{2.294343in}}%
\pgfpathlineto{\pgfqpoint{4.745819in}{2.219098in}}%
\pgfpathlineto{\pgfqpoint{4.748658in}{2.142208in}}%
\pgfpathlineto{\pgfqpoint{4.752272in}{2.112129in}}%
\pgfpathlineto{\pgfqpoint{4.752530in}{2.112903in}}%
\pgfpathlineto{\pgfqpoint{4.753563in}{2.128892in}}%
\pgfpathlineto{\pgfqpoint{4.758725in}{2.242135in}}%
\pgfpathlineto{\pgfqpoint{4.762339in}{2.265806in}}%
\pgfpathlineto{\pgfqpoint{4.763114in}{2.263735in}}%
\pgfpathlineto{\pgfqpoint{4.764404in}{2.248202in}}%
\pgfpathlineto{\pgfqpoint{4.768018in}{2.184724in}}%
\pgfpathlineto{\pgfqpoint{4.768534in}{2.187526in}}%
\pgfpathlineto{\pgfqpoint{4.770600in}{2.204373in}}%
\pgfpathlineto{\pgfqpoint{4.771116in}{2.202212in}}%
\pgfpathlineto{\pgfqpoint{4.773181in}{2.174677in}}%
\pgfpathlineto{\pgfqpoint{4.774213in}{2.167880in}}%
\pgfpathlineto{\pgfqpoint{4.774730in}{2.169728in}}%
\pgfpathlineto{\pgfqpoint{4.776279in}{2.192051in}}%
\pgfpathlineto{\pgfqpoint{4.778344in}{2.209278in}}%
\pgfpathlineto{\pgfqpoint{4.779376in}{2.209170in}}%
\pgfpathlineto{\pgfqpoint{4.779634in}{2.209707in}}%
\pgfpathlineto{\pgfqpoint{4.780925in}{2.217069in}}%
\pgfpathlineto{\pgfqpoint{4.783506in}{2.230452in}}%
\pgfpathlineto{\pgfqpoint{4.784281in}{2.228427in}}%
\pgfpathlineto{\pgfqpoint{4.786088in}{2.211703in}}%
\pgfpathlineto{\pgfqpoint{4.789443in}{2.188466in}}%
\pgfpathlineto{\pgfqpoint{4.790218in}{2.187415in}}%
\pgfpathlineto{\pgfqpoint{4.790734in}{2.188096in}}%
\pgfpathlineto{\pgfqpoint{4.792283in}{2.194607in}}%
\pgfpathlineto{\pgfqpoint{4.800285in}{2.237682in}}%
\pgfpathlineto{\pgfqpoint{4.800801in}{2.238240in}}%
\pgfpathlineto{\pgfqpoint{4.801059in}{2.237524in}}%
\pgfpathlineto{\pgfqpoint{4.802092in}{2.226397in}}%
\pgfpathlineto{\pgfqpoint{4.805448in}{2.184985in}}%
\pgfpathlineto{\pgfqpoint{4.809578in}{2.154458in}}%
\pgfpathlineto{\pgfqpoint{4.810352in}{2.155941in}}%
\pgfpathlineto{\pgfqpoint{4.811901in}{2.166526in}}%
\pgfpathlineto{\pgfqpoint{4.814224in}{2.205499in}}%
\pgfpathlineto{\pgfqpoint{4.818613in}{2.281927in}}%
\pgfpathlineto{\pgfqpoint{4.819129in}{2.282405in}}%
\pgfpathlineto{\pgfqpoint{4.819387in}{2.282041in}}%
\pgfpathlineto{\pgfqpoint{4.821452in}{2.274167in}}%
\pgfpathlineto{\pgfqpoint{4.823001in}{2.256689in}}%
\pgfpathlineto{\pgfqpoint{4.825582in}{2.188885in}}%
\pgfpathlineto{\pgfqpoint{4.828680in}{2.122218in}}%
\pgfpathlineto{\pgfqpoint{4.829196in}{2.123526in}}%
\pgfpathlineto{\pgfqpoint{4.834101in}{2.158937in}}%
\pgfpathlineto{\pgfqpoint{4.840038in}{2.270106in}}%
\pgfpathlineto{\pgfqpoint{4.841587in}{2.269530in}}%
\pgfpathlineto{\pgfqpoint{4.842619in}{2.266319in}}%
\pgfpathlineto{\pgfqpoint{4.844168in}{2.251097in}}%
\pgfpathlineto{\pgfqpoint{4.848556in}{2.201331in}}%
\pgfpathlineto{\pgfqpoint{4.852170in}{2.163421in}}%
\pgfpathlineto{\pgfqpoint{4.854493in}{2.144392in}}%
\pgfpathlineto{\pgfqpoint{4.855268in}{2.147140in}}%
\pgfpathlineto{\pgfqpoint{4.857075in}{2.167145in}}%
\pgfpathlineto{\pgfqpoint{4.864044in}{2.253783in}}%
\pgfpathlineto{\pgfqpoint{4.865593in}{2.263087in}}%
\pgfpathlineto{\pgfqpoint{4.865851in}{2.262945in}}%
\pgfpathlineto{\pgfqpoint{4.867142in}{2.257389in}}%
\pgfpathlineto{\pgfqpoint{4.869465in}{2.237343in}}%
\pgfpathlineto{\pgfqpoint{4.872563in}{2.175043in}}%
\pgfpathlineto{\pgfqpoint{4.875144in}{2.141905in}}%
\pgfpathlineto{\pgfqpoint{4.875660in}{2.141561in}}%
\pgfpathlineto{\pgfqpoint{4.875919in}{2.142258in}}%
\pgfpathlineto{\pgfqpoint{4.877209in}{2.153126in}}%
\pgfpathlineto{\pgfqpoint{4.885986in}{2.260871in}}%
\pgfpathlineto{\pgfqpoint{4.886760in}{2.258949in}}%
\pgfpathlineto{\pgfqpoint{4.891407in}{2.232806in}}%
\pgfpathlineto{\pgfqpoint{4.897344in}{2.147294in}}%
\pgfpathlineto{\pgfqpoint{4.897860in}{2.147283in}}%
\pgfpathlineto{\pgfqpoint{4.898118in}{2.147942in}}%
\pgfpathlineto{\pgfqpoint{4.899151in}{2.156227in}}%
\pgfpathlineto{\pgfqpoint{4.905604in}{2.237923in}}%
\pgfpathlineto{\pgfqpoint{4.906120in}{2.237305in}}%
\pgfpathlineto{\pgfqpoint{4.908185in}{2.228877in}}%
\pgfpathlineto{\pgfqpoint{4.911283in}{2.219183in}}%
\pgfpathlineto{\pgfqpoint{4.912316in}{2.216295in}}%
\pgfpathlineto{\pgfqpoint{4.913864in}{2.199665in}}%
\pgfpathlineto{\pgfqpoint{4.915929in}{2.180367in}}%
\pgfpathlineto{\pgfqpoint{4.916188in}{2.180666in}}%
\pgfpathlineto{\pgfqpoint{4.917220in}{2.187941in}}%
\pgfpathlineto{\pgfqpoint{4.920318in}{2.213088in}}%
\pgfpathlineto{\pgfqpoint{4.921350in}{2.212038in}}%
\pgfpathlineto{\pgfqpoint{4.929094in}{2.197517in}}%
\pgfpathlineto{\pgfqpoint{4.929869in}{2.198711in}}%
\pgfpathlineto{\pgfqpoint{4.932966in}{2.205372in}}%
\pgfpathlineto{\pgfqpoint{4.933483in}{2.204770in}}%
\pgfpathlineto{\pgfqpoint{4.935290in}{2.200122in}}%
\pgfpathlineto{\pgfqpoint{4.935806in}{2.201488in}}%
\pgfpathlineto{\pgfqpoint{4.937355in}{2.217611in}}%
\pgfpathlineto{\pgfqpoint{4.938903in}{2.228973in}}%
\pgfpathlineto{\pgfqpoint{4.939162in}{2.228783in}}%
\pgfpathlineto{\pgfqpoint{4.940710in}{2.221756in}}%
\pgfpathlineto{\pgfqpoint{4.943292in}{2.215466in}}%
\pgfpathlineto{\pgfqpoint{4.945099in}{2.209259in}}%
\pgfpathlineto{\pgfqpoint{4.948454in}{2.188269in}}%
\pgfpathlineto{\pgfqpoint{4.949229in}{2.189278in}}%
\pgfpathlineto{\pgfqpoint{4.952068in}{2.199055in}}%
\pgfpathlineto{\pgfqpoint{4.955166in}{2.214743in}}%
\pgfpathlineto{\pgfqpoint{4.955424in}{2.214697in}}%
\pgfpathlineto{\pgfqpoint{4.956715in}{2.212235in}}%
\pgfpathlineto{\pgfqpoint{4.959038in}{2.205552in}}%
\pgfpathlineto{\pgfqpoint{4.959554in}{2.206359in}}%
\pgfpathlineto{\pgfqpoint{4.960845in}{2.215609in}}%
\pgfpathlineto{\pgfqpoint{4.962394in}{2.226571in}}%
\pgfpathlineto{\pgfqpoint{4.962910in}{2.225652in}}%
\pgfpathlineto{\pgfqpoint{4.964201in}{2.210834in}}%
\pgfpathlineto{\pgfqpoint{4.967040in}{2.176742in}}%
\pgfpathlineto{\pgfqpoint{4.967815in}{2.181408in}}%
\pgfpathlineto{\pgfqpoint{4.970138in}{2.202312in}}%
\pgfpathlineto{\pgfqpoint{4.970654in}{2.201240in}}%
\pgfpathlineto{\pgfqpoint{4.972461in}{2.191284in}}%
\pgfpathlineto{\pgfqpoint{4.972977in}{2.192851in}}%
\pgfpathlineto{\pgfqpoint{4.974268in}{2.210562in}}%
\pgfpathlineto{\pgfqpoint{4.976849in}{2.242887in}}%
\pgfpathlineto{\pgfqpoint{4.977624in}{2.239106in}}%
\pgfpathlineto{\pgfqpoint{4.980721in}{2.210626in}}%
\pgfpathlineto{\pgfqpoint{4.981496in}{2.211540in}}%
\pgfpathlineto{\pgfqpoint{4.982270in}{2.212523in}}%
\pgfpathlineto{\pgfqpoint{4.982528in}{2.212171in}}%
\pgfpathlineto{\pgfqpoint{4.983561in}{2.205230in}}%
\pgfpathlineto{\pgfqpoint{4.986917in}{2.171536in}}%
\pgfpathlineto{\pgfqpoint{4.987175in}{2.171845in}}%
\pgfpathlineto{\pgfqpoint{4.988465in}{2.178641in}}%
\pgfpathlineto{\pgfqpoint{4.991821in}{2.210492in}}%
\pgfpathlineto{\pgfqpoint{4.994919in}{2.266450in}}%
\pgfpathlineto{\pgfqpoint{4.995693in}{2.260961in}}%
\pgfpathlineto{\pgfqpoint{5.000598in}{2.195921in}}%
\pgfpathlineto{\pgfqpoint{5.001372in}{2.193995in}}%
\pgfpathlineto{\pgfqpoint{5.002921in}{2.179273in}}%
\pgfpathlineto{\pgfqpoint{5.005244in}{2.159344in}}%
\pgfpathlineto{\pgfqpoint{5.005502in}{2.159770in}}%
\pgfpathlineto{\pgfqpoint{5.006535in}{2.168735in}}%
\pgfpathlineto{\pgfqpoint{5.011439in}{2.228344in}}%
\pgfpathlineto{\pgfqpoint{5.013504in}{2.232537in}}%
\pgfpathlineto{\pgfqpoint{5.015311in}{2.243779in}}%
\pgfpathlineto{\pgfqpoint{5.016086in}{2.247506in}}%
\pgfpathlineto{\pgfqpoint{5.016602in}{2.246235in}}%
\pgfpathlineto{\pgfqpoint{5.017893in}{2.226827in}}%
\pgfpathlineto{\pgfqpoint{5.020732in}{2.186206in}}%
\pgfpathlineto{\pgfqpoint{5.023055in}{2.182699in}}%
\pgfpathlineto{\pgfqpoint{5.026411in}{2.162440in}}%
\pgfpathlineto{\pgfqpoint{5.026927in}{2.164322in}}%
\pgfpathlineto{\pgfqpoint{5.028218in}{2.181232in}}%
\pgfpathlineto{\pgfqpoint{5.032090in}{2.238068in}}%
\pgfpathlineto{\pgfqpoint{5.034413in}{2.242647in}}%
\pgfpathlineto{\pgfqpoint{5.036737in}{2.252104in}}%
\pgfpathlineto{\pgfqpoint{5.037511in}{2.249077in}}%
\pgfpathlineto{\pgfqpoint{5.039060in}{2.222052in}}%
\pgfpathlineto{\pgfqpoint{5.042157in}{2.172297in}}%
\pgfpathlineto{\pgfqpoint{5.044481in}{2.161695in}}%
\pgfpathlineto{\pgfqpoint{5.045255in}{2.161533in}}%
\pgfpathlineto{\pgfqpoint{5.045513in}{2.161982in}}%
\pgfpathlineto{\pgfqpoint{5.046546in}{2.167712in}}%
\pgfpathlineto{\pgfqpoint{5.048611in}{2.199816in}}%
\pgfpathlineto{\pgfqpoint{5.051192in}{2.227633in}}%
\pgfpathlineto{\pgfqpoint{5.054032in}{2.234419in}}%
\pgfpathlineto{\pgfqpoint{5.055064in}{2.234966in}}%
\pgfpathlineto{\pgfqpoint{5.055322in}{2.234678in}}%
\pgfpathlineto{\pgfqpoint{5.056355in}{2.231001in}}%
\pgfpathlineto{\pgfqpoint{5.064099in}{2.177648in}}%
\pgfpathlineto{\pgfqpoint{5.065390in}{2.169326in}}%
\pgfpathlineto{\pgfqpoint{5.065648in}{2.169593in}}%
\pgfpathlineto{\pgfqpoint{5.066680in}{2.175834in}}%
\pgfpathlineto{\pgfqpoint{5.074682in}{2.246245in}}%
\pgfpathlineto{\pgfqpoint{5.075199in}{2.244746in}}%
\pgfpathlineto{\pgfqpoint{5.077264in}{2.227504in}}%
\pgfpathlineto{\pgfqpoint{5.079845in}{2.215356in}}%
\pgfpathlineto{\pgfqpoint{5.081652in}{2.209760in}}%
\pgfpathlineto{\pgfqpoint{5.089912in}{2.158263in}}%
\pgfpathlineto{\pgfqpoint{5.090687in}{2.160133in}}%
\pgfpathlineto{\pgfqpoint{5.092236in}{2.172470in}}%
\pgfpathlineto{\pgfqpoint{5.095591in}{2.222700in}}%
\pgfpathlineto{\pgfqpoint{5.098947in}{2.260331in}}%
\pgfpathlineto{\pgfqpoint{5.102561in}{2.273309in}}%
\pgfpathlineto{\pgfqpoint{5.103077in}{2.271675in}}%
\pgfpathlineto{\pgfqpoint{5.104110in}{2.258200in}}%
\pgfpathlineto{\pgfqpoint{5.110563in}{2.141726in}}%
\pgfpathlineto{\pgfqpoint{5.112370in}{2.137146in}}%
\pgfpathlineto{\pgfqpoint{5.112628in}{2.137531in}}%
\pgfpathlineto{\pgfqpoint{5.113661in}{2.143569in}}%
\pgfpathlineto{\pgfqpoint{5.115726in}{2.176466in}}%
\pgfpathlineto{\pgfqpoint{5.119082in}{2.219419in}}%
\pgfpathlineto{\pgfqpoint{5.123728in}{2.277179in}}%
\pgfpathlineto{\pgfqpoint{5.124502in}{2.273441in}}%
\pgfpathlineto{\pgfqpoint{5.130440in}{2.207408in}}%
\pgfpathlineto{\pgfqpoint{5.134053in}{2.137359in}}%
\pgfpathlineto{\pgfqpoint{5.134828in}{2.141883in}}%
\pgfpathlineto{\pgfqpoint{5.145153in}{2.250825in}}%
\pgfpathlineto{\pgfqpoint{5.145669in}{2.250000in}}%
\pgfpathlineto{\pgfqpoint{5.146702in}{2.241889in}}%
\pgfpathlineto{\pgfqpoint{5.151348in}{2.179199in}}%
\pgfpathlineto{\pgfqpoint{5.152123in}{2.180617in}}%
\pgfpathlineto{\pgfqpoint{5.160383in}{2.224334in}}%
\pgfpathlineto{\pgfqpoint{5.162190in}{2.237594in}}%
\pgfpathlineto{\pgfqpoint{5.162448in}{2.237154in}}%
\pgfpathlineto{\pgfqpoint{5.163739in}{2.226986in}}%
\pgfpathlineto{\pgfqpoint{5.165804in}{2.215257in}}%
\pgfpathlineto{\pgfqpoint{5.166837in}{2.216307in}}%
\pgfpathlineto{\pgfqpoint{5.167353in}{2.215271in}}%
\pgfpathlineto{\pgfqpoint{5.168385in}{2.206099in}}%
\pgfpathlineto{\pgfqpoint{5.171483in}{2.174542in}}%
\pgfpathlineto{\pgfqpoint{5.171741in}{2.174567in}}%
\pgfpathlineto{\pgfqpoint{5.173290in}{2.177570in}}%
\pgfpathlineto{\pgfqpoint{5.175871in}{2.186348in}}%
\pgfpathlineto{\pgfqpoint{5.177678in}{2.205327in}}%
\pgfpathlineto{\pgfqpoint{5.181034in}{2.239399in}}%
\pgfpathlineto{\pgfqpoint{5.181550in}{2.239674in}}%
\pgfpathlineto{\pgfqpoint{5.181808in}{2.239305in}}%
\pgfpathlineto{\pgfqpoint{5.183099in}{2.232961in}}%
\pgfpathlineto{\pgfqpoint{5.185164in}{2.223016in}}%
\pgfpathlineto{\pgfqpoint{5.185422in}{2.223255in}}%
\pgfpathlineto{\pgfqpoint{5.186971in}{2.226889in}}%
\pgfpathlineto{\pgfqpoint{5.187487in}{2.225622in}}%
\pgfpathlineto{\pgfqpoint{5.188520in}{2.214312in}}%
\pgfpathlineto{\pgfqpoint{5.191876in}{2.162504in}}%
\pgfpathlineto{\pgfqpoint{5.192134in}{2.163411in}}%
\pgfpathlineto{\pgfqpoint{5.193683in}{2.181060in}}%
\pgfpathlineto{\pgfqpoint{5.195748in}{2.196362in}}%
\pgfpathlineto{\pgfqpoint{5.197813in}{2.194781in}}%
\pgfpathlineto{\pgfqpoint{5.198071in}{2.195465in}}%
\pgfpathlineto{\pgfqpoint{5.199620in}{2.205564in}}%
\pgfpathlineto{\pgfqpoint{5.202717in}{2.223745in}}%
\pgfpathlineto{\pgfqpoint{5.204782in}{2.223596in}}%
\pgfpathlineto{\pgfqpoint{5.207364in}{2.231404in}}%
\pgfpathlineto{\pgfqpoint{5.208654in}{2.233522in}}%
\pgfpathlineto{\pgfqpoint{5.208913in}{2.233295in}}%
\pgfpathlineto{\pgfqpoint{5.209687in}{2.230104in}}%
\pgfpathlineto{\pgfqpoint{5.211236in}{2.210169in}}%
\pgfpathlineto{\pgfqpoint{5.213301in}{2.186718in}}%
\pgfpathlineto{\pgfqpoint{5.213559in}{2.186804in}}%
\pgfpathlineto{\pgfqpoint{5.214591in}{2.193180in}}%
\pgfpathlineto{\pgfqpoint{5.216398in}{2.204778in}}%
\pgfpathlineto{\pgfqpoint{5.216657in}{2.204378in}}%
\pgfpathlineto{\pgfqpoint{5.217689in}{2.196600in}}%
\pgfpathlineto{\pgfqpoint{5.219754in}{2.178219in}}%
\pgfpathlineto{\pgfqpoint{5.220270in}{2.179272in}}%
\pgfpathlineto{\pgfqpoint{5.222077in}{2.195986in}}%
\pgfpathlineto{\pgfqpoint{5.223626in}{2.203663in}}%
\pgfpathlineto{\pgfqpoint{5.223884in}{2.203187in}}%
\pgfpathlineto{\pgfqpoint{5.226208in}{2.192265in}}%
\pgfpathlineto{\pgfqpoint{5.226982in}{2.195985in}}%
\pgfpathlineto{\pgfqpoint{5.230854in}{2.228506in}}%
\pgfpathlineto{\pgfqpoint{5.231370in}{2.228180in}}%
\pgfpathlineto{\pgfqpoint{5.232661in}{2.228241in}}%
\pgfpathlineto{\pgfqpoint{5.234984in}{2.232137in}}%
\pgfpathlineto{\pgfqpoint{5.235500in}{2.231070in}}%
\pgfpathlineto{\pgfqpoint{5.236791in}{2.222313in}}%
\pgfpathlineto{\pgfqpoint{5.241179in}{2.188167in}}%
\pgfpathlineto{\pgfqpoint{5.242986in}{2.186662in}}%
\pgfpathlineto{\pgfqpoint{5.245051in}{2.188415in}}%
\pgfpathlineto{\pgfqpoint{5.245310in}{2.188016in}}%
\pgfpathlineto{\pgfqpoint{5.246342in}{2.186167in}}%
\pgfpathlineto{\pgfqpoint{5.246858in}{2.187040in}}%
\pgfpathlineto{\pgfqpoint{5.247891in}{2.196961in}}%
\pgfpathlineto{\pgfqpoint{5.250989in}{2.237283in}}%
\pgfpathlineto{\pgfqpoint{5.251247in}{2.236859in}}%
\pgfpathlineto{\pgfqpoint{5.252279in}{2.229701in}}%
\pgfpathlineto{\pgfqpoint{5.255119in}{2.207053in}}%
\pgfpathlineto{\pgfqpoint{5.255377in}{2.207385in}}%
\pgfpathlineto{\pgfqpoint{5.256926in}{2.211020in}}%
\pgfpathlineto{\pgfqpoint{5.257442in}{2.210108in}}%
\pgfpathlineto{\pgfqpoint{5.258733in}{2.201056in}}%
\pgfpathlineto{\pgfqpoint{5.261572in}{2.177757in}}%
\pgfpathlineto{\pgfqpoint{5.261830in}{2.177899in}}%
\pgfpathlineto{\pgfqpoint{5.262863in}{2.184123in}}%
\pgfpathlineto{\pgfqpoint{5.265960in}{2.207115in}}%
\pgfpathlineto{\pgfqpoint{5.267509in}{2.205337in}}%
\pgfpathlineto{\pgfqpoint{5.268025in}{2.206454in}}%
\pgfpathlineto{\pgfqpoint{5.269316in}{2.217565in}}%
\pgfpathlineto{\pgfqpoint{5.271639in}{2.242913in}}%
\pgfpathlineto{\pgfqpoint{5.271897in}{2.242314in}}%
\pgfpathlineto{\pgfqpoint{5.272930in}{2.231016in}}%
\pgfpathlineto{\pgfqpoint{5.275253in}{2.206515in}}%
\pgfpathlineto{\pgfqpoint{5.275511in}{2.206876in}}%
\pgfpathlineto{\pgfqpoint{5.277318in}{2.212129in}}%
\pgfpathlineto{\pgfqpoint{5.277835in}{2.210741in}}%
\pgfpathlineto{\pgfqpoint{5.279383in}{2.195495in}}%
\pgfpathlineto{\pgfqpoint{5.281190in}{2.181437in}}%
\pgfpathlineto{\pgfqpoint{5.281448in}{2.181819in}}%
\pgfpathlineto{\pgfqpoint{5.284030in}{2.192990in}}%
\pgfpathlineto{\pgfqpoint{5.285062in}{2.191108in}}%
\pgfpathlineto{\pgfqpoint{5.285837in}{2.190310in}}%
\pgfpathlineto{\pgfqpoint{5.286095in}{2.190605in}}%
\pgfpathlineto{\pgfqpoint{5.287127in}{2.195320in}}%
\pgfpathlineto{\pgfqpoint{5.289967in}{2.225114in}}%
\pgfpathlineto{\pgfqpoint{5.292032in}{2.238365in}}%
\pgfpathlineto{\pgfqpoint{5.292290in}{2.238327in}}%
\pgfpathlineto{\pgfqpoint{5.293323in}{2.234588in}}%
\pgfpathlineto{\pgfqpoint{5.302615in}{2.176028in}}%
\pgfpathlineto{\pgfqpoint{5.303132in}{2.176372in}}%
\pgfpathlineto{\pgfqpoint{5.305713in}{2.180798in}}%
\pgfpathlineto{\pgfqpoint{5.307262in}{2.191261in}}%
\pgfpathlineto{\pgfqpoint{5.313457in}{2.245665in}}%
\pgfpathlineto{\pgfqpoint{5.313973in}{2.244242in}}%
\pgfpathlineto{\pgfqpoint{5.318878in}{2.212587in}}%
\pgfpathlineto{\pgfqpoint{5.323266in}{2.188168in}}%
\pgfpathlineto{\pgfqpoint{5.323524in}{2.188362in}}%
\pgfpathlineto{\pgfqpoint{5.324815in}{2.189921in}}%
\pgfpathlineto{\pgfqpoint{5.325073in}{2.189604in}}%
\pgfpathlineto{\pgfqpoint{5.326106in}{2.183394in}}%
\pgfpathlineto{\pgfqpoint{5.328429in}{2.161543in}}%
\pgfpathlineto{\pgfqpoint{5.328945in}{2.163699in}}%
\pgfpathlineto{\pgfqpoint{5.330494in}{2.189922in}}%
\pgfpathlineto{\pgfqpoint{5.333334in}{2.230183in}}%
\pgfpathlineto{\pgfqpoint{5.335140in}{2.232655in}}%
\pgfpathlineto{\pgfqpoint{5.336431in}{2.235567in}}%
\pgfpathlineto{\pgfqpoint{5.338754in}{2.243269in}}%
\pgfpathlineto{\pgfqpoint{5.339012in}{2.242923in}}%
\pgfpathlineto{\pgfqpoint{5.340045in}{2.237770in}}%
\pgfpathlineto{\pgfqpoint{5.349854in}{2.169353in}}%
\pgfpathlineto{\pgfqpoint{5.351145in}{2.167786in}}%
\pgfpathlineto{\pgfqpoint{5.351403in}{2.168088in}}%
\pgfpathlineto{\pgfqpoint{5.352435in}{2.172123in}}%
\pgfpathlineto{\pgfqpoint{5.354759in}{2.194216in}}%
\pgfpathlineto{\pgfqpoint{5.360954in}{2.254265in}}%
\pgfpathlineto{\pgfqpoint{5.361728in}{2.253079in}}%
\pgfpathlineto{\pgfqpoint{5.363019in}{2.243929in}}%
\pgfpathlineto{\pgfqpoint{5.369472in}{2.180107in}}%
\pgfpathlineto{\pgfqpoint{5.370763in}{2.180864in}}%
\pgfpathlineto{\pgfqpoint{5.372312in}{2.181011in}}%
\pgfpathlineto{\pgfqpoint{5.373861in}{2.179725in}}%
\pgfpathlineto{\pgfqpoint{5.374119in}{2.179965in}}%
\pgfpathlineto{\pgfqpoint{5.375409in}{2.184490in}}%
\pgfpathlineto{\pgfqpoint{5.377733in}{2.204739in}}%
\pgfpathlineto{\pgfqpoint{5.380830in}{2.228841in}}%
\pgfpathlineto{\pgfqpoint{5.383412in}{2.233371in}}%
\pgfpathlineto{\pgfqpoint{5.383670in}{2.233156in}}%
\pgfpathlineto{\pgfqpoint{5.384702in}{2.229705in}}%
\pgfpathlineto{\pgfqpoint{5.387542in}{2.220211in}}%
\pgfpathlineto{\pgfqpoint{5.388574in}{2.219618in}}%
\pgfpathlineto{\pgfqpoint{5.389607in}{2.214241in}}%
\pgfpathlineto{\pgfqpoint{5.393479in}{2.176869in}}%
\pgfpathlineto{\pgfqpoint{5.394253in}{2.178646in}}%
\pgfpathlineto{\pgfqpoint{5.398125in}{2.194807in}}%
\pgfpathlineto{\pgfqpoint{5.399416in}{2.195363in}}%
\pgfpathlineto{\pgfqpoint{5.400449in}{2.199039in}}%
\pgfpathlineto{\pgfqpoint{5.402514in}{2.219264in}}%
\pgfpathlineto{\pgfqpoint{5.404062in}{2.228710in}}%
\pgfpathlineto{\pgfqpoint{5.404321in}{2.228399in}}%
\pgfpathlineto{\pgfqpoint{5.405611in}{2.221122in}}%
\pgfpathlineto{\pgfqpoint{5.408451in}{2.207502in}}%
\pgfpathlineto{\pgfqpoint{5.409483in}{2.207758in}}%
\pgfpathlineto{\pgfqpoint{5.411290in}{2.209854in}}%
\pgfpathlineto{\pgfqpoint{5.411548in}{2.209596in}}%
\pgfpathlineto{\pgfqpoint{5.413097in}{2.204898in}}%
\pgfpathlineto{\pgfqpoint{5.414646in}{2.201764in}}%
\pgfpathlineto{\pgfqpoint{5.414904in}{2.201948in}}%
\pgfpathlineto{\pgfqpoint{5.415937in}{2.205072in}}%
\pgfpathlineto{\pgfqpoint{5.418776in}{2.215793in}}%
\pgfpathlineto{\pgfqpoint{5.419034in}{2.215639in}}%
\pgfpathlineto{\pgfqpoint{5.421357in}{2.211111in}}%
\pgfpathlineto{\pgfqpoint{5.422906in}{2.204125in}}%
\pgfpathlineto{\pgfqpoint{5.425488in}{2.194151in}}%
\pgfpathlineto{\pgfqpoint{5.429360in}{2.193184in}}%
\pgfpathlineto{\pgfqpoint{5.431941in}{2.199032in}}%
\pgfpathlineto{\pgfqpoint{5.437878in}{2.232196in}}%
\pgfpathlineto{\pgfqpoint{5.438653in}{2.230950in}}%
\pgfpathlineto{\pgfqpoint{5.441234in}{2.221060in}}%
\pgfpathlineto{\pgfqpoint{5.443815in}{2.200530in}}%
\pgfpathlineto{\pgfqpoint{5.446397in}{2.185116in}}%
\pgfpathlineto{\pgfqpoint{5.447429in}{2.187033in}}%
\pgfpathlineto{\pgfqpoint{5.448978in}{2.190153in}}%
\pgfpathlineto{\pgfqpoint{5.449236in}{2.190036in}}%
\pgfpathlineto{\pgfqpoint{5.450269in}{2.187165in}}%
\pgfpathlineto{\pgfqpoint{5.452334in}{2.179112in}}%
\pgfpathlineto{\pgfqpoint{5.452592in}{2.179574in}}%
\pgfpathlineto{\pgfqpoint{5.453624in}{2.186979in}}%
\pgfpathlineto{\pgfqpoint{5.459561in}{2.245623in}}%
\pgfpathlineto{\pgfqpoint{5.460336in}{2.244879in}}%
\pgfpathlineto{\pgfqpoint{5.462143in}{2.235431in}}%
\pgfpathlineto{\pgfqpoint{5.466531in}{2.202714in}}%
\pgfpathlineto{\pgfqpoint{5.469371in}{2.173184in}}%
\pgfpathlineto{\pgfqpoint{5.470145in}{2.173440in}}%
\pgfpathlineto{\pgfqpoint{5.472726in}{2.176877in}}%
\pgfpathlineto{\pgfqpoint{5.472984in}{2.176700in}}%
\pgfpathlineto{\pgfqpoint{5.474017in}{2.176509in}}%
\pgfpathlineto{\pgfqpoint{5.474275in}{2.176955in}}%
\pgfpathlineto{\pgfqpoint{5.475308in}{2.182458in}}%
\pgfpathlineto{\pgfqpoint{5.481245in}{2.228860in}}%
\pgfpathlineto{\pgfqpoint{5.483568in}{2.232175in}}%
\pgfpathlineto{\pgfqpoint{5.485375in}{2.233880in}}%
\pgfpathlineto{\pgfqpoint{5.485633in}{2.233703in}}%
\pgfpathlineto{\pgfqpoint{5.486666in}{2.230944in}}%
\pgfpathlineto{\pgfqpoint{5.488214in}{2.216991in}}%
\pgfpathlineto{\pgfqpoint{5.491312in}{2.185581in}}%
\pgfpathlineto{\pgfqpoint{5.491570in}{2.185704in}}%
\pgfpathlineto{\pgfqpoint{5.492861in}{2.191010in}}%
\pgfpathlineto{\pgfqpoint{5.494152in}{2.194630in}}%
\pgfpathlineto{\pgfqpoint{5.494410in}{2.194128in}}%
\pgfpathlineto{\pgfqpoint{5.495700in}{2.185739in}}%
\pgfpathlineto{\pgfqpoint{5.496991in}{2.178833in}}%
\pgfpathlineto{\pgfqpoint{5.497507in}{2.179708in}}%
\pgfpathlineto{\pgfqpoint{5.498798in}{2.190886in}}%
\pgfpathlineto{\pgfqpoint{5.502154in}{2.220377in}}%
\pgfpathlineto{\pgfqpoint{5.503444in}{2.220100in}}%
\pgfpathlineto{\pgfqpoint{5.504477in}{2.220942in}}%
\pgfpathlineto{\pgfqpoint{5.506026in}{2.223381in}}%
\pgfpathlineto{\pgfqpoint{5.506542in}{2.222767in}}%
\pgfpathlineto{\pgfqpoint{5.512479in}{2.205519in}}%
\pgfpathlineto{\pgfqpoint{5.515835in}{2.184897in}}%
\pgfpathlineto{\pgfqpoint{5.516351in}{2.185822in}}%
\pgfpathlineto{\pgfqpoint{5.517900in}{2.197150in}}%
\pgfpathlineto{\pgfqpoint{5.520223in}{2.209112in}}%
\pgfpathlineto{\pgfqpoint{5.522030in}{2.212379in}}%
\pgfpathlineto{\pgfqpoint{5.525128in}{2.224857in}}%
\pgfpathlineto{\pgfqpoint{5.525644in}{2.223564in}}%
\pgfpathlineto{\pgfqpoint{5.527193in}{2.211478in}}%
\pgfpathlineto{\pgfqpoint{5.529774in}{2.195787in}}%
\pgfpathlineto{\pgfqpoint{5.531581in}{2.195236in}}%
\pgfpathlineto{\pgfqpoint{5.533388in}{2.194166in}}%
\pgfpathlineto{\pgfqpoint{5.534937in}{2.193206in}}%
\pgfpathlineto{\pgfqpoint{5.535195in}{2.193582in}}%
\pgfpathlineto{\pgfqpoint{5.536486in}{2.199072in}}%
\pgfpathlineto{\pgfqpoint{5.538809in}{2.207138in}}%
\pgfpathlineto{\pgfqpoint{5.540100in}{2.207703in}}%
\pgfpathlineto{\pgfqpoint{5.541390in}{2.213211in}}%
\pgfpathlineto{\pgfqpoint{5.545520in}{2.231869in}}%
\pgfpathlineto{\pgfqpoint{5.546553in}{2.232627in}}%
\pgfpathlineto{\pgfqpoint{5.546811in}{2.232263in}}%
\pgfpathlineto{\pgfqpoint{5.547844in}{2.228084in}}%
\pgfpathlineto{\pgfqpoint{5.549651in}{2.207227in}}%
\pgfpathlineto{\pgfqpoint{5.551974in}{2.182105in}}%
\pgfpathlineto{\pgfqpoint{5.552232in}{2.182170in}}%
\pgfpathlineto{\pgfqpoint{5.553523in}{2.187916in}}%
\pgfpathlineto{\pgfqpoint{5.554555in}{2.190997in}}%
\pgfpathlineto{\pgfqpoint{5.555071in}{2.190395in}}%
\pgfpathlineto{\pgfqpoint{5.557136in}{2.184172in}}%
\pgfpathlineto{\pgfqpoint{5.557653in}{2.185252in}}%
\pgfpathlineto{\pgfqpoint{5.559460in}{2.197940in}}%
\pgfpathlineto{\pgfqpoint{5.561525in}{2.206213in}}%
\pgfpathlineto{\pgfqpoint{5.562815in}{2.208345in}}%
\pgfpathlineto{\pgfqpoint{5.564364in}{2.219558in}}%
\pgfpathlineto{\pgfqpoint{5.566687in}{2.235237in}}%
\pgfpathlineto{\pgfqpoint{5.567720in}{2.230684in}}%
\pgfpathlineto{\pgfqpoint{5.570818in}{2.216590in}}%
\pgfpathlineto{\pgfqpoint{5.572108in}{2.211220in}}%
\pgfpathlineto{\pgfqpoint{5.575722in}{2.187377in}}%
\pgfpathlineto{\pgfqpoint{5.576238in}{2.188065in}}%
\pgfpathlineto{\pgfqpoint{5.578303in}{2.193282in}}%
\pgfpathlineto{\pgfqpoint{5.578820in}{2.192864in}}%
\pgfpathlineto{\pgfqpoint{5.580369in}{2.190077in}}%
\pgfpathlineto{\pgfqpoint{5.580885in}{2.190737in}}%
\pgfpathlineto{\pgfqpoint{5.582175in}{2.199019in}}%
\pgfpathlineto{\pgfqpoint{5.585273in}{2.222521in}}%
\pgfpathlineto{\pgfqpoint{5.585531in}{2.222377in}}%
\pgfpathlineto{\pgfqpoint{5.586564in}{2.218506in}}%
\pgfpathlineto{\pgfqpoint{5.588113in}{2.212898in}}%
\pgfpathlineto{\pgfqpoint{5.588371in}{2.213189in}}%
\pgfpathlineto{\pgfqpoint{5.589403in}{2.218713in}}%
\pgfpathlineto{\pgfqpoint{5.591468in}{2.231984in}}%
\pgfpathlineto{\pgfqpoint{5.591985in}{2.231302in}}%
\pgfpathlineto{\pgfqpoint{5.593275in}{2.220863in}}%
\pgfpathlineto{\pgfqpoint{5.597147in}{2.185698in}}%
\pgfpathlineto{\pgfqpoint{5.599212in}{2.181075in}}%
\pgfpathlineto{\pgfqpoint{5.600245in}{2.182818in}}%
\pgfpathlineto{\pgfqpoint{5.606182in}{2.205355in}}%
\pgfpathlineto{\pgfqpoint{5.611345in}{2.240621in}}%
\pgfpathlineto{\pgfqpoint{5.611603in}{2.240566in}}%
\pgfpathlineto{\pgfqpoint{5.612635in}{2.236987in}}%
\pgfpathlineto{\pgfqpoint{5.615217in}{2.213818in}}%
\pgfpathlineto{\pgfqpoint{5.618314in}{2.190299in}}%
\pgfpathlineto{\pgfqpoint{5.621412in}{2.187786in}}%
\pgfpathlineto{\pgfqpoint{5.624251in}{2.173143in}}%
\pgfpathlineto{\pgfqpoint{5.625026in}{2.175980in}}%
\pgfpathlineto{\pgfqpoint{5.627091in}{2.203106in}}%
\pgfpathlineto{\pgfqpoint{5.630705in}{2.238964in}}%
\pgfpathlineto{\pgfqpoint{5.631221in}{2.239750in}}%
\pgfpathlineto{\pgfqpoint{5.631737in}{2.239215in}}%
\pgfpathlineto{\pgfqpoint{5.633802in}{2.231779in}}%
\pgfpathlineto{\pgfqpoint{5.636384in}{2.213132in}}%
\pgfpathlineto{\pgfqpoint{5.641547in}{2.183182in}}%
\pgfpathlineto{\pgfqpoint{5.643870in}{2.178344in}}%
\pgfpathlineto{\pgfqpoint{5.644644in}{2.179972in}}%
\pgfpathlineto{\pgfqpoint{5.646193in}{2.192764in}}%
\pgfpathlineto{\pgfqpoint{5.649549in}{2.220671in}}%
\pgfpathlineto{\pgfqpoint{5.650581in}{2.218063in}}%
\pgfpathlineto{\pgfqpoint{5.651872in}{2.215206in}}%
\pgfpathlineto{\pgfqpoint{5.652130in}{2.215371in}}%
\pgfpathlineto{\pgfqpoint{5.653421in}{2.219838in}}%
\pgfpathlineto{\pgfqpoint{5.655228in}{2.225520in}}%
\pgfpathlineto{\pgfqpoint{5.655486in}{2.225406in}}%
\pgfpathlineto{\pgfqpoint{5.656776in}{2.221698in}}%
\pgfpathlineto{\pgfqpoint{5.662714in}{2.199364in}}%
\pgfpathlineto{\pgfqpoint{5.664004in}{2.198806in}}%
\pgfpathlineto{\pgfqpoint{5.665295in}{2.194508in}}%
\pgfpathlineto{\pgfqpoint{5.667102in}{2.188365in}}%
\pgfpathlineto{\pgfqpoint{5.667360in}{2.188850in}}%
\pgfpathlineto{\pgfqpoint{5.668651in}{2.197855in}}%
\pgfpathlineto{\pgfqpoint{5.670974in}{2.212650in}}%
\pgfpathlineto{\pgfqpoint{5.672006in}{2.209449in}}%
\pgfpathlineto{\pgfqpoint{5.673555in}{2.204355in}}%
\pgfpathlineto{\pgfqpoint{5.673813in}{2.204601in}}%
\pgfpathlineto{\pgfqpoint{5.674846in}{2.209470in}}%
\pgfpathlineto{\pgfqpoint{5.676653in}{2.218440in}}%
\pgfpathlineto{\pgfqpoint{5.676911in}{2.217984in}}%
\pgfpathlineto{\pgfqpoint{5.677944in}{2.210665in}}%
\pgfpathlineto{\pgfqpoint{5.680267in}{2.190656in}}%
\pgfpathlineto{\pgfqpoint{5.680783in}{2.191410in}}%
\pgfpathlineto{\pgfqpoint{5.682332in}{2.204311in}}%
\pgfpathlineto{\pgfqpoint{5.684655in}{2.217805in}}%
\pgfpathlineto{\pgfqpoint{5.686720in}{2.225369in}}%
\pgfpathlineto{\pgfqpoint{5.688269in}{2.228607in}}%
\pgfpathlineto{\pgfqpoint{5.688527in}{2.228396in}}%
\pgfpathlineto{\pgfqpoint{5.689818in}{2.224386in}}%
\pgfpathlineto{\pgfqpoint{5.692141in}{2.205380in}}%
\pgfpathlineto{\pgfqpoint{5.696271in}{2.177992in}}%
\pgfpathlineto{\pgfqpoint{5.696787in}{2.177848in}}%
\pgfpathlineto{\pgfqpoint{5.697046in}{2.178371in}}%
\pgfpathlineto{\pgfqpoint{5.699111in}{2.189247in}}%
\pgfpathlineto{\pgfqpoint{5.706338in}{2.221533in}}%
\pgfpathlineto{\pgfqpoint{5.707887in}{2.219411in}}%
\pgfpathlineto{\pgfqpoint{5.709436in}{2.218851in}}%
\pgfpathlineto{\pgfqpoint{5.710727in}{2.219100in}}%
\pgfpathlineto{\pgfqpoint{5.710985in}{2.218669in}}%
\pgfpathlineto{\pgfqpoint{5.712275in}{2.213156in}}%
\pgfpathlineto{\pgfqpoint{5.715889in}{2.192605in}}%
\pgfpathlineto{\pgfqpoint{5.716406in}{2.193352in}}%
\pgfpathlineto{\pgfqpoint{5.718213in}{2.203927in}}%
\pgfpathlineto{\pgfqpoint{5.720278in}{2.210466in}}%
\pgfpathlineto{\pgfqpoint{5.722601in}{2.212307in}}%
\pgfpathlineto{\pgfqpoint{5.724924in}{2.215057in}}%
\pgfpathlineto{\pgfqpoint{5.725182in}{2.214871in}}%
\pgfpathlineto{\pgfqpoint{5.726215in}{2.212134in}}%
\pgfpathlineto{\pgfqpoint{5.729570in}{2.192231in}}%
\pgfpathlineto{\pgfqpoint{5.730603in}{2.195566in}}%
\pgfpathlineto{\pgfqpoint{5.732668in}{2.207470in}}%
\pgfpathlineto{\pgfqpoint{5.733184in}{2.206809in}}%
\pgfpathlineto{\pgfqpoint{5.734733in}{2.195739in}}%
\pgfpathlineto{\pgfqpoint{5.736282in}{2.187667in}}%
\pgfpathlineto{\pgfqpoint{5.736540in}{2.187971in}}%
\pgfpathlineto{\pgfqpoint{5.737573in}{2.194204in}}%
\pgfpathlineto{\pgfqpoint{5.740670in}{2.215107in}}%
\pgfpathlineto{\pgfqpoint{5.742219in}{2.214884in}}%
\pgfpathlineto{\pgfqpoint{5.744026in}{2.214787in}}%
\pgfpathlineto{\pgfqpoint{5.745317in}{2.218330in}}%
\pgfpathlineto{\pgfqpoint{5.746091in}{2.222399in}}%
\pgfpathlineto{\pgfqpoint{5.746091in}{2.222399in}}%
\pgfusepath{stroke}%
\end{pgfscope}%
\begin{pgfscope}%
\pgfsetrectcap%
\pgfsetmiterjoin%
\pgfsetlinewidth{0.803000pt}%
\definecolor{currentstroke}{rgb}{0.737255,0.737255,0.737255}%
\pgfsetstrokecolor{currentstroke}%
\pgfsetdash{}{0pt}%
\pgfpathmoveto{\pgfqpoint{0.583136in}{1.796846in}}%
\pgfpathlineto{\pgfqpoint{0.583136in}{2.703703in}}%
\pgfusepath{stroke}%
\end{pgfscope}%
\begin{pgfscope}%
\pgfsetrectcap%
\pgfsetmiterjoin%
\pgfsetlinewidth{0.803000pt}%
\definecolor{currentstroke}{rgb}{0.737255,0.737255,0.737255}%
\pgfsetstrokecolor{currentstroke}%
\pgfsetdash{}{0pt}%
\pgfpathmoveto{\pgfqpoint{5.745833in}{1.796846in}}%
\pgfpathlineto{\pgfqpoint{5.745833in}{2.703703in}}%
\pgfusepath{stroke}%
\end{pgfscope}%
\begin{pgfscope}%
\pgfsetrectcap%
\pgfsetmiterjoin%
\pgfsetlinewidth{0.803000pt}%
\definecolor{currentstroke}{rgb}{0.737255,0.737255,0.737255}%
\pgfsetstrokecolor{currentstroke}%
\pgfsetdash{}{0pt}%
\pgfpathmoveto{\pgfqpoint{0.583136in}{1.796846in}}%
\pgfpathlineto{\pgfqpoint{5.745833in}{1.796846in}}%
\pgfusepath{stroke}%
\end{pgfscope}%
\begin{pgfscope}%
\pgfsetrectcap%
\pgfsetmiterjoin%
\pgfsetlinewidth{0.803000pt}%
\definecolor{currentstroke}{rgb}{0.737255,0.737255,0.737255}%
\pgfsetstrokecolor{currentstroke}%
\pgfsetdash{}{0pt}%
\pgfpathmoveto{\pgfqpoint{0.583136in}{2.703703in}}%
\pgfpathlineto{\pgfqpoint{5.745833in}{2.703703in}}%
\pgfusepath{stroke}%
\end{pgfscope}%
\begin{pgfscope}%
\pgfsetbuttcap%
\pgfsetmiterjoin%
\definecolor{currentfill}{rgb}{0.933333,0.933333,0.933333}%
\pgfsetfillcolor{currentfill}%
\pgfsetlinewidth{0.000000pt}%
\definecolor{currentstroke}{rgb}{0.000000,0.000000,0.000000}%
\pgfsetstrokecolor{currentstroke}%
\pgfsetstrokeopacity{0.000000}%
\pgfsetdash{}{0pt}%
\pgfpathmoveto{\pgfqpoint{0.583136in}{0.544166in}}%
\pgfpathlineto{\pgfqpoint{5.745833in}{0.544166in}}%
\pgfpathlineto{\pgfqpoint{5.745833in}{1.451024in}}%
\pgfpathlineto{\pgfqpoint{0.583136in}{1.451024in}}%
\pgfpathlineto{\pgfqpoint{0.583136in}{0.544166in}}%
\pgfpathclose%
\pgfusepath{fill}%
\end{pgfscope}%
\begin{pgfscope}%
\pgfpathrectangle{\pgfqpoint{0.583136in}{0.544166in}}{\pgfqpoint{5.162697in}{0.906858in}}%
\pgfusepath{clip}%
\pgfsetbuttcap%
\pgfsetroundjoin%
\pgfsetlinewidth{0.501875pt}%
\definecolor{currentstroke}{rgb}{0.698039,0.698039,0.698039}%
\pgfsetstrokecolor{currentstroke}%
\pgfsetdash{{1.850000pt}{0.800000pt}}{0.000000pt}%
\pgfpathmoveto{\pgfqpoint{0.583136in}{0.544166in}}%
\pgfpathlineto{\pgfqpoint{0.583136in}{1.451024in}}%
\pgfusepath{stroke}%
\end{pgfscope}%
\begin{pgfscope}%
\pgfsetbuttcap%
\pgfsetroundjoin%
\definecolor{currentfill}{rgb}{0.000000,0.000000,0.000000}%
\pgfsetfillcolor{currentfill}%
\pgfsetlinewidth{0.803000pt}%
\definecolor{currentstroke}{rgb}{0.000000,0.000000,0.000000}%
\pgfsetstrokecolor{currentstroke}%
\pgfsetdash{}{0pt}%
\pgfsys@defobject{currentmarker}{\pgfqpoint{0.000000in}{0.000000in}}{\pgfqpoint{0.000000in}{0.048611in}}{%
\pgfpathmoveto{\pgfqpoint{0.000000in}{0.000000in}}%
\pgfpathlineto{\pgfqpoint{0.000000in}{0.048611in}}%
\pgfusepath{stroke,fill}%
}%
\begin{pgfscope}%
\pgfsys@transformshift{0.583136in}{0.544166in}%
\pgfsys@useobject{currentmarker}{}%
\end{pgfscope}%
\end{pgfscope}%
\begin{pgfscope}%
\definecolor{textcolor}{rgb}{0.000000,0.000000,0.000000}%
\pgfsetstrokecolor{textcolor}%
\pgfsetfillcolor{textcolor}%
\pgftext[x=0.583136in,y=0.495555in,,top]{\color{textcolor}\rmfamily\fontsize{10.000000}{12.000000}\selectfont \(\displaystyle {0}\)}%
\end{pgfscope}%
\begin{pgfscope}%
\pgfpathrectangle{\pgfqpoint{0.583136in}{0.544166in}}{\pgfqpoint{5.162697in}{0.906858in}}%
\pgfusepath{clip}%
\pgfsetbuttcap%
\pgfsetroundjoin%
\pgfsetlinewidth{0.501875pt}%
\definecolor{currentstroke}{rgb}{0.698039,0.698039,0.698039}%
\pgfsetstrokecolor{currentstroke}%
\pgfsetdash{{1.850000pt}{0.800000pt}}{0.000000pt}%
\pgfpathmoveto{\pgfqpoint{1.615676in}{0.544166in}}%
\pgfpathlineto{\pgfqpoint{1.615676in}{1.451024in}}%
\pgfusepath{stroke}%
\end{pgfscope}%
\begin{pgfscope}%
\pgfsetbuttcap%
\pgfsetroundjoin%
\definecolor{currentfill}{rgb}{0.000000,0.000000,0.000000}%
\pgfsetfillcolor{currentfill}%
\pgfsetlinewidth{0.803000pt}%
\definecolor{currentstroke}{rgb}{0.000000,0.000000,0.000000}%
\pgfsetstrokecolor{currentstroke}%
\pgfsetdash{}{0pt}%
\pgfsys@defobject{currentmarker}{\pgfqpoint{0.000000in}{0.000000in}}{\pgfqpoint{0.000000in}{0.048611in}}{%
\pgfpathmoveto{\pgfqpoint{0.000000in}{0.000000in}}%
\pgfpathlineto{\pgfqpoint{0.000000in}{0.048611in}}%
\pgfusepath{stroke,fill}%
}%
\begin{pgfscope}%
\pgfsys@transformshift{1.615676in}{0.544166in}%
\pgfsys@useobject{currentmarker}{}%
\end{pgfscope}%
\end{pgfscope}%
\begin{pgfscope}%
\definecolor{textcolor}{rgb}{0.000000,0.000000,0.000000}%
\pgfsetstrokecolor{textcolor}%
\pgfsetfillcolor{textcolor}%
\pgftext[x=1.615676in,y=0.495555in,,top]{\color{textcolor}\rmfamily\fontsize{10.000000}{12.000000}\selectfont \(\displaystyle {20}\)}%
\end{pgfscope}%
\begin{pgfscope}%
\pgfpathrectangle{\pgfqpoint{0.583136in}{0.544166in}}{\pgfqpoint{5.162697in}{0.906858in}}%
\pgfusepath{clip}%
\pgfsetbuttcap%
\pgfsetroundjoin%
\pgfsetlinewidth{0.501875pt}%
\definecolor{currentstroke}{rgb}{0.698039,0.698039,0.698039}%
\pgfsetstrokecolor{currentstroke}%
\pgfsetdash{{1.850000pt}{0.800000pt}}{0.000000pt}%
\pgfpathmoveto{\pgfqpoint{2.648215in}{0.544166in}}%
\pgfpathlineto{\pgfqpoint{2.648215in}{1.451024in}}%
\pgfusepath{stroke}%
\end{pgfscope}%
\begin{pgfscope}%
\pgfsetbuttcap%
\pgfsetroundjoin%
\definecolor{currentfill}{rgb}{0.000000,0.000000,0.000000}%
\pgfsetfillcolor{currentfill}%
\pgfsetlinewidth{0.803000pt}%
\definecolor{currentstroke}{rgb}{0.000000,0.000000,0.000000}%
\pgfsetstrokecolor{currentstroke}%
\pgfsetdash{}{0pt}%
\pgfsys@defobject{currentmarker}{\pgfqpoint{0.000000in}{0.000000in}}{\pgfqpoint{0.000000in}{0.048611in}}{%
\pgfpathmoveto{\pgfqpoint{0.000000in}{0.000000in}}%
\pgfpathlineto{\pgfqpoint{0.000000in}{0.048611in}}%
\pgfusepath{stroke,fill}%
}%
\begin{pgfscope}%
\pgfsys@transformshift{2.648215in}{0.544166in}%
\pgfsys@useobject{currentmarker}{}%
\end{pgfscope}%
\end{pgfscope}%
\begin{pgfscope}%
\definecolor{textcolor}{rgb}{0.000000,0.000000,0.000000}%
\pgfsetstrokecolor{textcolor}%
\pgfsetfillcolor{textcolor}%
\pgftext[x=2.648215in,y=0.495555in,,top]{\color{textcolor}\rmfamily\fontsize{10.000000}{12.000000}\selectfont \(\displaystyle {40}\)}%
\end{pgfscope}%
\begin{pgfscope}%
\pgfpathrectangle{\pgfqpoint{0.583136in}{0.544166in}}{\pgfqpoint{5.162697in}{0.906858in}}%
\pgfusepath{clip}%
\pgfsetbuttcap%
\pgfsetroundjoin%
\pgfsetlinewidth{0.501875pt}%
\definecolor{currentstroke}{rgb}{0.698039,0.698039,0.698039}%
\pgfsetstrokecolor{currentstroke}%
\pgfsetdash{{1.850000pt}{0.800000pt}}{0.000000pt}%
\pgfpathmoveto{\pgfqpoint{3.680754in}{0.544166in}}%
\pgfpathlineto{\pgfqpoint{3.680754in}{1.451024in}}%
\pgfusepath{stroke}%
\end{pgfscope}%
\begin{pgfscope}%
\pgfsetbuttcap%
\pgfsetroundjoin%
\definecolor{currentfill}{rgb}{0.000000,0.000000,0.000000}%
\pgfsetfillcolor{currentfill}%
\pgfsetlinewidth{0.803000pt}%
\definecolor{currentstroke}{rgb}{0.000000,0.000000,0.000000}%
\pgfsetstrokecolor{currentstroke}%
\pgfsetdash{}{0pt}%
\pgfsys@defobject{currentmarker}{\pgfqpoint{0.000000in}{0.000000in}}{\pgfqpoint{0.000000in}{0.048611in}}{%
\pgfpathmoveto{\pgfqpoint{0.000000in}{0.000000in}}%
\pgfpathlineto{\pgfqpoint{0.000000in}{0.048611in}}%
\pgfusepath{stroke,fill}%
}%
\begin{pgfscope}%
\pgfsys@transformshift{3.680754in}{0.544166in}%
\pgfsys@useobject{currentmarker}{}%
\end{pgfscope}%
\end{pgfscope}%
\begin{pgfscope}%
\definecolor{textcolor}{rgb}{0.000000,0.000000,0.000000}%
\pgfsetstrokecolor{textcolor}%
\pgfsetfillcolor{textcolor}%
\pgftext[x=3.680754in,y=0.495555in,,top]{\color{textcolor}\rmfamily\fontsize{10.000000}{12.000000}\selectfont \(\displaystyle {60}\)}%
\end{pgfscope}%
\begin{pgfscope}%
\pgfpathrectangle{\pgfqpoint{0.583136in}{0.544166in}}{\pgfqpoint{5.162697in}{0.906858in}}%
\pgfusepath{clip}%
\pgfsetbuttcap%
\pgfsetroundjoin%
\pgfsetlinewidth{0.501875pt}%
\definecolor{currentstroke}{rgb}{0.698039,0.698039,0.698039}%
\pgfsetstrokecolor{currentstroke}%
\pgfsetdash{{1.850000pt}{0.800000pt}}{0.000000pt}%
\pgfpathmoveto{\pgfqpoint{4.713294in}{0.544166in}}%
\pgfpathlineto{\pgfqpoint{4.713294in}{1.451024in}}%
\pgfusepath{stroke}%
\end{pgfscope}%
\begin{pgfscope}%
\pgfsetbuttcap%
\pgfsetroundjoin%
\definecolor{currentfill}{rgb}{0.000000,0.000000,0.000000}%
\pgfsetfillcolor{currentfill}%
\pgfsetlinewidth{0.803000pt}%
\definecolor{currentstroke}{rgb}{0.000000,0.000000,0.000000}%
\pgfsetstrokecolor{currentstroke}%
\pgfsetdash{}{0pt}%
\pgfsys@defobject{currentmarker}{\pgfqpoint{0.000000in}{0.000000in}}{\pgfqpoint{0.000000in}{0.048611in}}{%
\pgfpathmoveto{\pgfqpoint{0.000000in}{0.000000in}}%
\pgfpathlineto{\pgfqpoint{0.000000in}{0.048611in}}%
\pgfusepath{stroke,fill}%
}%
\begin{pgfscope}%
\pgfsys@transformshift{4.713294in}{0.544166in}%
\pgfsys@useobject{currentmarker}{}%
\end{pgfscope}%
\end{pgfscope}%
\begin{pgfscope}%
\definecolor{textcolor}{rgb}{0.000000,0.000000,0.000000}%
\pgfsetstrokecolor{textcolor}%
\pgfsetfillcolor{textcolor}%
\pgftext[x=4.713294in,y=0.495555in,,top]{\color{textcolor}\rmfamily\fontsize{10.000000}{12.000000}\selectfont \(\displaystyle {80}\)}%
\end{pgfscope}%
\begin{pgfscope}%
\pgfpathrectangle{\pgfqpoint{0.583136in}{0.544166in}}{\pgfqpoint{5.162697in}{0.906858in}}%
\pgfusepath{clip}%
\pgfsetbuttcap%
\pgfsetroundjoin%
\pgfsetlinewidth{0.501875pt}%
\definecolor{currentstroke}{rgb}{0.698039,0.698039,0.698039}%
\pgfsetstrokecolor{currentstroke}%
\pgfsetdash{{1.850000pt}{0.800000pt}}{0.000000pt}%
\pgfpathmoveto{\pgfqpoint{5.745833in}{0.544166in}}%
\pgfpathlineto{\pgfqpoint{5.745833in}{1.451024in}}%
\pgfusepath{stroke}%
\end{pgfscope}%
\begin{pgfscope}%
\pgfsetbuttcap%
\pgfsetroundjoin%
\definecolor{currentfill}{rgb}{0.000000,0.000000,0.000000}%
\pgfsetfillcolor{currentfill}%
\pgfsetlinewidth{0.803000pt}%
\definecolor{currentstroke}{rgb}{0.000000,0.000000,0.000000}%
\pgfsetstrokecolor{currentstroke}%
\pgfsetdash{}{0pt}%
\pgfsys@defobject{currentmarker}{\pgfqpoint{0.000000in}{0.000000in}}{\pgfqpoint{0.000000in}{0.048611in}}{%
\pgfpathmoveto{\pgfqpoint{0.000000in}{0.000000in}}%
\pgfpathlineto{\pgfqpoint{0.000000in}{0.048611in}}%
\pgfusepath{stroke,fill}%
}%
\begin{pgfscope}%
\pgfsys@transformshift{5.745833in}{0.544166in}%
\pgfsys@useobject{currentmarker}{}%
\end{pgfscope}%
\end{pgfscope}%
\begin{pgfscope}%
\definecolor{textcolor}{rgb}{0.000000,0.000000,0.000000}%
\pgfsetstrokecolor{textcolor}%
\pgfsetfillcolor{textcolor}%
\pgftext[x=5.745833in,y=0.495555in,,top]{\color{textcolor}\rmfamily\fontsize{10.000000}{12.000000}\selectfont \(\displaystyle {100}\)}%
\end{pgfscope}%
\begin{pgfscope}%
\definecolor{textcolor}{rgb}{0.000000,0.000000,0.000000}%
\pgfsetstrokecolor{textcolor}%
\pgfsetfillcolor{textcolor}%
\pgftext[x=3.164485in,y=0.316666in,,top]{\color{textcolor}\rmfamily\fontsize{12.000000}{14.400000}\selectfont Temps [s]}%
\end{pgfscope}%
\begin{pgfscope}%
\pgfpathrectangle{\pgfqpoint{0.583136in}{0.544166in}}{\pgfqpoint{5.162697in}{0.906858in}}%
\pgfusepath{clip}%
\pgfsetbuttcap%
\pgfsetroundjoin%
\pgfsetlinewidth{0.501875pt}%
\definecolor{currentstroke}{rgb}{0.698039,0.698039,0.698039}%
\pgfsetstrokecolor{currentstroke}%
\pgfsetdash{{1.850000pt}{0.800000pt}}{0.000000pt}%
\pgfpathmoveto{\pgfqpoint{0.583136in}{0.585387in}}%
\pgfpathlineto{\pgfqpoint{5.745833in}{0.585387in}}%
\pgfusepath{stroke}%
\end{pgfscope}%
\begin{pgfscope}%
\pgfsetbuttcap%
\pgfsetroundjoin%
\definecolor{currentfill}{rgb}{0.000000,0.000000,0.000000}%
\pgfsetfillcolor{currentfill}%
\pgfsetlinewidth{0.803000pt}%
\definecolor{currentstroke}{rgb}{0.000000,0.000000,0.000000}%
\pgfsetstrokecolor{currentstroke}%
\pgfsetdash{}{0pt}%
\pgfsys@defobject{currentmarker}{\pgfqpoint{0.000000in}{0.000000in}}{\pgfqpoint{0.048611in}{0.000000in}}{%
\pgfpathmoveto{\pgfqpoint{0.000000in}{0.000000in}}%
\pgfpathlineto{\pgfqpoint{0.048611in}{0.000000in}}%
\pgfusepath{stroke,fill}%
}%
\begin{pgfscope}%
\pgfsys@transformshift{0.583136in}{0.585387in}%
\pgfsys@useobject{currentmarker}{}%
\end{pgfscope}%
\end{pgfscope}%
\begin{pgfscope}%
\definecolor{textcolor}{rgb}{0.000000,0.000000,0.000000}%
\pgfsetstrokecolor{textcolor}%
\pgfsetfillcolor{textcolor}%
\pgftext[x=0.465080in, y=0.537193in, left, base]{\color{textcolor}\rmfamily\fontsize{10.000000}{12.000000}\selectfont \(\displaystyle {0}\)}%
\end{pgfscope}%
\begin{pgfscope}%
\pgfpathrectangle{\pgfqpoint{0.583136in}{0.544166in}}{\pgfqpoint{5.162697in}{0.906858in}}%
\pgfusepath{clip}%
\pgfsetbuttcap%
\pgfsetroundjoin%
\pgfsetlinewidth{0.501875pt}%
\definecolor{currentstroke}{rgb}{0.698039,0.698039,0.698039}%
\pgfsetstrokecolor{currentstroke}%
\pgfsetdash{{1.850000pt}{0.800000pt}}{0.000000pt}%
\pgfpathmoveto{\pgfqpoint{0.583136in}{1.007704in}}%
\pgfpathlineto{\pgfqpoint{5.745833in}{1.007704in}}%
\pgfusepath{stroke}%
\end{pgfscope}%
\begin{pgfscope}%
\pgfsetbuttcap%
\pgfsetroundjoin%
\definecolor{currentfill}{rgb}{0.000000,0.000000,0.000000}%
\pgfsetfillcolor{currentfill}%
\pgfsetlinewidth{0.803000pt}%
\definecolor{currentstroke}{rgb}{0.000000,0.000000,0.000000}%
\pgfsetstrokecolor{currentstroke}%
\pgfsetdash{}{0pt}%
\pgfsys@defobject{currentmarker}{\pgfqpoint{0.000000in}{0.000000in}}{\pgfqpoint{0.048611in}{0.000000in}}{%
\pgfpathmoveto{\pgfqpoint{0.000000in}{0.000000in}}%
\pgfpathlineto{\pgfqpoint{0.048611in}{0.000000in}}%
\pgfusepath{stroke,fill}%
}%
\begin{pgfscope}%
\pgfsys@transformshift{0.583136in}{1.007704in}%
\pgfsys@useobject{currentmarker}{}%
\end{pgfscope}%
\end{pgfscope}%
\begin{pgfscope}%
\definecolor{textcolor}{rgb}{0.000000,0.000000,0.000000}%
\pgfsetstrokecolor{textcolor}%
\pgfsetfillcolor{textcolor}%
\pgftext[x=0.395636in, y=0.959510in, left, base]{\color{textcolor}\rmfamily\fontsize{10.000000}{12.000000}\selectfont \(\displaystyle {10}\)}%
\end{pgfscope}%
\begin{pgfscope}%
\pgfpathrectangle{\pgfqpoint{0.583136in}{0.544166in}}{\pgfqpoint{5.162697in}{0.906858in}}%
\pgfusepath{clip}%
\pgfsetbuttcap%
\pgfsetroundjoin%
\pgfsetlinewidth{0.501875pt}%
\definecolor{currentstroke}{rgb}{0.698039,0.698039,0.698039}%
\pgfsetstrokecolor{currentstroke}%
\pgfsetdash{{1.850000pt}{0.800000pt}}{0.000000pt}%
\pgfpathmoveto{\pgfqpoint{0.583136in}{1.430022in}}%
\pgfpathlineto{\pgfqpoint{5.745833in}{1.430022in}}%
\pgfusepath{stroke}%
\end{pgfscope}%
\begin{pgfscope}%
\pgfsetbuttcap%
\pgfsetroundjoin%
\definecolor{currentfill}{rgb}{0.000000,0.000000,0.000000}%
\pgfsetfillcolor{currentfill}%
\pgfsetlinewidth{0.803000pt}%
\definecolor{currentstroke}{rgb}{0.000000,0.000000,0.000000}%
\pgfsetstrokecolor{currentstroke}%
\pgfsetdash{}{0pt}%
\pgfsys@defobject{currentmarker}{\pgfqpoint{0.000000in}{0.000000in}}{\pgfqpoint{0.048611in}{0.000000in}}{%
\pgfpathmoveto{\pgfqpoint{0.000000in}{0.000000in}}%
\pgfpathlineto{\pgfqpoint{0.048611in}{0.000000in}}%
\pgfusepath{stroke,fill}%
}%
\begin{pgfscope}%
\pgfsys@transformshift{0.583136in}{1.430022in}%
\pgfsys@useobject{currentmarker}{}%
\end{pgfscope}%
\end{pgfscope}%
\begin{pgfscope}%
\definecolor{textcolor}{rgb}{0.000000,0.000000,0.000000}%
\pgfsetstrokecolor{textcolor}%
\pgfsetfillcolor{textcolor}%
\pgftext[x=0.395636in, y=1.381827in, left, base]{\color{textcolor}\rmfamily\fontsize{10.000000}{12.000000}\selectfont \(\displaystyle {20}\)}%
\end{pgfscope}%
\begin{pgfscope}%
\definecolor{textcolor}{rgb}{0.000000,0.000000,0.000000}%
\pgfsetstrokecolor{textcolor}%
\pgfsetfillcolor{textcolor}%
\pgftext[x=0.340080in,y=0.997595in,,bottom,rotate=90.000000]{\color{textcolor}\rmfamily\fontsize{12.000000}{14.400000}\selectfont STA/LTA}%
\end{pgfscope}%
\begin{pgfscope}%
\pgfpathrectangle{\pgfqpoint{0.583136in}{0.544166in}}{\pgfqpoint{5.162697in}{0.906858in}}%
\pgfusepath{clip}%
\pgfsetrectcap%
\pgfsetroundjoin%
\pgfsetlinewidth{1.505625pt}%
\definecolor{currentstroke}{rgb}{0.172549,0.627451,0.172549}%
\pgfsetstrokecolor{currentstroke}%
\pgfsetdash{}{0pt}%
\pgfpathmoveto{\pgfqpoint{0.583136in}{0.585387in}}%
\pgfpathlineto{\pgfqpoint{1.615417in}{0.585387in}}%
\pgfpathlineto{\pgfqpoint{1.616708in}{0.631191in}}%
\pgfpathlineto{\pgfqpoint{1.618515in}{0.633996in}}%
\pgfpathlineto{\pgfqpoint{1.620838in}{0.636023in}}%
\pgfpathlineto{\pgfqpoint{1.622129in}{0.633614in}}%
\pgfpathlineto{\pgfqpoint{1.625227in}{0.627413in}}%
\pgfpathlineto{\pgfqpoint{1.626517in}{0.626338in}}%
\pgfpathlineto{\pgfqpoint{1.630906in}{0.618747in}}%
\pgfpathlineto{\pgfqpoint{1.634003in}{0.618286in}}%
\pgfpathlineto{\pgfqpoint{1.636585in}{0.616226in}}%
\pgfpathlineto{\pgfqpoint{1.639424in}{0.616745in}}%
\pgfpathlineto{\pgfqpoint{1.642263in}{0.616079in}}%
\pgfpathlineto{\pgfqpoint{1.646652in}{0.610563in}}%
\pgfpathlineto{\pgfqpoint{1.647168in}{0.610996in}}%
\pgfpathlineto{\pgfqpoint{1.651298in}{0.614426in}}%
\pgfpathlineto{\pgfqpoint{1.652589in}{0.613097in}}%
\pgfpathlineto{\pgfqpoint{1.655945in}{0.608159in}}%
\pgfpathlineto{\pgfqpoint{1.658010in}{0.607626in}}%
\pgfpathlineto{\pgfqpoint{1.661882in}{0.605676in}}%
\pgfpathlineto{\pgfqpoint{1.667044in}{0.606968in}}%
\pgfpathlineto{\pgfqpoint{1.669626in}{0.609344in}}%
\pgfpathlineto{\pgfqpoint{1.673498in}{0.607385in}}%
\pgfpathlineto{\pgfqpoint{1.675047in}{0.608375in}}%
\pgfpathlineto{\pgfqpoint{1.679693in}{0.614156in}}%
\pgfpathlineto{\pgfqpoint{1.689760in}{0.615791in}}%
\pgfpathlineto{\pgfqpoint{1.692600in}{0.616481in}}%
\pgfpathlineto{\pgfqpoint{1.696988in}{0.618267in}}%
\pgfpathlineto{\pgfqpoint{1.699311in}{0.617931in}}%
\pgfpathlineto{\pgfqpoint{1.702409in}{0.615257in}}%
\pgfpathlineto{\pgfqpoint{1.710927in}{0.612978in}}%
\pgfpathlineto{\pgfqpoint{1.714541in}{0.610712in}}%
\pgfpathlineto{\pgfqpoint{1.719446in}{0.608161in}}%
\pgfpathlineto{\pgfqpoint{1.722802in}{0.603645in}}%
\pgfpathlineto{\pgfqpoint{1.725899in}{0.603173in}}%
\pgfpathlineto{\pgfqpoint{1.727190in}{0.602153in}}%
\pgfpathlineto{\pgfqpoint{1.730804in}{0.597520in}}%
\pgfpathlineto{\pgfqpoint{1.735708in}{0.598532in}}%
\pgfpathlineto{\pgfqpoint{1.736999in}{0.601473in}}%
\pgfpathlineto{\pgfqpoint{1.739580in}{0.606717in}}%
\pgfpathlineto{\pgfqpoint{1.741387in}{0.604595in}}%
\pgfpathlineto{\pgfqpoint{1.742936in}{0.604473in}}%
\pgfpathlineto{\pgfqpoint{1.746292in}{0.609198in}}%
\pgfpathlineto{\pgfqpoint{1.748615in}{0.612576in}}%
\pgfpathlineto{\pgfqpoint{1.750938in}{0.613191in}}%
\pgfpathlineto{\pgfqpoint{1.759457in}{0.624486in}}%
\pgfpathlineto{\pgfqpoint{1.761522in}{0.624927in}}%
\pgfpathlineto{\pgfqpoint{1.763071in}{0.629694in}}%
\pgfpathlineto{\pgfqpoint{1.769008in}{0.651320in}}%
\pgfpathlineto{\pgfqpoint{1.771331in}{0.652444in}}%
\pgfpathlineto{\pgfqpoint{1.774945in}{0.656992in}}%
\pgfpathlineto{\pgfqpoint{1.781656in}{0.668452in}}%
\pgfpathlineto{\pgfqpoint{1.785012in}{0.668397in}}%
\pgfpathlineto{\pgfqpoint{1.786303in}{0.669013in}}%
\pgfpathlineto{\pgfqpoint{1.787852in}{0.674268in}}%
\pgfpathlineto{\pgfqpoint{1.791724in}{0.686819in}}%
\pgfpathlineto{\pgfqpoint{1.793530in}{0.687269in}}%
\pgfpathlineto{\pgfqpoint{1.796112in}{0.685514in}}%
\pgfpathlineto{\pgfqpoint{1.798435in}{0.682362in}}%
\pgfpathlineto{\pgfqpoint{1.798951in}{0.683019in}}%
\pgfpathlineto{\pgfqpoint{1.802049in}{0.689962in}}%
\pgfpathlineto{\pgfqpoint{1.802565in}{0.689393in}}%
\pgfpathlineto{\pgfqpoint{1.805147in}{0.685692in}}%
\pgfpathlineto{\pgfqpoint{1.805663in}{0.686061in}}%
\pgfpathlineto{\pgfqpoint{1.807470in}{0.686312in}}%
\pgfpathlineto{\pgfqpoint{1.808760in}{0.686794in}}%
\pgfpathlineto{\pgfqpoint{1.812891in}{0.691150in}}%
\pgfpathlineto{\pgfqpoint{1.814181in}{0.689306in}}%
\pgfpathlineto{\pgfqpoint{1.816505in}{0.680417in}}%
\pgfpathlineto{\pgfqpoint{1.819860in}{0.668499in}}%
\pgfpathlineto{\pgfqpoint{1.823216in}{0.666861in}}%
\pgfpathlineto{\pgfqpoint{1.825797in}{0.666438in}}%
\pgfpathlineto{\pgfqpoint{1.827088in}{0.665975in}}%
\pgfpathlineto{\pgfqpoint{1.833283in}{0.657851in}}%
\pgfpathlineto{\pgfqpoint{1.838446in}{0.657543in}}%
\pgfpathlineto{\pgfqpoint{1.840253in}{0.650584in}}%
\pgfpathlineto{\pgfqpoint{1.843867in}{0.637053in}}%
\pgfpathlineto{\pgfqpoint{1.849546in}{0.635377in}}%
\pgfpathlineto{\pgfqpoint{1.851611in}{0.630904in}}%
\pgfpathlineto{\pgfqpoint{1.854192in}{0.626540in}}%
\pgfpathlineto{\pgfqpoint{1.860646in}{0.623340in}}%
\pgfpathlineto{\pgfqpoint{1.861936in}{0.622095in}}%
\pgfpathlineto{\pgfqpoint{1.862194in}{0.622264in}}%
\pgfpathlineto{\pgfqpoint{1.865550in}{0.624522in}}%
\pgfpathlineto{\pgfqpoint{1.869422in}{0.623342in}}%
\pgfpathlineto{\pgfqpoint{1.873810in}{0.623115in}}%
\pgfpathlineto{\pgfqpoint{1.875101in}{0.627315in}}%
\pgfpathlineto{\pgfqpoint{1.877424in}{0.635909in}}%
\pgfpathlineto{\pgfqpoint{1.877682in}{0.635673in}}%
\pgfpathlineto{\pgfqpoint{1.879231in}{0.634053in}}%
\pgfpathlineto{\pgfqpoint{1.879489in}{0.634315in}}%
\pgfpathlineto{\pgfqpoint{1.881554in}{0.636273in}}%
\pgfpathlineto{\pgfqpoint{1.881813in}{0.636091in}}%
\pgfpathlineto{\pgfqpoint{1.884652in}{0.635206in}}%
\pgfpathlineto{\pgfqpoint{1.890847in}{0.634541in}}%
\pgfpathlineto{\pgfqpoint{1.894461in}{0.628692in}}%
\pgfpathlineto{\pgfqpoint{1.896526in}{0.629269in}}%
\pgfpathlineto{\pgfqpoint{1.898850in}{0.630601in}}%
\pgfpathlineto{\pgfqpoint{1.901689in}{0.629466in}}%
\pgfpathlineto{\pgfqpoint{1.904012in}{0.627671in}}%
\pgfpathlineto{\pgfqpoint{1.909175in}{0.629363in}}%
\pgfpathlineto{\pgfqpoint{1.911240in}{0.628604in}}%
\pgfpathlineto{\pgfqpoint{1.912531in}{0.628896in}}%
\pgfpathlineto{\pgfqpoint{1.914338in}{0.623886in}}%
\pgfpathlineto{\pgfqpoint{1.916661in}{0.620980in}}%
\pgfpathlineto{\pgfqpoint{1.918984in}{0.620943in}}%
\pgfpathlineto{\pgfqpoint{1.924663in}{0.625004in}}%
\pgfpathlineto{\pgfqpoint{1.925696in}{0.624191in}}%
\pgfpathlineto{\pgfqpoint{1.926986in}{0.618829in}}%
\pgfpathlineto{\pgfqpoint{1.929051in}{0.610517in}}%
\pgfpathlineto{\pgfqpoint{1.929309in}{0.610814in}}%
\pgfpathlineto{\pgfqpoint{1.931375in}{0.614420in}}%
\pgfpathlineto{\pgfqpoint{1.931891in}{0.613715in}}%
\pgfpathlineto{\pgfqpoint{1.933181in}{0.612216in}}%
\pgfpathlineto{\pgfqpoint{1.933440in}{0.612412in}}%
\pgfpathlineto{\pgfqpoint{1.939635in}{0.619735in}}%
\pgfpathlineto{\pgfqpoint{1.943507in}{0.619039in}}%
\pgfpathlineto{\pgfqpoint{1.945572in}{0.620667in}}%
\pgfpathlineto{\pgfqpoint{1.947895in}{0.621829in}}%
\pgfpathlineto{\pgfqpoint{1.950993in}{0.619986in}}%
\pgfpathlineto{\pgfqpoint{1.955123in}{0.618984in}}%
\pgfpathlineto{\pgfqpoint{1.957446in}{0.617996in}}%
\pgfpathlineto{\pgfqpoint{1.959253in}{0.617422in}}%
\pgfpathlineto{\pgfqpoint{1.961318in}{0.616261in}}%
\pgfpathlineto{\pgfqpoint{1.963125in}{0.616617in}}%
\pgfpathlineto{\pgfqpoint{1.965448in}{0.615592in}}%
\pgfpathlineto{\pgfqpoint{1.970611in}{0.618469in}}%
\pgfpathlineto{\pgfqpoint{1.973709in}{0.618953in}}%
\pgfpathlineto{\pgfqpoint{1.977323in}{0.619236in}}%
\pgfpathlineto{\pgfqpoint{1.979129in}{0.620120in}}%
\pgfpathlineto{\pgfqpoint{1.980678in}{0.621620in}}%
\pgfpathlineto{\pgfqpoint{1.980936in}{0.621396in}}%
\pgfpathlineto{\pgfqpoint{1.984292in}{0.618290in}}%
\pgfpathlineto{\pgfqpoint{1.985325in}{0.617733in}}%
\pgfpathlineto{\pgfqpoint{1.988680in}{0.612462in}}%
\pgfpathlineto{\pgfqpoint{1.989197in}{0.612892in}}%
\pgfpathlineto{\pgfqpoint{1.991778in}{0.614031in}}%
\pgfpathlineto{\pgfqpoint{1.994876in}{0.614918in}}%
\pgfpathlineto{\pgfqpoint{1.996941in}{0.615670in}}%
\pgfpathlineto{\pgfqpoint{2.000038in}{0.614455in}}%
\pgfpathlineto{\pgfqpoint{2.012687in}{0.627592in}}%
\pgfpathlineto{\pgfqpoint{2.015268in}{0.627134in}}%
\pgfpathlineto{\pgfqpoint{2.021464in}{0.630337in}}%
\pgfpathlineto{\pgfqpoint{2.023012in}{0.630137in}}%
\pgfpathlineto{\pgfqpoint{2.030498in}{0.635545in}}%
\pgfpathlineto{\pgfqpoint{2.031015in}{0.634726in}}%
\pgfpathlineto{\pgfqpoint{2.033596in}{0.632011in}}%
\pgfpathlineto{\pgfqpoint{2.036952in}{0.631542in}}%
\pgfpathlineto{\pgfqpoint{2.039791in}{0.632595in}}%
\pgfpathlineto{\pgfqpoint{2.042889in}{0.630642in}}%
\pgfpathlineto{\pgfqpoint{2.043147in}{0.630981in}}%
\pgfpathlineto{\pgfqpoint{2.045470in}{0.632639in}}%
\pgfpathlineto{\pgfqpoint{2.049342in}{0.630844in}}%
\pgfpathlineto{\pgfqpoint{2.052182in}{0.634894in}}%
\pgfpathlineto{\pgfqpoint{2.052440in}{0.634722in}}%
\pgfpathlineto{\pgfqpoint{2.058635in}{0.627364in}}%
\pgfpathlineto{\pgfqpoint{2.060442in}{0.626075in}}%
\pgfpathlineto{\pgfqpoint{2.063281in}{0.626788in}}%
\pgfpathlineto{\pgfqpoint{2.066121in}{0.636037in}}%
\pgfpathlineto{\pgfqpoint{2.066637in}{0.635901in}}%
\pgfpathlineto{\pgfqpoint{2.070251in}{0.634848in}}%
\pgfpathlineto{\pgfqpoint{2.072058in}{0.633105in}}%
\pgfpathlineto{\pgfqpoint{2.072316in}{0.633393in}}%
\pgfpathlineto{\pgfqpoint{2.076188in}{0.640357in}}%
\pgfpathlineto{\pgfqpoint{2.077221in}{0.639678in}}%
\pgfpathlineto{\pgfqpoint{2.079028in}{0.637498in}}%
\pgfpathlineto{\pgfqpoint{2.080576in}{0.635622in}}%
\pgfpathlineto{\pgfqpoint{2.080835in}{0.635894in}}%
\pgfpathlineto{\pgfqpoint{2.082642in}{0.641727in}}%
\pgfpathlineto{\pgfqpoint{2.084448in}{0.643845in}}%
\pgfpathlineto{\pgfqpoint{2.089095in}{0.642163in}}%
\pgfpathlineto{\pgfqpoint{2.091160in}{0.641266in}}%
\pgfpathlineto{\pgfqpoint{2.093999in}{0.642195in}}%
\pgfpathlineto{\pgfqpoint{2.097871in}{0.644459in}}%
\pgfpathlineto{\pgfqpoint{2.100453in}{0.643421in}}%
\pgfpathlineto{\pgfqpoint{2.104325in}{0.638705in}}%
\pgfpathlineto{\pgfqpoint{2.104841in}{0.639262in}}%
\pgfpathlineto{\pgfqpoint{2.107164in}{0.640646in}}%
\pgfpathlineto{\pgfqpoint{2.114134in}{0.639699in}}%
\pgfpathlineto{\pgfqpoint{2.115683in}{0.635552in}}%
\pgfpathlineto{\pgfqpoint{2.117748in}{0.631312in}}%
\pgfpathlineto{\pgfqpoint{2.119555in}{0.630307in}}%
\pgfpathlineto{\pgfqpoint{2.122136in}{0.629210in}}%
\pgfpathlineto{\pgfqpoint{2.124459in}{0.627614in}}%
\pgfpathlineto{\pgfqpoint{2.129364in}{0.619615in}}%
\pgfpathlineto{\pgfqpoint{2.132462in}{0.619222in}}%
\pgfpathlineto{\pgfqpoint{2.135817in}{0.610272in}}%
\pgfpathlineto{\pgfqpoint{2.136592in}{0.610806in}}%
\pgfpathlineto{\pgfqpoint{2.138657in}{0.611551in}}%
\pgfpathlineto{\pgfqpoint{2.139689in}{0.612175in}}%
\pgfpathlineto{\pgfqpoint{2.144336in}{0.619213in}}%
\pgfpathlineto{\pgfqpoint{2.145626in}{0.618082in}}%
\pgfpathlineto{\pgfqpoint{2.148724in}{0.613207in}}%
\pgfpathlineto{\pgfqpoint{2.148982in}{0.613402in}}%
\pgfpathlineto{\pgfqpoint{2.152596in}{0.615695in}}%
\pgfpathlineto{\pgfqpoint{2.155694in}{0.615529in}}%
\pgfpathlineto{\pgfqpoint{2.158791in}{0.612671in}}%
\pgfpathlineto{\pgfqpoint{2.164728in}{0.613140in}}%
\pgfpathlineto{\pgfqpoint{2.166277in}{0.612149in}}%
\pgfpathlineto{\pgfqpoint{2.168859in}{0.609038in}}%
\pgfpathlineto{\pgfqpoint{2.169117in}{0.609224in}}%
\pgfpathlineto{\pgfqpoint{2.171956in}{0.610790in}}%
\pgfpathlineto{\pgfqpoint{2.173763in}{0.610844in}}%
\pgfpathlineto{\pgfqpoint{2.175828in}{0.614501in}}%
\pgfpathlineto{\pgfqpoint{2.177635in}{0.615402in}}%
\pgfpathlineto{\pgfqpoint{2.180991in}{0.615464in}}%
\pgfpathlineto{\pgfqpoint{2.182798in}{0.614821in}}%
\pgfpathlineto{\pgfqpoint{2.184605in}{0.614242in}}%
\pgfpathlineto{\pgfqpoint{2.191316in}{0.622291in}}%
\pgfpathlineto{\pgfqpoint{2.192091in}{0.620833in}}%
\pgfpathlineto{\pgfqpoint{2.194414in}{0.616295in}}%
\pgfpathlineto{\pgfqpoint{2.194672in}{0.616400in}}%
\pgfpathlineto{\pgfqpoint{2.196995in}{0.618873in}}%
\pgfpathlineto{\pgfqpoint{2.200609in}{0.623925in}}%
\pgfpathlineto{\pgfqpoint{2.203707in}{0.623471in}}%
\pgfpathlineto{\pgfqpoint{2.205514in}{0.623282in}}%
\pgfpathlineto{\pgfqpoint{2.206804in}{0.623751in}}%
\pgfpathlineto{\pgfqpoint{2.209902in}{0.626547in}}%
\pgfpathlineto{\pgfqpoint{2.213000in}{0.627189in}}%
\pgfpathlineto{\pgfqpoint{2.215839in}{0.628376in}}%
\pgfpathlineto{\pgfqpoint{2.225906in}{0.626359in}}%
\pgfpathlineto{\pgfqpoint{2.229262in}{0.621401in}}%
\pgfpathlineto{\pgfqpoint{2.229520in}{0.621549in}}%
\pgfpathlineto{\pgfqpoint{2.232360in}{0.622222in}}%
\pgfpathlineto{\pgfqpoint{2.234167in}{0.622081in}}%
\pgfpathlineto{\pgfqpoint{2.236490in}{0.624336in}}%
\pgfpathlineto{\pgfqpoint{2.236748in}{0.624064in}}%
\pgfpathlineto{\pgfqpoint{2.240620in}{0.617082in}}%
\pgfpathlineto{\pgfqpoint{2.243201in}{0.613971in}}%
\pgfpathlineto{\pgfqpoint{2.246299in}{0.614850in}}%
\pgfpathlineto{\pgfqpoint{2.247590in}{0.619658in}}%
\pgfpathlineto{\pgfqpoint{2.249397in}{0.625463in}}%
\pgfpathlineto{\pgfqpoint{2.249655in}{0.625177in}}%
\pgfpathlineto{\pgfqpoint{2.251204in}{0.623305in}}%
\pgfpathlineto{\pgfqpoint{2.251462in}{0.623531in}}%
\pgfpathlineto{\pgfqpoint{2.253785in}{0.624872in}}%
\pgfpathlineto{\pgfqpoint{2.257399in}{0.624306in}}%
\pgfpathlineto{\pgfqpoint{2.258689in}{0.623797in}}%
\pgfpathlineto{\pgfqpoint{2.261271in}{0.621553in}}%
\pgfpathlineto{\pgfqpoint{2.273661in}{0.620994in}}%
\pgfpathlineto{\pgfqpoint{2.276243in}{0.620411in}}%
\pgfpathlineto{\pgfqpoint{2.282696in}{0.623260in}}%
\pgfpathlineto{\pgfqpoint{2.285535in}{0.631760in}}%
\pgfpathlineto{\pgfqpoint{2.286052in}{0.631336in}}%
\pgfpathlineto{\pgfqpoint{2.288633in}{0.628975in}}%
\pgfpathlineto{\pgfqpoint{2.290698in}{0.631825in}}%
\pgfpathlineto{\pgfqpoint{2.293538in}{0.634643in}}%
\pgfpathlineto{\pgfqpoint{2.297926in}{0.635881in}}%
\pgfpathlineto{\pgfqpoint{2.299475in}{0.628633in}}%
\pgfpathlineto{\pgfqpoint{2.301282in}{0.622258in}}%
\pgfpathlineto{\pgfqpoint{2.301540in}{0.622461in}}%
\pgfpathlineto{\pgfqpoint{2.304896in}{0.628673in}}%
\pgfpathlineto{\pgfqpoint{2.306703in}{0.631377in}}%
\pgfpathlineto{\pgfqpoint{2.312123in}{0.633404in}}%
\pgfpathlineto{\pgfqpoint{2.316512in}{0.639934in}}%
\pgfpathlineto{\pgfqpoint{2.317028in}{0.639482in}}%
\pgfpathlineto{\pgfqpoint{2.319609in}{0.638184in}}%
\pgfpathlineto{\pgfqpoint{2.323481in}{0.639759in}}%
\pgfpathlineto{\pgfqpoint{2.325805in}{0.642496in}}%
\pgfpathlineto{\pgfqpoint{2.328128in}{0.643995in}}%
\pgfpathlineto{\pgfqpoint{2.333807in}{0.642424in}}%
\pgfpathlineto{\pgfqpoint{2.337937in}{0.633711in}}%
\pgfpathlineto{\pgfqpoint{2.340776in}{0.632940in}}%
\pgfpathlineto{\pgfqpoint{2.344390in}{0.628846in}}%
\pgfpathlineto{\pgfqpoint{2.344907in}{0.629502in}}%
\pgfpathlineto{\pgfqpoint{2.346713in}{0.630814in}}%
\pgfpathlineto{\pgfqpoint{2.346972in}{0.630626in}}%
\pgfpathlineto{\pgfqpoint{2.349811in}{0.629348in}}%
\pgfpathlineto{\pgfqpoint{2.352909in}{0.628875in}}%
\pgfpathlineto{\pgfqpoint{2.354458in}{0.626619in}}%
\pgfpathlineto{\pgfqpoint{2.358846in}{0.617061in}}%
\pgfpathlineto{\pgfqpoint{2.360911in}{0.615836in}}%
\pgfpathlineto{\pgfqpoint{2.364783in}{0.612146in}}%
\pgfpathlineto{\pgfqpoint{2.367364in}{0.609443in}}%
\pgfpathlineto{\pgfqpoint{2.372785in}{0.611772in}}%
\pgfpathlineto{\pgfqpoint{2.376657in}{0.611343in}}%
\pgfpathlineto{\pgfqpoint{2.381562in}{0.616326in}}%
\pgfpathlineto{\pgfqpoint{2.384143in}{0.615565in}}%
\pgfpathlineto{\pgfqpoint{2.388531in}{0.617304in}}%
\pgfpathlineto{\pgfqpoint{2.392403in}{0.624449in}}%
\pgfpathlineto{\pgfqpoint{2.396275in}{0.624262in}}%
\pgfpathlineto{\pgfqpoint{2.398599in}{0.622107in}}%
\pgfpathlineto{\pgfqpoint{2.401954in}{0.623297in}}%
\pgfpathlineto{\pgfqpoint{2.411763in}{0.622598in}}%
\pgfpathlineto{\pgfqpoint{2.416668in}{0.624775in}}%
\pgfpathlineto{\pgfqpoint{2.418991in}{0.623154in}}%
\pgfpathlineto{\pgfqpoint{2.422089in}{0.620426in}}%
\pgfpathlineto{\pgfqpoint{2.425445in}{0.621094in}}%
\pgfpathlineto{\pgfqpoint{2.427252in}{0.621402in}}%
\pgfpathlineto{\pgfqpoint{2.428800in}{0.617588in}}%
\pgfpathlineto{\pgfqpoint{2.431898in}{0.610306in}}%
\pgfpathlineto{\pgfqpoint{2.436028in}{0.610156in}}%
\pgfpathlineto{\pgfqpoint{2.438868in}{0.609213in}}%
\pgfpathlineto{\pgfqpoint{2.440675in}{0.605143in}}%
\pgfpathlineto{\pgfqpoint{2.443256in}{0.600344in}}%
\pgfpathlineto{\pgfqpoint{2.447902in}{0.599282in}}%
\pgfpathlineto{\pgfqpoint{2.452291in}{0.599053in}}%
\pgfpathlineto{\pgfqpoint{2.455388in}{0.599944in}}%
\pgfpathlineto{\pgfqpoint{2.460551in}{0.600637in}}%
\pgfpathlineto{\pgfqpoint{2.463132in}{0.602645in}}%
\pgfpathlineto{\pgfqpoint{2.468037in}{0.601561in}}%
\pgfpathlineto{\pgfqpoint{2.472683in}{0.601412in}}%
\pgfpathlineto{\pgfqpoint{2.475523in}{0.601881in}}%
\pgfpathlineto{\pgfqpoint{2.477072in}{0.601328in}}%
\pgfpathlineto{\pgfqpoint{2.478878in}{0.601335in}}%
\pgfpathlineto{\pgfqpoint{2.482751in}{0.602092in}}%
\pgfpathlineto{\pgfqpoint{2.490753in}{0.602308in}}%
\pgfpathlineto{\pgfqpoint{2.508048in}{0.598401in}}%
\pgfpathlineto{\pgfqpoint{2.510887in}{0.602524in}}%
\pgfpathlineto{\pgfqpoint{2.511403in}{0.601969in}}%
\pgfpathlineto{\pgfqpoint{2.513210in}{0.600499in}}%
\pgfpathlineto{\pgfqpoint{2.513469in}{0.600761in}}%
\pgfpathlineto{\pgfqpoint{2.518115in}{0.606285in}}%
\pgfpathlineto{\pgfqpoint{2.521729in}{0.606202in}}%
\pgfpathlineto{\pgfqpoint{2.526892in}{0.608467in}}%
\pgfpathlineto{\pgfqpoint{2.530247in}{0.609415in}}%
\pgfpathlineto{\pgfqpoint{2.533345in}{0.609005in}}%
\pgfpathlineto{\pgfqpoint{2.538766in}{0.609431in}}%
\pgfpathlineto{\pgfqpoint{2.542122in}{0.612304in}}%
\pgfpathlineto{\pgfqpoint{2.545219in}{0.614711in}}%
\pgfpathlineto{\pgfqpoint{2.547542in}{0.615159in}}%
\pgfpathlineto{\pgfqpoint{2.550898in}{0.616052in}}%
\pgfpathlineto{\pgfqpoint{2.552963in}{0.617586in}}%
\pgfpathlineto{\pgfqpoint{2.556577in}{0.621644in}}%
\pgfpathlineto{\pgfqpoint{2.559158in}{0.622743in}}%
\pgfpathlineto{\pgfqpoint{2.560449in}{0.623143in}}%
\pgfpathlineto{\pgfqpoint{2.560707in}{0.622885in}}%
\pgfpathlineto{\pgfqpoint{2.563805in}{0.620588in}}%
\pgfpathlineto{\pgfqpoint{2.565096in}{0.618958in}}%
\pgfpathlineto{\pgfqpoint{2.568193in}{0.614045in}}%
\pgfpathlineto{\pgfqpoint{2.576453in}{0.615627in}}%
\pgfpathlineto{\pgfqpoint{2.582391in}{0.611953in}}%
\pgfpathlineto{\pgfqpoint{2.590135in}{0.612453in}}%
\pgfpathlineto{\pgfqpoint{2.592458in}{0.613175in}}%
\pgfpathlineto{\pgfqpoint{2.596588in}{0.610532in}}%
\pgfpathlineto{\pgfqpoint{2.601751in}{0.611632in}}%
\pgfpathlineto{\pgfqpoint{2.603816in}{0.614247in}}%
\pgfpathlineto{\pgfqpoint{2.607688in}{0.612780in}}%
\pgfpathlineto{\pgfqpoint{2.610269in}{0.613182in}}%
\pgfpathlineto{\pgfqpoint{2.612334in}{0.612696in}}%
\pgfpathlineto{\pgfqpoint{2.615432in}{0.614886in}}%
\pgfpathlineto{\pgfqpoint{2.618013in}{0.615911in}}%
\pgfpathlineto{\pgfqpoint{2.620078in}{0.619834in}}%
\pgfpathlineto{\pgfqpoint{2.623176in}{0.626001in}}%
\pgfpathlineto{\pgfqpoint{2.624983in}{0.626759in}}%
\pgfpathlineto{\pgfqpoint{2.627306in}{0.627353in}}%
\pgfpathlineto{\pgfqpoint{2.629113in}{0.628253in}}%
\pgfpathlineto{\pgfqpoint{2.630404in}{0.632385in}}%
\pgfpathlineto{\pgfqpoint{2.633501in}{0.642802in}}%
\pgfpathlineto{\pgfqpoint{2.638148in}{0.645950in}}%
\pgfpathlineto{\pgfqpoint{2.640987in}{0.647758in}}%
\pgfpathlineto{\pgfqpoint{2.645375in}{0.656100in}}%
\pgfpathlineto{\pgfqpoint{2.652345in}{0.659163in}}%
\pgfpathlineto{\pgfqpoint{2.655185in}{0.664191in}}%
\pgfpathlineto{\pgfqpoint{2.656733in}{0.666271in}}%
\pgfpathlineto{\pgfqpoint{2.656992in}{0.666101in}}%
\pgfpathlineto{\pgfqpoint{2.659831in}{0.664374in}}%
\pgfpathlineto{\pgfqpoint{2.663187in}{0.666643in}}%
\pgfpathlineto{\pgfqpoint{2.663445in}{0.666197in}}%
\pgfpathlineto{\pgfqpoint{2.666284in}{0.662589in}}%
\pgfpathlineto{\pgfqpoint{2.669640in}{0.662013in}}%
\pgfpathlineto{\pgfqpoint{2.671189in}{0.658687in}}%
\pgfpathlineto{\pgfqpoint{2.675319in}{0.648191in}}%
\pgfpathlineto{\pgfqpoint{2.678675in}{0.646567in}}%
\pgfpathlineto{\pgfqpoint{2.680482in}{0.646161in}}%
\pgfpathlineto{\pgfqpoint{2.681772in}{0.642432in}}%
\pgfpathlineto{\pgfqpoint{2.685128in}{0.630840in}}%
\pgfpathlineto{\pgfqpoint{2.692872in}{0.627776in}}%
\pgfpathlineto{\pgfqpoint{2.696228in}{0.618724in}}%
\pgfpathlineto{\pgfqpoint{2.696744in}{0.619098in}}%
\pgfpathlineto{\pgfqpoint{2.700100in}{0.622761in}}%
\pgfpathlineto{\pgfqpoint{2.703972in}{0.620254in}}%
\pgfpathlineto{\pgfqpoint{2.708618in}{0.607543in}}%
\pgfpathlineto{\pgfqpoint{2.709393in}{0.608078in}}%
\pgfpathlineto{\pgfqpoint{2.711200in}{0.608280in}}%
\pgfpathlineto{\pgfqpoint{2.714814in}{0.605707in}}%
\pgfpathlineto{\pgfqpoint{2.717395in}{0.606449in}}%
\pgfpathlineto{\pgfqpoint{2.720235in}{0.607322in}}%
\pgfpathlineto{\pgfqpoint{2.731592in}{0.610574in}}%
\pgfpathlineto{\pgfqpoint{2.742950in}{0.609777in}}%
\pgfpathlineto{\pgfqpoint{2.746306in}{0.609246in}}%
\pgfpathlineto{\pgfqpoint{2.747855in}{0.611274in}}%
\pgfpathlineto{\pgfqpoint{2.748113in}{0.611032in}}%
\pgfpathlineto{\pgfqpoint{2.750694in}{0.605693in}}%
\pgfpathlineto{\pgfqpoint{2.751469in}{0.606758in}}%
\pgfpathlineto{\pgfqpoint{2.755341in}{0.612233in}}%
\pgfpathlineto{\pgfqpoint{2.766183in}{0.611219in}}%
\pgfpathlineto{\pgfqpoint{2.768506in}{0.609905in}}%
\pgfpathlineto{\pgfqpoint{2.778573in}{0.609262in}}%
\pgfpathlineto{\pgfqpoint{2.780380in}{0.609264in}}%
\pgfpathlineto{\pgfqpoint{2.784510in}{0.611594in}}%
\pgfpathlineto{\pgfqpoint{2.790447in}{0.612215in}}%
\pgfpathlineto{\pgfqpoint{2.793545in}{0.614516in}}%
\pgfpathlineto{\pgfqpoint{2.796642in}{0.616106in}}%
\pgfpathlineto{\pgfqpoint{2.798449in}{0.616140in}}%
\pgfpathlineto{\pgfqpoint{2.800256in}{0.615700in}}%
\pgfpathlineto{\pgfqpoint{2.802321in}{0.616342in}}%
\pgfpathlineto{\pgfqpoint{2.805161in}{0.610607in}}%
\pgfpathlineto{\pgfqpoint{2.807226in}{0.609454in}}%
\pgfpathlineto{\pgfqpoint{2.809549in}{0.609642in}}%
\pgfpathlineto{\pgfqpoint{2.811356in}{0.610030in}}%
\pgfpathlineto{\pgfqpoint{2.814454in}{0.611154in}}%
\pgfpathlineto{\pgfqpoint{2.816261in}{0.611206in}}%
\pgfpathlineto{\pgfqpoint{2.818842in}{0.613021in}}%
\pgfpathlineto{\pgfqpoint{2.822714in}{0.612719in}}%
\pgfpathlineto{\pgfqpoint{2.825295in}{0.612475in}}%
\pgfpathlineto{\pgfqpoint{2.834072in}{0.610974in}}%
\pgfpathlineto{\pgfqpoint{2.836653in}{0.610780in}}%
\pgfpathlineto{\pgfqpoint{2.841816in}{0.611760in}}%
\pgfpathlineto{\pgfqpoint{2.849818in}{0.605705in}}%
\pgfpathlineto{\pgfqpoint{2.854981in}{0.602311in}}%
\pgfpathlineto{\pgfqpoint{2.858595in}{0.603833in}}%
\pgfpathlineto{\pgfqpoint{2.860660in}{0.604257in}}%
\pgfpathlineto{\pgfqpoint{2.863758in}{0.603235in}}%
\pgfpathlineto{\pgfqpoint{2.869178in}{0.599910in}}%
\pgfpathlineto{\pgfqpoint{2.874083in}{0.607160in}}%
\pgfpathlineto{\pgfqpoint{2.876148in}{0.610300in}}%
\pgfpathlineto{\pgfqpoint{2.876922in}{0.611345in}}%
\pgfpathlineto{\pgfqpoint{2.878213in}{0.619963in}}%
\pgfpathlineto{\pgfqpoint{2.880020in}{0.654441in}}%
\pgfpathlineto{\pgfqpoint{2.881827in}{0.678089in}}%
\pgfpathlineto{\pgfqpoint{2.882085in}{0.678039in}}%
\pgfpathlineto{\pgfqpoint{2.883634in}{0.679498in}}%
\pgfpathlineto{\pgfqpoint{2.884150in}{0.688194in}}%
\pgfpathlineto{\pgfqpoint{2.885441in}{0.794259in}}%
\pgfpathlineto{\pgfqpoint{2.887248in}{0.860359in}}%
\pgfpathlineto{\pgfqpoint{2.888022in}{0.906218in}}%
\pgfpathlineto{\pgfqpoint{2.890604in}{1.091857in}}%
\pgfpathlineto{\pgfqpoint{2.892411in}{1.094436in}}%
\pgfpathlineto{\pgfqpoint{2.894476in}{1.099530in}}%
\pgfpathlineto{\pgfqpoint{2.895508in}{1.122606in}}%
\pgfpathlineto{\pgfqpoint{2.897831in}{1.170246in}}%
\pgfpathlineto{\pgfqpoint{2.898348in}{1.173379in}}%
\pgfpathlineto{\pgfqpoint{2.899638in}{1.216037in}}%
\pgfpathlineto{\pgfqpoint{2.901445in}{1.248063in}}%
\pgfpathlineto{\pgfqpoint{2.902994in}{1.248570in}}%
\pgfpathlineto{\pgfqpoint{2.905317in}{1.249225in}}%
\pgfpathlineto{\pgfqpoint{2.906092in}{1.253092in}}%
\pgfpathlineto{\pgfqpoint{2.913836in}{1.319491in}}%
\pgfpathlineto{\pgfqpoint{2.917450in}{1.333252in}}%
\pgfpathlineto{\pgfqpoint{2.918998in}{1.335707in}}%
\pgfpathlineto{\pgfqpoint{2.920031in}{1.337091in}}%
\pgfpathlineto{\pgfqpoint{2.922870in}{1.342349in}}%
\pgfpathlineto{\pgfqpoint{2.923903in}{1.343611in}}%
\pgfpathlineto{\pgfqpoint{2.925710in}{1.345350in}}%
\pgfpathlineto{\pgfqpoint{2.926484in}{1.345848in}}%
\pgfpathlineto{\pgfqpoint{2.928033in}{1.353100in}}%
\pgfpathlineto{\pgfqpoint{2.929324in}{1.356042in}}%
\pgfpathlineto{\pgfqpoint{2.929582in}{1.355864in}}%
\pgfpathlineto{\pgfqpoint{2.930356in}{1.356401in}}%
\pgfpathlineto{\pgfqpoint{2.933712in}{1.364826in}}%
\pgfpathlineto{\pgfqpoint{2.934745in}{1.369232in}}%
\pgfpathlineto{\pgfqpoint{2.936035in}{1.375634in}}%
\pgfpathlineto{\pgfqpoint{2.936552in}{1.373927in}}%
\pgfpathlineto{\pgfqpoint{2.937842in}{1.363925in}}%
\pgfpathlineto{\pgfqpoint{2.938359in}{1.365262in}}%
\pgfpathlineto{\pgfqpoint{2.939133in}{1.369751in}}%
\pgfpathlineto{\pgfqpoint{2.939649in}{1.367611in}}%
\pgfpathlineto{\pgfqpoint{2.941198in}{1.343757in}}%
\pgfpathlineto{\pgfqpoint{2.941972in}{1.348450in}}%
\pgfpathlineto{\pgfqpoint{2.943779in}{1.361668in}}%
\pgfpathlineto{\pgfqpoint{2.944037in}{1.361572in}}%
\pgfpathlineto{\pgfqpoint{2.944554in}{1.361805in}}%
\pgfpathlineto{\pgfqpoint{2.945586in}{1.369047in}}%
\pgfpathlineto{\pgfqpoint{2.946877in}{1.378709in}}%
\pgfpathlineto{\pgfqpoint{2.947393in}{1.377790in}}%
\pgfpathlineto{\pgfqpoint{2.948684in}{1.371645in}}%
\pgfpathlineto{\pgfqpoint{2.949200in}{1.372946in}}%
\pgfpathlineto{\pgfqpoint{2.949975in}{1.375521in}}%
\pgfpathlineto{\pgfqpoint{2.950233in}{1.375062in}}%
\pgfpathlineto{\pgfqpoint{2.951782in}{1.363097in}}%
\pgfpathlineto{\pgfqpoint{2.952556in}{1.368144in}}%
\pgfpathlineto{\pgfqpoint{2.954621in}{1.380403in}}%
\pgfpathlineto{\pgfqpoint{2.955395in}{1.380899in}}%
\pgfpathlineto{\pgfqpoint{2.956428in}{1.387307in}}%
\pgfpathlineto{\pgfqpoint{2.958235in}{1.398666in}}%
\pgfpathlineto{\pgfqpoint{2.958493in}{1.398465in}}%
\pgfpathlineto{\pgfqpoint{2.959784in}{1.396217in}}%
\pgfpathlineto{\pgfqpoint{2.960042in}{1.396655in}}%
\pgfpathlineto{\pgfqpoint{2.961074in}{1.399018in}}%
\pgfpathlineto{\pgfqpoint{2.961591in}{1.398234in}}%
\pgfpathlineto{\pgfqpoint{2.963398in}{1.392457in}}%
\pgfpathlineto{\pgfqpoint{2.963914in}{1.393144in}}%
\pgfpathlineto{\pgfqpoint{2.964946in}{1.393936in}}%
\pgfpathlineto{\pgfqpoint{2.965205in}{1.393734in}}%
\pgfpathlineto{\pgfqpoint{2.966237in}{1.393595in}}%
\pgfpathlineto{\pgfqpoint{2.970625in}{1.400214in}}%
\pgfpathlineto{\pgfqpoint{2.972949in}{1.401410in}}%
\pgfpathlineto{\pgfqpoint{2.975272in}{1.402323in}}%
\pgfpathlineto{\pgfqpoint{2.977853in}{1.401809in}}%
\pgfpathlineto{\pgfqpoint{2.985339in}{1.409283in}}%
\pgfpathlineto{\pgfqpoint{2.986630in}{1.407965in}}%
\pgfpathlineto{\pgfqpoint{2.987920in}{1.405334in}}%
\pgfpathlineto{\pgfqpoint{2.988437in}{1.406077in}}%
\pgfpathlineto{\pgfqpoint{2.989985in}{1.409803in}}%
\pgfpathlineto{\pgfqpoint{2.990244in}{1.409510in}}%
\pgfpathlineto{\pgfqpoint{2.991792in}{1.407565in}}%
\pgfpathlineto{\pgfqpoint{2.992051in}{1.407766in}}%
\pgfpathlineto{\pgfqpoint{2.992825in}{1.407715in}}%
\pgfpathlineto{\pgfqpoint{2.993083in}{1.407197in}}%
\pgfpathlineto{\pgfqpoint{2.999020in}{1.389012in}}%
\pgfpathlineto{\pgfqpoint{2.999795in}{1.390076in}}%
\pgfpathlineto{\pgfqpoint{3.000311in}{1.390152in}}%
\pgfpathlineto{\pgfqpoint{3.000569in}{1.389699in}}%
\pgfpathlineto{\pgfqpoint{3.004183in}{1.376955in}}%
\pgfpathlineto{\pgfqpoint{3.006248in}{1.362784in}}%
\pgfpathlineto{\pgfqpoint{3.007022in}{1.362202in}}%
\pgfpathlineto{\pgfqpoint{3.007797in}{1.356601in}}%
\pgfpathlineto{\pgfqpoint{3.011153in}{1.324859in}}%
\pgfpathlineto{\pgfqpoint{3.018897in}{1.287894in}}%
\pgfpathlineto{\pgfqpoint{3.021736in}{1.260805in}}%
\pgfpathlineto{\pgfqpoint{3.023027in}{1.260206in}}%
\pgfpathlineto{\pgfqpoint{3.024059in}{1.255470in}}%
\pgfpathlineto{\pgfqpoint{3.025608in}{1.242819in}}%
\pgfpathlineto{\pgfqpoint{3.026124in}{1.244347in}}%
\pgfpathlineto{\pgfqpoint{3.028189in}{1.251450in}}%
\pgfpathlineto{\pgfqpoint{3.028964in}{1.253103in}}%
\pgfpathlineto{\pgfqpoint{3.029738in}{1.254877in}}%
\pgfpathlineto{\pgfqpoint{3.029996in}{1.253904in}}%
\pgfpathlineto{\pgfqpoint{3.031029in}{1.237967in}}%
\pgfpathlineto{\pgfqpoint{3.032320in}{1.219210in}}%
\pgfpathlineto{\pgfqpoint{3.032836in}{1.220337in}}%
\pgfpathlineto{\pgfqpoint{3.033094in}{1.220561in}}%
\pgfpathlineto{\pgfqpoint{3.033610in}{1.215128in}}%
\pgfpathlineto{\pgfqpoint{3.037740in}{1.133441in}}%
\pgfpathlineto{\pgfqpoint{3.038515in}{1.127834in}}%
\pgfpathlineto{\pgfqpoint{3.039031in}{1.129132in}}%
\pgfpathlineto{\pgfqpoint{3.039289in}{1.129713in}}%
\pgfpathlineto{\pgfqpoint{3.039547in}{1.128767in}}%
\pgfpathlineto{\pgfqpoint{3.040322in}{1.109688in}}%
\pgfpathlineto{\pgfqpoint{3.042387in}{1.052053in}}%
\pgfpathlineto{\pgfqpoint{3.042645in}{1.052215in}}%
\pgfpathlineto{\pgfqpoint{3.043161in}{1.050968in}}%
\pgfpathlineto{\pgfqpoint{3.044452in}{1.031253in}}%
\pgfpathlineto{\pgfqpoint{3.046517in}{1.010520in}}%
\pgfpathlineto{\pgfqpoint{3.048066in}{1.010276in}}%
\pgfpathlineto{\pgfqpoint{3.049356in}{1.011549in}}%
\pgfpathlineto{\pgfqpoint{3.049615in}{1.010985in}}%
\pgfpathlineto{\pgfqpoint{3.050647in}{0.998228in}}%
\pgfpathlineto{\pgfqpoint{3.052454in}{0.978881in}}%
\pgfpathlineto{\pgfqpoint{3.052712in}{0.979158in}}%
\pgfpathlineto{\pgfqpoint{3.053745in}{0.982007in}}%
\pgfpathlineto{\pgfqpoint{3.054777in}{0.999335in}}%
\pgfpathlineto{\pgfqpoint{3.057101in}{1.030490in}}%
\pgfpathlineto{\pgfqpoint{3.058907in}{1.029509in}}%
\pgfpathlineto{\pgfqpoint{3.059940in}{1.021522in}}%
\pgfpathlineto{\pgfqpoint{3.061747in}{0.999301in}}%
\pgfpathlineto{\pgfqpoint{3.062263in}{1.003349in}}%
\pgfpathlineto{\pgfqpoint{3.066910in}{1.056225in}}%
\pgfpathlineto{\pgfqpoint{3.067942in}{1.053677in}}%
\pgfpathlineto{\pgfqpoint{3.068458in}{1.052859in}}%
\pgfpathlineto{\pgfqpoint{3.068975in}{1.054073in}}%
\pgfpathlineto{\pgfqpoint{3.069749in}{1.056743in}}%
\pgfpathlineto{\pgfqpoint{3.070265in}{1.055455in}}%
\pgfpathlineto{\pgfqpoint{3.072072in}{1.043723in}}%
\pgfpathlineto{\pgfqpoint{3.072589in}{1.045442in}}%
\pgfpathlineto{\pgfqpoint{3.074654in}{1.061596in}}%
\pgfpathlineto{\pgfqpoint{3.075428in}{1.059316in}}%
\pgfpathlineto{\pgfqpoint{3.075686in}{1.058896in}}%
\pgfpathlineto{\pgfqpoint{3.075944in}{1.059261in}}%
\pgfpathlineto{\pgfqpoint{3.077235in}{1.064770in}}%
\pgfpathlineto{\pgfqpoint{3.077493in}{1.063733in}}%
\pgfpathlineto{\pgfqpoint{3.080075in}{1.051982in}}%
\pgfpathlineto{\pgfqpoint{3.080849in}{1.044389in}}%
\pgfpathlineto{\pgfqpoint{3.083430in}{0.998909in}}%
\pgfpathlineto{\pgfqpoint{3.084205in}{0.999295in}}%
\pgfpathlineto{\pgfqpoint{3.085237in}{0.996030in}}%
\pgfpathlineto{\pgfqpoint{3.085753in}{0.997977in}}%
\pgfpathlineto{\pgfqpoint{3.087302in}{1.007500in}}%
\pgfpathlineto{\pgfqpoint{3.087819in}{1.006882in}}%
\pgfpathlineto{\pgfqpoint{3.088851in}{1.005961in}}%
\pgfpathlineto{\pgfqpoint{3.089109in}{1.006336in}}%
\pgfpathlineto{\pgfqpoint{3.090400in}{1.008812in}}%
\pgfpathlineto{\pgfqpoint{3.090658in}{1.008450in}}%
\pgfpathlineto{\pgfqpoint{3.091691in}{1.000736in}}%
\pgfpathlineto{\pgfqpoint{3.094014in}{0.983815in}}%
\pgfpathlineto{\pgfqpoint{3.094788in}{0.987891in}}%
\pgfpathlineto{\pgfqpoint{3.096337in}{1.017433in}}%
\pgfpathlineto{\pgfqpoint{3.098660in}{1.045820in}}%
\pgfpathlineto{\pgfqpoint{3.100467in}{1.051685in}}%
\pgfpathlineto{\pgfqpoint{3.100983in}{1.051180in}}%
\pgfpathlineto{\pgfqpoint{3.102016in}{1.049782in}}%
\pgfpathlineto{\pgfqpoint{3.102274in}{1.050273in}}%
\pgfpathlineto{\pgfqpoint{3.103823in}{1.059923in}}%
\pgfpathlineto{\pgfqpoint{3.105372in}{1.068381in}}%
\pgfpathlineto{\pgfqpoint{3.105630in}{1.067868in}}%
\pgfpathlineto{\pgfqpoint{3.106662in}{1.055667in}}%
\pgfpathlineto{\pgfqpoint{3.108211in}{1.039438in}}%
\pgfpathlineto{\pgfqpoint{3.108469in}{1.039907in}}%
\pgfpathlineto{\pgfqpoint{3.110793in}{1.044323in}}%
\pgfpathlineto{\pgfqpoint{3.111567in}{1.047473in}}%
\pgfpathlineto{\pgfqpoint{3.113116in}{1.058622in}}%
\pgfpathlineto{\pgfqpoint{3.113632in}{1.056975in}}%
\pgfpathlineto{\pgfqpoint{3.115181in}{1.036534in}}%
\pgfpathlineto{\pgfqpoint{3.117246in}{1.011634in}}%
\pgfpathlineto{\pgfqpoint{3.117504in}{1.011970in}}%
\pgfpathlineto{\pgfqpoint{3.119053in}{1.018338in}}%
\pgfpathlineto{\pgfqpoint{3.119827in}{1.017225in}}%
\pgfpathlineto{\pgfqpoint{3.120602in}{1.016096in}}%
\pgfpathlineto{\pgfqpoint{3.120860in}{1.016538in}}%
\pgfpathlineto{\pgfqpoint{3.122151in}{1.026507in}}%
\pgfpathlineto{\pgfqpoint{3.122925in}{1.030270in}}%
\pgfpathlineto{\pgfqpoint{3.123183in}{1.029510in}}%
\pgfpathlineto{\pgfqpoint{3.124732in}{1.010806in}}%
\pgfpathlineto{\pgfqpoint{3.128346in}{0.977086in}}%
\pgfpathlineto{\pgfqpoint{3.131443in}{0.962115in}}%
\pgfpathlineto{\pgfqpoint{3.132734in}{0.961548in}}%
\pgfpathlineto{\pgfqpoint{3.133508in}{0.960480in}}%
\pgfpathlineto{\pgfqpoint{3.133767in}{0.960726in}}%
\pgfpathlineto{\pgfqpoint{3.134799in}{0.966978in}}%
\pgfpathlineto{\pgfqpoint{3.136090in}{0.974520in}}%
\pgfpathlineto{\pgfqpoint{3.136348in}{0.973864in}}%
\pgfpathlineto{\pgfqpoint{3.140220in}{0.957849in}}%
\pgfpathlineto{\pgfqpoint{3.140736in}{0.958108in}}%
\pgfpathlineto{\pgfqpoint{3.142285in}{0.963413in}}%
\pgfpathlineto{\pgfqpoint{3.142801in}{0.962324in}}%
\pgfpathlineto{\pgfqpoint{3.143576in}{0.960166in}}%
\pgfpathlineto{\pgfqpoint{3.144092in}{0.961525in}}%
\pgfpathlineto{\pgfqpoint{3.146157in}{0.979470in}}%
\pgfpathlineto{\pgfqpoint{3.146673in}{0.975226in}}%
\pgfpathlineto{\pgfqpoint{3.148997in}{0.942774in}}%
\pgfpathlineto{\pgfqpoint{3.149513in}{0.943593in}}%
\pgfpathlineto{\pgfqpoint{3.150029in}{0.943443in}}%
\pgfpathlineto{\pgfqpoint{3.151836in}{0.936471in}}%
\pgfpathlineto{\pgfqpoint{3.152610in}{0.938534in}}%
\pgfpathlineto{\pgfqpoint{3.153385in}{0.940672in}}%
\pgfpathlineto{\pgfqpoint{3.153901in}{0.939230in}}%
\pgfpathlineto{\pgfqpoint{3.155966in}{0.919480in}}%
\pgfpathlineto{\pgfqpoint{3.157773in}{0.904088in}}%
\pgfpathlineto{\pgfqpoint{3.158289in}{0.905594in}}%
\pgfpathlineto{\pgfqpoint{3.160096in}{0.915226in}}%
\pgfpathlineto{\pgfqpoint{3.160613in}{0.913545in}}%
\pgfpathlineto{\pgfqpoint{3.161387in}{0.910892in}}%
\pgfpathlineto{\pgfqpoint{3.161645in}{0.911463in}}%
\pgfpathlineto{\pgfqpoint{3.163452in}{0.928008in}}%
\pgfpathlineto{\pgfqpoint{3.164226in}{0.922949in}}%
\pgfpathlineto{\pgfqpoint{3.165775in}{0.911874in}}%
\pgfpathlineto{\pgfqpoint{3.166292in}{0.912133in}}%
\pgfpathlineto{\pgfqpoint{3.167066in}{0.911990in}}%
\pgfpathlineto{\pgfqpoint{3.168615in}{0.907621in}}%
\pgfpathlineto{\pgfqpoint{3.169131in}{0.909228in}}%
\pgfpathlineto{\pgfqpoint{3.171454in}{0.929573in}}%
\pgfpathlineto{\pgfqpoint{3.172487in}{0.927179in}}%
\pgfpathlineto{\pgfqpoint{3.172745in}{0.926983in}}%
\pgfpathlineto{\pgfqpoint{3.173003in}{0.927324in}}%
\pgfpathlineto{\pgfqpoint{3.176359in}{0.937461in}}%
\pgfpathlineto{\pgfqpoint{3.176875in}{0.937308in}}%
\pgfpathlineto{\pgfqpoint{3.180231in}{0.937129in}}%
\pgfpathlineto{\pgfqpoint{3.183587in}{0.944005in}}%
\pgfpathlineto{\pgfqpoint{3.184877in}{0.948680in}}%
\pgfpathlineto{\pgfqpoint{3.185394in}{0.947925in}}%
\pgfpathlineto{\pgfqpoint{3.188491in}{0.930631in}}%
\pgfpathlineto{\pgfqpoint{3.189524in}{0.932947in}}%
\pgfpathlineto{\pgfqpoint{3.190814in}{0.948739in}}%
\pgfpathlineto{\pgfqpoint{3.192105in}{0.962149in}}%
\pgfpathlineto{\pgfqpoint{3.192621in}{0.960133in}}%
\pgfpathlineto{\pgfqpoint{3.193654in}{0.955407in}}%
\pgfpathlineto{\pgfqpoint{3.193912in}{0.956670in}}%
\pgfpathlineto{\pgfqpoint{3.195977in}{0.977882in}}%
\pgfpathlineto{\pgfqpoint{3.196493in}{0.975130in}}%
\pgfpathlineto{\pgfqpoint{3.198042in}{0.964320in}}%
\pgfpathlineto{\pgfqpoint{3.198558in}{0.964965in}}%
\pgfpathlineto{\pgfqpoint{3.199075in}{0.965992in}}%
\pgfpathlineto{\pgfqpoint{3.199591in}{0.964912in}}%
\pgfpathlineto{\pgfqpoint{3.201914in}{0.943732in}}%
\pgfpathlineto{\pgfqpoint{3.202689in}{0.947070in}}%
\pgfpathlineto{\pgfqpoint{3.204496in}{0.978721in}}%
\pgfpathlineto{\pgfqpoint{3.206561in}{0.996869in}}%
\pgfpathlineto{\pgfqpoint{3.208109in}{0.997367in}}%
\pgfpathlineto{\pgfqpoint{3.208884in}{0.998098in}}%
\pgfpathlineto{\pgfqpoint{3.209142in}{0.997526in}}%
\pgfpathlineto{\pgfqpoint{3.210433in}{0.986539in}}%
\pgfpathlineto{\pgfqpoint{3.212498in}{0.974431in}}%
\pgfpathlineto{\pgfqpoint{3.213272in}{0.973371in}}%
\pgfpathlineto{\pgfqpoint{3.214563in}{0.959970in}}%
\pgfpathlineto{\pgfqpoint{3.215337in}{0.953677in}}%
\pgfpathlineto{\pgfqpoint{3.215853in}{0.954880in}}%
\pgfpathlineto{\pgfqpoint{3.217919in}{0.967628in}}%
\pgfpathlineto{\pgfqpoint{3.218435in}{0.967141in}}%
\pgfpathlineto{\pgfqpoint{3.219209in}{0.967149in}}%
\pgfpathlineto{\pgfqpoint{3.219467in}{0.967575in}}%
\pgfpathlineto{\pgfqpoint{3.220242in}{0.968568in}}%
\pgfpathlineto{\pgfqpoint{3.220500in}{0.968232in}}%
\pgfpathlineto{\pgfqpoint{3.222307in}{0.959587in}}%
\pgfpathlineto{\pgfqpoint{3.223339in}{0.962372in}}%
\pgfpathlineto{\pgfqpoint{3.223597in}{0.962883in}}%
\pgfpathlineto{\pgfqpoint{3.224114in}{0.961446in}}%
\pgfpathlineto{\pgfqpoint{3.225663in}{0.939050in}}%
\pgfpathlineto{\pgfqpoint{3.227470in}{0.925554in}}%
\pgfpathlineto{\pgfqpoint{3.228244in}{0.926156in}}%
\pgfpathlineto{\pgfqpoint{3.230051in}{0.932389in}}%
\pgfpathlineto{\pgfqpoint{3.231858in}{0.935596in}}%
\pgfpathlineto{\pgfqpoint{3.233148in}{0.937380in}}%
\pgfpathlineto{\pgfqpoint{3.234181in}{0.938547in}}%
\pgfpathlineto{\pgfqpoint{3.234439in}{0.938225in}}%
\pgfpathlineto{\pgfqpoint{3.236762in}{0.930986in}}%
\pgfpathlineto{\pgfqpoint{3.237537in}{0.933338in}}%
\pgfpathlineto{\pgfqpoint{3.240893in}{0.953617in}}%
\pgfpathlineto{\pgfqpoint{3.241409in}{0.950404in}}%
\pgfpathlineto{\pgfqpoint{3.243732in}{0.921590in}}%
\pgfpathlineto{\pgfqpoint{3.244506in}{0.922546in}}%
\pgfpathlineto{\pgfqpoint{3.245023in}{0.921571in}}%
\pgfpathlineto{\pgfqpoint{3.246313in}{0.904801in}}%
\pgfpathlineto{\pgfqpoint{3.248120in}{0.888699in}}%
\pgfpathlineto{\pgfqpoint{3.249153in}{0.893643in}}%
\pgfpathlineto{\pgfqpoint{3.252509in}{0.926890in}}%
\pgfpathlineto{\pgfqpoint{3.253799in}{0.938421in}}%
\pgfpathlineto{\pgfqpoint{3.254316in}{0.936335in}}%
\pgfpathlineto{\pgfqpoint{3.255864in}{0.904205in}}%
\pgfpathlineto{\pgfqpoint{3.257413in}{0.885732in}}%
\pgfpathlineto{\pgfqpoint{3.257671in}{0.886148in}}%
\pgfpathlineto{\pgfqpoint{3.260769in}{0.901260in}}%
\pgfpathlineto{\pgfqpoint{3.265415in}{0.931940in}}%
\pgfpathlineto{\pgfqpoint{3.267222in}{0.933435in}}%
\pgfpathlineto{\pgfqpoint{3.267480in}{0.932849in}}%
\pgfpathlineto{\pgfqpoint{3.269287in}{0.926118in}}%
\pgfpathlineto{\pgfqpoint{3.270062in}{0.927556in}}%
\pgfpathlineto{\pgfqpoint{3.271094in}{0.930167in}}%
\pgfpathlineto{\pgfqpoint{3.271352in}{0.929854in}}%
\pgfpathlineto{\pgfqpoint{3.272385in}{0.922488in}}%
\pgfpathlineto{\pgfqpoint{3.275741in}{0.894002in}}%
\pgfpathlineto{\pgfqpoint{3.277548in}{0.896278in}}%
\pgfpathlineto{\pgfqpoint{3.279613in}{0.898078in}}%
\pgfpathlineto{\pgfqpoint{3.280645in}{0.895745in}}%
\pgfpathlineto{\pgfqpoint{3.286066in}{0.878818in}}%
\pgfpathlineto{\pgfqpoint{3.287873in}{0.876976in}}%
\pgfpathlineto{\pgfqpoint{3.288647in}{0.876179in}}%
\pgfpathlineto{\pgfqpoint{3.289938in}{0.868823in}}%
\pgfpathlineto{\pgfqpoint{3.294068in}{0.839568in}}%
\pgfpathlineto{\pgfqpoint{3.296650in}{0.834423in}}%
\pgfpathlineto{\pgfqpoint{3.296908in}{0.834670in}}%
\pgfpathlineto{\pgfqpoint{3.298457in}{0.836314in}}%
\pgfpathlineto{\pgfqpoint{3.298715in}{0.835875in}}%
\pgfpathlineto{\pgfqpoint{3.300005in}{0.828997in}}%
\pgfpathlineto{\pgfqpoint{3.302329in}{0.803220in}}%
\pgfpathlineto{\pgfqpoint{3.306201in}{0.749814in}}%
\pgfpathlineto{\pgfqpoint{3.306459in}{0.749851in}}%
\pgfpathlineto{\pgfqpoint{3.307491in}{0.749259in}}%
\pgfpathlineto{\pgfqpoint{3.309556in}{0.742881in}}%
\pgfpathlineto{\pgfqpoint{3.311880in}{0.726593in}}%
\pgfpathlineto{\pgfqpoint{3.314461in}{0.685033in}}%
\pgfpathlineto{\pgfqpoint{3.316268in}{0.675394in}}%
\pgfpathlineto{\pgfqpoint{3.317300in}{0.674552in}}%
\pgfpathlineto{\pgfqpoint{3.318333in}{0.673302in}}%
\pgfpathlineto{\pgfqpoint{3.318591in}{0.673829in}}%
\pgfpathlineto{\pgfqpoint{3.319882in}{0.683011in}}%
\pgfpathlineto{\pgfqpoint{3.321689in}{0.692522in}}%
\pgfpathlineto{\pgfqpoint{3.321947in}{0.692190in}}%
\pgfpathlineto{\pgfqpoint{3.323754in}{0.686002in}}%
\pgfpathlineto{\pgfqpoint{3.324270in}{0.687046in}}%
\pgfpathlineto{\pgfqpoint{3.327110in}{0.703437in}}%
\pgfpathlineto{\pgfqpoint{3.328142in}{0.702238in}}%
\pgfpathlineto{\pgfqpoint{3.330982in}{0.697084in}}%
\pgfpathlineto{\pgfqpoint{3.331498in}{0.697909in}}%
\pgfpathlineto{\pgfqpoint{3.334337in}{0.704862in}}%
\pgfpathlineto{\pgfqpoint{3.334595in}{0.704756in}}%
\pgfpathlineto{\pgfqpoint{3.336661in}{0.705120in}}%
\pgfpathlineto{\pgfqpoint{3.340533in}{0.706413in}}%
\pgfpathlineto{\pgfqpoint{3.343630in}{0.705708in}}%
\pgfpathlineto{\pgfqpoint{3.356021in}{0.707828in}}%
\pgfpathlineto{\pgfqpoint{3.358344in}{0.710326in}}%
\pgfpathlineto{\pgfqpoint{3.358602in}{0.710116in}}%
\pgfpathlineto{\pgfqpoint{3.359635in}{0.706021in}}%
\pgfpathlineto{\pgfqpoint{3.361958in}{0.695978in}}%
\pgfpathlineto{\pgfqpoint{3.362216in}{0.696208in}}%
\pgfpathlineto{\pgfqpoint{3.365572in}{0.699237in}}%
\pgfpathlineto{\pgfqpoint{3.367120in}{0.701003in}}%
\pgfpathlineto{\pgfqpoint{3.368669in}{0.702547in}}%
\pgfpathlineto{\pgfqpoint{3.368927in}{0.702279in}}%
\pgfpathlineto{\pgfqpoint{3.369960in}{0.697144in}}%
\pgfpathlineto{\pgfqpoint{3.375123in}{0.655234in}}%
\pgfpathlineto{\pgfqpoint{3.375897in}{0.652075in}}%
\pgfpathlineto{\pgfqpoint{3.378478in}{0.635634in}}%
\pgfpathlineto{\pgfqpoint{3.378995in}{0.635873in}}%
\pgfpathlineto{\pgfqpoint{3.381576in}{0.636260in}}%
\pgfpathlineto{\pgfqpoint{3.382609in}{0.635556in}}%
\pgfpathlineto{\pgfqpoint{3.386739in}{0.628539in}}%
\pgfpathlineto{\pgfqpoint{3.391901in}{0.626979in}}%
\pgfpathlineto{\pgfqpoint{3.393450in}{0.632801in}}%
\pgfpathlineto{\pgfqpoint{3.395773in}{0.638189in}}%
\pgfpathlineto{\pgfqpoint{3.400936in}{0.636086in}}%
\pgfpathlineto{\pgfqpoint{3.402485in}{0.635508in}}%
\pgfpathlineto{\pgfqpoint{3.404034in}{0.635211in}}%
\pgfpathlineto{\pgfqpoint{3.405324in}{0.635272in}}%
\pgfpathlineto{\pgfqpoint{3.406873in}{0.636080in}}%
\pgfpathlineto{\pgfqpoint{3.407131in}{0.635755in}}%
\pgfpathlineto{\pgfqpoint{3.409196in}{0.633924in}}%
\pgfpathlineto{\pgfqpoint{3.412294in}{0.634902in}}%
\pgfpathlineto{\pgfqpoint{3.413843in}{0.633498in}}%
\pgfpathlineto{\pgfqpoint{3.415134in}{0.632571in}}%
\pgfpathlineto{\pgfqpoint{3.415392in}{0.632911in}}%
\pgfpathlineto{\pgfqpoint{3.417457in}{0.637319in}}%
\pgfpathlineto{\pgfqpoint{3.417973in}{0.636647in}}%
\pgfpathlineto{\pgfqpoint{3.419522in}{0.634612in}}%
\pgfpathlineto{\pgfqpoint{3.419780in}{0.634878in}}%
\pgfpathlineto{\pgfqpoint{3.421845in}{0.637293in}}%
\pgfpathlineto{\pgfqpoint{3.422103in}{0.637109in}}%
\pgfpathlineto{\pgfqpoint{3.425975in}{0.633145in}}%
\pgfpathlineto{\pgfqpoint{3.426491in}{0.633941in}}%
\pgfpathlineto{\pgfqpoint{3.428815in}{0.638664in}}%
\pgfpathlineto{\pgfqpoint{3.429073in}{0.638397in}}%
\pgfpathlineto{\pgfqpoint{3.430880in}{0.636560in}}%
\pgfpathlineto{\pgfqpoint{3.431138in}{0.636763in}}%
\pgfpathlineto{\pgfqpoint{3.433461in}{0.637727in}}%
\pgfpathlineto{\pgfqpoint{3.438108in}{0.638189in}}%
\pgfpathlineto{\pgfqpoint{3.440689in}{0.640868in}}%
\pgfpathlineto{\pgfqpoint{3.442238in}{0.641041in}}%
\pgfpathlineto{\pgfqpoint{3.444561in}{0.645208in}}%
\pgfpathlineto{\pgfqpoint{3.445077in}{0.644770in}}%
\pgfpathlineto{\pgfqpoint{3.447659in}{0.641370in}}%
\pgfpathlineto{\pgfqpoint{3.447917in}{0.641690in}}%
\pgfpathlineto{\pgfqpoint{3.450756in}{0.646865in}}%
\pgfpathlineto{\pgfqpoint{3.451531in}{0.646378in}}%
\pgfpathlineto{\pgfqpoint{3.453079in}{0.646898in}}%
\pgfpathlineto{\pgfqpoint{3.455144in}{0.647567in}}%
\pgfpathlineto{\pgfqpoint{3.459016in}{0.645070in}}%
\pgfpathlineto{\pgfqpoint{3.459275in}{0.645568in}}%
\pgfpathlineto{\pgfqpoint{3.461598in}{0.653876in}}%
\pgfpathlineto{\pgfqpoint{3.463663in}{0.658431in}}%
\pgfpathlineto{\pgfqpoint{3.465212in}{0.659141in}}%
\pgfpathlineto{\pgfqpoint{3.466761in}{0.660471in}}%
\pgfpathlineto{\pgfqpoint{3.467019in}{0.660219in}}%
\pgfpathlineto{\pgfqpoint{3.468826in}{0.658395in}}%
\pgfpathlineto{\pgfqpoint{3.469084in}{0.658665in}}%
\pgfpathlineto{\pgfqpoint{3.470891in}{0.660177in}}%
\pgfpathlineto{\pgfqpoint{3.471149in}{0.659875in}}%
\pgfpathlineto{\pgfqpoint{3.473730in}{0.657441in}}%
\pgfpathlineto{\pgfqpoint{3.478635in}{0.658905in}}%
\pgfpathlineto{\pgfqpoint{3.481216in}{0.658296in}}%
\pgfpathlineto{\pgfqpoint{3.484056in}{0.660110in}}%
\pgfpathlineto{\pgfqpoint{3.486121in}{0.661441in}}%
\pgfpathlineto{\pgfqpoint{3.487153in}{0.662731in}}%
\pgfpathlineto{\pgfqpoint{3.489735in}{0.667444in}}%
\pgfpathlineto{\pgfqpoint{3.489993in}{0.667251in}}%
\pgfpathlineto{\pgfqpoint{3.493090in}{0.665481in}}%
\pgfpathlineto{\pgfqpoint{3.494381in}{0.664033in}}%
\pgfpathlineto{\pgfqpoint{3.498253in}{0.653162in}}%
\pgfpathlineto{\pgfqpoint{3.499286in}{0.653617in}}%
\pgfpathlineto{\pgfqpoint{3.500318in}{0.652137in}}%
\pgfpathlineto{\pgfqpoint{3.502383in}{0.649494in}}%
\pgfpathlineto{\pgfqpoint{3.503932in}{0.649281in}}%
\pgfpathlineto{\pgfqpoint{3.507546in}{0.646267in}}%
\pgfpathlineto{\pgfqpoint{3.510643in}{0.646270in}}%
\pgfpathlineto{\pgfqpoint{3.511676in}{0.643803in}}%
\pgfpathlineto{\pgfqpoint{3.515806in}{0.630029in}}%
\pgfpathlineto{\pgfqpoint{3.517097in}{0.628882in}}%
\pgfpathlineto{\pgfqpoint{3.519420in}{0.626800in}}%
\pgfpathlineto{\pgfqpoint{3.522776in}{0.628398in}}%
\pgfpathlineto{\pgfqpoint{3.527422in}{0.627200in}}%
\pgfpathlineto{\pgfqpoint{3.529745in}{0.627713in}}%
\pgfpathlineto{\pgfqpoint{3.533359in}{0.624849in}}%
\pgfpathlineto{\pgfqpoint{3.534908in}{0.622950in}}%
\pgfpathlineto{\pgfqpoint{3.537489in}{0.620083in}}%
\pgfpathlineto{\pgfqpoint{3.539038in}{0.619282in}}%
\pgfpathlineto{\pgfqpoint{3.543427in}{0.614304in}}%
\pgfpathlineto{\pgfqpoint{3.551171in}{0.612935in}}%
\pgfpathlineto{\pgfqpoint{3.555043in}{0.611222in}}%
\pgfpathlineto{\pgfqpoint{3.560980in}{0.614361in}}%
\pgfpathlineto{\pgfqpoint{3.562787in}{0.613994in}}%
\pgfpathlineto{\pgfqpoint{3.565626in}{0.613171in}}%
\pgfpathlineto{\pgfqpoint{3.570531in}{0.612085in}}%
\pgfpathlineto{\pgfqpoint{3.574403in}{0.604545in}}%
\pgfpathlineto{\pgfqpoint{3.575435in}{0.605370in}}%
\pgfpathlineto{\pgfqpoint{3.579307in}{0.608806in}}%
\pgfpathlineto{\pgfqpoint{3.579565in}{0.608566in}}%
\pgfpathlineto{\pgfqpoint{3.584212in}{0.604799in}}%
\pgfpathlineto{\pgfqpoint{3.587826in}{0.605154in}}%
\pgfpathlineto{\pgfqpoint{3.590665in}{0.610428in}}%
\pgfpathlineto{\pgfqpoint{3.591440in}{0.609409in}}%
\pgfpathlineto{\pgfqpoint{3.593505in}{0.608004in}}%
\pgfpathlineto{\pgfqpoint{3.596344in}{0.609614in}}%
\pgfpathlineto{\pgfqpoint{3.597635in}{0.613391in}}%
\pgfpathlineto{\pgfqpoint{3.601249in}{0.626506in}}%
\pgfpathlineto{\pgfqpoint{3.605121in}{0.627433in}}%
\pgfpathlineto{\pgfqpoint{3.606928in}{0.627378in}}%
\pgfpathlineto{\pgfqpoint{3.610283in}{0.625007in}}%
\pgfpathlineto{\pgfqpoint{3.612090in}{0.627922in}}%
\pgfpathlineto{\pgfqpoint{3.615704in}{0.634619in}}%
\pgfpathlineto{\pgfqpoint{3.620093in}{0.636555in}}%
\pgfpathlineto{\pgfqpoint{3.627062in}{0.648739in}}%
\pgfpathlineto{\pgfqpoint{3.630676in}{0.646323in}}%
\pgfpathlineto{\pgfqpoint{3.630934in}{0.646643in}}%
\pgfpathlineto{\pgfqpoint{3.638162in}{0.654358in}}%
\pgfpathlineto{\pgfqpoint{3.639453in}{0.653386in}}%
\pgfpathlineto{\pgfqpoint{3.643583in}{0.646472in}}%
\pgfpathlineto{\pgfqpoint{3.643841in}{0.646696in}}%
\pgfpathlineto{\pgfqpoint{3.645132in}{0.650037in}}%
\pgfpathlineto{\pgfqpoint{3.648229in}{0.661006in}}%
\pgfpathlineto{\pgfqpoint{3.648746in}{0.660421in}}%
\pgfpathlineto{\pgfqpoint{3.650294in}{0.653533in}}%
\pgfpathlineto{\pgfqpoint{3.652618in}{0.646436in}}%
\pgfpathlineto{\pgfqpoint{3.654166in}{0.648078in}}%
\pgfpathlineto{\pgfqpoint{3.659071in}{0.657053in}}%
\pgfpathlineto{\pgfqpoint{3.661910in}{0.666273in}}%
\pgfpathlineto{\pgfqpoint{3.662427in}{0.665840in}}%
\pgfpathlineto{\pgfqpoint{3.663201in}{0.665762in}}%
\pgfpathlineto{\pgfqpoint{3.664234in}{0.671151in}}%
\pgfpathlineto{\pgfqpoint{3.666815in}{0.684980in}}%
\pgfpathlineto{\pgfqpoint{3.669654in}{0.685143in}}%
\pgfpathlineto{\pgfqpoint{3.671461in}{0.684568in}}%
\pgfpathlineto{\pgfqpoint{3.673268in}{0.681880in}}%
\pgfpathlineto{\pgfqpoint{3.673785in}{0.682757in}}%
\pgfpathlineto{\pgfqpoint{3.676366in}{0.690779in}}%
\pgfpathlineto{\pgfqpoint{3.677140in}{0.689894in}}%
\pgfpathlineto{\pgfqpoint{3.678173in}{0.689248in}}%
\pgfpathlineto{\pgfqpoint{3.678431in}{0.689544in}}%
\pgfpathlineto{\pgfqpoint{3.680496in}{0.695676in}}%
\pgfpathlineto{\pgfqpoint{3.683078in}{0.700108in}}%
\pgfpathlineto{\pgfqpoint{3.684626in}{0.698648in}}%
\pgfpathlineto{\pgfqpoint{3.688240in}{0.695143in}}%
\pgfpathlineto{\pgfqpoint{3.689273in}{0.698102in}}%
\pgfpathlineto{\pgfqpoint{3.693403in}{0.723110in}}%
\pgfpathlineto{\pgfqpoint{3.697275in}{0.765985in}}%
\pgfpathlineto{\pgfqpoint{3.697533in}{0.765719in}}%
\pgfpathlineto{\pgfqpoint{3.698824in}{0.761734in}}%
\pgfpathlineto{\pgfqpoint{3.699082in}{0.762602in}}%
\pgfpathlineto{\pgfqpoint{3.699856in}{0.778213in}}%
\pgfpathlineto{\pgfqpoint{3.704503in}{0.945611in}}%
\pgfpathlineto{\pgfqpoint{3.705019in}{0.945382in}}%
\pgfpathlineto{\pgfqpoint{3.705793in}{0.946800in}}%
\pgfpathlineto{\pgfqpoint{3.707084in}{0.949503in}}%
\pgfpathlineto{\pgfqpoint{3.707342in}{0.949230in}}%
\pgfpathlineto{\pgfqpoint{3.707858in}{0.949053in}}%
\pgfpathlineto{\pgfqpoint{3.708117in}{0.949633in}}%
\pgfpathlineto{\pgfqpoint{3.711730in}{0.969475in}}%
\pgfpathlineto{\pgfqpoint{3.715086in}{1.024771in}}%
\pgfpathlineto{\pgfqpoint{3.720249in}{1.043595in}}%
\pgfpathlineto{\pgfqpoint{3.724895in}{1.096126in}}%
\pgfpathlineto{\pgfqpoint{3.727477in}{1.094104in}}%
\pgfpathlineto{\pgfqpoint{3.728251in}{1.093717in}}%
\pgfpathlineto{\pgfqpoint{3.728509in}{1.094297in}}%
\pgfpathlineto{\pgfqpoint{3.729800in}{1.105134in}}%
\pgfpathlineto{\pgfqpoint{3.731349in}{1.114333in}}%
\pgfpathlineto{\pgfqpoint{3.731607in}{1.114173in}}%
\pgfpathlineto{\pgfqpoint{3.732898in}{1.113263in}}%
\pgfpathlineto{\pgfqpoint{3.733156in}{1.113548in}}%
\pgfpathlineto{\pgfqpoint{3.735479in}{1.115304in}}%
\pgfpathlineto{\pgfqpoint{3.736253in}{1.115959in}}%
\pgfpathlineto{\pgfqpoint{3.737802in}{1.122969in}}%
\pgfpathlineto{\pgfqpoint{3.739609in}{1.127112in}}%
\pgfpathlineto{\pgfqpoint{3.741674in}{1.124817in}}%
\pgfpathlineto{\pgfqpoint{3.747869in}{1.110579in}}%
\pgfpathlineto{\pgfqpoint{3.749934in}{1.106789in}}%
\pgfpathlineto{\pgfqpoint{3.750709in}{1.105619in}}%
\pgfpathlineto{\pgfqpoint{3.751483in}{1.097207in}}%
\pgfpathlineto{\pgfqpoint{3.753290in}{1.036243in}}%
\pgfpathlineto{\pgfqpoint{3.755355in}{0.989636in}}%
\pgfpathlineto{\pgfqpoint{3.755872in}{0.990499in}}%
\pgfpathlineto{\pgfqpoint{3.756904in}{0.993498in}}%
\pgfpathlineto{\pgfqpoint{3.757162in}{0.993054in}}%
\pgfpathlineto{\pgfqpoint{3.758711in}{0.989045in}}%
\pgfpathlineto{\pgfqpoint{3.759227in}{0.989461in}}%
\pgfpathlineto{\pgfqpoint{3.759744in}{0.988494in}}%
\pgfpathlineto{\pgfqpoint{3.761292in}{0.979816in}}%
\pgfpathlineto{\pgfqpoint{3.762067in}{0.981928in}}%
\pgfpathlineto{\pgfqpoint{3.762325in}{0.982352in}}%
\pgfpathlineto{\pgfqpoint{3.762583in}{0.981686in}}%
\pgfpathlineto{\pgfqpoint{3.763357in}{0.969875in}}%
\pgfpathlineto{\pgfqpoint{3.766455in}{0.913816in}}%
\pgfpathlineto{\pgfqpoint{3.769553in}{0.892405in}}%
\pgfpathlineto{\pgfqpoint{3.769811in}{0.892497in}}%
\pgfpathlineto{\pgfqpoint{3.770843in}{0.893248in}}%
\pgfpathlineto{\pgfqpoint{3.771101in}{0.892584in}}%
\pgfpathlineto{\pgfqpoint{3.771876in}{0.885080in}}%
\pgfpathlineto{\pgfqpoint{3.776780in}{0.812225in}}%
\pgfpathlineto{\pgfqpoint{3.778846in}{0.808361in}}%
\pgfpathlineto{\pgfqpoint{3.779362in}{0.809603in}}%
\pgfpathlineto{\pgfqpoint{3.779878in}{0.810611in}}%
\pgfpathlineto{\pgfqpoint{3.780136in}{0.810136in}}%
\pgfpathlineto{\pgfqpoint{3.780911in}{0.801524in}}%
\pgfpathlineto{\pgfqpoint{3.782976in}{0.768874in}}%
\pgfpathlineto{\pgfqpoint{3.783492in}{0.769624in}}%
\pgfpathlineto{\pgfqpoint{3.786331in}{0.781965in}}%
\pgfpathlineto{\pgfqpoint{3.788138in}{0.797642in}}%
\pgfpathlineto{\pgfqpoint{3.788397in}{0.796981in}}%
\pgfpathlineto{\pgfqpoint{3.790720in}{0.779947in}}%
\pgfpathlineto{\pgfqpoint{3.791494in}{0.782676in}}%
\pgfpathlineto{\pgfqpoint{3.793817in}{0.789962in}}%
\pgfpathlineto{\pgfqpoint{3.795108in}{0.789348in}}%
\pgfpathlineto{\pgfqpoint{3.796657in}{0.788054in}}%
\pgfpathlineto{\pgfqpoint{3.797948in}{0.774275in}}%
\pgfpathlineto{\pgfqpoint{3.799238in}{0.765550in}}%
\pgfpathlineto{\pgfqpoint{3.799496in}{0.765841in}}%
\pgfpathlineto{\pgfqpoint{3.801820in}{0.774560in}}%
\pgfpathlineto{\pgfqpoint{3.803368in}{0.778722in}}%
\pgfpathlineto{\pgfqpoint{3.803626in}{0.778582in}}%
\pgfpathlineto{\pgfqpoint{3.804659in}{0.777770in}}%
\pgfpathlineto{\pgfqpoint{3.804917in}{0.778164in}}%
\pgfpathlineto{\pgfqpoint{3.806724in}{0.784624in}}%
\pgfpathlineto{\pgfqpoint{3.807498in}{0.782735in}}%
\pgfpathlineto{\pgfqpoint{3.809305in}{0.779809in}}%
\pgfpathlineto{\pgfqpoint{3.810596in}{0.778723in}}%
\pgfpathlineto{\pgfqpoint{3.811629in}{0.779015in}}%
\pgfpathlineto{\pgfqpoint{3.812403in}{0.779545in}}%
\pgfpathlineto{\pgfqpoint{3.812661in}{0.779226in}}%
\pgfpathlineto{\pgfqpoint{3.814468in}{0.774314in}}%
\pgfpathlineto{\pgfqpoint{3.815243in}{0.775727in}}%
\pgfpathlineto{\pgfqpoint{3.815759in}{0.776557in}}%
\pgfpathlineto{\pgfqpoint{3.816275in}{0.775751in}}%
\pgfpathlineto{\pgfqpoint{3.818082in}{0.767653in}}%
\pgfpathlineto{\pgfqpoint{3.818598in}{0.769947in}}%
\pgfpathlineto{\pgfqpoint{3.823245in}{0.798848in}}%
\pgfpathlineto{\pgfqpoint{3.824019in}{0.798037in}}%
\pgfpathlineto{\pgfqpoint{3.828666in}{0.786822in}}%
\pgfpathlineto{\pgfqpoint{3.829440in}{0.789718in}}%
\pgfpathlineto{\pgfqpoint{3.830989in}{0.813022in}}%
\pgfpathlineto{\pgfqpoint{3.833570in}{0.845043in}}%
\pgfpathlineto{\pgfqpoint{3.834345in}{0.845822in}}%
\pgfpathlineto{\pgfqpoint{3.834603in}{0.845590in}}%
\pgfpathlineto{\pgfqpoint{3.835893in}{0.842187in}}%
\pgfpathlineto{\pgfqpoint{3.837442in}{0.832018in}}%
\pgfpathlineto{\pgfqpoint{3.840023in}{0.808509in}}%
\pgfpathlineto{\pgfqpoint{3.840540in}{0.809500in}}%
\pgfpathlineto{\pgfqpoint{3.841830in}{0.825031in}}%
\pgfpathlineto{\pgfqpoint{3.844412in}{0.854624in}}%
\pgfpathlineto{\pgfqpoint{3.846219in}{0.855986in}}%
\pgfpathlineto{\pgfqpoint{3.849833in}{0.855370in}}%
\pgfpathlineto{\pgfqpoint{3.851123in}{0.853985in}}%
\pgfpathlineto{\pgfqpoint{3.851381in}{0.854337in}}%
\pgfpathlineto{\pgfqpoint{3.852414in}{0.859554in}}%
\pgfpathlineto{\pgfqpoint{3.854995in}{0.871623in}}%
\pgfpathlineto{\pgfqpoint{3.856286in}{0.872047in}}%
\pgfpathlineto{\pgfqpoint{3.856544in}{0.871691in}}%
\pgfpathlineto{\pgfqpoint{3.857835in}{0.865293in}}%
\pgfpathlineto{\pgfqpoint{3.860158in}{0.855526in}}%
\pgfpathlineto{\pgfqpoint{3.866611in}{0.852581in}}%
\pgfpathlineto{\pgfqpoint{3.867902in}{0.850700in}}%
\pgfpathlineto{\pgfqpoint{3.868160in}{0.850952in}}%
\pgfpathlineto{\pgfqpoint{3.869709in}{0.854288in}}%
\pgfpathlineto{\pgfqpoint{3.869967in}{0.853745in}}%
\pgfpathlineto{\pgfqpoint{3.871000in}{0.845030in}}%
\pgfpathlineto{\pgfqpoint{3.874614in}{0.808383in}}%
\pgfpathlineto{\pgfqpoint{3.878227in}{0.806167in}}%
\pgfpathlineto{\pgfqpoint{3.879002in}{0.806567in}}%
\pgfpathlineto{\pgfqpoint{3.880551in}{0.811310in}}%
\pgfpathlineto{\pgfqpoint{3.881067in}{0.810016in}}%
\pgfpathlineto{\pgfqpoint{3.882099in}{0.795589in}}%
\pgfpathlineto{\pgfqpoint{3.884165in}{0.764841in}}%
\pgfpathlineto{\pgfqpoint{3.884423in}{0.765466in}}%
\pgfpathlineto{\pgfqpoint{3.887778in}{0.778565in}}%
\pgfpathlineto{\pgfqpoint{3.890360in}{0.778992in}}%
\pgfpathlineto{\pgfqpoint{3.891392in}{0.778508in}}%
\pgfpathlineto{\pgfqpoint{3.892683in}{0.771959in}}%
\pgfpathlineto{\pgfqpoint{3.894490in}{0.746437in}}%
\pgfpathlineto{\pgfqpoint{3.896813in}{0.724106in}}%
\pgfpathlineto{\pgfqpoint{3.897846in}{0.724428in}}%
\pgfpathlineto{\pgfqpoint{3.900685in}{0.726478in}}%
\pgfpathlineto{\pgfqpoint{3.901718in}{0.727463in}}%
\pgfpathlineto{\pgfqpoint{3.902750in}{0.728721in}}%
\pgfpathlineto{\pgfqpoint{3.903008in}{0.728279in}}%
\pgfpathlineto{\pgfqpoint{3.904041in}{0.721356in}}%
\pgfpathlineto{\pgfqpoint{3.906880in}{0.702289in}}%
\pgfpathlineto{\pgfqpoint{3.914108in}{0.703724in}}%
\pgfpathlineto{\pgfqpoint{3.920562in}{0.722726in}}%
\pgfpathlineto{\pgfqpoint{3.922885in}{0.724281in}}%
\pgfpathlineto{\pgfqpoint{3.924434in}{0.725048in}}%
\pgfpathlineto{\pgfqpoint{3.926499in}{0.728600in}}%
\pgfpathlineto{\pgfqpoint{3.928047in}{0.742244in}}%
\pgfpathlineto{\pgfqpoint{3.931661in}{0.775799in}}%
\pgfpathlineto{\pgfqpoint{3.933726in}{0.776767in}}%
\pgfpathlineto{\pgfqpoint{3.934243in}{0.777051in}}%
\pgfpathlineto{\pgfqpoint{3.934501in}{0.776598in}}%
\pgfpathlineto{\pgfqpoint{3.935533in}{0.768934in}}%
\pgfpathlineto{\pgfqpoint{3.937340in}{0.755750in}}%
\pgfpathlineto{\pgfqpoint{3.937598in}{0.756443in}}%
\pgfpathlineto{\pgfqpoint{3.939147in}{0.769235in}}%
\pgfpathlineto{\pgfqpoint{3.943277in}{0.802703in}}%
\pgfpathlineto{\pgfqpoint{3.943794in}{0.803293in}}%
\pgfpathlineto{\pgfqpoint{3.944310in}{0.802646in}}%
\pgfpathlineto{\pgfqpoint{3.947666in}{0.791343in}}%
\pgfpathlineto{\pgfqpoint{3.948440in}{0.792379in}}%
\pgfpathlineto{\pgfqpoint{3.950763in}{0.801305in}}%
\pgfpathlineto{\pgfqpoint{3.953861in}{0.810114in}}%
\pgfpathlineto{\pgfqpoint{3.955152in}{0.808716in}}%
\pgfpathlineto{\pgfqpoint{3.959282in}{0.803476in}}%
\pgfpathlineto{\pgfqpoint{3.963412in}{0.807219in}}%
\pgfpathlineto{\pgfqpoint{3.964961in}{0.819278in}}%
\pgfpathlineto{\pgfqpoint{3.968316in}{0.843640in}}%
\pgfpathlineto{\pgfqpoint{3.970382in}{0.850081in}}%
\pgfpathlineto{\pgfqpoint{3.970640in}{0.849794in}}%
\pgfpathlineto{\pgfqpoint{3.972189in}{0.843408in}}%
\pgfpathlineto{\pgfqpoint{3.973221in}{0.840803in}}%
\pgfpathlineto{\pgfqpoint{3.973737in}{0.841634in}}%
\pgfpathlineto{\pgfqpoint{3.975028in}{0.850760in}}%
\pgfpathlineto{\pgfqpoint{3.977867in}{0.896860in}}%
\pgfpathlineto{\pgfqpoint{3.979416in}{0.910596in}}%
\pgfpathlineto{\pgfqpoint{3.979674in}{0.910494in}}%
\pgfpathlineto{\pgfqpoint{3.980707in}{0.904574in}}%
\pgfpathlineto{\pgfqpoint{3.984063in}{0.884431in}}%
\pgfpathlineto{\pgfqpoint{3.984837in}{0.885241in}}%
\pgfpathlineto{\pgfqpoint{3.986128in}{0.891653in}}%
\pgfpathlineto{\pgfqpoint{3.990258in}{0.915489in}}%
\pgfpathlineto{\pgfqpoint{3.991032in}{0.915710in}}%
\pgfpathlineto{\pgfqpoint{3.991291in}{0.915219in}}%
\pgfpathlineto{\pgfqpoint{3.992581in}{0.908528in}}%
\pgfpathlineto{\pgfqpoint{3.995163in}{0.897136in}}%
\pgfpathlineto{\pgfqpoint{3.995937in}{0.898280in}}%
\pgfpathlineto{\pgfqpoint{3.997486in}{0.906091in}}%
\pgfpathlineto{\pgfqpoint{4.001100in}{0.924444in}}%
\pgfpathlineto{\pgfqpoint{4.003939in}{0.924124in}}%
\pgfpathlineto{\pgfqpoint{4.005488in}{0.921925in}}%
\pgfpathlineto{\pgfqpoint{4.007037in}{0.920262in}}%
\pgfpathlineto{\pgfqpoint{4.007295in}{0.920541in}}%
\pgfpathlineto{\pgfqpoint{4.008844in}{0.925145in}}%
\pgfpathlineto{\pgfqpoint{4.015039in}{0.948776in}}%
\pgfpathlineto{\pgfqpoint{4.015813in}{0.947085in}}%
\pgfpathlineto{\pgfqpoint{4.017878in}{0.929219in}}%
\pgfpathlineto{\pgfqpoint{4.020460in}{0.916844in}}%
\pgfpathlineto{\pgfqpoint{4.021234in}{0.916718in}}%
\pgfpathlineto{\pgfqpoint{4.021492in}{0.917042in}}%
\pgfpathlineto{\pgfqpoint{4.023041in}{0.922016in}}%
\pgfpathlineto{\pgfqpoint{4.025106in}{0.927535in}}%
\pgfpathlineto{\pgfqpoint{4.025364in}{0.927282in}}%
\pgfpathlineto{\pgfqpoint{4.026397in}{0.922508in}}%
\pgfpathlineto{\pgfqpoint{4.028204in}{0.897845in}}%
\pgfpathlineto{\pgfqpoint{4.031560in}{0.851696in}}%
\pgfpathlineto{\pgfqpoint{4.032334in}{0.852425in}}%
\pgfpathlineto{\pgfqpoint{4.034915in}{0.856985in}}%
\pgfpathlineto{\pgfqpoint{4.035173in}{0.856816in}}%
\pgfpathlineto{\pgfqpoint{4.036206in}{0.854417in}}%
\pgfpathlineto{\pgfqpoint{4.038013in}{0.842987in}}%
\pgfpathlineto{\pgfqpoint{4.041627in}{0.802184in}}%
\pgfpathlineto{\pgfqpoint{4.044208in}{0.782063in}}%
\pgfpathlineto{\pgfqpoint{4.045499in}{0.782108in}}%
\pgfpathlineto{\pgfqpoint{4.046789in}{0.782029in}}%
\pgfpathlineto{\pgfqpoint{4.048080in}{0.777980in}}%
\pgfpathlineto{\pgfqpoint{4.050403in}{0.759025in}}%
\pgfpathlineto{\pgfqpoint{4.054017in}{0.734497in}}%
\pgfpathlineto{\pgfqpoint{4.056082in}{0.730518in}}%
\pgfpathlineto{\pgfqpoint{4.057373in}{0.732614in}}%
\pgfpathlineto{\pgfqpoint{4.058664in}{0.733901in}}%
\pgfpathlineto{\pgfqpoint{4.058922in}{0.733544in}}%
\pgfpathlineto{\pgfqpoint{4.060213in}{0.728421in}}%
\pgfpathlineto{\pgfqpoint{4.067440in}{0.693728in}}%
\pgfpathlineto{\pgfqpoint{4.069505in}{0.695457in}}%
\pgfpathlineto{\pgfqpoint{4.070022in}{0.694534in}}%
\pgfpathlineto{\pgfqpoint{4.071829in}{0.686501in}}%
\pgfpathlineto{\pgfqpoint{4.077508in}{0.658385in}}%
\pgfpathlineto{\pgfqpoint{4.080089in}{0.657760in}}%
\pgfpathlineto{\pgfqpoint{4.081380in}{0.656496in}}%
\pgfpathlineto{\pgfqpoint{4.082928in}{0.648803in}}%
\pgfpathlineto{\pgfqpoint{4.086026in}{0.634974in}}%
\pgfpathlineto{\pgfqpoint{4.088091in}{0.634199in}}%
\pgfpathlineto{\pgfqpoint{4.097384in}{0.636515in}}%
\pgfpathlineto{\pgfqpoint{4.099965in}{0.635084in}}%
\pgfpathlineto{\pgfqpoint{4.101256in}{0.637820in}}%
\pgfpathlineto{\pgfqpoint{4.105644in}{0.656492in}}%
\pgfpathlineto{\pgfqpoint{4.106677in}{0.655316in}}%
\pgfpathlineto{\pgfqpoint{4.110549in}{0.648133in}}%
\pgfpathlineto{\pgfqpoint{4.110807in}{0.648417in}}%
\pgfpathlineto{\pgfqpoint{4.112098in}{0.652764in}}%
\pgfpathlineto{\pgfqpoint{4.115453in}{0.677924in}}%
\pgfpathlineto{\pgfqpoint{4.117260in}{0.684108in}}%
\pgfpathlineto{\pgfqpoint{4.117518in}{0.683943in}}%
\pgfpathlineto{\pgfqpoint{4.119325in}{0.679479in}}%
\pgfpathlineto{\pgfqpoint{4.120874in}{0.677754in}}%
\pgfpathlineto{\pgfqpoint{4.121132in}{0.678054in}}%
\pgfpathlineto{\pgfqpoint{4.122165in}{0.681686in}}%
\pgfpathlineto{\pgfqpoint{4.123972in}{0.698509in}}%
\pgfpathlineto{\pgfqpoint{4.128102in}{0.738214in}}%
\pgfpathlineto{\pgfqpoint{4.129651in}{0.738609in}}%
\pgfpathlineto{\pgfqpoint{4.130683in}{0.739476in}}%
\pgfpathlineto{\pgfqpoint{4.132232in}{0.746257in}}%
\pgfpathlineto{\pgfqpoint{4.136879in}{0.768465in}}%
\pgfpathlineto{\pgfqpoint{4.139718in}{0.774017in}}%
\pgfpathlineto{\pgfqpoint{4.143074in}{0.775340in}}%
\pgfpathlineto{\pgfqpoint{4.151076in}{0.790398in}}%
\pgfpathlineto{\pgfqpoint{4.151592in}{0.789859in}}%
\pgfpathlineto{\pgfqpoint{4.153399in}{0.785171in}}%
\pgfpathlineto{\pgfqpoint{4.156755in}{0.775715in}}%
\pgfpathlineto{\pgfqpoint{4.157013in}{0.775791in}}%
\pgfpathlineto{\pgfqpoint{4.158820in}{0.778543in}}%
\pgfpathlineto{\pgfqpoint{4.161401in}{0.781192in}}%
\pgfpathlineto{\pgfqpoint{4.162692in}{0.779396in}}%
\pgfpathlineto{\pgfqpoint{4.164499in}{0.771697in}}%
\pgfpathlineto{\pgfqpoint{4.169145in}{0.746268in}}%
\pgfpathlineto{\pgfqpoint{4.169404in}{0.746528in}}%
\pgfpathlineto{\pgfqpoint{4.170952in}{0.751634in}}%
\pgfpathlineto{\pgfqpoint{4.172501in}{0.755079in}}%
\pgfpathlineto{\pgfqpoint{4.172759in}{0.754930in}}%
\pgfpathlineto{\pgfqpoint{4.173792in}{0.751795in}}%
\pgfpathlineto{\pgfqpoint{4.175599in}{0.736104in}}%
\pgfpathlineto{\pgfqpoint{4.179729in}{0.697199in}}%
\pgfpathlineto{\pgfqpoint{4.180761in}{0.697483in}}%
\pgfpathlineto{\pgfqpoint{4.182052in}{0.698509in}}%
\pgfpathlineto{\pgfqpoint{4.182310in}{0.698286in}}%
\pgfpathlineto{\pgfqpoint{4.183343in}{0.695043in}}%
\pgfpathlineto{\pgfqpoint{4.190571in}{0.662014in}}%
\pgfpathlineto{\pgfqpoint{4.191603in}{0.663343in}}%
\pgfpathlineto{\pgfqpoint{4.193926in}{0.666894in}}%
\pgfpathlineto{\pgfqpoint{4.194184in}{0.666635in}}%
\pgfpathlineto{\pgfqpoint{4.195991in}{0.661761in}}%
\pgfpathlineto{\pgfqpoint{4.200380in}{0.650713in}}%
\pgfpathlineto{\pgfqpoint{4.201670in}{0.652383in}}%
\pgfpathlineto{\pgfqpoint{4.204768in}{0.656655in}}%
\pgfpathlineto{\pgfqpoint{4.206317in}{0.655149in}}%
\pgfpathlineto{\pgfqpoint{4.210447in}{0.648577in}}%
\pgfpathlineto{\pgfqpoint{4.210963in}{0.649178in}}%
\pgfpathlineto{\pgfqpoint{4.212770in}{0.654988in}}%
\pgfpathlineto{\pgfqpoint{4.215868in}{0.663426in}}%
\pgfpathlineto{\pgfqpoint{4.218707in}{0.664812in}}%
\pgfpathlineto{\pgfqpoint{4.222321in}{0.669691in}}%
\pgfpathlineto{\pgfqpoint{4.223870in}{0.670558in}}%
\pgfpathlineto{\pgfqpoint{4.226968in}{0.673357in}}%
\pgfpathlineto{\pgfqpoint{4.229033in}{0.673072in}}%
\pgfpathlineto{\pgfqpoint{4.230581in}{0.677280in}}%
\pgfpathlineto{\pgfqpoint{4.234195in}{0.687397in}}%
\pgfpathlineto{\pgfqpoint{4.237551in}{0.687868in}}%
\pgfpathlineto{\pgfqpoint{4.239358in}{0.689015in}}%
\pgfpathlineto{\pgfqpoint{4.242972in}{0.693192in}}%
\pgfpathlineto{\pgfqpoint{4.243230in}{0.692962in}}%
\pgfpathlineto{\pgfqpoint{4.247102in}{0.690126in}}%
\pgfpathlineto{\pgfqpoint{4.252007in}{0.689442in}}%
\pgfpathlineto{\pgfqpoint{4.256395in}{0.681830in}}%
\pgfpathlineto{\pgfqpoint{4.261300in}{0.680358in}}%
\pgfpathlineto{\pgfqpoint{4.263106in}{0.676266in}}%
\pgfpathlineto{\pgfqpoint{4.267753in}{0.664061in}}%
\pgfpathlineto{\pgfqpoint{4.269818in}{0.663124in}}%
\pgfpathlineto{\pgfqpoint{4.271883in}{0.657330in}}%
\pgfpathlineto{\pgfqpoint{4.276529in}{0.644312in}}%
\pgfpathlineto{\pgfqpoint{4.278595in}{0.644254in}}%
\pgfpathlineto{\pgfqpoint{4.280143in}{0.643889in}}%
\pgfpathlineto{\pgfqpoint{4.281434in}{0.640563in}}%
\pgfpathlineto{\pgfqpoint{4.285048in}{0.629295in}}%
\pgfpathlineto{\pgfqpoint{4.285306in}{0.629369in}}%
\pgfpathlineto{\pgfqpoint{4.290727in}{0.631941in}}%
\pgfpathlineto{\pgfqpoint{4.294857in}{0.626464in}}%
\pgfpathlineto{\pgfqpoint{4.295115in}{0.626685in}}%
\pgfpathlineto{\pgfqpoint{4.297438in}{0.630940in}}%
\pgfpathlineto{\pgfqpoint{4.300278in}{0.634661in}}%
\pgfpathlineto{\pgfqpoint{4.304408in}{0.633769in}}%
\pgfpathlineto{\pgfqpoint{4.310345in}{0.641499in}}%
\pgfpathlineto{\pgfqpoint{4.312927in}{0.640853in}}%
\pgfpathlineto{\pgfqpoint{4.317315in}{0.639143in}}%
\pgfpathlineto{\pgfqpoint{4.324284in}{0.642821in}}%
\pgfpathlineto{\pgfqpoint{4.328415in}{0.640400in}}%
\pgfpathlineto{\pgfqpoint{4.331770in}{0.643293in}}%
\pgfpathlineto{\pgfqpoint{4.333835in}{0.643603in}}%
\pgfpathlineto{\pgfqpoint{4.335642in}{0.640554in}}%
\pgfpathlineto{\pgfqpoint{4.340031in}{0.632205in}}%
\pgfpathlineto{\pgfqpoint{4.342354in}{0.633776in}}%
\pgfpathlineto{\pgfqpoint{4.344161in}{0.633992in}}%
\pgfpathlineto{\pgfqpoint{4.345968in}{0.631334in}}%
\pgfpathlineto{\pgfqpoint{4.352421in}{0.619761in}}%
\pgfpathlineto{\pgfqpoint{4.354744in}{0.619297in}}%
\pgfpathlineto{\pgfqpoint{4.357068in}{0.614504in}}%
\pgfpathlineto{\pgfqpoint{4.360681in}{0.608380in}}%
\pgfpathlineto{\pgfqpoint{4.363263in}{0.607769in}}%
\pgfpathlineto{\pgfqpoint{4.367651in}{0.607175in}}%
\pgfpathlineto{\pgfqpoint{4.377460in}{0.602216in}}%
\pgfpathlineto{\pgfqpoint{4.379525in}{0.601542in}}%
\pgfpathlineto{\pgfqpoint{4.385721in}{0.597150in}}%
\pgfpathlineto{\pgfqpoint{4.388818in}{0.598151in}}%
\pgfpathlineto{\pgfqpoint{4.391658in}{0.598637in}}%
\pgfpathlineto{\pgfqpoint{4.399660in}{0.595779in}}%
\pgfpathlineto{\pgfqpoint{4.417987in}{0.595878in}}%
\pgfpathlineto{\pgfqpoint{4.421343in}{0.596152in}}%
\pgfpathlineto{\pgfqpoint{4.426764in}{0.595188in}}%
\pgfpathlineto{\pgfqpoint{4.432185in}{0.593897in}}%
\pgfpathlineto{\pgfqpoint{4.450512in}{0.589934in}}%
\pgfpathlineto{\pgfqpoint{4.469614in}{0.589073in}}%
\pgfpathlineto{\pgfqpoint{4.471679in}{0.589676in}}%
\pgfpathlineto{\pgfqpoint{4.476326in}{0.591780in}}%
\pgfpathlineto{\pgfqpoint{4.483295in}{0.593209in}}%
\pgfpathlineto{\pgfqpoint{4.489491in}{0.594788in}}%
\pgfpathlineto{\pgfqpoint{4.493105in}{0.595654in}}%
\pgfpathlineto{\pgfqpoint{4.498525in}{0.597459in}}%
\pgfpathlineto{\pgfqpoint{4.515820in}{0.598303in}}%
\pgfpathlineto{\pgfqpoint{4.547055in}{0.594747in}}%
\pgfpathlineto{\pgfqpoint{4.556864in}{0.596288in}}%
\pgfpathlineto{\pgfqpoint{4.560220in}{0.597020in}}%
\pgfpathlineto{\pgfqpoint{4.564866in}{0.597706in}}%
\pgfpathlineto{\pgfqpoint{4.570287in}{0.599671in}}%
\pgfpathlineto{\pgfqpoint{4.574675in}{0.599894in}}%
\pgfpathlineto{\pgfqpoint{4.585775in}{0.600856in}}%
\pgfpathlineto{\pgfqpoint{4.593519in}{0.601163in}}%
\pgfpathlineto{\pgfqpoint{4.607458in}{0.601063in}}%
\pgfpathlineto{\pgfqpoint{4.611847in}{0.599572in}}%
\pgfpathlineto{\pgfqpoint{4.617267in}{0.598644in}}%
\pgfpathlineto{\pgfqpoint{4.622172in}{0.597837in}}%
\pgfpathlineto{\pgfqpoint{4.626302in}{0.597097in}}%
\pgfpathlineto{\pgfqpoint{4.631207in}{0.596324in}}%
\pgfpathlineto{\pgfqpoint{4.636628in}{0.595666in}}%
\pgfpathlineto{\pgfqpoint{4.642823in}{0.594383in}}%
\pgfpathlineto{\pgfqpoint{4.646437in}{0.594182in}}%
\pgfpathlineto{\pgfqpoint{4.652374in}{0.592353in}}%
\pgfpathlineto{\pgfqpoint{4.658569in}{0.591620in}}%
\pgfpathlineto{\pgfqpoint{4.666055in}{0.590929in}}%
\pgfpathlineto{\pgfqpoint{4.672766in}{0.590033in}}%
\pgfpathlineto{\pgfqpoint{4.721038in}{0.589169in}}%
\pgfpathlineto{\pgfqpoint{4.738075in}{0.590881in}}%
\pgfpathlineto{\pgfqpoint{4.742979in}{0.592327in}}%
\pgfpathlineto{\pgfqpoint{4.747367in}{0.592993in}}%
\pgfpathlineto{\pgfqpoint{4.773181in}{0.593520in}}%
\pgfpathlineto{\pgfqpoint{4.802866in}{0.588157in}}%
\pgfpathlineto{\pgfqpoint{4.807771in}{0.587129in}}%
\pgfpathlineto{\pgfqpoint{4.815773in}{0.586905in}}%
\pgfpathlineto{\pgfqpoint{4.820678in}{0.587989in}}%
\pgfpathlineto{\pgfqpoint{4.834617in}{0.590475in}}%
\pgfpathlineto{\pgfqpoint{4.844168in}{0.591719in}}%
\pgfpathlineto{\pgfqpoint{4.862237in}{0.592156in}}%
\pgfpathlineto{\pgfqpoint{4.948971in}{0.586397in}}%
\pgfpathlineto{\pgfqpoint{4.958005in}{0.585823in}}%
\pgfpathlineto{\pgfqpoint{4.977107in}{0.585965in}}%
\pgfpathlineto{\pgfqpoint{5.018409in}{0.587897in}}%
\pgfpathlineto{\pgfqpoint{5.031316in}{0.588409in}}%
\pgfpathlineto{\pgfqpoint{5.055839in}{0.589398in}}%
\pgfpathlineto{\pgfqpoint{5.073134in}{0.588994in}}%
\pgfpathlineto{\pgfqpoint{5.083975in}{0.588662in}}%
\pgfpathlineto{\pgfqpoint{5.099205in}{0.588619in}}%
\pgfpathlineto{\pgfqpoint{5.118049in}{0.592527in}}%
\pgfpathlineto{\pgfqpoint{5.122954in}{0.593177in}}%
\pgfpathlineto{\pgfqpoint{5.127858in}{0.594116in}}%
\pgfpathlineto{\pgfqpoint{5.134570in}{0.595346in}}%
\pgfpathlineto{\pgfqpoint{5.138700in}{0.596130in}}%
\pgfpathlineto{\pgfqpoint{5.163223in}{0.593826in}}%
\pgfpathlineto{\pgfqpoint{5.168385in}{0.592652in}}%
\pgfpathlineto{\pgfqpoint{5.175097in}{0.592072in}}%
\pgfpathlineto{\pgfqpoint{5.179743in}{0.590769in}}%
\pgfpathlineto{\pgfqpoint{5.185939in}{0.589872in}}%
\pgfpathlineto{\pgfqpoint{5.189811in}{0.588927in}}%
\pgfpathlineto{\pgfqpoint{5.212526in}{0.589117in}}%
\pgfpathlineto{\pgfqpoint{5.252537in}{0.588578in}}%
\pgfpathlineto{\pgfqpoint{5.268800in}{0.588773in}}%
\pgfpathlineto{\pgfqpoint{5.305971in}{0.589959in}}%
\pgfpathlineto{\pgfqpoint{5.322234in}{0.591729in}}%
\pgfpathlineto{\pgfqpoint{5.327138in}{0.591768in}}%
\pgfpathlineto{\pgfqpoint{5.332817in}{0.593687in}}%
\pgfpathlineto{\pgfqpoint{5.337206in}{0.594583in}}%
\pgfpathlineto{\pgfqpoint{5.343143in}{0.596265in}}%
\pgfpathlineto{\pgfqpoint{5.347531in}{0.595894in}}%
\pgfpathlineto{\pgfqpoint{5.357340in}{0.597851in}}%
\pgfpathlineto{\pgfqpoint{5.359663in}{0.598884in}}%
\pgfpathlineto{\pgfqpoint{5.364310in}{0.601972in}}%
\pgfpathlineto{\pgfqpoint{5.369472in}{0.601042in}}%
\pgfpathlineto{\pgfqpoint{5.378765in}{0.602402in}}%
\pgfpathlineto{\pgfqpoint{5.382895in}{0.601173in}}%
\pgfpathlineto{\pgfqpoint{5.386767in}{0.601295in}}%
\pgfpathlineto{\pgfqpoint{5.410516in}{0.596412in}}%
\pgfpathlineto{\pgfqpoint{5.418002in}{0.592407in}}%
\pgfpathlineto{\pgfqpoint{5.423939in}{0.591116in}}%
\pgfpathlineto{\pgfqpoint{5.430908in}{0.590454in}}%
\pgfpathlineto{\pgfqpoint{5.451817in}{0.589875in}}%
\pgfpathlineto{\pgfqpoint{5.455689in}{0.590190in}}%
\pgfpathlineto{\pgfqpoint{5.458529in}{0.590718in}}%
\pgfpathlineto{\pgfqpoint{5.464982in}{0.593719in}}%
\pgfpathlineto{\pgfqpoint{5.470661in}{0.595281in}}%
\pgfpathlineto{\pgfqpoint{5.477373in}{0.597251in}}%
\pgfpathlineto{\pgfqpoint{5.483310in}{0.597915in}}%
\pgfpathlineto{\pgfqpoint{5.489505in}{0.598602in}}%
\pgfpathlineto{\pgfqpoint{5.494152in}{0.598496in}}%
\pgfpathlineto{\pgfqpoint{5.509123in}{0.598630in}}%
\pgfpathlineto{\pgfqpoint{5.518674in}{0.596106in}}%
\pgfpathlineto{\pgfqpoint{5.521772in}{0.594753in}}%
\pgfpathlineto{\pgfqpoint{5.527193in}{0.592873in}}%
\pgfpathlineto{\pgfqpoint{5.546295in}{0.590891in}}%
\pgfpathlineto{\pgfqpoint{5.549392in}{0.590568in}}%
\pgfpathlineto{\pgfqpoint{5.551974in}{0.590734in}}%
\pgfpathlineto{\pgfqpoint{5.557395in}{0.591485in}}%
\pgfpathlineto{\pgfqpoint{5.583466in}{0.593403in}}%
\pgfpathlineto{\pgfqpoint{5.590694in}{0.594174in}}%
\pgfpathlineto{\pgfqpoint{5.594824in}{0.594956in}}%
\pgfpathlineto{\pgfqpoint{5.600245in}{0.595088in}}%
\pgfpathlineto{\pgfqpoint{5.603859in}{0.595205in}}%
\pgfpathlineto{\pgfqpoint{5.609280in}{0.594599in}}%
\pgfpathlineto{\pgfqpoint{5.612119in}{0.596570in}}%
\pgfpathlineto{\pgfqpoint{5.615475in}{0.597758in}}%
\pgfpathlineto{\pgfqpoint{5.624251in}{0.598753in}}%
\pgfpathlineto{\pgfqpoint{5.626833in}{0.599727in}}%
\pgfpathlineto{\pgfqpoint{5.629930in}{0.600203in}}%
\pgfpathlineto{\pgfqpoint{5.636126in}{0.603315in}}%
\pgfpathlineto{\pgfqpoint{5.641288in}{0.603860in}}%
\pgfpathlineto{\pgfqpoint{5.648000in}{0.605004in}}%
\pgfpathlineto{\pgfqpoint{5.657293in}{0.604227in}}%
\pgfpathlineto{\pgfqpoint{5.660907in}{0.603727in}}%
\pgfpathlineto{\pgfqpoint{5.668134in}{0.601051in}}%
\pgfpathlineto{\pgfqpoint{5.672781in}{0.600232in}}%
\pgfpathlineto{\pgfqpoint{5.675878in}{0.598346in}}%
\pgfpathlineto{\pgfqpoint{5.678718in}{0.597227in}}%
\pgfpathlineto{\pgfqpoint{5.682074in}{0.596449in}}%
\pgfpathlineto{\pgfqpoint{5.686462in}{0.593646in}}%
\pgfpathlineto{\pgfqpoint{5.707371in}{0.594138in}}%
\pgfpathlineto{\pgfqpoint{5.714341in}{0.594313in}}%
\pgfpathlineto{\pgfqpoint{5.732152in}{0.594040in}}%
\pgfpathlineto{\pgfqpoint{5.736282in}{0.594260in}}%
\pgfpathlineto{\pgfqpoint{5.739121in}{0.593834in}}%
\pgfpathlineto{\pgfqpoint{5.743510in}{0.592951in}}%
\pgfpathlineto{\pgfqpoint{5.746091in}{0.593042in}}%
\pgfpathlineto{\pgfqpoint{5.746091in}{0.593042in}}%
\pgfusepath{stroke}%
\end{pgfscope}%
\begin{pgfscope}%
\pgfpathrectangle{\pgfqpoint{0.583136in}{0.544166in}}{\pgfqpoint{5.162697in}{0.906858in}}%
\pgfusepath{clip}%
\pgfsetrectcap%
\pgfsetroundjoin%
\pgfsetlinewidth{1.505625pt}%
\definecolor{currentstroke}{rgb}{0.121569,0.466667,0.705882}%
\pgfsetstrokecolor{currentstroke}%
\pgfsetdash{}{0pt}%
\pgfpathmoveto{\pgfqpoint{0.583136in}{0.585387in}}%
\pgfpathlineto{\pgfqpoint{1.615417in}{0.585387in}}%
\pgfpathlineto{\pgfqpoint{1.615676in}{0.648329in}}%
\pgfpathlineto{\pgfqpoint{1.616708in}{0.647631in}}%
\pgfpathlineto{\pgfqpoint{1.617999in}{0.648998in}}%
\pgfpathlineto{\pgfqpoint{1.620322in}{0.651219in}}%
\pgfpathlineto{\pgfqpoint{1.622129in}{0.649636in}}%
\pgfpathlineto{\pgfqpoint{1.634003in}{0.637941in}}%
\pgfpathlineto{\pgfqpoint{1.637359in}{0.636076in}}%
\pgfpathlineto{\pgfqpoint{1.640198in}{0.635058in}}%
\pgfpathlineto{\pgfqpoint{1.646652in}{0.630481in}}%
\pgfpathlineto{\pgfqpoint{1.647942in}{0.632117in}}%
\pgfpathlineto{\pgfqpoint{1.649749in}{0.634305in}}%
\pgfpathlineto{\pgfqpoint{1.650008in}{0.634152in}}%
\pgfpathlineto{\pgfqpoint{1.652331in}{0.633500in}}%
\pgfpathlineto{\pgfqpoint{1.653880in}{0.632888in}}%
\pgfpathlineto{\pgfqpoint{1.658268in}{0.630440in}}%
\pgfpathlineto{\pgfqpoint{1.664721in}{0.630606in}}%
\pgfpathlineto{\pgfqpoint{1.666786in}{0.630175in}}%
\pgfpathlineto{\pgfqpoint{1.672465in}{0.635455in}}%
\pgfpathlineto{\pgfqpoint{1.674788in}{0.634543in}}%
\pgfpathlineto{\pgfqpoint{1.675047in}{0.634919in}}%
\pgfpathlineto{\pgfqpoint{1.678402in}{0.640419in}}%
\pgfpathlineto{\pgfqpoint{1.678660in}{0.640246in}}%
\pgfpathlineto{\pgfqpoint{1.688728in}{0.633289in}}%
\pgfpathlineto{\pgfqpoint{1.691309in}{0.634615in}}%
\pgfpathlineto{\pgfqpoint{1.693374in}{0.634244in}}%
\pgfpathlineto{\pgfqpoint{1.695439in}{0.634491in}}%
\pgfpathlineto{\pgfqpoint{1.699053in}{0.633070in}}%
\pgfpathlineto{\pgfqpoint{1.700860in}{0.632319in}}%
\pgfpathlineto{\pgfqpoint{1.708604in}{0.627011in}}%
\pgfpathlineto{\pgfqpoint{1.711185in}{0.625472in}}%
\pgfpathlineto{\pgfqpoint{1.721769in}{0.619365in}}%
\pgfpathlineto{\pgfqpoint{1.727448in}{0.618304in}}%
\pgfpathlineto{\pgfqpoint{1.729513in}{0.617448in}}%
\pgfpathlineto{\pgfqpoint{1.735450in}{0.615481in}}%
\pgfpathlineto{\pgfqpoint{1.736483in}{0.617254in}}%
\pgfpathlineto{\pgfqpoint{1.739838in}{0.626323in}}%
\pgfpathlineto{\pgfqpoint{1.743194in}{0.625676in}}%
\pgfpathlineto{\pgfqpoint{1.745001in}{0.628404in}}%
\pgfpathlineto{\pgfqpoint{1.748099in}{0.636741in}}%
\pgfpathlineto{\pgfqpoint{1.748615in}{0.636362in}}%
\pgfpathlineto{\pgfqpoint{1.750422in}{0.635907in}}%
\pgfpathlineto{\pgfqpoint{1.751971in}{0.639263in}}%
\pgfpathlineto{\pgfqpoint{1.754294in}{0.643166in}}%
\pgfpathlineto{\pgfqpoint{1.757908in}{0.644341in}}%
\pgfpathlineto{\pgfqpoint{1.759715in}{0.644784in}}%
\pgfpathlineto{\pgfqpoint{1.761264in}{0.643844in}}%
\pgfpathlineto{\pgfqpoint{1.761522in}{0.644096in}}%
\pgfpathlineto{\pgfqpoint{1.762812in}{0.647980in}}%
\pgfpathlineto{\pgfqpoint{1.767975in}{0.672390in}}%
\pgfpathlineto{\pgfqpoint{1.768750in}{0.671594in}}%
\pgfpathlineto{\pgfqpoint{1.771073in}{0.669717in}}%
\pgfpathlineto{\pgfqpoint{1.774687in}{0.669406in}}%
\pgfpathlineto{\pgfqpoint{1.777268in}{0.674183in}}%
\pgfpathlineto{\pgfqpoint{1.779849in}{0.677213in}}%
\pgfpathlineto{\pgfqpoint{1.781398in}{0.675727in}}%
\pgfpathlineto{\pgfqpoint{1.786303in}{0.669258in}}%
\pgfpathlineto{\pgfqpoint{1.786561in}{0.669567in}}%
\pgfpathlineto{\pgfqpoint{1.787593in}{0.673714in}}%
\pgfpathlineto{\pgfqpoint{1.791724in}{0.696449in}}%
\pgfpathlineto{\pgfqpoint{1.791982in}{0.696282in}}%
\pgfpathlineto{\pgfqpoint{1.794821in}{0.691225in}}%
\pgfpathlineto{\pgfqpoint{1.797661in}{0.688117in}}%
\pgfpathlineto{\pgfqpoint{1.798951in}{0.690741in}}%
\pgfpathlineto{\pgfqpoint{1.801533in}{0.696459in}}%
\pgfpathlineto{\pgfqpoint{1.803340in}{0.694122in}}%
\pgfpathlineto{\pgfqpoint{1.805147in}{0.692980in}}%
\pgfpathlineto{\pgfqpoint{1.806695in}{0.692638in}}%
\pgfpathlineto{\pgfqpoint{1.808502in}{0.690599in}}%
\pgfpathlineto{\pgfqpoint{1.808760in}{0.690927in}}%
\pgfpathlineto{\pgfqpoint{1.811342in}{0.694696in}}%
\pgfpathlineto{\pgfqpoint{1.811600in}{0.694521in}}%
\pgfpathlineto{\pgfqpoint{1.817795in}{0.687247in}}%
\pgfpathlineto{\pgfqpoint{1.825539in}{0.676591in}}%
\pgfpathlineto{\pgfqpoint{1.827604in}{0.677067in}}%
\pgfpathlineto{\pgfqpoint{1.830444in}{0.674762in}}%
\pgfpathlineto{\pgfqpoint{1.832251in}{0.672959in}}%
\pgfpathlineto{\pgfqpoint{1.838962in}{0.664990in}}%
\pgfpathlineto{\pgfqpoint{1.840253in}{0.667614in}}%
\pgfpathlineto{\pgfqpoint{1.841544in}{0.669349in}}%
\pgfpathlineto{\pgfqpoint{1.841802in}{0.669165in}}%
\pgfpathlineto{\pgfqpoint{1.850062in}{0.659813in}}%
\pgfpathlineto{\pgfqpoint{1.850578in}{0.660301in}}%
\pgfpathlineto{\pgfqpoint{1.852385in}{0.660711in}}%
\pgfpathlineto{\pgfqpoint{1.857806in}{0.656645in}}%
\pgfpathlineto{\pgfqpoint{1.859097in}{0.656200in}}%
\pgfpathlineto{\pgfqpoint{1.860646in}{0.655046in}}%
\pgfpathlineto{\pgfqpoint{1.860904in}{0.655282in}}%
\pgfpathlineto{\pgfqpoint{1.862969in}{0.660399in}}%
\pgfpathlineto{\pgfqpoint{1.864776in}{0.661460in}}%
\pgfpathlineto{\pgfqpoint{1.868390in}{0.658239in}}%
\pgfpathlineto{\pgfqpoint{1.873810in}{0.652229in}}%
\pgfpathlineto{\pgfqpoint{1.874069in}{0.652493in}}%
\pgfpathlineto{\pgfqpoint{1.875101in}{0.656158in}}%
\pgfpathlineto{\pgfqpoint{1.877682in}{0.666700in}}%
\pgfpathlineto{\pgfqpoint{1.877941in}{0.666469in}}%
\pgfpathlineto{\pgfqpoint{1.879748in}{0.665186in}}%
\pgfpathlineto{\pgfqpoint{1.880006in}{0.665406in}}%
\pgfpathlineto{\pgfqpoint{1.881813in}{0.665894in}}%
\pgfpathlineto{\pgfqpoint{1.885943in}{0.660289in}}%
\pgfpathlineto{\pgfqpoint{1.890073in}{0.656459in}}%
\pgfpathlineto{\pgfqpoint{1.897043in}{0.650649in}}%
\pgfpathlineto{\pgfqpoint{1.898591in}{0.649985in}}%
\pgfpathlineto{\pgfqpoint{1.903496in}{0.645820in}}%
\pgfpathlineto{\pgfqpoint{1.905303in}{0.645269in}}%
\pgfpathlineto{\pgfqpoint{1.907626in}{0.644265in}}%
\pgfpathlineto{\pgfqpoint{1.909691in}{0.644738in}}%
\pgfpathlineto{\pgfqpoint{1.911756in}{0.643857in}}%
\pgfpathlineto{\pgfqpoint{1.913305in}{0.643551in}}%
\pgfpathlineto{\pgfqpoint{1.917951in}{0.640200in}}%
\pgfpathlineto{\pgfqpoint{1.919500in}{0.640447in}}%
\pgfpathlineto{\pgfqpoint{1.921049in}{0.640528in}}%
\pgfpathlineto{\pgfqpoint{1.923372in}{0.639716in}}%
\pgfpathlineto{\pgfqpoint{1.925179in}{0.639660in}}%
\pgfpathlineto{\pgfqpoint{1.928793in}{0.637386in}}%
\pgfpathlineto{\pgfqpoint{1.931891in}{0.642452in}}%
\pgfpathlineto{\pgfqpoint{1.932923in}{0.641582in}}%
\pgfpathlineto{\pgfqpoint{1.933956in}{0.642676in}}%
\pgfpathlineto{\pgfqpoint{1.937053in}{0.647243in}}%
\pgfpathlineto{\pgfqpoint{1.938860in}{0.646162in}}%
\pgfpathlineto{\pgfqpoint{1.944539in}{0.641473in}}%
\pgfpathlineto{\pgfqpoint{1.948153in}{0.641879in}}%
\pgfpathlineto{\pgfqpoint{1.960802in}{0.631548in}}%
\pgfpathlineto{\pgfqpoint{1.963641in}{0.631202in}}%
\pgfpathlineto{\pgfqpoint{1.965965in}{0.630501in}}%
\pgfpathlineto{\pgfqpoint{1.968030in}{0.631953in}}%
\pgfpathlineto{\pgfqpoint{1.968288in}{0.631739in}}%
\pgfpathlineto{\pgfqpoint{1.970095in}{0.631665in}}%
\pgfpathlineto{\pgfqpoint{1.973192in}{0.632590in}}%
\pgfpathlineto{\pgfqpoint{1.980162in}{0.634621in}}%
\pgfpathlineto{\pgfqpoint{1.982227in}{0.635953in}}%
\pgfpathlineto{\pgfqpoint{1.995908in}{0.634165in}}%
\pgfpathlineto{\pgfqpoint{1.997715in}{0.634235in}}%
\pgfpathlineto{\pgfqpoint{2.000555in}{0.633439in}}%
\pgfpathlineto{\pgfqpoint{2.003652in}{0.635560in}}%
\pgfpathlineto{\pgfqpoint{2.005717in}{0.635797in}}%
\pgfpathlineto{\pgfqpoint{2.008041in}{0.639681in}}%
\pgfpathlineto{\pgfqpoint{2.008557in}{0.639255in}}%
\pgfpathlineto{\pgfqpoint{2.010106in}{0.639202in}}%
\pgfpathlineto{\pgfqpoint{2.012171in}{0.640174in}}%
\pgfpathlineto{\pgfqpoint{2.016817in}{0.638363in}}%
\pgfpathlineto{\pgfqpoint{2.018882in}{0.641028in}}%
\pgfpathlineto{\pgfqpoint{2.020431in}{0.642375in}}%
\pgfpathlineto{\pgfqpoint{2.020689in}{0.642211in}}%
\pgfpathlineto{\pgfqpoint{2.022754in}{0.641541in}}%
\pgfpathlineto{\pgfqpoint{2.025594in}{0.642681in}}%
\pgfpathlineto{\pgfqpoint{2.027143in}{0.643849in}}%
\pgfpathlineto{\pgfqpoint{2.029724in}{0.647407in}}%
\pgfpathlineto{\pgfqpoint{2.029982in}{0.647240in}}%
\pgfpathlineto{\pgfqpoint{2.036435in}{0.642718in}}%
\pgfpathlineto{\pgfqpoint{2.039275in}{0.642513in}}%
\pgfpathlineto{\pgfqpoint{2.043147in}{0.640073in}}%
\pgfpathlineto{\pgfqpoint{2.045212in}{0.640406in}}%
\pgfpathlineto{\pgfqpoint{2.049084in}{0.638437in}}%
\pgfpathlineto{\pgfqpoint{2.053989in}{0.642120in}}%
\pgfpathlineto{\pgfqpoint{2.062249in}{0.635929in}}%
\pgfpathlineto{\pgfqpoint{2.063281in}{0.638845in}}%
\pgfpathlineto{\pgfqpoint{2.065605in}{0.648730in}}%
\pgfpathlineto{\pgfqpoint{2.066121in}{0.648447in}}%
\pgfpathlineto{\pgfqpoint{2.067412in}{0.647821in}}%
\pgfpathlineto{\pgfqpoint{2.067670in}{0.648045in}}%
\pgfpathlineto{\pgfqpoint{2.069735in}{0.648664in}}%
\pgfpathlineto{\pgfqpoint{2.072574in}{0.647966in}}%
\pgfpathlineto{\pgfqpoint{2.074123in}{0.652746in}}%
\pgfpathlineto{\pgfqpoint{2.075672in}{0.656382in}}%
\pgfpathlineto{\pgfqpoint{2.075930in}{0.656280in}}%
\pgfpathlineto{\pgfqpoint{2.080318in}{0.653244in}}%
\pgfpathlineto{\pgfqpoint{2.081609in}{0.656480in}}%
\pgfpathlineto{\pgfqpoint{2.083932in}{0.661678in}}%
\pgfpathlineto{\pgfqpoint{2.085481in}{0.660091in}}%
\pgfpathlineto{\pgfqpoint{2.093483in}{0.651100in}}%
\pgfpathlineto{\pgfqpoint{2.094774in}{0.652442in}}%
\pgfpathlineto{\pgfqpoint{2.096839in}{0.654168in}}%
\pgfpathlineto{\pgfqpoint{2.100969in}{0.649325in}}%
\pgfpathlineto{\pgfqpoint{2.103809in}{0.647559in}}%
\pgfpathlineto{\pgfqpoint{2.106906in}{0.648267in}}%
\pgfpathlineto{\pgfqpoint{2.114134in}{0.641374in}}%
\pgfpathlineto{\pgfqpoint{2.115683in}{0.642749in}}%
\pgfpathlineto{\pgfqpoint{2.117232in}{0.643241in}}%
\pgfpathlineto{\pgfqpoint{2.124459in}{0.638686in}}%
\pgfpathlineto{\pgfqpoint{2.127557in}{0.637735in}}%
\pgfpathlineto{\pgfqpoint{2.130655in}{0.636374in}}%
\pgfpathlineto{\pgfqpoint{2.132720in}{0.636488in}}%
\pgfpathlineto{\pgfqpoint{2.136592in}{0.634708in}}%
\pgfpathlineto{\pgfqpoint{2.138399in}{0.634352in}}%
\pgfpathlineto{\pgfqpoint{2.139689in}{0.634163in}}%
\pgfpathlineto{\pgfqpoint{2.142529in}{0.639872in}}%
\pgfpathlineto{\pgfqpoint{2.143561in}{0.638939in}}%
\pgfpathlineto{\pgfqpoint{2.149498in}{0.635253in}}%
\pgfpathlineto{\pgfqpoint{2.151305in}{0.635403in}}%
\pgfpathlineto{\pgfqpoint{2.155177in}{0.633490in}}%
\pgfpathlineto{\pgfqpoint{2.161115in}{0.629436in}}%
\pgfpathlineto{\pgfqpoint{2.169633in}{0.625177in}}%
\pgfpathlineto{\pgfqpoint{2.171956in}{0.625890in}}%
\pgfpathlineto{\pgfqpoint{2.174021in}{0.625339in}}%
\pgfpathlineto{\pgfqpoint{2.175828in}{0.629036in}}%
\pgfpathlineto{\pgfqpoint{2.177635in}{0.630374in}}%
\pgfpathlineto{\pgfqpoint{2.184605in}{0.627853in}}%
\pgfpathlineto{\pgfqpoint{2.187186in}{0.630966in}}%
\pgfpathlineto{\pgfqpoint{2.192091in}{0.636340in}}%
\pgfpathlineto{\pgfqpoint{2.196479in}{0.635178in}}%
\pgfpathlineto{\pgfqpoint{2.201642in}{0.639329in}}%
\pgfpathlineto{\pgfqpoint{2.207063in}{0.637985in}}%
\pgfpathlineto{\pgfqpoint{2.208869in}{0.639341in}}%
\pgfpathlineto{\pgfqpoint{2.213774in}{0.637066in}}%
\pgfpathlineto{\pgfqpoint{2.215839in}{0.635915in}}%
\pgfpathlineto{\pgfqpoint{2.221776in}{0.632144in}}%
\pgfpathlineto{\pgfqpoint{2.224358in}{0.631601in}}%
\pgfpathlineto{\pgfqpoint{2.229520in}{0.628883in}}%
\pgfpathlineto{\pgfqpoint{2.231843in}{0.628614in}}%
\pgfpathlineto{\pgfqpoint{2.234167in}{0.628078in}}%
\pgfpathlineto{\pgfqpoint{2.237006in}{0.630040in}}%
\pgfpathlineto{\pgfqpoint{2.244234in}{0.626195in}}%
\pgfpathlineto{\pgfqpoint{2.246299in}{0.626004in}}%
\pgfpathlineto{\pgfqpoint{2.247590in}{0.632759in}}%
\pgfpathlineto{\pgfqpoint{2.249655in}{0.642771in}}%
\pgfpathlineto{\pgfqpoint{2.249913in}{0.642615in}}%
\pgfpathlineto{\pgfqpoint{2.251204in}{0.642272in}}%
\pgfpathlineto{\pgfqpoint{2.253269in}{0.644277in}}%
\pgfpathlineto{\pgfqpoint{2.253527in}{0.644041in}}%
\pgfpathlineto{\pgfqpoint{2.257657in}{0.641002in}}%
\pgfpathlineto{\pgfqpoint{2.259722in}{0.640186in}}%
\pgfpathlineto{\pgfqpoint{2.265401in}{0.636130in}}%
\pgfpathlineto{\pgfqpoint{2.267466in}{0.635967in}}%
\pgfpathlineto{\pgfqpoint{2.278308in}{0.628782in}}%
\pgfpathlineto{\pgfqpoint{2.282438in}{0.629868in}}%
\pgfpathlineto{\pgfqpoint{2.285794in}{0.640174in}}%
\pgfpathlineto{\pgfqpoint{2.286310in}{0.639833in}}%
\pgfpathlineto{\pgfqpoint{2.289149in}{0.638329in}}%
\pgfpathlineto{\pgfqpoint{2.297152in}{0.641351in}}%
\pgfpathlineto{\pgfqpoint{2.300507in}{0.639658in}}%
\pgfpathlineto{\pgfqpoint{2.302314in}{0.640734in}}%
\pgfpathlineto{\pgfqpoint{2.303863in}{0.645469in}}%
\pgfpathlineto{\pgfqpoint{2.305928in}{0.651510in}}%
\pgfpathlineto{\pgfqpoint{2.306186in}{0.651431in}}%
\pgfpathlineto{\pgfqpoint{2.310575in}{0.649050in}}%
\pgfpathlineto{\pgfqpoint{2.312640in}{0.649677in}}%
\pgfpathlineto{\pgfqpoint{2.316254in}{0.652976in}}%
\pgfpathlineto{\pgfqpoint{2.316512in}{0.652733in}}%
\pgfpathlineto{\pgfqpoint{2.323223in}{0.646991in}}%
\pgfpathlineto{\pgfqpoint{2.325288in}{0.647598in}}%
\pgfpathlineto{\pgfqpoint{2.327353in}{0.647860in}}%
\pgfpathlineto{\pgfqpoint{2.336388in}{0.642342in}}%
\pgfpathlineto{\pgfqpoint{2.337937in}{0.641996in}}%
\pgfpathlineto{\pgfqpoint{2.344390in}{0.637029in}}%
\pgfpathlineto{\pgfqpoint{2.347230in}{0.639495in}}%
\pgfpathlineto{\pgfqpoint{2.356523in}{0.634204in}}%
\pgfpathlineto{\pgfqpoint{2.361685in}{0.630995in}}%
\pgfpathlineto{\pgfqpoint{2.365815in}{0.629120in}}%
\pgfpathlineto{\pgfqpoint{2.371753in}{0.630201in}}%
\pgfpathlineto{\pgfqpoint{2.375883in}{0.628328in}}%
\pgfpathlineto{\pgfqpoint{2.377432in}{0.632097in}}%
\pgfpathlineto{\pgfqpoint{2.380013in}{0.636758in}}%
\pgfpathlineto{\pgfqpoint{2.382336in}{0.635014in}}%
\pgfpathlineto{\pgfqpoint{2.387241in}{0.631938in}}%
\pgfpathlineto{\pgfqpoint{2.388531in}{0.634211in}}%
\pgfpathlineto{\pgfqpoint{2.391113in}{0.640684in}}%
\pgfpathlineto{\pgfqpoint{2.391371in}{0.640563in}}%
\pgfpathlineto{\pgfqpoint{2.397824in}{0.636579in}}%
\pgfpathlineto{\pgfqpoint{2.400922in}{0.636149in}}%
\pgfpathlineto{\pgfqpoint{2.407117in}{0.631893in}}%
\pgfpathlineto{\pgfqpoint{2.409956in}{0.630517in}}%
\pgfpathlineto{\pgfqpoint{2.412796in}{0.629644in}}%
\pgfpathlineto{\pgfqpoint{2.415377in}{0.629114in}}%
\pgfpathlineto{\pgfqpoint{2.421314in}{0.625728in}}%
\pgfpathlineto{\pgfqpoint{2.441965in}{0.617305in}}%
\pgfpathlineto{\pgfqpoint{2.444030in}{0.616637in}}%
\pgfpathlineto{\pgfqpoint{2.450226in}{0.614304in}}%
\pgfpathlineto{\pgfqpoint{2.454356in}{0.615396in}}%
\pgfpathlineto{\pgfqpoint{2.459518in}{0.614057in}}%
\pgfpathlineto{\pgfqpoint{2.463132in}{0.616447in}}%
\pgfpathlineto{\pgfqpoint{2.468037in}{0.615202in}}%
\pgfpathlineto{\pgfqpoint{2.472683in}{0.613482in}}%
\pgfpathlineto{\pgfqpoint{2.476297in}{0.612867in}}%
\pgfpathlineto{\pgfqpoint{2.482234in}{0.611694in}}%
\pgfpathlineto{\pgfqpoint{2.487913in}{0.609402in}}%
\pgfpathlineto{\pgfqpoint{2.494883in}{0.607588in}}%
\pgfpathlineto{\pgfqpoint{2.508306in}{0.605331in}}%
\pgfpathlineto{\pgfqpoint{2.511662in}{0.609360in}}%
\pgfpathlineto{\pgfqpoint{2.512952in}{0.609656in}}%
\pgfpathlineto{\pgfqpoint{2.518115in}{0.615778in}}%
\pgfpathlineto{\pgfqpoint{2.525085in}{0.614319in}}%
\pgfpathlineto{\pgfqpoint{2.529989in}{0.616620in}}%
\pgfpathlineto{\pgfqpoint{2.540831in}{0.613785in}}%
\pgfpathlineto{\pgfqpoint{2.545735in}{0.616150in}}%
\pgfpathlineto{\pgfqpoint{2.549349in}{0.616034in}}%
\pgfpathlineto{\pgfqpoint{2.553221in}{0.616057in}}%
\pgfpathlineto{\pgfqpoint{2.556061in}{0.618594in}}%
\pgfpathlineto{\pgfqpoint{2.558900in}{0.618531in}}%
\pgfpathlineto{\pgfqpoint{2.561740in}{0.621365in}}%
\pgfpathlineto{\pgfqpoint{2.570516in}{0.618569in}}%
\pgfpathlineto{\pgfqpoint{2.573872in}{0.619501in}}%
\pgfpathlineto{\pgfqpoint{2.579035in}{0.617857in}}%
\pgfpathlineto{\pgfqpoint{2.587811in}{0.613369in}}%
\pgfpathlineto{\pgfqpoint{2.590135in}{0.613275in}}%
\pgfpathlineto{\pgfqpoint{2.592974in}{0.614680in}}%
\pgfpathlineto{\pgfqpoint{2.594781in}{0.614921in}}%
\pgfpathlineto{\pgfqpoint{2.596588in}{0.614820in}}%
\pgfpathlineto{\pgfqpoint{2.600460in}{0.613629in}}%
\pgfpathlineto{\pgfqpoint{2.602267in}{0.616749in}}%
\pgfpathlineto{\pgfqpoint{2.603816in}{0.617914in}}%
\pgfpathlineto{\pgfqpoint{2.605881in}{0.617960in}}%
\pgfpathlineto{\pgfqpoint{2.608720in}{0.619524in}}%
\pgfpathlineto{\pgfqpoint{2.611302in}{0.620002in}}%
\pgfpathlineto{\pgfqpoint{2.614657in}{0.623684in}}%
\pgfpathlineto{\pgfqpoint{2.618013in}{0.622605in}}%
\pgfpathlineto{\pgfqpoint{2.619304in}{0.624865in}}%
\pgfpathlineto{\pgfqpoint{2.623434in}{0.634481in}}%
\pgfpathlineto{\pgfqpoint{2.626273in}{0.634450in}}%
\pgfpathlineto{\pgfqpoint{2.629113in}{0.633381in}}%
\pgfpathlineto{\pgfqpoint{2.630404in}{0.637819in}}%
\pgfpathlineto{\pgfqpoint{2.633243in}{0.646643in}}%
\pgfpathlineto{\pgfqpoint{2.636341in}{0.645562in}}%
\pgfpathlineto{\pgfqpoint{2.641245in}{0.644569in}}%
\pgfpathlineto{\pgfqpoint{2.644859in}{0.652152in}}%
\pgfpathlineto{\pgfqpoint{2.648473in}{0.651296in}}%
\pgfpathlineto{\pgfqpoint{2.652603in}{0.650337in}}%
\pgfpathlineto{\pgfqpoint{2.656217in}{0.658810in}}%
\pgfpathlineto{\pgfqpoint{2.656992in}{0.658159in}}%
\pgfpathlineto{\pgfqpoint{2.659831in}{0.656092in}}%
\pgfpathlineto{\pgfqpoint{2.664219in}{0.656076in}}%
\pgfpathlineto{\pgfqpoint{2.674028in}{0.645271in}}%
\pgfpathlineto{\pgfqpoint{2.683063in}{0.637218in}}%
\pgfpathlineto{\pgfqpoint{2.686935in}{0.635190in}}%
\pgfpathlineto{\pgfqpoint{2.690549in}{0.632261in}}%
\pgfpathlineto{\pgfqpoint{2.693130in}{0.631723in}}%
\pgfpathlineto{\pgfqpoint{2.695454in}{0.630477in}}%
\pgfpathlineto{\pgfqpoint{2.696486in}{0.630744in}}%
\pgfpathlineto{\pgfqpoint{2.699842in}{0.635940in}}%
\pgfpathlineto{\pgfqpoint{2.700358in}{0.635585in}}%
\pgfpathlineto{\pgfqpoint{2.709651in}{0.629237in}}%
\pgfpathlineto{\pgfqpoint{2.711458in}{0.629065in}}%
\pgfpathlineto{\pgfqpoint{2.713523in}{0.628627in}}%
\pgfpathlineto{\pgfqpoint{2.716363in}{0.629705in}}%
\pgfpathlineto{\pgfqpoint{2.723590in}{0.626425in}}%
\pgfpathlineto{\pgfqpoint{2.726172in}{0.626849in}}%
\pgfpathlineto{\pgfqpoint{2.728753in}{0.626328in}}%
\pgfpathlineto{\pgfqpoint{2.740111in}{0.620146in}}%
\pgfpathlineto{\pgfqpoint{2.745790in}{0.617611in}}%
\pgfpathlineto{\pgfqpoint{2.749662in}{0.619750in}}%
\pgfpathlineto{\pgfqpoint{2.750953in}{0.620963in}}%
\pgfpathlineto{\pgfqpoint{2.755083in}{0.625948in}}%
\pgfpathlineto{\pgfqpoint{2.762569in}{0.622550in}}%
\pgfpathlineto{\pgfqpoint{2.765666in}{0.623238in}}%
\pgfpathlineto{\pgfqpoint{2.770829in}{0.621855in}}%
\pgfpathlineto{\pgfqpoint{2.772636in}{0.621629in}}%
\pgfpathlineto{\pgfqpoint{2.777540in}{0.619509in}}%
\pgfpathlineto{\pgfqpoint{2.784768in}{0.618856in}}%
\pgfpathlineto{\pgfqpoint{2.789673in}{0.617106in}}%
\pgfpathlineto{\pgfqpoint{2.801031in}{0.620416in}}%
\pgfpathlineto{\pgfqpoint{2.803354in}{0.620818in}}%
\pgfpathlineto{\pgfqpoint{2.810324in}{0.617777in}}%
\pgfpathlineto{\pgfqpoint{2.812905in}{0.618140in}}%
\pgfpathlineto{\pgfqpoint{2.816519in}{0.619808in}}%
\pgfpathlineto{\pgfqpoint{2.818326in}{0.620752in}}%
\pgfpathlineto{\pgfqpoint{2.824263in}{0.618227in}}%
\pgfpathlineto{\pgfqpoint{2.829167in}{0.616897in}}%
\pgfpathlineto{\pgfqpoint{2.833039in}{0.615508in}}%
\pgfpathlineto{\pgfqpoint{2.838460in}{0.613766in}}%
\pgfpathlineto{\pgfqpoint{2.843881in}{0.612713in}}%
\pgfpathlineto{\pgfqpoint{2.857820in}{0.608956in}}%
\pgfpathlineto{\pgfqpoint{2.860660in}{0.609978in}}%
\pgfpathlineto{\pgfqpoint{2.867630in}{0.607944in}}%
\pgfpathlineto{\pgfqpoint{2.869436in}{0.610115in}}%
\pgfpathlineto{\pgfqpoint{2.872018in}{0.612374in}}%
\pgfpathlineto{\pgfqpoint{2.873309in}{0.613926in}}%
\pgfpathlineto{\pgfqpoint{2.875632in}{0.618850in}}%
\pgfpathlineto{\pgfqpoint{2.876148in}{0.618563in}}%
\pgfpathlineto{\pgfqpoint{2.876922in}{0.619279in}}%
\pgfpathlineto{\pgfqpoint{2.877955in}{0.625341in}}%
\pgfpathlineto{\pgfqpoint{2.879504in}{0.650687in}}%
\pgfpathlineto{\pgfqpoint{2.881569in}{0.686829in}}%
\pgfpathlineto{\pgfqpoint{2.881827in}{0.686664in}}%
\pgfpathlineto{\pgfqpoint{2.883634in}{0.684771in}}%
\pgfpathlineto{\pgfqpoint{2.884150in}{0.693011in}}%
\pgfpathlineto{\pgfqpoint{2.885441in}{0.801881in}}%
\pgfpathlineto{\pgfqpoint{2.886732in}{0.863476in}}%
\pgfpathlineto{\pgfqpoint{2.886990in}{0.862170in}}%
\pgfpathlineto{\pgfqpoint{2.887248in}{0.862576in}}%
\pgfpathlineto{\pgfqpoint{2.888022in}{0.907011in}}%
\pgfpathlineto{\pgfqpoint{2.890604in}{1.080987in}}%
\pgfpathlineto{\pgfqpoint{2.893185in}{1.062094in}}%
\pgfpathlineto{\pgfqpoint{2.894476in}{1.054350in}}%
\pgfpathlineto{\pgfqpoint{2.894734in}{1.055783in}}%
\pgfpathlineto{\pgfqpoint{2.895766in}{1.080458in}}%
\pgfpathlineto{\pgfqpoint{2.897057in}{1.111803in}}%
\pgfpathlineto{\pgfqpoint{2.897573in}{1.109727in}}%
\pgfpathlineto{\pgfqpoint{2.898089in}{1.105598in}}%
\pgfpathlineto{\pgfqpoint{2.898606in}{1.110201in}}%
\pgfpathlineto{\pgfqpoint{2.900929in}{1.176541in}}%
\pgfpathlineto{\pgfqpoint{2.901703in}{1.172271in}}%
\pgfpathlineto{\pgfqpoint{2.905834in}{1.134119in}}%
\pgfpathlineto{\pgfqpoint{2.906350in}{1.136437in}}%
\pgfpathlineto{\pgfqpoint{2.908157in}{1.155601in}}%
\pgfpathlineto{\pgfqpoint{2.908673in}{1.151806in}}%
\pgfpathlineto{\pgfqpoint{2.909447in}{1.146637in}}%
\pgfpathlineto{\pgfqpoint{2.909706in}{1.148406in}}%
\pgfpathlineto{\pgfqpoint{2.911771in}{1.190345in}}%
\pgfpathlineto{\pgfqpoint{2.912803in}{1.182483in}}%
\pgfpathlineto{\pgfqpoint{2.915901in}{1.160930in}}%
\pgfpathlineto{\pgfqpoint{2.916675in}{1.166165in}}%
\pgfpathlineto{\pgfqpoint{2.917708in}{1.173165in}}%
\pgfpathlineto{\pgfqpoint{2.917966in}{1.172443in}}%
\pgfpathlineto{\pgfqpoint{2.924419in}{1.137286in}}%
\pgfpathlineto{\pgfqpoint{2.925452in}{1.130342in}}%
\pgfpathlineto{\pgfqpoint{2.926484in}{1.122836in}}%
\pgfpathlineto{\pgfqpoint{2.927001in}{1.124392in}}%
\pgfpathlineto{\pgfqpoint{2.928808in}{1.139262in}}%
\pgfpathlineto{\pgfqpoint{2.929324in}{1.135762in}}%
\pgfpathlineto{\pgfqpoint{2.930098in}{1.130099in}}%
\pgfpathlineto{\pgfqpoint{2.930356in}{1.131190in}}%
\pgfpathlineto{\pgfqpoint{2.932421in}{1.162235in}}%
\pgfpathlineto{\pgfqpoint{2.933196in}{1.155694in}}%
\pgfpathlineto{\pgfqpoint{2.933712in}{1.152031in}}%
\pgfpathlineto{\pgfqpoint{2.934228in}{1.153792in}}%
\pgfpathlineto{\pgfqpoint{2.936552in}{1.185886in}}%
\pgfpathlineto{\pgfqpoint{2.937326in}{1.179191in}}%
\pgfpathlineto{\pgfqpoint{2.937842in}{1.175840in}}%
\pgfpathlineto{\pgfqpoint{2.938359in}{1.179642in}}%
\pgfpathlineto{\pgfqpoint{2.939649in}{1.196758in}}%
\pgfpathlineto{\pgfqpoint{2.940165in}{1.193268in}}%
\pgfpathlineto{\pgfqpoint{2.940682in}{1.188371in}}%
\pgfpathlineto{\pgfqpoint{2.941198in}{1.191238in}}%
\pgfpathlineto{\pgfqpoint{2.943263in}{1.234142in}}%
\pgfpathlineto{\pgfqpoint{2.943779in}{1.230766in}}%
\pgfpathlineto{\pgfqpoint{2.944812in}{1.222631in}}%
\pgfpathlineto{\pgfqpoint{2.945070in}{1.224517in}}%
\pgfpathlineto{\pgfqpoint{2.947135in}{1.261462in}}%
\pgfpathlineto{\pgfqpoint{2.947909in}{1.255751in}}%
\pgfpathlineto{\pgfqpoint{2.948942in}{1.247312in}}%
\pgfpathlineto{\pgfqpoint{2.949458in}{1.248363in}}%
\pgfpathlineto{\pgfqpoint{2.949975in}{1.248964in}}%
\pgfpathlineto{\pgfqpoint{2.950233in}{1.247900in}}%
\pgfpathlineto{\pgfqpoint{2.951523in}{1.236927in}}%
\pgfpathlineto{\pgfqpoint{2.951782in}{1.238869in}}%
\pgfpathlineto{\pgfqpoint{2.953847in}{1.276073in}}%
\pgfpathlineto{\pgfqpoint{2.954621in}{1.269876in}}%
\pgfpathlineto{\pgfqpoint{2.955654in}{1.261681in}}%
\pgfpathlineto{\pgfqpoint{2.955912in}{1.263421in}}%
\pgfpathlineto{\pgfqpoint{2.958235in}{1.301810in}}%
\pgfpathlineto{\pgfqpoint{2.959009in}{1.294590in}}%
\pgfpathlineto{\pgfqpoint{2.960300in}{1.286033in}}%
\pgfpathlineto{\pgfqpoint{2.960558in}{1.286227in}}%
\pgfpathlineto{\pgfqpoint{2.961074in}{1.285135in}}%
\pgfpathlineto{\pgfqpoint{2.966495in}{1.244338in}}%
\pgfpathlineto{\pgfqpoint{2.967011in}{1.247092in}}%
\pgfpathlineto{\pgfqpoint{2.969077in}{1.265447in}}%
\pgfpathlineto{\pgfqpoint{2.969593in}{1.263149in}}%
\pgfpathlineto{\pgfqpoint{2.972174in}{1.242615in}}%
\pgfpathlineto{\pgfqpoint{2.972690in}{1.243165in}}%
\pgfpathlineto{\pgfqpoint{2.973207in}{1.243193in}}%
\pgfpathlineto{\pgfqpoint{2.973465in}{1.242579in}}%
\pgfpathlineto{\pgfqpoint{2.974756in}{1.232348in}}%
\pgfpathlineto{\pgfqpoint{2.977853in}{1.200530in}}%
\pgfpathlineto{\pgfqpoint{2.978111in}{1.201439in}}%
\pgfpathlineto{\pgfqpoint{2.979402in}{1.224885in}}%
\pgfpathlineto{\pgfqpoint{2.980176in}{1.233444in}}%
\pgfpathlineto{\pgfqpoint{2.980693in}{1.230949in}}%
\pgfpathlineto{\pgfqpoint{2.981725in}{1.222693in}}%
\pgfpathlineto{\pgfqpoint{2.981983in}{1.224733in}}%
\pgfpathlineto{\pgfqpoint{2.984048in}{1.258431in}}%
\pgfpathlineto{\pgfqpoint{2.984565in}{1.254803in}}%
\pgfpathlineto{\pgfqpoint{2.987920in}{1.225683in}}%
\pgfpathlineto{\pgfqpoint{2.988437in}{1.229561in}}%
\pgfpathlineto{\pgfqpoint{2.989985in}{1.252348in}}%
\pgfpathlineto{\pgfqpoint{2.990502in}{1.249729in}}%
\pgfpathlineto{\pgfqpoint{2.993083in}{1.232375in}}%
\pgfpathlineto{\pgfqpoint{2.994374in}{1.224421in}}%
\pgfpathlineto{\pgfqpoint{2.998762in}{1.182763in}}%
\pgfpathlineto{\pgfqpoint{2.999536in}{1.183532in}}%
\pgfpathlineto{\pgfqpoint{3.000053in}{1.182128in}}%
\pgfpathlineto{\pgfqpoint{3.002634in}{1.157836in}}%
\pgfpathlineto{\pgfqpoint{3.008313in}{1.109061in}}%
\pgfpathlineto{\pgfqpoint{3.008571in}{1.109672in}}%
\pgfpathlineto{\pgfqpoint{3.009862in}{1.114166in}}%
\pgfpathlineto{\pgfqpoint{3.010120in}{1.113630in}}%
\pgfpathlineto{\pgfqpoint{3.011669in}{1.101091in}}%
\pgfpathlineto{\pgfqpoint{3.016573in}{1.067858in}}%
\pgfpathlineto{\pgfqpoint{3.018897in}{1.053549in}}%
\pgfpathlineto{\pgfqpoint{3.019671in}{1.055995in}}%
\pgfpathlineto{\pgfqpoint{3.020962in}{1.061066in}}%
\pgfpathlineto{\pgfqpoint{3.021220in}{1.060663in}}%
\pgfpathlineto{\pgfqpoint{3.023027in}{1.050232in}}%
\pgfpathlineto{\pgfqpoint{3.024059in}{1.052954in}}%
\pgfpathlineto{\pgfqpoint{3.024834in}{1.048700in}}%
\pgfpathlineto{\pgfqpoint{3.025608in}{1.044393in}}%
\pgfpathlineto{\pgfqpoint{3.026124in}{1.046492in}}%
\pgfpathlineto{\pgfqpoint{3.027157in}{1.051877in}}%
\pgfpathlineto{\pgfqpoint{3.027415in}{1.051151in}}%
\pgfpathlineto{\pgfqpoint{3.028706in}{1.042866in}}%
\pgfpathlineto{\pgfqpoint{3.029222in}{1.045379in}}%
\pgfpathlineto{\pgfqpoint{3.031287in}{1.072384in}}%
\pgfpathlineto{\pgfqpoint{3.031803in}{1.069247in}}%
\pgfpathlineto{\pgfqpoint{3.033610in}{1.062495in}}%
\pgfpathlineto{\pgfqpoint{3.034643in}{1.056301in}}%
\pgfpathlineto{\pgfqpoint{3.038773in}{1.030170in}}%
\pgfpathlineto{\pgfqpoint{3.039289in}{1.029306in}}%
\pgfpathlineto{\pgfqpoint{3.039547in}{1.029806in}}%
\pgfpathlineto{\pgfqpoint{3.041096in}{1.036177in}}%
\pgfpathlineto{\pgfqpoint{3.041612in}{1.035328in}}%
\pgfpathlineto{\pgfqpoint{3.043419in}{1.022899in}}%
\pgfpathlineto{\pgfqpoint{3.050389in}{0.975812in}}%
\pgfpathlineto{\pgfqpoint{3.053487in}{0.959776in}}%
\pgfpathlineto{\pgfqpoint{3.054003in}{0.961986in}}%
\pgfpathlineto{\pgfqpoint{3.056326in}{0.995809in}}%
\pgfpathlineto{\pgfqpoint{3.057359in}{0.990538in}}%
\pgfpathlineto{\pgfqpoint{3.061231in}{0.965329in}}%
\pgfpathlineto{\pgfqpoint{3.061489in}{0.965459in}}%
\pgfpathlineto{\pgfqpoint{3.062263in}{0.970246in}}%
\pgfpathlineto{\pgfqpoint{3.065877in}{1.009718in}}%
\pgfpathlineto{\pgfqpoint{3.066652in}{1.007121in}}%
\pgfpathlineto{\pgfqpoint{3.071298in}{0.984821in}}%
\pgfpathlineto{\pgfqpoint{3.071814in}{0.986643in}}%
\pgfpathlineto{\pgfqpoint{3.075428in}{1.007019in}}%
\pgfpathlineto{\pgfqpoint{3.077235in}{1.007635in}}%
\pgfpathlineto{\pgfqpoint{3.078526in}{1.005614in}}%
\pgfpathlineto{\pgfqpoint{3.080075in}{0.996705in}}%
\pgfpathlineto{\pgfqpoint{3.084205in}{0.973122in}}%
\pgfpathlineto{\pgfqpoint{3.084979in}{0.971945in}}%
\pgfpathlineto{\pgfqpoint{3.085237in}{0.972306in}}%
\pgfpathlineto{\pgfqpoint{3.087044in}{0.976420in}}%
\pgfpathlineto{\pgfqpoint{3.087560in}{0.975389in}}%
\pgfpathlineto{\pgfqpoint{3.092981in}{0.953560in}}%
\pgfpathlineto{\pgfqpoint{3.094530in}{0.946447in}}%
\pgfpathlineto{\pgfqpoint{3.094788in}{0.947291in}}%
\pgfpathlineto{\pgfqpoint{3.095821in}{0.961069in}}%
\pgfpathlineto{\pgfqpoint{3.098144in}{0.989193in}}%
\pgfpathlineto{\pgfqpoint{3.099177in}{0.987591in}}%
\pgfpathlineto{\pgfqpoint{3.100725in}{0.982316in}}%
\pgfpathlineto{\pgfqpoint{3.102532in}{0.973441in}}%
\pgfpathlineto{\pgfqpoint{3.102790in}{0.973721in}}%
\pgfpathlineto{\pgfqpoint{3.105372in}{0.982030in}}%
\pgfpathlineto{\pgfqpoint{3.106404in}{0.983915in}}%
\pgfpathlineto{\pgfqpoint{3.106921in}{0.983491in}}%
\pgfpathlineto{\pgfqpoint{3.109760in}{0.975999in}}%
\pgfpathlineto{\pgfqpoint{3.111309in}{0.967538in}}%
\pgfpathlineto{\pgfqpoint{3.111825in}{0.968704in}}%
\pgfpathlineto{\pgfqpoint{3.113374in}{0.975801in}}%
\pgfpathlineto{\pgfqpoint{3.113890in}{0.975104in}}%
\pgfpathlineto{\pgfqpoint{3.116472in}{0.964687in}}%
\pgfpathlineto{\pgfqpoint{3.117246in}{0.966175in}}%
\pgfpathlineto{\pgfqpoint{3.118537in}{0.970454in}}%
\pgfpathlineto{\pgfqpoint{3.118795in}{0.970004in}}%
\pgfpathlineto{\pgfqpoint{3.120602in}{0.961043in}}%
\pgfpathlineto{\pgfqpoint{3.121376in}{0.963443in}}%
\pgfpathlineto{\pgfqpoint{3.122925in}{0.969017in}}%
\pgfpathlineto{\pgfqpoint{3.123183in}{0.968743in}}%
\pgfpathlineto{\pgfqpoint{3.124474in}{0.963103in}}%
\pgfpathlineto{\pgfqpoint{3.134025in}{0.908745in}}%
\pgfpathlineto{\pgfqpoint{3.135057in}{0.913745in}}%
\pgfpathlineto{\pgfqpoint{3.135832in}{0.917185in}}%
\pgfpathlineto{\pgfqpoint{3.136348in}{0.916373in}}%
\pgfpathlineto{\pgfqpoint{3.137897in}{0.909830in}}%
\pgfpathlineto{\pgfqpoint{3.138413in}{0.910436in}}%
\pgfpathlineto{\pgfqpoint{3.138929in}{0.910787in}}%
\pgfpathlineto{\pgfqpoint{3.139187in}{0.910303in}}%
\pgfpathlineto{\pgfqpoint{3.140736in}{0.903720in}}%
\pgfpathlineto{\pgfqpoint{3.141252in}{0.904795in}}%
\pgfpathlineto{\pgfqpoint{3.142285in}{0.907479in}}%
\pgfpathlineto{\pgfqpoint{3.142543in}{0.906915in}}%
\pgfpathlineto{\pgfqpoint{3.143834in}{0.901059in}}%
\pgfpathlineto{\pgfqpoint{3.144350in}{0.902291in}}%
\pgfpathlineto{\pgfqpoint{3.146157in}{0.914720in}}%
\pgfpathlineto{\pgfqpoint{3.146673in}{0.913089in}}%
\pgfpathlineto{\pgfqpoint{3.147706in}{0.909023in}}%
\pgfpathlineto{\pgfqpoint{3.147964in}{0.910019in}}%
\pgfpathlineto{\pgfqpoint{3.150029in}{0.927761in}}%
\pgfpathlineto{\pgfqpoint{3.150803in}{0.924104in}}%
\pgfpathlineto{\pgfqpoint{3.152869in}{0.918224in}}%
\pgfpathlineto{\pgfqpoint{3.154159in}{0.912617in}}%
\pgfpathlineto{\pgfqpoint{3.157257in}{0.896703in}}%
\pgfpathlineto{\pgfqpoint{3.157773in}{0.897490in}}%
\pgfpathlineto{\pgfqpoint{3.160096in}{0.911513in}}%
\pgfpathlineto{\pgfqpoint{3.160871in}{0.908246in}}%
\pgfpathlineto{\pgfqpoint{3.161645in}{0.905028in}}%
\pgfpathlineto{\pgfqpoint{3.162161in}{0.906837in}}%
\pgfpathlineto{\pgfqpoint{3.163968in}{0.921935in}}%
\pgfpathlineto{\pgfqpoint{3.164485in}{0.920209in}}%
\pgfpathlineto{\pgfqpoint{3.168357in}{0.904000in}}%
\pgfpathlineto{\pgfqpoint{3.169131in}{0.910062in}}%
\pgfpathlineto{\pgfqpoint{3.170938in}{0.933335in}}%
\pgfpathlineto{\pgfqpoint{3.171454in}{0.931757in}}%
\pgfpathlineto{\pgfqpoint{3.172487in}{0.927641in}}%
\pgfpathlineto{\pgfqpoint{3.172745in}{0.928673in}}%
\pgfpathlineto{\pgfqpoint{3.175068in}{0.947883in}}%
\pgfpathlineto{\pgfqpoint{3.175843in}{0.945440in}}%
\pgfpathlineto{\pgfqpoint{3.183328in}{0.908957in}}%
\pgfpathlineto{\pgfqpoint{3.184361in}{0.908558in}}%
\pgfpathlineto{\pgfqpoint{3.185652in}{0.903023in}}%
\pgfpathlineto{\pgfqpoint{3.189007in}{0.888881in}}%
\pgfpathlineto{\pgfqpoint{3.189524in}{0.890346in}}%
\pgfpathlineto{\pgfqpoint{3.191847in}{0.919481in}}%
\pgfpathlineto{\pgfqpoint{3.192879in}{0.915007in}}%
\pgfpathlineto{\pgfqpoint{3.193396in}{0.913531in}}%
\pgfpathlineto{\pgfqpoint{3.193654in}{0.914089in}}%
\pgfpathlineto{\pgfqpoint{3.195203in}{0.929616in}}%
\pgfpathlineto{\pgfqpoint{3.196235in}{0.934626in}}%
\pgfpathlineto{\pgfqpoint{3.196751in}{0.933886in}}%
\pgfpathlineto{\pgfqpoint{3.202430in}{0.915614in}}%
\pgfpathlineto{\pgfqpoint{3.202689in}{0.916406in}}%
\pgfpathlineto{\pgfqpoint{3.203721in}{0.931480in}}%
\pgfpathlineto{\pgfqpoint{3.205786in}{0.960211in}}%
\pgfpathlineto{\pgfqpoint{3.206044in}{0.959749in}}%
\pgfpathlineto{\pgfqpoint{3.210433in}{0.937164in}}%
\pgfpathlineto{\pgfqpoint{3.215079in}{0.912202in}}%
\pgfpathlineto{\pgfqpoint{3.215337in}{0.912295in}}%
\pgfpathlineto{\pgfqpoint{3.216370in}{0.916182in}}%
\pgfpathlineto{\pgfqpoint{3.217660in}{0.920247in}}%
\pgfpathlineto{\pgfqpoint{3.217919in}{0.919669in}}%
\pgfpathlineto{\pgfqpoint{3.219725in}{0.911694in}}%
\pgfpathlineto{\pgfqpoint{3.220242in}{0.912979in}}%
\pgfpathlineto{\pgfqpoint{3.222823in}{0.931325in}}%
\pgfpathlineto{\pgfqpoint{3.223856in}{0.928558in}}%
\pgfpathlineto{\pgfqpoint{3.229793in}{0.900893in}}%
\pgfpathlineto{\pgfqpoint{3.231600in}{0.898062in}}%
\pgfpathlineto{\pgfqpoint{3.236504in}{0.880404in}}%
\pgfpathlineto{\pgfqpoint{3.237537in}{0.878170in}}%
\pgfpathlineto{\pgfqpoint{3.237795in}{0.878587in}}%
\pgfpathlineto{\pgfqpoint{3.239602in}{0.887531in}}%
\pgfpathlineto{\pgfqpoint{3.241151in}{0.891282in}}%
\pgfpathlineto{\pgfqpoint{3.241409in}{0.891101in}}%
\pgfpathlineto{\pgfqpoint{3.243216in}{0.887010in}}%
\pgfpathlineto{\pgfqpoint{3.245797in}{0.876867in}}%
\pgfpathlineto{\pgfqpoint{3.247604in}{0.870778in}}%
\pgfpathlineto{\pgfqpoint{3.247862in}{0.870879in}}%
\pgfpathlineto{\pgfqpoint{3.248895in}{0.873620in}}%
\pgfpathlineto{\pgfqpoint{3.250702in}{0.885885in}}%
\pgfpathlineto{\pgfqpoint{3.253541in}{0.914177in}}%
\pgfpathlineto{\pgfqpoint{3.254316in}{0.912341in}}%
\pgfpathlineto{\pgfqpoint{3.256122in}{0.905195in}}%
\pgfpathlineto{\pgfqpoint{3.256381in}{0.905388in}}%
\pgfpathlineto{\pgfqpoint{3.259220in}{0.907973in}}%
\pgfpathlineto{\pgfqpoint{3.260253in}{0.908973in}}%
\pgfpathlineto{\pgfqpoint{3.261543in}{0.916132in}}%
\pgfpathlineto{\pgfqpoint{3.263350in}{0.926768in}}%
\pgfpathlineto{\pgfqpoint{3.263608in}{0.926593in}}%
\pgfpathlineto{\pgfqpoint{3.264899in}{0.920264in}}%
\pgfpathlineto{\pgfqpoint{3.266706in}{0.915128in}}%
\pgfpathlineto{\pgfqpoint{3.269029in}{0.915280in}}%
\pgfpathlineto{\pgfqpoint{3.271611in}{0.910029in}}%
\pgfpathlineto{\pgfqpoint{3.273934in}{0.899913in}}%
\pgfpathlineto{\pgfqpoint{3.278580in}{0.879165in}}%
\pgfpathlineto{\pgfqpoint{3.282710in}{0.859518in}}%
\pgfpathlineto{\pgfqpoint{3.294326in}{0.810294in}}%
\pgfpathlineto{\pgfqpoint{3.298715in}{0.797264in}}%
\pgfpathlineto{\pgfqpoint{3.307491in}{0.770000in}}%
\pgfpathlineto{\pgfqpoint{3.307749in}{0.770392in}}%
\pgfpathlineto{\pgfqpoint{3.309556in}{0.775298in}}%
\pgfpathlineto{\pgfqpoint{3.310331in}{0.774233in}}%
\pgfpathlineto{\pgfqpoint{3.317559in}{0.753055in}}%
\pgfpathlineto{\pgfqpoint{3.318075in}{0.754121in}}%
\pgfpathlineto{\pgfqpoint{3.319366in}{0.766683in}}%
\pgfpathlineto{\pgfqpoint{3.321947in}{0.788027in}}%
\pgfpathlineto{\pgfqpoint{3.322979in}{0.785179in}}%
\pgfpathlineto{\pgfqpoint{3.324012in}{0.783306in}}%
\pgfpathlineto{\pgfqpoint{3.324270in}{0.783810in}}%
\pgfpathlineto{\pgfqpoint{3.326335in}{0.792126in}}%
\pgfpathlineto{\pgfqpoint{3.327110in}{0.790899in}}%
\pgfpathlineto{\pgfqpoint{3.331756in}{0.777338in}}%
\pgfpathlineto{\pgfqpoint{3.332272in}{0.777791in}}%
\pgfpathlineto{\pgfqpoint{3.333305in}{0.777735in}}%
\pgfpathlineto{\pgfqpoint{3.335628in}{0.770718in}}%
\pgfpathlineto{\pgfqpoint{3.349309in}{0.736875in}}%
\pgfpathlineto{\pgfqpoint{3.350342in}{0.735669in}}%
\pgfpathlineto{\pgfqpoint{3.355246in}{0.726209in}}%
\pgfpathlineto{\pgfqpoint{3.360925in}{0.717281in}}%
\pgfpathlineto{\pgfqpoint{3.363248in}{0.715177in}}%
\pgfpathlineto{\pgfqpoint{3.364797in}{0.713044in}}%
\pgfpathlineto{\pgfqpoint{3.366862in}{0.710609in}}%
\pgfpathlineto{\pgfqpoint{3.368153in}{0.710209in}}%
\pgfpathlineto{\pgfqpoint{3.377704in}{0.697116in}}%
\pgfpathlineto{\pgfqpoint{3.379511in}{0.696054in}}%
\pgfpathlineto{\pgfqpoint{3.385190in}{0.687913in}}%
\pgfpathlineto{\pgfqpoint{3.387771in}{0.683580in}}%
\pgfpathlineto{\pgfqpoint{3.391901in}{0.678028in}}%
\pgfpathlineto{\pgfqpoint{3.392934in}{0.680325in}}%
\pgfpathlineto{\pgfqpoint{3.395257in}{0.685563in}}%
\pgfpathlineto{\pgfqpoint{3.397064in}{0.682814in}}%
\pgfpathlineto{\pgfqpoint{3.400936in}{0.678487in}}%
\pgfpathlineto{\pgfqpoint{3.409971in}{0.673316in}}%
\pgfpathlineto{\pgfqpoint{3.411262in}{0.672064in}}%
\pgfpathlineto{\pgfqpoint{3.414875in}{0.667655in}}%
\pgfpathlineto{\pgfqpoint{3.415134in}{0.667929in}}%
\pgfpathlineto{\pgfqpoint{3.417199in}{0.671474in}}%
\pgfpathlineto{\pgfqpoint{3.417715in}{0.670995in}}%
\pgfpathlineto{\pgfqpoint{3.419522in}{0.669705in}}%
\pgfpathlineto{\pgfqpoint{3.421587in}{0.670471in}}%
\pgfpathlineto{\pgfqpoint{3.426233in}{0.665172in}}%
\pgfpathlineto{\pgfqpoint{3.426750in}{0.665973in}}%
\pgfpathlineto{\pgfqpoint{3.428557in}{0.668426in}}%
\pgfpathlineto{\pgfqpoint{3.428815in}{0.668187in}}%
\pgfpathlineto{\pgfqpoint{3.431654in}{0.666106in}}%
\pgfpathlineto{\pgfqpoint{3.433461in}{0.664237in}}%
\pgfpathlineto{\pgfqpoint{3.437075in}{0.660852in}}%
\pgfpathlineto{\pgfqpoint{3.440431in}{0.661611in}}%
\pgfpathlineto{\pgfqpoint{3.442238in}{0.659955in}}%
\pgfpathlineto{\pgfqpoint{3.442496in}{0.660180in}}%
\pgfpathlineto{\pgfqpoint{3.443787in}{0.663864in}}%
\pgfpathlineto{\pgfqpoint{3.446110in}{0.669850in}}%
\pgfpathlineto{\pgfqpoint{3.446368in}{0.669679in}}%
\pgfpathlineto{\pgfqpoint{3.448175in}{0.668333in}}%
\pgfpathlineto{\pgfqpoint{3.448433in}{0.668725in}}%
\pgfpathlineto{\pgfqpoint{3.450240in}{0.671244in}}%
\pgfpathlineto{\pgfqpoint{3.450498in}{0.671047in}}%
\pgfpathlineto{\pgfqpoint{3.453079in}{0.669408in}}%
\pgfpathlineto{\pgfqpoint{3.454628in}{0.669293in}}%
\pgfpathlineto{\pgfqpoint{3.459016in}{0.664046in}}%
\pgfpathlineto{\pgfqpoint{3.459533in}{0.664809in}}%
\pgfpathlineto{\pgfqpoint{3.461340in}{0.673320in}}%
\pgfpathlineto{\pgfqpoint{3.462888in}{0.678214in}}%
\pgfpathlineto{\pgfqpoint{3.463147in}{0.678160in}}%
\pgfpathlineto{\pgfqpoint{3.466244in}{0.676413in}}%
\pgfpathlineto{\pgfqpoint{3.470633in}{0.677747in}}%
\pgfpathlineto{\pgfqpoint{3.474246in}{0.673768in}}%
\pgfpathlineto{\pgfqpoint{3.477602in}{0.670817in}}%
\pgfpathlineto{\pgfqpoint{3.477860in}{0.671165in}}%
\pgfpathlineto{\pgfqpoint{3.480184in}{0.673109in}}%
\pgfpathlineto{\pgfqpoint{3.483281in}{0.671765in}}%
\pgfpathlineto{\pgfqpoint{3.485088in}{0.672608in}}%
\pgfpathlineto{\pgfqpoint{3.486895in}{0.670888in}}%
\pgfpathlineto{\pgfqpoint{3.487153in}{0.671212in}}%
\pgfpathlineto{\pgfqpoint{3.489993in}{0.676476in}}%
\pgfpathlineto{\pgfqpoint{3.490509in}{0.675970in}}%
\pgfpathlineto{\pgfqpoint{3.500576in}{0.664488in}}%
\pgfpathlineto{\pgfqpoint{3.504190in}{0.661229in}}%
\pgfpathlineto{\pgfqpoint{3.510902in}{0.654168in}}%
\pgfpathlineto{\pgfqpoint{3.513225in}{0.652607in}}%
\pgfpathlineto{\pgfqpoint{3.518646in}{0.647410in}}%
\pgfpathlineto{\pgfqpoint{3.519678in}{0.648871in}}%
\pgfpathlineto{\pgfqpoint{3.522001in}{0.653265in}}%
\pgfpathlineto{\pgfqpoint{3.522260in}{0.653117in}}%
\pgfpathlineto{\pgfqpoint{3.528455in}{0.647909in}}%
\pgfpathlineto{\pgfqpoint{3.528713in}{0.648188in}}%
\pgfpathlineto{\pgfqpoint{3.531036in}{0.649670in}}%
\pgfpathlineto{\pgfqpoint{3.535683in}{0.645838in}}%
\pgfpathlineto{\pgfqpoint{3.538780in}{0.644003in}}%
\pgfpathlineto{\pgfqpoint{3.539038in}{0.644335in}}%
\pgfpathlineto{\pgfqpoint{3.541103in}{0.645952in}}%
\pgfpathlineto{\pgfqpoint{3.556591in}{0.635462in}}%
\pgfpathlineto{\pgfqpoint{3.558140in}{0.635556in}}%
\pgfpathlineto{\pgfqpoint{3.571047in}{0.626924in}}%
\pgfpathlineto{\pgfqpoint{3.575177in}{0.625124in}}%
\pgfpathlineto{\pgfqpoint{3.579824in}{0.627322in}}%
\pgfpathlineto{\pgfqpoint{3.587568in}{0.624383in}}%
\pgfpathlineto{\pgfqpoint{3.589116in}{0.628185in}}%
\pgfpathlineto{\pgfqpoint{3.591440in}{0.631774in}}%
\pgfpathlineto{\pgfqpoint{3.596344in}{0.630322in}}%
\pgfpathlineto{\pgfqpoint{3.597635in}{0.634823in}}%
\pgfpathlineto{\pgfqpoint{3.600732in}{0.649207in}}%
\pgfpathlineto{\pgfqpoint{3.600991in}{0.649114in}}%
\pgfpathlineto{\pgfqpoint{3.611058in}{0.642351in}}%
\pgfpathlineto{\pgfqpoint{3.612349in}{0.645439in}}%
\pgfpathlineto{\pgfqpoint{3.614930in}{0.651162in}}%
\pgfpathlineto{\pgfqpoint{3.617253in}{0.649004in}}%
\pgfpathlineto{\pgfqpoint{3.619576in}{0.647781in}}%
\pgfpathlineto{\pgfqpoint{3.621641in}{0.651756in}}%
\pgfpathlineto{\pgfqpoint{3.625513in}{0.657631in}}%
\pgfpathlineto{\pgfqpoint{3.627320in}{0.657628in}}%
\pgfpathlineto{\pgfqpoint{3.630934in}{0.655220in}}%
\pgfpathlineto{\pgfqpoint{3.638162in}{0.658074in}}%
\pgfpathlineto{\pgfqpoint{3.644099in}{0.651876in}}%
\pgfpathlineto{\pgfqpoint{3.644357in}{0.652277in}}%
\pgfpathlineto{\pgfqpoint{3.645906in}{0.658873in}}%
\pgfpathlineto{\pgfqpoint{3.648746in}{0.668858in}}%
\pgfpathlineto{\pgfqpoint{3.650036in}{0.667823in}}%
\pgfpathlineto{\pgfqpoint{3.653650in}{0.664319in}}%
\pgfpathlineto{\pgfqpoint{3.656490in}{0.666941in}}%
\pgfpathlineto{\pgfqpoint{3.658813in}{0.670928in}}%
\pgfpathlineto{\pgfqpoint{3.661652in}{0.680358in}}%
\pgfpathlineto{\pgfqpoint{3.662427in}{0.679364in}}%
\pgfpathlineto{\pgfqpoint{3.662943in}{0.679108in}}%
\pgfpathlineto{\pgfqpoint{3.663201in}{0.679733in}}%
\pgfpathlineto{\pgfqpoint{3.664234in}{0.689332in}}%
\pgfpathlineto{\pgfqpoint{3.666557in}{0.709593in}}%
\pgfpathlineto{\pgfqpoint{3.667589in}{0.708277in}}%
\pgfpathlineto{\pgfqpoint{3.673010in}{0.698719in}}%
\pgfpathlineto{\pgfqpoint{3.673785in}{0.700795in}}%
\pgfpathlineto{\pgfqpoint{3.676366in}{0.712539in}}%
\pgfpathlineto{\pgfqpoint{3.676882in}{0.711800in}}%
\pgfpathlineto{\pgfqpoint{3.678431in}{0.709712in}}%
\pgfpathlineto{\pgfqpoint{3.678689in}{0.709961in}}%
\pgfpathlineto{\pgfqpoint{3.683078in}{0.716220in}}%
\pgfpathlineto{\pgfqpoint{3.683594in}{0.715618in}}%
\pgfpathlineto{\pgfqpoint{3.688240in}{0.706928in}}%
\pgfpathlineto{\pgfqpoint{3.689015in}{0.708528in}}%
\pgfpathlineto{\pgfqpoint{3.693145in}{0.730225in}}%
\pgfpathlineto{\pgfqpoint{3.697533in}{0.784899in}}%
\pgfpathlineto{\pgfqpoint{3.698049in}{0.783884in}}%
\pgfpathlineto{\pgfqpoint{3.698824in}{0.782037in}}%
\pgfpathlineto{\pgfqpoint{3.699082in}{0.783339in}}%
\pgfpathlineto{\pgfqpoint{3.699856in}{0.802932in}}%
\pgfpathlineto{\pgfqpoint{3.703728in}{0.978754in}}%
\pgfpathlineto{\pgfqpoint{3.704245in}{0.977745in}}%
\pgfpathlineto{\pgfqpoint{3.708117in}{0.959652in}}%
\pgfpathlineto{\pgfqpoint{3.708633in}{0.960300in}}%
\pgfpathlineto{\pgfqpoint{3.709407in}{0.962650in}}%
\pgfpathlineto{\pgfqpoint{3.709924in}{0.961335in}}%
\pgfpathlineto{\pgfqpoint{3.710956in}{0.956564in}}%
\pgfpathlineto{\pgfqpoint{3.711214in}{0.958122in}}%
\pgfpathlineto{\pgfqpoint{3.712505in}{0.988039in}}%
\pgfpathlineto{\pgfqpoint{3.714570in}{1.013530in}}%
\pgfpathlineto{\pgfqpoint{3.715861in}{1.015283in}}%
\pgfpathlineto{\pgfqpoint{3.717668in}{1.022494in}}%
\pgfpathlineto{\pgfqpoint{3.718184in}{1.021083in}}%
\pgfpathlineto{\pgfqpoint{3.719733in}{1.011570in}}%
\pgfpathlineto{\pgfqpoint{3.720249in}{1.013423in}}%
\pgfpathlineto{\pgfqpoint{3.721798in}{1.039611in}}%
\pgfpathlineto{\pgfqpoint{3.723088in}{1.052023in}}%
\pgfpathlineto{\pgfqpoint{3.723347in}{1.051860in}}%
\pgfpathlineto{\pgfqpoint{3.724637in}{1.044861in}}%
\pgfpathlineto{\pgfqpoint{3.728767in}{1.019970in}}%
\pgfpathlineto{\pgfqpoint{3.729026in}{1.020427in}}%
\pgfpathlineto{\pgfqpoint{3.730832in}{1.031075in}}%
\pgfpathlineto{\pgfqpoint{3.731349in}{1.029189in}}%
\pgfpathlineto{\pgfqpoint{3.736770in}{0.998163in}}%
\pgfpathlineto{\pgfqpoint{3.738835in}{0.997310in}}%
\pgfpathlineto{\pgfqpoint{3.744255in}{0.961815in}}%
\pgfpathlineto{\pgfqpoint{3.745288in}{0.957043in}}%
\pgfpathlineto{\pgfqpoint{3.745804in}{0.958089in}}%
\pgfpathlineto{\pgfqpoint{3.747095in}{0.962440in}}%
\pgfpathlineto{\pgfqpoint{3.747353in}{0.961998in}}%
\pgfpathlineto{\pgfqpoint{3.749160in}{0.951186in}}%
\pgfpathlineto{\pgfqpoint{3.754839in}{0.919904in}}%
\pgfpathlineto{\pgfqpoint{3.755872in}{0.921129in}}%
\pgfpathlineto{\pgfqpoint{3.756388in}{0.921145in}}%
\pgfpathlineto{\pgfqpoint{3.756646in}{0.920560in}}%
\pgfpathlineto{\pgfqpoint{3.759744in}{0.904568in}}%
\pgfpathlineto{\pgfqpoint{3.762067in}{0.897382in}}%
\pgfpathlineto{\pgfqpoint{3.763099in}{0.894159in}}%
\pgfpathlineto{\pgfqpoint{3.764648in}{0.888205in}}%
\pgfpathlineto{\pgfqpoint{3.765164in}{0.888730in}}%
\pgfpathlineto{\pgfqpoint{3.765681in}{0.889238in}}%
\pgfpathlineto{\pgfqpoint{3.766197in}{0.888405in}}%
\pgfpathlineto{\pgfqpoint{3.767746in}{0.881880in}}%
\pgfpathlineto{\pgfqpoint{3.768520in}{0.883585in}}%
\pgfpathlineto{\pgfqpoint{3.769811in}{0.887398in}}%
\pgfpathlineto{\pgfqpoint{3.770069in}{0.886968in}}%
\pgfpathlineto{\pgfqpoint{3.774715in}{0.875285in}}%
\pgfpathlineto{\pgfqpoint{3.776522in}{0.869254in}}%
\pgfpathlineto{\pgfqpoint{3.780394in}{0.857668in}}%
\pgfpathlineto{\pgfqpoint{3.784266in}{0.845202in}}%
\pgfpathlineto{\pgfqpoint{3.784783in}{0.845912in}}%
\pgfpathlineto{\pgfqpoint{3.786073in}{0.852911in}}%
\pgfpathlineto{\pgfqpoint{3.788397in}{0.865409in}}%
\pgfpathlineto{\pgfqpoint{3.788655in}{0.865076in}}%
\pgfpathlineto{\pgfqpoint{3.790462in}{0.857404in}}%
\pgfpathlineto{\pgfqpoint{3.791236in}{0.855853in}}%
\pgfpathlineto{\pgfqpoint{3.791752in}{0.856247in}}%
\pgfpathlineto{\pgfqpoint{3.792785in}{0.857279in}}%
\pgfpathlineto{\pgfqpoint{3.793043in}{0.857002in}}%
\pgfpathlineto{\pgfqpoint{3.794334in}{0.852128in}}%
\pgfpathlineto{\pgfqpoint{3.799238in}{0.832818in}}%
\pgfpathlineto{\pgfqpoint{3.801820in}{0.831592in}}%
\pgfpathlineto{\pgfqpoint{3.803110in}{0.830743in}}%
\pgfpathlineto{\pgfqpoint{3.805175in}{0.823947in}}%
\pgfpathlineto{\pgfqpoint{3.805950in}{0.825507in}}%
\pgfpathlineto{\pgfqpoint{3.807498in}{0.830289in}}%
\pgfpathlineto{\pgfqpoint{3.808015in}{0.829744in}}%
\pgfpathlineto{\pgfqpoint{3.810338in}{0.820846in}}%
\pgfpathlineto{\pgfqpoint{3.813952in}{0.810669in}}%
\pgfpathlineto{\pgfqpoint{3.818340in}{0.801586in}}%
\pgfpathlineto{\pgfqpoint{3.818598in}{0.802231in}}%
\pgfpathlineto{\pgfqpoint{3.819631in}{0.811936in}}%
\pgfpathlineto{\pgfqpoint{3.822470in}{0.837702in}}%
\pgfpathlineto{\pgfqpoint{3.823245in}{0.836539in}}%
\pgfpathlineto{\pgfqpoint{3.828924in}{0.816740in}}%
\pgfpathlineto{\pgfqpoint{3.829698in}{0.820013in}}%
\pgfpathlineto{\pgfqpoint{3.831505in}{0.851184in}}%
\pgfpathlineto{\pgfqpoint{3.833312in}{0.867649in}}%
\pgfpathlineto{\pgfqpoint{3.834086in}{0.865783in}}%
\pgfpathlineto{\pgfqpoint{3.840282in}{0.840977in}}%
\pgfpathlineto{\pgfqpoint{3.840540in}{0.841418in}}%
\pgfpathlineto{\pgfqpoint{3.841572in}{0.849426in}}%
\pgfpathlineto{\pgfqpoint{3.844670in}{0.883150in}}%
\pgfpathlineto{\pgfqpoint{3.844928in}{0.882735in}}%
\pgfpathlineto{\pgfqpoint{3.846735in}{0.874442in}}%
\pgfpathlineto{\pgfqpoint{3.851381in}{0.854118in}}%
\pgfpathlineto{\pgfqpoint{3.852156in}{0.857080in}}%
\pgfpathlineto{\pgfqpoint{3.854479in}{0.869646in}}%
\pgfpathlineto{\pgfqpoint{3.854737in}{0.869372in}}%
\pgfpathlineto{\pgfqpoint{3.856286in}{0.863134in}}%
\pgfpathlineto{\pgfqpoint{3.864804in}{0.825365in}}%
\pgfpathlineto{\pgfqpoint{3.879260in}{0.782476in}}%
\pgfpathlineto{\pgfqpoint{3.880034in}{0.783266in}}%
\pgfpathlineto{\pgfqpoint{3.881067in}{0.784343in}}%
\pgfpathlineto{\pgfqpoint{3.881325in}{0.784093in}}%
\pgfpathlineto{\pgfqpoint{3.882874in}{0.779961in}}%
\pgfpathlineto{\pgfqpoint{3.883390in}{0.781249in}}%
\pgfpathlineto{\pgfqpoint{3.886230in}{0.803757in}}%
\pgfpathlineto{\pgfqpoint{3.887520in}{0.801278in}}%
\pgfpathlineto{\pgfqpoint{3.892683in}{0.786681in}}%
\pgfpathlineto{\pgfqpoint{3.892941in}{0.786958in}}%
\pgfpathlineto{\pgfqpoint{3.895006in}{0.790813in}}%
\pgfpathlineto{\pgfqpoint{3.895522in}{0.790232in}}%
\pgfpathlineto{\pgfqpoint{3.899911in}{0.779748in}}%
\pgfpathlineto{\pgfqpoint{3.902492in}{0.774297in}}%
\pgfpathlineto{\pgfqpoint{3.905590in}{0.774359in}}%
\pgfpathlineto{\pgfqpoint{3.907397in}{0.771694in}}%
\pgfpathlineto{\pgfqpoint{3.912043in}{0.756956in}}%
\pgfpathlineto{\pgfqpoint{3.914108in}{0.752943in}}%
\pgfpathlineto{\pgfqpoint{3.914366in}{0.753150in}}%
\pgfpathlineto{\pgfqpoint{3.917980in}{0.756924in}}%
\pgfpathlineto{\pgfqpoint{3.919013in}{0.757760in}}%
\pgfpathlineto{\pgfqpoint{3.921336in}{0.764032in}}%
\pgfpathlineto{\pgfqpoint{3.921852in}{0.763271in}}%
\pgfpathlineto{\pgfqpoint{3.925724in}{0.754940in}}%
\pgfpathlineto{\pgfqpoint{3.925982in}{0.755098in}}%
\pgfpathlineto{\pgfqpoint{3.927015in}{0.758453in}}%
\pgfpathlineto{\pgfqpoint{3.928822in}{0.776744in}}%
\pgfpathlineto{\pgfqpoint{3.931661in}{0.798552in}}%
\pgfpathlineto{\pgfqpoint{3.932952in}{0.798789in}}%
\pgfpathlineto{\pgfqpoint{3.934243in}{0.796127in}}%
\pgfpathlineto{\pgfqpoint{3.936308in}{0.790710in}}%
\pgfpathlineto{\pgfqpoint{3.936566in}{0.791134in}}%
\pgfpathlineto{\pgfqpoint{3.937857in}{0.797870in}}%
\pgfpathlineto{\pgfqpoint{3.942761in}{0.830665in}}%
\pgfpathlineto{\pgfqpoint{3.943277in}{0.830110in}}%
\pgfpathlineto{\pgfqpoint{3.945342in}{0.823494in}}%
\pgfpathlineto{\pgfqpoint{3.948698in}{0.813130in}}%
\pgfpathlineto{\pgfqpoint{3.949731in}{0.814490in}}%
\pgfpathlineto{\pgfqpoint{3.952312in}{0.817779in}}%
\pgfpathlineto{\pgfqpoint{3.954119in}{0.816286in}}%
\pgfpathlineto{\pgfqpoint{3.956184in}{0.811064in}}%
\pgfpathlineto{\pgfqpoint{3.963154in}{0.789902in}}%
\pgfpathlineto{\pgfqpoint{3.963412in}{0.790098in}}%
\pgfpathlineto{\pgfqpoint{3.964444in}{0.794798in}}%
\pgfpathlineto{\pgfqpoint{3.969865in}{0.831699in}}%
\pgfpathlineto{\pgfqpoint{3.970640in}{0.831168in}}%
\pgfpathlineto{\pgfqpoint{3.972447in}{0.824502in}}%
\pgfpathlineto{\pgfqpoint{3.973737in}{0.820881in}}%
\pgfpathlineto{\pgfqpoint{3.974254in}{0.821630in}}%
\pgfpathlineto{\pgfqpoint{3.975286in}{0.829197in}}%
\pgfpathlineto{\pgfqpoint{3.978126in}{0.877905in}}%
\pgfpathlineto{\pgfqpoint{3.980191in}{0.894676in}}%
\pgfpathlineto{\pgfqpoint{3.980965in}{0.892909in}}%
\pgfpathlineto{\pgfqpoint{3.985095in}{0.877181in}}%
\pgfpathlineto{\pgfqpoint{3.985353in}{0.877425in}}%
\pgfpathlineto{\pgfqpoint{3.986644in}{0.881111in}}%
\pgfpathlineto{\pgfqpoint{3.989225in}{0.897046in}}%
\pgfpathlineto{\pgfqpoint{3.991807in}{0.908797in}}%
\pgfpathlineto{\pgfqpoint{3.992839in}{0.905877in}}%
\pgfpathlineto{\pgfqpoint{3.996453in}{0.891464in}}%
\pgfpathlineto{\pgfqpoint{3.996711in}{0.891561in}}%
\pgfpathlineto{\pgfqpoint{3.998002in}{0.894442in}}%
\pgfpathlineto{\pgfqpoint{4.000841in}{0.900434in}}%
\pgfpathlineto{\pgfqpoint{4.002648in}{0.899386in}}%
\pgfpathlineto{\pgfqpoint{4.004197in}{0.896150in}}%
\pgfpathlineto{\pgfqpoint{4.009618in}{0.879642in}}%
\pgfpathlineto{\pgfqpoint{4.012716in}{0.878675in}}%
\pgfpathlineto{\pgfqpoint{4.015039in}{0.882172in}}%
\pgfpathlineto{\pgfqpoint{4.015555in}{0.881362in}}%
\pgfpathlineto{\pgfqpoint{4.017620in}{0.871769in}}%
\pgfpathlineto{\pgfqpoint{4.019943in}{0.866596in}}%
\pgfpathlineto{\pgfqpoint{4.021234in}{0.867526in}}%
\pgfpathlineto{\pgfqpoint{4.023815in}{0.869790in}}%
\pgfpathlineto{\pgfqpoint{4.025106in}{0.867175in}}%
\pgfpathlineto{\pgfqpoint{4.028462in}{0.852175in}}%
\pgfpathlineto{\pgfqpoint{4.030527in}{0.846243in}}%
\pgfpathlineto{\pgfqpoint{4.030785in}{0.846370in}}%
\pgfpathlineto{\pgfqpoint{4.033108in}{0.848508in}}%
\pgfpathlineto{\pgfqpoint{4.033366in}{0.848118in}}%
\pgfpathlineto{\pgfqpoint{4.035173in}{0.842090in}}%
\pgfpathlineto{\pgfqpoint{4.046531in}{0.798090in}}%
\pgfpathlineto{\pgfqpoint{4.055824in}{0.773776in}}%
\pgfpathlineto{\pgfqpoint{4.058922in}{0.770029in}}%
\pgfpathlineto{\pgfqpoint{4.062536in}{0.759361in}}%
\pgfpathlineto{\pgfqpoint{4.065375in}{0.754701in}}%
\pgfpathlineto{\pgfqpoint{4.068731in}{0.754249in}}%
\pgfpathlineto{\pgfqpoint{4.079573in}{0.728963in}}%
\pgfpathlineto{\pgfqpoint{4.084477in}{0.719032in}}%
\pgfpathlineto{\pgfqpoint{4.095061in}{0.701610in}}%
\pgfpathlineto{\pgfqpoint{4.097384in}{0.698023in}}%
\pgfpathlineto{\pgfqpoint{4.100223in}{0.694327in}}%
\pgfpathlineto{\pgfqpoint{4.101514in}{0.696639in}}%
\pgfpathlineto{\pgfqpoint{4.105644in}{0.710086in}}%
\pgfpathlineto{\pgfqpoint{4.106419in}{0.709422in}}%
\pgfpathlineto{\pgfqpoint{4.110807in}{0.702743in}}%
\pgfpathlineto{\pgfqpoint{4.111323in}{0.703384in}}%
\pgfpathlineto{\pgfqpoint{4.112872in}{0.709287in}}%
\pgfpathlineto{\pgfqpoint{4.117518in}{0.733752in}}%
\pgfpathlineto{\pgfqpoint{4.117777in}{0.733644in}}%
\pgfpathlineto{\pgfqpoint{4.119325in}{0.730556in}}%
\pgfpathlineto{\pgfqpoint{4.121390in}{0.726880in}}%
\pgfpathlineto{\pgfqpoint{4.121649in}{0.727103in}}%
\pgfpathlineto{\pgfqpoint{4.122681in}{0.730973in}}%
\pgfpathlineto{\pgfqpoint{4.124746in}{0.752760in}}%
\pgfpathlineto{\pgfqpoint{4.127586in}{0.777238in}}%
\pgfpathlineto{\pgfqpoint{4.128102in}{0.777402in}}%
\pgfpathlineto{\pgfqpoint{4.128360in}{0.777055in}}%
\pgfpathlineto{\pgfqpoint{4.131200in}{0.770702in}}%
\pgfpathlineto{\pgfqpoint{4.131716in}{0.771577in}}%
\pgfpathlineto{\pgfqpoint{4.137395in}{0.784525in}}%
\pgfpathlineto{\pgfqpoint{4.138944in}{0.782931in}}%
\pgfpathlineto{\pgfqpoint{4.141267in}{0.775852in}}%
\pgfpathlineto{\pgfqpoint{4.143590in}{0.771845in}}%
\pgfpathlineto{\pgfqpoint{4.146946in}{0.772156in}}%
\pgfpathlineto{\pgfqpoint{4.149269in}{0.772421in}}%
\pgfpathlineto{\pgfqpoint{4.150818in}{0.769317in}}%
\pgfpathlineto{\pgfqpoint{4.155722in}{0.757617in}}%
\pgfpathlineto{\pgfqpoint{4.161401in}{0.752660in}}%
\pgfpathlineto{\pgfqpoint{4.169145in}{0.734747in}}%
\pgfpathlineto{\pgfqpoint{4.170436in}{0.736892in}}%
\pgfpathlineto{\pgfqpoint{4.171985in}{0.739009in}}%
\pgfpathlineto{\pgfqpoint{4.172243in}{0.738848in}}%
\pgfpathlineto{\pgfqpoint{4.173792in}{0.735479in}}%
\pgfpathlineto{\pgfqpoint{4.177922in}{0.727317in}}%
\pgfpathlineto{\pgfqpoint{4.185924in}{0.714970in}}%
\pgfpathlineto{\pgfqpoint{4.190054in}{0.707885in}}%
\pgfpathlineto{\pgfqpoint{4.191603in}{0.708197in}}%
\pgfpathlineto{\pgfqpoint{4.193668in}{0.708412in}}%
\pgfpathlineto{\pgfqpoint{4.195991in}{0.704094in}}%
\pgfpathlineto{\pgfqpoint{4.199347in}{0.699392in}}%
\pgfpathlineto{\pgfqpoint{4.200638in}{0.700869in}}%
\pgfpathlineto{\pgfqpoint{4.203219in}{0.704105in}}%
\pgfpathlineto{\pgfqpoint{4.204768in}{0.702334in}}%
\pgfpathlineto{\pgfqpoint{4.209931in}{0.695135in}}%
\pgfpathlineto{\pgfqpoint{4.211221in}{0.697377in}}%
\pgfpathlineto{\pgfqpoint{4.215093in}{0.706855in}}%
\pgfpathlineto{\pgfqpoint{4.215352in}{0.706678in}}%
\pgfpathlineto{\pgfqpoint{4.218707in}{0.702227in}}%
\pgfpathlineto{\pgfqpoint{4.219482in}{0.703249in}}%
\pgfpathlineto{\pgfqpoint{4.223096in}{0.713079in}}%
\pgfpathlineto{\pgfqpoint{4.224903in}{0.714964in}}%
\pgfpathlineto{\pgfqpoint{4.226451in}{0.712922in}}%
\pgfpathlineto{\pgfqpoint{4.229033in}{0.708698in}}%
\pgfpathlineto{\pgfqpoint{4.229291in}{0.708886in}}%
\pgfpathlineto{\pgfqpoint{4.230581in}{0.712239in}}%
\pgfpathlineto{\pgfqpoint{4.233679in}{0.720822in}}%
\pgfpathlineto{\pgfqpoint{4.234970in}{0.719826in}}%
\pgfpathlineto{\pgfqpoint{4.240132in}{0.711768in}}%
\pgfpathlineto{\pgfqpoint{4.240649in}{0.712131in}}%
\pgfpathlineto{\pgfqpoint{4.243230in}{0.713306in}}%
\pgfpathlineto{\pgfqpoint{4.245295in}{0.710592in}}%
\pgfpathlineto{\pgfqpoint{4.251232in}{0.701542in}}%
\pgfpathlineto{\pgfqpoint{4.253814in}{0.698736in}}%
\pgfpathlineto{\pgfqpoint{4.264397in}{0.683534in}}%
\pgfpathlineto{\pgfqpoint{4.267753in}{0.679490in}}%
\pgfpathlineto{\pgfqpoint{4.273174in}{0.672501in}}%
\pgfpathlineto{\pgfqpoint{4.279369in}{0.669715in}}%
\pgfpathlineto{\pgfqpoint{4.282983in}{0.666141in}}%
\pgfpathlineto{\pgfqpoint{4.284532in}{0.668128in}}%
\pgfpathlineto{\pgfqpoint{4.287887in}{0.672277in}}%
\pgfpathlineto{\pgfqpoint{4.289953in}{0.669915in}}%
\pgfpathlineto{\pgfqpoint{4.293308in}{0.666753in}}%
\pgfpathlineto{\pgfqpoint{4.295373in}{0.668807in}}%
\pgfpathlineto{\pgfqpoint{4.299503in}{0.673197in}}%
\pgfpathlineto{\pgfqpoint{4.301569in}{0.670489in}}%
\pgfpathlineto{\pgfqpoint{4.303376in}{0.669604in}}%
\pgfpathlineto{\pgfqpoint{4.305699in}{0.672423in}}%
\pgfpathlineto{\pgfqpoint{4.308280in}{0.673942in}}%
\pgfpathlineto{\pgfqpoint{4.310345in}{0.672239in}}%
\pgfpathlineto{\pgfqpoint{4.317057in}{0.664508in}}%
\pgfpathlineto{\pgfqpoint{4.322477in}{0.663478in}}%
\pgfpathlineto{\pgfqpoint{4.328415in}{0.658924in}}%
\pgfpathlineto{\pgfqpoint{4.332028in}{0.659658in}}%
\pgfpathlineto{\pgfqpoint{4.334610in}{0.657073in}}%
\pgfpathlineto{\pgfqpoint{4.340031in}{0.651761in}}%
\pgfpathlineto{\pgfqpoint{4.345193in}{0.649265in}}%
\pgfpathlineto{\pgfqpoint{4.354228in}{0.640603in}}%
\pgfpathlineto{\pgfqpoint{4.358875in}{0.636874in}}%
\pgfpathlineto{\pgfqpoint{4.368942in}{0.629120in}}%
\pgfpathlineto{\pgfqpoint{4.390109in}{0.622701in}}%
\pgfpathlineto{\pgfqpoint{4.392948in}{0.621994in}}%
\pgfpathlineto{\pgfqpoint{4.408695in}{0.613647in}}%
\pgfpathlineto{\pgfqpoint{4.469098in}{0.598494in}}%
\pgfpathlineto{\pgfqpoint{4.471163in}{0.599355in}}%
\pgfpathlineto{\pgfqpoint{4.475293in}{0.602148in}}%
\pgfpathlineto{\pgfqpoint{4.482521in}{0.602567in}}%
\pgfpathlineto{\pgfqpoint{4.485877in}{0.604109in}}%
\pgfpathlineto{\pgfqpoint{4.500332in}{0.604404in}}%
\pgfpathlineto{\pgfqpoint{4.505753in}{0.603668in}}%
\pgfpathlineto{\pgfqpoint{4.511432in}{0.602884in}}%
\pgfpathlineto{\pgfqpoint{4.521241in}{0.601026in}}%
\pgfpathlineto{\pgfqpoint{4.534406in}{0.601357in}}%
\pgfpathlineto{\pgfqpoint{4.538020in}{0.602191in}}%
\pgfpathlineto{\pgfqpoint{4.543699in}{0.601751in}}%
\pgfpathlineto{\pgfqpoint{4.550411in}{0.603153in}}%
\pgfpathlineto{\pgfqpoint{4.555315in}{0.603019in}}%
\pgfpathlineto{\pgfqpoint{4.559962in}{0.605713in}}%
\pgfpathlineto{\pgfqpoint{4.565124in}{0.605359in}}%
\pgfpathlineto{\pgfqpoint{4.569513in}{0.606806in}}%
\pgfpathlineto{\pgfqpoint{4.574159in}{0.606316in}}%
\pgfpathlineto{\pgfqpoint{4.580354in}{0.606838in}}%
\pgfpathlineto{\pgfqpoint{4.583710in}{0.606579in}}%
\pgfpathlineto{\pgfqpoint{4.591970in}{0.607297in}}%
\pgfpathlineto{\pgfqpoint{4.594810in}{0.607286in}}%
\pgfpathlineto{\pgfqpoint{4.600747in}{0.608646in}}%
\pgfpathlineto{\pgfqpoint{4.609265in}{0.607755in}}%
\pgfpathlineto{\pgfqpoint{4.613912in}{0.606703in}}%
\pgfpathlineto{\pgfqpoint{4.618558in}{0.605972in}}%
\pgfpathlineto{\pgfqpoint{4.623205in}{0.605239in}}%
\pgfpathlineto{\pgfqpoint{4.642307in}{0.602535in}}%
\pgfpathlineto{\pgfqpoint{4.649018in}{0.601633in}}%
\pgfpathlineto{\pgfqpoint{4.708905in}{0.593960in}}%
\pgfpathlineto{\pgfqpoint{4.712777in}{0.595137in}}%
\pgfpathlineto{\pgfqpoint{4.716649in}{0.594851in}}%
\pgfpathlineto{\pgfqpoint{4.723361in}{0.595527in}}%
\pgfpathlineto{\pgfqpoint{4.742979in}{0.598242in}}%
\pgfpathlineto{\pgfqpoint{4.745819in}{0.598676in}}%
\pgfpathlineto{\pgfqpoint{4.749949in}{0.598738in}}%
\pgfpathlineto{\pgfqpoint{4.757177in}{0.599696in}}%
\pgfpathlineto{\pgfqpoint{4.774730in}{0.597304in}}%
\pgfpathlineto{\pgfqpoint{4.793832in}{0.594092in}}%
\pgfpathlineto{\pgfqpoint{4.808545in}{0.592809in}}%
\pgfpathlineto{\pgfqpoint{4.849073in}{0.596220in}}%
\pgfpathlineto{\pgfqpoint{4.855010in}{0.596214in}}%
\pgfpathlineto{\pgfqpoint{4.859914in}{0.595871in}}%
\pgfpathlineto{\pgfqpoint{4.866626in}{0.595725in}}%
\pgfpathlineto{\pgfqpoint{4.880049in}{0.595707in}}%
\pgfpathlineto{\pgfqpoint{4.888309in}{0.595454in}}%
\pgfpathlineto{\pgfqpoint{4.908444in}{0.594435in}}%
\pgfpathlineto{\pgfqpoint{4.955166in}{0.589620in}}%
\pgfpathlineto{\pgfqpoint{4.985884in}{0.588531in}}%
\pgfpathlineto{\pgfqpoint{5.019183in}{0.589486in}}%
\pgfpathlineto{\pgfqpoint{5.033381in}{0.589700in}}%
\pgfpathlineto{\pgfqpoint{5.059194in}{0.590097in}}%
\pgfpathlineto{\pgfqpoint{5.074941in}{0.589843in}}%
\pgfpathlineto{\pgfqpoint{5.083201in}{0.589501in}}%
\pgfpathlineto{\pgfqpoint{5.091719in}{0.590083in}}%
\pgfpathlineto{\pgfqpoint{5.110047in}{0.592591in}}%
\pgfpathlineto{\pgfqpoint{5.117275in}{0.594064in}}%
\pgfpathlineto{\pgfqpoint{5.122954in}{0.594179in}}%
\pgfpathlineto{\pgfqpoint{5.127858in}{0.595138in}}%
\pgfpathlineto{\pgfqpoint{5.135344in}{0.595767in}}%
\pgfpathlineto{\pgfqpoint{5.139732in}{0.595486in}}%
\pgfpathlineto{\pgfqpoint{5.152123in}{0.594460in}}%
\pgfpathlineto{\pgfqpoint{5.207106in}{0.590593in}}%
\pgfpathlineto{\pgfqpoint{5.257958in}{0.588730in}}%
\pgfpathlineto{\pgfqpoint{5.281965in}{0.588361in}}%
\pgfpathlineto{\pgfqpoint{5.359405in}{0.589968in}}%
\pgfpathlineto{\pgfqpoint{5.367149in}{0.590540in}}%
\pgfpathlineto{\pgfqpoint{5.452076in}{0.588165in}}%
\pgfpathlineto{\pgfqpoint{5.482277in}{0.589276in}}%
\pgfpathlineto{\pgfqpoint{5.510156in}{0.588815in}}%
\pgfpathlineto{\pgfqpoint{5.547069in}{0.587955in}}%
\pgfpathlineto{\pgfqpoint{5.569269in}{0.588123in}}%
\pgfpathlineto{\pgfqpoint{5.612119in}{0.588384in}}%
\pgfpathlineto{\pgfqpoint{5.620121in}{0.588384in}}%
\pgfpathlineto{\pgfqpoint{5.746091in}{0.587207in}}%
\pgfpathlineto{\pgfqpoint{5.746091in}{0.587207in}}%
\pgfusepath{stroke}%
\end{pgfscope}%
\begin{pgfscope}%
\pgfsetrectcap%
\pgfsetmiterjoin%
\pgfsetlinewidth{0.803000pt}%
\definecolor{currentstroke}{rgb}{0.737255,0.737255,0.737255}%
\pgfsetstrokecolor{currentstroke}%
\pgfsetdash{}{0pt}%
\pgfpathmoveto{\pgfqpoint{0.583136in}{0.544166in}}%
\pgfpathlineto{\pgfqpoint{0.583136in}{1.451024in}}%
\pgfusepath{stroke}%
\end{pgfscope}%
\begin{pgfscope}%
\pgfsetrectcap%
\pgfsetmiterjoin%
\pgfsetlinewidth{0.803000pt}%
\definecolor{currentstroke}{rgb}{0.737255,0.737255,0.737255}%
\pgfsetstrokecolor{currentstroke}%
\pgfsetdash{}{0pt}%
\pgfpathmoveto{\pgfqpoint{5.745833in}{0.544166in}}%
\pgfpathlineto{\pgfqpoint{5.745833in}{1.451024in}}%
\pgfusepath{stroke}%
\end{pgfscope}%
\begin{pgfscope}%
\pgfsetrectcap%
\pgfsetmiterjoin%
\pgfsetlinewidth{0.803000pt}%
\definecolor{currentstroke}{rgb}{0.737255,0.737255,0.737255}%
\pgfsetstrokecolor{currentstroke}%
\pgfsetdash{}{0pt}%
\pgfpathmoveto{\pgfqpoint{0.583136in}{0.544166in}}%
\pgfpathlineto{\pgfqpoint{5.745833in}{0.544166in}}%
\pgfusepath{stroke}%
\end{pgfscope}%
\begin{pgfscope}%
\pgfsetrectcap%
\pgfsetmiterjoin%
\pgfsetlinewidth{0.803000pt}%
\definecolor{currentstroke}{rgb}{0.737255,0.737255,0.737255}%
\pgfsetstrokecolor{currentstroke}%
\pgfsetdash{}{0pt}%
\pgfpathmoveto{\pgfqpoint{0.583136in}{1.451024in}}%
\pgfpathlineto{\pgfqpoint{5.745833in}{1.451024in}}%
\pgfusepath{stroke}%
\end{pgfscope}%
\begin{pgfscope}%
\pgfsetbuttcap%
\pgfsetmiterjoin%
\definecolor{currentfill}{rgb}{0.933333,0.933333,0.933333}%
\pgfsetfillcolor{currentfill}%
\pgfsetfillopacity{0.800000}%
\pgfsetlinewidth{0.501875pt}%
\definecolor{currentstroke}{rgb}{0.800000,0.800000,0.800000}%
\pgfsetstrokecolor{currentstroke}%
\pgfsetstrokeopacity{0.800000}%
\pgfsetdash{}{0pt}%
\pgfpathmoveto{\pgfqpoint{4.635555in}{0.951996in}}%
\pgfpathlineto{\pgfqpoint{5.648611in}{0.951996in}}%
\pgfpathquadraticcurveto{\pgfqpoint{5.676389in}{0.951996in}}{\pgfqpoint{5.676389in}{0.979774in}}%
\pgfpathlineto{\pgfqpoint{5.676389in}{1.353802in}}%
\pgfpathquadraticcurveto{\pgfqpoint{5.676389in}{1.381579in}}{\pgfqpoint{5.648611in}{1.381579in}}%
\pgfpathlineto{\pgfqpoint{4.635555in}{1.381579in}}%
\pgfpathquadraticcurveto{\pgfqpoint{4.607777in}{1.381579in}}{\pgfqpoint{4.607777in}{1.353802in}}%
\pgfpathlineto{\pgfqpoint{4.607777in}{0.979774in}}%
\pgfpathquadraticcurveto{\pgfqpoint{4.607777in}{0.951996in}}{\pgfqpoint{4.635555in}{0.951996in}}%
\pgfpathlineto{\pgfqpoint{4.635555in}{0.951996in}}%
\pgfpathclose%
\pgfusepath{stroke,fill}%
\end{pgfscope}%
\begin{pgfscope}%
\pgfsetrectcap%
\pgfsetroundjoin%
\pgfsetlinewidth{1.505625pt}%
\definecolor{currentstroke}{rgb}{0.172549,0.627451,0.172549}%
\pgfsetstrokecolor{currentstroke}%
\pgfsetdash{}{0pt}%
\pgfpathmoveto{\pgfqpoint{4.663333in}{1.276718in}}%
\pgfpathlineto{\pgfqpoint{4.802222in}{1.276718in}}%
\pgfpathlineto{\pgfqpoint{4.941111in}{1.276718in}}%
\pgfusepath{stroke}%
\end{pgfscope}%
\begin{pgfscope}%
\definecolor{textcolor}{rgb}{0.000000,0.000000,0.000000}%
\pgfsetstrokecolor{textcolor}%
\pgfsetfillcolor{textcolor}%
\pgftext[x=5.052222in,y=1.228107in,left,base]{\color{textcolor}\rmfamily\fontsize{10.000000}{12.000000}\selectfont Classique}%
\end{pgfscope}%
\begin{pgfscope}%
\pgfsetrectcap%
\pgfsetroundjoin%
\pgfsetlinewidth{1.505625pt}%
\definecolor{currentstroke}{rgb}{0.121569,0.466667,0.705882}%
\pgfsetstrokecolor{currentstroke}%
\pgfsetdash{}{0pt}%
\pgfpathmoveto{\pgfqpoint{4.663333in}{1.083107in}}%
\pgfpathlineto{\pgfqpoint{4.802222in}{1.083107in}}%
\pgfpathlineto{\pgfqpoint{4.941111in}{1.083107in}}%
\pgfusepath{stroke}%
\end{pgfscope}%
\begin{pgfscope}%
\definecolor{textcolor}{rgb}{0.000000,0.000000,0.000000}%
\pgfsetstrokecolor{textcolor}%
\pgfsetfillcolor{textcolor}%
\pgftext[x=5.052222in,y=1.034496in,left,base]{\color{textcolor}\rmfamily\fontsize{10.000000}{12.000000}\selectfont Récursif}%
\end{pgfscope}%
\end{pgfpicture}%
\makeatother%
\endgroup%
}
    \caption{Comparaison de la fonction caractéristique STA/LTA obtenue par méthode classique (vert) et récursive (bleu) en utilisant des largeurs de fenêtres de $2$ et $10 \second$.}
    \label{fig:stalta}
\end{figure}

Une grande partie des algorithmes de calcul de fonction caractéristique se basent sur la méthode appelée STA/LTA. Cette méthode consiste à comparer la moyenne à court terme (STA = Short Term Average) avec la moyenne à long terme (LTA = Long Term Average). 
À court terme, nous mesurons en quelque sorte l'amplitude instantanée du signal. À long terme, nous mesurons plutôt l'amplitude locale du bruit de fond. Le rapport entre les deux permet de comparer ces deux grandeurs. Nous constatons que la fonction obtenue caractérise bien le signal puisqu'on observe 2 pics correspondant au début de la phase P et de la phase S, figure \ref{fig:stalta}.
Cet algorithme se base donc sur deux paramètres qui sont la longueur des fenêtres STA et LTA, respectivement $nsta$ et $nlta$. Dans la littérature nous trouvons généralement que la longueur de la fenêtre de LTA est 20 à 100 fois plus grande que la longueur de la fenêtre STA [d'après \cite{kuperkoch2010}, \cite{vassallo2012}, \cite{phasenet2018}].
Le principal défaut de cet algorithme c'est qu'il est coûteux en calcul en raison des nombreuses moyennes mobiles. Ce coût se fait particulièrement ressentir en utilisant un langage de haut niveau, ici Python. Si nous considérons un signal de $n$ points, auquel nous voulons calculer des moyennes glissantes de largeur $n_f$, il faudrait réaliser $n-n_f-1$ somme de $n_f$ éléments soit un total de $(n-n_f-1) \times n_f$ opérations d'addition puis $n-n_f-1$ opérations de divisions pour calculer les moyennes glissantes en chaque point. Une solution qui permet de limiter ce nombre consiste à calculer les moyennes mobiles à partir de la somme cumulée des valeurs des amplitudes du signal. En effet en utilisant une somme cumulée, nous devons réaliser $n-1$ opérations d'addition pour calculer la somme cumulée puis seulement $n-n_f-1$ opérations d'addition pour calculer les sommes spécifiques à chaque point, pour enfin calculer la moyenne en divisant $n-n_f-1$ fois. La méthode classique utilise donc pour un grand nombre d'opérations environ $n \times n_f$ opérations d'addition contre seulement $2 \times n_f$ opérations d'additions en utilisant les sommes cumulées. Nous avons $2 \times n_f > n \times n_f$ pour $n_f > 2$, qui est toujours atteint. L'implémentation du calcul des moyennes glissantes d'un signal $data$ à partir d'une somme cummulée peut se faire efficacement en utilisant le module Numpy :

\begin{minted}{python}
import numpy as np

cumsum_data = np.cumsum(data)

moyennes[nf:] = (cumsum_data[nf] - cumsum_data[:-nf])/nf
\end{minted}

\subsubsection{Algorithme d'Allen}

\cite{allen1982} propose un nouvel algorithme qui approxime l'algorithme STA/LTA trop coûteux en calculs, en calculant les coefficients STA et LTA par la relation de récurrence [d'après \cite{probatoire}, \cite{khalaf2016}, \cite{lumban2021}]:
\begin{equation}
   STA_i = STA_{i-1} + \frac{1}{nsta}(E_i^2 - STA_{i-1})
\end{equation}
\begin{equation}
   LTA_i = LTA_{i-1} + \frac{1}{nlta}(E_i^2 - LTA_{i-1})
\end{equation}
avec $E_i^2$ le signal d'entrée, $nsta$ et $nlta$ les largeurs des intervalles STA et LTA. 

Cette approximation astucieuse permet de réduire le nombre de calculs à également $2 \times n_f$ opérations d'addition pour chaque moyenne mobile. Même si elle a comme limite qu'elle caractérise moins le signal d'entrée, le pic est légèrement moins marqué lors de l'arrivée du champ d'ondes et elle bruite la fonction caractéristique, il y a plusieurs pics au niveau du maximum global, figure \ref{fig:stalta}.

Nous avons jusqu'à présent traduit ces algorithmes en Python, un langage haut niveau, pour des raisons d'aisance et de simplicité. Mais pour augmenter d'avantage l'efficacité de calcul de notre programme, nous pouvons le traduire dans un langage plus bas niveau. Nous avons choisi le langage C, qui présente une bonne compatibilité avec Python.

\begin{minted}{c}
#include <stdlib.h>

float *stalta_recursiv(float *data, int nsta, int nlta, int N)
{
    float *fc = malloc((sizeof(float) * N));
    float sta, lta;
    sta = 0;
    lta = 0;
    
    int i;
    for (i = 0; i < N; i++)
    {
        sta = (*(data + i) - sta) / nsta + sta;
        lta = (*(data + i) - lta) / nlta + lta;
        if (i > nlta && lta != 0)
            *(fc + i) = sta / lta;
        else
            *(fc + i) = 0;
    }

    return fc;
}
\end{minted}

Pour réaliser la passerelle entre Python et C, nous utilisons le module $ctypes$. Le code C doit d'abord être compilé
\begin{minted}{bash}
gcc -c -Wall -Werror -fPIC `python-config --cflags` stalta.c
gcc -shared -o stalta.so stalta.o `python-config --ldflags`
\end{minted}
Puis il peut être chargé en Python
\begin{minted}{python}
import ctypes

stalta_c = ctypes.CDLL('./stalta.so')

stalta_c.stalta_recursiv.argtype = (ctypes.POINTER(ctypes.c_float * n), 
    ctypes.c_int, ctypes.c_int, ctypes.c_int)
stalta_c.stalta_recursiv.restype = ctypes.POINTER(ctypes.c_float * n)

fc_allen_recursiv_ptr = stalta_c.stalta_recursiv((ctypes.c_float * n)(*sqa_data), 
    nsta, nlta, n) # exécution de la fonction
fc_allen_recursiv = fc_allen_recursiv_ptr.contents
\end{minted}

En comparant les temps d'exécution d'un signal de 36000 points avec des fenêtres de 200 et 4000 points, nous obtenons $0.07 \second$ en Python, poussé à $0.06 \second$ en utilisant une array Numpy, contre $0.01 \second$ en C. Pour donner un ordre de comparaison, l'algorithme STA/LTA classique écrit entièrement en Python a un temps d'exécution de plusieurs secondes (sans l'utilisation de Numpy). Nous avons ainsi résolu les problèmes de la puissance de calcul engendré par cet algorithme.

\subsubsection{Z-détecteur et algorithme de Baer et Kradolfer}

Swindell et Snell (1977) proposent le Z-détecteur, une méthode de calcul basé sur une unique fenêtre contrairement à la méthode STA/LTA. Cette fonction caractéristique estime l’écart des données sismiques à la valeur-moyenne, exprimée en unité de son écart-type [d'après \cite{wither1998}, \cite{probatoire}, \cite{kuperkoch2010}]
\begin{equation}
   Z_i = \frac{STA_i-\mu}{\sigma}
\end{equation}
avec $STA_i$ la moyenne glissante de la fenêtre, $\mu$ la moyenne et $\sigma$ l'écart-type des moyennes des fenêtres STA.

\begin{figure}[!ht]
    \centering
    %% Creator: Matplotlib, PGF backend
%%
%% To include the figure in your LaTeX document, write
%%   \input{<filename>.pgf}
%%
%% Make sure the required packages are loaded in your preamble
%%   \usepackage{pgf}
%%
%% Also ensure that all the required font packages are loaded; for instance,
%% the lmodern package is sometimes necessary when using math font.
%%   \usepackage{lmodern}
%%
%% Figures using additional raster images can only be included by \input if
%% they are in the same directory as the main LaTeX file. For loading figures
%% from other directories you can use the `import` package
%%   \usepackage{import}
%%
%% and then include the figures with
%%   \import{<path to file>}{<filename>.pgf}
%%
%% Matplotlib used the following preamble
%%   \usepackage{fontspec}
%%
\begingroup%
\makeatletter%
\begin{pgfpicture}%
\pgfpathrectangle{\pgfpointorigin}{\pgfqpoint{6.000000in}{4.000000in}}%
\pgfusepath{use as bounding box, clip}%
\begin{pgfscope}%
\pgfsetbuttcap%
\pgfsetmiterjoin%
\definecolor{currentfill}{rgb}{1.000000,1.000000,1.000000}%
\pgfsetfillcolor{currentfill}%
\pgfsetlinewidth{0.000000pt}%
\definecolor{currentstroke}{rgb}{1.000000,1.000000,1.000000}%
\pgfsetstrokecolor{currentstroke}%
\pgfsetdash{}{0pt}%
\pgfpathmoveto{\pgfqpoint{0.000000in}{0.000000in}}%
\pgfpathlineto{\pgfqpoint{6.000000in}{0.000000in}}%
\pgfpathlineto{\pgfqpoint{6.000000in}{4.000000in}}%
\pgfpathlineto{\pgfqpoint{0.000000in}{4.000000in}}%
\pgfpathlineto{\pgfqpoint{0.000000in}{0.000000in}}%
\pgfpathclose%
\pgfusepath{fill}%
\end{pgfscope}%
\begin{pgfscope}%
\pgfsetbuttcap%
\pgfsetmiterjoin%
\definecolor{currentfill}{rgb}{0.933333,0.933333,0.933333}%
\pgfsetfillcolor{currentfill}%
\pgfsetlinewidth{0.000000pt}%
\definecolor{currentstroke}{rgb}{0.000000,0.000000,0.000000}%
\pgfsetstrokecolor{currentstroke}%
\pgfsetstrokeopacity{0.000000}%
\pgfsetdash{}{0pt}%
\pgfpathmoveto{\pgfqpoint{0.610501in}{2.761814in}}%
\pgfpathlineto{\pgfqpoint{5.850000in}{2.761814in}}%
\pgfpathlineto{\pgfqpoint{5.850000in}{3.703703in}}%
\pgfpathlineto{\pgfqpoint{0.610501in}{3.703703in}}%
\pgfpathlineto{\pgfqpoint{0.610501in}{2.761814in}}%
\pgfpathclose%
\pgfusepath{fill}%
\end{pgfscope}%
\begin{pgfscope}%
\pgfpathrectangle{\pgfqpoint{0.610501in}{2.761814in}}{\pgfqpoint{5.239499in}{0.941890in}}%
\pgfusepath{clip}%
\pgfsetbuttcap%
\pgfsetroundjoin%
\pgfsetlinewidth{0.501875pt}%
\definecolor{currentstroke}{rgb}{0.698039,0.698039,0.698039}%
\pgfsetstrokecolor{currentstroke}%
\pgfsetdash{{1.850000pt}{0.800000pt}}{0.000000pt}%
\pgfpathmoveto{\pgfqpoint{0.610501in}{2.761814in}}%
\pgfpathlineto{\pgfqpoint{0.610501in}{3.703703in}}%
\pgfusepath{stroke}%
\end{pgfscope}%
\begin{pgfscope}%
\pgfsetbuttcap%
\pgfsetroundjoin%
\definecolor{currentfill}{rgb}{0.000000,0.000000,0.000000}%
\pgfsetfillcolor{currentfill}%
\pgfsetlinewidth{0.803000pt}%
\definecolor{currentstroke}{rgb}{0.000000,0.000000,0.000000}%
\pgfsetstrokecolor{currentstroke}%
\pgfsetdash{}{0pt}%
\pgfsys@defobject{currentmarker}{\pgfqpoint{0.000000in}{0.000000in}}{\pgfqpoint{0.000000in}{0.048611in}}{%
\pgfpathmoveto{\pgfqpoint{0.000000in}{0.000000in}}%
\pgfpathlineto{\pgfqpoint{0.000000in}{0.048611in}}%
\pgfusepath{stroke,fill}%
}%
\begin{pgfscope}%
\pgfsys@transformshift{0.610501in}{2.761814in}%
\pgfsys@useobject{currentmarker}{}%
\end{pgfscope}%
\end{pgfscope}%
\begin{pgfscope}%
\pgfpathrectangle{\pgfqpoint{0.610501in}{2.761814in}}{\pgfqpoint{5.239499in}{0.941890in}}%
\pgfusepath{clip}%
\pgfsetbuttcap%
\pgfsetroundjoin%
\pgfsetlinewidth{0.501875pt}%
\definecolor{currentstroke}{rgb}{0.698039,0.698039,0.698039}%
\pgfsetstrokecolor{currentstroke}%
\pgfsetdash{{1.850000pt}{0.800000pt}}{0.000000pt}%
\pgfpathmoveto{\pgfqpoint{1.192683in}{2.761814in}}%
\pgfpathlineto{\pgfqpoint{1.192683in}{3.703703in}}%
\pgfusepath{stroke}%
\end{pgfscope}%
\begin{pgfscope}%
\pgfsetbuttcap%
\pgfsetroundjoin%
\definecolor{currentfill}{rgb}{0.000000,0.000000,0.000000}%
\pgfsetfillcolor{currentfill}%
\pgfsetlinewidth{0.803000pt}%
\definecolor{currentstroke}{rgb}{0.000000,0.000000,0.000000}%
\pgfsetstrokecolor{currentstroke}%
\pgfsetdash{}{0pt}%
\pgfsys@defobject{currentmarker}{\pgfqpoint{0.000000in}{0.000000in}}{\pgfqpoint{0.000000in}{0.048611in}}{%
\pgfpathmoveto{\pgfqpoint{0.000000in}{0.000000in}}%
\pgfpathlineto{\pgfqpoint{0.000000in}{0.048611in}}%
\pgfusepath{stroke,fill}%
}%
\begin{pgfscope}%
\pgfsys@transformshift{1.192683in}{2.761814in}%
\pgfsys@useobject{currentmarker}{}%
\end{pgfscope}%
\end{pgfscope}%
\begin{pgfscope}%
\pgfpathrectangle{\pgfqpoint{0.610501in}{2.761814in}}{\pgfqpoint{5.239499in}{0.941890in}}%
\pgfusepath{clip}%
\pgfsetbuttcap%
\pgfsetroundjoin%
\pgfsetlinewidth{0.501875pt}%
\definecolor{currentstroke}{rgb}{0.698039,0.698039,0.698039}%
\pgfsetstrokecolor{currentstroke}%
\pgfsetdash{{1.850000pt}{0.800000pt}}{0.000000pt}%
\pgfpathmoveto{\pgfqpoint{1.774866in}{2.761814in}}%
\pgfpathlineto{\pgfqpoint{1.774866in}{3.703703in}}%
\pgfusepath{stroke}%
\end{pgfscope}%
\begin{pgfscope}%
\pgfsetbuttcap%
\pgfsetroundjoin%
\definecolor{currentfill}{rgb}{0.000000,0.000000,0.000000}%
\pgfsetfillcolor{currentfill}%
\pgfsetlinewidth{0.803000pt}%
\definecolor{currentstroke}{rgb}{0.000000,0.000000,0.000000}%
\pgfsetstrokecolor{currentstroke}%
\pgfsetdash{}{0pt}%
\pgfsys@defobject{currentmarker}{\pgfqpoint{0.000000in}{0.000000in}}{\pgfqpoint{0.000000in}{0.048611in}}{%
\pgfpathmoveto{\pgfqpoint{0.000000in}{0.000000in}}%
\pgfpathlineto{\pgfqpoint{0.000000in}{0.048611in}}%
\pgfusepath{stroke,fill}%
}%
\begin{pgfscope}%
\pgfsys@transformshift{1.774866in}{2.761814in}%
\pgfsys@useobject{currentmarker}{}%
\end{pgfscope}%
\end{pgfscope}%
\begin{pgfscope}%
\pgfpathrectangle{\pgfqpoint{0.610501in}{2.761814in}}{\pgfqpoint{5.239499in}{0.941890in}}%
\pgfusepath{clip}%
\pgfsetbuttcap%
\pgfsetroundjoin%
\pgfsetlinewidth{0.501875pt}%
\definecolor{currentstroke}{rgb}{0.698039,0.698039,0.698039}%
\pgfsetstrokecolor{currentstroke}%
\pgfsetdash{{1.850000pt}{0.800000pt}}{0.000000pt}%
\pgfpathmoveto{\pgfqpoint{2.357049in}{2.761814in}}%
\pgfpathlineto{\pgfqpoint{2.357049in}{3.703703in}}%
\pgfusepath{stroke}%
\end{pgfscope}%
\begin{pgfscope}%
\pgfsetbuttcap%
\pgfsetroundjoin%
\definecolor{currentfill}{rgb}{0.000000,0.000000,0.000000}%
\pgfsetfillcolor{currentfill}%
\pgfsetlinewidth{0.803000pt}%
\definecolor{currentstroke}{rgb}{0.000000,0.000000,0.000000}%
\pgfsetstrokecolor{currentstroke}%
\pgfsetdash{}{0pt}%
\pgfsys@defobject{currentmarker}{\pgfqpoint{0.000000in}{0.000000in}}{\pgfqpoint{0.000000in}{0.048611in}}{%
\pgfpathmoveto{\pgfqpoint{0.000000in}{0.000000in}}%
\pgfpathlineto{\pgfqpoint{0.000000in}{0.048611in}}%
\pgfusepath{stroke,fill}%
}%
\begin{pgfscope}%
\pgfsys@transformshift{2.357049in}{2.761814in}%
\pgfsys@useobject{currentmarker}{}%
\end{pgfscope}%
\end{pgfscope}%
\begin{pgfscope}%
\pgfpathrectangle{\pgfqpoint{0.610501in}{2.761814in}}{\pgfqpoint{5.239499in}{0.941890in}}%
\pgfusepath{clip}%
\pgfsetbuttcap%
\pgfsetroundjoin%
\pgfsetlinewidth{0.501875pt}%
\definecolor{currentstroke}{rgb}{0.698039,0.698039,0.698039}%
\pgfsetstrokecolor{currentstroke}%
\pgfsetdash{{1.850000pt}{0.800000pt}}{0.000000pt}%
\pgfpathmoveto{\pgfqpoint{2.939232in}{2.761814in}}%
\pgfpathlineto{\pgfqpoint{2.939232in}{3.703703in}}%
\pgfusepath{stroke}%
\end{pgfscope}%
\begin{pgfscope}%
\pgfsetbuttcap%
\pgfsetroundjoin%
\definecolor{currentfill}{rgb}{0.000000,0.000000,0.000000}%
\pgfsetfillcolor{currentfill}%
\pgfsetlinewidth{0.803000pt}%
\definecolor{currentstroke}{rgb}{0.000000,0.000000,0.000000}%
\pgfsetstrokecolor{currentstroke}%
\pgfsetdash{}{0pt}%
\pgfsys@defobject{currentmarker}{\pgfqpoint{0.000000in}{0.000000in}}{\pgfqpoint{0.000000in}{0.048611in}}{%
\pgfpathmoveto{\pgfqpoint{0.000000in}{0.000000in}}%
\pgfpathlineto{\pgfqpoint{0.000000in}{0.048611in}}%
\pgfusepath{stroke,fill}%
}%
\begin{pgfscope}%
\pgfsys@transformshift{2.939232in}{2.761814in}%
\pgfsys@useobject{currentmarker}{}%
\end{pgfscope}%
\end{pgfscope}%
\begin{pgfscope}%
\pgfpathrectangle{\pgfqpoint{0.610501in}{2.761814in}}{\pgfqpoint{5.239499in}{0.941890in}}%
\pgfusepath{clip}%
\pgfsetbuttcap%
\pgfsetroundjoin%
\pgfsetlinewidth{0.501875pt}%
\definecolor{currentstroke}{rgb}{0.698039,0.698039,0.698039}%
\pgfsetstrokecolor{currentstroke}%
\pgfsetdash{{1.850000pt}{0.800000pt}}{0.000000pt}%
\pgfpathmoveto{\pgfqpoint{3.521414in}{2.761814in}}%
\pgfpathlineto{\pgfqpoint{3.521414in}{3.703703in}}%
\pgfusepath{stroke}%
\end{pgfscope}%
\begin{pgfscope}%
\pgfsetbuttcap%
\pgfsetroundjoin%
\definecolor{currentfill}{rgb}{0.000000,0.000000,0.000000}%
\pgfsetfillcolor{currentfill}%
\pgfsetlinewidth{0.803000pt}%
\definecolor{currentstroke}{rgb}{0.000000,0.000000,0.000000}%
\pgfsetstrokecolor{currentstroke}%
\pgfsetdash{}{0pt}%
\pgfsys@defobject{currentmarker}{\pgfqpoint{0.000000in}{0.000000in}}{\pgfqpoint{0.000000in}{0.048611in}}{%
\pgfpathmoveto{\pgfqpoint{0.000000in}{0.000000in}}%
\pgfpathlineto{\pgfqpoint{0.000000in}{0.048611in}}%
\pgfusepath{stroke,fill}%
}%
\begin{pgfscope}%
\pgfsys@transformshift{3.521414in}{2.761814in}%
\pgfsys@useobject{currentmarker}{}%
\end{pgfscope}%
\end{pgfscope}%
\begin{pgfscope}%
\pgfpathrectangle{\pgfqpoint{0.610501in}{2.761814in}}{\pgfqpoint{5.239499in}{0.941890in}}%
\pgfusepath{clip}%
\pgfsetbuttcap%
\pgfsetroundjoin%
\pgfsetlinewidth{0.501875pt}%
\definecolor{currentstroke}{rgb}{0.698039,0.698039,0.698039}%
\pgfsetstrokecolor{currentstroke}%
\pgfsetdash{{1.850000pt}{0.800000pt}}{0.000000pt}%
\pgfpathmoveto{\pgfqpoint{4.103597in}{2.761814in}}%
\pgfpathlineto{\pgfqpoint{4.103597in}{3.703703in}}%
\pgfusepath{stroke}%
\end{pgfscope}%
\begin{pgfscope}%
\pgfsetbuttcap%
\pgfsetroundjoin%
\definecolor{currentfill}{rgb}{0.000000,0.000000,0.000000}%
\pgfsetfillcolor{currentfill}%
\pgfsetlinewidth{0.803000pt}%
\definecolor{currentstroke}{rgb}{0.000000,0.000000,0.000000}%
\pgfsetstrokecolor{currentstroke}%
\pgfsetdash{}{0pt}%
\pgfsys@defobject{currentmarker}{\pgfqpoint{0.000000in}{0.000000in}}{\pgfqpoint{0.000000in}{0.048611in}}{%
\pgfpathmoveto{\pgfqpoint{0.000000in}{0.000000in}}%
\pgfpathlineto{\pgfqpoint{0.000000in}{0.048611in}}%
\pgfusepath{stroke,fill}%
}%
\begin{pgfscope}%
\pgfsys@transformshift{4.103597in}{2.761814in}%
\pgfsys@useobject{currentmarker}{}%
\end{pgfscope}%
\end{pgfscope}%
\begin{pgfscope}%
\pgfpathrectangle{\pgfqpoint{0.610501in}{2.761814in}}{\pgfqpoint{5.239499in}{0.941890in}}%
\pgfusepath{clip}%
\pgfsetbuttcap%
\pgfsetroundjoin%
\pgfsetlinewidth{0.501875pt}%
\definecolor{currentstroke}{rgb}{0.698039,0.698039,0.698039}%
\pgfsetstrokecolor{currentstroke}%
\pgfsetdash{{1.850000pt}{0.800000pt}}{0.000000pt}%
\pgfpathmoveto{\pgfqpoint{4.685780in}{2.761814in}}%
\pgfpathlineto{\pgfqpoint{4.685780in}{3.703703in}}%
\pgfusepath{stroke}%
\end{pgfscope}%
\begin{pgfscope}%
\pgfsetbuttcap%
\pgfsetroundjoin%
\definecolor{currentfill}{rgb}{0.000000,0.000000,0.000000}%
\pgfsetfillcolor{currentfill}%
\pgfsetlinewidth{0.803000pt}%
\definecolor{currentstroke}{rgb}{0.000000,0.000000,0.000000}%
\pgfsetstrokecolor{currentstroke}%
\pgfsetdash{}{0pt}%
\pgfsys@defobject{currentmarker}{\pgfqpoint{0.000000in}{0.000000in}}{\pgfqpoint{0.000000in}{0.048611in}}{%
\pgfpathmoveto{\pgfqpoint{0.000000in}{0.000000in}}%
\pgfpathlineto{\pgfqpoint{0.000000in}{0.048611in}}%
\pgfusepath{stroke,fill}%
}%
\begin{pgfscope}%
\pgfsys@transformshift{4.685780in}{2.761814in}%
\pgfsys@useobject{currentmarker}{}%
\end{pgfscope}%
\end{pgfscope}%
\begin{pgfscope}%
\pgfpathrectangle{\pgfqpoint{0.610501in}{2.761814in}}{\pgfqpoint{5.239499in}{0.941890in}}%
\pgfusepath{clip}%
\pgfsetbuttcap%
\pgfsetroundjoin%
\pgfsetlinewidth{0.501875pt}%
\definecolor{currentstroke}{rgb}{0.698039,0.698039,0.698039}%
\pgfsetstrokecolor{currentstroke}%
\pgfsetdash{{1.850000pt}{0.800000pt}}{0.000000pt}%
\pgfpathmoveto{\pgfqpoint{5.267963in}{2.761814in}}%
\pgfpathlineto{\pgfqpoint{5.267963in}{3.703703in}}%
\pgfusepath{stroke}%
\end{pgfscope}%
\begin{pgfscope}%
\pgfsetbuttcap%
\pgfsetroundjoin%
\definecolor{currentfill}{rgb}{0.000000,0.000000,0.000000}%
\pgfsetfillcolor{currentfill}%
\pgfsetlinewidth{0.803000pt}%
\definecolor{currentstroke}{rgb}{0.000000,0.000000,0.000000}%
\pgfsetstrokecolor{currentstroke}%
\pgfsetdash{}{0pt}%
\pgfsys@defobject{currentmarker}{\pgfqpoint{0.000000in}{0.000000in}}{\pgfqpoint{0.000000in}{0.048611in}}{%
\pgfpathmoveto{\pgfqpoint{0.000000in}{0.000000in}}%
\pgfpathlineto{\pgfqpoint{0.000000in}{0.048611in}}%
\pgfusepath{stroke,fill}%
}%
\begin{pgfscope}%
\pgfsys@transformshift{5.267963in}{2.761814in}%
\pgfsys@useobject{currentmarker}{}%
\end{pgfscope}%
\end{pgfscope}%
\begin{pgfscope}%
\pgfpathrectangle{\pgfqpoint{0.610501in}{2.761814in}}{\pgfqpoint{5.239499in}{0.941890in}}%
\pgfusepath{clip}%
\pgfsetbuttcap%
\pgfsetroundjoin%
\pgfsetlinewidth{0.501875pt}%
\definecolor{currentstroke}{rgb}{0.698039,0.698039,0.698039}%
\pgfsetstrokecolor{currentstroke}%
\pgfsetdash{{1.850000pt}{0.800000pt}}{0.000000pt}%
\pgfpathmoveto{\pgfqpoint{0.610501in}{2.822332in}}%
\pgfpathlineto{\pgfqpoint{5.850000in}{2.822332in}}%
\pgfusepath{stroke}%
\end{pgfscope}%
\begin{pgfscope}%
\pgfsetbuttcap%
\pgfsetroundjoin%
\definecolor{currentfill}{rgb}{0.000000,0.000000,0.000000}%
\pgfsetfillcolor{currentfill}%
\pgfsetlinewidth{0.803000pt}%
\definecolor{currentstroke}{rgb}{0.000000,0.000000,0.000000}%
\pgfsetstrokecolor{currentstroke}%
\pgfsetdash{}{0pt}%
\pgfsys@defobject{currentmarker}{\pgfqpoint{0.000000in}{0.000000in}}{\pgfqpoint{0.048611in}{0.000000in}}{%
\pgfpathmoveto{\pgfqpoint{0.000000in}{0.000000in}}%
\pgfpathlineto{\pgfqpoint{0.048611in}{0.000000in}}%
\pgfusepath{stroke,fill}%
}%
\begin{pgfscope}%
\pgfsys@transformshift{0.610501in}{2.822332in}%
\pgfsys@useobject{currentmarker}{}%
\end{pgfscope}%
\end{pgfscope}%
\begin{pgfscope}%
\definecolor{textcolor}{rgb}{0.000000,0.000000,0.000000}%
\pgfsetstrokecolor{textcolor}%
\pgfsetfillcolor{textcolor}%
\pgftext[x=0.384420in, y=2.774137in, left, base]{\color{textcolor}\rmfamily\fontsize{10.000000}{12.000000}\selectfont \(\displaystyle {\ensuremath{-}5}\)}%
\end{pgfscope}%
\begin{pgfscope}%
\pgfpathrectangle{\pgfqpoint{0.610501in}{2.761814in}}{\pgfqpoint{5.239499in}{0.941890in}}%
\pgfusepath{clip}%
\pgfsetbuttcap%
\pgfsetroundjoin%
\pgfsetlinewidth{0.501875pt}%
\definecolor{currentstroke}{rgb}{0.698039,0.698039,0.698039}%
\pgfsetstrokecolor{currentstroke}%
\pgfsetdash{{1.850000pt}{0.800000pt}}{0.000000pt}%
\pgfpathmoveto{\pgfqpoint{0.610501in}{3.187411in}}%
\pgfpathlineto{\pgfqpoint{5.850000in}{3.187411in}}%
\pgfusepath{stroke}%
\end{pgfscope}%
\begin{pgfscope}%
\pgfsetbuttcap%
\pgfsetroundjoin%
\definecolor{currentfill}{rgb}{0.000000,0.000000,0.000000}%
\pgfsetfillcolor{currentfill}%
\pgfsetlinewidth{0.803000pt}%
\definecolor{currentstroke}{rgb}{0.000000,0.000000,0.000000}%
\pgfsetstrokecolor{currentstroke}%
\pgfsetdash{}{0pt}%
\pgfsys@defobject{currentmarker}{\pgfqpoint{0.000000in}{0.000000in}}{\pgfqpoint{0.048611in}{0.000000in}}{%
\pgfpathmoveto{\pgfqpoint{0.000000in}{0.000000in}}%
\pgfpathlineto{\pgfqpoint{0.048611in}{0.000000in}}%
\pgfusepath{stroke,fill}%
}%
\begin{pgfscope}%
\pgfsys@transformshift{0.610501in}{3.187411in}%
\pgfsys@useobject{currentmarker}{}%
\end{pgfscope}%
\end{pgfscope}%
\begin{pgfscope}%
\definecolor{textcolor}{rgb}{0.000000,0.000000,0.000000}%
\pgfsetstrokecolor{textcolor}%
\pgfsetfillcolor{textcolor}%
\pgftext[x=0.492445in, y=3.139217in, left, base]{\color{textcolor}\rmfamily\fontsize{10.000000}{12.000000}\selectfont \(\displaystyle {0}\)}%
\end{pgfscope}%
\begin{pgfscope}%
\pgfpathrectangle{\pgfqpoint{0.610501in}{2.761814in}}{\pgfqpoint{5.239499in}{0.941890in}}%
\pgfusepath{clip}%
\pgfsetbuttcap%
\pgfsetroundjoin%
\pgfsetlinewidth{0.501875pt}%
\definecolor{currentstroke}{rgb}{0.698039,0.698039,0.698039}%
\pgfsetstrokecolor{currentstroke}%
\pgfsetdash{{1.850000pt}{0.800000pt}}{0.000000pt}%
\pgfpathmoveto{\pgfqpoint{0.610501in}{3.552490in}}%
\pgfpathlineto{\pgfqpoint{5.850000in}{3.552490in}}%
\pgfusepath{stroke}%
\end{pgfscope}%
\begin{pgfscope}%
\pgfsetbuttcap%
\pgfsetroundjoin%
\definecolor{currentfill}{rgb}{0.000000,0.000000,0.000000}%
\pgfsetfillcolor{currentfill}%
\pgfsetlinewidth{0.803000pt}%
\definecolor{currentstroke}{rgb}{0.000000,0.000000,0.000000}%
\pgfsetstrokecolor{currentstroke}%
\pgfsetdash{}{0pt}%
\pgfsys@defobject{currentmarker}{\pgfqpoint{0.000000in}{0.000000in}}{\pgfqpoint{0.048611in}{0.000000in}}{%
\pgfpathmoveto{\pgfqpoint{0.000000in}{0.000000in}}%
\pgfpathlineto{\pgfqpoint{0.048611in}{0.000000in}}%
\pgfusepath{stroke,fill}%
}%
\begin{pgfscope}%
\pgfsys@transformshift{0.610501in}{3.552490in}%
\pgfsys@useobject{currentmarker}{}%
\end{pgfscope}%
\end{pgfscope}%
\begin{pgfscope}%
\definecolor{textcolor}{rgb}{0.000000,0.000000,0.000000}%
\pgfsetstrokecolor{textcolor}%
\pgfsetfillcolor{textcolor}%
\pgftext[x=0.492445in, y=3.504296in, left, base]{\color{textcolor}\rmfamily\fontsize{10.000000}{12.000000}\selectfont \(\displaystyle {5}\)}%
\end{pgfscope}%
\begin{pgfscope}%
\definecolor{textcolor}{rgb}{0.000000,0.000000,0.000000}%
\pgfsetstrokecolor{textcolor}%
\pgfsetfillcolor{textcolor}%
\pgftext[x=0.328864in,y=3.232758in,,bottom,rotate=90.000000]{\color{textcolor}\rmfamily\fontsize{12.000000}{14.400000}\selectfont Signal}%
\end{pgfscope}%
\begin{pgfscope}%
\definecolor{textcolor}{rgb}{0.000000,0.000000,0.000000}%
\pgfsetstrokecolor{textcolor}%
\pgfsetfillcolor{textcolor}%
\pgftext[x=0.610501in,y=3.745370in,left,base]{\color{textcolor}\rmfamily\fontsize{10.000000}{12.000000}\selectfont \(\displaystyle \times{10^{\ensuremath{-}5}}{}\)}%
\end{pgfscope}%
\begin{pgfscope}%
\pgfpathrectangle{\pgfqpoint{0.610501in}{2.761814in}}{\pgfqpoint{5.239499in}{0.941890in}}%
\pgfusepath{clip}%
\pgfsetrectcap%
\pgfsetroundjoin%
\pgfsetlinewidth{1.505625pt}%
\definecolor{currentstroke}{rgb}{0.498039,0.498039,0.498039}%
\pgfsetstrokecolor{currentstroke}%
\pgfsetdash{}{0pt}%
\pgfpathmoveto{\pgfqpoint{0.610501in}{3.187411in}}%
\pgfpathlineto{\pgfqpoint{1.060819in}{3.187177in}}%
\pgfpathlineto{\pgfqpoint{1.074355in}{3.187441in}}%
\pgfpathlineto{\pgfqpoint{1.204473in}{3.187320in}}%
\pgfpathlineto{\pgfqpoint{1.221792in}{3.187572in}}%
\pgfpathlineto{\pgfqpoint{1.235328in}{3.187446in}}%
\pgfpathlineto{\pgfqpoint{1.241441in}{3.187380in}}%
\pgfpathlineto{\pgfqpoint{1.338083in}{3.187519in}}%
\pgfpathlineto{\pgfqpoint{1.342013in}{3.187231in}}%
\pgfpathlineto{\pgfqpoint{1.363263in}{3.187326in}}%
\pgfpathlineto{\pgfqpoint{1.373451in}{3.187253in}}%
\pgfpathlineto{\pgfqpoint{1.389607in}{3.187355in}}%
\pgfpathlineto{\pgfqpoint{1.395428in}{3.187538in}}%
\pgfpathlineto{\pgfqpoint{1.482756in}{3.187289in}}%
\pgfpathlineto{\pgfqpoint{1.495273in}{3.187451in}}%
\pgfpathlineto{\pgfqpoint{1.505461in}{3.187558in}}%
\pgfpathlineto{\pgfqpoint{1.525110in}{3.187401in}}%
\pgfpathlineto{\pgfqpoint{1.530640in}{3.187470in}}%
\pgfpathlineto{\pgfqpoint{1.544467in}{3.187434in}}%
\pgfpathlineto{\pgfqpoint{1.602977in}{3.187355in}}%
\pgfpathlineto{\pgfqpoint{1.608071in}{3.187492in}}%
\pgfpathlineto{\pgfqpoint{1.720869in}{3.187521in}}%
\pgfpathlineto{\pgfqpoint{1.729019in}{3.187335in}}%
\pgfpathlineto{\pgfqpoint{1.737898in}{3.187424in}}%
\pgfpathlineto{\pgfqpoint{1.754199in}{3.187240in}}%
\pgfpathlineto{\pgfqpoint{1.768025in}{3.187310in}}%
\pgfpathlineto{\pgfqpoint{1.908040in}{3.188296in}}%
\pgfpathlineto{\pgfqpoint{1.908768in}{3.188913in}}%
\pgfpathlineto{\pgfqpoint{1.909205in}{3.188270in}}%
\pgfpathlineto{\pgfqpoint{1.910660in}{3.185029in}}%
\pgfpathlineto{\pgfqpoint{1.911242in}{3.185877in}}%
\pgfpathlineto{\pgfqpoint{1.912698in}{3.187829in}}%
\pgfpathlineto{\pgfqpoint{1.913135in}{3.187489in}}%
\pgfpathlineto{\pgfqpoint{1.914881in}{3.185512in}}%
\pgfpathlineto{\pgfqpoint{1.915318in}{3.186425in}}%
\pgfpathlineto{\pgfqpoint{1.916773in}{3.190150in}}%
\pgfpathlineto{\pgfqpoint{1.917210in}{3.189452in}}%
\pgfpathlineto{\pgfqpoint{1.919102in}{3.187336in}}%
\pgfpathlineto{\pgfqpoint{1.921285in}{3.189282in}}%
\pgfpathlineto{\pgfqpoint{1.921722in}{3.187909in}}%
\pgfpathlineto{\pgfqpoint{1.923032in}{3.184201in}}%
\pgfpathlineto{\pgfqpoint{1.923468in}{3.185131in}}%
\pgfpathlineto{\pgfqpoint{1.924778in}{3.188858in}}%
\pgfpathlineto{\pgfqpoint{1.925215in}{3.187942in}}%
\pgfpathlineto{\pgfqpoint{1.926379in}{3.184748in}}%
\pgfpathlineto{\pgfqpoint{1.926961in}{3.185713in}}%
\pgfpathlineto{\pgfqpoint{1.928708in}{3.189365in}}%
\pgfpathlineto{\pgfqpoint{1.928999in}{3.189047in}}%
\pgfpathlineto{\pgfqpoint{1.930600in}{3.185729in}}%
\pgfpathlineto{\pgfqpoint{1.931182in}{3.186984in}}%
\pgfpathlineto{\pgfqpoint{1.932638in}{3.190304in}}%
\pgfpathlineto{\pgfqpoint{1.933074in}{3.189458in}}%
\pgfpathlineto{\pgfqpoint{1.934530in}{3.183398in}}%
\pgfpathlineto{\pgfqpoint{1.935112in}{3.184960in}}%
\pgfpathlineto{\pgfqpoint{1.936858in}{3.191951in}}%
\pgfpathlineto{\pgfqpoint{1.937295in}{3.191056in}}%
\pgfpathlineto{\pgfqpoint{1.938751in}{3.182354in}}%
\pgfpathlineto{\pgfqpoint{1.939478in}{3.186108in}}%
\pgfpathlineto{\pgfqpoint{1.940643in}{3.194741in}}%
\pgfpathlineto{\pgfqpoint{1.941079in}{3.193321in}}%
\pgfpathlineto{\pgfqpoint{1.942971in}{3.178401in}}%
\pgfpathlineto{\pgfqpoint{1.943554in}{3.182154in}}%
\pgfpathlineto{\pgfqpoint{1.944718in}{3.193947in}}%
\pgfpathlineto{\pgfqpoint{1.945300in}{3.190809in}}%
\pgfpathlineto{\pgfqpoint{1.946610in}{3.174964in}}%
\pgfpathlineto{\pgfqpoint{1.947192in}{3.178572in}}%
\pgfpathlineto{\pgfqpoint{1.949084in}{3.204519in}}%
\pgfpathlineto{\pgfqpoint{1.949666in}{3.199603in}}%
\pgfpathlineto{\pgfqpoint{1.950976in}{3.175320in}}%
\pgfpathlineto{\pgfqpoint{1.951559in}{3.182115in}}%
\pgfpathlineto{\pgfqpoint{1.952723in}{3.198228in}}%
\pgfpathlineto{\pgfqpoint{1.953160in}{3.195227in}}%
\pgfpathlineto{\pgfqpoint{1.955197in}{3.169557in}}%
\pgfpathlineto{\pgfqpoint{1.955779in}{3.173014in}}%
\pgfpathlineto{\pgfqpoint{1.957817in}{3.201661in}}%
\pgfpathlineto{\pgfqpoint{1.958545in}{3.196928in}}%
\pgfpathlineto{\pgfqpoint{1.959855in}{3.186183in}}%
\pgfpathlineto{\pgfqpoint{1.960146in}{3.187875in}}%
\pgfpathlineto{\pgfqpoint{1.961601in}{3.211600in}}%
\pgfpathlineto{\pgfqpoint{1.962183in}{3.201910in}}%
\pgfpathlineto{\pgfqpoint{1.963639in}{3.155822in}}%
\pgfpathlineto{\pgfqpoint{1.964221in}{3.166075in}}%
\pgfpathlineto{\pgfqpoint{1.965385in}{3.202733in}}%
\pgfpathlineto{\pgfqpoint{1.965968in}{3.191984in}}%
\pgfpathlineto{\pgfqpoint{1.967132in}{3.151728in}}%
\pgfpathlineto{\pgfqpoint{1.967569in}{3.160215in}}%
\pgfpathlineto{\pgfqpoint{1.969170in}{3.208523in}}%
\pgfpathlineto{\pgfqpoint{1.969752in}{3.203742in}}%
\pgfpathlineto{\pgfqpoint{1.971207in}{3.184973in}}%
\pgfpathlineto{\pgfqpoint{1.971644in}{3.188203in}}%
\pgfpathlineto{\pgfqpoint{1.972808in}{3.209844in}}%
\pgfpathlineto{\pgfqpoint{1.973390in}{3.200666in}}%
\pgfpathlineto{\pgfqpoint{1.974409in}{3.179212in}}%
\pgfpathlineto{\pgfqpoint{1.974991in}{3.184187in}}%
\pgfpathlineto{\pgfqpoint{1.975865in}{3.190979in}}%
\pgfpathlineto{\pgfqpoint{1.976301in}{3.189150in}}%
\pgfpathlineto{\pgfqpoint{1.977757in}{3.171948in}}%
\pgfpathlineto{\pgfqpoint{1.978339in}{3.164975in}}%
\pgfpathlineto{\pgfqpoint{1.978921in}{3.170790in}}%
\pgfpathlineto{\pgfqpoint{1.980668in}{3.198751in}}%
\pgfpathlineto{\pgfqpoint{1.981104in}{3.195295in}}%
\pgfpathlineto{\pgfqpoint{1.982414in}{3.174799in}}%
\pgfpathlineto{\pgfqpoint{1.982996in}{3.181276in}}%
\pgfpathlineto{\pgfqpoint{1.984597in}{3.209170in}}%
\pgfpathlineto{\pgfqpoint{1.985034in}{3.205812in}}%
\pgfpathlineto{\pgfqpoint{1.986490in}{3.166617in}}%
\pgfpathlineto{\pgfqpoint{1.987217in}{3.183646in}}%
\pgfpathlineto{\pgfqpoint{1.988091in}{3.211641in}}%
\pgfpathlineto{\pgfqpoint{1.988673in}{3.202211in}}%
\pgfpathlineto{\pgfqpoint{1.990274in}{3.153122in}}%
\pgfpathlineto{\pgfqpoint{1.990710in}{3.160714in}}%
\pgfpathlineto{\pgfqpoint{1.992020in}{3.207541in}}%
\pgfpathlineto{\pgfqpoint{1.992602in}{3.195519in}}%
\pgfpathlineto{\pgfqpoint{1.993621in}{3.172750in}}%
\pgfpathlineto{\pgfqpoint{1.994058in}{3.176649in}}%
\pgfpathlineto{\pgfqpoint{1.995950in}{3.212450in}}%
\pgfpathlineto{\pgfqpoint{1.996678in}{3.203712in}}%
\pgfpathlineto{\pgfqpoint{1.997842in}{3.189874in}}%
\pgfpathlineto{\pgfqpoint{1.998424in}{3.190913in}}%
\pgfpathlineto{\pgfqpoint{1.998715in}{3.190972in}}%
\pgfpathlineto{\pgfqpoint{1.999006in}{3.190238in}}%
\pgfpathlineto{\pgfqpoint{2.000607in}{3.180034in}}%
\pgfpathlineto{\pgfqpoint{2.001335in}{3.176540in}}%
\pgfpathlineto{\pgfqpoint{2.001772in}{3.177695in}}%
\pgfpathlineto{\pgfqpoint{2.002645in}{3.181973in}}%
\pgfpathlineto{\pgfqpoint{2.002936in}{3.180495in}}%
\pgfpathlineto{\pgfqpoint{2.004392in}{3.151650in}}%
\pgfpathlineto{\pgfqpoint{2.004974in}{3.164630in}}%
\pgfpathlineto{\pgfqpoint{2.005993in}{3.191885in}}%
\pgfpathlineto{\pgfqpoint{2.006575in}{3.188265in}}%
\pgfpathlineto{\pgfqpoint{2.006866in}{3.187142in}}%
\pgfpathlineto{\pgfqpoint{2.007157in}{3.189376in}}%
\pgfpathlineto{\pgfqpoint{2.009486in}{3.222096in}}%
\pgfpathlineto{\pgfqpoint{2.009922in}{3.219370in}}%
\pgfpathlineto{\pgfqpoint{2.011232in}{3.196180in}}%
\pgfpathlineto{\pgfqpoint{2.011960in}{3.204289in}}%
\pgfpathlineto{\pgfqpoint{2.012251in}{3.205696in}}%
\pgfpathlineto{\pgfqpoint{2.012542in}{3.202136in}}%
\pgfpathlineto{\pgfqpoint{2.014143in}{3.154365in}}%
\pgfpathlineto{\pgfqpoint{2.015162in}{3.159876in}}%
\pgfpathlineto{\pgfqpoint{2.017927in}{3.174557in}}%
\pgfpathlineto{\pgfqpoint{2.019383in}{3.199906in}}%
\pgfpathlineto{\pgfqpoint{2.019965in}{3.197590in}}%
\pgfpathlineto{\pgfqpoint{2.020111in}{3.197478in}}%
\pgfpathlineto{\pgfqpoint{2.020256in}{3.197854in}}%
\pgfpathlineto{\pgfqpoint{2.021129in}{3.209736in}}%
\pgfpathlineto{\pgfqpoint{2.021712in}{3.215286in}}%
\pgfpathlineto{\pgfqpoint{2.022148in}{3.212204in}}%
\pgfpathlineto{\pgfqpoint{2.023313in}{3.200651in}}%
\pgfpathlineto{\pgfqpoint{2.023895in}{3.203420in}}%
\pgfpathlineto{\pgfqpoint{2.024477in}{3.206056in}}%
\pgfpathlineto{\pgfqpoint{2.024914in}{3.203415in}}%
\pgfpathlineto{\pgfqpoint{2.026224in}{3.175967in}}%
\pgfpathlineto{\pgfqpoint{2.027533in}{3.139989in}}%
\pgfpathlineto{\pgfqpoint{2.028116in}{3.148708in}}%
\pgfpathlineto{\pgfqpoint{2.031754in}{3.219011in}}%
\pgfpathlineto{\pgfqpoint{2.032628in}{3.219785in}}%
\pgfpathlineto{\pgfqpoint{2.032773in}{3.219357in}}%
\pgfpathlineto{\pgfqpoint{2.035102in}{3.203581in}}%
\pgfpathlineto{\pgfqpoint{2.036557in}{3.149891in}}%
\pgfpathlineto{\pgfqpoint{2.037285in}{3.160452in}}%
\pgfpathlineto{\pgfqpoint{2.038158in}{3.173109in}}%
\pgfpathlineto{\pgfqpoint{2.038595in}{3.168645in}}%
\pgfpathlineto{\pgfqpoint{2.039614in}{3.151151in}}%
\pgfpathlineto{\pgfqpoint{2.040050in}{3.157669in}}%
\pgfpathlineto{\pgfqpoint{2.041942in}{3.225641in}}%
\pgfpathlineto{\pgfqpoint{2.042670in}{3.214407in}}%
\pgfpathlineto{\pgfqpoint{2.043980in}{3.193179in}}%
\pgfpathlineto{\pgfqpoint{2.044417in}{3.195668in}}%
\pgfpathlineto{\pgfqpoint{2.045144in}{3.201510in}}%
\pgfpathlineto{\pgfqpoint{2.045581in}{3.197807in}}%
\pgfpathlineto{\pgfqpoint{2.046600in}{3.184520in}}%
\pgfpathlineto{\pgfqpoint{2.047037in}{3.188086in}}%
\pgfpathlineto{\pgfqpoint{2.047619in}{3.193186in}}%
\pgfpathlineto{\pgfqpoint{2.048055in}{3.187744in}}%
\pgfpathlineto{\pgfqpoint{2.049365in}{3.149757in}}%
\pgfpathlineto{\pgfqpoint{2.049947in}{3.161545in}}%
\pgfpathlineto{\pgfqpoint{2.051257in}{3.216852in}}%
\pgfpathlineto{\pgfqpoint{2.051840in}{3.203017in}}%
\pgfpathlineto{\pgfqpoint{2.053004in}{3.153533in}}%
\pgfpathlineto{\pgfqpoint{2.053586in}{3.166366in}}%
\pgfpathlineto{\pgfqpoint{2.055042in}{3.230991in}}%
\pgfpathlineto{\pgfqpoint{2.055624in}{3.218074in}}%
\pgfpathlineto{\pgfqpoint{2.057225in}{3.137066in}}%
\pgfpathlineto{\pgfqpoint{2.057807in}{3.157208in}}%
\pgfpathlineto{\pgfqpoint{2.059117in}{3.214113in}}%
\pgfpathlineto{\pgfqpoint{2.059699in}{3.206386in}}%
\pgfpathlineto{\pgfqpoint{2.060863in}{3.176801in}}%
\pgfpathlineto{\pgfqpoint{2.061300in}{3.183840in}}%
\pgfpathlineto{\pgfqpoint{2.062755in}{3.232933in}}%
\pgfpathlineto{\pgfqpoint{2.063338in}{3.222857in}}%
\pgfpathlineto{\pgfqpoint{2.065084in}{3.135168in}}%
\pgfpathlineto{\pgfqpoint{2.065812in}{3.162065in}}%
\pgfpathlineto{\pgfqpoint{2.066976in}{3.210688in}}%
\pgfpathlineto{\pgfqpoint{2.067413in}{3.200122in}}%
\pgfpathlineto{\pgfqpoint{2.069014in}{3.124435in}}%
\pgfpathlineto{\pgfqpoint{2.069596in}{3.146541in}}%
\pgfpathlineto{\pgfqpoint{2.071197in}{3.246771in}}%
\pgfpathlineto{\pgfqpoint{2.071779in}{3.232174in}}%
\pgfpathlineto{\pgfqpoint{2.073235in}{3.187378in}}%
\pgfpathlineto{\pgfqpoint{2.073671in}{3.191355in}}%
\pgfpathlineto{\pgfqpoint{2.074690in}{3.209262in}}%
\pgfpathlineto{\pgfqpoint{2.075127in}{3.203227in}}%
\pgfpathlineto{\pgfqpoint{2.076728in}{3.153566in}}%
\pgfpathlineto{\pgfqpoint{2.077310in}{3.161333in}}%
\pgfpathlineto{\pgfqpoint{2.078766in}{3.211180in}}%
\pgfpathlineto{\pgfqpoint{2.079348in}{3.194601in}}%
\pgfpathlineto{\pgfqpoint{2.080803in}{3.119085in}}%
\pgfpathlineto{\pgfqpoint{2.081240in}{3.134401in}}%
\pgfpathlineto{\pgfqpoint{2.083132in}{3.250833in}}%
\pgfpathlineto{\pgfqpoint{2.083714in}{3.235762in}}%
\pgfpathlineto{\pgfqpoint{2.084733in}{3.207816in}}%
\pgfpathlineto{\pgfqpoint{2.085315in}{3.214662in}}%
\pgfpathlineto{\pgfqpoint{2.086043in}{3.224332in}}%
\pgfpathlineto{\pgfqpoint{2.086334in}{3.219435in}}%
\pgfpathlineto{\pgfqpoint{2.087498in}{3.141007in}}%
\pgfpathlineto{\pgfqpoint{2.088372in}{3.100698in}}%
\pgfpathlineto{\pgfqpoint{2.088808in}{3.111659in}}%
\pgfpathlineto{\pgfqpoint{2.090991in}{3.216361in}}%
\pgfpathlineto{\pgfqpoint{2.091574in}{3.206070in}}%
\pgfpathlineto{\pgfqpoint{2.092738in}{3.184789in}}%
\pgfpathlineto{\pgfqpoint{2.093175in}{3.188388in}}%
\pgfpathlineto{\pgfqpoint{2.094630in}{3.228950in}}%
\pgfpathlineto{\pgfqpoint{2.095358in}{3.244112in}}%
\pgfpathlineto{\pgfqpoint{2.095794in}{3.235412in}}%
\pgfpathlineto{\pgfqpoint{2.098123in}{3.118055in}}%
\pgfpathlineto{\pgfqpoint{2.098851in}{3.136647in}}%
\pgfpathlineto{\pgfqpoint{2.101325in}{3.200061in}}%
\pgfpathlineto{\pgfqpoint{2.102926in}{3.226352in}}%
\pgfpathlineto{\pgfqpoint{2.103654in}{3.220310in}}%
\pgfpathlineto{\pgfqpoint{2.104382in}{3.214534in}}%
\pgfpathlineto{\pgfqpoint{2.104964in}{3.217351in}}%
\pgfpathlineto{\pgfqpoint{2.105255in}{3.218147in}}%
\pgfpathlineto{\pgfqpoint{2.105546in}{3.215866in}}%
\pgfpathlineto{\pgfqpoint{2.106565in}{3.175858in}}%
\pgfpathlineto{\pgfqpoint{2.108020in}{3.128215in}}%
\pgfpathlineto{\pgfqpoint{2.108457in}{3.130418in}}%
\pgfpathlineto{\pgfqpoint{2.111950in}{3.191688in}}%
\pgfpathlineto{\pgfqpoint{2.114570in}{3.259265in}}%
\pgfpathlineto{\pgfqpoint{2.115297in}{3.270576in}}%
\pgfpathlineto{\pgfqpoint{2.115734in}{3.265747in}}%
\pgfpathlineto{\pgfqpoint{2.116753in}{3.200520in}}%
\pgfpathlineto{\pgfqpoint{2.118208in}{3.130649in}}%
\pgfpathlineto{\pgfqpoint{2.118645in}{3.134304in}}%
\pgfpathlineto{\pgfqpoint{2.119227in}{3.139369in}}%
\pgfpathlineto{\pgfqpoint{2.119664in}{3.135746in}}%
\pgfpathlineto{\pgfqpoint{2.120974in}{3.107576in}}%
\pgfpathlineto{\pgfqpoint{2.121410in}{3.114751in}}%
\pgfpathlineto{\pgfqpoint{2.124030in}{3.243666in}}%
\pgfpathlineto{\pgfqpoint{2.125486in}{3.231878in}}%
\pgfpathlineto{\pgfqpoint{2.126796in}{3.220848in}}%
\pgfpathlineto{\pgfqpoint{2.129998in}{3.157419in}}%
\pgfpathlineto{\pgfqpoint{2.130580in}{3.161674in}}%
\pgfpathlineto{\pgfqpoint{2.133054in}{3.200792in}}%
\pgfpathlineto{\pgfqpoint{2.133782in}{3.196757in}}%
\pgfpathlineto{\pgfqpoint{2.136547in}{3.167698in}}%
\pgfpathlineto{\pgfqpoint{2.137129in}{3.173876in}}%
\pgfpathlineto{\pgfqpoint{2.138585in}{3.204719in}}%
\pgfpathlineto{\pgfqpoint{2.139167in}{3.198365in}}%
\pgfpathlineto{\pgfqpoint{2.140914in}{3.158632in}}%
\pgfpathlineto{\pgfqpoint{2.141496in}{3.164803in}}%
\pgfpathlineto{\pgfqpoint{2.143097in}{3.205783in}}%
\pgfpathlineto{\pgfqpoint{2.143679in}{3.199010in}}%
\pgfpathlineto{\pgfqpoint{2.145134in}{3.172899in}}%
\pgfpathlineto{\pgfqpoint{2.145571in}{3.178642in}}%
\pgfpathlineto{\pgfqpoint{2.147172in}{3.239706in}}%
\pgfpathlineto{\pgfqpoint{2.147900in}{3.222135in}}%
\pgfpathlineto{\pgfqpoint{2.149210in}{3.181184in}}%
\pgfpathlineto{\pgfqpoint{2.149792in}{3.186334in}}%
\pgfpathlineto{\pgfqpoint{2.150956in}{3.203387in}}%
\pgfpathlineto{\pgfqpoint{2.151393in}{3.198227in}}%
\pgfpathlineto{\pgfqpoint{2.152703in}{3.130599in}}%
\pgfpathlineto{\pgfqpoint{2.153576in}{3.104773in}}%
\pgfpathlineto{\pgfqpoint{2.154013in}{3.112502in}}%
\pgfpathlineto{\pgfqpoint{2.155468in}{3.197275in}}%
\pgfpathlineto{\pgfqpoint{2.156632in}{3.250738in}}%
\pgfpathlineto{\pgfqpoint{2.157069in}{3.240130in}}%
\pgfpathlineto{\pgfqpoint{2.158670in}{3.179234in}}%
\pgfpathlineto{\pgfqpoint{2.159252in}{3.189189in}}%
\pgfpathlineto{\pgfqpoint{2.160562in}{3.231532in}}%
\pgfpathlineto{\pgfqpoint{2.161144in}{3.219562in}}%
\pgfpathlineto{\pgfqpoint{2.162891in}{3.168960in}}%
\pgfpathlineto{\pgfqpoint{2.163328in}{3.169589in}}%
\pgfpathlineto{\pgfqpoint{2.164346in}{3.173520in}}%
\pgfpathlineto{\pgfqpoint{2.165220in}{3.178872in}}%
\pgfpathlineto{\pgfqpoint{2.165802in}{3.176660in}}%
\pgfpathlineto{\pgfqpoint{2.167985in}{3.170710in}}%
\pgfpathlineto{\pgfqpoint{2.168276in}{3.171755in}}%
\pgfpathlineto{\pgfqpoint{2.169149in}{3.193775in}}%
\pgfpathlineto{\pgfqpoint{2.170168in}{3.219908in}}%
\pgfpathlineto{\pgfqpoint{2.170605in}{3.210949in}}%
\pgfpathlineto{\pgfqpoint{2.172060in}{3.159735in}}%
\pgfpathlineto{\pgfqpoint{2.172642in}{3.173205in}}%
\pgfpathlineto{\pgfqpoint{2.174243in}{3.216453in}}%
\pgfpathlineto{\pgfqpoint{2.174680in}{3.213437in}}%
\pgfpathlineto{\pgfqpoint{2.176863in}{3.171908in}}%
\pgfpathlineto{\pgfqpoint{2.177882in}{3.159070in}}%
\pgfpathlineto{\pgfqpoint{2.178319in}{3.162526in}}%
\pgfpathlineto{\pgfqpoint{2.180211in}{3.219445in}}%
\pgfpathlineto{\pgfqpoint{2.181084in}{3.199287in}}%
\pgfpathlineto{\pgfqpoint{2.182831in}{3.161169in}}%
\pgfpathlineto{\pgfqpoint{2.183122in}{3.161701in}}%
\pgfpathlineto{\pgfqpoint{2.184577in}{3.174036in}}%
\pgfpathlineto{\pgfqpoint{2.186469in}{3.215538in}}%
\pgfpathlineto{\pgfqpoint{2.187197in}{3.201201in}}%
\pgfpathlineto{\pgfqpoint{2.188216in}{3.177359in}}%
\pgfpathlineto{\pgfqpoint{2.188653in}{3.182021in}}%
\pgfpathlineto{\pgfqpoint{2.189962in}{3.211668in}}%
\pgfpathlineto{\pgfqpoint{2.190545in}{3.205002in}}%
\pgfpathlineto{\pgfqpoint{2.191563in}{3.184951in}}%
\pgfpathlineto{\pgfqpoint{2.192146in}{3.191099in}}%
\pgfpathlineto{\pgfqpoint{2.192873in}{3.201839in}}%
\pgfpathlineto{\pgfqpoint{2.193310in}{3.196145in}}%
\pgfpathlineto{\pgfqpoint{2.195057in}{3.141434in}}%
\pgfpathlineto{\pgfqpoint{2.195784in}{3.153117in}}%
\pgfpathlineto{\pgfqpoint{2.198550in}{3.204773in}}%
\pgfpathlineto{\pgfqpoint{2.199277in}{3.207223in}}%
\pgfpathlineto{\pgfqpoint{2.199714in}{3.205809in}}%
\pgfpathlineto{\pgfqpoint{2.200296in}{3.203862in}}%
\pgfpathlineto{\pgfqpoint{2.200733in}{3.205367in}}%
\pgfpathlineto{\pgfqpoint{2.201752in}{3.212478in}}%
\pgfpathlineto{\pgfqpoint{2.202188in}{3.208415in}}%
\pgfpathlineto{\pgfqpoint{2.204080in}{3.158269in}}%
\pgfpathlineto{\pgfqpoint{2.204808in}{3.172776in}}%
\pgfpathlineto{\pgfqpoint{2.205827in}{3.194742in}}%
\pgfpathlineto{\pgfqpoint{2.206264in}{3.190205in}}%
\pgfpathlineto{\pgfqpoint{2.207865in}{3.148096in}}%
\pgfpathlineto{\pgfqpoint{2.208447in}{3.162175in}}%
\pgfpathlineto{\pgfqpoint{2.210048in}{3.219123in}}%
\pgfpathlineto{\pgfqpoint{2.210630in}{3.212081in}}%
\pgfpathlineto{\pgfqpoint{2.212231in}{3.175851in}}%
\pgfpathlineto{\pgfqpoint{2.212668in}{3.182642in}}%
\pgfpathlineto{\pgfqpoint{2.214123in}{3.225722in}}%
\pgfpathlineto{\pgfqpoint{2.214705in}{3.214297in}}%
\pgfpathlineto{\pgfqpoint{2.216306in}{3.163846in}}%
\pgfpathlineto{\pgfqpoint{2.216888in}{3.169575in}}%
\pgfpathlineto{\pgfqpoint{2.218053in}{3.187870in}}%
\pgfpathlineto{\pgfqpoint{2.218489in}{3.183317in}}%
\pgfpathlineto{\pgfqpoint{2.220381in}{3.158629in}}%
\pgfpathlineto{\pgfqpoint{2.220673in}{3.159763in}}%
\pgfpathlineto{\pgfqpoint{2.221546in}{3.175222in}}%
\pgfpathlineto{\pgfqpoint{2.223729in}{3.233511in}}%
\pgfpathlineto{\pgfqpoint{2.224311in}{3.226383in}}%
\pgfpathlineto{\pgfqpoint{2.226349in}{3.148917in}}%
\pgfpathlineto{\pgfqpoint{2.227222in}{3.169406in}}%
\pgfpathlineto{\pgfqpoint{2.228678in}{3.216864in}}%
\pgfpathlineto{\pgfqpoint{2.229260in}{3.210403in}}%
\pgfpathlineto{\pgfqpoint{2.233189in}{3.133692in}}%
\pgfpathlineto{\pgfqpoint{2.233626in}{3.137125in}}%
\pgfpathlineto{\pgfqpoint{2.234791in}{3.185076in}}%
\pgfpathlineto{\pgfqpoint{2.236100in}{3.217920in}}%
\pgfpathlineto{\pgfqpoint{2.236537in}{3.216821in}}%
\pgfpathlineto{\pgfqpoint{2.236828in}{3.216337in}}%
\pgfpathlineto{\pgfqpoint{2.237410in}{3.217425in}}%
\pgfpathlineto{\pgfqpoint{2.237701in}{3.217627in}}%
\pgfpathlineto{\pgfqpoint{2.237993in}{3.216432in}}%
\pgfpathlineto{\pgfqpoint{2.239594in}{3.200002in}}%
\pgfpathlineto{\pgfqpoint{2.240612in}{3.203582in}}%
\pgfpathlineto{\pgfqpoint{2.241195in}{3.195636in}}%
\pgfpathlineto{\pgfqpoint{2.242941in}{3.150575in}}%
\pgfpathlineto{\pgfqpoint{2.243669in}{3.160130in}}%
\pgfpathlineto{\pgfqpoint{2.244542in}{3.172839in}}%
\pgfpathlineto{\pgfqpoint{2.244979in}{3.168510in}}%
\pgfpathlineto{\pgfqpoint{2.245998in}{3.153592in}}%
\pgfpathlineto{\pgfqpoint{2.246434in}{3.159638in}}%
\pgfpathlineto{\pgfqpoint{2.248472in}{3.231973in}}%
\pgfpathlineto{\pgfqpoint{2.249345in}{3.214798in}}%
\pgfpathlineto{\pgfqpoint{2.250655in}{3.193394in}}%
\pgfpathlineto{\pgfqpoint{2.251092in}{3.194125in}}%
\pgfpathlineto{\pgfqpoint{2.252547in}{3.203151in}}%
\pgfpathlineto{\pgfqpoint{2.252984in}{3.199579in}}%
\pgfpathlineto{\pgfqpoint{2.254876in}{3.167518in}}%
\pgfpathlineto{\pgfqpoint{2.255749in}{3.168674in}}%
\pgfpathlineto{\pgfqpoint{2.256477in}{3.171674in}}%
\pgfpathlineto{\pgfqpoint{2.259388in}{3.201273in}}%
\pgfpathlineto{\pgfqpoint{2.259970in}{3.194903in}}%
\pgfpathlineto{\pgfqpoint{2.261425in}{3.166522in}}%
\pgfpathlineto{\pgfqpoint{2.262008in}{3.172112in}}%
\pgfpathlineto{\pgfqpoint{2.264336in}{3.193217in}}%
\pgfpathlineto{\pgfqpoint{2.264918in}{3.196475in}}%
\pgfpathlineto{\pgfqpoint{2.266665in}{3.228080in}}%
\pgfpathlineto{\pgfqpoint{2.267393in}{3.214371in}}%
\pgfpathlineto{\pgfqpoint{2.268848in}{3.177651in}}%
\pgfpathlineto{\pgfqpoint{2.269430in}{3.182069in}}%
\pgfpathlineto{\pgfqpoint{2.269867in}{3.184221in}}%
\pgfpathlineto{\pgfqpoint{2.270158in}{3.181938in}}%
\pgfpathlineto{\pgfqpoint{2.271614in}{3.155799in}}%
\pgfpathlineto{\pgfqpoint{2.272341in}{3.164213in}}%
\pgfpathlineto{\pgfqpoint{2.274088in}{3.177161in}}%
\pgfpathlineto{\pgfqpoint{2.274670in}{3.176969in}}%
\pgfpathlineto{\pgfqpoint{2.274816in}{3.176642in}}%
\pgfpathlineto{\pgfqpoint{2.275252in}{3.175741in}}%
\pgfpathlineto{\pgfqpoint{2.275543in}{3.176618in}}%
\pgfpathlineto{\pgfqpoint{2.276271in}{3.191199in}}%
\pgfpathlineto{\pgfqpoint{2.277435in}{3.220468in}}%
\pgfpathlineto{\pgfqpoint{2.277872in}{3.214791in}}%
\pgfpathlineto{\pgfqpoint{2.279327in}{3.188366in}}%
\pgfpathlineto{\pgfqpoint{2.279764in}{3.191124in}}%
\pgfpathlineto{\pgfqpoint{2.280928in}{3.200993in}}%
\pgfpathlineto{\pgfqpoint{2.281365in}{3.197200in}}%
\pgfpathlineto{\pgfqpoint{2.282821in}{3.177056in}}%
\pgfpathlineto{\pgfqpoint{2.283403in}{3.181554in}}%
\pgfpathlineto{\pgfqpoint{2.284713in}{3.195881in}}%
\pgfpathlineto{\pgfqpoint{2.285149in}{3.193211in}}%
\pgfpathlineto{\pgfqpoint{2.286896in}{3.160251in}}%
\pgfpathlineto{\pgfqpoint{2.287624in}{3.169160in}}%
\pgfpathlineto{\pgfqpoint{2.289516in}{3.204727in}}%
\pgfpathlineto{\pgfqpoint{2.290098in}{3.201106in}}%
\pgfpathlineto{\pgfqpoint{2.291844in}{3.184925in}}%
\pgfpathlineto{\pgfqpoint{2.292427in}{3.188536in}}%
\pgfpathlineto{\pgfqpoint{2.293300in}{3.195487in}}%
\pgfpathlineto{\pgfqpoint{2.293882in}{3.193084in}}%
\pgfpathlineto{\pgfqpoint{2.294464in}{3.190478in}}%
\pgfpathlineto{\pgfqpoint{2.295046in}{3.191762in}}%
\pgfpathlineto{\pgfqpoint{2.295338in}{3.192465in}}%
\pgfpathlineto{\pgfqpoint{2.295774in}{3.191129in}}%
\pgfpathlineto{\pgfqpoint{2.297375in}{3.167743in}}%
\pgfpathlineto{\pgfqpoint{2.298394in}{3.157850in}}%
\pgfpathlineto{\pgfqpoint{2.298831in}{3.160461in}}%
\pgfpathlineto{\pgfqpoint{2.301014in}{3.203419in}}%
\pgfpathlineto{\pgfqpoint{2.302178in}{3.196422in}}%
\pgfpathlineto{\pgfqpoint{2.303197in}{3.188781in}}%
\pgfpathlineto{\pgfqpoint{2.303634in}{3.190934in}}%
\pgfpathlineto{\pgfqpoint{2.305380in}{3.226415in}}%
\pgfpathlineto{\pgfqpoint{2.306253in}{3.214941in}}%
\pgfpathlineto{\pgfqpoint{2.310765in}{3.131151in}}%
\pgfpathlineto{\pgfqpoint{2.311202in}{3.136266in}}%
\pgfpathlineto{\pgfqpoint{2.314258in}{3.205039in}}%
\pgfpathlineto{\pgfqpoint{2.314695in}{3.203280in}}%
\pgfpathlineto{\pgfqpoint{2.316005in}{3.190608in}}%
\pgfpathlineto{\pgfqpoint{2.316587in}{3.194414in}}%
\pgfpathlineto{\pgfqpoint{2.318770in}{3.231917in}}%
\pgfpathlineto{\pgfqpoint{2.319498in}{3.221208in}}%
\pgfpathlineto{\pgfqpoint{2.323137in}{3.151148in}}%
\pgfpathlineto{\pgfqpoint{2.323573in}{3.149429in}}%
\pgfpathlineto{\pgfqpoint{2.324156in}{3.150530in}}%
\pgfpathlineto{\pgfqpoint{2.326484in}{3.163366in}}%
\pgfpathlineto{\pgfqpoint{2.330268in}{3.218479in}}%
\pgfpathlineto{\pgfqpoint{2.330560in}{3.218212in}}%
\pgfpathlineto{\pgfqpoint{2.332015in}{3.212407in}}%
\pgfpathlineto{\pgfqpoint{2.332888in}{3.203081in}}%
\pgfpathlineto{\pgfqpoint{2.334053in}{3.186929in}}%
\pgfpathlineto{\pgfqpoint{2.334489in}{3.190008in}}%
\pgfpathlineto{\pgfqpoint{2.335071in}{3.194503in}}%
\pgfpathlineto{\pgfqpoint{2.335508in}{3.189519in}}%
\pgfpathlineto{\pgfqpoint{2.337546in}{3.131878in}}%
\pgfpathlineto{\pgfqpoint{2.338419in}{3.142522in}}%
\pgfpathlineto{\pgfqpoint{2.342640in}{3.220553in}}%
\pgfpathlineto{\pgfqpoint{2.343659in}{3.222088in}}%
\pgfpathlineto{\pgfqpoint{2.344823in}{3.233013in}}%
\pgfpathlineto{\pgfqpoint{2.345551in}{3.240354in}}%
\pgfpathlineto{\pgfqpoint{2.345842in}{3.237297in}}%
\pgfpathlineto{\pgfqpoint{2.346715in}{3.185621in}}%
\pgfpathlineto{\pgfqpoint{2.348025in}{3.099089in}}%
\pgfpathlineto{\pgfqpoint{2.348462in}{3.111331in}}%
\pgfpathlineto{\pgfqpoint{2.350208in}{3.186209in}}%
\pgfpathlineto{\pgfqpoint{2.350790in}{3.177469in}}%
\pgfpathlineto{\pgfqpoint{2.351373in}{3.167363in}}%
\pgfpathlineto{\pgfqpoint{2.351809in}{3.172673in}}%
\pgfpathlineto{\pgfqpoint{2.353701in}{3.254891in}}%
\pgfpathlineto{\pgfqpoint{2.354429in}{3.231816in}}%
\pgfpathlineto{\pgfqpoint{2.356612in}{3.138744in}}%
\pgfpathlineto{\pgfqpoint{2.357049in}{3.139737in}}%
\pgfpathlineto{\pgfqpoint{2.357777in}{3.144275in}}%
\pgfpathlineto{\pgfqpoint{2.360979in}{3.202800in}}%
\pgfpathlineto{\pgfqpoint{2.362289in}{3.250658in}}%
\pgfpathlineto{\pgfqpoint{2.362725in}{3.246425in}}%
\pgfpathlineto{\pgfqpoint{2.363016in}{3.244241in}}%
\pgfpathlineto{\pgfqpoint{2.363453in}{3.249529in}}%
\pgfpathlineto{\pgfqpoint{2.364908in}{3.306234in}}%
\pgfpathlineto{\pgfqpoint{2.365636in}{3.285328in}}%
\pgfpathlineto{\pgfqpoint{2.366946in}{3.192352in}}%
\pgfpathlineto{\pgfqpoint{2.368693in}{2.915571in}}%
\pgfpathlineto{\pgfqpoint{2.369420in}{2.961967in}}%
\pgfpathlineto{\pgfqpoint{2.371312in}{3.276767in}}%
\pgfpathlineto{\pgfqpoint{2.372040in}{3.198630in}}%
\pgfpathlineto{\pgfqpoint{2.372913in}{3.062848in}}%
\pgfpathlineto{\pgfqpoint{2.373496in}{3.122152in}}%
\pgfpathlineto{\pgfqpoint{2.374951in}{3.432719in}}%
\pgfpathlineto{\pgfqpoint{2.375533in}{3.386955in}}%
\pgfpathlineto{\pgfqpoint{2.376406in}{3.318394in}}%
\pgfpathlineto{\pgfqpoint{2.376843in}{3.336039in}}%
\pgfpathlineto{\pgfqpoint{2.377425in}{3.356784in}}%
\pgfpathlineto{\pgfqpoint{2.377716in}{3.342958in}}%
\pgfpathlineto{\pgfqpoint{2.378881in}{3.115613in}}%
\pgfpathlineto{\pgfqpoint{2.380045in}{2.921981in}}%
\pgfpathlineto{\pgfqpoint{2.380482in}{2.943753in}}%
\pgfpathlineto{\pgfqpoint{2.381937in}{3.121458in}}%
\pgfpathlineto{\pgfqpoint{2.382665in}{3.093418in}}%
\pgfpathlineto{\pgfqpoint{2.382810in}{3.092289in}}%
\pgfpathlineto{\pgfqpoint{2.382956in}{3.096435in}}%
\pgfpathlineto{\pgfqpoint{2.383684in}{3.210205in}}%
\pgfpathlineto{\pgfqpoint{2.384848in}{3.405595in}}%
\pgfpathlineto{\pgfqpoint{2.385285in}{3.352418in}}%
\pgfpathlineto{\pgfqpoint{2.386886in}{3.075674in}}%
\pgfpathlineto{\pgfqpoint{2.387322in}{3.103437in}}%
\pgfpathlineto{\pgfqpoint{2.389069in}{3.348477in}}%
\pgfpathlineto{\pgfqpoint{2.389797in}{3.290718in}}%
\pgfpathlineto{\pgfqpoint{2.391107in}{3.119108in}}%
\pgfpathlineto{\pgfqpoint{2.391543in}{3.148711in}}%
\pgfpathlineto{\pgfqpoint{2.392416in}{3.238875in}}%
\pgfpathlineto{\pgfqpoint{2.392853in}{3.195839in}}%
\pgfpathlineto{\pgfqpoint{2.394017in}{3.006140in}}%
\pgfpathlineto{\pgfqpoint{2.394600in}{3.065494in}}%
\pgfpathlineto{\pgfqpoint{2.395764in}{3.203345in}}%
\pgfpathlineto{\pgfqpoint{2.396346in}{3.178954in}}%
\pgfpathlineto{\pgfqpoint{2.397219in}{3.126395in}}%
\pgfpathlineto{\pgfqpoint{2.397656in}{3.140531in}}%
\pgfpathlineto{\pgfqpoint{2.399257in}{3.332287in}}%
\pgfpathlineto{\pgfqpoint{2.399985in}{3.266486in}}%
\pgfpathlineto{\pgfqpoint{2.401004in}{3.126234in}}%
\pgfpathlineto{\pgfqpoint{2.401586in}{3.162908in}}%
\pgfpathlineto{\pgfqpoint{2.402750in}{3.303503in}}%
\pgfpathlineto{\pgfqpoint{2.403332in}{3.259363in}}%
\pgfpathlineto{\pgfqpoint{2.404788in}{3.052920in}}%
\pgfpathlineto{\pgfqpoint{2.405225in}{3.096736in}}%
\pgfpathlineto{\pgfqpoint{2.406680in}{3.350353in}}%
\pgfpathlineto{\pgfqpoint{2.407262in}{3.311773in}}%
\pgfpathlineto{\pgfqpoint{2.409445in}{3.069598in}}%
\pgfpathlineto{\pgfqpoint{2.409882in}{3.079385in}}%
\pgfpathlineto{\pgfqpoint{2.411192in}{3.134778in}}%
\pgfpathlineto{\pgfqpoint{2.411774in}{3.125936in}}%
\pgfpathlineto{\pgfqpoint{2.412793in}{3.095859in}}%
\pgfpathlineto{\pgfqpoint{2.413230in}{3.108231in}}%
\pgfpathlineto{\pgfqpoint{2.416723in}{3.379042in}}%
\pgfpathlineto{\pgfqpoint{2.417596in}{3.328423in}}%
\pgfpathlineto{\pgfqpoint{2.419779in}{3.049970in}}%
\pgfpathlineto{\pgfqpoint{2.420507in}{3.097196in}}%
\pgfpathlineto{\pgfqpoint{2.421817in}{3.215797in}}%
\pgfpathlineto{\pgfqpoint{2.422253in}{3.202261in}}%
\pgfpathlineto{\pgfqpoint{2.425310in}{3.073331in}}%
\pgfpathlineto{\pgfqpoint{2.425455in}{3.073414in}}%
\pgfpathlineto{\pgfqpoint{2.425892in}{3.087154in}}%
\pgfpathlineto{\pgfqpoint{2.426911in}{3.231418in}}%
\pgfpathlineto{\pgfqpoint{2.427930in}{3.346699in}}%
\pgfpathlineto{\pgfqpoint{2.428366in}{3.325666in}}%
\pgfpathlineto{\pgfqpoint{2.429822in}{3.208679in}}%
\pgfpathlineto{\pgfqpoint{2.430404in}{3.226594in}}%
\pgfpathlineto{\pgfqpoint{2.431423in}{3.261628in}}%
\pgfpathlineto{\pgfqpoint{2.432005in}{3.256698in}}%
\pgfpathlineto{\pgfqpoint{2.432150in}{3.256532in}}%
\pgfpathlineto{\pgfqpoint{2.432296in}{3.257845in}}%
\pgfpathlineto{\pgfqpoint{2.433169in}{3.283392in}}%
\pgfpathlineto{\pgfqpoint{2.433606in}{3.267714in}}%
\pgfpathlineto{\pgfqpoint{2.434916in}{3.023180in}}%
\pgfpathlineto{\pgfqpoint{2.435789in}{2.948125in}}%
\pgfpathlineto{\pgfqpoint{2.436226in}{2.972122in}}%
\pgfpathlineto{\pgfqpoint{2.438263in}{3.146042in}}%
\pgfpathlineto{\pgfqpoint{2.438846in}{3.131677in}}%
\pgfpathlineto{\pgfqpoint{2.439282in}{3.118461in}}%
\pgfpathlineto{\pgfqpoint{2.439719in}{3.132259in}}%
\pgfpathlineto{\pgfqpoint{2.440883in}{3.351345in}}%
\pgfpathlineto{\pgfqpoint{2.441902in}{3.483919in}}%
\pgfpathlineto{\pgfqpoint{2.442339in}{3.460812in}}%
\pgfpathlineto{\pgfqpoint{2.444085in}{3.222502in}}%
\pgfpathlineto{\pgfqpoint{2.445104in}{3.254757in}}%
\pgfpathlineto{\pgfqpoint{2.445250in}{3.255074in}}%
\pgfpathlineto{\pgfqpoint{2.445832in}{3.222605in}}%
\pgfpathlineto{\pgfqpoint{2.448306in}{2.886905in}}%
\pgfpathlineto{\pgfqpoint{2.449179in}{2.973369in}}%
\pgfpathlineto{\pgfqpoint{2.450780in}{3.173045in}}%
\pgfpathlineto{\pgfqpoint{2.451217in}{3.165887in}}%
\pgfpathlineto{\pgfqpoint{2.451799in}{3.154593in}}%
\pgfpathlineto{\pgfqpoint{2.452090in}{3.163727in}}%
\pgfpathlineto{\pgfqpoint{2.453109in}{3.307814in}}%
\pgfpathlineto{\pgfqpoint{2.454128in}{3.429517in}}%
\pgfpathlineto{\pgfqpoint{2.454565in}{3.407762in}}%
\pgfpathlineto{\pgfqpoint{2.457621in}{3.183365in}}%
\pgfpathlineto{\pgfqpoint{2.458785in}{3.131423in}}%
\pgfpathlineto{\pgfqpoint{2.459368in}{3.151439in}}%
\pgfpathlineto{\pgfqpoint{2.460969in}{3.267363in}}%
\pgfpathlineto{\pgfqpoint{2.461405in}{3.246631in}}%
\pgfpathlineto{\pgfqpoint{2.463006in}{3.053081in}}%
\pgfpathlineto{\pgfqpoint{2.463734in}{3.116980in}}%
\pgfpathlineto{\pgfqpoint{2.464753in}{3.218274in}}%
\pgfpathlineto{\pgfqpoint{2.465189in}{3.189767in}}%
\pgfpathlineto{\pgfqpoint{2.466208in}{3.108835in}}%
\pgfpathlineto{\pgfqpoint{2.466790in}{3.136050in}}%
\pgfpathlineto{\pgfqpoint{2.467518in}{3.176394in}}%
\pgfpathlineto{\pgfqpoint{2.468100in}{3.155487in}}%
\pgfpathlineto{\pgfqpoint{2.469556in}{3.039521in}}%
\pgfpathlineto{\pgfqpoint{2.469992in}{3.062708in}}%
\pgfpathlineto{\pgfqpoint{2.471448in}{3.394134in}}%
\pgfpathlineto{\pgfqpoint{2.472030in}{3.447399in}}%
\pgfpathlineto{\pgfqpoint{2.472467in}{3.410684in}}%
\pgfpathlineto{\pgfqpoint{2.473922in}{3.242905in}}%
\pgfpathlineto{\pgfqpoint{2.474504in}{3.254892in}}%
\pgfpathlineto{\pgfqpoint{2.474941in}{3.263826in}}%
\pgfpathlineto{\pgfqpoint{2.475232in}{3.255380in}}%
\pgfpathlineto{\pgfqpoint{2.476251in}{3.109393in}}%
\pgfpathlineto{\pgfqpoint{2.477124in}{3.014900in}}%
\pgfpathlineto{\pgfqpoint{2.477706in}{3.044172in}}%
\pgfpathlineto{\pgfqpoint{2.478871in}{3.117165in}}%
\pgfpathlineto{\pgfqpoint{2.479453in}{3.106994in}}%
\pgfpathlineto{\pgfqpoint{2.479889in}{3.100822in}}%
\pgfpathlineto{\pgfqpoint{2.480181in}{3.108318in}}%
\pgfpathlineto{\pgfqpoint{2.481199in}{3.225805in}}%
\pgfpathlineto{\pgfqpoint{2.482509in}{3.321408in}}%
\pgfpathlineto{\pgfqpoint{2.482946in}{3.315642in}}%
\pgfpathlineto{\pgfqpoint{2.484838in}{3.260529in}}%
\pgfpathlineto{\pgfqpoint{2.486293in}{3.187963in}}%
\pgfpathlineto{\pgfqpoint{2.486876in}{3.190476in}}%
\pgfpathlineto{\pgfqpoint{2.487167in}{3.188138in}}%
\pgfpathlineto{\pgfqpoint{2.487894in}{3.148149in}}%
\pgfpathlineto{\pgfqpoint{2.489350in}{3.028973in}}%
\pgfpathlineto{\pgfqpoint{2.489932in}{3.050522in}}%
\pgfpathlineto{\pgfqpoint{2.490660in}{3.077640in}}%
\pgfpathlineto{\pgfqpoint{2.491096in}{3.060132in}}%
\pgfpathlineto{\pgfqpoint{2.491970in}{3.006061in}}%
\pgfpathlineto{\pgfqpoint{2.492406in}{3.025504in}}%
\pgfpathlineto{\pgfqpoint{2.497064in}{3.471210in}}%
\pgfpathlineto{\pgfqpoint{2.497355in}{3.465201in}}%
\pgfpathlineto{\pgfqpoint{2.499538in}{3.300826in}}%
\pgfpathlineto{\pgfqpoint{2.500702in}{3.125852in}}%
\pgfpathlineto{\pgfqpoint{2.502740in}{2.919122in}}%
\pgfpathlineto{\pgfqpoint{2.503468in}{2.905839in}}%
\pgfpathlineto{\pgfqpoint{2.503759in}{2.911787in}}%
\pgfpathlineto{\pgfqpoint{2.504778in}{3.013210in}}%
\pgfpathlineto{\pgfqpoint{2.508853in}{3.389082in}}%
\pgfpathlineto{\pgfqpoint{2.508999in}{3.390344in}}%
\pgfpathlineto{\pgfqpoint{2.509290in}{3.386039in}}%
\pgfpathlineto{\pgfqpoint{2.512928in}{3.249123in}}%
\pgfpathlineto{\pgfqpoint{2.513656in}{3.270423in}}%
\pgfpathlineto{\pgfqpoint{2.514238in}{3.282091in}}%
\pgfpathlineto{\pgfqpoint{2.514529in}{3.274615in}}%
\pgfpathlineto{\pgfqpoint{2.515548in}{3.167667in}}%
\pgfpathlineto{\pgfqpoint{2.517877in}{2.871898in}}%
\pgfpathlineto{\pgfqpoint{2.518314in}{2.885868in}}%
\pgfpathlineto{\pgfqpoint{2.520497in}{3.036926in}}%
\pgfpathlineto{\pgfqpoint{2.524426in}{3.648689in}}%
\pgfpathlineto{\pgfqpoint{2.525300in}{3.569087in}}%
\pgfpathlineto{\pgfqpoint{2.530103in}{2.845920in}}%
\pgfpathlineto{\pgfqpoint{2.531122in}{2.804627in}}%
\pgfpathlineto{\pgfqpoint{2.531558in}{2.814990in}}%
\pgfpathlineto{\pgfqpoint{2.532577in}{2.950287in}}%
\pgfpathlineto{\pgfqpoint{2.536216in}{3.518900in}}%
\pgfpathlineto{\pgfqpoint{2.536507in}{3.515088in}}%
\pgfpathlineto{\pgfqpoint{2.539418in}{3.368849in}}%
\pgfpathlineto{\pgfqpoint{2.542183in}{2.933767in}}%
\pgfpathlineto{\pgfqpoint{2.543493in}{2.940331in}}%
\pgfpathlineto{\pgfqpoint{2.544803in}{2.871377in}}%
\pgfpathlineto{\pgfqpoint{2.545385in}{2.900662in}}%
\pgfpathlineto{\pgfqpoint{2.546986in}{3.267065in}}%
\pgfpathlineto{\pgfqpoint{2.549024in}{3.480428in}}%
\pgfpathlineto{\pgfqpoint{2.549460in}{3.487977in}}%
\pgfpathlineto{\pgfqpoint{2.549897in}{3.480276in}}%
\pgfpathlineto{\pgfqpoint{2.551061in}{3.379671in}}%
\pgfpathlineto{\pgfqpoint{2.554991in}{2.888947in}}%
\pgfpathlineto{\pgfqpoint{2.555719in}{2.952942in}}%
\pgfpathlineto{\pgfqpoint{2.558921in}{3.249748in}}%
\pgfpathlineto{\pgfqpoint{2.559648in}{3.276060in}}%
\pgfpathlineto{\pgfqpoint{2.560085in}{3.265498in}}%
\pgfpathlineto{\pgfqpoint{2.560813in}{3.247666in}}%
\pgfpathlineto{\pgfqpoint{2.561395in}{3.257487in}}%
\pgfpathlineto{\pgfqpoint{2.563724in}{3.300648in}}%
\pgfpathlineto{\pgfqpoint{2.564015in}{3.301989in}}%
\pgfpathlineto{\pgfqpoint{2.564306in}{3.300323in}}%
\pgfpathlineto{\pgfqpoint{2.565034in}{3.272270in}}%
\pgfpathlineto{\pgfqpoint{2.568672in}{3.005959in}}%
\pgfpathlineto{\pgfqpoint{2.569546in}{3.019697in}}%
\pgfpathlineto{\pgfqpoint{2.570710in}{3.094406in}}%
\pgfpathlineto{\pgfqpoint{2.574494in}{3.401948in}}%
\pgfpathlineto{\pgfqpoint{2.574640in}{3.401646in}}%
\pgfpathlineto{\pgfqpoint{2.575367in}{3.360699in}}%
\pgfpathlineto{\pgfqpoint{2.578861in}{3.080105in}}%
\pgfpathlineto{\pgfqpoint{2.579297in}{3.082141in}}%
\pgfpathlineto{\pgfqpoint{2.580462in}{3.102788in}}%
\pgfpathlineto{\pgfqpoint{2.581771in}{3.165006in}}%
\pgfpathlineto{\pgfqpoint{2.584100in}{3.292471in}}%
\pgfpathlineto{\pgfqpoint{2.584537in}{3.285186in}}%
\pgfpathlineto{\pgfqpoint{2.585701in}{3.191126in}}%
\pgfpathlineto{\pgfqpoint{2.587011in}{3.113327in}}%
\pgfpathlineto{\pgfqpoint{2.587448in}{3.117947in}}%
\pgfpathlineto{\pgfqpoint{2.588030in}{3.126246in}}%
\pgfpathlineto{\pgfqpoint{2.588467in}{3.120336in}}%
\pgfpathlineto{\pgfqpoint{2.590213in}{3.052126in}}%
\pgfpathlineto{\pgfqpoint{2.590795in}{3.070796in}}%
\pgfpathlineto{\pgfqpoint{2.592833in}{3.256325in}}%
\pgfpathlineto{\pgfqpoint{2.595307in}{3.469584in}}%
\pgfpathlineto{\pgfqpoint{2.595453in}{3.467022in}}%
\pgfpathlineto{\pgfqpoint{2.596326in}{3.390780in}}%
\pgfpathlineto{\pgfqpoint{2.601566in}{2.838347in}}%
\pgfpathlineto{\pgfqpoint{2.601711in}{2.837802in}}%
\pgfpathlineto{\pgfqpoint{2.601857in}{2.839629in}}%
\pgfpathlineto{\pgfqpoint{2.602584in}{2.886890in}}%
\pgfpathlineto{\pgfqpoint{2.605204in}{3.310515in}}%
\pgfpathlineto{\pgfqpoint{2.607533in}{3.660890in}}%
\pgfpathlineto{\pgfqpoint{2.607970in}{3.648142in}}%
\pgfpathlineto{\pgfqpoint{2.608988in}{3.483351in}}%
\pgfpathlineto{\pgfqpoint{2.612190in}{2.841432in}}%
\pgfpathlineto{\pgfqpoint{2.612627in}{2.853383in}}%
\pgfpathlineto{\pgfqpoint{2.615684in}{3.046264in}}%
\pgfpathlineto{\pgfqpoint{2.619322in}{3.426747in}}%
\pgfpathlineto{\pgfqpoint{2.620196in}{3.443661in}}%
\pgfpathlineto{\pgfqpoint{2.620632in}{3.434965in}}%
\pgfpathlineto{\pgfqpoint{2.621797in}{3.329331in}}%
\pgfpathlineto{\pgfqpoint{2.625581in}{2.964324in}}%
\pgfpathlineto{\pgfqpoint{2.625872in}{2.960912in}}%
\pgfpathlineto{\pgfqpoint{2.626308in}{2.968202in}}%
\pgfpathlineto{\pgfqpoint{2.627764in}{3.061137in}}%
\pgfpathlineto{\pgfqpoint{2.633295in}{3.472615in}}%
\pgfpathlineto{\pgfqpoint{2.633440in}{3.468323in}}%
\pgfpathlineto{\pgfqpoint{2.634313in}{3.347884in}}%
\pgfpathlineto{\pgfqpoint{2.636788in}{3.058737in}}%
\pgfpathlineto{\pgfqpoint{2.638680in}{3.031607in}}%
\pgfpathlineto{\pgfqpoint{2.639116in}{3.033103in}}%
\pgfpathlineto{\pgfqpoint{2.639844in}{3.045729in}}%
\pgfpathlineto{\pgfqpoint{2.641009in}{3.127158in}}%
\pgfpathlineto{\pgfqpoint{2.644938in}{3.396614in}}%
\pgfpathlineto{\pgfqpoint{2.645229in}{3.403053in}}%
\pgfpathlineto{\pgfqpoint{2.645666in}{3.389712in}}%
\pgfpathlineto{\pgfqpoint{2.647704in}{3.171284in}}%
\pgfpathlineto{\pgfqpoint{2.650469in}{2.932577in}}%
\pgfpathlineto{\pgfqpoint{2.651051in}{2.952848in}}%
\pgfpathlineto{\pgfqpoint{2.655418in}{3.386912in}}%
\pgfpathlineto{\pgfqpoint{2.657019in}{3.487130in}}%
\pgfpathlineto{\pgfqpoint{2.657310in}{3.480206in}}%
\pgfpathlineto{\pgfqpoint{2.658183in}{3.395707in}}%
\pgfpathlineto{\pgfqpoint{2.662113in}{2.891178in}}%
\pgfpathlineto{\pgfqpoint{2.662258in}{2.891250in}}%
\pgfpathlineto{\pgfqpoint{2.662840in}{2.904963in}}%
\pgfpathlineto{\pgfqpoint{2.664150in}{3.028225in}}%
\pgfpathlineto{\pgfqpoint{2.667498in}{3.519876in}}%
\pgfpathlineto{\pgfqpoint{2.668371in}{3.466261in}}%
\pgfpathlineto{\pgfqpoint{2.673029in}{2.947844in}}%
\pgfpathlineto{\pgfqpoint{2.673756in}{2.971280in}}%
\pgfpathlineto{\pgfqpoint{2.677249in}{3.285123in}}%
\pgfpathlineto{\pgfqpoint{2.678559in}{3.353228in}}%
\pgfpathlineto{\pgfqpoint{2.678850in}{3.348923in}}%
\pgfpathlineto{\pgfqpoint{2.680306in}{3.275764in}}%
\pgfpathlineto{\pgfqpoint{2.685254in}{3.047686in}}%
\pgfpathlineto{\pgfqpoint{2.686128in}{3.054570in}}%
\pgfpathlineto{\pgfqpoint{2.687147in}{3.088015in}}%
\pgfpathlineto{\pgfqpoint{2.692241in}{3.352270in}}%
\pgfpathlineto{\pgfqpoint{2.692532in}{3.351145in}}%
\pgfpathlineto{\pgfqpoint{2.693259in}{3.328466in}}%
\pgfpathlineto{\pgfqpoint{2.695443in}{3.152535in}}%
\pgfpathlineto{\pgfqpoint{2.697917in}{2.942853in}}%
\pgfpathlineto{\pgfqpoint{2.698208in}{2.947125in}}%
\pgfpathlineto{\pgfqpoint{2.699227in}{3.022370in}}%
\pgfpathlineto{\pgfqpoint{2.703884in}{3.414995in}}%
\pgfpathlineto{\pgfqpoint{2.704321in}{3.419225in}}%
\pgfpathlineto{\pgfqpoint{2.704758in}{3.415792in}}%
\pgfpathlineto{\pgfqpoint{2.705631in}{3.364352in}}%
\pgfpathlineto{\pgfqpoint{2.709124in}{2.960234in}}%
\pgfpathlineto{\pgfqpoint{2.710143in}{2.987887in}}%
\pgfpathlineto{\pgfqpoint{2.712180in}{3.133522in}}%
\pgfpathlineto{\pgfqpoint{2.714509in}{3.264295in}}%
\pgfpathlineto{\pgfqpoint{2.716692in}{3.361527in}}%
\pgfpathlineto{\pgfqpoint{2.717420in}{3.343306in}}%
\pgfpathlineto{\pgfqpoint{2.720185in}{3.177179in}}%
\pgfpathlineto{\pgfqpoint{2.722514in}{3.007000in}}%
\pgfpathlineto{\pgfqpoint{2.722805in}{3.008908in}}%
\pgfpathlineto{\pgfqpoint{2.723970in}{3.044023in}}%
\pgfpathlineto{\pgfqpoint{2.729646in}{3.324911in}}%
\pgfpathlineto{\pgfqpoint{2.730374in}{3.305696in}}%
\pgfpathlineto{\pgfqpoint{2.733867in}{3.151384in}}%
\pgfpathlineto{\pgfqpoint{2.734449in}{3.152783in}}%
\pgfpathlineto{\pgfqpoint{2.734594in}{3.153028in}}%
\pgfpathlineto{\pgfqpoint{2.734886in}{3.151951in}}%
\pgfpathlineto{\pgfqpoint{2.735613in}{3.134340in}}%
\pgfpathlineto{\pgfqpoint{2.736778in}{3.098578in}}%
\pgfpathlineto{\pgfqpoint{2.737214in}{3.104945in}}%
\pgfpathlineto{\pgfqpoint{2.740125in}{3.165315in}}%
\pgfpathlineto{\pgfqpoint{2.741144in}{3.196716in}}%
\pgfpathlineto{\pgfqpoint{2.742599in}{3.245048in}}%
\pgfpathlineto{\pgfqpoint{2.743036in}{3.237991in}}%
\pgfpathlineto{\pgfqpoint{2.743909in}{3.219847in}}%
\pgfpathlineto{\pgfqpoint{2.744346in}{3.227261in}}%
\pgfpathlineto{\pgfqpoint{2.745947in}{3.290996in}}%
\pgfpathlineto{\pgfqpoint{2.746529in}{3.273676in}}%
\pgfpathlineto{\pgfqpoint{2.750750in}{3.077130in}}%
\pgfpathlineto{\pgfqpoint{2.751187in}{3.082097in}}%
\pgfpathlineto{\pgfqpoint{2.753952in}{3.146158in}}%
\pgfpathlineto{\pgfqpoint{2.756572in}{3.310934in}}%
\pgfpathlineto{\pgfqpoint{2.757008in}{3.304213in}}%
\pgfpathlineto{\pgfqpoint{2.758609in}{3.227896in}}%
\pgfpathlineto{\pgfqpoint{2.760065in}{3.178135in}}%
\pgfpathlineto{\pgfqpoint{2.760502in}{3.182070in}}%
\pgfpathlineto{\pgfqpoint{2.761666in}{3.198514in}}%
\pgfpathlineto{\pgfqpoint{2.762103in}{3.194045in}}%
\pgfpathlineto{\pgfqpoint{2.765596in}{3.146405in}}%
\pgfpathlineto{\pgfqpoint{2.766323in}{3.144027in}}%
\pgfpathlineto{\pgfqpoint{2.767633in}{3.125413in}}%
\pgfpathlineto{\pgfqpoint{2.768070in}{3.122106in}}%
\pgfpathlineto{\pgfqpoint{2.768507in}{3.126405in}}%
\pgfpathlineto{\pgfqpoint{2.770399in}{3.193170in}}%
\pgfpathlineto{\pgfqpoint{2.772582in}{3.286627in}}%
\pgfpathlineto{\pgfqpoint{2.773310in}{3.274171in}}%
\pgfpathlineto{\pgfqpoint{2.777094in}{3.161896in}}%
\pgfpathlineto{\pgfqpoint{2.777676in}{3.164883in}}%
\pgfpathlineto{\pgfqpoint{2.777967in}{3.165534in}}%
\pgfpathlineto{\pgfqpoint{2.778258in}{3.163244in}}%
\pgfpathlineto{\pgfqpoint{2.779277in}{3.127805in}}%
\pgfpathlineto{\pgfqpoint{2.780150in}{3.106574in}}%
\pgfpathlineto{\pgfqpoint{2.780587in}{3.116085in}}%
\pgfpathlineto{\pgfqpoint{2.784225in}{3.220018in}}%
\pgfpathlineto{\pgfqpoint{2.784953in}{3.217404in}}%
\pgfpathlineto{\pgfqpoint{2.786263in}{3.200441in}}%
\pgfpathlineto{\pgfqpoint{2.788155in}{3.188652in}}%
\pgfpathlineto{\pgfqpoint{2.788737in}{3.187437in}}%
\pgfpathlineto{\pgfqpoint{2.789174in}{3.188809in}}%
\pgfpathlineto{\pgfqpoint{2.792522in}{3.216197in}}%
\pgfpathlineto{\pgfqpoint{2.793104in}{3.209674in}}%
\pgfpathlineto{\pgfqpoint{2.796015in}{3.147330in}}%
\pgfpathlineto{\pgfqpoint{2.796597in}{3.148646in}}%
\pgfpathlineto{\pgfqpoint{2.796888in}{3.148935in}}%
\pgfpathlineto{\pgfqpoint{2.797179in}{3.147762in}}%
\pgfpathlineto{\pgfqpoint{2.799362in}{3.130165in}}%
\pgfpathlineto{\pgfqpoint{2.799944in}{3.133126in}}%
\pgfpathlineto{\pgfqpoint{2.800672in}{3.156333in}}%
\pgfpathlineto{\pgfqpoint{2.803292in}{3.328490in}}%
\pgfpathlineto{\pgfqpoint{2.804165in}{3.301938in}}%
\pgfpathlineto{\pgfqpoint{2.809114in}{3.055059in}}%
\pgfpathlineto{\pgfqpoint{2.809405in}{3.058696in}}%
\pgfpathlineto{\pgfqpoint{2.813044in}{3.171917in}}%
\pgfpathlineto{\pgfqpoint{2.815227in}{3.296599in}}%
\pgfpathlineto{\pgfqpoint{2.815518in}{3.294915in}}%
\pgfpathlineto{\pgfqpoint{2.817992in}{3.244405in}}%
\pgfpathlineto{\pgfqpoint{2.820030in}{3.162359in}}%
\pgfpathlineto{\pgfqpoint{2.821922in}{3.113537in}}%
\pgfpathlineto{\pgfqpoint{2.822213in}{3.113663in}}%
\pgfpathlineto{\pgfqpoint{2.823086in}{3.116100in}}%
\pgfpathlineto{\pgfqpoint{2.823959in}{3.135671in}}%
\pgfpathlineto{\pgfqpoint{2.825706in}{3.204367in}}%
\pgfpathlineto{\pgfqpoint{2.826434in}{3.191817in}}%
\pgfpathlineto{\pgfqpoint{2.827453in}{3.169274in}}%
\pgfpathlineto{\pgfqpoint{2.827889in}{3.176231in}}%
\pgfpathlineto{\pgfqpoint{2.829927in}{3.254202in}}%
\pgfpathlineto{\pgfqpoint{2.830946in}{3.245692in}}%
\pgfpathlineto{\pgfqpoint{2.831237in}{3.244741in}}%
\pgfpathlineto{\pgfqpoint{2.831673in}{3.247408in}}%
\pgfpathlineto{\pgfqpoint{2.832547in}{3.255168in}}%
\pgfpathlineto{\pgfqpoint{2.832983in}{3.249783in}}%
\pgfpathlineto{\pgfqpoint{2.834875in}{3.169132in}}%
\pgfpathlineto{\pgfqpoint{2.838077in}{3.083968in}}%
\pgfpathlineto{\pgfqpoint{2.838514in}{3.079387in}}%
\pgfpathlineto{\pgfqpoint{2.838951in}{3.082413in}}%
\pgfpathlineto{\pgfqpoint{2.839969in}{3.123272in}}%
\pgfpathlineto{\pgfqpoint{2.843754in}{3.283436in}}%
\pgfpathlineto{\pgfqpoint{2.844627in}{3.291910in}}%
\pgfpathlineto{\pgfqpoint{2.845064in}{3.289450in}}%
\pgfpathlineto{\pgfqpoint{2.846374in}{3.255789in}}%
\pgfpathlineto{\pgfqpoint{2.848848in}{3.152672in}}%
\pgfpathlineto{\pgfqpoint{2.850885in}{3.048568in}}%
\pgfpathlineto{\pgfqpoint{2.851177in}{3.051796in}}%
\pgfpathlineto{\pgfqpoint{2.852341in}{3.109671in}}%
\pgfpathlineto{\pgfqpoint{2.856416in}{3.307983in}}%
\pgfpathlineto{\pgfqpoint{2.856998in}{3.296415in}}%
\pgfpathlineto{\pgfqpoint{2.858890in}{3.173135in}}%
\pgfpathlineto{\pgfqpoint{2.860783in}{3.094202in}}%
\pgfpathlineto{\pgfqpoint{2.861074in}{3.095094in}}%
\pgfpathlineto{\pgfqpoint{2.861801in}{3.098369in}}%
\pgfpathlineto{\pgfqpoint{2.862384in}{3.096383in}}%
\pgfpathlineto{\pgfqpoint{2.862675in}{3.096004in}}%
\pgfpathlineto{\pgfqpoint{2.862820in}{3.096711in}}%
\pgfpathlineto{\pgfqpoint{2.863548in}{3.116845in}}%
\pgfpathlineto{\pgfqpoint{2.867623in}{3.286757in}}%
\pgfpathlineto{\pgfqpoint{2.868205in}{3.291765in}}%
\pgfpathlineto{\pgfqpoint{2.868642in}{3.288323in}}%
\pgfpathlineto{\pgfqpoint{2.869661in}{3.247919in}}%
\pgfpathlineto{\pgfqpoint{2.873008in}{3.076599in}}%
\pgfpathlineto{\pgfqpoint{2.873445in}{3.078783in}}%
\pgfpathlineto{\pgfqpoint{2.875337in}{3.114822in}}%
\pgfpathlineto{\pgfqpoint{2.876647in}{3.181694in}}%
\pgfpathlineto{\pgfqpoint{2.878394in}{3.270082in}}%
\pgfpathlineto{\pgfqpoint{2.878830in}{3.267262in}}%
\pgfpathlineto{\pgfqpoint{2.879558in}{3.261713in}}%
\pgfpathlineto{\pgfqpoint{2.880286in}{3.263336in}}%
\pgfpathlineto{\pgfqpoint{2.880577in}{3.262581in}}%
\pgfpathlineto{\pgfqpoint{2.881450in}{3.247536in}}%
\pgfpathlineto{\pgfqpoint{2.884361in}{3.145068in}}%
\pgfpathlineto{\pgfqpoint{2.885671in}{3.107205in}}%
\pgfpathlineto{\pgfqpoint{2.886107in}{3.112185in}}%
\pgfpathlineto{\pgfqpoint{2.887708in}{3.147266in}}%
\pgfpathlineto{\pgfqpoint{2.888291in}{3.143332in}}%
\pgfpathlineto{\pgfqpoint{2.888873in}{3.139533in}}%
\pgfpathlineto{\pgfqpoint{2.889164in}{3.142555in}}%
\pgfpathlineto{\pgfqpoint{2.890183in}{3.190615in}}%
\pgfpathlineto{\pgfqpoint{2.891929in}{3.264659in}}%
\pgfpathlineto{\pgfqpoint{2.892220in}{3.263196in}}%
\pgfpathlineto{\pgfqpoint{2.893676in}{3.235507in}}%
\pgfpathlineto{\pgfqpoint{2.894695in}{3.242153in}}%
\pgfpathlineto{\pgfqpoint{2.895131in}{3.236690in}}%
\pgfpathlineto{\pgfqpoint{2.896587in}{3.160492in}}%
\pgfpathlineto{\pgfqpoint{2.898333in}{3.128061in}}%
\pgfpathlineto{\pgfqpoint{2.899352in}{3.119927in}}%
\pgfpathlineto{\pgfqpoint{2.899934in}{3.115795in}}%
\pgfpathlineto{\pgfqpoint{2.900371in}{3.120882in}}%
\pgfpathlineto{\pgfqpoint{2.901681in}{3.188742in}}%
\pgfpathlineto{\pgfqpoint{2.903864in}{3.244723in}}%
\pgfpathlineto{\pgfqpoint{2.904592in}{3.250947in}}%
\pgfpathlineto{\pgfqpoint{2.905028in}{3.248906in}}%
\pgfpathlineto{\pgfqpoint{2.906047in}{3.225441in}}%
\pgfpathlineto{\pgfqpoint{2.908667in}{3.166249in}}%
\pgfpathlineto{\pgfqpoint{2.911141in}{3.145932in}}%
\pgfpathlineto{\pgfqpoint{2.911432in}{3.146292in}}%
\pgfpathlineto{\pgfqpoint{2.912306in}{3.153409in}}%
\pgfpathlineto{\pgfqpoint{2.913907in}{3.188822in}}%
\pgfpathlineto{\pgfqpoint{2.915362in}{3.222792in}}%
\pgfpathlineto{\pgfqpoint{2.915799in}{3.221002in}}%
\pgfpathlineto{\pgfqpoint{2.916672in}{3.216766in}}%
\pgfpathlineto{\pgfqpoint{2.917109in}{3.218497in}}%
\pgfpathlineto{\pgfqpoint{2.918564in}{3.229769in}}%
\pgfpathlineto{\pgfqpoint{2.919001in}{3.226180in}}%
\pgfpathlineto{\pgfqpoint{2.921184in}{3.170591in}}%
\pgfpathlineto{\pgfqpoint{2.922203in}{3.153409in}}%
\pgfpathlineto{\pgfqpoint{2.922639in}{3.156169in}}%
\pgfpathlineto{\pgfqpoint{2.923804in}{3.168572in}}%
\pgfpathlineto{\pgfqpoint{2.924240in}{3.164914in}}%
\pgfpathlineto{\pgfqpoint{2.925987in}{3.138238in}}%
\pgfpathlineto{\pgfqpoint{2.926569in}{3.144202in}}%
\pgfpathlineto{\pgfqpoint{2.930644in}{3.239033in}}%
\pgfpathlineto{\pgfqpoint{2.931809in}{3.231820in}}%
\pgfpathlineto{\pgfqpoint{2.934429in}{3.201685in}}%
\pgfpathlineto{\pgfqpoint{2.938213in}{3.099313in}}%
\pgfpathlineto{\pgfqpoint{2.938941in}{3.112615in}}%
\pgfpathlineto{\pgfqpoint{2.943307in}{3.250945in}}%
\pgfpathlineto{\pgfqpoint{2.944180in}{3.253111in}}%
\pgfpathlineto{\pgfqpoint{2.944471in}{3.252388in}}%
\pgfpathlineto{\pgfqpoint{2.945199in}{3.241480in}}%
\pgfpathlineto{\pgfqpoint{2.949274in}{3.100704in}}%
\pgfpathlineto{\pgfqpoint{2.950148in}{3.115124in}}%
\pgfpathlineto{\pgfqpoint{2.955533in}{3.293523in}}%
\pgfpathlineto{\pgfqpoint{2.956261in}{3.283824in}}%
\pgfpathlineto{\pgfqpoint{2.957570in}{3.200378in}}%
\pgfpathlineto{\pgfqpoint{2.959754in}{3.111579in}}%
\pgfpathlineto{\pgfqpoint{2.961209in}{3.089276in}}%
\pgfpathlineto{\pgfqpoint{2.961646in}{3.095657in}}%
\pgfpathlineto{\pgfqpoint{2.963247in}{3.185830in}}%
\pgfpathlineto{\pgfqpoint{2.965139in}{3.227696in}}%
\pgfpathlineto{\pgfqpoint{2.966885in}{3.248890in}}%
\pgfpathlineto{\pgfqpoint{2.967468in}{3.244796in}}%
\pgfpathlineto{\pgfqpoint{2.968632in}{3.207346in}}%
\pgfpathlineto{\pgfqpoint{2.970087in}{3.164675in}}%
\pgfpathlineto{\pgfqpoint{2.970524in}{3.170194in}}%
\pgfpathlineto{\pgfqpoint{2.971543in}{3.185084in}}%
\pgfpathlineto{\pgfqpoint{2.971979in}{3.180480in}}%
\pgfpathlineto{\pgfqpoint{2.973580in}{3.147181in}}%
\pgfpathlineto{\pgfqpoint{2.974163in}{3.154668in}}%
\pgfpathlineto{\pgfqpoint{2.975909in}{3.190179in}}%
\pgfpathlineto{\pgfqpoint{2.976491in}{3.190066in}}%
\pgfpathlineto{\pgfqpoint{2.977219in}{3.196268in}}%
\pgfpathlineto{\pgfqpoint{2.978820in}{3.212171in}}%
\pgfpathlineto{\pgfqpoint{2.979257in}{3.210067in}}%
\pgfpathlineto{\pgfqpoint{2.982604in}{3.167470in}}%
\pgfpathlineto{\pgfqpoint{2.983623in}{3.173427in}}%
\pgfpathlineto{\pgfqpoint{2.988572in}{3.220259in}}%
\pgfpathlineto{\pgfqpoint{2.988863in}{3.217918in}}%
\pgfpathlineto{\pgfqpoint{2.993520in}{3.133241in}}%
\pgfpathlineto{\pgfqpoint{2.994393in}{3.138822in}}%
\pgfpathlineto{\pgfqpoint{2.995703in}{3.169723in}}%
\pgfpathlineto{\pgfqpoint{2.998905in}{3.266130in}}%
\pgfpathlineto{\pgfqpoint{2.999342in}{3.264262in}}%
\pgfpathlineto{\pgfqpoint{3.000652in}{3.250941in}}%
\pgfpathlineto{\pgfqpoint{3.002107in}{3.193831in}}%
\pgfpathlineto{\pgfqpoint{3.004291in}{3.099755in}}%
\pgfpathlineto{\pgfqpoint{3.004727in}{3.102733in}}%
\pgfpathlineto{\pgfqpoint{3.007784in}{3.158240in}}%
\pgfpathlineto{\pgfqpoint{3.010549in}{3.253277in}}%
\pgfpathlineto{\pgfqpoint{3.011422in}{3.253049in}}%
\pgfpathlineto{\pgfqpoint{3.011568in}{3.252757in}}%
\pgfpathlineto{\pgfqpoint{3.012296in}{3.247810in}}%
\pgfpathlineto{\pgfqpoint{3.013751in}{3.212847in}}%
\pgfpathlineto{\pgfqpoint{3.016808in}{3.159610in}}%
\pgfpathlineto{\pgfqpoint{3.018845in}{3.122786in}}%
\pgfpathlineto{\pgfqpoint{3.019282in}{3.125640in}}%
\pgfpathlineto{\pgfqpoint{3.020737in}{3.158491in}}%
\pgfpathlineto{\pgfqpoint{3.024813in}{3.244799in}}%
\pgfpathlineto{\pgfqpoint{3.025104in}{3.246066in}}%
\pgfpathlineto{\pgfqpoint{3.025540in}{3.244365in}}%
\pgfpathlineto{\pgfqpoint{3.027141in}{3.222852in}}%
\pgfpathlineto{\pgfqpoint{3.029324in}{3.144186in}}%
\pgfpathlineto{\pgfqpoint{3.030634in}{3.119721in}}%
\pgfpathlineto{\pgfqpoint{3.031071in}{3.121874in}}%
\pgfpathlineto{\pgfqpoint{3.032672in}{3.155318in}}%
\pgfpathlineto{\pgfqpoint{3.036602in}{3.243764in}}%
\pgfpathlineto{\pgfqpoint{3.037621in}{3.235915in}}%
\pgfpathlineto{\pgfqpoint{3.039949in}{3.205070in}}%
\pgfpathlineto{\pgfqpoint{3.043151in}{3.125586in}}%
\pgfpathlineto{\pgfqpoint{3.043588in}{3.127703in}}%
\pgfpathlineto{\pgfqpoint{3.044607in}{3.152877in}}%
\pgfpathlineto{\pgfqpoint{3.047518in}{3.219791in}}%
\pgfpathlineto{\pgfqpoint{3.047663in}{3.219930in}}%
\pgfpathlineto{\pgfqpoint{3.047954in}{3.219288in}}%
\pgfpathlineto{\pgfqpoint{3.049555in}{3.206638in}}%
\pgfpathlineto{\pgfqpoint{3.052612in}{3.172616in}}%
\pgfpathlineto{\pgfqpoint{3.053485in}{3.160150in}}%
\pgfpathlineto{\pgfqpoint{3.053922in}{3.163126in}}%
\pgfpathlineto{\pgfqpoint{3.055959in}{3.194135in}}%
\pgfpathlineto{\pgfqpoint{3.056833in}{3.192092in}}%
\pgfpathlineto{\pgfqpoint{3.060762in}{3.177863in}}%
\pgfpathlineto{\pgfqpoint{3.061636in}{3.180485in}}%
\pgfpathlineto{\pgfqpoint{3.063091in}{3.186122in}}%
\pgfpathlineto{\pgfqpoint{3.063528in}{3.184834in}}%
\pgfpathlineto{\pgfqpoint{3.064401in}{3.180669in}}%
\pgfpathlineto{\pgfqpoint{3.064838in}{3.183737in}}%
\pgfpathlineto{\pgfqpoint{3.066439in}{3.210634in}}%
\pgfpathlineto{\pgfqpoint{3.067166in}{3.205875in}}%
\pgfpathlineto{\pgfqpoint{3.069495in}{3.193771in}}%
\pgfpathlineto{\pgfqpoint{3.070805in}{3.177784in}}%
\pgfpathlineto{\pgfqpoint{3.071824in}{3.168358in}}%
\pgfpathlineto{\pgfqpoint{3.072406in}{3.170099in}}%
\pgfpathlineto{\pgfqpoint{3.074298in}{3.185008in}}%
\pgfpathlineto{\pgfqpoint{3.075608in}{3.195855in}}%
\pgfpathlineto{\pgfqpoint{3.076045in}{3.195166in}}%
\pgfpathlineto{\pgfqpoint{3.077791in}{3.186308in}}%
\pgfpathlineto{\pgfqpoint{3.078373in}{3.189837in}}%
\pgfpathlineto{\pgfqpoint{3.079683in}{3.208140in}}%
\pgfpathlineto{\pgfqpoint{3.080265in}{3.203028in}}%
\pgfpathlineto{\pgfqpoint{3.082303in}{3.156386in}}%
\pgfpathlineto{\pgfqpoint{3.082885in}{3.164088in}}%
\pgfpathlineto{\pgfqpoint{3.084050in}{3.182943in}}%
\pgfpathlineto{\pgfqpoint{3.084632in}{3.178621in}}%
\pgfpathlineto{\pgfqpoint{3.085360in}{3.171489in}}%
\pgfpathlineto{\pgfqpoint{3.085796in}{3.175305in}}%
\pgfpathlineto{\pgfqpoint{3.087834in}{3.225086in}}%
\pgfpathlineto{\pgfqpoint{3.088562in}{3.215574in}}%
\pgfpathlineto{\pgfqpoint{3.090017in}{3.191578in}}%
\pgfpathlineto{\pgfqpoint{3.090599in}{3.193078in}}%
\pgfpathlineto{\pgfqpoint{3.090890in}{3.193548in}}%
\pgfpathlineto{\pgfqpoint{3.091181in}{3.192276in}}%
\pgfpathlineto{\pgfqpoint{3.092200in}{3.171456in}}%
\pgfpathlineto{\pgfqpoint{3.093510in}{3.150978in}}%
\pgfpathlineto{\pgfqpoint{3.093947in}{3.153245in}}%
\pgfpathlineto{\pgfqpoint{3.095984in}{3.183441in}}%
\pgfpathlineto{\pgfqpoint{3.098022in}{3.249559in}}%
\pgfpathlineto{\pgfqpoint{3.098750in}{3.234042in}}%
\pgfpathlineto{\pgfqpoint{3.101224in}{3.176305in}}%
\pgfpathlineto{\pgfqpoint{3.101806in}{3.172787in}}%
\pgfpathlineto{\pgfqpoint{3.103844in}{3.138315in}}%
\pgfpathlineto{\pgfqpoint{3.104572in}{3.148069in}}%
\pgfpathlineto{\pgfqpoint{3.107774in}{3.211116in}}%
\pgfpathlineto{\pgfqpoint{3.108938in}{3.218141in}}%
\pgfpathlineto{\pgfqpoint{3.109957in}{3.229883in}}%
\pgfpathlineto{\pgfqpoint{3.110393in}{3.226349in}}%
\pgfpathlineto{\pgfqpoint{3.115051in}{3.146973in}}%
\pgfpathlineto{\pgfqpoint{3.115779in}{3.141531in}}%
\pgfpathlineto{\pgfqpoint{3.116215in}{3.145599in}}%
\pgfpathlineto{\pgfqpoint{3.117816in}{3.197818in}}%
\pgfpathlineto{\pgfqpoint{3.119708in}{3.223133in}}%
\pgfpathlineto{\pgfqpoint{3.120582in}{3.226637in}}%
\pgfpathlineto{\pgfqpoint{3.121600in}{3.234658in}}%
\pgfpathlineto{\pgfqpoint{3.122037in}{3.231515in}}%
\pgfpathlineto{\pgfqpoint{3.123347in}{3.185167in}}%
\pgfpathlineto{\pgfqpoint{3.125530in}{3.142689in}}%
\pgfpathlineto{\pgfqpoint{3.126258in}{3.140414in}}%
\pgfpathlineto{\pgfqpoint{3.126695in}{3.141870in}}%
\pgfpathlineto{\pgfqpoint{3.127568in}{3.156856in}}%
\pgfpathlineto{\pgfqpoint{3.130624in}{3.213985in}}%
\pgfpathlineto{\pgfqpoint{3.131789in}{3.216941in}}%
\pgfpathlineto{\pgfqpoint{3.132225in}{3.216013in}}%
\pgfpathlineto{\pgfqpoint{3.133244in}{3.205073in}}%
\pgfpathlineto{\pgfqpoint{3.137756in}{3.148683in}}%
\pgfpathlineto{\pgfqpoint{3.137902in}{3.148960in}}%
\pgfpathlineto{\pgfqpoint{3.138775in}{3.160242in}}%
\pgfpathlineto{\pgfqpoint{3.142996in}{3.228574in}}%
\pgfpathlineto{\pgfqpoint{3.143578in}{3.223662in}}%
\pgfpathlineto{\pgfqpoint{3.146780in}{3.192082in}}%
\pgfpathlineto{\pgfqpoint{3.151001in}{3.140752in}}%
\pgfpathlineto{\pgfqpoint{3.151583in}{3.137192in}}%
\pgfpathlineto{\pgfqpoint{3.152019in}{3.139135in}}%
\pgfpathlineto{\pgfqpoint{3.153329in}{3.161817in}}%
\pgfpathlineto{\pgfqpoint{3.157987in}{3.252771in}}%
\pgfpathlineto{\pgfqpoint{3.158715in}{3.256683in}}%
\pgfpathlineto{\pgfqpoint{3.159006in}{3.254986in}}%
\pgfpathlineto{\pgfqpoint{3.159879in}{3.228782in}}%
\pgfpathlineto{\pgfqpoint{3.163226in}{3.120017in}}%
\pgfpathlineto{\pgfqpoint{3.164245in}{3.115260in}}%
\pgfpathlineto{\pgfqpoint{3.164536in}{3.116489in}}%
\pgfpathlineto{\pgfqpoint{3.165410in}{3.132103in}}%
\pgfpathlineto{\pgfqpoint{3.170649in}{3.260702in}}%
\pgfpathlineto{\pgfqpoint{3.170795in}{3.260157in}}%
\pgfpathlineto{\pgfqpoint{3.171668in}{3.243761in}}%
\pgfpathlineto{\pgfqpoint{3.176471in}{3.115481in}}%
\pgfpathlineto{\pgfqpoint{3.177199in}{3.126627in}}%
\pgfpathlineto{\pgfqpoint{3.182147in}{3.229660in}}%
\pgfpathlineto{\pgfqpoint{3.182730in}{3.233330in}}%
\pgfpathlineto{\pgfqpoint{3.183166in}{3.231252in}}%
\pgfpathlineto{\pgfqpoint{3.184185in}{3.206727in}}%
\pgfpathlineto{\pgfqpoint{3.186223in}{3.158937in}}%
\pgfpathlineto{\pgfqpoint{3.186514in}{3.159505in}}%
\pgfpathlineto{\pgfqpoint{3.187969in}{3.175419in}}%
\pgfpathlineto{\pgfqpoint{3.192336in}{3.219588in}}%
\pgfpathlineto{\pgfqpoint{3.192918in}{3.213940in}}%
\pgfpathlineto{\pgfqpoint{3.194373in}{3.196389in}}%
\pgfpathlineto{\pgfqpoint{3.194810in}{3.197369in}}%
\pgfpathlineto{\pgfqpoint{3.194955in}{3.197479in}}%
\pgfpathlineto{\pgfqpoint{3.195247in}{3.196403in}}%
\pgfpathlineto{\pgfqpoint{3.196120in}{3.179048in}}%
\pgfpathlineto{\pgfqpoint{3.197575in}{3.154100in}}%
\pgfpathlineto{\pgfqpoint{3.198012in}{3.154840in}}%
\pgfpathlineto{\pgfqpoint{3.200050in}{3.166363in}}%
\pgfpathlineto{\pgfqpoint{3.201505in}{3.197776in}}%
\pgfpathlineto{\pgfqpoint{3.203106in}{3.221783in}}%
\pgfpathlineto{\pgfqpoint{3.203397in}{3.221365in}}%
\pgfpathlineto{\pgfqpoint{3.204416in}{3.210909in}}%
\pgfpathlineto{\pgfqpoint{3.205289in}{3.204447in}}%
\pgfpathlineto{\pgfqpoint{3.205726in}{3.206034in}}%
\pgfpathlineto{\pgfqpoint{3.206308in}{3.208469in}}%
\pgfpathlineto{\pgfqpoint{3.206745in}{3.205419in}}%
\pgfpathlineto{\pgfqpoint{3.207909in}{3.167572in}}%
\pgfpathlineto{\pgfqpoint{3.209073in}{3.141597in}}%
\pgfpathlineto{\pgfqpoint{3.209510in}{3.146882in}}%
\pgfpathlineto{\pgfqpoint{3.211257in}{3.176763in}}%
\pgfpathlineto{\pgfqpoint{3.211839in}{3.175241in}}%
\pgfpathlineto{\pgfqpoint{3.212130in}{3.174661in}}%
\pgfpathlineto{\pgfqpoint{3.212566in}{3.175832in}}%
\pgfpathlineto{\pgfqpoint{3.213731in}{3.191264in}}%
\pgfpathlineto{\pgfqpoint{3.215186in}{3.205204in}}%
\pgfpathlineto{\pgfqpoint{3.215623in}{3.204646in}}%
\pgfpathlineto{\pgfqpoint{3.216060in}{3.204432in}}%
\pgfpathlineto{\pgfqpoint{3.216351in}{3.205049in}}%
\pgfpathlineto{\pgfqpoint{3.218534in}{3.215359in}}%
\pgfpathlineto{\pgfqpoint{3.219116in}{3.211809in}}%
\pgfpathlineto{\pgfqpoint{3.221154in}{3.166747in}}%
\pgfpathlineto{\pgfqpoint{3.222172in}{3.178464in}}%
\pgfpathlineto{\pgfqpoint{3.222900in}{3.185505in}}%
\pgfpathlineto{\pgfqpoint{3.223337in}{3.182198in}}%
\pgfpathlineto{\pgfqpoint{3.224792in}{3.157919in}}%
\pgfpathlineto{\pgfqpoint{3.225374in}{3.162663in}}%
\pgfpathlineto{\pgfqpoint{3.226975in}{3.184347in}}%
\pgfpathlineto{\pgfqpoint{3.227412in}{3.181475in}}%
\pgfpathlineto{\pgfqpoint{3.228431in}{3.172508in}}%
\pgfpathlineto{\pgfqpoint{3.228868in}{3.176372in}}%
\pgfpathlineto{\pgfqpoint{3.231051in}{3.210149in}}%
\pgfpathlineto{\pgfqpoint{3.231779in}{3.209446in}}%
\pgfpathlineto{\pgfqpoint{3.232506in}{3.211553in}}%
\pgfpathlineto{\pgfqpoint{3.233234in}{3.214000in}}%
\pgfpathlineto{\pgfqpoint{3.233671in}{3.212812in}}%
\pgfpathlineto{\pgfqpoint{3.234689in}{3.197928in}}%
\pgfpathlineto{\pgfqpoint{3.237164in}{3.167367in}}%
\pgfpathlineto{\pgfqpoint{3.237746in}{3.166569in}}%
\pgfpathlineto{\pgfqpoint{3.238183in}{3.167311in}}%
\pgfpathlineto{\pgfqpoint{3.238910in}{3.168720in}}%
\pgfpathlineto{\pgfqpoint{3.239347in}{3.167537in}}%
\pgfpathlineto{\pgfqpoint{3.239784in}{3.166175in}}%
\pgfpathlineto{\pgfqpoint{3.240220in}{3.168513in}}%
\pgfpathlineto{\pgfqpoint{3.241385in}{3.201570in}}%
\pgfpathlineto{\pgfqpoint{3.242403in}{3.219265in}}%
\pgfpathlineto{\pgfqpoint{3.242840in}{3.216081in}}%
\pgfpathlineto{\pgfqpoint{3.244732in}{3.187868in}}%
\pgfpathlineto{\pgfqpoint{3.245460in}{3.191334in}}%
\pgfpathlineto{\pgfqpoint{3.245751in}{3.191988in}}%
\pgfpathlineto{\pgfqpoint{3.246188in}{3.189907in}}%
\pgfpathlineto{\pgfqpoint{3.248371in}{3.157439in}}%
\pgfpathlineto{\pgfqpoint{3.249244in}{3.166824in}}%
\pgfpathlineto{\pgfqpoint{3.250845in}{3.187932in}}%
\pgfpathlineto{\pgfqpoint{3.251282in}{3.186772in}}%
\pgfpathlineto{\pgfqpoint{3.251718in}{3.186085in}}%
\pgfpathlineto{\pgfqpoint{3.252009in}{3.187245in}}%
\pgfpathlineto{\pgfqpoint{3.252883in}{3.202455in}}%
\pgfpathlineto{\pgfqpoint{3.254047in}{3.225113in}}%
\pgfpathlineto{\pgfqpoint{3.254484in}{3.220202in}}%
\pgfpathlineto{\pgfqpoint{3.256085in}{3.187309in}}%
\pgfpathlineto{\pgfqpoint{3.256667in}{3.190694in}}%
\pgfpathlineto{\pgfqpoint{3.257249in}{3.193139in}}%
\pgfpathlineto{\pgfqpoint{3.257686in}{3.190151in}}%
\pgfpathlineto{\pgfqpoint{3.259432in}{3.161262in}}%
\pgfpathlineto{\pgfqpoint{3.260160in}{3.167530in}}%
\pgfpathlineto{\pgfqpoint{3.261033in}{3.173261in}}%
\pgfpathlineto{\pgfqpoint{3.261470in}{3.171910in}}%
\pgfpathlineto{\pgfqpoint{3.262052in}{3.170478in}}%
\pgfpathlineto{\pgfqpoint{3.262343in}{3.171458in}}%
\pgfpathlineto{\pgfqpoint{3.263362in}{3.185316in}}%
\pgfpathlineto{\pgfqpoint{3.265545in}{3.220389in}}%
\pgfpathlineto{\pgfqpoint{3.266127in}{3.217938in}}%
\pgfpathlineto{\pgfqpoint{3.270494in}{3.163097in}}%
\pgfpathlineto{\pgfqpoint{3.271512in}{3.155644in}}%
\pgfpathlineto{\pgfqpoint{3.271949in}{3.156320in}}%
\pgfpathlineto{\pgfqpoint{3.273405in}{3.161761in}}%
\pgfpathlineto{\pgfqpoint{3.275151in}{3.189152in}}%
\pgfpathlineto{\pgfqpoint{3.277625in}{3.227971in}}%
\pgfpathlineto{\pgfqpoint{3.278208in}{3.223310in}}%
\pgfpathlineto{\pgfqpoint{3.283156in}{3.168253in}}%
\pgfpathlineto{\pgfqpoint{3.284029in}{3.170073in}}%
\pgfpathlineto{\pgfqpoint{3.284321in}{3.168947in}}%
\pgfpathlineto{\pgfqpoint{3.285339in}{3.150374in}}%
\pgfpathlineto{\pgfqpoint{3.286067in}{3.140600in}}%
\pgfpathlineto{\pgfqpoint{3.286504in}{3.145501in}}%
\pgfpathlineto{\pgfqpoint{3.290433in}{3.216571in}}%
\pgfpathlineto{\pgfqpoint{3.291889in}{3.225482in}}%
\pgfpathlineto{\pgfqpoint{3.292326in}{3.223165in}}%
\pgfpathlineto{\pgfqpoint{3.294509in}{3.189166in}}%
\pgfpathlineto{\pgfqpoint{3.297420in}{3.153728in}}%
\pgfpathlineto{\pgfqpoint{3.298730in}{3.147020in}}%
\pgfpathlineto{\pgfqpoint{3.299166in}{3.147987in}}%
\pgfpathlineto{\pgfqpoint{3.300185in}{3.160323in}}%
\pgfpathlineto{\pgfqpoint{3.304406in}{3.236904in}}%
\pgfpathlineto{\pgfqpoint{3.305134in}{3.233097in}}%
\pgfpathlineto{\pgfqpoint{3.306443in}{3.203549in}}%
\pgfpathlineto{\pgfqpoint{3.308918in}{3.160568in}}%
\pgfpathlineto{\pgfqpoint{3.309354in}{3.159813in}}%
\pgfpathlineto{\pgfqpoint{3.309937in}{3.160667in}}%
\pgfpathlineto{\pgfqpoint{3.310519in}{3.161219in}}%
\pgfpathlineto{\pgfqpoint{3.310955in}{3.160493in}}%
\pgfpathlineto{\pgfqpoint{3.311683in}{3.159483in}}%
\pgfpathlineto{\pgfqpoint{3.311974in}{3.160205in}}%
\pgfpathlineto{\pgfqpoint{3.312993in}{3.169865in}}%
\pgfpathlineto{\pgfqpoint{3.317068in}{3.215202in}}%
\pgfpathlineto{\pgfqpoint{3.317650in}{3.212652in}}%
\pgfpathlineto{\pgfqpoint{3.321871in}{3.166644in}}%
\pgfpathlineto{\pgfqpoint{3.322745in}{3.156517in}}%
\pgfpathlineto{\pgfqpoint{3.323327in}{3.159827in}}%
\pgfpathlineto{\pgfqpoint{3.325364in}{3.175149in}}%
\pgfpathlineto{\pgfqpoint{3.325655in}{3.175124in}}%
\pgfpathlineto{\pgfqpoint{3.326238in}{3.176274in}}%
\pgfpathlineto{\pgfqpoint{3.327111in}{3.185749in}}%
\pgfpathlineto{\pgfqpoint{3.328712in}{3.210361in}}%
\pgfpathlineto{\pgfqpoint{3.329294in}{3.206299in}}%
\pgfpathlineto{\pgfqpoint{3.331477in}{3.188142in}}%
\pgfpathlineto{\pgfqpoint{3.331768in}{3.188600in}}%
\pgfpathlineto{\pgfqpoint{3.332787in}{3.190776in}}%
\pgfpathlineto{\pgfqpoint{3.333078in}{3.190025in}}%
\pgfpathlineto{\pgfqpoint{3.334679in}{3.182374in}}%
\pgfpathlineto{\pgfqpoint{3.335261in}{3.184657in}}%
\pgfpathlineto{\pgfqpoint{3.337008in}{3.196945in}}%
\pgfpathlineto{\pgfqpoint{3.337590in}{3.195476in}}%
\pgfpathlineto{\pgfqpoint{3.339337in}{3.184827in}}%
\pgfpathlineto{\pgfqpoint{3.341374in}{3.173647in}}%
\pgfpathlineto{\pgfqpoint{3.342248in}{3.171995in}}%
\pgfpathlineto{\pgfqpoint{3.342830in}{3.173024in}}%
\pgfpathlineto{\pgfqpoint{3.344576in}{3.180993in}}%
\pgfpathlineto{\pgfqpoint{3.347778in}{3.213982in}}%
\pgfpathlineto{\pgfqpoint{3.348361in}{3.211931in}}%
\pgfpathlineto{\pgfqpoint{3.350253in}{3.195463in}}%
\pgfpathlineto{\pgfqpoint{3.352581in}{3.165083in}}%
\pgfpathlineto{\pgfqpoint{3.353164in}{3.167074in}}%
\pgfpathlineto{\pgfqpoint{3.354037in}{3.170314in}}%
\pgfpathlineto{\pgfqpoint{3.354474in}{3.169190in}}%
\pgfpathlineto{\pgfqpoint{3.355929in}{3.158847in}}%
\pgfpathlineto{\pgfqpoint{3.356366in}{3.162065in}}%
\pgfpathlineto{\pgfqpoint{3.360150in}{3.228023in}}%
\pgfpathlineto{\pgfqpoint{3.361169in}{3.220637in}}%
\pgfpathlineto{\pgfqpoint{3.364225in}{3.174902in}}%
\pgfpathlineto{\pgfqpoint{3.365681in}{3.152498in}}%
\pgfpathlineto{\pgfqpoint{3.365972in}{3.152956in}}%
\pgfpathlineto{\pgfqpoint{3.367282in}{3.156636in}}%
\pgfpathlineto{\pgfqpoint{3.367864in}{3.155988in}}%
\pgfpathlineto{\pgfqpoint{3.368300in}{3.156607in}}%
\pgfpathlineto{\pgfqpoint{3.369174in}{3.167651in}}%
\pgfpathlineto{\pgfqpoint{3.372230in}{3.210517in}}%
\pgfpathlineto{\pgfqpoint{3.373831in}{3.214876in}}%
\pgfpathlineto{\pgfqpoint{3.374559in}{3.215731in}}%
\pgfpathlineto{\pgfqpoint{3.374850in}{3.215192in}}%
\pgfpathlineto{\pgfqpoint{3.375578in}{3.209371in}}%
\pgfpathlineto{\pgfqpoint{3.377906in}{3.165566in}}%
\pgfpathlineto{\pgfqpoint{3.378925in}{3.172541in}}%
\pgfpathlineto{\pgfqpoint{3.379507in}{3.174965in}}%
\pgfpathlineto{\pgfqpoint{3.379944in}{3.171928in}}%
\pgfpathlineto{\pgfqpoint{3.381108in}{3.158557in}}%
\pgfpathlineto{\pgfqpoint{3.381691in}{3.162767in}}%
\pgfpathlineto{\pgfqpoint{3.384165in}{3.201892in}}%
\pgfpathlineto{\pgfqpoint{3.385038in}{3.201534in}}%
\pgfpathlineto{\pgfqpoint{3.386202in}{3.204826in}}%
\pgfpathlineto{\pgfqpoint{3.386785in}{3.202716in}}%
\pgfpathlineto{\pgfqpoint{3.391151in}{3.168473in}}%
\pgfpathlineto{\pgfqpoint{3.391733in}{3.164856in}}%
\pgfpathlineto{\pgfqpoint{3.392170in}{3.167014in}}%
\pgfpathlineto{\pgfqpoint{3.396827in}{3.206381in}}%
\pgfpathlineto{\pgfqpoint{3.397118in}{3.205908in}}%
\pgfpathlineto{\pgfqpoint{3.397992in}{3.195062in}}%
\pgfpathlineto{\pgfqpoint{3.400175in}{3.175512in}}%
\pgfpathlineto{\pgfqpoint{3.401485in}{3.174795in}}%
\pgfpathlineto{\pgfqpoint{3.402358in}{3.173330in}}%
\pgfpathlineto{\pgfqpoint{3.402795in}{3.174532in}}%
\pgfpathlineto{\pgfqpoint{3.408908in}{3.214563in}}%
\pgfpathlineto{\pgfqpoint{3.409635in}{3.211272in}}%
\pgfpathlineto{\pgfqpoint{3.410800in}{3.188048in}}%
\pgfpathlineto{\pgfqpoint{3.412110in}{3.161956in}}%
\pgfpathlineto{\pgfqpoint{3.412692in}{3.165072in}}%
\pgfpathlineto{\pgfqpoint{3.413565in}{3.171191in}}%
\pgfpathlineto{\pgfqpoint{3.414147in}{3.168749in}}%
\pgfpathlineto{\pgfqpoint{3.415021in}{3.164103in}}%
\pgfpathlineto{\pgfqpoint{3.415457in}{3.166422in}}%
\pgfpathlineto{\pgfqpoint{3.420406in}{3.217141in}}%
\pgfpathlineto{\pgfqpoint{3.420842in}{3.214132in}}%
\pgfpathlineto{\pgfqpoint{3.425500in}{3.167432in}}%
\pgfpathlineto{\pgfqpoint{3.425791in}{3.168146in}}%
\pgfpathlineto{\pgfqpoint{3.426955in}{3.173564in}}%
\pgfpathlineto{\pgfqpoint{3.427537in}{3.172023in}}%
\pgfpathlineto{\pgfqpoint{3.428120in}{3.170235in}}%
\pgfpathlineto{\pgfqpoint{3.428556in}{3.171867in}}%
\pgfpathlineto{\pgfqpoint{3.430012in}{3.195809in}}%
\pgfpathlineto{\pgfqpoint{3.430885in}{3.203933in}}%
\pgfpathlineto{\pgfqpoint{3.431322in}{3.202315in}}%
\pgfpathlineto{\pgfqpoint{3.432486in}{3.193938in}}%
\pgfpathlineto{\pgfqpoint{3.432923in}{3.196295in}}%
\pgfpathlineto{\pgfqpoint{3.434378in}{3.213762in}}%
\pgfpathlineto{\pgfqpoint{3.434960in}{3.210114in}}%
\pgfpathlineto{\pgfqpoint{3.438744in}{3.160885in}}%
\pgfpathlineto{\pgfqpoint{3.439181in}{3.161979in}}%
\pgfpathlineto{\pgfqpoint{3.442383in}{3.182745in}}%
\pgfpathlineto{\pgfqpoint{3.445585in}{3.222732in}}%
\pgfpathlineto{\pgfqpoint{3.446458in}{3.217237in}}%
\pgfpathlineto{\pgfqpoint{3.451116in}{3.168733in}}%
\pgfpathlineto{\pgfqpoint{3.452862in}{3.152647in}}%
\pgfpathlineto{\pgfqpoint{3.453445in}{3.157926in}}%
\pgfpathlineto{\pgfqpoint{3.456792in}{3.221827in}}%
\pgfpathlineto{\pgfqpoint{3.457520in}{3.218953in}}%
\pgfpathlineto{\pgfqpoint{3.459266in}{3.200633in}}%
\pgfpathlineto{\pgfqpoint{3.462905in}{3.161397in}}%
\pgfpathlineto{\pgfqpoint{3.463924in}{3.158050in}}%
\pgfpathlineto{\pgfqpoint{3.464361in}{3.159740in}}%
\pgfpathlineto{\pgfqpoint{3.465525in}{3.179646in}}%
\pgfpathlineto{\pgfqpoint{3.467126in}{3.202011in}}%
\pgfpathlineto{\pgfqpoint{3.467417in}{3.201043in}}%
\pgfpathlineto{\pgfqpoint{3.468436in}{3.196336in}}%
\pgfpathlineto{\pgfqpoint{3.468872in}{3.197734in}}%
\pgfpathlineto{\pgfqpoint{3.470328in}{3.207048in}}%
\pgfpathlineto{\pgfqpoint{3.470910in}{3.205148in}}%
\pgfpathlineto{\pgfqpoint{3.477023in}{3.168458in}}%
\pgfpathlineto{\pgfqpoint{3.477314in}{3.169981in}}%
\pgfpathlineto{\pgfqpoint{3.479206in}{3.193680in}}%
\pgfpathlineto{\pgfqpoint{3.480079in}{3.187811in}}%
\pgfpathlineto{\pgfqpoint{3.480662in}{3.185066in}}%
\pgfpathlineto{\pgfqpoint{3.481098in}{3.187137in}}%
\pgfpathlineto{\pgfqpoint{3.482408in}{3.199694in}}%
\pgfpathlineto{\pgfqpoint{3.482845in}{3.196450in}}%
\pgfpathlineto{\pgfqpoint{3.484446in}{3.170837in}}%
\pgfpathlineto{\pgfqpoint{3.485174in}{3.177000in}}%
\pgfpathlineto{\pgfqpoint{3.487793in}{3.204510in}}%
\pgfpathlineto{\pgfqpoint{3.488958in}{3.210254in}}%
\pgfpathlineto{\pgfqpoint{3.489394in}{3.208984in}}%
\pgfpathlineto{\pgfqpoint{3.490559in}{3.196186in}}%
\pgfpathlineto{\pgfqpoint{3.493615in}{3.157398in}}%
\pgfpathlineto{\pgfqpoint{3.493906in}{3.158077in}}%
\pgfpathlineto{\pgfqpoint{3.495944in}{3.177478in}}%
\pgfpathlineto{\pgfqpoint{3.499146in}{3.202907in}}%
\pgfpathlineto{\pgfqpoint{3.499291in}{3.202755in}}%
\pgfpathlineto{\pgfqpoint{3.502493in}{3.194206in}}%
\pgfpathlineto{\pgfqpoint{3.504531in}{3.172861in}}%
\pgfpathlineto{\pgfqpoint{3.505113in}{3.175905in}}%
\pgfpathlineto{\pgfqpoint{3.507588in}{3.191799in}}%
\pgfpathlineto{\pgfqpoint{3.508752in}{3.194702in}}%
\pgfpathlineto{\pgfqpoint{3.509625in}{3.196180in}}%
\pgfpathlineto{\pgfqpoint{3.509916in}{3.195603in}}%
\pgfpathlineto{\pgfqpoint{3.510790in}{3.188653in}}%
\pgfpathlineto{\pgfqpoint{3.512245in}{3.172473in}}%
\pgfpathlineto{\pgfqpoint{3.512827in}{3.175936in}}%
\pgfpathlineto{\pgfqpoint{3.513992in}{3.188300in}}%
\pgfpathlineto{\pgfqpoint{3.514574in}{3.184929in}}%
\pgfpathlineto{\pgfqpoint{3.516029in}{3.167732in}}%
\pgfpathlineto{\pgfqpoint{3.516611in}{3.172246in}}%
\pgfpathlineto{\pgfqpoint{3.518795in}{3.196474in}}%
\pgfpathlineto{\pgfqpoint{3.519231in}{3.196164in}}%
\pgfpathlineto{\pgfqpoint{3.520250in}{3.195670in}}%
\pgfpathlineto{\pgfqpoint{3.520541in}{3.196276in}}%
\pgfpathlineto{\pgfqpoint{3.521560in}{3.203807in}}%
\pgfpathlineto{\pgfqpoint{3.522724in}{3.211923in}}%
\pgfpathlineto{\pgfqpoint{3.523161in}{3.209558in}}%
\pgfpathlineto{\pgfqpoint{3.527964in}{3.168375in}}%
\pgfpathlineto{\pgfqpoint{3.528837in}{3.163590in}}%
\pgfpathlineto{\pgfqpoint{3.529274in}{3.165551in}}%
\pgfpathlineto{\pgfqpoint{3.530875in}{3.191331in}}%
\pgfpathlineto{\pgfqpoint{3.531894in}{3.199833in}}%
\pgfpathlineto{\pgfqpoint{3.532330in}{3.198170in}}%
\pgfpathlineto{\pgfqpoint{3.533495in}{3.191509in}}%
\pgfpathlineto{\pgfqpoint{3.534077in}{3.192956in}}%
\pgfpathlineto{\pgfqpoint{3.534950in}{3.196058in}}%
\pgfpathlineto{\pgfqpoint{3.535387in}{3.194671in}}%
\pgfpathlineto{\pgfqpoint{3.537133in}{3.179876in}}%
\pgfpathlineto{\pgfqpoint{3.537716in}{3.184214in}}%
\pgfpathlineto{\pgfqpoint{3.539317in}{3.203911in}}%
\pgfpathlineto{\pgfqpoint{3.539899in}{3.201305in}}%
\pgfpathlineto{\pgfqpoint{3.542227in}{3.189829in}}%
\pgfpathlineto{\pgfqpoint{3.543392in}{3.181851in}}%
\pgfpathlineto{\pgfqpoint{3.546012in}{3.167418in}}%
\pgfpathlineto{\pgfqpoint{3.546448in}{3.166603in}}%
\pgfpathlineto{\pgfqpoint{3.546885in}{3.167517in}}%
\pgfpathlineto{\pgfqpoint{3.547904in}{3.179082in}}%
\pgfpathlineto{\pgfqpoint{3.550815in}{3.209048in}}%
\pgfpathlineto{\pgfqpoint{3.551688in}{3.212061in}}%
\pgfpathlineto{\pgfqpoint{3.552125in}{3.211398in}}%
\pgfpathlineto{\pgfqpoint{3.553143in}{3.204744in}}%
\pgfpathlineto{\pgfqpoint{3.555763in}{3.169675in}}%
\pgfpathlineto{\pgfqpoint{3.556636in}{3.172880in}}%
\pgfpathlineto{\pgfqpoint{3.557364in}{3.175564in}}%
\pgfpathlineto{\pgfqpoint{3.557801in}{3.174311in}}%
\pgfpathlineto{\pgfqpoint{3.558965in}{3.167200in}}%
\pgfpathlineto{\pgfqpoint{3.559547in}{3.170352in}}%
\pgfpathlineto{\pgfqpoint{3.564059in}{3.218999in}}%
\pgfpathlineto{\pgfqpoint{3.564496in}{3.215884in}}%
\pgfpathlineto{\pgfqpoint{3.567844in}{3.181771in}}%
\pgfpathlineto{\pgfqpoint{3.569590in}{3.171668in}}%
\pgfpathlineto{\pgfqpoint{3.570609in}{3.174247in}}%
\pgfpathlineto{\pgfqpoint{3.571046in}{3.173447in}}%
\pgfpathlineto{\pgfqpoint{3.571919in}{3.169490in}}%
\pgfpathlineto{\pgfqpoint{3.572355in}{3.171198in}}%
\pgfpathlineto{\pgfqpoint{3.574539in}{3.199582in}}%
\pgfpathlineto{\pgfqpoint{3.575849in}{3.197038in}}%
\pgfpathlineto{\pgfqpoint{3.576431in}{3.196683in}}%
\pgfpathlineto{\pgfqpoint{3.576867in}{3.197410in}}%
\pgfpathlineto{\pgfqpoint{3.578468in}{3.202857in}}%
\pgfpathlineto{\pgfqpoint{3.579051in}{3.200894in}}%
\pgfpathlineto{\pgfqpoint{3.585309in}{3.168145in}}%
\pgfpathlineto{\pgfqpoint{3.585600in}{3.168582in}}%
\pgfpathlineto{\pgfqpoint{3.586473in}{3.175823in}}%
\pgfpathlineto{\pgfqpoint{3.589093in}{3.195090in}}%
\pgfpathlineto{\pgfqpoint{3.591713in}{3.206886in}}%
\pgfpathlineto{\pgfqpoint{3.592441in}{3.205685in}}%
\pgfpathlineto{\pgfqpoint{3.593605in}{3.198333in}}%
\pgfpathlineto{\pgfqpoint{3.596807in}{3.161839in}}%
\pgfpathlineto{\pgfqpoint{3.597535in}{3.164346in}}%
\pgfpathlineto{\pgfqpoint{3.602629in}{3.208718in}}%
\pgfpathlineto{\pgfqpoint{3.603502in}{3.205272in}}%
\pgfpathlineto{\pgfqpoint{3.605831in}{3.183586in}}%
\pgfpathlineto{\pgfqpoint{3.607141in}{3.163162in}}%
\pgfpathlineto{\pgfqpoint{3.607723in}{3.166507in}}%
\pgfpathlineto{\pgfqpoint{3.608887in}{3.176992in}}%
\pgfpathlineto{\pgfqpoint{3.609470in}{3.174379in}}%
\pgfpathlineto{\pgfqpoint{3.610052in}{3.171388in}}%
\pgfpathlineto{\pgfqpoint{3.610488in}{3.172999in}}%
\pgfpathlineto{\pgfqpoint{3.612817in}{3.204439in}}%
\pgfpathlineto{\pgfqpoint{3.614127in}{3.200313in}}%
\pgfpathlineto{\pgfqpoint{3.614709in}{3.201805in}}%
\pgfpathlineto{\pgfqpoint{3.615583in}{3.205601in}}%
\pgfpathlineto{\pgfqpoint{3.616019in}{3.204285in}}%
\pgfpathlineto{\pgfqpoint{3.617038in}{3.188100in}}%
\pgfpathlineto{\pgfqpoint{3.618639in}{3.163786in}}%
\pgfpathlineto{\pgfqpoint{3.619076in}{3.166269in}}%
\pgfpathlineto{\pgfqpoint{3.620822in}{3.182673in}}%
\pgfpathlineto{\pgfqpoint{3.621404in}{3.181112in}}%
\pgfpathlineto{\pgfqpoint{3.621695in}{3.180694in}}%
\pgfpathlineto{\pgfqpoint{3.621987in}{3.181399in}}%
\pgfpathlineto{\pgfqpoint{3.623151in}{3.194654in}}%
\pgfpathlineto{\pgfqpoint{3.624024in}{3.199755in}}%
\pgfpathlineto{\pgfqpoint{3.624461in}{3.198535in}}%
\pgfpathlineto{\pgfqpoint{3.626207in}{3.192268in}}%
\pgfpathlineto{\pgfqpoint{3.626790in}{3.193072in}}%
\pgfpathlineto{\pgfqpoint{3.627517in}{3.194192in}}%
\pgfpathlineto{\pgfqpoint{3.627954in}{3.193104in}}%
\pgfpathlineto{\pgfqpoint{3.632611in}{3.174103in}}%
\pgfpathlineto{\pgfqpoint{3.633194in}{3.177669in}}%
\pgfpathlineto{\pgfqpoint{3.635231in}{3.195066in}}%
\pgfpathlineto{\pgfqpoint{3.635668in}{3.194429in}}%
\pgfpathlineto{\pgfqpoint{3.637123in}{3.190088in}}%
\pgfpathlineto{\pgfqpoint{3.637705in}{3.191623in}}%
\pgfpathlineto{\pgfqpoint{3.639015in}{3.197126in}}%
\pgfpathlineto{\pgfqpoint{3.639598in}{3.196150in}}%
\pgfpathlineto{\pgfqpoint{3.640762in}{3.188039in}}%
\pgfpathlineto{\pgfqpoint{3.642072in}{3.176064in}}%
\pgfpathlineto{\pgfqpoint{3.642654in}{3.178474in}}%
\pgfpathlineto{\pgfqpoint{3.645128in}{3.188419in}}%
\pgfpathlineto{\pgfqpoint{3.645856in}{3.187723in}}%
\pgfpathlineto{\pgfqpoint{3.646293in}{3.187587in}}%
\pgfpathlineto{\pgfqpoint{3.646584in}{3.188405in}}%
\pgfpathlineto{\pgfqpoint{3.647603in}{3.192579in}}%
\pgfpathlineto{\pgfqpoint{3.648039in}{3.191045in}}%
\pgfpathlineto{\pgfqpoint{3.649931in}{3.177990in}}%
\pgfpathlineto{\pgfqpoint{3.650513in}{3.180659in}}%
\pgfpathlineto{\pgfqpoint{3.652114in}{3.195625in}}%
\pgfpathlineto{\pgfqpoint{3.652842in}{3.192469in}}%
\pgfpathlineto{\pgfqpoint{3.654298in}{3.182622in}}%
\pgfpathlineto{\pgfqpoint{3.654880in}{3.184515in}}%
\pgfpathlineto{\pgfqpoint{3.656190in}{3.191388in}}%
\pgfpathlineto{\pgfqpoint{3.656772in}{3.190217in}}%
\pgfpathlineto{\pgfqpoint{3.658227in}{3.179283in}}%
\pgfpathlineto{\pgfqpoint{3.659101in}{3.173491in}}%
\pgfpathlineto{\pgfqpoint{3.659537in}{3.174789in}}%
\pgfpathlineto{\pgfqpoint{3.660702in}{3.192157in}}%
\pgfpathlineto{\pgfqpoint{3.662012in}{3.206971in}}%
\pgfpathlineto{\pgfqpoint{3.662448in}{3.205354in}}%
\pgfpathlineto{\pgfqpoint{3.664340in}{3.193605in}}%
\pgfpathlineto{\pgfqpoint{3.664777in}{3.194996in}}%
\pgfpathlineto{\pgfqpoint{3.665505in}{3.197990in}}%
\pgfpathlineto{\pgfqpoint{3.665941in}{3.196312in}}%
\pgfpathlineto{\pgfqpoint{3.668416in}{3.170416in}}%
\pgfpathlineto{\pgfqpoint{3.669580in}{3.171247in}}%
\pgfpathlineto{\pgfqpoint{3.670890in}{3.169927in}}%
\pgfpathlineto{\pgfqpoint{3.671181in}{3.171013in}}%
\pgfpathlineto{\pgfqpoint{3.672345in}{3.184783in}}%
\pgfpathlineto{\pgfqpoint{3.673946in}{3.197587in}}%
\pgfpathlineto{\pgfqpoint{3.674237in}{3.197367in}}%
\pgfpathlineto{\pgfqpoint{3.674820in}{3.197143in}}%
\pgfpathlineto{\pgfqpoint{3.675111in}{3.197665in}}%
\pgfpathlineto{\pgfqpoint{3.677003in}{3.202775in}}%
\pgfpathlineto{\pgfqpoint{3.677439in}{3.201838in}}%
\pgfpathlineto{\pgfqpoint{3.678604in}{3.191971in}}%
\pgfpathlineto{\pgfqpoint{3.680205in}{3.176846in}}%
\pgfpathlineto{\pgfqpoint{3.680641in}{3.177927in}}%
\pgfpathlineto{\pgfqpoint{3.682242in}{3.189668in}}%
\pgfpathlineto{\pgfqpoint{3.682825in}{3.186764in}}%
\pgfpathlineto{\pgfqpoint{3.684135in}{3.177850in}}%
\pgfpathlineto{\pgfqpoint{3.684571in}{3.179802in}}%
\pgfpathlineto{\pgfqpoint{3.685736in}{3.187783in}}%
\pgfpathlineto{\pgfqpoint{3.686318in}{3.185142in}}%
\pgfpathlineto{\pgfqpoint{3.687337in}{3.179398in}}%
\pgfpathlineto{\pgfqpoint{3.687773in}{3.181102in}}%
\pgfpathlineto{\pgfqpoint{3.689374in}{3.192749in}}%
\pgfpathlineto{\pgfqpoint{3.690102in}{3.190672in}}%
\pgfpathlineto{\pgfqpoint{3.690684in}{3.189150in}}%
\pgfpathlineto{\pgfqpoint{3.691121in}{3.191045in}}%
\pgfpathlineto{\pgfqpoint{3.692722in}{3.206066in}}%
\pgfpathlineto{\pgfqpoint{3.693304in}{3.202109in}}%
\pgfpathlineto{\pgfqpoint{3.695924in}{3.182975in}}%
\pgfpathlineto{\pgfqpoint{3.696651in}{3.181235in}}%
\pgfpathlineto{\pgfqpoint{3.698835in}{3.167455in}}%
\pgfpathlineto{\pgfqpoint{3.699562in}{3.171215in}}%
\pgfpathlineto{\pgfqpoint{3.705239in}{3.210789in}}%
\pgfpathlineto{\pgfqpoint{3.705821in}{3.210099in}}%
\pgfpathlineto{\pgfqpoint{3.706985in}{3.201768in}}%
\pgfpathlineto{\pgfqpoint{3.710915in}{3.161174in}}%
\pgfpathlineto{\pgfqpoint{3.711497in}{3.163027in}}%
\pgfpathlineto{\pgfqpoint{3.713535in}{3.183499in}}%
\pgfpathlineto{\pgfqpoint{3.716009in}{3.203095in}}%
\pgfpathlineto{\pgfqpoint{3.716155in}{3.202989in}}%
\pgfpathlineto{\pgfqpoint{3.717028in}{3.198967in}}%
\pgfpathlineto{\pgfqpoint{3.719065in}{3.189492in}}%
\pgfpathlineto{\pgfqpoint{3.719357in}{3.189897in}}%
\pgfpathlineto{\pgfqpoint{3.720084in}{3.191191in}}%
\pgfpathlineto{\pgfqpoint{3.720521in}{3.190327in}}%
\pgfpathlineto{\pgfqpoint{3.724305in}{3.178163in}}%
\pgfpathlineto{\pgfqpoint{3.724451in}{3.178280in}}%
\pgfpathlineto{\pgfqpoint{3.725324in}{3.181828in}}%
\pgfpathlineto{\pgfqpoint{3.727362in}{3.195189in}}%
\pgfpathlineto{\pgfqpoint{3.728089in}{3.193491in}}%
\pgfpathlineto{\pgfqpoint{3.730564in}{3.180239in}}%
\pgfpathlineto{\pgfqpoint{3.731437in}{3.182408in}}%
\pgfpathlineto{\pgfqpoint{3.732747in}{3.184968in}}%
\pgfpathlineto{\pgfqpoint{3.733183in}{3.184721in}}%
\pgfpathlineto{\pgfqpoint{3.733911in}{3.185499in}}%
\pgfpathlineto{\pgfqpoint{3.736385in}{3.199358in}}%
\pgfpathlineto{\pgfqpoint{3.737259in}{3.201930in}}%
\pgfpathlineto{\pgfqpoint{3.737695in}{3.200684in}}%
\pgfpathlineto{\pgfqpoint{3.742935in}{3.175432in}}%
\pgfpathlineto{\pgfqpoint{3.743954in}{3.176293in}}%
\pgfpathlineto{\pgfqpoint{3.746719in}{3.183439in}}%
\pgfpathlineto{\pgfqpoint{3.748611in}{3.202392in}}%
\pgfpathlineto{\pgfqpoint{3.749193in}{3.200622in}}%
\pgfpathlineto{\pgfqpoint{3.750212in}{3.197206in}}%
\pgfpathlineto{\pgfqpoint{3.750794in}{3.198060in}}%
\pgfpathlineto{\pgfqpoint{3.751231in}{3.198381in}}%
\pgfpathlineto{\pgfqpoint{3.751522in}{3.197697in}}%
\pgfpathlineto{\pgfqpoint{3.755597in}{3.174892in}}%
\pgfpathlineto{\pgfqpoint{3.756180in}{3.171791in}}%
\pgfpathlineto{\pgfqpoint{3.756616in}{3.173505in}}%
\pgfpathlineto{\pgfqpoint{3.758363in}{3.192637in}}%
\pgfpathlineto{\pgfqpoint{3.759236in}{3.188059in}}%
\pgfpathlineto{\pgfqpoint{3.761565in}{3.176980in}}%
\pgfpathlineto{\pgfqpoint{3.762147in}{3.178591in}}%
\pgfpathlineto{\pgfqpoint{3.764330in}{3.190449in}}%
\pgfpathlineto{\pgfqpoint{3.765058in}{3.189237in}}%
\pgfpathlineto{\pgfqpoint{3.765640in}{3.188387in}}%
\pgfpathlineto{\pgfqpoint{3.765931in}{3.189057in}}%
\pgfpathlineto{\pgfqpoint{3.767096in}{3.199649in}}%
\pgfpathlineto{\pgfqpoint{3.768697in}{3.209188in}}%
\pgfpathlineto{\pgfqpoint{3.768988in}{3.208648in}}%
\pgfpathlineto{\pgfqpoint{3.770006in}{3.200776in}}%
\pgfpathlineto{\pgfqpoint{3.773208in}{3.166652in}}%
\pgfpathlineto{\pgfqpoint{3.773791in}{3.167996in}}%
\pgfpathlineto{\pgfqpoint{3.775101in}{3.172497in}}%
\pgfpathlineto{\pgfqpoint{3.775828in}{3.172104in}}%
\pgfpathlineto{\pgfqpoint{3.776410in}{3.173912in}}%
\pgfpathlineto{\pgfqpoint{3.778157in}{3.193676in}}%
\pgfpathlineto{\pgfqpoint{3.779467in}{3.201143in}}%
\pgfpathlineto{\pgfqpoint{3.779904in}{3.200780in}}%
\pgfpathlineto{\pgfqpoint{3.780486in}{3.200510in}}%
\pgfpathlineto{\pgfqpoint{3.780777in}{3.201042in}}%
\pgfpathlineto{\pgfqpoint{3.781796in}{3.203772in}}%
\pgfpathlineto{\pgfqpoint{3.782232in}{3.202668in}}%
\pgfpathlineto{\pgfqpoint{3.787763in}{3.173211in}}%
\pgfpathlineto{\pgfqpoint{3.788200in}{3.174340in}}%
\pgfpathlineto{\pgfqpoint{3.792421in}{3.196086in}}%
\pgfpathlineto{\pgfqpoint{3.792857in}{3.194583in}}%
\pgfpathlineto{\pgfqpoint{3.794749in}{3.180238in}}%
\pgfpathlineto{\pgfqpoint{3.795477in}{3.182343in}}%
\pgfpathlineto{\pgfqpoint{3.799116in}{3.196205in}}%
\pgfpathlineto{\pgfqpoint{3.799407in}{3.195786in}}%
\pgfpathlineto{\pgfqpoint{3.804064in}{3.177395in}}%
\pgfpathlineto{\pgfqpoint{3.804646in}{3.178948in}}%
\pgfpathlineto{\pgfqpoint{3.809013in}{3.198285in}}%
\pgfpathlineto{\pgfqpoint{3.809304in}{3.197822in}}%
\pgfpathlineto{\pgfqpoint{3.810468in}{3.189405in}}%
\pgfpathlineto{\pgfqpoint{3.812069in}{3.180278in}}%
\pgfpathlineto{\pgfqpoint{3.812506in}{3.181014in}}%
\pgfpathlineto{\pgfqpoint{3.814689in}{3.191884in}}%
\pgfpathlineto{\pgfqpoint{3.815562in}{3.188810in}}%
\pgfpathlineto{\pgfqpoint{3.818037in}{3.182181in}}%
\pgfpathlineto{\pgfqpoint{3.818619in}{3.181761in}}%
\pgfpathlineto{\pgfqpoint{3.818910in}{3.182410in}}%
\pgfpathlineto{\pgfqpoint{3.820511in}{3.191511in}}%
\pgfpathlineto{\pgfqpoint{3.821384in}{3.187939in}}%
\pgfpathlineto{\pgfqpoint{3.822840in}{3.182811in}}%
\pgfpathlineto{\pgfqpoint{3.823131in}{3.183142in}}%
\pgfpathlineto{\pgfqpoint{3.824149in}{3.188631in}}%
\pgfpathlineto{\pgfqpoint{3.826333in}{3.201611in}}%
\pgfpathlineto{\pgfqpoint{3.826624in}{3.201356in}}%
\pgfpathlineto{\pgfqpoint{3.827497in}{3.195345in}}%
\pgfpathlineto{\pgfqpoint{3.829971in}{3.181277in}}%
\pgfpathlineto{\pgfqpoint{3.831718in}{3.168918in}}%
\pgfpathlineto{\pgfqpoint{3.832300in}{3.170360in}}%
\pgfpathlineto{\pgfqpoint{3.836230in}{3.196844in}}%
\pgfpathlineto{\pgfqpoint{3.838122in}{3.208808in}}%
\pgfpathlineto{\pgfqpoint{3.838413in}{3.208149in}}%
\pgfpathlineto{\pgfqpoint{3.839723in}{3.197011in}}%
\pgfpathlineto{\pgfqpoint{3.841906in}{3.186919in}}%
\pgfpathlineto{\pgfqpoint{3.843362in}{3.170167in}}%
\pgfpathlineto{\pgfqpoint{3.844235in}{3.164602in}}%
\pgfpathlineto{\pgfqpoint{3.844671in}{3.165941in}}%
\pgfpathlineto{\pgfqpoint{3.849620in}{3.203031in}}%
\pgfpathlineto{\pgfqpoint{3.850493in}{3.199554in}}%
\pgfpathlineto{\pgfqpoint{3.853841in}{3.181631in}}%
\pgfpathlineto{\pgfqpoint{3.855005in}{3.177401in}}%
\pgfpathlineto{\pgfqpoint{3.855442in}{3.178392in}}%
\pgfpathlineto{\pgfqpoint{3.859517in}{3.196472in}}%
\pgfpathlineto{\pgfqpoint{3.860099in}{3.195469in}}%
\pgfpathlineto{\pgfqpoint{3.861555in}{3.186405in}}%
\pgfpathlineto{\pgfqpoint{3.862574in}{3.182066in}}%
\pgfpathlineto{\pgfqpoint{3.863010in}{3.183000in}}%
\pgfpathlineto{\pgfqpoint{3.863883in}{3.185678in}}%
\pgfpathlineto{\pgfqpoint{3.864320in}{3.184266in}}%
\pgfpathlineto{\pgfqpoint{3.865630in}{3.175252in}}%
\pgfpathlineto{\pgfqpoint{3.866212in}{3.177799in}}%
\pgfpathlineto{\pgfqpoint{3.868104in}{3.192018in}}%
\pgfpathlineto{\pgfqpoint{3.868686in}{3.190714in}}%
\pgfpathlineto{\pgfqpoint{3.869269in}{3.189937in}}%
\pgfpathlineto{\pgfqpoint{3.869560in}{3.190702in}}%
\pgfpathlineto{\pgfqpoint{3.871015in}{3.199112in}}%
\pgfpathlineto{\pgfqpoint{3.871597in}{3.196888in}}%
\pgfpathlineto{\pgfqpoint{3.874072in}{3.186539in}}%
\pgfpathlineto{\pgfqpoint{3.875090in}{3.180424in}}%
\pgfpathlineto{\pgfqpoint{3.875964in}{3.176162in}}%
\pgfpathlineto{\pgfqpoint{3.876400in}{3.177827in}}%
\pgfpathlineto{\pgfqpoint{3.878147in}{3.189586in}}%
\pgfpathlineto{\pgfqpoint{3.878729in}{3.188314in}}%
\pgfpathlineto{\pgfqpoint{3.879748in}{3.184991in}}%
\pgfpathlineto{\pgfqpoint{3.880185in}{3.185896in}}%
\pgfpathlineto{\pgfqpoint{3.882077in}{3.196993in}}%
\pgfpathlineto{\pgfqpoint{3.882804in}{3.194278in}}%
\pgfpathlineto{\pgfqpoint{3.884405in}{3.183547in}}%
\pgfpathlineto{\pgfqpoint{3.884988in}{3.184668in}}%
\pgfpathlineto{\pgfqpoint{3.886006in}{3.187014in}}%
\pgfpathlineto{\pgfqpoint{3.886443in}{3.185878in}}%
\pgfpathlineto{\pgfqpoint{3.887898in}{3.177801in}}%
\pgfpathlineto{\pgfqpoint{3.888481in}{3.180376in}}%
\pgfpathlineto{\pgfqpoint{3.890082in}{3.191284in}}%
\pgfpathlineto{\pgfqpoint{3.890518in}{3.189988in}}%
\pgfpathlineto{\pgfqpoint{3.891683in}{3.185684in}}%
\pgfpathlineto{\pgfqpoint{3.892119in}{3.186716in}}%
\pgfpathlineto{\pgfqpoint{3.894303in}{3.202605in}}%
\pgfpathlineto{\pgfqpoint{3.895176in}{3.196910in}}%
\pgfpathlineto{\pgfqpoint{3.897068in}{3.178419in}}%
\pgfpathlineto{\pgfqpoint{3.897505in}{3.179463in}}%
\pgfpathlineto{\pgfqpoint{3.901143in}{3.188096in}}%
\pgfpathlineto{\pgfqpoint{3.901725in}{3.186728in}}%
\pgfpathlineto{\pgfqpoint{3.903326in}{3.179638in}}%
\pgfpathlineto{\pgfqpoint{3.903909in}{3.180829in}}%
\pgfpathlineto{\pgfqpoint{3.909876in}{3.201830in}}%
\pgfpathlineto{\pgfqpoint{3.910021in}{3.201599in}}%
\pgfpathlineto{\pgfqpoint{3.911040in}{3.195019in}}%
\pgfpathlineto{\pgfqpoint{3.914679in}{3.170635in}}%
\pgfpathlineto{\pgfqpoint{3.914824in}{3.170854in}}%
\pgfpathlineto{\pgfqpoint{3.918172in}{3.183617in}}%
\pgfpathlineto{\pgfqpoint{3.920210in}{3.200109in}}%
\pgfpathlineto{\pgfqpoint{3.920937in}{3.199249in}}%
\pgfpathlineto{\pgfqpoint{3.922393in}{3.199035in}}%
\pgfpathlineto{\pgfqpoint{3.923703in}{3.199721in}}%
\pgfpathlineto{\pgfqpoint{3.923994in}{3.198983in}}%
\pgfpathlineto{\pgfqpoint{3.925158in}{3.188708in}}%
\pgfpathlineto{\pgfqpoint{3.927778in}{3.169580in}}%
\pgfpathlineto{\pgfqpoint{3.928360in}{3.169105in}}%
\pgfpathlineto{\pgfqpoint{3.928797in}{3.169848in}}%
\pgfpathlineto{\pgfqpoint{3.929961in}{3.177625in}}%
\pgfpathlineto{\pgfqpoint{3.932727in}{3.197446in}}%
\pgfpathlineto{\pgfqpoint{3.932872in}{3.197425in}}%
\pgfpathlineto{\pgfqpoint{3.934036in}{3.197892in}}%
\pgfpathlineto{\pgfqpoint{3.935492in}{3.201616in}}%
\pgfpathlineto{\pgfqpoint{3.935929in}{3.200153in}}%
\pgfpathlineto{\pgfqpoint{3.940295in}{3.177414in}}%
\pgfpathlineto{\pgfqpoint{3.940877in}{3.176806in}}%
\pgfpathlineto{\pgfqpoint{3.941168in}{3.177344in}}%
\pgfpathlineto{\pgfqpoint{3.942187in}{3.184792in}}%
\pgfpathlineto{\pgfqpoint{3.943351in}{3.193275in}}%
\pgfpathlineto{\pgfqpoint{3.943934in}{3.191438in}}%
\pgfpathlineto{\pgfqpoint{3.945535in}{3.180508in}}%
\pgfpathlineto{\pgfqpoint{3.946117in}{3.182196in}}%
\pgfpathlineto{\pgfqpoint{3.948882in}{3.192529in}}%
\pgfpathlineto{\pgfqpoint{3.950338in}{3.193262in}}%
\pgfpathlineto{\pgfqpoint{3.950629in}{3.192341in}}%
\pgfpathlineto{\pgfqpoint{3.952084in}{3.185733in}}%
\pgfpathlineto{\pgfqpoint{3.952812in}{3.187171in}}%
\pgfpathlineto{\pgfqpoint{3.953831in}{3.188891in}}%
\pgfpathlineto{\pgfqpoint{3.954267in}{3.188589in}}%
\pgfpathlineto{\pgfqpoint{3.955868in}{3.186471in}}%
\pgfpathlineto{\pgfqpoint{3.957906in}{3.175136in}}%
\pgfpathlineto{\pgfqpoint{3.958488in}{3.177375in}}%
\pgfpathlineto{\pgfqpoint{3.960671in}{3.194291in}}%
\pgfpathlineto{\pgfqpoint{3.961399in}{3.192944in}}%
\pgfpathlineto{\pgfqpoint{3.961836in}{3.192566in}}%
\pgfpathlineto{\pgfqpoint{3.962272in}{3.193430in}}%
\pgfpathlineto{\pgfqpoint{3.963873in}{3.198820in}}%
\pgfpathlineto{\pgfqpoint{3.964310in}{3.197838in}}%
\pgfpathlineto{\pgfqpoint{3.966348in}{3.185751in}}%
\pgfpathlineto{\pgfqpoint{3.968240in}{3.172602in}}%
\pgfpathlineto{\pgfqpoint{3.968822in}{3.174159in}}%
\pgfpathlineto{\pgfqpoint{3.972606in}{3.187024in}}%
\pgfpathlineto{\pgfqpoint{3.973625in}{3.187829in}}%
\pgfpathlineto{\pgfqpoint{3.974935in}{3.194107in}}%
\pgfpathlineto{\pgfqpoint{3.976536in}{3.202416in}}%
\pgfpathlineto{\pgfqpoint{3.976972in}{3.201235in}}%
\pgfpathlineto{\pgfqpoint{3.982794in}{3.172759in}}%
\pgfpathlineto{\pgfqpoint{3.983522in}{3.174665in}}%
\pgfpathlineto{\pgfqpoint{3.985851in}{3.195024in}}%
\pgfpathlineto{\pgfqpoint{3.987161in}{3.191183in}}%
\pgfpathlineto{\pgfqpoint{3.987888in}{3.190083in}}%
\pgfpathlineto{\pgfqpoint{3.988325in}{3.190963in}}%
\pgfpathlineto{\pgfqpoint{3.989489in}{3.194413in}}%
\pgfpathlineto{\pgfqpoint{3.989926in}{3.193540in}}%
\pgfpathlineto{\pgfqpoint{3.992255in}{3.182202in}}%
\pgfpathlineto{\pgfqpoint{3.993128in}{3.183986in}}%
\pgfpathlineto{\pgfqpoint{3.995893in}{3.188499in}}%
\pgfpathlineto{\pgfqpoint{3.997203in}{3.189608in}}%
\pgfpathlineto{\pgfqpoint{3.997785in}{3.189871in}}%
\pgfpathlineto{\pgfqpoint{3.998222in}{3.189179in}}%
\pgfpathlineto{\pgfqpoint{3.999678in}{3.186492in}}%
\pgfpathlineto{\pgfqpoint{4.000114in}{3.187074in}}%
\pgfpathlineto{\pgfqpoint{4.001424in}{3.189103in}}%
\pgfpathlineto{\pgfqpoint{4.001715in}{3.188560in}}%
\pgfpathlineto{\pgfqpoint{4.003607in}{3.181181in}}%
\pgfpathlineto{\pgfqpoint{4.004481in}{3.182702in}}%
\pgfpathlineto{\pgfqpoint{4.006373in}{3.190254in}}%
\pgfpathlineto{\pgfqpoint{4.008119in}{3.192349in}}%
\pgfpathlineto{\pgfqpoint{4.009575in}{3.193612in}}%
\pgfpathlineto{\pgfqpoint{4.010448in}{3.194572in}}%
\pgfpathlineto{\pgfqpoint{4.010885in}{3.193986in}}%
\pgfpathlineto{\pgfqpoint{4.012486in}{3.187347in}}%
\pgfpathlineto{\pgfqpoint{4.015105in}{3.172908in}}%
\pgfpathlineto{\pgfqpoint{4.015542in}{3.174278in}}%
\pgfpathlineto{\pgfqpoint{4.017143in}{3.181726in}}%
\pgfpathlineto{\pgfqpoint{4.017725in}{3.180859in}}%
\pgfpathlineto{\pgfqpoint{4.018162in}{3.180570in}}%
\pgfpathlineto{\pgfqpoint{4.018453in}{3.181273in}}%
\pgfpathlineto{\pgfqpoint{4.019326in}{3.190011in}}%
\pgfpathlineto{\pgfqpoint{4.021218in}{3.209452in}}%
\pgfpathlineto{\pgfqpoint{4.021509in}{3.208831in}}%
\pgfpathlineto{\pgfqpoint{4.022819in}{3.200059in}}%
\pgfpathlineto{\pgfqpoint{4.026167in}{3.168957in}}%
\pgfpathlineto{\pgfqpoint{4.026895in}{3.170126in}}%
\pgfpathlineto{\pgfqpoint{4.028787in}{3.178431in}}%
\pgfpathlineto{\pgfqpoint{4.031843in}{3.211257in}}%
\pgfpathlineto{\pgfqpoint{4.032716in}{3.206241in}}%
\pgfpathlineto{\pgfqpoint{4.036646in}{3.174052in}}%
\pgfpathlineto{\pgfqpoint{4.037665in}{3.170401in}}%
\pgfpathlineto{\pgfqpoint{4.038247in}{3.171481in}}%
\pgfpathlineto{\pgfqpoint{4.040139in}{3.184132in}}%
\pgfpathlineto{\pgfqpoint{4.042759in}{3.195889in}}%
\pgfpathlineto{\pgfqpoint{4.044797in}{3.199779in}}%
\pgfpathlineto{\pgfqpoint{4.045379in}{3.198644in}}%
\pgfpathlineto{\pgfqpoint{4.049454in}{3.182850in}}%
\pgfpathlineto{\pgfqpoint{4.050473in}{3.179918in}}%
\pgfpathlineto{\pgfqpoint{4.051346in}{3.177757in}}%
\pgfpathlineto{\pgfqpoint{4.051928in}{3.178576in}}%
\pgfpathlineto{\pgfqpoint{4.053675in}{3.186415in}}%
\pgfpathlineto{\pgfqpoint{4.056004in}{3.195726in}}%
\pgfpathlineto{\pgfqpoint{4.057168in}{3.194210in}}%
\pgfpathlineto{\pgfqpoint{4.064445in}{3.180110in}}%
\pgfpathlineto{\pgfqpoint{4.064591in}{3.180229in}}%
\pgfpathlineto{\pgfqpoint{4.065610in}{3.184081in}}%
\pgfpathlineto{\pgfqpoint{4.067939in}{3.191059in}}%
\pgfpathlineto{\pgfqpoint{4.068084in}{3.190981in}}%
\pgfpathlineto{\pgfqpoint{4.069248in}{3.188007in}}%
\pgfpathlineto{\pgfqpoint{4.071286in}{3.184654in}}%
\pgfpathlineto{\pgfqpoint{4.071432in}{3.184751in}}%
\pgfpathlineto{\pgfqpoint{4.074197in}{3.188913in}}%
\pgfpathlineto{\pgfqpoint{4.076380in}{3.196032in}}%
\pgfpathlineto{\pgfqpoint{4.076817in}{3.195318in}}%
\pgfpathlineto{\pgfqpoint{4.080892in}{3.180211in}}%
\pgfpathlineto{\pgfqpoint{4.082056in}{3.175794in}}%
\pgfpathlineto{\pgfqpoint{4.082493in}{3.176773in}}%
\pgfpathlineto{\pgfqpoint{4.084676in}{3.187104in}}%
\pgfpathlineto{\pgfqpoint{4.085404in}{3.185692in}}%
\pgfpathlineto{\pgfqpoint{4.085986in}{3.184771in}}%
\pgfpathlineto{\pgfqpoint{4.086423in}{3.185537in}}%
\pgfpathlineto{\pgfqpoint{4.088606in}{3.197978in}}%
\pgfpathlineto{\pgfqpoint{4.089770in}{3.195748in}}%
\pgfpathlineto{\pgfqpoint{4.091517in}{3.194462in}}%
\pgfpathlineto{\pgfqpoint{4.092390in}{3.193886in}}%
\pgfpathlineto{\pgfqpoint{4.093409in}{3.186845in}}%
\pgfpathlineto{\pgfqpoint{4.095592in}{3.172212in}}%
\pgfpathlineto{\pgfqpoint{4.095738in}{3.172299in}}%
\pgfpathlineto{\pgfqpoint{4.096757in}{3.175197in}}%
\pgfpathlineto{\pgfqpoint{4.102724in}{3.202370in}}%
\pgfpathlineto{\pgfqpoint{4.103452in}{3.199424in}}%
\pgfpathlineto{\pgfqpoint{4.105780in}{3.186234in}}%
\pgfpathlineto{\pgfqpoint{4.106217in}{3.186397in}}%
\pgfpathlineto{\pgfqpoint{4.106945in}{3.185739in}}%
\pgfpathlineto{\pgfqpoint{4.108982in}{3.179997in}}%
\pgfpathlineto{\pgfqpoint{4.109710in}{3.180890in}}%
\pgfpathlineto{\pgfqpoint{4.111602in}{3.184214in}}%
\pgfpathlineto{\pgfqpoint{4.112184in}{3.183618in}}%
\pgfpathlineto{\pgfqpoint{4.112767in}{3.183453in}}%
\pgfpathlineto{\pgfqpoint{4.113203in}{3.184147in}}%
\pgfpathlineto{\pgfqpoint{4.115241in}{3.186082in}}%
\pgfpathlineto{\pgfqpoint{4.116260in}{3.189895in}}%
\pgfpathlineto{\pgfqpoint{4.117861in}{3.196817in}}%
\pgfpathlineto{\pgfqpoint{4.118443in}{3.196138in}}%
\pgfpathlineto{\pgfqpoint{4.121499in}{3.187689in}}%
\pgfpathlineto{\pgfqpoint{4.122373in}{3.184366in}}%
\pgfpathlineto{\pgfqpoint{4.122809in}{3.185328in}}%
\pgfpathlineto{\pgfqpoint{4.123683in}{3.187234in}}%
\pgfpathlineto{\pgfqpoint{4.124119in}{3.186052in}}%
\pgfpathlineto{\pgfqpoint{4.126593in}{3.174952in}}%
\pgfpathlineto{\pgfqpoint{4.127176in}{3.175917in}}%
\pgfpathlineto{\pgfqpoint{4.128777in}{3.183096in}}%
\pgfpathlineto{\pgfqpoint{4.131396in}{3.201487in}}%
\pgfpathlineto{\pgfqpoint{4.132124in}{3.200294in}}%
\pgfpathlineto{\pgfqpoint{4.136199in}{3.183235in}}%
\pgfpathlineto{\pgfqpoint{4.138092in}{3.171339in}}%
\pgfpathlineto{\pgfqpoint{4.138528in}{3.172207in}}%
\pgfpathlineto{\pgfqpoint{4.139693in}{3.183010in}}%
\pgfpathlineto{\pgfqpoint{4.141294in}{3.192869in}}%
\pgfpathlineto{\pgfqpoint{4.141585in}{3.192717in}}%
\pgfpathlineto{\pgfqpoint{4.143186in}{3.190983in}}%
\pgfpathlineto{\pgfqpoint{4.143622in}{3.191712in}}%
\pgfpathlineto{\pgfqpoint{4.145223in}{3.195760in}}%
\pgfpathlineto{\pgfqpoint{4.145660in}{3.194762in}}%
\pgfpathlineto{\pgfqpoint{4.147115in}{3.183220in}}%
\pgfpathlineto{\pgfqpoint{4.148134in}{3.178856in}}%
\pgfpathlineto{\pgfqpoint{4.148571in}{3.179955in}}%
\pgfpathlineto{\pgfqpoint{4.152209in}{3.190578in}}%
\pgfpathlineto{\pgfqpoint{4.155266in}{3.195488in}}%
\pgfpathlineto{\pgfqpoint{4.155848in}{3.193725in}}%
\pgfpathlineto{\pgfqpoint{4.158468in}{3.174718in}}%
\pgfpathlineto{\pgfqpoint{4.159487in}{3.177348in}}%
\pgfpathlineto{\pgfqpoint{4.164144in}{3.192145in}}%
\pgfpathlineto{\pgfqpoint{4.165745in}{3.194941in}}%
\pgfpathlineto{\pgfqpoint{4.167055in}{3.197313in}}%
\pgfpathlineto{\pgfqpoint{4.167492in}{3.196331in}}%
\pgfpathlineto{\pgfqpoint{4.168656in}{3.185467in}}%
\pgfpathlineto{\pgfqpoint{4.169821in}{3.176987in}}%
\pgfpathlineto{\pgfqpoint{4.170257in}{3.177923in}}%
\pgfpathlineto{\pgfqpoint{4.171567in}{3.183424in}}%
\pgfpathlineto{\pgfqpoint{4.172149in}{3.181735in}}%
\pgfpathlineto{\pgfqpoint{4.172877in}{3.179455in}}%
\pgfpathlineto{\pgfqpoint{4.173314in}{3.180462in}}%
\pgfpathlineto{\pgfqpoint{4.175497in}{3.195056in}}%
\pgfpathlineto{\pgfqpoint{4.176516in}{3.193120in}}%
\pgfpathlineto{\pgfqpoint{4.178117in}{3.192112in}}%
\pgfpathlineto{\pgfqpoint{4.178699in}{3.191140in}}%
\pgfpathlineto{\pgfqpoint{4.180445in}{3.182109in}}%
\pgfpathlineto{\pgfqpoint{4.181173in}{3.184371in}}%
\pgfpathlineto{\pgfqpoint{4.182337in}{3.189067in}}%
\pgfpathlineto{\pgfqpoint{4.182920in}{3.187927in}}%
\pgfpathlineto{\pgfqpoint{4.184375in}{3.182975in}}%
\pgfpathlineto{\pgfqpoint{4.184812in}{3.183686in}}%
\pgfpathlineto{\pgfqpoint{4.186849in}{3.187299in}}%
\pgfpathlineto{\pgfqpoint{4.187140in}{3.187030in}}%
\pgfpathlineto{\pgfqpoint{4.189178in}{3.183076in}}%
\pgfpathlineto{\pgfqpoint{4.189760in}{3.184302in}}%
\pgfpathlineto{\pgfqpoint{4.191798in}{3.195422in}}%
\pgfpathlineto{\pgfqpoint{4.193108in}{3.198223in}}%
\pgfpathlineto{\pgfqpoint{4.193399in}{3.198042in}}%
\pgfpathlineto{\pgfqpoint{4.194272in}{3.195501in}}%
\pgfpathlineto{\pgfqpoint{4.199512in}{3.174367in}}%
\pgfpathlineto{\pgfqpoint{4.200240in}{3.173470in}}%
\pgfpathlineto{\pgfqpoint{4.200676in}{3.174221in}}%
\pgfpathlineto{\pgfqpoint{4.201986in}{3.181984in}}%
\pgfpathlineto{\pgfqpoint{4.204897in}{3.195896in}}%
\pgfpathlineto{\pgfqpoint{4.207662in}{3.198141in}}%
\pgfpathlineto{\pgfqpoint{4.208827in}{3.196486in}}%
\pgfpathlineto{\pgfqpoint{4.210282in}{3.184777in}}%
\pgfpathlineto{\pgfqpoint{4.212465in}{3.176330in}}%
\pgfpathlineto{\pgfqpoint{4.213193in}{3.174870in}}%
\pgfpathlineto{\pgfqpoint{4.213630in}{3.175743in}}%
\pgfpathlineto{\pgfqpoint{4.218869in}{3.193407in}}%
\pgfpathlineto{\pgfqpoint{4.219452in}{3.194091in}}%
\pgfpathlineto{\pgfqpoint{4.220034in}{3.193383in}}%
\pgfpathlineto{\pgfqpoint{4.220616in}{3.192840in}}%
\pgfpathlineto{\pgfqpoint{4.221053in}{3.193469in}}%
\pgfpathlineto{\pgfqpoint{4.222508in}{3.196638in}}%
\pgfpathlineto{\pgfqpoint{4.222945in}{3.196073in}}%
\pgfpathlineto{\pgfqpoint{4.223964in}{3.190890in}}%
\pgfpathlineto{\pgfqpoint{4.225856in}{3.177716in}}%
\pgfpathlineto{\pgfqpoint{4.226292in}{3.178414in}}%
\pgfpathlineto{\pgfqpoint{4.228621in}{3.181527in}}%
\pgfpathlineto{\pgfqpoint{4.229494in}{3.182205in}}%
\pgfpathlineto{\pgfqpoint{4.233278in}{3.192592in}}%
\pgfpathlineto{\pgfqpoint{4.234297in}{3.192053in}}%
\pgfpathlineto{\pgfqpoint{4.237499in}{3.190559in}}%
\pgfpathlineto{\pgfqpoint{4.240410in}{3.182048in}}%
\pgfpathlineto{\pgfqpoint{4.242157in}{3.177841in}}%
\pgfpathlineto{\pgfqpoint{4.242448in}{3.178225in}}%
\pgfpathlineto{\pgfqpoint{4.243612in}{3.184345in}}%
\pgfpathlineto{\pgfqpoint{4.245068in}{3.189416in}}%
\pgfpathlineto{\pgfqpoint{4.245359in}{3.189215in}}%
\pgfpathlineto{\pgfqpoint{4.245941in}{3.189211in}}%
\pgfpathlineto{\pgfqpoint{4.246232in}{3.189887in}}%
\pgfpathlineto{\pgfqpoint{4.247833in}{3.195583in}}%
\pgfpathlineto{\pgfqpoint{4.248415in}{3.194539in}}%
\pgfpathlineto{\pgfqpoint{4.253509in}{3.178462in}}%
\pgfpathlineto{\pgfqpoint{4.254237in}{3.179843in}}%
\pgfpathlineto{\pgfqpoint{4.260641in}{3.198405in}}%
\pgfpathlineto{\pgfqpoint{4.260932in}{3.198005in}}%
\pgfpathlineto{\pgfqpoint{4.262096in}{3.191064in}}%
\pgfpathlineto{\pgfqpoint{4.263989in}{3.180875in}}%
\pgfpathlineto{\pgfqpoint{4.264280in}{3.181110in}}%
\pgfpathlineto{\pgfqpoint{4.265153in}{3.181795in}}%
\pgfpathlineto{\pgfqpoint{4.265590in}{3.181141in}}%
\pgfpathlineto{\pgfqpoint{4.267191in}{3.177600in}}%
\pgfpathlineto{\pgfqpoint{4.267773in}{3.178607in}}%
\pgfpathlineto{\pgfqpoint{4.269083in}{3.186810in}}%
\pgfpathlineto{\pgfqpoint{4.271120in}{3.200972in}}%
\pgfpathlineto{\pgfqpoint{4.271557in}{3.200321in}}%
\pgfpathlineto{\pgfqpoint{4.273303in}{3.190286in}}%
\pgfpathlineto{\pgfqpoint{4.274468in}{3.187025in}}%
\pgfpathlineto{\pgfqpoint{4.274904in}{3.187226in}}%
\pgfpathlineto{\pgfqpoint{4.275632in}{3.187102in}}%
\pgfpathlineto{\pgfqpoint{4.275778in}{3.186832in}}%
\pgfpathlineto{\pgfqpoint{4.277233in}{3.180157in}}%
\pgfpathlineto{\pgfqpoint{4.278980in}{3.172865in}}%
\pgfpathlineto{\pgfqpoint{4.279416in}{3.173685in}}%
\pgfpathlineto{\pgfqpoint{4.280581in}{3.183118in}}%
\pgfpathlineto{\pgfqpoint{4.283201in}{3.202114in}}%
\pgfpathlineto{\pgfqpoint{4.283346in}{3.201972in}}%
\pgfpathlineto{\pgfqpoint{4.284365in}{3.197823in}}%
\pgfpathlineto{\pgfqpoint{4.286694in}{3.190877in}}%
\pgfpathlineto{\pgfqpoint{4.288004in}{3.182852in}}%
\pgfpathlineto{\pgfqpoint{4.289605in}{3.176266in}}%
\pgfpathlineto{\pgfqpoint{4.289896in}{3.176406in}}%
\pgfpathlineto{\pgfqpoint{4.291642in}{3.179592in}}%
\pgfpathlineto{\pgfqpoint{4.293098in}{3.190371in}}%
\pgfpathlineto{\pgfqpoint{4.294262in}{3.195317in}}%
\pgfpathlineto{\pgfqpoint{4.294699in}{3.194847in}}%
\pgfpathlineto{\pgfqpoint{4.296445in}{3.191110in}}%
\pgfpathlineto{\pgfqpoint{4.297173in}{3.191828in}}%
\pgfpathlineto{\pgfqpoint{4.297755in}{3.191157in}}%
\pgfpathlineto{\pgfqpoint{4.299793in}{3.181279in}}%
\pgfpathlineto{\pgfqpoint{4.300957in}{3.183145in}}%
\pgfpathlineto{\pgfqpoint{4.303431in}{3.185423in}}%
\pgfpathlineto{\pgfqpoint{4.306051in}{3.194994in}}%
\pgfpathlineto{\pgfqpoint{4.306779in}{3.194108in}}%
\pgfpathlineto{\pgfqpoint{4.312601in}{3.183137in}}%
\pgfpathlineto{\pgfqpoint{4.313474in}{3.185060in}}%
\pgfpathlineto{\pgfqpoint{4.315657in}{3.188646in}}%
\pgfpathlineto{\pgfqpoint{4.316967in}{3.190698in}}%
\pgfpathlineto{\pgfqpoint{4.317840in}{3.191784in}}%
\pgfpathlineto{\pgfqpoint{4.318277in}{3.190985in}}%
\pgfpathlineto{\pgfqpoint{4.320751in}{3.186320in}}%
\pgfpathlineto{\pgfqpoint{4.322061in}{3.184764in}}%
\pgfpathlineto{\pgfqpoint{4.322789in}{3.184078in}}%
\pgfpathlineto{\pgfqpoint{4.323226in}{3.184735in}}%
\pgfpathlineto{\pgfqpoint{4.324827in}{3.190235in}}%
\pgfpathlineto{\pgfqpoint{4.325554in}{3.188379in}}%
\pgfpathlineto{\pgfqpoint{4.326864in}{3.185303in}}%
\pgfpathlineto{\pgfqpoint{4.327301in}{3.185532in}}%
\pgfpathlineto{\pgfqpoint{4.328465in}{3.185030in}}%
\pgfpathlineto{\pgfqpoint{4.329047in}{3.184783in}}%
\pgfpathlineto{\pgfqpoint{4.329484in}{3.185558in}}%
\pgfpathlineto{\pgfqpoint{4.331667in}{3.193213in}}%
\pgfpathlineto{\pgfqpoint{4.332395in}{3.191492in}}%
\pgfpathlineto{\pgfqpoint{4.334287in}{3.183327in}}%
\pgfpathlineto{\pgfqpoint{4.334869in}{3.184283in}}%
\pgfpathlineto{\pgfqpoint{4.336470in}{3.189348in}}%
\pgfpathlineto{\pgfqpoint{4.337053in}{3.188104in}}%
\pgfpathlineto{\pgfqpoint{4.338508in}{3.185111in}}%
\pgfpathlineto{\pgfqpoint{4.338945in}{3.185501in}}%
\pgfpathlineto{\pgfqpoint{4.340546in}{3.189479in}}%
\pgfpathlineto{\pgfqpoint{4.341856in}{3.192972in}}%
\pgfpathlineto{\pgfqpoint{4.342292in}{3.192139in}}%
\pgfpathlineto{\pgfqpoint{4.344475in}{3.184061in}}%
\pgfpathlineto{\pgfqpoint{4.345203in}{3.185389in}}%
\pgfpathlineto{\pgfqpoint{4.347095in}{3.189068in}}%
\pgfpathlineto{\pgfqpoint{4.347386in}{3.188840in}}%
\pgfpathlineto{\pgfqpoint{4.350879in}{3.184242in}}%
\pgfpathlineto{\pgfqpoint{4.351462in}{3.185350in}}%
\pgfpathlineto{\pgfqpoint{4.352626in}{3.188022in}}%
\pgfpathlineto{\pgfqpoint{4.353063in}{3.187403in}}%
\pgfpathlineto{\pgfqpoint{4.354664in}{3.181281in}}%
\pgfpathlineto{\pgfqpoint{4.355246in}{3.183017in}}%
\pgfpathlineto{\pgfqpoint{4.357866in}{3.192995in}}%
\pgfpathlineto{\pgfqpoint{4.358157in}{3.192799in}}%
\pgfpathlineto{\pgfqpoint{4.360631in}{3.189795in}}%
\pgfpathlineto{\pgfqpoint{4.361213in}{3.190684in}}%
\pgfpathlineto{\pgfqpoint{4.361941in}{3.191671in}}%
\pgfpathlineto{\pgfqpoint{4.362377in}{3.190890in}}%
\pgfpathlineto{\pgfqpoint{4.364124in}{3.183169in}}%
\pgfpathlineto{\pgfqpoint{4.364997in}{3.185048in}}%
\pgfpathlineto{\pgfqpoint{4.366016in}{3.186830in}}%
\pgfpathlineto{\pgfqpoint{4.366453in}{3.186210in}}%
\pgfpathlineto{\pgfqpoint{4.367472in}{3.184738in}}%
\pgfpathlineto{\pgfqpoint{4.367908in}{3.185280in}}%
\pgfpathlineto{\pgfqpoint{4.368636in}{3.186297in}}%
\pgfpathlineto{\pgfqpoint{4.369073in}{3.185519in}}%
\pgfpathlineto{\pgfqpoint{4.370674in}{3.178625in}}%
\pgfpathlineto{\pgfqpoint{4.371256in}{3.180976in}}%
\pgfpathlineto{\pgfqpoint{4.373148in}{3.194529in}}%
\pgfpathlineto{\pgfqpoint{4.373876in}{3.192937in}}%
\pgfpathlineto{\pgfqpoint{4.374458in}{3.192278in}}%
\pgfpathlineto{\pgfqpoint{4.374894in}{3.193015in}}%
\pgfpathlineto{\pgfqpoint{4.375913in}{3.194910in}}%
\pgfpathlineto{\pgfqpoint{4.376350in}{3.194323in}}%
\pgfpathlineto{\pgfqpoint{4.377514in}{3.187372in}}%
\pgfpathlineto{\pgfqpoint{4.378824in}{3.179925in}}%
\pgfpathlineto{\pgfqpoint{4.379261in}{3.180689in}}%
\pgfpathlineto{\pgfqpoint{4.381298in}{3.185714in}}%
\pgfpathlineto{\pgfqpoint{4.381735in}{3.185414in}}%
\pgfpathlineto{\pgfqpoint{4.382463in}{3.185462in}}%
\pgfpathlineto{\pgfqpoint{4.382608in}{3.185739in}}%
\pgfpathlineto{\pgfqpoint{4.384937in}{3.192821in}}%
\pgfpathlineto{\pgfqpoint{4.385956in}{3.191974in}}%
\pgfpathlineto{\pgfqpoint{4.387266in}{3.187356in}}%
\pgfpathlineto{\pgfqpoint{4.389595in}{3.179605in}}%
\pgfpathlineto{\pgfqpoint{4.389886in}{3.179761in}}%
\pgfpathlineto{\pgfqpoint{4.392069in}{3.182140in}}%
\pgfpathlineto{\pgfqpoint{4.394689in}{3.202855in}}%
\pgfpathlineto{\pgfqpoint{4.395707in}{3.200293in}}%
\pgfpathlineto{\pgfqpoint{4.398036in}{3.188280in}}%
\pgfpathlineto{\pgfqpoint{4.400365in}{3.171398in}}%
\pgfpathlineto{\pgfqpoint{4.400656in}{3.171653in}}%
\pgfpathlineto{\pgfqpoint{4.402257in}{3.178049in}}%
\pgfpathlineto{\pgfqpoint{4.407060in}{3.197048in}}%
\pgfpathlineto{\pgfqpoint{4.407933in}{3.196164in}}%
\pgfpathlineto{\pgfqpoint{4.412300in}{3.183959in}}%
\pgfpathlineto{\pgfqpoint{4.413173in}{3.182322in}}%
\pgfpathlineto{\pgfqpoint{4.413610in}{3.182739in}}%
\pgfpathlineto{\pgfqpoint{4.414628in}{3.183613in}}%
\pgfpathlineto{\pgfqpoint{4.415065in}{3.183236in}}%
\pgfpathlineto{\pgfqpoint{4.415647in}{3.183185in}}%
\pgfpathlineto{\pgfqpoint{4.415938in}{3.183672in}}%
\pgfpathlineto{\pgfqpoint{4.417539in}{3.186193in}}%
\pgfpathlineto{\pgfqpoint{4.417976in}{3.185952in}}%
\pgfpathlineto{\pgfqpoint{4.418849in}{3.186672in}}%
\pgfpathlineto{\pgfqpoint{4.421760in}{3.192532in}}%
\pgfpathlineto{\pgfqpoint{4.422197in}{3.192118in}}%
\pgfpathlineto{\pgfqpoint{4.427436in}{3.182465in}}%
\pgfpathlineto{\pgfqpoint{4.428455in}{3.181060in}}%
\pgfpathlineto{\pgfqpoint{4.428892in}{3.181551in}}%
\pgfpathlineto{\pgfqpoint{4.430347in}{3.187186in}}%
\pgfpathlineto{\pgfqpoint{4.432385in}{3.191549in}}%
\pgfpathlineto{\pgfqpoint{4.433986in}{3.192387in}}%
\pgfpathlineto{\pgfqpoint{4.434131in}{3.192216in}}%
\pgfpathlineto{\pgfqpoint{4.435587in}{3.188250in}}%
\pgfpathlineto{\pgfqpoint{4.438352in}{3.182139in}}%
\pgfpathlineto{\pgfqpoint{4.439226in}{3.182040in}}%
\pgfpathlineto{\pgfqpoint{4.439517in}{3.182468in}}%
\pgfpathlineto{\pgfqpoint{4.442137in}{3.190372in}}%
\pgfpathlineto{\pgfqpoint{4.444174in}{3.194545in}}%
\pgfpathlineto{\pgfqpoint{4.444320in}{3.194456in}}%
\pgfpathlineto{\pgfqpoint{4.445484in}{3.192041in}}%
\pgfpathlineto{\pgfqpoint{4.448977in}{3.180642in}}%
\pgfpathlineto{\pgfqpoint{4.449705in}{3.181884in}}%
\pgfpathlineto{\pgfqpoint{4.451015in}{3.184812in}}%
\pgfpathlineto{\pgfqpoint{4.451451in}{3.184231in}}%
\pgfpathlineto{\pgfqpoint{4.452325in}{3.183196in}}%
\pgfpathlineto{\pgfqpoint{4.452616in}{3.183768in}}%
\pgfpathlineto{\pgfqpoint{4.454071in}{3.192803in}}%
\pgfpathlineto{\pgfqpoint{4.455236in}{3.197027in}}%
\pgfpathlineto{\pgfqpoint{4.455672in}{3.196577in}}%
\pgfpathlineto{\pgfqpoint{4.458147in}{3.187821in}}%
\pgfpathlineto{\pgfqpoint{4.460912in}{3.176724in}}%
\pgfpathlineto{\pgfqpoint{4.461494in}{3.177540in}}%
\pgfpathlineto{\pgfqpoint{4.464259in}{3.190003in}}%
\pgfpathlineto{\pgfqpoint{4.466588in}{3.200695in}}%
\pgfpathlineto{\pgfqpoint{4.467025in}{3.199805in}}%
\pgfpathlineto{\pgfqpoint{4.470518in}{3.181103in}}%
\pgfpathlineto{\pgfqpoint{4.472410in}{3.176017in}}%
\pgfpathlineto{\pgfqpoint{4.473720in}{3.177193in}}%
\pgfpathlineto{\pgfqpoint{4.474884in}{3.182194in}}%
\pgfpathlineto{\pgfqpoint{4.478232in}{3.198939in}}%
\pgfpathlineto{\pgfqpoint{4.478814in}{3.199420in}}%
\pgfpathlineto{\pgfqpoint{4.479251in}{3.198649in}}%
\pgfpathlineto{\pgfqpoint{4.484054in}{3.177992in}}%
\pgfpathlineto{\pgfqpoint{4.484927in}{3.174788in}}%
\pgfpathlineto{\pgfqpoint{4.485509in}{3.175572in}}%
\pgfpathlineto{\pgfqpoint{4.488129in}{3.184216in}}%
\pgfpathlineto{\pgfqpoint{4.490894in}{3.195761in}}%
\pgfpathlineto{\pgfqpoint{4.492641in}{3.197186in}}%
\pgfpathlineto{\pgfqpoint{4.492932in}{3.196641in}}%
\pgfpathlineto{\pgfqpoint{4.495552in}{3.185963in}}%
\pgfpathlineto{\pgfqpoint{4.497881in}{3.178322in}}%
\pgfpathlineto{\pgfqpoint{4.498317in}{3.178918in}}%
\pgfpathlineto{\pgfqpoint{4.504721in}{3.191596in}}%
\pgfpathlineto{\pgfqpoint{4.505740in}{3.190928in}}%
\pgfpathlineto{\pgfqpoint{4.507487in}{3.189026in}}%
\pgfpathlineto{\pgfqpoint{4.507778in}{3.189204in}}%
\pgfpathlineto{\pgfqpoint{4.508796in}{3.188739in}}%
\pgfpathlineto{\pgfqpoint{4.510397in}{3.185912in}}%
\pgfpathlineto{\pgfqpoint{4.510980in}{3.186665in}}%
\pgfpathlineto{\pgfqpoint{4.512290in}{3.187898in}}%
\pgfpathlineto{\pgfqpoint{4.512726in}{3.187572in}}%
\pgfpathlineto{\pgfqpoint{4.517093in}{3.184604in}}%
\pgfpathlineto{\pgfqpoint{4.518111in}{3.186324in}}%
\pgfpathlineto{\pgfqpoint{4.520295in}{3.191383in}}%
\pgfpathlineto{\pgfqpoint{4.520731in}{3.190997in}}%
\pgfpathlineto{\pgfqpoint{4.526116in}{3.184441in}}%
\pgfpathlineto{\pgfqpoint{4.526990in}{3.185469in}}%
\pgfpathlineto{\pgfqpoint{4.529173in}{3.190891in}}%
\pgfpathlineto{\pgfqpoint{4.529755in}{3.190495in}}%
\pgfpathlineto{\pgfqpoint{4.532957in}{3.186633in}}%
\pgfpathlineto{\pgfqpoint{4.534995in}{3.181647in}}%
\pgfpathlineto{\pgfqpoint{4.535140in}{3.181733in}}%
\pgfpathlineto{\pgfqpoint{4.537323in}{3.185130in}}%
\pgfpathlineto{\pgfqpoint{4.540234in}{3.195612in}}%
\pgfpathlineto{\pgfqpoint{4.541108in}{3.193985in}}%
\pgfpathlineto{\pgfqpoint{4.546347in}{3.179196in}}%
\pgfpathlineto{\pgfqpoint{4.547075in}{3.180916in}}%
\pgfpathlineto{\pgfqpoint{4.550714in}{3.192518in}}%
\pgfpathlineto{\pgfqpoint{4.551441in}{3.192738in}}%
\pgfpathlineto{\pgfqpoint{4.551878in}{3.192174in}}%
\pgfpathlineto{\pgfqpoint{4.553042in}{3.190728in}}%
\pgfpathlineto{\pgfqpoint{4.553479in}{3.191161in}}%
\pgfpathlineto{\pgfqpoint{4.554498in}{3.192175in}}%
\pgfpathlineto{\pgfqpoint{4.554789in}{3.191673in}}%
\pgfpathlineto{\pgfqpoint{4.557845in}{3.180944in}}%
\pgfpathlineto{\pgfqpoint{4.559010in}{3.182258in}}%
\pgfpathlineto{\pgfqpoint{4.562212in}{3.185817in}}%
\pgfpathlineto{\pgfqpoint{4.564395in}{3.195874in}}%
\pgfpathlineto{\pgfqpoint{4.564977in}{3.194932in}}%
\pgfpathlineto{\pgfqpoint{4.566724in}{3.191943in}}%
\pgfpathlineto{\pgfqpoint{4.567160in}{3.192161in}}%
\pgfpathlineto{\pgfqpoint{4.568034in}{3.191589in}}%
\pgfpathlineto{\pgfqpoint{4.569489in}{3.184976in}}%
\pgfpathlineto{\pgfqpoint{4.571236in}{3.179502in}}%
\pgfpathlineto{\pgfqpoint{4.571527in}{3.179671in}}%
\pgfpathlineto{\pgfqpoint{4.573855in}{3.184388in}}%
\pgfpathlineto{\pgfqpoint{4.575893in}{3.191994in}}%
\pgfpathlineto{\pgfqpoint{4.576766in}{3.191168in}}%
\pgfpathlineto{\pgfqpoint{4.579241in}{3.187798in}}%
\pgfpathlineto{\pgfqpoint{4.580550in}{3.184785in}}%
\pgfpathlineto{\pgfqpoint{4.580987in}{3.185256in}}%
\pgfpathlineto{\pgfqpoint{4.583607in}{3.187727in}}%
\pgfpathlineto{\pgfqpoint{4.585353in}{3.188597in}}%
\pgfpathlineto{\pgfqpoint{4.585645in}{3.188292in}}%
\pgfpathlineto{\pgfqpoint{4.587537in}{3.187808in}}%
\pgfpathlineto{\pgfqpoint{4.588701in}{3.186978in}}%
\pgfpathlineto{\pgfqpoint{4.589720in}{3.185735in}}%
\pgfpathlineto{\pgfqpoint{4.590156in}{3.186225in}}%
\pgfpathlineto{\pgfqpoint{4.591757in}{3.188345in}}%
\pgfpathlineto{\pgfqpoint{4.592194in}{3.188059in}}%
\pgfpathlineto{\pgfqpoint{4.595833in}{3.183738in}}%
\pgfpathlineto{\pgfqpoint{4.597143in}{3.180981in}}%
\pgfpathlineto{\pgfqpoint{4.597579in}{3.181712in}}%
\pgfpathlineto{\pgfqpoint{4.599035in}{3.190816in}}%
\pgfpathlineto{\pgfqpoint{4.600345in}{3.194957in}}%
\pgfpathlineto{\pgfqpoint{4.600636in}{3.194669in}}%
\pgfpathlineto{\pgfqpoint{4.601800in}{3.193678in}}%
\pgfpathlineto{\pgfqpoint{4.602237in}{3.194180in}}%
\pgfpathlineto{\pgfqpoint{4.602819in}{3.194550in}}%
\pgfpathlineto{\pgfqpoint{4.603256in}{3.193732in}}%
\pgfpathlineto{\pgfqpoint{4.604857in}{3.183768in}}%
\pgfpathlineto{\pgfqpoint{4.606312in}{3.179020in}}%
\pgfpathlineto{\pgfqpoint{4.606603in}{3.179306in}}%
\pgfpathlineto{\pgfqpoint{4.610096in}{3.186315in}}%
\pgfpathlineto{\pgfqpoint{4.612425in}{3.192701in}}%
\pgfpathlineto{\pgfqpoint{4.614172in}{3.195560in}}%
\pgfpathlineto{\pgfqpoint{4.614754in}{3.194437in}}%
\pgfpathlineto{\pgfqpoint{4.619557in}{3.179881in}}%
\pgfpathlineto{\pgfqpoint{4.619702in}{3.179935in}}%
\pgfpathlineto{\pgfqpoint{4.620867in}{3.182403in}}%
\pgfpathlineto{\pgfqpoint{4.626106in}{3.194328in}}%
\pgfpathlineto{\pgfqpoint{4.626543in}{3.193801in}}%
\pgfpathlineto{\pgfqpoint{4.628581in}{3.187343in}}%
\pgfpathlineto{\pgfqpoint{4.629890in}{3.185184in}}%
\pgfpathlineto{\pgfqpoint{4.630182in}{3.185422in}}%
\pgfpathlineto{\pgfqpoint{4.632074in}{3.186097in}}%
\pgfpathlineto{\pgfqpoint{4.634257in}{3.185415in}}%
\pgfpathlineto{\pgfqpoint{4.634402in}{3.185623in}}%
\pgfpathlineto{\pgfqpoint{4.636731in}{3.187659in}}%
\pgfpathlineto{\pgfqpoint{4.638478in}{3.188036in}}%
\pgfpathlineto{\pgfqpoint{4.639788in}{3.187807in}}%
\pgfpathlineto{\pgfqpoint{4.640806in}{3.187605in}}%
\pgfpathlineto{\pgfqpoint{4.641097in}{3.188068in}}%
\pgfpathlineto{\pgfqpoint{4.642407in}{3.190393in}}%
\pgfpathlineto{\pgfqpoint{4.642844in}{3.189787in}}%
\pgfpathlineto{\pgfqpoint{4.644736in}{3.182343in}}%
\pgfpathlineto{\pgfqpoint{4.645464in}{3.184858in}}%
\pgfpathlineto{\pgfqpoint{4.646628in}{3.188961in}}%
\pgfpathlineto{\pgfqpoint{4.647065in}{3.188118in}}%
\pgfpathlineto{\pgfqpoint{4.648375in}{3.184093in}}%
\pgfpathlineto{\pgfqpoint{4.648811in}{3.184905in}}%
\pgfpathlineto{\pgfqpoint{4.650703in}{3.191628in}}%
\pgfpathlineto{\pgfqpoint{4.651286in}{3.190810in}}%
\pgfpathlineto{\pgfqpoint{4.653469in}{3.188709in}}%
\pgfpathlineto{\pgfqpoint{4.654924in}{3.187155in}}%
\pgfpathlineto{\pgfqpoint{4.657253in}{3.184444in}}%
\pgfpathlineto{\pgfqpoint{4.657544in}{3.184669in}}%
\pgfpathlineto{\pgfqpoint{4.660455in}{3.188929in}}%
\pgfpathlineto{\pgfqpoint{4.661910in}{3.193047in}}%
\pgfpathlineto{\pgfqpoint{4.662347in}{3.192356in}}%
\pgfpathlineto{\pgfqpoint{4.664239in}{3.186693in}}%
\pgfpathlineto{\pgfqpoint{4.665113in}{3.187142in}}%
\pgfpathlineto{\pgfqpoint{4.665986in}{3.184883in}}%
\pgfpathlineto{\pgfqpoint{4.667441in}{3.181598in}}%
\pgfpathlineto{\pgfqpoint{4.667878in}{3.181909in}}%
\pgfpathlineto{\pgfqpoint{4.669479in}{3.185956in}}%
\pgfpathlineto{\pgfqpoint{4.671808in}{3.193543in}}%
\pgfpathlineto{\pgfqpoint{4.672535in}{3.192337in}}%
\pgfpathlineto{\pgfqpoint{4.674573in}{3.189845in}}%
\pgfpathlineto{\pgfqpoint{4.675592in}{3.188450in}}%
\pgfpathlineto{\pgfqpoint{4.677047in}{3.185257in}}%
\pgfpathlineto{\pgfqpoint{4.677629in}{3.185987in}}%
\pgfpathlineto{\pgfqpoint{4.678648in}{3.187537in}}%
\pgfpathlineto{\pgfqpoint{4.679085in}{3.186613in}}%
\pgfpathlineto{\pgfqpoint{4.680686in}{3.181510in}}%
\pgfpathlineto{\pgfqpoint{4.681268in}{3.182292in}}%
\pgfpathlineto{\pgfqpoint{4.684325in}{3.186962in}}%
\pgfpathlineto{\pgfqpoint{4.685489in}{3.189455in}}%
\pgfpathlineto{\pgfqpoint{4.687235in}{3.193682in}}%
\pgfpathlineto{\pgfqpoint{4.687672in}{3.193017in}}%
\pgfpathlineto{\pgfqpoint{4.692475in}{3.183061in}}%
\pgfpathlineto{\pgfqpoint{4.692766in}{3.183355in}}%
\pgfpathlineto{\pgfqpoint{4.694658in}{3.186485in}}%
\pgfpathlineto{\pgfqpoint{4.695386in}{3.185729in}}%
\pgfpathlineto{\pgfqpoint{4.696259in}{3.185591in}}%
\pgfpathlineto{\pgfqpoint{4.696550in}{3.185990in}}%
\pgfpathlineto{\pgfqpoint{4.699170in}{3.191402in}}%
\pgfpathlineto{\pgfqpoint{4.699898in}{3.190337in}}%
\pgfpathlineto{\pgfqpoint{4.702663in}{3.183954in}}%
\pgfpathlineto{\pgfqpoint{4.703391in}{3.184514in}}%
\pgfpathlineto{\pgfqpoint{4.705138in}{3.187253in}}%
\pgfpathlineto{\pgfqpoint{4.706302in}{3.188692in}}%
\pgfpathlineto{\pgfqpoint{4.706593in}{3.188390in}}%
\pgfpathlineto{\pgfqpoint{4.708048in}{3.186649in}}%
\pgfpathlineto{\pgfqpoint{4.708485in}{3.187223in}}%
\pgfpathlineto{\pgfqpoint{4.710523in}{3.193008in}}%
\pgfpathlineto{\pgfqpoint{4.711250in}{3.191800in}}%
\pgfpathlineto{\pgfqpoint{4.716636in}{3.180287in}}%
\pgfpathlineto{\pgfqpoint{4.717363in}{3.180324in}}%
\pgfpathlineto{\pgfqpoint{4.717654in}{3.180759in}}%
\pgfpathlineto{\pgfqpoint{4.720857in}{3.188871in}}%
\pgfpathlineto{\pgfqpoint{4.723040in}{3.199182in}}%
\pgfpathlineto{\pgfqpoint{4.723476in}{3.198549in}}%
\pgfpathlineto{\pgfqpoint{4.725223in}{3.190161in}}%
\pgfpathlineto{\pgfqpoint{4.728570in}{3.179663in}}%
\pgfpathlineto{\pgfqpoint{4.729589in}{3.178827in}}%
\pgfpathlineto{\pgfqpoint{4.730026in}{3.179389in}}%
\pgfpathlineto{\pgfqpoint{4.731772in}{3.186296in}}%
\pgfpathlineto{\pgfqpoint{4.734247in}{3.192638in}}%
\pgfpathlineto{\pgfqpoint{4.736430in}{3.191817in}}%
\pgfpathlineto{\pgfqpoint{4.741378in}{3.179779in}}%
\pgfpathlineto{\pgfqpoint{4.742252in}{3.181851in}}%
\pgfpathlineto{\pgfqpoint{4.745308in}{3.188942in}}%
\pgfpathlineto{\pgfqpoint{4.747200in}{3.190775in}}%
\pgfpathlineto{\pgfqpoint{4.748510in}{3.192990in}}%
\pgfpathlineto{\pgfqpoint{4.748801in}{3.192552in}}%
\pgfpathlineto{\pgfqpoint{4.750839in}{3.185672in}}%
\pgfpathlineto{\pgfqpoint{4.751858in}{3.187187in}}%
\pgfpathlineto{\pgfqpoint{4.752440in}{3.187314in}}%
\pgfpathlineto{\pgfqpoint{4.752731in}{3.186910in}}%
\pgfpathlineto{\pgfqpoint{4.754186in}{3.184966in}}%
\pgfpathlineto{\pgfqpoint{4.754623in}{3.185417in}}%
\pgfpathlineto{\pgfqpoint{4.756224in}{3.187054in}}%
\pgfpathlineto{\pgfqpoint{4.756661in}{3.186692in}}%
\pgfpathlineto{\pgfqpoint{4.758116in}{3.186263in}}%
\pgfpathlineto{\pgfqpoint{4.758262in}{3.186401in}}%
\pgfpathlineto{\pgfqpoint{4.760299in}{3.186989in}}%
\pgfpathlineto{\pgfqpoint{4.762628in}{3.186966in}}%
\pgfpathlineto{\pgfqpoint{4.765102in}{3.190139in}}%
\pgfpathlineto{\pgfqpoint{4.769032in}{3.188633in}}%
\pgfpathlineto{\pgfqpoint{4.773399in}{3.180683in}}%
\pgfpathlineto{\pgfqpoint{4.773835in}{3.181499in}}%
\pgfpathlineto{\pgfqpoint{4.776892in}{3.191592in}}%
\pgfpathlineto{\pgfqpoint{4.777910in}{3.191097in}}%
\pgfpathlineto{\pgfqpoint{4.780239in}{3.190874in}}%
\pgfpathlineto{\pgfqpoint{4.782277in}{3.185595in}}%
\pgfpathlineto{\pgfqpoint{4.782713in}{3.185909in}}%
\pgfpathlineto{\pgfqpoint{4.783587in}{3.185831in}}%
\pgfpathlineto{\pgfqpoint{4.783732in}{3.185569in}}%
\pgfpathlineto{\pgfqpoint{4.785188in}{3.182542in}}%
\pgfpathlineto{\pgfqpoint{4.785770in}{3.183618in}}%
\pgfpathlineto{\pgfqpoint{4.787516in}{3.187586in}}%
\pgfpathlineto{\pgfqpoint{4.787953in}{3.187323in}}%
\pgfpathlineto{\pgfqpoint{4.789409in}{3.187392in}}%
\pgfpathlineto{\pgfqpoint{4.793484in}{3.190820in}}%
\pgfpathlineto{\pgfqpoint{4.793775in}{3.190284in}}%
\pgfpathlineto{\pgfqpoint{4.796831in}{3.184488in}}%
\pgfpathlineto{\pgfqpoint{4.796977in}{3.184564in}}%
\pgfpathlineto{\pgfqpoint{4.799015in}{3.187372in}}%
\pgfpathlineto{\pgfqpoint{4.801925in}{3.190278in}}%
\pgfpathlineto{\pgfqpoint{4.803090in}{3.188479in}}%
\pgfpathlineto{\pgfqpoint{4.804400in}{3.186425in}}%
\pgfpathlineto{\pgfqpoint{4.804836in}{3.186899in}}%
\pgfpathlineto{\pgfqpoint{4.806001in}{3.187931in}}%
\pgfpathlineto{\pgfqpoint{4.806292in}{3.187561in}}%
\pgfpathlineto{\pgfqpoint{4.809639in}{3.183461in}}%
\pgfpathlineto{\pgfqpoint{4.810949in}{3.184418in}}%
\pgfpathlineto{\pgfqpoint{4.815898in}{3.194000in}}%
\pgfpathlineto{\pgfqpoint{4.816480in}{3.193427in}}%
\pgfpathlineto{\pgfqpoint{4.817790in}{3.188956in}}%
\pgfpathlineto{\pgfqpoint{4.820701in}{3.180348in}}%
\pgfpathlineto{\pgfqpoint{4.821720in}{3.179706in}}%
\pgfpathlineto{\pgfqpoint{4.822011in}{3.180117in}}%
\pgfpathlineto{\pgfqpoint{4.823466in}{3.185932in}}%
\pgfpathlineto{\pgfqpoint{4.825358in}{3.190539in}}%
\pgfpathlineto{\pgfqpoint{4.825504in}{3.190489in}}%
\pgfpathlineto{\pgfqpoint{4.826668in}{3.191111in}}%
\pgfpathlineto{\pgfqpoint{4.828124in}{3.193002in}}%
\pgfpathlineto{\pgfqpoint{4.828560in}{3.192457in}}%
\pgfpathlineto{\pgfqpoint{4.833800in}{3.180805in}}%
\pgfpathlineto{\pgfqpoint{4.834528in}{3.182257in}}%
\pgfpathlineto{\pgfqpoint{4.838457in}{3.191825in}}%
\pgfpathlineto{\pgfqpoint{4.838749in}{3.191576in}}%
\pgfpathlineto{\pgfqpoint{4.843697in}{3.183939in}}%
\pgfpathlineto{\pgfqpoint{4.844279in}{3.184738in}}%
\pgfpathlineto{\pgfqpoint{4.847627in}{3.189864in}}%
\pgfpathlineto{\pgfqpoint{4.847918in}{3.189553in}}%
\pgfpathlineto{\pgfqpoint{4.849956in}{3.186445in}}%
\pgfpathlineto{\pgfqpoint{4.850392in}{3.187155in}}%
\pgfpathlineto{\pgfqpoint{4.852139in}{3.190889in}}%
\pgfpathlineto{\pgfqpoint{4.852575in}{3.190292in}}%
\pgfpathlineto{\pgfqpoint{4.856068in}{3.184739in}}%
\pgfpathlineto{\pgfqpoint{4.858688in}{3.185118in}}%
\pgfpathlineto{\pgfqpoint{4.860726in}{3.187447in}}%
\pgfpathlineto{\pgfqpoint{4.862181in}{3.189893in}}%
\pgfpathlineto{\pgfqpoint{4.864073in}{3.191822in}}%
\pgfpathlineto{\pgfqpoint{4.864219in}{3.191739in}}%
\pgfpathlineto{\pgfqpoint{4.866257in}{3.189029in}}%
\pgfpathlineto{\pgfqpoint{4.868294in}{3.187202in}}%
\pgfpathlineto{\pgfqpoint{4.870041in}{3.184974in}}%
\pgfpathlineto{\pgfqpoint{4.871351in}{3.184701in}}%
\pgfpathlineto{\pgfqpoint{4.871496in}{3.184841in}}%
\pgfpathlineto{\pgfqpoint{4.872806in}{3.185554in}}%
\pgfpathlineto{\pgfqpoint{4.873097in}{3.185197in}}%
\pgfpathlineto{\pgfqpoint{4.874407in}{3.183449in}}%
\pgfpathlineto{\pgfqpoint{4.874844in}{3.184048in}}%
\pgfpathlineto{\pgfqpoint{4.877027in}{3.190656in}}%
\pgfpathlineto{\pgfqpoint{4.877755in}{3.189397in}}%
\pgfpathlineto{\pgfqpoint{4.878919in}{3.187923in}}%
\pgfpathlineto{\pgfqpoint{4.879356in}{3.188346in}}%
\pgfpathlineto{\pgfqpoint{4.880520in}{3.189398in}}%
\pgfpathlineto{\pgfqpoint{4.880957in}{3.188961in}}%
\pgfpathlineto{\pgfqpoint{4.882121in}{3.188288in}}%
\pgfpathlineto{\pgfqpoint{4.882412in}{3.188602in}}%
\pgfpathlineto{\pgfqpoint{4.884159in}{3.189108in}}%
\pgfpathlineto{\pgfqpoint{4.889107in}{3.184321in}}%
\pgfpathlineto{\pgfqpoint{4.891145in}{3.182554in}}%
\pgfpathlineto{\pgfqpoint{4.891436in}{3.182795in}}%
\pgfpathlineto{\pgfqpoint{4.893474in}{3.186754in}}%
\pgfpathlineto{\pgfqpoint{4.896676in}{3.194684in}}%
\pgfpathlineto{\pgfqpoint{4.897258in}{3.194197in}}%
\pgfpathlineto{\pgfqpoint{4.899732in}{3.189987in}}%
\pgfpathlineto{\pgfqpoint{4.902934in}{3.179452in}}%
\pgfpathlineto{\pgfqpoint{4.903371in}{3.179670in}}%
\pgfpathlineto{\pgfqpoint{4.905700in}{3.182352in}}%
\pgfpathlineto{\pgfqpoint{4.909047in}{3.195581in}}%
\pgfpathlineto{\pgfqpoint{4.910066in}{3.194139in}}%
\pgfpathlineto{\pgfqpoint{4.914432in}{3.184359in}}%
\pgfpathlineto{\pgfqpoint{4.915451in}{3.185111in}}%
\pgfpathlineto{\pgfqpoint{4.921273in}{3.188276in}}%
\pgfpathlineto{\pgfqpoint{4.923311in}{3.190741in}}%
\pgfpathlineto{\pgfqpoint{4.925057in}{3.189389in}}%
\pgfpathlineto{\pgfqpoint{4.930297in}{3.184690in}}%
\pgfpathlineto{\pgfqpoint{4.931607in}{3.185672in}}%
\pgfpathlineto{\pgfqpoint{4.933353in}{3.186886in}}%
\pgfpathlineto{\pgfqpoint{4.933644in}{3.186586in}}%
\pgfpathlineto{\pgfqpoint{4.934809in}{3.185731in}}%
\pgfpathlineto{\pgfqpoint{4.935100in}{3.186186in}}%
\pgfpathlineto{\pgfqpoint{4.936992in}{3.189698in}}%
\pgfpathlineto{\pgfqpoint{4.937574in}{3.189122in}}%
\pgfpathlineto{\pgfqpoint{4.938738in}{3.188357in}}%
\pgfpathlineto{\pgfqpoint{4.939030in}{3.188658in}}%
\pgfpathlineto{\pgfqpoint{4.940339in}{3.189958in}}%
\pgfpathlineto{\pgfqpoint{4.940776in}{3.189426in}}%
\pgfpathlineto{\pgfqpoint{4.943833in}{3.185882in}}%
\pgfpathlineto{\pgfqpoint{4.945725in}{3.187243in}}%
\pgfpathlineto{\pgfqpoint{4.947617in}{3.188401in}}%
\pgfpathlineto{\pgfqpoint{4.949363in}{3.186233in}}%
\pgfpathlineto{\pgfqpoint{4.951255in}{3.185014in}}%
\pgfpathlineto{\pgfqpoint{4.955476in}{3.189541in}}%
\pgfpathlineto{\pgfqpoint{4.957223in}{3.192711in}}%
\pgfpathlineto{\pgfqpoint{4.957659in}{3.192210in}}%
\pgfpathlineto{\pgfqpoint{4.963772in}{3.180601in}}%
\pgfpathlineto{\pgfqpoint{4.963918in}{3.180673in}}%
\pgfpathlineto{\pgfqpoint{4.964791in}{3.183423in}}%
\pgfpathlineto{\pgfqpoint{4.967120in}{3.189338in}}%
\pgfpathlineto{\pgfqpoint{4.968866in}{3.191890in}}%
\pgfpathlineto{\pgfqpoint{4.970758in}{3.193451in}}%
\pgfpathlineto{\pgfqpoint{4.970904in}{3.193351in}}%
\pgfpathlineto{\pgfqpoint{4.971923in}{3.190836in}}%
\pgfpathlineto{\pgfqpoint{4.975707in}{3.180504in}}%
\pgfpathlineto{\pgfqpoint{4.976580in}{3.182448in}}%
\pgfpathlineto{\pgfqpoint{4.978472in}{3.187448in}}%
\pgfpathlineto{\pgfqpoint{4.978909in}{3.187110in}}%
\pgfpathlineto{\pgfqpoint{4.980073in}{3.186280in}}%
\pgfpathlineto{\pgfqpoint{4.980510in}{3.186866in}}%
\pgfpathlineto{\pgfqpoint{4.983275in}{3.193034in}}%
\pgfpathlineto{\pgfqpoint{4.984003in}{3.192060in}}%
\pgfpathlineto{\pgfqpoint{4.987933in}{3.183000in}}%
\pgfpathlineto{\pgfqpoint{4.988806in}{3.184622in}}%
\pgfpathlineto{\pgfqpoint{4.990116in}{3.186625in}}%
\pgfpathlineto{\pgfqpoint{4.990407in}{3.186316in}}%
\pgfpathlineto{\pgfqpoint{4.991717in}{3.184964in}}%
\pgfpathlineto{\pgfqpoint{4.992008in}{3.185396in}}%
\pgfpathlineto{\pgfqpoint{4.994337in}{3.189572in}}%
\pgfpathlineto{\pgfqpoint{4.994774in}{3.189309in}}%
\pgfpathlineto{\pgfqpoint{4.998703in}{3.187457in}}%
\pgfpathlineto{\pgfqpoint{5.000886in}{3.187375in}}%
\pgfpathlineto{\pgfqpoint{5.002487in}{3.187870in}}%
\pgfpathlineto{\pgfqpoint{5.004380in}{3.191464in}}%
\pgfpathlineto{\pgfqpoint{5.004816in}{3.190789in}}%
\pgfpathlineto{\pgfqpoint{5.009474in}{3.181030in}}%
\pgfpathlineto{\pgfqpoint{5.009765in}{3.181299in}}%
\pgfpathlineto{\pgfqpoint{5.011366in}{3.185814in}}%
\pgfpathlineto{\pgfqpoint{5.015441in}{3.195182in}}%
\pgfpathlineto{\pgfqpoint{5.015587in}{3.195087in}}%
\pgfpathlineto{\pgfqpoint{5.016751in}{3.192562in}}%
\pgfpathlineto{\pgfqpoint{5.020681in}{3.181796in}}%
\pgfpathlineto{\pgfqpoint{5.020972in}{3.181910in}}%
\pgfpathlineto{\pgfqpoint{5.023155in}{3.183515in}}%
\pgfpathlineto{\pgfqpoint{5.026211in}{3.189959in}}%
\pgfpathlineto{\pgfqpoint{5.028686in}{3.191543in}}%
\pgfpathlineto{\pgfqpoint{5.028977in}{3.191119in}}%
\pgfpathlineto{\pgfqpoint{5.031742in}{3.187944in}}%
\pgfpathlineto{\pgfqpoint{5.032906in}{3.187033in}}%
\pgfpathlineto{\pgfqpoint{5.035235in}{3.183603in}}%
\pgfpathlineto{\pgfqpoint{5.035672in}{3.183913in}}%
\pgfpathlineto{\pgfqpoint{5.038728in}{3.187688in}}%
\pgfpathlineto{\pgfqpoint{5.040475in}{3.192745in}}%
\pgfpathlineto{\pgfqpoint{5.040911in}{3.192384in}}%
\pgfpathlineto{\pgfqpoint{5.042804in}{3.189662in}}%
\pgfpathlineto{\pgfqpoint{5.043386in}{3.190056in}}%
\pgfpathlineto{\pgfqpoint{5.044113in}{3.189462in}}%
\pgfpathlineto{\pgfqpoint{5.046733in}{3.183213in}}%
\pgfpathlineto{\pgfqpoint{5.047607in}{3.183679in}}%
\pgfpathlineto{\pgfqpoint{5.050372in}{3.185540in}}%
\pgfpathlineto{\pgfqpoint{5.054011in}{3.192267in}}%
\pgfpathlineto{\pgfqpoint{5.055612in}{3.191032in}}%
\pgfpathlineto{\pgfqpoint{5.060706in}{3.183623in}}%
\pgfpathlineto{\pgfqpoint{5.062307in}{3.184842in}}%
\pgfpathlineto{\pgfqpoint{5.067255in}{3.191257in}}%
\pgfpathlineto{\pgfqpoint{5.067546in}{3.191020in}}%
\pgfpathlineto{\pgfqpoint{5.068856in}{3.187960in}}%
\pgfpathlineto{\pgfqpoint{5.071185in}{3.184455in}}%
\pgfpathlineto{\pgfqpoint{5.072495in}{3.183337in}}%
\pgfpathlineto{\pgfqpoint{5.072786in}{3.183688in}}%
\pgfpathlineto{\pgfqpoint{5.076425in}{3.188269in}}%
\pgfpathlineto{\pgfqpoint{5.081228in}{3.187816in}}%
\pgfpathlineto{\pgfqpoint{5.082392in}{3.187390in}}%
\pgfpathlineto{\pgfqpoint{5.082683in}{3.187702in}}%
\pgfpathlineto{\pgfqpoint{5.084284in}{3.188885in}}%
\pgfpathlineto{\pgfqpoint{5.084575in}{3.188579in}}%
\pgfpathlineto{\pgfqpoint{5.087777in}{3.185923in}}%
\pgfpathlineto{\pgfqpoint{5.090251in}{3.185695in}}%
\pgfpathlineto{\pgfqpoint{5.091416in}{3.184591in}}%
\pgfpathlineto{\pgfqpoint{5.091707in}{3.184944in}}%
\pgfpathlineto{\pgfqpoint{5.094472in}{3.187676in}}%
\pgfpathlineto{\pgfqpoint{5.095782in}{3.189172in}}%
\pgfpathlineto{\pgfqpoint{5.097383in}{3.191657in}}%
\pgfpathlineto{\pgfqpoint{5.097820in}{3.191221in}}%
\pgfpathlineto{\pgfqpoint{5.102914in}{3.184475in}}%
\pgfpathlineto{\pgfqpoint{5.104078in}{3.185733in}}%
\pgfpathlineto{\pgfqpoint{5.108154in}{3.189585in}}%
\pgfpathlineto{\pgfqpoint{5.109755in}{3.188349in}}%
\pgfpathlineto{\pgfqpoint{5.112520in}{3.183528in}}%
\pgfpathlineto{\pgfqpoint{5.113248in}{3.184819in}}%
\pgfpathlineto{\pgfqpoint{5.116595in}{3.190321in}}%
\pgfpathlineto{\pgfqpoint{5.117905in}{3.191226in}}%
\pgfpathlineto{\pgfqpoint{5.118196in}{3.190895in}}%
\pgfpathlineto{\pgfqpoint{5.123727in}{3.183873in}}%
\pgfpathlineto{\pgfqpoint{5.125328in}{3.184937in}}%
\pgfpathlineto{\pgfqpoint{5.127220in}{3.188822in}}%
\pgfpathlineto{\pgfqpoint{5.128384in}{3.189944in}}%
\pgfpathlineto{\pgfqpoint{5.128676in}{3.189723in}}%
\pgfpathlineto{\pgfqpoint{5.129985in}{3.188985in}}%
\pgfpathlineto{\pgfqpoint{5.130277in}{3.189319in}}%
\pgfpathlineto{\pgfqpoint{5.131732in}{3.191099in}}%
\pgfpathlineto{\pgfqpoint{5.132169in}{3.190597in}}%
\pgfpathlineto{\pgfqpoint{5.136972in}{3.183259in}}%
\pgfpathlineto{\pgfqpoint{5.137117in}{3.183351in}}%
\pgfpathlineto{\pgfqpoint{5.139446in}{3.186876in}}%
\pgfpathlineto{\pgfqpoint{5.142066in}{3.189822in}}%
\pgfpathlineto{\pgfqpoint{5.143958in}{3.189217in}}%
\pgfpathlineto{\pgfqpoint{5.146287in}{3.185954in}}%
\pgfpathlineto{\pgfqpoint{5.146869in}{3.186543in}}%
\pgfpathlineto{\pgfqpoint{5.149343in}{3.188345in}}%
\pgfpathlineto{\pgfqpoint{5.153127in}{3.187129in}}%
\pgfpathlineto{\pgfqpoint{5.155747in}{3.183514in}}%
\pgfpathlineto{\pgfqpoint{5.156329in}{3.184221in}}%
\pgfpathlineto{\pgfqpoint{5.160259in}{3.190843in}}%
\pgfpathlineto{\pgfqpoint{5.161569in}{3.192477in}}%
\pgfpathlineto{\pgfqpoint{5.161860in}{3.192296in}}%
\pgfpathlineto{\pgfqpoint{5.165353in}{3.187871in}}%
\pgfpathlineto{\pgfqpoint{5.168264in}{3.182690in}}%
\pgfpathlineto{\pgfqpoint{5.169719in}{3.183215in}}%
\pgfpathlineto{\pgfqpoint{5.174959in}{3.192031in}}%
\pgfpathlineto{\pgfqpoint{5.175687in}{3.190975in}}%
\pgfpathlineto{\pgfqpoint{5.178743in}{3.186446in}}%
\pgfpathlineto{\pgfqpoint{5.179908in}{3.185672in}}%
\pgfpathlineto{\pgfqpoint{5.181800in}{3.183671in}}%
\pgfpathlineto{\pgfqpoint{5.182091in}{3.183894in}}%
\pgfpathlineto{\pgfqpoint{5.184565in}{3.188037in}}%
\pgfpathlineto{\pgfqpoint{5.186894in}{3.190602in}}%
\pgfpathlineto{\pgfqpoint{5.189368in}{3.189890in}}%
\pgfpathlineto{\pgfqpoint{5.190824in}{3.186738in}}%
\pgfpathlineto{\pgfqpoint{5.192133in}{3.184329in}}%
\pgfpathlineto{\pgfqpoint{5.192570in}{3.184843in}}%
\pgfpathlineto{\pgfqpoint{5.194608in}{3.188313in}}%
\pgfpathlineto{\pgfqpoint{5.195044in}{3.187969in}}%
\pgfpathlineto{\pgfqpoint{5.198683in}{3.185101in}}%
\pgfpathlineto{\pgfqpoint{5.207125in}{3.189955in}}%
\pgfpathlineto{\pgfqpoint{5.213529in}{3.184331in}}%
\pgfpathlineto{\pgfqpoint{5.214111in}{3.185376in}}%
\pgfpathlineto{\pgfqpoint{5.216003in}{3.188203in}}%
\pgfpathlineto{\pgfqpoint{5.216294in}{3.188039in}}%
\pgfpathlineto{\pgfqpoint{5.218623in}{3.187312in}}%
\pgfpathlineto{\pgfqpoint{5.220369in}{3.188264in}}%
\pgfpathlineto{\pgfqpoint{5.220660in}{3.187947in}}%
\pgfpathlineto{\pgfqpoint{5.222261in}{3.187048in}}%
\pgfpathlineto{\pgfqpoint{5.222407in}{3.187187in}}%
\pgfpathlineto{\pgfqpoint{5.225609in}{3.189280in}}%
\pgfpathlineto{\pgfqpoint{5.226919in}{3.187825in}}%
\pgfpathlineto{\pgfqpoint{5.228957in}{3.185725in}}%
\pgfpathlineto{\pgfqpoint{5.230412in}{3.185079in}}%
\pgfpathlineto{\pgfqpoint{5.231722in}{3.184268in}}%
\pgfpathlineto{\pgfqpoint{5.232013in}{3.184650in}}%
\pgfpathlineto{\pgfqpoint{5.235943in}{3.190352in}}%
\pgfpathlineto{\pgfqpoint{5.241037in}{3.185746in}}%
\pgfpathlineto{\pgfqpoint{5.242638in}{3.184873in}}%
\pgfpathlineto{\pgfqpoint{5.243802in}{3.183872in}}%
\pgfpathlineto{\pgfqpoint{5.244821in}{3.183098in}}%
\pgfpathlineto{\pgfqpoint{5.245112in}{3.183431in}}%
\pgfpathlineto{\pgfqpoint{5.247004in}{3.189455in}}%
\pgfpathlineto{\pgfqpoint{5.249770in}{3.193429in}}%
\pgfpathlineto{\pgfqpoint{5.250643in}{3.193279in}}%
\pgfpathlineto{\pgfqpoint{5.250788in}{3.193054in}}%
\pgfpathlineto{\pgfqpoint{5.252244in}{3.187725in}}%
\pgfpathlineto{\pgfqpoint{5.254573in}{3.181819in}}%
\pgfpathlineto{\pgfqpoint{5.256610in}{3.182053in}}%
\pgfpathlineto{\pgfqpoint{5.259230in}{3.189620in}}%
\pgfpathlineto{\pgfqpoint{5.261122in}{3.193011in}}%
\pgfpathlineto{\pgfqpoint{5.261413in}{3.192855in}}%
\pgfpathlineto{\pgfqpoint{5.264906in}{3.187755in}}%
\pgfpathlineto{\pgfqpoint{5.267381in}{3.183983in}}%
\pgfpathlineto{\pgfqpoint{5.269273in}{3.185391in}}%
\pgfpathlineto{\pgfqpoint{5.272038in}{3.188522in}}%
\pgfpathlineto{\pgfqpoint{5.272329in}{3.188298in}}%
\pgfpathlineto{\pgfqpoint{5.276696in}{3.185880in}}%
\pgfpathlineto{\pgfqpoint{5.278442in}{3.188742in}}%
\pgfpathlineto{\pgfqpoint{5.280625in}{3.190078in}}%
\pgfpathlineto{\pgfqpoint{5.282372in}{3.189239in}}%
\pgfpathlineto{\pgfqpoint{5.287466in}{3.184077in}}%
\pgfpathlineto{\pgfqpoint{5.287903in}{3.184609in}}%
\pgfpathlineto{\pgfqpoint{5.292851in}{3.191798in}}%
\pgfpathlineto{\pgfqpoint{5.293433in}{3.190911in}}%
\pgfpathlineto{\pgfqpoint{5.297654in}{3.183777in}}%
\pgfpathlineto{\pgfqpoint{5.297800in}{3.183870in}}%
\pgfpathlineto{\pgfqpoint{5.299110in}{3.186548in}}%
\pgfpathlineto{\pgfqpoint{5.301293in}{3.189169in}}%
\pgfpathlineto{\pgfqpoint{5.303913in}{3.188128in}}%
\pgfpathlineto{\pgfqpoint{5.305805in}{3.186978in}}%
\pgfpathlineto{\pgfqpoint{5.307988in}{3.187207in}}%
\pgfpathlineto{\pgfqpoint{5.311190in}{3.184515in}}%
\pgfpathlineto{\pgfqpoint{5.312354in}{3.185485in}}%
\pgfpathlineto{\pgfqpoint{5.317885in}{3.192410in}}%
\pgfpathlineto{\pgfqpoint{5.318904in}{3.191172in}}%
\pgfpathlineto{\pgfqpoint{5.322251in}{3.180799in}}%
\pgfpathlineto{\pgfqpoint{5.323125in}{3.181675in}}%
\pgfpathlineto{\pgfqpoint{5.326181in}{3.188346in}}%
\pgfpathlineto{\pgfqpoint{5.328946in}{3.192993in}}%
\pgfpathlineto{\pgfqpoint{5.330256in}{3.192174in}}%
\pgfpathlineto{\pgfqpoint{5.335205in}{3.183925in}}%
\pgfpathlineto{\pgfqpoint{5.336951in}{3.186051in}}%
\pgfpathlineto{\pgfqpoint{5.339280in}{3.189876in}}%
\pgfpathlineto{\pgfqpoint{5.339717in}{3.189408in}}%
\pgfpathlineto{\pgfqpoint{5.341900in}{3.188074in}}%
\pgfpathlineto{\pgfqpoint{5.343210in}{3.186801in}}%
\pgfpathlineto{\pgfqpoint{5.344956in}{3.185979in}}%
\pgfpathlineto{\pgfqpoint{5.349759in}{3.188294in}}%
\pgfpathlineto{\pgfqpoint{5.352379in}{3.186603in}}%
\pgfpathlineto{\pgfqpoint{5.354854in}{3.187443in}}%
\pgfpathlineto{\pgfqpoint{5.356600in}{3.188096in}}%
\pgfpathlineto{\pgfqpoint{5.359074in}{3.188168in}}%
\pgfpathlineto{\pgfqpoint{5.360530in}{3.187444in}}%
\pgfpathlineto{\pgfqpoint{5.362131in}{3.187379in}}%
\pgfpathlineto{\pgfqpoint{5.363441in}{3.187760in}}%
\pgfpathlineto{\pgfqpoint{5.363586in}{3.187573in}}%
\pgfpathlineto{\pgfqpoint{5.365624in}{3.184664in}}%
\pgfpathlineto{\pgfqpoint{5.366061in}{3.185284in}}%
\pgfpathlineto{\pgfqpoint{5.368535in}{3.189163in}}%
\pgfpathlineto{\pgfqpoint{5.368826in}{3.189013in}}%
\pgfpathlineto{\pgfqpoint{5.376685in}{3.185741in}}%
\pgfpathlineto{\pgfqpoint{5.376831in}{3.186033in}}%
\pgfpathlineto{\pgfqpoint{5.379596in}{3.191670in}}%
\pgfpathlineto{\pgfqpoint{5.380178in}{3.191136in}}%
\pgfpathlineto{\pgfqpoint{5.387165in}{3.182525in}}%
\pgfpathlineto{\pgfqpoint{5.387601in}{3.182986in}}%
\pgfpathlineto{\pgfqpoint{5.389930in}{3.187750in}}%
\pgfpathlineto{\pgfqpoint{5.392841in}{3.192709in}}%
\pgfpathlineto{\pgfqpoint{5.394005in}{3.191950in}}%
\pgfpathlineto{\pgfqpoint{5.399391in}{3.183899in}}%
\pgfpathlineto{\pgfqpoint{5.399536in}{3.184004in}}%
\pgfpathlineto{\pgfqpoint{5.401574in}{3.187350in}}%
\pgfpathlineto{\pgfqpoint{5.404630in}{3.190440in}}%
\pgfpathlineto{\pgfqpoint{5.405940in}{3.189661in}}%
\pgfpathlineto{\pgfqpoint{5.408705in}{3.186719in}}%
\pgfpathlineto{\pgfqpoint{5.408997in}{3.186925in}}%
\pgfpathlineto{\pgfqpoint{5.410452in}{3.186527in}}%
\pgfpathlineto{\pgfqpoint{5.412781in}{3.185145in}}%
\pgfpathlineto{\pgfqpoint{5.416419in}{3.189048in}}%
\pgfpathlineto{\pgfqpoint{5.417729in}{3.190568in}}%
\pgfpathlineto{\pgfqpoint{5.418166in}{3.190260in}}%
\pgfpathlineto{\pgfqpoint{5.420640in}{3.189236in}}%
\pgfpathlineto{\pgfqpoint{5.421950in}{3.187790in}}%
\pgfpathlineto{\pgfqpoint{5.423842in}{3.184711in}}%
\pgfpathlineto{\pgfqpoint{5.424279in}{3.185100in}}%
\pgfpathlineto{\pgfqpoint{5.427044in}{3.187316in}}%
\pgfpathlineto{\pgfqpoint{5.434176in}{3.187049in}}%
\pgfpathlineto{\pgfqpoint{5.438542in}{3.189319in}}%
\pgfpathlineto{\pgfqpoint{5.441890in}{3.187251in}}%
\pgfpathlineto{\pgfqpoint{5.443491in}{3.186519in}}%
\pgfpathlineto{\pgfqpoint{5.445092in}{3.186658in}}%
\pgfpathlineto{\pgfqpoint{5.448876in}{3.188492in}}%
\pgfpathlineto{\pgfqpoint{5.450768in}{3.187187in}}%
\pgfpathlineto{\pgfqpoint{5.453242in}{3.185748in}}%
\pgfpathlineto{\pgfqpoint{5.456881in}{3.188617in}}%
\pgfpathlineto{\pgfqpoint{5.458628in}{3.190314in}}%
\pgfpathlineto{\pgfqpoint{5.458919in}{3.190008in}}%
\pgfpathlineto{\pgfqpoint{5.462412in}{3.184848in}}%
\pgfpathlineto{\pgfqpoint{5.463140in}{3.185486in}}%
\pgfpathlineto{\pgfqpoint{5.469252in}{3.190549in}}%
\pgfpathlineto{\pgfqpoint{5.469398in}{3.190379in}}%
\pgfpathlineto{\pgfqpoint{5.474929in}{3.184834in}}%
\pgfpathlineto{\pgfqpoint{5.475948in}{3.185595in}}%
\pgfpathlineto{\pgfqpoint{5.478276in}{3.190218in}}%
\pgfpathlineto{\pgfqpoint{5.478858in}{3.189563in}}%
\pgfpathlineto{\pgfqpoint{5.482060in}{3.186729in}}%
\pgfpathlineto{\pgfqpoint{5.483516in}{3.186428in}}%
\pgfpathlineto{\pgfqpoint{5.483661in}{3.186577in}}%
\pgfpathlineto{\pgfqpoint{5.487446in}{3.189196in}}%
\pgfpathlineto{\pgfqpoint{5.488901in}{3.188729in}}%
\pgfpathlineto{\pgfqpoint{5.494577in}{3.184984in}}%
\pgfpathlineto{\pgfqpoint{5.497488in}{3.188750in}}%
\pgfpathlineto{\pgfqpoint{5.499380in}{3.192404in}}%
\pgfpathlineto{\pgfqpoint{5.499672in}{3.192223in}}%
\pgfpathlineto{\pgfqpoint{5.504620in}{3.184958in}}%
\pgfpathlineto{\pgfqpoint{5.506221in}{3.182698in}}%
\pgfpathlineto{\pgfqpoint{5.506658in}{3.183047in}}%
\pgfpathlineto{\pgfqpoint{5.512625in}{3.189756in}}%
\pgfpathlineto{\pgfqpoint{5.515245in}{3.188158in}}%
\pgfpathlineto{\pgfqpoint{5.520485in}{3.185085in}}%
\pgfpathlineto{\pgfqpoint{5.522522in}{3.188249in}}%
\pgfpathlineto{\pgfqpoint{5.524705in}{3.190798in}}%
\pgfpathlineto{\pgfqpoint{5.527616in}{3.188601in}}%
\pgfpathlineto{\pgfqpoint{5.530818in}{3.183461in}}%
\pgfpathlineto{\pgfqpoint{5.532128in}{3.185315in}}%
\pgfpathlineto{\pgfqpoint{5.536058in}{3.191484in}}%
\pgfpathlineto{\pgfqpoint{5.537077in}{3.190012in}}%
\pgfpathlineto{\pgfqpoint{5.540861in}{3.183162in}}%
\pgfpathlineto{\pgfqpoint{5.542462in}{3.183447in}}%
\pgfpathlineto{\pgfqpoint{5.544208in}{3.188598in}}%
\pgfpathlineto{\pgfqpoint{5.546537in}{3.193038in}}%
\pgfpathlineto{\pgfqpoint{5.547993in}{3.192284in}}%
\pgfpathlineto{\pgfqpoint{5.550904in}{3.185447in}}%
\pgfpathlineto{\pgfqpoint{5.553815in}{3.180551in}}%
\pgfpathlineto{\pgfqpoint{5.554979in}{3.182161in}}%
\pgfpathlineto{\pgfqpoint{5.556580in}{3.188640in}}%
\pgfpathlineto{\pgfqpoint{5.558618in}{3.195301in}}%
\pgfpathlineto{\pgfqpoint{5.558763in}{3.195203in}}%
\pgfpathlineto{\pgfqpoint{5.560219in}{3.192045in}}%
\pgfpathlineto{\pgfqpoint{5.564294in}{3.182164in}}%
\pgfpathlineto{\pgfqpoint{5.565022in}{3.182799in}}%
\pgfpathlineto{\pgfqpoint{5.570261in}{3.190377in}}%
\pgfpathlineto{\pgfqpoint{5.570989in}{3.189560in}}%
\pgfpathlineto{\pgfqpoint{5.574482in}{3.186412in}}%
\pgfpathlineto{\pgfqpoint{5.576520in}{3.187262in}}%
\pgfpathlineto{\pgfqpoint{5.578121in}{3.187037in}}%
\pgfpathlineto{\pgfqpoint{5.580158in}{3.186574in}}%
\pgfpathlineto{\pgfqpoint{5.582924in}{3.188199in}}%
\pgfpathlineto{\pgfqpoint{5.585689in}{3.187092in}}%
\pgfpathlineto{\pgfqpoint{5.589473in}{3.187753in}}%
\pgfpathlineto{\pgfqpoint{5.594276in}{3.186698in}}%
\pgfpathlineto{\pgfqpoint{5.597042in}{3.185278in}}%
\pgfpathlineto{\pgfqpoint{5.598934in}{3.186862in}}%
\pgfpathlineto{\pgfqpoint{5.601845in}{3.189645in}}%
\pgfpathlineto{\pgfqpoint{5.601990in}{3.189532in}}%
\pgfpathlineto{\pgfqpoint{5.604756in}{3.188421in}}%
\pgfpathlineto{\pgfqpoint{5.606357in}{3.187014in}}%
\pgfpathlineto{\pgfqpoint{5.609122in}{3.185097in}}%
\pgfpathlineto{\pgfqpoint{5.611451in}{3.187519in}}%
\pgfpathlineto{\pgfqpoint{5.614362in}{3.188503in}}%
\pgfpathlineto{\pgfqpoint{5.617709in}{3.188165in}}%
\pgfpathlineto{\pgfqpoint{5.620766in}{3.188597in}}%
\pgfpathlineto{\pgfqpoint{5.624113in}{3.186304in}}%
\pgfpathlineto{\pgfqpoint{5.625860in}{3.185876in}}%
\pgfpathlineto{\pgfqpoint{5.627461in}{3.186133in}}%
\pgfpathlineto{\pgfqpoint{5.631681in}{3.187972in}}%
\pgfpathlineto{\pgfqpoint{5.633428in}{3.188012in}}%
\pgfpathlineto{\pgfqpoint{5.636048in}{3.188253in}}%
\pgfpathlineto{\pgfqpoint{5.642161in}{3.186123in}}%
\pgfpathlineto{\pgfqpoint{5.647983in}{3.187659in}}%
\pgfpathlineto{\pgfqpoint{5.649729in}{3.185715in}}%
\pgfpathlineto{\pgfqpoint{5.650020in}{3.185933in}}%
\pgfpathlineto{\pgfqpoint{5.652786in}{3.187343in}}%
\pgfpathlineto{\pgfqpoint{5.660645in}{3.186883in}}%
\pgfpathlineto{\pgfqpoint{5.665157in}{3.187031in}}%
\pgfpathlineto{\pgfqpoint{5.667631in}{3.188042in}}%
\pgfpathlineto{\pgfqpoint{5.669378in}{3.188821in}}%
\pgfpathlineto{\pgfqpoint{5.671124in}{3.188995in}}%
\pgfpathlineto{\pgfqpoint{5.675054in}{3.186126in}}%
\pgfpathlineto{\pgfqpoint{5.682914in}{3.188945in}}%
\pgfpathlineto{\pgfqpoint{5.684223in}{3.187765in}}%
\pgfpathlineto{\pgfqpoint{5.686261in}{3.184685in}}%
\pgfpathlineto{\pgfqpoint{5.686552in}{3.184892in}}%
\pgfpathlineto{\pgfqpoint{5.692083in}{3.188972in}}%
\pgfpathlineto{\pgfqpoint{5.696013in}{3.187151in}}%
\pgfpathlineto{\pgfqpoint{5.697905in}{3.187379in}}%
\pgfpathlineto{\pgfqpoint{5.699797in}{3.187343in}}%
\pgfpathlineto{\pgfqpoint{5.701689in}{3.187194in}}%
\pgfpathlineto{\pgfqpoint{5.704891in}{3.187480in}}%
\pgfpathlineto{\pgfqpoint{5.711004in}{3.186995in}}%
\pgfpathlineto{\pgfqpoint{5.713915in}{3.186540in}}%
\pgfpathlineto{\pgfqpoint{5.717845in}{3.189146in}}%
\pgfpathlineto{\pgfqpoint{5.719737in}{3.188841in}}%
\pgfpathlineto{\pgfqpoint{5.724103in}{3.184590in}}%
\pgfpathlineto{\pgfqpoint{5.726141in}{3.187136in}}%
\pgfpathlineto{\pgfqpoint{5.729197in}{3.191027in}}%
\pgfpathlineto{\pgfqpoint{5.731089in}{3.190362in}}%
\pgfpathlineto{\pgfqpoint{5.736038in}{3.182947in}}%
\pgfpathlineto{\pgfqpoint{5.737202in}{3.184602in}}%
\pgfpathlineto{\pgfqpoint{5.740986in}{3.192218in}}%
\pgfpathlineto{\pgfqpoint{5.742296in}{3.191152in}}%
\pgfpathlineto{\pgfqpoint{5.746808in}{3.182427in}}%
\pgfpathlineto{\pgfqpoint{5.747245in}{3.182790in}}%
\pgfpathlineto{\pgfqpoint{5.748700in}{3.186625in}}%
\pgfpathlineto{\pgfqpoint{5.751320in}{3.191957in}}%
\pgfpathlineto{\pgfqpoint{5.752775in}{3.190747in}}%
\pgfpathlineto{\pgfqpoint{5.756996in}{3.182879in}}%
\pgfpathlineto{\pgfqpoint{5.757724in}{3.183953in}}%
\pgfpathlineto{\pgfqpoint{5.762236in}{3.191334in}}%
\pgfpathlineto{\pgfqpoint{5.763255in}{3.190892in}}%
\pgfpathlineto{\pgfqpoint{5.767912in}{3.184542in}}%
\pgfpathlineto{\pgfqpoint{5.768058in}{3.184628in}}%
\pgfpathlineto{\pgfqpoint{5.769659in}{3.187180in}}%
\pgfpathlineto{\pgfqpoint{5.771405in}{3.188297in}}%
\pgfpathlineto{\pgfqpoint{5.775481in}{3.186812in}}%
\pgfpathlineto{\pgfqpoint{5.777227in}{3.186214in}}%
\pgfpathlineto{\pgfqpoint{5.783195in}{3.188184in}}%
\pgfpathlineto{\pgfqpoint{5.788143in}{3.187329in}}%
\pgfpathlineto{\pgfqpoint{5.789890in}{3.187621in}}%
\pgfpathlineto{\pgfqpoint{5.794110in}{3.186777in}}%
\pgfpathlineto{\pgfqpoint{5.796439in}{3.188233in}}%
\pgfpathlineto{\pgfqpoint{5.801242in}{3.186182in}}%
\pgfpathlineto{\pgfqpoint{5.806918in}{3.190163in}}%
\pgfpathlineto{\pgfqpoint{5.807064in}{3.189984in}}%
\pgfpathlineto{\pgfqpoint{5.812886in}{3.184225in}}%
\pgfpathlineto{\pgfqpoint{5.814923in}{3.186203in}}%
\pgfpathlineto{\pgfqpoint{5.818562in}{3.190749in}}%
\pgfpathlineto{\pgfqpoint{5.820163in}{3.189375in}}%
\pgfpathlineto{\pgfqpoint{5.824530in}{3.184970in}}%
\pgfpathlineto{\pgfqpoint{5.826713in}{3.186445in}}%
\pgfpathlineto{\pgfqpoint{5.830351in}{3.188750in}}%
\pgfpathlineto{\pgfqpoint{5.836173in}{3.185740in}}%
\pgfpathlineto{\pgfqpoint{5.837920in}{3.187761in}}%
\pgfpathlineto{\pgfqpoint{5.839666in}{3.188253in}}%
\pgfpathlineto{\pgfqpoint{5.848108in}{3.186593in}}%
\pgfpathlineto{\pgfqpoint{5.850000in}{3.187159in}}%
\pgfpathlineto{\pgfqpoint{5.850000in}{3.187159in}}%
\pgfusepath{stroke}%
\end{pgfscope}%
\begin{pgfscope}%
\pgfsetrectcap%
\pgfsetmiterjoin%
\pgfsetlinewidth{0.803000pt}%
\definecolor{currentstroke}{rgb}{0.737255,0.737255,0.737255}%
\pgfsetstrokecolor{currentstroke}%
\pgfsetdash{}{0pt}%
\pgfpathmoveto{\pgfqpoint{0.610501in}{2.761814in}}%
\pgfpathlineto{\pgfqpoint{0.610501in}{3.703703in}}%
\pgfusepath{stroke}%
\end{pgfscope}%
\begin{pgfscope}%
\pgfsetrectcap%
\pgfsetmiterjoin%
\pgfsetlinewidth{0.803000pt}%
\definecolor{currentstroke}{rgb}{0.737255,0.737255,0.737255}%
\pgfsetstrokecolor{currentstroke}%
\pgfsetdash{}{0pt}%
\pgfpathmoveto{\pgfqpoint{5.850000in}{2.761814in}}%
\pgfpathlineto{\pgfqpoint{5.850000in}{3.703703in}}%
\pgfusepath{stroke}%
\end{pgfscope}%
\begin{pgfscope}%
\pgfsetrectcap%
\pgfsetmiterjoin%
\pgfsetlinewidth{0.803000pt}%
\definecolor{currentstroke}{rgb}{0.737255,0.737255,0.737255}%
\pgfsetstrokecolor{currentstroke}%
\pgfsetdash{}{0pt}%
\pgfpathmoveto{\pgfqpoint{0.610501in}{2.761814in}}%
\pgfpathlineto{\pgfqpoint{5.850000in}{2.761814in}}%
\pgfusepath{stroke}%
\end{pgfscope}%
\begin{pgfscope}%
\pgfsetrectcap%
\pgfsetmiterjoin%
\pgfsetlinewidth{0.803000pt}%
\definecolor{currentstroke}{rgb}{0.737255,0.737255,0.737255}%
\pgfsetstrokecolor{currentstroke}%
\pgfsetdash{}{0pt}%
\pgfpathmoveto{\pgfqpoint{0.610501in}{3.703703in}}%
\pgfpathlineto{\pgfqpoint{5.850000in}{3.703703in}}%
\pgfusepath{stroke}%
\end{pgfscope}%
\begin{pgfscope}%
\pgfsetbuttcap%
\pgfsetmiterjoin%
\definecolor{currentfill}{rgb}{0.933333,0.933333,0.933333}%
\pgfsetfillcolor{currentfill}%
\pgfsetlinewidth{0.000000pt}%
\definecolor{currentstroke}{rgb}{0.000000,0.000000,0.000000}%
\pgfsetstrokecolor{currentstroke}%
\pgfsetstrokeopacity{0.000000}%
\pgfsetdash{}{0pt}%
\pgfpathmoveto{\pgfqpoint{0.610501in}{1.652990in}}%
\pgfpathlineto{\pgfqpoint{5.850000in}{1.652990in}}%
\pgfpathlineto{\pgfqpoint{5.850000in}{2.594880in}}%
\pgfpathlineto{\pgfqpoint{0.610501in}{2.594880in}}%
\pgfpathlineto{\pgfqpoint{0.610501in}{1.652990in}}%
\pgfpathclose%
\pgfusepath{fill}%
\end{pgfscope}%
\begin{pgfscope}%
\pgfpathrectangle{\pgfqpoint{0.610501in}{1.652990in}}{\pgfqpoint{5.239499in}{0.941890in}}%
\pgfusepath{clip}%
\pgfsetbuttcap%
\pgfsetroundjoin%
\pgfsetlinewidth{0.501875pt}%
\definecolor{currentstroke}{rgb}{0.698039,0.698039,0.698039}%
\pgfsetstrokecolor{currentstroke}%
\pgfsetdash{{1.850000pt}{0.800000pt}}{0.000000pt}%
\pgfpathmoveto{\pgfqpoint{0.610501in}{1.652990in}}%
\pgfpathlineto{\pgfqpoint{0.610501in}{2.594880in}}%
\pgfusepath{stroke}%
\end{pgfscope}%
\begin{pgfscope}%
\pgfsetbuttcap%
\pgfsetroundjoin%
\definecolor{currentfill}{rgb}{0.000000,0.000000,0.000000}%
\pgfsetfillcolor{currentfill}%
\pgfsetlinewidth{0.803000pt}%
\definecolor{currentstroke}{rgb}{0.000000,0.000000,0.000000}%
\pgfsetstrokecolor{currentstroke}%
\pgfsetdash{}{0pt}%
\pgfsys@defobject{currentmarker}{\pgfqpoint{0.000000in}{0.000000in}}{\pgfqpoint{0.000000in}{0.048611in}}{%
\pgfpathmoveto{\pgfqpoint{0.000000in}{0.000000in}}%
\pgfpathlineto{\pgfqpoint{0.000000in}{0.048611in}}%
\pgfusepath{stroke,fill}%
}%
\begin{pgfscope}%
\pgfsys@transformshift{0.610501in}{1.652990in}%
\pgfsys@useobject{currentmarker}{}%
\end{pgfscope}%
\end{pgfscope}%
\begin{pgfscope}%
\pgfpathrectangle{\pgfqpoint{0.610501in}{1.652990in}}{\pgfqpoint{5.239499in}{0.941890in}}%
\pgfusepath{clip}%
\pgfsetbuttcap%
\pgfsetroundjoin%
\pgfsetlinewidth{0.501875pt}%
\definecolor{currentstroke}{rgb}{0.698039,0.698039,0.698039}%
\pgfsetstrokecolor{currentstroke}%
\pgfsetdash{{1.850000pt}{0.800000pt}}{0.000000pt}%
\pgfpathmoveto{\pgfqpoint{1.192683in}{1.652990in}}%
\pgfpathlineto{\pgfqpoint{1.192683in}{2.594880in}}%
\pgfusepath{stroke}%
\end{pgfscope}%
\begin{pgfscope}%
\pgfsetbuttcap%
\pgfsetroundjoin%
\definecolor{currentfill}{rgb}{0.000000,0.000000,0.000000}%
\pgfsetfillcolor{currentfill}%
\pgfsetlinewidth{0.803000pt}%
\definecolor{currentstroke}{rgb}{0.000000,0.000000,0.000000}%
\pgfsetstrokecolor{currentstroke}%
\pgfsetdash{}{0pt}%
\pgfsys@defobject{currentmarker}{\pgfqpoint{0.000000in}{0.000000in}}{\pgfqpoint{0.000000in}{0.048611in}}{%
\pgfpathmoveto{\pgfqpoint{0.000000in}{0.000000in}}%
\pgfpathlineto{\pgfqpoint{0.000000in}{0.048611in}}%
\pgfusepath{stroke,fill}%
}%
\begin{pgfscope}%
\pgfsys@transformshift{1.192683in}{1.652990in}%
\pgfsys@useobject{currentmarker}{}%
\end{pgfscope}%
\end{pgfscope}%
\begin{pgfscope}%
\pgfpathrectangle{\pgfqpoint{0.610501in}{1.652990in}}{\pgfqpoint{5.239499in}{0.941890in}}%
\pgfusepath{clip}%
\pgfsetbuttcap%
\pgfsetroundjoin%
\pgfsetlinewidth{0.501875pt}%
\definecolor{currentstroke}{rgb}{0.698039,0.698039,0.698039}%
\pgfsetstrokecolor{currentstroke}%
\pgfsetdash{{1.850000pt}{0.800000pt}}{0.000000pt}%
\pgfpathmoveto{\pgfqpoint{1.774866in}{1.652990in}}%
\pgfpathlineto{\pgfqpoint{1.774866in}{2.594880in}}%
\pgfusepath{stroke}%
\end{pgfscope}%
\begin{pgfscope}%
\pgfsetbuttcap%
\pgfsetroundjoin%
\definecolor{currentfill}{rgb}{0.000000,0.000000,0.000000}%
\pgfsetfillcolor{currentfill}%
\pgfsetlinewidth{0.803000pt}%
\definecolor{currentstroke}{rgb}{0.000000,0.000000,0.000000}%
\pgfsetstrokecolor{currentstroke}%
\pgfsetdash{}{0pt}%
\pgfsys@defobject{currentmarker}{\pgfqpoint{0.000000in}{0.000000in}}{\pgfqpoint{0.000000in}{0.048611in}}{%
\pgfpathmoveto{\pgfqpoint{0.000000in}{0.000000in}}%
\pgfpathlineto{\pgfqpoint{0.000000in}{0.048611in}}%
\pgfusepath{stroke,fill}%
}%
\begin{pgfscope}%
\pgfsys@transformshift{1.774866in}{1.652990in}%
\pgfsys@useobject{currentmarker}{}%
\end{pgfscope}%
\end{pgfscope}%
\begin{pgfscope}%
\pgfpathrectangle{\pgfqpoint{0.610501in}{1.652990in}}{\pgfqpoint{5.239499in}{0.941890in}}%
\pgfusepath{clip}%
\pgfsetbuttcap%
\pgfsetroundjoin%
\pgfsetlinewidth{0.501875pt}%
\definecolor{currentstroke}{rgb}{0.698039,0.698039,0.698039}%
\pgfsetstrokecolor{currentstroke}%
\pgfsetdash{{1.850000pt}{0.800000pt}}{0.000000pt}%
\pgfpathmoveto{\pgfqpoint{2.357049in}{1.652990in}}%
\pgfpathlineto{\pgfqpoint{2.357049in}{2.594880in}}%
\pgfusepath{stroke}%
\end{pgfscope}%
\begin{pgfscope}%
\pgfsetbuttcap%
\pgfsetroundjoin%
\definecolor{currentfill}{rgb}{0.000000,0.000000,0.000000}%
\pgfsetfillcolor{currentfill}%
\pgfsetlinewidth{0.803000pt}%
\definecolor{currentstroke}{rgb}{0.000000,0.000000,0.000000}%
\pgfsetstrokecolor{currentstroke}%
\pgfsetdash{}{0pt}%
\pgfsys@defobject{currentmarker}{\pgfqpoint{0.000000in}{0.000000in}}{\pgfqpoint{0.000000in}{0.048611in}}{%
\pgfpathmoveto{\pgfqpoint{0.000000in}{0.000000in}}%
\pgfpathlineto{\pgfqpoint{0.000000in}{0.048611in}}%
\pgfusepath{stroke,fill}%
}%
\begin{pgfscope}%
\pgfsys@transformshift{2.357049in}{1.652990in}%
\pgfsys@useobject{currentmarker}{}%
\end{pgfscope}%
\end{pgfscope}%
\begin{pgfscope}%
\pgfpathrectangle{\pgfqpoint{0.610501in}{1.652990in}}{\pgfqpoint{5.239499in}{0.941890in}}%
\pgfusepath{clip}%
\pgfsetbuttcap%
\pgfsetroundjoin%
\pgfsetlinewidth{0.501875pt}%
\definecolor{currentstroke}{rgb}{0.698039,0.698039,0.698039}%
\pgfsetstrokecolor{currentstroke}%
\pgfsetdash{{1.850000pt}{0.800000pt}}{0.000000pt}%
\pgfpathmoveto{\pgfqpoint{2.939232in}{1.652990in}}%
\pgfpathlineto{\pgfqpoint{2.939232in}{2.594880in}}%
\pgfusepath{stroke}%
\end{pgfscope}%
\begin{pgfscope}%
\pgfsetbuttcap%
\pgfsetroundjoin%
\definecolor{currentfill}{rgb}{0.000000,0.000000,0.000000}%
\pgfsetfillcolor{currentfill}%
\pgfsetlinewidth{0.803000pt}%
\definecolor{currentstroke}{rgb}{0.000000,0.000000,0.000000}%
\pgfsetstrokecolor{currentstroke}%
\pgfsetdash{}{0pt}%
\pgfsys@defobject{currentmarker}{\pgfqpoint{0.000000in}{0.000000in}}{\pgfqpoint{0.000000in}{0.048611in}}{%
\pgfpathmoveto{\pgfqpoint{0.000000in}{0.000000in}}%
\pgfpathlineto{\pgfqpoint{0.000000in}{0.048611in}}%
\pgfusepath{stroke,fill}%
}%
\begin{pgfscope}%
\pgfsys@transformshift{2.939232in}{1.652990in}%
\pgfsys@useobject{currentmarker}{}%
\end{pgfscope}%
\end{pgfscope}%
\begin{pgfscope}%
\pgfpathrectangle{\pgfqpoint{0.610501in}{1.652990in}}{\pgfqpoint{5.239499in}{0.941890in}}%
\pgfusepath{clip}%
\pgfsetbuttcap%
\pgfsetroundjoin%
\pgfsetlinewidth{0.501875pt}%
\definecolor{currentstroke}{rgb}{0.698039,0.698039,0.698039}%
\pgfsetstrokecolor{currentstroke}%
\pgfsetdash{{1.850000pt}{0.800000pt}}{0.000000pt}%
\pgfpathmoveto{\pgfqpoint{3.521414in}{1.652990in}}%
\pgfpathlineto{\pgfqpoint{3.521414in}{2.594880in}}%
\pgfusepath{stroke}%
\end{pgfscope}%
\begin{pgfscope}%
\pgfsetbuttcap%
\pgfsetroundjoin%
\definecolor{currentfill}{rgb}{0.000000,0.000000,0.000000}%
\pgfsetfillcolor{currentfill}%
\pgfsetlinewidth{0.803000pt}%
\definecolor{currentstroke}{rgb}{0.000000,0.000000,0.000000}%
\pgfsetstrokecolor{currentstroke}%
\pgfsetdash{}{0pt}%
\pgfsys@defobject{currentmarker}{\pgfqpoint{0.000000in}{0.000000in}}{\pgfqpoint{0.000000in}{0.048611in}}{%
\pgfpathmoveto{\pgfqpoint{0.000000in}{0.000000in}}%
\pgfpathlineto{\pgfqpoint{0.000000in}{0.048611in}}%
\pgfusepath{stroke,fill}%
}%
\begin{pgfscope}%
\pgfsys@transformshift{3.521414in}{1.652990in}%
\pgfsys@useobject{currentmarker}{}%
\end{pgfscope}%
\end{pgfscope}%
\begin{pgfscope}%
\pgfpathrectangle{\pgfqpoint{0.610501in}{1.652990in}}{\pgfqpoint{5.239499in}{0.941890in}}%
\pgfusepath{clip}%
\pgfsetbuttcap%
\pgfsetroundjoin%
\pgfsetlinewidth{0.501875pt}%
\definecolor{currentstroke}{rgb}{0.698039,0.698039,0.698039}%
\pgfsetstrokecolor{currentstroke}%
\pgfsetdash{{1.850000pt}{0.800000pt}}{0.000000pt}%
\pgfpathmoveto{\pgfqpoint{4.103597in}{1.652990in}}%
\pgfpathlineto{\pgfqpoint{4.103597in}{2.594880in}}%
\pgfusepath{stroke}%
\end{pgfscope}%
\begin{pgfscope}%
\pgfsetbuttcap%
\pgfsetroundjoin%
\definecolor{currentfill}{rgb}{0.000000,0.000000,0.000000}%
\pgfsetfillcolor{currentfill}%
\pgfsetlinewidth{0.803000pt}%
\definecolor{currentstroke}{rgb}{0.000000,0.000000,0.000000}%
\pgfsetstrokecolor{currentstroke}%
\pgfsetdash{}{0pt}%
\pgfsys@defobject{currentmarker}{\pgfqpoint{0.000000in}{0.000000in}}{\pgfqpoint{0.000000in}{0.048611in}}{%
\pgfpathmoveto{\pgfqpoint{0.000000in}{0.000000in}}%
\pgfpathlineto{\pgfqpoint{0.000000in}{0.048611in}}%
\pgfusepath{stroke,fill}%
}%
\begin{pgfscope}%
\pgfsys@transformshift{4.103597in}{1.652990in}%
\pgfsys@useobject{currentmarker}{}%
\end{pgfscope}%
\end{pgfscope}%
\begin{pgfscope}%
\pgfpathrectangle{\pgfqpoint{0.610501in}{1.652990in}}{\pgfqpoint{5.239499in}{0.941890in}}%
\pgfusepath{clip}%
\pgfsetbuttcap%
\pgfsetroundjoin%
\pgfsetlinewidth{0.501875pt}%
\definecolor{currentstroke}{rgb}{0.698039,0.698039,0.698039}%
\pgfsetstrokecolor{currentstroke}%
\pgfsetdash{{1.850000pt}{0.800000pt}}{0.000000pt}%
\pgfpathmoveto{\pgfqpoint{4.685780in}{1.652990in}}%
\pgfpathlineto{\pgfqpoint{4.685780in}{2.594880in}}%
\pgfusepath{stroke}%
\end{pgfscope}%
\begin{pgfscope}%
\pgfsetbuttcap%
\pgfsetroundjoin%
\definecolor{currentfill}{rgb}{0.000000,0.000000,0.000000}%
\pgfsetfillcolor{currentfill}%
\pgfsetlinewidth{0.803000pt}%
\definecolor{currentstroke}{rgb}{0.000000,0.000000,0.000000}%
\pgfsetstrokecolor{currentstroke}%
\pgfsetdash{}{0pt}%
\pgfsys@defobject{currentmarker}{\pgfqpoint{0.000000in}{0.000000in}}{\pgfqpoint{0.000000in}{0.048611in}}{%
\pgfpathmoveto{\pgfqpoint{0.000000in}{0.000000in}}%
\pgfpathlineto{\pgfqpoint{0.000000in}{0.048611in}}%
\pgfusepath{stroke,fill}%
}%
\begin{pgfscope}%
\pgfsys@transformshift{4.685780in}{1.652990in}%
\pgfsys@useobject{currentmarker}{}%
\end{pgfscope}%
\end{pgfscope}%
\begin{pgfscope}%
\pgfpathrectangle{\pgfqpoint{0.610501in}{1.652990in}}{\pgfqpoint{5.239499in}{0.941890in}}%
\pgfusepath{clip}%
\pgfsetbuttcap%
\pgfsetroundjoin%
\pgfsetlinewidth{0.501875pt}%
\definecolor{currentstroke}{rgb}{0.698039,0.698039,0.698039}%
\pgfsetstrokecolor{currentstroke}%
\pgfsetdash{{1.850000pt}{0.800000pt}}{0.000000pt}%
\pgfpathmoveto{\pgfqpoint{5.267963in}{1.652990in}}%
\pgfpathlineto{\pgfqpoint{5.267963in}{2.594880in}}%
\pgfusepath{stroke}%
\end{pgfscope}%
\begin{pgfscope}%
\pgfsetbuttcap%
\pgfsetroundjoin%
\definecolor{currentfill}{rgb}{0.000000,0.000000,0.000000}%
\pgfsetfillcolor{currentfill}%
\pgfsetlinewidth{0.803000pt}%
\definecolor{currentstroke}{rgb}{0.000000,0.000000,0.000000}%
\pgfsetstrokecolor{currentstroke}%
\pgfsetdash{}{0pt}%
\pgfsys@defobject{currentmarker}{\pgfqpoint{0.000000in}{0.000000in}}{\pgfqpoint{0.000000in}{0.048611in}}{%
\pgfpathmoveto{\pgfqpoint{0.000000in}{0.000000in}}%
\pgfpathlineto{\pgfqpoint{0.000000in}{0.048611in}}%
\pgfusepath{stroke,fill}%
}%
\begin{pgfscope}%
\pgfsys@transformshift{5.267963in}{1.652990in}%
\pgfsys@useobject{currentmarker}{}%
\end{pgfscope}%
\end{pgfscope}%
\begin{pgfscope}%
\pgfpathrectangle{\pgfqpoint{0.610501in}{1.652990in}}{\pgfqpoint{5.239499in}{0.941890in}}%
\pgfusepath{clip}%
\pgfsetbuttcap%
\pgfsetroundjoin%
\pgfsetlinewidth{0.501875pt}%
\definecolor{currentstroke}{rgb}{0.698039,0.698039,0.698039}%
\pgfsetstrokecolor{currentstroke}%
\pgfsetdash{{1.850000pt}{0.800000pt}}{0.000000pt}%
\pgfpathmoveto{\pgfqpoint{0.610501in}{1.722112in}}%
\pgfpathlineto{\pgfqpoint{5.850000in}{1.722112in}}%
\pgfusepath{stroke}%
\end{pgfscope}%
\begin{pgfscope}%
\pgfsetbuttcap%
\pgfsetroundjoin%
\definecolor{currentfill}{rgb}{0.000000,0.000000,0.000000}%
\pgfsetfillcolor{currentfill}%
\pgfsetlinewidth{0.803000pt}%
\definecolor{currentstroke}{rgb}{0.000000,0.000000,0.000000}%
\pgfsetstrokecolor{currentstroke}%
\pgfsetdash{}{0pt}%
\pgfsys@defobject{currentmarker}{\pgfqpoint{0.000000in}{0.000000in}}{\pgfqpoint{0.048611in}{0.000000in}}{%
\pgfpathmoveto{\pgfqpoint{0.000000in}{0.000000in}}%
\pgfpathlineto{\pgfqpoint{0.048611in}{0.000000in}}%
\pgfusepath{stroke,fill}%
}%
\begin{pgfscope}%
\pgfsys@transformshift{0.610501in}{1.722112in}%
\pgfsys@useobject{currentmarker}{}%
\end{pgfscope}%
\end{pgfscope}%
\begin{pgfscope}%
\definecolor{textcolor}{rgb}{0.000000,0.000000,0.000000}%
\pgfsetstrokecolor{textcolor}%
\pgfsetfillcolor{textcolor}%
\pgftext[x=0.492445in, y=1.673918in, left, base]{\color{textcolor}\rmfamily\fontsize{10.000000}{12.000000}\selectfont \(\displaystyle {0}\)}%
\end{pgfscope}%
\begin{pgfscope}%
\pgfpathrectangle{\pgfqpoint{0.610501in}{1.652990in}}{\pgfqpoint{5.239499in}{0.941890in}}%
\pgfusepath{clip}%
\pgfsetbuttcap%
\pgfsetroundjoin%
\pgfsetlinewidth{0.501875pt}%
\definecolor{currentstroke}{rgb}{0.698039,0.698039,0.698039}%
\pgfsetstrokecolor{currentstroke}%
\pgfsetdash{{1.850000pt}{0.800000pt}}{0.000000pt}%
\pgfpathmoveto{\pgfqpoint{0.610501in}{2.201607in}}%
\pgfpathlineto{\pgfqpoint{5.850000in}{2.201607in}}%
\pgfusepath{stroke}%
\end{pgfscope}%
\begin{pgfscope}%
\pgfsetbuttcap%
\pgfsetroundjoin%
\definecolor{currentfill}{rgb}{0.000000,0.000000,0.000000}%
\pgfsetfillcolor{currentfill}%
\pgfsetlinewidth{0.803000pt}%
\definecolor{currentstroke}{rgb}{0.000000,0.000000,0.000000}%
\pgfsetstrokecolor{currentstroke}%
\pgfsetdash{}{0pt}%
\pgfsys@defobject{currentmarker}{\pgfqpoint{0.000000in}{0.000000in}}{\pgfqpoint{0.048611in}{0.000000in}}{%
\pgfpathmoveto{\pgfqpoint{0.000000in}{0.000000in}}%
\pgfpathlineto{\pgfqpoint{0.048611in}{0.000000in}}%
\pgfusepath{stroke,fill}%
}%
\begin{pgfscope}%
\pgfsys@transformshift{0.610501in}{2.201607in}%
\pgfsys@useobject{currentmarker}{}%
\end{pgfscope}%
\end{pgfscope}%
\begin{pgfscope}%
\definecolor{textcolor}{rgb}{0.000000,0.000000,0.000000}%
\pgfsetstrokecolor{textcolor}%
\pgfsetfillcolor{textcolor}%
\pgftext[x=0.492445in, y=2.153412in, left, base]{\color{textcolor}\rmfamily\fontsize{10.000000}{12.000000}\selectfont \(\displaystyle {5}\)}%
\end{pgfscope}%
\begin{pgfscope}%
\definecolor{textcolor}{rgb}{0.000000,0.000000,0.000000}%
\pgfsetstrokecolor{textcolor}%
\pgfsetfillcolor{textcolor}%
\pgftext[x=0.436889in,y=2.123935in,,bottom,rotate=90.000000]{\color{textcolor}\rmfamily\fontsize{12.000000}{14.400000}\selectfont Z-detect}%
\end{pgfscope}%
\begin{pgfscope}%
\pgfpathrectangle{\pgfqpoint{0.610501in}{1.652990in}}{\pgfqpoint{5.239499in}{0.941890in}}%
\pgfusepath{clip}%
\pgfsetrectcap%
\pgfsetroundjoin%
\pgfsetlinewidth{1.505625pt}%
\definecolor{currentstroke}{rgb}{0.121569,0.466667,0.705882}%
\pgfsetstrokecolor{currentstroke}%
\pgfsetdash{}{0pt}%
\pgfpathmoveto{\pgfqpoint{0.610501in}{1.722112in}}%
\pgfpathlineto{\pgfqpoint{0.639464in}{1.722112in}}%
\pgfpathlineto{\pgfqpoint{0.640629in}{1.695803in}}%
\pgfpathlineto{\pgfqpoint{1.963348in}{1.696886in}}%
\pgfpathlineto{\pgfqpoint{1.970188in}{1.698049in}}%
\pgfpathlineto{\pgfqpoint{2.027533in}{1.700576in}}%
\pgfpathlineto{\pgfqpoint{2.030153in}{1.701606in}}%
\pgfpathlineto{\pgfqpoint{2.036557in}{1.702522in}}%
\pgfpathlineto{\pgfqpoint{2.038595in}{1.702474in}}%
\pgfpathlineto{\pgfqpoint{2.041215in}{1.702670in}}%
\pgfpathlineto{\pgfqpoint{2.046745in}{1.702291in}}%
\pgfpathlineto{\pgfqpoint{2.052567in}{1.702608in}}%
\pgfpathlineto{\pgfqpoint{2.061737in}{1.702819in}}%
\pgfpathlineto{\pgfqpoint{2.063483in}{1.703319in}}%
\pgfpathlineto{\pgfqpoint{2.071488in}{1.705616in}}%
\pgfpathlineto{\pgfqpoint{2.073671in}{1.706031in}}%
\pgfpathlineto{\pgfqpoint{2.080512in}{1.706711in}}%
\pgfpathlineto{\pgfqpoint{2.083860in}{1.709710in}}%
\pgfpathlineto{\pgfqpoint{2.087062in}{1.708999in}}%
\pgfpathlineto{\pgfqpoint{2.087935in}{1.709658in}}%
\pgfpathlineto{\pgfqpoint{2.090700in}{1.713148in}}%
\pgfpathlineto{\pgfqpoint{2.092447in}{1.712505in}}%
\pgfpathlineto{\pgfqpoint{2.095358in}{1.712134in}}%
\pgfpathlineto{\pgfqpoint{2.097395in}{1.713084in}}%
\pgfpathlineto{\pgfqpoint{2.100161in}{1.713629in}}%
\pgfpathlineto{\pgfqpoint{2.102198in}{1.712469in}}%
\pgfpathlineto{\pgfqpoint{2.110203in}{1.714578in}}%
\pgfpathlineto{\pgfqpoint{2.113260in}{1.712682in}}%
\pgfpathlineto{\pgfqpoint{2.114861in}{1.714840in}}%
\pgfpathlineto{\pgfqpoint{2.116607in}{1.718136in}}%
\pgfpathlineto{\pgfqpoint{2.117190in}{1.717223in}}%
\pgfpathlineto{\pgfqpoint{2.118499in}{1.715289in}}%
\pgfpathlineto{\pgfqpoint{2.118936in}{1.715649in}}%
\pgfpathlineto{\pgfqpoint{2.121410in}{1.719583in}}%
\pgfpathlineto{\pgfqpoint{2.123157in}{1.720819in}}%
\pgfpathlineto{\pgfqpoint{2.126941in}{1.721531in}}%
\pgfpathlineto{\pgfqpoint{2.129415in}{1.719575in}}%
\pgfpathlineto{\pgfqpoint{2.132326in}{1.719395in}}%
\pgfpathlineto{\pgfqpoint{2.141350in}{1.715453in}}%
\pgfpathlineto{\pgfqpoint{2.142806in}{1.714583in}}%
\pgfpathlineto{\pgfqpoint{2.144698in}{1.710560in}}%
\pgfpathlineto{\pgfqpoint{2.146444in}{1.708756in}}%
\pgfpathlineto{\pgfqpoint{2.148919in}{1.707555in}}%
\pgfpathlineto{\pgfqpoint{2.152994in}{1.703014in}}%
\pgfpathlineto{\pgfqpoint{2.153430in}{1.703503in}}%
\pgfpathlineto{\pgfqpoint{2.155177in}{1.704888in}}%
\pgfpathlineto{\pgfqpoint{2.155468in}{1.704686in}}%
\pgfpathlineto{\pgfqpoint{2.156632in}{1.705188in}}%
\pgfpathlineto{\pgfqpoint{2.158525in}{1.706163in}}%
\pgfpathlineto{\pgfqpoint{2.161144in}{1.706463in}}%
\pgfpathlineto{\pgfqpoint{2.164637in}{1.706774in}}%
\pgfpathlineto{\pgfqpoint{2.176863in}{1.706196in}}%
\pgfpathlineto{\pgfqpoint{2.178610in}{1.706357in}}%
\pgfpathlineto{\pgfqpoint{2.182249in}{1.705732in}}%
\pgfpathlineto{\pgfqpoint{2.184868in}{1.702388in}}%
\pgfpathlineto{\pgfqpoint{2.194620in}{1.699904in}}%
\pgfpathlineto{\pgfqpoint{2.197967in}{1.700826in}}%
\pgfpathlineto{\pgfqpoint{2.212085in}{1.700793in}}%
\pgfpathlineto{\pgfqpoint{2.216161in}{1.700951in}}%
\pgfpathlineto{\pgfqpoint{2.233917in}{1.703353in}}%
\pgfpathlineto{\pgfqpoint{2.236537in}{1.703912in}}%
\pgfpathlineto{\pgfqpoint{2.244397in}{1.703813in}}%
\pgfpathlineto{\pgfqpoint{2.253566in}{1.703608in}}%
\pgfpathlineto{\pgfqpoint{2.256913in}{1.703073in}}%
\pgfpathlineto{\pgfqpoint{2.262153in}{1.702153in}}%
\pgfpathlineto{\pgfqpoint{2.264773in}{1.700784in}}%
\pgfpathlineto{\pgfqpoint{2.268848in}{1.700693in}}%
\pgfpathlineto{\pgfqpoint{2.278454in}{1.699292in}}%
\pgfpathlineto{\pgfqpoint{2.283694in}{1.699087in}}%
\pgfpathlineto{\pgfqpoint{2.289807in}{1.699214in}}%
\pgfpathlineto{\pgfqpoint{2.295920in}{1.698441in}}%
\pgfpathlineto{\pgfqpoint{2.298103in}{1.698226in}}%
\pgfpathlineto{\pgfqpoint{2.301596in}{1.698160in}}%
\pgfpathlineto{\pgfqpoint{2.306108in}{1.698796in}}%
\pgfpathlineto{\pgfqpoint{2.311202in}{1.700261in}}%
\pgfpathlineto{\pgfqpoint{2.313240in}{1.700850in}}%
\pgfpathlineto{\pgfqpoint{2.323428in}{1.702433in}}%
\pgfpathlineto{\pgfqpoint{2.327503in}{1.703694in}}%
\pgfpathlineto{\pgfqpoint{2.329977in}{1.703882in}}%
\pgfpathlineto{\pgfqpoint{2.334635in}{1.704163in}}%
\pgfpathlineto{\pgfqpoint{2.336818in}{1.703928in}}%
\pgfpathlineto{\pgfqpoint{2.339729in}{1.705177in}}%
\pgfpathlineto{\pgfqpoint{2.339874in}{1.704992in}}%
\pgfpathlineto{\pgfqpoint{2.341912in}{1.703997in}}%
\pgfpathlineto{\pgfqpoint{2.346133in}{1.706649in}}%
\pgfpathlineto{\pgfqpoint{2.348753in}{1.709470in}}%
\pgfpathlineto{\pgfqpoint{2.350499in}{1.709893in}}%
\pgfpathlineto{\pgfqpoint{2.354429in}{1.710979in}}%
\pgfpathlineto{\pgfqpoint{2.360542in}{1.712169in}}%
\pgfpathlineto{\pgfqpoint{2.362143in}{1.712781in}}%
\pgfpathlineto{\pgfqpoint{2.364617in}{1.718410in}}%
\pgfpathlineto{\pgfqpoint{2.366655in}{1.725646in}}%
\pgfpathlineto{\pgfqpoint{2.366946in}{1.725409in}}%
\pgfpathlineto{\pgfqpoint{2.367383in}{1.725049in}}%
\pgfpathlineto{\pgfqpoint{2.367674in}{1.725872in}}%
\pgfpathlineto{\pgfqpoint{2.368401in}{1.737856in}}%
\pgfpathlineto{\pgfqpoint{2.370585in}{1.778780in}}%
\pgfpathlineto{\pgfqpoint{2.370876in}{1.778742in}}%
\pgfpathlineto{\pgfqpoint{2.371749in}{1.780587in}}%
\pgfpathlineto{\pgfqpoint{2.373059in}{1.783219in}}%
\pgfpathlineto{\pgfqpoint{2.375097in}{1.800794in}}%
\pgfpathlineto{\pgfqpoint{2.377134in}{1.824342in}}%
\pgfpathlineto{\pgfqpoint{2.380191in}{1.855225in}}%
\pgfpathlineto{\pgfqpoint{2.382083in}{1.878919in}}%
\pgfpathlineto{\pgfqpoint{2.384266in}{1.882171in}}%
\pgfpathlineto{\pgfqpoint{2.387322in}{1.905213in}}%
\pgfpathlineto{\pgfqpoint{2.389214in}{1.912783in}}%
\pgfpathlineto{\pgfqpoint{2.390379in}{1.919165in}}%
\pgfpathlineto{\pgfqpoint{2.390815in}{1.918990in}}%
\pgfpathlineto{\pgfqpoint{2.392562in}{1.918099in}}%
\pgfpathlineto{\pgfqpoint{2.393435in}{1.916672in}}%
\pgfpathlineto{\pgfqpoint{2.393872in}{1.918290in}}%
\pgfpathlineto{\pgfqpoint{2.394891in}{1.923224in}}%
\pgfpathlineto{\pgfqpoint{2.395473in}{1.922040in}}%
\pgfpathlineto{\pgfqpoint{2.397074in}{1.917581in}}%
\pgfpathlineto{\pgfqpoint{2.399112in}{1.873303in}}%
\pgfpathlineto{\pgfqpoint{2.400713in}{1.876431in}}%
\pgfpathlineto{\pgfqpoint{2.402459in}{1.872978in}}%
\pgfpathlineto{\pgfqpoint{2.402896in}{1.874343in}}%
\pgfpathlineto{\pgfqpoint{2.403187in}{1.875660in}}%
\pgfpathlineto{\pgfqpoint{2.403624in}{1.873014in}}%
\pgfpathlineto{\pgfqpoint{2.406534in}{1.841392in}}%
\pgfpathlineto{\pgfqpoint{2.406826in}{1.841087in}}%
\pgfpathlineto{\pgfqpoint{2.407262in}{1.841852in}}%
\pgfpathlineto{\pgfqpoint{2.407990in}{1.843049in}}%
\pgfpathlineto{\pgfqpoint{2.408281in}{1.842047in}}%
\pgfpathlineto{\pgfqpoint{2.409154in}{1.827858in}}%
\pgfpathlineto{\pgfqpoint{2.411046in}{1.807304in}}%
\pgfpathlineto{\pgfqpoint{2.412502in}{1.805779in}}%
\pgfpathlineto{\pgfqpoint{2.412938in}{1.807136in}}%
\pgfpathlineto{\pgfqpoint{2.413084in}{1.807421in}}%
\pgfpathlineto{\pgfqpoint{2.413375in}{1.806624in}}%
\pgfpathlineto{\pgfqpoint{2.414831in}{1.787189in}}%
\pgfpathlineto{\pgfqpoint{2.415995in}{1.790693in}}%
\pgfpathlineto{\pgfqpoint{2.416868in}{1.799441in}}%
\pgfpathlineto{\pgfqpoint{2.417741in}{1.808682in}}%
\pgfpathlineto{\pgfqpoint{2.418324in}{1.805547in}}%
\pgfpathlineto{\pgfqpoint{2.419197in}{1.799964in}}%
\pgfpathlineto{\pgfqpoint{2.419634in}{1.801940in}}%
\pgfpathlineto{\pgfqpoint{2.421235in}{1.807623in}}%
\pgfpathlineto{\pgfqpoint{2.421380in}{1.807534in}}%
\pgfpathlineto{\pgfqpoint{2.422690in}{1.805746in}}%
\pgfpathlineto{\pgfqpoint{2.424000in}{1.794796in}}%
\pgfpathlineto{\pgfqpoint{2.424728in}{1.797569in}}%
\pgfpathlineto{\pgfqpoint{2.428366in}{1.810173in}}%
\pgfpathlineto{\pgfqpoint{2.432587in}{1.806307in}}%
\pgfpathlineto{\pgfqpoint{2.433024in}{1.807423in}}%
\pgfpathlineto{\pgfqpoint{2.433460in}{1.808199in}}%
\pgfpathlineto{\pgfqpoint{2.433751in}{1.807334in}}%
\pgfpathlineto{\pgfqpoint{2.434479in}{1.803347in}}%
\pgfpathlineto{\pgfqpoint{2.434916in}{1.805173in}}%
\pgfpathlineto{\pgfqpoint{2.437681in}{1.832576in}}%
\pgfpathlineto{\pgfqpoint{2.438263in}{1.831064in}}%
\pgfpathlineto{\pgfqpoint{2.440447in}{1.825569in}}%
\pgfpathlineto{\pgfqpoint{2.440883in}{1.826877in}}%
\pgfpathlineto{\pgfqpoint{2.441756in}{1.845084in}}%
\pgfpathlineto{\pgfqpoint{2.443794in}{1.883219in}}%
\pgfpathlineto{\pgfqpoint{2.443940in}{1.883096in}}%
\pgfpathlineto{\pgfqpoint{2.444958in}{1.879172in}}%
\pgfpathlineto{\pgfqpoint{2.446414in}{1.861439in}}%
\pgfpathlineto{\pgfqpoint{2.446996in}{1.857399in}}%
\pgfpathlineto{\pgfqpoint{2.447433in}{1.860551in}}%
\pgfpathlineto{\pgfqpoint{2.450635in}{1.913390in}}%
\pgfpathlineto{\pgfqpoint{2.451508in}{1.913314in}}%
\pgfpathlineto{\pgfqpoint{2.452672in}{1.912714in}}%
\pgfpathlineto{\pgfqpoint{2.453109in}{1.912184in}}%
\pgfpathlineto{\pgfqpoint{2.453400in}{1.913193in}}%
\pgfpathlineto{\pgfqpoint{2.454565in}{1.931662in}}%
\pgfpathlineto{\pgfqpoint{2.456020in}{1.940079in}}%
\pgfpathlineto{\pgfqpoint{2.456166in}{1.939993in}}%
\pgfpathlineto{\pgfqpoint{2.456748in}{1.937129in}}%
\pgfpathlineto{\pgfqpoint{2.458494in}{1.925927in}}%
\pgfpathlineto{\pgfqpoint{2.458931in}{1.926273in}}%
\pgfpathlineto{\pgfqpoint{2.459659in}{1.926618in}}%
\pgfpathlineto{\pgfqpoint{2.460095in}{1.926099in}}%
\pgfpathlineto{\pgfqpoint{2.462570in}{1.923450in}}%
\pgfpathlineto{\pgfqpoint{2.462861in}{1.924132in}}%
\pgfpathlineto{\pgfqpoint{2.463734in}{1.928731in}}%
\pgfpathlineto{\pgfqpoint{2.464171in}{1.925739in}}%
\pgfpathlineto{\pgfqpoint{2.467664in}{1.888885in}}%
\pgfpathlineto{\pgfqpoint{2.468682in}{1.888097in}}%
\pgfpathlineto{\pgfqpoint{2.468974in}{1.888791in}}%
\pgfpathlineto{\pgfqpoint{2.469992in}{1.894931in}}%
\pgfpathlineto{\pgfqpoint{2.470283in}{1.892568in}}%
\pgfpathlineto{\pgfqpoint{2.471739in}{1.857053in}}%
\pgfpathlineto{\pgfqpoint{2.472612in}{1.868634in}}%
\pgfpathlineto{\pgfqpoint{2.473922in}{1.876269in}}%
\pgfpathlineto{\pgfqpoint{2.474213in}{1.876186in}}%
\pgfpathlineto{\pgfqpoint{2.475086in}{1.876703in}}%
\pgfpathlineto{\pgfqpoint{2.475669in}{1.877515in}}%
\pgfpathlineto{\pgfqpoint{2.476105in}{1.876280in}}%
\pgfpathlineto{\pgfqpoint{2.477124in}{1.864169in}}%
\pgfpathlineto{\pgfqpoint{2.479162in}{1.828965in}}%
\pgfpathlineto{\pgfqpoint{2.479744in}{1.830024in}}%
\pgfpathlineto{\pgfqpoint{2.482364in}{1.835063in}}%
\pgfpathlineto{\pgfqpoint{2.482800in}{1.832797in}}%
\pgfpathlineto{\pgfqpoint{2.484692in}{1.811283in}}%
\pgfpathlineto{\pgfqpoint{2.485566in}{1.811987in}}%
\pgfpathlineto{\pgfqpoint{2.488768in}{1.813251in}}%
\pgfpathlineto{\pgfqpoint{2.493571in}{1.839925in}}%
\pgfpathlineto{\pgfqpoint{2.494444in}{1.840560in}}%
\pgfpathlineto{\pgfqpoint{2.495754in}{1.845762in}}%
\pgfpathlineto{\pgfqpoint{2.496918in}{1.872612in}}%
\pgfpathlineto{\pgfqpoint{2.499101in}{1.912347in}}%
\pgfpathlineto{\pgfqpoint{2.500120in}{1.915255in}}%
\pgfpathlineto{\pgfqpoint{2.500411in}{1.913598in}}%
\pgfpathlineto{\pgfqpoint{2.501721in}{1.892894in}}%
\pgfpathlineto{\pgfqpoint{2.502303in}{1.900137in}}%
\pgfpathlineto{\pgfqpoint{2.505505in}{1.971423in}}%
\pgfpathlineto{\pgfqpoint{2.506233in}{1.967198in}}%
\pgfpathlineto{\pgfqpoint{2.507543in}{1.957062in}}%
\pgfpathlineto{\pgfqpoint{2.507980in}{1.958263in}}%
\pgfpathlineto{\pgfqpoint{2.509290in}{1.973676in}}%
\pgfpathlineto{\pgfqpoint{2.511473in}{1.992742in}}%
\pgfpathlineto{\pgfqpoint{2.511618in}{1.992599in}}%
\pgfpathlineto{\pgfqpoint{2.512928in}{1.988047in}}%
\pgfpathlineto{\pgfqpoint{2.513802in}{1.986002in}}%
\pgfpathlineto{\pgfqpoint{2.514238in}{1.986564in}}%
\pgfpathlineto{\pgfqpoint{2.516713in}{1.995794in}}%
\pgfpathlineto{\pgfqpoint{2.519332in}{2.070784in}}%
\pgfpathlineto{\pgfqpoint{2.520642in}{2.087608in}}%
\pgfpathlineto{\pgfqpoint{2.521079in}{2.084452in}}%
\pgfpathlineto{\pgfqpoint{2.522098in}{2.075350in}}%
\pgfpathlineto{\pgfqpoint{2.522534in}{2.079562in}}%
\pgfpathlineto{\pgfqpoint{2.523553in}{2.120712in}}%
\pgfpathlineto{\pgfqpoint{2.525882in}{2.261290in}}%
\pgfpathlineto{\pgfqpoint{2.526464in}{2.252105in}}%
\pgfpathlineto{\pgfqpoint{2.528065in}{2.222717in}}%
\pgfpathlineto{\pgfqpoint{2.528502in}{2.224890in}}%
\pgfpathlineto{\pgfqpoint{2.529521in}{2.251397in}}%
\pgfpathlineto{\pgfqpoint{2.532431in}{2.347974in}}%
\pgfpathlineto{\pgfqpoint{2.532868in}{2.343830in}}%
\pgfpathlineto{\pgfqpoint{2.534324in}{2.319823in}}%
\pgfpathlineto{\pgfqpoint{2.534906in}{2.324869in}}%
\pgfpathlineto{\pgfqpoint{2.536216in}{2.370230in}}%
\pgfpathlineto{\pgfqpoint{2.538544in}{2.429316in}}%
\pgfpathlineto{\pgfqpoint{2.539709in}{2.437029in}}%
\pgfpathlineto{\pgfqpoint{2.540145in}{2.435177in}}%
\pgfpathlineto{\pgfqpoint{2.540873in}{2.430679in}}%
\pgfpathlineto{\pgfqpoint{2.541310in}{2.432851in}}%
\pgfpathlineto{\pgfqpoint{2.542911in}{2.468394in}}%
\pgfpathlineto{\pgfqpoint{2.545239in}{2.539587in}}%
\pgfpathlineto{\pgfqpoint{2.545822in}{2.552066in}}%
\pgfpathlineto{\pgfqpoint{2.546404in}{2.543239in}}%
\pgfpathlineto{\pgfqpoint{2.548441in}{2.490736in}}%
\pgfpathlineto{\pgfqpoint{2.548878in}{2.493273in}}%
\pgfpathlineto{\pgfqpoint{2.550042in}{2.520682in}}%
\pgfpathlineto{\pgfqpoint{2.551352in}{2.549672in}}%
\pgfpathlineto{\pgfqpoint{2.551789in}{2.546631in}}%
\pgfpathlineto{\pgfqpoint{2.552662in}{2.510307in}}%
\pgfpathlineto{\pgfqpoint{2.554845in}{2.372458in}}%
\pgfpathlineto{\pgfqpoint{2.555573in}{2.382756in}}%
\pgfpathlineto{\pgfqpoint{2.556883in}{2.398611in}}%
\pgfpathlineto{\pgfqpoint{2.557320in}{2.396398in}}%
\pgfpathlineto{\pgfqpoint{2.558339in}{2.373516in}}%
\pgfpathlineto{\pgfqpoint{2.560958in}{2.242246in}}%
\pgfpathlineto{\pgfqpoint{2.562414in}{2.206199in}}%
\pgfpathlineto{\pgfqpoint{2.562705in}{2.207085in}}%
\pgfpathlineto{\pgfqpoint{2.563287in}{2.209006in}}%
\pgfpathlineto{\pgfqpoint{2.563724in}{2.207422in}}%
\pgfpathlineto{\pgfqpoint{2.564597in}{2.191097in}}%
\pgfpathlineto{\pgfqpoint{2.568672in}{2.078756in}}%
\pgfpathlineto{\pgfqpoint{2.569109in}{2.081307in}}%
\pgfpathlineto{\pgfqpoint{2.570128in}{2.091203in}}%
\pgfpathlineto{\pgfqpoint{2.570710in}{2.086735in}}%
\pgfpathlineto{\pgfqpoint{2.574058in}{2.003378in}}%
\pgfpathlineto{\pgfqpoint{2.575076in}{1.988617in}}%
\pgfpathlineto{\pgfqpoint{2.575513in}{1.991002in}}%
\pgfpathlineto{\pgfqpoint{2.576241in}{1.995297in}}%
\pgfpathlineto{\pgfqpoint{2.576677in}{1.992425in}}%
\pgfpathlineto{\pgfqpoint{2.578133in}{1.957613in}}%
\pgfpathlineto{\pgfqpoint{2.581189in}{1.894479in}}%
\pgfpathlineto{\pgfqpoint{2.583081in}{1.890312in}}%
\pgfpathlineto{\pgfqpoint{2.584246in}{1.859846in}}%
\pgfpathlineto{\pgfqpoint{2.586283in}{1.829661in}}%
\pgfpathlineto{\pgfqpoint{2.586720in}{1.829605in}}%
\pgfpathlineto{\pgfqpoint{2.587157in}{1.830360in}}%
\pgfpathlineto{\pgfqpoint{2.588175in}{1.831764in}}%
\pgfpathlineto{\pgfqpoint{2.588612in}{1.831350in}}%
\pgfpathlineto{\pgfqpoint{2.589194in}{1.831183in}}%
\pgfpathlineto{\pgfqpoint{2.589485in}{1.831852in}}%
\pgfpathlineto{\pgfqpoint{2.591232in}{1.839530in}}%
\pgfpathlineto{\pgfqpoint{2.591960in}{1.837886in}}%
\pgfpathlineto{\pgfqpoint{2.593270in}{1.832893in}}%
\pgfpathlineto{\pgfqpoint{2.593706in}{1.834333in}}%
\pgfpathlineto{\pgfqpoint{2.594725in}{1.852072in}}%
\pgfpathlineto{\pgfqpoint{2.596763in}{1.899202in}}%
\pgfpathlineto{\pgfqpoint{2.597345in}{1.896124in}}%
\pgfpathlineto{\pgfqpoint{2.599382in}{1.875530in}}%
\pgfpathlineto{\pgfqpoint{2.599819in}{1.878421in}}%
\pgfpathlineto{\pgfqpoint{2.600838in}{1.905972in}}%
\pgfpathlineto{\pgfqpoint{2.603312in}{1.995324in}}%
\pgfpathlineto{\pgfqpoint{2.603894in}{1.990783in}}%
\pgfpathlineto{\pgfqpoint{2.605204in}{1.976644in}}%
\pgfpathlineto{\pgfqpoint{2.605641in}{1.979925in}}%
\pgfpathlineto{\pgfqpoint{2.606514in}{2.010302in}}%
\pgfpathlineto{\pgfqpoint{2.611026in}{2.212669in}}%
\pgfpathlineto{\pgfqpoint{2.612482in}{2.272875in}}%
\pgfpathlineto{\pgfqpoint{2.615392in}{2.355954in}}%
\pgfpathlineto{\pgfqpoint{2.616120in}{2.360911in}}%
\pgfpathlineto{\pgfqpoint{2.616702in}{2.360073in}}%
\pgfpathlineto{\pgfqpoint{2.616993in}{2.359961in}}%
\pgfpathlineto{\pgfqpoint{2.617285in}{2.361023in}}%
\pgfpathlineto{\pgfqpoint{2.618449in}{2.377993in}}%
\pgfpathlineto{\pgfqpoint{2.622233in}{2.443340in}}%
\pgfpathlineto{\pgfqpoint{2.622815in}{2.439588in}}%
\pgfpathlineto{\pgfqpoint{2.625435in}{2.395013in}}%
\pgfpathlineto{\pgfqpoint{2.626308in}{2.404092in}}%
\pgfpathlineto{\pgfqpoint{2.628055in}{2.424608in}}%
\pgfpathlineto{\pgfqpoint{2.628492in}{2.423324in}}%
\pgfpathlineto{\pgfqpoint{2.629510in}{2.404771in}}%
\pgfpathlineto{\pgfqpoint{2.632421in}{2.297139in}}%
\pgfpathlineto{\pgfqpoint{2.633295in}{2.314266in}}%
\pgfpathlineto{\pgfqpoint{2.634459in}{2.340674in}}%
\pgfpathlineto{\pgfqpoint{2.634896in}{2.336527in}}%
\pgfpathlineto{\pgfqpoint{2.635914in}{2.290794in}}%
\pgfpathlineto{\pgfqpoint{2.638680in}{2.123019in}}%
\pgfpathlineto{\pgfqpoint{2.639262in}{2.126870in}}%
\pgfpathlineto{\pgfqpoint{2.639844in}{2.130196in}}%
\pgfpathlineto{\pgfqpoint{2.640135in}{2.127913in}}%
\pgfpathlineto{\pgfqpoint{2.641154in}{2.092829in}}%
\pgfpathlineto{\pgfqpoint{2.644211in}{1.993209in}}%
\pgfpathlineto{\pgfqpoint{2.644502in}{1.991690in}}%
\pgfpathlineto{\pgfqpoint{2.644938in}{1.993766in}}%
\pgfpathlineto{\pgfqpoint{2.646394in}{2.013667in}}%
\pgfpathlineto{\pgfqpoint{2.646976in}{2.008783in}}%
\pgfpathlineto{\pgfqpoint{2.650032in}{1.951850in}}%
\pgfpathlineto{\pgfqpoint{2.650760in}{1.957796in}}%
\pgfpathlineto{\pgfqpoint{2.652652in}{1.978760in}}%
\pgfpathlineto{\pgfqpoint{2.653089in}{1.976916in}}%
\pgfpathlineto{\pgfqpoint{2.654690in}{1.955976in}}%
\pgfpathlineto{\pgfqpoint{2.655563in}{1.947360in}}%
\pgfpathlineto{\pgfqpoint{2.656145in}{1.950949in}}%
\pgfpathlineto{\pgfqpoint{2.657601in}{1.991473in}}%
\pgfpathlineto{\pgfqpoint{2.659056in}{2.015114in}}%
\pgfpathlineto{\pgfqpoint{2.659347in}{2.014948in}}%
\pgfpathlineto{\pgfqpoint{2.659638in}{2.014784in}}%
\pgfpathlineto{\pgfqpoint{2.659929in}{2.015493in}}%
\pgfpathlineto{\pgfqpoint{2.660803in}{2.026500in}}%
\pgfpathlineto{\pgfqpoint{2.663131in}{2.050937in}}%
\pgfpathlineto{\pgfqpoint{2.664878in}{2.068516in}}%
\pgfpathlineto{\pgfqpoint{2.665606in}{2.065789in}}%
\pgfpathlineto{\pgfqpoint{2.665897in}{2.065023in}}%
\pgfpathlineto{\pgfqpoint{2.666333in}{2.067181in}}%
\pgfpathlineto{\pgfqpoint{2.667352in}{2.093206in}}%
\pgfpathlineto{\pgfqpoint{2.669681in}{2.141561in}}%
\pgfpathlineto{\pgfqpoint{2.672155in}{2.148073in}}%
\pgfpathlineto{\pgfqpoint{2.674047in}{2.172039in}}%
\pgfpathlineto{\pgfqpoint{2.674630in}{2.168615in}}%
\pgfpathlineto{\pgfqpoint{2.676667in}{2.159658in}}%
\pgfpathlineto{\pgfqpoint{2.677395in}{2.160736in}}%
\pgfpathlineto{\pgfqpoint{2.678559in}{2.164719in}}%
\pgfpathlineto{\pgfqpoint{2.678996in}{2.162516in}}%
\pgfpathlineto{\pgfqpoint{2.683508in}{2.124867in}}%
\pgfpathlineto{\pgfqpoint{2.684527in}{2.119616in}}%
\pgfpathlineto{\pgfqpoint{2.685691in}{2.098281in}}%
\pgfpathlineto{\pgfqpoint{2.688311in}{2.048699in}}%
\pgfpathlineto{\pgfqpoint{2.688456in}{2.048730in}}%
\pgfpathlineto{\pgfqpoint{2.688893in}{2.048085in}}%
\pgfpathlineto{\pgfqpoint{2.689621in}{2.039996in}}%
\pgfpathlineto{\pgfqpoint{2.693551in}{1.960811in}}%
\pgfpathlineto{\pgfqpoint{2.694569in}{1.962621in}}%
\pgfpathlineto{\pgfqpoint{2.695006in}{1.961655in}}%
\pgfpathlineto{\pgfqpoint{2.695734in}{1.949640in}}%
\pgfpathlineto{\pgfqpoint{2.697917in}{1.895811in}}%
\pgfpathlineto{\pgfqpoint{2.698499in}{1.900199in}}%
\pgfpathlineto{\pgfqpoint{2.700246in}{1.914455in}}%
\pgfpathlineto{\pgfqpoint{2.700682in}{1.913398in}}%
\pgfpathlineto{\pgfqpoint{2.701701in}{1.900621in}}%
\pgfpathlineto{\pgfqpoint{2.703157in}{1.884157in}}%
\pgfpathlineto{\pgfqpoint{2.703593in}{1.886455in}}%
\pgfpathlineto{\pgfqpoint{2.705194in}{1.913246in}}%
\pgfpathlineto{\pgfqpoint{2.706359in}{1.926193in}}%
\pgfpathlineto{\pgfqpoint{2.706795in}{1.924506in}}%
\pgfpathlineto{\pgfqpoint{2.707960in}{1.916403in}}%
\pgfpathlineto{\pgfqpoint{2.708542in}{1.919244in}}%
\pgfpathlineto{\pgfqpoint{2.712471in}{1.962258in}}%
\pgfpathlineto{\pgfqpoint{2.713199in}{1.960907in}}%
\pgfpathlineto{\pgfqpoint{2.715528in}{1.950078in}}%
\pgfpathlineto{\pgfqpoint{2.716256in}{1.952515in}}%
\pgfpathlineto{\pgfqpoint{2.719167in}{1.974379in}}%
\pgfpathlineto{\pgfqpoint{2.719749in}{1.972856in}}%
\pgfpathlineto{\pgfqpoint{2.721932in}{1.959092in}}%
\pgfpathlineto{\pgfqpoint{2.722660in}{1.961908in}}%
\pgfpathlineto{\pgfqpoint{2.724988in}{1.978771in}}%
\pgfpathlineto{\pgfqpoint{2.725571in}{1.976294in}}%
\pgfpathlineto{\pgfqpoint{2.726880in}{1.952622in}}%
\pgfpathlineto{\pgfqpoint{2.728773in}{1.922579in}}%
\pgfpathlineto{\pgfqpoint{2.729064in}{1.922956in}}%
\pgfpathlineto{\pgfqpoint{2.730810in}{1.933072in}}%
\pgfpathlineto{\pgfqpoint{2.731538in}{1.929739in}}%
\pgfpathlineto{\pgfqpoint{2.733285in}{1.898166in}}%
\pgfpathlineto{\pgfqpoint{2.735759in}{1.862707in}}%
\pgfpathlineto{\pgfqpoint{2.735904in}{1.862856in}}%
\pgfpathlineto{\pgfqpoint{2.736632in}{1.863742in}}%
\pgfpathlineto{\pgfqpoint{2.736923in}{1.862854in}}%
\pgfpathlineto{\pgfqpoint{2.737942in}{1.849502in}}%
\pgfpathlineto{\pgfqpoint{2.741290in}{1.804645in}}%
\pgfpathlineto{\pgfqpoint{2.742017in}{1.804633in}}%
\pgfpathlineto{\pgfqpoint{2.742308in}{1.804902in}}%
\pgfpathlineto{\pgfqpoint{2.743473in}{1.804870in}}%
\pgfpathlineto{\pgfqpoint{2.743618in}{1.804643in}}%
\pgfpathlineto{\pgfqpoint{2.744783in}{1.800088in}}%
\pgfpathlineto{\pgfqpoint{2.748858in}{1.776177in}}%
\pgfpathlineto{\pgfqpoint{2.749295in}{1.776554in}}%
\pgfpathlineto{\pgfqpoint{2.750313in}{1.778028in}}%
\pgfpathlineto{\pgfqpoint{2.750750in}{1.777177in}}%
\pgfpathlineto{\pgfqpoint{2.752060in}{1.766614in}}%
\pgfpathlineto{\pgfqpoint{2.754534in}{1.753447in}}%
\pgfpathlineto{\pgfqpoint{2.755262in}{1.753719in}}%
\pgfpathlineto{\pgfqpoint{2.755407in}{1.754025in}}%
\pgfpathlineto{\pgfqpoint{2.757591in}{1.761796in}}%
\pgfpathlineto{\pgfqpoint{2.758318in}{1.759434in}}%
\pgfpathlineto{\pgfqpoint{2.761375in}{1.746970in}}%
\pgfpathlineto{\pgfqpoint{2.765013in}{1.746045in}}%
\pgfpathlineto{\pgfqpoint{2.767488in}{1.742513in}}%
\pgfpathlineto{\pgfqpoint{2.768361in}{1.743664in}}%
\pgfpathlineto{\pgfqpoint{2.770399in}{1.744713in}}%
\pgfpathlineto{\pgfqpoint{2.772291in}{1.745469in}}%
\pgfpathlineto{\pgfqpoint{2.774183in}{1.749979in}}%
\pgfpathlineto{\pgfqpoint{2.774619in}{1.749248in}}%
\pgfpathlineto{\pgfqpoint{2.777530in}{1.744082in}}%
\pgfpathlineto{\pgfqpoint{2.778695in}{1.743177in}}%
\pgfpathlineto{\pgfqpoint{2.785827in}{1.725516in}}%
\pgfpathlineto{\pgfqpoint{2.788010in}{1.718256in}}%
\pgfpathlineto{\pgfqpoint{2.792085in}{1.718363in}}%
\pgfpathlineto{\pgfqpoint{2.794414in}{1.717701in}}%
\pgfpathlineto{\pgfqpoint{2.799071in}{1.716308in}}%
\pgfpathlineto{\pgfqpoint{2.800672in}{1.717627in}}%
\pgfpathlineto{\pgfqpoint{2.800963in}{1.717137in}}%
\pgfpathlineto{\pgfqpoint{2.802273in}{1.713750in}}%
\pgfpathlineto{\pgfqpoint{2.802710in}{1.714686in}}%
\pgfpathlineto{\pgfqpoint{2.806640in}{1.727707in}}%
\pgfpathlineto{\pgfqpoint{2.808241in}{1.728502in}}%
\pgfpathlineto{\pgfqpoint{2.814062in}{1.742753in}}%
\pgfpathlineto{\pgfqpoint{2.815227in}{1.746965in}}%
\pgfpathlineto{\pgfqpoint{2.818138in}{1.755893in}}%
\pgfpathlineto{\pgfqpoint{2.820030in}{1.756541in}}%
\pgfpathlineto{\pgfqpoint{2.821340in}{1.757720in}}%
\pgfpathlineto{\pgfqpoint{2.824833in}{1.762863in}}%
\pgfpathlineto{\pgfqpoint{2.825269in}{1.762578in}}%
\pgfpathlineto{\pgfqpoint{2.829927in}{1.759651in}}%
\pgfpathlineto{\pgfqpoint{2.830363in}{1.760426in}}%
\pgfpathlineto{\pgfqpoint{2.831091in}{1.761011in}}%
\pgfpathlineto{\pgfqpoint{2.831382in}{1.760523in}}%
\pgfpathlineto{\pgfqpoint{2.832838in}{1.752590in}}%
\pgfpathlineto{\pgfqpoint{2.834875in}{1.746696in}}%
\pgfpathlineto{\pgfqpoint{2.836040in}{1.747047in}}%
\pgfpathlineto{\pgfqpoint{2.837350in}{1.749128in}}%
\pgfpathlineto{\pgfqpoint{2.837932in}{1.748128in}}%
\pgfpathlineto{\pgfqpoint{2.840406in}{1.744456in}}%
\pgfpathlineto{\pgfqpoint{2.842153in}{1.741806in}}%
\pgfpathlineto{\pgfqpoint{2.842589in}{1.742327in}}%
\pgfpathlineto{\pgfqpoint{2.844190in}{1.742983in}}%
\pgfpathlineto{\pgfqpoint{2.846082in}{1.743165in}}%
\pgfpathlineto{\pgfqpoint{2.850012in}{1.744867in}}%
\pgfpathlineto{\pgfqpoint{2.852632in}{1.754921in}}%
\pgfpathlineto{\pgfqpoint{2.853505in}{1.754428in}}%
\pgfpathlineto{\pgfqpoint{2.854670in}{1.754859in}}%
\pgfpathlineto{\pgfqpoint{2.855834in}{1.758990in}}%
\pgfpathlineto{\pgfqpoint{2.858163in}{1.768949in}}%
\pgfpathlineto{\pgfqpoint{2.858599in}{1.768602in}}%
\pgfpathlineto{\pgfqpoint{2.859909in}{1.767225in}}%
\pgfpathlineto{\pgfqpoint{2.860346in}{1.767792in}}%
\pgfpathlineto{\pgfqpoint{2.864421in}{1.775469in}}%
\pgfpathlineto{\pgfqpoint{2.865149in}{1.774991in}}%
\pgfpathlineto{\pgfqpoint{2.868788in}{1.774780in}}%
\pgfpathlineto{\pgfqpoint{2.871262in}{1.776432in}}%
\pgfpathlineto{\pgfqpoint{2.872572in}{1.777263in}}%
\pgfpathlineto{\pgfqpoint{2.874464in}{1.778106in}}%
\pgfpathlineto{\pgfqpoint{2.877520in}{1.777553in}}%
\pgfpathlineto{\pgfqpoint{2.878830in}{1.779225in}}%
\pgfpathlineto{\pgfqpoint{2.879267in}{1.778329in}}%
\pgfpathlineto{\pgfqpoint{2.882323in}{1.768847in}}%
\pgfpathlineto{\pgfqpoint{2.882905in}{1.769018in}}%
\pgfpathlineto{\pgfqpoint{2.883924in}{1.768228in}}%
\pgfpathlineto{\pgfqpoint{2.888873in}{1.760074in}}%
\pgfpathlineto{\pgfqpoint{2.889892in}{1.757738in}}%
\pgfpathlineto{\pgfqpoint{2.892511in}{1.752245in}}%
\pgfpathlineto{\pgfqpoint{2.893821in}{1.752841in}}%
\pgfpathlineto{\pgfqpoint{2.894695in}{1.752742in}}%
\pgfpathlineto{\pgfqpoint{2.894840in}{1.752526in}}%
\pgfpathlineto{\pgfqpoint{2.896587in}{1.747235in}}%
\pgfpathlineto{\pgfqpoint{2.898916in}{1.741314in}}%
\pgfpathlineto{\pgfqpoint{2.899207in}{1.741535in}}%
\pgfpathlineto{\pgfqpoint{2.900808in}{1.743799in}}%
\pgfpathlineto{\pgfqpoint{2.901244in}{1.742960in}}%
\pgfpathlineto{\pgfqpoint{2.905028in}{1.731815in}}%
\pgfpathlineto{\pgfqpoint{2.905465in}{1.732116in}}%
\pgfpathlineto{\pgfqpoint{2.906629in}{1.732580in}}%
\pgfpathlineto{\pgfqpoint{2.906921in}{1.732132in}}%
\pgfpathlineto{\pgfqpoint{2.911724in}{1.724023in}}%
\pgfpathlineto{\pgfqpoint{2.914052in}{1.723770in}}%
\pgfpathlineto{\pgfqpoint{2.917545in}{1.719494in}}%
\pgfpathlineto{\pgfqpoint{2.921330in}{1.717031in}}%
\pgfpathlineto{\pgfqpoint{2.923658in}{1.714466in}}%
\pgfpathlineto{\pgfqpoint{2.925550in}{1.713995in}}%
\pgfpathlineto{\pgfqpoint{2.927151in}{1.713895in}}%
\pgfpathlineto{\pgfqpoint{2.930499in}{1.709842in}}%
\pgfpathlineto{\pgfqpoint{2.931372in}{1.710731in}}%
\pgfpathlineto{\pgfqpoint{2.933119in}{1.710592in}}%
\pgfpathlineto{\pgfqpoint{2.936903in}{1.708262in}}%
\pgfpathlineto{\pgfqpoint{2.938213in}{1.710800in}}%
\pgfpathlineto{\pgfqpoint{2.939814in}{1.713451in}}%
\pgfpathlineto{\pgfqpoint{2.940105in}{1.713318in}}%
\pgfpathlineto{\pgfqpoint{2.942143in}{1.713095in}}%
\pgfpathlineto{\pgfqpoint{2.949565in}{1.721334in}}%
\pgfpathlineto{\pgfqpoint{2.951603in}{1.724039in}}%
\pgfpathlineto{\pgfqpoint{2.953932in}{1.724425in}}%
\pgfpathlineto{\pgfqpoint{2.955533in}{1.728957in}}%
\pgfpathlineto{\pgfqpoint{2.958007in}{1.734273in}}%
\pgfpathlineto{\pgfqpoint{2.959608in}{1.735197in}}%
\pgfpathlineto{\pgfqpoint{2.961209in}{1.738658in}}%
\pgfpathlineto{\pgfqpoint{2.963247in}{1.741832in}}%
\pgfpathlineto{\pgfqpoint{2.965575in}{1.742499in}}%
\pgfpathlineto{\pgfqpoint{2.967176in}{1.742096in}}%
\pgfpathlineto{\pgfqpoint{2.969942in}{1.739570in}}%
\pgfpathlineto{\pgfqpoint{2.971543in}{1.738494in}}%
\pgfpathlineto{\pgfqpoint{2.976055in}{1.734286in}}%
\pgfpathlineto{\pgfqpoint{2.977365in}{1.732703in}}%
\pgfpathlineto{\pgfqpoint{2.981440in}{1.726174in}}%
\pgfpathlineto{\pgfqpoint{2.983041in}{1.725428in}}%
\pgfpathlineto{\pgfqpoint{2.984496in}{1.720358in}}%
\pgfpathlineto{\pgfqpoint{2.986971in}{1.713852in}}%
\pgfpathlineto{\pgfqpoint{2.988281in}{1.713149in}}%
\pgfpathlineto{\pgfqpoint{2.990027in}{1.708817in}}%
\pgfpathlineto{\pgfqpoint{2.992065in}{1.703898in}}%
\pgfpathlineto{\pgfqpoint{2.992356in}{1.704066in}}%
\pgfpathlineto{\pgfqpoint{2.995412in}{1.705372in}}%
\pgfpathlineto{\pgfqpoint{2.997595in}{1.703366in}}%
\pgfpathlineto{\pgfqpoint{2.998032in}{1.704013in}}%
\pgfpathlineto{\pgfqpoint{3.002253in}{1.710935in}}%
\pgfpathlineto{\pgfqpoint{3.002398in}{1.710855in}}%
\pgfpathlineto{\pgfqpoint{3.003563in}{1.711120in}}%
\pgfpathlineto{\pgfqpoint{3.009676in}{1.721011in}}%
\pgfpathlineto{\pgfqpoint{3.014915in}{1.726785in}}%
\pgfpathlineto{\pgfqpoint{3.018554in}{1.727623in}}%
\pgfpathlineto{\pgfqpoint{3.021319in}{1.730026in}}%
\pgfpathlineto{\pgfqpoint{3.025104in}{1.728728in}}%
\pgfpathlineto{\pgfqpoint{3.026850in}{1.729964in}}%
\pgfpathlineto{\pgfqpoint{3.026996in}{1.729827in}}%
\pgfpathlineto{\pgfqpoint{3.028742in}{1.725963in}}%
\pgfpathlineto{\pgfqpoint{3.030198in}{1.724412in}}%
\pgfpathlineto{\pgfqpoint{3.030489in}{1.724634in}}%
\pgfpathlineto{\pgfqpoint{3.032381in}{1.726734in}}%
\pgfpathlineto{\pgfqpoint{3.032818in}{1.726136in}}%
\pgfpathlineto{\pgfqpoint{3.036602in}{1.718798in}}%
\pgfpathlineto{\pgfqpoint{3.037038in}{1.719126in}}%
\pgfpathlineto{\pgfqpoint{3.039076in}{1.719785in}}%
\pgfpathlineto{\pgfqpoint{3.042569in}{1.716131in}}%
\pgfpathlineto{\pgfqpoint{3.043151in}{1.716838in}}%
\pgfpathlineto{\pgfqpoint{3.045626in}{1.718754in}}%
\pgfpathlineto{\pgfqpoint{3.047809in}{1.717297in}}%
\pgfpathlineto{\pgfqpoint{3.050720in}{1.715289in}}%
\pgfpathlineto{\pgfqpoint{3.053631in}{1.714465in}}%
\pgfpathlineto{\pgfqpoint{3.063964in}{1.706332in}}%
\pgfpathlineto{\pgfqpoint{3.065711in}{1.704850in}}%
\pgfpathlineto{\pgfqpoint{3.069058in}{1.702390in}}%
\pgfpathlineto{\pgfqpoint{3.071678in}{1.701095in}}%
\pgfpathlineto{\pgfqpoint{3.075026in}{1.698388in}}%
\pgfpathlineto{\pgfqpoint{3.093656in}{1.699091in}}%
\pgfpathlineto{\pgfqpoint{3.096567in}{1.699395in}}%
\pgfpathlineto{\pgfqpoint{3.098022in}{1.700459in}}%
\pgfpathlineto{\pgfqpoint{3.100205in}{1.701869in}}%
\pgfpathlineto{\pgfqpoint{3.103844in}{1.702695in}}%
\pgfpathlineto{\pgfqpoint{3.106318in}{1.703655in}}%
\pgfpathlineto{\pgfqpoint{3.115633in}{1.705869in}}%
\pgfpathlineto{\pgfqpoint{3.117525in}{1.706055in}}%
\pgfpathlineto{\pgfqpoint{3.119563in}{1.706381in}}%
\pgfpathlineto{\pgfqpoint{3.124366in}{1.707648in}}%
\pgfpathlineto{\pgfqpoint{3.128296in}{1.707626in}}%
\pgfpathlineto{\pgfqpoint{3.131206in}{1.707927in}}%
\pgfpathlineto{\pgfqpoint{3.133244in}{1.707251in}}%
\pgfpathlineto{\pgfqpoint{3.135718in}{1.706570in}}%
\pgfpathlineto{\pgfqpoint{3.143141in}{1.706527in}}%
\pgfpathlineto{\pgfqpoint{3.144888in}{1.706197in}}%
\pgfpathlineto{\pgfqpoint{3.147653in}{1.705371in}}%
\pgfpathlineto{\pgfqpoint{3.152311in}{1.704984in}}%
\pgfpathlineto{\pgfqpoint{3.154203in}{1.705184in}}%
\pgfpathlineto{\pgfqpoint{3.156822in}{1.704593in}}%
\pgfpathlineto{\pgfqpoint{3.168029in}{1.715190in}}%
\pgfpathlineto{\pgfqpoint{3.169630in}{1.715842in}}%
\pgfpathlineto{\pgfqpoint{3.173706in}{1.719640in}}%
\pgfpathlineto{\pgfqpoint{3.176035in}{1.720776in}}%
\pgfpathlineto{\pgfqpoint{3.178800in}{1.724094in}}%
\pgfpathlineto{\pgfqpoint{3.180692in}{1.722972in}}%
\pgfpathlineto{\pgfqpoint{3.182584in}{1.722406in}}%
\pgfpathlineto{\pgfqpoint{3.185058in}{1.722946in}}%
\pgfpathlineto{\pgfqpoint{3.187824in}{1.719375in}}%
\pgfpathlineto{\pgfqpoint{3.190152in}{1.717260in}}%
\pgfpathlineto{\pgfqpoint{3.191753in}{1.715991in}}%
\pgfpathlineto{\pgfqpoint{3.197139in}{1.710619in}}%
\pgfpathlineto{\pgfqpoint{3.199031in}{1.710178in}}%
\pgfpathlineto{\pgfqpoint{3.203106in}{1.706046in}}%
\pgfpathlineto{\pgfqpoint{3.204998in}{1.705770in}}%
\pgfpathlineto{\pgfqpoint{3.208346in}{1.702249in}}%
\pgfpathlineto{\pgfqpoint{3.211839in}{1.702616in}}%
\pgfpathlineto{\pgfqpoint{3.214604in}{1.701698in}}%
\pgfpathlineto{\pgfqpoint{3.225083in}{1.701860in}}%
\pgfpathlineto{\pgfqpoint{3.227267in}{1.701444in}}%
\pgfpathlineto{\pgfqpoint{3.231196in}{1.700977in}}%
\pgfpathlineto{\pgfqpoint{3.238910in}{1.699649in}}%
\pgfpathlineto{\pgfqpoint{3.242112in}{1.699715in}}%
\pgfpathlineto{\pgfqpoint{3.245751in}{1.699923in}}%
\pgfpathlineto{\pgfqpoint{3.254338in}{1.699783in}}%
\pgfpathlineto{\pgfqpoint{3.260160in}{1.699982in}}%
\pgfpathlineto{\pgfqpoint{3.274132in}{1.700360in}}%
\pgfpathlineto{\pgfqpoint{3.292471in}{1.703389in}}%
\pgfpathlineto{\pgfqpoint{3.294800in}{1.703327in}}%
\pgfpathlineto{\pgfqpoint{3.297420in}{1.703339in}}%
\pgfpathlineto{\pgfqpoint{3.301786in}{1.704201in}}%
\pgfpathlineto{\pgfqpoint{3.303824in}{1.704649in}}%
\pgfpathlineto{\pgfqpoint{3.306735in}{1.705940in}}%
\pgfpathlineto{\pgfqpoint{3.309354in}{1.705450in}}%
\pgfpathlineto{\pgfqpoint{3.314740in}{1.705801in}}%
\pgfpathlineto{\pgfqpoint{3.317359in}{1.705280in}}%
\pgfpathlineto{\pgfqpoint{3.320416in}{1.704750in}}%
\pgfpathlineto{\pgfqpoint{3.323618in}{1.704289in}}%
\pgfpathlineto{\pgfqpoint{3.326674in}{1.703952in}}%
\pgfpathlineto{\pgfqpoint{3.338900in}{1.699213in}}%
\pgfpathlineto{\pgfqpoint{3.346905in}{1.697899in}}%
\pgfpathlineto{\pgfqpoint{3.359277in}{1.698423in}}%
\pgfpathlineto{\pgfqpoint{3.369174in}{1.701929in}}%
\pgfpathlineto{\pgfqpoint{3.384165in}{1.702642in}}%
\pgfpathlineto{\pgfqpoint{3.395517in}{1.700026in}}%
\pgfpathlineto{\pgfqpoint{3.400466in}{1.699123in}}%
\pgfpathlineto{\pgfqpoint{3.415166in}{1.698430in}}%
\pgfpathlineto{\pgfqpoint{3.432486in}{1.698983in}}%
\pgfpathlineto{\pgfqpoint{3.447186in}{1.699966in}}%
\pgfpathlineto{\pgfqpoint{3.466253in}{1.701600in}}%
\pgfpathlineto{\pgfqpoint{3.487066in}{1.698158in}}%
\pgfpathlineto{\pgfqpoint{3.490559in}{1.698331in}}%
\pgfpathlineto{\pgfqpoint{3.545575in}{1.697553in}}%
\pgfpathlineto{\pgfqpoint{3.626935in}{1.698035in}}%
\pgfpathlineto{\pgfqpoint{3.639452in}{1.697137in}}%
\pgfpathlineto{\pgfqpoint{3.665650in}{1.696523in}}%
\pgfpathlineto{\pgfqpoint{3.727507in}{1.697878in}}%
\pgfpathlineto{\pgfqpoint{3.748029in}{1.696437in}}%
\pgfpathlineto{\pgfqpoint{3.777866in}{1.697371in}}%
\pgfpathlineto{\pgfqpoint{3.788491in}{1.697534in}}%
\pgfpathlineto{\pgfqpoint{3.800717in}{1.697101in}}%
\pgfpathlineto{\pgfqpoint{3.838704in}{1.696765in}}%
\pgfpathlineto{\pgfqpoint{3.853695in}{1.697451in}}%
\pgfpathlineto{\pgfqpoint{3.878584in}{1.696369in}}%
\pgfpathlineto{\pgfqpoint{3.891392in}{1.696207in}}%
\pgfpathlineto{\pgfqpoint{3.928506in}{1.696940in}}%
\pgfpathlineto{\pgfqpoint{3.939131in}{1.697213in}}%
\pgfpathlineto{\pgfqpoint{3.968531in}{1.696348in}}%
\pgfpathlineto{\pgfqpoint{3.990072in}{1.696636in}}%
\pgfpathlineto{\pgfqpoint{4.033590in}{1.697521in}}%
\pgfpathlineto{\pgfqpoint{4.060225in}{1.696963in}}%
\pgfpathlineto{\pgfqpoint{4.070413in}{1.696266in}}%
\pgfpathlineto{\pgfqpoint{4.291206in}{1.696662in}}%
\pgfpathlineto{\pgfqpoint{4.313329in}{1.696193in}}%
\pgfpathlineto{\pgfqpoint{4.341710in}{1.695885in}}%
\pgfpathlineto{\pgfqpoint{4.583461in}{1.696057in}}%
\pgfpathlineto{\pgfqpoint{4.711396in}{1.695905in}}%
\pgfpathlineto{\pgfqpoint{4.945725in}{1.695843in}}%
\pgfpathlineto{\pgfqpoint{5.333313in}{1.695947in}}%
\pgfpathlineto{\pgfqpoint{5.527616in}{1.695889in}}%
\pgfpathlineto{\pgfqpoint{5.706637in}{1.695819in}}%
\pgfpathlineto{\pgfqpoint{5.850000in}{1.695819in}}%
\pgfpathlineto{\pgfqpoint{5.850000in}{1.695819in}}%
\pgfusepath{stroke}%
\end{pgfscope}%
\begin{pgfscope}%
\pgfsetrectcap%
\pgfsetmiterjoin%
\pgfsetlinewidth{0.803000pt}%
\definecolor{currentstroke}{rgb}{0.737255,0.737255,0.737255}%
\pgfsetstrokecolor{currentstroke}%
\pgfsetdash{}{0pt}%
\pgfpathmoveto{\pgfqpoint{0.610501in}{1.652990in}}%
\pgfpathlineto{\pgfqpoint{0.610501in}{2.594880in}}%
\pgfusepath{stroke}%
\end{pgfscope}%
\begin{pgfscope}%
\pgfsetrectcap%
\pgfsetmiterjoin%
\pgfsetlinewidth{0.803000pt}%
\definecolor{currentstroke}{rgb}{0.737255,0.737255,0.737255}%
\pgfsetstrokecolor{currentstroke}%
\pgfsetdash{}{0pt}%
\pgfpathmoveto{\pgfqpoint{5.850000in}{1.652990in}}%
\pgfpathlineto{\pgfqpoint{5.850000in}{2.594880in}}%
\pgfusepath{stroke}%
\end{pgfscope}%
\begin{pgfscope}%
\pgfsetrectcap%
\pgfsetmiterjoin%
\pgfsetlinewidth{0.803000pt}%
\definecolor{currentstroke}{rgb}{0.737255,0.737255,0.737255}%
\pgfsetstrokecolor{currentstroke}%
\pgfsetdash{}{0pt}%
\pgfpathmoveto{\pgfqpoint{0.610501in}{1.652990in}}%
\pgfpathlineto{\pgfqpoint{5.850000in}{1.652990in}}%
\pgfusepath{stroke}%
\end{pgfscope}%
\begin{pgfscope}%
\pgfsetrectcap%
\pgfsetmiterjoin%
\pgfsetlinewidth{0.803000pt}%
\definecolor{currentstroke}{rgb}{0.737255,0.737255,0.737255}%
\pgfsetstrokecolor{currentstroke}%
\pgfsetdash{}{0pt}%
\pgfpathmoveto{\pgfqpoint{0.610501in}{2.594880in}}%
\pgfpathlineto{\pgfqpoint{5.850000in}{2.594880in}}%
\pgfusepath{stroke}%
\end{pgfscope}%
\begin{pgfscope}%
\pgfsetbuttcap%
\pgfsetmiterjoin%
\definecolor{currentfill}{rgb}{0.933333,0.933333,0.933333}%
\pgfsetfillcolor{currentfill}%
\pgfsetlinewidth{0.000000pt}%
\definecolor{currentstroke}{rgb}{0.000000,0.000000,0.000000}%
\pgfsetstrokecolor{currentstroke}%
\pgfsetstrokeopacity{0.000000}%
\pgfsetdash{}{0pt}%
\pgfpathmoveto{\pgfqpoint{0.610501in}{0.544166in}}%
\pgfpathlineto{\pgfqpoint{5.850000in}{0.544166in}}%
\pgfpathlineto{\pgfqpoint{5.850000in}{1.486056in}}%
\pgfpathlineto{\pgfqpoint{0.610501in}{1.486056in}}%
\pgfpathlineto{\pgfqpoint{0.610501in}{0.544166in}}%
\pgfpathclose%
\pgfusepath{fill}%
\end{pgfscope}%
\begin{pgfscope}%
\pgfpathrectangle{\pgfqpoint{0.610501in}{0.544166in}}{\pgfqpoint{5.239499in}{0.941890in}}%
\pgfusepath{clip}%
\pgfsetbuttcap%
\pgfsetroundjoin%
\pgfsetlinewidth{0.501875pt}%
\definecolor{currentstroke}{rgb}{0.698039,0.698039,0.698039}%
\pgfsetstrokecolor{currentstroke}%
\pgfsetdash{{1.850000pt}{0.800000pt}}{0.000000pt}%
\pgfpathmoveto{\pgfqpoint{0.610501in}{0.544166in}}%
\pgfpathlineto{\pgfqpoint{0.610501in}{1.486056in}}%
\pgfusepath{stroke}%
\end{pgfscope}%
\begin{pgfscope}%
\pgfsetbuttcap%
\pgfsetroundjoin%
\definecolor{currentfill}{rgb}{0.000000,0.000000,0.000000}%
\pgfsetfillcolor{currentfill}%
\pgfsetlinewidth{0.803000pt}%
\definecolor{currentstroke}{rgb}{0.000000,0.000000,0.000000}%
\pgfsetstrokecolor{currentstroke}%
\pgfsetdash{}{0pt}%
\pgfsys@defobject{currentmarker}{\pgfqpoint{0.000000in}{0.000000in}}{\pgfqpoint{0.000000in}{0.048611in}}{%
\pgfpathmoveto{\pgfqpoint{0.000000in}{0.000000in}}%
\pgfpathlineto{\pgfqpoint{0.000000in}{0.048611in}}%
\pgfusepath{stroke,fill}%
}%
\begin{pgfscope}%
\pgfsys@transformshift{0.610501in}{0.544166in}%
\pgfsys@useobject{currentmarker}{}%
\end{pgfscope}%
\end{pgfscope}%
\begin{pgfscope}%
\definecolor{textcolor}{rgb}{0.000000,0.000000,0.000000}%
\pgfsetstrokecolor{textcolor}%
\pgfsetfillcolor{textcolor}%
\pgftext[x=0.610501in,y=0.495555in,,top]{\color{textcolor}\rmfamily\fontsize{10.000000}{12.000000}\selectfont \(\displaystyle {0}\)}%
\end{pgfscope}%
\begin{pgfscope}%
\pgfpathrectangle{\pgfqpoint{0.610501in}{0.544166in}}{\pgfqpoint{5.239499in}{0.941890in}}%
\pgfusepath{clip}%
\pgfsetbuttcap%
\pgfsetroundjoin%
\pgfsetlinewidth{0.501875pt}%
\definecolor{currentstroke}{rgb}{0.698039,0.698039,0.698039}%
\pgfsetstrokecolor{currentstroke}%
\pgfsetdash{{1.850000pt}{0.800000pt}}{0.000000pt}%
\pgfpathmoveto{\pgfqpoint{1.192683in}{0.544166in}}%
\pgfpathlineto{\pgfqpoint{1.192683in}{1.486056in}}%
\pgfusepath{stroke}%
\end{pgfscope}%
\begin{pgfscope}%
\pgfsetbuttcap%
\pgfsetroundjoin%
\definecolor{currentfill}{rgb}{0.000000,0.000000,0.000000}%
\pgfsetfillcolor{currentfill}%
\pgfsetlinewidth{0.803000pt}%
\definecolor{currentstroke}{rgb}{0.000000,0.000000,0.000000}%
\pgfsetstrokecolor{currentstroke}%
\pgfsetdash{}{0pt}%
\pgfsys@defobject{currentmarker}{\pgfqpoint{0.000000in}{0.000000in}}{\pgfqpoint{0.000000in}{0.048611in}}{%
\pgfpathmoveto{\pgfqpoint{0.000000in}{0.000000in}}%
\pgfpathlineto{\pgfqpoint{0.000000in}{0.048611in}}%
\pgfusepath{stroke,fill}%
}%
\begin{pgfscope}%
\pgfsys@transformshift{1.192683in}{0.544166in}%
\pgfsys@useobject{currentmarker}{}%
\end{pgfscope}%
\end{pgfscope}%
\begin{pgfscope}%
\definecolor{textcolor}{rgb}{0.000000,0.000000,0.000000}%
\pgfsetstrokecolor{textcolor}%
\pgfsetfillcolor{textcolor}%
\pgftext[x=1.192683in,y=0.495555in,,top]{\color{textcolor}\rmfamily\fontsize{10.000000}{12.000000}\selectfont \(\displaystyle {20}\)}%
\end{pgfscope}%
\begin{pgfscope}%
\pgfpathrectangle{\pgfqpoint{0.610501in}{0.544166in}}{\pgfqpoint{5.239499in}{0.941890in}}%
\pgfusepath{clip}%
\pgfsetbuttcap%
\pgfsetroundjoin%
\pgfsetlinewidth{0.501875pt}%
\definecolor{currentstroke}{rgb}{0.698039,0.698039,0.698039}%
\pgfsetstrokecolor{currentstroke}%
\pgfsetdash{{1.850000pt}{0.800000pt}}{0.000000pt}%
\pgfpathmoveto{\pgfqpoint{1.774866in}{0.544166in}}%
\pgfpathlineto{\pgfqpoint{1.774866in}{1.486056in}}%
\pgfusepath{stroke}%
\end{pgfscope}%
\begin{pgfscope}%
\pgfsetbuttcap%
\pgfsetroundjoin%
\definecolor{currentfill}{rgb}{0.000000,0.000000,0.000000}%
\pgfsetfillcolor{currentfill}%
\pgfsetlinewidth{0.803000pt}%
\definecolor{currentstroke}{rgb}{0.000000,0.000000,0.000000}%
\pgfsetstrokecolor{currentstroke}%
\pgfsetdash{}{0pt}%
\pgfsys@defobject{currentmarker}{\pgfqpoint{0.000000in}{0.000000in}}{\pgfqpoint{0.000000in}{0.048611in}}{%
\pgfpathmoveto{\pgfqpoint{0.000000in}{0.000000in}}%
\pgfpathlineto{\pgfqpoint{0.000000in}{0.048611in}}%
\pgfusepath{stroke,fill}%
}%
\begin{pgfscope}%
\pgfsys@transformshift{1.774866in}{0.544166in}%
\pgfsys@useobject{currentmarker}{}%
\end{pgfscope}%
\end{pgfscope}%
\begin{pgfscope}%
\definecolor{textcolor}{rgb}{0.000000,0.000000,0.000000}%
\pgfsetstrokecolor{textcolor}%
\pgfsetfillcolor{textcolor}%
\pgftext[x=1.774866in,y=0.495555in,,top]{\color{textcolor}\rmfamily\fontsize{10.000000}{12.000000}\selectfont \(\displaystyle {40}\)}%
\end{pgfscope}%
\begin{pgfscope}%
\pgfpathrectangle{\pgfqpoint{0.610501in}{0.544166in}}{\pgfqpoint{5.239499in}{0.941890in}}%
\pgfusepath{clip}%
\pgfsetbuttcap%
\pgfsetroundjoin%
\pgfsetlinewidth{0.501875pt}%
\definecolor{currentstroke}{rgb}{0.698039,0.698039,0.698039}%
\pgfsetstrokecolor{currentstroke}%
\pgfsetdash{{1.850000pt}{0.800000pt}}{0.000000pt}%
\pgfpathmoveto{\pgfqpoint{2.357049in}{0.544166in}}%
\pgfpathlineto{\pgfqpoint{2.357049in}{1.486056in}}%
\pgfusepath{stroke}%
\end{pgfscope}%
\begin{pgfscope}%
\pgfsetbuttcap%
\pgfsetroundjoin%
\definecolor{currentfill}{rgb}{0.000000,0.000000,0.000000}%
\pgfsetfillcolor{currentfill}%
\pgfsetlinewidth{0.803000pt}%
\definecolor{currentstroke}{rgb}{0.000000,0.000000,0.000000}%
\pgfsetstrokecolor{currentstroke}%
\pgfsetdash{}{0pt}%
\pgfsys@defobject{currentmarker}{\pgfqpoint{0.000000in}{0.000000in}}{\pgfqpoint{0.000000in}{0.048611in}}{%
\pgfpathmoveto{\pgfqpoint{0.000000in}{0.000000in}}%
\pgfpathlineto{\pgfqpoint{0.000000in}{0.048611in}}%
\pgfusepath{stroke,fill}%
}%
\begin{pgfscope}%
\pgfsys@transformshift{2.357049in}{0.544166in}%
\pgfsys@useobject{currentmarker}{}%
\end{pgfscope}%
\end{pgfscope}%
\begin{pgfscope}%
\definecolor{textcolor}{rgb}{0.000000,0.000000,0.000000}%
\pgfsetstrokecolor{textcolor}%
\pgfsetfillcolor{textcolor}%
\pgftext[x=2.357049in,y=0.495555in,,top]{\color{textcolor}\rmfamily\fontsize{10.000000}{12.000000}\selectfont \(\displaystyle {60}\)}%
\end{pgfscope}%
\begin{pgfscope}%
\pgfpathrectangle{\pgfqpoint{0.610501in}{0.544166in}}{\pgfqpoint{5.239499in}{0.941890in}}%
\pgfusepath{clip}%
\pgfsetbuttcap%
\pgfsetroundjoin%
\pgfsetlinewidth{0.501875pt}%
\definecolor{currentstroke}{rgb}{0.698039,0.698039,0.698039}%
\pgfsetstrokecolor{currentstroke}%
\pgfsetdash{{1.850000pt}{0.800000pt}}{0.000000pt}%
\pgfpathmoveto{\pgfqpoint{2.939232in}{0.544166in}}%
\pgfpathlineto{\pgfqpoint{2.939232in}{1.486056in}}%
\pgfusepath{stroke}%
\end{pgfscope}%
\begin{pgfscope}%
\pgfsetbuttcap%
\pgfsetroundjoin%
\definecolor{currentfill}{rgb}{0.000000,0.000000,0.000000}%
\pgfsetfillcolor{currentfill}%
\pgfsetlinewidth{0.803000pt}%
\definecolor{currentstroke}{rgb}{0.000000,0.000000,0.000000}%
\pgfsetstrokecolor{currentstroke}%
\pgfsetdash{}{0pt}%
\pgfsys@defobject{currentmarker}{\pgfqpoint{0.000000in}{0.000000in}}{\pgfqpoint{0.000000in}{0.048611in}}{%
\pgfpathmoveto{\pgfqpoint{0.000000in}{0.000000in}}%
\pgfpathlineto{\pgfqpoint{0.000000in}{0.048611in}}%
\pgfusepath{stroke,fill}%
}%
\begin{pgfscope}%
\pgfsys@transformshift{2.939232in}{0.544166in}%
\pgfsys@useobject{currentmarker}{}%
\end{pgfscope}%
\end{pgfscope}%
\begin{pgfscope}%
\definecolor{textcolor}{rgb}{0.000000,0.000000,0.000000}%
\pgfsetstrokecolor{textcolor}%
\pgfsetfillcolor{textcolor}%
\pgftext[x=2.939232in,y=0.495555in,,top]{\color{textcolor}\rmfamily\fontsize{10.000000}{12.000000}\selectfont \(\displaystyle {80}\)}%
\end{pgfscope}%
\begin{pgfscope}%
\pgfpathrectangle{\pgfqpoint{0.610501in}{0.544166in}}{\pgfqpoint{5.239499in}{0.941890in}}%
\pgfusepath{clip}%
\pgfsetbuttcap%
\pgfsetroundjoin%
\pgfsetlinewidth{0.501875pt}%
\definecolor{currentstroke}{rgb}{0.698039,0.698039,0.698039}%
\pgfsetstrokecolor{currentstroke}%
\pgfsetdash{{1.850000pt}{0.800000pt}}{0.000000pt}%
\pgfpathmoveto{\pgfqpoint{3.521414in}{0.544166in}}%
\pgfpathlineto{\pgfqpoint{3.521414in}{1.486056in}}%
\pgfusepath{stroke}%
\end{pgfscope}%
\begin{pgfscope}%
\pgfsetbuttcap%
\pgfsetroundjoin%
\definecolor{currentfill}{rgb}{0.000000,0.000000,0.000000}%
\pgfsetfillcolor{currentfill}%
\pgfsetlinewidth{0.803000pt}%
\definecolor{currentstroke}{rgb}{0.000000,0.000000,0.000000}%
\pgfsetstrokecolor{currentstroke}%
\pgfsetdash{}{0pt}%
\pgfsys@defobject{currentmarker}{\pgfqpoint{0.000000in}{0.000000in}}{\pgfqpoint{0.000000in}{0.048611in}}{%
\pgfpathmoveto{\pgfqpoint{0.000000in}{0.000000in}}%
\pgfpathlineto{\pgfqpoint{0.000000in}{0.048611in}}%
\pgfusepath{stroke,fill}%
}%
\begin{pgfscope}%
\pgfsys@transformshift{3.521414in}{0.544166in}%
\pgfsys@useobject{currentmarker}{}%
\end{pgfscope}%
\end{pgfscope}%
\begin{pgfscope}%
\definecolor{textcolor}{rgb}{0.000000,0.000000,0.000000}%
\pgfsetstrokecolor{textcolor}%
\pgfsetfillcolor{textcolor}%
\pgftext[x=3.521414in,y=0.495555in,,top]{\color{textcolor}\rmfamily\fontsize{10.000000}{12.000000}\selectfont \(\displaystyle {100}\)}%
\end{pgfscope}%
\begin{pgfscope}%
\pgfpathrectangle{\pgfqpoint{0.610501in}{0.544166in}}{\pgfqpoint{5.239499in}{0.941890in}}%
\pgfusepath{clip}%
\pgfsetbuttcap%
\pgfsetroundjoin%
\pgfsetlinewidth{0.501875pt}%
\definecolor{currentstroke}{rgb}{0.698039,0.698039,0.698039}%
\pgfsetstrokecolor{currentstroke}%
\pgfsetdash{{1.850000pt}{0.800000pt}}{0.000000pt}%
\pgfpathmoveto{\pgfqpoint{4.103597in}{0.544166in}}%
\pgfpathlineto{\pgfqpoint{4.103597in}{1.486056in}}%
\pgfusepath{stroke}%
\end{pgfscope}%
\begin{pgfscope}%
\pgfsetbuttcap%
\pgfsetroundjoin%
\definecolor{currentfill}{rgb}{0.000000,0.000000,0.000000}%
\pgfsetfillcolor{currentfill}%
\pgfsetlinewidth{0.803000pt}%
\definecolor{currentstroke}{rgb}{0.000000,0.000000,0.000000}%
\pgfsetstrokecolor{currentstroke}%
\pgfsetdash{}{0pt}%
\pgfsys@defobject{currentmarker}{\pgfqpoint{0.000000in}{0.000000in}}{\pgfqpoint{0.000000in}{0.048611in}}{%
\pgfpathmoveto{\pgfqpoint{0.000000in}{0.000000in}}%
\pgfpathlineto{\pgfqpoint{0.000000in}{0.048611in}}%
\pgfusepath{stroke,fill}%
}%
\begin{pgfscope}%
\pgfsys@transformshift{4.103597in}{0.544166in}%
\pgfsys@useobject{currentmarker}{}%
\end{pgfscope}%
\end{pgfscope}%
\begin{pgfscope}%
\definecolor{textcolor}{rgb}{0.000000,0.000000,0.000000}%
\pgfsetstrokecolor{textcolor}%
\pgfsetfillcolor{textcolor}%
\pgftext[x=4.103597in,y=0.495555in,,top]{\color{textcolor}\rmfamily\fontsize{10.000000}{12.000000}\selectfont \(\displaystyle {120}\)}%
\end{pgfscope}%
\begin{pgfscope}%
\pgfpathrectangle{\pgfqpoint{0.610501in}{0.544166in}}{\pgfqpoint{5.239499in}{0.941890in}}%
\pgfusepath{clip}%
\pgfsetbuttcap%
\pgfsetroundjoin%
\pgfsetlinewidth{0.501875pt}%
\definecolor{currentstroke}{rgb}{0.698039,0.698039,0.698039}%
\pgfsetstrokecolor{currentstroke}%
\pgfsetdash{{1.850000pt}{0.800000pt}}{0.000000pt}%
\pgfpathmoveto{\pgfqpoint{4.685780in}{0.544166in}}%
\pgfpathlineto{\pgfqpoint{4.685780in}{1.486056in}}%
\pgfusepath{stroke}%
\end{pgfscope}%
\begin{pgfscope}%
\pgfsetbuttcap%
\pgfsetroundjoin%
\definecolor{currentfill}{rgb}{0.000000,0.000000,0.000000}%
\pgfsetfillcolor{currentfill}%
\pgfsetlinewidth{0.803000pt}%
\definecolor{currentstroke}{rgb}{0.000000,0.000000,0.000000}%
\pgfsetstrokecolor{currentstroke}%
\pgfsetdash{}{0pt}%
\pgfsys@defobject{currentmarker}{\pgfqpoint{0.000000in}{0.000000in}}{\pgfqpoint{0.000000in}{0.048611in}}{%
\pgfpathmoveto{\pgfqpoint{0.000000in}{0.000000in}}%
\pgfpathlineto{\pgfqpoint{0.000000in}{0.048611in}}%
\pgfusepath{stroke,fill}%
}%
\begin{pgfscope}%
\pgfsys@transformshift{4.685780in}{0.544166in}%
\pgfsys@useobject{currentmarker}{}%
\end{pgfscope}%
\end{pgfscope}%
\begin{pgfscope}%
\definecolor{textcolor}{rgb}{0.000000,0.000000,0.000000}%
\pgfsetstrokecolor{textcolor}%
\pgfsetfillcolor{textcolor}%
\pgftext[x=4.685780in,y=0.495555in,,top]{\color{textcolor}\rmfamily\fontsize{10.000000}{12.000000}\selectfont \(\displaystyle {140}\)}%
\end{pgfscope}%
\begin{pgfscope}%
\pgfpathrectangle{\pgfqpoint{0.610501in}{0.544166in}}{\pgfqpoint{5.239499in}{0.941890in}}%
\pgfusepath{clip}%
\pgfsetbuttcap%
\pgfsetroundjoin%
\pgfsetlinewidth{0.501875pt}%
\definecolor{currentstroke}{rgb}{0.698039,0.698039,0.698039}%
\pgfsetstrokecolor{currentstroke}%
\pgfsetdash{{1.850000pt}{0.800000pt}}{0.000000pt}%
\pgfpathmoveto{\pgfqpoint{5.267963in}{0.544166in}}%
\pgfpathlineto{\pgfqpoint{5.267963in}{1.486056in}}%
\pgfusepath{stroke}%
\end{pgfscope}%
\begin{pgfscope}%
\pgfsetbuttcap%
\pgfsetroundjoin%
\definecolor{currentfill}{rgb}{0.000000,0.000000,0.000000}%
\pgfsetfillcolor{currentfill}%
\pgfsetlinewidth{0.803000pt}%
\definecolor{currentstroke}{rgb}{0.000000,0.000000,0.000000}%
\pgfsetstrokecolor{currentstroke}%
\pgfsetdash{}{0pt}%
\pgfsys@defobject{currentmarker}{\pgfqpoint{0.000000in}{0.000000in}}{\pgfqpoint{0.000000in}{0.048611in}}{%
\pgfpathmoveto{\pgfqpoint{0.000000in}{0.000000in}}%
\pgfpathlineto{\pgfqpoint{0.000000in}{0.048611in}}%
\pgfusepath{stroke,fill}%
}%
\begin{pgfscope}%
\pgfsys@transformshift{5.267963in}{0.544166in}%
\pgfsys@useobject{currentmarker}{}%
\end{pgfscope}%
\end{pgfscope}%
\begin{pgfscope}%
\definecolor{textcolor}{rgb}{0.000000,0.000000,0.000000}%
\pgfsetstrokecolor{textcolor}%
\pgfsetfillcolor{textcolor}%
\pgftext[x=5.267963in,y=0.495555in,,top]{\color{textcolor}\rmfamily\fontsize{10.000000}{12.000000}\selectfont \(\displaystyle {160}\)}%
\end{pgfscope}%
\begin{pgfscope}%
\definecolor{textcolor}{rgb}{0.000000,0.000000,0.000000}%
\pgfsetstrokecolor{textcolor}%
\pgfsetfillcolor{textcolor}%
\pgftext[x=3.230250in,y=0.316666in,,top]{\color{textcolor}\rmfamily\fontsize{12.000000}{14.400000}\selectfont Temps [s]}%
\end{pgfscope}%
\begin{pgfscope}%
\pgfpathrectangle{\pgfqpoint{0.610501in}{0.544166in}}{\pgfqpoint{5.239499in}{0.941890in}}%
\pgfusepath{clip}%
\pgfsetbuttcap%
\pgfsetroundjoin%
\pgfsetlinewidth{0.501875pt}%
\definecolor{currentstroke}{rgb}{0.698039,0.698039,0.698039}%
\pgfsetstrokecolor{currentstroke}%
\pgfsetdash{{1.850000pt}{0.800000pt}}{0.000000pt}%
\pgfpathmoveto{\pgfqpoint{0.610501in}{0.673675in}}%
\pgfpathlineto{\pgfqpoint{5.850000in}{0.673675in}}%
\pgfusepath{stroke}%
\end{pgfscope}%
\begin{pgfscope}%
\pgfsetbuttcap%
\pgfsetroundjoin%
\definecolor{currentfill}{rgb}{0.000000,0.000000,0.000000}%
\pgfsetfillcolor{currentfill}%
\pgfsetlinewidth{0.803000pt}%
\definecolor{currentstroke}{rgb}{0.000000,0.000000,0.000000}%
\pgfsetstrokecolor{currentstroke}%
\pgfsetdash{}{0pt}%
\pgfsys@defobject{currentmarker}{\pgfqpoint{0.000000in}{0.000000in}}{\pgfqpoint{0.048611in}{0.000000in}}{%
\pgfpathmoveto{\pgfqpoint{0.000000in}{0.000000in}}%
\pgfpathlineto{\pgfqpoint{0.048611in}{0.000000in}}%
\pgfusepath{stroke,fill}%
}%
\begin{pgfscope}%
\pgfsys@transformshift{0.610501in}{0.673675in}%
\pgfsys@useobject{currentmarker}{}%
\end{pgfscope}%
\end{pgfscope}%
\begin{pgfscope}%
\definecolor{textcolor}{rgb}{0.000000,0.000000,0.000000}%
\pgfsetstrokecolor{textcolor}%
\pgfsetfillcolor{textcolor}%
\pgftext[x=0.492445in, y=0.625481in, left, base]{\color{textcolor}\rmfamily\fontsize{10.000000}{12.000000}\selectfont \(\displaystyle {0}\)}%
\end{pgfscope}%
\begin{pgfscope}%
\pgfpathrectangle{\pgfqpoint{0.610501in}{0.544166in}}{\pgfqpoint{5.239499in}{0.941890in}}%
\pgfusepath{clip}%
\pgfsetbuttcap%
\pgfsetroundjoin%
\pgfsetlinewidth{0.501875pt}%
\definecolor{currentstroke}{rgb}{0.698039,0.698039,0.698039}%
\pgfsetstrokecolor{currentstroke}%
\pgfsetdash{{1.850000pt}{0.800000pt}}{0.000000pt}%
\pgfpathmoveto{\pgfqpoint{0.610501in}{1.025340in}}%
\pgfpathlineto{\pgfqpoint{5.850000in}{1.025340in}}%
\pgfusepath{stroke}%
\end{pgfscope}%
\begin{pgfscope}%
\pgfsetbuttcap%
\pgfsetroundjoin%
\definecolor{currentfill}{rgb}{0.000000,0.000000,0.000000}%
\pgfsetfillcolor{currentfill}%
\pgfsetlinewidth{0.803000pt}%
\definecolor{currentstroke}{rgb}{0.000000,0.000000,0.000000}%
\pgfsetstrokecolor{currentstroke}%
\pgfsetdash{}{0pt}%
\pgfsys@defobject{currentmarker}{\pgfqpoint{0.000000in}{0.000000in}}{\pgfqpoint{0.048611in}{0.000000in}}{%
\pgfpathmoveto{\pgfqpoint{0.000000in}{0.000000in}}%
\pgfpathlineto{\pgfqpoint{0.048611in}{0.000000in}}%
\pgfusepath{stroke,fill}%
}%
\begin{pgfscope}%
\pgfsys@transformshift{0.610501in}{1.025340in}%
\pgfsys@useobject{currentmarker}{}%
\end{pgfscope}%
\end{pgfscope}%
\begin{pgfscope}%
\definecolor{textcolor}{rgb}{0.000000,0.000000,0.000000}%
\pgfsetstrokecolor{textcolor}%
\pgfsetfillcolor{textcolor}%
\pgftext[x=0.353555in, y=0.977146in, left, base]{\color{textcolor}\rmfamily\fontsize{10.000000}{12.000000}\selectfont \(\displaystyle {100}\)}%
\end{pgfscope}%
\begin{pgfscope}%
\pgfpathrectangle{\pgfqpoint{0.610501in}{0.544166in}}{\pgfqpoint{5.239499in}{0.941890in}}%
\pgfusepath{clip}%
\pgfsetbuttcap%
\pgfsetroundjoin%
\pgfsetlinewidth{0.501875pt}%
\definecolor{currentstroke}{rgb}{0.698039,0.698039,0.698039}%
\pgfsetstrokecolor{currentstroke}%
\pgfsetdash{{1.850000pt}{0.800000pt}}{0.000000pt}%
\pgfpathmoveto{\pgfqpoint{0.610501in}{1.377005in}}%
\pgfpathlineto{\pgfqpoint{5.850000in}{1.377005in}}%
\pgfusepath{stroke}%
\end{pgfscope}%
\begin{pgfscope}%
\pgfsetbuttcap%
\pgfsetroundjoin%
\definecolor{currentfill}{rgb}{0.000000,0.000000,0.000000}%
\pgfsetfillcolor{currentfill}%
\pgfsetlinewidth{0.803000pt}%
\definecolor{currentstroke}{rgb}{0.000000,0.000000,0.000000}%
\pgfsetstrokecolor{currentstroke}%
\pgfsetdash{}{0pt}%
\pgfsys@defobject{currentmarker}{\pgfqpoint{0.000000in}{0.000000in}}{\pgfqpoint{0.048611in}{0.000000in}}{%
\pgfpathmoveto{\pgfqpoint{0.000000in}{0.000000in}}%
\pgfpathlineto{\pgfqpoint{0.048611in}{0.000000in}}%
\pgfusepath{stroke,fill}%
}%
\begin{pgfscope}%
\pgfsys@transformshift{0.610501in}{1.377005in}%
\pgfsys@useobject{currentmarker}{}%
\end{pgfscope}%
\end{pgfscope}%
\begin{pgfscope}%
\definecolor{textcolor}{rgb}{0.000000,0.000000,0.000000}%
\pgfsetstrokecolor{textcolor}%
\pgfsetfillcolor{textcolor}%
\pgftext[x=0.353555in, y=1.328811in, left, base]{\color{textcolor}\rmfamily\fontsize{10.000000}{12.000000}\selectfont \(\displaystyle {200}\)}%
\end{pgfscope}%
\begin{pgfscope}%
\definecolor{textcolor}{rgb}{0.000000,0.000000,0.000000}%
\pgfsetstrokecolor{textcolor}%
\pgfsetfillcolor{textcolor}%
\pgftext[x=0.298000in,y=1.015111in,,bottom,rotate=90.000000]{\color{textcolor}\rmfamily\fontsize{12.000000}{14.400000}\selectfont FC Baer}%
\end{pgfscope}%
\begin{pgfscope}%
\pgfpathrectangle{\pgfqpoint{0.610501in}{0.544166in}}{\pgfqpoint{5.239499in}{0.941890in}}%
\pgfusepath{clip}%
\pgfsetrectcap%
\pgfsetroundjoin%
\pgfsetlinewidth{1.505625pt}%
\definecolor{currentstroke}{rgb}{0.121569,0.466667,0.705882}%
\pgfsetstrokecolor{currentstroke}%
\pgfsetdash{}{0pt}%
\pgfpathmoveto{\pgfqpoint{0.610501in}{0.673675in}}%
\pgfpathlineto{\pgfqpoint{0.639464in}{0.673675in}}%
\pgfpathlineto{\pgfqpoint{0.640192in}{0.668741in}}%
\pgfpathlineto{\pgfqpoint{0.640483in}{0.671077in}}%
\pgfpathlineto{\pgfqpoint{0.641065in}{0.687094in}}%
\pgfpathlineto{\pgfqpoint{0.641647in}{0.677273in}}%
\pgfpathlineto{\pgfqpoint{0.642521in}{0.671654in}}%
\pgfpathlineto{\pgfqpoint{0.642957in}{0.671910in}}%
\pgfpathlineto{\pgfqpoint{0.644413in}{0.674445in}}%
\pgfpathlineto{\pgfqpoint{0.644995in}{0.703394in}}%
\pgfpathlineto{\pgfqpoint{0.645577in}{0.765681in}}%
\pgfpathlineto{\pgfqpoint{0.646014in}{0.721272in}}%
\pgfpathlineto{\pgfqpoint{0.646741in}{0.676268in}}%
\pgfpathlineto{\pgfqpoint{0.647324in}{0.681460in}}%
\pgfpathlineto{\pgfqpoint{0.647469in}{0.681910in}}%
\pgfpathlineto{\pgfqpoint{0.647615in}{0.680894in}}%
\pgfpathlineto{\pgfqpoint{0.648634in}{0.667545in}}%
\pgfpathlineto{\pgfqpoint{0.649216in}{0.669860in}}%
\pgfpathlineto{\pgfqpoint{0.650089in}{0.678833in}}%
\pgfpathlineto{\pgfqpoint{0.650526in}{0.674245in}}%
\pgfpathlineto{\pgfqpoint{0.650671in}{0.673418in}}%
\pgfpathlineto{\pgfqpoint{0.650962in}{0.677324in}}%
\pgfpathlineto{\pgfqpoint{0.651836in}{0.748465in}}%
\pgfpathlineto{\pgfqpoint{0.652272in}{0.697928in}}%
\pgfpathlineto{\pgfqpoint{0.653000in}{0.669508in}}%
\pgfpathlineto{\pgfqpoint{0.653582in}{0.669912in}}%
\pgfpathlineto{\pgfqpoint{0.655038in}{0.668656in}}%
\pgfpathlineto{\pgfqpoint{0.655329in}{0.668822in}}%
\pgfpathlineto{\pgfqpoint{0.657075in}{0.670692in}}%
\pgfpathlineto{\pgfqpoint{0.657512in}{0.673828in}}%
\pgfpathlineto{\pgfqpoint{0.658094in}{0.670886in}}%
\pgfpathlineto{\pgfqpoint{0.658385in}{0.670527in}}%
\pgfpathlineto{\pgfqpoint{0.658531in}{0.671194in}}%
\pgfpathlineto{\pgfqpoint{0.659113in}{0.682737in}}%
\pgfpathlineto{\pgfqpoint{0.659549in}{0.674275in}}%
\pgfpathlineto{\pgfqpoint{0.660423in}{0.668977in}}%
\pgfpathlineto{\pgfqpoint{0.660859in}{0.669024in}}%
\pgfpathlineto{\pgfqpoint{0.661733in}{0.669401in}}%
\pgfpathlineto{\pgfqpoint{0.662169in}{0.681642in}}%
\pgfpathlineto{\pgfqpoint{0.662606in}{0.714887in}}%
\pgfpathlineto{\pgfqpoint{0.663188in}{0.677592in}}%
\pgfpathlineto{\pgfqpoint{0.663334in}{0.672985in}}%
\pgfpathlineto{\pgfqpoint{0.663625in}{0.681900in}}%
\pgfpathlineto{\pgfqpoint{0.664207in}{0.851465in}}%
\pgfpathlineto{\pgfqpoint{0.664789in}{0.737317in}}%
\pgfpathlineto{\pgfqpoint{0.665517in}{0.677375in}}%
\pgfpathlineto{\pgfqpoint{0.666099in}{0.690818in}}%
\pgfpathlineto{\pgfqpoint{0.666536in}{0.679363in}}%
\pgfpathlineto{\pgfqpoint{0.667263in}{0.670578in}}%
\pgfpathlineto{\pgfqpoint{0.667846in}{0.670633in}}%
\pgfpathlineto{\pgfqpoint{0.669592in}{0.670949in}}%
\pgfpathlineto{\pgfqpoint{0.671193in}{0.684944in}}%
\pgfpathlineto{\pgfqpoint{0.671630in}{0.706340in}}%
\pgfpathlineto{\pgfqpoint{0.672212in}{0.680993in}}%
\pgfpathlineto{\pgfqpoint{0.673522in}{0.673932in}}%
\pgfpathlineto{\pgfqpoint{0.674977in}{0.672155in}}%
\pgfpathlineto{\pgfqpoint{0.676578in}{0.670741in}}%
\pgfpathlineto{\pgfqpoint{0.678179in}{0.671716in}}%
\pgfpathlineto{\pgfqpoint{0.678761in}{0.673551in}}%
\pgfpathlineto{\pgfqpoint{0.679344in}{0.671907in}}%
\pgfpathlineto{\pgfqpoint{0.681090in}{0.671374in}}%
\pgfpathlineto{\pgfqpoint{0.683564in}{0.675073in}}%
\pgfpathlineto{\pgfqpoint{0.685311in}{0.703195in}}%
\pgfpathlineto{\pgfqpoint{0.685748in}{0.690461in}}%
\pgfpathlineto{\pgfqpoint{0.686184in}{0.681661in}}%
\pgfpathlineto{\pgfqpoint{0.686475in}{0.691134in}}%
\pgfpathlineto{\pgfqpoint{0.686912in}{0.710779in}}%
\pgfpathlineto{\pgfqpoint{0.687349in}{0.682507in}}%
\pgfpathlineto{\pgfqpoint{0.687931in}{0.669979in}}%
\pgfpathlineto{\pgfqpoint{0.688513in}{0.674366in}}%
\pgfpathlineto{\pgfqpoint{0.688659in}{0.675231in}}%
\pgfpathlineto{\pgfqpoint{0.689095in}{0.671933in}}%
\pgfpathlineto{\pgfqpoint{0.689386in}{0.670683in}}%
\pgfpathlineto{\pgfqpoint{0.689823in}{0.674165in}}%
\pgfpathlineto{\pgfqpoint{0.690405in}{0.690470in}}%
\pgfpathlineto{\pgfqpoint{0.690987in}{0.676711in}}%
\pgfpathlineto{\pgfqpoint{0.692297in}{0.670293in}}%
\pgfpathlineto{\pgfqpoint{0.692879in}{0.670032in}}%
\pgfpathlineto{\pgfqpoint{0.693171in}{0.670632in}}%
\pgfpathlineto{\pgfqpoint{0.694480in}{0.676577in}}%
\pgfpathlineto{\pgfqpoint{0.695208in}{0.752465in}}%
\pgfpathlineto{\pgfqpoint{0.695790in}{0.697266in}}%
\pgfpathlineto{\pgfqpoint{0.696373in}{0.672327in}}%
\pgfpathlineto{\pgfqpoint{0.697100in}{0.676001in}}%
\pgfpathlineto{\pgfqpoint{0.697682in}{0.668887in}}%
\pgfpathlineto{\pgfqpoint{0.697974in}{0.672101in}}%
\pgfpathlineto{\pgfqpoint{0.698556in}{0.705122in}}%
\pgfpathlineto{\pgfqpoint{0.699138in}{0.681029in}}%
\pgfpathlineto{\pgfqpoint{0.699283in}{0.679127in}}%
\pgfpathlineto{\pgfqpoint{0.699575in}{0.687170in}}%
\pgfpathlineto{\pgfqpoint{0.700302in}{0.771899in}}%
\pgfpathlineto{\pgfqpoint{0.700884in}{0.716027in}}%
\pgfpathlineto{\pgfqpoint{0.702340in}{0.672778in}}%
\pgfpathlineto{\pgfqpoint{0.702922in}{0.670583in}}%
\pgfpathlineto{\pgfqpoint{0.703504in}{0.672142in}}%
\pgfpathlineto{\pgfqpoint{0.704960in}{0.676017in}}%
\pgfpathlineto{\pgfqpoint{0.705251in}{0.674803in}}%
\pgfpathlineto{\pgfqpoint{0.706124in}{0.669620in}}%
\pgfpathlineto{\pgfqpoint{0.706706in}{0.670841in}}%
\pgfpathlineto{\pgfqpoint{0.706997in}{0.671246in}}%
\pgfpathlineto{\pgfqpoint{0.707434in}{0.670062in}}%
\pgfpathlineto{\pgfqpoint{0.709035in}{0.668621in}}%
\pgfpathlineto{\pgfqpoint{0.710054in}{0.668053in}}%
\pgfpathlineto{\pgfqpoint{0.710345in}{0.668298in}}%
\pgfpathlineto{\pgfqpoint{0.712674in}{0.672942in}}%
\pgfpathlineto{\pgfqpoint{0.713256in}{0.683044in}}%
\pgfpathlineto{\pgfqpoint{0.713692in}{0.675318in}}%
\pgfpathlineto{\pgfqpoint{0.714275in}{0.668749in}}%
\pgfpathlineto{\pgfqpoint{0.714857in}{0.673482in}}%
\pgfpathlineto{\pgfqpoint{0.715148in}{0.671803in}}%
\pgfpathlineto{\pgfqpoint{0.715730in}{0.668444in}}%
\pgfpathlineto{\pgfqpoint{0.716312in}{0.669424in}}%
\pgfpathlineto{\pgfqpoint{0.716894in}{0.671055in}}%
\pgfpathlineto{\pgfqpoint{0.717477in}{0.670010in}}%
\pgfpathlineto{\pgfqpoint{0.719805in}{0.669358in}}%
\pgfpathlineto{\pgfqpoint{0.720679in}{0.670277in}}%
\pgfpathlineto{\pgfqpoint{0.721406in}{0.675607in}}%
\pgfpathlineto{\pgfqpoint{0.721843in}{0.671680in}}%
\pgfpathlineto{\pgfqpoint{0.723153in}{0.669689in}}%
\pgfpathlineto{\pgfqpoint{0.724463in}{0.670554in}}%
\pgfpathlineto{\pgfqpoint{0.725191in}{0.673242in}}%
\pgfpathlineto{\pgfqpoint{0.725627in}{0.671700in}}%
\pgfpathlineto{\pgfqpoint{0.726355in}{0.669837in}}%
\pgfpathlineto{\pgfqpoint{0.726792in}{0.670548in}}%
\pgfpathlineto{\pgfqpoint{0.727519in}{0.684708in}}%
\pgfpathlineto{\pgfqpoint{0.728101in}{0.673189in}}%
\pgfpathlineto{\pgfqpoint{0.728538in}{0.670710in}}%
\pgfpathlineto{\pgfqpoint{0.729266in}{0.672293in}}%
\pgfpathlineto{\pgfqpoint{0.730576in}{0.664009in}}%
\pgfpathlineto{\pgfqpoint{0.731886in}{0.665501in}}%
\pgfpathlineto{\pgfqpoint{0.733341in}{0.666347in}}%
\pgfpathlineto{\pgfqpoint{0.733778in}{0.667120in}}%
\pgfpathlineto{\pgfqpoint{0.734069in}{0.674455in}}%
\pgfpathlineto{\pgfqpoint{0.734942in}{0.817389in}}%
\pgfpathlineto{\pgfqpoint{0.735524in}{0.727684in}}%
\pgfpathlineto{\pgfqpoint{0.735961in}{0.698003in}}%
\pgfpathlineto{\pgfqpoint{0.736543in}{0.734860in}}%
\pgfpathlineto{\pgfqpoint{0.736834in}{0.750862in}}%
\pgfpathlineto{\pgfqpoint{0.737271in}{0.717117in}}%
\pgfpathlineto{\pgfqpoint{0.738435in}{0.670036in}}%
\pgfpathlineto{\pgfqpoint{0.738726in}{0.670152in}}%
\pgfpathlineto{\pgfqpoint{0.741928in}{0.671670in}}%
\pgfpathlineto{\pgfqpoint{0.744257in}{0.690360in}}%
\pgfpathlineto{\pgfqpoint{0.744548in}{0.696721in}}%
\pgfpathlineto{\pgfqpoint{0.744985in}{0.685890in}}%
\pgfpathlineto{\pgfqpoint{0.746149in}{0.670929in}}%
\pgfpathlineto{\pgfqpoint{0.746440in}{0.671092in}}%
\pgfpathlineto{\pgfqpoint{0.747896in}{0.671789in}}%
\pgfpathlineto{\pgfqpoint{0.748187in}{0.671078in}}%
\pgfpathlineto{\pgfqpoint{0.748769in}{0.670216in}}%
\pgfpathlineto{\pgfqpoint{0.749206in}{0.671050in}}%
\pgfpathlineto{\pgfqpoint{0.749788in}{0.674210in}}%
\pgfpathlineto{\pgfqpoint{0.750370in}{0.671922in}}%
\pgfpathlineto{\pgfqpoint{0.750661in}{0.672888in}}%
\pgfpathlineto{\pgfqpoint{0.751534in}{0.696691in}}%
\pgfpathlineto{\pgfqpoint{0.751971in}{0.682260in}}%
\pgfpathlineto{\pgfqpoint{0.752553in}{0.670526in}}%
\pgfpathlineto{\pgfqpoint{0.753281in}{0.673173in}}%
\pgfpathlineto{\pgfqpoint{0.753718in}{0.671318in}}%
\pgfpathlineto{\pgfqpoint{0.754300in}{0.669882in}}%
\pgfpathlineto{\pgfqpoint{0.754882in}{0.670223in}}%
\pgfpathlineto{\pgfqpoint{0.757211in}{0.670652in}}%
\pgfpathlineto{\pgfqpoint{0.758084in}{0.671865in}}%
\pgfpathlineto{\pgfqpoint{0.758521in}{0.672705in}}%
\pgfpathlineto{\pgfqpoint{0.759103in}{0.671602in}}%
\pgfpathlineto{\pgfqpoint{0.759539in}{0.672675in}}%
\pgfpathlineto{\pgfqpoint{0.760413in}{0.681736in}}%
\pgfpathlineto{\pgfqpoint{0.760995in}{0.676146in}}%
\pgfpathlineto{\pgfqpoint{0.762014in}{0.671769in}}%
\pgfpathlineto{\pgfqpoint{0.762305in}{0.672492in}}%
\pgfpathlineto{\pgfqpoint{0.763032in}{0.686963in}}%
\pgfpathlineto{\pgfqpoint{0.763178in}{0.689189in}}%
\pgfpathlineto{\pgfqpoint{0.763615in}{0.679836in}}%
\pgfpathlineto{\pgfqpoint{0.764197in}{0.669862in}}%
\pgfpathlineto{\pgfqpoint{0.764633in}{0.676111in}}%
\pgfpathlineto{\pgfqpoint{0.765070in}{0.688652in}}%
\pgfpathlineto{\pgfqpoint{0.765507in}{0.674780in}}%
\pgfpathlineto{\pgfqpoint{0.766089in}{0.669419in}}%
\pgfpathlineto{\pgfqpoint{0.766671in}{0.671932in}}%
\pgfpathlineto{\pgfqpoint{0.766817in}{0.672165in}}%
\pgfpathlineto{\pgfqpoint{0.767108in}{0.671077in}}%
\pgfpathlineto{\pgfqpoint{0.767981in}{0.668422in}}%
\pgfpathlineto{\pgfqpoint{0.768563in}{0.668750in}}%
\pgfpathlineto{\pgfqpoint{0.769582in}{0.669113in}}%
\pgfpathlineto{\pgfqpoint{0.771183in}{0.680584in}}%
\pgfpathlineto{\pgfqpoint{0.770310in}{0.668517in}}%
\pgfpathlineto{\pgfqpoint{0.771620in}{0.673412in}}%
\pgfpathlineto{\pgfqpoint{0.772056in}{0.668947in}}%
\pgfpathlineto{\pgfqpoint{0.772784in}{0.671517in}}%
\pgfpathlineto{\pgfqpoint{0.773512in}{0.683956in}}%
\pgfpathlineto{\pgfqpoint{0.773948in}{0.694091in}}%
\pgfpathlineto{\pgfqpoint{0.774385in}{0.677699in}}%
\pgfpathlineto{\pgfqpoint{0.774676in}{0.672613in}}%
\pgfpathlineto{\pgfqpoint{0.775258in}{0.683405in}}%
\pgfpathlineto{\pgfqpoint{0.775404in}{0.683165in}}%
\pgfpathlineto{\pgfqpoint{0.776423in}{0.668535in}}%
\pgfpathlineto{\pgfqpoint{0.777005in}{0.669325in}}%
\pgfpathlineto{\pgfqpoint{0.777587in}{0.670710in}}%
\pgfpathlineto{\pgfqpoint{0.778169in}{0.669724in}}%
\pgfpathlineto{\pgfqpoint{0.779479in}{0.669785in}}%
\pgfpathlineto{\pgfqpoint{0.779916in}{0.672467in}}%
\pgfpathlineto{\pgfqpoint{0.780789in}{0.710584in}}%
\pgfpathlineto{\pgfqpoint{0.781371in}{0.683568in}}%
\pgfpathlineto{\pgfqpoint{0.782681in}{0.673234in}}%
\pgfpathlineto{\pgfqpoint{0.782972in}{0.673813in}}%
\pgfpathlineto{\pgfqpoint{0.783409in}{0.689252in}}%
\pgfpathlineto{\pgfqpoint{0.783991in}{0.742711in}}%
\pgfpathlineto{\pgfqpoint{0.784428in}{0.693389in}}%
\pgfpathlineto{\pgfqpoint{0.785010in}{0.669655in}}%
\pgfpathlineto{\pgfqpoint{0.785738in}{0.671603in}}%
\pgfpathlineto{\pgfqpoint{0.786611in}{0.666989in}}%
\pgfpathlineto{\pgfqpoint{0.787047in}{0.667107in}}%
\pgfpathlineto{\pgfqpoint{0.788357in}{0.668440in}}%
\pgfpathlineto{\pgfqpoint{0.788794in}{0.671126in}}%
\pgfpathlineto{\pgfqpoint{0.789522in}{0.668939in}}%
\pgfpathlineto{\pgfqpoint{0.791123in}{0.669496in}}%
\pgfpathlineto{\pgfqpoint{0.791850in}{0.671573in}}%
\pgfpathlineto{\pgfqpoint{0.792287in}{0.669641in}}%
\pgfpathlineto{\pgfqpoint{0.792724in}{0.669006in}}%
\pgfpathlineto{\pgfqpoint{0.793451in}{0.669270in}}%
\pgfpathlineto{\pgfqpoint{0.794179in}{0.669997in}}%
\pgfpathlineto{\pgfqpoint{0.796799in}{0.691606in}}%
\pgfpathlineto{\pgfqpoint{0.797090in}{0.686969in}}%
\pgfpathlineto{\pgfqpoint{0.798546in}{0.668710in}}%
\pgfpathlineto{\pgfqpoint{0.802766in}{0.670162in}}%
\pgfpathlineto{\pgfqpoint{0.803058in}{0.672954in}}%
\pgfpathlineto{\pgfqpoint{0.803785in}{0.768552in}}%
\pgfpathlineto{\pgfqpoint{0.804367in}{0.698638in}}%
\pgfpathlineto{\pgfqpoint{0.804513in}{0.690953in}}%
\pgfpathlineto{\pgfqpoint{0.804950in}{0.715806in}}%
\pgfpathlineto{\pgfqpoint{0.805386in}{0.757832in}}%
\pgfpathlineto{\pgfqpoint{0.805823in}{0.701405in}}%
\pgfpathlineto{\pgfqpoint{0.806551in}{0.671005in}}%
\pgfpathlineto{\pgfqpoint{0.807133in}{0.671191in}}%
\pgfpathlineto{\pgfqpoint{0.807715in}{0.671288in}}%
\pgfpathlineto{\pgfqpoint{0.808443in}{0.682107in}}%
\pgfpathlineto{\pgfqpoint{0.809170in}{0.673433in}}%
\pgfpathlineto{\pgfqpoint{0.809462in}{0.673957in}}%
\pgfpathlineto{\pgfqpoint{0.809898in}{0.675451in}}%
\pgfpathlineto{\pgfqpoint{0.810335in}{0.673507in}}%
\pgfpathlineto{\pgfqpoint{0.811208in}{0.671171in}}%
\pgfpathlineto{\pgfqpoint{0.811645in}{0.671429in}}%
\pgfpathlineto{\pgfqpoint{0.813537in}{0.673758in}}%
\pgfpathlineto{\pgfqpoint{0.813682in}{0.673395in}}%
\pgfpathlineto{\pgfqpoint{0.815138in}{0.671735in}}%
\pgfpathlineto{\pgfqpoint{0.817175in}{0.672838in}}%
\pgfpathlineto{\pgfqpoint{0.817467in}{0.673167in}}%
\pgfpathlineto{\pgfqpoint{0.817903in}{0.671912in}}%
\pgfpathlineto{\pgfqpoint{0.818194in}{0.671811in}}%
\pgfpathlineto{\pgfqpoint{0.818485in}{0.672651in}}%
\pgfpathlineto{\pgfqpoint{0.819504in}{0.690505in}}%
\pgfpathlineto{\pgfqpoint{0.820086in}{0.703662in}}%
\pgfpathlineto{\pgfqpoint{0.820523in}{0.692133in}}%
\pgfpathlineto{\pgfqpoint{0.820960in}{0.685139in}}%
\pgfpathlineto{\pgfqpoint{0.821251in}{0.693170in}}%
\pgfpathlineto{\pgfqpoint{0.821687in}{0.706973in}}%
\pgfpathlineto{\pgfqpoint{0.822124in}{0.686400in}}%
\pgfpathlineto{\pgfqpoint{0.822997in}{0.670987in}}%
\pgfpathlineto{\pgfqpoint{0.823434in}{0.671054in}}%
\pgfpathlineto{\pgfqpoint{0.824453in}{0.671689in}}%
\pgfpathlineto{\pgfqpoint{0.825326in}{0.687767in}}%
\pgfpathlineto{\pgfqpoint{0.826054in}{0.677158in}}%
\pgfpathlineto{\pgfqpoint{0.827946in}{0.672120in}}%
\pgfpathlineto{\pgfqpoint{0.829110in}{0.671724in}}%
\pgfpathlineto{\pgfqpoint{0.829256in}{0.671923in}}%
\pgfpathlineto{\pgfqpoint{0.829692in}{0.678252in}}%
\pgfpathlineto{\pgfqpoint{0.830275in}{0.702551in}}%
\pgfpathlineto{\pgfqpoint{0.830711in}{0.677570in}}%
\pgfpathlineto{\pgfqpoint{0.831148in}{0.670794in}}%
\pgfpathlineto{\pgfqpoint{0.831730in}{0.680311in}}%
\pgfpathlineto{\pgfqpoint{0.832458in}{0.669653in}}%
\pgfpathlineto{\pgfqpoint{0.833331in}{0.674037in}}%
\pgfpathlineto{\pgfqpoint{0.834932in}{0.681354in}}%
\pgfpathlineto{\pgfqpoint{0.835369in}{0.680104in}}%
\pgfpathlineto{\pgfqpoint{0.836970in}{0.674564in}}%
\pgfpathlineto{\pgfqpoint{0.838862in}{0.668793in}}%
\pgfpathlineto{\pgfqpoint{0.840026in}{0.669309in}}%
\pgfpathlineto{\pgfqpoint{0.840317in}{0.673158in}}%
\pgfpathlineto{\pgfqpoint{0.841045in}{0.714103in}}%
\pgfpathlineto{\pgfqpoint{0.841627in}{0.688039in}}%
\pgfpathlineto{\pgfqpoint{0.842209in}{0.689660in}}%
\pgfpathlineto{\pgfqpoint{0.843956in}{0.668620in}}%
\pgfpathlineto{\pgfqpoint{0.844392in}{0.670458in}}%
\pgfpathlineto{\pgfqpoint{0.845266in}{0.684940in}}%
\pgfpathlineto{\pgfqpoint{0.845848in}{0.675728in}}%
\pgfpathlineto{\pgfqpoint{0.846430in}{0.668938in}}%
\pgfpathlineto{\pgfqpoint{0.847158in}{0.671104in}}%
\pgfpathlineto{\pgfqpoint{0.847886in}{0.665453in}}%
\pgfpathlineto{\pgfqpoint{0.848177in}{0.667401in}}%
\pgfpathlineto{\pgfqpoint{0.850069in}{0.734184in}}%
\pgfpathlineto{\pgfqpoint{0.850505in}{0.811337in}}%
\pgfpathlineto{\pgfqpoint{0.850942in}{0.731560in}}%
\pgfpathlineto{\pgfqpoint{0.851670in}{0.669556in}}%
\pgfpathlineto{\pgfqpoint{0.852252in}{0.676508in}}%
\pgfpathlineto{\pgfqpoint{0.852397in}{0.676779in}}%
\pgfpathlineto{\pgfqpoint{0.853125in}{0.667285in}}%
\pgfpathlineto{\pgfqpoint{0.853562in}{0.669838in}}%
\pgfpathlineto{\pgfqpoint{0.854144in}{0.684803in}}%
\pgfpathlineto{\pgfqpoint{0.854726in}{0.671848in}}%
\pgfpathlineto{\pgfqpoint{0.854872in}{0.670957in}}%
\pgfpathlineto{\pgfqpoint{0.855163in}{0.673409in}}%
\pgfpathlineto{\pgfqpoint{0.855745in}{0.683869in}}%
\pgfpathlineto{\pgfqpoint{0.856182in}{0.675434in}}%
\pgfpathlineto{\pgfqpoint{0.856909in}{0.667202in}}%
\pgfpathlineto{\pgfqpoint{0.857637in}{0.667515in}}%
\pgfpathlineto{\pgfqpoint{0.858365in}{0.667981in}}%
\pgfpathlineto{\pgfqpoint{0.859093in}{0.675338in}}%
\pgfpathlineto{\pgfqpoint{0.859675in}{0.670430in}}%
\pgfpathlineto{\pgfqpoint{0.861567in}{0.668267in}}%
\pgfpathlineto{\pgfqpoint{0.862149in}{0.668979in}}%
\pgfpathlineto{\pgfqpoint{0.863022in}{0.679411in}}%
\pgfpathlineto{\pgfqpoint{0.863605in}{0.673864in}}%
\pgfpathlineto{\pgfqpoint{0.864187in}{0.670092in}}%
\pgfpathlineto{\pgfqpoint{0.864769in}{0.672481in}}%
\pgfpathlineto{\pgfqpoint{0.864914in}{0.672651in}}%
\pgfpathlineto{\pgfqpoint{0.865060in}{0.671986in}}%
\pgfpathlineto{\pgfqpoint{0.865642in}{0.668474in}}%
\pgfpathlineto{\pgfqpoint{0.866224in}{0.671491in}}%
\pgfpathlineto{\pgfqpoint{0.866661in}{0.677630in}}%
\pgfpathlineto{\pgfqpoint{0.867243in}{0.671644in}}%
\pgfpathlineto{\pgfqpoint{0.867389in}{0.671377in}}%
\pgfpathlineto{\pgfqpoint{0.867534in}{0.672042in}}%
\pgfpathlineto{\pgfqpoint{0.868116in}{0.690002in}}%
\pgfpathlineto{\pgfqpoint{0.868699in}{0.718298in}}%
\pgfpathlineto{\pgfqpoint{0.869135in}{0.702488in}}%
\pgfpathlineto{\pgfqpoint{0.869863in}{0.672824in}}%
\pgfpathlineto{\pgfqpoint{0.870591in}{0.675186in}}%
\pgfpathlineto{\pgfqpoint{0.871610in}{0.668203in}}%
\pgfpathlineto{\pgfqpoint{0.872192in}{0.668511in}}%
\pgfpathlineto{\pgfqpoint{0.875103in}{0.669255in}}%
\pgfpathlineto{\pgfqpoint{0.875976in}{0.670103in}}%
\pgfpathlineto{\pgfqpoint{0.876267in}{0.670739in}}%
\pgfpathlineto{\pgfqpoint{0.876995in}{0.669703in}}%
\pgfpathlineto{\pgfqpoint{0.885145in}{0.671179in}}%
\pgfpathlineto{\pgfqpoint{0.888638in}{0.672839in}}%
\pgfpathlineto{\pgfqpoint{0.889366in}{0.678187in}}%
\pgfpathlineto{\pgfqpoint{0.889948in}{0.674331in}}%
\pgfpathlineto{\pgfqpoint{0.890530in}{0.672370in}}%
\pgfpathlineto{\pgfqpoint{0.890967in}{0.673649in}}%
\pgfpathlineto{\pgfqpoint{0.891695in}{0.688691in}}%
\pgfpathlineto{\pgfqpoint{0.892131in}{0.699560in}}%
\pgfpathlineto{\pgfqpoint{0.892568in}{0.686243in}}%
\pgfpathlineto{\pgfqpoint{0.893005in}{0.674610in}}%
\pgfpathlineto{\pgfqpoint{0.893296in}{0.682063in}}%
\pgfpathlineto{\pgfqpoint{0.894024in}{0.806969in}}%
\pgfpathlineto{\pgfqpoint{0.894606in}{0.709217in}}%
\pgfpathlineto{\pgfqpoint{0.895188in}{0.677462in}}%
\pgfpathlineto{\pgfqpoint{0.895770in}{0.699037in}}%
\pgfpathlineto{\pgfqpoint{0.896061in}{0.712449in}}%
\pgfpathlineto{\pgfqpoint{0.896498in}{0.695091in}}%
\pgfpathlineto{\pgfqpoint{0.897226in}{0.671974in}}%
\pgfpathlineto{\pgfqpoint{0.897808in}{0.674284in}}%
\pgfpathlineto{\pgfqpoint{0.897953in}{0.674503in}}%
\pgfpathlineto{\pgfqpoint{0.898244in}{0.673225in}}%
\pgfpathlineto{\pgfqpoint{0.898972in}{0.672046in}}%
\pgfpathlineto{\pgfqpoint{0.899554in}{0.672186in}}%
\pgfpathlineto{\pgfqpoint{0.901301in}{0.673608in}}%
\pgfpathlineto{\pgfqpoint{0.902029in}{0.683456in}}%
\pgfpathlineto{\pgfqpoint{0.902320in}{0.687165in}}%
\pgfpathlineto{\pgfqpoint{0.902902in}{0.681381in}}%
\pgfpathlineto{\pgfqpoint{0.903047in}{0.680918in}}%
\pgfpathlineto{\pgfqpoint{0.903193in}{0.681931in}}%
\pgfpathlineto{\pgfqpoint{0.903921in}{0.699279in}}%
\pgfpathlineto{\pgfqpoint{0.904357in}{0.685921in}}%
\pgfpathlineto{\pgfqpoint{0.905231in}{0.672641in}}%
\pgfpathlineto{\pgfqpoint{0.905813in}{0.673662in}}%
\pgfpathlineto{\pgfqpoint{0.906832in}{0.679049in}}%
\pgfpathlineto{\pgfqpoint{0.907705in}{0.677526in}}%
\pgfpathlineto{\pgfqpoint{0.907996in}{0.677699in}}%
\pgfpathlineto{\pgfqpoint{0.908142in}{0.677298in}}%
\pgfpathlineto{\pgfqpoint{0.910034in}{0.671023in}}%
\pgfpathlineto{\pgfqpoint{0.910179in}{0.671180in}}%
\pgfpathlineto{\pgfqpoint{0.910616in}{0.674691in}}%
\pgfpathlineto{\pgfqpoint{0.911344in}{0.697364in}}%
\pgfpathlineto{\pgfqpoint{0.911926in}{0.679689in}}%
\pgfpathlineto{\pgfqpoint{0.913236in}{0.671032in}}%
\pgfpathlineto{\pgfqpoint{0.913527in}{0.669801in}}%
\pgfpathlineto{\pgfqpoint{0.913963in}{0.671405in}}%
\pgfpathlineto{\pgfqpoint{0.915564in}{0.696262in}}%
\pgfpathlineto{\pgfqpoint{0.916147in}{0.765507in}}%
\pgfpathlineto{\pgfqpoint{0.916583in}{0.722445in}}%
\pgfpathlineto{\pgfqpoint{0.917311in}{0.674203in}}%
\pgfpathlineto{\pgfqpoint{0.917748in}{0.691415in}}%
\pgfpathlineto{\pgfqpoint{0.918184in}{0.714900in}}%
\pgfpathlineto{\pgfqpoint{0.918621in}{0.682726in}}%
\pgfpathlineto{\pgfqpoint{0.919203in}{0.668117in}}%
\pgfpathlineto{\pgfqpoint{0.919785in}{0.670794in}}%
\pgfpathlineto{\pgfqpoint{0.919931in}{0.671298in}}%
\pgfpathlineto{\pgfqpoint{0.920367in}{0.669106in}}%
\pgfpathlineto{\pgfqpoint{0.920804in}{0.668048in}}%
\pgfpathlineto{\pgfqpoint{0.921532in}{0.668789in}}%
\pgfpathlineto{\pgfqpoint{0.922696in}{0.669865in}}%
\pgfpathlineto{\pgfqpoint{0.923133in}{0.669325in}}%
\pgfpathlineto{\pgfqpoint{0.923715in}{0.669024in}}%
\pgfpathlineto{\pgfqpoint{0.924152in}{0.669507in}}%
\pgfpathlineto{\pgfqpoint{0.925461in}{0.671935in}}%
\pgfpathlineto{\pgfqpoint{0.926044in}{0.670403in}}%
\pgfpathlineto{\pgfqpoint{0.926626in}{0.669631in}}%
\pgfpathlineto{\pgfqpoint{0.927062in}{0.670316in}}%
\pgfpathlineto{\pgfqpoint{0.927499in}{0.671691in}}%
\pgfpathlineto{\pgfqpoint{0.927936in}{0.670141in}}%
\pgfpathlineto{\pgfqpoint{0.928663in}{0.669001in}}%
\pgfpathlineto{\pgfqpoint{0.929246in}{0.669290in}}%
\pgfpathlineto{\pgfqpoint{0.931138in}{0.669308in}}%
\pgfpathlineto{\pgfqpoint{0.933612in}{0.671235in}}%
\pgfpathlineto{\pgfqpoint{0.934340in}{0.678640in}}%
\pgfpathlineto{\pgfqpoint{0.934922in}{0.673492in}}%
\pgfpathlineto{\pgfqpoint{0.935504in}{0.671046in}}%
\pgfpathlineto{\pgfqpoint{0.936232in}{0.671444in}}%
\pgfpathlineto{\pgfqpoint{0.937542in}{0.670706in}}%
\pgfpathlineto{\pgfqpoint{0.944382in}{0.673515in}}%
\pgfpathlineto{\pgfqpoint{0.945256in}{0.672984in}}%
\pgfpathlineto{\pgfqpoint{0.945838in}{0.680372in}}%
\pgfpathlineto{\pgfqpoint{0.946129in}{0.683429in}}%
\pgfpathlineto{\pgfqpoint{0.946566in}{0.674633in}}%
\pgfpathlineto{\pgfqpoint{0.948021in}{0.668651in}}%
\pgfpathlineto{\pgfqpoint{0.952970in}{0.670546in}}%
\pgfpathlineto{\pgfqpoint{0.953697in}{0.671618in}}%
\pgfpathlineto{\pgfqpoint{0.954134in}{0.673146in}}%
\pgfpathlineto{\pgfqpoint{0.954571in}{0.671406in}}%
\pgfpathlineto{\pgfqpoint{0.955153in}{0.670714in}}%
\pgfpathlineto{\pgfqpoint{0.955735in}{0.670952in}}%
\pgfpathlineto{\pgfqpoint{0.956608in}{0.672852in}}%
\pgfpathlineto{\pgfqpoint{0.956754in}{0.673250in}}%
\pgfpathlineto{\pgfqpoint{0.957190in}{0.671656in}}%
\pgfpathlineto{\pgfqpoint{0.957336in}{0.671366in}}%
\pgfpathlineto{\pgfqpoint{0.957481in}{0.671686in}}%
\pgfpathlineto{\pgfqpoint{0.957773in}{0.689658in}}%
\pgfpathlineto{\pgfqpoint{0.958209in}{0.802623in}}%
\pgfpathlineto{\pgfqpoint{0.958791in}{0.689777in}}%
\pgfpathlineto{\pgfqpoint{0.959082in}{0.676756in}}%
\pgfpathlineto{\pgfqpoint{0.959665in}{0.698647in}}%
\pgfpathlineto{\pgfqpoint{0.959810in}{0.696105in}}%
\pgfpathlineto{\pgfqpoint{0.960683in}{0.671252in}}%
\pgfpathlineto{\pgfqpoint{0.961266in}{0.671336in}}%
\pgfpathlineto{\pgfqpoint{0.961848in}{0.672176in}}%
\pgfpathlineto{\pgfqpoint{0.962284in}{0.674089in}}%
\pgfpathlineto{\pgfqpoint{0.962867in}{0.671800in}}%
\pgfpathlineto{\pgfqpoint{0.963303in}{0.671771in}}%
\pgfpathlineto{\pgfqpoint{0.963449in}{0.672115in}}%
\pgfpathlineto{\pgfqpoint{0.964031in}{0.681625in}}%
\pgfpathlineto{\pgfqpoint{0.964468in}{0.693498in}}%
\pgfpathlineto{\pgfqpoint{0.965050in}{0.678845in}}%
\pgfpathlineto{\pgfqpoint{0.965341in}{0.674403in}}%
\pgfpathlineto{\pgfqpoint{0.965778in}{0.679988in}}%
\pgfpathlineto{\pgfqpoint{0.966360in}{0.698564in}}%
\pgfpathlineto{\pgfqpoint{0.966796in}{0.682290in}}%
\pgfpathlineto{\pgfqpoint{0.968106in}{0.671478in}}%
\pgfpathlineto{\pgfqpoint{0.971308in}{0.671515in}}%
\pgfpathlineto{\pgfqpoint{0.974074in}{0.673381in}}%
\pgfpathlineto{\pgfqpoint{0.974365in}{0.681526in}}%
\pgfpathlineto{\pgfqpoint{0.975093in}{0.845595in}}%
\pgfpathlineto{\pgfqpoint{0.975675in}{0.727843in}}%
\pgfpathlineto{\pgfqpoint{0.976111in}{0.683539in}}%
\pgfpathlineto{\pgfqpoint{0.976839in}{0.703774in}}%
\pgfpathlineto{\pgfqpoint{0.977858in}{0.671899in}}%
\pgfpathlineto{\pgfqpoint{0.978877in}{0.673989in}}%
\pgfpathlineto{\pgfqpoint{0.980332in}{0.672501in}}%
\pgfpathlineto{\pgfqpoint{0.981060in}{0.671981in}}%
\pgfpathlineto{\pgfqpoint{0.981351in}{0.672672in}}%
\pgfpathlineto{\pgfqpoint{0.981933in}{0.684582in}}%
\pgfpathlineto{\pgfqpoint{0.982515in}{0.704275in}}%
\pgfpathlineto{\pgfqpoint{0.983098in}{0.688215in}}%
\pgfpathlineto{\pgfqpoint{0.983534in}{0.683649in}}%
\pgfpathlineto{\pgfqpoint{0.984262in}{0.686207in}}%
\pgfpathlineto{\pgfqpoint{0.987027in}{0.670788in}}%
\pgfpathlineto{\pgfqpoint{0.987901in}{0.671109in}}%
\pgfpathlineto{\pgfqpoint{0.988483in}{0.671336in}}%
\pgfpathlineto{\pgfqpoint{0.988919in}{0.670641in}}%
\pgfpathlineto{\pgfqpoint{0.989502in}{0.670938in}}%
\pgfpathlineto{\pgfqpoint{0.989647in}{0.671579in}}%
\pgfpathlineto{\pgfqpoint{0.990084in}{0.673592in}}%
\pgfpathlineto{\pgfqpoint{0.990666in}{0.671033in}}%
\pgfpathlineto{\pgfqpoint{0.990957in}{0.671272in}}%
\pgfpathlineto{\pgfqpoint{0.991103in}{0.671984in}}%
\pgfpathlineto{\pgfqpoint{0.992121in}{0.682645in}}%
\pgfpathlineto{\pgfqpoint{0.992704in}{0.676721in}}%
\pgfpathlineto{\pgfqpoint{0.994159in}{0.672006in}}%
\pgfpathlineto{\pgfqpoint{0.995032in}{0.670823in}}%
\pgfpathlineto{\pgfqpoint{0.995323in}{0.671435in}}%
\pgfpathlineto{\pgfqpoint{0.995906in}{0.674041in}}%
\pgfpathlineto{\pgfqpoint{0.996342in}{0.671213in}}%
\pgfpathlineto{\pgfqpoint{0.996779in}{0.670255in}}%
\pgfpathlineto{\pgfqpoint{0.997070in}{0.671370in}}%
\pgfpathlineto{\pgfqpoint{0.997652in}{0.676670in}}%
\pgfpathlineto{\pgfqpoint{0.998089in}{0.670935in}}%
\pgfpathlineto{\pgfqpoint{0.998380in}{0.669751in}}%
\pgfpathlineto{\pgfqpoint{0.998671in}{0.672008in}}%
\pgfpathlineto{\pgfqpoint{0.999253in}{0.746211in}}%
\pgfpathlineto{\pgfqpoint{0.999544in}{0.763523in}}%
\pgfpathlineto{\pgfqpoint{0.999981in}{0.706928in}}%
\pgfpathlineto{\pgfqpoint{1.000417in}{0.681404in}}%
\pgfpathlineto{\pgfqpoint{1.001000in}{0.701423in}}%
\pgfpathlineto{\pgfqpoint{1.001145in}{0.704988in}}%
\pgfpathlineto{\pgfqpoint{1.001582in}{0.692310in}}%
\pgfpathlineto{\pgfqpoint{1.003037in}{0.670209in}}%
\pgfpathlineto{\pgfqpoint{1.004784in}{0.670137in}}%
\pgfpathlineto{\pgfqpoint{1.005220in}{0.670084in}}%
\pgfpathlineto{\pgfqpoint{1.005512in}{0.669391in}}%
\pgfpathlineto{\pgfqpoint{1.006094in}{0.668980in}}%
\pgfpathlineto{\pgfqpoint{1.006530in}{0.669572in}}%
\pgfpathlineto{\pgfqpoint{1.008714in}{0.671477in}}%
\pgfpathlineto{\pgfqpoint{1.009296in}{0.672358in}}%
\pgfpathlineto{\pgfqpoint{1.011916in}{0.689436in}}%
\pgfpathlineto{\pgfqpoint{1.012643in}{0.738552in}}%
\pgfpathlineto{\pgfqpoint{1.013225in}{0.701415in}}%
\pgfpathlineto{\pgfqpoint{1.014099in}{0.671069in}}%
\pgfpathlineto{\pgfqpoint{1.014535in}{0.673927in}}%
\pgfpathlineto{\pgfqpoint{1.015118in}{0.677657in}}%
\pgfpathlineto{\pgfqpoint{1.015554in}{0.673170in}}%
\pgfpathlineto{\pgfqpoint{1.016282in}{0.669087in}}%
\pgfpathlineto{\pgfqpoint{1.016864in}{0.669980in}}%
\pgfpathlineto{\pgfqpoint{1.019630in}{0.673305in}}%
\pgfpathlineto{\pgfqpoint{1.019775in}{0.673905in}}%
\pgfpathlineto{\pgfqpoint{1.020066in}{0.672204in}}%
\pgfpathlineto{\pgfqpoint{1.020648in}{0.668529in}}%
\pgfpathlineto{\pgfqpoint{1.021376in}{0.669053in}}%
\pgfpathlineto{\pgfqpoint{1.024141in}{0.670940in}}%
\pgfpathlineto{\pgfqpoint{1.024724in}{0.673385in}}%
\pgfpathlineto{\pgfqpoint{1.027052in}{0.704326in}}%
\pgfpathlineto{\pgfqpoint{1.027343in}{0.697390in}}%
\pgfpathlineto{\pgfqpoint{1.028799in}{0.668605in}}%
\pgfpathlineto{\pgfqpoint{1.029527in}{0.668661in}}%
\pgfpathlineto{\pgfqpoint{1.029672in}{0.668858in}}%
\pgfpathlineto{\pgfqpoint{1.031419in}{0.675446in}}%
\pgfpathlineto{\pgfqpoint{1.033456in}{0.685478in}}%
\pgfpathlineto{\pgfqpoint{1.033747in}{0.683651in}}%
\pgfpathlineto{\pgfqpoint{1.035785in}{0.668719in}}%
\pgfpathlineto{\pgfqpoint{1.036658in}{0.669618in}}%
\pgfpathlineto{\pgfqpoint{1.038696in}{0.679658in}}%
\pgfpathlineto{\pgfqpoint{1.039278in}{0.697905in}}%
\pgfpathlineto{\pgfqpoint{1.039715in}{0.679608in}}%
\pgfpathlineto{\pgfqpoint{1.040297in}{0.667552in}}%
\pgfpathlineto{\pgfqpoint{1.041025in}{0.668166in}}%
\pgfpathlineto{\pgfqpoint{1.042480in}{0.669469in}}%
\pgfpathlineto{\pgfqpoint{1.042917in}{0.691682in}}%
\pgfpathlineto{\pgfqpoint{1.043645in}{0.861234in}}%
\pgfpathlineto{\pgfqpoint{1.044227in}{0.737608in}}%
\pgfpathlineto{\pgfqpoint{1.044518in}{0.713215in}}%
\pgfpathlineto{\pgfqpoint{1.044809in}{0.748981in}}%
\pgfpathlineto{\pgfqpoint{1.045246in}{0.865146in}}%
\pgfpathlineto{\pgfqpoint{1.045828in}{0.759807in}}%
\pgfpathlineto{\pgfqpoint{1.046847in}{0.673306in}}%
\pgfpathlineto{\pgfqpoint{1.047283in}{0.674685in}}%
\pgfpathlineto{\pgfqpoint{1.047574in}{0.674067in}}%
\pgfpathlineto{\pgfqpoint{1.048302in}{0.671662in}}%
\pgfpathlineto{\pgfqpoint{1.048884in}{0.672598in}}%
\pgfpathlineto{\pgfqpoint{1.049321in}{0.673720in}}%
\pgfpathlineto{\pgfqpoint{1.050049in}{0.672924in}}%
\pgfpathlineto{\pgfqpoint{1.051358in}{0.673082in}}%
\pgfpathlineto{\pgfqpoint{1.051941in}{0.680981in}}%
\pgfpathlineto{\pgfqpoint{1.052523in}{0.697412in}}%
\pgfpathlineto{\pgfqpoint{1.053105in}{0.685203in}}%
\pgfpathlineto{\pgfqpoint{1.054560in}{0.675422in}}%
\pgfpathlineto{\pgfqpoint{1.056016in}{0.670868in}}%
\pgfpathlineto{\pgfqpoint{1.056307in}{0.671008in}}%
\pgfpathlineto{\pgfqpoint{1.057762in}{0.673497in}}%
\pgfpathlineto{\pgfqpoint{1.058199in}{0.677131in}}%
\pgfpathlineto{\pgfqpoint{1.058781in}{0.673254in}}%
\pgfpathlineto{\pgfqpoint{1.059655in}{0.671125in}}%
\pgfpathlineto{\pgfqpoint{1.060091in}{0.671375in}}%
\pgfpathlineto{\pgfqpoint{1.062274in}{0.672862in}}%
\pgfpathlineto{\pgfqpoint{1.062420in}{0.672747in}}%
\pgfpathlineto{\pgfqpoint{1.063584in}{0.671846in}}%
\pgfpathlineto{\pgfqpoint{1.063875in}{0.672307in}}%
\pgfpathlineto{\pgfqpoint{1.064312in}{0.672901in}}%
\pgfpathlineto{\pgfqpoint{1.064749in}{0.671844in}}%
\pgfpathlineto{\pgfqpoint{1.066204in}{0.670324in}}%
\pgfpathlineto{\pgfqpoint{1.066932in}{0.670927in}}%
\pgfpathlineto{\pgfqpoint{1.068969in}{0.675402in}}%
\pgfpathlineto{\pgfqpoint{1.069115in}{0.674916in}}%
\pgfpathlineto{\pgfqpoint{1.070425in}{0.670128in}}%
\pgfpathlineto{\pgfqpoint{1.070862in}{0.670291in}}%
\pgfpathlineto{\pgfqpoint{1.074791in}{0.670066in}}%
\pgfpathlineto{\pgfqpoint{1.075519in}{0.668472in}}%
\pgfpathlineto{\pgfqpoint{1.076247in}{0.668941in}}%
\pgfpathlineto{\pgfqpoint{1.079303in}{0.670131in}}%
\pgfpathlineto{\pgfqpoint{1.081195in}{0.670708in}}%
\pgfpathlineto{\pgfqpoint{1.082651in}{0.674401in}}%
\pgfpathlineto{\pgfqpoint{1.083233in}{0.672103in}}%
\pgfpathlineto{\pgfqpoint{1.084106in}{0.669671in}}%
\pgfpathlineto{\pgfqpoint{1.084688in}{0.669859in}}%
\pgfpathlineto{\pgfqpoint{1.085562in}{0.668266in}}%
\pgfpathlineto{\pgfqpoint{1.085998in}{0.669545in}}%
\pgfpathlineto{\pgfqpoint{1.088473in}{0.685976in}}%
\pgfpathlineto{\pgfqpoint{1.088909in}{0.684071in}}%
\pgfpathlineto{\pgfqpoint{1.089055in}{0.684000in}}%
\pgfpathlineto{\pgfqpoint{1.089200in}{0.684943in}}%
\pgfpathlineto{\pgfqpoint{1.089783in}{0.706401in}}%
\pgfpathlineto{\pgfqpoint{1.090365in}{0.739271in}}%
\pgfpathlineto{\pgfqpoint{1.090801in}{0.713923in}}%
\pgfpathlineto{\pgfqpoint{1.092257in}{0.683612in}}%
\pgfpathlineto{\pgfqpoint{1.093858in}{0.669242in}}%
\pgfpathlineto{\pgfqpoint{1.094149in}{0.669945in}}%
\pgfpathlineto{\pgfqpoint{1.095459in}{0.676483in}}%
\pgfpathlineto{\pgfqpoint{1.096187in}{0.700534in}}%
\pgfpathlineto{\pgfqpoint{1.096623in}{0.687032in}}%
\pgfpathlineto{\pgfqpoint{1.097642in}{0.672161in}}%
\pgfpathlineto{\pgfqpoint{1.098079in}{0.672513in}}%
\pgfpathlineto{\pgfqpoint{1.099680in}{0.676863in}}%
\pgfpathlineto{\pgfqpoint{1.100116in}{0.680090in}}%
\pgfpathlineto{\pgfqpoint{1.100407in}{0.676081in}}%
\pgfpathlineto{\pgfqpoint{1.101426in}{0.667578in}}%
\pgfpathlineto{\pgfqpoint{1.101863in}{0.668119in}}%
\pgfpathlineto{\pgfqpoint{1.102299in}{0.669167in}}%
\pgfpathlineto{\pgfqpoint{1.102882in}{0.667856in}}%
\pgfpathlineto{\pgfqpoint{1.103173in}{0.667754in}}%
\pgfpathlineto{\pgfqpoint{1.103609in}{0.668443in}}%
\pgfpathlineto{\pgfqpoint{1.104046in}{0.668653in}}%
\pgfpathlineto{\pgfqpoint{1.104483in}{0.667946in}}%
\pgfpathlineto{\pgfqpoint{1.105793in}{0.666181in}}%
\pgfpathlineto{\pgfqpoint{1.106229in}{0.666434in}}%
\pgfpathlineto{\pgfqpoint{1.110596in}{0.671024in}}%
\pgfpathlineto{\pgfqpoint{1.111178in}{0.676393in}}%
\pgfpathlineto{\pgfqpoint{1.113215in}{0.704083in}}%
\pgfpathlineto{\pgfqpoint{1.113506in}{0.699864in}}%
\pgfpathlineto{\pgfqpoint{1.115253in}{0.672583in}}%
\pgfpathlineto{\pgfqpoint{1.115690in}{0.671240in}}%
\pgfpathlineto{\pgfqpoint{1.116126in}{0.673385in}}%
\pgfpathlineto{\pgfqpoint{1.116417in}{0.672345in}}%
\pgfpathlineto{\pgfqpoint{1.117145in}{0.667450in}}%
\pgfpathlineto{\pgfqpoint{1.117582in}{0.669549in}}%
\pgfpathlineto{\pgfqpoint{1.119037in}{0.684996in}}%
\pgfpathlineto{\pgfqpoint{1.119619in}{0.746677in}}%
\pgfpathlineto{\pgfqpoint{1.120056in}{0.699022in}}%
\pgfpathlineto{\pgfqpoint{1.120638in}{0.669562in}}%
\pgfpathlineto{\pgfqpoint{1.121220in}{0.685158in}}%
\pgfpathlineto{\pgfqpoint{1.121366in}{0.687038in}}%
\pgfpathlineto{\pgfqpoint{1.121657in}{0.680874in}}%
\pgfpathlineto{\pgfqpoint{1.122385in}{0.667673in}}%
\pgfpathlineto{\pgfqpoint{1.122821in}{0.671051in}}%
\pgfpathlineto{\pgfqpoint{1.123549in}{0.701109in}}%
\pgfpathlineto{\pgfqpoint{1.124131in}{0.678087in}}%
\pgfpathlineto{\pgfqpoint{1.124422in}{0.672878in}}%
\pgfpathlineto{\pgfqpoint{1.124859in}{0.679866in}}%
\pgfpathlineto{\pgfqpoint{1.125296in}{0.689636in}}%
\pgfpathlineto{\pgfqpoint{1.125732in}{0.678037in}}%
\pgfpathlineto{\pgfqpoint{1.126606in}{0.667805in}}%
\pgfpathlineto{\pgfqpoint{1.127042in}{0.668380in}}%
\pgfpathlineto{\pgfqpoint{1.129371in}{0.676698in}}%
\pgfpathlineto{\pgfqpoint{1.130244in}{0.757353in}}%
\pgfpathlineto{\pgfqpoint{1.130826in}{0.700716in}}%
\pgfpathlineto{\pgfqpoint{1.131554in}{0.672827in}}%
\pgfpathlineto{\pgfqpoint{1.131991in}{0.681608in}}%
\pgfpathlineto{\pgfqpoint{1.132282in}{0.688331in}}%
\pgfpathlineto{\pgfqpoint{1.132719in}{0.679043in}}%
\pgfpathlineto{\pgfqpoint{1.133446in}{0.667684in}}%
\pgfpathlineto{\pgfqpoint{1.134028in}{0.668218in}}%
\pgfpathlineto{\pgfqpoint{1.134611in}{0.674511in}}%
\pgfpathlineto{\pgfqpoint{1.135338in}{0.693231in}}%
\pgfpathlineto{\pgfqpoint{1.135775in}{0.680149in}}%
\pgfpathlineto{\pgfqpoint{1.136066in}{0.674544in}}%
\pgfpathlineto{\pgfqpoint{1.136357in}{0.686071in}}%
\pgfpathlineto{\pgfqpoint{1.137230in}{0.854506in}}%
\pgfpathlineto{\pgfqpoint{1.137813in}{0.759670in}}%
\pgfpathlineto{\pgfqpoint{1.138540in}{0.685979in}}%
\pgfpathlineto{\pgfqpoint{1.139123in}{0.711618in}}%
\pgfpathlineto{\pgfqpoint{1.139414in}{0.719674in}}%
\pgfpathlineto{\pgfqpoint{1.139705in}{0.700715in}}%
\pgfpathlineto{\pgfqpoint{1.140432in}{0.668984in}}%
\pgfpathlineto{\pgfqpoint{1.141160in}{0.669377in}}%
\pgfpathlineto{\pgfqpoint{1.141888in}{0.670187in}}%
\pgfpathlineto{\pgfqpoint{1.142325in}{0.673687in}}%
\pgfpathlineto{\pgfqpoint{1.142907in}{0.670553in}}%
\pgfpathlineto{\pgfqpoint{1.144217in}{0.670070in}}%
\pgfpathlineto{\pgfqpoint{1.146982in}{0.671473in}}%
\pgfpathlineto{\pgfqpoint{1.147855in}{0.683098in}}%
\pgfpathlineto{\pgfqpoint{1.148292in}{0.675562in}}%
\pgfpathlineto{\pgfqpoint{1.148583in}{0.672243in}}%
\pgfpathlineto{\pgfqpoint{1.149020in}{0.676674in}}%
\pgfpathlineto{\pgfqpoint{1.149456in}{0.688232in}}%
\pgfpathlineto{\pgfqpoint{1.150038in}{0.674796in}}%
\pgfpathlineto{\pgfqpoint{1.150621in}{0.669924in}}%
\pgfpathlineto{\pgfqpoint{1.151348in}{0.670727in}}%
\pgfpathlineto{\pgfqpoint{1.152076in}{0.671674in}}%
\pgfpathlineto{\pgfqpoint{1.152513in}{0.670908in}}%
\pgfpathlineto{\pgfqpoint{1.153095in}{0.670284in}}%
\pgfpathlineto{\pgfqpoint{1.153386in}{0.670865in}}%
\pgfpathlineto{\pgfqpoint{1.153823in}{0.672284in}}%
\pgfpathlineto{\pgfqpoint{1.154405in}{0.670560in}}%
\pgfpathlineto{\pgfqpoint{1.155424in}{0.670472in}}%
\pgfpathlineto{\pgfqpoint{1.155569in}{0.670583in}}%
\pgfpathlineto{\pgfqpoint{1.157170in}{0.673040in}}%
\pgfpathlineto{\pgfqpoint{1.157607in}{0.675449in}}%
\pgfpathlineto{\pgfqpoint{1.158043in}{0.672397in}}%
\pgfpathlineto{\pgfqpoint{1.158771in}{0.670269in}}%
\pgfpathlineto{\pgfqpoint{1.159353in}{0.670939in}}%
\pgfpathlineto{\pgfqpoint{1.159790in}{0.672122in}}%
\pgfpathlineto{\pgfqpoint{1.160372in}{0.670659in}}%
\pgfpathlineto{\pgfqpoint{1.160809in}{0.670784in}}%
\pgfpathlineto{\pgfqpoint{1.160954in}{0.671500in}}%
\pgfpathlineto{\pgfqpoint{1.161391in}{0.676710in}}%
\pgfpathlineto{\pgfqpoint{1.161973in}{0.671550in}}%
\pgfpathlineto{\pgfqpoint{1.162264in}{0.671093in}}%
\pgfpathlineto{\pgfqpoint{1.162555in}{0.672646in}}%
\pgfpathlineto{\pgfqpoint{1.162992in}{0.676350in}}%
\pgfpathlineto{\pgfqpoint{1.163429in}{0.672548in}}%
\pgfpathlineto{\pgfqpoint{1.164011in}{0.670501in}}%
\pgfpathlineto{\pgfqpoint{1.164593in}{0.671066in}}%
\pgfpathlineto{\pgfqpoint{1.165612in}{0.671063in}}%
\pgfpathlineto{\pgfqpoint{1.165757in}{0.671531in}}%
\pgfpathlineto{\pgfqpoint{1.168086in}{0.685446in}}%
\pgfpathlineto{\pgfqpoint{1.168959in}{0.817667in}}%
\pgfpathlineto{\pgfqpoint{1.169542in}{0.714045in}}%
\pgfpathlineto{\pgfqpoint{1.170560in}{0.675923in}}%
\pgfpathlineto{\pgfqpoint{1.170851in}{0.677505in}}%
\pgfpathlineto{\pgfqpoint{1.171434in}{0.683606in}}%
\pgfpathlineto{\pgfqpoint{1.171870in}{0.678651in}}%
\pgfpathlineto{\pgfqpoint{1.172598in}{0.669415in}}%
\pgfpathlineto{\pgfqpoint{1.173180in}{0.672191in}}%
\pgfpathlineto{\pgfqpoint{1.173617in}{0.680804in}}%
\pgfpathlineto{\pgfqpoint{1.174199in}{0.671986in}}%
\pgfpathlineto{\pgfqpoint{1.174490in}{0.670664in}}%
\pgfpathlineto{\pgfqpoint{1.174781in}{0.674519in}}%
\pgfpathlineto{\pgfqpoint{1.175363in}{0.691790in}}%
\pgfpathlineto{\pgfqpoint{1.175800in}{0.677407in}}%
\pgfpathlineto{\pgfqpoint{1.176382in}{0.668710in}}%
\pgfpathlineto{\pgfqpoint{1.177110in}{0.669462in}}%
\pgfpathlineto{\pgfqpoint{1.177547in}{0.669043in}}%
\pgfpathlineto{\pgfqpoint{1.177838in}{0.669552in}}%
\pgfpathlineto{\pgfqpoint{1.178565in}{0.682441in}}%
\pgfpathlineto{\pgfqpoint{1.179293in}{0.674063in}}%
\pgfpathlineto{\pgfqpoint{1.179439in}{0.673896in}}%
\pgfpathlineto{\pgfqpoint{1.179584in}{0.674719in}}%
\pgfpathlineto{\pgfqpoint{1.180312in}{0.683502in}}%
\pgfpathlineto{\pgfqpoint{1.180749in}{0.679041in}}%
\pgfpathlineto{\pgfqpoint{1.181622in}{0.668791in}}%
\pgfpathlineto{\pgfqpoint{1.182350in}{0.669263in}}%
\pgfpathlineto{\pgfqpoint{1.182932in}{0.669003in}}%
\pgfpathlineto{\pgfqpoint{1.183077in}{0.669208in}}%
\pgfpathlineto{\pgfqpoint{1.183514in}{0.677688in}}%
\pgfpathlineto{\pgfqpoint{1.183951in}{0.687461in}}%
\pgfpathlineto{\pgfqpoint{1.184533in}{0.675767in}}%
\pgfpathlineto{\pgfqpoint{1.184824in}{0.673763in}}%
\pgfpathlineto{\pgfqpoint{1.185552in}{0.676335in}}%
\pgfpathlineto{\pgfqpoint{1.186716in}{0.679950in}}%
\pgfpathlineto{\pgfqpoint{1.187153in}{0.677272in}}%
\pgfpathlineto{\pgfqpoint{1.188317in}{0.669851in}}%
\pgfpathlineto{\pgfqpoint{1.188899in}{0.669965in}}%
\pgfpathlineto{\pgfqpoint{1.191956in}{0.668326in}}%
\pgfpathlineto{\pgfqpoint{1.192392in}{0.669294in}}%
\pgfpathlineto{\pgfqpoint{1.193266in}{0.681376in}}%
\pgfpathlineto{\pgfqpoint{1.193848in}{0.672705in}}%
\pgfpathlineto{\pgfqpoint{1.195158in}{0.669927in}}%
\pgfpathlineto{\pgfqpoint{1.195594in}{0.670455in}}%
\pgfpathlineto{\pgfqpoint{1.196176in}{0.672414in}}%
\pgfpathlineto{\pgfqpoint{1.196468in}{0.670912in}}%
\pgfpathlineto{\pgfqpoint{1.197050in}{0.668159in}}%
\pgfpathlineto{\pgfqpoint{1.197777in}{0.668574in}}%
\pgfpathlineto{\pgfqpoint{1.199087in}{0.666160in}}%
\pgfpathlineto{\pgfqpoint{1.200252in}{0.667339in}}%
\pgfpathlineto{\pgfqpoint{1.203745in}{0.671097in}}%
\pgfpathlineto{\pgfqpoint{1.204036in}{0.673496in}}%
\pgfpathlineto{\pgfqpoint{1.204618in}{0.670257in}}%
\pgfpathlineto{\pgfqpoint{1.205782in}{0.669655in}}%
\pgfpathlineto{\pgfqpoint{1.205928in}{0.669702in}}%
\pgfpathlineto{\pgfqpoint{1.206365in}{0.671029in}}%
\pgfpathlineto{\pgfqpoint{1.206801in}{0.674735in}}%
\pgfpathlineto{\pgfqpoint{1.207238in}{0.671468in}}%
\pgfpathlineto{\pgfqpoint{1.207675in}{0.669107in}}%
\pgfpathlineto{\pgfqpoint{1.208402in}{0.670970in}}%
\pgfpathlineto{\pgfqpoint{1.209130in}{0.669494in}}%
\pgfpathlineto{\pgfqpoint{1.209421in}{0.670057in}}%
\pgfpathlineto{\pgfqpoint{1.210003in}{0.689904in}}%
\pgfpathlineto{\pgfqpoint{1.210294in}{0.700122in}}%
\pgfpathlineto{\pgfqpoint{1.210877in}{0.681237in}}%
\pgfpathlineto{\pgfqpoint{1.211168in}{0.676717in}}%
\pgfpathlineto{\pgfqpoint{1.211459in}{0.683470in}}%
\pgfpathlineto{\pgfqpoint{1.212186in}{0.738575in}}%
\pgfpathlineto{\pgfqpoint{1.212623in}{0.702411in}}%
\pgfpathlineto{\pgfqpoint{1.213351in}{0.670316in}}%
\pgfpathlineto{\pgfqpoint{1.213933in}{0.683370in}}%
\pgfpathlineto{\pgfqpoint{1.214224in}{0.691401in}}%
\pgfpathlineto{\pgfqpoint{1.214661in}{0.678277in}}%
\pgfpathlineto{\pgfqpoint{1.215243in}{0.669715in}}%
\pgfpathlineto{\pgfqpoint{1.215680in}{0.673358in}}%
\pgfpathlineto{\pgfqpoint{1.216262in}{0.687727in}}%
\pgfpathlineto{\pgfqpoint{1.216844in}{0.677956in}}%
\pgfpathlineto{\pgfqpoint{1.218299in}{0.671469in}}%
\pgfpathlineto{\pgfqpoint{1.219609in}{0.670200in}}%
\pgfpathlineto{\pgfqpoint{1.219755in}{0.670317in}}%
\pgfpathlineto{\pgfqpoint{1.220191in}{0.672706in}}%
\pgfpathlineto{\pgfqpoint{1.221210in}{0.687481in}}%
\pgfpathlineto{\pgfqpoint{1.221792in}{0.682797in}}%
\pgfpathlineto{\pgfqpoint{1.223830in}{0.671134in}}%
\pgfpathlineto{\pgfqpoint{1.224267in}{0.670183in}}%
\pgfpathlineto{\pgfqpoint{1.224703in}{0.671702in}}%
\pgfpathlineto{\pgfqpoint{1.225286in}{0.696172in}}%
\pgfpathlineto{\pgfqpoint{1.225722in}{0.719133in}}%
\pgfpathlineto{\pgfqpoint{1.226304in}{0.697274in}}%
\pgfpathlineto{\pgfqpoint{1.226887in}{0.680593in}}%
\pgfpathlineto{\pgfqpoint{1.227469in}{0.689112in}}%
\pgfpathlineto{\pgfqpoint{1.227614in}{0.691078in}}%
\pgfpathlineto{\pgfqpoint{1.228051in}{0.683717in}}%
\pgfpathlineto{\pgfqpoint{1.229361in}{0.668112in}}%
\pgfpathlineto{\pgfqpoint{1.229506in}{0.668116in}}%
\pgfpathlineto{\pgfqpoint{1.232854in}{0.669122in}}%
\pgfpathlineto{\pgfqpoint{1.234746in}{0.677035in}}%
\pgfpathlineto{\pgfqpoint{1.235474in}{0.718434in}}%
\pgfpathlineto{\pgfqpoint{1.236056in}{0.688173in}}%
\pgfpathlineto{\pgfqpoint{1.236784in}{0.669604in}}%
\pgfpathlineto{\pgfqpoint{1.237366in}{0.673436in}}%
\pgfpathlineto{\pgfqpoint{1.237511in}{0.673932in}}%
\pgfpathlineto{\pgfqpoint{1.238094in}{0.671974in}}%
\pgfpathlineto{\pgfqpoint{1.239695in}{0.667963in}}%
\pgfpathlineto{\pgfqpoint{1.239986in}{0.669257in}}%
\pgfpathlineto{\pgfqpoint{1.240568in}{0.673985in}}%
\pgfpathlineto{\pgfqpoint{1.241005in}{0.669297in}}%
\pgfpathlineto{\pgfqpoint{1.241732in}{0.666907in}}%
\pgfpathlineto{\pgfqpoint{1.242314in}{0.667338in}}%
\pgfpathlineto{\pgfqpoint{1.244061in}{0.669364in}}%
\pgfpathlineto{\pgfqpoint{1.244498in}{0.671652in}}%
\pgfpathlineto{\pgfqpoint{1.245080in}{0.668792in}}%
\pgfpathlineto{\pgfqpoint{1.245516in}{0.672047in}}%
\pgfpathlineto{\pgfqpoint{1.245808in}{0.674754in}}%
\pgfpathlineto{\pgfqpoint{1.246390in}{0.668902in}}%
\pgfpathlineto{\pgfqpoint{1.246826in}{0.668972in}}%
\pgfpathlineto{\pgfqpoint{1.247117in}{0.669529in}}%
\pgfpathlineto{\pgfqpoint{1.248718in}{0.669712in}}%
\pgfpathlineto{\pgfqpoint{1.249446in}{0.669391in}}%
\pgfpathlineto{\pgfqpoint{1.249592in}{0.669654in}}%
\pgfpathlineto{\pgfqpoint{1.250174in}{0.672031in}}%
\pgfpathlineto{\pgfqpoint{1.250756in}{0.669938in}}%
\pgfpathlineto{\pgfqpoint{1.251193in}{0.670880in}}%
\pgfpathlineto{\pgfqpoint{1.251775in}{0.675799in}}%
\pgfpathlineto{\pgfqpoint{1.252212in}{0.671140in}}%
\pgfpathlineto{\pgfqpoint{1.252503in}{0.670207in}}%
\pgfpathlineto{\pgfqpoint{1.253085in}{0.671797in}}%
\pgfpathlineto{\pgfqpoint{1.253230in}{0.672088in}}%
\pgfpathlineto{\pgfqpoint{1.253521in}{0.670785in}}%
\pgfpathlineto{\pgfqpoint{1.253813in}{0.670293in}}%
\pgfpathlineto{\pgfqpoint{1.254104in}{0.671152in}}%
\pgfpathlineto{\pgfqpoint{1.254686in}{0.687008in}}%
\pgfpathlineto{\pgfqpoint{1.255268in}{0.672456in}}%
\pgfpathlineto{\pgfqpoint{1.255414in}{0.671932in}}%
\pgfpathlineto{\pgfqpoint{1.255559in}{0.672895in}}%
\pgfpathlineto{\pgfqpoint{1.256141in}{0.683987in}}%
\pgfpathlineto{\pgfqpoint{1.256578in}{0.675265in}}%
\pgfpathlineto{\pgfqpoint{1.257160in}{0.670123in}}%
\pgfpathlineto{\pgfqpoint{1.257742in}{0.671902in}}%
\pgfpathlineto{\pgfqpoint{1.258179in}{0.674009in}}%
\pgfpathlineto{\pgfqpoint{1.258761in}{0.671493in}}%
\pgfpathlineto{\pgfqpoint{1.259052in}{0.670586in}}%
\pgfpathlineto{\pgfqpoint{1.259343in}{0.671652in}}%
\pgfpathlineto{\pgfqpoint{1.259780in}{0.709338in}}%
\pgfpathlineto{\pgfqpoint{1.260362in}{0.825151in}}%
\pgfpathlineto{\pgfqpoint{1.260799in}{0.744208in}}%
\pgfpathlineto{\pgfqpoint{1.261526in}{0.688134in}}%
\pgfpathlineto{\pgfqpoint{1.262109in}{0.700174in}}%
\pgfpathlineto{\pgfqpoint{1.262400in}{0.696902in}}%
\pgfpathlineto{\pgfqpoint{1.264437in}{0.672543in}}%
\pgfpathlineto{\pgfqpoint{1.264874in}{0.673354in}}%
\pgfpathlineto{\pgfqpoint{1.265456in}{0.682589in}}%
\pgfpathlineto{\pgfqpoint{1.267348in}{0.735667in}}%
\pgfpathlineto{\pgfqpoint{1.267639in}{0.721125in}}%
\pgfpathlineto{\pgfqpoint{1.269240in}{0.680152in}}%
\pgfpathlineto{\pgfqpoint{1.271569in}{0.670261in}}%
\pgfpathlineto{\pgfqpoint{1.273316in}{0.673828in}}%
\pgfpathlineto{\pgfqpoint{1.274334in}{0.713740in}}%
\pgfpathlineto{\pgfqpoint{1.275353in}{0.694049in}}%
\pgfpathlineto{\pgfqpoint{1.276809in}{0.690843in}}%
\pgfpathlineto{\pgfqpoint{1.277536in}{0.689629in}}%
\pgfpathlineto{\pgfqpoint{1.277682in}{0.689077in}}%
\pgfpathlineto{\pgfqpoint{1.277973in}{0.691199in}}%
\pgfpathlineto{\pgfqpoint{1.278846in}{0.753939in}}%
\pgfpathlineto{\pgfqpoint{1.279283in}{0.706641in}}%
\pgfpathlineto{\pgfqpoint{1.280593in}{0.667672in}}%
\pgfpathlineto{\pgfqpoint{1.281321in}{0.667904in}}%
\pgfpathlineto{\pgfqpoint{1.281466in}{0.668257in}}%
\pgfpathlineto{\pgfqpoint{1.283649in}{0.671751in}}%
\pgfpathlineto{\pgfqpoint{1.284668in}{0.670381in}}%
\pgfpathlineto{\pgfqpoint{1.287725in}{0.665479in}}%
\pgfpathlineto{\pgfqpoint{1.288016in}{0.665707in}}%
\pgfpathlineto{\pgfqpoint{1.288452in}{0.670251in}}%
\pgfpathlineto{\pgfqpoint{1.289471in}{0.740824in}}%
\pgfpathlineto{\pgfqpoint{1.290344in}{0.703806in}}%
\pgfpathlineto{\pgfqpoint{1.292819in}{0.671859in}}%
\pgfpathlineto{\pgfqpoint{1.293546in}{0.666330in}}%
\pgfpathlineto{\pgfqpoint{1.294129in}{0.666534in}}%
\pgfpathlineto{\pgfqpoint{1.294711in}{0.668177in}}%
\pgfpathlineto{\pgfqpoint{1.295730in}{0.680317in}}%
\pgfpathlineto{\pgfqpoint{1.296457in}{0.676167in}}%
\pgfpathlineto{\pgfqpoint{1.296749in}{0.676732in}}%
\pgfpathlineto{\pgfqpoint{1.297331in}{0.679960in}}%
\pgfpathlineto{\pgfqpoint{1.297622in}{0.676584in}}%
\pgfpathlineto{\pgfqpoint{1.298350in}{0.666915in}}%
\pgfpathlineto{\pgfqpoint{1.298932in}{0.670204in}}%
\pgfpathlineto{\pgfqpoint{1.299805in}{0.669038in}}%
\pgfpathlineto{\pgfqpoint{1.300387in}{0.695563in}}%
\pgfpathlineto{\pgfqpoint{1.300824in}{0.735402in}}%
\pgfpathlineto{\pgfqpoint{1.301406in}{0.697999in}}%
\pgfpathlineto{\pgfqpoint{1.302134in}{0.671941in}}%
\pgfpathlineto{\pgfqpoint{1.302716in}{0.674582in}}%
\pgfpathlineto{\pgfqpoint{1.303735in}{0.666880in}}%
\pgfpathlineto{\pgfqpoint{1.305045in}{0.668091in}}%
\pgfpathlineto{\pgfqpoint{1.305772in}{0.670045in}}%
\pgfpathlineto{\pgfqpoint{1.306355in}{0.671774in}}%
\pgfpathlineto{\pgfqpoint{1.306791in}{0.669938in}}%
\pgfpathlineto{\pgfqpoint{1.307373in}{0.668510in}}%
\pgfpathlineto{\pgfqpoint{1.307956in}{0.669349in}}%
\pgfpathlineto{\pgfqpoint{1.310430in}{0.671853in}}%
\pgfpathlineto{\pgfqpoint{1.310866in}{0.673668in}}%
\pgfpathlineto{\pgfqpoint{1.311303in}{0.671244in}}%
\pgfpathlineto{\pgfqpoint{1.311449in}{0.670681in}}%
\pgfpathlineto{\pgfqpoint{1.311740in}{0.671690in}}%
\pgfpathlineto{\pgfqpoint{1.312613in}{0.703367in}}%
\pgfpathlineto{\pgfqpoint{1.313050in}{0.681276in}}%
\pgfpathlineto{\pgfqpoint{1.313632in}{0.669397in}}%
\pgfpathlineto{\pgfqpoint{1.314360in}{0.670446in}}%
\pgfpathlineto{\pgfqpoint{1.315233in}{0.668958in}}%
\pgfpathlineto{\pgfqpoint{1.315815in}{0.669138in}}%
\pgfpathlineto{\pgfqpoint{1.317562in}{0.669920in}}%
\pgfpathlineto{\pgfqpoint{1.318144in}{0.679040in}}%
\pgfpathlineto{\pgfqpoint{1.318580in}{0.689332in}}%
\pgfpathlineto{\pgfqpoint{1.319163in}{0.676486in}}%
\pgfpathlineto{\pgfqpoint{1.319454in}{0.674066in}}%
\pgfpathlineto{\pgfqpoint{1.319745in}{0.681092in}}%
\pgfpathlineto{\pgfqpoint{1.320327in}{0.706998in}}%
\pgfpathlineto{\pgfqpoint{1.320764in}{0.680187in}}%
\pgfpathlineto{\pgfqpoint{1.321346in}{0.668937in}}%
\pgfpathlineto{\pgfqpoint{1.322073in}{0.670718in}}%
\pgfpathlineto{\pgfqpoint{1.322510in}{0.669924in}}%
\pgfpathlineto{\pgfqpoint{1.323238in}{0.669197in}}%
\pgfpathlineto{\pgfqpoint{1.323529in}{0.669750in}}%
\pgfpathlineto{\pgfqpoint{1.324402in}{0.679175in}}%
\pgfpathlineto{\pgfqpoint{1.324984in}{0.672353in}}%
\pgfpathlineto{\pgfqpoint{1.325275in}{0.671002in}}%
\pgfpathlineto{\pgfqpoint{1.325858in}{0.673747in}}%
\pgfpathlineto{\pgfqpoint{1.326149in}{0.675279in}}%
\pgfpathlineto{\pgfqpoint{1.326585in}{0.672514in}}%
\pgfpathlineto{\pgfqpoint{1.327168in}{0.669509in}}%
\pgfpathlineto{\pgfqpoint{1.327750in}{0.670336in}}%
\pgfpathlineto{\pgfqpoint{1.328477in}{0.676204in}}%
\pgfpathlineto{\pgfqpoint{1.328914in}{0.671897in}}%
\pgfpathlineto{\pgfqpoint{1.329205in}{0.670873in}}%
\pgfpathlineto{\pgfqpoint{1.329496in}{0.673688in}}%
\pgfpathlineto{\pgfqpoint{1.330515in}{0.713291in}}%
\pgfpathlineto{\pgfqpoint{1.331097in}{0.690980in}}%
\pgfpathlineto{\pgfqpoint{1.331825in}{0.676721in}}%
\pgfpathlineto{\pgfqpoint{1.332553in}{0.678519in}}%
\pgfpathlineto{\pgfqpoint{1.334008in}{0.668720in}}%
\pgfpathlineto{\pgfqpoint{1.334445in}{0.668790in}}%
\pgfpathlineto{\pgfqpoint{1.337647in}{0.670502in}}%
\pgfpathlineto{\pgfqpoint{1.338083in}{0.689856in}}%
\pgfpathlineto{\pgfqpoint{1.338666in}{0.730054in}}%
\pgfpathlineto{\pgfqpoint{1.339102in}{0.703395in}}%
\pgfpathlineto{\pgfqpoint{1.339539in}{0.689605in}}%
\pgfpathlineto{\pgfqpoint{1.339976in}{0.712317in}}%
\pgfpathlineto{\pgfqpoint{1.340558in}{0.753694in}}%
\pgfpathlineto{\pgfqpoint{1.340994in}{0.721957in}}%
\pgfpathlineto{\pgfqpoint{1.342013in}{0.671289in}}%
\pgfpathlineto{\pgfqpoint{1.342450in}{0.673362in}}%
\pgfpathlineto{\pgfqpoint{1.342741in}{0.674041in}}%
\pgfpathlineto{\pgfqpoint{1.343178in}{0.671699in}}%
\pgfpathlineto{\pgfqpoint{1.344487in}{0.669979in}}%
\pgfpathlineto{\pgfqpoint{1.349727in}{0.671132in}}%
\pgfpathlineto{\pgfqpoint{1.350455in}{0.680225in}}%
\pgfpathlineto{\pgfqpoint{1.351037in}{0.672396in}}%
\pgfpathlineto{\pgfqpoint{1.351328in}{0.671356in}}%
\pgfpathlineto{\pgfqpoint{1.351765in}{0.673309in}}%
\pgfpathlineto{\pgfqpoint{1.352056in}{0.674955in}}%
\pgfpathlineto{\pgfqpoint{1.352638in}{0.672051in}}%
\pgfpathlineto{\pgfqpoint{1.353220in}{0.670566in}}%
\pgfpathlineto{\pgfqpoint{1.353802in}{0.671813in}}%
\pgfpathlineto{\pgfqpoint{1.354385in}{0.674607in}}%
\pgfpathlineto{\pgfqpoint{1.354821in}{0.672007in}}%
\pgfpathlineto{\pgfqpoint{1.355258in}{0.671181in}}%
\pgfpathlineto{\pgfqpoint{1.355695in}{0.673017in}}%
\pgfpathlineto{\pgfqpoint{1.356422in}{0.678048in}}%
\pgfpathlineto{\pgfqpoint{1.356859in}{0.675522in}}%
\pgfpathlineto{\pgfqpoint{1.357732in}{0.671380in}}%
\pgfpathlineto{\pgfqpoint{1.358023in}{0.672424in}}%
\pgfpathlineto{\pgfqpoint{1.358897in}{0.690576in}}%
\pgfpathlineto{\pgfqpoint{1.359333in}{0.677731in}}%
\pgfpathlineto{\pgfqpoint{1.359770in}{0.671113in}}%
\pgfpathlineto{\pgfqpoint{1.360206in}{0.676502in}}%
\pgfpathlineto{\pgfqpoint{1.360643in}{0.684895in}}%
\pgfpathlineto{\pgfqpoint{1.361080in}{0.674511in}}%
\pgfpathlineto{\pgfqpoint{1.361516in}{0.670790in}}%
\pgfpathlineto{\pgfqpoint{1.362390in}{0.670942in}}%
\pgfpathlineto{\pgfqpoint{1.362826in}{0.670884in}}%
\pgfpathlineto{\pgfqpoint{1.363117in}{0.671671in}}%
\pgfpathlineto{\pgfqpoint{1.363554in}{0.673195in}}%
\pgfpathlineto{\pgfqpoint{1.364136in}{0.671341in}}%
\pgfpathlineto{\pgfqpoint{1.364427in}{0.672153in}}%
\pgfpathlineto{\pgfqpoint{1.365155in}{0.685714in}}%
\pgfpathlineto{\pgfqpoint{1.365737in}{0.675285in}}%
\pgfpathlineto{\pgfqpoint{1.366465in}{0.670596in}}%
\pgfpathlineto{\pgfqpoint{1.366902in}{0.672131in}}%
\pgfpathlineto{\pgfqpoint{1.367193in}{0.673252in}}%
\pgfpathlineto{\pgfqpoint{1.367629in}{0.671115in}}%
\pgfpathlineto{\pgfqpoint{1.368066in}{0.670048in}}%
\pgfpathlineto{\pgfqpoint{1.368357in}{0.670652in}}%
\pgfpathlineto{\pgfqpoint{1.369085in}{0.687367in}}%
\pgfpathlineto{\pgfqpoint{1.369812in}{0.676838in}}%
\pgfpathlineto{\pgfqpoint{1.369958in}{0.675840in}}%
\pgfpathlineto{\pgfqpoint{1.370249in}{0.678009in}}%
\pgfpathlineto{\pgfqpoint{1.370831in}{0.729991in}}%
\pgfpathlineto{\pgfqpoint{1.371268in}{0.784896in}}%
\pgfpathlineto{\pgfqpoint{1.371850in}{0.714606in}}%
\pgfpathlineto{\pgfqpoint{1.372578in}{0.674413in}}%
\pgfpathlineto{\pgfqpoint{1.373160in}{0.674557in}}%
\pgfpathlineto{\pgfqpoint{1.375198in}{0.668478in}}%
\pgfpathlineto{\pgfqpoint{1.376653in}{0.668573in}}%
\pgfpathlineto{\pgfqpoint{1.377090in}{0.672026in}}%
\pgfpathlineto{\pgfqpoint{1.377526in}{0.677108in}}%
\pgfpathlineto{\pgfqpoint{1.378109in}{0.671451in}}%
\pgfpathlineto{\pgfqpoint{1.378982in}{0.668883in}}%
\pgfpathlineto{\pgfqpoint{1.379418in}{0.669457in}}%
\pgfpathlineto{\pgfqpoint{1.379855in}{0.670312in}}%
\pgfpathlineto{\pgfqpoint{1.380292in}{0.669219in}}%
\pgfpathlineto{\pgfqpoint{1.380874in}{0.668058in}}%
\pgfpathlineto{\pgfqpoint{1.381456in}{0.668459in}}%
\pgfpathlineto{\pgfqpoint{1.385240in}{0.671514in}}%
\pgfpathlineto{\pgfqpoint{1.385677in}{0.669998in}}%
\pgfpathlineto{\pgfqpoint{1.385968in}{0.669999in}}%
\pgfpathlineto{\pgfqpoint{1.386114in}{0.670441in}}%
\pgfpathlineto{\pgfqpoint{1.386841in}{0.680095in}}%
\pgfpathlineto{\pgfqpoint{1.387423in}{0.672580in}}%
\pgfpathlineto{\pgfqpoint{1.387860in}{0.670555in}}%
\pgfpathlineto{\pgfqpoint{1.388297in}{0.673325in}}%
\pgfpathlineto{\pgfqpoint{1.388442in}{0.674294in}}%
\pgfpathlineto{\pgfqpoint{1.388879in}{0.670773in}}%
\pgfpathlineto{\pgfqpoint{1.389170in}{0.670042in}}%
\pgfpathlineto{\pgfqpoint{1.389461in}{0.671316in}}%
\pgfpathlineto{\pgfqpoint{1.390043in}{0.686440in}}%
\pgfpathlineto{\pgfqpoint{1.390480in}{0.673701in}}%
\pgfpathlineto{\pgfqpoint{1.390625in}{0.672414in}}%
\pgfpathlineto{\pgfqpoint{1.390917in}{0.677738in}}%
\pgfpathlineto{\pgfqpoint{1.391499in}{0.722316in}}%
\pgfpathlineto{\pgfqpoint{1.392081in}{0.682038in}}%
\pgfpathlineto{\pgfqpoint{1.393391in}{0.670462in}}%
\pgfpathlineto{\pgfqpoint{1.394701in}{0.670685in}}%
\pgfpathlineto{\pgfqpoint{1.396302in}{0.672616in}}%
\pgfpathlineto{\pgfqpoint{1.397175in}{0.707659in}}%
\pgfpathlineto{\pgfqpoint{1.397612in}{0.684690in}}%
\pgfpathlineto{\pgfqpoint{1.398194in}{0.672870in}}%
\pgfpathlineto{\pgfqpoint{1.398922in}{0.673106in}}%
\pgfpathlineto{\pgfqpoint{1.399795in}{0.670683in}}%
\pgfpathlineto{\pgfqpoint{1.400377in}{0.670800in}}%
\pgfpathlineto{\pgfqpoint{1.401396in}{0.670948in}}%
\pgfpathlineto{\pgfqpoint{1.401832in}{0.677208in}}%
\pgfpathlineto{\pgfqpoint{1.402415in}{0.698417in}}%
\pgfpathlineto{\pgfqpoint{1.402851in}{0.684670in}}%
\pgfpathlineto{\pgfqpoint{1.403433in}{0.672527in}}%
\pgfpathlineto{\pgfqpoint{1.404016in}{0.681741in}}%
\pgfpathlineto{\pgfqpoint{1.404161in}{0.682208in}}%
\pgfpathlineto{\pgfqpoint{1.404307in}{0.679819in}}%
\pgfpathlineto{\pgfqpoint{1.405035in}{0.669716in}}%
\pgfpathlineto{\pgfqpoint{1.405617in}{0.673985in}}%
\pgfpathlineto{\pgfqpoint{1.405908in}{0.676380in}}%
\pgfpathlineto{\pgfqpoint{1.406490in}{0.672998in}}%
\pgfpathlineto{\pgfqpoint{1.406927in}{0.671706in}}%
\pgfpathlineto{\pgfqpoint{1.407509in}{0.672704in}}%
\pgfpathlineto{\pgfqpoint{1.408528in}{0.678392in}}%
\pgfpathlineto{\pgfqpoint{1.409110in}{0.682727in}}%
\pgfpathlineto{\pgfqpoint{1.409546in}{0.679111in}}%
\pgfpathlineto{\pgfqpoint{1.410274in}{0.672674in}}%
\pgfpathlineto{\pgfqpoint{1.410856in}{0.674887in}}%
\pgfpathlineto{\pgfqpoint{1.411293in}{0.671132in}}%
\pgfpathlineto{\pgfqpoint{1.411730in}{0.668612in}}%
\pgfpathlineto{\pgfqpoint{1.412021in}{0.670437in}}%
\pgfpathlineto{\pgfqpoint{1.413767in}{0.728790in}}%
\pgfpathlineto{\pgfqpoint{1.414204in}{0.799766in}}%
\pgfpathlineto{\pgfqpoint{1.414641in}{0.721947in}}%
\pgfpathlineto{\pgfqpoint{1.415368in}{0.670476in}}%
\pgfpathlineto{\pgfqpoint{1.415950in}{0.673469in}}%
\pgfpathlineto{\pgfqpoint{1.416533in}{0.669481in}}%
\pgfpathlineto{\pgfqpoint{1.417115in}{0.668738in}}%
\pgfpathlineto{\pgfqpoint{1.417697in}{0.669404in}}%
\pgfpathlineto{\pgfqpoint{1.420462in}{0.673858in}}%
\pgfpathlineto{\pgfqpoint{1.421045in}{0.708367in}}%
\pgfpathlineto{\pgfqpoint{1.421336in}{0.726959in}}%
\pgfpathlineto{\pgfqpoint{1.421772in}{0.698819in}}%
\pgfpathlineto{\pgfqpoint{1.422646in}{0.670614in}}%
\pgfpathlineto{\pgfqpoint{1.423082in}{0.670719in}}%
\pgfpathlineto{\pgfqpoint{1.423664in}{0.669142in}}%
\pgfpathlineto{\pgfqpoint{1.424101in}{0.670707in}}%
\pgfpathlineto{\pgfqpoint{1.426575in}{0.696312in}}%
\pgfpathlineto{\pgfqpoint{1.426721in}{0.694012in}}%
\pgfpathlineto{\pgfqpoint{1.427740in}{0.668512in}}%
\pgfpathlineto{\pgfqpoint{1.428467in}{0.669605in}}%
\pgfpathlineto{\pgfqpoint{1.428904in}{0.668237in}}%
\pgfpathlineto{\pgfqpoint{1.429341in}{0.669867in}}%
\pgfpathlineto{\pgfqpoint{1.429777in}{0.674500in}}%
\pgfpathlineto{\pgfqpoint{1.430359in}{0.669351in}}%
\pgfpathlineto{\pgfqpoint{1.430505in}{0.669189in}}%
\pgfpathlineto{\pgfqpoint{1.430651in}{0.669696in}}%
\pgfpathlineto{\pgfqpoint{1.431233in}{0.674618in}}%
\pgfpathlineto{\pgfqpoint{1.431669in}{0.670486in}}%
\pgfpathlineto{\pgfqpoint{1.432106in}{0.668072in}}%
\pgfpathlineto{\pgfqpoint{1.432688in}{0.669994in}}%
\pgfpathlineto{\pgfqpoint{1.433270in}{0.678933in}}%
\pgfpathlineto{\pgfqpoint{1.433707in}{0.671984in}}%
\pgfpathlineto{\pgfqpoint{1.433998in}{0.669865in}}%
\pgfpathlineto{\pgfqpoint{1.434435in}{0.672781in}}%
\pgfpathlineto{\pgfqpoint{1.434871in}{0.676304in}}%
\pgfpathlineto{\pgfqpoint{1.435308in}{0.671650in}}%
\pgfpathlineto{\pgfqpoint{1.436036in}{0.668543in}}%
\pgfpathlineto{\pgfqpoint{1.436472in}{0.669603in}}%
\pgfpathlineto{\pgfqpoint{1.437200in}{0.675276in}}%
\pgfpathlineto{\pgfqpoint{1.437928in}{0.672363in}}%
\pgfpathlineto{\pgfqpoint{1.438801in}{0.672032in}}%
\pgfpathlineto{\pgfqpoint{1.440693in}{0.668468in}}%
\pgfpathlineto{\pgfqpoint{1.443895in}{0.669665in}}%
\pgfpathlineto{\pgfqpoint{1.444332in}{0.695334in}}%
\pgfpathlineto{\pgfqpoint{1.444914in}{0.747715in}}%
\pgfpathlineto{\pgfqpoint{1.445496in}{0.704248in}}%
\pgfpathlineto{\pgfqpoint{1.445642in}{0.698662in}}%
\pgfpathlineto{\pgfqpoint{1.445933in}{0.714941in}}%
\pgfpathlineto{\pgfqpoint{1.446515in}{0.864245in}}%
\pgfpathlineto{\pgfqpoint{1.447097in}{0.735058in}}%
\pgfpathlineto{\pgfqpoint{1.447825in}{0.671783in}}%
\pgfpathlineto{\pgfqpoint{1.448407in}{0.673204in}}%
\pgfpathlineto{\pgfqpoint{1.449135in}{0.670716in}}%
\pgfpathlineto{\pgfqpoint{1.449717in}{0.671859in}}%
\pgfpathlineto{\pgfqpoint{1.451900in}{0.695081in}}%
\pgfpathlineto{\pgfqpoint{1.452191in}{0.708391in}}%
\pgfpathlineto{\pgfqpoint{1.452628in}{0.687667in}}%
\pgfpathlineto{\pgfqpoint{1.453501in}{0.670563in}}%
\pgfpathlineto{\pgfqpoint{1.453938in}{0.670748in}}%
\pgfpathlineto{\pgfqpoint{1.454374in}{0.673276in}}%
\pgfpathlineto{\pgfqpoint{1.454957in}{0.681286in}}%
\pgfpathlineto{\pgfqpoint{1.455539in}{0.676602in}}%
\pgfpathlineto{\pgfqpoint{1.455684in}{0.676106in}}%
\pgfpathlineto{\pgfqpoint{1.456121in}{0.677443in}}%
\pgfpathlineto{\pgfqpoint{1.456703in}{0.679110in}}%
\pgfpathlineto{\pgfqpoint{1.457140in}{0.677824in}}%
\pgfpathlineto{\pgfqpoint{1.458450in}{0.670264in}}%
\pgfpathlineto{\pgfqpoint{1.459323in}{0.670525in}}%
\pgfpathlineto{\pgfqpoint{1.461361in}{0.671502in}}%
\pgfpathlineto{\pgfqpoint{1.462234in}{0.675354in}}%
\pgfpathlineto{\pgfqpoint{1.462816in}{0.673098in}}%
\pgfpathlineto{\pgfqpoint{1.463107in}{0.672460in}}%
\pgfpathlineto{\pgfqpoint{1.463544in}{0.673947in}}%
\pgfpathlineto{\pgfqpoint{1.463981in}{0.677091in}}%
\pgfpathlineto{\pgfqpoint{1.464417in}{0.673600in}}%
\pgfpathlineto{\pgfqpoint{1.465290in}{0.670466in}}%
\pgfpathlineto{\pgfqpoint{1.465727in}{0.670524in}}%
\pgfpathlineto{\pgfqpoint{1.467183in}{0.671624in}}%
\pgfpathlineto{\pgfqpoint{1.467910in}{0.672725in}}%
\pgfpathlineto{\pgfqpoint{1.468492in}{0.671817in}}%
\pgfpathlineto{\pgfqpoint{1.468929in}{0.672667in}}%
\pgfpathlineto{\pgfqpoint{1.469511in}{0.676814in}}%
\pgfpathlineto{\pgfqpoint{1.469948in}{0.673332in}}%
\pgfpathlineto{\pgfqpoint{1.470821in}{0.670422in}}%
\pgfpathlineto{\pgfqpoint{1.471258in}{0.670547in}}%
\pgfpathlineto{\pgfqpoint{1.473150in}{0.671830in}}%
\pgfpathlineto{\pgfqpoint{1.473732in}{0.674580in}}%
\pgfpathlineto{\pgfqpoint{1.474314in}{0.672219in}}%
\pgfpathlineto{\pgfqpoint{1.474605in}{0.671634in}}%
\pgfpathlineto{\pgfqpoint{1.475042in}{0.672821in}}%
\pgfpathlineto{\pgfqpoint{1.475915in}{0.681292in}}%
\pgfpathlineto{\pgfqpoint{1.476352in}{0.675742in}}%
\pgfpathlineto{\pgfqpoint{1.477080in}{0.668800in}}%
\pgfpathlineto{\pgfqpoint{1.477662in}{0.669564in}}%
\pgfpathlineto{\pgfqpoint{1.478535in}{0.668791in}}%
\pgfpathlineto{\pgfqpoint{1.478826in}{0.669302in}}%
\pgfpathlineto{\pgfqpoint{1.479263in}{0.670193in}}%
\pgfpathlineto{\pgfqpoint{1.480136in}{0.669745in}}%
\pgfpathlineto{\pgfqpoint{1.481592in}{0.671609in}}%
\pgfpathlineto{\pgfqpoint{1.482319in}{0.678356in}}%
\pgfpathlineto{\pgfqpoint{1.482756in}{0.674591in}}%
\pgfpathlineto{\pgfqpoint{1.483338in}{0.670015in}}%
\pgfpathlineto{\pgfqpoint{1.483775in}{0.672779in}}%
\pgfpathlineto{\pgfqpoint{1.484066in}{0.675200in}}%
\pgfpathlineto{\pgfqpoint{1.484502in}{0.670375in}}%
\pgfpathlineto{\pgfqpoint{1.484939in}{0.668021in}}%
\pgfpathlineto{\pgfqpoint{1.485376in}{0.670882in}}%
\pgfpathlineto{\pgfqpoint{1.485812in}{0.678314in}}%
\pgfpathlineto{\pgfqpoint{1.486395in}{0.670986in}}%
\pgfpathlineto{\pgfqpoint{1.486540in}{0.670407in}}%
\pgfpathlineto{\pgfqpoint{1.486686in}{0.671290in}}%
\pgfpathlineto{\pgfqpoint{1.487122in}{0.698609in}}%
\pgfpathlineto{\pgfqpoint{1.487850in}{0.798098in}}%
\pgfpathlineto{\pgfqpoint{1.488287in}{0.738671in}}%
\pgfpathlineto{\pgfqpoint{1.489014in}{0.682833in}}%
\pgfpathlineto{\pgfqpoint{1.489451in}{0.726503in}}%
\pgfpathlineto{\pgfqpoint{1.489742in}{0.760188in}}%
\pgfpathlineto{\pgfqpoint{1.490179in}{0.699599in}}%
\pgfpathlineto{\pgfqpoint{1.490761in}{0.669146in}}%
\pgfpathlineto{\pgfqpoint{1.491489in}{0.670518in}}%
\pgfpathlineto{\pgfqpoint{1.491925in}{0.669569in}}%
\pgfpathlineto{\pgfqpoint{1.492362in}{0.670314in}}%
\pgfpathlineto{\pgfqpoint{1.493090in}{0.677889in}}%
\pgfpathlineto{\pgfqpoint{1.493672in}{0.672197in}}%
\pgfpathlineto{\pgfqpoint{1.493963in}{0.671403in}}%
\pgfpathlineto{\pgfqpoint{1.494254in}{0.674219in}}%
\pgfpathlineto{\pgfqpoint{1.494836in}{0.684974in}}%
\pgfpathlineto{\pgfqpoint{1.495273in}{0.676186in}}%
\pgfpathlineto{\pgfqpoint{1.496292in}{0.669373in}}%
\pgfpathlineto{\pgfqpoint{1.496583in}{0.669455in}}%
\pgfpathlineto{\pgfqpoint{1.501677in}{0.670324in}}%
\pgfpathlineto{\pgfqpoint{1.502987in}{0.670453in}}%
\pgfpathlineto{\pgfqpoint{1.504006in}{0.671774in}}%
\pgfpathlineto{\pgfqpoint{1.504588in}{0.675356in}}%
\pgfpathlineto{\pgfqpoint{1.505170in}{0.672701in}}%
\pgfpathlineto{\pgfqpoint{1.505315in}{0.672408in}}%
\pgfpathlineto{\pgfqpoint{1.505607in}{0.673236in}}%
\pgfpathlineto{\pgfqpoint{1.507499in}{0.684814in}}%
\pgfpathlineto{\pgfqpoint{1.507790in}{0.683406in}}%
\pgfpathlineto{\pgfqpoint{1.508372in}{0.675930in}}%
\pgfpathlineto{\pgfqpoint{1.508954in}{0.681722in}}%
\pgfpathlineto{\pgfqpoint{1.509100in}{0.682840in}}%
\pgfpathlineto{\pgfqpoint{1.509391in}{0.679376in}}%
\pgfpathlineto{\pgfqpoint{1.510264in}{0.669862in}}%
\pgfpathlineto{\pgfqpoint{1.510846in}{0.670504in}}%
\pgfpathlineto{\pgfqpoint{1.512447in}{0.676939in}}%
\pgfpathlineto{\pgfqpoint{1.513029in}{0.697814in}}%
\pgfpathlineto{\pgfqpoint{1.513466in}{0.682833in}}%
\pgfpathlineto{\pgfqpoint{1.514485in}{0.672026in}}%
\pgfpathlineto{\pgfqpoint{1.514776in}{0.672208in}}%
\pgfpathlineto{\pgfqpoint{1.516086in}{0.673266in}}%
\pgfpathlineto{\pgfqpoint{1.516377in}{0.673041in}}%
\pgfpathlineto{\pgfqpoint{1.517541in}{0.672027in}}%
\pgfpathlineto{\pgfqpoint{1.517832in}{0.672978in}}%
\pgfpathlineto{\pgfqpoint{1.518560in}{0.678367in}}%
\pgfpathlineto{\pgfqpoint{1.519142in}{0.674597in}}%
\pgfpathlineto{\pgfqpoint{1.519579in}{0.672318in}}%
\pgfpathlineto{\pgfqpoint{1.520016in}{0.674444in}}%
\pgfpathlineto{\pgfqpoint{1.520598in}{0.679004in}}%
\pgfpathlineto{\pgfqpoint{1.521326in}{0.677299in}}%
\pgfpathlineto{\pgfqpoint{1.521617in}{0.677643in}}%
\pgfpathlineto{\pgfqpoint{1.521908in}{0.676548in}}%
\pgfpathlineto{\pgfqpoint{1.523218in}{0.667162in}}%
\pgfpathlineto{\pgfqpoint{1.523800in}{0.667843in}}%
\pgfpathlineto{\pgfqpoint{1.524528in}{0.668983in}}%
\pgfpathlineto{\pgfqpoint{1.524964in}{0.693654in}}%
\pgfpathlineto{\pgfqpoint{1.525546in}{0.778500in}}%
\pgfpathlineto{\pgfqpoint{1.525983in}{0.705454in}}%
\pgfpathlineto{\pgfqpoint{1.526420in}{0.674990in}}%
\pgfpathlineto{\pgfqpoint{1.527002in}{0.708926in}}%
\pgfpathlineto{\pgfqpoint{1.527147in}{0.711392in}}%
\pgfpathlineto{\pgfqpoint{1.527293in}{0.703128in}}%
\pgfpathlineto{\pgfqpoint{1.528021in}{0.667544in}}%
\pgfpathlineto{\pgfqpoint{1.528603in}{0.676625in}}%
\pgfpathlineto{\pgfqpoint{1.528748in}{0.679200in}}%
\pgfpathlineto{\pgfqpoint{1.529331in}{0.670417in}}%
\pgfpathlineto{\pgfqpoint{1.529622in}{0.668386in}}%
\pgfpathlineto{\pgfqpoint{1.530349in}{0.670788in}}%
\pgfpathlineto{\pgfqpoint{1.532678in}{0.666177in}}%
\pgfpathlineto{\pgfqpoint{1.532824in}{0.666556in}}%
\pgfpathlineto{\pgfqpoint{1.533551in}{0.677228in}}%
\pgfpathlineto{\pgfqpoint{1.534134in}{0.670246in}}%
\pgfpathlineto{\pgfqpoint{1.534716in}{0.667127in}}%
\pgfpathlineto{\pgfqpoint{1.535298in}{0.668357in}}%
\pgfpathlineto{\pgfqpoint{1.535589in}{0.668544in}}%
\pgfpathlineto{\pgfqpoint{1.535880in}{0.667787in}}%
\pgfpathlineto{\pgfqpoint{1.536462in}{0.666945in}}%
\pgfpathlineto{\pgfqpoint{1.537044in}{0.667415in}}%
\pgfpathlineto{\pgfqpoint{1.539519in}{0.671323in}}%
\pgfpathlineto{\pgfqpoint{1.541120in}{0.689541in}}%
\pgfpathlineto{\pgfqpoint{1.541556in}{0.681411in}}%
\pgfpathlineto{\pgfqpoint{1.541993in}{0.673385in}}%
\pgfpathlineto{\pgfqpoint{1.542430in}{0.684947in}}%
\pgfpathlineto{\pgfqpoint{1.542721in}{0.695351in}}%
\pgfpathlineto{\pgfqpoint{1.543157in}{0.676164in}}%
\pgfpathlineto{\pgfqpoint{1.543594in}{0.666975in}}%
\pgfpathlineto{\pgfqpoint{1.544467in}{0.667624in}}%
\pgfpathlineto{\pgfqpoint{1.545777in}{0.669401in}}%
\pgfpathlineto{\pgfqpoint{1.547233in}{0.677950in}}%
\pgfpathlineto{\pgfqpoint{1.548106in}{0.853431in}}%
\pgfpathlineto{\pgfqpoint{1.548834in}{0.738653in}}%
\pgfpathlineto{\pgfqpoint{1.549270in}{0.698371in}}%
\pgfpathlineto{\pgfqpoint{1.549707in}{0.733991in}}%
\pgfpathlineto{\pgfqpoint{1.550144in}{0.833929in}}%
\pgfpathlineto{\pgfqpoint{1.550726in}{0.738652in}}%
\pgfpathlineto{\pgfqpoint{1.551453in}{0.673142in}}%
\pgfpathlineto{\pgfqpoint{1.552036in}{0.682123in}}%
\pgfpathlineto{\pgfqpoint{1.552181in}{0.683246in}}%
\pgfpathlineto{\pgfqpoint{1.552472in}{0.677944in}}%
\pgfpathlineto{\pgfqpoint{1.553054in}{0.671732in}}%
\pgfpathlineto{\pgfqpoint{1.553637in}{0.672919in}}%
\pgfpathlineto{\pgfqpoint{1.553928in}{0.673805in}}%
\pgfpathlineto{\pgfqpoint{1.554510in}{0.672192in}}%
\pgfpathlineto{\pgfqpoint{1.555965in}{0.672056in}}%
\pgfpathlineto{\pgfqpoint{1.557712in}{0.672156in}}%
\pgfpathlineto{\pgfqpoint{1.558440in}{0.674290in}}%
\pgfpathlineto{\pgfqpoint{1.559022in}{0.672471in}}%
\pgfpathlineto{\pgfqpoint{1.560914in}{0.672016in}}%
\pgfpathlineto{\pgfqpoint{1.567027in}{0.672819in}}%
\pgfpathlineto{\pgfqpoint{1.567755in}{0.682028in}}%
\pgfpathlineto{\pgfqpoint{1.568628in}{0.677339in}}%
\pgfpathlineto{\pgfqpoint{1.569501in}{0.675486in}}%
\pgfpathlineto{\pgfqpoint{1.569792in}{0.676290in}}%
\pgfpathlineto{\pgfqpoint{1.570520in}{0.682883in}}%
\pgfpathlineto{\pgfqpoint{1.570957in}{0.676754in}}%
\pgfpathlineto{\pgfqpoint{1.571830in}{0.672008in}}%
\pgfpathlineto{\pgfqpoint{1.572267in}{0.672347in}}%
\pgfpathlineto{\pgfqpoint{1.573576in}{0.672284in}}%
\pgfpathlineto{\pgfqpoint{1.576778in}{0.673497in}}%
\pgfpathlineto{\pgfqpoint{1.577215in}{0.675715in}}%
\pgfpathlineto{\pgfqpoint{1.577797in}{0.672855in}}%
\pgfpathlineto{\pgfqpoint{1.578525in}{0.671792in}}%
\pgfpathlineto{\pgfqpoint{1.579107in}{0.672093in}}%
\pgfpathlineto{\pgfqpoint{1.579689in}{0.675068in}}%
\pgfpathlineto{\pgfqpoint{1.581727in}{0.729736in}}%
\pgfpathlineto{\pgfqpoint{1.582164in}{0.761798in}}%
\pgfpathlineto{\pgfqpoint{1.582600in}{0.706042in}}%
\pgfpathlineto{\pgfqpoint{1.583910in}{0.670483in}}%
\pgfpathlineto{\pgfqpoint{1.584492in}{0.670531in}}%
\pgfpathlineto{\pgfqpoint{1.584783in}{0.671084in}}%
\pgfpathlineto{\pgfqpoint{1.587403in}{0.677594in}}%
\pgfpathlineto{\pgfqpoint{1.587694in}{0.676040in}}%
\pgfpathlineto{\pgfqpoint{1.588713in}{0.669567in}}%
\pgfpathlineto{\pgfqpoint{1.589295in}{0.669736in}}%
\pgfpathlineto{\pgfqpoint{1.594244in}{0.673142in}}%
\pgfpathlineto{\pgfqpoint{1.594826in}{0.678668in}}%
\pgfpathlineto{\pgfqpoint{1.595263in}{0.673357in}}%
\pgfpathlineto{\pgfqpoint{1.595699in}{0.668717in}}%
\pgfpathlineto{\pgfqpoint{1.596136in}{0.672781in}}%
\pgfpathlineto{\pgfqpoint{1.596427in}{0.678572in}}%
\pgfpathlineto{\pgfqpoint{1.597009in}{0.668692in}}%
\pgfpathlineto{\pgfqpoint{1.597155in}{0.668196in}}%
\pgfpathlineto{\pgfqpoint{1.597446in}{0.668711in}}%
\pgfpathlineto{\pgfqpoint{1.598174in}{0.683332in}}%
\pgfpathlineto{\pgfqpoint{1.598610in}{0.673394in}}%
\pgfpathlineto{\pgfqpoint{1.599047in}{0.669998in}}%
\pgfpathlineto{\pgfqpoint{1.599629in}{0.674789in}}%
\pgfpathlineto{\pgfqpoint{1.599920in}{0.676011in}}%
\pgfpathlineto{\pgfqpoint{1.600211in}{0.673177in}}%
\pgfpathlineto{\pgfqpoint{1.600939in}{0.668474in}}%
\pgfpathlineto{\pgfqpoint{1.601521in}{0.668873in}}%
\pgfpathlineto{\pgfqpoint{1.602686in}{0.671132in}}%
\pgfpathlineto{\pgfqpoint{1.603268in}{0.701307in}}%
\pgfpathlineto{\pgfqpoint{1.603850in}{0.677277in}}%
\pgfpathlineto{\pgfqpoint{1.604141in}{0.673406in}}%
\pgfpathlineto{\pgfqpoint{1.604723in}{0.678565in}}%
\pgfpathlineto{\pgfqpoint{1.604869in}{0.678020in}}%
\pgfpathlineto{\pgfqpoint{1.606761in}{0.669448in}}%
\pgfpathlineto{\pgfqpoint{1.608798in}{0.669450in}}%
\pgfpathlineto{\pgfqpoint{1.611273in}{0.669949in}}%
\pgfpathlineto{\pgfqpoint{1.611418in}{0.669606in}}%
\pgfpathlineto{\pgfqpoint{1.613019in}{0.668324in}}%
\pgfpathlineto{\pgfqpoint{1.614329in}{0.668904in}}%
\pgfpathlineto{\pgfqpoint{1.617240in}{0.673502in}}%
\pgfpathlineto{\pgfqpoint{1.617822in}{0.674358in}}%
\pgfpathlineto{\pgfqpoint{1.618113in}{0.673176in}}%
\pgfpathlineto{\pgfqpoint{1.618987in}{0.668926in}}%
\pgfpathlineto{\pgfqpoint{1.619714in}{0.669154in}}%
\pgfpathlineto{\pgfqpoint{1.620005in}{0.669876in}}%
\pgfpathlineto{\pgfqpoint{1.620588in}{0.685328in}}%
\pgfpathlineto{\pgfqpoint{1.621315in}{0.714474in}}%
\pgfpathlineto{\pgfqpoint{1.621898in}{0.699214in}}%
\pgfpathlineto{\pgfqpoint{1.622480in}{0.689521in}}%
\pgfpathlineto{\pgfqpoint{1.623062in}{0.694405in}}%
\pgfpathlineto{\pgfqpoint{1.623207in}{0.695831in}}%
\pgfpathlineto{\pgfqpoint{1.623644in}{0.689767in}}%
\pgfpathlineto{\pgfqpoint{1.624808in}{0.669244in}}%
\pgfpathlineto{\pgfqpoint{1.625245in}{0.669389in}}%
\pgfpathlineto{\pgfqpoint{1.626555in}{0.671143in}}%
\pgfpathlineto{\pgfqpoint{1.627574in}{0.681796in}}%
\pgfpathlineto{\pgfqpoint{1.629903in}{0.764353in}}%
\pgfpathlineto{\pgfqpoint{1.630194in}{0.737641in}}%
\pgfpathlineto{\pgfqpoint{1.631649in}{0.670277in}}%
\pgfpathlineto{\pgfqpoint{1.633105in}{0.670982in}}%
\pgfpathlineto{\pgfqpoint{1.634997in}{0.678584in}}%
\pgfpathlineto{\pgfqpoint{1.635288in}{0.675735in}}%
\pgfpathlineto{\pgfqpoint{1.636016in}{0.670723in}}%
\pgfpathlineto{\pgfqpoint{1.636598in}{0.670786in}}%
\pgfpathlineto{\pgfqpoint{1.639654in}{0.671925in}}%
\pgfpathlineto{\pgfqpoint{1.640236in}{0.673613in}}%
\pgfpathlineto{\pgfqpoint{1.640819in}{0.672669in}}%
\pgfpathlineto{\pgfqpoint{1.642711in}{0.671346in}}%
\pgfpathlineto{\pgfqpoint{1.643002in}{0.672046in}}%
\pgfpathlineto{\pgfqpoint{1.643875in}{0.676003in}}%
\pgfpathlineto{\pgfqpoint{1.644312in}{0.673033in}}%
\pgfpathlineto{\pgfqpoint{1.644894in}{0.670181in}}%
\pgfpathlineto{\pgfqpoint{1.645476in}{0.671650in}}%
\pgfpathlineto{\pgfqpoint{1.645913in}{0.673240in}}%
\pgfpathlineto{\pgfqpoint{1.646349in}{0.671578in}}%
\pgfpathlineto{\pgfqpoint{1.647077in}{0.670227in}}%
\pgfpathlineto{\pgfqpoint{1.647514in}{0.670700in}}%
\pgfpathlineto{\pgfqpoint{1.648387in}{0.675056in}}%
\pgfpathlineto{\pgfqpoint{1.648969in}{0.671696in}}%
\pgfpathlineto{\pgfqpoint{1.649406in}{0.670469in}}%
\pgfpathlineto{\pgfqpoint{1.649988in}{0.672201in}}%
\pgfpathlineto{\pgfqpoint{1.650279in}{0.673428in}}%
\pgfpathlineto{\pgfqpoint{1.650861in}{0.670926in}}%
\pgfpathlineto{\pgfqpoint{1.651298in}{0.670221in}}%
\pgfpathlineto{\pgfqpoint{1.652026in}{0.670564in}}%
\pgfpathlineto{\pgfqpoint{1.654500in}{0.671042in}}%
\pgfpathlineto{\pgfqpoint{1.656392in}{0.671475in}}%
\pgfpathlineto{\pgfqpoint{1.657847in}{0.672838in}}%
\pgfpathlineto{\pgfqpoint{1.658138in}{0.672657in}}%
\pgfpathlineto{\pgfqpoint{1.658575in}{0.673083in}}%
\pgfpathlineto{\pgfqpoint{1.659157in}{0.681444in}}%
\pgfpathlineto{\pgfqpoint{1.659448in}{0.684877in}}%
\pgfpathlineto{\pgfqpoint{1.659739in}{0.673868in}}%
\pgfpathlineto{\pgfqpoint{1.660176in}{0.666880in}}%
\pgfpathlineto{\pgfqpoint{1.660904in}{0.667384in}}%
\pgfpathlineto{\pgfqpoint{1.661632in}{0.667524in}}%
\pgfpathlineto{\pgfqpoint{1.661777in}{0.667914in}}%
\pgfpathlineto{\pgfqpoint{1.662359in}{0.678416in}}%
\pgfpathlineto{\pgfqpoint{1.662796in}{0.685448in}}%
\pgfpathlineto{\pgfqpoint{1.663378in}{0.678528in}}%
\pgfpathlineto{\pgfqpoint{1.663960in}{0.675852in}}%
\pgfpathlineto{\pgfqpoint{1.664542in}{0.676885in}}%
\pgfpathlineto{\pgfqpoint{1.664979in}{0.678491in}}%
\pgfpathlineto{\pgfqpoint{1.665270in}{0.676380in}}%
\pgfpathlineto{\pgfqpoint{1.666143in}{0.666144in}}%
\pgfpathlineto{\pgfqpoint{1.666871in}{0.666477in}}%
\pgfpathlineto{\pgfqpoint{1.667162in}{0.669986in}}%
\pgfpathlineto{\pgfqpoint{1.667890in}{0.734363in}}%
\pgfpathlineto{\pgfqpoint{1.668618in}{0.692346in}}%
\pgfpathlineto{\pgfqpoint{1.669200in}{0.683684in}}%
\pgfpathlineto{\pgfqpoint{1.669782in}{0.688812in}}%
\pgfpathlineto{\pgfqpoint{1.670364in}{0.702503in}}%
\pgfpathlineto{\pgfqpoint{1.670801in}{0.689634in}}%
\pgfpathlineto{\pgfqpoint{1.671674in}{0.668509in}}%
\pgfpathlineto{\pgfqpoint{1.672256in}{0.670373in}}%
\pgfpathlineto{\pgfqpoint{1.672984in}{0.669719in}}%
\pgfpathlineto{\pgfqpoint{1.673421in}{0.677137in}}%
\pgfpathlineto{\pgfqpoint{1.674003in}{0.705635in}}%
\pgfpathlineto{\pgfqpoint{1.674440in}{0.679298in}}%
\pgfpathlineto{\pgfqpoint{1.675022in}{0.667884in}}%
\pgfpathlineto{\pgfqpoint{1.675604in}{0.669748in}}%
\pgfpathlineto{\pgfqpoint{1.676477in}{0.684632in}}%
\pgfpathlineto{\pgfqpoint{1.676768in}{0.687206in}}%
\pgfpathlineto{\pgfqpoint{1.677059in}{0.678716in}}%
\pgfpathlineto{\pgfqpoint{1.677496in}{0.669823in}}%
\pgfpathlineto{\pgfqpoint{1.677787in}{0.680443in}}%
\pgfpathlineto{\pgfqpoint{1.678369in}{0.809316in}}%
\pgfpathlineto{\pgfqpoint{1.678806in}{0.713887in}}%
\pgfpathlineto{\pgfqpoint{1.679243in}{0.671478in}}%
\pgfpathlineto{\pgfqpoint{1.679970in}{0.697506in}}%
\pgfpathlineto{\pgfqpoint{1.680116in}{0.698782in}}%
\pgfpathlineto{\pgfqpoint{1.680261in}{0.694944in}}%
\pgfpathlineto{\pgfqpoint{1.681280in}{0.668867in}}%
\pgfpathlineto{\pgfqpoint{1.681862in}{0.670948in}}%
\pgfpathlineto{\pgfqpoint{1.682154in}{0.673161in}}%
\pgfpathlineto{\pgfqpoint{1.682590in}{0.669515in}}%
\pgfpathlineto{\pgfqpoint{1.683027in}{0.668279in}}%
\pgfpathlineto{\pgfqpoint{1.683609in}{0.670135in}}%
\pgfpathlineto{\pgfqpoint{1.684337in}{0.668182in}}%
\pgfpathlineto{\pgfqpoint{1.684628in}{0.669636in}}%
\pgfpathlineto{\pgfqpoint{1.685064in}{0.674457in}}%
\pgfpathlineto{\pgfqpoint{1.685501in}{0.669259in}}%
\pgfpathlineto{\pgfqpoint{1.685792in}{0.667866in}}%
\pgfpathlineto{\pgfqpoint{1.686229in}{0.670442in}}%
\pgfpathlineto{\pgfqpoint{1.686665in}{0.674812in}}%
\pgfpathlineto{\pgfqpoint{1.687248in}{0.670476in}}%
\pgfpathlineto{\pgfqpoint{1.688121in}{0.668512in}}%
\pgfpathlineto{\pgfqpoint{1.688412in}{0.669411in}}%
\pgfpathlineto{\pgfqpoint{1.689140in}{0.681420in}}%
\pgfpathlineto{\pgfqpoint{1.689576in}{0.673420in}}%
\pgfpathlineto{\pgfqpoint{1.690013in}{0.668047in}}%
\pgfpathlineto{\pgfqpoint{1.690595in}{0.672903in}}%
\pgfpathlineto{\pgfqpoint{1.690741in}{0.673765in}}%
\pgfpathlineto{\pgfqpoint{1.691032in}{0.670184in}}%
\pgfpathlineto{\pgfqpoint{1.691468in}{0.667418in}}%
\pgfpathlineto{\pgfqpoint{1.692196in}{0.667957in}}%
\pgfpathlineto{\pgfqpoint{1.694816in}{0.670850in}}%
\pgfpathlineto{\pgfqpoint{1.695253in}{0.678078in}}%
\pgfpathlineto{\pgfqpoint{1.695980in}{0.715235in}}%
\pgfpathlineto{\pgfqpoint{1.696563in}{0.689592in}}%
\pgfpathlineto{\pgfqpoint{1.696708in}{0.689476in}}%
\pgfpathlineto{\pgfqpoint{1.697145in}{0.721364in}}%
\pgfpathlineto{\pgfqpoint{1.697872in}{0.800960in}}%
\pgfpathlineto{\pgfqpoint{1.698309in}{0.763132in}}%
\pgfpathlineto{\pgfqpoint{1.699473in}{0.681617in}}%
\pgfpathlineto{\pgfqpoint{1.699910in}{0.684258in}}%
\pgfpathlineto{\pgfqpoint{1.700056in}{0.684477in}}%
\pgfpathlineto{\pgfqpoint{1.700201in}{0.683375in}}%
\pgfpathlineto{\pgfqpoint{1.702093in}{0.670518in}}%
\pgfpathlineto{\pgfqpoint{1.703694in}{0.671507in}}%
\pgfpathlineto{\pgfqpoint{1.704568in}{0.683427in}}%
\pgfpathlineto{\pgfqpoint{1.705295in}{0.676117in}}%
\pgfpathlineto{\pgfqpoint{1.706896in}{0.672853in}}%
\pgfpathlineto{\pgfqpoint{1.709807in}{0.670948in}}%
\pgfpathlineto{\pgfqpoint{1.715775in}{0.674731in}}%
\pgfpathlineto{\pgfqpoint{1.716357in}{0.689760in}}%
\pgfpathlineto{\pgfqpoint{1.716648in}{0.698968in}}%
\pgfpathlineto{\pgfqpoint{1.717084in}{0.685770in}}%
\pgfpathlineto{\pgfqpoint{1.718103in}{0.670873in}}%
\pgfpathlineto{\pgfqpoint{1.718540in}{0.670926in}}%
\pgfpathlineto{\pgfqpoint{1.720432in}{0.672016in}}%
\pgfpathlineto{\pgfqpoint{1.721014in}{0.682683in}}%
\pgfpathlineto{\pgfqpoint{1.721742in}{0.675179in}}%
\pgfpathlineto{\pgfqpoint{1.721887in}{0.675100in}}%
\pgfpathlineto{\pgfqpoint{1.722033in}{0.675755in}}%
\pgfpathlineto{\pgfqpoint{1.723634in}{0.691886in}}%
\pgfpathlineto{\pgfqpoint{1.724071in}{0.704063in}}%
\pgfpathlineto{\pgfqpoint{1.724507in}{0.688218in}}%
\pgfpathlineto{\pgfqpoint{1.725089in}{0.675485in}}%
\pgfpathlineto{\pgfqpoint{1.725672in}{0.683476in}}%
\pgfpathlineto{\pgfqpoint{1.725817in}{0.683465in}}%
\pgfpathlineto{\pgfqpoint{1.727855in}{0.668062in}}%
\pgfpathlineto{\pgfqpoint{1.728146in}{0.668290in}}%
\pgfpathlineto{\pgfqpoint{1.728583in}{0.671930in}}%
\pgfpathlineto{\pgfqpoint{1.729165in}{0.685252in}}%
\pgfpathlineto{\pgfqpoint{1.729601in}{0.673017in}}%
\pgfpathlineto{\pgfqpoint{1.729747in}{0.671175in}}%
\pgfpathlineto{\pgfqpoint{1.730038in}{0.679202in}}%
\pgfpathlineto{\pgfqpoint{1.730620in}{0.825678in}}%
\pgfpathlineto{\pgfqpoint{1.731202in}{0.705975in}}%
\pgfpathlineto{\pgfqpoint{1.731639in}{0.673069in}}%
\pgfpathlineto{\pgfqpoint{1.732367in}{0.693373in}}%
\pgfpathlineto{\pgfqpoint{1.732512in}{0.695780in}}%
\pgfpathlineto{\pgfqpoint{1.732803in}{0.688710in}}%
\pgfpathlineto{\pgfqpoint{1.733531in}{0.669198in}}%
\pgfpathlineto{\pgfqpoint{1.734113in}{0.673308in}}%
\pgfpathlineto{\pgfqpoint{1.734550in}{0.679799in}}%
\pgfpathlineto{\pgfqpoint{1.734987in}{0.672743in}}%
\pgfpathlineto{\pgfqpoint{1.735423in}{0.669832in}}%
\pgfpathlineto{\pgfqpoint{1.736005in}{0.673664in}}%
\pgfpathlineto{\pgfqpoint{1.736151in}{0.674533in}}%
\pgfpathlineto{\pgfqpoint{1.736588in}{0.671139in}}%
\pgfpathlineto{\pgfqpoint{1.737024in}{0.669384in}}%
\pgfpathlineto{\pgfqpoint{1.737752in}{0.669889in}}%
\pgfpathlineto{\pgfqpoint{1.741391in}{0.670763in}}%
\pgfpathlineto{\pgfqpoint{1.742118in}{0.677973in}}%
\pgfpathlineto{\pgfqpoint{1.742555in}{0.672821in}}%
\pgfpathlineto{\pgfqpoint{1.742701in}{0.672179in}}%
\pgfpathlineto{\pgfqpoint{1.742846in}{0.672956in}}%
\pgfpathlineto{\pgfqpoint{1.743283in}{0.695843in}}%
\pgfpathlineto{\pgfqpoint{1.743719in}{0.727918in}}%
\pgfpathlineto{\pgfqpoint{1.744156in}{0.692840in}}%
\pgfpathlineto{\pgfqpoint{1.744738in}{0.670198in}}%
\pgfpathlineto{\pgfqpoint{1.745320in}{0.676500in}}%
\pgfpathlineto{\pgfqpoint{1.745611in}{0.679049in}}%
\pgfpathlineto{\pgfqpoint{1.746048in}{0.671434in}}%
\pgfpathlineto{\pgfqpoint{1.746339in}{0.669271in}}%
\pgfpathlineto{\pgfqpoint{1.746921in}{0.671075in}}%
\pgfpathlineto{\pgfqpoint{1.748231in}{0.682584in}}%
\pgfpathlineto{\pgfqpoint{1.748522in}{0.680368in}}%
\pgfpathlineto{\pgfqpoint{1.748959in}{0.675401in}}%
\pgfpathlineto{\pgfqpoint{1.749250in}{0.681995in}}%
\pgfpathlineto{\pgfqpoint{1.749978in}{0.742950in}}%
\pgfpathlineto{\pgfqpoint{1.750414in}{0.702183in}}%
\pgfpathlineto{\pgfqpoint{1.751433in}{0.670684in}}%
\pgfpathlineto{\pgfqpoint{1.751870in}{0.671117in}}%
\pgfpathlineto{\pgfqpoint{1.752161in}{0.671093in}}%
\pgfpathlineto{\pgfqpoint{1.752452in}{0.670315in}}%
\pgfpathlineto{\pgfqpoint{1.753034in}{0.669288in}}%
\pgfpathlineto{\pgfqpoint{1.753325in}{0.670000in}}%
\pgfpathlineto{\pgfqpoint{1.754053in}{0.681420in}}%
\pgfpathlineto{\pgfqpoint{1.755509in}{0.680284in}}%
\pgfpathlineto{\pgfqpoint{1.755654in}{0.680593in}}%
\pgfpathlineto{\pgfqpoint{1.755945in}{0.679538in}}%
\pgfpathlineto{\pgfqpoint{1.757110in}{0.669950in}}%
\pgfpathlineto{\pgfqpoint{1.757546in}{0.671704in}}%
\pgfpathlineto{\pgfqpoint{1.758565in}{0.689314in}}%
\pgfpathlineto{\pgfqpoint{1.759584in}{0.683474in}}%
\pgfpathlineto{\pgfqpoint{1.759875in}{0.684402in}}%
\pgfpathlineto{\pgfqpoint{1.760166in}{0.682108in}}%
\pgfpathlineto{\pgfqpoint{1.761621in}{0.669026in}}%
\pgfpathlineto{\pgfqpoint{1.761913in}{0.669208in}}%
\pgfpathlineto{\pgfqpoint{1.762786in}{0.672199in}}%
\pgfpathlineto{\pgfqpoint{1.763514in}{0.686576in}}%
\pgfpathlineto{\pgfqpoint{1.764241in}{0.744184in}}%
\pgfpathlineto{\pgfqpoint{1.764823in}{0.711825in}}%
\pgfpathlineto{\pgfqpoint{1.766424in}{0.680362in}}%
\pgfpathlineto{\pgfqpoint{1.767734in}{0.667675in}}%
\pgfpathlineto{\pgfqpoint{1.769335in}{0.668092in}}%
\pgfpathlineto{\pgfqpoint{1.769772in}{0.680209in}}%
\pgfpathlineto{\pgfqpoint{1.770500in}{0.744875in}}%
\pgfpathlineto{\pgfqpoint{1.770936in}{0.703713in}}%
\pgfpathlineto{\pgfqpoint{1.772246in}{0.678141in}}%
\pgfpathlineto{\pgfqpoint{1.773993in}{0.667322in}}%
\pgfpathlineto{\pgfqpoint{1.774138in}{0.667350in}}%
\pgfpathlineto{\pgfqpoint{1.775157in}{0.668333in}}%
\pgfpathlineto{\pgfqpoint{1.775739in}{0.675898in}}%
\pgfpathlineto{\pgfqpoint{1.776322in}{0.685434in}}%
\pgfpathlineto{\pgfqpoint{1.776904in}{0.681204in}}%
\pgfpathlineto{\pgfqpoint{1.777340in}{0.679355in}}%
\pgfpathlineto{\pgfqpoint{1.778068in}{0.680518in}}%
\pgfpathlineto{\pgfqpoint{1.778505in}{0.679514in}}%
\pgfpathlineto{\pgfqpoint{1.778650in}{0.679978in}}%
\pgfpathlineto{\pgfqpoint{1.779232in}{0.696704in}}%
\pgfpathlineto{\pgfqpoint{1.779669in}{0.709551in}}%
\pgfpathlineto{\pgfqpoint{1.780106in}{0.690069in}}%
\pgfpathlineto{\pgfqpoint{1.781125in}{0.667576in}}%
\pgfpathlineto{\pgfqpoint{1.781561in}{0.667984in}}%
\pgfpathlineto{\pgfqpoint{1.783162in}{0.673188in}}%
\pgfpathlineto{\pgfqpoint{1.783599in}{0.682223in}}%
\pgfpathlineto{\pgfqpoint{1.784035in}{0.672933in}}%
\pgfpathlineto{\pgfqpoint{1.784618in}{0.665225in}}%
\pgfpathlineto{\pgfqpoint{1.785345in}{0.665897in}}%
\pgfpathlineto{\pgfqpoint{1.791313in}{0.670966in}}%
\pgfpathlineto{\pgfqpoint{1.791749in}{0.669579in}}%
\pgfpathlineto{\pgfqpoint{1.792186in}{0.669272in}}%
\pgfpathlineto{\pgfqpoint{1.792477in}{0.669842in}}%
\pgfpathlineto{\pgfqpoint{1.792914in}{0.671224in}}%
\pgfpathlineto{\pgfqpoint{1.793496in}{0.669933in}}%
\pgfpathlineto{\pgfqpoint{1.794078in}{0.669717in}}%
\pgfpathlineto{\pgfqpoint{1.794515in}{0.670255in}}%
\pgfpathlineto{\pgfqpoint{1.795388in}{0.673000in}}%
\pgfpathlineto{\pgfqpoint{1.795825in}{0.671481in}}%
\pgfpathlineto{\pgfqpoint{1.796407in}{0.670470in}}%
\pgfpathlineto{\pgfqpoint{1.796844in}{0.671267in}}%
\pgfpathlineto{\pgfqpoint{1.797280in}{0.671788in}}%
\pgfpathlineto{\pgfqpoint{1.797571in}{0.670907in}}%
\pgfpathlineto{\pgfqpoint{1.798299in}{0.669997in}}%
\pgfpathlineto{\pgfqpoint{1.798590in}{0.670665in}}%
\pgfpathlineto{\pgfqpoint{1.799172in}{0.672835in}}%
\pgfpathlineto{\pgfqpoint{1.799609in}{0.671329in}}%
\pgfpathlineto{\pgfqpoint{1.800046in}{0.670398in}}%
\pgfpathlineto{\pgfqpoint{1.800628in}{0.671730in}}%
\pgfpathlineto{\pgfqpoint{1.801064in}{0.677747in}}%
\pgfpathlineto{\pgfqpoint{1.801938in}{0.745972in}}%
\pgfpathlineto{\pgfqpoint{1.802374in}{0.703346in}}%
\pgfpathlineto{\pgfqpoint{1.802811in}{0.680646in}}%
\pgfpathlineto{\pgfqpoint{1.803539in}{0.692560in}}%
\pgfpathlineto{\pgfqpoint{1.804703in}{0.669978in}}%
\pgfpathlineto{\pgfqpoint{1.805285in}{0.670090in}}%
\pgfpathlineto{\pgfqpoint{1.808196in}{0.671704in}}%
\pgfpathlineto{\pgfqpoint{1.808924in}{0.693269in}}%
\pgfpathlineto{\pgfqpoint{1.809506in}{0.674610in}}%
\pgfpathlineto{\pgfqpoint{1.809797in}{0.672537in}}%
\pgfpathlineto{\pgfqpoint{1.810088in}{0.678143in}}%
\pgfpathlineto{\pgfqpoint{1.810670in}{0.698016in}}%
\pgfpathlineto{\pgfqpoint{1.811253in}{0.689335in}}%
\pgfpathlineto{\pgfqpoint{1.812126in}{0.673852in}}%
\pgfpathlineto{\pgfqpoint{1.812708in}{0.678011in}}%
\pgfpathlineto{\pgfqpoint{1.812854in}{0.678776in}}%
\pgfpathlineto{\pgfqpoint{1.813145in}{0.675392in}}%
\pgfpathlineto{\pgfqpoint{1.813727in}{0.670949in}}%
\pgfpathlineto{\pgfqpoint{1.814455in}{0.671774in}}%
\pgfpathlineto{\pgfqpoint{1.816056in}{0.671726in}}%
\pgfpathlineto{\pgfqpoint{1.817074in}{0.675367in}}%
\pgfpathlineto{\pgfqpoint{1.817657in}{0.673131in}}%
\pgfpathlineto{\pgfqpoint{1.819112in}{0.671044in}}%
\pgfpathlineto{\pgfqpoint{1.822168in}{0.670790in}}%
\pgfpathlineto{\pgfqpoint{1.822314in}{0.671370in}}%
\pgfpathlineto{\pgfqpoint{1.822460in}{0.671686in}}%
\pgfpathlineto{\pgfqpoint{1.822896in}{0.670548in}}%
\pgfpathlineto{\pgfqpoint{1.824061in}{0.670526in}}%
\pgfpathlineto{\pgfqpoint{1.826535in}{0.672507in}}%
\pgfpathlineto{\pgfqpoint{1.826826in}{0.671397in}}%
\pgfpathlineto{\pgfqpoint{1.827408in}{0.670413in}}%
\pgfpathlineto{\pgfqpoint{1.827990in}{0.670872in}}%
\pgfpathlineto{\pgfqpoint{1.828572in}{0.670681in}}%
\pgfpathlineto{\pgfqpoint{1.828718in}{0.671101in}}%
\pgfpathlineto{\pgfqpoint{1.829446in}{0.686186in}}%
\pgfpathlineto{\pgfqpoint{1.829737in}{0.690010in}}%
\pgfpathlineto{\pgfqpoint{1.830173in}{0.682914in}}%
\pgfpathlineto{\pgfqpoint{1.830610in}{0.675116in}}%
\pgfpathlineto{\pgfqpoint{1.831047in}{0.688922in}}%
\pgfpathlineto{\pgfqpoint{1.831629in}{0.776928in}}%
\pgfpathlineto{\pgfqpoint{1.832211in}{0.715012in}}%
\pgfpathlineto{\pgfqpoint{1.833667in}{0.677075in}}%
\pgfpathlineto{\pgfqpoint{1.834103in}{0.676799in}}%
\pgfpathlineto{\pgfqpoint{1.834249in}{0.677158in}}%
\pgfpathlineto{\pgfqpoint{1.835268in}{0.688413in}}%
\pgfpathlineto{\pgfqpoint{1.835704in}{0.680789in}}%
\pgfpathlineto{\pgfqpoint{1.836723in}{0.669534in}}%
\pgfpathlineto{\pgfqpoint{1.837160in}{0.670623in}}%
\pgfpathlineto{\pgfqpoint{1.837451in}{0.671102in}}%
\pgfpathlineto{\pgfqpoint{1.837887in}{0.669355in}}%
\pgfpathlineto{\pgfqpoint{1.838324in}{0.669734in}}%
\pgfpathlineto{\pgfqpoint{1.839052in}{0.678452in}}%
\pgfpathlineto{\pgfqpoint{1.839634in}{0.672306in}}%
\pgfpathlineto{\pgfqpoint{1.839925in}{0.671618in}}%
\pgfpathlineto{\pgfqpoint{1.840507in}{0.673211in}}%
\pgfpathlineto{\pgfqpoint{1.841817in}{0.669981in}}%
\pgfpathlineto{\pgfqpoint{1.842399in}{0.671404in}}%
\pgfpathlineto{\pgfqpoint{1.843127in}{0.679067in}}%
\pgfpathlineto{\pgfqpoint{1.843709in}{0.673114in}}%
\pgfpathlineto{\pgfqpoint{1.844000in}{0.671646in}}%
\pgfpathlineto{\pgfqpoint{1.844291in}{0.673501in}}%
\pgfpathlineto{\pgfqpoint{1.844874in}{0.685965in}}%
\pgfpathlineto{\pgfqpoint{1.845310in}{0.678379in}}%
\pgfpathlineto{\pgfqpoint{1.845892in}{0.669524in}}%
\pgfpathlineto{\pgfqpoint{1.846475in}{0.671545in}}%
\pgfpathlineto{\pgfqpoint{1.847057in}{0.678613in}}%
\pgfpathlineto{\pgfqpoint{1.847493in}{0.673490in}}%
\pgfpathlineto{\pgfqpoint{1.847930in}{0.670140in}}%
\pgfpathlineto{\pgfqpoint{1.848221in}{0.675353in}}%
\pgfpathlineto{\pgfqpoint{1.848803in}{0.714232in}}%
\pgfpathlineto{\pgfqpoint{1.849240in}{0.685538in}}%
\pgfpathlineto{\pgfqpoint{1.849822in}{0.670903in}}%
\pgfpathlineto{\pgfqpoint{1.850550in}{0.670973in}}%
\pgfpathlineto{\pgfqpoint{1.853170in}{0.669264in}}%
\pgfpathlineto{\pgfqpoint{1.853315in}{0.669459in}}%
\pgfpathlineto{\pgfqpoint{1.854043in}{0.670239in}}%
\pgfpathlineto{\pgfqpoint{1.854771in}{0.669790in}}%
\pgfpathlineto{\pgfqpoint{1.856372in}{0.670604in}}%
\pgfpathlineto{\pgfqpoint{1.857245in}{0.671370in}}%
\pgfpathlineto{\pgfqpoint{1.857682in}{0.691796in}}%
\pgfpathlineto{\pgfqpoint{1.858118in}{0.730718in}}%
\pgfpathlineto{\pgfqpoint{1.858555in}{0.690887in}}%
\pgfpathlineto{\pgfqpoint{1.859865in}{0.670218in}}%
\pgfpathlineto{\pgfqpoint{1.860156in}{0.669883in}}%
\pgfpathlineto{\pgfqpoint{1.860738in}{0.670873in}}%
\pgfpathlineto{\pgfqpoint{1.861757in}{0.672647in}}%
\pgfpathlineto{\pgfqpoint{1.862485in}{0.689608in}}%
\pgfpathlineto{\pgfqpoint{1.862921in}{0.673990in}}%
\pgfpathlineto{\pgfqpoint{1.863358in}{0.668266in}}%
\pgfpathlineto{\pgfqpoint{1.863795in}{0.675696in}}%
\pgfpathlineto{\pgfqpoint{1.864086in}{0.686683in}}%
\pgfpathlineto{\pgfqpoint{1.864668in}{0.670794in}}%
\pgfpathlineto{\pgfqpoint{1.864959in}{0.669102in}}%
\pgfpathlineto{\pgfqpoint{1.865250in}{0.672641in}}%
\pgfpathlineto{\pgfqpoint{1.865687in}{0.683948in}}%
\pgfpathlineto{\pgfqpoint{1.866269in}{0.674657in}}%
\pgfpathlineto{\pgfqpoint{1.867579in}{0.669431in}}%
\pgfpathlineto{\pgfqpoint{1.867724in}{0.669474in}}%
\pgfpathlineto{\pgfqpoint{1.868889in}{0.671108in}}%
\pgfpathlineto{\pgfqpoint{1.870344in}{0.708463in}}%
\pgfpathlineto{\pgfqpoint{1.870781in}{0.771582in}}%
\pgfpathlineto{\pgfqpoint{1.871217in}{0.692414in}}%
\pgfpathlineto{\pgfqpoint{1.871800in}{0.668803in}}%
\pgfpathlineto{\pgfqpoint{1.872527in}{0.669040in}}%
\pgfpathlineto{\pgfqpoint{1.874710in}{0.669949in}}%
\pgfpathlineto{\pgfqpoint{1.876748in}{0.670529in}}%
\pgfpathlineto{\pgfqpoint{1.883734in}{0.672475in}}%
\pgfpathlineto{\pgfqpoint{1.884316in}{0.677863in}}%
\pgfpathlineto{\pgfqpoint{1.884753in}{0.672321in}}%
\pgfpathlineto{\pgfqpoint{1.885190in}{0.670813in}}%
\pgfpathlineto{\pgfqpoint{1.885917in}{0.671189in}}%
\pgfpathlineto{\pgfqpoint{1.887955in}{0.671413in}}%
\pgfpathlineto{\pgfqpoint{1.888392in}{0.672782in}}%
\pgfpathlineto{\pgfqpoint{1.888974in}{0.671420in}}%
\pgfpathlineto{\pgfqpoint{1.890429in}{0.671542in}}%
\pgfpathlineto{\pgfqpoint{1.891594in}{0.673552in}}%
\pgfpathlineto{\pgfqpoint{1.892467in}{0.692238in}}%
\pgfpathlineto{\pgfqpoint{1.893049in}{0.679603in}}%
\pgfpathlineto{\pgfqpoint{1.893631in}{0.674576in}}%
\pgfpathlineto{\pgfqpoint{1.893922in}{0.677854in}}%
\pgfpathlineto{\pgfqpoint{1.894505in}{0.696733in}}%
\pgfpathlineto{\pgfqpoint{1.894941in}{0.680843in}}%
\pgfpathlineto{\pgfqpoint{1.895523in}{0.671507in}}%
\pgfpathlineto{\pgfqpoint{1.896106in}{0.676953in}}%
\pgfpathlineto{\pgfqpoint{1.896251in}{0.677323in}}%
\pgfpathlineto{\pgfqpoint{1.896397in}{0.676059in}}%
\pgfpathlineto{\pgfqpoint{1.896979in}{0.671621in}}%
\pgfpathlineto{\pgfqpoint{1.897561in}{0.674702in}}%
\pgfpathlineto{\pgfqpoint{1.898725in}{0.687446in}}%
\pgfpathlineto{\pgfqpoint{1.899162in}{0.683856in}}%
\pgfpathlineto{\pgfqpoint{1.899599in}{0.679173in}}%
\pgfpathlineto{\pgfqpoint{1.900035in}{0.683674in}}%
\pgfpathlineto{\pgfqpoint{1.900326in}{0.687728in}}%
\pgfpathlineto{\pgfqpoint{1.900618in}{0.679158in}}%
\pgfpathlineto{\pgfqpoint{1.900909in}{0.672937in}}%
\pgfpathlineto{\pgfqpoint{1.901200in}{0.678632in}}%
\pgfpathlineto{\pgfqpoint{1.902801in}{0.874334in}}%
\pgfpathlineto{\pgfqpoint{1.903092in}{1.126747in}}%
\pgfpathlineto{\pgfqpoint{1.903820in}{0.804609in}}%
\pgfpathlineto{\pgfqpoint{1.904256in}{0.736152in}}%
\pgfpathlineto{\pgfqpoint{1.904838in}{0.785139in}}%
\pgfpathlineto{\pgfqpoint{1.905130in}{0.804839in}}%
\pgfpathlineto{\pgfqpoint{1.905566in}{0.742883in}}%
\pgfpathlineto{\pgfqpoint{1.905712in}{0.728515in}}%
\pgfpathlineto{\pgfqpoint{1.906003in}{0.775163in}}%
\pgfpathlineto{\pgfqpoint{1.906439in}{1.062927in}}%
\pgfpathlineto{\pgfqpoint{1.906876in}{0.735364in}}%
\pgfpathlineto{\pgfqpoint{1.907167in}{0.673186in}}%
\pgfpathlineto{\pgfqpoint{1.907458in}{0.776485in}}%
\pgfpathlineto{\pgfqpoint{1.907749in}{1.443243in}}%
\pgfpathlineto{\pgfqpoint{1.908477in}{0.732429in}}%
\pgfpathlineto{\pgfqpoint{1.908768in}{0.688823in}}%
\pgfpathlineto{\pgfqpoint{1.909059in}{0.787928in}}%
\pgfpathlineto{\pgfqpoint{1.909496in}{1.142773in}}%
\pgfpathlineto{\pgfqpoint{1.910078in}{0.775319in}}%
\pgfpathlineto{\pgfqpoint{1.910660in}{0.680117in}}%
\pgfpathlineto{\pgfqpoint{1.911242in}{0.704022in}}%
\pgfpathlineto{\pgfqpoint{1.911388in}{0.707269in}}%
\pgfpathlineto{\pgfqpoint{1.911679in}{0.697388in}}%
\pgfpathlineto{\pgfqpoint{1.912698in}{0.673150in}}%
\pgfpathlineto{\pgfqpoint{1.913135in}{0.674161in}}%
\pgfpathlineto{\pgfqpoint{1.914153in}{0.685297in}}%
\pgfpathlineto{\pgfqpoint{1.914590in}{0.676986in}}%
\pgfpathlineto{\pgfqpoint{1.914736in}{0.675546in}}%
\pgfpathlineto{\pgfqpoint{1.915027in}{0.681651in}}%
\pgfpathlineto{\pgfqpoint{1.915754in}{0.811891in}}%
\pgfpathlineto{\pgfqpoint{1.916191in}{0.725288in}}%
\pgfpathlineto{\pgfqpoint{1.916773in}{0.677014in}}%
\pgfpathlineto{\pgfqpoint{1.917355in}{0.692431in}}%
\pgfpathlineto{\pgfqpoint{1.917646in}{0.687082in}}%
\pgfpathlineto{\pgfqpoint{1.918956in}{0.672538in}}%
\pgfpathlineto{\pgfqpoint{1.919102in}{0.672545in}}%
\pgfpathlineto{\pgfqpoint{1.919684in}{0.673562in}}%
\pgfpathlineto{\pgfqpoint{1.920412in}{0.680198in}}%
\pgfpathlineto{\pgfqpoint{1.920848in}{0.674922in}}%
\pgfpathlineto{\pgfqpoint{1.920994in}{0.674039in}}%
\pgfpathlineto{\pgfqpoint{1.921140in}{0.674816in}}%
\pgfpathlineto{\pgfqpoint{1.921576in}{0.714000in}}%
\pgfpathlineto{\pgfqpoint{1.922013in}{0.775909in}}%
\pgfpathlineto{\pgfqpoint{1.922595in}{0.698683in}}%
\pgfpathlineto{\pgfqpoint{1.923032in}{0.676430in}}%
\pgfpathlineto{\pgfqpoint{1.923468in}{0.700684in}}%
\pgfpathlineto{\pgfqpoint{1.923905in}{0.738971in}}%
\pgfpathlineto{\pgfqpoint{1.924342in}{0.684792in}}%
\pgfpathlineto{\pgfqpoint{1.924778in}{0.672021in}}%
\pgfpathlineto{\pgfqpoint{1.925215in}{0.684066in}}%
\pgfpathlineto{\pgfqpoint{1.925651in}{0.706420in}}%
\pgfpathlineto{\pgfqpoint{1.926088in}{0.681020in}}%
\pgfpathlineto{\pgfqpoint{1.926525in}{0.672954in}}%
\pgfpathlineto{\pgfqpoint{1.926961in}{0.683097in}}%
\pgfpathlineto{\pgfqpoint{1.927252in}{0.691351in}}%
\pgfpathlineto{\pgfqpoint{1.927835in}{0.679254in}}%
\pgfpathlineto{\pgfqpoint{1.928708in}{0.671327in}}%
\pgfpathlineto{\pgfqpoint{1.929145in}{0.673879in}}%
\pgfpathlineto{\pgfqpoint{1.929727in}{0.684257in}}%
\pgfpathlineto{\pgfqpoint{1.930163in}{0.674902in}}%
\pgfpathlineto{\pgfqpoint{1.930600in}{0.670558in}}%
\pgfpathlineto{\pgfqpoint{1.931037in}{0.677761in}}%
\pgfpathlineto{\pgfqpoint{1.931473in}{0.692422in}}%
\pgfpathlineto{\pgfqpoint{1.932055in}{0.680808in}}%
\pgfpathlineto{\pgfqpoint{1.932638in}{0.671522in}}%
\pgfpathlineto{\pgfqpoint{1.932929in}{0.675788in}}%
\pgfpathlineto{\pgfqpoint{1.933511in}{0.791417in}}%
\pgfpathlineto{\pgfqpoint{1.933656in}{0.812642in}}%
\pgfpathlineto{\pgfqpoint{1.934093in}{0.729718in}}%
\pgfpathlineto{\pgfqpoint{1.934675in}{0.674577in}}%
\pgfpathlineto{\pgfqpoint{1.935112in}{0.716995in}}%
\pgfpathlineto{\pgfqpoint{1.935549in}{0.770697in}}%
\pgfpathlineto{\pgfqpoint{1.936131in}{0.702283in}}%
\pgfpathlineto{\pgfqpoint{1.936858in}{0.674734in}}%
\pgfpathlineto{\pgfqpoint{1.937295in}{0.687497in}}%
\pgfpathlineto{\pgfqpoint{1.938023in}{0.841020in}}%
\pgfpathlineto{\pgfqpoint{1.938459in}{0.720474in}}%
\pgfpathlineto{\pgfqpoint{1.938896in}{0.674168in}}%
\pgfpathlineto{\pgfqpoint{1.939187in}{0.706256in}}%
\pgfpathlineto{\pgfqpoint{1.939624in}{0.911453in}}%
\pgfpathlineto{\pgfqpoint{1.940206in}{0.727171in}}%
\pgfpathlineto{\pgfqpoint{1.940788in}{0.676913in}}%
\pgfpathlineto{\pgfqpoint{1.941225in}{0.705154in}}%
\pgfpathlineto{\pgfqpoint{1.941807in}{0.778090in}}%
\pgfpathlineto{\pgfqpoint{1.942389in}{0.717573in}}%
\pgfpathlineto{\pgfqpoint{1.942971in}{0.677566in}}%
\pgfpathlineto{\pgfqpoint{1.943408in}{0.701670in}}%
\pgfpathlineto{\pgfqpoint{1.943845in}{0.779841in}}%
\pgfpathlineto{\pgfqpoint{1.944427in}{0.698529in}}%
\pgfpathlineto{\pgfqpoint{1.944863in}{0.672754in}}%
\pgfpathlineto{\pgfqpoint{1.945300in}{0.706531in}}%
\pgfpathlineto{\pgfqpoint{1.945737in}{0.820184in}}%
\pgfpathlineto{\pgfqpoint{1.946319in}{0.707105in}}%
\pgfpathlineto{\pgfqpoint{1.946756in}{0.677844in}}%
\pgfpathlineto{\pgfqpoint{1.947192in}{0.703134in}}%
\pgfpathlineto{\pgfqpoint{1.947920in}{0.761705in}}%
\pgfpathlineto{\pgfqpoint{1.948502in}{0.728894in}}%
\pgfpathlineto{\pgfqpoint{1.949230in}{0.681726in}}%
\pgfpathlineto{\pgfqpoint{1.949521in}{0.699324in}}%
\pgfpathlineto{\pgfqpoint{1.950103in}{0.847192in}}%
\pgfpathlineto{\pgfqpoint{1.950540in}{0.729228in}}%
\pgfpathlineto{\pgfqpoint{1.951122in}{0.673438in}}%
\pgfpathlineto{\pgfqpoint{1.951559in}{0.705920in}}%
\pgfpathlineto{\pgfqpoint{1.951850in}{0.721847in}}%
\pgfpathlineto{\pgfqpoint{1.952286in}{0.683846in}}%
\pgfpathlineto{\pgfqpoint{1.952723in}{0.672468in}}%
\pgfpathlineto{\pgfqpoint{1.953305in}{0.679366in}}%
\pgfpathlineto{\pgfqpoint{1.953887in}{0.692832in}}%
\pgfpathlineto{\pgfqpoint{1.954469in}{0.684313in}}%
\pgfpathlineto{\pgfqpoint{1.955343in}{0.675176in}}%
\pgfpathlineto{\pgfqpoint{1.955779in}{0.680454in}}%
\pgfpathlineto{\pgfqpoint{1.956507in}{0.701470in}}%
\pgfpathlineto{\pgfqpoint{1.956944in}{0.688306in}}%
\pgfpathlineto{\pgfqpoint{1.957963in}{0.671908in}}%
\pgfpathlineto{\pgfqpoint{1.958399in}{0.674203in}}%
\pgfpathlineto{\pgfqpoint{1.958690in}{0.675779in}}%
\pgfpathlineto{\pgfqpoint{1.959127in}{0.673019in}}%
\pgfpathlineto{\pgfqpoint{1.959855in}{0.669891in}}%
\pgfpathlineto{\pgfqpoint{1.960146in}{0.670729in}}%
\pgfpathlineto{\pgfqpoint{1.960582in}{0.709240in}}%
\pgfpathlineto{\pgfqpoint{1.960874in}{0.745440in}}%
\pgfpathlineto{\pgfqpoint{1.961456in}{0.687810in}}%
\pgfpathlineto{\pgfqpoint{1.961601in}{0.680803in}}%
\pgfpathlineto{\pgfqpoint{1.961892in}{0.692009in}}%
\pgfpathlineto{\pgfqpoint{1.962620in}{0.961813in}}%
\pgfpathlineto{\pgfqpoint{1.963202in}{0.738806in}}%
\pgfpathlineto{\pgfqpoint{1.963784in}{0.678486in}}%
\pgfpathlineto{\pgfqpoint{1.964221in}{0.712267in}}%
\pgfpathlineto{\pgfqpoint{1.964658in}{0.752427in}}%
\pgfpathlineto{\pgfqpoint{1.965094in}{0.693804in}}%
\pgfpathlineto{\pgfqpoint{1.965531in}{0.671817in}}%
\pgfpathlineto{\pgfqpoint{1.965968in}{0.709025in}}%
\pgfpathlineto{\pgfqpoint{1.966404in}{0.789231in}}%
\pgfpathlineto{\pgfqpoint{1.966986in}{0.683390in}}%
\pgfpathlineto{\pgfqpoint{1.967278in}{0.676660in}}%
\pgfpathlineto{\pgfqpoint{1.967569in}{0.696494in}}%
\pgfpathlineto{\pgfqpoint{1.968005in}{0.740686in}}%
\pgfpathlineto{\pgfqpoint{1.968442in}{0.694674in}}%
\pgfpathlineto{\pgfqpoint{1.969315in}{0.671773in}}%
\pgfpathlineto{\pgfqpoint{1.969752in}{0.672141in}}%
\pgfpathlineto{\pgfqpoint{1.970480in}{0.672059in}}%
\pgfpathlineto{\pgfqpoint{1.970771in}{0.671560in}}%
\pgfpathlineto{\pgfqpoint{1.971498in}{0.671214in}}%
\pgfpathlineto{\pgfqpoint{1.971644in}{0.671546in}}%
\pgfpathlineto{\pgfqpoint{1.972226in}{0.677321in}}%
\pgfpathlineto{\pgfqpoint{1.972663in}{0.672456in}}%
\pgfpathlineto{\pgfqpoint{1.972954in}{0.671552in}}%
\pgfpathlineto{\pgfqpoint{1.973245in}{0.674381in}}%
\pgfpathlineto{\pgfqpoint{1.973682in}{0.681345in}}%
\pgfpathlineto{\pgfqpoint{1.974118in}{0.672254in}}%
\pgfpathlineto{\pgfqpoint{1.974555in}{0.670632in}}%
\pgfpathlineto{\pgfqpoint{1.975283in}{0.670961in}}%
\pgfpathlineto{\pgfqpoint{1.976738in}{0.670828in}}%
\pgfpathlineto{\pgfqpoint{1.977466in}{0.671834in}}%
\pgfpathlineto{\pgfqpoint{1.977902in}{0.673001in}}%
\pgfpathlineto{\pgfqpoint{1.978339in}{0.671526in}}%
\pgfpathlineto{\pgfqpoint{1.978485in}{0.671334in}}%
\pgfpathlineto{\pgfqpoint{1.978776in}{0.672179in}}%
\pgfpathlineto{\pgfqpoint{1.979212in}{0.675602in}}%
\pgfpathlineto{\pgfqpoint{1.979794in}{0.672596in}}%
\pgfpathlineto{\pgfqpoint{1.980668in}{0.670760in}}%
\pgfpathlineto{\pgfqpoint{1.981104in}{0.671159in}}%
\pgfpathlineto{\pgfqpoint{1.981687in}{0.672992in}}%
\pgfpathlineto{\pgfqpoint{1.982123in}{0.671547in}}%
\pgfpathlineto{\pgfqpoint{1.982560in}{0.670768in}}%
\pgfpathlineto{\pgfqpoint{1.982851in}{0.671475in}}%
\pgfpathlineto{\pgfqpoint{1.983433in}{0.677457in}}%
\pgfpathlineto{\pgfqpoint{1.984015in}{0.672735in}}%
\pgfpathlineto{\pgfqpoint{1.984743in}{0.671048in}}%
\pgfpathlineto{\pgfqpoint{1.985034in}{0.671616in}}%
\pgfpathlineto{\pgfqpoint{1.985471in}{0.679161in}}%
\pgfpathlineto{\pgfqpoint{1.985907in}{0.692650in}}%
\pgfpathlineto{\pgfqpoint{1.986344in}{0.674596in}}%
\pgfpathlineto{\pgfqpoint{1.986635in}{0.670686in}}%
\pgfpathlineto{\pgfqpoint{1.986926in}{0.676749in}}%
\pgfpathlineto{\pgfqpoint{1.987363in}{0.716211in}}%
\pgfpathlineto{\pgfqpoint{1.987945in}{0.675390in}}%
\pgfpathlineto{\pgfqpoint{1.988236in}{0.670612in}}%
\pgfpathlineto{\pgfqpoint{1.988673in}{0.680748in}}%
\pgfpathlineto{\pgfqpoint{1.989255in}{0.704115in}}%
\pgfpathlineto{\pgfqpoint{1.989692in}{0.688660in}}%
\pgfpathlineto{\pgfqpoint{1.990419in}{0.672138in}}%
\pgfpathlineto{\pgfqpoint{1.990710in}{0.681741in}}%
\pgfpathlineto{\pgfqpoint{1.991147in}{0.729881in}}%
\pgfpathlineto{\pgfqpoint{1.991729in}{0.678052in}}%
\pgfpathlineto{\pgfqpoint{1.992020in}{0.669797in}}%
\pgfpathlineto{\pgfqpoint{1.992602in}{0.679477in}}%
\pgfpathlineto{\pgfqpoint{1.992748in}{0.682439in}}%
\pgfpathlineto{\pgfqpoint{1.993185in}{0.673712in}}%
\pgfpathlineto{\pgfqpoint{1.993621in}{0.669661in}}%
\pgfpathlineto{\pgfqpoint{1.994349in}{0.671358in}}%
\pgfpathlineto{\pgfqpoint{1.995222in}{0.674007in}}%
\pgfpathlineto{\pgfqpoint{1.995513in}{0.672705in}}%
\pgfpathlineto{\pgfqpoint{1.996096in}{0.670480in}}%
\pgfpathlineto{\pgfqpoint{1.996678in}{0.672351in}}%
\pgfpathlineto{\pgfqpoint{1.997114in}{0.671011in}}%
\pgfpathlineto{\pgfqpoint{1.997697in}{0.669709in}}%
\pgfpathlineto{\pgfqpoint{1.998424in}{0.669945in}}%
\pgfpathlineto{\pgfqpoint{2.003082in}{0.671424in}}%
\pgfpathlineto{\pgfqpoint{2.003809in}{0.687267in}}%
\pgfpathlineto{\pgfqpoint{2.004392in}{0.674463in}}%
\pgfpathlineto{\pgfqpoint{2.004683in}{0.678419in}}%
\pgfpathlineto{\pgfqpoint{2.005119in}{0.700560in}}%
\pgfpathlineto{\pgfqpoint{2.005556in}{0.680892in}}%
\pgfpathlineto{\pgfqpoint{2.006138in}{0.670023in}}%
\pgfpathlineto{\pgfqpoint{2.006866in}{0.670123in}}%
\pgfpathlineto{\pgfqpoint{2.007303in}{0.671254in}}%
\pgfpathlineto{\pgfqpoint{2.007885in}{0.678081in}}%
\pgfpathlineto{\pgfqpoint{2.008467in}{0.673748in}}%
\pgfpathlineto{\pgfqpoint{2.009777in}{0.673245in}}%
\pgfpathlineto{\pgfqpoint{2.010505in}{0.677405in}}%
\pgfpathlineto{\pgfqpoint{2.010796in}{0.673142in}}%
\pgfpathlineto{\pgfqpoint{2.011232in}{0.669688in}}%
\pgfpathlineto{\pgfqpoint{2.011960in}{0.670721in}}%
\pgfpathlineto{\pgfqpoint{2.012251in}{0.669933in}}%
\pgfpathlineto{\pgfqpoint{2.012542in}{0.671541in}}%
\pgfpathlineto{\pgfqpoint{2.013270in}{0.713717in}}%
\pgfpathlineto{\pgfqpoint{2.013707in}{0.683578in}}%
\pgfpathlineto{\pgfqpoint{2.015017in}{0.671271in}}%
\pgfpathlineto{\pgfqpoint{2.016763in}{0.670555in}}%
\pgfpathlineto{\pgfqpoint{2.017782in}{0.671509in}}%
\pgfpathlineto{\pgfqpoint{2.018364in}{0.676751in}}%
\pgfpathlineto{\pgfqpoint{2.018801in}{0.672720in}}%
\pgfpathlineto{\pgfqpoint{2.019383in}{0.670192in}}%
\pgfpathlineto{\pgfqpoint{2.020111in}{0.670384in}}%
\pgfpathlineto{\pgfqpoint{2.020547in}{0.670575in}}%
\pgfpathlineto{\pgfqpoint{2.021129in}{0.675913in}}%
\pgfpathlineto{\pgfqpoint{2.022003in}{0.672748in}}%
\pgfpathlineto{\pgfqpoint{2.022294in}{0.673832in}}%
\pgfpathlineto{\pgfqpoint{2.022730in}{0.671744in}}%
\pgfpathlineto{\pgfqpoint{2.023313in}{0.669871in}}%
\pgfpathlineto{\pgfqpoint{2.023895in}{0.670536in}}%
\pgfpathlineto{\pgfqpoint{2.025496in}{0.673334in}}%
\pgfpathlineto{\pgfqpoint{2.026078in}{0.685723in}}%
\pgfpathlineto{\pgfqpoint{2.026806in}{0.735460in}}%
\pgfpathlineto{\pgfqpoint{2.027242in}{0.705323in}}%
\pgfpathlineto{\pgfqpoint{2.028698in}{0.675691in}}%
\pgfpathlineto{\pgfqpoint{2.029426in}{0.670104in}}%
\pgfpathlineto{\pgfqpoint{2.029862in}{0.673759in}}%
\pgfpathlineto{\pgfqpoint{2.030444in}{0.724107in}}%
\pgfpathlineto{\pgfqpoint{2.031027in}{0.684242in}}%
\pgfpathlineto{\pgfqpoint{2.031900in}{0.671654in}}%
\pgfpathlineto{\pgfqpoint{2.032336in}{0.671779in}}%
\pgfpathlineto{\pgfqpoint{2.033501in}{0.671265in}}%
\pgfpathlineto{\pgfqpoint{2.034083in}{0.669361in}}%
\pgfpathlineto{\pgfqpoint{2.034811in}{0.670051in}}%
\pgfpathlineto{\pgfqpoint{2.035102in}{0.678165in}}%
\pgfpathlineto{\pgfqpoint{2.035684in}{0.778799in}}%
\pgfpathlineto{\pgfqpoint{2.036266in}{0.696243in}}%
\pgfpathlineto{\pgfqpoint{2.036703in}{0.673831in}}%
\pgfpathlineto{\pgfqpoint{2.037576in}{0.674736in}}%
\pgfpathlineto{\pgfqpoint{2.038158in}{0.669148in}}%
\pgfpathlineto{\pgfqpoint{2.038740in}{0.673658in}}%
\pgfpathlineto{\pgfqpoint{2.039759in}{0.673350in}}%
\pgfpathlineto{\pgfqpoint{2.040341in}{0.715899in}}%
\pgfpathlineto{\pgfqpoint{2.040778in}{0.762771in}}%
\pgfpathlineto{\pgfqpoint{2.041215in}{0.712268in}}%
\pgfpathlineto{\pgfqpoint{2.041942in}{0.672884in}}%
\pgfpathlineto{\pgfqpoint{2.042525in}{0.675099in}}%
\pgfpathlineto{\pgfqpoint{2.042670in}{0.675164in}}%
\pgfpathlineto{\pgfqpoint{2.042816in}{0.674720in}}%
\pgfpathlineto{\pgfqpoint{2.043980in}{0.669480in}}%
\pgfpathlineto{\pgfqpoint{2.044708in}{0.669952in}}%
\pgfpathlineto{\pgfqpoint{2.045581in}{0.670194in}}%
\pgfpathlineto{\pgfqpoint{2.045727in}{0.670614in}}%
\pgfpathlineto{\pgfqpoint{2.045872in}{0.670903in}}%
\pgfpathlineto{\pgfqpoint{2.046454in}{0.669735in}}%
\pgfpathlineto{\pgfqpoint{2.047910in}{0.670296in}}%
\pgfpathlineto{\pgfqpoint{2.048346in}{0.681793in}}%
\pgfpathlineto{\pgfqpoint{2.048783in}{0.694291in}}%
\pgfpathlineto{\pgfqpoint{2.049220in}{0.676938in}}%
\pgfpathlineto{\pgfqpoint{2.049511in}{0.672793in}}%
\pgfpathlineto{\pgfqpoint{2.049802in}{0.680387in}}%
\pgfpathlineto{\pgfqpoint{2.050384in}{0.743525in}}%
\pgfpathlineto{\pgfqpoint{2.050966in}{0.688074in}}%
\pgfpathlineto{\pgfqpoint{2.051403in}{0.670482in}}%
\pgfpathlineto{\pgfqpoint{2.051840in}{0.694885in}}%
\pgfpathlineto{\pgfqpoint{2.052276in}{0.734894in}}%
\pgfpathlineto{\pgfqpoint{2.052713in}{0.687313in}}%
\pgfpathlineto{\pgfqpoint{2.053149in}{0.670643in}}%
\pgfpathlineto{\pgfqpoint{2.053586in}{0.689249in}}%
\pgfpathlineto{\pgfqpoint{2.054023in}{0.739585in}}%
\pgfpathlineto{\pgfqpoint{2.054605in}{0.695623in}}%
\pgfpathlineto{\pgfqpoint{2.055187in}{0.672056in}}%
\pgfpathlineto{\pgfqpoint{2.055624in}{0.689045in}}%
\pgfpathlineto{\pgfqpoint{2.056206in}{0.774917in}}%
\pgfpathlineto{\pgfqpoint{2.056788in}{0.699708in}}%
\pgfpathlineto{\pgfqpoint{2.057225in}{0.673470in}}%
\pgfpathlineto{\pgfqpoint{2.057807in}{0.697884in}}%
\pgfpathlineto{\pgfqpoint{2.058098in}{0.710741in}}%
\pgfpathlineto{\pgfqpoint{2.058535in}{0.687374in}}%
\pgfpathlineto{\pgfqpoint{2.059262in}{0.669799in}}%
\pgfpathlineto{\pgfqpoint{2.059845in}{0.672212in}}%
\pgfpathlineto{\pgfqpoint{2.060136in}{0.673263in}}%
\pgfpathlineto{\pgfqpoint{2.060572in}{0.670589in}}%
\pgfpathlineto{\pgfqpoint{2.061009in}{0.669739in}}%
\pgfpathlineto{\pgfqpoint{2.061300in}{0.671169in}}%
\pgfpathlineto{\pgfqpoint{2.061882in}{0.685206in}}%
\pgfpathlineto{\pgfqpoint{2.062464in}{0.674509in}}%
\pgfpathlineto{\pgfqpoint{2.062901in}{0.671385in}}%
\pgfpathlineto{\pgfqpoint{2.063338in}{0.675162in}}%
\pgfpathlineto{\pgfqpoint{2.064211in}{0.726140in}}%
\pgfpathlineto{\pgfqpoint{2.064793in}{0.686004in}}%
\pgfpathlineto{\pgfqpoint{2.065230in}{0.672255in}}%
\pgfpathlineto{\pgfqpoint{2.065666in}{0.692754in}}%
\pgfpathlineto{\pgfqpoint{2.065957in}{0.710340in}}%
\pgfpathlineto{\pgfqpoint{2.066540in}{0.676512in}}%
\pgfpathlineto{\pgfqpoint{2.066976in}{0.668924in}}%
\pgfpathlineto{\pgfqpoint{2.067558in}{0.678326in}}%
\pgfpathlineto{\pgfqpoint{2.068141in}{0.708995in}}%
\pgfpathlineto{\pgfqpoint{2.068723in}{0.684223in}}%
\pgfpathlineto{\pgfqpoint{2.069160in}{0.674176in}}%
\pgfpathlineto{\pgfqpoint{2.069451in}{0.689707in}}%
\pgfpathlineto{\pgfqpoint{2.070033in}{0.779245in}}%
\pgfpathlineto{\pgfqpoint{2.070615in}{0.703548in}}%
\pgfpathlineto{\pgfqpoint{2.071343in}{0.672114in}}%
\pgfpathlineto{\pgfqpoint{2.071925in}{0.676446in}}%
\pgfpathlineto{\pgfqpoint{2.072362in}{0.673405in}}%
\pgfpathlineto{\pgfqpoint{2.073380in}{0.668673in}}%
\pgfpathlineto{\pgfqpoint{2.073817in}{0.668902in}}%
\pgfpathlineto{\pgfqpoint{2.075564in}{0.672570in}}%
\pgfpathlineto{\pgfqpoint{2.075709in}{0.673389in}}%
\pgfpathlineto{\pgfqpoint{2.076146in}{0.670967in}}%
\pgfpathlineto{\pgfqpoint{2.076873in}{0.668164in}}%
\pgfpathlineto{\pgfqpoint{2.077310in}{0.668821in}}%
\pgfpathlineto{\pgfqpoint{2.078038in}{0.673641in}}%
\pgfpathlineto{\pgfqpoint{2.078474in}{0.669501in}}%
\pgfpathlineto{\pgfqpoint{2.078911in}{0.667748in}}%
\pgfpathlineto{\pgfqpoint{2.079202in}{0.670511in}}%
\pgfpathlineto{\pgfqpoint{2.079930in}{0.717253in}}%
\pgfpathlineto{\pgfqpoint{2.080512in}{0.679722in}}%
\pgfpathlineto{\pgfqpoint{2.080803in}{0.672761in}}%
\pgfpathlineto{\pgfqpoint{2.081240in}{0.686390in}}%
\pgfpathlineto{\pgfqpoint{2.081822in}{0.742715in}}%
\pgfpathlineto{\pgfqpoint{2.082404in}{0.701124in}}%
\pgfpathlineto{\pgfqpoint{2.083132in}{0.670926in}}%
\pgfpathlineto{\pgfqpoint{2.083714in}{0.673532in}}%
\pgfpathlineto{\pgfqpoint{2.084005in}{0.672457in}}%
\pgfpathlineto{\pgfqpoint{2.084733in}{0.668173in}}%
\pgfpathlineto{\pgfqpoint{2.085315in}{0.668950in}}%
\pgfpathlineto{\pgfqpoint{2.086479in}{0.671211in}}%
\pgfpathlineto{\pgfqpoint{2.087207in}{0.714080in}}%
\pgfpathlineto{\pgfqpoint{2.087498in}{0.720872in}}%
\pgfpathlineto{\pgfqpoint{2.087789in}{0.705126in}}%
\pgfpathlineto{\pgfqpoint{2.088517in}{0.677228in}}%
\pgfpathlineto{\pgfqpoint{2.088954in}{0.686956in}}%
\pgfpathlineto{\pgfqpoint{2.089390in}{0.694836in}}%
\pgfpathlineto{\pgfqpoint{2.089827in}{0.683505in}}%
\pgfpathlineto{\pgfqpoint{2.090846in}{0.668051in}}%
\pgfpathlineto{\pgfqpoint{2.091282in}{0.668319in}}%
\pgfpathlineto{\pgfqpoint{2.092592in}{0.668362in}}%
\pgfpathlineto{\pgfqpoint{2.094048in}{0.670066in}}%
\pgfpathlineto{\pgfqpoint{2.094776in}{0.672182in}}%
\pgfpathlineto{\pgfqpoint{2.095358in}{0.670761in}}%
\pgfpathlineto{\pgfqpoint{2.095649in}{0.671251in}}%
\pgfpathlineto{\pgfqpoint{2.096231in}{0.682860in}}%
\pgfpathlineto{\pgfqpoint{2.096668in}{0.691305in}}%
\pgfpathlineto{\pgfqpoint{2.097250in}{0.681789in}}%
\pgfpathlineto{\pgfqpoint{2.098269in}{0.672141in}}%
\pgfpathlineto{\pgfqpoint{2.098705in}{0.674008in}}%
\pgfpathlineto{\pgfqpoint{2.099142in}{0.675918in}}%
\pgfpathlineto{\pgfqpoint{2.099579in}{0.672647in}}%
\pgfpathlineto{\pgfqpoint{2.100452in}{0.669010in}}%
\pgfpathlineto{\pgfqpoint{2.100888in}{0.669177in}}%
\pgfpathlineto{\pgfqpoint{2.103945in}{0.669558in}}%
\pgfpathlineto{\pgfqpoint{2.105546in}{0.669820in}}%
\pgfpathlineto{\pgfqpoint{2.105983in}{0.671085in}}%
\pgfpathlineto{\pgfqpoint{2.106710in}{0.677208in}}%
\pgfpathlineto{\pgfqpoint{2.107292in}{0.673984in}}%
\pgfpathlineto{\pgfqpoint{2.108748in}{0.671131in}}%
\pgfpathlineto{\pgfqpoint{2.111513in}{0.671095in}}%
\pgfpathlineto{\pgfqpoint{2.112241in}{0.674208in}}%
\pgfpathlineto{\pgfqpoint{2.112969in}{0.672889in}}%
\pgfpathlineto{\pgfqpoint{2.113405in}{0.672577in}}%
\pgfpathlineto{\pgfqpoint{2.113842in}{0.673277in}}%
\pgfpathlineto{\pgfqpoint{2.115880in}{0.683937in}}%
\pgfpathlineto{\pgfqpoint{2.116753in}{0.739970in}}%
\pgfpathlineto{\pgfqpoint{2.117335in}{0.699342in}}%
\pgfpathlineto{\pgfqpoint{2.118791in}{0.671294in}}%
\pgfpathlineto{\pgfqpoint{2.119227in}{0.670437in}}%
\pgfpathlineto{\pgfqpoint{2.119664in}{0.671814in}}%
\pgfpathlineto{\pgfqpoint{2.121847in}{0.701770in}}%
\pgfpathlineto{\pgfqpoint{2.122284in}{0.720219in}}%
\pgfpathlineto{\pgfqpoint{2.122866in}{0.697362in}}%
\pgfpathlineto{\pgfqpoint{2.124321in}{0.669935in}}%
\pgfpathlineto{\pgfqpoint{2.126941in}{0.669252in}}%
\pgfpathlineto{\pgfqpoint{2.127814in}{0.670078in}}%
\pgfpathlineto{\pgfqpoint{2.128251in}{0.669550in}}%
\pgfpathlineto{\pgfqpoint{2.130143in}{0.669045in}}%
\pgfpathlineto{\pgfqpoint{2.135674in}{0.669663in}}%
\pgfpathlineto{\pgfqpoint{2.137712in}{0.671338in}}%
\pgfpathlineto{\pgfqpoint{2.138294in}{0.670461in}}%
\pgfpathlineto{\pgfqpoint{2.139458in}{0.671063in}}%
\pgfpathlineto{\pgfqpoint{2.140040in}{0.671505in}}%
\pgfpathlineto{\pgfqpoint{2.140622in}{0.670850in}}%
\pgfpathlineto{\pgfqpoint{2.141350in}{0.670850in}}%
\pgfpathlineto{\pgfqpoint{2.141496in}{0.671143in}}%
\pgfpathlineto{\pgfqpoint{2.142078in}{0.672933in}}%
\pgfpathlineto{\pgfqpoint{2.142660in}{0.671223in}}%
\pgfpathlineto{\pgfqpoint{2.143388in}{0.670917in}}%
\pgfpathlineto{\pgfqpoint{2.143824in}{0.671314in}}%
\pgfpathlineto{\pgfqpoint{2.145717in}{0.672662in}}%
\pgfpathlineto{\pgfqpoint{2.146444in}{0.692148in}}%
\pgfpathlineto{\pgfqpoint{2.147026in}{0.676851in}}%
\pgfpathlineto{\pgfqpoint{2.147318in}{0.674162in}}%
\pgfpathlineto{\pgfqpoint{2.147754in}{0.679770in}}%
\pgfpathlineto{\pgfqpoint{2.148045in}{0.684259in}}%
\pgfpathlineto{\pgfqpoint{2.148482in}{0.677725in}}%
\pgfpathlineto{\pgfqpoint{2.149210in}{0.670882in}}%
\pgfpathlineto{\pgfqpoint{2.149792in}{0.671218in}}%
\pgfpathlineto{\pgfqpoint{2.151538in}{0.673735in}}%
\pgfpathlineto{\pgfqpoint{2.151975in}{0.718125in}}%
\pgfpathlineto{\pgfqpoint{2.152557in}{0.817791in}}%
\pgfpathlineto{\pgfqpoint{2.153139in}{0.732013in}}%
\pgfpathlineto{\pgfqpoint{2.153722in}{0.702308in}}%
\pgfpathlineto{\pgfqpoint{2.154304in}{0.710951in}}%
\pgfpathlineto{\pgfqpoint{2.155468in}{0.742832in}}%
\pgfpathlineto{\pgfqpoint{2.155759in}{0.730348in}}%
\pgfpathlineto{\pgfqpoint{2.156632in}{0.676652in}}%
\pgfpathlineto{\pgfqpoint{2.157215in}{0.687236in}}%
\pgfpathlineto{\pgfqpoint{2.157506in}{0.691334in}}%
\pgfpathlineto{\pgfqpoint{2.157942in}{0.679259in}}%
\pgfpathlineto{\pgfqpoint{2.158670in}{0.670947in}}%
\pgfpathlineto{\pgfqpoint{2.159252in}{0.672114in}}%
\pgfpathlineto{\pgfqpoint{2.159834in}{0.678219in}}%
\pgfpathlineto{\pgfqpoint{2.160417in}{0.673027in}}%
\pgfpathlineto{\pgfqpoint{2.160708in}{0.671906in}}%
\pgfpathlineto{\pgfqpoint{2.160999in}{0.673633in}}%
\pgfpathlineto{\pgfqpoint{2.161581in}{0.679750in}}%
\pgfpathlineto{\pgfqpoint{2.162018in}{0.673522in}}%
\pgfpathlineto{\pgfqpoint{2.163328in}{0.670281in}}%
\pgfpathlineto{\pgfqpoint{2.168713in}{0.671114in}}%
\pgfpathlineto{\pgfqpoint{2.169440in}{0.674730in}}%
\pgfpathlineto{\pgfqpoint{2.169877in}{0.671853in}}%
\pgfpathlineto{\pgfqpoint{2.170168in}{0.670819in}}%
\pgfpathlineto{\pgfqpoint{2.170605in}{0.672966in}}%
\pgfpathlineto{\pgfqpoint{2.171187in}{0.681489in}}%
\pgfpathlineto{\pgfqpoint{2.171624in}{0.673541in}}%
\pgfpathlineto{\pgfqpoint{2.172060in}{0.670466in}}%
\pgfpathlineto{\pgfqpoint{2.172642in}{0.673142in}}%
\pgfpathlineto{\pgfqpoint{2.173079in}{0.675346in}}%
\pgfpathlineto{\pgfqpoint{2.173516in}{0.672253in}}%
\pgfpathlineto{\pgfqpoint{2.174680in}{0.670181in}}%
\pgfpathlineto{\pgfqpoint{2.174826in}{0.670206in}}%
\pgfpathlineto{\pgfqpoint{2.178610in}{0.671962in}}%
\pgfpathlineto{\pgfqpoint{2.179338in}{0.677346in}}%
\pgfpathlineto{\pgfqpoint{2.179774in}{0.672764in}}%
\pgfpathlineto{\pgfqpoint{2.180356in}{0.670661in}}%
\pgfpathlineto{\pgfqpoint{2.180793in}{0.671988in}}%
\pgfpathlineto{\pgfqpoint{2.181230in}{0.673195in}}%
\pgfpathlineto{\pgfqpoint{2.181812in}{0.671774in}}%
\pgfpathlineto{\pgfqpoint{2.183267in}{0.670795in}}%
\pgfpathlineto{\pgfqpoint{2.185305in}{0.672375in}}%
\pgfpathlineto{\pgfqpoint{2.185742in}{0.675882in}}%
\pgfpathlineto{\pgfqpoint{2.186178in}{0.672492in}}%
\pgfpathlineto{\pgfqpoint{2.186615in}{0.670159in}}%
\pgfpathlineto{\pgfqpoint{2.186906in}{0.672759in}}%
\pgfpathlineto{\pgfqpoint{2.187343in}{0.682357in}}%
\pgfpathlineto{\pgfqpoint{2.187779in}{0.673382in}}%
\pgfpathlineto{\pgfqpoint{2.188216in}{0.668624in}}%
\pgfpathlineto{\pgfqpoint{2.188798in}{0.671596in}}%
\pgfpathlineto{\pgfqpoint{2.189235in}{0.677631in}}%
\pgfpathlineto{\pgfqpoint{2.189817in}{0.670611in}}%
\pgfpathlineto{\pgfqpoint{2.190108in}{0.669514in}}%
\pgfpathlineto{\pgfqpoint{2.190545in}{0.672391in}}%
\pgfpathlineto{\pgfqpoint{2.190836in}{0.675202in}}%
\pgfpathlineto{\pgfqpoint{2.191272in}{0.670242in}}%
\pgfpathlineto{\pgfqpoint{2.191563in}{0.668782in}}%
\pgfpathlineto{\pgfqpoint{2.192146in}{0.670291in}}%
\pgfpathlineto{\pgfqpoint{2.192437in}{0.671160in}}%
\pgfpathlineto{\pgfqpoint{2.193019in}{0.668835in}}%
\pgfpathlineto{\pgfqpoint{2.193310in}{0.672176in}}%
\pgfpathlineto{\pgfqpoint{2.194038in}{0.725663in}}%
\pgfpathlineto{\pgfqpoint{2.194620in}{0.690981in}}%
\pgfpathlineto{\pgfqpoint{2.195202in}{0.677069in}}%
\pgfpathlineto{\pgfqpoint{2.195784in}{0.684786in}}%
\pgfpathlineto{\pgfqpoint{2.196075in}{0.687562in}}%
\pgfpathlineto{\pgfqpoint{2.196512in}{0.680759in}}%
\pgfpathlineto{\pgfqpoint{2.197822in}{0.666536in}}%
\pgfpathlineto{\pgfqpoint{2.198113in}{0.666750in}}%
\pgfpathlineto{\pgfqpoint{2.200151in}{0.667870in}}%
\pgfpathlineto{\pgfqpoint{2.200296in}{0.667737in}}%
\pgfpathlineto{\pgfqpoint{2.200587in}{0.667581in}}%
\pgfpathlineto{\pgfqpoint{2.201024in}{0.668475in}}%
\pgfpathlineto{\pgfqpoint{2.202334in}{0.673020in}}%
\pgfpathlineto{\pgfqpoint{2.203062in}{0.703741in}}%
\pgfpathlineto{\pgfqpoint{2.203498in}{0.685913in}}%
\pgfpathlineto{\pgfqpoint{2.204080in}{0.669524in}}%
\pgfpathlineto{\pgfqpoint{2.204663in}{0.678394in}}%
\pgfpathlineto{\pgfqpoint{2.204954in}{0.682713in}}%
\pgfpathlineto{\pgfqpoint{2.205390in}{0.672522in}}%
\pgfpathlineto{\pgfqpoint{2.205827in}{0.667297in}}%
\pgfpathlineto{\pgfqpoint{2.206409in}{0.673221in}}%
\pgfpathlineto{\pgfqpoint{2.206991in}{0.705268in}}%
\pgfpathlineto{\pgfqpoint{2.207573in}{0.680543in}}%
\pgfpathlineto{\pgfqpoint{2.207865in}{0.673487in}}%
\pgfpathlineto{\pgfqpoint{2.208301in}{0.690180in}}%
\pgfpathlineto{\pgfqpoint{2.208883in}{0.753249in}}%
\pgfpathlineto{\pgfqpoint{2.209466in}{0.697603in}}%
\pgfpathlineto{\pgfqpoint{2.210193in}{0.670082in}}%
\pgfpathlineto{\pgfqpoint{2.210775in}{0.676971in}}%
\pgfpathlineto{\pgfqpoint{2.211067in}{0.680964in}}%
\pgfpathlineto{\pgfqpoint{2.211649in}{0.673152in}}%
\pgfpathlineto{\pgfqpoint{2.212231in}{0.667951in}}%
\pgfpathlineto{\pgfqpoint{2.212668in}{0.671626in}}%
\pgfpathlineto{\pgfqpoint{2.213395in}{0.705473in}}%
\pgfpathlineto{\pgfqpoint{2.213832in}{0.680867in}}%
\pgfpathlineto{\pgfqpoint{2.214269in}{0.671991in}}%
\pgfpathlineto{\pgfqpoint{2.214705in}{0.687759in}}%
\pgfpathlineto{\pgfqpoint{2.215142in}{0.708192in}}%
\pgfpathlineto{\pgfqpoint{2.215578in}{0.689660in}}%
\pgfpathlineto{\pgfqpoint{2.216452in}{0.667992in}}%
\pgfpathlineto{\pgfqpoint{2.216888in}{0.669764in}}%
\pgfpathlineto{\pgfqpoint{2.217179in}{0.671069in}}%
\pgfpathlineto{\pgfqpoint{2.217616in}{0.668433in}}%
\pgfpathlineto{\pgfqpoint{2.217907in}{0.667472in}}%
\pgfpathlineto{\pgfqpoint{2.218489in}{0.668421in}}%
\pgfpathlineto{\pgfqpoint{2.219072in}{0.670752in}}%
\pgfpathlineto{\pgfqpoint{2.219799in}{0.669531in}}%
\pgfpathlineto{\pgfqpoint{2.220818in}{0.669350in}}%
\pgfpathlineto{\pgfqpoint{2.220964in}{0.669559in}}%
\pgfpathlineto{\pgfqpoint{2.221546in}{0.674096in}}%
\pgfpathlineto{\pgfqpoint{2.222419in}{0.693888in}}%
\pgfpathlineto{\pgfqpoint{2.223001in}{0.685475in}}%
\pgfpathlineto{\pgfqpoint{2.223875in}{0.676347in}}%
\pgfpathlineto{\pgfqpoint{2.224311in}{0.679947in}}%
\pgfpathlineto{\pgfqpoint{2.225039in}{0.727430in}}%
\pgfpathlineto{\pgfqpoint{2.225330in}{0.746445in}}%
\pgfpathlineto{\pgfqpoint{2.225767in}{0.713008in}}%
\pgfpathlineto{\pgfqpoint{2.226494in}{0.671154in}}%
\pgfpathlineto{\pgfqpoint{2.227077in}{0.686595in}}%
\pgfpathlineto{\pgfqpoint{2.227513in}{0.700543in}}%
\pgfpathlineto{\pgfqpoint{2.228095in}{0.683752in}}%
\pgfpathlineto{\pgfqpoint{2.228823in}{0.667480in}}%
\pgfpathlineto{\pgfqpoint{2.229405in}{0.669839in}}%
\pgfpathlineto{\pgfqpoint{2.230279in}{0.665771in}}%
\pgfpathlineto{\pgfqpoint{2.231297in}{0.667590in}}%
\pgfpathlineto{\pgfqpoint{2.232171in}{0.699689in}}%
\pgfpathlineto{\pgfqpoint{2.232898in}{0.683128in}}%
\pgfpathlineto{\pgfqpoint{2.233335in}{0.678645in}}%
\pgfpathlineto{\pgfqpoint{2.233772in}{0.682764in}}%
\pgfpathlineto{\pgfqpoint{2.234645in}{0.723003in}}%
\pgfpathlineto{\pgfqpoint{2.235082in}{0.696260in}}%
\pgfpathlineto{\pgfqpoint{2.236100in}{0.666671in}}%
\pgfpathlineto{\pgfqpoint{2.236537in}{0.666856in}}%
\pgfpathlineto{\pgfqpoint{2.241049in}{0.668606in}}%
\pgfpathlineto{\pgfqpoint{2.242068in}{0.680061in}}%
\pgfpathlineto{\pgfqpoint{2.242650in}{0.672604in}}%
\pgfpathlineto{\pgfqpoint{2.243087in}{0.670085in}}%
\pgfpathlineto{\pgfqpoint{2.243814in}{0.671603in}}%
\pgfpathlineto{\pgfqpoint{2.244542in}{0.667989in}}%
\pgfpathlineto{\pgfqpoint{2.244979in}{0.669343in}}%
\pgfpathlineto{\pgfqpoint{2.245415in}{0.672443in}}%
\pgfpathlineto{\pgfqpoint{2.246143in}{0.669967in}}%
\pgfpathlineto{\pgfqpoint{2.246580in}{0.676709in}}%
\pgfpathlineto{\pgfqpoint{2.247453in}{0.713761in}}%
\pgfpathlineto{\pgfqpoint{2.247890in}{0.692531in}}%
\pgfpathlineto{\pgfqpoint{2.248617in}{0.672207in}}%
\pgfpathlineto{\pgfqpoint{2.249200in}{0.677657in}}%
\pgfpathlineto{\pgfqpoint{2.249345in}{0.677599in}}%
\pgfpathlineto{\pgfqpoint{2.250364in}{0.666695in}}%
\pgfpathlineto{\pgfqpoint{2.251237in}{0.667336in}}%
\pgfpathlineto{\pgfqpoint{2.253275in}{0.670478in}}%
\pgfpathlineto{\pgfqpoint{2.253711in}{0.672443in}}%
\pgfpathlineto{\pgfqpoint{2.254148in}{0.670035in}}%
\pgfpathlineto{\pgfqpoint{2.255021in}{0.669003in}}%
\pgfpathlineto{\pgfqpoint{2.255458in}{0.669183in}}%
\pgfpathlineto{\pgfqpoint{2.259970in}{0.671275in}}%
\pgfpathlineto{\pgfqpoint{2.260552in}{0.674809in}}%
\pgfpathlineto{\pgfqpoint{2.260989in}{0.671791in}}%
\pgfpathlineto{\pgfqpoint{2.261425in}{0.670371in}}%
\pgfpathlineto{\pgfqpoint{2.262153in}{0.671264in}}%
\pgfpathlineto{\pgfqpoint{2.263754in}{0.670528in}}%
\pgfpathlineto{\pgfqpoint{2.264918in}{0.670737in}}%
\pgfpathlineto{\pgfqpoint{2.265355in}{0.673993in}}%
\pgfpathlineto{\pgfqpoint{2.266083in}{0.685834in}}%
\pgfpathlineto{\pgfqpoint{2.266519in}{0.679182in}}%
\pgfpathlineto{\pgfqpoint{2.266811in}{0.677460in}}%
\pgfpathlineto{\pgfqpoint{2.267102in}{0.682173in}}%
\pgfpathlineto{\pgfqpoint{2.267684in}{0.699039in}}%
\pgfpathlineto{\pgfqpoint{2.268120in}{0.684064in}}%
\pgfpathlineto{\pgfqpoint{2.268848in}{0.670075in}}%
\pgfpathlineto{\pgfqpoint{2.269430in}{0.670237in}}%
\pgfpathlineto{\pgfqpoint{2.270158in}{0.670332in}}%
\pgfpathlineto{\pgfqpoint{2.270886in}{0.679009in}}%
\pgfpathlineto{\pgfqpoint{2.271614in}{0.672793in}}%
\pgfpathlineto{\pgfqpoint{2.271759in}{0.672715in}}%
\pgfpathlineto{\pgfqpoint{2.272050in}{0.673679in}}%
\pgfpathlineto{\pgfqpoint{2.272341in}{0.674232in}}%
\pgfpathlineto{\pgfqpoint{2.272632in}{0.672783in}}%
\pgfpathlineto{\pgfqpoint{2.274088in}{0.670538in}}%
\pgfpathlineto{\pgfqpoint{2.275834in}{0.671745in}}%
\pgfpathlineto{\pgfqpoint{2.276271in}{0.683370in}}%
\pgfpathlineto{\pgfqpoint{2.276708in}{0.706162in}}%
\pgfpathlineto{\pgfqpoint{2.277290in}{0.678879in}}%
\pgfpathlineto{\pgfqpoint{2.277581in}{0.675621in}}%
\pgfpathlineto{\pgfqpoint{2.277872in}{0.683178in}}%
\pgfpathlineto{\pgfqpoint{2.278163in}{0.691272in}}%
\pgfpathlineto{\pgfqpoint{2.278600in}{0.679129in}}%
\pgfpathlineto{\pgfqpoint{2.279182in}{0.669939in}}%
\pgfpathlineto{\pgfqpoint{2.279910in}{0.670498in}}%
\pgfpathlineto{\pgfqpoint{2.280346in}{0.670800in}}%
\pgfpathlineto{\pgfqpoint{2.280783in}{0.670106in}}%
\pgfpathlineto{\pgfqpoint{2.281074in}{0.670095in}}%
\pgfpathlineto{\pgfqpoint{2.281220in}{0.670384in}}%
\pgfpathlineto{\pgfqpoint{2.281802in}{0.674977in}}%
\pgfpathlineto{\pgfqpoint{2.282384in}{0.670773in}}%
\pgfpathlineto{\pgfqpoint{2.282821in}{0.669933in}}%
\pgfpathlineto{\pgfqpoint{2.283403in}{0.670644in}}%
\pgfpathlineto{\pgfqpoint{2.283985in}{0.671167in}}%
\pgfpathlineto{\pgfqpoint{2.284422in}{0.670509in}}%
\pgfpathlineto{\pgfqpoint{2.285004in}{0.670293in}}%
\pgfpathlineto{\pgfqpoint{2.285149in}{0.670743in}}%
\pgfpathlineto{\pgfqpoint{2.286023in}{0.688803in}}%
\pgfpathlineto{\pgfqpoint{2.286605in}{0.675159in}}%
\pgfpathlineto{\pgfqpoint{2.287041in}{0.672543in}}%
\pgfpathlineto{\pgfqpoint{2.287478in}{0.675601in}}%
\pgfpathlineto{\pgfqpoint{2.288206in}{0.684395in}}%
\pgfpathlineto{\pgfqpoint{2.288642in}{0.679539in}}%
\pgfpathlineto{\pgfqpoint{2.289516in}{0.669697in}}%
\pgfpathlineto{\pgfqpoint{2.290098in}{0.670168in}}%
\pgfpathlineto{\pgfqpoint{2.293737in}{0.669851in}}%
\pgfpathlineto{\pgfqpoint{2.296065in}{0.671006in}}%
\pgfpathlineto{\pgfqpoint{2.298976in}{0.681260in}}%
\pgfpathlineto{\pgfqpoint{2.299558in}{0.708060in}}%
\pgfpathlineto{\pgfqpoint{2.300141in}{0.684930in}}%
\pgfpathlineto{\pgfqpoint{2.301014in}{0.670328in}}%
\pgfpathlineto{\pgfqpoint{2.301450in}{0.670534in}}%
\pgfpathlineto{\pgfqpoint{2.303779in}{0.672438in}}%
\pgfpathlineto{\pgfqpoint{2.304507in}{0.714404in}}%
\pgfpathlineto{\pgfqpoint{2.305235in}{0.688708in}}%
\pgfpathlineto{\pgfqpoint{2.305526in}{0.684841in}}%
\pgfpathlineto{\pgfqpoint{2.305962in}{0.692198in}}%
\pgfpathlineto{\pgfqpoint{2.306399in}{0.699964in}}%
\pgfpathlineto{\pgfqpoint{2.306981in}{0.691869in}}%
\pgfpathlineto{\pgfqpoint{2.308291in}{0.669903in}}%
\pgfpathlineto{\pgfqpoint{2.308728in}{0.670770in}}%
\pgfpathlineto{\pgfqpoint{2.309164in}{0.682497in}}%
\pgfpathlineto{\pgfqpoint{2.310038in}{0.765276in}}%
\pgfpathlineto{\pgfqpoint{2.310620in}{0.727392in}}%
\pgfpathlineto{\pgfqpoint{2.310911in}{0.716179in}}%
\pgfpathlineto{\pgfqpoint{2.311348in}{0.733395in}}%
\pgfpathlineto{\pgfqpoint{2.311784in}{0.756634in}}%
\pgfpathlineto{\pgfqpoint{2.312221in}{0.718649in}}%
\pgfpathlineto{\pgfqpoint{2.313676in}{0.670603in}}%
\pgfpathlineto{\pgfqpoint{2.316733in}{0.671337in}}%
\pgfpathlineto{\pgfqpoint{2.317460in}{0.676024in}}%
\pgfpathlineto{\pgfqpoint{2.318188in}{0.686927in}}%
\pgfpathlineto{\pgfqpoint{2.318770in}{0.681289in}}%
\pgfpathlineto{\pgfqpoint{2.318916in}{0.680673in}}%
\pgfpathlineto{\pgfqpoint{2.319207in}{0.683167in}}%
\pgfpathlineto{\pgfqpoint{2.319644in}{0.688016in}}%
\pgfpathlineto{\pgfqpoint{2.319935in}{0.682320in}}%
\pgfpathlineto{\pgfqpoint{2.320954in}{0.670205in}}%
\pgfpathlineto{\pgfqpoint{2.321390in}{0.670667in}}%
\pgfpathlineto{\pgfqpoint{2.322409in}{0.681276in}}%
\pgfpathlineto{\pgfqpoint{2.323282in}{0.675184in}}%
\pgfpathlineto{\pgfqpoint{2.326048in}{0.670963in}}%
\pgfpathlineto{\pgfqpoint{2.326484in}{0.671866in}}%
\pgfpathlineto{\pgfqpoint{2.327358in}{0.676697in}}%
\pgfpathlineto{\pgfqpoint{2.327940in}{0.673724in}}%
\pgfpathlineto{\pgfqpoint{2.328813in}{0.670320in}}%
\pgfpathlineto{\pgfqpoint{2.329395in}{0.670560in}}%
\pgfpathlineto{\pgfqpoint{2.330996in}{0.670049in}}%
\pgfpathlineto{\pgfqpoint{2.332452in}{0.668394in}}%
\pgfpathlineto{\pgfqpoint{2.332743in}{0.668999in}}%
\pgfpathlineto{\pgfqpoint{2.333179in}{0.670179in}}%
\pgfpathlineto{\pgfqpoint{2.333470in}{0.668751in}}%
\pgfpathlineto{\pgfqpoint{2.333907in}{0.667018in}}%
\pgfpathlineto{\pgfqpoint{2.334635in}{0.667911in}}%
\pgfpathlineto{\pgfqpoint{2.335363in}{0.668414in}}%
\pgfpathlineto{\pgfqpoint{2.335799in}{0.686066in}}%
\pgfpathlineto{\pgfqpoint{2.336381in}{0.725796in}}%
\pgfpathlineto{\pgfqpoint{2.336964in}{0.707663in}}%
\pgfpathlineto{\pgfqpoint{2.337837in}{0.691008in}}%
\pgfpathlineto{\pgfqpoint{2.338419in}{0.691554in}}%
\pgfpathlineto{\pgfqpoint{2.339001in}{0.685912in}}%
\pgfpathlineto{\pgfqpoint{2.340311in}{0.669113in}}%
\pgfpathlineto{\pgfqpoint{2.340893in}{0.670803in}}%
\pgfpathlineto{\pgfqpoint{2.341475in}{0.673777in}}%
\pgfpathlineto{\pgfqpoint{2.342203in}{0.671964in}}%
\pgfpathlineto{\pgfqpoint{2.343222in}{0.671578in}}%
\pgfpathlineto{\pgfqpoint{2.343513in}{0.671991in}}%
\pgfpathlineto{\pgfqpoint{2.344241in}{0.676885in}}%
\pgfpathlineto{\pgfqpoint{2.345987in}{0.721208in}}%
\pgfpathlineto{\pgfqpoint{2.346715in}{1.149003in}}%
\pgfpathlineto{\pgfqpoint{2.347297in}{0.833065in}}%
\pgfpathlineto{\pgfqpoint{2.348171in}{0.686832in}}%
\pgfpathlineto{\pgfqpoint{2.348607in}{0.700690in}}%
\pgfpathlineto{\pgfqpoint{2.348753in}{0.703489in}}%
\pgfpathlineto{\pgfqpoint{2.349189in}{0.691223in}}%
\pgfpathlineto{\pgfqpoint{2.350208in}{0.672029in}}%
\pgfpathlineto{\pgfqpoint{2.350645in}{0.672182in}}%
\pgfpathlineto{\pgfqpoint{2.351955in}{0.673886in}}%
\pgfpathlineto{\pgfqpoint{2.352683in}{0.692365in}}%
\pgfpathlineto{\pgfqpoint{2.353265in}{0.680948in}}%
\pgfpathlineto{\pgfqpoint{2.353847in}{0.675700in}}%
\pgfpathlineto{\pgfqpoint{2.354284in}{0.680841in}}%
\pgfpathlineto{\pgfqpoint{2.355011in}{0.695393in}}%
\pgfpathlineto{\pgfqpoint{2.355448in}{0.684535in}}%
\pgfpathlineto{\pgfqpoint{2.356903in}{0.672196in}}%
\pgfpathlineto{\pgfqpoint{2.360542in}{0.671850in}}%
\pgfpathlineto{\pgfqpoint{2.361270in}{0.684970in}}%
\pgfpathlineto{\pgfqpoint{2.361852in}{0.676660in}}%
\pgfpathlineto{\pgfqpoint{2.363162in}{0.672440in}}%
\pgfpathlineto{\pgfqpoint{2.363598in}{0.678941in}}%
\pgfpathlineto{\pgfqpoint{2.364326in}{0.718641in}}%
\pgfpathlineto{\pgfqpoint{2.365054in}{0.699634in}}%
\pgfpathlineto{\pgfqpoint{2.365345in}{0.700410in}}%
\pgfpathlineto{\pgfqpoint{2.365491in}{0.699566in}}%
\pgfpathlineto{\pgfqpoint{2.366218in}{0.684500in}}%
\pgfpathlineto{\pgfqpoint{2.366509in}{0.695450in}}%
\pgfpathlineto{\pgfqpoint{2.366946in}{0.892242in}}%
\pgfpathlineto{\pgfqpoint{2.367383in}{1.140333in}}%
\pgfpathlineto{\pgfqpoint{2.367965in}{0.835458in}}%
\pgfpathlineto{\pgfqpoint{2.368984in}{0.703681in}}%
\pgfpathlineto{\pgfqpoint{2.369420in}{0.705791in}}%
\pgfpathlineto{\pgfqpoint{2.370294in}{0.724087in}}%
\pgfpathlineto{\pgfqpoint{2.370585in}{0.715817in}}%
\pgfpathlineto{\pgfqpoint{2.371458in}{0.672653in}}%
\pgfpathlineto{\pgfqpoint{2.371895in}{0.684607in}}%
\pgfpathlineto{\pgfqpoint{2.372331in}{0.705716in}}%
\pgfpathlineto{\pgfqpoint{2.372768in}{0.678003in}}%
\pgfpathlineto{\pgfqpoint{2.373059in}{0.672902in}}%
\pgfpathlineto{\pgfqpoint{2.373350in}{0.680374in}}%
\pgfpathlineto{\pgfqpoint{2.373932in}{0.783641in}}%
\pgfpathlineto{\pgfqpoint{2.374514in}{0.708312in}}%
\pgfpathlineto{\pgfqpoint{2.375824in}{0.678199in}}%
\pgfpathlineto{\pgfqpoint{2.376406in}{0.672410in}}%
\pgfpathlineto{\pgfqpoint{2.377134in}{0.673713in}}%
\pgfpathlineto{\pgfqpoint{2.377571in}{0.673304in}}%
\pgfpathlineto{\pgfqpoint{2.377716in}{0.673833in}}%
\pgfpathlineto{\pgfqpoint{2.378299in}{0.687453in}}%
\pgfpathlineto{\pgfqpoint{2.379026in}{0.707902in}}%
\pgfpathlineto{\pgfqpoint{2.379463in}{0.696804in}}%
\pgfpathlineto{\pgfqpoint{2.380191in}{0.678300in}}%
\pgfpathlineto{\pgfqpoint{2.380773in}{0.684745in}}%
\pgfpathlineto{\pgfqpoint{2.380918in}{0.685390in}}%
\pgfpathlineto{\pgfqpoint{2.381064in}{0.683817in}}%
\pgfpathlineto{\pgfqpoint{2.381937in}{0.670757in}}%
\pgfpathlineto{\pgfqpoint{2.382665in}{0.670912in}}%
\pgfpathlineto{\pgfqpoint{2.382956in}{0.670856in}}%
\pgfpathlineto{\pgfqpoint{2.383102in}{0.671182in}}%
\pgfpathlineto{\pgfqpoint{2.383538in}{0.683261in}}%
\pgfpathlineto{\pgfqpoint{2.383975in}{0.706229in}}%
\pgfpathlineto{\pgfqpoint{2.384557in}{0.679050in}}%
\pgfpathlineto{\pgfqpoint{2.384848in}{0.673270in}}%
\pgfpathlineto{\pgfqpoint{2.385285in}{0.683611in}}%
\pgfpathlineto{\pgfqpoint{2.385721in}{0.702258in}}%
\pgfpathlineto{\pgfqpoint{2.386158in}{0.684036in}}%
\pgfpathlineto{\pgfqpoint{2.387031in}{0.669636in}}%
\pgfpathlineto{\pgfqpoint{2.387468in}{0.671659in}}%
\pgfpathlineto{\pgfqpoint{2.388196in}{0.678667in}}%
\pgfpathlineto{\pgfqpoint{2.388632in}{0.673520in}}%
\pgfpathlineto{\pgfqpoint{2.389214in}{0.669340in}}%
\pgfpathlineto{\pgfqpoint{2.389651in}{0.671344in}}%
\pgfpathlineto{\pgfqpoint{2.390233in}{0.676365in}}%
\pgfpathlineto{\pgfqpoint{2.390670in}{0.670625in}}%
\pgfpathlineto{\pgfqpoint{2.391252in}{0.667501in}}%
\pgfpathlineto{\pgfqpoint{2.391689in}{0.670002in}}%
\pgfpathlineto{\pgfqpoint{2.391834in}{0.670852in}}%
\pgfpathlineto{\pgfqpoint{2.392271in}{0.667372in}}%
\pgfpathlineto{\pgfqpoint{2.392562in}{0.667060in}}%
\pgfpathlineto{\pgfqpoint{2.392708in}{0.667949in}}%
\pgfpathlineto{\pgfqpoint{2.393290in}{0.693851in}}%
\pgfpathlineto{\pgfqpoint{2.393872in}{0.670007in}}%
\pgfpathlineto{\pgfqpoint{2.394163in}{0.667961in}}%
\pgfpathlineto{\pgfqpoint{2.394600in}{0.674799in}}%
\pgfpathlineto{\pgfqpoint{2.394745in}{0.676481in}}%
\pgfpathlineto{\pgfqpoint{2.395182in}{0.669698in}}%
\pgfpathlineto{\pgfqpoint{2.395764in}{0.665148in}}%
\pgfpathlineto{\pgfqpoint{2.396492in}{0.666068in}}%
\pgfpathlineto{\pgfqpoint{2.397947in}{0.668578in}}%
\pgfpathlineto{\pgfqpoint{2.398529in}{0.673941in}}%
\pgfpathlineto{\pgfqpoint{2.398966in}{0.670218in}}%
\pgfpathlineto{\pgfqpoint{2.399403in}{0.668038in}}%
\pgfpathlineto{\pgfqpoint{2.399839in}{0.671882in}}%
\pgfpathlineto{\pgfqpoint{2.400276in}{0.678156in}}%
\pgfpathlineto{\pgfqpoint{2.400713in}{0.669941in}}%
\pgfpathlineto{\pgfqpoint{2.401149in}{0.667381in}}%
\pgfpathlineto{\pgfqpoint{2.401586in}{0.669570in}}%
\pgfpathlineto{\pgfqpoint{2.402023in}{0.674190in}}%
\pgfpathlineto{\pgfqpoint{2.402605in}{0.668600in}}%
\pgfpathlineto{\pgfqpoint{2.402896in}{0.668020in}}%
\pgfpathlineto{\pgfqpoint{2.403187in}{0.669349in}}%
\pgfpathlineto{\pgfqpoint{2.403915in}{0.687810in}}%
\pgfpathlineto{\pgfqpoint{2.404351in}{0.674342in}}%
\pgfpathlineto{\pgfqpoint{2.404788in}{0.667757in}}%
\pgfpathlineto{\pgfqpoint{2.405225in}{0.674746in}}%
\pgfpathlineto{\pgfqpoint{2.405807in}{0.702164in}}%
\pgfpathlineto{\pgfqpoint{2.406243in}{0.682458in}}%
\pgfpathlineto{\pgfqpoint{2.406826in}{0.668140in}}%
\pgfpathlineto{\pgfqpoint{2.407408in}{0.675113in}}%
\pgfpathlineto{\pgfqpoint{2.407699in}{0.678737in}}%
\pgfpathlineto{\pgfqpoint{2.408281in}{0.672144in}}%
\pgfpathlineto{\pgfqpoint{2.409300in}{0.667190in}}%
\pgfpathlineto{\pgfqpoint{2.409736in}{0.667431in}}%
\pgfpathlineto{\pgfqpoint{2.411483in}{0.667627in}}%
\pgfpathlineto{\pgfqpoint{2.413521in}{0.668985in}}%
\pgfpathlineto{\pgfqpoint{2.414103in}{0.671098in}}%
\pgfpathlineto{\pgfqpoint{2.414685in}{0.669312in}}%
\pgfpathlineto{\pgfqpoint{2.415122in}{0.668699in}}%
\pgfpathlineto{\pgfqpoint{2.415413in}{0.669361in}}%
\pgfpathlineto{\pgfqpoint{2.418906in}{0.683271in}}%
\pgfpathlineto{\pgfqpoint{2.421817in}{0.667171in}}%
\pgfpathlineto{\pgfqpoint{2.421962in}{0.667256in}}%
\pgfpathlineto{\pgfqpoint{2.424582in}{0.668605in}}%
\pgfpathlineto{\pgfqpoint{2.426038in}{0.670111in}}%
\pgfpathlineto{\pgfqpoint{2.426620in}{0.691244in}}%
\pgfpathlineto{\pgfqpoint{2.426911in}{0.704903in}}%
\pgfpathlineto{\pgfqpoint{2.427493in}{0.682051in}}%
\pgfpathlineto{\pgfqpoint{2.428075in}{0.670704in}}%
\pgfpathlineto{\pgfqpoint{2.428657in}{0.675188in}}%
\pgfpathlineto{\pgfqpoint{2.428948in}{0.673699in}}%
\pgfpathlineto{\pgfqpoint{2.429676in}{0.667549in}}%
\pgfpathlineto{\pgfqpoint{2.430404in}{0.668095in}}%
\pgfpathlineto{\pgfqpoint{2.433169in}{0.669372in}}%
\pgfpathlineto{\pgfqpoint{2.433460in}{0.669552in}}%
\pgfpathlineto{\pgfqpoint{2.433751in}{0.673993in}}%
\pgfpathlineto{\pgfqpoint{2.434625in}{0.750445in}}%
\pgfpathlineto{\pgfqpoint{2.435207in}{0.709585in}}%
\pgfpathlineto{\pgfqpoint{2.435935in}{0.686482in}}%
\pgfpathlineto{\pgfqpoint{2.436517in}{0.688806in}}%
\pgfpathlineto{\pgfqpoint{2.437390in}{0.674052in}}%
\pgfpathlineto{\pgfqpoint{2.438263in}{0.669169in}}%
\pgfpathlineto{\pgfqpoint{2.438700in}{0.669389in}}%
\pgfpathlineto{\pgfqpoint{2.439719in}{0.670181in}}%
\pgfpathlineto{\pgfqpoint{2.440155in}{0.690987in}}%
\pgfpathlineto{\pgfqpoint{2.440738in}{0.764850in}}%
\pgfpathlineto{\pgfqpoint{2.441320in}{0.727113in}}%
\pgfpathlineto{\pgfqpoint{2.442048in}{0.693423in}}%
\pgfpathlineto{\pgfqpoint{2.442630in}{0.703881in}}%
\pgfpathlineto{\pgfqpoint{2.442775in}{0.706262in}}%
\pgfpathlineto{\pgfqpoint{2.443066in}{0.701719in}}%
\pgfpathlineto{\pgfqpoint{2.444085in}{0.669366in}}%
\pgfpathlineto{\pgfqpoint{2.444813in}{0.669691in}}%
\pgfpathlineto{\pgfqpoint{2.445686in}{0.670225in}}%
\pgfpathlineto{\pgfqpoint{2.446559in}{0.677587in}}%
\pgfpathlineto{\pgfqpoint{2.447578in}{0.702976in}}%
\pgfpathlineto{\pgfqpoint{2.448306in}{0.689274in}}%
\pgfpathlineto{\pgfqpoint{2.448452in}{0.687991in}}%
\pgfpathlineto{\pgfqpoint{2.448888in}{0.693073in}}%
\pgfpathlineto{\pgfqpoint{2.449179in}{0.697505in}}%
\pgfpathlineto{\pgfqpoint{2.449616in}{0.689506in}}%
\pgfpathlineto{\pgfqpoint{2.450780in}{0.669621in}}%
\pgfpathlineto{\pgfqpoint{2.451217in}{0.669637in}}%
\pgfpathlineto{\pgfqpoint{2.452381in}{0.670770in}}%
\pgfpathlineto{\pgfqpoint{2.453255in}{0.690549in}}%
\pgfpathlineto{\pgfqpoint{2.453982in}{0.677772in}}%
\pgfpathlineto{\pgfqpoint{2.454273in}{0.675761in}}%
\pgfpathlineto{\pgfqpoint{2.454856in}{0.680316in}}%
\pgfpathlineto{\pgfqpoint{2.455001in}{0.681021in}}%
\pgfpathlineto{\pgfqpoint{2.455292in}{0.679147in}}%
\pgfpathlineto{\pgfqpoint{2.456311in}{0.668814in}}%
\pgfpathlineto{\pgfqpoint{2.457039in}{0.669107in}}%
\pgfpathlineto{\pgfqpoint{2.461696in}{0.671710in}}%
\pgfpathlineto{\pgfqpoint{2.462278in}{0.677443in}}%
\pgfpathlineto{\pgfqpoint{2.462715in}{0.671925in}}%
\pgfpathlineto{\pgfqpoint{2.463152in}{0.669521in}}%
\pgfpathlineto{\pgfqpoint{2.463588in}{0.673007in}}%
\pgfpathlineto{\pgfqpoint{2.463879in}{0.676072in}}%
\pgfpathlineto{\pgfqpoint{2.464316in}{0.671059in}}%
\pgfpathlineto{\pgfqpoint{2.464753in}{0.669173in}}%
\pgfpathlineto{\pgfqpoint{2.465335in}{0.671068in}}%
\pgfpathlineto{\pgfqpoint{2.465480in}{0.671515in}}%
\pgfpathlineto{\pgfqpoint{2.466063in}{0.669821in}}%
\pgfpathlineto{\pgfqpoint{2.467955in}{0.669847in}}%
\pgfpathlineto{\pgfqpoint{2.469992in}{0.672456in}}%
\pgfpathlineto{\pgfqpoint{2.470429in}{0.694892in}}%
\pgfpathlineto{\pgfqpoint{2.471011in}{0.752578in}}%
\pgfpathlineto{\pgfqpoint{2.471448in}{0.714461in}}%
\pgfpathlineto{\pgfqpoint{2.472030in}{0.679256in}}%
\pgfpathlineto{\pgfqpoint{2.472758in}{0.687625in}}%
\pgfpathlineto{\pgfqpoint{2.474068in}{0.669844in}}%
\pgfpathlineto{\pgfqpoint{2.475086in}{0.669935in}}%
\pgfpathlineto{\pgfqpoint{2.475523in}{0.671298in}}%
\pgfpathlineto{\pgfqpoint{2.476251in}{0.684446in}}%
\pgfpathlineto{\pgfqpoint{2.476833in}{0.674722in}}%
\pgfpathlineto{\pgfqpoint{2.477270in}{0.671888in}}%
\pgfpathlineto{\pgfqpoint{2.478143in}{0.672041in}}%
\pgfpathlineto{\pgfqpoint{2.478871in}{0.670346in}}%
\pgfpathlineto{\pgfqpoint{2.479453in}{0.670636in}}%
\pgfpathlineto{\pgfqpoint{2.480472in}{0.671394in}}%
\pgfpathlineto{\pgfqpoint{2.481199in}{0.676918in}}%
\pgfpathlineto{\pgfqpoint{2.481782in}{0.672727in}}%
\pgfpathlineto{\pgfqpoint{2.483091in}{0.671129in}}%
\pgfpathlineto{\pgfqpoint{2.487749in}{0.671310in}}%
\pgfpathlineto{\pgfqpoint{2.488622in}{0.675393in}}%
\pgfpathlineto{\pgfqpoint{2.489204in}{0.672923in}}%
\pgfpathlineto{\pgfqpoint{2.490660in}{0.670989in}}%
\pgfpathlineto{\pgfqpoint{2.490951in}{0.671547in}}%
\pgfpathlineto{\pgfqpoint{2.492989in}{0.685518in}}%
\pgfpathlineto{\pgfqpoint{2.493280in}{0.682446in}}%
\pgfpathlineto{\pgfqpoint{2.494444in}{0.670781in}}%
\pgfpathlineto{\pgfqpoint{2.494735in}{0.671058in}}%
\pgfpathlineto{\pgfqpoint{2.495317in}{0.674622in}}%
\pgfpathlineto{\pgfqpoint{2.496482in}{0.691217in}}%
\pgfpathlineto{\pgfqpoint{2.497209in}{0.687881in}}%
\pgfpathlineto{\pgfqpoint{2.498083in}{0.683384in}}%
\pgfpathlineto{\pgfqpoint{2.499393in}{0.671043in}}%
\pgfpathlineto{\pgfqpoint{2.499829in}{0.672469in}}%
\pgfpathlineto{\pgfqpoint{2.500411in}{0.693145in}}%
\pgfpathlineto{\pgfqpoint{2.500994in}{0.721842in}}%
\pgfpathlineto{\pgfqpoint{2.501576in}{0.697588in}}%
\pgfpathlineto{\pgfqpoint{2.502158in}{0.690306in}}%
\pgfpathlineto{\pgfqpoint{2.502740in}{0.693570in}}%
\pgfpathlineto{\pgfqpoint{2.503322in}{0.696957in}}%
\pgfpathlineto{\pgfqpoint{2.503904in}{0.694809in}}%
\pgfpathlineto{\pgfqpoint{2.504341in}{0.695275in}}%
\pgfpathlineto{\pgfqpoint{2.504487in}{0.694144in}}%
\pgfpathlineto{\pgfqpoint{2.505942in}{0.667815in}}%
\pgfpathlineto{\pgfqpoint{2.506815in}{0.670402in}}%
\pgfpathlineto{\pgfqpoint{2.507980in}{0.677745in}}%
\pgfpathlineto{\pgfqpoint{2.508562in}{0.675968in}}%
\pgfpathlineto{\pgfqpoint{2.511182in}{0.668074in}}%
\pgfpathlineto{\pgfqpoint{2.512201in}{0.666725in}}%
\pgfpathlineto{\pgfqpoint{2.512928in}{0.665957in}}%
\pgfpathlineto{\pgfqpoint{2.513365in}{0.666299in}}%
\pgfpathlineto{\pgfqpoint{2.514966in}{0.668527in}}%
\pgfpathlineto{\pgfqpoint{2.515839in}{0.688032in}}%
\pgfpathlineto{\pgfqpoint{2.516858in}{0.727985in}}%
\pgfpathlineto{\pgfqpoint{2.517440in}{0.712983in}}%
\pgfpathlineto{\pgfqpoint{2.519915in}{0.678521in}}%
\pgfpathlineto{\pgfqpoint{2.521079in}{0.678695in}}%
\pgfpathlineto{\pgfqpoint{2.521516in}{0.684358in}}%
\pgfpathlineto{\pgfqpoint{2.523117in}{0.764950in}}%
\pgfpathlineto{\pgfqpoint{2.524281in}{0.747478in}}%
\pgfpathlineto{\pgfqpoint{2.527483in}{0.671777in}}%
\pgfpathlineto{\pgfqpoint{2.527628in}{0.672016in}}%
\pgfpathlineto{\pgfqpoint{2.529957in}{0.682340in}}%
\pgfpathlineto{\pgfqpoint{2.530976in}{0.689793in}}%
\pgfpathlineto{\pgfqpoint{2.531558in}{0.688033in}}%
\pgfpathlineto{\pgfqpoint{2.532140in}{0.686746in}}%
\pgfpathlineto{\pgfqpoint{2.532577in}{0.687993in}}%
\pgfpathlineto{\pgfqpoint{2.532868in}{0.688665in}}%
\pgfpathlineto{\pgfqpoint{2.533159in}{0.687331in}}%
\pgfpathlineto{\pgfqpoint{2.534469in}{0.673805in}}%
\pgfpathlineto{\pgfqpoint{2.535197in}{0.676152in}}%
\pgfpathlineto{\pgfqpoint{2.535925in}{0.677973in}}%
\pgfpathlineto{\pgfqpoint{2.536507in}{0.677033in}}%
\pgfpathlineto{\pgfqpoint{2.537380in}{0.672138in}}%
\pgfpathlineto{\pgfqpoint{2.538981in}{0.667361in}}%
\pgfpathlineto{\pgfqpoint{2.539127in}{0.667397in}}%
\pgfpathlineto{\pgfqpoint{2.539709in}{0.670410in}}%
\pgfpathlineto{\pgfqpoint{2.540582in}{0.681526in}}%
\pgfpathlineto{\pgfqpoint{2.541164in}{0.675339in}}%
\pgfpathlineto{\pgfqpoint{2.542911in}{0.662092in}}%
\pgfpathlineto{\pgfqpoint{2.543202in}{0.661459in}}%
\pgfpathlineto{\pgfqpoint{2.543638in}{0.662783in}}%
\pgfpathlineto{\pgfqpoint{2.545822in}{0.687057in}}%
\pgfpathlineto{\pgfqpoint{2.546404in}{0.704990in}}%
\pgfpathlineto{\pgfqpoint{2.546840in}{0.688415in}}%
\pgfpathlineto{\pgfqpoint{2.547859in}{0.661327in}}%
\pgfpathlineto{\pgfqpoint{2.548296in}{0.662428in}}%
\pgfpathlineto{\pgfqpoint{2.549315in}{0.665633in}}%
\pgfpathlineto{\pgfqpoint{2.549751in}{0.664877in}}%
\pgfpathlineto{\pgfqpoint{2.551207in}{0.656766in}}%
\pgfpathlineto{\pgfqpoint{2.552080in}{0.653024in}}%
\pgfpathlineto{\pgfqpoint{2.552517in}{0.653871in}}%
\pgfpathlineto{\pgfqpoint{2.553099in}{0.658550in}}%
\pgfpathlineto{\pgfqpoint{2.554118in}{0.690807in}}%
\pgfpathlineto{\pgfqpoint{2.554700in}{0.673931in}}%
\pgfpathlineto{\pgfqpoint{2.557029in}{0.644265in}}%
\pgfpathlineto{\pgfqpoint{2.557756in}{0.648155in}}%
\pgfpathlineto{\pgfqpoint{2.560231in}{0.657862in}}%
\pgfpathlineto{\pgfqpoint{2.566198in}{0.668120in}}%
\pgfpathlineto{\pgfqpoint{2.568672in}{0.669105in}}%
\pgfpathlineto{\pgfqpoint{2.572020in}{0.668908in}}%
\pgfpathlineto{\pgfqpoint{2.572602in}{0.668782in}}%
\pgfpathlineto{\pgfqpoint{2.572893in}{0.669199in}}%
\pgfpathlineto{\pgfqpoint{2.575659in}{0.674205in}}%
\pgfpathlineto{\pgfqpoint{2.575804in}{0.674137in}}%
\pgfpathlineto{\pgfqpoint{2.576823in}{0.672328in}}%
\pgfpathlineto{\pgfqpoint{2.578278in}{0.668445in}}%
\pgfpathlineto{\pgfqpoint{2.578715in}{0.668592in}}%
\pgfpathlineto{\pgfqpoint{2.583809in}{0.669268in}}%
\pgfpathlineto{\pgfqpoint{2.584537in}{0.668673in}}%
\pgfpathlineto{\pgfqpoint{2.584828in}{0.669334in}}%
\pgfpathlineto{\pgfqpoint{2.585701in}{0.676338in}}%
\pgfpathlineto{\pgfqpoint{2.586138in}{0.671313in}}%
\pgfpathlineto{\pgfqpoint{2.587011in}{0.664128in}}%
\pgfpathlineto{\pgfqpoint{2.587593in}{0.664422in}}%
\pgfpathlineto{\pgfqpoint{2.588467in}{0.665348in}}%
\pgfpathlineto{\pgfqpoint{2.589631in}{0.673163in}}%
\pgfpathlineto{\pgfqpoint{2.591086in}{0.672468in}}%
\pgfpathlineto{\pgfqpoint{2.591814in}{0.670912in}}%
\pgfpathlineto{\pgfqpoint{2.592105in}{0.672889in}}%
\pgfpathlineto{\pgfqpoint{2.592833in}{0.701561in}}%
\pgfpathlineto{\pgfqpoint{2.594579in}{0.760302in}}%
\pgfpathlineto{\pgfqpoint{2.594725in}{0.760010in}}%
\pgfpathlineto{\pgfqpoint{2.596472in}{0.713213in}}%
\pgfpathlineto{\pgfqpoint{2.597781in}{0.672245in}}%
\pgfpathlineto{\pgfqpoint{2.598218in}{0.674502in}}%
\pgfpathlineto{\pgfqpoint{2.599528in}{0.698593in}}%
\pgfpathlineto{\pgfqpoint{2.600838in}{0.745486in}}%
\pgfpathlineto{\pgfqpoint{2.601566in}{0.735565in}}%
\pgfpathlineto{\pgfqpoint{2.604477in}{0.679313in}}%
\pgfpathlineto{\pgfqpoint{2.604768in}{0.681542in}}%
\pgfpathlineto{\pgfqpoint{2.605495in}{0.712684in}}%
\pgfpathlineto{\pgfqpoint{2.606514in}{0.753669in}}%
\pgfpathlineto{\pgfqpoint{2.606951in}{0.748068in}}%
\pgfpathlineto{\pgfqpoint{2.610589in}{0.692163in}}%
\pgfpathlineto{\pgfqpoint{2.612918in}{0.677347in}}%
\pgfpathlineto{\pgfqpoint{2.615247in}{0.669441in}}%
\pgfpathlineto{\pgfqpoint{2.615392in}{0.669495in}}%
\pgfpathlineto{\pgfqpoint{2.616120in}{0.671660in}}%
\pgfpathlineto{\pgfqpoint{2.616848in}{0.673798in}}%
\pgfpathlineto{\pgfqpoint{2.617285in}{0.671972in}}%
\pgfpathlineto{\pgfqpoint{2.619031in}{0.667419in}}%
\pgfpathlineto{\pgfqpoint{2.621069in}{0.666914in}}%
\pgfpathlineto{\pgfqpoint{2.623980in}{0.663806in}}%
\pgfpathlineto{\pgfqpoint{2.624125in}{0.663937in}}%
\pgfpathlineto{\pgfqpoint{2.626308in}{0.666199in}}%
\pgfpathlineto{\pgfqpoint{2.626745in}{0.665708in}}%
\pgfpathlineto{\pgfqpoint{2.628928in}{0.664078in}}%
\pgfpathlineto{\pgfqpoint{2.630820in}{0.666075in}}%
\pgfpathlineto{\pgfqpoint{2.631839in}{0.668915in}}%
\pgfpathlineto{\pgfqpoint{2.632712in}{0.675215in}}%
\pgfpathlineto{\pgfqpoint{2.633440in}{0.672891in}}%
\pgfpathlineto{\pgfqpoint{2.634313in}{0.678215in}}%
\pgfpathlineto{\pgfqpoint{2.634896in}{0.673917in}}%
\pgfpathlineto{\pgfqpoint{2.636497in}{0.667224in}}%
\pgfpathlineto{\pgfqpoint{2.639408in}{0.665973in}}%
\pgfpathlineto{\pgfqpoint{2.642610in}{0.663081in}}%
\pgfpathlineto{\pgfqpoint{2.643483in}{0.664952in}}%
\pgfpathlineto{\pgfqpoint{2.646248in}{0.677725in}}%
\pgfpathlineto{\pgfqpoint{2.646539in}{0.676486in}}%
\pgfpathlineto{\pgfqpoint{2.647995in}{0.665891in}}%
\pgfpathlineto{\pgfqpoint{2.648577in}{0.668100in}}%
\pgfpathlineto{\pgfqpoint{2.649450in}{0.688059in}}%
\pgfpathlineto{\pgfqpoint{2.649887in}{0.695154in}}%
\pgfpathlineto{\pgfqpoint{2.650615in}{0.689456in}}%
\pgfpathlineto{\pgfqpoint{2.651633in}{0.684471in}}%
\pgfpathlineto{\pgfqpoint{2.653234in}{0.667411in}}%
\pgfpathlineto{\pgfqpoint{2.653525in}{0.667626in}}%
\pgfpathlineto{\pgfqpoint{2.654399in}{0.670885in}}%
\pgfpathlineto{\pgfqpoint{2.655272in}{0.685498in}}%
\pgfpathlineto{\pgfqpoint{2.656436in}{0.707864in}}%
\pgfpathlineto{\pgfqpoint{2.657019in}{0.705185in}}%
\pgfpathlineto{\pgfqpoint{2.657746in}{0.699932in}}%
\pgfpathlineto{\pgfqpoint{2.658183in}{0.702595in}}%
\pgfpathlineto{\pgfqpoint{2.659056in}{0.714420in}}%
\pgfpathlineto{\pgfqpoint{2.659493in}{0.709487in}}%
\pgfpathlineto{\pgfqpoint{2.660948in}{0.688538in}}%
\pgfpathlineto{\pgfqpoint{2.661385in}{0.690093in}}%
\pgfpathlineto{\pgfqpoint{2.662113in}{0.693212in}}%
\pgfpathlineto{\pgfqpoint{2.662695in}{0.691162in}}%
\pgfpathlineto{\pgfqpoint{2.664587in}{0.676183in}}%
\pgfpathlineto{\pgfqpoint{2.665024in}{0.680399in}}%
\pgfpathlineto{\pgfqpoint{2.666188in}{0.728458in}}%
\pgfpathlineto{\pgfqpoint{2.667061in}{0.710606in}}%
\pgfpathlineto{\pgfqpoint{2.670409in}{0.663848in}}%
\pgfpathlineto{\pgfqpoint{2.670554in}{0.664147in}}%
\pgfpathlineto{\pgfqpoint{2.672010in}{0.671890in}}%
\pgfpathlineto{\pgfqpoint{2.672883in}{0.669300in}}%
\pgfpathlineto{\pgfqpoint{2.674630in}{0.659423in}}%
\pgfpathlineto{\pgfqpoint{2.675357in}{0.655967in}}%
\pgfpathlineto{\pgfqpoint{2.675940in}{0.657020in}}%
\pgfpathlineto{\pgfqpoint{2.676958in}{0.662037in}}%
\pgfpathlineto{\pgfqpoint{2.677686in}{0.659436in}}%
\pgfpathlineto{\pgfqpoint{2.678268in}{0.658126in}}%
\pgfpathlineto{\pgfqpoint{2.678850in}{0.658674in}}%
\pgfpathlineto{\pgfqpoint{2.685837in}{0.665362in}}%
\pgfpathlineto{\pgfqpoint{2.691658in}{0.670179in}}%
\pgfpathlineto{\pgfqpoint{2.693551in}{0.670454in}}%
\pgfpathlineto{\pgfqpoint{2.694715in}{0.669900in}}%
\pgfpathlineto{\pgfqpoint{2.695006in}{0.670436in}}%
\pgfpathlineto{\pgfqpoint{2.695879in}{0.677701in}}%
\pgfpathlineto{\pgfqpoint{2.698208in}{0.705444in}}%
\pgfpathlineto{\pgfqpoint{2.698790in}{0.703001in}}%
\pgfpathlineto{\pgfqpoint{2.699955in}{0.677297in}}%
\pgfpathlineto{\pgfqpoint{2.700682in}{0.670065in}}%
\pgfpathlineto{\pgfqpoint{2.701119in}{0.672014in}}%
\pgfpathlineto{\pgfqpoint{2.704175in}{0.706265in}}%
\pgfpathlineto{\pgfqpoint{2.704758in}{0.702781in}}%
\pgfpathlineto{\pgfqpoint{2.705485in}{0.697864in}}%
\pgfpathlineto{\pgfqpoint{2.705922in}{0.702148in}}%
\pgfpathlineto{\pgfqpoint{2.706941in}{0.719788in}}%
\pgfpathlineto{\pgfqpoint{2.707377in}{0.712906in}}%
\pgfpathlineto{\pgfqpoint{2.709415in}{0.688497in}}%
\pgfpathlineto{\pgfqpoint{2.710725in}{0.674512in}}%
\pgfpathlineto{\pgfqpoint{2.713054in}{0.659948in}}%
\pgfpathlineto{\pgfqpoint{2.713490in}{0.659194in}}%
\pgfpathlineto{\pgfqpoint{2.714072in}{0.659979in}}%
\pgfpathlineto{\pgfqpoint{2.715091in}{0.664065in}}%
\pgfpathlineto{\pgfqpoint{2.716110in}{0.670623in}}%
\pgfpathlineto{\pgfqpoint{2.716838in}{0.669554in}}%
\pgfpathlineto{\pgfqpoint{2.717711in}{0.665625in}}%
\pgfpathlineto{\pgfqpoint{2.719021in}{0.658937in}}%
\pgfpathlineto{\pgfqpoint{2.719458in}{0.659809in}}%
\pgfpathlineto{\pgfqpoint{2.721495in}{0.672301in}}%
\pgfpathlineto{\pgfqpoint{2.722660in}{0.669772in}}%
\pgfpathlineto{\pgfqpoint{2.723533in}{0.665941in}}%
\pgfpathlineto{\pgfqpoint{2.725862in}{0.658365in}}%
\pgfpathlineto{\pgfqpoint{2.726298in}{0.658222in}}%
\pgfpathlineto{\pgfqpoint{2.726735in}{0.658968in}}%
\pgfpathlineto{\pgfqpoint{2.730665in}{0.666192in}}%
\pgfpathlineto{\pgfqpoint{2.730956in}{0.665852in}}%
\pgfpathlineto{\pgfqpoint{2.732411in}{0.664052in}}%
\pgfpathlineto{\pgfqpoint{2.732848in}{0.664392in}}%
\pgfpathlineto{\pgfqpoint{2.737360in}{0.667239in}}%
\pgfpathlineto{\pgfqpoint{2.738088in}{0.666354in}}%
\pgfpathlineto{\pgfqpoint{2.739397in}{0.665294in}}%
\pgfpathlineto{\pgfqpoint{2.739543in}{0.665357in}}%
\pgfpathlineto{\pgfqpoint{2.741144in}{0.667339in}}%
\pgfpathlineto{\pgfqpoint{2.741872in}{0.668262in}}%
\pgfpathlineto{\pgfqpoint{2.742454in}{0.667718in}}%
\pgfpathlineto{\pgfqpoint{2.744346in}{0.668294in}}%
\pgfpathlineto{\pgfqpoint{2.746675in}{0.674081in}}%
\pgfpathlineto{\pgfqpoint{2.746820in}{0.674297in}}%
\pgfpathlineto{\pgfqpoint{2.747111in}{0.673645in}}%
\pgfpathlineto{\pgfqpoint{2.748276in}{0.669631in}}%
\pgfpathlineto{\pgfqpoint{2.748712in}{0.670161in}}%
\pgfpathlineto{\pgfqpoint{2.751041in}{0.675488in}}%
\pgfpathlineto{\pgfqpoint{2.751332in}{0.675066in}}%
\pgfpathlineto{\pgfqpoint{2.753079in}{0.671626in}}%
\pgfpathlineto{\pgfqpoint{2.753224in}{0.671830in}}%
\pgfpathlineto{\pgfqpoint{2.753806in}{0.678588in}}%
\pgfpathlineto{\pgfqpoint{2.754971in}{0.719795in}}%
\pgfpathlineto{\pgfqpoint{2.755699in}{0.704291in}}%
\pgfpathlineto{\pgfqpoint{2.757445in}{0.685248in}}%
\pgfpathlineto{\pgfqpoint{2.759774in}{0.664785in}}%
\pgfpathlineto{\pgfqpoint{2.760647in}{0.665691in}}%
\pgfpathlineto{\pgfqpoint{2.763121in}{0.667240in}}%
\pgfpathlineto{\pgfqpoint{2.765305in}{0.667581in}}%
\pgfpathlineto{\pgfqpoint{2.769234in}{0.670293in}}%
\pgfpathlineto{\pgfqpoint{2.769671in}{0.669267in}}%
\pgfpathlineto{\pgfqpoint{2.770253in}{0.668602in}}%
\pgfpathlineto{\pgfqpoint{2.770544in}{0.669279in}}%
\pgfpathlineto{\pgfqpoint{2.771272in}{0.678162in}}%
\pgfpathlineto{\pgfqpoint{2.771854in}{0.684403in}}%
\pgfpathlineto{\pgfqpoint{2.772436in}{0.679575in}}%
\pgfpathlineto{\pgfqpoint{2.774765in}{0.668005in}}%
\pgfpathlineto{\pgfqpoint{2.775202in}{0.668486in}}%
\pgfpathlineto{\pgfqpoint{2.775784in}{0.669765in}}%
\pgfpathlineto{\pgfqpoint{2.776220in}{0.668708in}}%
\pgfpathlineto{\pgfqpoint{2.776948in}{0.667589in}}%
\pgfpathlineto{\pgfqpoint{2.777530in}{0.667920in}}%
\pgfpathlineto{\pgfqpoint{2.778549in}{0.669122in}}%
\pgfpathlineto{\pgfqpoint{2.781023in}{0.677295in}}%
\pgfpathlineto{\pgfqpoint{2.783789in}{0.669611in}}%
\pgfpathlineto{\pgfqpoint{2.792958in}{0.670299in}}%
\pgfpathlineto{\pgfqpoint{2.793832in}{0.672497in}}%
\pgfpathlineto{\pgfqpoint{2.794414in}{0.671221in}}%
\pgfpathlineto{\pgfqpoint{2.796015in}{0.670722in}}%
\pgfpathlineto{\pgfqpoint{2.797470in}{0.671623in}}%
\pgfpathlineto{\pgfqpoint{2.799362in}{0.674081in}}%
\pgfpathlineto{\pgfqpoint{2.800090in}{0.674167in}}%
\pgfpathlineto{\pgfqpoint{2.800236in}{0.674895in}}%
\pgfpathlineto{\pgfqpoint{2.800672in}{0.687238in}}%
\pgfpathlineto{\pgfqpoint{2.801691in}{0.832452in}}%
\pgfpathlineto{\pgfqpoint{2.802419in}{0.760512in}}%
\pgfpathlineto{\pgfqpoint{2.804165in}{0.705776in}}%
\pgfpathlineto{\pgfqpoint{2.806203in}{0.672066in}}%
\pgfpathlineto{\pgfqpoint{2.806785in}{0.673354in}}%
\pgfpathlineto{\pgfqpoint{2.807367in}{0.691328in}}%
\pgfpathlineto{\pgfqpoint{2.807949in}{0.718250in}}%
\pgfpathlineto{\pgfqpoint{2.808532in}{0.701645in}}%
\pgfpathlineto{\pgfqpoint{2.810278in}{0.680735in}}%
\pgfpathlineto{\pgfqpoint{2.811879in}{0.673216in}}%
\pgfpathlineto{\pgfqpoint{2.812316in}{0.673826in}}%
\pgfpathlineto{\pgfqpoint{2.812898in}{0.682751in}}%
\pgfpathlineto{\pgfqpoint{2.813480in}{0.694606in}}%
\pgfpathlineto{\pgfqpoint{2.814062in}{0.686260in}}%
\pgfpathlineto{\pgfqpoint{2.815663in}{0.677384in}}%
\pgfpathlineto{\pgfqpoint{2.816682in}{0.673073in}}%
\pgfpathlineto{\pgfqpoint{2.818283in}{0.670392in}}%
\pgfpathlineto{\pgfqpoint{2.822504in}{0.669603in}}%
\pgfpathlineto{\pgfqpoint{2.824251in}{0.669938in}}%
\pgfpathlineto{\pgfqpoint{2.824542in}{0.670216in}}%
\pgfpathlineto{\pgfqpoint{2.824833in}{0.669547in}}%
\pgfpathlineto{\pgfqpoint{2.825852in}{0.666647in}}%
\pgfpathlineto{\pgfqpoint{2.826434in}{0.667012in}}%
\pgfpathlineto{\pgfqpoint{2.828326in}{0.669008in}}%
\pgfpathlineto{\pgfqpoint{2.828908in}{0.671309in}}%
\pgfpathlineto{\pgfqpoint{2.829345in}{0.669541in}}%
\pgfpathlineto{\pgfqpoint{2.830800in}{0.668001in}}%
\pgfpathlineto{\pgfqpoint{2.831673in}{0.668176in}}%
\pgfpathlineto{\pgfqpoint{2.831819in}{0.668421in}}%
\pgfpathlineto{\pgfqpoint{2.833857in}{0.671388in}}%
\pgfpathlineto{\pgfqpoint{2.834148in}{0.670531in}}%
\pgfpathlineto{\pgfqpoint{2.834875in}{0.669050in}}%
\pgfpathlineto{\pgfqpoint{2.835312in}{0.670085in}}%
\pgfpathlineto{\pgfqpoint{2.837786in}{0.683915in}}%
\pgfpathlineto{\pgfqpoint{2.839387in}{0.692821in}}%
\pgfpathlineto{\pgfqpoint{2.839824in}{0.693427in}}%
\pgfpathlineto{\pgfqpoint{2.840115in}{0.692105in}}%
\pgfpathlineto{\pgfqpoint{2.841716in}{0.668631in}}%
\pgfpathlineto{\pgfqpoint{2.842589in}{0.674889in}}%
\pgfpathlineto{\pgfqpoint{2.844627in}{0.709333in}}%
\pgfpathlineto{\pgfqpoint{2.845355in}{0.703468in}}%
\pgfpathlineto{\pgfqpoint{2.847392in}{0.661723in}}%
\pgfpathlineto{\pgfqpoint{2.848120in}{0.674026in}}%
\pgfpathlineto{\pgfqpoint{2.849867in}{0.779388in}}%
\pgfpathlineto{\pgfqpoint{2.850594in}{0.750544in}}%
\pgfpathlineto{\pgfqpoint{2.853214in}{0.676469in}}%
\pgfpathlineto{\pgfqpoint{2.853505in}{0.675589in}}%
\pgfpathlineto{\pgfqpoint{2.853942in}{0.678321in}}%
\pgfpathlineto{\pgfqpoint{2.856125in}{0.695219in}}%
\pgfpathlineto{\pgfqpoint{2.856416in}{0.694401in}}%
\pgfpathlineto{\pgfqpoint{2.856853in}{0.693313in}}%
\pgfpathlineto{\pgfqpoint{2.857144in}{0.695121in}}%
\pgfpathlineto{\pgfqpoint{2.857581in}{0.698144in}}%
\pgfpathlineto{\pgfqpoint{2.857872in}{0.694759in}}%
\pgfpathlineto{\pgfqpoint{2.858890in}{0.677928in}}%
\pgfpathlineto{\pgfqpoint{2.859618in}{0.678619in}}%
\pgfpathlineto{\pgfqpoint{2.861947in}{0.664602in}}%
\pgfpathlineto{\pgfqpoint{2.862238in}{0.665129in}}%
\pgfpathlineto{\pgfqpoint{2.862966in}{0.664635in}}%
\pgfpathlineto{\pgfqpoint{2.863402in}{0.667157in}}%
\pgfpathlineto{\pgfqpoint{2.864130in}{0.706947in}}%
\pgfpathlineto{\pgfqpoint{2.864567in}{0.734000in}}%
\pgfpathlineto{\pgfqpoint{2.865149in}{0.699429in}}%
\pgfpathlineto{\pgfqpoint{2.866459in}{0.661170in}}%
\pgfpathlineto{\pgfqpoint{2.866604in}{0.661175in}}%
\pgfpathlineto{\pgfqpoint{2.867041in}{0.662559in}}%
\pgfpathlineto{\pgfqpoint{2.868205in}{0.672164in}}%
\pgfpathlineto{\pgfqpoint{2.869079in}{0.670835in}}%
\pgfpathlineto{\pgfqpoint{2.871698in}{0.679695in}}%
\pgfpathlineto{\pgfqpoint{2.871990in}{0.680691in}}%
\pgfpathlineto{\pgfqpoint{2.872572in}{0.678629in}}%
\pgfpathlineto{\pgfqpoint{2.873882in}{0.669915in}}%
\pgfpathlineto{\pgfqpoint{2.875483in}{0.651292in}}%
\pgfpathlineto{\pgfqpoint{2.875919in}{0.653193in}}%
\pgfpathlineto{\pgfqpoint{2.876647in}{0.676351in}}%
\pgfpathlineto{\pgfqpoint{2.877084in}{0.690967in}}%
\pgfpathlineto{\pgfqpoint{2.877520in}{0.674496in}}%
\pgfpathlineto{\pgfqpoint{2.879121in}{0.643574in}}%
\pgfpathlineto{\pgfqpoint{2.879703in}{0.641026in}}%
\pgfpathlineto{\pgfqpoint{2.880431in}{0.641758in}}%
\pgfpathlineto{\pgfqpoint{2.881887in}{0.638298in}}%
\pgfpathlineto{\pgfqpoint{2.882469in}{0.640098in}}%
\pgfpathlineto{\pgfqpoint{2.884361in}{0.657745in}}%
\pgfpathlineto{\pgfqpoint{2.884943in}{0.662912in}}%
\pgfpathlineto{\pgfqpoint{2.885525in}{0.659733in}}%
\pgfpathlineto{\pgfqpoint{2.887417in}{0.649776in}}%
\pgfpathlineto{\pgfqpoint{2.888145in}{0.652897in}}%
\pgfpathlineto{\pgfqpoint{2.889746in}{0.665966in}}%
\pgfpathlineto{\pgfqpoint{2.890474in}{0.682717in}}%
\pgfpathlineto{\pgfqpoint{2.891056in}{0.675161in}}%
\pgfpathlineto{\pgfqpoint{2.892657in}{0.665003in}}%
\pgfpathlineto{\pgfqpoint{2.893967in}{0.657725in}}%
\pgfpathlineto{\pgfqpoint{2.894986in}{0.659213in}}%
\pgfpathlineto{\pgfqpoint{2.895568in}{0.668762in}}%
\pgfpathlineto{\pgfqpoint{2.896150in}{0.681715in}}%
\pgfpathlineto{\pgfqpoint{2.896587in}{0.673491in}}%
\pgfpathlineto{\pgfqpoint{2.897606in}{0.661375in}}%
\pgfpathlineto{\pgfqpoint{2.898042in}{0.661900in}}%
\pgfpathlineto{\pgfqpoint{2.899207in}{0.665931in}}%
\pgfpathlineto{\pgfqpoint{2.900808in}{0.678912in}}%
\pgfpathlineto{\pgfqpoint{2.901244in}{0.692814in}}%
\pgfpathlineto{\pgfqpoint{2.901681in}{0.681315in}}%
\pgfpathlineto{\pgfqpoint{2.902700in}{0.659223in}}%
\pgfpathlineto{\pgfqpoint{2.903136in}{0.660411in}}%
\pgfpathlineto{\pgfqpoint{2.904592in}{0.666024in}}%
\pgfpathlineto{\pgfqpoint{2.905174in}{0.665707in}}%
\pgfpathlineto{\pgfqpoint{2.905756in}{0.666243in}}%
\pgfpathlineto{\pgfqpoint{2.906338in}{0.668484in}}%
\pgfpathlineto{\pgfqpoint{2.906629in}{0.666103in}}%
\pgfpathlineto{\pgfqpoint{2.907503in}{0.653180in}}%
\pgfpathlineto{\pgfqpoint{2.908230in}{0.655724in}}%
\pgfpathlineto{\pgfqpoint{2.911141in}{0.663261in}}%
\pgfpathlineto{\pgfqpoint{2.913616in}{0.665862in}}%
\pgfpathlineto{\pgfqpoint{2.914343in}{0.669269in}}%
\pgfpathlineto{\pgfqpoint{2.914780in}{0.666798in}}%
\pgfpathlineto{\pgfqpoint{2.915362in}{0.665267in}}%
\pgfpathlineto{\pgfqpoint{2.916090in}{0.665663in}}%
\pgfpathlineto{\pgfqpoint{2.917545in}{0.666809in}}%
\pgfpathlineto{\pgfqpoint{2.919583in}{0.668980in}}%
\pgfpathlineto{\pgfqpoint{2.919729in}{0.668863in}}%
\pgfpathlineto{\pgfqpoint{2.920602in}{0.667204in}}%
\pgfpathlineto{\pgfqpoint{2.921038in}{0.668149in}}%
\pgfpathlineto{\pgfqpoint{2.921475in}{0.668793in}}%
\pgfpathlineto{\pgfqpoint{2.921912in}{0.667491in}}%
\pgfpathlineto{\pgfqpoint{2.923367in}{0.666805in}}%
\pgfpathlineto{\pgfqpoint{2.923804in}{0.666712in}}%
\pgfpathlineto{\pgfqpoint{2.924095in}{0.667130in}}%
\pgfpathlineto{\pgfqpoint{2.927151in}{0.674396in}}%
\pgfpathlineto{\pgfqpoint{2.927442in}{0.673846in}}%
\pgfpathlineto{\pgfqpoint{2.928461in}{0.670367in}}%
\pgfpathlineto{\pgfqpoint{2.928898in}{0.672059in}}%
\pgfpathlineto{\pgfqpoint{2.931081in}{0.682256in}}%
\pgfpathlineto{\pgfqpoint{2.931227in}{0.681785in}}%
\pgfpathlineto{\pgfqpoint{2.933846in}{0.662594in}}%
\pgfpathlineto{\pgfqpoint{2.934283in}{0.664371in}}%
\pgfpathlineto{\pgfqpoint{2.934720in}{0.667020in}}%
\pgfpathlineto{\pgfqpoint{2.935156in}{0.664604in}}%
\pgfpathlineto{\pgfqpoint{2.935739in}{0.660199in}}%
\pgfpathlineto{\pgfqpoint{2.936030in}{0.665644in}}%
\pgfpathlineto{\pgfqpoint{2.936612in}{0.754753in}}%
\pgfpathlineto{\pgfqpoint{2.937194in}{0.972032in}}%
\pgfpathlineto{\pgfqpoint{2.937776in}{0.862204in}}%
\pgfpathlineto{\pgfqpoint{2.938649in}{0.754567in}}%
\pgfpathlineto{\pgfqpoint{2.939232in}{0.759068in}}%
\pgfpathlineto{\pgfqpoint{2.939668in}{0.746568in}}%
\pgfpathlineto{\pgfqpoint{2.941560in}{0.673910in}}%
\pgfpathlineto{\pgfqpoint{2.941997in}{0.674764in}}%
\pgfpathlineto{\pgfqpoint{2.944035in}{0.681250in}}%
\pgfpathlineto{\pgfqpoint{2.944471in}{0.680660in}}%
\pgfpathlineto{\pgfqpoint{2.944908in}{0.679805in}}%
\pgfpathlineto{\pgfqpoint{2.945199in}{0.681305in}}%
\pgfpathlineto{\pgfqpoint{2.946218in}{0.703648in}}%
\pgfpathlineto{\pgfqpoint{2.946800in}{0.691280in}}%
\pgfpathlineto{\pgfqpoint{2.947237in}{0.686425in}}%
\pgfpathlineto{\pgfqpoint{2.947819in}{0.690552in}}%
\pgfpathlineto{\pgfqpoint{2.948838in}{0.700671in}}%
\pgfpathlineto{\pgfqpoint{2.949420in}{0.698404in}}%
\pgfpathlineto{\pgfqpoint{2.951021in}{0.680805in}}%
\pgfpathlineto{\pgfqpoint{2.952185in}{0.666549in}}%
\pgfpathlineto{\pgfqpoint{2.952622in}{0.668765in}}%
\pgfpathlineto{\pgfqpoint{2.953350in}{0.692789in}}%
\pgfpathlineto{\pgfqpoint{2.954951in}{0.723116in}}%
\pgfpathlineto{\pgfqpoint{2.955242in}{0.723639in}}%
\pgfpathlineto{\pgfqpoint{2.955533in}{0.721823in}}%
\pgfpathlineto{\pgfqpoint{2.956261in}{0.714061in}}%
\pgfpathlineto{\pgfqpoint{2.956552in}{0.719262in}}%
\pgfpathlineto{\pgfqpoint{2.957279in}{0.740471in}}%
\pgfpathlineto{\pgfqpoint{2.957716in}{0.721710in}}%
\pgfpathlineto{\pgfqpoint{2.959171in}{0.672656in}}%
\pgfpathlineto{\pgfqpoint{2.959463in}{0.673189in}}%
\pgfpathlineto{\pgfqpoint{2.960190in}{0.680938in}}%
\pgfpathlineto{\pgfqpoint{2.962373in}{0.706248in}}%
\pgfpathlineto{\pgfqpoint{2.962519in}{0.705532in}}%
\pgfpathlineto{\pgfqpoint{2.964557in}{0.658352in}}%
\pgfpathlineto{\pgfqpoint{2.965430in}{0.659304in}}%
\pgfpathlineto{\pgfqpoint{2.968777in}{0.667839in}}%
\pgfpathlineto{\pgfqpoint{2.969069in}{0.666498in}}%
\pgfpathlineto{\pgfqpoint{2.969942in}{0.661088in}}%
\pgfpathlineto{\pgfqpoint{2.970524in}{0.662029in}}%
\pgfpathlineto{\pgfqpoint{2.974599in}{0.666674in}}%
\pgfpathlineto{\pgfqpoint{2.974745in}{0.666536in}}%
\pgfpathlineto{\pgfqpoint{2.975327in}{0.665383in}}%
\pgfpathlineto{\pgfqpoint{2.976055in}{0.666101in}}%
\pgfpathlineto{\pgfqpoint{2.980567in}{0.669348in}}%
\pgfpathlineto{\pgfqpoint{2.987553in}{0.671284in}}%
\pgfpathlineto{\pgfqpoint{2.989299in}{0.673492in}}%
\pgfpathlineto{\pgfqpoint{2.989736in}{0.674802in}}%
\pgfpathlineto{\pgfqpoint{2.990173in}{0.672796in}}%
\pgfpathlineto{\pgfqpoint{2.990900in}{0.671681in}}%
\pgfpathlineto{\pgfqpoint{2.991337in}{0.672089in}}%
\pgfpathlineto{\pgfqpoint{2.991919in}{0.676618in}}%
\pgfpathlineto{\pgfqpoint{2.993084in}{0.688883in}}%
\pgfpathlineto{\pgfqpoint{2.993666in}{0.687971in}}%
\pgfpathlineto{\pgfqpoint{2.994976in}{0.682472in}}%
\pgfpathlineto{\pgfqpoint{2.995412in}{0.684473in}}%
\pgfpathlineto{\pgfqpoint{2.996286in}{0.705572in}}%
\pgfpathlineto{\pgfqpoint{2.997741in}{0.741444in}}%
\pgfpathlineto{\pgfqpoint{2.998032in}{0.739592in}}%
\pgfpathlineto{\pgfqpoint{3.000943in}{0.693543in}}%
\pgfpathlineto{\pgfqpoint{3.001816in}{0.702700in}}%
\pgfpathlineto{\pgfqpoint{3.002981in}{0.740234in}}%
\pgfpathlineto{\pgfqpoint{3.003563in}{0.723655in}}%
\pgfpathlineto{\pgfqpoint{3.005309in}{0.688361in}}%
\pgfpathlineto{\pgfqpoint{3.006765in}{0.680798in}}%
\pgfpathlineto{\pgfqpoint{3.006910in}{0.680925in}}%
\pgfpathlineto{\pgfqpoint{3.007493in}{0.685951in}}%
\pgfpathlineto{\pgfqpoint{3.008366in}{0.696134in}}%
\pgfpathlineto{\pgfqpoint{3.008803in}{0.690975in}}%
\pgfpathlineto{\pgfqpoint{3.010695in}{0.675851in}}%
\pgfpathlineto{\pgfqpoint{3.011859in}{0.674601in}}%
\pgfpathlineto{\pgfqpoint{3.013897in}{0.666057in}}%
\pgfpathlineto{\pgfqpoint{3.015352in}{0.662924in}}%
\pgfpathlineto{\pgfqpoint{3.016662in}{0.663978in}}%
\pgfpathlineto{\pgfqpoint{3.017535in}{0.669654in}}%
\pgfpathlineto{\pgfqpoint{3.018263in}{0.673526in}}%
\pgfpathlineto{\pgfqpoint{3.018845in}{0.672673in}}%
\pgfpathlineto{\pgfqpoint{3.020010in}{0.668442in}}%
\pgfpathlineto{\pgfqpoint{3.022193in}{0.660151in}}%
\pgfpathlineto{\pgfqpoint{3.022338in}{0.660183in}}%
\pgfpathlineto{\pgfqpoint{3.023212in}{0.661382in}}%
\pgfpathlineto{\pgfqpoint{3.024667in}{0.668674in}}%
\pgfpathlineto{\pgfqpoint{3.025831in}{0.666948in}}%
\pgfpathlineto{\pgfqpoint{3.026996in}{0.664303in}}%
\pgfpathlineto{\pgfqpoint{3.027432in}{0.665410in}}%
\pgfpathlineto{\pgfqpoint{3.029907in}{0.677884in}}%
\pgfpathlineto{\pgfqpoint{3.030634in}{0.677017in}}%
\pgfpathlineto{\pgfqpoint{3.031799in}{0.673158in}}%
\pgfpathlineto{\pgfqpoint{3.033400in}{0.660372in}}%
\pgfpathlineto{\pgfqpoint{3.033982in}{0.661862in}}%
\pgfpathlineto{\pgfqpoint{3.036165in}{0.673223in}}%
\pgfpathlineto{\pgfqpoint{3.036456in}{0.673034in}}%
\pgfpathlineto{\pgfqpoint{3.037329in}{0.670830in}}%
\pgfpathlineto{\pgfqpoint{3.038930in}{0.660099in}}%
\pgfpathlineto{\pgfqpoint{3.039367in}{0.662496in}}%
\pgfpathlineto{\pgfqpoint{3.040823in}{0.693429in}}%
\pgfpathlineto{\pgfqpoint{3.041841in}{0.685904in}}%
\pgfpathlineto{\pgfqpoint{3.041987in}{0.685764in}}%
\pgfpathlineto{\pgfqpoint{3.042132in}{0.686209in}}%
\pgfpathlineto{\pgfqpoint{3.043006in}{0.691866in}}%
\pgfpathlineto{\pgfqpoint{3.043588in}{0.689262in}}%
\pgfpathlineto{\pgfqpoint{3.043733in}{0.688907in}}%
\pgfpathlineto{\pgfqpoint{3.044025in}{0.690442in}}%
\pgfpathlineto{\pgfqpoint{3.044607in}{0.697068in}}%
\pgfpathlineto{\pgfqpoint{3.044898in}{0.691458in}}%
\pgfpathlineto{\pgfqpoint{3.046353in}{0.643410in}}%
\pgfpathlineto{\pgfqpoint{3.046935in}{0.644727in}}%
\pgfpathlineto{\pgfqpoint{3.052612in}{0.664985in}}%
\pgfpathlineto{\pgfqpoint{3.053194in}{0.661085in}}%
\pgfpathlineto{\pgfqpoint{3.053485in}{0.660474in}}%
\pgfpathlineto{\pgfqpoint{3.053922in}{0.661636in}}%
\pgfpathlineto{\pgfqpoint{3.054504in}{0.664143in}}%
\pgfpathlineto{\pgfqpoint{3.055086in}{0.662591in}}%
\pgfpathlineto{\pgfqpoint{3.055523in}{0.662106in}}%
\pgfpathlineto{\pgfqpoint{3.056105in}{0.663032in}}%
\pgfpathlineto{\pgfqpoint{3.059743in}{0.667215in}}%
\pgfpathlineto{\pgfqpoint{3.065129in}{0.671446in}}%
\pgfpathlineto{\pgfqpoint{3.065565in}{0.673938in}}%
\pgfpathlineto{\pgfqpoint{3.066148in}{0.671538in}}%
\pgfpathlineto{\pgfqpoint{3.067457in}{0.670999in}}%
\pgfpathlineto{\pgfqpoint{3.069204in}{0.671182in}}%
\pgfpathlineto{\pgfqpoint{3.074735in}{0.671794in}}%
\pgfpathlineto{\pgfqpoint{3.075754in}{0.670055in}}%
\pgfpathlineto{\pgfqpoint{3.076190in}{0.670241in}}%
\pgfpathlineto{\pgfqpoint{3.078228in}{0.671204in}}%
\pgfpathlineto{\pgfqpoint{3.078810in}{0.685120in}}%
\pgfpathlineto{\pgfqpoint{3.079101in}{0.689202in}}%
\pgfpathlineto{\pgfqpoint{3.079538in}{0.678361in}}%
\pgfpathlineto{\pgfqpoint{3.079829in}{0.674607in}}%
\pgfpathlineto{\pgfqpoint{3.080120in}{0.679333in}}%
\pgfpathlineto{\pgfqpoint{3.080993in}{0.775895in}}%
\pgfpathlineto{\pgfqpoint{3.081721in}{0.712586in}}%
\pgfpathlineto{\pgfqpoint{3.082303in}{0.682885in}}%
\pgfpathlineto{\pgfqpoint{3.082885in}{0.696484in}}%
\pgfpathlineto{\pgfqpoint{3.083031in}{0.699064in}}%
\pgfpathlineto{\pgfqpoint{3.083322in}{0.692085in}}%
\pgfpathlineto{\pgfqpoint{3.084050in}{0.671402in}}%
\pgfpathlineto{\pgfqpoint{3.084632in}{0.673700in}}%
\pgfpathlineto{\pgfqpoint{3.085505in}{0.672171in}}%
\pgfpathlineto{\pgfqpoint{3.085942in}{0.686232in}}%
\pgfpathlineto{\pgfqpoint{3.086669in}{0.765531in}}%
\pgfpathlineto{\pgfqpoint{3.087252in}{0.712242in}}%
\pgfpathlineto{\pgfqpoint{3.087979in}{0.685147in}}%
\pgfpathlineto{\pgfqpoint{3.088562in}{0.692699in}}%
\pgfpathlineto{\pgfqpoint{3.088707in}{0.693181in}}%
\pgfpathlineto{\pgfqpoint{3.088853in}{0.691715in}}%
\pgfpathlineto{\pgfqpoint{3.090163in}{0.670663in}}%
\pgfpathlineto{\pgfqpoint{3.090745in}{0.670696in}}%
\pgfpathlineto{\pgfqpoint{3.091327in}{0.671124in}}%
\pgfpathlineto{\pgfqpoint{3.091764in}{0.679041in}}%
\pgfpathlineto{\pgfqpoint{3.092346in}{0.697595in}}%
\pgfpathlineto{\pgfqpoint{3.092928in}{0.688241in}}%
\pgfpathlineto{\pgfqpoint{3.094529in}{0.678133in}}%
\pgfpathlineto{\pgfqpoint{3.095402in}{0.672772in}}%
\pgfpathlineto{\pgfqpoint{3.095839in}{0.674776in}}%
\pgfpathlineto{\pgfqpoint{3.096275in}{0.695064in}}%
\pgfpathlineto{\pgfqpoint{3.097149in}{0.814698in}}%
\pgfpathlineto{\pgfqpoint{3.097585in}{0.755471in}}%
\pgfpathlineto{\pgfqpoint{3.098168in}{0.706105in}}%
\pgfpathlineto{\pgfqpoint{3.098750in}{0.731866in}}%
\pgfpathlineto{\pgfqpoint{3.098895in}{0.737027in}}%
\pgfpathlineto{\pgfqpoint{3.099332in}{0.718263in}}%
\pgfpathlineto{\pgfqpoint{3.100787in}{0.670694in}}%
\pgfpathlineto{\pgfqpoint{3.100933in}{0.670702in}}%
\pgfpathlineto{\pgfqpoint{3.101952in}{0.671521in}}%
\pgfpathlineto{\pgfqpoint{3.105154in}{0.694210in}}%
\pgfpathlineto{\pgfqpoint{3.105590in}{0.687474in}}%
\pgfpathlineto{\pgfqpoint{3.107191in}{0.669183in}}%
\pgfpathlineto{\pgfqpoint{3.108647in}{0.670071in}}%
\pgfpathlineto{\pgfqpoint{3.110393in}{0.675686in}}%
\pgfpathlineto{\pgfqpoint{3.111121in}{0.694780in}}%
\pgfpathlineto{\pgfqpoint{3.111558in}{0.684911in}}%
\pgfpathlineto{\pgfqpoint{3.113013in}{0.668745in}}%
\pgfpathlineto{\pgfqpoint{3.114032in}{0.669658in}}%
\pgfpathlineto{\pgfqpoint{3.116652in}{0.682363in}}%
\pgfpathlineto{\pgfqpoint{3.117089in}{0.687998in}}%
\pgfpathlineto{\pgfqpoint{3.117671in}{0.680897in}}%
\pgfpathlineto{\pgfqpoint{3.119272in}{0.669883in}}%
\pgfpathlineto{\pgfqpoint{3.120436in}{0.670677in}}%
\pgfpathlineto{\pgfqpoint{3.122328in}{0.681549in}}%
\pgfpathlineto{\pgfqpoint{3.122910in}{0.699184in}}%
\pgfpathlineto{\pgfqpoint{3.123493in}{0.685773in}}%
\pgfpathlineto{\pgfqpoint{3.124511in}{0.670850in}}%
\pgfpathlineto{\pgfqpoint{3.124948in}{0.671504in}}%
\pgfpathlineto{\pgfqpoint{3.127568in}{0.678877in}}%
\pgfpathlineto{\pgfqpoint{3.128150in}{0.684382in}}%
\pgfpathlineto{\pgfqpoint{3.128441in}{0.679788in}}%
\pgfpathlineto{\pgfqpoint{3.130188in}{0.661899in}}%
\pgfpathlineto{\pgfqpoint{3.131789in}{0.662623in}}%
\pgfpathlineto{\pgfqpoint{3.133535in}{0.663670in}}%
\pgfpathlineto{\pgfqpoint{3.133972in}{0.663109in}}%
\pgfpathlineto{\pgfqpoint{3.134700in}{0.662745in}}%
\pgfpathlineto{\pgfqpoint{3.135136in}{0.663368in}}%
\pgfpathlineto{\pgfqpoint{3.136301in}{0.666753in}}%
\pgfpathlineto{\pgfqpoint{3.137028in}{0.678040in}}%
\pgfpathlineto{\pgfqpoint{3.137610in}{0.672606in}}%
\pgfpathlineto{\pgfqpoint{3.139648in}{0.665185in}}%
\pgfpathlineto{\pgfqpoint{3.140521in}{0.665939in}}%
\pgfpathlineto{\pgfqpoint{3.141395in}{0.672118in}}%
\pgfpathlineto{\pgfqpoint{3.142122in}{0.678368in}}%
\pgfpathlineto{\pgfqpoint{3.142705in}{0.676018in}}%
\pgfpathlineto{\pgfqpoint{3.145324in}{0.663036in}}%
\pgfpathlineto{\pgfqpoint{3.146052in}{0.663781in}}%
\pgfpathlineto{\pgfqpoint{3.147362in}{0.666078in}}%
\pgfpathlineto{\pgfqpoint{3.147944in}{0.667850in}}%
\pgfpathlineto{\pgfqpoint{3.148672in}{0.666841in}}%
\pgfpathlineto{\pgfqpoint{3.149400in}{0.667590in}}%
\pgfpathlineto{\pgfqpoint{3.149982in}{0.673469in}}%
\pgfpathlineto{\pgfqpoint{3.151001in}{0.695105in}}%
\pgfpathlineto{\pgfqpoint{3.151874in}{0.693204in}}%
\pgfpathlineto{\pgfqpoint{3.152311in}{0.693688in}}%
\pgfpathlineto{\pgfqpoint{3.152602in}{0.693022in}}%
\pgfpathlineto{\pgfqpoint{3.153620in}{0.686789in}}%
\pgfpathlineto{\pgfqpoint{3.154057in}{0.688883in}}%
\pgfpathlineto{\pgfqpoint{3.158278in}{0.744138in}}%
\pgfpathlineto{\pgfqpoint{3.158860in}{0.732779in}}%
\pgfpathlineto{\pgfqpoint{3.159151in}{0.728252in}}%
\pgfpathlineto{\pgfqpoint{3.159588in}{0.739237in}}%
\pgfpathlineto{\pgfqpoint{3.160170in}{0.757394in}}%
\pgfpathlineto{\pgfqpoint{3.160607in}{0.739229in}}%
\pgfpathlineto{\pgfqpoint{3.162208in}{0.688841in}}%
\pgfpathlineto{\pgfqpoint{3.162353in}{0.688982in}}%
\pgfpathlineto{\pgfqpoint{3.163226in}{0.694856in}}%
\pgfpathlineto{\pgfqpoint{3.163954in}{0.699508in}}%
\pgfpathlineto{\pgfqpoint{3.164536in}{0.697253in}}%
\pgfpathlineto{\pgfqpoint{3.166428in}{0.675882in}}%
\pgfpathlineto{\pgfqpoint{3.167593in}{0.666875in}}%
\pgfpathlineto{\pgfqpoint{3.167884in}{0.667213in}}%
\pgfpathlineto{\pgfqpoint{3.168466in}{0.670556in}}%
\pgfpathlineto{\pgfqpoint{3.170067in}{0.699805in}}%
\pgfpathlineto{\pgfqpoint{3.171231in}{0.693198in}}%
\pgfpathlineto{\pgfqpoint{3.171959in}{0.679640in}}%
\pgfpathlineto{\pgfqpoint{3.173269in}{0.665690in}}%
\pgfpathlineto{\pgfqpoint{3.173560in}{0.665859in}}%
\pgfpathlineto{\pgfqpoint{3.173997in}{0.669287in}}%
\pgfpathlineto{\pgfqpoint{3.174870in}{0.717462in}}%
\pgfpathlineto{\pgfqpoint{3.175161in}{0.724020in}}%
\pgfpathlineto{\pgfqpoint{3.175598in}{0.710397in}}%
\pgfpathlineto{\pgfqpoint{3.177199in}{0.686832in}}%
\pgfpathlineto{\pgfqpoint{3.178218in}{0.664024in}}%
\pgfpathlineto{\pgfqpoint{3.179091in}{0.656740in}}%
\pgfpathlineto{\pgfqpoint{3.179673in}{0.657480in}}%
\pgfpathlineto{\pgfqpoint{3.181420in}{0.664559in}}%
\pgfpathlineto{\pgfqpoint{3.183021in}{0.663578in}}%
\pgfpathlineto{\pgfqpoint{3.183457in}{0.664122in}}%
\pgfpathlineto{\pgfqpoint{3.184331in}{0.669755in}}%
\pgfpathlineto{\pgfqpoint{3.184913in}{0.665957in}}%
\pgfpathlineto{\pgfqpoint{3.185932in}{0.660445in}}%
\pgfpathlineto{\pgfqpoint{3.186368in}{0.660917in}}%
\pgfpathlineto{\pgfqpoint{3.191753in}{0.668854in}}%
\pgfpathlineto{\pgfqpoint{3.191899in}{0.668657in}}%
\pgfpathlineto{\pgfqpoint{3.192481in}{0.668174in}}%
\pgfpathlineto{\pgfqpoint{3.192918in}{0.668875in}}%
\pgfpathlineto{\pgfqpoint{3.193209in}{0.669048in}}%
\pgfpathlineto{\pgfqpoint{3.193646in}{0.668238in}}%
\pgfpathlineto{\pgfqpoint{3.194810in}{0.668496in}}%
\pgfpathlineto{\pgfqpoint{3.195829in}{0.670383in}}%
\pgfpathlineto{\pgfqpoint{3.196411in}{0.673177in}}%
\pgfpathlineto{\pgfqpoint{3.196993in}{0.670929in}}%
\pgfpathlineto{\pgfqpoint{3.198449in}{0.670060in}}%
\pgfpathlineto{\pgfqpoint{3.200632in}{0.670894in}}%
\pgfpathlineto{\pgfqpoint{3.201651in}{0.673660in}}%
\pgfpathlineto{\pgfqpoint{3.202233in}{0.672466in}}%
\pgfpathlineto{\pgfqpoint{3.203834in}{0.671601in}}%
\pgfpathlineto{\pgfqpoint{3.204561in}{0.670554in}}%
\pgfpathlineto{\pgfqpoint{3.206454in}{0.667852in}}%
\pgfpathlineto{\pgfqpoint{3.206745in}{0.668556in}}%
\pgfpathlineto{\pgfqpoint{3.207181in}{0.688802in}}%
\pgfpathlineto{\pgfqpoint{3.207909in}{0.767014in}}%
\pgfpathlineto{\pgfqpoint{3.208491in}{0.714654in}}%
\pgfpathlineto{\pgfqpoint{3.209073in}{0.685514in}}%
\pgfpathlineto{\pgfqpoint{3.209801in}{0.692097in}}%
\pgfpathlineto{\pgfqpoint{3.211111in}{0.667077in}}%
\pgfpathlineto{\pgfqpoint{3.212858in}{0.668574in}}%
\pgfpathlineto{\pgfqpoint{3.213876in}{0.670670in}}%
\pgfpathlineto{\pgfqpoint{3.214604in}{0.669450in}}%
\pgfpathlineto{\pgfqpoint{3.216060in}{0.669356in}}%
\pgfpathlineto{\pgfqpoint{3.217661in}{0.671366in}}%
\pgfpathlineto{\pgfqpoint{3.219262in}{0.673535in}}%
\pgfpathlineto{\pgfqpoint{3.220135in}{0.693092in}}%
\pgfpathlineto{\pgfqpoint{3.220571in}{0.681173in}}%
\pgfpathlineto{\pgfqpoint{3.221299in}{0.668911in}}%
\pgfpathlineto{\pgfqpoint{3.221881in}{0.671231in}}%
\pgfpathlineto{\pgfqpoint{3.222027in}{0.671474in}}%
\pgfpathlineto{\pgfqpoint{3.222318in}{0.670216in}}%
\pgfpathlineto{\pgfqpoint{3.222900in}{0.668454in}}%
\pgfpathlineto{\pgfqpoint{3.223337in}{0.669207in}}%
\pgfpathlineto{\pgfqpoint{3.224065in}{0.679922in}}%
\pgfpathlineto{\pgfqpoint{3.224647in}{0.673132in}}%
\pgfpathlineto{\pgfqpoint{3.224938in}{0.671541in}}%
\pgfpathlineto{\pgfqpoint{3.225520in}{0.674943in}}%
\pgfpathlineto{\pgfqpoint{3.225666in}{0.675438in}}%
\pgfpathlineto{\pgfqpoint{3.225957in}{0.673971in}}%
\pgfpathlineto{\pgfqpoint{3.226830in}{0.668150in}}%
\pgfpathlineto{\pgfqpoint{3.227412in}{0.668806in}}%
\pgfpathlineto{\pgfqpoint{3.229304in}{0.676815in}}%
\pgfpathlineto{\pgfqpoint{3.229595in}{0.679226in}}%
\pgfpathlineto{\pgfqpoint{3.230032in}{0.674970in}}%
\pgfpathlineto{\pgfqpoint{3.231051in}{0.670354in}}%
\pgfpathlineto{\pgfqpoint{3.231487in}{0.670392in}}%
\pgfpathlineto{\pgfqpoint{3.232506in}{0.671390in}}%
\pgfpathlineto{\pgfqpoint{3.234981in}{0.675880in}}%
\pgfpathlineto{\pgfqpoint{3.235272in}{0.674787in}}%
\pgfpathlineto{\pgfqpoint{3.237018in}{0.670035in}}%
\pgfpathlineto{\pgfqpoint{3.239056in}{0.667946in}}%
\pgfpathlineto{\pgfqpoint{3.239929in}{0.669099in}}%
\pgfpathlineto{\pgfqpoint{3.240220in}{0.671802in}}%
\pgfpathlineto{\pgfqpoint{3.240657in}{0.716091in}}%
\pgfpathlineto{\pgfqpoint{3.241093in}{0.802439in}}%
\pgfpathlineto{\pgfqpoint{3.241676in}{0.732052in}}%
\pgfpathlineto{\pgfqpoint{3.242549in}{0.678420in}}%
\pgfpathlineto{\pgfqpoint{3.242986in}{0.683432in}}%
\pgfpathlineto{\pgfqpoint{3.243422in}{0.687079in}}%
\pgfpathlineto{\pgfqpoint{3.243713in}{0.682313in}}%
\pgfpathlineto{\pgfqpoint{3.244587in}{0.665929in}}%
\pgfpathlineto{\pgfqpoint{3.245314in}{0.666334in}}%
\pgfpathlineto{\pgfqpoint{3.246333in}{0.667512in}}%
\pgfpathlineto{\pgfqpoint{3.247643in}{0.683702in}}%
\pgfpathlineto{\pgfqpoint{3.248371in}{0.675702in}}%
\pgfpathlineto{\pgfqpoint{3.248516in}{0.675141in}}%
\pgfpathlineto{\pgfqpoint{3.248807in}{0.676810in}}%
\pgfpathlineto{\pgfqpoint{3.249390in}{0.686183in}}%
\pgfpathlineto{\pgfqpoint{3.249826in}{0.678157in}}%
\pgfpathlineto{\pgfqpoint{3.250554in}{0.666330in}}%
\pgfpathlineto{\pgfqpoint{3.251282in}{0.666895in}}%
\pgfpathlineto{\pgfqpoint{3.252300in}{0.668738in}}%
\pgfpathlineto{\pgfqpoint{3.252883in}{0.700136in}}%
\pgfpathlineto{\pgfqpoint{3.253319in}{0.728862in}}%
\pgfpathlineto{\pgfqpoint{3.253901in}{0.693355in}}%
\pgfpathlineto{\pgfqpoint{3.254193in}{0.687147in}}%
\pgfpathlineto{\pgfqpoint{3.254484in}{0.699535in}}%
\pgfpathlineto{\pgfqpoint{3.254920in}{0.724489in}}%
\pgfpathlineto{\pgfqpoint{3.255357in}{0.691106in}}%
\pgfpathlineto{\pgfqpoint{3.256085in}{0.667929in}}%
\pgfpathlineto{\pgfqpoint{3.256667in}{0.668154in}}%
\pgfpathlineto{\pgfqpoint{3.257686in}{0.668810in}}%
\pgfpathlineto{\pgfqpoint{3.258559in}{0.685312in}}%
\pgfpathlineto{\pgfqpoint{3.259287in}{0.672990in}}%
\pgfpathlineto{\pgfqpoint{3.261033in}{0.668518in}}%
\pgfpathlineto{\pgfqpoint{3.262780in}{0.670371in}}%
\pgfpathlineto{\pgfqpoint{3.264526in}{0.682923in}}%
\pgfpathlineto{\pgfqpoint{3.265400in}{0.678226in}}%
\pgfpathlineto{\pgfqpoint{3.268456in}{0.667989in}}%
\pgfpathlineto{\pgfqpoint{3.269038in}{0.668422in}}%
\pgfpathlineto{\pgfqpoint{3.269620in}{0.673518in}}%
\pgfpathlineto{\pgfqpoint{3.270348in}{0.682266in}}%
\pgfpathlineto{\pgfqpoint{3.270930in}{0.679018in}}%
\pgfpathlineto{\pgfqpoint{3.272968in}{0.674635in}}%
\pgfpathlineto{\pgfqpoint{3.273550in}{0.674114in}}%
\pgfpathlineto{\pgfqpoint{3.273987in}{0.674862in}}%
\pgfpathlineto{\pgfqpoint{3.274278in}{0.674707in}}%
\pgfpathlineto{\pgfqpoint{3.274423in}{0.674150in}}%
\pgfpathlineto{\pgfqpoint{3.275151in}{0.671124in}}%
\pgfpathlineto{\pgfqpoint{3.275442in}{0.672424in}}%
\pgfpathlineto{\pgfqpoint{3.276315in}{0.695117in}}%
\pgfpathlineto{\pgfqpoint{3.276752in}{0.703732in}}%
\pgfpathlineto{\pgfqpoint{3.277334in}{0.697021in}}%
\pgfpathlineto{\pgfqpoint{3.280536in}{0.666319in}}%
\pgfpathlineto{\pgfqpoint{3.281846in}{0.667380in}}%
\pgfpathlineto{\pgfqpoint{3.282283in}{0.667637in}}%
\pgfpathlineto{\pgfqpoint{3.282720in}{0.666721in}}%
\pgfpathlineto{\pgfqpoint{3.284175in}{0.665995in}}%
\pgfpathlineto{\pgfqpoint{3.284466in}{0.669531in}}%
\pgfpathlineto{\pgfqpoint{3.285048in}{0.731098in}}%
\pgfpathlineto{\pgfqpoint{3.285485in}{0.768107in}}%
\pgfpathlineto{\pgfqpoint{3.286067in}{0.722401in}}%
\pgfpathlineto{\pgfqpoint{3.286213in}{0.719451in}}%
\pgfpathlineto{\pgfqpoint{3.286504in}{0.735977in}}%
\pgfpathlineto{\pgfqpoint{3.287086in}{0.799926in}}%
\pgfpathlineto{\pgfqpoint{3.287523in}{0.754602in}}%
\pgfpathlineto{\pgfqpoint{3.289124in}{0.670361in}}%
\pgfpathlineto{\pgfqpoint{3.290142in}{0.671011in}}%
\pgfpathlineto{\pgfqpoint{3.291016in}{0.676721in}}%
\pgfpathlineto{\pgfqpoint{3.291598in}{0.679585in}}%
\pgfpathlineto{\pgfqpoint{3.292326in}{0.678572in}}%
\pgfpathlineto{\pgfqpoint{3.292762in}{0.677736in}}%
\pgfpathlineto{\pgfqpoint{3.294072in}{0.669138in}}%
\pgfpathlineto{\pgfqpoint{3.294800in}{0.671444in}}%
\pgfpathlineto{\pgfqpoint{3.294945in}{0.671577in}}%
\pgfpathlineto{\pgfqpoint{3.295236in}{0.670850in}}%
\pgfpathlineto{\pgfqpoint{3.295964in}{0.669099in}}%
\pgfpathlineto{\pgfqpoint{3.296401in}{0.670042in}}%
\pgfpathlineto{\pgfqpoint{3.298730in}{0.682247in}}%
\pgfpathlineto{\pgfqpoint{3.299748in}{0.679480in}}%
\pgfpathlineto{\pgfqpoint{3.301058in}{0.672488in}}%
\pgfpathlineto{\pgfqpoint{3.301640in}{0.676348in}}%
\pgfpathlineto{\pgfqpoint{3.303969in}{0.698463in}}%
\pgfpathlineto{\pgfqpoint{3.304260in}{0.698865in}}%
\pgfpathlineto{\pgfqpoint{3.304551in}{0.697923in}}%
\pgfpathlineto{\pgfqpoint{3.307171in}{0.673753in}}%
\pgfpathlineto{\pgfqpoint{3.308627in}{0.666646in}}%
\pgfpathlineto{\pgfqpoint{3.313575in}{0.665264in}}%
\pgfpathlineto{\pgfqpoint{3.313721in}{0.665741in}}%
\pgfpathlineto{\pgfqpoint{3.314303in}{0.667142in}}%
\pgfpathlineto{\pgfqpoint{3.314740in}{0.665430in}}%
\pgfpathlineto{\pgfqpoint{3.315322in}{0.663618in}}%
\pgfpathlineto{\pgfqpoint{3.316049in}{0.664362in}}%
\pgfpathlineto{\pgfqpoint{3.316923in}{0.663827in}}%
\pgfpathlineto{\pgfqpoint{3.318378in}{0.660900in}}%
\pgfpathlineto{\pgfqpoint{3.318960in}{0.658751in}}%
\pgfpathlineto{\pgfqpoint{3.319543in}{0.659532in}}%
\pgfpathlineto{\pgfqpoint{3.320561in}{0.663452in}}%
\pgfpathlineto{\pgfqpoint{3.321726in}{0.684008in}}%
\pgfpathlineto{\pgfqpoint{3.322308in}{0.675332in}}%
\pgfpathlineto{\pgfqpoint{3.323909in}{0.667315in}}%
\pgfpathlineto{\pgfqpoint{3.324782in}{0.661956in}}%
\pgfpathlineto{\pgfqpoint{3.325364in}{0.662453in}}%
\pgfpathlineto{\pgfqpoint{3.326820in}{0.665180in}}%
\pgfpathlineto{\pgfqpoint{3.327839in}{0.675506in}}%
\pgfpathlineto{\pgfqpoint{3.328421in}{0.668715in}}%
\pgfpathlineto{\pgfqpoint{3.328857in}{0.667087in}}%
\pgfpathlineto{\pgfqpoint{3.329440in}{0.668742in}}%
\pgfpathlineto{\pgfqpoint{3.329876in}{0.667727in}}%
\pgfpathlineto{\pgfqpoint{3.330604in}{0.665996in}}%
\pgfpathlineto{\pgfqpoint{3.331186in}{0.666413in}}%
\pgfpathlineto{\pgfqpoint{3.333806in}{0.667559in}}%
\pgfpathlineto{\pgfqpoint{3.338172in}{0.667096in}}%
\pgfpathlineto{\pgfqpoint{3.339191in}{0.669126in}}%
\pgfpathlineto{\pgfqpoint{3.339628in}{0.670736in}}%
\pgfpathlineto{\pgfqpoint{3.340210in}{0.669264in}}%
\pgfpathlineto{\pgfqpoint{3.340938in}{0.669078in}}%
\pgfpathlineto{\pgfqpoint{3.341229in}{0.669318in}}%
\pgfpathlineto{\pgfqpoint{3.343267in}{0.670068in}}%
\pgfpathlineto{\pgfqpoint{3.344285in}{0.670460in}}%
\pgfpathlineto{\pgfqpoint{3.346469in}{0.679427in}}%
\pgfpathlineto{\pgfqpoint{3.347051in}{0.686075in}}%
\pgfpathlineto{\pgfqpoint{3.347778in}{0.682908in}}%
\pgfpathlineto{\pgfqpoint{3.348797in}{0.678005in}}%
\pgfpathlineto{\pgfqpoint{3.349816in}{0.674357in}}%
\pgfpathlineto{\pgfqpoint{3.350253in}{0.675516in}}%
\pgfpathlineto{\pgfqpoint{3.350980in}{0.686979in}}%
\pgfpathlineto{\pgfqpoint{3.351417in}{0.695336in}}%
\pgfpathlineto{\pgfqpoint{3.351999in}{0.686472in}}%
\pgfpathlineto{\pgfqpoint{3.353600in}{0.674859in}}%
\pgfpathlineto{\pgfqpoint{3.354037in}{0.672041in}}%
\pgfpathlineto{\pgfqpoint{3.354474in}{0.674667in}}%
\pgfpathlineto{\pgfqpoint{3.356657in}{0.732809in}}%
\pgfpathlineto{\pgfqpoint{3.357093in}{0.761832in}}%
\pgfpathlineto{\pgfqpoint{3.357676in}{0.726077in}}%
\pgfpathlineto{\pgfqpoint{3.358112in}{0.710689in}}%
\pgfpathlineto{\pgfqpoint{3.358694in}{0.723777in}}%
\pgfpathlineto{\pgfqpoint{3.359277in}{0.736164in}}%
\pgfpathlineto{\pgfqpoint{3.359713in}{0.727821in}}%
\pgfpathlineto{\pgfqpoint{3.362770in}{0.672408in}}%
\pgfpathlineto{\pgfqpoint{3.363061in}{0.673412in}}%
\pgfpathlineto{\pgfqpoint{3.363643in}{0.695171in}}%
\pgfpathlineto{\pgfqpoint{3.364225in}{0.735234in}}%
\pgfpathlineto{\pgfqpoint{3.364807in}{0.708251in}}%
\pgfpathlineto{\pgfqpoint{3.366263in}{0.683781in}}%
\pgfpathlineto{\pgfqpoint{3.367282in}{0.677839in}}%
\pgfpathlineto{\pgfqpoint{3.368009in}{0.678680in}}%
\pgfpathlineto{\pgfqpoint{3.368446in}{0.678010in}}%
\pgfpathlineto{\pgfqpoint{3.368737in}{0.679528in}}%
\pgfpathlineto{\pgfqpoint{3.369610in}{0.695517in}}%
\pgfpathlineto{\pgfqpoint{3.370192in}{0.686208in}}%
\pgfpathlineto{\pgfqpoint{3.371793in}{0.665249in}}%
\pgfpathlineto{\pgfqpoint{3.371939in}{0.665259in}}%
\pgfpathlineto{\pgfqpoint{3.372958in}{0.666262in}}%
\pgfpathlineto{\pgfqpoint{3.374268in}{0.669695in}}%
\pgfpathlineto{\pgfqpoint{3.374850in}{0.668710in}}%
\pgfpathlineto{\pgfqpoint{3.375141in}{0.668013in}}%
\pgfpathlineto{\pgfqpoint{3.375432in}{0.668797in}}%
\pgfpathlineto{\pgfqpoint{3.376014in}{0.683555in}}%
\pgfpathlineto{\pgfqpoint{3.376596in}{0.706507in}}%
\pgfpathlineto{\pgfqpoint{3.377033in}{0.692821in}}%
\pgfpathlineto{\pgfqpoint{3.378052in}{0.659463in}}%
\pgfpathlineto{\pgfqpoint{3.378634in}{0.660242in}}%
\pgfpathlineto{\pgfqpoint{3.379362in}{0.655930in}}%
\pgfpathlineto{\pgfqpoint{3.379798in}{0.657920in}}%
\pgfpathlineto{\pgfqpoint{3.382273in}{0.689869in}}%
\pgfpathlineto{\pgfqpoint{3.382418in}{0.689739in}}%
\pgfpathlineto{\pgfqpoint{3.384019in}{0.653301in}}%
\pgfpathlineto{\pgfqpoint{3.385329in}{0.657171in}}%
\pgfpathlineto{\pgfqpoint{3.387221in}{0.660685in}}%
\pgfpathlineto{\pgfqpoint{3.388240in}{0.661740in}}%
\pgfpathlineto{\pgfqpoint{3.390132in}{0.666270in}}%
\pgfpathlineto{\pgfqpoint{3.391005in}{0.672228in}}%
\pgfpathlineto{\pgfqpoint{3.391588in}{0.668898in}}%
\pgfpathlineto{\pgfqpoint{3.391879in}{0.668035in}}%
\pgfpathlineto{\pgfqpoint{3.392315in}{0.670526in}}%
\pgfpathlineto{\pgfqpoint{3.392752in}{0.673357in}}%
\pgfpathlineto{\pgfqpoint{3.393189in}{0.668966in}}%
\pgfpathlineto{\pgfqpoint{3.394062in}{0.662932in}}%
\pgfpathlineto{\pgfqpoint{3.394644in}{0.663629in}}%
\pgfpathlineto{\pgfqpoint{3.395663in}{0.666734in}}%
\pgfpathlineto{\pgfqpoint{3.396245in}{0.669665in}}%
\pgfpathlineto{\pgfqpoint{3.396973in}{0.668159in}}%
\pgfpathlineto{\pgfqpoint{3.397410in}{0.670113in}}%
\pgfpathlineto{\pgfqpoint{3.398137in}{0.678753in}}%
\pgfpathlineto{\pgfqpoint{3.398574in}{0.673667in}}%
\pgfpathlineto{\pgfqpoint{3.399593in}{0.665895in}}%
\pgfpathlineto{\pgfqpoint{3.400029in}{0.666147in}}%
\pgfpathlineto{\pgfqpoint{3.403522in}{0.670163in}}%
\pgfpathlineto{\pgfqpoint{3.404105in}{0.668140in}}%
\pgfpathlineto{\pgfqpoint{3.405123in}{0.668542in}}%
\pgfpathlineto{\pgfqpoint{3.405997in}{0.670175in}}%
\pgfpathlineto{\pgfqpoint{3.408617in}{0.688423in}}%
\pgfpathlineto{\pgfqpoint{3.409199in}{0.686928in}}%
\pgfpathlineto{\pgfqpoint{3.409344in}{0.686648in}}%
\pgfpathlineto{\pgfqpoint{3.409635in}{0.687618in}}%
\pgfpathlineto{\pgfqpoint{3.410218in}{0.710781in}}%
\pgfpathlineto{\pgfqpoint{3.410945in}{0.787469in}}%
\pgfpathlineto{\pgfqpoint{3.411382in}{0.747533in}}%
\pgfpathlineto{\pgfqpoint{3.412255in}{0.679676in}}%
\pgfpathlineto{\pgfqpoint{3.412837in}{0.680531in}}%
\pgfpathlineto{\pgfqpoint{3.413711in}{0.670895in}}%
\pgfpathlineto{\pgfqpoint{3.414147in}{0.674377in}}%
\pgfpathlineto{\pgfqpoint{3.415748in}{0.680568in}}%
\pgfpathlineto{\pgfqpoint{3.415894in}{0.681231in}}%
\pgfpathlineto{\pgfqpoint{3.416185in}{0.679062in}}%
\pgfpathlineto{\pgfqpoint{3.417349in}{0.668387in}}%
\pgfpathlineto{\pgfqpoint{3.417786in}{0.668437in}}%
\pgfpathlineto{\pgfqpoint{3.418368in}{0.669036in}}%
\pgfpathlineto{\pgfqpoint{3.418805in}{0.676671in}}%
\pgfpathlineto{\pgfqpoint{3.419532in}{0.706894in}}%
\pgfpathlineto{\pgfqpoint{3.420115in}{0.690603in}}%
\pgfpathlineto{\pgfqpoint{3.422152in}{0.669737in}}%
\pgfpathlineto{\pgfqpoint{3.422880in}{0.669109in}}%
\pgfpathlineto{\pgfqpoint{3.423171in}{0.669399in}}%
\pgfpathlineto{\pgfqpoint{3.423753in}{0.673685in}}%
\pgfpathlineto{\pgfqpoint{3.424335in}{0.680259in}}%
\pgfpathlineto{\pgfqpoint{3.424918in}{0.675559in}}%
\pgfpathlineto{\pgfqpoint{3.426519in}{0.669701in}}%
\pgfpathlineto{\pgfqpoint{3.426955in}{0.668740in}}%
\pgfpathlineto{\pgfqpoint{3.427537in}{0.669861in}}%
\pgfpathlineto{\pgfqpoint{3.428993in}{0.676028in}}%
\pgfpathlineto{\pgfqpoint{3.429575in}{0.684985in}}%
\pgfpathlineto{\pgfqpoint{3.430157in}{0.678143in}}%
\pgfpathlineto{\pgfqpoint{3.431031in}{0.668793in}}%
\pgfpathlineto{\pgfqpoint{3.431613in}{0.669952in}}%
\pgfpathlineto{\pgfqpoint{3.432486in}{0.667039in}}%
\pgfpathlineto{\pgfqpoint{3.432923in}{0.668643in}}%
\pgfpathlineto{\pgfqpoint{3.433796in}{0.688257in}}%
\pgfpathlineto{\pgfqpoint{3.434524in}{0.677446in}}%
\pgfpathlineto{\pgfqpoint{3.434815in}{0.679611in}}%
\pgfpathlineto{\pgfqpoint{3.435688in}{0.709196in}}%
\pgfpathlineto{\pgfqpoint{3.436270in}{0.691353in}}%
\pgfpathlineto{\pgfqpoint{3.437289in}{0.672317in}}%
\pgfpathlineto{\pgfqpoint{3.437726in}{0.674041in}}%
\pgfpathlineto{\pgfqpoint{3.438599in}{0.676354in}}%
\pgfpathlineto{\pgfqpoint{3.439036in}{0.675810in}}%
\pgfpathlineto{\pgfqpoint{3.440054in}{0.671614in}}%
\pgfpathlineto{\pgfqpoint{3.441655in}{0.663417in}}%
\pgfpathlineto{\pgfqpoint{3.442092in}{0.663966in}}%
\pgfpathlineto{\pgfqpoint{3.442674in}{0.668260in}}%
\pgfpathlineto{\pgfqpoint{3.445148in}{0.712595in}}%
\pgfpathlineto{\pgfqpoint{3.445585in}{0.710098in}}%
\pgfpathlineto{\pgfqpoint{3.449660in}{0.664371in}}%
\pgfpathlineto{\pgfqpoint{3.450534in}{0.664959in}}%
\pgfpathlineto{\pgfqpoint{3.451261in}{0.669231in}}%
\pgfpathlineto{\pgfqpoint{3.451989in}{0.697423in}}%
\pgfpathlineto{\pgfqpoint{3.453881in}{0.750712in}}%
\pgfpathlineto{\pgfqpoint{3.454027in}{0.749975in}}%
\pgfpathlineto{\pgfqpoint{3.455191in}{0.693466in}}%
\pgfpathlineto{\pgfqpoint{3.456210in}{0.696187in}}%
\pgfpathlineto{\pgfqpoint{3.458684in}{0.672510in}}%
\pgfpathlineto{\pgfqpoint{3.459412in}{0.668979in}}%
\pgfpathlineto{\pgfqpoint{3.460431in}{0.661224in}}%
\pgfpathlineto{\pgfqpoint{3.461013in}{0.662550in}}%
\pgfpathlineto{\pgfqpoint{3.463778in}{0.674961in}}%
\pgfpathlineto{\pgfqpoint{3.463924in}{0.674755in}}%
\pgfpathlineto{\pgfqpoint{3.464215in}{0.674281in}}%
\pgfpathlineto{\pgfqpoint{3.464506in}{0.675393in}}%
\pgfpathlineto{\pgfqpoint{3.465525in}{0.695982in}}%
\pgfpathlineto{\pgfqpoint{3.465962in}{0.686328in}}%
\pgfpathlineto{\pgfqpoint{3.466980in}{0.659929in}}%
\pgfpathlineto{\pgfqpoint{3.467563in}{0.660847in}}%
\pgfpathlineto{\pgfqpoint{3.469164in}{0.663759in}}%
\pgfpathlineto{\pgfqpoint{3.469746in}{0.665722in}}%
\pgfpathlineto{\pgfqpoint{3.470328in}{0.664763in}}%
\pgfpathlineto{\pgfqpoint{3.473384in}{0.662893in}}%
\pgfpathlineto{\pgfqpoint{3.475859in}{0.666967in}}%
\pgfpathlineto{\pgfqpoint{3.478042in}{0.678015in}}%
\pgfpathlineto{\pgfqpoint{3.478333in}{0.674963in}}%
\pgfpathlineto{\pgfqpoint{3.479206in}{0.666532in}}%
\pgfpathlineto{\pgfqpoint{3.479788in}{0.667189in}}%
\pgfpathlineto{\pgfqpoint{3.481389in}{0.669506in}}%
\pgfpathlineto{\pgfqpoint{3.481680in}{0.671083in}}%
\pgfpathlineto{\pgfqpoint{3.482263in}{0.668498in}}%
\pgfpathlineto{\pgfqpoint{3.482408in}{0.668279in}}%
\pgfpathlineto{\pgfqpoint{3.482699in}{0.669023in}}%
\pgfpathlineto{\pgfqpoint{3.483281in}{0.685350in}}%
\pgfpathlineto{\pgfqpoint{3.483573in}{0.693085in}}%
\pgfpathlineto{\pgfqpoint{3.484009in}{0.680265in}}%
\pgfpathlineto{\pgfqpoint{3.484591in}{0.668743in}}%
\pgfpathlineto{\pgfqpoint{3.485028in}{0.674239in}}%
\pgfpathlineto{\pgfqpoint{3.485610in}{0.687316in}}%
\pgfpathlineto{\pgfqpoint{3.486047in}{0.675326in}}%
\pgfpathlineto{\pgfqpoint{3.486775in}{0.667163in}}%
\pgfpathlineto{\pgfqpoint{3.487357in}{0.668913in}}%
\pgfpathlineto{\pgfqpoint{3.488376in}{0.677952in}}%
\pgfpathlineto{\pgfqpoint{3.489249in}{0.676516in}}%
\pgfpathlineto{\pgfqpoint{3.489685in}{0.676102in}}%
\pgfpathlineto{\pgfqpoint{3.489977in}{0.676982in}}%
\pgfpathlineto{\pgfqpoint{3.490850in}{0.685816in}}%
\pgfpathlineto{\pgfqpoint{3.491432in}{0.680259in}}%
\pgfpathlineto{\pgfqpoint{3.491723in}{0.678164in}}%
\pgfpathlineto{\pgfqpoint{3.492160in}{0.681418in}}%
\pgfpathlineto{\pgfqpoint{3.493033in}{0.703992in}}%
\pgfpathlineto{\pgfqpoint{3.493906in}{0.696704in}}%
\pgfpathlineto{\pgfqpoint{3.494197in}{0.697503in}}%
\pgfpathlineto{\pgfqpoint{3.494488in}{0.695114in}}%
\pgfpathlineto{\pgfqpoint{3.495944in}{0.670052in}}%
\pgfpathlineto{\pgfqpoint{3.496526in}{0.672261in}}%
\pgfpathlineto{\pgfqpoint{3.496672in}{0.672579in}}%
\pgfpathlineto{\pgfqpoint{3.496963in}{0.671685in}}%
\pgfpathlineto{\pgfqpoint{3.498127in}{0.666643in}}%
\pgfpathlineto{\pgfqpoint{3.498709in}{0.667019in}}%
\pgfpathlineto{\pgfqpoint{3.500165in}{0.666180in}}%
\pgfpathlineto{\pgfqpoint{3.500601in}{0.665933in}}%
\pgfpathlineto{\pgfqpoint{3.501184in}{0.666578in}}%
\pgfpathlineto{\pgfqpoint{3.502202in}{0.668186in}}%
\pgfpathlineto{\pgfqpoint{3.503367in}{0.674606in}}%
\pgfpathlineto{\pgfqpoint{3.503803in}{0.672937in}}%
\pgfpathlineto{\pgfqpoint{3.504677in}{0.665425in}}%
\pgfpathlineto{\pgfqpoint{3.505113in}{0.668528in}}%
\pgfpathlineto{\pgfqpoint{3.505550in}{0.672212in}}%
\pgfpathlineto{\pgfqpoint{3.505987in}{0.667409in}}%
\pgfpathlineto{\pgfqpoint{3.506569in}{0.663548in}}%
\pgfpathlineto{\pgfqpoint{3.507297in}{0.664376in}}%
\pgfpathlineto{\pgfqpoint{3.509771in}{0.665823in}}%
\pgfpathlineto{\pgfqpoint{3.510353in}{0.666609in}}%
\pgfpathlineto{\pgfqpoint{3.511081in}{0.676157in}}%
\pgfpathlineto{\pgfqpoint{3.511372in}{0.680266in}}%
\pgfpathlineto{\pgfqpoint{3.511954in}{0.672109in}}%
\pgfpathlineto{\pgfqpoint{3.512391in}{0.667842in}}%
\pgfpathlineto{\pgfqpoint{3.512827in}{0.673509in}}%
\pgfpathlineto{\pgfqpoint{3.513118in}{0.679585in}}%
\pgfpathlineto{\pgfqpoint{3.513555in}{0.671394in}}%
\pgfpathlineto{\pgfqpoint{3.513992in}{0.664889in}}%
\pgfpathlineto{\pgfqpoint{3.514574in}{0.669946in}}%
\pgfpathlineto{\pgfqpoint{3.515302in}{0.699426in}}%
\pgfpathlineto{\pgfqpoint{3.515738in}{0.682409in}}%
\pgfpathlineto{\pgfqpoint{3.516175in}{0.674181in}}%
\pgfpathlineto{\pgfqpoint{3.516611in}{0.688492in}}%
\pgfpathlineto{\pgfqpoint{3.517048in}{0.708149in}}%
\pgfpathlineto{\pgfqpoint{3.517485in}{0.689410in}}%
\pgfpathlineto{\pgfqpoint{3.518504in}{0.665984in}}%
\pgfpathlineto{\pgfqpoint{3.518940in}{0.666257in}}%
\pgfpathlineto{\pgfqpoint{3.520832in}{0.668731in}}%
\pgfpathlineto{\pgfqpoint{3.521560in}{0.682890in}}%
\pgfpathlineto{\pgfqpoint{3.522142in}{0.694924in}}%
\pgfpathlineto{\pgfqpoint{3.523015in}{0.693663in}}%
\pgfpathlineto{\pgfqpoint{3.523889in}{0.735347in}}%
\pgfpathlineto{\pgfqpoint{3.524471in}{0.710291in}}%
\pgfpathlineto{\pgfqpoint{3.525926in}{0.668598in}}%
\pgfpathlineto{\pgfqpoint{3.526072in}{0.668636in}}%
\pgfpathlineto{\pgfqpoint{3.527091in}{0.669187in}}%
\pgfpathlineto{\pgfqpoint{3.527818in}{0.684521in}}%
\pgfpathlineto{\pgfqpoint{3.528255in}{0.691698in}}%
\pgfpathlineto{\pgfqpoint{3.528983in}{0.686435in}}%
\pgfpathlineto{\pgfqpoint{3.529419in}{0.690727in}}%
\pgfpathlineto{\pgfqpoint{3.530293in}{0.708025in}}%
\pgfpathlineto{\pgfqpoint{3.530729in}{0.697381in}}%
\pgfpathlineto{\pgfqpoint{3.531894in}{0.667584in}}%
\pgfpathlineto{\pgfqpoint{3.532476in}{0.668784in}}%
\pgfpathlineto{\pgfqpoint{3.532621in}{0.668897in}}%
\pgfpathlineto{\pgfqpoint{3.532913in}{0.668159in}}%
\pgfpathlineto{\pgfqpoint{3.533495in}{0.667085in}}%
\pgfpathlineto{\pgfqpoint{3.534077in}{0.667653in}}%
\pgfpathlineto{\pgfqpoint{3.535969in}{0.672638in}}%
\pgfpathlineto{\pgfqpoint{3.536115in}{0.673441in}}%
\pgfpathlineto{\pgfqpoint{3.536551in}{0.670893in}}%
\pgfpathlineto{\pgfqpoint{3.537133in}{0.667391in}}%
\pgfpathlineto{\pgfqpoint{3.537570in}{0.669511in}}%
\pgfpathlineto{\pgfqpoint{3.538298in}{0.691764in}}%
\pgfpathlineto{\pgfqpoint{3.538880in}{0.678805in}}%
\pgfpathlineto{\pgfqpoint{3.539462in}{0.669784in}}%
\pgfpathlineto{\pgfqpoint{3.540190in}{0.672389in}}%
\pgfpathlineto{\pgfqpoint{3.541209in}{0.666310in}}%
\pgfpathlineto{\pgfqpoint{3.542082in}{0.666895in}}%
\pgfpathlineto{\pgfqpoint{3.543246in}{0.669465in}}%
\pgfpathlineto{\pgfqpoint{3.547467in}{0.684119in}}%
\pgfpathlineto{\pgfqpoint{3.548049in}{0.691178in}}%
\pgfpathlineto{\pgfqpoint{3.548340in}{0.686028in}}%
\pgfpathlineto{\pgfqpoint{3.549359in}{0.669103in}}%
\pgfpathlineto{\pgfqpoint{3.549796in}{0.671015in}}%
\pgfpathlineto{\pgfqpoint{3.551251in}{0.688486in}}%
\pgfpathlineto{\pgfqpoint{3.552125in}{0.686772in}}%
\pgfpathlineto{\pgfqpoint{3.552707in}{0.683897in}}%
\pgfpathlineto{\pgfqpoint{3.553143in}{0.686554in}}%
\pgfpathlineto{\pgfqpoint{3.554162in}{0.712512in}}%
\pgfpathlineto{\pgfqpoint{3.554599in}{0.701653in}}%
\pgfpathlineto{\pgfqpoint{3.556345in}{0.670415in}}%
\pgfpathlineto{\pgfqpoint{3.557364in}{0.665423in}}%
\pgfpathlineto{\pgfqpoint{3.557801in}{0.667813in}}%
\pgfpathlineto{\pgfqpoint{3.559984in}{0.695329in}}%
\pgfpathlineto{\pgfqpoint{3.560130in}{0.694174in}}%
\pgfpathlineto{\pgfqpoint{3.561585in}{0.664262in}}%
\pgfpathlineto{\pgfqpoint{3.562167in}{0.666887in}}%
\pgfpathlineto{\pgfqpoint{3.562604in}{0.685743in}}%
\pgfpathlineto{\pgfqpoint{3.563332in}{0.781374in}}%
\pgfpathlineto{\pgfqpoint{3.563914in}{0.732560in}}%
\pgfpathlineto{\pgfqpoint{3.564205in}{0.720078in}}%
\pgfpathlineto{\pgfqpoint{3.564642in}{0.743538in}}%
\pgfpathlineto{\pgfqpoint{3.564933in}{0.757692in}}%
\pgfpathlineto{\pgfqpoint{3.565369in}{0.726026in}}%
\pgfpathlineto{\pgfqpoint{3.566679in}{0.665475in}}%
\pgfpathlineto{\pgfqpoint{3.566970in}{0.665641in}}%
\pgfpathlineto{\pgfqpoint{3.568280in}{0.668053in}}%
\pgfpathlineto{\pgfqpoint{3.568862in}{0.670465in}}%
\pgfpathlineto{\pgfqpoint{3.569590in}{0.669194in}}%
\pgfpathlineto{\pgfqpoint{3.571046in}{0.668362in}}%
\pgfpathlineto{\pgfqpoint{3.571191in}{0.669106in}}%
\pgfpathlineto{\pgfqpoint{3.572647in}{0.679329in}}%
\pgfpathlineto{\pgfqpoint{3.573229in}{0.697994in}}%
\pgfpathlineto{\pgfqpoint{3.573665in}{0.681041in}}%
\pgfpathlineto{\pgfqpoint{3.574539in}{0.665935in}}%
\pgfpathlineto{\pgfqpoint{3.575121in}{0.666143in}}%
\pgfpathlineto{\pgfqpoint{3.577013in}{0.667418in}}%
\pgfpathlineto{\pgfqpoint{3.578905in}{0.669252in}}%
\pgfpathlineto{\pgfqpoint{3.583999in}{0.670927in}}%
\pgfpathlineto{\pgfqpoint{3.586619in}{0.676873in}}%
\pgfpathlineto{\pgfqpoint{3.587056in}{0.675070in}}%
\pgfpathlineto{\pgfqpoint{3.588220in}{0.669338in}}%
\pgfpathlineto{\pgfqpoint{3.588657in}{0.669512in}}%
\pgfpathlineto{\pgfqpoint{3.589675in}{0.671446in}}%
\pgfpathlineto{\pgfqpoint{3.591567in}{0.674613in}}%
\pgfpathlineto{\pgfqpoint{3.593460in}{0.674678in}}%
\pgfpathlineto{\pgfqpoint{3.594042in}{0.688428in}}%
\pgfpathlineto{\pgfqpoint{3.594769in}{0.750008in}}%
\pgfpathlineto{\pgfqpoint{3.595497in}{0.714106in}}%
\pgfpathlineto{\pgfqpoint{3.596079in}{0.701997in}}%
\pgfpathlineto{\pgfqpoint{3.596807in}{0.702594in}}%
\pgfpathlineto{\pgfqpoint{3.598554in}{0.690307in}}%
\pgfpathlineto{\pgfqpoint{3.600446in}{0.666988in}}%
\pgfpathlineto{\pgfqpoint{3.600882in}{0.667494in}}%
\pgfpathlineto{\pgfqpoint{3.601610in}{0.674495in}}%
\pgfpathlineto{\pgfqpoint{3.602338in}{0.683872in}}%
\pgfpathlineto{\pgfqpoint{3.603066in}{0.682703in}}%
\pgfpathlineto{\pgfqpoint{3.603357in}{0.681508in}}%
\pgfpathlineto{\pgfqpoint{3.604667in}{0.666998in}}%
\pgfpathlineto{\pgfqpoint{3.605103in}{0.671685in}}%
\pgfpathlineto{\pgfqpoint{3.605685in}{0.733660in}}%
\pgfpathlineto{\pgfqpoint{3.606122in}{0.790485in}}%
\pgfpathlineto{\pgfqpoint{3.606704in}{0.723046in}}%
\pgfpathlineto{\pgfqpoint{3.607286in}{0.685419in}}%
\pgfpathlineto{\pgfqpoint{3.608014in}{0.691820in}}%
\pgfpathlineto{\pgfqpoint{3.608887in}{0.664730in}}%
\pgfpathlineto{\pgfqpoint{3.609615in}{0.669772in}}%
\pgfpathlineto{\pgfqpoint{3.609761in}{0.669903in}}%
\pgfpathlineto{\pgfqpoint{3.609906in}{0.669164in}}%
\pgfpathlineto{\pgfqpoint{3.610197in}{0.667642in}}%
\pgfpathlineto{\pgfqpoint{3.610488in}{0.669909in}}%
\pgfpathlineto{\pgfqpoint{3.611362in}{0.720173in}}%
\pgfpathlineto{\pgfqpoint{3.611944in}{0.693110in}}%
\pgfpathlineto{\pgfqpoint{3.613399in}{0.666351in}}%
\pgfpathlineto{\pgfqpoint{3.614127in}{0.663747in}}%
\pgfpathlineto{\pgfqpoint{3.614564in}{0.665086in}}%
\pgfpathlineto{\pgfqpoint{3.616456in}{0.683055in}}%
\pgfpathlineto{\pgfqpoint{3.617329in}{0.742271in}}%
\pgfpathlineto{\pgfqpoint{3.617911in}{0.712707in}}%
\pgfpathlineto{\pgfqpoint{3.618785in}{0.676479in}}%
\pgfpathlineto{\pgfqpoint{3.619367in}{0.684525in}}%
\pgfpathlineto{\pgfqpoint{3.619512in}{0.684267in}}%
\pgfpathlineto{\pgfqpoint{3.620531in}{0.661120in}}%
\pgfpathlineto{\pgfqpoint{3.621695in}{0.662772in}}%
\pgfpathlineto{\pgfqpoint{3.622278in}{0.664662in}}%
\pgfpathlineto{\pgfqpoint{3.623005in}{0.672373in}}%
\pgfpathlineto{\pgfqpoint{3.623442in}{0.667427in}}%
\pgfpathlineto{\pgfqpoint{3.624024in}{0.664829in}}%
\pgfpathlineto{\pgfqpoint{3.624752in}{0.665423in}}%
\pgfpathlineto{\pgfqpoint{3.626498in}{0.666328in}}%
\pgfpathlineto{\pgfqpoint{3.628973in}{0.668912in}}%
\pgfpathlineto{\pgfqpoint{3.629555in}{0.668265in}}%
\pgfpathlineto{\pgfqpoint{3.631156in}{0.668792in}}%
\pgfpathlineto{\pgfqpoint{3.633339in}{0.673743in}}%
\pgfpathlineto{\pgfqpoint{3.633630in}{0.675696in}}%
\pgfpathlineto{\pgfqpoint{3.634212in}{0.671876in}}%
\pgfpathlineto{\pgfqpoint{3.635522in}{0.669395in}}%
\pgfpathlineto{\pgfqpoint{3.637705in}{0.670273in}}%
\pgfpathlineto{\pgfqpoint{3.639161in}{0.670652in}}%
\pgfpathlineto{\pgfqpoint{3.640325in}{0.671485in}}%
\pgfpathlineto{\pgfqpoint{3.641199in}{0.678593in}}%
\pgfpathlineto{\pgfqpoint{3.641781in}{0.673066in}}%
\pgfpathlineto{\pgfqpoint{3.642217in}{0.671416in}}%
\pgfpathlineto{\pgfqpoint{3.642945in}{0.672119in}}%
\pgfpathlineto{\pgfqpoint{3.644401in}{0.670919in}}%
\pgfpathlineto{\pgfqpoint{3.647311in}{0.670930in}}%
\pgfpathlineto{\pgfqpoint{3.647748in}{0.670399in}}%
\pgfpathlineto{\pgfqpoint{3.648039in}{0.671459in}}%
\pgfpathlineto{\pgfqpoint{3.648912in}{0.682241in}}%
\pgfpathlineto{\pgfqpoint{3.649349in}{0.676801in}}%
\pgfpathlineto{\pgfqpoint{3.650077in}{0.671299in}}%
\pgfpathlineto{\pgfqpoint{3.650368in}{0.674349in}}%
\pgfpathlineto{\pgfqpoint{3.651096in}{0.709964in}}%
\pgfpathlineto{\pgfqpoint{3.651678in}{0.682525in}}%
\pgfpathlineto{\pgfqpoint{3.652260in}{0.669740in}}%
\pgfpathlineto{\pgfqpoint{3.652842in}{0.676820in}}%
\pgfpathlineto{\pgfqpoint{3.653279in}{0.681712in}}%
\pgfpathlineto{\pgfqpoint{3.653715in}{0.675028in}}%
\pgfpathlineto{\pgfqpoint{3.654443in}{0.668573in}}%
\pgfpathlineto{\pgfqpoint{3.654880in}{0.670817in}}%
\pgfpathlineto{\pgfqpoint{3.655316in}{0.674012in}}%
\pgfpathlineto{\pgfqpoint{3.655753in}{0.669887in}}%
\pgfpathlineto{\pgfqpoint{3.656335in}{0.668262in}}%
\pgfpathlineto{\pgfqpoint{3.656917in}{0.669029in}}%
\pgfpathlineto{\pgfqpoint{3.657500in}{0.673713in}}%
\pgfpathlineto{\pgfqpoint{3.658373in}{0.700630in}}%
\pgfpathlineto{\pgfqpoint{3.658955in}{0.685288in}}%
\pgfpathlineto{\pgfqpoint{3.659246in}{0.679978in}}%
\pgfpathlineto{\pgfqpoint{3.659683in}{0.694430in}}%
\pgfpathlineto{\pgfqpoint{3.660556in}{0.818589in}}%
\pgfpathlineto{\pgfqpoint{3.661138in}{0.757872in}}%
\pgfpathlineto{\pgfqpoint{3.662157in}{0.686534in}}%
\pgfpathlineto{\pgfqpoint{3.662594in}{0.688434in}}%
\pgfpathlineto{\pgfqpoint{3.662885in}{0.686836in}}%
\pgfpathlineto{\pgfqpoint{3.664340in}{0.670381in}}%
\pgfpathlineto{\pgfqpoint{3.664922in}{0.672195in}}%
\pgfpathlineto{\pgfqpoint{3.665650in}{0.671504in}}%
\pgfpathlineto{\pgfqpoint{3.666087in}{0.679149in}}%
\pgfpathlineto{\pgfqpoint{3.666815in}{0.724705in}}%
\pgfpathlineto{\pgfqpoint{3.667397in}{0.692273in}}%
\pgfpathlineto{\pgfqpoint{3.668852in}{0.676704in}}%
\pgfpathlineto{\pgfqpoint{3.669580in}{0.675674in}}%
\pgfpathlineto{\pgfqpoint{3.669871in}{0.676323in}}%
\pgfpathlineto{\pgfqpoint{3.671763in}{0.687780in}}%
\pgfpathlineto{\pgfqpoint{3.672200in}{0.695688in}}%
\pgfpathlineto{\pgfqpoint{3.672636in}{0.688694in}}%
\pgfpathlineto{\pgfqpoint{3.674237in}{0.669128in}}%
\pgfpathlineto{\pgfqpoint{3.674965in}{0.668940in}}%
\pgfpathlineto{\pgfqpoint{3.675256in}{0.669568in}}%
\pgfpathlineto{\pgfqpoint{3.676712in}{0.672862in}}%
\pgfpathlineto{\pgfqpoint{3.677148in}{0.672463in}}%
\pgfpathlineto{\pgfqpoint{3.677585in}{0.672348in}}%
\pgfpathlineto{\pgfqpoint{3.677876in}{0.673212in}}%
\pgfpathlineto{\pgfqpoint{3.678895in}{0.681638in}}%
\pgfpathlineto{\pgfqpoint{3.679477in}{0.676134in}}%
\pgfpathlineto{\pgfqpoint{3.680350in}{0.667352in}}%
\pgfpathlineto{\pgfqpoint{3.680787in}{0.669658in}}%
\pgfpathlineto{\pgfqpoint{3.681224in}{0.673986in}}%
\pgfpathlineto{\pgfqpoint{3.681660in}{0.669418in}}%
\pgfpathlineto{\pgfqpoint{3.682097in}{0.665983in}}%
\pgfpathlineto{\pgfqpoint{3.682825in}{0.668437in}}%
\pgfpathlineto{\pgfqpoint{3.683261in}{0.672599in}}%
\pgfpathlineto{\pgfqpoint{3.683843in}{0.667968in}}%
\pgfpathlineto{\pgfqpoint{3.684135in}{0.666654in}}%
\pgfpathlineto{\pgfqpoint{3.684571in}{0.668518in}}%
\pgfpathlineto{\pgfqpoint{3.684862in}{0.670963in}}%
\pgfpathlineto{\pgfqpoint{3.685299in}{0.667668in}}%
\pgfpathlineto{\pgfqpoint{3.685736in}{0.665188in}}%
\pgfpathlineto{\pgfqpoint{3.686318in}{0.667094in}}%
\pgfpathlineto{\pgfqpoint{3.686609in}{0.668897in}}%
\pgfpathlineto{\pgfqpoint{3.687191in}{0.665970in}}%
\pgfpathlineto{\pgfqpoint{3.687337in}{0.665787in}}%
\pgfpathlineto{\pgfqpoint{3.687628in}{0.666422in}}%
\pgfpathlineto{\pgfqpoint{3.688355in}{0.671806in}}%
\pgfpathlineto{\pgfqpoint{3.688792in}{0.668568in}}%
\pgfpathlineto{\pgfqpoint{3.690247in}{0.665709in}}%
\pgfpathlineto{\pgfqpoint{3.690684in}{0.665013in}}%
\pgfpathlineto{\pgfqpoint{3.690975in}{0.665694in}}%
\pgfpathlineto{\pgfqpoint{3.691412in}{0.683031in}}%
\pgfpathlineto{\pgfqpoint{3.691994in}{0.717960in}}%
\pgfpathlineto{\pgfqpoint{3.692576in}{0.687860in}}%
\pgfpathlineto{\pgfqpoint{3.692867in}{0.684505in}}%
\pgfpathlineto{\pgfqpoint{3.693158in}{0.693273in}}%
\pgfpathlineto{\pgfqpoint{3.693595in}{0.711141in}}%
\pgfpathlineto{\pgfqpoint{3.694032in}{0.696956in}}%
\pgfpathlineto{\pgfqpoint{3.695196in}{0.665240in}}%
\pgfpathlineto{\pgfqpoint{3.695633in}{0.665638in}}%
\pgfpathlineto{\pgfqpoint{3.696651in}{0.666466in}}%
\pgfpathlineto{\pgfqpoint{3.697379in}{0.676613in}}%
\pgfpathlineto{\pgfqpoint{3.698252in}{0.698952in}}%
\pgfpathlineto{\pgfqpoint{3.698835in}{0.691626in}}%
\pgfpathlineto{\pgfqpoint{3.699417in}{0.693175in}}%
\pgfpathlineto{\pgfqpoint{3.699999in}{0.681128in}}%
\pgfpathlineto{\pgfqpoint{3.701018in}{0.667583in}}%
\pgfpathlineto{\pgfqpoint{3.701454in}{0.668395in}}%
\pgfpathlineto{\pgfqpoint{3.702328in}{0.680331in}}%
\pgfpathlineto{\pgfqpoint{3.704948in}{0.710620in}}%
\pgfpathlineto{\pgfqpoint{3.705093in}{0.710485in}}%
\pgfpathlineto{\pgfqpoint{3.705821in}{0.704149in}}%
\pgfpathlineto{\pgfqpoint{3.708004in}{0.679004in}}%
\pgfpathlineto{\pgfqpoint{3.708441in}{0.681997in}}%
\pgfpathlineto{\pgfqpoint{3.710042in}{0.717444in}}%
\pgfpathlineto{\pgfqpoint{3.710915in}{0.709937in}}%
\pgfpathlineto{\pgfqpoint{3.715718in}{0.669546in}}%
\pgfpathlineto{\pgfqpoint{3.719065in}{0.662498in}}%
\pgfpathlineto{\pgfqpoint{3.719357in}{0.662644in}}%
\pgfpathlineto{\pgfqpoint{3.721976in}{0.664967in}}%
\pgfpathlineto{\pgfqpoint{3.725033in}{0.667225in}}%
\pgfpathlineto{\pgfqpoint{3.726197in}{0.669127in}}%
\pgfpathlineto{\pgfqpoint{3.726779in}{0.667414in}}%
\pgfpathlineto{\pgfqpoint{3.727362in}{0.666792in}}%
\pgfpathlineto{\pgfqpoint{3.727944in}{0.667328in}}%
\pgfpathlineto{\pgfqpoint{3.729399in}{0.669343in}}%
\pgfpathlineto{\pgfqpoint{3.730127in}{0.668168in}}%
\pgfpathlineto{\pgfqpoint{3.731582in}{0.668341in}}%
\pgfpathlineto{\pgfqpoint{3.734202in}{0.669929in}}%
\pgfpathlineto{\pgfqpoint{3.738278in}{0.676722in}}%
\pgfpathlineto{\pgfqpoint{3.738423in}{0.676218in}}%
\pgfpathlineto{\pgfqpoint{3.739879in}{0.670428in}}%
\pgfpathlineto{\pgfqpoint{3.740315in}{0.670710in}}%
\pgfpathlineto{\pgfqpoint{3.742062in}{0.674425in}}%
\pgfpathlineto{\pgfqpoint{3.743372in}{0.678055in}}%
\pgfpathlineto{\pgfqpoint{3.743808in}{0.677464in}}%
\pgfpathlineto{\pgfqpoint{3.745118in}{0.671475in}}%
\pgfpathlineto{\pgfqpoint{3.745991in}{0.672494in}}%
\pgfpathlineto{\pgfqpoint{3.746428in}{0.688843in}}%
\pgfpathlineto{\pgfqpoint{3.747156in}{0.839561in}}%
\pgfpathlineto{\pgfqpoint{3.747884in}{0.736517in}}%
\pgfpathlineto{\pgfqpoint{3.749339in}{0.684960in}}%
\pgfpathlineto{\pgfqpoint{3.750212in}{0.671057in}}%
\pgfpathlineto{\pgfqpoint{3.750940in}{0.673256in}}%
\pgfpathlineto{\pgfqpoint{3.751377in}{0.672734in}}%
\pgfpathlineto{\pgfqpoint{3.751522in}{0.673281in}}%
\pgfpathlineto{\pgfqpoint{3.751959in}{0.676260in}}%
\pgfpathlineto{\pgfqpoint{3.752250in}{0.673978in}}%
\pgfpathlineto{\pgfqpoint{3.752978in}{0.666434in}}%
\pgfpathlineto{\pgfqpoint{3.753705in}{0.666861in}}%
\pgfpathlineto{\pgfqpoint{3.754142in}{0.667676in}}%
\pgfpathlineto{\pgfqpoint{3.754579in}{0.680819in}}%
\pgfpathlineto{\pgfqpoint{3.755306in}{0.731363in}}%
\pgfpathlineto{\pgfqpoint{3.755889in}{0.699334in}}%
\pgfpathlineto{\pgfqpoint{3.756325in}{0.685306in}}%
\pgfpathlineto{\pgfqpoint{3.756762in}{0.705512in}}%
\pgfpathlineto{\pgfqpoint{3.757198in}{0.745070in}}%
\pgfpathlineto{\pgfqpoint{3.757781in}{0.698646in}}%
\pgfpathlineto{\pgfqpoint{3.758508in}{0.668196in}}%
\pgfpathlineto{\pgfqpoint{3.759091in}{0.672825in}}%
\pgfpathlineto{\pgfqpoint{3.759382in}{0.674639in}}%
\pgfpathlineto{\pgfqpoint{3.759964in}{0.671609in}}%
\pgfpathlineto{\pgfqpoint{3.761419in}{0.670252in}}%
\pgfpathlineto{\pgfqpoint{3.761856in}{0.670019in}}%
\pgfpathlineto{\pgfqpoint{3.762147in}{0.670690in}}%
\pgfpathlineto{\pgfqpoint{3.762729in}{0.672425in}}%
\pgfpathlineto{\pgfqpoint{3.763166in}{0.670222in}}%
\pgfpathlineto{\pgfqpoint{3.764185in}{0.666288in}}%
\pgfpathlineto{\pgfqpoint{3.764621in}{0.666587in}}%
\pgfpathlineto{\pgfqpoint{3.766077in}{0.668527in}}%
\pgfpathlineto{\pgfqpoint{3.766659in}{0.689039in}}%
\pgfpathlineto{\pgfqpoint{3.767532in}{0.714527in}}%
\pgfpathlineto{\pgfqpoint{3.768114in}{0.711873in}}%
\pgfpathlineto{\pgfqpoint{3.769861in}{0.700544in}}%
\pgfpathlineto{\pgfqpoint{3.770443in}{0.702250in}}%
\pgfpathlineto{\pgfqpoint{3.771316in}{0.707114in}}%
\pgfpathlineto{\pgfqpoint{3.771607in}{0.705398in}}%
\pgfpathlineto{\pgfqpoint{3.775246in}{0.672059in}}%
\pgfpathlineto{\pgfqpoint{3.776556in}{0.674860in}}%
\pgfpathlineto{\pgfqpoint{3.777429in}{0.685244in}}%
\pgfpathlineto{\pgfqpoint{3.778011in}{0.680224in}}%
\pgfpathlineto{\pgfqpoint{3.780049in}{0.669445in}}%
\pgfpathlineto{\pgfqpoint{3.780486in}{0.669148in}}%
\pgfpathlineto{\pgfqpoint{3.780777in}{0.670251in}}%
\pgfpathlineto{\pgfqpoint{3.782815in}{0.676987in}}%
\pgfpathlineto{\pgfqpoint{3.783251in}{0.673059in}}%
\pgfpathlineto{\pgfqpoint{3.784124in}{0.663088in}}%
\pgfpathlineto{\pgfqpoint{3.784852in}{0.664007in}}%
\pgfpathlineto{\pgfqpoint{3.786308in}{0.668067in}}%
\pgfpathlineto{\pgfqpoint{3.787035in}{0.674724in}}%
\pgfpathlineto{\pgfqpoint{3.787618in}{0.671645in}}%
\pgfpathlineto{\pgfqpoint{3.789946in}{0.665749in}}%
\pgfpathlineto{\pgfqpoint{3.791256in}{0.665977in}}%
\pgfpathlineto{\pgfqpoint{3.793003in}{0.669007in}}%
\pgfpathlineto{\pgfqpoint{3.793585in}{0.676169in}}%
\pgfpathlineto{\pgfqpoint{3.794022in}{0.670210in}}%
\pgfpathlineto{\pgfqpoint{3.794895in}{0.664153in}}%
\pgfpathlineto{\pgfqpoint{3.795477in}{0.664440in}}%
\pgfpathlineto{\pgfqpoint{3.797660in}{0.666259in}}%
\pgfpathlineto{\pgfqpoint{3.798533in}{0.669220in}}%
\pgfpathlineto{\pgfqpoint{3.799261in}{0.667505in}}%
\pgfpathlineto{\pgfqpoint{3.799989in}{0.669920in}}%
\pgfpathlineto{\pgfqpoint{3.800426in}{0.668117in}}%
\pgfpathlineto{\pgfqpoint{3.801153in}{0.666288in}}%
\pgfpathlineto{\pgfqpoint{3.801735in}{0.666453in}}%
\pgfpathlineto{\pgfqpoint{3.802463in}{0.667244in}}%
\pgfpathlineto{\pgfqpoint{3.803336in}{0.681454in}}%
\pgfpathlineto{\pgfqpoint{3.803919in}{0.670980in}}%
\pgfpathlineto{\pgfqpoint{3.804792in}{0.671450in}}%
\pgfpathlineto{\pgfqpoint{3.805229in}{0.669392in}}%
\pgfpathlineto{\pgfqpoint{3.806247in}{0.667210in}}%
\pgfpathlineto{\pgfqpoint{3.806684in}{0.667324in}}%
\pgfpathlineto{\pgfqpoint{3.807266in}{0.668805in}}%
\pgfpathlineto{\pgfqpoint{3.808285in}{0.679180in}}%
\pgfpathlineto{\pgfqpoint{3.809013in}{0.673901in}}%
\pgfpathlineto{\pgfqpoint{3.809158in}{0.673669in}}%
\pgfpathlineto{\pgfqpoint{3.809304in}{0.674191in}}%
\pgfpathlineto{\pgfqpoint{3.810323in}{0.686037in}}%
\pgfpathlineto{\pgfqpoint{3.810905in}{0.679791in}}%
\pgfpathlineto{\pgfqpoint{3.812215in}{0.667945in}}%
\pgfpathlineto{\pgfqpoint{3.812651in}{0.669827in}}%
\pgfpathlineto{\pgfqpoint{3.813379in}{0.676344in}}%
\pgfpathlineto{\pgfqpoint{3.813816in}{0.673275in}}%
\pgfpathlineto{\pgfqpoint{3.814689in}{0.665379in}}%
\pgfpathlineto{\pgfqpoint{3.815271in}{0.667015in}}%
\pgfpathlineto{\pgfqpoint{3.815562in}{0.667766in}}%
\pgfpathlineto{\pgfqpoint{3.816144in}{0.666256in}}%
\pgfpathlineto{\pgfqpoint{3.816727in}{0.665760in}}%
\pgfpathlineto{\pgfqpoint{3.817309in}{0.666381in}}%
\pgfpathlineto{\pgfqpoint{3.819055in}{0.670000in}}%
\pgfpathlineto{\pgfqpoint{3.819783in}{0.685995in}}%
\pgfpathlineto{\pgfqpoint{3.820220in}{0.671245in}}%
\pgfpathlineto{\pgfqpoint{3.820656in}{0.666805in}}%
\pgfpathlineto{\pgfqpoint{3.821093in}{0.671180in}}%
\pgfpathlineto{\pgfqpoint{3.821530in}{0.675607in}}%
\pgfpathlineto{\pgfqpoint{3.821966in}{0.670420in}}%
\pgfpathlineto{\pgfqpoint{3.822985in}{0.666876in}}%
\pgfpathlineto{\pgfqpoint{3.823276in}{0.667359in}}%
\pgfpathlineto{\pgfqpoint{3.823858in}{0.678475in}}%
\pgfpathlineto{\pgfqpoint{3.824732in}{0.713128in}}%
\pgfpathlineto{\pgfqpoint{3.825605in}{0.711397in}}%
\pgfpathlineto{\pgfqpoint{3.825896in}{0.712159in}}%
\pgfpathlineto{\pgfqpoint{3.826187in}{0.709600in}}%
\pgfpathlineto{\pgfqpoint{3.826624in}{0.704613in}}%
\pgfpathlineto{\pgfqpoint{3.826915in}{0.710381in}}%
\pgfpathlineto{\pgfqpoint{3.827497in}{0.734311in}}%
\pgfpathlineto{\pgfqpoint{3.827934in}{0.708425in}}%
\pgfpathlineto{\pgfqpoint{3.828952in}{0.667128in}}%
\pgfpathlineto{\pgfqpoint{3.829389in}{0.667586in}}%
\pgfpathlineto{\pgfqpoint{3.829826in}{0.673174in}}%
\pgfpathlineto{\pgfqpoint{3.830845in}{0.763069in}}%
\pgfpathlineto{\pgfqpoint{3.831863in}{0.718621in}}%
\pgfpathlineto{\pgfqpoint{3.833610in}{0.677748in}}%
\pgfpathlineto{\pgfqpoint{3.834483in}{0.669839in}}%
\pgfpathlineto{\pgfqpoint{3.834920in}{0.670790in}}%
\pgfpathlineto{\pgfqpoint{3.835648in}{0.681866in}}%
\pgfpathlineto{\pgfqpoint{3.837394in}{0.732597in}}%
\pgfpathlineto{\pgfqpoint{3.838122in}{0.719146in}}%
\pgfpathlineto{\pgfqpoint{3.841178in}{0.668674in}}%
\pgfpathlineto{\pgfqpoint{3.842052in}{0.670767in}}%
\pgfpathlineto{\pgfqpoint{3.842634in}{0.700104in}}%
\pgfpathlineto{\pgfqpoint{3.843216in}{0.734754in}}%
\pgfpathlineto{\pgfqpoint{3.843798in}{0.715933in}}%
\pgfpathlineto{\pgfqpoint{3.844526in}{0.701658in}}%
\pgfpathlineto{\pgfqpoint{3.845254in}{0.701684in}}%
\pgfpathlineto{\pgfqpoint{3.846127in}{0.680302in}}%
\pgfpathlineto{\pgfqpoint{3.847291in}{0.667654in}}%
\pgfpathlineto{\pgfqpoint{3.847582in}{0.667746in}}%
\pgfpathlineto{\pgfqpoint{3.848019in}{0.669061in}}%
\pgfpathlineto{\pgfqpoint{3.848892in}{0.678201in}}%
\pgfpathlineto{\pgfqpoint{3.849620in}{0.674201in}}%
\pgfpathlineto{\pgfqpoint{3.849911in}{0.673752in}}%
\pgfpathlineto{\pgfqpoint{3.850493in}{0.674807in}}%
\pgfpathlineto{\pgfqpoint{3.850784in}{0.674848in}}%
\pgfpathlineto{\pgfqpoint{3.850930in}{0.674490in}}%
\pgfpathlineto{\pgfqpoint{3.852676in}{0.665068in}}%
\pgfpathlineto{\pgfqpoint{3.853695in}{0.666167in}}%
\pgfpathlineto{\pgfqpoint{3.854423in}{0.667895in}}%
\pgfpathlineto{\pgfqpoint{3.855296in}{0.667562in}}%
\pgfpathlineto{\pgfqpoint{3.856024in}{0.668544in}}%
\pgfpathlineto{\pgfqpoint{3.856606in}{0.667699in}}%
\pgfpathlineto{\pgfqpoint{3.858062in}{0.667756in}}%
\pgfpathlineto{\pgfqpoint{3.861118in}{0.669870in}}%
\pgfpathlineto{\pgfqpoint{3.861555in}{0.670026in}}%
\pgfpathlineto{\pgfqpoint{3.861846in}{0.669482in}}%
\pgfpathlineto{\pgfqpoint{3.862574in}{0.668556in}}%
\pgfpathlineto{\pgfqpoint{3.863156in}{0.668969in}}%
\pgfpathlineto{\pgfqpoint{3.864320in}{0.669439in}}%
\pgfpathlineto{\pgfqpoint{3.865048in}{0.676772in}}%
\pgfpathlineto{\pgfqpoint{3.865630in}{0.671651in}}%
\pgfpathlineto{\pgfqpoint{3.865776in}{0.671381in}}%
\pgfpathlineto{\pgfqpoint{3.866067in}{0.673154in}}%
\pgfpathlineto{\pgfqpoint{3.866649in}{0.681466in}}%
\pgfpathlineto{\pgfqpoint{3.867231in}{0.674103in}}%
\pgfpathlineto{\pgfqpoint{3.867959in}{0.669208in}}%
\pgfpathlineto{\pgfqpoint{3.868541in}{0.669508in}}%
\pgfpathlineto{\pgfqpoint{3.869705in}{0.670561in}}%
\pgfpathlineto{\pgfqpoint{3.870433in}{0.676106in}}%
\pgfpathlineto{\pgfqpoint{3.871015in}{0.672146in}}%
\pgfpathlineto{\pgfqpoint{3.871161in}{0.671954in}}%
\pgfpathlineto{\pgfqpoint{3.871452in}{0.673110in}}%
\pgfpathlineto{\pgfqpoint{3.871888in}{0.675533in}}%
\pgfpathlineto{\pgfqpoint{3.872325in}{0.672790in}}%
\pgfpathlineto{\pgfqpoint{3.873781in}{0.668645in}}%
\pgfpathlineto{\pgfqpoint{3.874363in}{0.668819in}}%
\pgfpathlineto{\pgfqpoint{3.874799in}{0.675887in}}%
\pgfpathlineto{\pgfqpoint{3.875236in}{0.688861in}}%
\pgfpathlineto{\pgfqpoint{3.875818in}{0.676522in}}%
\pgfpathlineto{\pgfqpoint{3.875964in}{0.674647in}}%
\pgfpathlineto{\pgfqpoint{3.876400in}{0.680949in}}%
\pgfpathlineto{\pgfqpoint{3.876837in}{0.694976in}}%
\pgfpathlineto{\pgfqpoint{3.877274in}{0.683060in}}%
\pgfpathlineto{\pgfqpoint{3.878147in}{0.666184in}}%
\pgfpathlineto{\pgfqpoint{3.878729in}{0.667018in}}%
\pgfpathlineto{\pgfqpoint{3.879166in}{0.667761in}}%
\pgfpathlineto{\pgfqpoint{3.879893in}{0.667022in}}%
\pgfpathlineto{\pgfqpoint{3.880330in}{0.669093in}}%
\pgfpathlineto{\pgfqpoint{3.881203in}{0.685135in}}%
\pgfpathlineto{\pgfqpoint{3.881786in}{0.676266in}}%
\pgfpathlineto{\pgfqpoint{3.882222in}{0.671908in}}%
\pgfpathlineto{\pgfqpoint{3.882659in}{0.677475in}}%
\pgfpathlineto{\pgfqpoint{3.883241in}{0.696829in}}%
\pgfpathlineto{\pgfqpoint{3.883678in}{0.682213in}}%
\pgfpathlineto{\pgfqpoint{3.884405in}{0.666417in}}%
\pgfpathlineto{\pgfqpoint{3.884988in}{0.667240in}}%
\pgfpathlineto{\pgfqpoint{3.886589in}{0.670324in}}%
\pgfpathlineto{\pgfqpoint{3.887171in}{0.684818in}}%
\pgfpathlineto{\pgfqpoint{3.887753in}{0.674189in}}%
\pgfpathlineto{\pgfqpoint{3.888044in}{0.672223in}}%
\pgfpathlineto{\pgfqpoint{3.888335in}{0.677959in}}%
\pgfpathlineto{\pgfqpoint{3.888917in}{0.700304in}}%
\pgfpathlineto{\pgfqpoint{3.889354in}{0.682836in}}%
\pgfpathlineto{\pgfqpoint{3.890082in}{0.667357in}}%
\pgfpathlineto{\pgfqpoint{3.890664in}{0.669391in}}%
\pgfpathlineto{\pgfqpoint{3.890809in}{0.669791in}}%
\pgfpathlineto{\pgfqpoint{3.891246in}{0.668519in}}%
\pgfpathlineto{\pgfqpoint{3.891828in}{0.667878in}}%
\pgfpathlineto{\pgfqpoint{3.892119in}{0.668528in}}%
\pgfpathlineto{\pgfqpoint{3.892702in}{0.685403in}}%
\pgfpathlineto{\pgfqpoint{3.893429in}{0.718270in}}%
\pgfpathlineto{\pgfqpoint{3.894011in}{0.699754in}}%
\pgfpathlineto{\pgfqpoint{3.894448in}{0.691129in}}%
\pgfpathlineto{\pgfqpoint{3.894885in}{0.706443in}}%
\pgfpathlineto{\pgfqpoint{3.895612in}{0.762034in}}%
\pgfpathlineto{\pgfqpoint{3.896049in}{0.725495in}}%
\pgfpathlineto{\pgfqpoint{3.897068in}{0.670884in}}%
\pgfpathlineto{\pgfqpoint{3.897505in}{0.671070in}}%
\pgfpathlineto{\pgfqpoint{3.899397in}{0.669084in}}%
\pgfpathlineto{\pgfqpoint{3.899833in}{0.669456in}}%
\pgfpathlineto{\pgfqpoint{3.902453in}{0.673244in}}%
\pgfpathlineto{\pgfqpoint{3.902890in}{0.671664in}}%
\pgfpathlineto{\pgfqpoint{3.903472in}{0.670492in}}%
\pgfpathlineto{\pgfqpoint{3.904054in}{0.671123in}}%
\pgfpathlineto{\pgfqpoint{3.904782in}{0.671715in}}%
\pgfpathlineto{\pgfqpoint{3.905218in}{0.671038in}}%
\pgfpathlineto{\pgfqpoint{3.906674in}{0.670270in}}%
\pgfpathlineto{\pgfqpoint{3.907838in}{0.671383in}}%
\pgfpathlineto{\pgfqpoint{3.908712in}{0.679797in}}%
\pgfpathlineto{\pgfqpoint{3.909294in}{0.685524in}}%
\pgfpathlineto{\pgfqpoint{3.909876in}{0.682833in}}%
\pgfpathlineto{\pgfqpoint{3.910021in}{0.682675in}}%
\pgfpathlineto{\pgfqpoint{3.910167in}{0.683134in}}%
\pgfpathlineto{\pgfqpoint{3.910749in}{0.686971in}}%
\pgfpathlineto{\pgfqpoint{3.911040in}{0.684722in}}%
\pgfpathlineto{\pgfqpoint{3.912205in}{0.670924in}}%
\pgfpathlineto{\pgfqpoint{3.912641in}{0.672413in}}%
\pgfpathlineto{\pgfqpoint{3.913369in}{0.693582in}}%
\pgfpathlineto{\pgfqpoint{3.913806in}{0.704804in}}%
\pgfpathlineto{\pgfqpoint{3.914388in}{0.693591in}}%
\pgfpathlineto{\pgfqpoint{3.916862in}{0.670796in}}%
\pgfpathlineto{\pgfqpoint{3.917444in}{0.670257in}}%
\pgfpathlineto{\pgfqpoint{3.917735in}{0.670708in}}%
\pgfpathlineto{\pgfqpoint{3.918318in}{0.677999in}}%
\pgfpathlineto{\pgfqpoint{3.919045in}{0.697342in}}%
\pgfpathlineto{\pgfqpoint{3.919482in}{0.686970in}}%
\pgfpathlineto{\pgfqpoint{3.920937in}{0.673093in}}%
\pgfpathlineto{\pgfqpoint{3.921665in}{0.671675in}}%
\pgfpathlineto{\pgfqpoint{3.922247in}{0.672102in}}%
\pgfpathlineto{\pgfqpoint{3.924139in}{0.675511in}}%
\pgfpathlineto{\pgfqpoint{3.924722in}{0.690217in}}%
\pgfpathlineto{\pgfqpoint{3.925449in}{0.719248in}}%
\pgfpathlineto{\pgfqpoint{3.926031in}{0.702780in}}%
\pgfpathlineto{\pgfqpoint{3.926614in}{0.693610in}}%
\pgfpathlineto{\pgfqpoint{3.927196in}{0.697400in}}%
\pgfpathlineto{\pgfqpoint{3.928069in}{0.703841in}}%
\pgfpathlineto{\pgfqpoint{3.928506in}{0.701899in}}%
\pgfpathlineto{\pgfqpoint{3.933454in}{0.663763in}}%
\pgfpathlineto{\pgfqpoint{3.933891in}{0.664536in}}%
\pgfpathlineto{\pgfqpoint{3.936074in}{0.682590in}}%
\pgfpathlineto{\pgfqpoint{3.936656in}{0.711471in}}%
\pgfpathlineto{\pgfqpoint{3.937093in}{0.691237in}}%
\pgfpathlineto{\pgfqpoint{3.937966in}{0.652685in}}%
\pgfpathlineto{\pgfqpoint{3.938548in}{0.654379in}}%
\pgfpathlineto{\pgfqpoint{3.941750in}{0.670469in}}%
\pgfpathlineto{\pgfqpoint{3.942333in}{0.682343in}}%
\pgfpathlineto{\pgfqpoint{3.942769in}{0.671276in}}%
\pgfpathlineto{\pgfqpoint{3.943351in}{0.658040in}}%
\pgfpathlineto{\pgfqpoint{3.943934in}{0.662543in}}%
\pgfpathlineto{\pgfqpoint{3.944516in}{0.676887in}}%
\pgfpathlineto{\pgfqpoint{3.945098in}{0.664984in}}%
\pgfpathlineto{\pgfqpoint{3.945535in}{0.660180in}}%
\pgfpathlineto{\pgfqpoint{3.946262in}{0.663636in}}%
\pgfpathlineto{\pgfqpoint{3.946699in}{0.665402in}}%
\pgfpathlineto{\pgfqpoint{3.947136in}{0.663085in}}%
\pgfpathlineto{\pgfqpoint{3.947863in}{0.660752in}}%
\pgfpathlineto{\pgfqpoint{3.948446in}{0.661438in}}%
\pgfpathlineto{\pgfqpoint{3.951211in}{0.667075in}}%
\pgfpathlineto{\pgfqpoint{3.951793in}{0.664898in}}%
\pgfpathlineto{\pgfqpoint{3.952230in}{0.665042in}}%
\pgfpathlineto{\pgfqpoint{3.952521in}{0.665441in}}%
\pgfpathlineto{\pgfqpoint{3.955432in}{0.667864in}}%
\pgfpathlineto{\pgfqpoint{3.956014in}{0.669342in}}%
\pgfpathlineto{\pgfqpoint{3.957033in}{0.682117in}}%
\pgfpathlineto{\pgfqpoint{3.957906in}{0.676941in}}%
\pgfpathlineto{\pgfqpoint{3.958197in}{0.677925in}}%
\pgfpathlineto{\pgfqpoint{3.958925in}{0.706501in}}%
\pgfpathlineto{\pgfqpoint{3.959216in}{0.714252in}}%
\pgfpathlineto{\pgfqpoint{3.959653in}{0.695630in}}%
\pgfpathlineto{\pgfqpoint{3.960671in}{0.668624in}}%
\pgfpathlineto{\pgfqpoint{3.961108in}{0.668898in}}%
\pgfpathlineto{\pgfqpoint{3.961981in}{0.668506in}}%
\pgfpathlineto{\pgfqpoint{3.962127in}{0.668904in}}%
\pgfpathlineto{\pgfqpoint{3.963146in}{0.678351in}}%
\pgfpathlineto{\pgfqpoint{3.964601in}{0.676750in}}%
\pgfpathlineto{\pgfqpoint{3.965620in}{0.673029in}}%
\pgfpathlineto{\pgfqpoint{3.966057in}{0.675322in}}%
\pgfpathlineto{\pgfqpoint{3.966784in}{0.705382in}}%
\pgfpathlineto{\pgfqpoint{3.967221in}{0.722159in}}%
\pgfpathlineto{\pgfqpoint{3.967803in}{0.704704in}}%
\pgfpathlineto{\pgfqpoint{3.969695in}{0.676214in}}%
\pgfpathlineto{\pgfqpoint{3.971296in}{0.668114in}}%
\pgfpathlineto{\pgfqpoint{3.971878in}{0.667802in}}%
\pgfpathlineto{\pgfqpoint{3.972461in}{0.668194in}}%
\pgfpathlineto{\pgfqpoint{3.974062in}{0.669554in}}%
\pgfpathlineto{\pgfqpoint{3.974644in}{0.674772in}}%
\pgfpathlineto{\pgfqpoint{3.975663in}{0.702343in}}%
\pgfpathlineto{\pgfqpoint{3.976390in}{0.692693in}}%
\pgfpathlineto{\pgfqpoint{3.979883in}{0.668362in}}%
\pgfpathlineto{\pgfqpoint{3.980466in}{0.669081in}}%
\pgfpathlineto{\pgfqpoint{3.981193in}{0.677112in}}%
\pgfpathlineto{\pgfqpoint{3.982212in}{0.688529in}}%
\pgfpathlineto{\pgfqpoint{3.982794in}{0.687564in}}%
\pgfpathlineto{\pgfqpoint{3.983231in}{0.686608in}}%
\pgfpathlineto{\pgfqpoint{3.983522in}{0.688612in}}%
\pgfpathlineto{\pgfqpoint{3.984541in}{0.728518in}}%
\pgfpathlineto{\pgfqpoint{3.984977in}{0.700692in}}%
\pgfpathlineto{\pgfqpoint{3.985996in}{0.665976in}}%
\pgfpathlineto{\pgfqpoint{3.986433in}{0.666742in}}%
\pgfpathlineto{\pgfqpoint{3.986724in}{0.666857in}}%
\pgfpathlineto{\pgfqpoint{3.987161in}{0.666126in}}%
\pgfpathlineto{\pgfqpoint{3.987597in}{0.665840in}}%
\pgfpathlineto{\pgfqpoint{3.988179in}{0.666499in}}%
\pgfpathlineto{\pgfqpoint{3.990799in}{0.673999in}}%
\pgfpathlineto{\pgfqpoint{3.991090in}{0.671997in}}%
\pgfpathlineto{\pgfqpoint{3.992109in}{0.666732in}}%
\pgfpathlineto{\pgfqpoint{3.992546in}{0.667065in}}%
\pgfpathlineto{\pgfqpoint{3.996621in}{0.669188in}}%
\pgfpathlineto{\pgfqpoint{4.002297in}{0.673011in}}%
\pgfpathlineto{\pgfqpoint{4.002589in}{0.673855in}}%
\pgfpathlineto{\pgfqpoint{4.003025in}{0.672049in}}%
\pgfpathlineto{\pgfqpoint{4.004335in}{0.670994in}}%
\pgfpathlineto{\pgfqpoint{4.005499in}{0.671883in}}%
\pgfpathlineto{\pgfqpoint{4.005936in}{0.672290in}}%
\pgfpathlineto{\pgfqpoint{4.006518in}{0.671492in}}%
\pgfpathlineto{\pgfqpoint{4.008119in}{0.671574in}}%
\pgfpathlineto{\pgfqpoint{4.011903in}{0.673398in}}%
\pgfpathlineto{\pgfqpoint{4.012631in}{0.676327in}}%
\pgfpathlineto{\pgfqpoint{4.013504in}{0.695913in}}%
\pgfpathlineto{\pgfqpoint{4.014378in}{0.776729in}}%
\pgfpathlineto{\pgfqpoint{4.015105in}{0.733455in}}%
\pgfpathlineto{\pgfqpoint{4.017143in}{0.671572in}}%
\pgfpathlineto{\pgfqpoint{4.018453in}{0.674267in}}%
\pgfpathlineto{\pgfqpoint{4.018890in}{0.710108in}}%
\pgfpathlineto{\pgfqpoint{4.019617in}{0.899898in}}%
\pgfpathlineto{\pgfqpoint{4.020200in}{0.784639in}}%
\pgfpathlineto{\pgfqpoint{4.021801in}{0.700244in}}%
\pgfpathlineto{\pgfqpoint{4.023693in}{0.684259in}}%
\pgfpathlineto{\pgfqpoint{4.023838in}{0.684384in}}%
\pgfpathlineto{\pgfqpoint{4.024857in}{0.689180in}}%
\pgfpathlineto{\pgfqpoint{4.025439in}{0.686803in}}%
\pgfpathlineto{\pgfqpoint{4.028205in}{0.674903in}}%
\pgfpathlineto{\pgfqpoint{4.028496in}{0.674699in}}%
\pgfpathlineto{\pgfqpoint{4.028787in}{0.675783in}}%
\pgfpathlineto{\pgfqpoint{4.029514in}{0.690821in}}%
\pgfpathlineto{\pgfqpoint{4.030388in}{0.707692in}}%
\pgfpathlineto{\pgfqpoint{4.030970in}{0.705254in}}%
\pgfpathlineto{\pgfqpoint{4.033299in}{0.688257in}}%
\pgfpathlineto{\pgfqpoint{4.035045in}{0.668897in}}%
\pgfpathlineto{\pgfqpoint{4.035773in}{0.668845in}}%
\pgfpathlineto{\pgfqpoint{4.036064in}{0.669397in}}%
\pgfpathlineto{\pgfqpoint{4.037374in}{0.674322in}}%
\pgfpathlineto{\pgfqpoint{4.038247in}{0.673009in}}%
\pgfpathlineto{\pgfqpoint{4.039848in}{0.666891in}}%
\pgfpathlineto{\pgfqpoint{4.041595in}{0.663209in}}%
\pgfpathlineto{\pgfqpoint{4.042031in}{0.663516in}}%
\pgfpathlineto{\pgfqpoint{4.045961in}{0.670268in}}%
\pgfpathlineto{\pgfqpoint{4.046107in}{0.670838in}}%
\pgfpathlineto{\pgfqpoint{4.046543in}{0.668153in}}%
\pgfpathlineto{\pgfqpoint{4.047562in}{0.663208in}}%
\pgfpathlineto{\pgfqpoint{4.047999in}{0.663454in}}%
\pgfpathlineto{\pgfqpoint{4.052802in}{0.667596in}}%
\pgfpathlineto{\pgfqpoint{4.059206in}{0.669855in}}%
\pgfpathlineto{\pgfqpoint{4.061826in}{0.669388in}}%
\pgfpathlineto{\pgfqpoint{4.063136in}{0.668067in}}%
\pgfpathlineto{\pgfqpoint{4.063718in}{0.668904in}}%
\pgfpathlineto{\pgfqpoint{4.065464in}{0.671321in}}%
\pgfpathlineto{\pgfqpoint{4.065755in}{0.670850in}}%
\pgfpathlineto{\pgfqpoint{4.067793in}{0.667485in}}%
\pgfpathlineto{\pgfqpoint{4.069103in}{0.667843in}}%
\pgfpathlineto{\pgfqpoint{4.070849in}{0.668081in}}%
\pgfpathlineto{\pgfqpoint{4.073760in}{0.669676in}}%
\pgfpathlineto{\pgfqpoint{4.074343in}{0.672368in}}%
\pgfpathlineto{\pgfqpoint{4.075944in}{0.689340in}}%
\pgfpathlineto{\pgfqpoint{4.076962in}{0.687180in}}%
\pgfpathlineto{\pgfqpoint{4.077399in}{0.683984in}}%
\pgfpathlineto{\pgfqpoint{4.078854in}{0.666074in}}%
\pgfpathlineto{\pgfqpoint{4.079291in}{0.666951in}}%
\pgfpathlineto{\pgfqpoint{4.080019in}{0.676260in}}%
\pgfpathlineto{\pgfqpoint{4.080892in}{0.723429in}}%
\pgfpathlineto{\pgfqpoint{4.081329in}{0.751505in}}%
\pgfpathlineto{\pgfqpoint{4.081911in}{0.728699in}}%
\pgfpathlineto{\pgfqpoint{4.082202in}{0.718697in}}%
\pgfpathlineto{\pgfqpoint{4.082784in}{0.736852in}}%
\pgfpathlineto{\pgfqpoint{4.082930in}{0.739351in}}%
\pgfpathlineto{\pgfqpoint{4.083221in}{0.727708in}}%
\pgfpathlineto{\pgfqpoint{4.084385in}{0.668344in}}%
\pgfpathlineto{\pgfqpoint{4.084967in}{0.668749in}}%
\pgfpathlineto{\pgfqpoint{4.086568in}{0.673634in}}%
\pgfpathlineto{\pgfqpoint{4.087442in}{0.737561in}}%
\pgfpathlineto{\pgfqpoint{4.088169in}{0.698551in}}%
\pgfpathlineto{\pgfqpoint{4.089770in}{0.678809in}}%
\pgfpathlineto{\pgfqpoint{4.091226in}{0.672351in}}%
\pgfpathlineto{\pgfqpoint{4.091371in}{0.672367in}}%
\pgfpathlineto{\pgfqpoint{4.092390in}{0.672810in}}%
\pgfpathlineto{\pgfqpoint{4.092827in}{0.682638in}}%
\pgfpathlineto{\pgfqpoint{4.093846in}{0.778985in}}%
\pgfpathlineto{\pgfqpoint{4.094428in}{0.735451in}}%
\pgfpathlineto{\pgfqpoint{4.096029in}{0.699211in}}%
\pgfpathlineto{\pgfqpoint{4.099376in}{0.669826in}}%
\pgfpathlineto{\pgfqpoint{4.099667in}{0.670056in}}%
\pgfpathlineto{\pgfqpoint{4.100541in}{0.673646in}}%
\pgfpathlineto{\pgfqpoint{4.101560in}{0.692063in}}%
\pgfpathlineto{\pgfqpoint{4.102142in}{0.700696in}}%
\pgfpathlineto{\pgfqpoint{4.102724in}{0.696706in}}%
\pgfpathlineto{\pgfqpoint{4.102869in}{0.696078in}}%
\pgfpathlineto{\pgfqpoint{4.103306in}{0.699074in}}%
\pgfpathlineto{\pgfqpoint{4.103743in}{0.702795in}}%
\pgfpathlineto{\pgfqpoint{4.104034in}{0.697786in}}%
\pgfpathlineto{\pgfqpoint{4.105344in}{0.666270in}}%
\pgfpathlineto{\pgfqpoint{4.105926in}{0.666589in}}%
\pgfpathlineto{\pgfqpoint{4.107672in}{0.669131in}}%
\pgfpathlineto{\pgfqpoint{4.107964in}{0.669487in}}%
\pgfpathlineto{\pgfqpoint{4.108546in}{0.668671in}}%
\pgfpathlineto{\pgfqpoint{4.110875in}{0.667704in}}%
\pgfpathlineto{\pgfqpoint{4.112476in}{0.668628in}}%
\pgfpathlineto{\pgfqpoint{4.116260in}{0.671725in}}%
\pgfpathlineto{\pgfqpoint{4.116987in}{0.675475in}}%
\pgfpathlineto{\pgfqpoint{4.117570in}{0.673617in}}%
\pgfpathlineto{\pgfqpoint{4.119607in}{0.670704in}}%
\pgfpathlineto{\pgfqpoint{4.120626in}{0.671755in}}%
\pgfpathlineto{\pgfqpoint{4.121354in}{0.677862in}}%
\pgfpathlineto{\pgfqpoint{4.121790in}{0.672279in}}%
\pgfpathlineto{\pgfqpoint{4.122373in}{0.669089in}}%
\pgfpathlineto{\pgfqpoint{4.123100in}{0.669655in}}%
\pgfpathlineto{\pgfqpoint{4.123828in}{0.669298in}}%
\pgfpathlineto{\pgfqpoint{4.123974in}{0.669522in}}%
\pgfpathlineto{\pgfqpoint{4.124556in}{0.677003in}}%
\pgfpathlineto{\pgfqpoint{4.125575in}{0.689818in}}%
\pgfpathlineto{\pgfqpoint{4.126157in}{0.688366in}}%
\pgfpathlineto{\pgfqpoint{4.127467in}{0.682605in}}%
\pgfpathlineto{\pgfqpoint{4.128340in}{0.677554in}}%
\pgfpathlineto{\pgfqpoint{4.128777in}{0.680490in}}%
\pgfpathlineto{\pgfqpoint{4.130087in}{0.710196in}}%
\pgfpathlineto{\pgfqpoint{4.130814in}{0.699904in}}%
\pgfpathlineto{\pgfqpoint{4.132852in}{0.683282in}}%
\pgfpathlineto{\pgfqpoint{4.134890in}{0.668591in}}%
\pgfpathlineto{\pgfqpoint{4.135181in}{0.668727in}}%
\pgfpathlineto{\pgfqpoint{4.135617in}{0.671644in}}%
\pgfpathlineto{\pgfqpoint{4.136927in}{0.722480in}}%
\pgfpathlineto{\pgfqpoint{4.138237in}{0.705210in}}%
\pgfpathlineto{\pgfqpoint{4.138383in}{0.704342in}}%
\pgfpathlineto{\pgfqpoint{4.138674in}{0.706765in}}%
\pgfpathlineto{\pgfqpoint{4.139401in}{0.725991in}}%
\pgfpathlineto{\pgfqpoint{4.139838in}{0.709377in}}%
\pgfpathlineto{\pgfqpoint{4.141439in}{0.667816in}}%
\pgfpathlineto{\pgfqpoint{4.144059in}{0.668714in}}%
\pgfpathlineto{\pgfqpoint{4.144641in}{0.669633in}}%
\pgfpathlineto{\pgfqpoint{4.145223in}{0.668825in}}%
\pgfpathlineto{\pgfqpoint{4.145660in}{0.669632in}}%
\pgfpathlineto{\pgfqpoint{4.146388in}{0.680193in}}%
\pgfpathlineto{\pgfqpoint{4.146824in}{0.686947in}}%
\pgfpathlineto{\pgfqpoint{4.147261in}{0.681204in}}%
\pgfpathlineto{\pgfqpoint{4.148134in}{0.668650in}}%
\pgfpathlineto{\pgfqpoint{4.148716in}{0.669652in}}%
\pgfpathlineto{\pgfqpoint{4.149153in}{0.668477in}}%
\pgfpathlineto{\pgfqpoint{4.150026in}{0.666309in}}%
\pgfpathlineto{\pgfqpoint{4.150608in}{0.666722in}}%
\pgfpathlineto{\pgfqpoint{4.152937in}{0.667291in}}%
\pgfpathlineto{\pgfqpoint{4.154102in}{0.668060in}}%
\pgfpathlineto{\pgfqpoint{4.155994in}{0.674456in}}%
\pgfpathlineto{\pgfqpoint{4.157012in}{0.700485in}}%
\pgfpathlineto{\pgfqpoint{4.157595in}{0.688241in}}%
\pgfpathlineto{\pgfqpoint{4.159196in}{0.676370in}}%
\pgfpathlineto{\pgfqpoint{4.161961in}{0.669636in}}%
\pgfpathlineto{\pgfqpoint{4.165163in}{0.670039in}}%
\pgfpathlineto{\pgfqpoint{4.167346in}{0.674234in}}%
\pgfpathlineto{\pgfqpoint{4.167783in}{0.681648in}}%
\pgfpathlineto{\pgfqpoint{4.168656in}{0.732439in}}%
\pgfpathlineto{\pgfqpoint{4.169238in}{0.702041in}}%
\pgfpathlineto{\pgfqpoint{4.169966in}{0.675697in}}%
\pgfpathlineto{\pgfqpoint{4.170548in}{0.680010in}}%
\pgfpathlineto{\pgfqpoint{4.170985in}{0.674048in}}%
\pgfpathlineto{\pgfqpoint{4.171567in}{0.668395in}}%
\pgfpathlineto{\pgfqpoint{4.172149in}{0.671477in}}%
\pgfpathlineto{\pgfqpoint{4.173023in}{0.670296in}}%
\pgfpathlineto{\pgfqpoint{4.173750in}{0.687600in}}%
\pgfpathlineto{\pgfqpoint{4.174187in}{0.698712in}}%
\pgfpathlineto{\pgfqpoint{4.174624in}{0.685110in}}%
\pgfpathlineto{\pgfqpoint{4.175642in}{0.669643in}}%
\pgfpathlineto{\pgfqpoint{4.176079in}{0.669872in}}%
\pgfpathlineto{\pgfqpoint{4.178553in}{0.669092in}}%
\pgfpathlineto{\pgfqpoint{4.178699in}{0.669417in}}%
\pgfpathlineto{\pgfqpoint{4.179572in}{0.677546in}}%
\pgfpathlineto{\pgfqpoint{4.180009in}{0.672110in}}%
\pgfpathlineto{\pgfqpoint{4.180591in}{0.668417in}}%
\pgfpathlineto{\pgfqpoint{4.181173in}{0.670811in}}%
\pgfpathlineto{\pgfqpoint{4.181319in}{0.671322in}}%
\pgfpathlineto{\pgfqpoint{4.181755in}{0.669667in}}%
\pgfpathlineto{\pgfqpoint{4.182483in}{0.667616in}}%
\pgfpathlineto{\pgfqpoint{4.182920in}{0.668267in}}%
\pgfpathlineto{\pgfqpoint{4.183502in}{0.670172in}}%
\pgfpathlineto{\pgfqpoint{4.183938in}{0.668582in}}%
\pgfpathlineto{\pgfqpoint{4.184375in}{0.668065in}}%
\pgfpathlineto{\pgfqpoint{4.185103in}{0.668599in}}%
\pgfpathlineto{\pgfqpoint{4.189760in}{0.670570in}}%
\pgfpathlineto{\pgfqpoint{4.190634in}{0.674632in}}%
\pgfpathlineto{\pgfqpoint{4.191652in}{0.682207in}}%
\pgfpathlineto{\pgfqpoint{4.192235in}{0.680567in}}%
\pgfpathlineto{\pgfqpoint{4.193981in}{0.678487in}}%
\pgfpathlineto{\pgfqpoint{4.194272in}{0.678809in}}%
\pgfpathlineto{\pgfqpoint{4.194563in}{0.679268in}}%
\pgfpathlineto{\pgfqpoint{4.194854in}{0.678621in}}%
\pgfpathlineto{\pgfqpoint{4.196455in}{0.668965in}}%
\pgfpathlineto{\pgfqpoint{4.197183in}{0.669382in}}%
\pgfpathlineto{\pgfqpoint{4.197765in}{0.670758in}}%
\pgfpathlineto{\pgfqpoint{4.198347in}{0.687553in}}%
\pgfpathlineto{\pgfqpoint{4.199221in}{0.719625in}}%
\pgfpathlineto{\pgfqpoint{4.199803in}{0.716438in}}%
\pgfpathlineto{\pgfqpoint{4.202277in}{0.687651in}}%
\pgfpathlineto{\pgfqpoint{4.203587in}{0.673132in}}%
\pgfpathlineto{\pgfqpoint{4.203878in}{0.673310in}}%
\pgfpathlineto{\pgfqpoint{4.205334in}{0.675841in}}%
\pgfpathlineto{\pgfqpoint{4.207080in}{0.682661in}}%
\pgfpathlineto{\pgfqpoint{4.207662in}{0.681590in}}%
\pgfpathlineto{\pgfqpoint{4.208536in}{0.679034in}}%
\pgfpathlineto{\pgfqpoint{4.208827in}{0.680045in}}%
\pgfpathlineto{\pgfqpoint{4.209409in}{0.699314in}}%
\pgfpathlineto{\pgfqpoint{4.209846in}{0.716379in}}%
\pgfpathlineto{\pgfqpoint{4.210282in}{0.699877in}}%
\pgfpathlineto{\pgfqpoint{4.211301in}{0.672098in}}%
\pgfpathlineto{\pgfqpoint{4.211738in}{0.674011in}}%
\pgfpathlineto{\pgfqpoint{4.214066in}{0.693708in}}%
\pgfpathlineto{\pgfqpoint{4.214212in}{0.692604in}}%
\pgfpathlineto{\pgfqpoint{4.216395in}{0.662084in}}%
\pgfpathlineto{\pgfqpoint{4.216686in}{0.662130in}}%
\pgfpathlineto{\pgfqpoint{4.217559in}{0.663651in}}%
\pgfpathlineto{\pgfqpoint{4.218433in}{0.666177in}}%
\pgfpathlineto{\pgfqpoint{4.218869in}{0.664932in}}%
\pgfpathlineto{\pgfqpoint{4.220470in}{0.663108in}}%
\pgfpathlineto{\pgfqpoint{4.220907in}{0.664155in}}%
\pgfpathlineto{\pgfqpoint{4.221926in}{0.671802in}}%
\pgfpathlineto{\pgfqpoint{4.222799in}{0.670813in}}%
\pgfpathlineto{\pgfqpoint{4.223236in}{0.671128in}}%
\pgfpathlineto{\pgfqpoint{4.223672in}{0.677170in}}%
\pgfpathlineto{\pgfqpoint{4.224546in}{0.730566in}}%
\pgfpathlineto{\pgfqpoint{4.225128in}{0.694712in}}%
\pgfpathlineto{\pgfqpoint{4.226583in}{0.668941in}}%
\pgfpathlineto{\pgfqpoint{4.227893in}{0.663351in}}%
\pgfpathlineto{\pgfqpoint{4.228330in}{0.663641in}}%
\pgfpathlineto{\pgfqpoint{4.230368in}{0.666004in}}%
\pgfpathlineto{\pgfqpoint{4.231095in}{0.665050in}}%
\pgfpathlineto{\pgfqpoint{4.239246in}{0.669436in}}%
\pgfpathlineto{\pgfqpoint{4.240265in}{0.673227in}}%
\pgfpathlineto{\pgfqpoint{4.243467in}{0.690393in}}%
\pgfpathlineto{\pgfqpoint{4.243758in}{0.685595in}}%
\pgfpathlineto{\pgfqpoint{4.245359in}{0.668650in}}%
\pgfpathlineto{\pgfqpoint{4.246232in}{0.669510in}}%
\pgfpathlineto{\pgfqpoint{4.247105in}{0.682393in}}%
\pgfpathlineto{\pgfqpoint{4.247833in}{0.675225in}}%
\pgfpathlineto{\pgfqpoint{4.248852in}{0.673603in}}%
\pgfpathlineto{\pgfqpoint{4.250307in}{0.668436in}}%
\pgfpathlineto{\pgfqpoint{4.250744in}{0.668639in}}%
\pgfpathlineto{\pgfqpoint{4.251326in}{0.669658in}}%
\pgfpathlineto{\pgfqpoint{4.251908in}{0.679678in}}%
\pgfpathlineto{\pgfqpoint{4.252490in}{0.689219in}}%
\pgfpathlineto{\pgfqpoint{4.253073in}{0.681234in}}%
\pgfpathlineto{\pgfqpoint{4.253509in}{0.679385in}}%
\pgfpathlineto{\pgfqpoint{4.253946in}{0.681809in}}%
\pgfpathlineto{\pgfqpoint{4.254237in}{0.683182in}}%
\pgfpathlineto{\pgfqpoint{4.254674in}{0.681294in}}%
\pgfpathlineto{\pgfqpoint{4.256566in}{0.665949in}}%
\pgfpathlineto{\pgfqpoint{4.257585in}{0.666730in}}%
\pgfpathlineto{\pgfqpoint{4.258458in}{0.669213in}}%
\pgfpathlineto{\pgfqpoint{4.259040in}{0.680049in}}%
\pgfpathlineto{\pgfqpoint{4.259913in}{0.721822in}}%
\pgfpathlineto{\pgfqpoint{4.260641in}{0.706043in}}%
\pgfpathlineto{\pgfqpoint{4.260787in}{0.704503in}}%
\pgfpathlineto{\pgfqpoint{4.261223in}{0.708784in}}%
\pgfpathlineto{\pgfqpoint{4.262096in}{0.724485in}}%
\pgfpathlineto{\pgfqpoint{4.262533in}{0.717159in}}%
\pgfpathlineto{\pgfqpoint{4.264716in}{0.669407in}}%
\pgfpathlineto{\pgfqpoint{4.265153in}{0.668815in}}%
\pgfpathlineto{\pgfqpoint{4.265444in}{0.670003in}}%
\pgfpathlineto{\pgfqpoint{4.268937in}{0.711346in}}%
\pgfpathlineto{\pgfqpoint{4.269810in}{0.758464in}}%
\pgfpathlineto{\pgfqpoint{4.270393in}{0.733262in}}%
\pgfpathlineto{\pgfqpoint{4.271994in}{0.697738in}}%
\pgfpathlineto{\pgfqpoint{4.275196in}{0.667000in}}%
\pgfpathlineto{\pgfqpoint{4.275923in}{0.667553in}}%
\pgfpathlineto{\pgfqpoint{4.276651in}{0.672514in}}%
\pgfpathlineto{\pgfqpoint{4.280581in}{0.722708in}}%
\pgfpathlineto{\pgfqpoint{4.281017in}{0.718691in}}%
\pgfpathlineto{\pgfqpoint{4.283637in}{0.688020in}}%
\pgfpathlineto{\pgfqpoint{4.284365in}{0.681405in}}%
\pgfpathlineto{\pgfqpoint{4.286403in}{0.665468in}}%
\pgfpathlineto{\pgfqpoint{4.286985in}{0.666145in}}%
\pgfpathlineto{\pgfqpoint{4.287858in}{0.676299in}}%
\pgfpathlineto{\pgfqpoint{4.288149in}{0.678305in}}%
\pgfpathlineto{\pgfqpoint{4.288731in}{0.675173in}}%
\pgfpathlineto{\pgfqpoint{4.290623in}{0.669896in}}%
\pgfpathlineto{\pgfqpoint{4.291497in}{0.668246in}}%
\pgfpathlineto{\pgfqpoint{4.291788in}{0.669102in}}%
\pgfpathlineto{\pgfqpoint{4.292807in}{0.682889in}}%
\pgfpathlineto{\pgfqpoint{4.293389in}{0.673830in}}%
\pgfpathlineto{\pgfqpoint{4.294844in}{0.664942in}}%
\pgfpathlineto{\pgfqpoint{4.295718in}{0.663673in}}%
\pgfpathlineto{\pgfqpoint{4.296154in}{0.663452in}}%
\pgfpathlineto{\pgfqpoint{4.296736in}{0.664131in}}%
\pgfpathlineto{\pgfqpoint{4.298046in}{0.666776in}}%
\pgfpathlineto{\pgfqpoint{4.298774in}{0.672282in}}%
\pgfpathlineto{\pgfqpoint{4.299211in}{0.669169in}}%
\pgfpathlineto{\pgfqpoint{4.299793in}{0.666458in}}%
\pgfpathlineto{\pgfqpoint{4.300521in}{0.666883in}}%
\pgfpathlineto{\pgfqpoint{4.301830in}{0.667458in}}%
\pgfpathlineto{\pgfqpoint{4.304159in}{0.670007in}}%
\pgfpathlineto{\pgfqpoint{4.304596in}{0.671448in}}%
\pgfpathlineto{\pgfqpoint{4.305178in}{0.669719in}}%
\pgfpathlineto{\pgfqpoint{4.306633in}{0.668586in}}%
\pgfpathlineto{\pgfqpoint{4.312455in}{0.668145in}}%
\pgfpathlineto{\pgfqpoint{4.316239in}{0.668473in}}%
\pgfpathlineto{\pgfqpoint{4.318859in}{0.672382in}}%
\pgfpathlineto{\pgfqpoint{4.319150in}{0.670759in}}%
\pgfpathlineto{\pgfqpoint{4.319733in}{0.669033in}}%
\pgfpathlineto{\pgfqpoint{4.320460in}{0.669394in}}%
\pgfpathlineto{\pgfqpoint{4.322352in}{0.669620in}}%
\pgfpathlineto{\pgfqpoint{4.322498in}{0.669343in}}%
\pgfpathlineto{\pgfqpoint{4.322935in}{0.668984in}}%
\pgfpathlineto{\pgfqpoint{4.323080in}{0.669399in}}%
\pgfpathlineto{\pgfqpoint{4.323662in}{0.680200in}}%
\pgfpathlineto{\pgfqpoint{4.323953in}{0.685233in}}%
\pgfpathlineto{\pgfqpoint{4.324390in}{0.674770in}}%
\pgfpathlineto{\pgfqpoint{4.324827in}{0.668861in}}%
\pgfpathlineto{\pgfqpoint{4.325409in}{0.673845in}}%
\pgfpathlineto{\pgfqpoint{4.325700in}{0.677252in}}%
\pgfpathlineto{\pgfqpoint{4.326282in}{0.670229in}}%
\pgfpathlineto{\pgfqpoint{4.326864in}{0.668628in}}%
\pgfpathlineto{\pgfqpoint{4.327446in}{0.668897in}}%
\pgfpathlineto{\pgfqpoint{4.329339in}{0.669838in}}%
\pgfpathlineto{\pgfqpoint{4.329921in}{0.687145in}}%
\pgfpathlineto{\pgfqpoint{4.330357in}{0.702970in}}%
\pgfpathlineto{\pgfqpoint{4.330940in}{0.691089in}}%
\pgfpathlineto{\pgfqpoint{4.331813in}{0.677250in}}%
\pgfpathlineto{\pgfqpoint{4.332250in}{0.684477in}}%
\pgfpathlineto{\pgfqpoint{4.332977in}{0.715678in}}%
\pgfpathlineto{\pgfqpoint{4.333559in}{0.695224in}}%
\pgfpathlineto{\pgfqpoint{4.334433in}{0.670307in}}%
\pgfpathlineto{\pgfqpoint{4.334869in}{0.675422in}}%
\pgfpathlineto{\pgfqpoint{4.335452in}{0.688976in}}%
\pgfpathlineto{\pgfqpoint{4.335888in}{0.677459in}}%
\pgfpathlineto{\pgfqpoint{4.336470in}{0.668112in}}%
\pgfpathlineto{\pgfqpoint{4.337053in}{0.673800in}}%
\pgfpathlineto{\pgfqpoint{4.337344in}{0.677143in}}%
\pgfpathlineto{\pgfqpoint{4.337926in}{0.670113in}}%
\pgfpathlineto{\pgfqpoint{4.338508in}{0.668564in}}%
\pgfpathlineto{\pgfqpoint{4.339090in}{0.668955in}}%
\pgfpathlineto{\pgfqpoint{4.339818in}{0.670312in}}%
\pgfpathlineto{\pgfqpoint{4.340400in}{0.678296in}}%
\pgfpathlineto{\pgfqpoint{4.340982in}{0.692152in}}%
\pgfpathlineto{\pgfqpoint{4.341564in}{0.682043in}}%
\pgfpathlineto{\pgfqpoint{4.341856in}{0.677624in}}%
\pgfpathlineto{\pgfqpoint{4.342292in}{0.685078in}}%
\pgfpathlineto{\pgfqpoint{4.343020in}{0.716240in}}%
\pgfpathlineto{\pgfqpoint{4.343457in}{0.695550in}}%
\pgfpathlineto{\pgfqpoint{4.344475in}{0.668811in}}%
\pgfpathlineto{\pgfqpoint{4.344912in}{0.669861in}}%
\pgfpathlineto{\pgfqpoint{4.345494in}{0.672070in}}%
\pgfpathlineto{\pgfqpoint{4.345931in}{0.670405in}}%
\pgfpathlineto{\pgfqpoint{4.346950in}{0.667599in}}%
\pgfpathlineto{\pgfqpoint{4.347386in}{0.667907in}}%
\pgfpathlineto{\pgfqpoint{4.350006in}{0.670001in}}%
\pgfpathlineto{\pgfqpoint{4.350297in}{0.669489in}}%
\pgfpathlineto{\pgfqpoint{4.350734in}{0.669000in}}%
\pgfpathlineto{\pgfqpoint{4.351170in}{0.669733in}}%
\pgfpathlineto{\pgfqpoint{4.351753in}{0.672125in}}%
\pgfpathlineto{\pgfqpoint{4.352189in}{0.669902in}}%
\pgfpathlineto{\pgfqpoint{4.352626in}{0.667897in}}%
\pgfpathlineto{\pgfqpoint{4.353063in}{0.669912in}}%
\pgfpathlineto{\pgfqpoint{4.353790in}{0.724566in}}%
\pgfpathlineto{\pgfqpoint{4.354518in}{0.682425in}}%
\pgfpathlineto{\pgfqpoint{4.354664in}{0.679158in}}%
\pgfpathlineto{\pgfqpoint{4.355100in}{0.692606in}}%
\pgfpathlineto{\pgfqpoint{4.355682in}{0.732711in}}%
\pgfpathlineto{\pgfqpoint{4.356265in}{0.705542in}}%
\pgfpathlineto{\pgfqpoint{4.357866in}{0.673508in}}%
\pgfpathlineto{\pgfqpoint{4.359030in}{0.670842in}}%
\pgfpathlineto{\pgfqpoint{4.360340in}{0.667916in}}%
\pgfpathlineto{\pgfqpoint{4.360631in}{0.668185in}}%
\pgfpathlineto{\pgfqpoint{4.362232in}{0.671733in}}%
\pgfpathlineto{\pgfqpoint{4.362814in}{0.708435in}}%
\pgfpathlineto{\pgfqpoint{4.363105in}{0.724557in}}%
\pgfpathlineto{\pgfqpoint{4.363542in}{0.695251in}}%
\pgfpathlineto{\pgfqpoint{4.364270in}{0.670371in}}%
\pgfpathlineto{\pgfqpoint{4.364852in}{0.674370in}}%
\pgfpathlineto{\pgfqpoint{4.365143in}{0.673340in}}%
\pgfpathlineto{\pgfqpoint{4.365871in}{0.668423in}}%
\pgfpathlineto{\pgfqpoint{4.366453in}{0.669248in}}%
\pgfpathlineto{\pgfqpoint{4.366889in}{0.670084in}}%
\pgfpathlineto{\pgfqpoint{4.367472in}{0.669075in}}%
\pgfpathlineto{\pgfqpoint{4.368927in}{0.669306in}}%
\pgfpathlineto{\pgfqpoint{4.369364in}{0.695432in}}%
\pgfpathlineto{\pgfqpoint{4.369946in}{0.751144in}}%
\pgfpathlineto{\pgfqpoint{4.370382in}{0.711069in}}%
\pgfpathlineto{\pgfqpoint{4.370674in}{0.695448in}}%
\pgfpathlineto{\pgfqpoint{4.371110in}{0.719748in}}%
\pgfpathlineto{\pgfqpoint{4.371692in}{0.825448in}}%
\pgfpathlineto{\pgfqpoint{4.372275in}{0.751321in}}%
\pgfpathlineto{\pgfqpoint{4.373293in}{0.675177in}}%
\pgfpathlineto{\pgfqpoint{4.373730in}{0.675585in}}%
\pgfpathlineto{\pgfqpoint{4.374458in}{0.671922in}}%
\pgfpathlineto{\pgfqpoint{4.374894in}{0.674004in}}%
\pgfpathlineto{\pgfqpoint{4.375477in}{0.676759in}}%
\pgfpathlineto{\pgfqpoint{4.376204in}{0.675752in}}%
\pgfpathlineto{\pgfqpoint{4.376641in}{0.678469in}}%
\pgfpathlineto{\pgfqpoint{4.377369in}{0.704741in}}%
\pgfpathlineto{\pgfqpoint{4.377660in}{0.715665in}}%
\pgfpathlineto{\pgfqpoint{4.378096in}{0.701586in}}%
\pgfpathlineto{\pgfqpoint{4.378970in}{0.674059in}}%
\pgfpathlineto{\pgfqpoint{4.379552in}{0.676128in}}%
\pgfpathlineto{\pgfqpoint{4.382317in}{0.669564in}}%
\pgfpathlineto{\pgfqpoint{4.382463in}{0.669624in}}%
\pgfpathlineto{\pgfqpoint{4.382899in}{0.670833in}}%
\pgfpathlineto{\pgfqpoint{4.383627in}{0.675882in}}%
\pgfpathlineto{\pgfqpoint{4.384064in}{0.672759in}}%
\pgfpathlineto{\pgfqpoint{4.385374in}{0.671050in}}%
\pgfpathlineto{\pgfqpoint{4.386538in}{0.671714in}}%
\pgfpathlineto{\pgfqpoint{4.388576in}{0.677211in}}%
\pgfpathlineto{\pgfqpoint{4.389012in}{0.675914in}}%
\pgfpathlineto{\pgfqpoint{4.390468in}{0.673687in}}%
\pgfpathlineto{\pgfqpoint{4.390613in}{0.673845in}}%
\pgfpathlineto{\pgfqpoint{4.391778in}{0.675961in}}%
\pgfpathlineto{\pgfqpoint{4.392360in}{0.701945in}}%
\pgfpathlineto{\pgfqpoint{4.393233in}{0.778927in}}%
\pgfpathlineto{\pgfqpoint{4.393815in}{0.744915in}}%
\pgfpathlineto{\pgfqpoint{4.395562in}{0.694423in}}%
\pgfpathlineto{\pgfqpoint{4.397017in}{0.675804in}}%
\pgfpathlineto{\pgfqpoint{4.397308in}{0.677094in}}%
\pgfpathlineto{\pgfqpoint{4.398182in}{0.699804in}}%
\pgfpathlineto{\pgfqpoint{4.398909in}{0.709367in}}%
\pgfpathlineto{\pgfqpoint{4.399492in}{0.706353in}}%
\pgfpathlineto{\pgfqpoint{4.403421in}{0.670834in}}%
\pgfpathlineto{\pgfqpoint{4.403858in}{0.671350in}}%
\pgfpathlineto{\pgfqpoint{4.404586in}{0.672478in}}%
\pgfpathlineto{\pgfqpoint{4.405022in}{0.671819in}}%
\pgfpathlineto{\pgfqpoint{4.406041in}{0.670913in}}%
\pgfpathlineto{\pgfqpoint{4.406478in}{0.671235in}}%
\pgfpathlineto{\pgfqpoint{4.407933in}{0.670914in}}%
\pgfpathlineto{\pgfqpoint{4.411717in}{0.667885in}}%
\pgfpathlineto{\pgfqpoint{4.411863in}{0.668038in}}%
\pgfpathlineto{\pgfqpoint{4.413318in}{0.667992in}}%
\pgfpathlineto{\pgfqpoint{4.426563in}{0.671163in}}%
\pgfpathlineto{\pgfqpoint{4.430202in}{0.675019in}}%
\pgfpathlineto{\pgfqpoint{4.430493in}{0.673826in}}%
\pgfpathlineto{\pgfqpoint{4.431657in}{0.669151in}}%
\pgfpathlineto{\pgfqpoint{4.432094in}{0.669544in}}%
\pgfpathlineto{\pgfqpoint{4.435878in}{0.674570in}}%
\pgfpathlineto{\pgfqpoint{4.436024in}{0.674468in}}%
\pgfpathlineto{\pgfqpoint{4.437188in}{0.671620in}}%
\pgfpathlineto{\pgfqpoint{4.437625in}{0.673211in}}%
\pgfpathlineto{\pgfqpoint{4.439953in}{0.685373in}}%
\pgfpathlineto{\pgfqpoint{4.440099in}{0.684875in}}%
\pgfpathlineto{\pgfqpoint{4.441263in}{0.670345in}}%
\pgfpathlineto{\pgfqpoint{4.441700in}{0.675247in}}%
\pgfpathlineto{\pgfqpoint{4.443155in}{0.725040in}}%
\pgfpathlineto{\pgfqpoint{4.444174in}{0.714441in}}%
\pgfpathlineto{\pgfqpoint{4.446066in}{0.686085in}}%
\pgfpathlineto{\pgfqpoint{4.446503in}{0.693009in}}%
\pgfpathlineto{\pgfqpoint{4.447376in}{0.714931in}}%
\pgfpathlineto{\pgfqpoint{4.447958in}{0.709526in}}%
\pgfpathlineto{\pgfqpoint{4.450433in}{0.669013in}}%
\pgfpathlineto{\pgfqpoint{4.451015in}{0.655387in}}%
\pgfpathlineto{\pgfqpoint{4.451597in}{0.664213in}}%
\pgfpathlineto{\pgfqpoint{4.452325in}{0.661465in}}%
\pgfpathlineto{\pgfqpoint{4.452761in}{0.682678in}}%
\pgfpathlineto{\pgfqpoint{4.453635in}{0.877218in}}%
\pgfpathlineto{\pgfqpoint{4.454362in}{0.776662in}}%
\pgfpathlineto{\pgfqpoint{4.455963in}{0.708228in}}%
\pgfpathlineto{\pgfqpoint{4.457710in}{0.669067in}}%
\pgfpathlineto{\pgfqpoint{4.458438in}{0.677676in}}%
\pgfpathlineto{\pgfqpoint{4.459748in}{0.721670in}}%
\pgfpathlineto{\pgfqpoint{4.460475in}{0.717403in}}%
\pgfpathlineto{\pgfqpoint{4.462367in}{0.695241in}}%
\pgfpathlineto{\pgfqpoint{4.463532in}{0.668653in}}%
\pgfpathlineto{\pgfqpoint{4.463968in}{0.671850in}}%
\pgfpathlineto{\pgfqpoint{4.464696in}{0.707047in}}%
\pgfpathlineto{\pgfqpoint{4.465569in}{0.750173in}}%
\pgfpathlineto{\pgfqpoint{4.466152in}{0.740371in}}%
\pgfpathlineto{\pgfqpoint{4.469208in}{0.667503in}}%
\pgfpathlineto{\pgfqpoint{4.469499in}{0.668514in}}%
\pgfpathlineto{\pgfqpoint{4.472264in}{0.690436in}}%
\pgfpathlineto{\pgfqpoint{4.472701in}{0.689558in}}%
\pgfpathlineto{\pgfqpoint{4.474302in}{0.679608in}}%
\pgfpathlineto{\pgfqpoint{4.474739in}{0.685369in}}%
\pgfpathlineto{\pgfqpoint{4.475466in}{0.705937in}}%
\pgfpathlineto{\pgfqpoint{4.476049in}{0.692005in}}%
\pgfpathlineto{\pgfqpoint{4.476922in}{0.674159in}}%
\pgfpathlineto{\pgfqpoint{4.477359in}{0.677095in}}%
\pgfpathlineto{\pgfqpoint{4.478523in}{0.693256in}}%
\pgfpathlineto{\pgfqpoint{4.479396in}{0.691269in}}%
\pgfpathlineto{\pgfqpoint{4.479542in}{0.691585in}}%
\pgfpathlineto{\pgfqpoint{4.479833in}{0.689837in}}%
\pgfpathlineto{\pgfqpoint{4.481288in}{0.650716in}}%
\pgfpathlineto{\pgfqpoint{4.482453in}{0.654146in}}%
\pgfpathlineto{\pgfqpoint{4.483035in}{0.674965in}}%
\pgfpathlineto{\pgfqpoint{4.483908in}{0.735201in}}%
\pgfpathlineto{\pgfqpoint{4.484490in}{0.706609in}}%
\pgfpathlineto{\pgfqpoint{4.486382in}{0.672315in}}%
\pgfpathlineto{\pgfqpoint{4.487110in}{0.667639in}}%
\pgfpathlineto{\pgfqpoint{4.487547in}{0.670666in}}%
\pgfpathlineto{\pgfqpoint{4.488129in}{0.675346in}}%
\pgfpathlineto{\pgfqpoint{4.488566in}{0.670819in}}%
\pgfpathlineto{\pgfqpoint{4.490167in}{0.659644in}}%
\pgfpathlineto{\pgfqpoint{4.490312in}{0.659825in}}%
\pgfpathlineto{\pgfqpoint{4.491331in}{0.664456in}}%
\pgfpathlineto{\pgfqpoint{4.492204in}{0.669242in}}%
\pgfpathlineto{\pgfqpoint{4.492786in}{0.667756in}}%
\pgfpathlineto{\pgfqpoint{4.493805in}{0.664243in}}%
\pgfpathlineto{\pgfqpoint{4.494970in}{0.654033in}}%
\pgfpathlineto{\pgfqpoint{4.495552in}{0.655828in}}%
\pgfpathlineto{\pgfqpoint{4.497007in}{0.669665in}}%
\pgfpathlineto{\pgfqpoint{4.498026in}{0.666852in}}%
\pgfpathlineto{\pgfqpoint{4.499336in}{0.661694in}}%
\pgfpathlineto{\pgfqpoint{4.500500in}{0.656023in}}%
\pgfpathlineto{\pgfqpoint{4.500937in}{0.656981in}}%
\pgfpathlineto{\pgfqpoint{4.504576in}{0.665206in}}%
\pgfpathlineto{\pgfqpoint{4.507341in}{0.667080in}}%
\pgfpathlineto{\pgfqpoint{4.510980in}{0.668962in}}%
\pgfpathlineto{\pgfqpoint{4.517529in}{0.670413in}}%
\pgfpathlineto{\pgfqpoint{4.518548in}{0.672516in}}%
\pgfpathlineto{\pgfqpoint{4.518985in}{0.672904in}}%
\pgfpathlineto{\pgfqpoint{4.519421in}{0.672307in}}%
\pgfpathlineto{\pgfqpoint{4.520586in}{0.671619in}}%
\pgfpathlineto{\pgfqpoint{4.520877in}{0.671821in}}%
\pgfpathlineto{\pgfqpoint{4.521896in}{0.671279in}}%
\pgfpathlineto{\pgfqpoint{4.523497in}{0.671152in}}%
\pgfpathlineto{\pgfqpoint{4.526844in}{0.673258in}}%
\pgfpathlineto{\pgfqpoint{4.527572in}{0.690865in}}%
\pgfpathlineto{\pgfqpoint{4.527863in}{0.694748in}}%
\pgfpathlineto{\pgfqpoint{4.528300in}{0.685048in}}%
\pgfpathlineto{\pgfqpoint{4.529755in}{0.673812in}}%
\pgfpathlineto{\pgfqpoint{4.530628in}{0.670252in}}%
\pgfpathlineto{\pgfqpoint{4.531356in}{0.670900in}}%
\pgfpathlineto{\pgfqpoint{4.531647in}{0.670773in}}%
\pgfpathlineto{\pgfqpoint{4.531793in}{0.671441in}}%
\pgfpathlineto{\pgfqpoint{4.532229in}{0.684004in}}%
\pgfpathlineto{\pgfqpoint{4.533248in}{0.748589in}}%
\pgfpathlineto{\pgfqpoint{4.533976in}{0.730159in}}%
\pgfpathlineto{\pgfqpoint{4.536305in}{0.680794in}}%
\pgfpathlineto{\pgfqpoint{4.536887in}{0.676183in}}%
\pgfpathlineto{\pgfqpoint{4.537178in}{0.678659in}}%
\pgfpathlineto{\pgfqpoint{4.537906in}{0.723292in}}%
\pgfpathlineto{\pgfqpoint{4.538633in}{0.765835in}}%
\pgfpathlineto{\pgfqpoint{4.539215in}{0.756184in}}%
\pgfpathlineto{\pgfqpoint{4.542854in}{0.672573in}}%
\pgfpathlineto{\pgfqpoint{4.543291in}{0.671202in}}%
\pgfpathlineto{\pgfqpoint{4.543727in}{0.673256in}}%
\pgfpathlineto{\pgfqpoint{4.547512in}{0.723074in}}%
\pgfpathlineto{\pgfqpoint{4.547803in}{0.715347in}}%
\pgfpathlineto{\pgfqpoint{4.549258in}{0.668532in}}%
\pgfpathlineto{\pgfqpoint{4.549549in}{0.668723in}}%
\pgfpathlineto{\pgfqpoint{4.550859in}{0.672607in}}%
\pgfpathlineto{\pgfqpoint{4.551878in}{0.671586in}}%
\pgfpathlineto{\pgfqpoint{4.553042in}{0.666339in}}%
\pgfpathlineto{\pgfqpoint{4.553770in}{0.669525in}}%
\pgfpathlineto{\pgfqpoint{4.554498in}{0.668664in}}%
\pgfpathlineto{\pgfqpoint{4.554934in}{0.673451in}}%
\pgfpathlineto{\pgfqpoint{4.555808in}{0.701289in}}%
\pgfpathlineto{\pgfqpoint{4.556390in}{0.691452in}}%
\pgfpathlineto{\pgfqpoint{4.558136in}{0.674710in}}%
\pgfpathlineto{\pgfqpoint{4.561193in}{0.664985in}}%
\pgfpathlineto{\pgfqpoint{4.561484in}{0.665498in}}%
\pgfpathlineto{\pgfqpoint{4.562066in}{0.678148in}}%
\pgfpathlineto{\pgfqpoint{4.563085in}{0.758892in}}%
\pgfpathlineto{\pgfqpoint{4.563667in}{0.723078in}}%
\pgfpathlineto{\pgfqpoint{4.565268in}{0.685277in}}%
\pgfpathlineto{\pgfqpoint{4.566578in}{0.662144in}}%
\pgfpathlineto{\pgfqpoint{4.567160in}{0.663270in}}%
\pgfpathlineto{\pgfqpoint{4.568179in}{0.666054in}}%
\pgfpathlineto{\pgfqpoint{4.569635in}{0.695343in}}%
\pgfpathlineto{\pgfqpoint{4.570799in}{0.687735in}}%
\pgfpathlineto{\pgfqpoint{4.573419in}{0.663955in}}%
\pgfpathlineto{\pgfqpoint{4.574001in}{0.669596in}}%
\pgfpathlineto{\pgfqpoint{4.574729in}{0.688681in}}%
\pgfpathlineto{\pgfqpoint{4.575165in}{0.676069in}}%
\pgfpathlineto{\pgfqpoint{4.576621in}{0.663325in}}%
\pgfpathlineto{\pgfqpoint{4.577203in}{0.662472in}}%
\pgfpathlineto{\pgfqpoint{4.577785in}{0.663183in}}%
\pgfpathlineto{\pgfqpoint{4.578513in}{0.664262in}}%
\pgfpathlineto{\pgfqpoint{4.579386in}{0.670086in}}%
\pgfpathlineto{\pgfqpoint{4.579823in}{0.666164in}}%
\pgfpathlineto{\pgfqpoint{4.580405in}{0.662820in}}%
\pgfpathlineto{\pgfqpoint{4.581133in}{0.663877in}}%
\pgfpathlineto{\pgfqpoint{4.587391in}{0.668408in}}%
\pgfpathlineto{\pgfqpoint{4.591903in}{0.670125in}}%
\pgfpathlineto{\pgfqpoint{4.594959in}{0.671344in}}%
\pgfpathlineto{\pgfqpoint{4.595833in}{0.679528in}}%
\pgfpathlineto{\pgfqpoint{4.596560in}{0.685687in}}%
\pgfpathlineto{\pgfqpoint{4.596997in}{0.682860in}}%
\pgfpathlineto{\pgfqpoint{4.597288in}{0.681810in}}%
\pgfpathlineto{\pgfqpoint{4.597579in}{0.685987in}}%
\pgfpathlineto{\pgfqpoint{4.598598in}{0.758705in}}%
\pgfpathlineto{\pgfqpoint{4.599326in}{0.716102in}}%
\pgfpathlineto{\pgfqpoint{4.600781in}{0.683745in}}%
\pgfpathlineto{\pgfqpoint{4.601655in}{0.677538in}}%
\pgfpathlineto{\pgfqpoint{4.602091in}{0.680478in}}%
\pgfpathlineto{\pgfqpoint{4.603692in}{0.700609in}}%
\pgfpathlineto{\pgfqpoint{4.604420in}{0.722825in}}%
\pgfpathlineto{\pgfqpoint{4.604857in}{0.714484in}}%
\pgfpathlineto{\pgfqpoint{4.607040in}{0.680680in}}%
\pgfpathlineto{\pgfqpoint{4.608641in}{0.675416in}}%
\pgfpathlineto{\pgfqpoint{4.609077in}{0.675938in}}%
\pgfpathlineto{\pgfqpoint{4.609951in}{0.681951in}}%
\pgfpathlineto{\pgfqpoint{4.610533in}{0.677575in}}%
\pgfpathlineto{\pgfqpoint{4.611697in}{0.671552in}}%
\pgfpathlineto{\pgfqpoint{4.612134in}{0.671901in}}%
\pgfpathlineto{\pgfqpoint{4.612862in}{0.675433in}}%
\pgfpathlineto{\pgfqpoint{4.613735in}{0.682342in}}%
\pgfpathlineto{\pgfqpoint{4.614608in}{0.681938in}}%
\pgfpathlineto{\pgfqpoint{4.615190in}{0.685347in}}%
\pgfpathlineto{\pgfqpoint{4.615481in}{0.682520in}}%
\pgfpathlineto{\pgfqpoint{4.617082in}{0.667369in}}%
\pgfpathlineto{\pgfqpoint{4.617228in}{0.667427in}}%
\pgfpathlineto{\pgfqpoint{4.617810in}{0.669659in}}%
\pgfpathlineto{\pgfqpoint{4.618829in}{0.680032in}}%
\pgfpathlineto{\pgfqpoint{4.619557in}{0.676522in}}%
\pgfpathlineto{\pgfqpoint{4.620721in}{0.671771in}}%
\pgfpathlineto{\pgfqpoint{4.622613in}{0.662873in}}%
\pgfpathlineto{\pgfqpoint{4.623195in}{0.663870in}}%
\pgfpathlineto{\pgfqpoint{4.624360in}{0.674695in}}%
\pgfpathlineto{\pgfqpoint{4.625379in}{0.671184in}}%
\pgfpathlineto{\pgfqpoint{4.627271in}{0.668434in}}%
\pgfpathlineto{\pgfqpoint{4.629454in}{0.659197in}}%
\pgfpathlineto{\pgfqpoint{4.630327in}{0.661061in}}%
\pgfpathlineto{\pgfqpoint{4.633675in}{0.665236in}}%
\pgfpathlineto{\pgfqpoint{4.635567in}{0.665188in}}%
\pgfpathlineto{\pgfqpoint{4.641243in}{0.670317in}}%
\pgfpathlineto{\pgfqpoint{4.641680in}{0.671439in}}%
\pgfpathlineto{\pgfqpoint{4.642407in}{0.670335in}}%
\pgfpathlineto{\pgfqpoint{4.642698in}{0.670777in}}%
\pgfpathlineto{\pgfqpoint{4.643135in}{0.678935in}}%
\pgfpathlineto{\pgfqpoint{4.643863in}{0.711674in}}%
\pgfpathlineto{\pgfqpoint{4.644299in}{0.688147in}}%
\pgfpathlineto{\pgfqpoint{4.644736in}{0.674618in}}%
\pgfpathlineto{\pgfqpoint{4.645173in}{0.686232in}}%
\pgfpathlineto{\pgfqpoint{4.645609in}{0.706154in}}%
\pgfpathlineto{\pgfqpoint{4.646046in}{0.684059in}}%
\pgfpathlineto{\pgfqpoint{4.646628in}{0.669696in}}%
\pgfpathlineto{\pgfqpoint{4.647210in}{0.675906in}}%
\pgfpathlineto{\pgfqpoint{4.647647in}{0.682857in}}%
\pgfpathlineto{\pgfqpoint{4.648084in}{0.673641in}}%
\pgfpathlineto{\pgfqpoint{4.648375in}{0.670739in}}%
\pgfpathlineto{\pgfqpoint{4.648811in}{0.675160in}}%
\pgfpathlineto{\pgfqpoint{4.649539in}{0.704942in}}%
\pgfpathlineto{\pgfqpoint{4.649976in}{0.686459in}}%
\pgfpathlineto{\pgfqpoint{4.650703in}{0.672319in}}%
\pgfpathlineto{\pgfqpoint{4.651286in}{0.673139in}}%
\pgfpathlineto{\pgfqpoint{4.651868in}{0.671167in}}%
\pgfpathlineto{\pgfqpoint{4.652596in}{0.669970in}}%
\pgfpathlineto{\pgfqpoint{4.653178in}{0.670204in}}%
\pgfpathlineto{\pgfqpoint{4.659873in}{0.673260in}}%
\pgfpathlineto{\pgfqpoint{4.661037in}{0.693627in}}%
\pgfpathlineto{\pgfqpoint{4.661619in}{0.683668in}}%
\pgfpathlineto{\pgfqpoint{4.662056in}{0.679897in}}%
\pgfpathlineto{\pgfqpoint{4.662347in}{0.683968in}}%
\pgfpathlineto{\pgfqpoint{4.662929in}{0.697937in}}%
\pgfpathlineto{\pgfqpoint{4.663366in}{0.683183in}}%
\pgfpathlineto{\pgfqpoint{4.664094in}{0.670442in}}%
\pgfpathlineto{\pgfqpoint{4.664676in}{0.670535in}}%
\pgfpathlineto{\pgfqpoint{4.665404in}{0.670962in}}%
\pgfpathlineto{\pgfqpoint{4.666277in}{0.688513in}}%
\pgfpathlineto{\pgfqpoint{4.667296in}{0.679510in}}%
\pgfpathlineto{\pgfqpoint{4.669624in}{0.674466in}}%
\pgfpathlineto{\pgfqpoint{4.669916in}{0.675500in}}%
\pgfpathlineto{\pgfqpoint{4.670934in}{0.696062in}}%
\pgfpathlineto{\pgfqpoint{4.671517in}{0.684917in}}%
\pgfpathlineto{\pgfqpoint{4.673409in}{0.668872in}}%
\pgfpathlineto{\pgfqpoint{4.673991in}{0.667470in}}%
\pgfpathlineto{\pgfqpoint{4.674573in}{0.667803in}}%
\pgfpathlineto{\pgfqpoint{4.675301in}{0.668659in}}%
\pgfpathlineto{\pgfqpoint{4.676028in}{0.672556in}}%
\pgfpathlineto{\pgfqpoint{4.676611in}{0.669519in}}%
\pgfpathlineto{\pgfqpoint{4.677047in}{0.667921in}}%
\pgfpathlineto{\pgfqpoint{4.677629in}{0.669437in}}%
\pgfpathlineto{\pgfqpoint{4.677921in}{0.670124in}}%
\pgfpathlineto{\pgfqpoint{4.678357in}{0.668245in}}%
\pgfpathlineto{\pgfqpoint{4.678648in}{0.668076in}}%
\pgfpathlineto{\pgfqpoint{4.678794in}{0.668223in}}%
\pgfpathlineto{\pgfqpoint{4.679085in}{0.674356in}}%
\pgfpathlineto{\pgfqpoint{4.679667in}{0.716586in}}%
\pgfpathlineto{\pgfqpoint{4.680395in}{0.687944in}}%
\pgfpathlineto{\pgfqpoint{4.681705in}{0.681471in}}%
\pgfpathlineto{\pgfqpoint{4.683451in}{0.666974in}}%
\pgfpathlineto{\pgfqpoint{4.683888in}{0.667060in}}%
\pgfpathlineto{\pgfqpoint{4.684907in}{0.668477in}}%
\pgfpathlineto{\pgfqpoint{4.685634in}{0.679053in}}%
\pgfpathlineto{\pgfqpoint{4.686362in}{0.695687in}}%
\pgfpathlineto{\pgfqpoint{4.686944in}{0.687259in}}%
\pgfpathlineto{\pgfqpoint{4.689273in}{0.666025in}}%
\pgfpathlineto{\pgfqpoint{4.689564in}{0.666261in}}%
\pgfpathlineto{\pgfqpoint{4.690146in}{0.670018in}}%
\pgfpathlineto{\pgfqpoint{4.690874in}{0.677653in}}%
\pgfpathlineto{\pgfqpoint{4.691456in}{0.674842in}}%
\pgfpathlineto{\pgfqpoint{4.692475in}{0.670446in}}%
\pgfpathlineto{\pgfqpoint{4.692912in}{0.671784in}}%
\pgfpathlineto{\pgfqpoint{4.693203in}{0.672912in}}%
\pgfpathlineto{\pgfqpoint{4.693639in}{0.670118in}}%
\pgfpathlineto{\pgfqpoint{4.694513in}{0.664235in}}%
\pgfpathlineto{\pgfqpoint{4.695095in}{0.665148in}}%
\pgfpathlineto{\pgfqpoint{4.697133in}{0.670460in}}%
\pgfpathlineto{\pgfqpoint{4.698006in}{0.675463in}}%
\pgfpathlineto{\pgfqpoint{4.698442in}{0.673204in}}%
\pgfpathlineto{\pgfqpoint{4.699170in}{0.670488in}}%
\pgfpathlineto{\pgfqpoint{4.699752in}{0.671863in}}%
\pgfpathlineto{\pgfqpoint{4.701062in}{0.681116in}}%
\pgfpathlineto{\pgfqpoint{4.701499in}{0.677429in}}%
\pgfpathlineto{\pgfqpoint{4.703245in}{0.668570in}}%
\pgfpathlineto{\pgfqpoint{4.704119in}{0.667380in}}%
\pgfpathlineto{\pgfqpoint{4.704555in}{0.668253in}}%
\pgfpathlineto{\pgfqpoint{4.705138in}{0.670425in}}%
\pgfpathlineto{\pgfqpoint{4.705574in}{0.668232in}}%
\pgfpathlineto{\pgfqpoint{4.706156in}{0.666601in}}%
\pgfpathlineto{\pgfqpoint{4.706739in}{0.667315in}}%
\pgfpathlineto{\pgfqpoint{4.707175in}{0.667777in}}%
\pgfpathlineto{\pgfqpoint{4.707757in}{0.666877in}}%
\pgfpathlineto{\pgfqpoint{4.708194in}{0.666952in}}%
\pgfpathlineto{\pgfqpoint{4.708340in}{0.667343in}}%
\pgfpathlineto{\pgfqpoint{4.708776in}{0.675062in}}%
\pgfpathlineto{\pgfqpoint{4.709649in}{0.718783in}}%
\pgfpathlineto{\pgfqpoint{4.710232in}{0.693840in}}%
\pgfpathlineto{\pgfqpoint{4.711978in}{0.673773in}}%
\pgfpathlineto{\pgfqpoint{4.713434in}{0.666184in}}%
\pgfpathlineto{\pgfqpoint{4.714016in}{0.667139in}}%
\pgfpathlineto{\pgfqpoint{4.714744in}{0.676515in}}%
\pgfpathlineto{\pgfqpoint{4.716345in}{0.733220in}}%
\pgfpathlineto{\pgfqpoint{4.717218in}{0.720904in}}%
\pgfpathlineto{\pgfqpoint{4.718528in}{0.684818in}}%
\pgfpathlineto{\pgfqpoint{4.719692in}{0.671526in}}%
\pgfpathlineto{\pgfqpoint{4.719983in}{0.672917in}}%
\pgfpathlineto{\pgfqpoint{4.720565in}{0.695887in}}%
\pgfpathlineto{\pgfqpoint{4.721584in}{0.859698in}}%
\pgfpathlineto{\pgfqpoint{4.722312in}{0.775831in}}%
\pgfpathlineto{\pgfqpoint{4.724059in}{0.709307in}}%
\pgfpathlineto{\pgfqpoint{4.726242in}{0.672201in}}%
\pgfpathlineto{\pgfqpoint{4.726824in}{0.673382in}}%
\pgfpathlineto{\pgfqpoint{4.729298in}{0.681723in}}%
\pgfpathlineto{\pgfqpoint{4.730026in}{0.680635in}}%
\pgfpathlineto{\pgfqpoint{4.730899in}{0.678500in}}%
\pgfpathlineto{\pgfqpoint{4.733665in}{0.670187in}}%
\pgfpathlineto{\pgfqpoint{4.737303in}{0.667772in}}%
\pgfpathlineto{\pgfqpoint{4.738468in}{0.667600in}}%
\pgfpathlineto{\pgfqpoint{4.738613in}{0.667712in}}%
\pgfpathlineto{\pgfqpoint{4.739341in}{0.669761in}}%
\pgfpathlineto{\pgfqpoint{4.740505in}{0.682793in}}%
\pgfpathlineto{\pgfqpoint{4.741087in}{0.677174in}}%
\pgfpathlineto{\pgfqpoint{4.743562in}{0.666701in}}%
\pgfpathlineto{\pgfqpoint{4.745017in}{0.666576in}}%
\pgfpathlineto{\pgfqpoint{4.747200in}{0.669371in}}%
\pgfpathlineto{\pgfqpoint{4.747928in}{0.671129in}}%
\pgfpathlineto{\pgfqpoint{4.748510in}{0.670215in}}%
\pgfpathlineto{\pgfqpoint{4.748947in}{0.671460in}}%
\pgfpathlineto{\pgfqpoint{4.749529in}{0.676018in}}%
\pgfpathlineto{\pgfqpoint{4.749966in}{0.673096in}}%
\pgfpathlineto{\pgfqpoint{4.750984in}{0.667888in}}%
\pgfpathlineto{\pgfqpoint{4.751421in}{0.667929in}}%
\pgfpathlineto{\pgfqpoint{4.755642in}{0.667097in}}%
\pgfpathlineto{\pgfqpoint{4.762774in}{0.671875in}}%
\pgfpathlineto{\pgfqpoint{4.763065in}{0.672456in}}%
\pgfpathlineto{\pgfqpoint{4.763792in}{0.671428in}}%
\pgfpathlineto{\pgfqpoint{4.765539in}{0.671581in}}%
\pgfpathlineto{\pgfqpoint{4.770779in}{0.673124in}}%
\pgfpathlineto{\pgfqpoint{4.771506in}{0.683401in}}%
\pgfpathlineto{\pgfqpoint{4.772671in}{0.719719in}}%
\pgfpathlineto{\pgfqpoint{4.773253in}{0.710098in}}%
\pgfpathlineto{\pgfqpoint{4.773544in}{0.707669in}}%
\pgfpathlineto{\pgfqpoint{4.773835in}{0.712414in}}%
\pgfpathlineto{\pgfqpoint{4.774563in}{0.733086in}}%
\pgfpathlineto{\pgfqpoint{4.775000in}{0.717709in}}%
\pgfpathlineto{\pgfqpoint{4.776892in}{0.674119in}}%
\pgfpathlineto{\pgfqpoint{4.778347in}{0.674281in}}%
\pgfpathlineto{\pgfqpoint{4.780676in}{0.692452in}}%
\pgfpathlineto{\pgfqpoint{4.780967in}{0.687812in}}%
\pgfpathlineto{\pgfqpoint{4.782131in}{0.670581in}}%
\pgfpathlineto{\pgfqpoint{4.782568in}{0.670698in}}%
\pgfpathlineto{\pgfqpoint{4.783587in}{0.671075in}}%
\pgfpathlineto{\pgfqpoint{4.786061in}{0.686132in}}%
\pgfpathlineto{\pgfqpoint{4.786352in}{0.682194in}}%
\pgfpathlineto{\pgfqpoint{4.787371in}{0.669441in}}%
\pgfpathlineto{\pgfqpoint{4.787953in}{0.669512in}}%
\pgfpathlineto{\pgfqpoint{4.791883in}{0.670640in}}%
\pgfpathlineto{\pgfqpoint{4.794066in}{0.677779in}}%
\pgfpathlineto{\pgfqpoint{4.794503in}{0.681774in}}%
\pgfpathlineto{\pgfqpoint{4.794939in}{0.676659in}}%
\pgfpathlineto{\pgfqpoint{4.796395in}{0.669647in}}%
\pgfpathlineto{\pgfqpoint{4.801343in}{0.669960in}}%
\pgfpathlineto{\pgfqpoint{4.802799in}{0.671492in}}%
\pgfpathlineto{\pgfqpoint{4.803235in}{0.672969in}}%
\pgfpathlineto{\pgfqpoint{4.803672in}{0.671330in}}%
\pgfpathlineto{\pgfqpoint{4.804545in}{0.667829in}}%
\pgfpathlineto{\pgfqpoint{4.804982in}{0.668900in}}%
\pgfpathlineto{\pgfqpoint{4.805127in}{0.669041in}}%
\pgfpathlineto{\pgfqpoint{4.805564in}{0.668112in}}%
\pgfpathlineto{\pgfqpoint{4.806146in}{0.668555in}}%
\pgfpathlineto{\pgfqpoint{4.807165in}{0.677097in}}%
\pgfpathlineto{\pgfqpoint{4.808621in}{0.675487in}}%
\pgfpathlineto{\pgfqpoint{4.811240in}{0.682615in}}%
\pgfpathlineto{\pgfqpoint{4.815170in}{0.736798in}}%
\pgfpathlineto{\pgfqpoint{4.815752in}{0.730624in}}%
\pgfpathlineto{\pgfqpoint{4.816771in}{0.712749in}}%
\pgfpathlineto{\pgfqpoint{4.817208in}{0.718857in}}%
\pgfpathlineto{\pgfqpoint{4.817644in}{0.725357in}}%
\pgfpathlineto{\pgfqpoint{4.818227in}{0.717029in}}%
\pgfpathlineto{\pgfqpoint{4.819536in}{0.695693in}}%
\pgfpathlineto{\pgfqpoint{4.819973in}{0.699382in}}%
\pgfpathlineto{\pgfqpoint{4.821137in}{0.723509in}}%
\pgfpathlineto{\pgfqpoint{4.821865in}{0.714778in}}%
\pgfpathlineto{\pgfqpoint{4.822593in}{0.715338in}}%
\pgfpathlineto{\pgfqpoint{4.823321in}{0.702338in}}%
\pgfpathlineto{\pgfqpoint{4.825504in}{0.668242in}}%
\pgfpathlineto{\pgfqpoint{4.826232in}{0.668412in}}%
\pgfpathlineto{\pgfqpoint{4.826377in}{0.668959in}}%
\pgfpathlineto{\pgfqpoint{4.827541in}{0.678299in}}%
\pgfpathlineto{\pgfqpoint{4.828851in}{0.677817in}}%
\pgfpathlineto{\pgfqpoint{4.829288in}{0.674756in}}%
\pgfpathlineto{\pgfqpoint{4.830452in}{0.665244in}}%
\pgfpathlineto{\pgfqpoint{4.830889in}{0.665385in}}%
\pgfpathlineto{\pgfqpoint{4.831471in}{0.666870in}}%
\pgfpathlineto{\pgfqpoint{4.832199in}{0.685565in}}%
\pgfpathlineto{\pgfqpoint{4.832781in}{0.698054in}}%
\pgfpathlineto{\pgfqpoint{4.833363in}{0.689265in}}%
\pgfpathlineto{\pgfqpoint{4.836129in}{0.662400in}}%
\pgfpathlineto{\pgfqpoint{4.836420in}{0.662539in}}%
\pgfpathlineto{\pgfqpoint{4.836565in}{0.662999in}}%
\pgfpathlineto{\pgfqpoint{4.837584in}{0.670525in}}%
\pgfpathlineto{\pgfqpoint{4.838166in}{0.665843in}}%
\pgfpathlineto{\pgfqpoint{4.840204in}{0.659553in}}%
\pgfpathlineto{\pgfqpoint{4.841368in}{0.661388in}}%
\pgfpathlineto{\pgfqpoint{4.842096in}{0.665872in}}%
\pgfpathlineto{\pgfqpoint{4.842678in}{0.670035in}}%
\pgfpathlineto{\pgfqpoint{4.843115in}{0.666563in}}%
\pgfpathlineto{\pgfqpoint{4.843552in}{0.664338in}}%
\pgfpathlineto{\pgfqpoint{4.844279in}{0.665711in}}%
\pgfpathlineto{\pgfqpoint{4.846171in}{0.666178in}}%
\pgfpathlineto{\pgfqpoint{4.846754in}{0.667201in}}%
\pgfpathlineto{\pgfqpoint{4.847336in}{0.666367in}}%
\pgfpathlineto{\pgfqpoint{4.848063in}{0.667412in}}%
\pgfpathlineto{\pgfqpoint{4.848791in}{0.668441in}}%
\pgfpathlineto{\pgfqpoint{4.849228in}{0.667668in}}%
\pgfpathlineto{\pgfqpoint{4.849810in}{0.667390in}}%
\pgfpathlineto{\pgfqpoint{4.850247in}{0.668104in}}%
\pgfpathlineto{\pgfqpoint{4.851120in}{0.676756in}}%
\pgfpathlineto{\pgfqpoint{4.851702in}{0.671428in}}%
\pgfpathlineto{\pgfqpoint{4.852139in}{0.669243in}}%
\pgfpathlineto{\pgfqpoint{4.852575in}{0.671696in}}%
\pgfpathlineto{\pgfqpoint{4.853449in}{0.681117in}}%
\pgfpathlineto{\pgfqpoint{4.853885in}{0.676419in}}%
\pgfpathlineto{\pgfqpoint{4.855486in}{0.667203in}}%
\pgfpathlineto{\pgfqpoint{4.856360in}{0.667779in}}%
\pgfpathlineto{\pgfqpoint{4.858688in}{0.676142in}}%
\pgfpathlineto{\pgfqpoint{4.858979in}{0.673462in}}%
\pgfpathlineto{\pgfqpoint{4.859998in}{0.666647in}}%
\pgfpathlineto{\pgfqpoint{4.860435in}{0.666746in}}%
\pgfpathlineto{\pgfqpoint{4.861308in}{0.668245in}}%
\pgfpathlineto{\pgfqpoint{4.864073in}{0.685785in}}%
\pgfpathlineto{\pgfqpoint{4.864801in}{0.680267in}}%
\pgfpathlineto{\pgfqpoint{4.867275in}{0.661125in}}%
\pgfpathlineto{\pgfqpoint{4.868731in}{0.664212in}}%
\pgfpathlineto{\pgfqpoint{4.870041in}{0.670126in}}%
\pgfpathlineto{\pgfqpoint{4.870623in}{0.668937in}}%
\pgfpathlineto{\pgfqpoint{4.872515in}{0.663892in}}%
\pgfpathlineto{\pgfqpoint{4.872806in}{0.665083in}}%
\pgfpathlineto{\pgfqpoint{4.874844in}{0.697437in}}%
\pgfpathlineto{\pgfqpoint{4.875572in}{0.792912in}}%
\pgfpathlineto{\pgfqpoint{4.876154in}{0.734263in}}%
\pgfpathlineto{\pgfqpoint{4.877027in}{0.670956in}}%
\pgfpathlineto{\pgfqpoint{4.877609in}{0.677381in}}%
\pgfpathlineto{\pgfqpoint{4.877755in}{0.677407in}}%
\pgfpathlineto{\pgfqpoint{4.878774in}{0.666074in}}%
\pgfpathlineto{\pgfqpoint{4.879647in}{0.668472in}}%
\pgfpathlineto{\pgfqpoint{4.879792in}{0.668633in}}%
\pgfpathlineto{\pgfqpoint{4.880083in}{0.667842in}}%
\pgfpathlineto{\pgfqpoint{4.880375in}{0.667248in}}%
\pgfpathlineto{\pgfqpoint{4.880957in}{0.668470in}}%
\pgfpathlineto{\pgfqpoint{4.881102in}{0.668586in}}%
\pgfpathlineto{\pgfqpoint{4.881393in}{0.667791in}}%
\pgfpathlineto{\pgfqpoint{4.881684in}{0.667241in}}%
\pgfpathlineto{\pgfqpoint{4.882267in}{0.668077in}}%
\pgfpathlineto{\pgfqpoint{4.882849in}{0.669356in}}%
\pgfpathlineto{\pgfqpoint{4.883577in}{0.668874in}}%
\pgfpathlineto{\pgfqpoint{4.888380in}{0.672301in}}%
\pgfpathlineto{\pgfqpoint{4.889398in}{0.683078in}}%
\pgfpathlineto{\pgfqpoint{4.890417in}{0.694474in}}%
\pgfpathlineto{\pgfqpoint{4.890999in}{0.692527in}}%
\pgfpathlineto{\pgfqpoint{4.893037in}{0.678876in}}%
\pgfpathlineto{\pgfqpoint{4.893765in}{0.683705in}}%
\pgfpathlineto{\pgfqpoint{4.894638in}{0.721741in}}%
\pgfpathlineto{\pgfqpoint{4.895366in}{0.751445in}}%
\pgfpathlineto{\pgfqpoint{4.895948in}{0.735819in}}%
\pgfpathlineto{\pgfqpoint{4.898859in}{0.681678in}}%
\pgfpathlineto{\pgfqpoint{4.899150in}{0.680802in}}%
\pgfpathlineto{\pgfqpoint{4.899441in}{0.682148in}}%
\pgfpathlineto{\pgfqpoint{4.900023in}{0.700485in}}%
\pgfpathlineto{\pgfqpoint{4.900751in}{0.743055in}}%
\pgfpathlineto{\pgfqpoint{4.901333in}{0.725227in}}%
\pgfpathlineto{\pgfqpoint{4.902934in}{0.699881in}}%
\pgfpathlineto{\pgfqpoint{4.905408in}{0.681769in}}%
\pgfpathlineto{\pgfqpoint{4.905845in}{0.685144in}}%
\pgfpathlineto{\pgfqpoint{4.907301in}{0.709525in}}%
\pgfpathlineto{\pgfqpoint{4.907883in}{0.703045in}}%
\pgfpathlineto{\pgfqpoint{4.910648in}{0.675910in}}%
\pgfpathlineto{\pgfqpoint{4.912395in}{0.668569in}}%
\pgfpathlineto{\pgfqpoint{4.913122in}{0.667596in}}%
\pgfpathlineto{\pgfqpoint{4.915160in}{0.664062in}}%
\pgfpathlineto{\pgfqpoint{4.917198in}{0.663659in}}%
\pgfpathlineto{\pgfqpoint{4.919381in}{0.664447in}}%
\pgfpathlineto{\pgfqpoint{4.923311in}{0.666638in}}%
\pgfpathlineto{\pgfqpoint{4.928987in}{0.669635in}}%
\pgfpathlineto{\pgfqpoint{4.931461in}{0.670545in}}%
\pgfpathlineto{\pgfqpoint{4.934954in}{0.671656in}}%
\pgfpathlineto{\pgfqpoint{4.935827in}{0.674457in}}%
\pgfpathlineto{\pgfqpoint{4.936410in}{0.672317in}}%
\pgfpathlineto{\pgfqpoint{4.938011in}{0.671402in}}%
\pgfpathlineto{\pgfqpoint{4.938884in}{0.670611in}}%
\pgfpathlineto{\pgfqpoint{4.939030in}{0.670985in}}%
\pgfpathlineto{\pgfqpoint{4.939612in}{0.674638in}}%
\pgfpathlineto{\pgfqpoint{4.940194in}{0.671524in}}%
\pgfpathlineto{\pgfqpoint{4.940339in}{0.671492in}}%
\pgfpathlineto{\pgfqpoint{4.940485in}{0.672398in}}%
\pgfpathlineto{\pgfqpoint{4.941504in}{0.692409in}}%
\pgfpathlineto{\pgfqpoint{4.941940in}{0.683551in}}%
\pgfpathlineto{\pgfqpoint{4.943105in}{0.665866in}}%
\pgfpathlineto{\pgfqpoint{4.943541in}{0.666012in}}%
\pgfpathlineto{\pgfqpoint{4.944851in}{0.666460in}}%
\pgfpathlineto{\pgfqpoint{4.945434in}{0.671044in}}%
\pgfpathlineto{\pgfqpoint{4.945870in}{0.674669in}}%
\pgfpathlineto{\pgfqpoint{4.946307in}{0.669481in}}%
\pgfpathlineto{\pgfqpoint{4.946889in}{0.666352in}}%
\pgfpathlineto{\pgfqpoint{4.947617in}{0.666824in}}%
\pgfpathlineto{\pgfqpoint{4.948490in}{0.668725in}}%
\pgfpathlineto{\pgfqpoint{4.949654in}{0.679105in}}%
\pgfpathlineto{\pgfqpoint{4.950382in}{0.674983in}}%
\pgfpathlineto{\pgfqpoint{4.950964in}{0.674127in}}%
\pgfpathlineto{\pgfqpoint{4.951401in}{0.674690in}}%
\pgfpathlineto{\pgfqpoint{4.952274in}{0.679148in}}%
\pgfpathlineto{\pgfqpoint{4.952711in}{0.675324in}}%
\pgfpathlineto{\pgfqpoint{4.953730in}{0.666037in}}%
\pgfpathlineto{\pgfqpoint{4.954312in}{0.666581in}}%
\pgfpathlineto{\pgfqpoint{4.954748in}{0.667835in}}%
\pgfpathlineto{\pgfqpoint{4.955185in}{0.682585in}}%
\pgfpathlineto{\pgfqpoint{4.956204in}{0.810768in}}%
\pgfpathlineto{\pgfqpoint{4.956932in}{0.751232in}}%
\pgfpathlineto{\pgfqpoint{4.957514in}{0.731630in}}%
\pgfpathlineto{\pgfqpoint{4.957950in}{0.743078in}}%
\pgfpathlineto{\pgfqpoint{4.958387in}{0.751723in}}%
\pgfpathlineto{\pgfqpoint{4.958678in}{0.742040in}}%
\pgfpathlineto{\pgfqpoint{4.960425in}{0.671182in}}%
\pgfpathlineto{\pgfqpoint{4.960716in}{0.671237in}}%
\pgfpathlineto{\pgfqpoint{4.961589in}{0.670849in}}%
\pgfpathlineto{\pgfqpoint{4.961735in}{0.671489in}}%
\pgfpathlineto{\pgfqpoint{4.962171in}{0.684721in}}%
\pgfpathlineto{\pgfqpoint{4.963045in}{0.765311in}}%
\pgfpathlineto{\pgfqpoint{4.963627in}{0.727175in}}%
\pgfpathlineto{\pgfqpoint{4.963918in}{0.716606in}}%
\pgfpathlineto{\pgfqpoint{4.964500in}{0.733697in}}%
\pgfpathlineto{\pgfqpoint{4.964646in}{0.737070in}}%
\pgfpathlineto{\pgfqpoint{4.964937in}{0.730477in}}%
\pgfpathlineto{\pgfqpoint{4.966829in}{0.671083in}}%
\pgfpathlineto{\pgfqpoint{4.966974in}{0.671106in}}%
\pgfpathlineto{\pgfqpoint{4.967848in}{0.672484in}}%
\pgfpathlineto{\pgfqpoint{4.971923in}{0.692483in}}%
\pgfpathlineto{\pgfqpoint{4.972214in}{0.690687in}}%
\pgfpathlineto{\pgfqpoint{4.973378in}{0.673442in}}%
\pgfpathlineto{\pgfqpoint{4.973815in}{0.676608in}}%
\pgfpathlineto{\pgfqpoint{4.974979in}{0.701480in}}%
\pgfpathlineto{\pgfqpoint{4.975853in}{0.693674in}}%
\pgfpathlineto{\pgfqpoint{4.976580in}{0.697579in}}%
\pgfpathlineto{\pgfqpoint{4.976871in}{0.695357in}}%
\pgfpathlineto{\pgfqpoint{4.978327in}{0.666660in}}%
\pgfpathlineto{\pgfqpoint{4.979200in}{0.667144in}}%
\pgfpathlineto{\pgfqpoint{4.980510in}{0.667830in}}%
\pgfpathlineto{\pgfqpoint{4.981383in}{0.677515in}}%
\pgfpathlineto{\pgfqpoint{4.981965in}{0.681878in}}%
\pgfpathlineto{\pgfqpoint{4.982548in}{0.679568in}}%
\pgfpathlineto{\pgfqpoint{4.985313in}{0.666934in}}%
\pgfpathlineto{\pgfqpoint{4.985750in}{0.668243in}}%
\pgfpathlineto{\pgfqpoint{4.986768in}{0.676040in}}%
\pgfpathlineto{\pgfqpoint{4.987351in}{0.671649in}}%
\pgfpathlineto{\pgfqpoint{4.987933in}{0.669044in}}%
\pgfpathlineto{\pgfqpoint{4.988515in}{0.670968in}}%
\pgfpathlineto{\pgfqpoint{4.988661in}{0.671309in}}%
\pgfpathlineto{\pgfqpoint{4.988952in}{0.670499in}}%
\pgfpathlineto{\pgfqpoint{4.989970in}{0.663598in}}%
\pgfpathlineto{\pgfqpoint{4.990698in}{0.664896in}}%
\pgfpathlineto{\pgfqpoint{4.991426in}{0.664286in}}%
\pgfpathlineto{\pgfqpoint{4.992445in}{0.670120in}}%
\pgfpathlineto{\pgfqpoint{4.992736in}{0.671208in}}%
\pgfpathlineto{\pgfqpoint{4.993172in}{0.667852in}}%
\pgfpathlineto{\pgfqpoint{4.994191in}{0.664839in}}%
\pgfpathlineto{\pgfqpoint{4.994628in}{0.665095in}}%
\pgfpathlineto{\pgfqpoint{4.996666in}{0.665571in}}%
\pgfpathlineto{\pgfqpoint{4.999431in}{0.668000in}}%
\pgfpathlineto{\pgfqpoint{5.002633in}{0.671453in}}%
\pgfpathlineto{\pgfqpoint{5.003361in}{0.679834in}}%
\pgfpathlineto{\pgfqpoint{5.003943in}{0.675938in}}%
\pgfpathlineto{\pgfqpoint{5.004234in}{0.674392in}}%
\pgfpathlineto{\pgfqpoint{5.004816in}{0.677539in}}%
\pgfpathlineto{\pgfqpoint{5.005253in}{0.680938in}}%
\pgfpathlineto{\pgfqpoint{5.005835in}{0.677300in}}%
\pgfpathlineto{\pgfqpoint{5.006563in}{0.673405in}}%
\pgfpathlineto{\pgfqpoint{5.006999in}{0.675727in}}%
\pgfpathlineto{\pgfqpoint{5.008164in}{0.703034in}}%
\pgfpathlineto{\pgfqpoint{5.008891in}{0.711323in}}%
\pgfpathlineto{\pgfqpoint{5.009328in}{0.708113in}}%
\pgfpathlineto{\pgfqpoint{5.012093in}{0.673154in}}%
\pgfpathlineto{\pgfqpoint{5.012967in}{0.682019in}}%
\pgfpathlineto{\pgfqpoint{5.014568in}{0.733388in}}%
\pgfpathlineto{\pgfqpoint{5.015441in}{0.720304in}}%
\pgfpathlineto{\pgfqpoint{5.018206in}{0.690072in}}%
\pgfpathlineto{\pgfqpoint{5.020390in}{0.678923in}}%
\pgfpathlineto{\pgfqpoint{5.022573in}{0.676023in}}%
\pgfpathlineto{\pgfqpoint{5.022864in}{0.676806in}}%
\pgfpathlineto{\pgfqpoint{5.023592in}{0.682148in}}%
\pgfpathlineto{\pgfqpoint{5.023883in}{0.679269in}}%
\pgfpathlineto{\pgfqpoint{5.025629in}{0.666758in}}%
\pgfpathlineto{\pgfqpoint{5.026939in}{0.667780in}}%
\pgfpathlineto{\pgfqpoint{5.027958in}{0.669439in}}%
\pgfpathlineto{\pgfqpoint{5.028540in}{0.668738in}}%
\pgfpathlineto{\pgfqpoint{5.029704in}{0.667953in}}%
\pgfpathlineto{\pgfqpoint{5.031305in}{0.661499in}}%
\pgfpathlineto{\pgfqpoint{5.031451in}{0.661531in}}%
\pgfpathlineto{\pgfqpoint{5.032470in}{0.662870in}}%
\pgfpathlineto{\pgfqpoint{5.033925in}{0.668736in}}%
\pgfpathlineto{\pgfqpoint{5.035235in}{0.666934in}}%
\pgfpathlineto{\pgfqpoint{5.037709in}{0.666156in}}%
\pgfpathlineto{\pgfqpoint{5.037855in}{0.666554in}}%
\pgfpathlineto{\pgfqpoint{5.038437in}{0.672306in}}%
\pgfpathlineto{\pgfqpoint{5.039310in}{0.691349in}}%
\pgfpathlineto{\pgfqpoint{5.039893in}{0.684198in}}%
\pgfpathlineto{\pgfqpoint{5.040620in}{0.675765in}}%
\pgfpathlineto{\pgfqpoint{5.041348in}{0.676574in}}%
\pgfpathlineto{\pgfqpoint{5.042512in}{0.665841in}}%
\pgfpathlineto{\pgfqpoint{5.043386in}{0.666946in}}%
\pgfpathlineto{\pgfqpoint{5.043968in}{0.667568in}}%
\pgfpathlineto{\pgfqpoint{5.044550in}{0.677969in}}%
\pgfpathlineto{\pgfqpoint{5.045132in}{0.689779in}}%
\pgfpathlineto{\pgfqpoint{5.045569in}{0.681876in}}%
\pgfpathlineto{\pgfqpoint{5.047170in}{0.670772in}}%
\pgfpathlineto{\pgfqpoint{5.048043in}{0.668708in}}%
\pgfpathlineto{\pgfqpoint{5.048480in}{0.669592in}}%
\pgfpathlineto{\pgfqpoint{5.051973in}{0.680577in}}%
\pgfpathlineto{\pgfqpoint{5.052846in}{0.684653in}}%
\pgfpathlineto{\pgfqpoint{5.053428in}{0.684583in}}%
\pgfpathlineto{\pgfqpoint{5.054011in}{0.684038in}}%
\pgfpathlineto{\pgfqpoint{5.055757in}{0.676000in}}%
\pgfpathlineto{\pgfqpoint{5.056485in}{0.679228in}}%
\pgfpathlineto{\pgfqpoint{5.056776in}{0.679764in}}%
\pgfpathlineto{\pgfqpoint{5.057067in}{0.677613in}}%
\pgfpathlineto{\pgfqpoint{5.058377in}{0.664022in}}%
\pgfpathlineto{\pgfqpoint{5.058959in}{0.664823in}}%
\pgfpathlineto{\pgfqpoint{5.060415in}{0.669356in}}%
\pgfpathlineto{\pgfqpoint{5.061142in}{0.667825in}}%
\pgfpathlineto{\pgfqpoint{5.062452in}{0.662571in}}%
\pgfpathlineto{\pgfqpoint{5.063180in}{0.663681in}}%
\pgfpathlineto{\pgfqpoint{5.065072in}{0.668304in}}%
\pgfpathlineto{\pgfqpoint{5.066382in}{0.670578in}}%
\pgfpathlineto{\pgfqpoint{5.066528in}{0.670480in}}%
\pgfpathlineto{\pgfqpoint{5.067546in}{0.667852in}}%
\pgfpathlineto{\pgfqpoint{5.067837in}{0.668791in}}%
\pgfpathlineto{\pgfqpoint{5.068711in}{0.687683in}}%
\pgfpathlineto{\pgfqpoint{5.069147in}{0.674624in}}%
\pgfpathlineto{\pgfqpoint{5.070021in}{0.651406in}}%
\pgfpathlineto{\pgfqpoint{5.070603in}{0.654021in}}%
\pgfpathlineto{\pgfqpoint{5.073514in}{0.704405in}}%
\pgfpathlineto{\pgfqpoint{5.073950in}{0.687630in}}%
\pgfpathlineto{\pgfqpoint{5.074824in}{0.636854in}}%
\pgfpathlineto{\pgfqpoint{5.075551in}{0.643233in}}%
\pgfpathlineto{\pgfqpoint{5.078171in}{0.657851in}}%
\pgfpathlineto{\pgfqpoint{5.081228in}{0.664496in}}%
\pgfpathlineto{\pgfqpoint{5.081373in}{0.664349in}}%
\pgfpathlineto{\pgfqpoint{5.081810in}{0.664065in}}%
\pgfpathlineto{\pgfqpoint{5.082392in}{0.664798in}}%
\pgfpathlineto{\pgfqpoint{5.083265in}{0.667354in}}%
\pgfpathlineto{\pgfqpoint{5.083993in}{0.666010in}}%
\pgfpathlineto{\pgfqpoint{5.084721in}{0.668076in}}%
\pgfpathlineto{\pgfqpoint{5.085157in}{0.669100in}}%
\pgfpathlineto{\pgfqpoint{5.085740in}{0.667829in}}%
\pgfpathlineto{\pgfqpoint{5.086613in}{0.667160in}}%
\pgfpathlineto{\pgfqpoint{5.087049in}{0.667431in}}%
\pgfpathlineto{\pgfqpoint{5.089815in}{0.669282in}}%
\pgfpathlineto{\pgfqpoint{5.092144in}{0.681256in}}%
\pgfpathlineto{\pgfqpoint{5.093308in}{0.669011in}}%
\pgfpathlineto{\pgfqpoint{5.094181in}{0.669465in}}%
\pgfpathlineto{\pgfqpoint{5.095200in}{0.670815in}}%
\pgfpathlineto{\pgfqpoint{5.095782in}{0.683553in}}%
\pgfpathlineto{\pgfqpoint{5.096510in}{0.707090in}}%
\pgfpathlineto{\pgfqpoint{5.097092in}{0.696474in}}%
\pgfpathlineto{\pgfqpoint{5.097383in}{0.693259in}}%
\pgfpathlineto{\pgfqpoint{5.097965in}{0.698259in}}%
\pgfpathlineto{\pgfqpoint{5.098111in}{0.699336in}}%
\pgfpathlineto{\pgfqpoint{5.098402in}{0.696950in}}%
\pgfpathlineto{\pgfqpoint{5.100149in}{0.670194in}}%
\pgfpathlineto{\pgfqpoint{5.100585in}{0.670363in}}%
\pgfpathlineto{\pgfqpoint{5.101313in}{0.671494in}}%
\pgfpathlineto{\pgfqpoint{5.102186in}{0.682750in}}%
\pgfpathlineto{\pgfqpoint{5.103351in}{0.679031in}}%
\pgfpathlineto{\pgfqpoint{5.103642in}{0.680024in}}%
\pgfpathlineto{\pgfqpoint{5.103933in}{0.678388in}}%
\pgfpathlineto{\pgfqpoint{5.105679in}{0.670133in}}%
\pgfpathlineto{\pgfqpoint{5.106262in}{0.670802in}}%
\pgfpathlineto{\pgfqpoint{5.107135in}{0.676863in}}%
\pgfpathlineto{\pgfqpoint{5.107717in}{0.673153in}}%
\pgfpathlineto{\pgfqpoint{5.109172in}{0.671374in}}%
\pgfpathlineto{\pgfqpoint{5.109609in}{0.671885in}}%
\pgfpathlineto{\pgfqpoint{5.110191in}{0.678562in}}%
\pgfpathlineto{\pgfqpoint{5.111356in}{0.715624in}}%
\pgfpathlineto{\pgfqpoint{5.111938in}{0.700936in}}%
\pgfpathlineto{\pgfqpoint{5.112520in}{0.691344in}}%
\pgfpathlineto{\pgfqpoint{5.112957in}{0.698509in}}%
\pgfpathlineto{\pgfqpoint{5.113248in}{0.702124in}}%
\pgfpathlineto{\pgfqpoint{5.113684in}{0.693707in}}%
\pgfpathlineto{\pgfqpoint{5.115140in}{0.669449in}}%
\pgfpathlineto{\pgfqpoint{5.115431in}{0.669773in}}%
\pgfpathlineto{\pgfqpoint{5.116159in}{0.675418in}}%
\pgfpathlineto{\pgfqpoint{5.117177in}{0.689674in}}%
\pgfpathlineto{\pgfqpoint{5.117905in}{0.686288in}}%
\pgfpathlineto{\pgfqpoint{5.118487in}{0.691149in}}%
\pgfpathlineto{\pgfqpoint{5.118778in}{0.688365in}}%
\pgfpathlineto{\pgfqpoint{5.119943in}{0.666373in}}%
\pgfpathlineto{\pgfqpoint{5.120525in}{0.666896in}}%
\pgfpathlineto{\pgfqpoint{5.121253in}{0.671614in}}%
\pgfpathlineto{\pgfqpoint{5.123145in}{0.678473in}}%
\pgfpathlineto{\pgfqpoint{5.123581in}{0.679390in}}%
\pgfpathlineto{\pgfqpoint{5.124164in}{0.678342in}}%
\pgfpathlineto{\pgfqpoint{5.125619in}{0.671273in}}%
\pgfpathlineto{\pgfqpoint{5.126056in}{0.674185in}}%
\pgfpathlineto{\pgfqpoint{5.126929in}{0.690395in}}%
\pgfpathlineto{\pgfqpoint{5.127511in}{0.680771in}}%
\pgfpathlineto{\pgfqpoint{5.128239in}{0.668749in}}%
\pgfpathlineto{\pgfqpoint{5.128967in}{0.669041in}}%
\pgfpathlineto{\pgfqpoint{5.129840in}{0.663098in}}%
\pgfpathlineto{\pgfqpoint{5.130131in}{0.665642in}}%
\pgfpathlineto{\pgfqpoint{5.132605in}{0.704787in}}%
\pgfpathlineto{\pgfqpoint{5.133042in}{0.692359in}}%
\pgfpathlineto{\pgfqpoint{5.134206in}{0.663256in}}%
\pgfpathlineto{\pgfqpoint{5.134643in}{0.664567in}}%
\pgfpathlineto{\pgfqpoint{5.135371in}{0.684941in}}%
\pgfpathlineto{\pgfqpoint{5.136098in}{0.704977in}}%
\pgfpathlineto{\pgfqpoint{5.136681in}{0.699215in}}%
\pgfpathlineto{\pgfqpoint{5.138136in}{0.678129in}}%
\pgfpathlineto{\pgfqpoint{5.139009in}{0.663732in}}%
\pgfpathlineto{\pgfqpoint{5.139591in}{0.667197in}}%
\pgfpathlineto{\pgfqpoint{5.140028in}{0.670016in}}%
\pgfpathlineto{\pgfqpoint{5.140465in}{0.665636in}}%
\pgfpathlineto{\pgfqpoint{5.141192in}{0.659471in}}%
\pgfpathlineto{\pgfqpoint{5.141920in}{0.660619in}}%
\pgfpathlineto{\pgfqpoint{5.144103in}{0.666802in}}%
\pgfpathlineto{\pgfqpoint{5.144831in}{0.675322in}}%
\pgfpathlineto{\pgfqpoint{5.145122in}{0.671096in}}%
\pgfpathlineto{\pgfqpoint{5.145995in}{0.656619in}}%
\pgfpathlineto{\pgfqpoint{5.146578in}{0.658846in}}%
\pgfpathlineto{\pgfqpoint{5.147160in}{0.662419in}}%
\pgfpathlineto{\pgfqpoint{5.147742in}{0.660124in}}%
\pgfpathlineto{\pgfqpoint{5.148179in}{0.659187in}}%
\pgfpathlineto{\pgfqpoint{5.148761in}{0.660078in}}%
\pgfpathlineto{\pgfqpoint{5.151963in}{0.664506in}}%
\pgfpathlineto{\pgfqpoint{5.153709in}{0.674635in}}%
\pgfpathlineto{\pgfqpoint{5.154583in}{0.718102in}}%
\pgfpathlineto{\pgfqpoint{5.155310in}{0.697710in}}%
\pgfpathlineto{\pgfqpoint{5.157930in}{0.668121in}}%
\pgfpathlineto{\pgfqpoint{5.158221in}{0.668542in}}%
\pgfpathlineto{\pgfqpoint{5.158803in}{0.675873in}}%
\pgfpathlineto{\pgfqpoint{5.159677in}{0.744546in}}%
\pgfpathlineto{\pgfqpoint{5.160113in}{0.767976in}}%
\pgfpathlineto{\pgfqpoint{5.160550in}{0.743259in}}%
\pgfpathlineto{\pgfqpoint{5.162297in}{0.695922in}}%
\pgfpathlineto{\pgfqpoint{5.164480in}{0.670896in}}%
\pgfpathlineto{\pgfqpoint{5.164916in}{0.673544in}}%
\pgfpathlineto{\pgfqpoint{5.165935in}{0.698797in}}%
\pgfpathlineto{\pgfqpoint{5.166809in}{0.687938in}}%
\pgfpathlineto{\pgfqpoint{5.167100in}{0.686614in}}%
\pgfpathlineto{\pgfqpoint{5.167682in}{0.689101in}}%
\pgfpathlineto{\pgfqpoint{5.168555in}{0.693978in}}%
\pgfpathlineto{\pgfqpoint{5.169137in}{0.692119in}}%
\pgfpathlineto{\pgfqpoint{5.172339in}{0.668959in}}%
\pgfpathlineto{\pgfqpoint{5.173067in}{0.673588in}}%
\pgfpathlineto{\pgfqpoint{5.174231in}{0.689013in}}%
\pgfpathlineto{\pgfqpoint{5.174959in}{0.686496in}}%
\pgfpathlineto{\pgfqpoint{5.175250in}{0.686361in}}%
\pgfpathlineto{\pgfqpoint{5.175541in}{0.687107in}}%
\pgfpathlineto{\pgfqpoint{5.175978in}{0.688609in}}%
\pgfpathlineto{\pgfqpoint{5.176415in}{0.686114in}}%
\pgfpathlineto{\pgfqpoint{5.178743in}{0.662655in}}%
\pgfpathlineto{\pgfqpoint{5.179617in}{0.663480in}}%
\pgfpathlineto{\pgfqpoint{5.180781in}{0.674690in}}%
\pgfpathlineto{\pgfqpoint{5.181800in}{0.671086in}}%
\pgfpathlineto{\pgfqpoint{5.182527in}{0.668208in}}%
\pgfpathlineto{\pgfqpoint{5.183546in}{0.662490in}}%
\pgfpathlineto{\pgfqpoint{5.184128in}{0.663927in}}%
\pgfpathlineto{\pgfqpoint{5.184856in}{0.666569in}}%
\pgfpathlineto{\pgfqpoint{5.185438in}{0.665294in}}%
\pgfpathlineto{\pgfqpoint{5.186166in}{0.664836in}}%
\pgfpathlineto{\pgfqpoint{5.186603in}{0.665135in}}%
\pgfpathlineto{\pgfqpoint{5.187767in}{0.664553in}}%
\pgfpathlineto{\pgfqpoint{5.189514in}{0.662645in}}%
\pgfpathlineto{\pgfqpoint{5.189659in}{0.663076in}}%
\pgfpathlineto{\pgfqpoint{5.190241in}{0.672229in}}%
\pgfpathlineto{\pgfqpoint{5.191115in}{0.698071in}}%
\pgfpathlineto{\pgfqpoint{5.191551in}{0.681194in}}%
\pgfpathlineto{\pgfqpoint{5.192133in}{0.664188in}}%
\pgfpathlineto{\pgfqpoint{5.192716in}{0.677498in}}%
\pgfpathlineto{\pgfqpoint{5.193007in}{0.685923in}}%
\pgfpathlineto{\pgfqpoint{5.193589in}{0.669405in}}%
\pgfpathlineto{\pgfqpoint{5.194317in}{0.656528in}}%
\pgfpathlineto{\pgfqpoint{5.194899in}{0.658554in}}%
\pgfpathlineto{\pgfqpoint{5.196936in}{0.662500in}}%
\pgfpathlineto{\pgfqpoint{5.199120in}{0.665465in}}%
\pgfpathlineto{\pgfqpoint{5.200284in}{0.663232in}}%
\pgfpathlineto{\pgfqpoint{5.201157in}{0.664538in}}%
\pgfpathlineto{\pgfqpoint{5.201885in}{0.666218in}}%
\pgfpathlineto{\pgfqpoint{5.202758in}{0.665619in}}%
\pgfpathlineto{\pgfqpoint{5.204796in}{0.669016in}}%
\pgfpathlineto{\pgfqpoint{5.205960in}{0.674667in}}%
\pgfpathlineto{\pgfqpoint{5.206542in}{0.673974in}}%
\pgfpathlineto{\pgfqpoint{5.207707in}{0.671013in}}%
\pgfpathlineto{\pgfqpoint{5.209599in}{0.664413in}}%
\pgfpathlineto{\pgfqpoint{5.210763in}{0.665523in}}%
\pgfpathlineto{\pgfqpoint{5.211345in}{0.666562in}}%
\pgfpathlineto{\pgfqpoint{5.211782in}{0.676383in}}%
\pgfpathlineto{\pgfqpoint{5.212364in}{0.702024in}}%
\pgfpathlineto{\pgfqpoint{5.212946in}{0.688458in}}%
\pgfpathlineto{\pgfqpoint{5.213383in}{0.682243in}}%
\pgfpathlineto{\pgfqpoint{5.213820in}{0.692015in}}%
\pgfpathlineto{\pgfqpoint{5.214256in}{0.701090in}}%
\pgfpathlineto{\pgfqpoint{5.214693in}{0.689125in}}%
\pgfpathlineto{\pgfqpoint{5.215712in}{0.667014in}}%
\pgfpathlineto{\pgfqpoint{5.216294in}{0.667603in}}%
\pgfpathlineto{\pgfqpoint{5.216876in}{0.668705in}}%
\pgfpathlineto{\pgfqpoint{5.217604in}{0.668124in}}%
\pgfpathlineto{\pgfqpoint{5.218623in}{0.669370in}}%
\pgfpathlineto{\pgfqpoint{5.219205in}{0.670335in}}%
\pgfpathlineto{\pgfqpoint{5.219642in}{0.669520in}}%
\pgfpathlineto{\pgfqpoint{5.220224in}{0.669213in}}%
\pgfpathlineto{\pgfqpoint{5.220515in}{0.670003in}}%
\pgfpathlineto{\pgfqpoint{5.220952in}{0.671749in}}%
\pgfpathlineto{\pgfqpoint{5.221388in}{0.669820in}}%
\pgfpathlineto{\pgfqpoint{5.221970in}{0.668977in}}%
\pgfpathlineto{\pgfqpoint{5.222407in}{0.669820in}}%
\pgfpathlineto{\pgfqpoint{5.223135in}{0.673259in}}%
\pgfpathlineto{\pgfqpoint{5.223717in}{0.672010in}}%
\pgfpathlineto{\pgfqpoint{5.224590in}{0.671429in}}%
\pgfpathlineto{\pgfqpoint{5.225027in}{0.671678in}}%
\pgfpathlineto{\pgfqpoint{5.226191in}{0.672504in}}%
\pgfpathlineto{\pgfqpoint{5.226773in}{0.686045in}}%
\pgfpathlineto{\pgfqpoint{5.227210in}{0.701885in}}%
\pgfpathlineto{\pgfqpoint{5.227792in}{0.682129in}}%
\pgfpathlineto{\pgfqpoint{5.229102in}{0.670645in}}%
\pgfpathlineto{\pgfqpoint{5.229393in}{0.670298in}}%
\pgfpathlineto{\pgfqpoint{5.229684in}{0.670933in}}%
\pgfpathlineto{\pgfqpoint{5.230266in}{0.680465in}}%
\pgfpathlineto{\pgfqpoint{5.232741in}{0.758660in}}%
\pgfpathlineto{\pgfqpoint{5.233032in}{0.737748in}}%
\pgfpathlineto{\pgfqpoint{5.234196in}{0.676221in}}%
\pgfpathlineto{\pgfqpoint{5.234633in}{0.677421in}}%
\pgfpathlineto{\pgfqpoint{5.235361in}{0.680267in}}%
\pgfpathlineto{\pgfqpoint{5.236088in}{0.679436in}}%
\pgfpathlineto{\pgfqpoint{5.236525in}{0.678164in}}%
\pgfpathlineto{\pgfqpoint{5.238271in}{0.668738in}}%
\pgfpathlineto{\pgfqpoint{5.238854in}{0.668822in}}%
\pgfpathlineto{\pgfqpoint{5.239727in}{0.669225in}}%
\pgfpathlineto{\pgfqpoint{5.240164in}{0.674005in}}%
\pgfpathlineto{\pgfqpoint{5.240891in}{0.694894in}}%
\pgfpathlineto{\pgfqpoint{5.241473in}{0.681230in}}%
\pgfpathlineto{\pgfqpoint{5.242783in}{0.673921in}}%
\pgfpathlineto{\pgfqpoint{5.243220in}{0.680208in}}%
\pgfpathlineto{\pgfqpoint{5.245985in}{0.765432in}}%
\pgfpathlineto{\pgfqpoint{5.246131in}{0.761378in}}%
\pgfpathlineto{\pgfqpoint{5.247586in}{0.685419in}}%
\pgfpathlineto{\pgfqpoint{5.248314in}{0.694581in}}%
\pgfpathlineto{\pgfqpoint{5.249624in}{0.736293in}}%
\pgfpathlineto{\pgfqpoint{5.250352in}{0.724624in}}%
\pgfpathlineto{\pgfqpoint{5.250934in}{0.718040in}}%
\pgfpathlineto{\pgfqpoint{5.251371in}{0.723557in}}%
\pgfpathlineto{\pgfqpoint{5.252244in}{0.742238in}}%
\pgfpathlineto{\pgfqpoint{5.252680in}{0.728863in}}%
\pgfpathlineto{\pgfqpoint{5.253990in}{0.685556in}}%
\pgfpathlineto{\pgfqpoint{5.254427in}{0.686384in}}%
\pgfpathlineto{\pgfqpoint{5.255591in}{0.693862in}}%
\pgfpathlineto{\pgfqpoint{5.256319in}{0.690886in}}%
\pgfpathlineto{\pgfqpoint{5.258211in}{0.676533in}}%
\pgfpathlineto{\pgfqpoint{5.259230in}{0.679938in}}%
\pgfpathlineto{\pgfqpoint{5.260540in}{0.684412in}}%
\pgfpathlineto{\pgfqpoint{5.260977in}{0.683370in}}%
\pgfpathlineto{\pgfqpoint{5.262141in}{0.676455in}}%
\pgfpathlineto{\pgfqpoint{5.263742in}{0.665693in}}%
\pgfpathlineto{\pgfqpoint{5.263887in}{0.665829in}}%
\pgfpathlineto{\pgfqpoint{5.264906in}{0.668542in}}%
\pgfpathlineto{\pgfqpoint{5.265488in}{0.667113in}}%
\pgfpathlineto{\pgfqpoint{5.269418in}{0.660908in}}%
\pgfpathlineto{\pgfqpoint{5.270292in}{0.661180in}}%
\pgfpathlineto{\pgfqpoint{5.270583in}{0.660721in}}%
\pgfpathlineto{\pgfqpoint{5.271456in}{0.659950in}}%
\pgfpathlineto{\pgfqpoint{5.271893in}{0.660299in}}%
\pgfpathlineto{\pgfqpoint{5.274367in}{0.663868in}}%
\pgfpathlineto{\pgfqpoint{5.278297in}{0.668408in}}%
\pgfpathlineto{\pgfqpoint{5.279898in}{0.668638in}}%
\pgfpathlineto{\pgfqpoint{5.281935in}{0.668883in}}%
\pgfpathlineto{\pgfqpoint{5.284409in}{0.668847in}}%
\pgfpathlineto{\pgfqpoint{5.285137in}{0.670024in}}%
\pgfpathlineto{\pgfqpoint{5.286447in}{0.676800in}}%
\pgfpathlineto{\pgfqpoint{5.287466in}{0.676143in}}%
\pgfpathlineto{\pgfqpoint{5.288048in}{0.674801in}}%
\pgfpathlineto{\pgfqpoint{5.289795in}{0.668107in}}%
\pgfpathlineto{\pgfqpoint{5.290231in}{0.668893in}}%
\pgfpathlineto{\pgfqpoint{5.290668in}{0.676345in}}%
\pgfpathlineto{\pgfqpoint{5.291687in}{0.761053in}}%
\pgfpathlineto{\pgfqpoint{5.292560in}{0.720193in}}%
\pgfpathlineto{\pgfqpoint{5.295616in}{0.674907in}}%
\pgfpathlineto{\pgfqpoint{5.295762in}{0.675214in}}%
\pgfpathlineto{\pgfqpoint{5.298818in}{0.696423in}}%
\pgfpathlineto{\pgfqpoint{5.299255in}{0.688901in}}%
\pgfpathlineto{\pgfqpoint{5.301147in}{0.666326in}}%
\pgfpathlineto{\pgfqpoint{5.303185in}{0.666611in}}%
\pgfpathlineto{\pgfqpoint{5.303913in}{0.670383in}}%
\pgfpathlineto{\pgfqpoint{5.304349in}{0.667050in}}%
\pgfpathlineto{\pgfqpoint{5.305077in}{0.664073in}}%
\pgfpathlineto{\pgfqpoint{5.305659in}{0.664402in}}%
\pgfpathlineto{\pgfqpoint{5.307842in}{0.666838in}}%
\pgfpathlineto{\pgfqpoint{5.308424in}{0.669171in}}%
\pgfpathlineto{\pgfqpoint{5.309152in}{0.668009in}}%
\pgfpathlineto{\pgfqpoint{5.309734in}{0.668608in}}%
\pgfpathlineto{\pgfqpoint{5.312209in}{0.673608in}}%
\pgfpathlineto{\pgfqpoint{5.312354in}{0.673508in}}%
\pgfpathlineto{\pgfqpoint{5.313082in}{0.672261in}}%
\pgfpathlineto{\pgfqpoint{5.313519in}{0.673579in}}%
\pgfpathlineto{\pgfqpoint{5.316429in}{0.697996in}}%
\pgfpathlineto{\pgfqpoint{5.317448in}{0.716468in}}%
\pgfpathlineto{\pgfqpoint{5.318030in}{0.712527in}}%
\pgfpathlineto{\pgfqpoint{5.318467in}{0.708630in}}%
\pgfpathlineto{\pgfqpoint{5.318758in}{0.712866in}}%
\pgfpathlineto{\pgfqpoint{5.319923in}{0.798802in}}%
\pgfpathlineto{\pgfqpoint{5.320650in}{0.750461in}}%
\pgfpathlineto{\pgfqpoint{5.322397in}{0.704458in}}%
\pgfpathlineto{\pgfqpoint{5.325308in}{0.674540in}}%
\pgfpathlineto{\pgfqpoint{5.326036in}{0.675332in}}%
\pgfpathlineto{\pgfqpoint{5.328219in}{0.682555in}}%
\pgfpathlineto{\pgfqpoint{5.329238in}{0.684623in}}%
\pgfpathlineto{\pgfqpoint{5.329674in}{0.683605in}}%
\pgfpathlineto{\pgfqpoint{5.330256in}{0.682529in}}%
\pgfpathlineto{\pgfqpoint{5.330693in}{0.683671in}}%
\pgfpathlineto{\pgfqpoint{5.331130in}{0.684707in}}%
\pgfpathlineto{\pgfqpoint{5.331566in}{0.683495in}}%
\pgfpathlineto{\pgfqpoint{5.334914in}{0.668717in}}%
\pgfpathlineto{\pgfqpoint{5.336951in}{0.666029in}}%
\pgfpathlineto{\pgfqpoint{5.337388in}{0.667010in}}%
\pgfpathlineto{\pgfqpoint{5.337970in}{0.668336in}}%
\pgfpathlineto{\pgfqpoint{5.338261in}{0.667188in}}%
\pgfpathlineto{\pgfqpoint{5.339135in}{0.663569in}}%
\pgfpathlineto{\pgfqpoint{5.339862in}{0.664044in}}%
\pgfpathlineto{\pgfqpoint{5.341609in}{0.663658in}}%
\pgfpathlineto{\pgfqpoint{5.349614in}{0.668842in}}%
\pgfpathlineto{\pgfqpoint{5.354271in}{0.669832in}}%
\pgfpathlineto{\pgfqpoint{5.356163in}{0.670402in}}%
\pgfpathlineto{\pgfqpoint{5.360675in}{0.671253in}}%
\pgfpathlineto{\pgfqpoint{5.361840in}{0.670422in}}%
\pgfpathlineto{\pgfqpoint{5.362131in}{0.670810in}}%
\pgfpathlineto{\pgfqpoint{5.363586in}{0.673808in}}%
\pgfpathlineto{\pgfqpoint{5.364460in}{0.721697in}}%
\pgfpathlineto{\pgfqpoint{5.365042in}{0.697681in}}%
\pgfpathlineto{\pgfqpoint{5.365478in}{0.686031in}}%
\pgfpathlineto{\pgfqpoint{5.365915in}{0.697699in}}%
\pgfpathlineto{\pgfqpoint{5.366643in}{0.744670in}}%
\pgfpathlineto{\pgfqpoint{5.367079in}{0.719206in}}%
\pgfpathlineto{\pgfqpoint{5.368244in}{0.673308in}}%
\pgfpathlineto{\pgfqpoint{5.368680in}{0.673368in}}%
\pgfpathlineto{\pgfqpoint{5.370718in}{0.672190in}}%
\pgfpathlineto{\pgfqpoint{5.371155in}{0.673222in}}%
\pgfpathlineto{\pgfqpoint{5.372028in}{0.678275in}}%
\pgfpathlineto{\pgfqpoint{5.372901in}{0.677197in}}%
\pgfpathlineto{\pgfqpoint{5.373774in}{0.677240in}}%
\pgfpathlineto{\pgfqpoint{5.373920in}{0.677533in}}%
\pgfpathlineto{\pgfqpoint{5.376394in}{0.686035in}}%
\pgfpathlineto{\pgfqpoint{5.377268in}{0.718898in}}%
\pgfpathlineto{\pgfqpoint{5.377995in}{0.755805in}}%
\pgfpathlineto{\pgfqpoint{5.378577in}{0.737840in}}%
\pgfpathlineto{\pgfqpoint{5.380470in}{0.690310in}}%
\pgfpathlineto{\pgfqpoint{5.382362in}{0.670943in}}%
\pgfpathlineto{\pgfqpoint{5.382944in}{0.671007in}}%
\pgfpathlineto{\pgfqpoint{5.383089in}{0.671404in}}%
\pgfpathlineto{\pgfqpoint{5.383672in}{0.679943in}}%
\pgfpathlineto{\pgfqpoint{5.384836in}{0.712412in}}%
\pgfpathlineto{\pgfqpoint{5.385564in}{0.711111in}}%
\pgfpathlineto{\pgfqpoint{5.386146in}{0.712469in}}%
\pgfpathlineto{\pgfqpoint{5.386437in}{0.711404in}}%
\pgfpathlineto{\pgfqpoint{5.387747in}{0.694132in}}%
\pgfpathlineto{\pgfqpoint{5.388911in}{0.679788in}}%
\pgfpathlineto{\pgfqpoint{5.389348in}{0.681801in}}%
\pgfpathlineto{\pgfqpoint{5.389930in}{0.685116in}}%
\pgfpathlineto{\pgfqpoint{5.390658in}{0.683566in}}%
\pgfpathlineto{\pgfqpoint{5.391094in}{0.686114in}}%
\pgfpathlineto{\pgfqpoint{5.392695in}{0.710152in}}%
\pgfpathlineto{\pgfqpoint{5.393423in}{0.705178in}}%
\pgfpathlineto{\pgfqpoint{5.394442in}{0.697730in}}%
\pgfpathlineto{\pgfqpoint{5.395024in}{0.699170in}}%
\pgfpathlineto{\pgfqpoint{5.395461in}{0.696043in}}%
\pgfpathlineto{\pgfqpoint{5.397353in}{0.667869in}}%
\pgfpathlineto{\pgfqpoint{5.397935in}{0.668613in}}%
\pgfpathlineto{\pgfqpoint{5.398808in}{0.670805in}}%
\pgfpathlineto{\pgfqpoint{5.399391in}{0.669847in}}%
\pgfpathlineto{\pgfqpoint{5.400846in}{0.665019in}}%
\pgfpathlineto{\pgfqpoint{5.402593in}{0.659145in}}%
\pgfpathlineto{\pgfqpoint{5.403029in}{0.659690in}}%
\pgfpathlineto{\pgfqpoint{5.404485in}{0.663625in}}%
\pgfpathlineto{\pgfqpoint{5.405503in}{0.662818in}}%
\pgfpathlineto{\pgfqpoint{5.406522in}{0.663584in}}%
\pgfpathlineto{\pgfqpoint{5.406813in}{0.663829in}}%
\pgfpathlineto{\pgfqpoint{5.407104in}{0.662993in}}%
\pgfpathlineto{\pgfqpoint{5.407832in}{0.659832in}}%
\pgfpathlineto{\pgfqpoint{5.408414in}{0.660986in}}%
\pgfpathlineto{\pgfqpoint{5.411180in}{0.667840in}}%
\pgfpathlineto{\pgfqpoint{5.411762in}{0.666660in}}%
\pgfpathlineto{\pgfqpoint{5.413654in}{0.664965in}}%
\pgfpathlineto{\pgfqpoint{5.415109in}{0.666485in}}%
\pgfpathlineto{\pgfqpoint{5.415837in}{0.672161in}}%
\pgfpathlineto{\pgfqpoint{5.416565in}{0.679229in}}%
\pgfpathlineto{\pgfqpoint{5.417002in}{0.675789in}}%
\pgfpathlineto{\pgfqpoint{5.418748in}{0.669365in}}%
\pgfpathlineto{\pgfqpoint{5.419476in}{0.667035in}}%
\pgfpathlineto{\pgfqpoint{5.420058in}{0.667463in}}%
\pgfpathlineto{\pgfqpoint{5.421659in}{0.670108in}}%
\pgfpathlineto{\pgfqpoint{5.422823in}{0.684088in}}%
\pgfpathlineto{\pgfqpoint{5.423260in}{0.677683in}}%
\pgfpathlineto{\pgfqpoint{5.423842in}{0.671343in}}%
\pgfpathlineto{\pgfqpoint{5.424279in}{0.674946in}}%
\pgfpathlineto{\pgfqpoint{5.424715in}{0.679241in}}%
\pgfpathlineto{\pgfqpoint{5.425007in}{0.675273in}}%
\pgfpathlineto{\pgfqpoint{5.425880in}{0.660100in}}%
\pgfpathlineto{\pgfqpoint{5.426608in}{0.661662in}}%
\pgfpathlineto{\pgfqpoint{5.429664in}{0.665921in}}%
\pgfpathlineto{\pgfqpoint{5.430683in}{0.669190in}}%
\pgfpathlineto{\pgfqpoint{5.431265in}{0.670335in}}%
\pgfpathlineto{\pgfqpoint{5.431847in}{0.669284in}}%
\pgfpathlineto{\pgfqpoint{5.432575in}{0.668786in}}%
\pgfpathlineto{\pgfqpoint{5.433012in}{0.669229in}}%
\pgfpathlineto{\pgfqpoint{5.433594in}{0.669415in}}%
\pgfpathlineto{\pgfqpoint{5.433885in}{0.668940in}}%
\pgfpathlineto{\pgfqpoint{5.434904in}{0.668145in}}%
\pgfpathlineto{\pgfqpoint{5.435340in}{0.668442in}}%
\pgfpathlineto{\pgfqpoint{5.436359in}{0.670958in}}%
\pgfpathlineto{\pgfqpoint{5.437378in}{0.674870in}}%
\pgfpathlineto{\pgfqpoint{5.437960in}{0.673821in}}%
\pgfpathlineto{\pgfqpoint{5.438251in}{0.673798in}}%
\pgfpathlineto{\pgfqpoint{5.438542in}{0.674752in}}%
\pgfpathlineto{\pgfqpoint{5.439270in}{0.679408in}}%
\pgfpathlineto{\pgfqpoint{5.439707in}{0.675903in}}%
\pgfpathlineto{\pgfqpoint{5.440580in}{0.667815in}}%
\pgfpathlineto{\pgfqpoint{5.441162in}{0.668279in}}%
\pgfpathlineto{\pgfqpoint{5.444219in}{0.669670in}}%
\pgfpathlineto{\pgfqpoint{5.450186in}{0.671677in}}%
\pgfpathlineto{\pgfqpoint{5.451350in}{0.677505in}}%
\pgfpathlineto{\pgfqpoint{5.452369in}{0.676071in}}%
\pgfpathlineto{\pgfqpoint{5.453679in}{0.676555in}}%
\pgfpathlineto{\pgfqpoint{5.454261in}{0.678782in}}%
\pgfpathlineto{\pgfqpoint{5.454552in}{0.677788in}}%
\pgfpathlineto{\pgfqpoint{5.455862in}{0.667313in}}%
\pgfpathlineto{\pgfqpoint{5.456299in}{0.669105in}}%
\pgfpathlineto{\pgfqpoint{5.456736in}{0.689597in}}%
\pgfpathlineto{\pgfqpoint{5.457463in}{0.794677in}}%
\pgfpathlineto{\pgfqpoint{5.458191in}{0.735392in}}%
\pgfpathlineto{\pgfqpoint{5.458628in}{0.721986in}}%
\pgfpathlineto{\pgfqpoint{5.459064in}{0.742322in}}%
\pgfpathlineto{\pgfqpoint{5.459646in}{0.764811in}}%
\pgfpathlineto{\pgfqpoint{5.460083in}{0.749566in}}%
\pgfpathlineto{\pgfqpoint{5.462412in}{0.683205in}}%
\pgfpathlineto{\pgfqpoint{5.463140in}{0.678377in}}%
\pgfpathlineto{\pgfqpoint{5.465032in}{0.669503in}}%
\pgfpathlineto{\pgfqpoint{5.465759in}{0.670342in}}%
\pgfpathlineto{\pgfqpoint{5.466342in}{0.682546in}}%
\pgfpathlineto{\pgfqpoint{5.467215in}{0.721281in}}%
\pgfpathlineto{\pgfqpoint{5.467797in}{0.710381in}}%
\pgfpathlineto{\pgfqpoint{5.469544in}{0.692188in}}%
\pgfpathlineto{\pgfqpoint{5.473910in}{0.670965in}}%
\pgfpathlineto{\pgfqpoint{5.474201in}{0.672015in}}%
\pgfpathlineto{\pgfqpoint{5.475948in}{0.686991in}}%
\pgfpathlineto{\pgfqpoint{5.476821in}{0.740197in}}%
\pgfpathlineto{\pgfqpoint{5.477403in}{0.706877in}}%
\pgfpathlineto{\pgfqpoint{5.478276in}{0.675038in}}%
\pgfpathlineto{\pgfqpoint{5.478858in}{0.678103in}}%
\pgfpathlineto{\pgfqpoint{5.480314in}{0.664487in}}%
\pgfpathlineto{\pgfqpoint{5.481333in}{0.665402in}}%
\pgfpathlineto{\pgfqpoint{5.482206in}{0.666931in}}%
\pgfpathlineto{\pgfqpoint{5.482788in}{0.665791in}}%
\pgfpathlineto{\pgfqpoint{5.483225in}{0.665815in}}%
\pgfpathlineto{\pgfqpoint{5.483516in}{0.666438in}}%
\pgfpathlineto{\pgfqpoint{5.483807in}{0.666935in}}%
\pgfpathlineto{\pgfqpoint{5.484389in}{0.665673in}}%
\pgfpathlineto{\pgfqpoint{5.485699in}{0.665467in}}%
\pgfpathlineto{\pgfqpoint{5.486427in}{0.667268in}}%
\pgfpathlineto{\pgfqpoint{5.487009in}{0.669170in}}%
\pgfpathlineto{\pgfqpoint{5.487737in}{0.668296in}}%
\pgfpathlineto{\pgfqpoint{5.490211in}{0.669027in}}%
\pgfpathlineto{\pgfqpoint{5.491230in}{0.671878in}}%
\pgfpathlineto{\pgfqpoint{5.492685in}{0.685764in}}%
\pgfpathlineto{\pgfqpoint{5.493704in}{0.683269in}}%
\pgfpathlineto{\pgfqpoint{5.494577in}{0.675618in}}%
\pgfpathlineto{\pgfqpoint{5.495887in}{0.668050in}}%
\pgfpathlineto{\pgfqpoint{5.496178in}{0.668350in}}%
\pgfpathlineto{\pgfqpoint{5.496615in}{0.671511in}}%
\pgfpathlineto{\pgfqpoint{5.497197in}{0.697995in}}%
\pgfpathlineto{\pgfqpoint{5.498071in}{0.774253in}}%
\pgfpathlineto{\pgfqpoint{5.498653in}{0.749235in}}%
\pgfpathlineto{\pgfqpoint{5.500690in}{0.691730in}}%
\pgfpathlineto{\pgfqpoint{5.502437in}{0.669743in}}%
\pgfpathlineto{\pgfqpoint{5.503456in}{0.670206in}}%
\pgfpathlineto{\pgfqpoint{5.504038in}{0.674235in}}%
\pgfpathlineto{\pgfqpoint{5.505348in}{0.702123in}}%
\pgfpathlineto{\pgfqpoint{5.506367in}{0.695825in}}%
\pgfpathlineto{\pgfqpoint{5.507094in}{0.693298in}}%
\pgfpathlineto{\pgfqpoint{5.509860in}{0.672855in}}%
\pgfpathlineto{\pgfqpoint{5.511897in}{0.671090in}}%
\pgfpathlineto{\pgfqpoint{5.514517in}{0.668402in}}%
\pgfpathlineto{\pgfqpoint{5.515827in}{0.667947in}}%
\pgfpathlineto{\pgfqpoint{5.515973in}{0.668123in}}%
\pgfpathlineto{\pgfqpoint{5.516846in}{0.671750in}}%
\pgfpathlineto{\pgfqpoint{5.517719in}{0.676288in}}%
\pgfpathlineto{\pgfqpoint{5.518301in}{0.675361in}}%
\pgfpathlineto{\pgfqpoint{5.520193in}{0.669585in}}%
\pgfpathlineto{\pgfqpoint{5.521212in}{0.667704in}}%
\pgfpathlineto{\pgfqpoint{5.521503in}{0.668180in}}%
\pgfpathlineto{\pgfqpoint{5.522231in}{0.673606in}}%
\pgfpathlineto{\pgfqpoint{5.522959in}{0.680749in}}%
\pgfpathlineto{\pgfqpoint{5.523541in}{0.677916in}}%
\pgfpathlineto{\pgfqpoint{5.525724in}{0.670005in}}%
\pgfpathlineto{\pgfqpoint{5.526597in}{0.669081in}}%
\pgfpathlineto{\pgfqpoint{5.526889in}{0.669606in}}%
\pgfpathlineto{\pgfqpoint{5.527616in}{0.678316in}}%
\pgfpathlineto{\pgfqpoint{5.529217in}{0.687572in}}%
\pgfpathlineto{\pgfqpoint{5.530236in}{0.691096in}}%
\pgfpathlineto{\pgfqpoint{5.530673in}{0.689806in}}%
\pgfpathlineto{\pgfqpoint{5.532856in}{0.675009in}}%
\pgfpathlineto{\pgfqpoint{5.533438in}{0.670362in}}%
\pgfpathlineto{\pgfqpoint{5.533875in}{0.673067in}}%
\pgfpathlineto{\pgfqpoint{5.537077in}{0.730108in}}%
\pgfpathlineto{\pgfqpoint{5.537513in}{0.740936in}}%
\pgfpathlineto{\pgfqpoint{5.537950in}{0.725280in}}%
\pgfpathlineto{\pgfqpoint{5.539260in}{0.683088in}}%
\pgfpathlineto{\pgfqpoint{5.539697in}{0.685739in}}%
\pgfpathlineto{\pgfqpoint{5.543335in}{0.743480in}}%
\pgfpathlineto{\pgfqpoint{5.544063in}{0.763986in}}%
\pgfpathlineto{\pgfqpoint{5.544645in}{0.757075in}}%
\pgfpathlineto{\pgfqpoint{5.545518in}{0.744543in}}%
\pgfpathlineto{\pgfqpoint{5.546101in}{0.747165in}}%
\pgfpathlineto{\pgfqpoint{5.546246in}{0.747650in}}%
\pgfpathlineto{\pgfqpoint{5.546683in}{0.745326in}}%
\pgfpathlineto{\pgfqpoint{5.548429in}{0.701851in}}%
\pgfpathlineto{\pgfqpoint{5.550030in}{0.663905in}}%
\pgfpathlineto{\pgfqpoint{5.550467in}{0.669216in}}%
\pgfpathlineto{\pgfqpoint{5.551922in}{0.729684in}}%
\pgfpathlineto{\pgfqpoint{5.552941in}{0.781712in}}%
\pgfpathlineto{\pgfqpoint{5.553523in}{0.768900in}}%
\pgfpathlineto{\pgfqpoint{5.555270in}{0.708310in}}%
\pgfpathlineto{\pgfqpoint{5.555707in}{0.717871in}}%
\pgfpathlineto{\pgfqpoint{5.556725in}{0.810385in}}%
\pgfpathlineto{\pgfqpoint{5.557453in}{0.767721in}}%
\pgfpathlineto{\pgfqpoint{5.559345in}{0.703265in}}%
\pgfpathlineto{\pgfqpoint{5.561237in}{0.668500in}}%
\pgfpathlineto{\pgfqpoint{5.561674in}{0.669867in}}%
\pgfpathlineto{\pgfqpoint{5.562693in}{0.676435in}}%
\pgfpathlineto{\pgfqpoint{5.563129in}{0.674321in}}%
\pgfpathlineto{\pgfqpoint{5.565895in}{0.663480in}}%
\pgfpathlineto{\pgfqpoint{5.567059in}{0.661107in}}%
\pgfpathlineto{\pgfqpoint{5.567496in}{0.661239in}}%
\pgfpathlineto{\pgfqpoint{5.572735in}{0.664037in}}%
\pgfpathlineto{\pgfqpoint{5.577684in}{0.668679in}}%
\pgfpathlineto{\pgfqpoint{5.583360in}{0.670842in}}%
\pgfpathlineto{\pgfqpoint{5.588454in}{0.671466in}}%
\pgfpathlineto{\pgfqpoint{5.591365in}{0.671983in}}%
\pgfpathlineto{\pgfqpoint{5.594276in}{0.672947in}}%
\pgfpathlineto{\pgfqpoint{5.598934in}{0.685574in}}%
\pgfpathlineto{\pgfqpoint{5.600535in}{0.724503in}}%
\pgfpathlineto{\pgfqpoint{5.601117in}{0.710382in}}%
\pgfpathlineto{\pgfqpoint{5.603591in}{0.668549in}}%
\pgfpathlineto{\pgfqpoint{5.604901in}{0.668811in}}%
\pgfpathlineto{\pgfqpoint{5.605483in}{0.671289in}}%
\pgfpathlineto{\pgfqpoint{5.608394in}{0.696272in}}%
\pgfpathlineto{\pgfqpoint{5.608685in}{0.695322in}}%
\pgfpathlineto{\pgfqpoint{5.610577in}{0.681698in}}%
\pgfpathlineto{\pgfqpoint{5.612033in}{0.667208in}}%
\pgfpathlineto{\pgfqpoint{5.612178in}{0.667241in}}%
\pgfpathlineto{\pgfqpoint{5.615089in}{0.669079in}}%
\pgfpathlineto{\pgfqpoint{5.615380in}{0.668258in}}%
\pgfpathlineto{\pgfqpoint{5.616254in}{0.665833in}}%
\pgfpathlineto{\pgfqpoint{5.616836in}{0.666267in}}%
\pgfpathlineto{\pgfqpoint{5.617564in}{0.667285in}}%
\pgfpathlineto{\pgfqpoint{5.618291in}{0.666798in}}%
\pgfpathlineto{\pgfqpoint{5.619165in}{0.667618in}}%
\pgfpathlineto{\pgfqpoint{5.620766in}{0.675144in}}%
\pgfpathlineto{\pgfqpoint{5.621493in}{0.709546in}}%
\pgfpathlineto{\pgfqpoint{5.622075in}{0.690125in}}%
\pgfpathlineto{\pgfqpoint{5.623240in}{0.667354in}}%
\pgfpathlineto{\pgfqpoint{5.623531in}{0.667461in}}%
\pgfpathlineto{\pgfqpoint{5.624113in}{0.665868in}}%
\pgfpathlineto{\pgfqpoint{5.624550in}{0.664734in}}%
\pgfpathlineto{\pgfqpoint{5.624986in}{0.665625in}}%
\pgfpathlineto{\pgfqpoint{5.626005in}{0.673656in}}%
\pgfpathlineto{\pgfqpoint{5.627170in}{0.672566in}}%
\pgfpathlineto{\pgfqpoint{5.628043in}{0.670126in}}%
\pgfpathlineto{\pgfqpoint{5.628479in}{0.671675in}}%
\pgfpathlineto{\pgfqpoint{5.629353in}{0.681097in}}%
\pgfpathlineto{\pgfqpoint{5.629789in}{0.675343in}}%
\pgfpathlineto{\pgfqpoint{5.630517in}{0.668570in}}%
\pgfpathlineto{\pgfqpoint{5.631099in}{0.671331in}}%
\pgfpathlineto{\pgfqpoint{5.631245in}{0.671733in}}%
\pgfpathlineto{\pgfqpoint{5.631536in}{0.670480in}}%
\pgfpathlineto{\pgfqpoint{5.632264in}{0.664675in}}%
\pgfpathlineto{\pgfqpoint{5.632991in}{0.665466in}}%
\pgfpathlineto{\pgfqpoint{5.634738in}{0.667235in}}%
\pgfpathlineto{\pgfqpoint{5.637067in}{0.666752in}}%
\pgfpathlineto{\pgfqpoint{5.638522in}{0.666622in}}%
\pgfpathlineto{\pgfqpoint{5.639686in}{0.668028in}}%
\pgfpathlineto{\pgfqpoint{5.641142in}{0.674647in}}%
\pgfpathlineto{\pgfqpoint{5.642015in}{0.673744in}}%
\pgfpathlineto{\pgfqpoint{5.642743in}{0.674753in}}%
\pgfpathlineto{\pgfqpoint{5.643034in}{0.675176in}}%
\pgfpathlineto{\pgfqpoint{5.643325in}{0.674434in}}%
\pgfpathlineto{\pgfqpoint{5.644344in}{0.668860in}}%
\pgfpathlineto{\pgfqpoint{5.644926in}{0.670781in}}%
\pgfpathlineto{\pgfqpoint{5.647255in}{0.696395in}}%
\pgfpathlineto{\pgfqpoint{5.648128in}{0.786756in}}%
\pgfpathlineto{\pgfqpoint{5.648710in}{0.736189in}}%
\pgfpathlineto{\pgfqpoint{5.649584in}{0.677490in}}%
\pgfpathlineto{\pgfqpoint{5.650166in}{0.687725in}}%
\pgfpathlineto{\pgfqpoint{5.650457in}{0.690791in}}%
\pgfpathlineto{\pgfqpoint{5.650748in}{0.684125in}}%
\pgfpathlineto{\pgfqpoint{5.651621in}{0.667091in}}%
\pgfpathlineto{\pgfqpoint{5.652203in}{0.667567in}}%
\pgfpathlineto{\pgfqpoint{5.656715in}{0.669750in}}%
\pgfpathlineto{\pgfqpoint{5.657880in}{0.670233in}}%
\pgfpathlineto{\pgfqpoint{5.661955in}{0.671167in}}%
\pgfpathlineto{\pgfqpoint{5.663119in}{0.671355in}}%
\pgfpathlineto{\pgfqpoint{5.663702in}{0.673850in}}%
\pgfpathlineto{\pgfqpoint{5.664284in}{0.671844in}}%
\pgfpathlineto{\pgfqpoint{5.664575in}{0.672118in}}%
\pgfpathlineto{\pgfqpoint{5.665303in}{0.677289in}}%
\pgfpathlineto{\pgfqpoint{5.665885in}{0.673582in}}%
\pgfpathlineto{\pgfqpoint{5.667340in}{0.671152in}}%
\pgfpathlineto{\pgfqpoint{5.668505in}{0.671631in}}%
\pgfpathlineto{\pgfqpoint{5.669087in}{0.681464in}}%
\pgfpathlineto{\pgfqpoint{5.669669in}{0.691878in}}%
\pgfpathlineto{\pgfqpoint{5.670251in}{0.686759in}}%
\pgfpathlineto{\pgfqpoint{5.670542in}{0.685416in}}%
\pgfpathlineto{\pgfqpoint{5.670979in}{0.687673in}}%
\pgfpathlineto{\pgfqpoint{5.671561in}{0.715140in}}%
\pgfpathlineto{\pgfqpoint{5.672289in}{0.791023in}}%
\pgfpathlineto{\pgfqpoint{5.672725in}{0.746522in}}%
\pgfpathlineto{\pgfqpoint{5.673744in}{0.684275in}}%
\pgfpathlineto{\pgfqpoint{5.674181in}{0.684913in}}%
\pgfpathlineto{\pgfqpoint{5.675782in}{0.671738in}}%
\pgfpathlineto{\pgfqpoint{5.676655in}{0.672181in}}%
\pgfpathlineto{\pgfqpoint{5.677528in}{0.672944in}}%
\pgfpathlineto{\pgfqpoint{5.678838in}{0.680234in}}%
\pgfpathlineto{\pgfqpoint{5.679712in}{0.677041in}}%
\pgfpathlineto{\pgfqpoint{5.681167in}{0.676281in}}%
\pgfpathlineto{\pgfqpoint{5.683350in}{0.673336in}}%
\pgfpathlineto{\pgfqpoint{5.683641in}{0.674996in}}%
\pgfpathlineto{\pgfqpoint{5.684078in}{0.695377in}}%
\pgfpathlineto{\pgfqpoint{5.684806in}{0.763112in}}%
\pgfpathlineto{\pgfqpoint{5.685242in}{0.726908in}}%
\pgfpathlineto{\pgfqpoint{5.686261in}{0.686383in}}%
\pgfpathlineto{\pgfqpoint{5.686698in}{0.687093in}}%
\pgfpathlineto{\pgfqpoint{5.687280in}{0.681844in}}%
\pgfpathlineto{\pgfqpoint{5.688881in}{0.669999in}}%
\pgfpathlineto{\pgfqpoint{5.689026in}{0.670013in}}%
\pgfpathlineto{\pgfqpoint{5.689754in}{0.671507in}}%
\pgfpathlineto{\pgfqpoint{5.690482in}{0.675384in}}%
\pgfpathlineto{\pgfqpoint{5.691064in}{0.673537in}}%
\pgfpathlineto{\pgfqpoint{5.692520in}{0.672121in}}%
\pgfpathlineto{\pgfqpoint{5.693102in}{0.671752in}}%
\pgfpathlineto{\pgfqpoint{5.693538in}{0.672546in}}%
\pgfpathlineto{\pgfqpoint{5.693829in}{0.672747in}}%
\pgfpathlineto{\pgfqpoint{5.694121in}{0.671624in}}%
\pgfpathlineto{\pgfqpoint{5.694848in}{0.668452in}}%
\pgfpathlineto{\pgfqpoint{5.695430in}{0.669305in}}%
\pgfpathlineto{\pgfqpoint{5.695722in}{0.669759in}}%
\pgfpathlineto{\pgfqpoint{5.696158in}{0.668923in}}%
\pgfpathlineto{\pgfqpoint{5.697468in}{0.667698in}}%
\pgfpathlineto{\pgfqpoint{5.697614in}{0.667755in}}%
\pgfpathlineto{\pgfqpoint{5.704163in}{0.670320in}}%
\pgfpathlineto{\pgfqpoint{5.713769in}{0.674227in}}%
\pgfpathlineto{\pgfqpoint{5.714788in}{0.684060in}}%
\pgfpathlineto{\pgfqpoint{5.715370in}{0.678040in}}%
\pgfpathlineto{\pgfqpoint{5.715952in}{0.674813in}}%
\pgfpathlineto{\pgfqpoint{5.716535in}{0.676105in}}%
\pgfpathlineto{\pgfqpoint{5.717553in}{0.685791in}}%
\pgfpathlineto{\pgfqpoint{5.720028in}{0.741636in}}%
\pgfpathlineto{\pgfqpoint{5.720755in}{0.786256in}}%
\pgfpathlineto{\pgfqpoint{5.721338in}{0.769961in}}%
\pgfpathlineto{\pgfqpoint{5.723666in}{0.703101in}}%
\pgfpathlineto{\pgfqpoint{5.725267in}{0.684552in}}%
\pgfpathlineto{\pgfqpoint{5.725704in}{0.686983in}}%
\pgfpathlineto{\pgfqpoint{5.727014in}{0.715933in}}%
\pgfpathlineto{\pgfqpoint{5.727887in}{0.703637in}}%
\pgfpathlineto{\pgfqpoint{5.728615in}{0.700215in}}%
\pgfpathlineto{\pgfqpoint{5.729197in}{0.700940in}}%
\pgfpathlineto{\pgfqpoint{5.729925in}{0.703613in}}%
\pgfpathlineto{\pgfqpoint{5.730361in}{0.701029in}}%
\pgfpathlineto{\pgfqpoint{5.730653in}{0.699544in}}%
\pgfpathlineto{\pgfqpoint{5.731089in}{0.702885in}}%
\pgfpathlineto{\pgfqpoint{5.732108in}{0.716537in}}%
\pgfpathlineto{\pgfqpoint{5.732836in}{0.714370in}}%
\pgfpathlineto{\pgfqpoint{5.736620in}{0.694236in}}%
\pgfpathlineto{\pgfqpoint{5.736911in}{0.695937in}}%
\pgfpathlineto{\pgfqpoint{5.738075in}{0.737409in}}%
\pgfpathlineto{\pgfqpoint{5.738803in}{0.714267in}}%
\pgfpathlineto{\pgfqpoint{5.739967in}{0.693493in}}%
\pgfpathlineto{\pgfqpoint{5.740404in}{0.693582in}}%
\pgfpathlineto{\pgfqpoint{5.741132in}{0.691949in}}%
\pgfpathlineto{\pgfqpoint{5.742005in}{0.687666in}}%
\pgfpathlineto{\pgfqpoint{5.742442in}{0.690742in}}%
\pgfpathlineto{\pgfqpoint{5.743024in}{0.697325in}}%
\pgfpathlineto{\pgfqpoint{5.743461in}{0.692331in}}%
\pgfpathlineto{\pgfqpoint{5.744479in}{0.679525in}}%
\pgfpathlineto{\pgfqpoint{5.744916in}{0.681422in}}%
\pgfpathlineto{\pgfqpoint{5.748700in}{0.706726in}}%
\pgfpathlineto{\pgfqpoint{5.749282in}{0.701857in}}%
\pgfpathlineto{\pgfqpoint{5.752630in}{0.665069in}}%
\pgfpathlineto{\pgfqpoint{5.754376in}{0.656100in}}%
\pgfpathlineto{\pgfqpoint{5.754522in}{0.656617in}}%
\pgfpathlineto{\pgfqpoint{5.755250in}{0.668669in}}%
\pgfpathlineto{\pgfqpoint{5.756123in}{0.684718in}}%
\pgfpathlineto{\pgfqpoint{5.756560in}{0.679922in}}%
\pgfpathlineto{\pgfqpoint{5.758452in}{0.634644in}}%
\pgfpathlineto{\pgfqpoint{5.759325in}{0.586979in}}%
\pgfpathlineto{\pgfqpoint{5.759907in}{0.597319in}}%
\pgfpathlineto{\pgfqpoint{5.762090in}{0.658747in}}%
\pgfpathlineto{\pgfqpoint{5.762964in}{0.655853in}}%
\pgfpathlineto{\pgfqpoint{5.763546in}{0.660957in}}%
\pgfpathlineto{\pgfqpoint{5.763982in}{0.657437in}}%
\pgfpathlineto{\pgfqpoint{5.765147in}{0.640096in}}%
\pgfpathlineto{\pgfqpoint{5.765729in}{0.642544in}}%
\pgfpathlineto{\pgfqpoint{5.767330in}{0.648080in}}%
\pgfpathlineto{\pgfqpoint{5.767767in}{0.647843in}}%
\pgfpathlineto{\pgfqpoint{5.768349in}{0.648728in}}%
\pgfpathlineto{\pgfqpoint{5.769222in}{0.654234in}}%
\pgfpathlineto{\pgfqpoint{5.769804in}{0.650595in}}%
\pgfpathlineto{\pgfqpoint{5.770241in}{0.648258in}}%
\pgfpathlineto{\pgfqpoint{5.770823in}{0.650851in}}%
\pgfpathlineto{\pgfqpoint{5.773589in}{0.660888in}}%
\pgfpathlineto{\pgfqpoint{5.775190in}{0.665384in}}%
\pgfpathlineto{\pgfqpoint{5.776645in}{0.664249in}}%
\pgfpathlineto{\pgfqpoint{5.781594in}{0.667842in}}%
\pgfpathlineto{\pgfqpoint{5.786251in}{0.670486in}}%
\pgfpathlineto{\pgfqpoint{5.791636in}{0.671391in}}%
\pgfpathlineto{\pgfqpoint{5.793965in}{0.675879in}}%
\pgfpathlineto{\pgfqpoint{5.794547in}{0.682244in}}%
\pgfpathlineto{\pgfqpoint{5.794984in}{0.678420in}}%
\pgfpathlineto{\pgfqpoint{5.795857in}{0.671188in}}%
\pgfpathlineto{\pgfqpoint{5.796439in}{0.671932in}}%
\pgfpathlineto{\pgfqpoint{5.798477in}{0.670376in}}%
\pgfpathlineto{\pgfqpoint{5.798622in}{0.670708in}}%
\pgfpathlineto{\pgfqpoint{5.799350in}{0.680632in}}%
\pgfpathlineto{\pgfqpoint{5.799787in}{0.684025in}}%
\pgfpathlineto{\pgfqpoint{5.800223in}{0.680385in}}%
\pgfpathlineto{\pgfqpoint{5.800806in}{0.676307in}}%
\pgfpathlineto{\pgfqpoint{5.801242in}{0.680076in}}%
\pgfpathlineto{\pgfqpoint{5.801679in}{0.686054in}}%
\pgfpathlineto{\pgfqpoint{5.802115in}{0.678553in}}%
\pgfpathlineto{\pgfqpoint{5.802843in}{0.668582in}}%
\pgfpathlineto{\pgfqpoint{5.803425in}{0.668943in}}%
\pgfpathlineto{\pgfqpoint{5.804008in}{0.670216in}}%
\pgfpathlineto{\pgfqpoint{5.804299in}{0.686718in}}%
\pgfpathlineto{\pgfqpoint{5.805026in}{0.901621in}}%
\pgfpathlineto{\pgfqpoint{5.805754in}{0.770909in}}%
\pgfpathlineto{\pgfqpoint{5.807355in}{0.698836in}}%
\pgfpathlineto{\pgfqpoint{5.808665in}{0.673355in}}%
\pgfpathlineto{\pgfqpoint{5.808956in}{0.673660in}}%
\pgfpathlineto{\pgfqpoint{5.809538in}{0.681858in}}%
\pgfpathlineto{\pgfqpoint{5.810703in}{0.732418in}}%
\pgfpathlineto{\pgfqpoint{5.811285in}{0.715582in}}%
\pgfpathlineto{\pgfqpoint{5.813322in}{0.687354in}}%
\pgfpathlineto{\pgfqpoint{5.814196in}{0.682858in}}%
\pgfpathlineto{\pgfqpoint{5.814778in}{0.684081in}}%
\pgfpathlineto{\pgfqpoint{5.815069in}{0.683698in}}%
\pgfpathlineto{\pgfqpoint{5.816088in}{0.678642in}}%
\pgfpathlineto{\pgfqpoint{5.816670in}{0.681261in}}%
\pgfpathlineto{\pgfqpoint{5.818125in}{0.691449in}}%
\pgfpathlineto{\pgfqpoint{5.818562in}{0.690537in}}%
\pgfpathlineto{\pgfqpoint{5.819872in}{0.682047in}}%
\pgfpathlineto{\pgfqpoint{5.821910in}{0.668547in}}%
\pgfpathlineto{\pgfqpoint{5.822346in}{0.668911in}}%
\pgfpathlineto{\pgfqpoint{5.823947in}{0.673407in}}%
\pgfpathlineto{\pgfqpoint{5.824821in}{0.672401in}}%
\pgfpathlineto{\pgfqpoint{5.827586in}{0.665809in}}%
\pgfpathlineto{\pgfqpoint{5.828459in}{0.666683in}}%
\pgfpathlineto{\pgfqpoint{5.833117in}{0.662588in}}%
\pgfpathlineto{\pgfqpoint{5.833699in}{0.663498in}}%
\pgfpathlineto{\pgfqpoint{5.834718in}{0.668488in}}%
\pgfpathlineto{\pgfqpoint{5.835445in}{0.666757in}}%
\pgfpathlineto{\pgfqpoint{5.836028in}{0.666264in}}%
\pgfpathlineto{\pgfqpoint{5.836464in}{0.666993in}}%
\pgfpathlineto{\pgfqpoint{5.837629in}{0.671270in}}%
\pgfpathlineto{\pgfqpoint{5.837920in}{0.669578in}}%
\pgfpathlineto{\pgfqpoint{5.838793in}{0.664588in}}%
\pgfpathlineto{\pgfqpoint{5.839375in}{0.665162in}}%
\pgfpathlineto{\pgfqpoint{5.845925in}{0.670151in}}%
\pgfpathlineto{\pgfqpoint{5.846507in}{0.671209in}}%
\pgfpathlineto{\pgfqpoint{5.847089in}{0.670592in}}%
\pgfpathlineto{\pgfqpoint{5.847671in}{0.672021in}}%
\pgfpathlineto{\pgfqpoint{5.848108in}{0.673482in}}%
\pgfpathlineto{\pgfqpoint{5.848545in}{0.670846in}}%
\pgfpathlineto{\pgfqpoint{5.849418in}{0.664606in}}%
\pgfpathlineto{\pgfqpoint{5.849854in}{0.665605in}}%
\pgfpathlineto{\pgfqpoint{5.850000in}{0.666512in}}%
\pgfpathlineto{\pgfqpoint{5.850000in}{0.666512in}}%
\pgfusepath{stroke}%
\end{pgfscope}%
\begin{pgfscope}%
\pgfsetrectcap%
\pgfsetmiterjoin%
\pgfsetlinewidth{0.803000pt}%
\definecolor{currentstroke}{rgb}{0.737255,0.737255,0.737255}%
\pgfsetstrokecolor{currentstroke}%
\pgfsetdash{}{0pt}%
\pgfpathmoveto{\pgfqpoint{0.610501in}{0.544166in}}%
\pgfpathlineto{\pgfqpoint{0.610501in}{1.486056in}}%
\pgfusepath{stroke}%
\end{pgfscope}%
\begin{pgfscope}%
\pgfsetrectcap%
\pgfsetmiterjoin%
\pgfsetlinewidth{0.803000pt}%
\definecolor{currentstroke}{rgb}{0.737255,0.737255,0.737255}%
\pgfsetstrokecolor{currentstroke}%
\pgfsetdash{}{0pt}%
\pgfpathmoveto{\pgfqpoint{5.850000in}{0.544166in}}%
\pgfpathlineto{\pgfqpoint{5.850000in}{1.486056in}}%
\pgfusepath{stroke}%
\end{pgfscope}%
\begin{pgfscope}%
\pgfsetrectcap%
\pgfsetmiterjoin%
\pgfsetlinewidth{0.803000pt}%
\definecolor{currentstroke}{rgb}{0.737255,0.737255,0.737255}%
\pgfsetstrokecolor{currentstroke}%
\pgfsetdash{}{0pt}%
\pgfpathmoveto{\pgfqpoint{0.610501in}{0.544166in}}%
\pgfpathlineto{\pgfqpoint{5.850000in}{0.544166in}}%
\pgfusepath{stroke}%
\end{pgfscope}%
\begin{pgfscope}%
\pgfsetrectcap%
\pgfsetmiterjoin%
\pgfsetlinewidth{0.803000pt}%
\definecolor{currentstroke}{rgb}{0.737255,0.737255,0.737255}%
\pgfsetstrokecolor{currentstroke}%
\pgfsetdash{}{0pt}%
\pgfpathmoveto{\pgfqpoint{0.610501in}{1.486056in}}%
\pgfpathlineto{\pgfqpoint{5.850000in}{1.486056in}}%
\pgfusepath{stroke}%
\end{pgfscope}%
\end{pgfpicture}%
\makeatother%
\endgroup%

    \caption{Comparaison de la fonction caractéristique de Swindell et Snell avec celle de Baer et Kradolfer}
    \label{fig:fc-z}
\end{figure}

\cite{baer1987} introduisent une variante 
\begin{equation}
   FC_i = \frac{E_i^4-\bar{E_i^4}}{\sigma(E_i^4)}
\end{equation}
avec $\bar{E_i^4}$ la moyenne de $E^4$ sur l'intervalle de la fenêtre choisie et $\sigma(E_i^4)$ l'écart-type de $E^4$ sur l'intervalle de la fenêtre choisie.
Cette autre fonction caractéristique présentent des pics très marqués lors du début des phases P et S, mais présentent beaucoup de bruit, qui pourraient être liés à la fenêtre choisie trop petite. 

\section{Détection et pointage}

\begin{figure}[ht]
    \centering
    \scalebox{.9}{%% Creator: Matplotlib, PGF backend
%%
%% To include the figure in your LaTeX document, write
%%   \input{<filename>.pgf}
%%
%% Make sure the required packages are loaded in your preamble
%%   \usepackage{pgf}
%%
%% Also ensure that all the required font packages are loaded; for instance,
%% the lmodern package is sometimes necessary when using math font.
%%   \usepackage{lmodern}
%%
%% Figures using additional raster images can only be included by \input if
%% they are in the same directory as the main LaTeX file. For loading figures
%% from other directories you can use the `import` package
%%   \usepackage{import}
%%
%% and then include the figures with
%%   \import{<path to file>}{<filename>.pgf}
%%
%% Matplotlib used the following preamble
%%   \usepackage{fontspec}
%%
\begingroup%
\makeatletter%
\begin{pgfpicture}%
\pgfpathrectangle{\pgfpointorigin}{\pgfqpoint{6.000000in}{6.000000in}}%
\pgfusepath{use as bounding box, clip}%
\begin{pgfscope}%
\pgfsetbuttcap%
\pgfsetmiterjoin%
\definecolor{currentfill}{rgb}{1.000000,1.000000,1.000000}%
\pgfsetfillcolor{currentfill}%
\pgfsetlinewidth{0.000000pt}%
\definecolor{currentstroke}{rgb}{1.000000,1.000000,1.000000}%
\pgfsetstrokecolor{currentstroke}%
\pgfsetdash{}{0pt}%
\pgfpathmoveto{\pgfqpoint{0.000000in}{0.000000in}}%
\pgfpathlineto{\pgfqpoint{6.000000in}{0.000000in}}%
\pgfpathlineto{\pgfqpoint{6.000000in}{6.000000in}}%
\pgfpathlineto{\pgfqpoint{0.000000in}{6.000000in}}%
\pgfpathlineto{\pgfqpoint{0.000000in}{0.000000in}}%
\pgfpathclose%
\pgfusepath{fill}%
\end{pgfscope}%
\begin{pgfscope}%
\pgfsetbuttcap%
\pgfsetmiterjoin%
\definecolor{currentfill}{rgb}{0.933333,0.933333,0.933333}%
\pgfsetfillcolor{currentfill}%
\pgfsetlinewidth{0.000000pt}%
\definecolor{currentstroke}{rgb}{0.000000,0.000000,0.000000}%
\pgfsetstrokecolor{currentstroke}%
\pgfsetstrokeopacity{0.000000}%
\pgfsetdash{}{0pt}%
\pgfpathmoveto{\pgfqpoint{0.691161in}{4.791796in}}%
\pgfpathlineto{\pgfqpoint{5.745833in}{4.791796in}}%
\pgfpathlineto{\pgfqpoint{5.745833in}{5.703703in}}%
\pgfpathlineto{\pgfqpoint{0.691161in}{5.703703in}}%
\pgfpathlineto{\pgfqpoint{0.691161in}{4.791796in}}%
\pgfpathclose%
\pgfusepath{fill}%
\end{pgfscope}%
\begin{pgfscope}%
\pgfpathrectangle{\pgfqpoint{0.691161in}{4.791796in}}{\pgfqpoint{5.054672in}{0.911907in}}%
\pgfusepath{clip}%
\pgfsetbuttcap%
\pgfsetroundjoin%
\pgfsetlinewidth{0.501875pt}%
\definecolor{currentstroke}{rgb}{0.698039,0.698039,0.698039}%
\pgfsetstrokecolor{currentstroke}%
\pgfsetdash{{1.850000pt}{0.800000pt}}{0.000000pt}%
\pgfpathmoveto{\pgfqpoint{0.691161in}{4.791796in}}%
\pgfpathlineto{\pgfqpoint{0.691161in}{5.703703in}}%
\pgfusepath{stroke}%
\end{pgfscope}%
\begin{pgfscope}%
\pgfsetbuttcap%
\pgfsetroundjoin%
\definecolor{currentfill}{rgb}{0.000000,0.000000,0.000000}%
\pgfsetfillcolor{currentfill}%
\pgfsetlinewidth{0.803000pt}%
\definecolor{currentstroke}{rgb}{0.000000,0.000000,0.000000}%
\pgfsetstrokecolor{currentstroke}%
\pgfsetdash{}{0pt}%
\pgfsys@defobject{currentmarker}{\pgfqpoint{0.000000in}{0.000000in}}{\pgfqpoint{0.000000in}{0.048611in}}{%
\pgfpathmoveto{\pgfqpoint{0.000000in}{0.000000in}}%
\pgfpathlineto{\pgfqpoint{0.000000in}{0.048611in}}%
\pgfusepath{stroke,fill}%
}%
\begin{pgfscope}%
\pgfsys@transformshift{0.691161in}{4.791796in}%
\pgfsys@useobject{currentmarker}{}%
\end{pgfscope}%
\end{pgfscope}%
\begin{pgfscope}%
\pgfpathrectangle{\pgfqpoint{0.691161in}{4.791796in}}{\pgfqpoint{5.054672in}{0.911907in}}%
\pgfusepath{clip}%
\pgfsetbuttcap%
\pgfsetroundjoin%
\pgfsetlinewidth{0.501875pt}%
\definecolor{currentstroke}{rgb}{0.698039,0.698039,0.698039}%
\pgfsetstrokecolor{currentstroke}%
\pgfsetdash{{1.850000pt}{0.800000pt}}{0.000000pt}%
\pgfpathmoveto{\pgfqpoint{1.702096in}{4.791796in}}%
\pgfpathlineto{\pgfqpoint{1.702096in}{5.703703in}}%
\pgfusepath{stroke}%
\end{pgfscope}%
\begin{pgfscope}%
\pgfsetbuttcap%
\pgfsetroundjoin%
\definecolor{currentfill}{rgb}{0.000000,0.000000,0.000000}%
\pgfsetfillcolor{currentfill}%
\pgfsetlinewidth{0.803000pt}%
\definecolor{currentstroke}{rgb}{0.000000,0.000000,0.000000}%
\pgfsetstrokecolor{currentstroke}%
\pgfsetdash{}{0pt}%
\pgfsys@defobject{currentmarker}{\pgfqpoint{0.000000in}{0.000000in}}{\pgfqpoint{0.000000in}{0.048611in}}{%
\pgfpathmoveto{\pgfqpoint{0.000000in}{0.000000in}}%
\pgfpathlineto{\pgfqpoint{0.000000in}{0.048611in}}%
\pgfusepath{stroke,fill}%
}%
\begin{pgfscope}%
\pgfsys@transformshift{1.702096in}{4.791796in}%
\pgfsys@useobject{currentmarker}{}%
\end{pgfscope}%
\end{pgfscope}%
\begin{pgfscope}%
\pgfpathrectangle{\pgfqpoint{0.691161in}{4.791796in}}{\pgfqpoint{5.054672in}{0.911907in}}%
\pgfusepath{clip}%
\pgfsetbuttcap%
\pgfsetroundjoin%
\pgfsetlinewidth{0.501875pt}%
\definecolor{currentstroke}{rgb}{0.698039,0.698039,0.698039}%
\pgfsetstrokecolor{currentstroke}%
\pgfsetdash{{1.850000pt}{0.800000pt}}{0.000000pt}%
\pgfpathmoveto{\pgfqpoint{2.713030in}{4.791796in}}%
\pgfpathlineto{\pgfqpoint{2.713030in}{5.703703in}}%
\pgfusepath{stroke}%
\end{pgfscope}%
\begin{pgfscope}%
\pgfsetbuttcap%
\pgfsetroundjoin%
\definecolor{currentfill}{rgb}{0.000000,0.000000,0.000000}%
\pgfsetfillcolor{currentfill}%
\pgfsetlinewidth{0.803000pt}%
\definecolor{currentstroke}{rgb}{0.000000,0.000000,0.000000}%
\pgfsetstrokecolor{currentstroke}%
\pgfsetdash{}{0pt}%
\pgfsys@defobject{currentmarker}{\pgfqpoint{0.000000in}{0.000000in}}{\pgfqpoint{0.000000in}{0.048611in}}{%
\pgfpathmoveto{\pgfqpoint{0.000000in}{0.000000in}}%
\pgfpathlineto{\pgfqpoint{0.000000in}{0.048611in}}%
\pgfusepath{stroke,fill}%
}%
\begin{pgfscope}%
\pgfsys@transformshift{2.713030in}{4.791796in}%
\pgfsys@useobject{currentmarker}{}%
\end{pgfscope}%
\end{pgfscope}%
\begin{pgfscope}%
\pgfpathrectangle{\pgfqpoint{0.691161in}{4.791796in}}{\pgfqpoint{5.054672in}{0.911907in}}%
\pgfusepath{clip}%
\pgfsetbuttcap%
\pgfsetroundjoin%
\pgfsetlinewidth{0.501875pt}%
\definecolor{currentstroke}{rgb}{0.698039,0.698039,0.698039}%
\pgfsetstrokecolor{currentstroke}%
\pgfsetdash{{1.850000pt}{0.800000pt}}{0.000000pt}%
\pgfpathmoveto{\pgfqpoint{3.723964in}{4.791796in}}%
\pgfpathlineto{\pgfqpoint{3.723964in}{5.703703in}}%
\pgfusepath{stroke}%
\end{pgfscope}%
\begin{pgfscope}%
\pgfsetbuttcap%
\pgfsetroundjoin%
\definecolor{currentfill}{rgb}{0.000000,0.000000,0.000000}%
\pgfsetfillcolor{currentfill}%
\pgfsetlinewidth{0.803000pt}%
\definecolor{currentstroke}{rgb}{0.000000,0.000000,0.000000}%
\pgfsetstrokecolor{currentstroke}%
\pgfsetdash{}{0pt}%
\pgfsys@defobject{currentmarker}{\pgfqpoint{0.000000in}{0.000000in}}{\pgfqpoint{0.000000in}{0.048611in}}{%
\pgfpathmoveto{\pgfqpoint{0.000000in}{0.000000in}}%
\pgfpathlineto{\pgfqpoint{0.000000in}{0.048611in}}%
\pgfusepath{stroke,fill}%
}%
\begin{pgfscope}%
\pgfsys@transformshift{3.723964in}{4.791796in}%
\pgfsys@useobject{currentmarker}{}%
\end{pgfscope}%
\end{pgfscope}%
\begin{pgfscope}%
\pgfpathrectangle{\pgfqpoint{0.691161in}{4.791796in}}{\pgfqpoint{5.054672in}{0.911907in}}%
\pgfusepath{clip}%
\pgfsetbuttcap%
\pgfsetroundjoin%
\pgfsetlinewidth{0.501875pt}%
\definecolor{currentstroke}{rgb}{0.698039,0.698039,0.698039}%
\pgfsetstrokecolor{currentstroke}%
\pgfsetdash{{1.850000pt}{0.800000pt}}{0.000000pt}%
\pgfpathmoveto{\pgfqpoint{4.734899in}{4.791796in}}%
\pgfpathlineto{\pgfqpoint{4.734899in}{5.703703in}}%
\pgfusepath{stroke}%
\end{pgfscope}%
\begin{pgfscope}%
\pgfsetbuttcap%
\pgfsetroundjoin%
\definecolor{currentfill}{rgb}{0.000000,0.000000,0.000000}%
\pgfsetfillcolor{currentfill}%
\pgfsetlinewidth{0.803000pt}%
\definecolor{currentstroke}{rgb}{0.000000,0.000000,0.000000}%
\pgfsetstrokecolor{currentstroke}%
\pgfsetdash{}{0pt}%
\pgfsys@defobject{currentmarker}{\pgfqpoint{0.000000in}{0.000000in}}{\pgfqpoint{0.000000in}{0.048611in}}{%
\pgfpathmoveto{\pgfqpoint{0.000000in}{0.000000in}}%
\pgfpathlineto{\pgfqpoint{0.000000in}{0.048611in}}%
\pgfusepath{stroke,fill}%
}%
\begin{pgfscope}%
\pgfsys@transformshift{4.734899in}{4.791796in}%
\pgfsys@useobject{currentmarker}{}%
\end{pgfscope}%
\end{pgfscope}%
\begin{pgfscope}%
\pgfpathrectangle{\pgfqpoint{0.691161in}{4.791796in}}{\pgfqpoint{5.054672in}{0.911907in}}%
\pgfusepath{clip}%
\pgfsetbuttcap%
\pgfsetroundjoin%
\pgfsetlinewidth{0.501875pt}%
\definecolor{currentstroke}{rgb}{0.698039,0.698039,0.698039}%
\pgfsetstrokecolor{currentstroke}%
\pgfsetdash{{1.850000pt}{0.800000pt}}{0.000000pt}%
\pgfpathmoveto{\pgfqpoint{5.745833in}{4.791796in}}%
\pgfpathlineto{\pgfqpoint{5.745833in}{5.703703in}}%
\pgfusepath{stroke}%
\end{pgfscope}%
\begin{pgfscope}%
\pgfsetbuttcap%
\pgfsetroundjoin%
\definecolor{currentfill}{rgb}{0.000000,0.000000,0.000000}%
\pgfsetfillcolor{currentfill}%
\pgfsetlinewidth{0.803000pt}%
\definecolor{currentstroke}{rgb}{0.000000,0.000000,0.000000}%
\pgfsetstrokecolor{currentstroke}%
\pgfsetdash{}{0pt}%
\pgfsys@defobject{currentmarker}{\pgfqpoint{0.000000in}{0.000000in}}{\pgfqpoint{0.000000in}{0.048611in}}{%
\pgfpathmoveto{\pgfqpoint{0.000000in}{0.000000in}}%
\pgfpathlineto{\pgfqpoint{0.000000in}{0.048611in}}%
\pgfusepath{stroke,fill}%
}%
\begin{pgfscope}%
\pgfsys@transformshift{5.745833in}{4.791796in}%
\pgfsys@useobject{currentmarker}{}%
\end{pgfscope}%
\end{pgfscope}%
\begin{pgfscope}%
\pgfpathrectangle{\pgfqpoint{0.691161in}{4.791796in}}{\pgfqpoint{5.054672in}{0.911907in}}%
\pgfusepath{clip}%
\pgfsetbuttcap%
\pgfsetroundjoin%
\pgfsetlinewidth{0.501875pt}%
\definecolor{currentstroke}{rgb}{0.698039,0.698039,0.698039}%
\pgfsetstrokecolor{currentstroke}%
\pgfsetdash{{1.850000pt}{0.800000pt}}{0.000000pt}%
\pgfpathmoveto{\pgfqpoint{0.691161in}{4.896335in}}%
\pgfpathlineto{\pgfqpoint{5.745833in}{4.896335in}}%
\pgfusepath{stroke}%
\end{pgfscope}%
\begin{pgfscope}%
\pgfsetbuttcap%
\pgfsetroundjoin%
\definecolor{currentfill}{rgb}{0.000000,0.000000,0.000000}%
\pgfsetfillcolor{currentfill}%
\pgfsetlinewidth{0.803000pt}%
\definecolor{currentstroke}{rgb}{0.000000,0.000000,0.000000}%
\pgfsetstrokecolor{currentstroke}%
\pgfsetdash{}{0pt}%
\pgfsys@defobject{currentmarker}{\pgfqpoint{0.000000in}{0.000000in}}{\pgfqpoint{0.048611in}{0.000000in}}{%
\pgfpathmoveto{\pgfqpoint{0.000000in}{0.000000in}}%
\pgfpathlineto{\pgfqpoint{0.048611in}{0.000000in}}%
\pgfusepath{stroke,fill}%
}%
\begin{pgfscope}%
\pgfsys@transformshift{0.691161in}{4.896335in}%
\pgfsys@useobject{currentmarker}{}%
\end{pgfscope}%
\end{pgfscope}%
\begin{pgfscope}%
\definecolor{textcolor}{rgb}{0.000000,0.000000,0.000000}%
\pgfsetstrokecolor{textcolor}%
\pgfsetfillcolor{textcolor}%
\pgftext[x=0.357055in, y=4.848141in, left, base]{\color{textcolor}\rmfamily\fontsize{10.000000}{12.000000}\selectfont \(\displaystyle {\ensuremath{-}2.5}\)}%
\end{pgfscope}%
\begin{pgfscope}%
\pgfpathrectangle{\pgfqpoint{0.691161in}{4.791796in}}{\pgfqpoint{5.054672in}{0.911907in}}%
\pgfusepath{clip}%
\pgfsetbuttcap%
\pgfsetroundjoin%
\pgfsetlinewidth{0.501875pt}%
\definecolor{currentstroke}{rgb}{0.698039,0.698039,0.698039}%
\pgfsetstrokecolor{currentstroke}%
\pgfsetdash{{1.850000pt}{0.800000pt}}{0.000000pt}%
\pgfpathmoveto{\pgfqpoint{0.691161in}{5.229899in}}%
\pgfpathlineto{\pgfqpoint{5.745833in}{5.229899in}}%
\pgfusepath{stroke}%
\end{pgfscope}%
\begin{pgfscope}%
\pgfsetbuttcap%
\pgfsetroundjoin%
\definecolor{currentfill}{rgb}{0.000000,0.000000,0.000000}%
\pgfsetfillcolor{currentfill}%
\pgfsetlinewidth{0.803000pt}%
\definecolor{currentstroke}{rgb}{0.000000,0.000000,0.000000}%
\pgfsetstrokecolor{currentstroke}%
\pgfsetdash{}{0pt}%
\pgfsys@defobject{currentmarker}{\pgfqpoint{0.000000in}{0.000000in}}{\pgfqpoint{0.048611in}{0.000000in}}{%
\pgfpathmoveto{\pgfqpoint{0.000000in}{0.000000in}}%
\pgfpathlineto{\pgfqpoint{0.048611in}{0.000000in}}%
\pgfusepath{stroke,fill}%
}%
\begin{pgfscope}%
\pgfsys@transformshift{0.691161in}{5.229899in}%
\pgfsys@useobject{currentmarker}{}%
\end{pgfscope}%
\end{pgfscope}%
\begin{pgfscope}%
\definecolor{textcolor}{rgb}{0.000000,0.000000,0.000000}%
\pgfsetstrokecolor{textcolor}%
\pgfsetfillcolor{textcolor}%
\pgftext[x=0.465080in, y=5.181704in, left, base]{\color{textcolor}\rmfamily\fontsize{10.000000}{12.000000}\selectfont \(\displaystyle {0.0}\)}%
\end{pgfscope}%
\begin{pgfscope}%
\pgfpathrectangle{\pgfqpoint{0.691161in}{4.791796in}}{\pgfqpoint{5.054672in}{0.911907in}}%
\pgfusepath{clip}%
\pgfsetbuttcap%
\pgfsetroundjoin%
\pgfsetlinewidth{0.501875pt}%
\definecolor{currentstroke}{rgb}{0.698039,0.698039,0.698039}%
\pgfsetstrokecolor{currentstroke}%
\pgfsetdash{{1.850000pt}{0.800000pt}}{0.000000pt}%
\pgfpathmoveto{\pgfqpoint{0.691161in}{5.563462in}}%
\pgfpathlineto{\pgfqpoint{5.745833in}{5.563462in}}%
\pgfusepath{stroke}%
\end{pgfscope}%
\begin{pgfscope}%
\pgfsetbuttcap%
\pgfsetroundjoin%
\definecolor{currentfill}{rgb}{0.000000,0.000000,0.000000}%
\pgfsetfillcolor{currentfill}%
\pgfsetlinewidth{0.803000pt}%
\definecolor{currentstroke}{rgb}{0.000000,0.000000,0.000000}%
\pgfsetstrokecolor{currentstroke}%
\pgfsetdash{}{0pt}%
\pgfsys@defobject{currentmarker}{\pgfqpoint{0.000000in}{0.000000in}}{\pgfqpoint{0.048611in}{0.000000in}}{%
\pgfpathmoveto{\pgfqpoint{0.000000in}{0.000000in}}%
\pgfpathlineto{\pgfqpoint{0.048611in}{0.000000in}}%
\pgfusepath{stroke,fill}%
}%
\begin{pgfscope}%
\pgfsys@transformshift{0.691161in}{5.563462in}%
\pgfsys@useobject{currentmarker}{}%
\end{pgfscope}%
\end{pgfscope}%
\begin{pgfscope}%
\definecolor{textcolor}{rgb}{0.000000,0.000000,0.000000}%
\pgfsetstrokecolor{textcolor}%
\pgfsetfillcolor{textcolor}%
\pgftext[x=0.465080in, y=5.515268in, left, base]{\color{textcolor}\rmfamily\fontsize{10.000000}{12.000000}\selectfont \(\displaystyle {2.5}\)}%
\end{pgfscope}%
\begin{pgfscope}%
\definecolor{textcolor}{rgb}{0.000000,0.000000,0.000000}%
\pgfsetstrokecolor{textcolor}%
\pgfsetfillcolor{textcolor}%
\pgftext[x=0.301500in,y=5.247750in,,bottom,rotate=90.000000]{\color{textcolor}\rmfamily\fontsize{12.000000}{14.400000}\selectfont Signal}%
\end{pgfscope}%
\begin{pgfscope}%
\definecolor{textcolor}{rgb}{0.000000,0.000000,0.000000}%
\pgfsetstrokecolor{textcolor}%
\pgfsetfillcolor{textcolor}%
\pgftext[x=0.691161in,y=5.745370in,left,base]{\color{textcolor}\rmfamily\fontsize{10.000000}{12.000000}\selectfont \(\displaystyle \times{10^{\ensuremath{-}5}}{}\)}%
\end{pgfscope}%
\begin{pgfscope}%
\pgfpathrectangle{\pgfqpoint{0.691161in}{4.791796in}}{\pgfqpoint{5.054672in}{0.911907in}}%
\pgfusepath{clip}%
\pgfsetrectcap%
\pgfsetroundjoin%
\pgfsetlinewidth{1.505625pt}%
\definecolor{currentstroke}{rgb}{0.498039,0.498039,0.498039}%
\pgfsetstrokecolor{currentstroke}%
\pgfsetdash{}{0pt}%
\pgfpathmoveto{\pgfqpoint{0.691060in}{5.230115in}}%
\pgfpathlineto{\pgfqpoint{0.696216in}{5.229797in}}%
\pgfpathlineto{\pgfqpoint{0.708145in}{5.229873in}}%
\pgfpathlineto{\pgfqpoint{0.714817in}{5.229927in}}%
\pgfpathlineto{\pgfqpoint{0.718659in}{5.230112in}}%
\pgfpathlineto{\pgfqpoint{0.721186in}{5.230017in}}%
\pgfpathlineto{\pgfqpoint{0.723208in}{5.229661in}}%
\pgfpathlineto{\pgfqpoint{0.728161in}{5.230042in}}%
\pgfpathlineto{\pgfqpoint{0.730487in}{5.229993in}}%
\pgfpathlineto{\pgfqpoint{0.734935in}{5.229930in}}%
\pgfpathlineto{\pgfqpoint{0.738776in}{5.229760in}}%
\pgfpathlineto{\pgfqpoint{0.742112in}{5.229845in}}%
\pgfpathlineto{\pgfqpoint{0.764151in}{5.229592in}}%
\pgfpathlineto{\pgfqpoint{0.766071in}{5.230170in}}%
\pgfpathlineto{\pgfqpoint{0.771126in}{5.229838in}}%
\pgfpathlineto{\pgfqpoint{0.778304in}{5.229732in}}%
\pgfpathlineto{\pgfqpoint{0.781134in}{5.229876in}}%
\pgfpathlineto{\pgfqpoint{0.787200in}{5.229865in}}%
\pgfpathlineto{\pgfqpoint{0.930652in}{5.229629in}}%
\pgfpathlineto{\pgfqpoint{0.935706in}{5.229888in}}%
\pgfpathlineto{\pgfqpoint{0.939245in}{5.229778in}}%
\pgfpathlineto{\pgfqpoint{0.975335in}{5.229742in}}%
\pgfpathlineto{\pgfqpoint{0.977660in}{5.230130in}}%
\pgfpathlineto{\pgfqpoint{0.987163in}{5.230017in}}%
\pgfpathlineto{\pgfqpoint{0.988881in}{5.229833in}}%
\pgfpathlineto{\pgfqpoint{0.991308in}{5.230018in}}%
\pgfpathlineto{\pgfqpoint{1.000507in}{5.229766in}}%
\pgfpathlineto{\pgfqpoint{1.012032in}{5.229892in}}%
\pgfpathlineto{\pgfqpoint{1.054693in}{5.229784in}}%
\pgfpathlineto{\pgfqpoint{1.061264in}{5.230079in}}%
\pgfpathlineto{\pgfqpoint{1.069352in}{5.229894in}}%
\pgfpathlineto{\pgfqpoint{1.124549in}{5.229745in}}%
\pgfpathlineto{\pgfqpoint{1.126874in}{5.229882in}}%
\pgfpathlineto{\pgfqpoint{1.133445in}{5.229705in}}%
\pgfpathlineto{\pgfqpoint{1.136680in}{5.229999in}}%
\pgfpathlineto{\pgfqpoint{1.140319in}{5.229822in}}%
\pgfpathlineto{\pgfqpoint{1.151237in}{5.229884in}}%
\pgfpathlineto{\pgfqpoint{1.156595in}{5.229829in}}%
\pgfpathlineto{\pgfqpoint{1.160942in}{5.230053in}}%
\pgfpathlineto{\pgfqpoint{1.165289in}{5.229642in}}%
\pgfpathlineto{\pgfqpoint{1.169434in}{5.229979in}}%
\pgfpathlineto{\pgfqpoint{1.177724in}{5.229887in}}%
\pgfpathlineto{\pgfqpoint{1.180555in}{5.229788in}}%
\pgfpathlineto{\pgfqpoint{1.184902in}{5.229884in}}%
\pgfpathlineto{\pgfqpoint{1.188642in}{5.229875in}}%
\pgfpathlineto{\pgfqpoint{1.193393in}{5.229819in}}%
\pgfpathlineto{\pgfqpoint{1.200672in}{5.229794in}}%
\pgfpathlineto{\pgfqpoint{1.203503in}{5.230251in}}%
\pgfpathlineto{\pgfqpoint{1.208153in}{5.229645in}}%
\pgfpathlineto{\pgfqpoint{1.211186in}{5.230026in}}%
\pgfpathlineto{\pgfqpoint{1.218262in}{5.230100in}}%
\pgfpathlineto{\pgfqpoint{1.221093in}{5.229750in}}%
\pgfpathlineto{\pgfqpoint{1.240503in}{5.229900in}}%
\pgfpathlineto{\pgfqpoint{1.241817in}{5.229651in}}%
\pgfpathlineto{\pgfqpoint{1.242019in}{5.229878in}}%
\pgfpathlineto{\pgfqpoint{1.243637in}{5.230015in}}%
\pgfpathlineto{\pgfqpoint{1.251623in}{5.229932in}}%
\pgfpathlineto{\pgfqpoint{1.257082in}{5.229886in}}%
\pgfpathlineto{\pgfqpoint{1.264361in}{5.229819in}}%
\pgfpathlineto{\pgfqpoint{1.266585in}{5.229924in}}%
\pgfpathlineto{\pgfqpoint{1.270123in}{5.229890in}}%
\pgfpathlineto{\pgfqpoint{1.275077in}{5.230012in}}%
\pgfpathlineto{\pgfqpoint{1.280637in}{5.229687in}}%
\pgfpathlineto{\pgfqpoint{1.295296in}{5.229936in}}%
\pgfpathlineto{\pgfqpoint{1.299946in}{5.229871in}}%
\pgfpathlineto{\pgfqpoint{1.314301in}{5.229816in}}%
\pgfpathlineto{\pgfqpoint{1.316930in}{5.230158in}}%
\pgfpathlineto{\pgfqpoint{1.327848in}{5.229957in}}%
\pgfpathlineto{\pgfqpoint{1.332397in}{5.229705in}}%
\pgfpathlineto{\pgfqpoint{1.335126in}{5.229775in}}%
\pgfpathlineto{\pgfqpoint{1.339575in}{5.229817in}}%
\pgfpathlineto{\pgfqpoint{1.342607in}{5.229825in}}%
\pgfpathlineto{\pgfqpoint{1.348066in}{5.230015in}}%
\pgfpathlineto{\pgfqpoint{1.352818in}{5.230051in}}%
\pgfpathlineto{\pgfqpoint{1.400938in}{5.229709in}}%
\pgfpathlineto{\pgfqpoint{1.404274in}{5.230231in}}%
\pgfpathlineto{\pgfqpoint{1.407712in}{5.229892in}}%
\pgfpathlineto{\pgfqpoint{1.438141in}{5.229581in}}%
\pgfpathlineto{\pgfqpoint{1.439152in}{5.229825in}}%
\pgfpathlineto{\pgfqpoint{1.439253in}{5.229946in}}%
\pgfpathlineto{\pgfqpoint{1.440567in}{5.229849in}}%
\pgfpathlineto{\pgfqpoint{1.442184in}{5.229934in}}%
\pgfpathlineto{\pgfqpoint{1.444307in}{5.229685in}}%
\pgfpathlineto{\pgfqpoint{1.450980in}{5.229634in}}%
\pgfpathlineto{\pgfqpoint{1.457449in}{5.229934in}}%
\pgfpathlineto{\pgfqpoint{1.459067in}{5.229973in}}%
\pgfpathlineto{\pgfqpoint{1.461493in}{5.230072in}}%
\pgfpathlineto{\pgfqpoint{1.501324in}{5.229805in}}%
\pgfpathlineto{\pgfqpoint{1.503548in}{5.230039in}}%
\pgfpathlineto{\pgfqpoint{1.505166in}{5.229773in}}%
\pgfpathlineto{\pgfqpoint{1.512444in}{5.229881in}}%
\pgfpathlineto{\pgfqpoint{1.527305in}{5.229818in}}%
\pgfpathlineto{\pgfqpoint{1.529529in}{5.230005in}}%
\pgfpathlineto{\pgfqpoint{1.537010in}{5.229727in}}%
\pgfpathlineto{\pgfqpoint{1.538729in}{5.230135in}}%
\pgfpathlineto{\pgfqpoint{1.540649in}{5.229750in}}%
\pgfpathlineto{\pgfqpoint{1.558644in}{5.229650in}}%
\pgfpathlineto{\pgfqpoint{1.574516in}{5.229959in}}%
\pgfpathlineto{\pgfqpoint{1.608382in}{5.229756in}}%
\pgfpathlineto{\pgfqpoint{1.612527in}{5.230017in}}%
\pgfpathlineto{\pgfqpoint{1.616368in}{5.229965in}}%
\pgfpathlineto{\pgfqpoint{1.620614in}{5.229698in}}%
\pgfpathlineto{\pgfqpoint{1.624759in}{5.230008in}}%
\pgfpathlineto{\pgfqpoint{1.639620in}{5.229707in}}%
\pgfpathlineto{\pgfqpoint{1.642248in}{5.229922in}}%
\pgfpathlineto{\pgfqpoint{1.651751in}{5.229783in}}%
\pgfpathlineto{\pgfqpoint{1.656300in}{5.230037in}}%
\pgfpathlineto{\pgfqpoint{1.668735in}{5.230113in}}%
\pgfpathlineto{\pgfqpoint{1.673992in}{5.229856in}}%
\pgfpathlineto{\pgfqpoint{1.676923in}{5.229937in}}%
\pgfpathlineto{\pgfqpoint{1.681978in}{5.230068in}}%
\pgfpathlineto{\pgfqpoint{1.683697in}{5.229851in}}%
\pgfpathlineto{\pgfqpoint{1.686527in}{5.229832in}}%
\pgfpathlineto{\pgfqpoint{1.690571in}{5.229828in}}%
\pgfpathlineto{\pgfqpoint{1.692795in}{5.229749in}}%
\pgfpathlineto{\pgfqpoint{1.703309in}{5.229848in}}%
\pgfpathlineto{\pgfqpoint{1.705128in}{5.229911in}}%
\pgfpathlineto{\pgfqpoint{1.706746in}{5.229898in}}%
\pgfpathlineto{\pgfqpoint{1.708363in}{5.229913in}}%
\pgfpathlineto{\pgfqpoint{1.713014in}{5.229879in}}%
\pgfpathlineto{\pgfqpoint{1.717462in}{5.229820in}}%
\pgfpathlineto{\pgfqpoint{1.719686in}{5.229871in}}%
\pgfpathlineto{\pgfqpoint{1.726762in}{5.229867in}}%
\pgfpathlineto{\pgfqpoint{1.737984in}{5.229737in}}%
\pgfpathlineto{\pgfqpoint{1.740814in}{5.230107in}}%
\pgfpathlineto{\pgfqpoint{1.745060in}{5.230015in}}%
\pgfpathlineto{\pgfqpoint{1.748801in}{5.229708in}}%
\pgfpathlineto{\pgfqpoint{1.753350in}{5.230099in}}%
\pgfpathlineto{\pgfqpoint{1.758910in}{5.229879in}}%
\pgfpathlineto{\pgfqpoint{1.802279in}{5.230066in}}%
\pgfpathlineto{\pgfqpoint{1.806020in}{5.230080in}}%
\pgfpathlineto{\pgfqpoint{1.836247in}{5.229936in}}%
\pgfpathlineto{\pgfqpoint{1.854747in}{5.229849in}}%
\pgfpathlineto{\pgfqpoint{1.858083in}{5.229928in}}%
\pgfpathlineto{\pgfqpoint{1.863441in}{5.229875in}}%
\pgfpathlineto{\pgfqpoint{1.866878in}{5.230000in}}%
\pgfpathlineto{\pgfqpoint{1.883963in}{5.230100in}}%
\pgfpathlineto{\pgfqpoint{1.890736in}{5.229868in}}%
\pgfpathlineto{\pgfqpoint{1.892556in}{5.229827in}}%
\pgfpathlineto{\pgfqpoint{1.894881in}{5.229794in}}%
\pgfpathlineto{\pgfqpoint{1.899026in}{5.229892in}}%
\pgfpathlineto{\pgfqpoint{1.945427in}{5.230098in}}%
\pgfpathlineto{\pgfqpoint{1.947854in}{5.229796in}}%
\pgfpathlineto{\pgfqpoint{1.969387in}{5.229818in}}%
\pgfpathlineto{\pgfqpoint{1.972622in}{5.229951in}}%
\pgfpathlineto{\pgfqpoint{1.979597in}{5.230074in}}%
\pgfpathlineto{\pgfqpoint{1.981417in}{5.229607in}}%
\pgfpathlineto{\pgfqpoint{1.983236in}{5.230094in}}%
\pgfpathlineto{\pgfqpoint{1.995873in}{5.229838in}}%
\pgfpathlineto{\pgfqpoint{1.997996in}{5.230026in}}%
\pgfpathlineto{\pgfqpoint{2.004163in}{5.229670in}}%
\pgfpathlineto{\pgfqpoint{2.006690in}{5.230205in}}%
\pgfpathlineto{\pgfqpoint{2.033076in}{5.229720in}}%
\pgfpathlineto{\pgfqpoint{2.037625in}{5.229964in}}%
\pgfpathlineto{\pgfqpoint{2.061887in}{5.229926in}}%
\pgfpathlineto{\pgfqpoint{2.064414in}{5.229659in}}%
\pgfpathlineto{\pgfqpoint{2.077051in}{5.229909in}}%
\pgfpathlineto{\pgfqpoint{2.203620in}{5.229618in}}%
\pgfpathlineto{\pgfqpoint{2.205339in}{5.229881in}}%
\pgfpathlineto{\pgfqpoint{2.207462in}{5.229692in}}%
\pgfpathlineto{\pgfqpoint{2.214033in}{5.229875in}}%
\pgfpathlineto{\pgfqpoint{2.219492in}{5.229904in}}%
\pgfpathlineto{\pgfqpoint{2.286011in}{5.229871in}}%
\pgfpathlineto{\pgfqpoint{2.292178in}{5.229902in}}%
\pgfpathlineto{\pgfqpoint{2.296222in}{5.229955in}}%
\pgfpathlineto{\pgfqpoint{2.306432in}{5.229948in}}%
\pgfpathlineto{\pgfqpoint{2.310274in}{5.229960in}}%
\pgfpathlineto{\pgfqpoint{2.321495in}{5.229845in}}%
\pgfpathlineto{\pgfqpoint{2.327156in}{5.229898in}}%
\pgfpathlineto{\pgfqpoint{2.331604in}{5.229918in}}%
\pgfpathlineto{\pgfqpoint{2.339389in}{5.229855in}}%
\pgfpathlineto{\pgfqpoint{2.342523in}{5.229731in}}%
\pgfpathlineto{\pgfqpoint{2.628920in}{5.229552in}}%
\pgfpathlineto{\pgfqpoint{2.633368in}{5.230058in}}%
\pgfpathlineto{\pgfqpoint{2.645500in}{5.229825in}}%
\pgfpathlineto{\pgfqpoint{2.649341in}{5.229636in}}%
\pgfpathlineto{\pgfqpoint{2.657226in}{5.229659in}}%
\pgfpathlineto{\pgfqpoint{2.660664in}{5.229825in}}%
\pgfpathlineto{\pgfqpoint{2.665415in}{5.229738in}}%
\pgfpathlineto{\pgfqpoint{2.667841in}{5.230001in}}%
\pgfpathlineto{\pgfqpoint{2.675120in}{5.229838in}}%
\pgfpathlineto{\pgfqpoint{2.683005in}{5.230040in}}%
\pgfpathlineto{\pgfqpoint{2.685836in}{5.229835in}}%
\pgfpathlineto{\pgfqpoint{2.695945in}{5.230079in}}%
\pgfpathlineto{\pgfqpoint{2.698574in}{5.230211in}}%
\pgfpathlineto{\pgfqpoint{2.716770in}{5.229792in}}%
\pgfpathlineto{\pgfqpoint{2.719702in}{5.229769in}}%
\pgfpathlineto{\pgfqpoint{3.002663in}{5.229547in}}%
\pgfpathlineto{\pgfqpoint{3.003674in}{5.226994in}}%
\pgfpathlineto{\pgfqpoint{3.004078in}{5.228853in}}%
\pgfpathlineto{\pgfqpoint{3.004988in}{5.236540in}}%
\pgfpathlineto{\pgfqpoint{3.005594in}{5.232707in}}%
\pgfpathlineto{\pgfqpoint{3.007313in}{5.226267in}}%
\pgfpathlineto{\pgfqpoint{3.007616in}{5.225775in}}%
\pgfpathlineto{\pgfqpoint{3.007919in}{5.227145in}}%
\pgfpathlineto{\pgfqpoint{3.008627in}{5.232350in}}%
\pgfpathlineto{\pgfqpoint{3.009234in}{5.228811in}}%
\pgfpathlineto{\pgfqpoint{3.009739in}{5.226299in}}%
\pgfpathlineto{\pgfqpoint{3.010245in}{5.228952in}}%
\pgfpathlineto{\pgfqpoint{3.010851in}{5.231484in}}%
\pgfpathlineto{\pgfqpoint{3.011458in}{5.230332in}}%
\pgfpathlineto{\pgfqpoint{3.012671in}{5.230396in}}%
\pgfpathlineto{\pgfqpoint{3.014187in}{5.229748in}}%
\pgfpathlineto{\pgfqpoint{3.014491in}{5.230801in}}%
\pgfpathlineto{\pgfqpoint{3.015502in}{5.233355in}}%
\pgfpathlineto{\pgfqpoint{3.016007in}{5.232335in}}%
\pgfpathlineto{\pgfqpoint{3.018939in}{5.225258in}}%
\pgfpathlineto{\pgfqpoint{3.019343in}{5.226513in}}%
\pgfpathlineto{\pgfqpoint{3.021365in}{5.235253in}}%
\pgfpathlineto{\pgfqpoint{3.021870in}{5.232765in}}%
\pgfpathlineto{\pgfqpoint{3.023084in}{5.217798in}}%
\pgfpathlineto{\pgfqpoint{3.023589in}{5.224877in}}%
\pgfpathlineto{\pgfqpoint{3.024600in}{5.246190in}}%
\pgfpathlineto{\pgfqpoint{3.025105in}{5.239803in}}%
\pgfpathlineto{\pgfqpoint{3.026217in}{5.214054in}}%
\pgfpathlineto{\pgfqpoint{3.026723in}{5.222538in}}%
\pgfpathlineto{\pgfqpoint{3.027835in}{5.249514in}}%
\pgfpathlineto{\pgfqpoint{3.028340in}{5.242471in}}%
\pgfpathlineto{\pgfqpoint{3.029655in}{5.215367in}}%
\pgfpathlineto{\pgfqpoint{3.030160in}{5.221321in}}%
\pgfpathlineto{\pgfqpoint{3.030868in}{5.231062in}}%
\pgfpathlineto{\pgfqpoint{3.031373in}{5.224175in}}%
\pgfpathlineto{\pgfqpoint{3.031980in}{5.218159in}}%
\pgfpathlineto{\pgfqpoint{3.032384in}{5.222378in}}%
\pgfpathlineto{\pgfqpoint{3.033395in}{5.241202in}}%
\pgfpathlineto{\pgfqpoint{3.033901in}{5.234449in}}%
\pgfpathlineto{\pgfqpoint{3.034507in}{5.225975in}}%
\pgfpathlineto{\pgfqpoint{3.035114in}{5.231578in}}%
\pgfpathlineto{\pgfqpoint{3.036832in}{5.261896in}}%
\pgfpathlineto{\pgfqpoint{3.037135in}{5.251504in}}%
\pgfpathlineto{\pgfqpoint{3.038146in}{5.163614in}}%
\pgfpathlineto{\pgfqpoint{3.038753in}{5.209086in}}%
\pgfpathlineto{\pgfqpoint{3.039461in}{5.256150in}}%
\pgfpathlineto{\pgfqpoint{3.040067in}{5.236556in}}%
\pgfpathlineto{\pgfqpoint{3.040269in}{5.233085in}}%
\pgfpathlineto{\pgfqpoint{3.040775in}{5.243259in}}%
\pgfpathlineto{\pgfqpoint{3.041280in}{5.254145in}}%
\pgfpathlineto{\pgfqpoint{3.041887in}{5.246237in}}%
\pgfpathlineto{\pgfqpoint{3.044313in}{5.205691in}}%
\pgfpathlineto{\pgfqpoint{3.044718in}{5.211531in}}%
\pgfpathlineto{\pgfqpoint{3.045728in}{5.250830in}}%
\pgfpathlineto{\pgfqpoint{3.046335in}{5.231830in}}%
\pgfpathlineto{\pgfqpoint{3.046942in}{5.213505in}}%
\pgfpathlineto{\pgfqpoint{3.047750in}{5.215121in}}%
\pgfpathlineto{\pgfqpoint{3.048458in}{5.194005in}}%
\pgfpathlineto{\pgfqpoint{3.048963in}{5.211667in}}%
\pgfpathlineto{\pgfqpoint{3.050581in}{5.241332in}}%
\pgfpathlineto{\pgfqpoint{3.051592in}{5.271754in}}%
\pgfpathlineto{\pgfqpoint{3.052198in}{5.264048in}}%
\pgfpathlineto{\pgfqpoint{3.054321in}{5.190324in}}%
\pgfpathlineto{\pgfqpoint{3.054726in}{5.179942in}}%
\pgfpathlineto{\pgfqpoint{3.055130in}{5.194496in}}%
\pgfpathlineto{\pgfqpoint{3.056040in}{5.245681in}}%
\pgfpathlineto{\pgfqpoint{3.056748in}{5.233301in}}%
\pgfpathlineto{\pgfqpoint{3.058163in}{5.229146in}}%
\pgfpathlineto{\pgfqpoint{3.057354in}{5.235337in}}%
\pgfpathlineto{\pgfqpoint{3.058365in}{5.230512in}}%
\pgfpathlineto{\pgfqpoint{3.058668in}{5.233067in}}%
\pgfpathlineto{\pgfqpoint{3.059073in}{5.226546in}}%
\pgfpathlineto{\pgfqpoint{3.059983in}{5.190929in}}%
\pgfpathlineto{\pgfqpoint{3.060387in}{5.207711in}}%
\pgfpathlineto{\pgfqpoint{3.061398in}{5.297230in}}%
\pgfpathlineto{\pgfqpoint{3.061903in}{5.260296in}}%
\pgfpathlineto{\pgfqpoint{3.062712in}{5.213415in}}%
\pgfpathlineto{\pgfqpoint{3.063218in}{5.225714in}}%
\pgfpathlineto{\pgfqpoint{3.063925in}{5.240389in}}%
\pgfpathlineto{\pgfqpoint{3.064330in}{5.227882in}}%
\pgfpathlineto{\pgfqpoint{3.065341in}{5.153460in}}%
\pgfpathlineto{\pgfqpoint{3.065947in}{5.185629in}}%
\pgfpathlineto{\pgfqpoint{3.067464in}{5.323871in}}%
\pgfpathlineto{\pgfqpoint{3.067969in}{5.284926in}}%
\pgfpathlineto{\pgfqpoint{3.068879in}{5.183888in}}%
\pgfpathlineto{\pgfqpoint{3.069384in}{5.225633in}}%
\pgfpathlineto{\pgfqpoint{3.069991in}{5.279376in}}%
\pgfpathlineto{\pgfqpoint{3.070597in}{5.239857in}}%
\pgfpathlineto{\pgfqpoint{3.071507in}{5.185801in}}%
\pgfpathlineto{\pgfqpoint{3.072114in}{5.195421in}}%
\pgfpathlineto{\pgfqpoint{3.073832in}{5.222896in}}%
\pgfpathlineto{\pgfqpoint{3.074237in}{5.215451in}}%
\pgfpathlineto{\pgfqpoint{3.075147in}{5.176522in}}%
\pgfpathlineto{\pgfqpoint{3.075551in}{5.203064in}}%
\pgfpathlineto{\pgfqpoint{3.077371in}{5.296495in}}%
\pgfpathlineto{\pgfqpoint{3.077977in}{5.310650in}}%
\pgfpathlineto{\pgfqpoint{3.078382in}{5.296529in}}%
\pgfpathlineto{\pgfqpoint{3.080606in}{5.171217in}}%
\pgfpathlineto{\pgfqpoint{3.081010in}{5.182115in}}%
\pgfpathlineto{\pgfqpoint{3.081920in}{5.237108in}}%
\pgfpathlineto{\pgfqpoint{3.082425in}{5.211759in}}%
\pgfpathlineto{\pgfqpoint{3.083133in}{5.164129in}}%
\pgfpathlineto{\pgfqpoint{3.083740in}{5.189503in}}%
\pgfpathlineto{\pgfqpoint{3.085256in}{5.277694in}}%
\pgfpathlineto{\pgfqpoint{3.085761in}{5.268337in}}%
\pgfpathlineto{\pgfqpoint{3.085964in}{5.265185in}}%
\pgfpathlineto{\pgfqpoint{3.086368in}{5.273123in}}%
\pgfpathlineto{\pgfqpoint{3.087076in}{5.300375in}}%
\pgfpathlineto{\pgfqpoint{3.087480in}{5.281930in}}%
\pgfpathlineto{\pgfqpoint{3.088895in}{5.184937in}}%
\pgfpathlineto{\pgfqpoint{3.089603in}{5.185291in}}%
\pgfpathlineto{\pgfqpoint{3.089906in}{5.184196in}}%
\pgfpathlineto{\pgfqpoint{3.090210in}{5.186056in}}%
\pgfpathlineto{\pgfqpoint{3.090917in}{5.225660in}}%
\pgfpathlineto{\pgfqpoint{3.091625in}{5.261590in}}%
\pgfpathlineto{\pgfqpoint{3.092130in}{5.246278in}}%
\pgfpathlineto{\pgfqpoint{3.092939in}{5.223560in}}%
\pgfpathlineto{\pgfqpoint{3.093546in}{5.231292in}}%
\pgfpathlineto{\pgfqpoint{3.093748in}{5.229779in}}%
\pgfpathlineto{\pgfqpoint{3.094557in}{5.204720in}}%
\pgfpathlineto{\pgfqpoint{3.095062in}{5.223037in}}%
\pgfpathlineto{\pgfqpoint{3.095871in}{5.274410in}}%
\pgfpathlineto{\pgfqpoint{3.096376in}{5.242214in}}%
\pgfpathlineto{\pgfqpoint{3.097084in}{5.186041in}}%
\pgfpathlineto{\pgfqpoint{3.097589in}{5.212514in}}%
\pgfpathlineto{\pgfqpoint{3.098499in}{5.303079in}}%
\pgfpathlineto{\pgfqpoint{3.099106in}{5.266720in}}%
\pgfpathlineto{\pgfqpoint{3.100723in}{5.153075in}}%
\pgfpathlineto{\pgfqpoint{3.101128in}{5.169223in}}%
\pgfpathlineto{\pgfqpoint{3.102139in}{5.254919in}}%
\pgfpathlineto{\pgfqpoint{3.102846in}{5.231364in}}%
\pgfpathlineto{\pgfqpoint{3.103048in}{5.228654in}}%
\pgfpathlineto{\pgfqpoint{3.103352in}{5.238243in}}%
\pgfpathlineto{\pgfqpoint{3.104565in}{5.321182in}}%
\pgfpathlineto{\pgfqpoint{3.105070in}{5.287118in}}%
\pgfpathlineto{\pgfqpoint{3.106182in}{5.145146in}}%
\pgfpathlineto{\pgfqpoint{3.106688in}{5.191388in}}%
\pgfpathlineto{\pgfqpoint{3.107395in}{5.254163in}}%
\pgfpathlineto{\pgfqpoint{3.108002in}{5.219517in}}%
\pgfpathlineto{\pgfqpoint{3.109114in}{5.145038in}}%
\pgfpathlineto{\pgfqpoint{3.109619in}{5.168034in}}%
\pgfpathlineto{\pgfqpoint{3.111035in}{5.333525in}}%
\pgfpathlineto{\pgfqpoint{3.111641in}{5.282270in}}%
\pgfpathlineto{\pgfqpoint{3.112551in}{5.202180in}}%
\pgfpathlineto{\pgfqpoint{3.113057in}{5.225637in}}%
\pgfpathlineto{\pgfqpoint{3.113967in}{5.255202in}}%
\pgfpathlineto{\pgfqpoint{3.114573in}{5.254178in}}%
\pgfpathlineto{\pgfqpoint{3.115180in}{5.253101in}}%
\pgfpathlineto{\pgfqpoint{3.115685in}{5.236056in}}%
\pgfpathlineto{\pgfqpoint{3.116999in}{5.119772in}}%
\pgfpathlineto{\pgfqpoint{3.117606in}{5.154708in}}%
\pgfpathlineto{\pgfqpoint{3.119122in}{5.341316in}}%
\pgfpathlineto{\pgfqpoint{3.119628in}{5.299487in}}%
\pgfpathlineto{\pgfqpoint{3.120639in}{5.179012in}}%
\pgfpathlineto{\pgfqpoint{3.121245in}{5.209290in}}%
\pgfpathlineto{\pgfqpoint{3.122762in}{5.249597in}}%
\pgfpathlineto{\pgfqpoint{3.122863in}{5.249579in}}%
\pgfpathlineto{\pgfqpoint{3.123267in}{5.243068in}}%
\pgfpathlineto{\pgfqpoint{3.124682in}{5.165976in}}%
\pgfpathlineto{\pgfqpoint{3.125289in}{5.205225in}}%
\pgfpathlineto{\pgfqpoint{3.126502in}{5.282261in}}%
\pgfpathlineto{\pgfqpoint{3.127008in}{5.275349in}}%
\pgfpathlineto{\pgfqpoint{3.128322in}{5.238117in}}%
\pgfpathlineto{\pgfqpoint{3.129535in}{5.175348in}}%
\pgfpathlineto{\pgfqpoint{3.130040in}{5.194216in}}%
\pgfpathlineto{\pgfqpoint{3.130849in}{5.231137in}}%
\pgfpathlineto{\pgfqpoint{3.131557in}{5.222897in}}%
\pgfpathlineto{\pgfqpoint{3.131961in}{5.228592in}}%
\pgfpathlineto{\pgfqpoint{3.132467in}{5.236235in}}%
\pgfpathlineto{\pgfqpoint{3.133073in}{5.229249in}}%
\pgfpathlineto{\pgfqpoint{3.133174in}{5.229113in}}%
\pgfpathlineto{\pgfqpoint{3.133275in}{5.229993in}}%
\pgfpathlineto{\pgfqpoint{3.134084in}{5.268506in}}%
\pgfpathlineto{\pgfqpoint{3.134387in}{5.275675in}}%
\pgfpathlineto{\pgfqpoint{3.134792in}{5.258292in}}%
\pgfpathlineto{\pgfqpoint{3.135904in}{5.203596in}}%
\pgfpathlineto{\pgfqpoint{3.136510in}{5.206902in}}%
\pgfpathlineto{\pgfqpoint{3.137218in}{5.225887in}}%
\pgfpathlineto{\pgfqpoint{3.138330in}{5.275186in}}%
\pgfpathlineto{\pgfqpoint{3.138835in}{5.259755in}}%
\pgfpathlineto{\pgfqpoint{3.140453in}{5.131495in}}%
\pgfpathlineto{\pgfqpoint{3.141060in}{5.171616in}}%
\pgfpathlineto{\pgfqpoint{3.142576in}{5.325192in}}%
\pgfpathlineto{\pgfqpoint{3.143081in}{5.311197in}}%
\pgfpathlineto{\pgfqpoint{3.146215in}{5.187536in}}%
\pgfpathlineto{\pgfqpoint{3.146417in}{5.186719in}}%
\pgfpathlineto{\pgfqpoint{3.146620in}{5.188905in}}%
\pgfpathlineto{\pgfqpoint{3.148945in}{5.240061in}}%
\pgfpathlineto{\pgfqpoint{3.149450in}{5.235121in}}%
\pgfpathlineto{\pgfqpoint{3.151169in}{5.182406in}}%
\pgfpathlineto{\pgfqpoint{3.151573in}{5.202249in}}%
\pgfpathlineto{\pgfqpoint{3.152887in}{5.394286in}}%
\pgfpathlineto{\pgfqpoint{3.153494in}{5.330522in}}%
\pgfpathlineto{\pgfqpoint{3.155010in}{5.045666in}}%
\pgfpathlineto{\pgfqpoint{3.155516in}{5.106416in}}%
\pgfpathlineto{\pgfqpoint{3.156830in}{5.275505in}}%
\pgfpathlineto{\pgfqpoint{3.157336in}{5.257242in}}%
\pgfpathlineto{\pgfqpoint{3.158852in}{5.222456in}}%
\pgfpathlineto{\pgfqpoint{3.159054in}{5.225423in}}%
\pgfpathlineto{\pgfqpoint{3.160166in}{5.292659in}}%
\pgfpathlineto{\pgfqpoint{3.160975in}{5.329942in}}%
\pgfpathlineto{\pgfqpoint{3.161379in}{5.313638in}}%
\pgfpathlineto{\pgfqpoint{3.163401in}{5.113417in}}%
\pgfpathlineto{\pgfqpoint{3.164210in}{5.166908in}}%
\pgfpathlineto{\pgfqpoint{3.165120in}{5.242733in}}%
\pgfpathlineto{\pgfqpoint{3.165726in}{5.227663in}}%
\pgfpathlineto{\pgfqpoint{3.165929in}{5.224965in}}%
\pgfpathlineto{\pgfqpoint{3.166333in}{5.233099in}}%
\pgfpathlineto{\pgfqpoint{3.168153in}{5.290744in}}%
\pgfpathlineto{\pgfqpoint{3.168456in}{5.279892in}}%
\pgfpathlineto{\pgfqpoint{3.169972in}{5.162434in}}%
\pgfpathlineto{\pgfqpoint{3.170579in}{5.193026in}}%
\pgfpathlineto{\pgfqpoint{3.171893in}{5.335604in}}%
\pgfpathlineto{\pgfqpoint{3.172601in}{5.292327in}}%
\pgfpathlineto{\pgfqpoint{3.174926in}{5.111785in}}%
\pgfpathlineto{\pgfqpoint{3.175229in}{5.119143in}}%
\pgfpathlineto{\pgfqpoint{3.176139in}{5.249887in}}%
\pgfpathlineto{\pgfqpoint{3.177049in}{5.361733in}}%
\pgfpathlineto{\pgfqpoint{3.177554in}{5.318002in}}%
\pgfpathlineto{\pgfqpoint{3.178868in}{5.150017in}}%
\pgfpathlineto{\pgfqpoint{3.179475in}{5.179141in}}%
\pgfpathlineto{\pgfqpoint{3.180991in}{5.308924in}}%
\pgfpathlineto{\pgfqpoint{3.181497in}{5.275738in}}%
\pgfpathlineto{\pgfqpoint{3.182609in}{5.159784in}}%
\pgfpathlineto{\pgfqpoint{3.183216in}{5.190631in}}%
\pgfpathlineto{\pgfqpoint{3.184530in}{5.269128in}}%
\pgfpathlineto{\pgfqpoint{3.184934in}{5.256820in}}%
\pgfpathlineto{\pgfqpoint{3.186147in}{5.167007in}}%
\pgfpathlineto{\pgfqpoint{3.186754in}{5.202030in}}%
\pgfpathlineto{\pgfqpoint{3.188068in}{5.319774in}}%
\pgfpathlineto{\pgfqpoint{3.188675in}{5.300614in}}%
\pgfpathlineto{\pgfqpoint{3.191000in}{5.149983in}}%
\pgfpathlineto{\pgfqpoint{3.191707in}{5.162080in}}%
\pgfpathlineto{\pgfqpoint{3.192617in}{5.178638in}}%
\pgfpathlineto{\pgfqpoint{3.194538in}{5.353777in}}%
\pgfpathlineto{\pgfqpoint{3.195347in}{5.297778in}}%
\pgfpathlineto{\pgfqpoint{3.196964in}{5.125892in}}%
\pgfpathlineto{\pgfqpoint{3.197470in}{5.142741in}}%
\pgfpathlineto{\pgfqpoint{3.199087in}{5.300119in}}%
\pgfpathlineto{\pgfqpoint{3.200098in}{5.263278in}}%
\pgfpathlineto{\pgfqpoint{3.201008in}{5.246890in}}%
\pgfpathlineto{\pgfqpoint{3.202221in}{5.181390in}}%
\pgfpathlineto{\pgfqpoint{3.202828in}{5.194460in}}%
\pgfpathlineto{\pgfqpoint{3.204748in}{5.266560in}}%
\pgfpathlineto{\pgfqpoint{3.205456in}{5.246033in}}%
\pgfpathlineto{\pgfqpoint{3.206063in}{5.233961in}}%
\pgfpathlineto{\pgfqpoint{3.206669in}{5.239287in}}%
\pgfpathlineto{\pgfqpoint{3.206972in}{5.235384in}}%
\pgfpathlineto{\pgfqpoint{3.208388in}{5.176750in}}%
\pgfpathlineto{\pgfqpoint{3.209095in}{5.195920in}}%
\pgfpathlineto{\pgfqpoint{3.210713in}{5.310909in}}%
\pgfpathlineto{\pgfqpoint{3.211421in}{5.277476in}}%
\pgfpathlineto{\pgfqpoint{3.213341in}{5.148827in}}%
\pgfpathlineto{\pgfqpoint{3.213847in}{5.175931in}}%
\pgfpathlineto{\pgfqpoint{3.214959in}{5.263942in}}%
\pgfpathlineto{\pgfqpoint{3.215667in}{5.254170in}}%
\pgfpathlineto{\pgfqpoint{3.217587in}{5.224457in}}%
\pgfpathlineto{\pgfqpoint{3.218093in}{5.235399in}}%
\pgfpathlineto{\pgfqpoint{3.218598in}{5.249883in}}%
\pgfpathlineto{\pgfqpoint{3.219104in}{5.234055in}}%
\pgfpathlineto{\pgfqpoint{3.220216in}{5.163297in}}%
\pgfpathlineto{\pgfqpoint{3.220721in}{5.187607in}}%
\pgfpathlineto{\pgfqpoint{3.222238in}{5.297442in}}%
\pgfpathlineto{\pgfqpoint{3.222743in}{5.284203in}}%
\pgfpathlineto{\pgfqpoint{3.224158in}{5.183350in}}%
\pgfpathlineto{\pgfqpoint{3.224866in}{5.215523in}}%
\pgfpathlineto{\pgfqpoint{3.226382in}{5.246351in}}%
\pgfpathlineto{\pgfqpoint{3.226483in}{5.246139in}}%
\pgfpathlineto{\pgfqpoint{3.226989in}{5.235885in}}%
\pgfpathlineto{\pgfqpoint{3.228000in}{5.204327in}}%
\pgfpathlineto{\pgfqpoint{3.228606in}{5.213223in}}%
\pgfpathlineto{\pgfqpoint{3.230022in}{5.262297in}}%
\pgfpathlineto{\pgfqpoint{3.230527in}{5.242352in}}%
\pgfpathlineto{\pgfqpoint{3.231336in}{5.199828in}}%
\pgfpathlineto{\pgfqpoint{3.231943in}{5.215792in}}%
\pgfpathlineto{\pgfqpoint{3.233762in}{5.249602in}}%
\pgfpathlineto{\pgfqpoint{3.233964in}{5.248248in}}%
\pgfpathlineto{\pgfqpoint{3.234975in}{5.233517in}}%
\pgfpathlineto{\pgfqpoint{3.235582in}{5.238908in}}%
\pgfpathlineto{\pgfqpoint{3.235784in}{5.239557in}}%
\pgfpathlineto{\pgfqpoint{3.236087in}{5.236765in}}%
\pgfpathlineto{\pgfqpoint{3.236997in}{5.219859in}}%
\pgfpathlineto{\pgfqpoint{3.237503in}{5.227645in}}%
\pgfpathlineto{\pgfqpoint{3.237806in}{5.230974in}}%
\pgfpathlineto{\pgfqpoint{3.238210in}{5.224453in}}%
\pgfpathlineto{\pgfqpoint{3.239221in}{5.198286in}}%
\pgfpathlineto{\pgfqpoint{3.239727in}{5.205885in}}%
\pgfpathlineto{\pgfqpoint{3.241648in}{5.279881in}}%
\pgfpathlineto{\pgfqpoint{3.242254in}{5.257073in}}%
\pgfpathlineto{\pgfqpoint{3.243366in}{5.164700in}}%
\pgfpathlineto{\pgfqpoint{3.243973in}{5.191605in}}%
\pgfpathlineto{\pgfqpoint{3.246096in}{5.288182in}}%
\pgfpathlineto{\pgfqpoint{3.246197in}{5.287192in}}%
\pgfpathlineto{\pgfqpoint{3.246803in}{5.251774in}}%
\pgfpathlineto{\pgfqpoint{3.247814in}{5.192412in}}%
\pgfpathlineto{\pgfqpoint{3.248421in}{5.202743in}}%
\pgfpathlineto{\pgfqpoint{3.249937in}{5.249805in}}%
\pgfpathlineto{\pgfqpoint{3.250443in}{5.234462in}}%
\pgfpathlineto{\pgfqpoint{3.251555in}{5.177064in}}%
\pgfpathlineto{\pgfqpoint{3.252060in}{5.202772in}}%
\pgfpathlineto{\pgfqpoint{3.253172in}{5.288027in}}%
\pgfpathlineto{\pgfqpoint{3.253779in}{5.263986in}}%
\pgfpathlineto{\pgfqpoint{3.255396in}{5.198616in}}%
\pgfpathlineto{\pgfqpoint{3.255700in}{5.200997in}}%
\pgfpathlineto{\pgfqpoint{3.256609in}{5.239287in}}%
\pgfpathlineto{\pgfqpoint{3.257115in}{5.253777in}}%
\pgfpathlineto{\pgfqpoint{3.257721in}{5.240862in}}%
\pgfpathlineto{\pgfqpoint{3.258227in}{5.229891in}}%
\pgfpathlineto{\pgfqpoint{3.258833in}{5.238137in}}%
\pgfpathlineto{\pgfqpoint{3.260148in}{5.253243in}}%
\pgfpathlineto{\pgfqpoint{3.260451in}{5.250719in}}%
\pgfpathlineto{\pgfqpoint{3.262068in}{5.200345in}}%
\pgfpathlineto{\pgfqpoint{3.263079in}{5.211555in}}%
\pgfpathlineto{\pgfqpoint{3.263888in}{5.220500in}}%
\pgfpathlineto{\pgfqpoint{3.264495in}{5.219154in}}%
\pgfpathlineto{\pgfqpoint{3.264899in}{5.218014in}}%
\pgfpathlineto{\pgfqpoint{3.265202in}{5.219661in}}%
\pgfpathlineto{\pgfqpoint{3.266011in}{5.251882in}}%
\pgfpathlineto{\pgfqpoint{3.266516in}{5.264443in}}%
\pgfpathlineto{\pgfqpoint{3.267022in}{5.253795in}}%
\pgfpathlineto{\pgfqpoint{3.268639in}{5.219330in}}%
\pgfpathlineto{\pgfqpoint{3.268943in}{5.223691in}}%
\pgfpathlineto{\pgfqpoint{3.269954in}{5.247172in}}%
\pgfpathlineto{\pgfqpoint{3.270560in}{5.238531in}}%
\pgfpathlineto{\pgfqpoint{3.271268in}{5.225471in}}%
\pgfpathlineto{\pgfqpoint{3.271874in}{5.230524in}}%
\pgfpathlineto{\pgfqpoint{3.272279in}{5.233213in}}%
\pgfpathlineto{\pgfqpoint{3.272683in}{5.228752in}}%
\pgfpathlineto{\pgfqpoint{3.273593in}{5.199627in}}%
\pgfpathlineto{\pgfqpoint{3.274099in}{5.217143in}}%
\pgfpathlineto{\pgfqpoint{3.274705in}{5.241993in}}%
\pgfpathlineto{\pgfqpoint{3.275211in}{5.223289in}}%
\pgfpathlineto{\pgfqpoint{3.275817in}{5.201319in}}%
\pgfpathlineto{\pgfqpoint{3.276323in}{5.219172in}}%
\pgfpathlineto{\pgfqpoint{3.278547in}{5.269552in}}%
\pgfpathlineto{\pgfqpoint{3.278648in}{5.270096in}}%
\pgfpathlineto{\pgfqpoint{3.278850in}{5.267888in}}%
\pgfpathlineto{\pgfqpoint{3.280872in}{5.196780in}}%
\pgfpathlineto{\pgfqpoint{3.281377in}{5.180737in}}%
\pgfpathlineto{\pgfqpoint{3.281782in}{5.196620in}}%
\pgfpathlineto{\pgfqpoint{3.283399in}{5.291097in}}%
\pgfpathlineto{\pgfqpoint{3.283905in}{5.273928in}}%
\pgfpathlineto{\pgfqpoint{3.285320in}{5.138175in}}%
\pgfpathlineto{\pgfqpoint{3.286028in}{5.196561in}}%
\pgfpathlineto{\pgfqpoint{3.287544in}{5.251420in}}%
\pgfpathlineto{\pgfqpoint{3.288049in}{5.254965in}}%
\pgfpathlineto{\pgfqpoint{3.288454in}{5.251357in}}%
\pgfpathlineto{\pgfqpoint{3.289161in}{5.241164in}}%
\pgfpathlineto{\pgfqpoint{3.289667in}{5.249029in}}%
\pgfpathlineto{\pgfqpoint{3.289970in}{5.252242in}}%
\pgfpathlineto{\pgfqpoint{3.290375in}{5.243267in}}%
\pgfpathlineto{\pgfqpoint{3.291385in}{5.209178in}}%
\pgfpathlineto{\pgfqpoint{3.291891in}{5.218460in}}%
\pgfpathlineto{\pgfqpoint{3.292599in}{5.229592in}}%
\pgfpathlineto{\pgfqpoint{3.293205in}{5.223915in}}%
\pgfpathlineto{\pgfqpoint{3.294014in}{5.212778in}}%
\pgfpathlineto{\pgfqpoint{3.294519in}{5.219350in}}%
\pgfpathlineto{\pgfqpoint{3.295935in}{5.252130in}}%
\pgfpathlineto{\pgfqpoint{3.296440in}{5.247738in}}%
\pgfpathlineto{\pgfqpoint{3.297855in}{5.206873in}}%
\pgfpathlineto{\pgfqpoint{3.299271in}{5.211430in}}%
\pgfpathlineto{\pgfqpoint{3.299574in}{5.209078in}}%
\pgfpathlineto{\pgfqpoint{3.299877in}{5.214731in}}%
\pgfpathlineto{\pgfqpoint{3.301192in}{5.292976in}}%
\pgfpathlineto{\pgfqpoint{3.301697in}{5.259165in}}%
\pgfpathlineto{\pgfqpoint{3.302607in}{5.172260in}}%
\pgfpathlineto{\pgfqpoint{3.303112in}{5.201647in}}%
\pgfpathlineto{\pgfqpoint{3.305033in}{5.291509in}}%
\pgfpathlineto{\pgfqpoint{3.305235in}{5.289514in}}%
\pgfpathlineto{\pgfqpoint{3.305943in}{5.218830in}}%
\pgfpathlineto{\pgfqpoint{3.306651in}{5.150837in}}%
\pgfpathlineto{\pgfqpoint{3.307257in}{5.188443in}}%
\pgfpathlineto{\pgfqpoint{3.307965in}{5.223003in}}%
\pgfpathlineto{\pgfqpoint{3.308672in}{5.217865in}}%
\pgfpathlineto{\pgfqpoint{3.309582in}{5.244356in}}%
\pgfpathlineto{\pgfqpoint{3.310189in}{5.226734in}}%
\pgfpathlineto{\pgfqpoint{3.310492in}{5.222021in}}%
\pgfpathlineto{\pgfqpoint{3.310795in}{5.232958in}}%
\pgfpathlineto{\pgfqpoint{3.311604in}{5.280567in}}%
\pgfpathlineto{\pgfqpoint{3.312009in}{5.255889in}}%
\pgfpathlineto{\pgfqpoint{3.312817in}{5.184278in}}%
\pgfpathlineto{\pgfqpoint{3.313424in}{5.216314in}}%
\pgfpathlineto{\pgfqpoint{3.314738in}{5.304520in}}%
\pgfpathlineto{\pgfqpoint{3.315244in}{5.291287in}}%
\pgfpathlineto{\pgfqpoint{3.317063in}{5.134699in}}%
\pgfpathlineto{\pgfqpoint{3.318175in}{5.185955in}}%
\pgfpathlineto{\pgfqpoint{3.319692in}{5.310641in}}%
\pgfpathlineto{\pgfqpoint{3.320298in}{5.275002in}}%
\pgfpathlineto{\pgfqpoint{3.321815in}{5.087857in}}%
\pgfpathlineto{\pgfqpoint{3.322421in}{5.158681in}}%
\pgfpathlineto{\pgfqpoint{3.323836in}{5.448070in}}%
\pgfpathlineto{\pgfqpoint{3.324342in}{5.387536in}}%
\pgfpathlineto{\pgfqpoint{3.326667in}{4.956086in}}%
\pgfpathlineto{\pgfqpoint{3.327071in}{5.019684in}}%
\pgfpathlineto{\pgfqpoint{3.328386in}{5.605580in}}%
\pgfpathlineto{\pgfqpoint{3.329093in}{5.427708in}}%
\pgfpathlineto{\pgfqpoint{3.330509in}{4.951102in}}%
\pgfpathlineto{\pgfqpoint{3.331014in}{5.039563in}}%
\pgfpathlineto{\pgfqpoint{3.332227in}{5.407223in}}%
\pgfpathlineto{\pgfqpoint{3.332834in}{5.336363in}}%
\pgfpathlineto{\pgfqpoint{3.334047in}{4.991194in}}%
\pgfpathlineto{\pgfqpoint{3.334552in}{5.127348in}}%
\pgfpathlineto{\pgfqpoint{3.335462in}{5.511865in}}%
\pgfpathlineto{\pgfqpoint{3.335968in}{5.375773in}}%
\pgfpathlineto{\pgfqpoint{3.336979in}{4.991827in}}%
\pgfpathlineto{\pgfqpoint{3.337484in}{5.094746in}}%
\pgfpathlineto{\pgfqpoint{3.338293in}{5.237666in}}%
\pgfpathlineto{\pgfqpoint{3.338798in}{5.195700in}}%
\pgfpathlineto{\pgfqpoint{3.339304in}{5.168810in}}%
\pgfpathlineto{\pgfqpoint{3.339708in}{5.194697in}}%
\pgfpathlineto{\pgfqpoint{3.340719in}{5.467843in}}%
\pgfpathlineto{\pgfqpoint{3.341427in}{5.662253in}}%
\pgfpathlineto{\pgfqpoint{3.341932in}{5.519837in}}%
\pgfpathlineto{\pgfqpoint{3.343145in}{4.916285in}}%
\pgfpathlineto{\pgfqpoint{3.343853in}{4.969629in}}%
\pgfpathlineto{\pgfqpoint{3.344763in}{5.145113in}}%
\pgfpathlineto{\pgfqpoint{3.345572in}{5.352956in}}%
\pgfpathlineto{\pgfqpoint{3.346077in}{5.254197in}}%
\pgfpathlineto{\pgfqpoint{3.346987in}{5.010200in}}%
\pgfpathlineto{\pgfqpoint{3.347492in}{5.086548in}}%
\pgfpathlineto{\pgfqpoint{3.349716in}{5.658087in}}%
\pgfpathlineto{\pgfqpoint{3.350525in}{5.490814in}}%
\pgfpathlineto{\pgfqpoint{3.352042in}{4.901056in}}%
\pgfpathlineto{\pgfqpoint{3.352749in}{5.024762in}}%
\pgfpathlineto{\pgfqpoint{3.353558in}{5.112388in}}%
\pgfpathlineto{\pgfqpoint{3.354165in}{5.095257in}}%
\pgfpathlineto{\pgfqpoint{3.354569in}{5.086224in}}%
\pgfpathlineto{\pgfqpoint{3.354872in}{5.098573in}}%
\pgfpathlineto{\pgfqpoint{3.356085in}{5.284251in}}%
\pgfpathlineto{\pgfqpoint{3.357703in}{5.656719in}}%
\pgfpathlineto{\pgfqpoint{3.358208in}{5.538327in}}%
\pgfpathlineto{\pgfqpoint{3.359624in}{4.958860in}}%
\pgfpathlineto{\pgfqpoint{3.360230in}{5.037095in}}%
\pgfpathlineto{\pgfqpoint{3.361140in}{5.133017in}}%
\pgfpathlineto{\pgfqpoint{3.361645in}{5.096370in}}%
\pgfpathlineto{\pgfqpoint{3.361949in}{5.081388in}}%
\pgfpathlineto{\pgfqpoint{3.362252in}{5.111985in}}%
\pgfpathlineto{\pgfqpoint{3.363263in}{5.350154in}}%
\pgfpathlineto{\pgfqpoint{3.364274in}{5.318862in}}%
\pgfpathlineto{\pgfqpoint{3.364880in}{5.373835in}}%
\pgfpathlineto{\pgfqpoint{3.365285in}{5.328058in}}%
\pgfpathlineto{\pgfqpoint{3.366195in}{5.141901in}}%
\pgfpathlineto{\pgfqpoint{3.366801in}{5.214177in}}%
\pgfpathlineto{\pgfqpoint{3.367509in}{5.299783in}}%
\pgfpathlineto{\pgfqpoint{3.368115in}{5.257369in}}%
\pgfpathlineto{\pgfqpoint{3.370339in}{5.025714in}}%
\pgfpathlineto{\pgfqpoint{3.371047in}{5.105547in}}%
\pgfpathlineto{\pgfqpoint{3.372564in}{5.554895in}}%
\pgfpathlineto{\pgfqpoint{3.373372in}{5.395315in}}%
\pgfpathlineto{\pgfqpoint{3.374585in}{5.022553in}}%
\pgfpathlineto{\pgfqpoint{3.375091in}{5.154805in}}%
\pgfpathlineto{\pgfqpoint{3.375799in}{5.386011in}}%
\pgfpathlineto{\pgfqpoint{3.376304in}{5.269688in}}%
\pgfpathlineto{\pgfqpoint{3.377315in}{4.981509in}}%
\pgfpathlineto{\pgfqpoint{3.377921in}{5.022358in}}%
\pgfpathlineto{\pgfqpoint{3.379236in}{5.273982in}}%
\pgfpathlineto{\pgfqpoint{3.380348in}{5.553578in}}%
\pgfpathlineto{\pgfqpoint{3.380853in}{5.481383in}}%
\pgfpathlineto{\pgfqpoint{3.382673in}{5.018582in}}%
\pgfpathlineto{\pgfqpoint{3.383178in}{5.080185in}}%
\pgfpathlineto{\pgfqpoint{3.384290in}{5.348721in}}%
\pgfpathlineto{\pgfqpoint{3.384796in}{5.272006in}}%
\pgfpathlineto{\pgfqpoint{3.386009in}{5.141729in}}%
\pgfpathlineto{\pgfqpoint{3.386413in}{5.155385in}}%
\pgfpathlineto{\pgfqpoint{3.387728in}{5.375700in}}%
\pgfpathlineto{\pgfqpoint{3.388637in}{5.283087in}}%
\pgfpathlineto{\pgfqpoint{3.390255in}{5.132145in}}%
\pgfpathlineto{\pgfqpoint{3.390861in}{5.153260in}}%
\pgfpathlineto{\pgfqpoint{3.390963in}{5.154491in}}%
\pgfpathlineto{\pgfqpoint{3.391266in}{5.145977in}}%
\pgfpathlineto{\pgfqpoint{3.392075in}{5.095152in}}%
\pgfpathlineto{\pgfqpoint{3.392479in}{5.138904in}}%
\pgfpathlineto{\pgfqpoint{3.393894in}{5.382835in}}%
\pgfpathlineto{\pgfqpoint{3.394602in}{5.367640in}}%
\pgfpathlineto{\pgfqpoint{3.395411in}{5.244990in}}%
\pgfpathlineto{\pgfqpoint{3.395916in}{5.184536in}}%
\pgfpathlineto{\pgfqpoint{3.396523in}{5.232197in}}%
\pgfpathlineto{\pgfqpoint{3.396927in}{5.253245in}}%
\pgfpathlineto{\pgfqpoint{3.397432in}{5.227869in}}%
\pgfpathlineto{\pgfqpoint{3.398140in}{5.204737in}}%
\pgfpathlineto{\pgfqpoint{3.398646in}{5.216568in}}%
\pgfpathlineto{\pgfqpoint{3.399859in}{5.336976in}}%
\pgfpathlineto{\pgfqpoint{3.400465in}{5.276837in}}%
\pgfpathlineto{\pgfqpoint{3.402083in}{4.936559in}}%
\pgfpathlineto{\pgfqpoint{3.402689in}{5.026285in}}%
\pgfpathlineto{\pgfqpoint{3.403801in}{5.340509in}}%
\pgfpathlineto{\pgfqpoint{3.404509in}{5.246540in}}%
\pgfpathlineto{\pgfqpoint{3.404812in}{5.222299in}}%
\pgfpathlineto{\pgfqpoint{3.405217in}{5.275392in}}%
\pgfpathlineto{\pgfqpoint{3.406329in}{5.468060in}}%
\pgfpathlineto{\pgfqpoint{3.407036in}{5.451474in}}%
\pgfpathlineto{\pgfqpoint{3.407744in}{5.389263in}}%
\pgfpathlineto{\pgfqpoint{3.409362in}{5.061917in}}%
\pgfpathlineto{\pgfqpoint{3.410069in}{5.155226in}}%
\pgfpathlineto{\pgfqpoint{3.410372in}{5.180906in}}%
\pgfpathlineto{\pgfqpoint{3.410878in}{5.122439in}}%
\pgfpathlineto{\pgfqpoint{3.411889in}{4.998381in}}%
\pgfpathlineto{\pgfqpoint{3.412394in}{5.011217in}}%
\pgfpathlineto{\pgfqpoint{3.413102in}{5.105026in}}%
\pgfpathlineto{\pgfqpoint{3.414416in}{5.455490in}}%
\pgfpathlineto{\pgfqpoint{3.415023in}{5.343600in}}%
\pgfpathlineto{\pgfqpoint{3.415629in}{5.262902in}}%
\pgfpathlineto{\pgfqpoint{3.416236in}{5.307866in}}%
\pgfpathlineto{\pgfqpoint{3.418157in}{5.493307in}}%
\pgfpathlineto{\pgfqpoint{3.418561in}{5.440194in}}%
\pgfpathlineto{\pgfqpoint{3.421291in}{4.833246in}}%
\pgfpathlineto{\pgfqpoint{3.421493in}{4.841357in}}%
\pgfpathlineto{\pgfqpoint{3.422301in}{5.055858in}}%
\pgfpathlineto{\pgfqpoint{3.423919in}{5.616457in}}%
\pgfpathlineto{\pgfqpoint{3.424526in}{5.528952in}}%
\pgfpathlineto{\pgfqpoint{3.425739in}{5.321056in}}%
\pgfpathlineto{\pgfqpoint{3.426345in}{5.334931in}}%
\pgfpathlineto{\pgfqpoint{3.426851in}{5.292591in}}%
\pgfpathlineto{\pgfqpoint{3.428771in}{4.861442in}}%
\pgfpathlineto{\pgfqpoint{3.429580in}{5.022201in}}%
\pgfpathlineto{\pgfqpoint{3.430996in}{5.351667in}}%
\pgfpathlineto{\pgfqpoint{3.431501in}{5.320753in}}%
\pgfpathlineto{\pgfqpoint{3.431804in}{5.307681in}}%
\pgfpathlineto{\pgfqpoint{3.432209in}{5.334468in}}%
\pgfpathlineto{\pgfqpoint{3.433523in}{5.556031in}}%
\pgfpathlineto{\pgfqpoint{3.434028in}{5.482723in}}%
\pgfpathlineto{\pgfqpoint{3.436353in}{4.898428in}}%
\pgfpathlineto{\pgfqpoint{3.436859in}{4.942844in}}%
\pgfpathlineto{\pgfqpoint{3.440498in}{5.461323in}}%
\pgfpathlineto{\pgfqpoint{3.441206in}{5.511791in}}%
\pgfpathlineto{\pgfqpoint{3.441711in}{5.475258in}}%
\pgfpathlineto{\pgfqpoint{3.444037in}{5.077330in}}%
\pgfpathlineto{\pgfqpoint{3.444845in}{5.099074in}}%
\pgfpathlineto{\pgfqpoint{3.445149in}{5.090091in}}%
\pgfpathlineto{\pgfqpoint{3.446058in}{5.013080in}}%
\pgfpathlineto{\pgfqpoint{3.446564in}{5.063359in}}%
\pgfpathlineto{\pgfqpoint{3.448384in}{5.446505in}}%
\pgfpathlineto{\pgfqpoint{3.448990in}{5.398190in}}%
\pgfpathlineto{\pgfqpoint{3.450304in}{5.112586in}}%
\pgfpathlineto{\pgfqpoint{3.451012in}{5.204602in}}%
\pgfpathlineto{\pgfqpoint{3.451922in}{5.342875in}}%
\pgfpathlineto{\pgfqpoint{3.452427in}{5.292554in}}%
\pgfpathlineto{\pgfqpoint{3.453843in}{5.129348in}}%
\pgfpathlineto{\pgfqpoint{3.454247in}{5.142041in}}%
\pgfpathlineto{\pgfqpoint{3.455763in}{5.295632in}}%
\pgfpathlineto{\pgfqpoint{3.456774in}{5.260872in}}%
\pgfpathlineto{\pgfqpoint{3.456875in}{5.259853in}}%
\pgfpathlineto{\pgfqpoint{3.457280in}{5.266953in}}%
\pgfpathlineto{\pgfqpoint{3.457886in}{5.284352in}}%
\pgfpathlineto{\pgfqpoint{3.458190in}{5.269645in}}%
\pgfpathlineto{\pgfqpoint{3.459605in}{5.093262in}}%
\pgfpathlineto{\pgfqpoint{3.460414in}{5.147472in}}%
\pgfpathlineto{\pgfqpoint{3.462537in}{5.309011in}}%
\pgfpathlineto{\pgfqpoint{3.462941in}{5.293826in}}%
\pgfpathlineto{\pgfqpoint{3.463649in}{5.250760in}}%
\pgfpathlineto{\pgfqpoint{3.464154in}{5.284153in}}%
\pgfpathlineto{\pgfqpoint{3.464761in}{5.328837in}}%
\pgfpathlineto{\pgfqpoint{3.465367in}{5.300421in}}%
\pgfpathlineto{\pgfqpoint{3.467389in}{5.086919in}}%
\pgfpathlineto{\pgfqpoint{3.468097in}{5.146012in}}%
\pgfpathlineto{\pgfqpoint{3.468804in}{5.195921in}}%
\pgfpathlineto{\pgfqpoint{3.469613in}{5.194946in}}%
\pgfpathlineto{\pgfqpoint{3.471736in}{5.374797in}}%
\pgfpathlineto{\pgfqpoint{3.472444in}{5.317188in}}%
\pgfpathlineto{\pgfqpoint{3.473758in}{5.125170in}}%
\pgfpathlineto{\pgfqpoint{3.474466in}{5.154132in}}%
\pgfpathlineto{\pgfqpoint{3.474769in}{5.161799in}}%
\pgfpathlineto{\pgfqpoint{3.475376in}{5.146399in}}%
\pgfpathlineto{\pgfqpoint{3.475780in}{5.139704in}}%
\pgfpathlineto{\pgfqpoint{3.476285in}{5.150519in}}%
\pgfpathlineto{\pgfqpoint{3.477296in}{5.246415in}}%
\pgfpathlineto{\pgfqpoint{3.479116in}{5.436419in}}%
\pgfpathlineto{\pgfqpoint{3.479520in}{5.417040in}}%
\pgfpathlineto{\pgfqpoint{3.481947in}{5.080549in}}%
\pgfpathlineto{\pgfqpoint{3.482856in}{5.127543in}}%
\pgfpathlineto{\pgfqpoint{3.483665in}{5.165695in}}%
\pgfpathlineto{\pgfqpoint{3.484171in}{5.154519in}}%
\pgfpathlineto{\pgfqpoint{3.484474in}{5.148854in}}%
\pgfpathlineto{\pgfqpoint{3.484878in}{5.161455in}}%
\pgfpathlineto{\pgfqpoint{3.487507in}{5.426238in}}%
\pgfpathlineto{\pgfqpoint{3.488214in}{5.361001in}}%
\pgfpathlineto{\pgfqpoint{3.490540in}{5.112652in}}%
\pgfpathlineto{\pgfqpoint{3.490742in}{5.114697in}}%
\pgfpathlineto{\pgfqpoint{3.491449in}{5.157250in}}%
\pgfpathlineto{\pgfqpoint{3.493269in}{5.306712in}}%
\pgfpathlineto{\pgfqpoint{3.493775in}{5.287573in}}%
\pgfpathlineto{\pgfqpoint{3.495089in}{5.131518in}}%
\pgfpathlineto{\pgfqpoint{3.495796in}{5.180419in}}%
\pgfpathlineto{\pgfqpoint{3.496807in}{5.240894in}}%
\pgfpathlineto{\pgfqpoint{3.497212in}{5.223866in}}%
\pgfpathlineto{\pgfqpoint{3.498324in}{5.146290in}}%
\pgfpathlineto{\pgfqpoint{3.498829in}{5.178008in}}%
\pgfpathlineto{\pgfqpoint{3.500750in}{5.490936in}}%
\pgfpathlineto{\pgfqpoint{3.501458in}{5.396049in}}%
\pgfpathlineto{\pgfqpoint{3.503378in}{5.009328in}}%
\pgfpathlineto{\pgfqpoint{3.503884in}{5.042913in}}%
\pgfpathlineto{\pgfqpoint{3.505198in}{5.248788in}}%
\pgfpathlineto{\pgfqpoint{3.506007in}{5.214736in}}%
\pgfpathlineto{\pgfqpoint{3.506108in}{5.214823in}}%
\pgfpathlineto{\pgfqpoint{3.506715in}{5.253530in}}%
\pgfpathlineto{\pgfqpoint{3.507422in}{5.287004in}}%
\pgfpathlineto{\pgfqpoint{3.507928in}{5.271143in}}%
\pgfpathlineto{\pgfqpoint{3.509848in}{5.187129in}}%
\pgfpathlineto{\pgfqpoint{3.510354in}{5.215648in}}%
\pgfpathlineto{\pgfqpoint{3.511870in}{5.315943in}}%
\pgfpathlineto{\pgfqpoint{3.512275in}{5.310611in}}%
\pgfpathlineto{\pgfqpoint{3.512982in}{5.250213in}}%
\pgfpathlineto{\pgfqpoint{3.513993in}{5.145706in}}%
\pgfpathlineto{\pgfqpoint{3.514499in}{5.180121in}}%
\pgfpathlineto{\pgfqpoint{3.515307in}{5.243570in}}%
\pgfpathlineto{\pgfqpoint{3.515914in}{5.226473in}}%
\pgfpathlineto{\pgfqpoint{3.517633in}{5.178785in}}%
\pgfpathlineto{\pgfqpoint{3.517936in}{5.188254in}}%
\pgfpathlineto{\pgfqpoint{3.518947in}{5.270538in}}%
\pgfpathlineto{\pgfqpoint{3.519553in}{5.232141in}}%
\pgfpathlineto{\pgfqpoint{3.520362in}{5.163575in}}%
\pgfpathlineto{\pgfqpoint{3.520969in}{5.191799in}}%
\pgfpathlineto{\pgfqpoint{3.523193in}{5.315732in}}%
\pgfpathlineto{\pgfqpoint{3.523496in}{5.311138in}}%
\pgfpathlineto{\pgfqpoint{3.525619in}{5.231564in}}%
\pgfpathlineto{\pgfqpoint{3.527236in}{5.140969in}}%
\pgfpathlineto{\pgfqpoint{3.527843in}{5.151470in}}%
\pgfpathlineto{\pgfqpoint{3.529865in}{5.230448in}}%
\pgfpathlineto{\pgfqpoint{3.530775in}{5.342519in}}%
\pgfpathlineto{\pgfqpoint{3.531280in}{5.304325in}}%
\pgfpathlineto{\pgfqpoint{3.533302in}{5.205800in}}%
\pgfpathlineto{\pgfqpoint{3.533403in}{5.205101in}}%
\pgfpathlineto{\pgfqpoint{3.533504in}{5.207823in}}%
\pgfpathlineto{\pgfqpoint{3.534616in}{5.332882in}}%
\pgfpathlineto{\pgfqpoint{3.535324in}{5.262649in}}%
\pgfpathlineto{\pgfqpoint{3.537346in}{5.102858in}}%
\pgfpathlineto{\pgfqpoint{3.537548in}{5.100516in}}%
\pgfpathlineto{\pgfqpoint{3.537851in}{5.112234in}}%
\pgfpathlineto{\pgfqpoint{3.541390in}{5.313750in}}%
\pgfpathlineto{\pgfqpoint{3.541996in}{5.318130in}}%
\pgfpathlineto{\pgfqpoint{3.542400in}{5.314064in}}%
\pgfpathlineto{\pgfqpoint{3.543209in}{5.269718in}}%
\pgfpathlineto{\pgfqpoint{3.545838in}{5.113254in}}%
\pgfpathlineto{\pgfqpoint{3.546040in}{5.118860in}}%
\pgfpathlineto{\pgfqpoint{3.547051in}{5.250604in}}%
\pgfpathlineto{\pgfqpoint{3.547758in}{5.293306in}}%
\pgfpathlineto{\pgfqpoint{3.548365in}{5.276789in}}%
\pgfpathlineto{\pgfqpoint{3.549679in}{5.221178in}}%
\pgfpathlineto{\pgfqpoint{3.550488in}{5.163145in}}%
\pgfpathlineto{\pgfqpoint{3.550993in}{5.183477in}}%
\pgfpathlineto{\pgfqpoint{3.552712in}{5.361965in}}%
\pgfpathlineto{\pgfqpoint{3.553420in}{5.301406in}}%
\pgfpathlineto{\pgfqpoint{3.555037in}{5.120154in}}%
\pgfpathlineto{\pgfqpoint{3.555543in}{5.139634in}}%
\pgfpathlineto{\pgfqpoint{3.556554in}{5.210168in}}%
\pgfpathlineto{\pgfqpoint{3.557160in}{5.186861in}}%
\pgfpathlineto{\pgfqpoint{3.557362in}{5.182811in}}%
\pgfpathlineto{\pgfqpoint{3.557767in}{5.199923in}}%
\pgfpathlineto{\pgfqpoint{3.560193in}{5.319491in}}%
\pgfpathlineto{\pgfqpoint{3.560294in}{5.320410in}}%
\pgfpathlineto{\pgfqpoint{3.560597in}{5.313245in}}%
\pgfpathlineto{\pgfqpoint{3.563630in}{5.174738in}}%
\pgfpathlineto{\pgfqpoint{3.564034in}{5.179948in}}%
\pgfpathlineto{\pgfqpoint{3.564338in}{5.182991in}}%
\pgfpathlineto{\pgfqpoint{3.564944in}{5.176969in}}%
\pgfpathlineto{\pgfqpoint{3.565045in}{5.176735in}}%
\pgfpathlineto{\pgfqpoint{3.565147in}{5.177413in}}%
\pgfpathlineto{\pgfqpoint{3.568179in}{5.242464in}}%
\pgfpathlineto{\pgfqpoint{3.569595in}{5.306599in}}%
\pgfpathlineto{\pgfqpoint{3.569999in}{5.300601in}}%
\pgfpathlineto{\pgfqpoint{3.571313in}{5.206358in}}%
\pgfpathlineto{\pgfqpoint{3.571718in}{5.193394in}}%
\pgfpathlineto{\pgfqpoint{3.572223in}{5.213868in}}%
\pgfpathlineto{\pgfqpoint{3.573335in}{5.291092in}}%
\pgfpathlineto{\pgfqpoint{3.573942in}{5.268812in}}%
\pgfpathlineto{\pgfqpoint{3.576065in}{5.141474in}}%
\pgfpathlineto{\pgfqpoint{3.576671in}{5.153848in}}%
\pgfpathlineto{\pgfqpoint{3.578491in}{5.217423in}}%
\pgfpathlineto{\pgfqpoint{3.581321in}{5.426273in}}%
\pgfpathlineto{\pgfqpoint{3.581625in}{5.411545in}}%
\pgfpathlineto{\pgfqpoint{3.582939in}{5.176951in}}%
\pgfpathlineto{\pgfqpoint{3.584051in}{5.102536in}}%
\pgfpathlineto{\pgfqpoint{3.584455in}{5.104300in}}%
\pgfpathlineto{\pgfqpoint{3.584759in}{5.105745in}}%
\pgfpathlineto{\pgfqpoint{3.585163in}{5.101615in}}%
\pgfpathlineto{\pgfqpoint{3.585770in}{5.090059in}}%
\pgfpathlineto{\pgfqpoint{3.586275in}{5.101236in}}%
\pgfpathlineto{\pgfqpoint{3.587893in}{5.240809in}}%
\pgfpathlineto{\pgfqpoint{3.589712in}{5.459817in}}%
\pgfpathlineto{\pgfqpoint{3.590319in}{5.427920in}}%
\pgfpathlineto{\pgfqpoint{3.593149in}{5.083204in}}%
\pgfpathlineto{\pgfqpoint{3.594261in}{4.996869in}}%
\pgfpathlineto{\pgfqpoint{3.594767in}{5.018832in}}%
\pgfpathlineto{\pgfqpoint{3.598811in}{5.405755in}}%
\pgfpathlineto{\pgfqpoint{3.599619in}{5.367093in}}%
\pgfpathlineto{\pgfqpoint{3.604067in}{5.072036in}}%
\pgfpathlineto{\pgfqpoint{3.604169in}{5.073787in}}%
\pgfpathlineto{\pgfqpoint{3.605078in}{5.140546in}}%
\pgfpathlineto{\pgfqpoint{3.607707in}{5.277612in}}%
\pgfpathlineto{\pgfqpoint{3.609122in}{5.343877in}}%
\pgfpathlineto{\pgfqpoint{3.609931in}{5.317953in}}%
\pgfpathlineto{\pgfqpoint{3.612458in}{5.136303in}}%
\pgfpathlineto{\pgfqpoint{3.613469in}{5.166676in}}%
\pgfpathlineto{\pgfqpoint{3.615087in}{5.289356in}}%
\pgfpathlineto{\pgfqpoint{3.615693in}{5.253836in}}%
\pgfpathlineto{\pgfqpoint{3.616502in}{5.217644in}}%
\pgfpathlineto{\pgfqpoint{3.617109in}{5.228324in}}%
\pgfpathlineto{\pgfqpoint{3.617917in}{5.238295in}}%
\pgfpathlineto{\pgfqpoint{3.618322in}{5.232475in}}%
\pgfpathlineto{\pgfqpoint{3.619535in}{5.183072in}}%
\pgfpathlineto{\pgfqpoint{3.620141in}{5.201001in}}%
\pgfpathlineto{\pgfqpoint{3.620950in}{5.226201in}}%
\pgfpathlineto{\pgfqpoint{3.621557in}{5.216024in}}%
\pgfpathlineto{\pgfqpoint{3.621860in}{5.210664in}}%
\pgfpathlineto{\pgfqpoint{3.622264in}{5.222715in}}%
\pgfpathlineto{\pgfqpoint{3.623680in}{5.361139in}}%
\pgfpathlineto{\pgfqpoint{3.624387in}{5.305565in}}%
\pgfpathlineto{\pgfqpoint{3.626308in}{5.135374in}}%
\pgfpathlineto{\pgfqpoint{3.626712in}{5.148253in}}%
\pgfpathlineto{\pgfqpoint{3.628431in}{5.235617in}}%
\pgfpathlineto{\pgfqpoint{3.629038in}{5.215240in}}%
\pgfpathlineto{\pgfqpoint{3.629745in}{5.187446in}}%
\pgfpathlineto{\pgfqpoint{3.630150in}{5.204574in}}%
\pgfpathlineto{\pgfqpoint{3.631363in}{5.294220in}}%
\pgfpathlineto{\pgfqpoint{3.632070in}{5.272595in}}%
\pgfpathlineto{\pgfqpoint{3.632879in}{5.250604in}}%
\pgfpathlineto{\pgfqpoint{3.633587in}{5.252294in}}%
\pgfpathlineto{\pgfqpoint{3.634497in}{5.231091in}}%
\pgfpathlineto{\pgfqpoint{3.637833in}{5.178021in}}%
\pgfpathlineto{\pgfqpoint{3.638136in}{5.187979in}}%
\pgfpathlineto{\pgfqpoint{3.639248in}{5.259506in}}%
\pgfpathlineto{\pgfqpoint{3.639855in}{5.230356in}}%
\pgfpathlineto{\pgfqpoint{3.640259in}{5.217667in}}%
\pgfpathlineto{\pgfqpoint{3.640764in}{5.238073in}}%
\pgfpathlineto{\pgfqpoint{3.642483in}{5.277911in}}%
\pgfpathlineto{\pgfqpoint{3.642887in}{5.285554in}}%
\pgfpathlineto{\pgfqpoint{3.643292in}{5.269288in}}%
\pgfpathlineto{\pgfqpoint{3.644707in}{5.136279in}}%
\pgfpathlineto{\pgfqpoint{3.645415in}{5.174263in}}%
\pgfpathlineto{\pgfqpoint{3.646931in}{5.284718in}}%
\pgfpathlineto{\pgfqpoint{3.647437in}{5.256224in}}%
\pgfpathlineto{\pgfqpoint{3.648346in}{5.183026in}}%
\pgfpathlineto{\pgfqpoint{3.648852in}{5.213818in}}%
\pgfpathlineto{\pgfqpoint{3.650065in}{5.333138in}}%
\pgfpathlineto{\pgfqpoint{3.650570in}{5.296719in}}%
\pgfpathlineto{\pgfqpoint{3.652491in}{5.144796in}}%
\pgfpathlineto{\pgfqpoint{3.652795in}{5.151336in}}%
\pgfpathlineto{\pgfqpoint{3.655827in}{5.274636in}}%
\pgfpathlineto{\pgfqpoint{3.656232in}{5.268567in}}%
\pgfpathlineto{\pgfqpoint{3.656737in}{5.258284in}}%
\pgfpathlineto{\pgfqpoint{3.657344in}{5.267874in}}%
\pgfpathlineto{\pgfqpoint{3.657849in}{5.274953in}}%
\pgfpathlineto{\pgfqpoint{3.658355in}{5.265785in}}%
\pgfpathlineto{\pgfqpoint{3.660781in}{5.156596in}}%
\pgfpathlineto{\pgfqpoint{3.661387in}{5.185781in}}%
\pgfpathlineto{\pgfqpoint{3.662499in}{5.249277in}}%
\pgfpathlineto{\pgfqpoint{3.663106in}{5.243488in}}%
\pgfpathlineto{\pgfqpoint{3.663207in}{5.243128in}}%
\pgfpathlineto{\pgfqpoint{3.663612in}{5.245733in}}%
\pgfpathlineto{\pgfqpoint{3.664218in}{5.250787in}}%
\pgfpathlineto{\pgfqpoint{3.664724in}{5.246908in}}%
\pgfpathlineto{\pgfqpoint{3.665330in}{5.240224in}}%
\pgfpathlineto{\pgfqpoint{3.665836in}{5.245935in}}%
\pgfpathlineto{\pgfqpoint{3.666543in}{5.258359in}}%
\pgfpathlineto{\pgfqpoint{3.666948in}{5.250440in}}%
\pgfpathlineto{\pgfqpoint{3.668363in}{5.202255in}}%
\pgfpathlineto{\pgfqpoint{3.668969in}{5.212823in}}%
\pgfpathlineto{\pgfqpoint{3.669576in}{5.226017in}}%
\pgfpathlineto{\pgfqpoint{3.670081in}{5.216560in}}%
\pgfpathlineto{\pgfqpoint{3.670890in}{5.199986in}}%
\pgfpathlineto{\pgfqpoint{3.671396in}{5.207321in}}%
\pgfpathlineto{\pgfqpoint{3.673013in}{5.268388in}}%
\pgfpathlineto{\pgfqpoint{3.673721in}{5.246812in}}%
\pgfpathlineto{\pgfqpoint{3.674327in}{5.232912in}}%
\pgfpathlineto{\pgfqpoint{3.674934in}{5.242057in}}%
\pgfpathlineto{\pgfqpoint{3.675136in}{5.244165in}}%
\pgfpathlineto{\pgfqpoint{3.675541in}{5.239071in}}%
\pgfpathlineto{\pgfqpoint{3.676349in}{5.221179in}}%
\pgfpathlineto{\pgfqpoint{3.676956in}{5.229037in}}%
\pgfpathlineto{\pgfqpoint{3.677057in}{5.229496in}}%
\pgfpathlineto{\pgfqpoint{3.677259in}{5.227761in}}%
\pgfpathlineto{\pgfqpoint{3.678270in}{5.196142in}}%
\pgfpathlineto{\pgfqpoint{3.678674in}{5.211608in}}%
\pgfpathlineto{\pgfqpoint{3.679584in}{5.267164in}}%
\pgfpathlineto{\pgfqpoint{3.680191in}{5.244416in}}%
\pgfpathlineto{\pgfqpoint{3.681404in}{5.199033in}}%
\pgfpathlineto{\pgfqpoint{3.681909in}{5.202745in}}%
\pgfpathlineto{\pgfqpoint{3.683123in}{5.220792in}}%
\pgfpathlineto{\pgfqpoint{3.684538in}{5.256822in}}%
\pgfpathlineto{\pgfqpoint{3.685144in}{5.256784in}}%
\pgfpathlineto{\pgfqpoint{3.685852in}{5.260608in}}%
\pgfpathlineto{\pgfqpoint{3.686155in}{5.257862in}}%
\pgfpathlineto{\pgfqpoint{3.690603in}{5.176930in}}%
\pgfpathlineto{\pgfqpoint{3.691008in}{5.183270in}}%
\pgfpathlineto{\pgfqpoint{3.693232in}{5.296531in}}%
\pgfpathlineto{\pgfqpoint{3.694142in}{5.264291in}}%
\pgfpathlineto{\pgfqpoint{3.695759in}{5.221108in}}%
\pgfpathlineto{\pgfqpoint{3.696063in}{5.222656in}}%
\pgfpathlineto{\pgfqpoint{3.696972in}{5.233201in}}%
\pgfpathlineto{\pgfqpoint{3.697377in}{5.225937in}}%
\pgfpathlineto{\pgfqpoint{3.698792in}{5.163326in}}%
\pgfpathlineto{\pgfqpoint{3.699500in}{5.183723in}}%
\pgfpathlineto{\pgfqpoint{3.701319in}{5.265092in}}%
\pgfpathlineto{\pgfqpoint{3.701926in}{5.255698in}}%
\pgfpathlineto{\pgfqpoint{3.702735in}{5.240462in}}%
\pgfpathlineto{\pgfqpoint{3.703240in}{5.247566in}}%
\pgfpathlineto{\pgfqpoint{3.703948in}{5.266246in}}%
\pgfpathlineto{\pgfqpoint{3.704453in}{5.252942in}}%
\pgfpathlineto{\pgfqpoint{3.705565in}{5.212972in}}%
\pgfpathlineto{\pgfqpoint{3.706172in}{5.219610in}}%
\pgfpathlineto{\pgfqpoint{3.707992in}{5.252179in}}%
\pgfpathlineto{\pgfqpoint{3.708699in}{5.249354in}}%
\pgfpathlineto{\pgfqpoint{3.709407in}{5.229961in}}%
\pgfpathlineto{\pgfqpoint{3.711227in}{5.189123in}}%
\pgfpathlineto{\pgfqpoint{3.711328in}{5.189363in}}%
\pgfpathlineto{\pgfqpoint{3.712035in}{5.199351in}}%
\pgfpathlineto{\pgfqpoint{3.715675in}{5.281708in}}%
\pgfpathlineto{\pgfqpoint{3.716483in}{5.289490in}}%
\pgfpathlineto{\pgfqpoint{3.716888in}{5.285514in}}%
\pgfpathlineto{\pgfqpoint{3.718000in}{5.228037in}}%
\pgfpathlineto{\pgfqpoint{3.718707in}{5.211956in}}%
\pgfpathlineto{\pgfqpoint{3.719314in}{5.214014in}}%
\pgfpathlineto{\pgfqpoint{3.720123in}{5.224866in}}%
\pgfpathlineto{\pgfqpoint{3.720426in}{5.228968in}}%
\pgfpathlineto{\pgfqpoint{3.720830in}{5.222149in}}%
\pgfpathlineto{\pgfqpoint{3.721841in}{5.177735in}}%
\pgfpathlineto{\pgfqpoint{3.722347in}{5.197206in}}%
\pgfpathlineto{\pgfqpoint{3.723459in}{5.238113in}}%
\pgfpathlineto{\pgfqpoint{3.723964in}{5.235464in}}%
\pgfpathlineto{\pgfqpoint{3.724571in}{5.233632in}}%
\pgfpathlineto{\pgfqpoint{3.724975in}{5.235216in}}%
\pgfpathlineto{\pgfqpoint{3.725683in}{5.252839in}}%
\pgfpathlineto{\pgfqpoint{3.726391in}{5.271485in}}%
\pgfpathlineto{\pgfqpoint{3.726795in}{5.259944in}}%
\pgfpathlineto{\pgfqpoint{3.728412in}{5.196711in}}%
\pgfpathlineto{\pgfqpoint{3.728918in}{5.200169in}}%
\pgfpathlineto{\pgfqpoint{3.729929in}{5.237888in}}%
\pgfpathlineto{\pgfqpoint{3.730839in}{5.259029in}}%
\pgfpathlineto{\pgfqpoint{3.731344in}{5.253245in}}%
\pgfpathlineto{\pgfqpoint{3.732962in}{5.215144in}}%
\pgfpathlineto{\pgfqpoint{3.733568in}{5.227909in}}%
\pgfpathlineto{\pgfqpoint{3.734074in}{5.236136in}}%
\pgfpathlineto{\pgfqpoint{3.734680in}{5.229654in}}%
\pgfpathlineto{\pgfqpoint{3.735186in}{5.225443in}}%
\pgfpathlineto{\pgfqpoint{3.735994in}{5.226259in}}%
\pgfpathlineto{\pgfqpoint{3.736500in}{5.224712in}}%
\pgfpathlineto{\pgfqpoint{3.736904in}{5.226693in}}%
\pgfpathlineto{\pgfqpoint{3.737309in}{5.228200in}}%
\pgfpathlineto{\pgfqpoint{3.737713in}{5.225402in}}%
\pgfpathlineto{\pgfqpoint{3.738117in}{5.222628in}}%
\pgfpathlineto{\pgfqpoint{3.738522in}{5.226595in}}%
\pgfpathlineto{\pgfqpoint{3.739128in}{5.234539in}}%
\pgfpathlineto{\pgfqpoint{3.739634in}{5.228475in}}%
\pgfpathlineto{\pgfqpoint{3.740847in}{5.208326in}}%
\pgfpathlineto{\pgfqpoint{3.741352in}{5.212171in}}%
\pgfpathlineto{\pgfqpoint{3.742869in}{5.252328in}}%
\pgfpathlineto{\pgfqpoint{3.743779in}{5.236866in}}%
\pgfpathlineto{\pgfqpoint{3.744284in}{5.231919in}}%
\pgfpathlineto{\pgfqpoint{3.744790in}{5.237598in}}%
\pgfpathlineto{\pgfqpoint{3.746104in}{5.266141in}}%
\pgfpathlineto{\pgfqpoint{3.746710in}{5.259039in}}%
\pgfpathlineto{\pgfqpoint{3.749339in}{5.192780in}}%
\pgfpathlineto{\pgfqpoint{3.750046in}{5.200731in}}%
\pgfpathlineto{\pgfqpoint{3.752877in}{5.247941in}}%
\pgfpathlineto{\pgfqpoint{3.753180in}{5.244942in}}%
\pgfpathlineto{\pgfqpoint{3.753787in}{5.236319in}}%
\pgfpathlineto{\pgfqpoint{3.754191in}{5.246600in}}%
\pgfpathlineto{\pgfqpoint{3.755101in}{5.281444in}}%
\pgfpathlineto{\pgfqpoint{3.755607in}{5.267880in}}%
\pgfpathlineto{\pgfqpoint{3.757527in}{5.190038in}}%
\pgfpathlineto{\pgfqpoint{3.757932in}{5.197013in}}%
\pgfpathlineto{\pgfqpoint{3.758943in}{5.222788in}}%
\pgfpathlineto{\pgfqpoint{3.759549in}{5.216833in}}%
\pgfpathlineto{\pgfqpoint{3.759954in}{5.214155in}}%
\pgfpathlineto{\pgfqpoint{3.760560in}{5.217509in}}%
\pgfpathlineto{\pgfqpoint{3.761773in}{5.240665in}}%
\pgfpathlineto{\pgfqpoint{3.763290in}{5.278608in}}%
\pgfpathlineto{\pgfqpoint{3.763795in}{5.273425in}}%
\pgfpathlineto{\pgfqpoint{3.765109in}{5.216201in}}%
\pgfpathlineto{\pgfqpoint{3.766019in}{5.186307in}}%
\pgfpathlineto{\pgfqpoint{3.766525in}{5.197892in}}%
\pgfpathlineto{\pgfqpoint{3.767434in}{5.231766in}}%
\pgfpathlineto{\pgfqpoint{3.767940in}{5.217504in}}%
\pgfpathlineto{\pgfqpoint{3.768850in}{5.192301in}}%
\pgfpathlineto{\pgfqpoint{3.769355in}{5.199673in}}%
\pgfpathlineto{\pgfqpoint{3.771074in}{5.278745in}}%
\pgfpathlineto{\pgfqpoint{3.771781in}{5.247698in}}%
\pgfpathlineto{\pgfqpoint{3.772792in}{5.202410in}}%
\pgfpathlineto{\pgfqpoint{3.773298in}{5.216425in}}%
\pgfpathlineto{\pgfqpoint{3.774511in}{5.253679in}}%
\pgfpathlineto{\pgfqpoint{3.775118in}{5.251565in}}%
\pgfpathlineto{\pgfqpoint{3.775421in}{5.251393in}}%
\pgfpathlineto{\pgfqpoint{3.776027in}{5.251877in}}%
\pgfpathlineto{\pgfqpoint{3.776836in}{5.255295in}}%
\pgfpathlineto{\pgfqpoint{3.777038in}{5.253099in}}%
\pgfpathlineto{\pgfqpoint{3.777847in}{5.208652in}}%
\pgfpathlineto{\pgfqpoint{3.778656in}{5.178662in}}%
\pgfpathlineto{\pgfqpoint{3.779161in}{5.187445in}}%
\pgfpathlineto{\pgfqpoint{3.781183in}{5.243460in}}%
\pgfpathlineto{\pgfqpoint{3.781992in}{5.231953in}}%
\pgfpathlineto{\pgfqpoint{3.783104in}{5.216491in}}%
\pgfpathlineto{\pgfqpoint{3.783508in}{5.222023in}}%
\pgfpathlineto{\pgfqpoint{3.785833in}{5.256516in}}%
\pgfpathlineto{\pgfqpoint{3.785935in}{5.256644in}}%
\pgfpathlineto{\pgfqpoint{3.786137in}{5.255755in}}%
\pgfpathlineto{\pgfqpoint{3.786743in}{5.250841in}}%
\pgfpathlineto{\pgfqpoint{3.787249in}{5.255214in}}%
\pgfpathlineto{\pgfqpoint{3.787451in}{5.256727in}}%
\pgfpathlineto{\pgfqpoint{3.787855in}{5.250817in}}%
\pgfpathlineto{\pgfqpoint{3.789978in}{5.186673in}}%
\pgfpathlineto{\pgfqpoint{3.790484in}{5.194985in}}%
\pgfpathlineto{\pgfqpoint{3.791596in}{5.227024in}}%
\pgfpathlineto{\pgfqpoint{3.792303in}{5.223310in}}%
\pgfpathlineto{\pgfqpoint{3.793011in}{5.239993in}}%
\pgfpathlineto{\pgfqpoint{3.793618in}{5.251831in}}%
\pgfpathlineto{\pgfqpoint{3.794224in}{5.243604in}}%
\pgfpathlineto{\pgfqpoint{3.794730in}{5.236029in}}%
\pgfpathlineto{\pgfqpoint{3.795235in}{5.242585in}}%
\pgfpathlineto{\pgfqpoint{3.795943in}{5.254555in}}%
\pgfpathlineto{\pgfqpoint{3.796347in}{5.248154in}}%
\pgfpathlineto{\pgfqpoint{3.798066in}{5.190191in}}%
\pgfpathlineto{\pgfqpoint{3.798672in}{5.200564in}}%
\pgfpathlineto{\pgfqpoint{3.800290in}{5.272944in}}%
\pgfpathlineto{\pgfqpoint{3.800997in}{5.249219in}}%
\pgfpathlineto{\pgfqpoint{3.802110in}{5.205706in}}%
\pgfpathlineto{\pgfqpoint{3.802615in}{5.216737in}}%
\pgfpathlineto{\pgfqpoint{3.803929in}{5.255688in}}%
\pgfpathlineto{\pgfqpoint{3.804435in}{5.246565in}}%
\pgfpathlineto{\pgfqpoint{3.805648in}{5.213519in}}%
\pgfpathlineto{\pgfqpoint{3.806153in}{5.220714in}}%
\pgfpathlineto{\pgfqpoint{3.806861in}{5.230823in}}%
\pgfpathlineto{\pgfqpoint{3.807366in}{5.222988in}}%
\pgfpathlineto{\pgfqpoint{3.809793in}{5.191020in}}%
\pgfpathlineto{\pgfqpoint{3.810298in}{5.211385in}}%
\pgfpathlineto{\pgfqpoint{3.811309in}{5.265298in}}%
\pgfpathlineto{\pgfqpoint{3.811916in}{5.254200in}}%
\pgfpathlineto{\pgfqpoint{3.812118in}{5.252850in}}%
\pgfpathlineto{\pgfqpoint{3.812522in}{5.258188in}}%
\pgfpathlineto{\pgfqpoint{3.812927in}{5.263092in}}%
\pgfpathlineto{\pgfqpoint{3.813432in}{5.256037in}}%
\pgfpathlineto{\pgfqpoint{3.817274in}{5.180136in}}%
\pgfpathlineto{\pgfqpoint{3.817678in}{5.187095in}}%
\pgfpathlineto{\pgfqpoint{3.819093in}{5.231146in}}%
\pgfpathlineto{\pgfqpoint{3.819801in}{5.229045in}}%
\pgfpathlineto{\pgfqpoint{3.820306in}{5.230871in}}%
\pgfpathlineto{\pgfqpoint{3.821317in}{5.243569in}}%
\pgfpathlineto{\pgfqpoint{3.823541in}{5.265299in}}%
\pgfpathlineto{\pgfqpoint{3.823642in}{5.265375in}}%
\pgfpathlineto{\pgfqpoint{3.823744in}{5.264867in}}%
\pgfpathlineto{\pgfqpoint{3.824350in}{5.248436in}}%
\pgfpathlineto{\pgfqpoint{3.826372in}{5.204934in}}%
\pgfpathlineto{\pgfqpoint{3.827080in}{5.201097in}}%
\pgfpathlineto{\pgfqpoint{3.827484in}{5.203437in}}%
\pgfpathlineto{\pgfqpoint{3.830921in}{5.259114in}}%
\pgfpathlineto{\pgfqpoint{3.831629in}{5.244342in}}%
\pgfpathlineto{\pgfqpoint{3.832640in}{5.228347in}}%
\pgfpathlineto{\pgfqpoint{3.833145in}{5.231092in}}%
\pgfpathlineto{\pgfqpoint{3.834561in}{5.251540in}}%
\pgfpathlineto{\pgfqpoint{3.835066in}{5.242787in}}%
\pgfpathlineto{\pgfqpoint{3.837189in}{5.206467in}}%
\pgfpathlineto{\pgfqpoint{3.837290in}{5.206507in}}%
\pgfpathlineto{\pgfqpoint{3.837795in}{5.215549in}}%
\pgfpathlineto{\pgfqpoint{3.838806in}{5.241068in}}%
\pgfpathlineto{\pgfqpoint{3.839413in}{5.232526in}}%
\pgfpathlineto{\pgfqpoint{3.839918in}{5.227292in}}%
\pgfpathlineto{\pgfqpoint{3.840525in}{5.232370in}}%
\pgfpathlineto{\pgfqpoint{3.840727in}{5.233065in}}%
\pgfpathlineto{\pgfqpoint{3.841030in}{5.230312in}}%
\pgfpathlineto{\pgfqpoint{3.841738in}{5.219647in}}%
\pgfpathlineto{\pgfqpoint{3.842244in}{5.225527in}}%
\pgfpathlineto{\pgfqpoint{3.842850in}{5.232976in}}%
\pgfpathlineto{\pgfqpoint{3.843457in}{5.229363in}}%
\pgfpathlineto{\pgfqpoint{3.843558in}{5.229340in}}%
\pgfpathlineto{\pgfqpoint{3.843659in}{5.229855in}}%
\pgfpathlineto{\pgfqpoint{3.844569in}{5.251902in}}%
\pgfpathlineto{\pgfqpoint{3.844872in}{5.254308in}}%
\pgfpathlineto{\pgfqpoint{3.845276in}{5.247555in}}%
\pgfpathlineto{\pgfqpoint{3.846692in}{5.215839in}}%
\pgfpathlineto{\pgfqpoint{3.847298in}{5.219493in}}%
\pgfpathlineto{\pgfqpoint{3.849522in}{5.239406in}}%
\pgfpathlineto{\pgfqpoint{3.849927in}{5.235584in}}%
\pgfpathlineto{\pgfqpoint{3.851949in}{5.196119in}}%
\pgfpathlineto{\pgfqpoint{3.852454in}{5.205498in}}%
\pgfpathlineto{\pgfqpoint{3.854476in}{5.281772in}}%
\pgfpathlineto{\pgfqpoint{3.854981in}{5.267090in}}%
\pgfpathlineto{\pgfqpoint{3.856296in}{5.221908in}}%
\pgfpathlineto{\pgfqpoint{3.856801in}{5.226764in}}%
\pgfpathlineto{\pgfqpoint{3.857509in}{5.234626in}}%
\pgfpathlineto{\pgfqpoint{3.858014in}{5.230081in}}%
\pgfpathlineto{\pgfqpoint{3.859531in}{5.199777in}}%
\pgfpathlineto{\pgfqpoint{3.860238in}{5.206759in}}%
\pgfpathlineto{\pgfqpoint{3.861957in}{5.245056in}}%
\pgfpathlineto{\pgfqpoint{3.862664in}{5.263760in}}%
\pgfpathlineto{\pgfqpoint{3.863271in}{5.253484in}}%
\pgfpathlineto{\pgfqpoint{3.865091in}{5.207996in}}%
\pgfpathlineto{\pgfqpoint{3.865596in}{5.210943in}}%
\pgfpathlineto{\pgfqpoint{3.868730in}{5.243763in}}%
\pgfpathlineto{\pgfqpoint{3.868831in}{5.243576in}}%
\pgfpathlineto{\pgfqpoint{3.869640in}{5.239394in}}%
\pgfpathlineto{\pgfqpoint{3.870044in}{5.242831in}}%
\pgfpathlineto{\pgfqpoint{3.871055in}{5.261569in}}%
\pgfpathlineto{\pgfqpoint{3.871561in}{5.253338in}}%
\pgfpathlineto{\pgfqpoint{3.873987in}{5.174334in}}%
\pgfpathlineto{\pgfqpoint{3.874695in}{5.187093in}}%
\pgfpathlineto{\pgfqpoint{3.876514in}{5.265630in}}%
\pgfpathlineto{\pgfqpoint{3.877424in}{5.255296in}}%
\pgfpathlineto{\pgfqpoint{3.879446in}{5.228682in}}%
\pgfpathlineto{\pgfqpoint{3.879648in}{5.228742in}}%
\pgfpathlineto{\pgfqpoint{3.880053in}{5.227650in}}%
\pgfpathlineto{\pgfqpoint{3.880962in}{5.218905in}}%
\pgfpathlineto{\pgfqpoint{3.881468in}{5.224914in}}%
\pgfpathlineto{\pgfqpoint{3.882681in}{5.254488in}}%
\pgfpathlineto{\pgfqpoint{3.883288in}{5.248374in}}%
\pgfpathlineto{\pgfqpoint{3.884804in}{5.201619in}}%
\pgfpathlineto{\pgfqpoint{3.885613in}{5.215891in}}%
\pgfpathlineto{\pgfqpoint{3.886219in}{5.225857in}}%
\pgfpathlineto{\pgfqpoint{3.886725in}{5.219336in}}%
\pgfpathlineto{\pgfqpoint{3.887432in}{5.209200in}}%
\pgfpathlineto{\pgfqpoint{3.888039in}{5.214294in}}%
\pgfpathlineto{\pgfqpoint{3.889656in}{5.245765in}}%
\pgfpathlineto{\pgfqpoint{3.890971in}{5.282816in}}%
\pgfpathlineto{\pgfqpoint{3.891476in}{5.272938in}}%
\pgfpathlineto{\pgfqpoint{3.893801in}{5.205659in}}%
\pgfpathlineto{\pgfqpoint{3.894206in}{5.206912in}}%
\pgfpathlineto{\pgfqpoint{3.894711in}{5.208632in}}%
\pgfpathlineto{\pgfqpoint{3.895014in}{5.206619in}}%
\pgfpathlineto{\pgfqpoint{3.896126in}{5.188828in}}%
\pgfpathlineto{\pgfqpoint{3.896531in}{5.197060in}}%
\pgfpathlineto{\pgfqpoint{3.898755in}{5.264439in}}%
\pgfpathlineto{\pgfqpoint{3.899159in}{5.260678in}}%
\pgfpathlineto{\pgfqpoint{3.902192in}{5.215617in}}%
\pgfpathlineto{\pgfqpoint{3.902495in}{5.216823in}}%
\pgfpathlineto{\pgfqpoint{3.903911in}{5.235789in}}%
\pgfpathlineto{\pgfqpoint{3.904618in}{5.228006in}}%
\pgfpathlineto{\pgfqpoint{3.905023in}{5.225096in}}%
\pgfpathlineto{\pgfqpoint{3.905427in}{5.229374in}}%
\pgfpathlineto{\pgfqpoint{3.906741in}{5.251511in}}%
\pgfpathlineto{\pgfqpoint{3.907247in}{5.243318in}}%
\pgfpathlineto{\pgfqpoint{3.908662in}{5.198904in}}%
\pgfpathlineto{\pgfqpoint{3.909370in}{5.207100in}}%
\pgfpathlineto{\pgfqpoint{3.910886in}{5.239782in}}%
\pgfpathlineto{\pgfqpoint{3.911594in}{5.226940in}}%
\pgfpathlineto{\pgfqpoint{3.912301in}{5.217107in}}%
\pgfpathlineto{\pgfqpoint{3.913009in}{5.218442in}}%
\pgfpathlineto{\pgfqpoint{3.913514in}{5.225990in}}%
\pgfpathlineto{\pgfqpoint{3.914930in}{5.288244in}}%
\pgfpathlineto{\pgfqpoint{3.915637in}{5.261321in}}%
\pgfpathlineto{\pgfqpoint{3.917760in}{5.206874in}}%
\pgfpathlineto{\pgfqpoint{3.918974in}{5.197948in}}%
\pgfpathlineto{\pgfqpoint{3.919277in}{5.199579in}}%
\pgfpathlineto{\pgfqpoint{3.920692in}{5.235155in}}%
\pgfpathlineto{\pgfqpoint{3.921905in}{5.224731in}}%
\pgfpathlineto{\pgfqpoint{3.922310in}{5.223110in}}%
\pgfpathlineto{\pgfqpoint{3.922714in}{5.224998in}}%
\pgfpathlineto{\pgfqpoint{3.923624in}{5.236614in}}%
\pgfpathlineto{\pgfqpoint{3.924028in}{5.230760in}}%
\pgfpathlineto{\pgfqpoint{3.924635in}{5.218776in}}%
\pgfpathlineto{\pgfqpoint{3.925140in}{5.228868in}}%
\pgfpathlineto{\pgfqpoint{3.926454in}{5.277495in}}%
\pgfpathlineto{\pgfqpoint{3.927061in}{5.270541in}}%
\pgfpathlineto{\pgfqpoint{3.928375in}{5.219913in}}%
\pgfpathlineto{\pgfqpoint{3.929487in}{5.183658in}}%
\pgfpathlineto{\pgfqpoint{3.929993in}{5.191444in}}%
\pgfpathlineto{\pgfqpoint{3.931408in}{5.230632in}}%
\pgfpathlineto{\pgfqpoint{3.932015in}{5.222884in}}%
\pgfpathlineto{\pgfqpoint{3.932823in}{5.209395in}}%
\pgfpathlineto{\pgfqpoint{3.933228in}{5.216290in}}%
\pgfpathlineto{\pgfqpoint{3.934542in}{5.270104in}}%
\pgfpathlineto{\pgfqpoint{3.935250in}{5.254807in}}%
\pgfpathlineto{\pgfqpoint{3.937069in}{5.228802in}}%
\pgfpathlineto{\pgfqpoint{3.937979in}{5.220723in}}%
\pgfpathlineto{\pgfqpoint{3.938788in}{5.212192in}}%
\pgfpathlineto{\pgfqpoint{3.939192in}{5.216839in}}%
\pgfpathlineto{\pgfqpoint{3.940001in}{5.230686in}}%
\pgfpathlineto{\pgfqpoint{3.940709in}{5.225338in}}%
\pgfpathlineto{\pgfqpoint{3.940911in}{5.224934in}}%
\pgfpathlineto{\pgfqpoint{3.941214in}{5.227328in}}%
\pgfpathlineto{\pgfqpoint{3.943337in}{5.243080in}}%
\pgfpathlineto{\pgfqpoint{3.943438in}{5.243055in}}%
\pgfpathlineto{\pgfqpoint{3.943944in}{5.234198in}}%
\pgfpathlineto{\pgfqpoint{3.944955in}{5.212480in}}%
\pgfpathlineto{\pgfqpoint{3.945460in}{5.217578in}}%
\pgfpathlineto{\pgfqpoint{3.947684in}{5.254951in}}%
\pgfpathlineto{\pgfqpoint{3.948190in}{5.245106in}}%
\pgfpathlineto{\pgfqpoint{3.949403in}{5.216980in}}%
\pgfpathlineto{\pgfqpoint{3.950009in}{5.218420in}}%
\pgfpathlineto{\pgfqpoint{3.951020in}{5.222486in}}%
\pgfpathlineto{\pgfqpoint{3.951425in}{5.224149in}}%
\pgfpathlineto{\pgfqpoint{3.951829in}{5.221832in}}%
\pgfpathlineto{\pgfqpoint{3.952334in}{5.218015in}}%
\pgfpathlineto{\pgfqpoint{3.952739in}{5.222722in}}%
\pgfpathlineto{\pgfqpoint{3.953750in}{5.244000in}}%
\pgfpathlineto{\pgfqpoint{3.954356in}{5.237564in}}%
\pgfpathlineto{\pgfqpoint{3.954660in}{5.235708in}}%
\pgfpathlineto{\pgfqpoint{3.955165in}{5.238935in}}%
\pgfpathlineto{\pgfqpoint{3.955468in}{5.240339in}}%
\pgfpathlineto{\pgfqpoint{3.955873in}{5.237237in}}%
\pgfpathlineto{\pgfqpoint{3.956378in}{5.233268in}}%
\pgfpathlineto{\pgfqpoint{3.956782in}{5.238401in}}%
\pgfpathlineto{\pgfqpoint{3.957389in}{5.247133in}}%
\pgfpathlineto{\pgfqpoint{3.957793in}{5.239014in}}%
\pgfpathlineto{\pgfqpoint{3.958905in}{5.202178in}}%
\pgfpathlineto{\pgfqpoint{3.959613in}{5.210449in}}%
\pgfpathlineto{\pgfqpoint{3.963657in}{5.244812in}}%
\pgfpathlineto{\pgfqpoint{3.963758in}{5.244659in}}%
\pgfpathlineto{\pgfqpoint{3.964567in}{5.238453in}}%
\pgfpathlineto{\pgfqpoint{3.966892in}{5.221015in}}%
\pgfpathlineto{\pgfqpoint{3.967195in}{5.222038in}}%
\pgfpathlineto{\pgfqpoint{3.968206in}{5.228548in}}%
\pgfpathlineto{\pgfqpoint{3.968813in}{5.226288in}}%
\pgfpathlineto{\pgfqpoint{3.969824in}{5.223978in}}%
\pgfpathlineto{\pgfqpoint{3.970228in}{5.225000in}}%
\pgfpathlineto{\pgfqpoint{3.971744in}{5.237032in}}%
\pgfpathlineto{\pgfqpoint{3.972351in}{5.230336in}}%
\pgfpathlineto{\pgfqpoint{3.973362in}{5.213042in}}%
\pgfpathlineto{\pgfqpoint{3.973968in}{5.218340in}}%
\pgfpathlineto{\pgfqpoint{3.977001in}{5.264884in}}%
\pgfpathlineto{\pgfqpoint{3.977406in}{5.259882in}}%
\pgfpathlineto{\pgfqpoint{3.979326in}{5.217291in}}%
\pgfpathlineto{\pgfqpoint{3.980034in}{5.225784in}}%
\pgfpathlineto{\pgfqpoint{3.980337in}{5.227762in}}%
\pgfpathlineto{\pgfqpoint{3.980641in}{5.222970in}}%
\pgfpathlineto{\pgfqpoint{3.981753in}{5.184227in}}%
\pgfpathlineto{\pgfqpoint{3.982359in}{5.199954in}}%
\pgfpathlineto{\pgfqpoint{3.983673in}{5.236726in}}%
\pgfpathlineto{\pgfqpoint{3.984179in}{5.236518in}}%
\pgfpathlineto{\pgfqpoint{3.986605in}{5.251431in}}%
\pgfpathlineto{\pgfqpoint{3.987111in}{5.247356in}}%
\pgfpathlineto{\pgfqpoint{3.990143in}{5.213965in}}%
\pgfpathlineto{\pgfqpoint{3.990244in}{5.213778in}}%
\pgfpathlineto{\pgfqpoint{3.990447in}{5.214866in}}%
\pgfpathlineto{\pgfqpoint{3.991660in}{5.234852in}}%
\pgfpathlineto{\pgfqpoint{3.992772in}{5.230612in}}%
\pgfpathlineto{\pgfqpoint{3.993176in}{5.232342in}}%
\pgfpathlineto{\pgfqpoint{3.993682in}{5.229898in}}%
\pgfpathlineto{\pgfqpoint{3.994288in}{5.226883in}}%
\pgfpathlineto{\pgfqpoint{3.994794in}{5.228770in}}%
\pgfpathlineto{\pgfqpoint{3.995400in}{5.231087in}}%
\pgfpathlineto{\pgfqpoint{3.995805in}{5.228627in}}%
\pgfpathlineto{\pgfqpoint{3.996815in}{5.216014in}}%
\pgfpathlineto{\pgfqpoint{3.997523in}{5.220792in}}%
\pgfpathlineto{\pgfqpoint{3.999141in}{5.245028in}}%
\pgfpathlineto{\pgfqpoint{3.999646in}{5.249654in}}%
\pgfpathlineto{\pgfqpoint{4.000152in}{5.244840in}}%
\pgfpathlineto{\pgfqpoint{4.001769in}{5.228618in}}%
\pgfpathlineto{\pgfqpoint{4.002173in}{5.229647in}}%
\pgfpathlineto{\pgfqpoint{4.003387in}{5.237559in}}%
\pgfpathlineto{\pgfqpoint{4.003892in}{5.233316in}}%
\pgfpathlineto{\pgfqpoint{4.005004in}{5.215417in}}%
\pgfpathlineto{\pgfqpoint{4.005611in}{5.221458in}}%
\pgfpathlineto{\pgfqpoint{4.006419in}{5.232583in}}%
\pgfpathlineto{\pgfqpoint{4.006925in}{5.226105in}}%
\pgfpathlineto{\pgfqpoint{4.007936in}{5.208005in}}%
\pgfpathlineto{\pgfqpoint{4.008441in}{5.213416in}}%
\pgfpathlineto{\pgfqpoint{4.010160in}{5.252291in}}%
\pgfpathlineto{\pgfqpoint{4.010867in}{5.241629in}}%
\pgfpathlineto{\pgfqpoint{4.011373in}{5.235084in}}%
\pgfpathlineto{\pgfqpoint{4.012081in}{5.238812in}}%
\pgfpathlineto{\pgfqpoint{4.012384in}{5.237283in}}%
\pgfpathlineto{\pgfqpoint{4.014305in}{5.219475in}}%
\pgfpathlineto{\pgfqpoint{4.014709in}{5.221640in}}%
\pgfpathlineto{\pgfqpoint{4.016529in}{5.243186in}}%
\pgfpathlineto{\pgfqpoint{4.017135in}{5.240479in}}%
\pgfpathlineto{\pgfqpoint{4.020168in}{5.204580in}}%
\pgfpathlineto{\pgfqpoint{4.020674in}{5.212121in}}%
\pgfpathlineto{\pgfqpoint{4.023302in}{5.260516in}}%
\pgfpathlineto{\pgfqpoint{4.023706in}{5.255745in}}%
\pgfpathlineto{\pgfqpoint{4.026133in}{5.225181in}}%
\pgfpathlineto{\pgfqpoint{4.026537in}{5.226949in}}%
\pgfpathlineto{\pgfqpoint{4.027144in}{5.230872in}}%
\pgfpathlineto{\pgfqpoint{4.027548in}{5.226376in}}%
\pgfpathlineto{\pgfqpoint{4.028660in}{5.203724in}}%
\pgfpathlineto{\pgfqpoint{4.029165in}{5.211761in}}%
\pgfpathlineto{\pgfqpoint{4.031086in}{5.245528in}}%
\pgfpathlineto{\pgfqpoint{4.031389in}{5.244410in}}%
\pgfpathlineto{\pgfqpoint{4.032400in}{5.224029in}}%
\pgfpathlineto{\pgfqpoint{4.032906in}{5.218200in}}%
\pgfpathlineto{\pgfqpoint{4.033411in}{5.224653in}}%
\pgfpathlineto{\pgfqpoint{4.034726in}{5.246332in}}%
\pgfpathlineto{\pgfqpoint{4.035231in}{5.243423in}}%
\pgfpathlineto{\pgfqpoint{4.036646in}{5.217972in}}%
\pgfpathlineto{\pgfqpoint{4.037455in}{5.225939in}}%
\pgfpathlineto{\pgfqpoint{4.039881in}{5.242092in}}%
\pgfpathlineto{\pgfqpoint{4.038264in}{5.225354in}}%
\pgfpathlineto{\pgfqpoint{4.040387in}{5.237445in}}%
\pgfpathlineto{\pgfqpoint{4.041903in}{5.215214in}}%
\pgfpathlineto{\pgfqpoint{4.042510in}{5.221563in}}%
\pgfpathlineto{\pgfqpoint{4.044228in}{5.234016in}}%
\pgfpathlineto{\pgfqpoint{4.044633in}{5.235527in}}%
\pgfpathlineto{\pgfqpoint{4.045037in}{5.232849in}}%
\pgfpathlineto{\pgfqpoint{4.045846in}{5.222984in}}%
\pgfpathlineto{\pgfqpoint{4.046452in}{5.227370in}}%
\pgfpathlineto{\pgfqpoint{4.046857in}{5.229956in}}%
\pgfpathlineto{\pgfqpoint{4.047463in}{5.227069in}}%
\pgfpathlineto{\pgfqpoint{4.047868in}{5.225676in}}%
\pgfpathlineto{\pgfqpoint{4.048272in}{5.227861in}}%
\pgfpathlineto{\pgfqpoint{4.050799in}{5.246589in}}%
\pgfpathlineto{\pgfqpoint{4.051002in}{5.245333in}}%
\pgfpathlineto{\pgfqpoint{4.052619in}{5.222335in}}%
\pgfpathlineto{\pgfqpoint{4.053529in}{5.226985in}}%
\pgfpathlineto{\pgfqpoint{4.053933in}{5.227875in}}%
\pgfpathlineto{\pgfqpoint{4.054338in}{5.225923in}}%
\pgfpathlineto{\pgfqpoint{4.055652in}{5.212353in}}%
\pgfpathlineto{\pgfqpoint{4.056258in}{5.218305in}}%
\pgfpathlineto{\pgfqpoint{4.057977in}{5.252580in}}%
\pgfpathlineto{\pgfqpoint{4.058584in}{5.246828in}}%
\pgfpathlineto{\pgfqpoint{4.059898in}{5.211784in}}%
\pgfpathlineto{\pgfqpoint{4.060605in}{5.223584in}}%
\pgfpathlineto{\pgfqpoint{4.062324in}{5.243179in}}%
\pgfpathlineto{\pgfqpoint{4.062526in}{5.241730in}}%
\pgfpathlineto{\pgfqpoint{4.063739in}{5.214614in}}%
\pgfpathlineto{\pgfqpoint{4.064447in}{5.226591in}}%
\pgfpathlineto{\pgfqpoint{4.065155in}{5.237146in}}%
\pgfpathlineto{\pgfqpoint{4.065761in}{5.232667in}}%
\pgfpathlineto{\pgfqpoint{4.067783in}{5.215625in}}%
\pgfpathlineto{\pgfqpoint{4.068086in}{5.218224in}}%
\pgfpathlineto{\pgfqpoint{4.069401in}{5.242668in}}%
\pgfpathlineto{\pgfqpoint{4.070108in}{5.236865in}}%
\pgfpathlineto{\pgfqpoint{4.071827in}{5.227349in}}%
\pgfpathlineto{\pgfqpoint{4.072130in}{5.228632in}}%
\pgfpathlineto{\pgfqpoint{4.073242in}{5.234476in}}%
\pgfpathlineto{\pgfqpoint{4.073849in}{5.233495in}}%
\pgfpathlineto{\pgfqpoint{4.075466in}{5.230113in}}%
\pgfpathlineto{\pgfqpoint{4.076376in}{5.224998in}}%
\pgfpathlineto{\pgfqpoint{4.076881in}{5.226438in}}%
\pgfpathlineto{\pgfqpoint{4.077387in}{5.227393in}}%
\pgfpathlineto{\pgfqpoint{4.077791in}{5.225909in}}%
\pgfpathlineto{\pgfqpoint{4.078903in}{5.218113in}}%
\pgfpathlineto{\pgfqpoint{4.079712in}{5.219428in}}%
\pgfpathlineto{\pgfqpoint{4.080218in}{5.221465in}}%
\pgfpathlineto{\pgfqpoint{4.081228in}{5.243935in}}%
\pgfpathlineto{\pgfqpoint{4.081835in}{5.253971in}}%
\pgfpathlineto{\pgfqpoint{4.082341in}{5.247844in}}%
\pgfpathlineto{\pgfqpoint{4.084059in}{5.234753in}}%
\pgfpathlineto{\pgfqpoint{4.085980in}{5.210184in}}%
\pgfpathlineto{\pgfqpoint{4.086890in}{5.215667in}}%
\pgfpathlineto{\pgfqpoint{4.090024in}{5.253279in}}%
\pgfpathlineto{\pgfqpoint{4.090327in}{5.251136in}}%
\pgfpathlineto{\pgfqpoint{4.091439in}{5.235229in}}%
\pgfpathlineto{\pgfqpoint{4.092045in}{5.239182in}}%
\pgfpathlineto{\pgfqpoint{4.092349in}{5.240290in}}%
\pgfpathlineto{\pgfqpoint{4.092652in}{5.237804in}}%
\pgfpathlineto{\pgfqpoint{4.094067in}{5.202380in}}%
\pgfpathlineto{\pgfqpoint{4.094876in}{5.214021in}}%
\pgfpathlineto{\pgfqpoint{4.096494in}{5.223629in}}%
\pgfpathlineto{\pgfqpoint{4.096898in}{5.222830in}}%
\pgfpathlineto{\pgfqpoint{4.097606in}{5.218852in}}%
\pgfpathlineto{\pgfqpoint{4.098010in}{5.221293in}}%
\pgfpathlineto{\pgfqpoint{4.099830in}{5.267629in}}%
\pgfpathlineto{\pgfqpoint{4.100841in}{5.251180in}}%
\pgfpathlineto{\pgfqpoint{4.102357in}{5.209258in}}%
\pgfpathlineto{\pgfqpoint{4.103065in}{5.211285in}}%
\pgfpathlineto{\pgfqpoint{4.103873in}{5.212323in}}%
\pgfpathlineto{\pgfqpoint{4.107311in}{5.239492in}}%
\pgfpathlineto{\pgfqpoint{4.107513in}{5.239265in}}%
\pgfpathlineto{\pgfqpoint{4.108220in}{5.232835in}}%
\pgfpathlineto{\pgfqpoint{4.108928in}{5.226362in}}%
\pgfpathlineto{\pgfqpoint{4.109332in}{5.229869in}}%
\pgfpathlineto{\pgfqpoint{4.110141in}{5.239322in}}%
\pgfpathlineto{\pgfqpoint{4.110546in}{5.234588in}}%
\pgfpathlineto{\pgfqpoint{4.111354in}{5.222455in}}%
\pgfpathlineto{\pgfqpoint{4.111961in}{5.227029in}}%
\pgfpathlineto{\pgfqpoint{4.112567in}{5.230342in}}%
\pgfpathlineto{\pgfqpoint{4.113174in}{5.228320in}}%
\pgfpathlineto{\pgfqpoint{4.114690in}{5.220349in}}%
\pgfpathlineto{\pgfqpoint{4.115196in}{5.223639in}}%
\pgfpathlineto{\pgfqpoint{4.117824in}{5.246078in}}%
\pgfpathlineto{\pgfqpoint{4.118128in}{5.243957in}}%
\pgfpathlineto{\pgfqpoint{4.119543in}{5.210661in}}%
\pgfpathlineto{\pgfqpoint{4.120453in}{5.221287in}}%
\pgfpathlineto{\pgfqpoint{4.122272in}{5.261144in}}%
\pgfpathlineto{\pgfqpoint{4.122980in}{5.244370in}}%
\pgfpathlineto{\pgfqpoint{4.124496in}{5.213059in}}%
\pgfpathlineto{\pgfqpoint{4.124800in}{5.214050in}}%
\pgfpathlineto{\pgfqpoint{4.126013in}{5.225568in}}%
\pgfpathlineto{\pgfqpoint{4.126619in}{5.220298in}}%
\pgfpathlineto{\pgfqpoint{4.127327in}{5.212091in}}%
\pgfpathlineto{\pgfqpoint{4.127833in}{5.217028in}}%
\pgfpathlineto{\pgfqpoint{4.129652in}{5.252469in}}%
\pgfpathlineto{\pgfqpoint{4.130259in}{5.244499in}}%
\pgfpathlineto{\pgfqpoint{4.131472in}{5.229575in}}%
\pgfpathlineto{\pgfqpoint{4.131876in}{5.230606in}}%
\pgfpathlineto{\pgfqpoint{4.132685in}{5.233815in}}%
\pgfpathlineto{\pgfqpoint{4.133089in}{5.231513in}}%
\pgfpathlineto{\pgfqpoint{4.134100in}{5.220144in}}%
\pgfpathlineto{\pgfqpoint{4.134606in}{5.224910in}}%
\pgfpathlineto{\pgfqpoint{4.135819in}{5.236958in}}%
\pgfpathlineto{\pgfqpoint{4.136223in}{5.234873in}}%
\pgfpathlineto{\pgfqpoint{4.137538in}{5.220524in}}%
\pgfpathlineto{\pgfqpoint{4.138346in}{5.222375in}}%
\pgfpathlineto{\pgfqpoint{4.138953in}{5.225413in}}%
\pgfpathlineto{\pgfqpoint{4.140166in}{5.241163in}}%
\pgfpathlineto{\pgfqpoint{4.140874in}{5.236862in}}%
\pgfpathlineto{\pgfqpoint{4.143199in}{5.228935in}}%
\pgfpathlineto{\pgfqpoint{4.143805in}{5.229460in}}%
\pgfpathlineto{\pgfqpoint{4.144311in}{5.226399in}}%
\pgfpathlineto{\pgfqpoint{4.145423in}{5.210417in}}%
\pgfpathlineto{\pgfqpoint{4.146029in}{5.215427in}}%
\pgfpathlineto{\pgfqpoint{4.148253in}{5.237813in}}%
\pgfpathlineto{\pgfqpoint{4.149062in}{5.235882in}}%
\pgfpathlineto{\pgfqpoint{4.149467in}{5.237226in}}%
\pgfpathlineto{\pgfqpoint{4.150680in}{5.246293in}}%
\pgfpathlineto{\pgfqpoint{4.151185in}{5.242470in}}%
\pgfpathlineto{\pgfqpoint{4.152095in}{5.235135in}}%
\pgfpathlineto{\pgfqpoint{4.152702in}{5.236911in}}%
\pgfpathlineto{\pgfqpoint{4.152803in}{5.236995in}}%
\pgfpathlineto{\pgfqpoint{4.153005in}{5.236476in}}%
\pgfpathlineto{\pgfqpoint{4.153814in}{5.223678in}}%
\pgfpathlineto{\pgfqpoint{4.155229in}{5.208770in}}%
\pgfpathlineto{\pgfqpoint{4.155532in}{5.209290in}}%
\pgfpathlineto{\pgfqpoint{4.156240in}{5.217796in}}%
\pgfpathlineto{\pgfqpoint{4.158969in}{5.248509in}}%
\pgfpathlineto{\pgfqpoint{4.159273in}{5.250388in}}%
\pgfpathlineto{\pgfqpoint{4.159778in}{5.245590in}}%
\pgfpathlineto{\pgfqpoint{4.162508in}{5.211418in}}%
\pgfpathlineto{\pgfqpoint{4.162912in}{5.213804in}}%
\pgfpathlineto{\pgfqpoint{4.165540in}{5.241827in}}%
\pgfpathlineto{\pgfqpoint{4.166147in}{5.232993in}}%
\pgfpathlineto{\pgfqpoint{4.166754in}{5.226216in}}%
\pgfpathlineto{\pgfqpoint{4.167259in}{5.232312in}}%
\pgfpathlineto{\pgfqpoint{4.168169in}{5.243125in}}%
\pgfpathlineto{\pgfqpoint{4.168674in}{5.240191in}}%
\pgfpathlineto{\pgfqpoint{4.171404in}{5.219828in}}%
\pgfpathlineto{\pgfqpoint{4.172111in}{5.223651in}}%
\pgfpathlineto{\pgfqpoint{4.173628in}{5.235733in}}%
\pgfpathlineto{\pgfqpoint{4.174032in}{5.232504in}}%
\pgfpathlineto{\pgfqpoint{4.175144in}{5.218182in}}%
\pgfpathlineto{\pgfqpoint{4.175751in}{5.222832in}}%
\pgfpathlineto{\pgfqpoint{4.177469in}{5.241607in}}%
\pgfpathlineto{\pgfqpoint{4.177975in}{5.237800in}}%
\pgfpathlineto{\pgfqpoint{4.178885in}{5.228714in}}%
\pgfpathlineto{\pgfqpoint{4.179491in}{5.231885in}}%
\pgfpathlineto{\pgfqpoint{4.179795in}{5.232692in}}%
\pgfpathlineto{\pgfqpoint{4.180199in}{5.230235in}}%
\pgfpathlineto{\pgfqpoint{4.182423in}{5.216413in}}%
\pgfpathlineto{\pgfqpoint{4.182524in}{5.216679in}}%
\pgfpathlineto{\pgfqpoint{4.183232in}{5.227652in}}%
\pgfpathlineto{\pgfqpoint{4.184849in}{5.245725in}}%
\pgfpathlineto{\pgfqpoint{4.185051in}{5.245547in}}%
\pgfpathlineto{\pgfqpoint{4.185658in}{5.240687in}}%
\pgfpathlineto{\pgfqpoint{4.186871in}{5.223469in}}%
\pgfpathlineto{\pgfqpoint{4.187579in}{5.226953in}}%
\pgfpathlineto{\pgfqpoint{4.187781in}{5.227108in}}%
\pgfpathlineto{\pgfqpoint{4.188084in}{5.225613in}}%
\pgfpathlineto{\pgfqpoint{4.189095in}{5.218364in}}%
\pgfpathlineto{\pgfqpoint{4.189702in}{5.220984in}}%
\pgfpathlineto{\pgfqpoint{4.190713in}{5.237283in}}%
\pgfpathlineto{\pgfqpoint{4.191420in}{5.250810in}}%
\pgfpathlineto{\pgfqpoint{4.191926in}{5.241192in}}%
\pgfpathlineto{\pgfqpoint{4.192836in}{5.218336in}}%
\pgfpathlineto{\pgfqpoint{4.193341in}{5.226466in}}%
\pgfpathlineto{\pgfqpoint{4.193948in}{5.233827in}}%
\pgfpathlineto{\pgfqpoint{4.194554in}{5.230925in}}%
\pgfpathlineto{\pgfqpoint{4.196374in}{5.222405in}}%
\pgfpathlineto{\pgfqpoint{4.196778in}{5.224402in}}%
\pgfpathlineto{\pgfqpoint{4.197991in}{5.242327in}}%
\pgfpathlineto{\pgfqpoint{4.198598in}{5.234671in}}%
\pgfpathlineto{\pgfqpoint{4.199609in}{5.223424in}}%
\pgfpathlineto{\pgfqpoint{4.200114in}{5.224838in}}%
\pgfpathlineto{\pgfqpoint{4.201631in}{5.236768in}}%
\pgfpathlineto{\pgfqpoint{4.202237in}{5.231420in}}%
\pgfpathlineto{\pgfqpoint{4.203248in}{5.220466in}}%
\pgfpathlineto{\pgfqpoint{4.203855in}{5.222627in}}%
\pgfpathlineto{\pgfqpoint{4.205573in}{5.234141in}}%
\pgfpathlineto{\pgfqpoint{4.206787in}{5.245114in}}%
\pgfpathlineto{\pgfqpoint{4.207292in}{5.242634in}}%
\pgfpathlineto{\pgfqpoint{4.209516in}{5.215657in}}%
\pgfpathlineto{\pgfqpoint{4.210527in}{5.222205in}}%
\pgfpathlineto{\pgfqpoint{4.211032in}{5.219042in}}%
\pgfpathlineto{\pgfqpoint{4.211437in}{5.216616in}}%
\pgfpathlineto{\pgfqpoint{4.211841in}{5.219449in}}%
\pgfpathlineto{\pgfqpoint{4.214470in}{5.244037in}}%
\pgfpathlineto{\pgfqpoint{4.214874in}{5.245794in}}%
\pgfpathlineto{\pgfqpoint{4.215278in}{5.243524in}}%
\pgfpathlineto{\pgfqpoint{4.217705in}{5.217436in}}%
\pgfpathlineto{\pgfqpoint{4.218210in}{5.220812in}}%
\pgfpathlineto{\pgfqpoint{4.219423in}{5.235505in}}%
\pgfpathlineto{\pgfqpoint{4.219929in}{5.231524in}}%
\pgfpathlineto{\pgfqpoint{4.220940in}{5.219896in}}%
\pgfpathlineto{\pgfqpoint{4.221445in}{5.222846in}}%
\pgfpathlineto{\pgfqpoint{4.222961in}{5.245311in}}%
\pgfpathlineto{\pgfqpoint{4.223669in}{5.236774in}}%
\pgfpathlineto{\pgfqpoint{4.225691in}{5.213206in}}%
\pgfpathlineto{\pgfqpoint{4.225893in}{5.213661in}}%
\pgfpathlineto{\pgfqpoint{4.226500in}{5.224363in}}%
\pgfpathlineto{\pgfqpoint{4.227713in}{5.261695in}}%
\pgfpathlineto{\pgfqpoint{4.228319in}{5.251457in}}%
\pgfpathlineto{\pgfqpoint{4.230240in}{5.210070in}}%
\pgfpathlineto{\pgfqpoint{4.230543in}{5.211195in}}%
\pgfpathlineto{\pgfqpoint{4.231858in}{5.227929in}}%
\pgfpathlineto{\pgfqpoint{4.232565in}{5.220157in}}%
\pgfpathlineto{\pgfqpoint{4.233071in}{5.215485in}}%
\pgfpathlineto{\pgfqpoint{4.233576in}{5.221710in}}%
\pgfpathlineto{\pgfqpoint{4.234992in}{5.250098in}}%
\pgfpathlineto{\pgfqpoint{4.235598in}{5.245310in}}%
\pgfpathlineto{\pgfqpoint{4.236811in}{5.233514in}}%
\pgfpathlineto{\pgfqpoint{4.237216in}{5.236606in}}%
\pgfpathlineto{\pgfqpoint{4.238126in}{5.247175in}}%
\pgfpathlineto{\pgfqpoint{4.238631in}{5.241806in}}%
\pgfpathlineto{\pgfqpoint{4.241158in}{5.192218in}}%
\pgfpathlineto{\pgfqpoint{4.241765in}{5.202004in}}%
\pgfpathlineto{\pgfqpoint{4.243382in}{5.246920in}}%
\pgfpathlineto{\pgfqpoint{4.244090in}{5.242204in}}%
\pgfpathlineto{\pgfqpoint{4.245809in}{5.229453in}}%
\pgfpathlineto{\pgfqpoint{4.246314in}{5.231864in}}%
\pgfpathlineto{\pgfqpoint{4.248437in}{5.242456in}}%
\pgfpathlineto{\pgfqpoint{4.248841in}{5.240645in}}%
\pgfpathlineto{\pgfqpoint{4.252279in}{5.215453in}}%
\pgfpathlineto{\pgfqpoint{4.252481in}{5.215256in}}%
\pgfpathlineto{\pgfqpoint{4.252784in}{5.216257in}}%
\pgfpathlineto{\pgfqpoint{4.254907in}{5.233114in}}%
\pgfpathlineto{\pgfqpoint{4.255817in}{5.247062in}}%
\pgfpathlineto{\pgfqpoint{4.256423in}{5.243765in}}%
\pgfpathlineto{\pgfqpoint{4.259658in}{5.224496in}}%
\pgfpathlineto{\pgfqpoint{4.260669in}{5.221945in}}%
\pgfpathlineto{\pgfqpoint{4.260973in}{5.223107in}}%
\pgfpathlineto{\pgfqpoint{4.262590in}{5.243343in}}%
\pgfpathlineto{\pgfqpoint{4.263399in}{5.236090in}}%
\pgfpathlineto{\pgfqpoint{4.265320in}{5.221600in}}%
\pgfpathlineto{\pgfqpoint{4.266229in}{5.218610in}}%
\pgfpathlineto{\pgfqpoint{4.266836in}{5.220570in}}%
\pgfpathlineto{\pgfqpoint{4.267948in}{5.233771in}}%
\pgfpathlineto{\pgfqpoint{4.269161in}{5.248281in}}%
\pgfpathlineto{\pgfqpoint{4.269566in}{5.244608in}}%
\pgfpathlineto{\pgfqpoint{4.270981in}{5.214555in}}%
\pgfpathlineto{\pgfqpoint{4.271689in}{5.223394in}}%
\pgfpathlineto{\pgfqpoint{4.272497in}{5.233437in}}%
\pgfpathlineto{\pgfqpoint{4.273003in}{5.229887in}}%
\pgfpathlineto{\pgfqpoint{4.273811in}{5.223433in}}%
\pgfpathlineto{\pgfqpoint{4.274317in}{5.225683in}}%
\pgfpathlineto{\pgfqpoint{4.276440in}{5.240497in}}%
\pgfpathlineto{\pgfqpoint{4.276945in}{5.238089in}}%
\pgfpathlineto{\pgfqpoint{4.277855in}{5.230894in}}%
\pgfpathlineto{\pgfqpoint{4.278361in}{5.233768in}}%
\pgfpathlineto{\pgfqpoint{4.278664in}{5.235323in}}%
\pgfpathlineto{\pgfqpoint{4.279169in}{5.231911in}}%
\pgfpathlineto{\pgfqpoint{4.280989in}{5.221471in}}%
\pgfpathlineto{\pgfqpoint{4.281798in}{5.216005in}}%
\pgfpathlineto{\pgfqpoint{4.282303in}{5.219268in}}%
\pgfpathlineto{\pgfqpoint{4.283921in}{5.238616in}}%
\pgfpathlineto{\pgfqpoint{4.284527in}{5.236699in}}%
\pgfpathlineto{\pgfqpoint{4.285943in}{5.229600in}}%
\pgfpathlineto{\pgfqpoint{4.286448in}{5.231069in}}%
\pgfpathlineto{\pgfqpoint{4.287358in}{5.235825in}}%
\pgfpathlineto{\pgfqpoint{4.287863in}{5.232974in}}%
\pgfpathlineto{\pgfqpoint{4.288975in}{5.222639in}}%
\pgfpathlineto{\pgfqpoint{4.289481in}{5.225238in}}%
\pgfpathlineto{\pgfqpoint{4.290492in}{5.232925in}}%
\pgfpathlineto{\pgfqpoint{4.290997in}{5.230573in}}%
\pgfpathlineto{\pgfqpoint{4.291806in}{5.226859in}}%
\pgfpathlineto{\pgfqpoint{4.292413in}{5.228256in}}%
\pgfpathlineto{\pgfqpoint{4.293626in}{5.233017in}}%
\pgfpathlineto{\pgfqpoint{4.294435in}{5.238603in}}%
\pgfpathlineto{\pgfqpoint{4.294940in}{5.235436in}}%
\pgfpathlineto{\pgfqpoint{4.295749in}{5.230355in}}%
\pgfpathlineto{\pgfqpoint{4.296456in}{5.230871in}}%
\pgfpathlineto{\pgfqpoint{4.297872in}{5.223073in}}%
\pgfpathlineto{\pgfqpoint{4.298478in}{5.225935in}}%
\pgfpathlineto{\pgfqpoint{4.299186in}{5.229238in}}%
\pgfpathlineto{\pgfqpoint{4.299691in}{5.226672in}}%
\pgfpathlineto{\pgfqpoint{4.300702in}{5.221034in}}%
\pgfpathlineto{\pgfqpoint{4.301107in}{5.222693in}}%
\pgfpathlineto{\pgfqpoint{4.302623in}{5.240439in}}%
\pgfpathlineto{\pgfqpoint{4.303432in}{5.233692in}}%
\pgfpathlineto{\pgfqpoint{4.304443in}{5.226919in}}%
\pgfpathlineto{\pgfqpoint{4.304948in}{5.228843in}}%
\pgfpathlineto{\pgfqpoint{4.306262in}{5.235972in}}%
\pgfpathlineto{\pgfqpoint{4.306869in}{5.235694in}}%
\pgfpathlineto{\pgfqpoint{4.307678in}{5.232434in}}%
\pgfpathlineto{\pgfqpoint{4.307880in}{5.231954in}}%
\pgfpathlineto{\pgfqpoint{4.308183in}{5.233329in}}%
\pgfpathlineto{\pgfqpoint{4.309093in}{5.243819in}}%
\pgfpathlineto{\pgfqpoint{4.309497in}{5.238110in}}%
\pgfpathlineto{\pgfqpoint{4.311014in}{5.205050in}}%
\pgfpathlineto{\pgfqpoint{4.311620in}{5.210696in}}%
\pgfpathlineto{\pgfqpoint{4.313844in}{5.237695in}}%
\pgfpathlineto{\pgfqpoint{4.314148in}{5.237055in}}%
\pgfpathlineto{\pgfqpoint{4.315058in}{5.232309in}}%
\pgfpathlineto{\pgfqpoint{4.315563in}{5.235128in}}%
\pgfpathlineto{\pgfqpoint{4.316170in}{5.237648in}}%
\pgfpathlineto{\pgfqpoint{4.316776in}{5.235816in}}%
\pgfpathlineto{\pgfqpoint{4.318090in}{5.229055in}}%
\pgfpathlineto{\pgfqpoint{4.318596in}{5.226628in}}%
\pgfpathlineto{\pgfqpoint{4.319101in}{5.229564in}}%
\pgfpathlineto{\pgfqpoint{4.319506in}{5.231138in}}%
\pgfpathlineto{\pgfqpoint{4.320112in}{5.228859in}}%
\pgfpathlineto{\pgfqpoint{4.320314in}{5.228554in}}%
\pgfpathlineto{\pgfqpoint{4.320719in}{5.230265in}}%
\pgfpathlineto{\pgfqpoint{4.321325in}{5.233607in}}%
\pgfpathlineto{\pgfqpoint{4.321831in}{5.231447in}}%
\pgfpathlineto{\pgfqpoint{4.322539in}{5.228153in}}%
\pgfpathlineto{\pgfqpoint{4.323145in}{5.229944in}}%
\pgfpathlineto{\pgfqpoint{4.323448in}{5.230383in}}%
\pgfpathlineto{\pgfqpoint{4.323752in}{5.229281in}}%
\pgfpathlineto{\pgfqpoint{4.325268in}{5.215826in}}%
\pgfpathlineto{\pgfqpoint{4.325976in}{5.219287in}}%
\pgfpathlineto{\pgfqpoint{4.328301in}{5.247436in}}%
\pgfpathlineto{\pgfqpoint{4.329211in}{5.241863in}}%
\pgfpathlineto{\pgfqpoint{4.331940in}{5.214691in}}%
\pgfpathlineto{\pgfqpoint{4.332547in}{5.218521in}}%
\pgfpathlineto{\pgfqpoint{4.333962in}{5.239846in}}%
\pgfpathlineto{\pgfqpoint{4.334670in}{5.235027in}}%
\pgfpathlineto{\pgfqpoint{4.336793in}{5.225342in}}%
\pgfpathlineto{\pgfqpoint{4.336894in}{5.225407in}}%
\pgfpathlineto{\pgfqpoint{4.340735in}{5.236357in}}%
\pgfpathlineto{\pgfqpoint{4.341241in}{5.234263in}}%
\pgfpathlineto{\pgfqpoint{4.342959in}{5.225628in}}%
\pgfpathlineto{\pgfqpoint{4.343364in}{5.226978in}}%
\pgfpathlineto{\pgfqpoint{4.343869in}{5.229223in}}%
\pgfpathlineto{\pgfqpoint{4.344274in}{5.226702in}}%
\pgfpathlineto{\pgfqpoint{4.345487in}{5.204019in}}%
\pgfpathlineto{\pgfqpoint{4.346093in}{5.212198in}}%
\pgfpathlineto{\pgfqpoint{4.347812in}{5.255479in}}%
\pgfpathlineto{\pgfqpoint{4.348418in}{5.248256in}}%
\pgfpathlineto{\pgfqpoint{4.350238in}{5.232894in}}%
\pgfpathlineto{\pgfqpoint{4.350339in}{5.232924in}}%
\pgfpathlineto{\pgfqpoint{4.350845in}{5.232993in}}%
\pgfpathlineto{\pgfqpoint{4.351047in}{5.232277in}}%
\pgfpathlineto{\pgfqpoint{4.352260in}{5.219700in}}%
\pgfpathlineto{\pgfqpoint{4.353069in}{5.226125in}}%
\pgfpathlineto{\pgfqpoint{4.353473in}{5.228255in}}%
\pgfpathlineto{\pgfqpoint{4.354080in}{5.225206in}}%
\pgfpathlineto{\pgfqpoint{4.354888in}{5.221134in}}%
\pgfpathlineto{\pgfqpoint{4.355394in}{5.222519in}}%
\pgfpathlineto{\pgfqpoint{4.357011in}{5.236827in}}%
\pgfpathlineto{\pgfqpoint{4.357618in}{5.240536in}}%
\pgfpathlineto{\pgfqpoint{4.358123in}{5.237964in}}%
\pgfpathlineto{\pgfqpoint{4.361055in}{5.218365in}}%
\pgfpathlineto{\pgfqpoint{4.361459in}{5.220018in}}%
\pgfpathlineto{\pgfqpoint{4.363987in}{5.242241in}}%
\pgfpathlineto{\pgfqpoint{4.364796in}{5.240065in}}%
\pgfpathlineto{\pgfqpoint{4.365705in}{5.233164in}}%
\pgfpathlineto{\pgfqpoint{4.366919in}{5.214610in}}%
\pgfpathlineto{\pgfqpoint{4.367525in}{5.221652in}}%
\pgfpathlineto{\pgfqpoint{4.368435in}{5.236900in}}%
\pgfpathlineto{\pgfqpoint{4.368940in}{5.232236in}}%
\pgfpathlineto{\pgfqpoint{4.369850in}{5.219596in}}%
\pgfpathlineto{\pgfqpoint{4.370457in}{5.224218in}}%
\pgfpathlineto{\pgfqpoint{4.372074in}{5.244488in}}%
\pgfpathlineto{\pgfqpoint{4.372580in}{5.240503in}}%
\pgfpathlineto{\pgfqpoint{4.374197in}{5.214703in}}%
\pgfpathlineto{\pgfqpoint{4.374905in}{5.220786in}}%
\pgfpathlineto{\pgfqpoint{4.376219in}{5.236144in}}%
\pgfpathlineto{\pgfqpoint{4.376725in}{5.233905in}}%
\pgfpathlineto{\pgfqpoint{4.377736in}{5.227572in}}%
\pgfpathlineto{\pgfqpoint{4.378241in}{5.230150in}}%
\pgfpathlineto{\pgfqpoint{4.378949in}{5.235271in}}%
\pgfpathlineto{\pgfqpoint{4.379454in}{5.232306in}}%
\pgfpathlineto{\pgfqpoint{4.380263in}{5.225994in}}%
\pgfpathlineto{\pgfqpoint{4.380768in}{5.228705in}}%
\pgfpathlineto{\pgfqpoint{4.381476in}{5.231873in}}%
\pgfpathlineto{\pgfqpoint{4.382184in}{5.231112in}}%
\pgfpathlineto{\pgfqpoint{4.382790in}{5.233217in}}%
\pgfpathlineto{\pgfqpoint{4.383599in}{5.236327in}}%
\pgfpathlineto{\pgfqpoint{4.384104in}{5.234652in}}%
\pgfpathlineto{\pgfqpoint{4.386025in}{5.218757in}}%
\pgfpathlineto{\pgfqpoint{4.386834in}{5.224816in}}%
\pgfpathlineto{\pgfqpoint{4.387845in}{5.231171in}}%
\pgfpathlineto{\pgfqpoint{4.388451in}{5.230504in}}%
\pgfpathlineto{\pgfqpoint{4.389462in}{5.228581in}}%
\pgfpathlineto{\pgfqpoint{4.390170in}{5.225458in}}%
\pgfpathlineto{\pgfqpoint{4.390675in}{5.228057in}}%
\pgfpathlineto{\pgfqpoint{4.392192in}{5.239451in}}%
\pgfpathlineto{\pgfqpoint{4.392596in}{5.237370in}}%
\pgfpathlineto{\pgfqpoint{4.393304in}{5.232055in}}%
\pgfpathlineto{\pgfqpoint{4.393809in}{5.235403in}}%
\pgfpathlineto{\pgfqpoint{4.394315in}{5.240082in}}%
\pgfpathlineto{\pgfqpoint{4.394820in}{5.235174in}}%
\pgfpathlineto{\pgfqpoint{4.395932in}{5.216510in}}%
\pgfpathlineto{\pgfqpoint{4.396539in}{5.221449in}}%
\pgfpathlineto{\pgfqpoint{4.397550in}{5.228997in}}%
\pgfpathlineto{\pgfqpoint{4.398055in}{5.226875in}}%
\pgfpathlineto{\pgfqpoint{4.398662in}{5.224353in}}%
\pgfpathlineto{\pgfqpoint{4.399268in}{5.225975in}}%
\pgfpathlineto{\pgfqpoint{4.401695in}{5.238864in}}%
\pgfpathlineto{\pgfqpoint{4.402402in}{5.234418in}}%
\pgfpathlineto{\pgfqpoint{4.402908in}{5.232382in}}%
\pgfpathlineto{\pgfqpoint{4.403413in}{5.234912in}}%
\pgfpathlineto{\pgfqpoint{4.403919in}{5.236839in}}%
\pgfpathlineto{\pgfqpoint{4.404323in}{5.234160in}}%
\pgfpathlineto{\pgfqpoint{4.405536in}{5.217402in}}%
\pgfpathlineto{\pgfqpoint{4.406143in}{5.222903in}}%
\pgfpathlineto{\pgfqpoint{4.407962in}{5.235593in}}%
\pgfpathlineto{\pgfqpoint{4.408468in}{5.235249in}}%
\pgfpathlineto{\pgfqpoint{4.408670in}{5.234745in}}%
\pgfpathlineto{\pgfqpoint{4.410490in}{5.226257in}}%
\pgfpathlineto{\pgfqpoint{4.412309in}{5.213319in}}%
\pgfpathlineto{\pgfqpoint{4.412815in}{5.216975in}}%
\pgfpathlineto{\pgfqpoint{4.414837in}{5.252640in}}%
\pgfpathlineto{\pgfqpoint{4.415646in}{5.246349in}}%
\pgfpathlineto{\pgfqpoint{4.417162in}{5.221109in}}%
\pgfpathlineto{\pgfqpoint{4.417971in}{5.224203in}}%
\pgfpathlineto{\pgfqpoint{4.418274in}{5.224477in}}%
\pgfpathlineto{\pgfqpoint{4.418678in}{5.223465in}}%
\pgfpathlineto{\pgfqpoint{4.419993in}{5.218169in}}%
\pgfpathlineto{\pgfqpoint{4.420498in}{5.220162in}}%
\pgfpathlineto{\pgfqpoint{4.422924in}{5.248055in}}%
\pgfpathlineto{\pgfqpoint{4.423733in}{5.238313in}}%
\pgfpathlineto{\pgfqpoint{4.425249in}{5.219123in}}%
\pgfpathlineto{\pgfqpoint{4.425654in}{5.220218in}}%
\pgfpathlineto{\pgfqpoint{4.428080in}{5.237349in}}%
\pgfpathlineto{\pgfqpoint{4.429091in}{5.246815in}}%
\pgfpathlineto{\pgfqpoint{4.429596in}{5.244340in}}%
\pgfpathlineto{\pgfqpoint{4.431113in}{5.220635in}}%
\pgfpathlineto{\pgfqpoint{4.431719in}{5.215767in}}%
\pgfpathlineto{\pgfqpoint{4.432326in}{5.218316in}}%
\pgfpathlineto{\pgfqpoint{4.434348in}{5.232515in}}%
\pgfpathlineto{\pgfqpoint{4.435359in}{5.240315in}}%
\pgfpathlineto{\pgfqpoint{4.435864in}{5.238668in}}%
\pgfpathlineto{\pgfqpoint{4.437280in}{5.227592in}}%
\pgfpathlineto{\pgfqpoint{4.437886in}{5.230621in}}%
\pgfpathlineto{\pgfqpoint{4.438291in}{5.232270in}}%
\pgfpathlineto{\pgfqpoint{4.438796in}{5.229988in}}%
\pgfpathlineto{\pgfqpoint{4.439605in}{5.225414in}}%
\pgfpathlineto{\pgfqpoint{4.440211in}{5.227299in}}%
\pgfpathlineto{\pgfqpoint{4.441930in}{5.236799in}}%
\pgfpathlineto{\pgfqpoint{4.442435in}{5.232088in}}%
\pgfpathlineto{\pgfqpoint{4.443446in}{5.219978in}}%
\pgfpathlineto{\pgfqpoint{4.444053in}{5.223502in}}%
\pgfpathlineto{\pgfqpoint{4.445670in}{5.242209in}}%
\pgfpathlineto{\pgfqpoint{4.446277in}{5.238588in}}%
\pgfpathlineto{\pgfqpoint{4.447793in}{5.220946in}}%
\pgfpathlineto{\pgfqpoint{4.448400in}{5.225057in}}%
\pgfpathlineto{\pgfqpoint{4.449613in}{5.235104in}}%
\pgfpathlineto{\pgfqpoint{4.450118in}{5.231646in}}%
\pgfpathlineto{\pgfqpoint{4.451332in}{5.220446in}}%
\pgfpathlineto{\pgfqpoint{4.451837in}{5.222104in}}%
\pgfpathlineto{\pgfqpoint{4.454769in}{5.240280in}}%
\pgfpathlineto{\pgfqpoint{4.455274in}{5.238655in}}%
\pgfpathlineto{\pgfqpoint{4.456993in}{5.225262in}}%
\pgfpathlineto{\pgfqpoint{4.457903in}{5.213959in}}%
\pgfpathlineto{\pgfqpoint{4.458408in}{5.219189in}}%
\pgfpathlineto{\pgfqpoint{4.460632in}{5.243894in}}%
\pgfpathlineto{\pgfqpoint{4.460834in}{5.243500in}}%
\pgfpathlineto{\pgfqpoint{4.463665in}{5.216396in}}%
\pgfpathlineto{\pgfqpoint{4.464474in}{5.225531in}}%
\pgfpathlineto{\pgfqpoint{4.465282in}{5.233514in}}%
\pgfpathlineto{\pgfqpoint{4.465889in}{5.231168in}}%
\pgfpathlineto{\pgfqpoint{4.467507in}{5.229455in}}%
\pgfpathlineto{\pgfqpoint{4.467810in}{5.229265in}}%
\pgfpathlineto{\pgfqpoint{4.468113in}{5.230204in}}%
\pgfpathlineto{\pgfqpoint{4.470034in}{5.241747in}}%
\pgfpathlineto{\pgfqpoint{4.470539in}{5.239245in}}%
\pgfpathlineto{\pgfqpoint{4.471955in}{5.217729in}}%
\pgfpathlineto{\pgfqpoint{4.472966in}{5.221982in}}%
\pgfpathlineto{\pgfqpoint{4.473471in}{5.221699in}}%
\pgfpathlineto{\pgfqpoint{4.473673in}{5.222292in}}%
\pgfpathlineto{\pgfqpoint{4.476201in}{5.235446in}}%
\pgfpathlineto{\pgfqpoint{4.476908in}{5.234483in}}%
\pgfpathlineto{\pgfqpoint{4.478728in}{5.230662in}}%
\pgfpathlineto{\pgfqpoint{4.480042in}{5.215585in}}%
\pgfpathlineto{\pgfqpoint{4.480548in}{5.212161in}}%
\pgfpathlineto{\pgfqpoint{4.480952in}{5.216150in}}%
\pgfpathlineto{\pgfqpoint{4.482671in}{5.250546in}}%
\pgfpathlineto{\pgfqpoint{4.483277in}{5.244002in}}%
\pgfpathlineto{\pgfqpoint{4.484793in}{5.222083in}}%
\pgfpathlineto{\pgfqpoint{4.485299in}{5.225190in}}%
\pgfpathlineto{\pgfqpoint{4.486310in}{5.235426in}}%
\pgfpathlineto{\pgfqpoint{4.486815in}{5.230700in}}%
\pgfpathlineto{\pgfqpoint{4.487826in}{5.215235in}}%
\pgfpathlineto{\pgfqpoint{4.488332in}{5.220645in}}%
\pgfpathlineto{\pgfqpoint{4.490354in}{5.237038in}}%
\pgfpathlineto{\pgfqpoint{4.490758in}{5.237908in}}%
\pgfpathlineto{\pgfqpoint{4.491162in}{5.236518in}}%
\pgfpathlineto{\pgfqpoint{4.492274in}{5.229897in}}%
\pgfpathlineto{\pgfqpoint{4.492982in}{5.231648in}}%
\pgfpathlineto{\pgfqpoint{4.493285in}{5.232091in}}%
\pgfpathlineto{\pgfqpoint{4.493690in}{5.230831in}}%
\pgfpathlineto{\pgfqpoint{4.495408in}{5.219698in}}%
\pgfpathlineto{\pgfqpoint{4.496015in}{5.223083in}}%
\pgfpathlineto{\pgfqpoint{4.497127in}{5.234078in}}%
\pgfpathlineto{\pgfqpoint{4.497733in}{5.231262in}}%
\pgfpathlineto{\pgfqpoint{4.498239in}{5.229322in}}%
\pgfpathlineto{\pgfqpoint{4.498845in}{5.231055in}}%
\pgfpathlineto{\pgfqpoint{4.501373in}{5.238142in}}%
\pgfpathlineto{\pgfqpoint{4.501979in}{5.239534in}}%
\pgfpathlineto{\pgfqpoint{4.502384in}{5.238108in}}%
\pgfpathlineto{\pgfqpoint{4.504305in}{5.217413in}}%
\pgfpathlineto{\pgfqpoint{4.505113in}{5.223519in}}%
\pgfpathlineto{\pgfqpoint{4.506023in}{5.232963in}}%
\pgfpathlineto{\pgfqpoint{4.506427in}{5.229405in}}%
\pgfpathlineto{\pgfqpoint{4.507438in}{5.217257in}}%
\pgfpathlineto{\pgfqpoint{4.507944in}{5.222165in}}%
\pgfpathlineto{\pgfqpoint{4.509359in}{5.243168in}}%
\pgfpathlineto{\pgfqpoint{4.509966in}{5.239544in}}%
\pgfpathlineto{\pgfqpoint{4.511583in}{5.222717in}}%
\pgfpathlineto{\pgfqpoint{4.512089in}{5.226664in}}%
\pgfpathlineto{\pgfqpoint{4.513100in}{5.236517in}}%
\pgfpathlineto{\pgfqpoint{4.513605in}{5.233180in}}%
\pgfpathlineto{\pgfqpoint{4.514313in}{5.228879in}}%
\pgfpathlineto{\pgfqpoint{4.514818in}{5.230813in}}%
\pgfpathlineto{\pgfqpoint{4.516335in}{5.235144in}}%
\pgfpathlineto{\pgfqpoint{4.516537in}{5.234940in}}%
\pgfpathlineto{\pgfqpoint{4.517143in}{5.229820in}}%
\pgfpathlineto{\pgfqpoint{4.518154in}{5.218226in}}%
\pgfpathlineto{\pgfqpoint{4.518761in}{5.222016in}}%
\pgfpathlineto{\pgfqpoint{4.519266in}{5.224585in}}%
\pgfpathlineto{\pgfqpoint{4.519974in}{5.222983in}}%
\pgfpathlineto{\pgfqpoint{4.520277in}{5.223552in}}%
\pgfpathlineto{\pgfqpoint{4.522906in}{5.243866in}}%
\pgfpathlineto{\pgfqpoint{4.523613in}{5.237889in}}%
\pgfpathlineto{\pgfqpoint{4.524624in}{5.229640in}}%
\pgfpathlineto{\pgfqpoint{4.525231in}{5.230836in}}%
\pgfpathlineto{\pgfqpoint{4.525635in}{5.228336in}}%
\pgfpathlineto{\pgfqpoint{4.526747in}{5.214482in}}%
\pgfpathlineto{\pgfqpoint{4.527354in}{5.219352in}}%
\pgfpathlineto{\pgfqpoint{4.529477in}{5.234323in}}%
\pgfpathlineto{\pgfqpoint{4.529578in}{5.234185in}}%
\pgfpathlineto{\pgfqpoint{4.530589in}{5.230728in}}%
\pgfpathlineto{\pgfqpoint{4.531094in}{5.232936in}}%
\pgfpathlineto{\pgfqpoint{4.532712in}{5.236501in}}%
\pgfpathlineto{\pgfqpoint{4.533116in}{5.236782in}}%
\pgfpathlineto{\pgfqpoint{4.533419in}{5.235619in}}%
\pgfpathlineto{\pgfqpoint{4.534835in}{5.216728in}}%
\pgfpathlineto{\pgfqpoint{4.535542in}{5.224983in}}%
\pgfpathlineto{\pgfqpoint{4.536654in}{5.235957in}}%
\pgfpathlineto{\pgfqpoint{4.537160in}{5.234799in}}%
\pgfpathlineto{\pgfqpoint{4.538373in}{5.224909in}}%
\pgfpathlineto{\pgfqpoint{4.538878in}{5.221535in}}%
\pgfpathlineto{\pgfqpoint{4.539384in}{5.224605in}}%
\pgfpathlineto{\pgfqpoint{4.540496in}{5.236043in}}%
\pgfpathlineto{\pgfqpoint{4.541103in}{5.233604in}}%
\pgfpathlineto{\pgfqpoint{4.542316in}{5.225518in}}%
\pgfpathlineto{\pgfqpoint{4.542922in}{5.228108in}}%
\pgfpathlineto{\pgfqpoint{4.544540in}{5.236451in}}%
\pgfpathlineto{\pgfqpoint{4.544843in}{5.235539in}}%
\pgfpathlineto{\pgfqpoint{4.546157in}{5.223194in}}%
\pgfpathlineto{\pgfqpoint{4.546865in}{5.228342in}}%
\pgfpathlineto{\pgfqpoint{4.547674in}{5.232864in}}%
\pgfpathlineto{\pgfqpoint{4.548179in}{5.231061in}}%
\pgfpathlineto{\pgfqpoint{4.549695in}{5.216496in}}%
\pgfpathlineto{\pgfqpoint{4.550302in}{5.223322in}}%
\pgfpathlineto{\pgfqpoint{4.551515in}{5.240910in}}%
\pgfpathlineto{\pgfqpoint{4.552122in}{5.239696in}}%
\pgfpathlineto{\pgfqpoint{4.553234in}{5.237355in}}%
\pgfpathlineto{\pgfqpoint{4.555559in}{5.226773in}}%
\pgfpathlineto{\pgfqpoint{4.556570in}{5.222824in}}%
\pgfpathlineto{\pgfqpoint{4.557075in}{5.224177in}}%
\pgfpathlineto{\pgfqpoint{4.559805in}{5.231666in}}%
\pgfpathlineto{\pgfqpoint{4.561018in}{5.231606in}}%
\pgfpathlineto{\pgfqpoint{4.562130in}{5.231160in}}%
\pgfpathlineto{\pgfqpoint{4.562332in}{5.231781in}}%
\pgfpathlineto{\pgfqpoint{4.563242in}{5.235815in}}%
\pgfpathlineto{\pgfqpoint{4.563747in}{5.233829in}}%
\pgfpathlineto{\pgfqpoint{4.565466in}{5.227849in}}%
\pgfpathlineto{\pgfqpoint{4.565668in}{5.228068in}}%
\pgfpathlineto{\pgfqpoint{4.567084in}{5.231843in}}%
\pgfpathlineto{\pgfqpoint{4.567690in}{5.229979in}}%
\pgfpathlineto{\pgfqpoint{4.569105in}{5.222474in}}%
\pgfpathlineto{\pgfqpoint{4.569712in}{5.225185in}}%
\pgfpathlineto{\pgfqpoint{4.571835in}{5.235167in}}%
\pgfpathlineto{\pgfqpoint{4.571936in}{5.235032in}}%
\pgfpathlineto{\pgfqpoint{4.572644in}{5.229232in}}%
\pgfpathlineto{\pgfqpoint{4.573149in}{5.226011in}}%
\pgfpathlineto{\pgfqpoint{4.573655in}{5.229711in}}%
\pgfpathlineto{\pgfqpoint{4.574362in}{5.236767in}}%
\pgfpathlineto{\pgfqpoint{4.574868in}{5.233561in}}%
\pgfpathlineto{\pgfqpoint{4.575980in}{5.223183in}}%
\pgfpathlineto{\pgfqpoint{4.576586in}{5.224761in}}%
\pgfpathlineto{\pgfqpoint{4.577698in}{5.235108in}}%
\pgfpathlineto{\pgfqpoint{4.578406in}{5.240113in}}%
\pgfpathlineto{\pgfqpoint{4.579013in}{5.237811in}}%
\pgfpathlineto{\pgfqpoint{4.583360in}{5.217995in}}%
\pgfpathlineto{\pgfqpoint{4.583562in}{5.218156in}}%
\pgfpathlineto{\pgfqpoint{4.584472in}{5.222848in}}%
\pgfpathlineto{\pgfqpoint{4.586999in}{5.242037in}}%
\pgfpathlineto{\pgfqpoint{4.587808in}{5.239984in}}%
\pgfpathlineto{\pgfqpoint{4.589324in}{5.232653in}}%
\pgfpathlineto{\pgfqpoint{4.591144in}{5.224821in}}%
\pgfpathlineto{\pgfqpoint{4.592256in}{5.221132in}}%
\pgfpathlineto{\pgfqpoint{4.592761in}{5.222976in}}%
\pgfpathlineto{\pgfqpoint{4.594177in}{5.234975in}}%
\pgfpathlineto{\pgfqpoint{4.594884in}{5.231349in}}%
\pgfpathlineto{\pgfqpoint{4.595693in}{5.226087in}}%
\pgfpathlineto{\pgfqpoint{4.596198in}{5.228238in}}%
\pgfpathlineto{\pgfqpoint{4.597209in}{5.231788in}}%
\pgfpathlineto{\pgfqpoint{4.597715in}{5.231334in}}%
\pgfpathlineto{\pgfqpoint{4.598423in}{5.227522in}}%
\pgfpathlineto{\pgfqpoint{4.599332in}{5.220901in}}%
\pgfpathlineto{\pgfqpoint{4.599838in}{5.224287in}}%
\pgfpathlineto{\pgfqpoint{4.601860in}{5.248903in}}%
\pgfpathlineto{\pgfqpoint{4.602567in}{5.244486in}}%
\pgfpathlineto{\pgfqpoint{4.606005in}{5.213199in}}%
\pgfpathlineto{\pgfqpoint{4.606409in}{5.214211in}}%
\pgfpathlineto{\pgfqpoint{4.607521in}{5.228577in}}%
\pgfpathlineto{\pgfqpoint{4.609644in}{5.245604in}}%
\pgfpathlineto{\pgfqpoint{4.609846in}{5.245854in}}%
\pgfpathlineto{\pgfqpoint{4.610250in}{5.244891in}}%
\pgfpathlineto{\pgfqpoint{4.611666in}{5.232692in}}%
\pgfpathlineto{\pgfqpoint{4.613182in}{5.218026in}}%
\pgfpathlineto{\pgfqpoint{4.613789in}{5.219819in}}%
\pgfpathlineto{\pgfqpoint{4.616417in}{5.228004in}}%
\pgfpathlineto{\pgfqpoint{4.617731in}{5.236284in}}%
\pgfpathlineto{\pgfqpoint{4.618641in}{5.239461in}}%
\pgfpathlineto{\pgfqpoint{4.619046in}{5.238470in}}%
\pgfpathlineto{\pgfqpoint{4.620158in}{5.228015in}}%
\pgfpathlineto{\pgfqpoint{4.620865in}{5.223620in}}%
\pgfpathlineto{\pgfqpoint{4.621472in}{5.225303in}}%
\pgfpathlineto{\pgfqpoint{4.622179in}{5.226458in}}%
\pgfpathlineto{\pgfqpoint{4.622786in}{5.226105in}}%
\pgfpathlineto{\pgfqpoint{4.623393in}{5.227244in}}%
\pgfpathlineto{\pgfqpoint{4.624909in}{5.233253in}}%
\pgfpathlineto{\pgfqpoint{4.625516in}{5.231755in}}%
\pgfpathlineto{\pgfqpoint{4.626223in}{5.229589in}}%
\pgfpathlineto{\pgfqpoint{4.626830in}{5.230856in}}%
\pgfpathlineto{\pgfqpoint{4.628144in}{5.232800in}}%
\pgfpathlineto{\pgfqpoint{4.628346in}{5.232513in}}%
\pgfpathlineto{\pgfqpoint{4.629256in}{5.226419in}}%
\pgfpathlineto{\pgfqpoint{4.629660in}{5.224962in}}%
\pgfpathlineto{\pgfqpoint{4.630166in}{5.227367in}}%
\pgfpathlineto{\pgfqpoint{4.630772in}{5.229896in}}%
\pgfpathlineto{\pgfqpoint{4.631278in}{5.228345in}}%
\pgfpathlineto{\pgfqpoint{4.631783in}{5.227042in}}%
\pgfpathlineto{\pgfqpoint{4.632289in}{5.228744in}}%
\pgfpathlineto{\pgfqpoint{4.633603in}{5.237364in}}%
\pgfpathlineto{\pgfqpoint{4.634210in}{5.234559in}}%
\pgfpathlineto{\pgfqpoint{4.635625in}{5.223098in}}%
\pgfpathlineto{\pgfqpoint{4.636231in}{5.225767in}}%
\pgfpathlineto{\pgfqpoint{4.637950in}{5.235696in}}%
\pgfpathlineto{\pgfqpoint{4.638354in}{5.234195in}}%
\pgfpathlineto{\pgfqpoint{4.639568in}{5.224782in}}%
\pgfpathlineto{\pgfqpoint{4.640275in}{5.226692in}}%
\pgfpathlineto{\pgfqpoint{4.640578in}{5.226970in}}%
\pgfpathlineto{\pgfqpoint{4.641084in}{5.226121in}}%
\pgfpathlineto{\pgfqpoint{4.641286in}{5.225968in}}%
\pgfpathlineto{\pgfqpoint{4.641589in}{5.226683in}}%
\pgfpathlineto{\pgfqpoint{4.644420in}{5.235589in}}%
\pgfpathlineto{\pgfqpoint{4.645633in}{5.234769in}}%
\pgfpathlineto{\pgfqpoint{4.646341in}{5.230998in}}%
\pgfpathlineto{\pgfqpoint{4.647958in}{5.214377in}}%
\pgfpathlineto{\pgfqpoint{4.648565in}{5.217502in}}%
\pgfpathlineto{\pgfqpoint{4.650486in}{5.240168in}}%
\pgfpathlineto{\pgfqpoint{4.651395in}{5.237460in}}%
\pgfpathlineto{\pgfqpoint{4.654024in}{5.225566in}}%
\pgfpathlineto{\pgfqpoint{4.654630in}{5.223516in}}%
\pgfpathlineto{\pgfqpoint{4.655035in}{5.224916in}}%
\pgfpathlineto{\pgfqpoint{4.656147in}{5.232824in}}%
\pgfpathlineto{\pgfqpoint{4.656956in}{5.231679in}}%
\pgfpathlineto{\pgfqpoint{4.657967in}{5.235775in}}%
\pgfpathlineto{\pgfqpoint{4.658573in}{5.232988in}}%
\pgfpathlineto{\pgfqpoint{4.661202in}{5.221807in}}%
\pgfpathlineto{\pgfqpoint{4.661707in}{5.224398in}}%
\pgfpathlineto{\pgfqpoint{4.662617in}{5.229961in}}%
\pgfpathlineto{\pgfqpoint{4.663223in}{5.228588in}}%
\pgfpathlineto{\pgfqpoint{4.663527in}{5.228100in}}%
\pgfpathlineto{\pgfqpoint{4.664032in}{5.229312in}}%
\pgfpathlineto{\pgfqpoint{4.665852in}{5.232373in}}%
\pgfpathlineto{\pgfqpoint{4.666762in}{5.234037in}}%
\pgfpathlineto{\pgfqpoint{4.667267in}{5.232636in}}%
\pgfpathlineto{\pgfqpoint{4.668177in}{5.229620in}}%
\pgfpathlineto{\pgfqpoint{4.668885in}{5.230143in}}%
\pgfpathlineto{\pgfqpoint{4.671109in}{5.228620in}}%
\pgfpathlineto{\pgfqpoint{4.672120in}{5.225840in}}%
\pgfpathlineto{\pgfqpoint{4.672726in}{5.226382in}}%
\pgfpathlineto{\pgfqpoint{4.673535in}{5.228526in}}%
\pgfpathlineto{\pgfqpoint{4.674849in}{5.239604in}}%
\pgfpathlineto{\pgfqpoint{4.675456in}{5.235441in}}%
\pgfpathlineto{\pgfqpoint{4.676871in}{5.222274in}}%
\pgfpathlineto{\pgfqpoint{4.677376in}{5.223281in}}%
\pgfpathlineto{\pgfqpoint{4.680915in}{5.232393in}}%
\pgfpathlineto{\pgfqpoint{4.681825in}{5.234288in}}%
\pgfpathlineto{\pgfqpoint{4.682229in}{5.233200in}}%
\pgfpathlineto{\pgfqpoint{4.684857in}{5.224462in}}%
\pgfpathlineto{\pgfqpoint{4.685262in}{5.225890in}}%
\pgfpathlineto{\pgfqpoint{4.686677in}{5.235166in}}%
\pgfpathlineto{\pgfqpoint{4.687385in}{5.233392in}}%
\pgfpathlineto{\pgfqpoint{4.689912in}{5.226185in}}%
\pgfpathlineto{\pgfqpoint{4.690114in}{5.226572in}}%
\pgfpathlineto{\pgfqpoint{4.691024in}{5.231446in}}%
\pgfpathlineto{\pgfqpoint{4.691530in}{5.229106in}}%
\pgfpathlineto{\pgfqpoint{4.692439in}{5.224178in}}%
\pgfpathlineto{\pgfqpoint{4.692945in}{5.225930in}}%
\pgfpathlineto{\pgfqpoint{4.695371in}{5.237971in}}%
\pgfpathlineto{\pgfqpoint{4.695674in}{5.236734in}}%
\pgfpathlineto{\pgfqpoint{4.698101in}{5.225711in}}%
\pgfpathlineto{\pgfqpoint{4.698606in}{5.226457in}}%
\pgfpathlineto{\pgfqpoint{4.699314in}{5.227916in}}%
\pgfpathlineto{\pgfqpoint{4.699920in}{5.227173in}}%
\pgfpathlineto{\pgfqpoint{4.700426in}{5.227895in}}%
\pgfpathlineto{\pgfqpoint{4.702751in}{5.234877in}}%
\pgfpathlineto{\pgfqpoint{4.703155in}{5.233818in}}%
\pgfpathlineto{\pgfqpoint{4.704672in}{5.224368in}}%
\pgfpathlineto{\pgfqpoint{4.705379in}{5.227669in}}%
\pgfpathlineto{\pgfqpoint{4.707300in}{5.232950in}}%
\pgfpathlineto{\pgfqpoint{4.707401in}{5.232892in}}%
\pgfpathlineto{\pgfqpoint{4.708109in}{5.229034in}}%
\pgfpathlineto{\pgfqpoint{4.708817in}{5.225998in}}%
\pgfpathlineto{\pgfqpoint{4.709221in}{5.227926in}}%
\pgfpathlineto{\pgfqpoint{4.710535in}{5.242913in}}%
\pgfpathlineto{\pgfqpoint{4.711142in}{5.236766in}}%
\pgfpathlineto{\pgfqpoint{4.712658in}{5.221995in}}%
\pgfpathlineto{\pgfqpoint{4.713062in}{5.223014in}}%
\pgfpathlineto{\pgfqpoint{4.714377in}{5.231814in}}%
\pgfpathlineto{\pgfqpoint{4.714983in}{5.228749in}}%
\pgfpathlineto{\pgfqpoint{4.715893in}{5.224730in}}%
\pgfpathlineto{\pgfqpoint{4.716500in}{5.225779in}}%
\pgfpathlineto{\pgfqpoint{4.720442in}{5.233232in}}%
\pgfpathlineto{\pgfqpoint{4.721352in}{5.234474in}}%
\pgfpathlineto{\pgfqpoint{4.722161in}{5.236289in}}%
\pgfpathlineto{\pgfqpoint{4.722666in}{5.235104in}}%
\pgfpathlineto{\pgfqpoint{4.724486in}{5.226702in}}%
\pgfpathlineto{\pgfqpoint{4.725194in}{5.227959in}}%
\pgfpathlineto{\pgfqpoint{4.725598in}{5.228441in}}%
\pgfpathlineto{\pgfqpoint{4.726002in}{5.227565in}}%
\pgfpathlineto{\pgfqpoint{4.727317in}{5.219946in}}%
\pgfpathlineto{\pgfqpoint{4.727923in}{5.222958in}}%
\pgfpathlineto{\pgfqpoint{4.730552in}{5.237263in}}%
\pgfpathlineto{\pgfqpoint{4.731461in}{5.236515in}}%
\pgfpathlineto{\pgfqpoint{4.733382in}{5.225860in}}%
\pgfpathlineto{\pgfqpoint{4.734494in}{5.228201in}}%
\pgfpathlineto{\pgfqpoint{4.735404in}{5.231742in}}%
\pgfpathlineto{\pgfqpoint{4.735910in}{5.229769in}}%
\pgfpathlineto{\pgfqpoint{4.737022in}{5.225789in}}%
\pgfpathlineto{\pgfqpoint{4.737527in}{5.226012in}}%
\pgfpathlineto{\pgfqpoint{4.738134in}{5.228474in}}%
\pgfpathlineto{\pgfqpoint{4.739145in}{5.235760in}}%
\pgfpathlineto{\pgfqpoint{4.739751in}{5.232689in}}%
\pgfpathlineto{\pgfqpoint{4.740459in}{5.228927in}}%
\pgfpathlineto{\pgfqpoint{4.740964in}{5.230538in}}%
\pgfpathlineto{\pgfqpoint{4.741773in}{5.234506in}}%
\pgfpathlineto{\pgfqpoint{4.742278in}{5.232743in}}%
\pgfpathlineto{\pgfqpoint{4.743593in}{5.224750in}}%
\pgfpathlineto{\pgfqpoint{4.744199in}{5.226617in}}%
\pgfpathlineto{\pgfqpoint{4.745615in}{5.231978in}}%
\pgfpathlineto{\pgfqpoint{4.746120in}{5.231083in}}%
\pgfpathlineto{\pgfqpoint{4.747535in}{5.224653in}}%
\pgfpathlineto{\pgfqpoint{4.748142in}{5.226778in}}%
\pgfpathlineto{\pgfqpoint{4.749860in}{5.235537in}}%
\pgfpathlineto{\pgfqpoint{4.750366in}{5.234631in}}%
\pgfpathlineto{\pgfqpoint{4.752893in}{5.225796in}}%
\pgfpathlineto{\pgfqpoint{4.753500in}{5.228560in}}%
\pgfpathlineto{\pgfqpoint{4.754207in}{5.231234in}}%
\pgfpathlineto{\pgfqpoint{4.754713in}{5.229698in}}%
\pgfpathlineto{\pgfqpoint{4.755926in}{5.223377in}}%
\pgfpathlineto{\pgfqpoint{4.756432in}{5.225231in}}%
\pgfpathlineto{\pgfqpoint{4.758049in}{5.237727in}}%
\pgfpathlineto{\pgfqpoint{4.758656in}{5.234926in}}%
\pgfpathlineto{\pgfqpoint{4.759970in}{5.225605in}}%
\pgfpathlineto{\pgfqpoint{4.760576in}{5.227956in}}%
\pgfpathlineto{\pgfqpoint{4.761284in}{5.230398in}}%
\pgfpathlineto{\pgfqpoint{4.761789in}{5.229210in}}%
\pgfpathlineto{\pgfqpoint{4.763306in}{5.223948in}}%
\pgfpathlineto{\pgfqpoint{4.763811in}{5.226009in}}%
\pgfpathlineto{\pgfqpoint{4.765429in}{5.238421in}}%
\pgfpathlineto{\pgfqpoint{4.766035in}{5.237075in}}%
\pgfpathlineto{\pgfqpoint{4.769877in}{5.220988in}}%
\pgfpathlineto{\pgfqpoint{4.770484in}{5.223638in}}%
\pgfpathlineto{\pgfqpoint{4.772404in}{5.229874in}}%
\pgfpathlineto{\pgfqpoint{4.773011in}{5.230555in}}%
\pgfpathlineto{\pgfqpoint{4.773415in}{5.229724in}}%
\pgfpathlineto{\pgfqpoint{4.773921in}{5.228786in}}%
\pgfpathlineto{\pgfqpoint{4.774325in}{5.229882in}}%
\pgfpathlineto{\pgfqpoint{4.775437in}{5.234776in}}%
\pgfpathlineto{\pgfqpoint{4.775943in}{5.233311in}}%
\pgfpathlineto{\pgfqpoint{4.776954in}{5.230458in}}%
\pgfpathlineto{\pgfqpoint{4.777358in}{5.231197in}}%
\pgfpathlineto{\pgfqpoint{4.778874in}{5.236171in}}%
\pgfpathlineto{\pgfqpoint{4.779380in}{5.234277in}}%
\pgfpathlineto{\pgfqpoint{4.781806in}{5.221688in}}%
\pgfpathlineto{\pgfqpoint{4.782109in}{5.222271in}}%
\pgfpathlineto{\pgfqpoint{4.785850in}{5.235617in}}%
\pgfpathlineto{\pgfqpoint{4.786355in}{5.234887in}}%
\pgfpathlineto{\pgfqpoint{4.787973in}{5.226536in}}%
\pgfpathlineto{\pgfqpoint{4.788883in}{5.229550in}}%
\pgfpathlineto{\pgfqpoint{4.789287in}{5.230391in}}%
\pgfpathlineto{\pgfqpoint{4.789792in}{5.228843in}}%
\pgfpathlineto{\pgfqpoint{4.791005in}{5.221191in}}%
\pgfpathlineto{\pgfqpoint{4.791612in}{5.224250in}}%
\pgfpathlineto{\pgfqpoint{4.793432in}{5.238998in}}%
\pgfpathlineto{\pgfqpoint{4.793937in}{5.237977in}}%
\pgfpathlineto{\pgfqpoint{4.798992in}{5.221086in}}%
\pgfpathlineto{\pgfqpoint{4.799396in}{5.222104in}}%
\pgfpathlineto{\pgfqpoint{4.802429in}{5.237754in}}%
\pgfpathlineto{\pgfqpoint{4.802732in}{5.237250in}}%
\pgfpathlineto{\pgfqpoint{4.805057in}{5.224790in}}%
\pgfpathlineto{\pgfqpoint{4.805563in}{5.223367in}}%
\pgfpathlineto{\pgfqpoint{4.806068in}{5.224935in}}%
\pgfpathlineto{\pgfqpoint{4.807989in}{5.233215in}}%
\pgfpathlineto{\pgfqpoint{4.808495in}{5.231528in}}%
\pgfpathlineto{\pgfqpoint{4.809405in}{5.227597in}}%
\pgfpathlineto{\pgfqpoint{4.809910in}{5.229109in}}%
\pgfpathlineto{\pgfqpoint{4.810921in}{5.231719in}}%
\pgfpathlineto{\pgfqpoint{4.811426in}{5.231276in}}%
\pgfpathlineto{\pgfqpoint{4.812336in}{5.227470in}}%
\pgfpathlineto{\pgfqpoint{4.813246in}{5.223611in}}%
\pgfpathlineto{\pgfqpoint{4.813650in}{5.224931in}}%
\pgfpathlineto{\pgfqpoint{4.815672in}{5.239901in}}%
\pgfpathlineto{\pgfqpoint{4.816481in}{5.235732in}}%
\pgfpathlineto{\pgfqpoint{4.817896in}{5.224840in}}%
\pgfpathlineto{\pgfqpoint{4.818503in}{5.226225in}}%
\pgfpathlineto{\pgfqpoint{4.819211in}{5.227898in}}%
\pgfpathlineto{\pgfqpoint{4.819716in}{5.226713in}}%
\pgfpathlineto{\pgfqpoint{4.820828in}{5.223464in}}%
\pgfpathlineto{\pgfqpoint{4.821232in}{5.224769in}}%
\pgfpathlineto{\pgfqpoint{4.822850in}{5.237604in}}%
\pgfpathlineto{\pgfqpoint{4.823760in}{5.234857in}}%
\pgfpathlineto{\pgfqpoint{4.827804in}{5.220708in}}%
\pgfpathlineto{\pgfqpoint{4.828410in}{5.224520in}}%
\pgfpathlineto{\pgfqpoint{4.829825in}{5.237851in}}%
\pgfpathlineto{\pgfqpoint{4.830432in}{5.235633in}}%
\pgfpathlineto{\pgfqpoint{4.831948in}{5.228012in}}%
\pgfpathlineto{\pgfqpoint{4.832555in}{5.228984in}}%
\pgfpathlineto{\pgfqpoint{4.833869in}{5.230666in}}%
\pgfpathlineto{\pgfqpoint{4.834172in}{5.230329in}}%
\pgfpathlineto{\pgfqpoint{4.835487in}{5.228141in}}%
\pgfpathlineto{\pgfqpoint{4.836093in}{5.228844in}}%
\pgfpathlineto{\pgfqpoint{4.838115in}{5.229983in}}%
\pgfpathlineto{\pgfqpoint{4.839328in}{5.231182in}}%
\pgfpathlineto{\pgfqpoint{4.839834in}{5.230363in}}%
\pgfpathlineto{\pgfqpoint{4.842260in}{5.226331in}}%
\pgfpathlineto{\pgfqpoint{4.842563in}{5.226891in}}%
\pgfpathlineto{\pgfqpoint{4.844282in}{5.234920in}}%
\pgfpathlineto{\pgfqpoint{4.845090in}{5.232899in}}%
\pgfpathlineto{\pgfqpoint{4.847112in}{5.229177in}}%
\pgfpathlineto{\pgfqpoint{4.847921in}{5.228478in}}%
\pgfpathlineto{\pgfqpoint{4.848325in}{5.229213in}}%
\pgfpathlineto{\pgfqpoint{4.849235in}{5.231158in}}%
\pgfpathlineto{\pgfqpoint{4.849741in}{5.230282in}}%
\pgfpathlineto{\pgfqpoint{4.851864in}{5.223086in}}%
\pgfpathlineto{\pgfqpoint{4.852470in}{5.225880in}}%
\pgfpathlineto{\pgfqpoint{4.854694in}{5.234756in}}%
\pgfpathlineto{\pgfqpoint{4.855200in}{5.235274in}}%
\pgfpathlineto{\pgfqpoint{4.855705in}{5.234621in}}%
\pgfpathlineto{\pgfqpoint{4.858637in}{5.228694in}}%
\pgfpathlineto{\pgfqpoint{4.860558in}{5.221523in}}%
\pgfpathlineto{\pgfqpoint{4.861265in}{5.222907in}}%
\pgfpathlineto{\pgfqpoint{4.865916in}{5.237118in}}%
\pgfpathlineto{\pgfqpoint{4.866421in}{5.236196in}}%
\pgfpathlineto{\pgfqpoint{4.868949in}{5.226871in}}%
\pgfpathlineto{\pgfqpoint{4.869959in}{5.223245in}}%
\pgfpathlineto{\pgfqpoint{4.870465in}{5.224065in}}%
\pgfpathlineto{\pgfqpoint{4.871880in}{5.226808in}}%
\pgfpathlineto{\pgfqpoint{4.872386in}{5.226087in}}%
\pgfpathlineto{\pgfqpoint{4.872891in}{5.225515in}}%
\pgfpathlineto{\pgfqpoint{4.873194in}{5.226364in}}%
\pgfpathlineto{\pgfqpoint{4.876227in}{5.236502in}}%
\pgfpathlineto{\pgfqpoint{4.876834in}{5.235423in}}%
\pgfpathlineto{\pgfqpoint{4.880776in}{5.223551in}}%
\pgfpathlineto{\pgfqpoint{4.881282in}{5.225316in}}%
\pgfpathlineto{\pgfqpoint{4.882394in}{5.230508in}}%
\pgfpathlineto{\pgfqpoint{4.883001in}{5.229749in}}%
\pgfpathlineto{\pgfqpoint{4.883405in}{5.229599in}}%
\pgfpathlineto{\pgfqpoint{4.883809in}{5.230353in}}%
\pgfpathlineto{\pgfqpoint{4.885427in}{5.236316in}}%
\pgfpathlineto{\pgfqpoint{4.886134in}{5.234095in}}%
\pgfpathlineto{\pgfqpoint{4.887954in}{5.230201in}}%
\pgfpathlineto{\pgfqpoint{4.889066in}{5.222617in}}%
\pgfpathlineto{\pgfqpoint{4.889572in}{5.220974in}}%
\pgfpathlineto{\pgfqpoint{4.890178in}{5.222141in}}%
\pgfpathlineto{\pgfqpoint{4.891290in}{5.228918in}}%
\pgfpathlineto{\pgfqpoint{4.892503in}{5.236212in}}%
\pgfpathlineto{\pgfqpoint{4.893009in}{5.235400in}}%
\pgfpathlineto{\pgfqpoint{4.895132in}{5.228252in}}%
\pgfpathlineto{\pgfqpoint{4.895940in}{5.231209in}}%
\pgfpathlineto{\pgfqpoint{4.896648in}{5.232708in}}%
\pgfpathlineto{\pgfqpoint{4.897154in}{5.231808in}}%
\pgfpathlineto{\pgfqpoint{4.899277in}{5.226777in}}%
\pgfpathlineto{\pgfqpoint{4.899580in}{5.227018in}}%
\pgfpathlineto{\pgfqpoint{4.900793in}{5.230120in}}%
\pgfpathlineto{\pgfqpoint{4.901298in}{5.228398in}}%
\pgfpathlineto{\pgfqpoint{4.901905in}{5.226592in}}%
\pgfpathlineto{\pgfqpoint{4.902410in}{5.227939in}}%
\pgfpathlineto{\pgfqpoint{4.904028in}{5.232216in}}%
\pgfpathlineto{\pgfqpoint{4.904432in}{5.232033in}}%
\pgfpathlineto{\pgfqpoint{4.905039in}{5.232764in}}%
\pgfpathlineto{\pgfqpoint{4.906353in}{5.237225in}}%
\pgfpathlineto{\pgfqpoint{4.906859in}{5.235666in}}%
\pgfpathlineto{\pgfqpoint{4.908678in}{5.220557in}}%
\pgfpathlineto{\pgfqpoint{4.909588in}{5.223854in}}%
\pgfpathlineto{\pgfqpoint{4.912115in}{5.233019in}}%
\pgfpathlineto{\pgfqpoint{4.912520in}{5.232651in}}%
\pgfpathlineto{\pgfqpoint{4.913935in}{5.230215in}}%
\pgfpathlineto{\pgfqpoint{4.914339in}{5.231127in}}%
\pgfpathlineto{\pgfqpoint{4.915350in}{5.235528in}}%
\pgfpathlineto{\pgfqpoint{4.915856in}{5.232904in}}%
\pgfpathlineto{\pgfqpoint{4.916766in}{5.227299in}}%
\pgfpathlineto{\pgfqpoint{4.917372in}{5.228822in}}%
\pgfpathlineto{\pgfqpoint{4.918181in}{5.230628in}}%
\pgfpathlineto{\pgfqpoint{4.918687in}{5.229651in}}%
\pgfpathlineto{\pgfqpoint{4.919596in}{5.227576in}}%
\pgfpathlineto{\pgfqpoint{4.920102in}{5.228430in}}%
\pgfpathlineto{\pgfqpoint{4.920911in}{5.229250in}}%
\pgfpathlineto{\pgfqpoint{4.921416in}{5.228764in}}%
\pgfpathlineto{\pgfqpoint{4.922932in}{5.226279in}}%
\pgfpathlineto{\pgfqpoint{4.923438in}{5.227611in}}%
\pgfpathlineto{\pgfqpoint{4.924853in}{5.234801in}}%
\pgfpathlineto{\pgfqpoint{4.925561in}{5.232926in}}%
\pgfpathlineto{\pgfqpoint{4.926774in}{5.230304in}}%
\pgfpathlineto{\pgfqpoint{4.927178in}{5.230699in}}%
\pgfpathlineto{\pgfqpoint{4.928796in}{5.232207in}}%
\pgfpathlineto{\pgfqpoint{4.928998in}{5.232008in}}%
\pgfpathlineto{\pgfqpoint{4.929807in}{5.228360in}}%
\pgfpathlineto{\pgfqpoint{4.930717in}{5.224531in}}%
\pgfpathlineto{\pgfqpoint{4.931222in}{5.225763in}}%
\pgfpathlineto{\pgfqpoint{4.933749in}{5.230557in}}%
\pgfpathlineto{\pgfqpoint{4.934760in}{5.231616in}}%
\pgfpathlineto{\pgfqpoint{4.935872in}{5.233443in}}%
\pgfpathlineto{\pgfqpoint{4.936277in}{5.232796in}}%
\pgfpathlineto{\pgfqpoint{4.938703in}{5.225437in}}%
\pgfpathlineto{\pgfqpoint{4.939208in}{5.227031in}}%
\pgfpathlineto{\pgfqpoint{4.940422in}{5.231729in}}%
\pgfpathlineto{\pgfqpoint{4.941028in}{5.231338in}}%
\pgfpathlineto{\pgfqpoint{4.941736in}{5.232915in}}%
\pgfpathlineto{\pgfqpoint{4.942747in}{5.235218in}}%
\pgfpathlineto{\pgfqpoint{4.943252in}{5.234275in}}%
\pgfpathlineto{\pgfqpoint{4.945173in}{5.224063in}}%
\pgfpathlineto{\pgfqpoint{4.946184in}{5.226960in}}%
\pgfpathlineto{\pgfqpoint{4.948610in}{5.232416in}}%
\pgfpathlineto{\pgfqpoint{4.949419in}{5.231013in}}%
\pgfpathlineto{\pgfqpoint{4.951340in}{5.228847in}}%
\pgfpathlineto{\pgfqpoint{4.951744in}{5.229291in}}%
\pgfpathlineto{\pgfqpoint{4.953766in}{5.231504in}}%
\pgfpathlineto{\pgfqpoint{4.954271in}{5.231038in}}%
\pgfpathlineto{\pgfqpoint{4.955990in}{5.228299in}}%
\pgfpathlineto{\pgfqpoint{4.956698in}{5.229626in}}%
\pgfpathlineto{\pgfqpoint{4.958315in}{5.232892in}}%
\pgfpathlineto{\pgfqpoint{4.958720in}{5.232138in}}%
\pgfpathlineto{\pgfqpoint{4.961449in}{5.226866in}}%
\pgfpathlineto{\pgfqpoint{4.961550in}{5.226900in}}%
\pgfpathlineto{\pgfqpoint{4.962864in}{5.229126in}}%
\pgfpathlineto{\pgfqpoint{4.964785in}{5.232601in}}%
\pgfpathlineto{\pgfqpoint{4.965291in}{5.232054in}}%
\pgfpathlineto{\pgfqpoint{4.966807in}{5.227450in}}%
\pgfpathlineto{\pgfqpoint{4.968323in}{5.226015in}}%
\pgfpathlineto{\pgfqpoint{4.969031in}{5.226945in}}%
\pgfpathlineto{\pgfqpoint{4.970547in}{5.236301in}}%
\pgfpathlineto{\pgfqpoint{4.971457in}{5.233217in}}%
\pgfpathlineto{\pgfqpoint{4.972771in}{5.227564in}}%
\pgfpathlineto{\pgfqpoint{4.973277in}{5.228795in}}%
\pgfpathlineto{\pgfqpoint{4.974389in}{5.231712in}}%
\pgfpathlineto{\pgfqpoint{4.974894in}{5.231343in}}%
\pgfpathlineto{\pgfqpoint{4.976411in}{5.229199in}}%
\pgfpathlineto{\pgfqpoint{4.977017in}{5.228194in}}%
\pgfpathlineto{\pgfqpoint{4.977523in}{5.229119in}}%
\pgfpathlineto{\pgfqpoint{4.978231in}{5.230112in}}%
\pgfpathlineto{\pgfqpoint{4.978635in}{5.229232in}}%
\pgfpathlineto{\pgfqpoint{4.979949in}{5.224053in}}%
\pgfpathlineto{\pgfqpoint{4.980455in}{5.225766in}}%
\pgfpathlineto{\pgfqpoint{4.982476in}{5.231066in}}%
\pgfpathlineto{\pgfqpoint{4.983993in}{5.233899in}}%
\pgfpathlineto{\pgfqpoint{4.984802in}{5.236469in}}%
\pgfpathlineto{\pgfqpoint{4.985206in}{5.235112in}}%
\pgfpathlineto{\pgfqpoint{4.987632in}{5.226554in}}%
\pgfpathlineto{\pgfqpoint{4.988037in}{5.226358in}}%
\pgfpathlineto{\pgfqpoint{4.988441in}{5.227142in}}%
\pgfpathlineto{\pgfqpoint{4.990058in}{5.231049in}}%
\pgfpathlineto{\pgfqpoint{4.990564in}{5.230400in}}%
\pgfpathlineto{\pgfqpoint{4.992080in}{5.224899in}}%
\pgfpathlineto{\pgfqpoint{4.992889in}{5.227517in}}%
\pgfpathlineto{\pgfqpoint{4.994507in}{5.234871in}}%
\pgfpathlineto{\pgfqpoint{4.995113in}{5.233855in}}%
\pgfpathlineto{\pgfqpoint{4.998045in}{5.225860in}}%
\pgfpathlineto{\pgfqpoint{4.998449in}{5.225166in}}%
\pgfpathlineto{\pgfqpoint{4.998955in}{5.226292in}}%
\pgfpathlineto{\pgfqpoint{5.000370in}{5.229450in}}%
\pgfpathlineto{\pgfqpoint{5.000774in}{5.229139in}}%
\pgfpathlineto{\pgfqpoint{5.001785in}{5.228292in}}%
\pgfpathlineto{\pgfqpoint{5.002190in}{5.229059in}}%
\pgfpathlineto{\pgfqpoint{5.003403in}{5.233416in}}%
\pgfpathlineto{\pgfqpoint{5.004009in}{5.231738in}}%
\pgfpathlineto{\pgfqpoint{5.005222in}{5.226803in}}%
\pgfpathlineto{\pgfqpoint{5.005728in}{5.228439in}}%
\pgfpathlineto{\pgfqpoint{5.006739in}{5.231912in}}%
\pgfpathlineto{\pgfqpoint{5.007345in}{5.230996in}}%
\pgfpathlineto{\pgfqpoint{5.008356in}{5.230027in}}%
\pgfpathlineto{\pgfqpoint{5.008761in}{5.230432in}}%
\pgfpathlineto{\pgfqpoint{5.009367in}{5.230626in}}%
\pgfpathlineto{\pgfqpoint{5.009671in}{5.230121in}}%
\pgfpathlineto{\pgfqpoint{5.010479in}{5.228815in}}%
\pgfpathlineto{\pgfqpoint{5.010985in}{5.229524in}}%
\pgfpathlineto{\pgfqpoint{5.013108in}{5.231196in}}%
\pgfpathlineto{\pgfqpoint{5.013917in}{5.230099in}}%
\pgfpathlineto{\pgfqpoint{5.014927in}{5.229150in}}%
\pgfpathlineto{\pgfqpoint{5.015332in}{5.229539in}}%
\pgfpathlineto{\pgfqpoint{5.016646in}{5.229470in}}%
\pgfpathlineto{\pgfqpoint{5.017657in}{5.230226in}}%
\pgfpathlineto{\pgfqpoint{5.018365in}{5.230042in}}%
\pgfpathlineto{\pgfqpoint{5.018668in}{5.229613in}}%
\pgfpathlineto{\pgfqpoint{5.021600in}{5.226187in}}%
\pgfpathlineto{\pgfqpoint{5.022004in}{5.226928in}}%
\pgfpathlineto{\pgfqpoint{5.023824in}{5.235665in}}%
\pgfpathlineto{\pgfqpoint{5.024936in}{5.233365in}}%
\pgfpathlineto{\pgfqpoint{5.029081in}{5.223917in}}%
\pgfpathlineto{\pgfqpoint{5.029485in}{5.225129in}}%
\pgfpathlineto{\pgfqpoint{5.031203in}{5.236083in}}%
\pgfpathlineto{\pgfqpoint{5.032012in}{5.233750in}}%
\pgfpathlineto{\pgfqpoint{5.034135in}{5.227852in}}%
\pgfpathlineto{\pgfqpoint{5.034337in}{5.227956in}}%
\pgfpathlineto{\pgfqpoint{5.037168in}{5.231734in}}%
\pgfpathlineto{\pgfqpoint{5.037775in}{5.230000in}}%
\pgfpathlineto{\pgfqpoint{5.038786in}{5.227340in}}%
\pgfpathlineto{\pgfqpoint{5.039291in}{5.228377in}}%
\pgfpathlineto{\pgfqpoint{5.040403in}{5.231853in}}%
\pgfpathlineto{\pgfqpoint{5.040908in}{5.230776in}}%
\pgfpathlineto{\pgfqpoint{5.041818in}{5.227490in}}%
\pgfpathlineto{\pgfqpoint{5.042324in}{5.229277in}}%
\pgfpathlineto{\pgfqpoint{5.043031in}{5.231938in}}%
\pgfpathlineto{\pgfqpoint{5.043638in}{5.230357in}}%
\pgfpathlineto{\pgfqpoint{5.044042in}{5.229576in}}%
\pgfpathlineto{\pgfqpoint{5.044548in}{5.230695in}}%
\pgfpathlineto{\pgfqpoint{5.045458in}{5.232485in}}%
\pgfpathlineto{\pgfqpoint{5.045963in}{5.231905in}}%
\pgfpathlineto{\pgfqpoint{5.047176in}{5.227256in}}%
\pgfpathlineto{\pgfqpoint{5.047884in}{5.225524in}}%
\pgfpathlineto{\pgfqpoint{5.048389in}{5.226434in}}%
\pgfpathlineto{\pgfqpoint{5.051321in}{5.234293in}}%
\pgfpathlineto{\pgfqpoint{5.051827in}{5.233117in}}%
\pgfpathlineto{\pgfqpoint{5.053444in}{5.228328in}}%
\pgfpathlineto{\pgfqpoint{5.053848in}{5.229170in}}%
\pgfpathlineto{\pgfqpoint{5.054859in}{5.232396in}}%
\pgfpathlineto{\pgfqpoint{5.055365in}{5.230681in}}%
\pgfpathlineto{\pgfqpoint{5.056477in}{5.226700in}}%
\pgfpathlineto{\pgfqpoint{5.057083in}{5.227009in}}%
\pgfpathlineto{\pgfqpoint{5.058094in}{5.228399in}}%
\pgfpathlineto{\pgfqpoint{5.059611in}{5.233280in}}%
\pgfpathlineto{\pgfqpoint{5.060116in}{5.232539in}}%
\pgfpathlineto{\pgfqpoint{5.062138in}{5.226636in}}%
\pgfpathlineto{\pgfqpoint{5.062947in}{5.228424in}}%
\pgfpathlineto{\pgfqpoint{5.064362in}{5.232144in}}%
\pgfpathlineto{\pgfqpoint{5.064868in}{5.231643in}}%
\pgfpathlineto{\pgfqpoint{5.067092in}{5.229147in}}%
\pgfpathlineto{\pgfqpoint{5.067395in}{5.229529in}}%
\pgfpathlineto{\pgfqpoint{5.069114in}{5.232789in}}%
\pgfpathlineto{\pgfqpoint{5.069619in}{5.231841in}}%
\pgfpathlineto{\pgfqpoint{5.072146in}{5.223794in}}%
\pgfpathlineto{\pgfqpoint{5.072652in}{5.225453in}}%
\pgfpathlineto{\pgfqpoint{5.075179in}{5.234540in}}%
\pgfpathlineto{\pgfqpoint{5.075280in}{5.234474in}}%
\pgfpathlineto{\pgfqpoint{5.076190in}{5.231339in}}%
\pgfpathlineto{\pgfqpoint{5.076898in}{5.230172in}}%
\pgfpathlineto{\pgfqpoint{5.077504in}{5.230503in}}%
\pgfpathlineto{\pgfqpoint{5.079627in}{5.228779in}}%
\pgfpathlineto{\pgfqpoint{5.080638in}{5.226127in}}%
\pgfpathlineto{\pgfqpoint{5.081245in}{5.227491in}}%
\pgfpathlineto{\pgfqpoint{5.083267in}{5.231934in}}%
\pgfpathlineto{\pgfqpoint{5.083469in}{5.231736in}}%
\pgfpathlineto{\pgfqpoint{5.084682in}{5.229099in}}%
\pgfpathlineto{\pgfqpoint{5.085288in}{5.230705in}}%
\pgfpathlineto{\pgfqpoint{5.085996in}{5.232323in}}%
\pgfpathlineto{\pgfqpoint{5.086502in}{5.231491in}}%
\pgfpathlineto{\pgfqpoint{5.087917in}{5.227738in}}%
\pgfpathlineto{\pgfqpoint{5.088422in}{5.228774in}}%
\pgfpathlineto{\pgfqpoint{5.090040in}{5.231855in}}%
\pgfpathlineto{\pgfqpoint{5.090444in}{5.231447in}}%
\pgfpathlineto{\pgfqpoint{5.092365in}{5.230550in}}%
\pgfpathlineto{\pgfqpoint{5.093477in}{5.227447in}}%
\pgfpathlineto{\pgfqpoint{5.094084in}{5.226570in}}%
\pgfpathlineto{\pgfqpoint{5.094791in}{5.226978in}}%
\pgfpathlineto{\pgfqpoint{5.096510in}{5.227587in}}%
\pgfpathlineto{\pgfqpoint{5.100149in}{5.236110in}}%
\pgfpathlineto{\pgfqpoint{5.100554in}{5.235486in}}%
\pgfpathlineto{\pgfqpoint{5.102070in}{5.228635in}}%
\pgfpathlineto{\pgfqpoint{5.103182in}{5.224418in}}%
\pgfpathlineto{\pgfqpoint{5.103687in}{5.224850in}}%
\pgfpathlineto{\pgfqpoint{5.104698in}{5.224038in}}%
\pgfpathlineto{\pgfqpoint{5.105204in}{5.224636in}}%
\pgfpathlineto{\pgfqpoint{5.105305in}{5.225007in}}%
\pgfpathlineto{\pgfqpoint{5.106821in}{5.231599in}}%
\pgfpathlineto{\pgfqpoint{5.107529in}{5.230406in}}%
\pgfpathlineto{\pgfqpoint{5.108136in}{5.229897in}}%
\pgfpathlineto{\pgfqpoint{5.108641in}{5.230573in}}%
\pgfpathlineto{\pgfqpoint{5.110764in}{5.236353in}}%
\pgfpathlineto{\pgfqpoint{5.111472in}{5.233821in}}%
\pgfpathlineto{\pgfqpoint{5.113797in}{5.226644in}}%
\pgfpathlineto{\pgfqpoint{5.114909in}{5.227467in}}%
\pgfpathlineto{\pgfqpoint{5.116324in}{5.230528in}}%
\pgfpathlineto{\pgfqpoint{5.116830in}{5.230166in}}%
\pgfpathlineto{\pgfqpoint{5.118245in}{5.228429in}}%
\pgfpathlineto{\pgfqpoint{5.118750in}{5.229459in}}%
\pgfpathlineto{\pgfqpoint{5.119761in}{5.231817in}}%
\pgfpathlineto{\pgfqpoint{5.120267in}{5.230845in}}%
\pgfpathlineto{\pgfqpoint{5.121177in}{5.228167in}}%
\pgfpathlineto{\pgfqpoint{5.121783in}{5.229450in}}%
\pgfpathlineto{\pgfqpoint{5.123704in}{5.233585in}}%
\pgfpathlineto{\pgfqpoint{5.124007in}{5.233316in}}%
\pgfpathlineto{\pgfqpoint{5.126838in}{5.226851in}}%
\pgfpathlineto{\pgfqpoint{5.127546in}{5.228211in}}%
\pgfpathlineto{\pgfqpoint{5.128152in}{5.229300in}}%
\pgfpathlineto{\pgfqpoint{5.128658in}{5.228336in}}%
\pgfpathlineto{\pgfqpoint{5.129163in}{5.227335in}}%
\pgfpathlineto{\pgfqpoint{5.129669in}{5.228752in}}%
\pgfpathlineto{\pgfqpoint{5.130578in}{5.231747in}}%
\pgfpathlineto{\pgfqpoint{5.131084in}{5.230715in}}%
\pgfpathlineto{\pgfqpoint{5.132600in}{5.227852in}}%
\pgfpathlineto{\pgfqpoint{5.132903in}{5.228303in}}%
\pgfpathlineto{\pgfqpoint{5.134420in}{5.235423in}}%
\pgfpathlineto{\pgfqpoint{5.135330in}{5.232651in}}%
\pgfpathlineto{\pgfqpoint{5.137554in}{5.228607in}}%
\pgfpathlineto{\pgfqpoint{5.138969in}{5.226204in}}%
\pgfpathlineto{\pgfqpoint{5.139272in}{5.226558in}}%
\pgfpathlineto{\pgfqpoint{5.140890in}{5.230965in}}%
\pgfpathlineto{\pgfqpoint{5.141699in}{5.229723in}}%
\pgfpathlineto{\pgfqpoint{5.143316in}{5.227568in}}%
\pgfpathlineto{\pgfqpoint{5.143720in}{5.228190in}}%
\pgfpathlineto{\pgfqpoint{5.145237in}{5.233244in}}%
\pgfpathlineto{\pgfqpoint{5.145945in}{5.231501in}}%
\pgfpathlineto{\pgfqpoint{5.146854in}{5.228960in}}%
\pgfpathlineto{\pgfqpoint{5.147360in}{5.229849in}}%
\pgfpathlineto{\pgfqpoint{5.148674in}{5.233746in}}%
\pgfpathlineto{\pgfqpoint{5.149180in}{5.232709in}}%
\pgfpathlineto{\pgfqpoint{5.151606in}{5.226135in}}%
\pgfpathlineto{\pgfqpoint{5.151808in}{5.226285in}}%
\pgfpathlineto{\pgfqpoint{5.155144in}{5.232623in}}%
\pgfpathlineto{\pgfqpoint{5.156559in}{5.236234in}}%
\pgfpathlineto{\pgfqpoint{5.156964in}{5.235782in}}%
\pgfpathlineto{\pgfqpoint{5.157975in}{5.231242in}}%
\pgfpathlineto{\pgfqpoint{5.159794in}{5.223011in}}%
\pgfpathlineto{\pgfqpoint{5.160199in}{5.223272in}}%
\pgfpathlineto{\pgfqpoint{5.161412in}{5.226328in}}%
\pgfpathlineto{\pgfqpoint{5.165253in}{5.237595in}}%
\pgfpathlineto{\pgfqpoint{5.165557in}{5.237842in}}%
\pgfpathlineto{\pgfqpoint{5.165961in}{5.236735in}}%
\pgfpathlineto{\pgfqpoint{5.168589in}{5.224467in}}%
\pgfpathlineto{\pgfqpoint{5.169095in}{5.225454in}}%
\pgfpathlineto{\pgfqpoint{5.171926in}{5.231881in}}%
\pgfpathlineto{\pgfqpoint{5.175363in}{5.228075in}}%
\pgfpathlineto{\pgfqpoint{5.175969in}{5.227163in}}%
\pgfpathlineto{\pgfqpoint{5.176475in}{5.228102in}}%
\pgfpathlineto{\pgfqpoint{5.178396in}{5.233843in}}%
\pgfpathlineto{\pgfqpoint{5.178901in}{5.232860in}}%
\pgfpathlineto{\pgfqpoint{5.181024in}{5.227232in}}%
\pgfpathlineto{\pgfqpoint{5.181529in}{5.227646in}}%
\pgfpathlineto{\pgfqpoint{5.183248in}{5.230619in}}%
\pgfpathlineto{\pgfqpoint{5.183956in}{5.229519in}}%
\pgfpathlineto{\pgfqpoint{5.184764in}{5.228181in}}%
\pgfpathlineto{\pgfqpoint{5.185169in}{5.228967in}}%
\pgfpathlineto{\pgfqpoint{5.186382in}{5.232661in}}%
\pgfpathlineto{\pgfqpoint{5.187090in}{5.232117in}}%
\pgfpathlineto{\pgfqpoint{5.188202in}{5.229737in}}%
\pgfpathlineto{\pgfqpoint{5.189111in}{5.228126in}}%
\pgfpathlineto{\pgfqpoint{5.189617in}{5.228689in}}%
\pgfpathlineto{\pgfqpoint{5.190223in}{5.229133in}}%
\pgfpathlineto{\pgfqpoint{5.190628in}{5.228390in}}%
\pgfpathlineto{\pgfqpoint{5.191639in}{5.226241in}}%
\pgfpathlineto{\pgfqpoint{5.192144in}{5.227128in}}%
\pgfpathlineto{\pgfqpoint{5.193357in}{5.233773in}}%
\pgfpathlineto{\pgfqpoint{5.194065in}{5.237537in}}%
\pgfpathlineto{\pgfqpoint{5.194570in}{5.235527in}}%
\pgfpathlineto{\pgfqpoint{5.196795in}{5.227261in}}%
\pgfpathlineto{\pgfqpoint{5.197300in}{5.227577in}}%
\pgfpathlineto{\pgfqpoint{5.197502in}{5.228035in}}%
\pgfpathlineto{\pgfqpoint{5.198715in}{5.230091in}}%
\pgfpathlineto{\pgfqpoint{5.199221in}{5.229587in}}%
\pgfpathlineto{\pgfqpoint{5.201040in}{5.227667in}}%
\pgfpathlineto{\pgfqpoint{5.201445in}{5.228323in}}%
\pgfpathlineto{\pgfqpoint{5.202658in}{5.231143in}}%
\pgfpathlineto{\pgfqpoint{5.203366in}{5.230634in}}%
\pgfpathlineto{\pgfqpoint{5.204478in}{5.231511in}}%
\pgfpathlineto{\pgfqpoint{5.204882in}{5.230490in}}%
\pgfpathlineto{\pgfqpoint{5.205590in}{5.229333in}}%
\pgfpathlineto{\pgfqpoint{5.206095in}{5.229981in}}%
\pgfpathlineto{\pgfqpoint{5.207713in}{5.232162in}}%
\pgfpathlineto{\pgfqpoint{5.208117in}{5.231432in}}%
\pgfpathlineto{\pgfqpoint{5.210341in}{5.228140in}}%
\pgfpathlineto{\pgfqpoint{5.210442in}{5.228193in}}%
\pgfpathlineto{\pgfqpoint{5.214082in}{5.229308in}}%
\pgfpathlineto{\pgfqpoint{5.215295in}{5.229614in}}%
\pgfpathlineto{\pgfqpoint{5.216912in}{5.230556in}}%
\pgfpathlineto{\pgfqpoint{5.218226in}{5.234302in}}%
\pgfpathlineto{\pgfqpoint{5.218732in}{5.233474in}}%
\pgfpathlineto{\pgfqpoint{5.221562in}{5.227136in}}%
\pgfpathlineto{\pgfqpoint{5.221664in}{5.227168in}}%
\pgfpathlineto{\pgfqpoint{5.222371in}{5.229068in}}%
\pgfpathlineto{\pgfqpoint{5.223382in}{5.231668in}}%
\pgfpathlineto{\pgfqpoint{5.223888in}{5.230852in}}%
\pgfpathlineto{\pgfqpoint{5.225505in}{5.228407in}}%
\pgfpathlineto{\pgfqpoint{5.225909in}{5.228714in}}%
\pgfpathlineto{\pgfqpoint{5.230256in}{5.231982in}}%
\pgfpathlineto{\pgfqpoint{5.231065in}{5.231179in}}%
\pgfpathlineto{\pgfqpoint{5.232177in}{5.229372in}}%
\pgfpathlineto{\pgfqpoint{5.232683in}{5.230059in}}%
\pgfpathlineto{\pgfqpoint{5.233188in}{5.230400in}}%
\pgfpathlineto{\pgfqpoint{5.233593in}{5.229685in}}%
\pgfpathlineto{\pgfqpoint{5.235311in}{5.226082in}}%
\pgfpathlineto{\pgfqpoint{5.235716in}{5.226585in}}%
\pgfpathlineto{\pgfqpoint{5.238445in}{5.233370in}}%
\pgfpathlineto{\pgfqpoint{5.238849in}{5.232357in}}%
\pgfpathlineto{\pgfqpoint{5.240164in}{5.225353in}}%
\pgfpathlineto{\pgfqpoint{5.240770in}{5.227640in}}%
\pgfpathlineto{\pgfqpoint{5.242792in}{5.233001in}}%
\pgfpathlineto{\pgfqpoint{5.242893in}{5.232957in}}%
\pgfpathlineto{\pgfqpoint{5.243702in}{5.230590in}}%
\pgfpathlineto{\pgfqpoint{5.244410in}{5.229277in}}%
\pgfpathlineto{\pgfqpoint{5.244915in}{5.230033in}}%
\pgfpathlineto{\pgfqpoint{5.246330in}{5.232660in}}%
\pgfpathlineto{\pgfqpoint{5.246836in}{5.231539in}}%
\pgfpathlineto{\pgfqpoint{5.248757in}{5.224680in}}%
\pgfpathlineto{\pgfqpoint{5.249262in}{5.225615in}}%
\pgfpathlineto{\pgfqpoint{5.251082in}{5.232179in}}%
\pgfpathlineto{\pgfqpoint{5.251890in}{5.231153in}}%
\pgfpathlineto{\pgfqpoint{5.253710in}{5.230706in}}%
\pgfpathlineto{\pgfqpoint{5.255024in}{5.232899in}}%
\pgfpathlineto{\pgfqpoint{5.255631in}{5.231368in}}%
\pgfpathlineto{\pgfqpoint{5.257451in}{5.225159in}}%
\pgfpathlineto{\pgfqpoint{5.258057in}{5.226362in}}%
\pgfpathlineto{\pgfqpoint{5.260787in}{5.234188in}}%
\pgfpathlineto{\pgfqpoint{5.261090in}{5.233683in}}%
\pgfpathlineto{\pgfqpoint{5.262809in}{5.228389in}}%
\pgfpathlineto{\pgfqpoint{5.263516in}{5.229315in}}%
\pgfpathlineto{\pgfqpoint{5.264325in}{5.230308in}}%
\pgfpathlineto{\pgfqpoint{5.264729in}{5.229673in}}%
\pgfpathlineto{\pgfqpoint{5.266246in}{5.226311in}}%
\pgfpathlineto{\pgfqpoint{5.266751in}{5.227170in}}%
\pgfpathlineto{\pgfqpoint{5.269582in}{5.233634in}}%
\pgfpathlineto{\pgfqpoint{5.269885in}{5.233410in}}%
\pgfpathlineto{\pgfqpoint{5.271401in}{5.230607in}}%
\pgfpathlineto{\pgfqpoint{5.272615in}{5.225349in}}%
\pgfpathlineto{\pgfqpoint{5.273120in}{5.226570in}}%
\pgfpathlineto{\pgfqpoint{5.275142in}{5.229806in}}%
\pgfpathlineto{\pgfqpoint{5.277164in}{5.231369in}}%
\pgfpathlineto{\pgfqpoint{5.277973in}{5.232764in}}%
\pgfpathlineto{\pgfqpoint{5.278478in}{5.232131in}}%
\pgfpathlineto{\pgfqpoint{5.280904in}{5.227147in}}%
\pgfpathlineto{\pgfqpoint{5.282016in}{5.228711in}}%
\pgfpathlineto{\pgfqpoint{5.283836in}{5.230187in}}%
\pgfpathlineto{\pgfqpoint{5.285656in}{5.233385in}}%
\pgfpathlineto{\pgfqpoint{5.286161in}{5.232575in}}%
\pgfpathlineto{\pgfqpoint{5.288790in}{5.225277in}}%
\pgfpathlineto{\pgfqpoint{5.289497in}{5.226880in}}%
\pgfpathlineto{\pgfqpoint{5.291115in}{5.232540in}}%
\pgfpathlineto{\pgfqpoint{5.291822in}{5.231807in}}%
\pgfpathlineto{\pgfqpoint{5.292530in}{5.231985in}}%
\pgfpathlineto{\pgfqpoint{5.292833in}{5.232405in}}%
\pgfpathlineto{\pgfqpoint{5.293440in}{5.232670in}}%
\pgfpathlineto{\pgfqpoint{5.293844in}{5.231994in}}%
\pgfpathlineto{\pgfqpoint{5.296068in}{5.225460in}}%
\pgfpathlineto{\pgfqpoint{5.296675in}{5.226918in}}%
\pgfpathlineto{\pgfqpoint{5.299000in}{5.234148in}}%
\pgfpathlineto{\pgfqpoint{5.299505in}{5.233317in}}%
\pgfpathlineto{\pgfqpoint{5.302943in}{5.224947in}}%
\pgfpathlineto{\pgfqpoint{5.303145in}{5.225210in}}%
\pgfpathlineto{\pgfqpoint{5.304156in}{5.230526in}}%
\pgfpathlineto{\pgfqpoint{5.305470in}{5.235286in}}%
\pgfpathlineto{\pgfqpoint{5.305874in}{5.234751in}}%
\pgfpathlineto{\pgfqpoint{5.309210in}{5.226959in}}%
\pgfpathlineto{\pgfqpoint{5.309615in}{5.227478in}}%
\pgfpathlineto{\pgfqpoint{5.310727in}{5.229410in}}%
\pgfpathlineto{\pgfqpoint{5.311333in}{5.228689in}}%
\pgfpathlineto{\pgfqpoint{5.312041in}{5.228378in}}%
\pgfpathlineto{\pgfqpoint{5.312445in}{5.228976in}}%
\pgfpathlineto{\pgfqpoint{5.315175in}{5.232364in}}%
\pgfpathlineto{\pgfqpoint{5.316287in}{5.231937in}}%
\pgfpathlineto{\pgfqpoint{5.318511in}{5.227642in}}%
\pgfpathlineto{\pgfqpoint{5.319320in}{5.229337in}}%
\pgfpathlineto{\pgfqpoint{5.320432in}{5.231207in}}%
\pgfpathlineto{\pgfqpoint{5.320937in}{5.230829in}}%
\pgfpathlineto{\pgfqpoint{5.322150in}{5.227497in}}%
\pgfpathlineto{\pgfqpoint{5.322757in}{5.226305in}}%
\pgfpathlineto{\pgfqpoint{5.323161in}{5.227425in}}%
\pgfpathlineto{\pgfqpoint{5.324273in}{5.232190in}}%
\pgfpathlineto{\pgfqpoint{5.324880in}{5.230666in}}%
\pgfpathlineto{\pgfqpoint{5.325588in}{5.229342in}}%
\pgfpathlineto{\pgfqpoint{5.326093in}{5.230349in}}%
\pgfpathlineto{\pgfqpoint{5.326801in}{5.231426in}}%
\pgfpathlineto{\pgfqpoint{5.327306in}{5.230554in}}%
\pgfpathlineto{\pgfqpoint{5.328317in}{5.228970in}}%
\pgfpathlineto{\pgfqpoint{5.328721in}{5.229513in}}%
\pgfpathlineto{\pgfqpoint{5.329935in}{5.230850in}}%
\pgfpathlineto{\pgfqpoint{5.330339in}{5.230403in}}%
\pgfpathlineto{\pgfqpoint{5.331451in}{5.229000in}}%
\pgfpathlineto{\pgfqpoint{5.331956in}{5.229715in}}%
\pgfpathlineto{\pgfqpoint{5.333271in}{5.232050in}}%
\pgfpathlineto{\pgfqpoint{5.333776in}{5.231235in}}%
\pgfpathlineto{\pgfqpoint{5.335495in}{5.225723in}}%
\pgfpathlineto{\pgfqpoint{5.336101in}{5.227709in}}%
\pgfpathlineto{\pgfqpoint{5.337112in}{5.231492in}}%
\pgfpathlineto{\pgfqpoint{5.337618in}{5.230529in}}%
\pgfpathlineto{\pgfqpoint{5.339134in}{5.228984in}}%
\pgfpathlineto{\pgfqpoint{5.339336in}{5.229105in}}%
\pgfpathlineto{\pgfqpoint{5.341965in}{5.231648in}}%
\pgfpathlineto{\pgfqpoint{5.342571in}{5.231077in}}%
\pgfpathlineto{\pgfqpoint{5.345806in}{5.227749in}}%
\pgfpathlineto{\pgfqpoint{5.347120in}{5.225059in}}%
\pgfpathlineto{\pgfqpoint{5.347525in}{5.225817in}}%
\pgfpathlineto{\pgfqpoint{5.350457in}{5.233629in}}%
\pgfpathlineto{\pgfqpoint{5.350760in}{5.233017in}}%
\pgfpathlineto{\pgfqpoint{5.351872in}{5.228604in}}%
\pgfpathlineto{\pgfqpoint{5.352478in}{5.230251in}}%
\pgfpathlineto{\pgfqpoint{5.353793in}{5.234095in}}%
\pgfpathlineto{\pgfqpoint{5.354197in}{5.233432in}}%
\pgfpathlineto{\pgfqpoint{5.356320in}{5.224101in}}%
\pgfpathlineto{\pgfqpoint{5.357331in}{5.227007in}}%
\pgfpathlineto{\pgfqpoint{5.359353in}{5.230740in}}%
\pgfpathlineto{\pgfqpoint{5.360768in}{5.232941in}}%
\pgfpathlineto{\pgfqpoint{5.361577in}{5.232276in}}%
\pgfpathlineto{\pgfqpoint{5.362790in}{5.230911in}}%
\pgfpathlineto{\pgfqpoint{5.364509in}{5.226346in}}%
\pgfpathlineto{\pgfqpoint{5.365115in}{5.227433in}}%
\pgfpathlineto{\pgfqpoint{5.366632in}{5.230303in}}%
\pgfpathlineto{\pgfqpoint{5.367137in}{5.229920in}}%
\pgfpathlineto{\pgfqpoint{5.368552in}{5.229271in}}%
\pgfpathlineto{\pgfqpoint{5.368754in}{5.229596in}}%
\pgfpathlineto{\pgfqpoint{5.370170in}{5.232751in}}%
\pgfpathlineto{\pgfqpoint{5.370776in}{5.231858in}}%
\pgfpathlineto{\pgfqpoint{5.372798in}{5.229952in}}%
\pgfpathlineto{\pgfqpoint{5.374517in}{5.229506in}}%
\pgfpathlineto{\pgfqpoint{5.376033in}{5.225919in}}%
\pgfpathlineto{\pgfqpoint{5.376640in}{5.226955in}}%
\pgfpathlineto{\pgfqpoint{5.379470in}{5.232566in}}%
\pgfpathlineto{\pgfqpoint{5.379673in}{5.232446in}}%
\pgfpathlineto{\pgfqpoint{5.381189in}{5.229469in}}%
\pgfpathlineto{\pgfqpoint{5.381998in}{5.229049in}}%
\pgfpathlineto{\pgfqpoint{5.382402in}{5.229477in}}%
\pgfpathlineto{\pgfqpoint{5.383716in}{5.230969in}}%
\pgfpathlineto{\pgfqpoint{5.384121in}{5.230215in}}%
\pgfpathlineto{\pgfqpoint{5.385839in}{5.225765in}}%
\pgfpathlineto{\pgfqpoint{5.386345in}{5.226811in}}%
\pgfpathlineto{\pgfqpoint{5.388063in}{5.234458in}}%
\pgfpathlineto{\pgfqpoint{5.388771in}{5.232406in}}%
\pgfpathlineto{\pgfqpoint{5.389580in}{5.230795in}}%
\pgfpathlineto{\pgfqpoint{5.390186in}{5.231355in}}%
\pgfpathlineto{\pgfqpoint{5.390894in}{5.231068in}}%
\pgfpathlineto{\pgfqpoint{5.391096in}{5.230663in}}%
\pgfpathlineto{\pgfqpoint{5.392511in}{5.227617in}}%
\pgfpathlineto{\pgfqpoint{5.393118in}{5.228305in}}%
\pgfpathlineto{\pgfqpoint{5.393927in}{5.228683in}}%
\pgfpathlineto{\pgfqpoint{5.394331in}{5.228255in}}%
\pgfpathlineto{\pgfqpoint{5.395645in}{5.227851in}}%
\pgfpathlineto{\pgfqpoint{5.395848in}{5.228102in}}%
\pgfpathlineto{\pgfqpoint{5.398577in}{5.232196in}}%
\pgfpathlineto{\pgfqpoint{5.398981in}{5.231748in}}%
\pgfpathlineto{\pgfqpoint{5.400902in}{5.230556in}}%
\pgfpathlineto{\pgfqpoint{5.402014in}{5.229737in}}%
\pgfpathlineto{\pgfqpoint{5.403227in}{5.227702in}}%
\pgfpathlineto{\pgfqpoint{5.403733in}{5.228378in}}%
\pgfpathlineto{\pgfqpoint{5.405856in}{5.230576in}}%
\pgfpathlineto{\pgfqpoint{5.406058in}{5.230427in}}%
\pgfpathlineto{\pgfqpoint{5.407878in}{5.228443in}}%
\pgfpathlineto{\pgfqpoint{5.408383in}{5.229339in}}%
\pgfpathlineto{\pgfqpoint{5.410102in}{5.232339in}}%
\pgfpathlineto{\pgfqpoint{5.410405in}{5.232118in}}%
\pgfpathlineto{\pgfqpoint{5.413842in}{5.226639in}}%
\pgfpathlineto{\pgfqpoint{5.414449in}{5.226238in}}%
\pgfpathlineto{\pgfqpoint{5.414954in}{5.226922in}}%
\pgfpathlineto{\pgfqpoint{5.418189in}{5.232636in}}%
\pgfpathlineto{\pgfqpoint{5.418594in}{5.232271in}}%
\pgfpathlineto{\pgfqpoint{5.420312in}{5.229819in}}%
\pgfpathlineto{\pgfqpoint{5.420919in}{5.230768in}}%
\pgfpathlineto{\pgfqpoint{5.421829in}{5.231536in}}%
\pgfpathlineto{\pgfqpoint{5.422233in}{5.230921in}}%
\pgfpathlineto{\pgfqpoint{5.424154in}{5.226957in}}%
\pgfpathlineto{\pgfqpoint{5.424659in}{5.227661in}}%
\pgfpathlineto{\pgfqpoint{5.426176in}{5.231006in}}%
\pgfpathlineto{\pgfqpoint{5.426782in}{5.230610in}}%
\pgfpathlineto{\pgfqpoint{5.427995in}{5.228553in}}%
\pgfpathlineto{\pgfqpoint{5.428703in}{5.227553in}}%
\pgfpathlineto{\pgfqpoint{5.429107in}{5.228240in}}%
\pgfpathlineto{\pgfqpoint{5.430624in}{5.232155in}}%
\pgfpathlineto{\pgfqpoint{5.431331in}{5.231615in}}%
\pgfpathlineto{\pgfqpoint{5.436386in}{5.228967in}}%
\pgfpathlineto{\pgfqpoint{5.438307in}{5.230247in}}%
\pgfpathlineto{\pgfqpoint{5.440025in}{5.231531in}}%
\pgfpathlineto{\pgfqpoint{5.440329in}{5.231125in}}%
\pgfpathlineto{\pgfqpoint{5.443159in}{5.227500in}}%
\pgfpathlineto{\pgfqpoint{5.443463in}{5.227797in}}%
\pgfpathlineto{\pgfqpoint{5.446192in}{5.232279in}}%
\pgfpathlineto{\pgfqpoint{5.446900in}{5.231427in}}%
\pgfpathlineto{\pgfqpoint{5.449023in}{5.228719in}}%
\pgfpathlineto{\pgfqpoint{5.449528in}{5.229387in}}%
\pgfpathlineto{\pgfqpoint{5.450034in}{5.229739in}}%
\pgfpathlineto{\pgfqpoint{5.450438in}{5.229126in}}%
\pgfpathlineto{\pgfqpoint{5.451348in}{5.227425in}}%
\pgfpathlineto{\pgfqpoint{5.451752in}{5.228414in}}%
\pgfpathlineto{\pgfqpoint{5.453066in}{5.232915in}}%
\pgfpathlineto{\pgfqpoint{5.453673in}{5.231899in}}%
\pgfpathlineto{\pgfqpoint{5.455189in}{5.228599in}}%
\pgfpathlineto{\pgfqpoint{5.455695in}{5.229241in}}%
\pgfpathlineto{\pgfqpoint{5.456605in}{5.229946in}}%
\pgfpathlineto{\pgfqpoint{5.457009in}{5.229509in}}%
\pgfpathlineto{\pgfqpoint{5.458222in}{5.229023in}}%
\pgfpathlineto{\pgfqpoint{5.458525in}{5.229282in}}%
\pgfpathlineto{\pgfqpoint{5.460851in}{5.231387in}}%
\pgfpathlineto{\pgfqpoint{5.461255in}{5.230789in}}%
\pgfpathlineto{\pgfqpoint{5.463580in}{5.228926in}}%
\pgfpathlineto{\pgfqpoint{5.467017in}{5.229367in}}%
\pgfpathlineto{\pgfqpoint{5.468736in}{5.232552in}}%
\pgfpathlineto{\pgfqpoint{5.468938in}{5.232460in}}%
\pgfpathlineto{\pgfqpoint{5.472274in}{5.228582in}}%
\pgfpathlineto{\pgfqpoint{5.473386in}{5.226391in}}%
\pgfpathlineto{\pgfqpoint{5.473791in}{5.227118in}}%
\pgfpathlineto{\pgfqpoint{5.475812in}{5.229799in}}%
\pgfpathlineto{\pgfqpoint{5.476722in}{5.230874in}}%
\pgfpathlineto{\pgfqpoint{5.478036in}{5.233566in}}%
\pgfpathlineto{\pgfqpoint{5.478542in}{5.232743in}}%
\pgfpathlineto{\pgfqpoint{5.480261in}{5.227862in}}%
\pgfpathlineto{\pgfqpoint{5.480867in}{5.228668in}}%
\pgfpathlineto{\pgfqpoint{5.481878in}{5.229721in}}%
\pgfpathlineto{\pgfqpoint{5.482282in}{5.229303in}}%
\pgfpathlineto{\pgfqpoint{5.483091in}{5.228534in}}%
\pgfpathlineto{\pgfqpoint{5.483496in}{5.229174in}}%
\pgfpathlineto{\pgfqpoint{5.484608in}{5.231212in}}%
\pgfpathlineto{\pgfqpoint{5.485214in}{5.230547in}}%
\pgfpathlineto{\pgfqpoint{5.487135in}{5.229753in}}%
\pgfpathlineto{\pgfqpoint{5.489359in}{5.229095in}}%
\pgfpathlineto{\pgfqpoint{5.490471in}{5.229667in}}%
\pgfpathlineto{\pgfqpoint{5.492291in}{5.231502in}}%
\pgfpathlineto{\pgfqpoint{5.492594in}{5.231072in}}%
\pgfpathlineto{\pgfqpoint{5.494110in}{5.227951in}}%
\pgfpathlineto{\pgfqpoint{5.494717in}{5.228811in}}%
\pgfpathlineto{\pgfqpoint{5.496840in}{5.230464in}}%
\pgfpathlineto{\pgfqpoint{5.498255in}{5.229407in}}%
\pgfpathlineto{\pgfqpoint{5.498963in}{5.229128in}}%
\pgfpathlineto{\pgfqpoint{5.499367in}{5.229648in}}%
\pgfpathlineto{\pgfqpoint{5.501793in}{5.231681in}}%
\pgfpathlineto{\pgfqpoint{5.502501in}{5.230595in}}%
\pgfpathlineto{\pgfqpoint{5.504523in}{5.228247in}}%
\pgfpathlineto{\pgfqpoint{5.505938in}{5.229518in}}%
\pgfpathlineto{\pgfqpoint{5.511498in}{5.230969in}}%
\pgfpathlineto{\pgfqpoint{5.515441in}{5.228515in}}%
\pgfpathlineto{\pgfqpoint{5.515542in}{5.228622in}}%
\pgfpathlineto{\pgfqpoint{5.519384in}{5.231524in}}%
\pgfpathlineto{\pgfqpoint{5.521001in}{5.230024in}}%
\pgfpathlineto{\pgfqpoint{5.521810in}{5.229064in}}%
\pgfpathlineto{\pgfqpoint{5.522315in}{5.229781in}}%
\pgfpathlineto{\pgfqpoint{5.523023in}{5.230441in}}%
\pgfpathlineto{\pgfqpoint{5.523427in}{5.229872in}}%
\pgfpathlineto{\pgfqpoint{5.525550in}{5.228143in}}%
\pgfpathlineto{\pgfqpoint{5.526966in}{5.229683in}}%
\pgfpathlineto{\pgfqpoint{5.529190in}{5.233126in}}%
\pgfpathlineto{\pgfqpoint{5.529796in}{5.232329in}}%
\pgfpathlineto{\pgfqpoint{5.532829in}{5.226655in}}%
\pgfpathlineto{\pgfqpoint{5.533436in}{5.227264in}}%
\pgfpathlineto{\pgfqpoint{5.536671in}{5.232602in}}%
\pgfpathlineto{\pgfqpoint{5.537378in}{5.231725in}}%
\pgfpathlineto{\pgfqpoint{5.540209in}{5.228873in}}%
\pgfpathlineto{\pgfqpoint{5.540310in}{5.228951in}}%
\pgfpathlineto{\pgfqpoint{5.541624in}{5.229288in}}%
\pgfpathlineto{\pgfqpoint{5.541826in}{5.229042in}}%
\pgfpathlineto{\pgfqpoint{5.543141in}{5.227480in}}%
\pgfpathlineto{\pgfqpoint{5.543545in}{5.228313in}}%
\pgfpathlineto{\pgfqpoint{5.545163in}{5.232364in}}%
\pgfpathlineto{\pgfqpoint{5.545668in}{5.231686in}}%
\pgfpathlineto{\pgfqpoint{5.547791in}{5.228484in}}%
\pgfpathlineto{\pgfqpoint{5.548195in}{5.228762in}}%
\pgfpathlineto{\pgfqpoint{5.550015in}{5.231607in}}%
\pgfpathlineto{\pgfqpoint{5.550824in}{5.230147in}}%
\pgfpathlineto{\pgfqpoint{5.551329in}{5.229654in}}%
\pgfpathlineto{\pgfqpoint{5.551835in}{5.230432in}}%
\pgfpathlineto{\pgfqpoint{5.552643in}{5.231658in}}%
\pgfpathlineto{\pgfqpoint{5.553149in}{5.230975in}}%
\pgfpathlineto{\pgfqpoint{5.554867in}{5.226694in}}%
\pgfpathlineto{\pgfqpoint{5.555474in}{5.227806in}}%
\pgfpathlineto{\pgfqpoint{5.557294in}{5.231562in}}%
\pgfpathlineto{\pgfqpoint{5.557799in}{5.230919in}}%
\pgfpathlineto{\pgfqpoint{5.558810in}{5.229817in}}%
\pgfpathlineto{\pgfqpoint{5.559316in}{5.230268in}}%
\pgfpathlineto{\pgfqpoint{5.560630in}{5.230081in}}%
\pgfpathlineto{\pgfqpoint{5.565786in}{5.230421in}}%
\pgfpathlineto{\pgfqpoint{5.566999in}{5.231335in}}%
\pgfpathlineto{\pgfqpoint{5.567302in}{5.230893in}}%
\pgfpathlineto{\pgfqpoint{5.569223in}{5.227280in}}%
\pgfpathlineto{\pgfqpoint{5.569829in}{5.227940in}}%
\pgfpathlineto{\pgfqpoint{5.573165in}{5.231498in}}%
\pgfpathlineto{\pgfqpoint{5.573974in}{5.231276in}}%
\pgfpathlineto{\pgfqpoint{5.574176in}{5.230910in}}%
\pgfpathlineto{\pgfqpoint{5.576097in}{5.227968in}}%
\pgfpathlineto{\pgfqpoint{5.576501in}{5.228543in}}%
\pgfpathlineto{\pgfqpoint{5.577917in}{5.230904in}}%
\pgfpathlineto{\pgfqpoint{5.578422in}{5.230490in}}%
\pgfpathlineto{\pgfqpoint{5.580849in}{5.228424in}}%
\pgfpathlineto{\pgfqpoint{5.581152in}{5.228776in}}%
\pgfpathlineto{\pgfqpoint{5.583275in}{5.231123in}}%
\pgfpathlineto{\pgfqpoint{5.583679in}{5.230830in}}%
\pgfpathlineto{\pgfqpoint{5.584993in}{5.231244in}}%
\pgfpathlineto{\pgfqpoint{5.585802in}{5.231356in}}%
\pgfpathlineto{\pgfqpoint{5.586105in}{5.230877in}}%
\pgfpathlineto{\pgfqpoint{5.587824in}{5.227381in}}%
\pgfpathlineto{\pgfqpoint{5.588431in}{5.228308in}}%
\pgfpathlineto{\pgfqpoint{5.590553in}{5.230422in}}%
\pgfpathlineto{\pgfqpoint{5.590756in}{5.230262in}}%
\pgfpathlineto{\pgfqpoint{5.591969in}{5.230501in}}%
\pgfpathlineto{\pgfqpoint{5.592980in}{5.230557in}}%
\pgfpathlineto{\pgfqpoint{5.593182in}{5.230322in}}%
\pgfpathlineto{\pgfqpoint{5.594900in}{5.229145in}}%
\pgfpathlineto{\pgfqpoint{5.595204in}{5.229423in}}%
\pgfpathlineto{\pgfqpoint{5.597428in}{5.231847in}}%
\pgfpathlineto{\pgfqpoint{5.598034in}{5.230867in}}%
\pgfpathlineto{\pgfqpoint{5.600258in}{5.228979in}}%
\pgfpathlineto{\pgfqpoint{5.603089in}{5.229782in}}%
\pgfpathlineto{\pgfqpoint{5.604707in}{5.231430in}}%
\pgfpathlineto{\pgfqpoint{5.605111in}{5.230911in}}%
\pgfpathlineto{\pgfqpoint{5.607234in}{5.229300in}}%
\pgfpathlineto{\pgfqpoint{5.608447in}{5.228081in}}%
\pgfpathlineto{\pgfqpoint{5.608851in}{5.228984in}}%
\pgfpathlineto{\pgfqpoint{5.610267in}{5.232690in}}%
\pgfpathlineto{\pgfqpoint{5.610772in}{5.231796in}}%
\pgfpathlineto{\pgfqpoint{5.612289in}{5.228474in}}%
\pgfpathlineto{\pgfqpoint{5.612794in}{5.229106in}}%
\pgfpathlineto{\pgfqpoint{5.613906in}{5.230637in}}%
\pgfpathlineto{\pgfqpoint{5.614412in}{5.229968in}}%
\pgfpathlineto{\pgfqpoint{5.616332in}{5.228744in}}%
\pgfpathlineto{\pgfqpoint{5.617242in}{5.230073in}}%
\pgfpathlineto{\pgfqpoint{5.618455in}{5.232486in}}%
\pgfpathlineto{\pgfqpoint{5.618961in}{5.231614in}}%
\pgfpathlineto{\pgfqpoint{5.620275in}{5.228345in}}%
\pgfpathlineto{\pgfqpoint{5.620882in}{5.229176in}}%
\pgfpathlineto{\pgfqpoint{5.621690in}{5.229965in}}%
\pgfpathlineto{\pgfqpoint{5.622196in}{5.229313in}}%
\pgfpathlineto{\pgfqpoint{5.623207in}{5.228227in}}%
\pgfpathlineto{\pgfqpoint{5.623611in}{5.228718in}}%
\pgfpathlineto{\pgfqpoint{5.625532in}{5.232013in}}%
\pgfpathlineto{\pgfqpoint{5.626138in}{5.231155in}}%
\pgfpathlineto{\pgfqpoint{5.627857in}{5.230249in}}%
\pgfpathlineto{\pgfqpoint{5.629474in}{5.228844in}}%
\pgfpathlineto{\pgfqpoint{5.632204in}{5.228452in}}%
\pgfpathlineto{\pgfqpoint{5.633417in}{5.229433in}}%
\pgfpathlineto{\pgfqpoint{5.636046in}{5.232313in}}%
\pgfpathlineto{\pgfqpoint{5.636248in}{5.232155in}}%
\pgfpathlineto{\pgfqpoint{5.641201in}{5.227178in}}%
\pgfpathlineto{\pgfqpoint{5.641606in}{5.227620in}}%
\pgfpathlineto{\pgfqpoint{5.645346in}{5.230533in}}%
\pgfpathlineto{\pgfqpoint{5.648682in}{5.230197in}}%
\pgfpathlineto{\pgfqpoint{5.651816in}{5.227799in}}%
\pgfpathlineto{\pgfqpoint{5.652726in}{5.227747in}}%
\pgfpathlineto{\pgfqpoint{5.653029in}{5.228159in}}%
\pgfpathlineto{\pgfqpoint{5.656567in}{5.231852in}}%
\pgfpathlineto{\pgfqpoint{5.657680in}{5.229706in}}%
\pgfpathlineto{\pgfqpoint{5.659095in}{5.228166in}}%
\pgfpathlineto{\pgfqpoint{5.659398in}{5.228396in}}%
\pgfpathlineto{\pgfqpoint{5.663442in}{5.232110in}}%
\pgfpathlineto{\pgfqpoint{5.663745in}{5.231623in}}%
\pgfpathlineto{\pgfqpoint{5.666374in}{5.228175in}}%
\pgfpathlineto{\pgfqpoint{5.667587in}{5.228726in}}%
\pgfpathlineto{\pgfqpoint{5.670114in}{5.230711in}}%
\pgfpathlineto{\pgfqpoint{5.672844in}{5.230456in}}%
\pgfpathlineto{\pgfqpoint{5.673854in}{5.229948in}}%
\pgfpathlineto{\pgfqpoint{5.675169in}{5.228663in}}%
\pgfpathlineto{\pgfqpoint{5.675573in}{5.229006in}}%
\pgfpathlineto{\pgfqpoint{5.676786in}{5.229479in}}%
\pgfpathlineto{\pgfqpoint{5.677089in}{5.229081in}}%
\pgfpathlineto{\pgfqpoint{5.677999in}{5.228041in}}%
\pgfpathlineto{\pgfqpoint{5.678404in}{5.228688in}}%
\pgfpathlineto{\pgfqpoint{5.680021in}{5.233516in}}%
\pgfpathlineto{\pgfqpoint{5.680628in}{5.232056in}}%
\pgfpathlineto{\pgfqpoint{5.682346in}{5.230219in}}%
\pgfpathlineto{\pgfqpoint{5.684065in}{5.229443in}}%
\pgfpathlineto{\pgfqpoint{5.686289in}{5.228204in}}%
\pgfpathlineto{\pgfqpoint{5.687805in}{5.229877in}}%
\pgfpathlineto{\pgfqpoint{5.688715in}{5.230719in}}%
\pgfpathlineto{\pgfqpoint{5.689221in}{5.230270in}}%
\pgfpathlineto{\pgfqpoint{5.689827in}{5.230559in}}%
\pgfpathlineto{\pgfqpoint{5.690029in}{5.231013in}}%
\pgfpathlineto{\pgfqpoint{5.691040in}{5.232664in}}%
\pgfpathlineto{\pgfqpoint{5.691445in}{5.232075in}}%
\pgfpathlineto{\pgfqpoint{5.693163in}{5.228501in}}%
\pgfpathlineto{\pgfqpoint{5.693871in}{5.229043in}}%
\pgfpathlineto{\pgfqpoint{5.695893in}{5.229357in}}%
\pgfpathlineto{\pgfqpoint{5.697207in}{5.228407in}}%
\pgfpathlineto{\pgfqpoint{5.697510in}{5.228862in}}%
\pgfpathlineto{\pgfqpoint{5.699027in}{5.232096in}}%
\pgfpathlineto{\pgfqpoint{5.699633in}{5.231220in}}%
\pgfpathlineto{\pgfqpoint{5.701857in}{5.228970in}}%
\pgfpathlineto{\pgfqpoint{5.703070in}{5.229775in}}%
\pgfpathlineto{\pgfqpoint{5.703980in}{5.230507in}}%
\pgfpathlineto{\pgfqpoint{5.704486in}{5.229952in}}%
\pgfpathlineto{\pgfqpoint{5.705800in}{5.229824in}}%
\pgfpathlineto{\pgfqpoint{5.705901in}{5.229913in}}%
\pgfpathlineto{\pgfqpoint{5.707721in}{5.230985in}}%
\pgfpathlineto{\pgfqpoint{5.708024in}{5.230521in}}%
\pgfpathlineto{\pgfqpoint{5.709439in}{5.228543in}}%
\pgfpathlineto{\pgfqpoint{5.709945in}{5.229064in}}%
\pgfpathlineto{\pgfqpoint{5.711461in}{5.229274in}}%
\pgfpathlineto{\pgfqpoint{5.714292in}{5.230517in}}%
\pgfpathlineto{\pgfqpoint{5.715909in}{5.231260in}}%
\pgfpathlineto{\pgfqpoint{5.717527in}{5.230450in}}%
\pgfpathlineto{\pgfqpoint{5.719751in}{5.228001in}}%
\pgfpathlineto{\pgfqpoint{5.719953in}{5.228133in}}%
\pgfpathlineto{\pgfqpoint{5.724704in}{5.231284in}}%
\pgfpathlineto{\pgfqpoint{5.725008in}{5.230825in}}%
\pgfpathlineto{\pgfqpoint{5.727030in}{5.229625in}}%
\pgfpathlineto{\pgfqpoint{5.730265in}{5.228989in}}%
\pgfpathlineto{\pgfqpoint{5.733500in}{5.230396in}}%
\pgfpathlineto{\pgfqpoint{5.741183in}{5.229522in}}%
\pgfpathlineto{\pgfqpoint{5.742497in}{5.232457in}}%
\pgfpathlineto{\pgfqpoint{5.743002in}{5.231692in}}%
\pgfpathlineto{\pgfqpoint{5.744215in}{5.229258in}}%
\pgfpathlineto{\pgfqpoint{5.744721in}{5.229835in}}%
\pgfpathlineto{\pgfqpoint{5.745631in}{5.230363in}}%
\pgfpathlineto{\pgfqpoint{5.745833in}{5.230184in}}%
\pgfusepath{stroke}%
\end{pgfscope}%
\begin{pgfscope}%
\pgfsetrectcap%
\pgfsetmiterjoin%
\pgfsetlinewidth{0.803000pt}%
\definecolor{currentstroke}{rgb}{0.737255,0.737255,0.737255}%
\pgfsetstrokecolor{currentstroke}%
\pgfsetdash{}{0pt}%
\pgfpathmoveto{\pgfqpoint{0.691161in}{4.791796in}}%
\pgfpathlineto{\pgfqpoint{0.691161in}{5.703703in}}%
\pgfusepath{stroke}%
\end{pgfscope}%
\begin{pgfscope}%
\pgfsetrectcap%
\pgfsetmiterjoin%
\pgfsetlinewidth{0.803000pt}%
\definecolor{currentstroke}{rgb}{0.737255,0.737255,0.737255}%
\pgfsetstrokecolor{currentstroke}%
\pgfsetdash{}{0pt}%
\pgfpathmoveto{\pgfqpoint{5.745833in}{4.791796in}}%
\pgfpathlineto{\pgfqpoint{5.745833in}{5.703703in}}%
\pgfusepath{stroke}%
\end{pgfscope}%
\begin{pgfscope}%
\pgfsetrectcap%
\pgfsetmiterjoin%
\pgfsetlinewidth{0.803000pt}%
\definecolor{currentstroke}{rgb}{0.737255,0.737255,0.737255}%
\pgfsetstrokecolor{currentstroke}%
\pgfsetdash{}{0pt}%
\pgfpathmoveto{\pgfqpoint{0.691161in}{4.791796in}}%
\pgfpathlineto{\pgfqpoint{5.745833in}{4.791796in}}%
\pgfusepath{stroke}%
\end{pgfscope}%
\begin{pgfscope}%
\pgfsetrectcap%
\pgfsetmiterjoin%
\pgfsetlinewidth{0.803000pt}%
\definecolor{currentstroke}{rgb}{0.737255,0.737255,0.737255}%
\pgfsetstrokecolor{currentstroke}%
\pgfsetdash{}{0pt}%
\pgfpathmoveto{\pgfqpoint{0.691161in}{5.703703in}}%
\pgfpathlineto{\pgfqpoint{5.745833in}{5.703703in}}%
\pgfusepath{stroke}%
\end{pgfscope}%
\begin{pgfscope}%
\pgfsetbuttcap%
\pgfsetmiterjoin%
\definecolor{currentfill}{rgb}{0.933333,0.933333,0.933333}%
\pgfsetfillcolor{currentfill}%
\pgfsetlinewidth{0.000000pt}%
\definecolor{currentstroke}{rgb}{0.000000,0.000000,0.000000}%
\pgfsetstrokecolor{currentstroke}%
\pgfsetstrokeopacity{0.000000}%
\pgfsetdash{}{0pt}%
\pgfpathmoveto{\pgfqpoint{0.691161in}{3.729888in}}%
\pgfpathlineto{\pgfqpoint{5.745833in}{3.729888in}}%
\pgfpathlineto{\pgfqpoint{5.745833in}{4.641796in}}%
\pgfpathlineto{\pgfqpoint{0.691161in}{4.641796in}}%
\pgfpathlineto{\pgfqpoint{0.691161in}{3.729888in}}%
\pgfpathclose%
\pgfusepath{fill}%
\end{pgfscope}%
\begin{pgfscope}%
\pgfpathrectangle{\pgfqpoint{0.691161in}{3.729888in}}{\pgfqpoint{5.054672in}{0.911907in}}%
\pgfusepath{clip}%
\pgfsetbuttcap%
\pgfsetroundjoin%
\pgfsetlinewidth{0.501875pt}%
\definecolor{currentstroke}{rgb}{0.698039,0.698039,0.698039}%
\pgfsetstrokecolor{currentstroke}%
\pgfsetdash{{1.850000pt}{0.800000pt}}{0.000000pt}%
\pgfpathmoveto{\pgfqpoint{0.691161in}{3.729888in}}%
\pgfpathlineto{\pgfqpoint{0.691161in}{4.641796in}}%
\pgfusepath{stroke}%
\end{pgfscope}%
\begin{pgfscope}%
\pgfsetbuttcap%
\pgfsetroundjoin%
\definecolor{currentfill}{rgb}{0.000000,0.000000,0.000000}%
\pgfsetfillcolor{currentfill}%
\pgfsetlinewidth{0.803000pt}%
\definecolor{currentstroke}{rgb}{0.000000,0.000000,0.000000}%
\pgfsetstrokecolor{currentstroke}%
\pgfsetdash{}{0pt}%
\pgfsys@defobject{currentmarker}{\pgfqpoint{0.000000in}{0.000000in}}{\pgfqpoint{0.000000in}{0.048611in}}{%
\pgfpathmoveto{\pgfqpoint{0.000000in}{0.000000in}}%
\pgfpathlineto{\pgfqpoint{0.000000in}{0.048611in}}%
\pgfusepath{stroke,fill}%
}%
\begin{pgfscope}%
\pgfsys@transformshift{0.691161in}{3.729888in}%
\pgfsys@useobject{currentmarker}{}%
\end{pgfscope}%
\end{pgfscope}%
\begin{pgfscope}%
\pgfpathrectangle{\pgfqpoint{0.691161in}{3.729888in}}{\pgfqpoint{5.054672in}{0.911907in}}%
\pgfusepath{clip}%
\pgfsetbuttcap%
\pgfsetroundjoin%
\pgfsetlinewidth{0.501875pt}%
\definecolor{currentstroke}{rgb}{0.698039,0.698039,0.698039}%
\pgfsetstrokecolor{currentstroke}%
\pgfsetdash{{1.850000pt}{0.800000pt}}{0.000000pt}%
\pgfpathmoveto{\pgfqpoint{1.702096in}{3.729888in}}%
\pgfpathlineto{\pgfqpoint{1.702096in}{4.641796in}}%
\pgfusepath{stroke}%
\end{pgfscope}%
\begin{pgfscope}%
\pgfsetbuttcap%
\pgfsetroundjoin%
\definecolor{currentfill}{rgb}{0.000000,0.000000,0.000000}%
\pgfsetfillcolor{currentfill}%
\pgfsetlinewidth{0.803000pt}%
\definecolor{currentstroke}{rgb}{0.000000,0.000000,0.000000}%
\pgfsetstrokecolor{currentstroke}%
\pgfsetdash{}{0pt}%
\pgfsys@defobject{currentmarker}{\pgfqpoint{0.000000in}{0.000000in}}{\pgfqpoint{0.000000in}{0.048611in}}{%
\pgfpathmoveto{\pgfqpoint{0.000000in}{0.000000in}}%
\pgfpathlineto{\pgfqpoint{0.000000in}{0.048611in}}%
\pgfusepath{stroke,fill}%
}%
\begin{pgfscope}%
\pgfsys@transformshift{1.702096in}{3.729888in}%
\pgfsys@useobject{currentmarker}{}%
\end{pgfscope}%
\end{pgfscope}%
\begin{pgfscope}%
\pgfpathrectangle{\pgfqpoint{0.691161in}{3.729888in}}{\pgfqpoint{5.054672in}{0.911907in}}%
\pgfusepath{clip}%
\pgfsetbuttcap%
\pgfsetroundjoin%
\pgfsetlinewidth{0.501875pt}%
\definecolor{currentstroke}{rgb}{0.698039,0.698039,0.698039}%
\pgfsetstrokecolor{currentstroke}%
\pgfsetdash{{1.850000pt}{0.800000pt}}{0.000000pt}%
\pgfpathmoveto{\pgfqpoint{2.713030in}{3.729888in}}%
\pgfpathlineto{\pgfqpoint{2.713030in}{4.641796in}}%
\pgfusepath{stroke}%
\end{pgfscope}%
\begin{pgfscope}%
\pgfsetbuttcap%
\pgfsetroundjoin%
\definecolor{currentfill}{rgb}{0.000000,0.000000,0.000000}%
\pgfsetfillcolor{currentfill}%
\pgfsetlinewidth{0.803000pt}%
\definecolor{currentstroke}{rgb}{0.000000,0.000000,0.000000}%
\pgfsetstrokecolor{currentstroke}%
\pgfsetdash{}{0pt}%
\pgfsys@defobject{currentmarker}{\pgfqpoint{0.000000in}{0.000000in}}{\pgfqpoint{0.000000in}{0.048611in}}{%
\pgfpathmoveto{\pgfqpoint{0.000000in}{0.000000in}}%
\pgfpathlineto{\pgfqpoint{0.000000in}{0.048611in}}%
\pgfusepath{stroke,fill}%
}%
\begin{pgfscope}%
\pgfsys@transformshift{2.713030in}{3.729888in}%
\pgfsys@useobject{currentmarker}{}%
\end{pgfscope}%
\end{pgfscope}%
\begin{pgfscope}%
\pgfpathrectangle{\pgfqpoint{0.691161in}{3.729888in}}{\pgfqpoint{5.054672in}{0.911907in}}%
\pgfusepath{clip}%
\pgfsetbuttcap%
\pgfsetroundjoin%
\pgfsetlinewidth{0.501875pt}%
\definecolor{currentstroke}{rgb}{0.698039,0.698039,0.698039}%
\pgfsetstrokecolor{currentstroke}%
\pgfsetdash{{1.850000pt}{0.800000pt}}{0.000000pt}%
\pgfpathmoveto{\pgfqpoint{3.723964in}{3.729888in}}%
\pgfpathlineto{\pgfqpoint{3.723964in}{4.641796in}}%
\pgfusepath{stroke}%
\end{pgfscope}%
\begin{pgfscope}%
\pgfsetbuttcap%
\pgfsetroundjoin%
\definecolor{currentfill}{rgb}{0.000000,0.000000,0.000000}%
\pgfsetfillcolor{currentfill}%
\pgfsetlinewidth{0.803000pt}%
\definecolor{currentstroke}{rgb}{0.000000,0.000000,0.000000}%
\pgfsetstrokecolor{currentstroke}%
\pgfsetdash{}{0pt}%
\pgfsys@defobject{currentmarker}{\pgfqpoint{0.000000in}{0.000000in}}{\pgfqpoint{0.000000in}{0.048611in}}{%
\pgfpathmoveto{\pgfqpoint{0.000000in}{0.000000in}}%
\pgfpathlineto{\pgfqpoint{0.000000in}{0.048611in}}%
\pgfusepath{stroke,fill}%
}%
\begin{pgfscope}%
\pgfsys@transformshift{3.723964in}{3.729888in}%
\pgfsys@useobject{currentmarker}{}%
\end{pgfscope}%
\end{pgfscope}%
\begin{pgfscope}%
\pgfpathrectangle{\pgfqpoint{0.691161in}{3.729888in}}{\pgfqpoint{5.054672in}{0.911907in}}%
\pgfusepath{clip}%
\pgfsetbuttcap%
\pgfsetroundjoin%
\pgfsetlinewidth{0.501875pt}%
\definecolor{currentstroke}{rgb}{0.698039,0.698039,0.698039}%
\pgfsetstrokecolor{currentstroke}%
\pgfsetdash{{1.850000pt}{0.800000pt}}{0.000000pt}%
\pgfpathmoveto{\pgfqpoint{4.734899in}{3.729888in}}%
\pgfpathlineto{\pgfqpoint{4.734899in}{4.641796in}}%
\pgfusepath{stroke}%
\end{pgfscope}%
\begin{pgfscope}%
\pgfsetbuttcap%
\pgfsetroundjoin%
\definecolor{currentfill}{rgb}{0.000000,0.000000,0.000000}%
\pgfsetfillcolor{currentfill}%
\pgfsetlinewidth{0.803000pt}%
\definecolor{currentstroke}{rgb}{0.000000,0.000000,0.000000}%
\pgfsetstrokecolor{currentstroke}%
\pgfsetdash{}{0pt}%
\pgfsys@defobject{currentmarker}{\pgfqpoint{0.000000in}{0.000000in}}{\pgfqpoint{0.000000in}{0.048611in}}{%
\pgfpathmoveto{\pgfqpoint{0.000000in}{0.000000in}}%
\pgfpathlineto{\pgfqpoint{0.000000in}{0.048611in}}%
\pgfusepath{stroke,fill}%
}%
\begin{pgfscope}%
\pgfsys@transformshift{4.734899in}{3.729888in}%
\pgfsys@useobject{currentmarker}{}%
\end{pgfscope}%
\end{pgfscope}%
\begin{pgfscope}%
\pgfpathrectangle{\pgfqpoint{0.691161in}{3.729888in}}{\pgfqpoint{5.054672in}{0.911907in}}%
\pgfusepath{clip}%
\pgfsetbuttcap%
\pgfsetroundjoin%
\pgfsetlinewidth{0.501875pt}%
\definecolor{currentstroke}{rgb}{0.698039,0.698039,0.698039}%
\pgfsetstrokecolor{currentstroke}%
\pgfsetdash{{1.850000pt}{0.800000pt}}{0.000000pt}%
\pgfpathmoveto{\pgfqpoint{5.745833in}{3.729888in}}%
\pgfpathlineto{\pgfqpoint{5.745833in}{4.641796in}}%
\pgfusepath{stroke}%
\end{pgfscope}%
\begin{pgfscope}%
\pgfsetbuttcap%
\pgfsetroundjoin%
\definecolor{currentfill}{rgb}{0.000000,0.000000,0.000000}%
\pgfsetfillcolor{currentfill}%
\pgfsetlinewidth{0.803000pt}%
\definecolor{currentstroke}{rgb}{0.000000,0.000000,0.000000}%
\pgfsetstrokecolor{currentstroke}%
\pgfsetdash{}{0pt}%
\pgfsys@defobject{currentmarker}{\pgfqpoint{0.000000in}{0.000000in}}{\pgfqpoint{0.000000in}{0.048611in}}{%
\pgfpathmoveto{\pgfqpoint{0.000000in}{0.000000in}}%
\pgfpathlineto{\pgfqpoint{0.000000in}{0.048611in}}%
\pgfusepath{stroke,fill}%
}%
\begin{pgfscope}%
\pgfsys@transformshift{5.745833in}{3.729888in}%
\pgfsys@useobject{currentmarker}{}%
\end{pgfscope}%
\end{pgfscope}%
\begin{pgfscope}%
\pgfpathrectangle{\pgfqpoint{0.691161in}{3.729888in}}{\pgfqpoint{5.054672in}{0.911907in}}%
\pgfusepath{clip}%
\pgfsetbuttcap%
\pgfsetroundjoin%
\pgfsetlinewidth{0.501875pt}%
\definecolor{currentstroke}{rgb}{0.698039,0.698039,0.698039}%
\pgfsetstrokecolor{currentstroke}%
\pgfsetdash{{1.850000pt}{0.800000pt}}{0.000000pt}%
\pgfpathmoveto{\pgfqpoint{0.691161in}{3.901446in}}%
\pgfpathlineto{\pgfqpoint{5.745833in}{3.901446in}}%
\pgfusepath{stroke}%
\end{pgfscope}%
\begin{pgfscope}%
\pgfsetbuttcap%
\pgfsetroundjoin%
\definecolor{currentfill}{rgb}{0.000000,0.000000,0.000000}%
\pgfsetfillcolor{currentfill}%
\pgfsetlinewidth{0.803000pt}%
\definecolor{currentstroke}{rgb}{0.000000,0.000000,0.000000}%
\pgfsetstrokecolor{currentstroke}%
\pgfsetdash{}{0pt}%
\pgfsys@defobject{currentmarker}{\pgfqpoint{0.000000in}{0.000000in}}{\pgfqpoint{0.048611in}{0.000000in}}{%
\pgfpathmoveto{\pgfqpoint{0.000000in}{0.000000in}}%
\pgfpathlineto{\pgfqpoint{0.048611in}{0.000000in}}%
\pgfusepath{stroke,fill}%
}%
\begin{pgfscope}%
\pgfsys@transformshift{0.691161in}{3.901446in}%
\pgfsys@useobject{currentmarker}{}%
\end{pgfscope}%
\end{pgfscope}%
\begin{pgfscope}%
\definecolor{textcolor}{rgb}{0.000000,0.000000,0.000000}%
\pgfsetstrokecolor{textcolor}%
\pgfsetfillcolor{textcolor}%
\pgftext[x=0.573105in, y=3.853252in, left, base]{\color{textcolor}\rmfamily\fontsize{10.000000}{12.000000}\selectfont \(\displaystyle {0}\)}%
\end{pgfscope}%
\begin{pgfscope}%
\pgfpathrectangle{\pgfqpoint{0.691161in}{3.729888in}}{\pgfqpoint{5.054672in}{0.911907in}}%
\pgfusepath{clip}%
\pgfsetbuttcap%
\pgfsetroundjoin%
\pgfsetlinewidth{0.501875pt}%
\definecolor{currentstroke}{rgb}{0.698039,0.698039,0.698039}%
\pgfsetstrokecolor{currentstroke}%
\pgfsetdash{{1.850000pt}{0.800000pt}}{0.000000pt}%
\pgfpathmoveto{\pgfqpoint{0.691161in}{4.226716in}}%
\pgfpathlineto{\pgfqpoint{5.745833in}{4.226716in}}%
\pgfusepath{stroke}%
\end{pgfscope}%
\begin{pgfscope}%
\pgfsetbuttcap%
\pgfsetroundjoin%
\definecolor{currentfill}{rgb}{0.000000,0.000000,0.000000}%
\pgfsetfillcolor{currentfill}%
\pgfsetlinewidth{0.803000pt}%
\definecolor{currentstroke}{rgb}{0.000000,0.000000,0.000000}%
\pgfsetstrokecolor{currentstroke}%
\pgfsetdash{}{0pt}%
\pgfsys@defobject{currentmarker}{\pgfqpoint{0.000000in}{0.000000in}}{\pgfqpoint{0.048611in}{0.000000in}}{%
\pgfpathmoveto{\pgfqpoint{0.000000in}{0.000000in}}%
\pgfpathlineto{\pgfqpoint{0.048611in}{0.000000in}}%
\pgfusepath{stroke,fill}%
}%
\begin{pgfscope}%
\pgfsys@transformshift{0.691161in}{4.226716in}%
\pgfsys@useobject{currentmarker}{}%
\end{pgfscope}%
\end{pgfscope}%
\begin{pgfscope}%
\definecolor{textcolor}{rgb}{0.000000,0.000000,0.000000}%
\pgfsetstrokecolor{textcolor}%
\pgfsetfillcolor{textcolor}%
\pgftext[x=0.573105in, y=4.178521in, left, base]{\color{textcolor}\rmfamily\fontsize{10.000000}{12.000000}\selectfont \(\displaystyle {5}\)}%
\end{pgfscope}%
\begin{pgfscope}%
\pgfpathrectangle{\pgfqpoint{0.691161in}{3.729888in}}{\pgfqpoint{5.054672in}{0.911907in}}%
\pgfusepath{clip}%
\pgfsetbuttcap%
\pgfsetroundjoin%
\pgfsetlinewidth{0.501875pt}%
\definecolor{currentstroke}{rgb}{0.698039,0.698039,0.698039}%
\pgfsetstrokecolor{currentstroke}%
\pgfsetdash{{1.850000pt}{0.800000pt}}{0.000000pt}%
\pgfpathmoveto{\pgfqpoint{0.691161in}{4.551985in}}%
\pgfpathlineto{\pgfqpoint{5.745833in}{4.551985in}}%
\pgfusepath{stroke}%
\end{pgfscope}%
\begin{pgfscope}%
\pgfsetbuttcap%
\pgfsetroundjoin%
\definecolor{currentfill}{rgb}{0.000000,0.000000,0.000000}%
\pgfsetfillcolor{currentfill}%
\pgfsetlinewidth{0.803000pt}%
\definecolor{currentstroke}{rgb}{0.000000,0.000000,0.000000}%
\pgfsetstrokecolor{currentstroke}%
\pgfsetdash{}{0pt}%
\pgfsys@defobject{currentmarker}{\pgfqpoint{0.000000in}{0.000000in}}{\pgfqpoint{0.048611in}{0.000000in}}{%
\pgfpathmoveto{\pgfqpoint{0.000000in}{0.000000in}}%
\pgfpathlineto{\pgfqpoint{0.048611in}{0.000000in}}%
\pgfusepath{stroke,fill}%
}%
\begin{pgfscope}%
\pgfsys@transformshift{0.691161in}{4.551985in}%
\pgfsys@useobject{currentmarker}{}%
\end{pgfscope}%
\end{pgfscope}%
\begin{pgfscope}%
\definecolor{textcolor}{rgb}{0.000000,0.000000,0.000000}%
\pgfsetstrokecolor{textcolor}%
\pgfsetfillcolor{textcolor}%
\pgftext[x=0.503661in, y=4.503791in, left, base]{\color{textcolor}\rmfamily\fontsize{10.000000}{12.000000}\selectfont \(\displaystyle {10}\)}%
\end{pgfscope}%
\begin{pgfscope}%
\definecolor{textcolor}{rgb}{0.000000,0.000000,0.000000}%
\pgfsetstrokecolor{textcolor}%
\pgfsetfillcolor{textcolor}%
\pgftext[x=0.448105in,y=4.185842in,,bottom,rotate=90.000000]{\color{textcolor}\rmfamily\fontsize{12.000000}{14.400000}\selectfont Z-detect}%
\end{pgfscope}%
\begin{pgfscope}%
\pgfpathrectangle{\pgfqpoint{0.691161in}{3.729888in}}{\pgfqpoint{5.054672in}{0.911907in}}%
\pgfusepath{clip}%
\pgfsetrectcap%
\pgfsetroundjoin%
\pgfsetlinewidth{1.505625pt}%
\definecolor{currentstroke}{rgb}{0.121569,0.466667,0.705882}%
\pgfsetstrokecolor{currentstroke}%
\pgfsetdash{}{0pt}%
\pgfpathmoveto{\pgfqpoint{0.691060in}{3.888018in}}%
\pgfpathlineto{\pgfqpoint{3.036630in}{3.889025in}}%
\pgfpathlineto{\pgfqpoint{3.044920in}{3.892344in}}%
\pgfpathlineto{\pgfqpoint{3.050783in}{3.892856in}}%
\pgfpathlineto{\pgfqpoint{3.054625in}{3.894893in}}%
\pgfpathlineto{\pgfqpoint{3.056444in}{3.895844in}}%
\pgfpathlineto{\pgfqpoint{3.058264in}{3.894795in}}%
\pgfpathlineto{\pgfqpoint{3.059780in}{3.893334in}}%
\pgfpathlineto{\pgfqpoint{3.059882in}{3.893376in}}%
\pgfpathlineto{\pgfqpoint{3.065745in}{3.898181in}}%
\pgfpathlineto{\pgfqpoint{3.069485in}{3.904192in}}%
\pgfpathlineto{\pgfqpoint{3.071507in}{3.905373in}}%
\pgfpathlineto{\pgfqpoint{3.074641in}{3.905152in}}%
\pgfpathlineto{\pgfqpoint{3.076259in}{3.905534in}}%
\pgfpathlineto{\pgfqpoint{3.077876in}{3.909831in}}%
\pgfpathlineto{\pgfqpoint{3.079696in}{3.912904in}}%
\pgfpathlineto{\pgfqpoint{3.081313in}{3.914531in}}%
\pgfpathlineto{\pgfqpoint{3.081920in}{3.913042in}}%
\pgfpathlineto{\pgfqpoint{3.082628in}{3.912750in}}%
\pgfpathlineto{\pgfqpoint{3.083032in}{3.913337in}}%
\pgfpathlineto{\pgfqpoint{3.085054in}{3.915419in}}%
\pgfpathlineto{\pgfqpoint{3.085761in}{3.914328in}}%
\pgfpathlineto{\pgfqpoint{3.086368in}{3.913539in}}%
\pgfpathlineto{\pgfqpoint{3.086873in}{3.914394in}}%
\pgfpathlineto{\pgfqpoint{3.087278in}{3.914854in}}%
\pgfpathlineto{\pgfqpoint{3.087682in}{3.913670in}}%
\pgfpathlineto{\pgfqpoint{3.088390in}{3.911219in}}%
\pgfpathlineto{\pgfqpoint{3.089199in}{3.911642in}}%
\pgfpathlineto{\pgfqpoint{3.094961in}{3.910939in}}%
\pgfpathlineto{\pgfqpoint{3.100420in}{3.906736in}}%
\pgfpathlineto{\pgfqpoint{3.100521in}{3.906866in}}%
\pgfpathlineto{\pgfqpoint{3.102543in}{3.907939in}}%
\pgfpathlineto{\pgfqpoint{3.103352in}{3.906971in}}%
\pgfpathlineto{\pgfqpoint{3.103958in}{3.905927in}}%
\pgfpathlineto{\pgfqpoint{3.104363in}{3.906994in}}%
\pgfpathlineto{\pgfqpoint{3.106486in}{3.911874in}}%
\pgfpathlineto{\pgfqpoint{3.106789in}{3.912042in}}%
\pgfpathlineto{\pgfqpoint{3.107193in}{3.911126in}}%
\pgfpathlineto{\pgfqpoint{3.108204in}{3.909450in}}%
\pgfpathlineto{\pgfqpoint{3.108710in}{3.910067in}}%
\pgfpathlineto{\pgfqpoint{3.113360in}{3.918108in}}%
\pgfpathlineto{\pgfqpoint{3.116696in}{3.919449in}}%
\pgfpathlineto{\pgfqpoint{3.118920in}{3.926251in}}%
\pgfpathlineto{\pgfqpoint{3.120032in}{3.930657in}}%
\pgfpathlineto{\pgfqpoint{3.120841in}{3.929878in}}%
\pgfpathlineto{\pgfqpoint{3.122863in}{3.927909in}}%
\pgfpathlineto{\pgfqpoint{3.124682in}{3.927152in}}%
\pgfpathlineto{\pgfqpoint{3.126906in}{3.923319in}}%
\pgfpathlineto{\pgfqpoint{3.127210in}{3.923580in}}%
\pgfpathlineto{\pgfqpoint{3.128524in}{3.924081in}}%
\pgfpathlineto{\pgfqpoint{3.128726in}{3.923847in}}%
\pgfpathlineto{\pgfqpoint{3.131355in}{3.917155in}}%
\pgfpathlineto{\pgfqpoint{3.132871in}{3.914237in}}%
\pgfpathlineto{\pgfqpoint{3.134691in}{3.914652in}}%
\pgfpathlineto{\pgfqpoint{3.136611in}{3.914456in}}%
\pgfpathlineto{\pgfqpoint{3.137521in}{3.908891in}}%
\pgfpathlineto{\pgfqpoint{3.139846in}{3.900696in}}%
\pgfpathlineto{\pgfqpoint{3.139948in}{3.900617in}}%
\pgfpathlineto{\pgfqpoint{3.140251in}{3.901527in}}%
\pgfpathlineto{\pgfqpoint{3.142172in}{3.906067in}}%
\pgfpathlineto{\pgfqpoint{3.144092in}{3.912496in}}%
\pgfpathlineto{\pgfqpoint{3.144699in}{3.911789in}}%
\pgfpathlineto{\pgfqpoint{3.145508in}{3.910774in}}%
\pgfpathlineto{\pgfqpoint{3.146114in}{3.911273in}}%
\pgfpathlineto{\pgfqpoint{3.147226in}{3.910599in}}%
\pgfpathlineto{\pgfqpoint{3.150765in}{3.907932in}}%
\pgfpathlineto{\pgfqpoint{3.152382in}{3.910385in}}%
\pgfpathlineto{\pgfqpoint{3.157740in}{3.949678in}}%
\pgfpathlineto{\pgfqpoint{3.159054in}{3.948571in}}%
\pgfpathlineto{\pgfqpoint{3.161076in}{3.948164in}}%
\pgfpathlineto{\pgfqpoint{3.161177in}{3.948377in}}%
\pgfpathlineto{\pgfqpoint{3.161986in}{3.950172in}}%
\pgfpathlineto{\pgfqpoint{3.162491in}{3.948939in}}%
\pgfpathlineto{\pgfqpoint{3.162997in}{3.947789in}}%
\pgfpathlineto{\pgfqpoint{3.163502in}{3.949323in}}%
\pgfpathlineto{\pgfqpoint{3.164817in}{3.953494in}}%
\pgfpathlineto{\pgfqpoint{3.165221in}{3.953327in}}%
\pgfpathlineto{\pgfqpoint{3.167647in}{3.951888in}}%
\pgfpathlineto{\pgfqpoint{3.168051in}{3.952666in}}%
\pgfpathlineto{\pgfqpoint{3.170882in}{3.956696in}}%
\pgfpathlineto{\pgfqpoint{3.171489in}{3.956729in}}%
\pgfpathlineto{\pgfqpoint{3.171691in}{3.957420in}}%
\pgfpathlineto{\pgfqpoint{3.172500in}{3.961608in}}%
\pgfpathlineto{\pgfqpoint{3.172904in}{3.958249in}}%
\pgfpathlineto{\pgfqpoint{3.173915in}{3.945785in}}%
\pgfpathlineto{\pgfqpoint{3.174623in}{3.947765in}}%
\pgfpathlineto{\pgfqpoint{3.175027in}{3.945555in}}%
\pgfpathlineto{\pgfqpoint{3.176240in}{3.934456in}}%
\pgfpathlineto{\pgfqpoint{3.176847in}{3.937124in}}%
\pgfpathlineto{\pgfqpoint{3.178970in}{3.946231in}}%
\pgfpathlineto{\pgfqpoint{3.179879in}{3.948232in}}%
\pgfpathlineto{\pgfqpoint{3.180385in}{3.947790in}}%
\pgfpathlineto{\pgfqpoint{3.181699in}{3.944660in}}%
\pgfpathlineto{\pgfqpoint{3.182205in}{3.943792in}}%
\pgfpathlineto{\pgfqpoint{3.182710in}{3.944891in}}%
\pgfpathlineto{\pgfqpoint{3.182912in}{3.945134in}}%
\pgfpathlineto{\pgfqpoint{3.183317in}{3.943919in}}%
\pgfpathlineto{\pgfqpoint{3.184732in}{3.936107in}}%
\pgfpathlineto{\pgfqpoint{3.185541in}{3.936269in}}%
\pgfpathlineto{\pgfqpoint{3.186349in}{3.937661in}}%
\pgfpathlineto{\pgfqpoint{3.189483in}{3.942277in}}%
\pgfpathlineto{\pgfqpoint{3.190696in}{3.941245in}}%
\pgfpathlineto{\pgfqpoint{3.191101in}{3.942578in}}%
\pgfpathlineto{\pgfqpoint{3.191707in}{3.944084in}}%
\pgfpathlineto{\pgfqpoint{3.192213in}{3.942508in}}%
\pgfpathlineto{\pgfqpoint{3.193426in}{3.939955in}}%
\pgfpathlineto{\pgfqpoint{3.193729in}{3.940276in}}%
\pgfpathlineto{\pgfqpoint{3.195043in}{3.944819in}}%
\pgfpathlineto{\pgfqpoint{3.195650in}{3.941980in}}%
\pgfpathlineto{\pgfqpoint{3.196155in}{3.940505in}}%
\pgfpathlineto{\pgfqpoint{3.196661in}{3.941909in}}%
\pgfpathlineto{\pgfqpoint{3.196964in}{3.942198in}}%
\pgfpathlineto{\pgfqpoint{3.197267in}{3.941238in}}%
\pgfpathlineto{\pgfqpoint{3.198986in}{3.938075in}}%
\pgfpathlineto{\pgfqpoint{3.208388in}{3.928978in}}%
\pgfpathlineto{\pgfqpoint{3.209803in}{3.926721in}}%
\pgfpathlineto{\pgfqpoint{3.210106in}{3.926903in}}%
\pgfpathlineto{\pgfqpoint{3.211117in}{3.928058in}}%
\pgfpathlineto{\pgfqpoint{3.211623in}{3.927041in}}%
\pgfpathlineto{\pgfqpoint{3.212836in}{3.924225in}}%
\pgfpathlineto{\pgfqpoint{3.213341in}{3.925455in}}%
\pgfpathlineto{\pgfqpoint{3.213948in}{3.927079in}}%
\pgfpathlineto{\pgfqpoint{3.214352in}{3.925460in}}%
\pgfpathlineto{\pgfqpoint{3.217789in}{3.908212in}}%
\pgfpathlineto{\pgfqpoint{3.219710in}{3.904856in}}%
\pgfpathlineto{\pgfqpoint{3.220014in}{3.904874in}}%
\pgfpathlineto{\pgfqpoint{3.220418in}{3.905630in}}%
\pgfpathlineto{\pgfqpoint{3.222844in}{3.907801in}}%
\pgfpathlineto{\pgfqpoint{3.226888in}{3.908161in}}%
\pgfpathlineto{\pgfqpoint{3.228910in}{3.907075in}}%
\pgfpathlineto{\pgfqpoint{3.236188in}{3.897577in}}%
\pgfpathlineto{\pgfqpoint{3.239525in}{3.897749in}}%
\pgfpathlineto{\pgfqpoint{3.240131in}{3.897447in}}%
\pgfpathlineto{\pgfqpoint{3.240333in}{3.896963in}}%
\pgfpathlineto{\pgfqpoint{3.241142in}{3.895487in}}%
\pgfpathlineto{\pgfqpoint{3.241749in}{3.896151in}}%
\pgfpathlineto{\pgfqpoint{3.242153in}{3.896118in}}%
\pgfpathlineto{\pgfqpoint{3.242456in}{3.895464in}}%
\pgfpathlineto{\pgfqpoint{3.243164in}{3.894177in}}%
\pgfpathlineto{\pgfqpoint{3.243669in}{3.895272in}}%
\pgfpathlineto{\pgfqpoint{3.244074in}{3.895759in}}%
\pgfpathlineto{\pgfqpoint{3.244680in}{3.895101in}}%
\pgfpathlineto{\pgfqpoint{3.245489in}{3.895223in}}%
\pgfpathlineto{\pgfqpoint{3.245691in}{3.895496in}}%
\pgfpathlineto{\pgfqpoint{3.248320in}{3.897773in}}%
\pgfpathlineto{\pgfqpoint{3.253273in}{3.899592in}}%
\pgfpathlineto{\pgfqpoint{3.254891in}{3.900492in}}%
\pgfpathlineto{\pgfqpoint{3.259137in}{3.901262in}}%
\pgfpathlineto{\pgfqpoint{3.263989in}{3.898312in}}%
\pgfpathlineto{\pgfqpoint{3.265708in}{3.897512in}}%
\pgfpathlineto{\pgfqpoint{3.269044in}{3.895149in}}%
\pgfpathlineto{\pgfqpoint{3.272178in}{3.893679in}}%
\pgfpathlineto{\pgfqpoint{3.276525in}{3.891574in}}%
\pgfpathlineto{\pgfqpoint{3.281276in}{3.892490in}}%
\pgfpathlineto{\pgfqpoint{3.284309in}{3.895234in}}%
\pgfpathlineto{\pgfqpoint{3.285017in}{3.896008in}}%
\pgfpathlineto{\pgfqpoint{3.286331in}{3.900067in}}%
\pgfpathlineto{\pgfqpoint{3.287038in}{3.899697in}}%
\pgfpathlineto{\pgfqpoint{3.289667in}{3.900080in}}%
\pgfpathlineto{\pgfqpoint{3.295935in}{3.900085in}}%
\pgfpathlineto{\pgfqpoint{3.300989in}{3.899996in}}%
\pgfpathlineto{\pgfqpoint{3.303011in}{3.901397in}}%
\pgfpathlineto{\pgfqpoint{3.303820in}{3.899915in}}%
\pgfpathlineto{\pgfqpoint{3.304427in}{3.899364in}}%
\pgfpathlineto{\pgfqpoint{3.304932in}{3.900050in}}%
\pgfpathlineto{\pgfqpoint{3.305235in}{3.900071in}}%
\pgfpathlineto{\pgfqpoint{3.305539in}{3.899207in}}%
\pgfpathlineto{\pgfqpoint{3.306246in}{3.896963in}}%
\pgfpathlineto{\pgfqpoint{3.306752in}{3.898449in}}%
\pgfpathlineto{\pgfqpoint{3.307864in}{3.899893in}}%
\pgfpathlineto{\pgfqpoint{3.308167in}{3.899798in}}%
\pgfpathlineto{\pgfqpoint{3.311503in}{3.899694in}}%
\pgfpathlineto{\pgfqpoint{3.317265in}{3.909245in}}%
\pgfpathlineto{\pgfqpoint{3.318984in}{3.912828in}}%
\pgfpathlineto{\pgfqpoint{3.319692in}{3.914697in}}%
\pgfpathlineto{\pgfqpoint{3.323533in}{3.933345in}}%
\pgfpathlineto{\pgfqpoint{3.325656in}{3.960543in}}%
\pgfpathlineto{\pgfqpoint{3.326465in}{3.980951in}}%
\pgfpathlineto{\pgfqpoint{3.331418in}{4.147647in}}%
\pgfpathlineto{\pgfqpoint{3.331722in}{4.147570in}}%
\pgfpathlineto{\pgfqpoint{3.331823in}{4.147927in}}%
\pgfpathlineto{\pgfqpoint{3.333845in}{4.169975in}}%
\pgfpathlineto{\pgfqpoint{3.337484in}{4.248766in}}%
\pgfpathlineto{\pgfqpoint{3.337686in}{4.249553in}}%
\pgfpathlineto{\pgfqpoint{3.338394in}{4.247692in}}%
\pgfpathlineto{\pgfqpoint{3.338899in}{4.247548in}}%
\pgfpathlineto{\pgfqpoint{3.339304in}{4.248137in}}%
\pgfpathlineto{\pgfqpoint{3.339607in}{4.248103in}}%
\pgfpathlineto{\pgfqpoint{3.339809in}{4.247559in}}%
\pgfpathlineto{\pgfqpoint{3.340315in}{4.245988in}}%
\pgfpathlineto{\pgfqpoint{3.340618in}{4.248643in}}%
\pgfpathlineto{\pgfqpoint{3.341225in}{4.284351in}}%
\pgfpathlineto{\pgfqpoint{3.343348in}{4.394286in}}%
\pgfpathlineto{\pgfqpoint{3.344460in}{4.416060in}}%
\pgfpathlineto{\pgfqpoint{3.344864in}{4.415153in}}%
\pgfpathlineto{\pgfqpoint{3.345167in}{4.414637in}}%
\pgfpathlineto{\pgfqpoint{3.345572in}{4.416274in}}%
\pgfpathlineto{\pgfqpoint{3.345774in}{4.416907in}}%
\pgfpathlineto{\pgfqpoint{3.346077in}{4.414436in}}%
\pgfpathlineto{\pgfqpoint{3.348807in}{4.342693in}}%
\pgfpathlineto{\pgfqpoint{3.349110in}{4.333115in}}%
\pgfpathlineto{\pgfqpoint{3.349514in}{4.355848in}}%
\pgfpathlineto{\pgfqpoint{3.350626in}{4.444263in}}%
\pgfpathlineto{\pgfqpoint{3.351233in}{4.424224in}}%
\pgfpathlineto{\pgfqpoint{3.351435in}{4.421707in}}%
\pgfpathlineto{\pgfqpoint{3.351738in}{4.430044in}}%
\pgfpathlineto{\pgfqpoint{3.353760in}{4.483116in}}%
\pgfpathlineto{\pgfqpoint{3.353962in}{4.482239in}}%
\pgfpathlineto{\pgfqpoint{3.356287in}{4.436863in}}%
\pgfpathlineto{\pgfqpoint{3.357197in}{4.450194in}}%
\pgfpathlineto{\pgfqpoint{3.358208in}{4.535625in}}%
\pgfpathlineto{\pgfqpoint{3.360230in}{4.593709in}}%
\pgfpathlineto{\pgfqpoint{3.360736in}{4.600345in}}%
\pgfpathlineto{\pgfqpoint{3.361140in}{4.590201in}}%
\pgfpathlineto{\pgfqpoint{3.364880in}{4.417967in}}%
\pgfpathlineto{\pgfqpoint{3.365083in}{4.419255in}}%
\pgfpathlineto{\pgfqpoint{3.365487in}{4.421936in}}%
\pgfpathlineto{\pgfqpoint{3.365992in}{4.418441in}}%
\pgfpathlineto{\pgfqpoint{3.366599in}{4.419182in}}%
\pgfpathlineto{\pgfqpoint{3.367104in}{4.410419in}}%
\pgfpathlineto{\pgfqpoint{3.368823in}{4.391884in}}%
\pgfpathlineto{\pgfqpoint{3.369329in}{4.376261in}}%
\pgfpathlineto{\pgfqpoint{3.372462in}{4.242914in}}%
\pgfpathlineto{\pgfqpoint{3.372665in}{4.244254in}}%
\pgfpathlineto{\pgfqpoint{3.374585in}{4.259409in}}%
\pgfpathlineto{\pgfqpoint{3.376203in}{4.267141in}}%
\pgfpathlineto{\pgfqpoint{3.376304in}{4.267478in}}%
\pgfpathlineto{\pgfqpoint{3.376809in}{4.266041in}}%
\pgfpathlineto{\pgfqpoint{3.377012in}{4.266077in}}%
\pgfpathlineto{\pgfqpoint{3.377315in}{4.263911in}}%
\pgfpathlineto{\pgfqpoint{3.378023in}{4.220238in}}%
\pgfpathlineto{\pgfqpoint{3.379640in}{4.168823in}}%
\pgfpathlineto{\pgfqpoint{3.379943in}{4.163910in}}%
\pgfpathlineto{\pgfqpoint{3.380449in}{4.175149in}}%
\pgfpathlineto{\pgfqpoint{3.381359in}{4.200144in}}%
\pgfpathlineto{\pgfqpoint{3.381965in}{4.196267in}}%
\pgfpathlineto{\pgfqpoint{3.382268in}{4.195070in}}%
\pgfpathlineto{\pgfqpoint{3.382572in}{4.198169in}}%
\pgfpathlineto{\pgfqpoint{3.383381in}{4.214054in}}%
\pgfpathlineto{\pgfqpoint{3.384189in}{4.210812in}}%
\pgfpathlineto{\pgfqpoint{3.384594in}{4.212442in}}%
\pgfpathlineto{\pgfqpoint{3.384998in}{4.209621in}}%
\pgfpathlineto{\pgfqpoint{3.385605in}{4.203792in}}%
\pgfpathlineto{\pgfqpoint{3.386312in}{4.205960in}}%
\pgfpathlineto{\pgfqpoint{3.387525in}{4.206797in}}%
\pgfpathlineto{\pgfqpoint{3.389143in}{4.215861in}}%
\pgfpathlineto{\pgfqpoint{3.389547in}{4.215689in}}%
\pgfpathlineto{\pgfqpoint{3.390053in}{4.213448in}}%
\pgfpathlineto{\pgfqpoint{3.392782in}{4.171017in}}%
\pgfpathlineto{\pgfqpoint{3.393692in}{4.132079in}}%
\pgfpathlineto{\pgfqpoint{3.394299in}{4.138901in}}%
\pgfpathlineto{\pgfqpoint{3.394501in}{4.139524in}}%
\pgfpathlineto{\pgfqpoint{3.394804in}{4.136030in}}%
\pgfpathlineto{\pgfqpoint{3.396624in}{4.120133in}}%
\pgfpathlineto{\pgfqpoint{3.396725in}{4.120141in}}%
\pgfpathlineto{\pgfqpoint{3.397028in}{4.118367in}}%
\pgfpathlineto{\pgfqpoint{3.400566in}{4.036887in}}%
\pgfpathlineto{\pgfqpoint{3.401375in}{4.000616in}}%
\pgfpathlineto{\pgfqpoint{3.401881in}{4.016309in}}%
\pgfpathlineto{\pgfqpoint{3.402790in}{4.048886in}}%
\pgfpathlineto{\pgfqpoint{3.403296in}{4.039287in}}%
\pgfpathlineto{\pgfqpoint{3.403599in}{4.035304in}}%
\pgfpathlineto{\pgfqpoint{3.404509in}{4.036352in}}%
\pgfpathlineto{\pgfqpoint{3.405217in}{4.033124in}}%
\pgfpathlineto{\pgfqpoint{3.405621in}{4.034962in}}%
\pgfpathlineto{\pgfqpoint{3.406632in}{4.061281in}}%
\pgfpathlineto{\pgfqpoint{3.407946in}{4.091385in}}%
\pgfpathlineto{\pgfqpoint{3.408452in}{4.087532in}}%
\pgfpathlineto{\pgfqpoint{3.408755in}{4.086055in}}%
\pgfpathlineto{\pgfqpoint{3.409159in}{4.088962in}}%
\pgfpathlineto{\pgfqpoint{3.410069in}{4.100553in}}%
\pgfpathlineto{\pgfqpoint{3.410777in}{4.098294in}}%
\pgfpathlineto{\pgfqpoint{3.411080in}{4.099263in}}%
\pgfpathlineto{\pgfqpoint{3.412091in}{4.122004in}}%
\pgfpathlineto{\pgfqpoint{3.413405in}{4.143260in}}%
\pgfpathlineto{\pgfqpoint{3.413911in}{4.140007in}}%
\pgfpathlineto{\pgfqpoint{3.414113in}{4.140527in}}%
\pgfpathlineto{\pgfqpoint{3.415023in}{4.151565in}}%
\pgfpathlineto{\pgfqpoint{3.416135in}{4.151036in}}%
\pgfpathlineto{\pgfqpoint{3.416741in}{4.153768in}}%
\pgfpathlineto{\pgfqpoint{3.417853in}{4.171211in}}%
\pgfpathlineto{\pgfqpoint{3.422099in}{4.342930in}}%
\pgfpathlineto{\pgfqpoint{3.422908in}{4.314102in}}%
\pgfpathlineto{\pgfqpoint{3.423110in}{4.311149in}}%
\pgfpathlineto{\pgfqpoint{3.423515in}{4.324501in}}%
\pgfpathlineto{\pgfqpoint{3.425638in}{4.431483in}}%
\pgfpathlineto{\pgfqpoint{3.426042in}{4.428584in}}%
\pgfpathlineto{\pgfqpoint{3.427862in}{4.379653in}}%
\pgfpathlineto{\pgfqpoint{3.428468in}{4.400244in}}%
\pgfpathlineto{\pgfqpoint{3.429782in}{4.473571in}}%
\pgfpathlineto{\pgfqpoint{3.430490in}{4.470322in}}%
\pgfpathlineto{\pgfqpoint{3.431198in}{4.473355in}}%
\pgfpathlineto{\pgfqpoint{3.431501in}{4.471137in}}%
\pgfpathlineto{\pgfqpoint{3.432815in}{4.442328in}}%
\pgfpathlineto{\pgfqpoint{3.433321in}{4.455352in}}%
\pgfpathlineto{\pgfqpoint{3.434332in}{4.503037in}}%
\pgfpathlineto{\pgfqpoint{3.434938in}{4.487522in}}%
\pgfpathlineto{\pgfqpoint{3.435140in}{4.485120in}}%
\pgfpathlineto{\pgfqpoint{3.435545in}{4.492823in}}%
\pgfpathlineto{\pgfqpoint{3.437465in}{4.586433in}}%
\pgfpathlineto{\pgfqpoint{3.438375in}{4.563723in}}%
\pgfpathlineto{\pgfqpoint{3.440094in}{4.536220in}}%
\pgfpathlineto{\pgfqpoint{3.441307in}{4.514075in}}%
\pgfpathlineto{\pgfqpoint{3.445351in}{4.352374in}}%
\pgfpathlineto{\pgfqpoint{3.445856in}{4.358489in}}%
\pgfpathlineto{\pgfqpoint{3.447878in}{4.379465in}}%
\pgfpathlineto{\pgfqpoint{3.448181in}{4.381923in}}%
\pgfpathlineto{\pgfqpoint{3.448586in}{4.375694in}}%
\pgfpathlineto{\pgfqpoint{3.451517in}{4.298478in}}%
\pgfpathlineto{\pgfqpoint{3.451720in}{4.298124in}}%
\pgfpathlineto{\pgfqpoint{3.452225in}{4.299779in}}%
\pgfpathlineto{\pgfqpoint{3.452326in}{4.299879in}}%
\pgfpathlineto{\pgfqpoint{3.452528in}{4.299263in}}%
\pgfpathlineto{\pgfqpoint{3.453135in}{4.287587in}}%
\pgfpathlineto{\pgfqpoint{3.457987in}{4.104551in}}%
\pgfpathlineto{\pgfqpoint{3.458998in}{4.105342in}}%
\pgfpathlineto{\pgfqpoint{3.460009in}{4.113514in}}%
\pgfpathlineto{\pgfqpoint{3.460515in}{4.108184in}}%
\pgfpathlineto{\pgfqpoint{3.465165in}{4.019735in}}%
\pgfpathlineto{\pgfqpoint{3.465974in}{4.008636in}}%
\pgfpathlineto{\pgfqpoint{3.466985in}{3.985311in}}%
\pgfpathlineto{\pgfqpoint{3.467692in}{3.991709in}}%
\pgfpathlineto{\pgfqpoint{3.467895in}{3.992202in}}%
\pgfpathlineto{\pgfqpoint{3.468198in}{3.988864in}}%
\pgfpathlineto{\pgfqpoint{3.470927in}{3.954168in}}%
\pgfpathlineto{\pgfqpoint{3.471433in}{3.958137in}}%
\pgfpathlineto{\pgfqpoint{3.472343in}{3.964610in}}%
\pgfpathlineto{\pgfqpoint{3.472949in}{3.964261in}}%
\pgfpathlineto{\pgfqpoint{3.473859in}{3.965379in}}%
\pgfpathlineto{\pgfqpoint{3.476285in}{3.969123in}}%
\pgfpathlineto{\pgfqpoint{3.478408in}{3.974595in}}%
\pgfpathlineto{\pgfqpoint{3.479925in}{4.000815in}}%
\pgfpathlineto{\pgfqpoint{3.480936in}{3.996142in}}%
\pgfpathlineto{\pgfqpoint{3.481542in}{4.001119in}}%
\pgfpathlineto{\pgfqpoint{3.483766in}{4.016625in}}%
\pgfpathlineto{\pgfqpoint{3.484676in}{4.018722in}}%
\pgfpathlineto{\pgfqpoint{3.485182in}{4.017290in}}%
\pgfpathlineto{\pgfqpoint{3.485889in}{4.014785in}}%
\pgfpathlineto{\pgfqpoint{3.486395in}{4.016438in}}%
\pgfpathlineto{\pgfqpoint{3.491146in}{4.046611in}}%
\pgfpathlineto{\pgfqpoint{3.491348in}{4.046125in}}%
\pgfpathlineto{\pgfqpoint{3.492460in}{4.034041in}}%
\pgfpathlineto{\pgfqpoint{3.492865in}{4.032337in}}%
\pgfpathlineto{\pgfqpoint{3.493572in}{4.034099in}}%
\pgfpathlineto{\pgfqpoint{3.493977in}{4.032861in}}%
\pgfpathlineto{\pgfqpoint{3.494785in}{4.029107in}}%
\pgfpathlineto{\pgfqpoint{3.495392in}{4.030611in}}%
\pgfpathlineto{\pgfqpoint{3.495594in}{4.030719in}}%
\pgfpathlineto{\pgfqpoint{3.495898in}{4.029973in}}%
\pgfpathlineto{\pgfqpoint{3.497414in}{4.025367in}}%
\pgfpathlineto{\pgfqpoint{3.498020in}{4.025542in}}%
\pgfpathlineto{\pgfqpoint{3.498526in}{4.023554in}}%
\pgfpathlineto{\pgfqpoint{3.499436in}{4.005308in}}%
\pgfpathlineto{\pgfqpoint{3.499941in}{3.996166in}}%
\pgfpathlineto{\pgfqpoint{3.500447in}{4.004418in}}%
\pgfpathlineto{\pgfqpoint{3.501660in}{4.040821in}}%
\pgfpathlineto{\pgfqpoint{3.502266in}{4.034057in}}%
\pgfpathlineto{\pgfqpoint{3.502671in}{4.031245in}}%
\pgfpathlineto{\pgfqpoint{3.503075in}{4.035294in}}%
\pgfpathlineto{\pgfqpoint{3.504490in}{4.062087in}}%
\pgfpathlineto{\pgfqpoint{3.505299in}{4.059605in}}%
\pgfpathlineto{\pgfqpoint{3.506816in}{4.055688in}}%
\pgfpathlineto{\pgfqpoint{3.508130in}{4.031231in}}%
\pgfpathlineto{\pgfqpoint{3.509646in}{4.022902in}}%
\pgfpathlineto{\pgfqpoint{3.510354in}{4.019803in}}%
\pgfpathlineto{\pgfqpoint{3.511567in}{4.012385in}}%
\pgfpathlineto{\pgfqpoint{3.512174in}{4.013908in}}%
\pgfpathlineto{\pgfqpoint{3.512881in}{4.015769in}}%
\pgfpathlineto{\pgfqpoint{3.513387in}{4.014513in}}%
\pgfpathlineto{\pgfqpoint{3.513791in}{4.013842in}}%
\pgfpathlineto{\pgfqpoint{3.514297in}{4.014944in}}%
\pgfpathlineto{\pgfqpoint{3.514701in}{4.015681in}}%
\pgfpathlineto{\pgfqpoint{3.515105in}{4.014595in}}%
\pgfpathlineto{\pgfqpoint{3.516521in}{4.009891in}}%
\pgfpathlineto{\pgfqpoint{3.517026in}{4.010115in}}%
\pgfpathlineto{\pgfqpoint{3.518037in}{4.011055in}}%
\pgfpathlineto{\pgfqpoint{3.518441in}{4.009836in}}%
\pgfpathlineto{\pgfqpoint{3.520564in}{3.996517in}}%
\pgfpathlineto{\pgfqpoint{3.523193in}{3.946683in}}%
\pgfpathlineto{\pgfqpoint{3.524911in}{3.916543in}}%
\pgfpathlineto{\pgfqpoint{3.525720in}{3.916651in}}%
\pgfpathlineto{\pgfqpoint{3.526630in}{3.917613in}}%
\pgfpathlineto{\pgfqpoint{3.529764in}{3.925520in}}%
\pgfpathlineto{\pgfqpoint{3.529966in}{3.925364in}}%
\pgfpathlineto{\pgfqpoint{3.530269in}{3.925297in}}%
\pgfpathlineto{\pgfqpoint{3.530471in}{3.925923in}}%
\pgfpathlineto{\pgfqpoint{3.531381in}{3.931154in}}%
\pgfpathlineto{\pgfqpoint{3.532089in}{3.929108in}}%
\pgfpathlineto{\pgfqpoint{3.534313in}{3.924294in}}%
\pgfpathlineto{\pgfqpoint{3.534717in}{3.925222in}}%
\pgfpathlineto{\pgfqpoint{3.538862in}{3.943481in}}%
\pgfpathlineto{\pgfqpoint{3.539267in}{3.943198in}}%
\pgfpathlineto{\pgfqpoint{3.540682in}{3.943485in}}%
\pgfpathlineto{\pgfqpoint{3.541288in}{3.944450in}}%
\pgfpathlineto{\pgfqpoint{3.542805in}{3.950150in}}%
\pgfpathlineto{\pgfqpoint{3.543411in}{3.948687in}}%
\pgfpathlineto{\pgfqpoint{3.544523in}{3.945300in}}%
\pgfpathlineto{\pgfqpoint{3.545029in}{3.946259in}}%
\pgfpathlineto{\pgfqpoint{3.546646in}{3.956036in}}%
\pgfpathlineto{\pgfqpoint{3.547657in}{3.952777in}}%
\pgfpathlineto{\pgfqpoint{3.549982in}{3.948852in}}%
\pgfpathlineto{\pgfqpoint{3.550286in}{3.949296in}}%
\pgfpathlineto{\pgfqpoint{3.550589in}{3.949545in}}%
\pgfpathlineto{\pgfqpoint{3.550892in}{3.948492in}}%
\pgfpathlineto{\pgfqpoint{3.551903in}{3.944034in}}%
\pgfpathlineto{\pgfqpoint{3.552308in}{3.945512in}}%
\pgfpathlineto{\pgfqpoint{3.554835in}{3.956259in}}%
\pgfpathlineto{\pgfqpoint{3.555442in}{3.957541in}}%
\pgfpathlineto{\pgfqpoint{3.556149in}{3.959587in}}%
\pgfpathlineto{\pgfqpoint{3.556655in}{3.958105in}}%
\pgfpathlineto{\pgfqpoint{3.558778in}{3.943993in}}%
\pgfpathlineto{\pgfqpoint{3.560092in}{3.946759in}}%
\pgfpathlineto{\pgfqpoint{3.560901in}{3.949105in}}%
\pgfpathlineto{\pgfqpoint{3.561406in}{3.948125in}}%
\pgfpathlineto{\pgfqpoint{3.563428in}{3.940538in}}%
\pgfpathlineto{\pgfqpoint{3.564237in}{3.941753in}}%
\pgfpathlineto{\pgfqpoint{3.564843in}{3.941852in}}%
\pgfpathlineto{\pgfqpoint{3.565147in}{3.941395in}}%
\pgfpathlineto{\pgfqpoint{3.566056in}{3.936230in}}%
\pgfpathlineto{\pgfqpoint{3.567674in}{3.931309in}}%
\pgfpathlineto{\pgfqpoint{3.568786in}{3.929463in}}%
\pgfpathlineto{\pgfqpoint{3.569291in}{3.930215in}}%
\pgfpathlineto{\pgfqpoint{3.570403in}{3.933311in}}%
\pgfpathlineto{\pgfqpoint{3.571111in}{3.931890in}}%
\pgfpathlineto{\pgfqpoint{3.572830in}{3.927221in}}%
\pgfpathlineto{\pgfqpoint{3.573841in}{3.921320in}}%
\pgfpathlineto{\pgfqpoint{3.574447in}{3.921395in}}%
\pgfpathlineto{\pgfqpoint{3.575256in}{3.917817in}}%
\pgfpathlineto{\pgfqpoint{3.575862in}{3.915657in}}%
\pgfpathlineto{\pgfqpoint{3.576469in}{3.917203in}}%
\pgfpathlineto{\pgfqpoint{3.578592in}{3.921673in}}%
\pgfpathlineto{\pgfqpoint{3.578693in}{3.921631in}}%
\pgfpathlineto{\pgfqpoint{3.580311in}{3.920481in}}%
\pgfpathlineto{\pgfqpoint{3.580513in}{3.921221in}}%
\pgfpathlineto{\pgfqpoint{3.581321in}{3.933827in}}%
\pgfpathlineto{\pgfqpoint{3.583141in}{3.951325in}}%
\pgfpathlineto{\pgfqpoint{3.584354in}{3.960087in}}%
\pgfpathlineto{\pgfqpoint{3.588398in}{3.986237in}}%
\pgfpathlineto{\pgfqpoint{3.589106in}{3.994082in}}%
\pgfpathlineto{\pgfqpoint{3.592644in}{4.047094in}}%
\pgfpathlineto{\pgfqpoint{3.593453in}{4.055347in}}%
\pgfpathlineto{\pgfqpoint{3.595879in}{4.109851in}}%
\pgfpathlineto{\pgfqpoint{3.597193in}{4.104928in}}%
\pgfpathlineto{\pgfqpoint{3.597597in}{4.104556in}}%
\pgfpathlineto{\pgfqpoint{3.597901in}{4.105368in}}%
\pgfpathlineto{\pgfqpoint{3.598710in}{4.116168in}}%
\pgfpathlineto{\pgfqpoint{3.600327in}{4.136649in}}%
\pgfpathlineto{\pgfqpoint{3.600731in}{4.134568in}}%
\pgfpathlineto{\pgfqpoint{3.602753in}{4.101646in}}%
\pgfpathlineto{\pgfqpoint{3.604169in}{4.107693in}}%
\pgfpathlineto{\pgfqpoint{3.604775in}{4.109434in}}%
\pgfpathlineto{\pgfqpoint{3.605180in}{4.108179in}}%
\pgfpathlineto{\pgfqpoint{3.606595in}{4.093252in}}%
\pgfpathlineto{\pgfqpoint{3.607707in}{4.088298in}}%
\pgfpathlineto{\pgfqpoint{3.608111in}{4.088934in}}%
\pgfpathlineto{\pgfqpoint{3.608819in}{4.090629in}}%
\pgfpathlineto{\pgfqpoint{3.609122in}{4.089552in}}%
\pgfpathlineto{\pgfqpoint{3.609931in}{4.073842in}}%
\pgfpathlineto{\pgfqpoint{3.611953in}{4.037853in}}%
\pgfpathlineto{\pgfqpoint{3.612155in}{4.038178in}}%
\pgfpathlineto{\pgfqpoint{3.612964in}{4.041469in}}%
\pgfpathlineto{\pgfqpoint{3.613368in}{4.039587in}}%
\pgfpathlineto{\pgfqpoint{3.614278in}{4.019444in}}%
\pgfpathlineto{\pgfqpoint{3.616502in}{3.979401in}}%
\pgfpathlineto{\pgfqpoint{3.617715in}{3.978553in}}%
\pgfpathlineto{\pgfqpoint{3.618423in}{3.972962in}}%
\pgfpathlineto{\pgfqpoint{3.621456in}{3.942872in}}%
\pgfpathlineto{\pgfqpoint{3.621557in}{3.942880in}}%
\pgfpathlineto{\pgfqpoint{3.623174in}{3.943987in}}%
\pgfpathlineto{\pgfqpoint{3.623781in}{3.945991in}}%
\pgfpathlineto{\pgfqpoint{3.624185in}{3.944099in}}%
\pgfpathlineto{\pgfqpoint{3.625701in}{3.931649in}}%
\pgfpathlineto{\pgfqpoint{3.626308in}{3.934018in}}%
\pgfpathlineto{\pgfqpoint{3.627521in}{3.937751in}}%
\pgfpathlineto{\pgfqpoint{3.627926in}{3.937443in}}%
\pgfpathlineto{\pgfqpoint{3.628835in}{3.934548in}}%
\pgfpathlineto{\pgfqpoint{3.630857in}{3.923586in}}%
\pgfpathlineto{\pgfqpoint{3.631464in}{3.924646in}}%
\pgfpathlineto{\pgfqpoint{3.631868in}{3.925240in}}%
\pgfpathlineto{\pgfqpoint{3.632374in}{3.924076in}}%
\pgfpathlineto{\pgfqpoint{3.635103in}{3.916816in}}%
\pgfpathlineto{\pgfqpoint{3.636822in}{3.915655in}}%
\pgfpathlineto{\pgfqpoint{3.638136in}{3.917108in}}%
\pgfpathlineto{\pgfqpoint{3.639551in}{3.917292in}}%
\pgfpathlineto{\pgfqpoint{3.639652in}{3.917184in}}%
\pgfpathlineto{\pgfqpoint{3.641270in}{3.916409in}}%
\pgfpathlineto{\pgfqpoint{3.641472in}{3.916577in}}%
\pgfpathlineto{\pgfqpoint{3.643191in}{3.918233in}}%
\pgfpathlineto{\pgfqpoint{3.643393in}{3.917544in}}%
\pgfpathlineto{\pgfqpoint{3.644505in}{3.909050in}}%
\pgfpathlineto{\pgfqpoint{3.645415in}{3.911777in}}%
\pgfpathlineto{\pgfqpoint{3.645718in}{3.912037in}}%
\pgfpathlineto{\pgfqpoint{3.646122in}{3.910999in}}%
\pgfpathlineto{\pgfqpoint{3.647942in}{3.906061in}}%
\pgfpathlineto{\pgfqpoint{3.648245in}{3.906268in}}%
\pgfpathlineto{\pgfqpoint{3.650166in}{3.910349in}}%
\pgfpathlineto{\pgfqpoint{3.650975in}{3.912900in}}%
\pgfpathlineto{\pgfqpoint{3.651581in}{3.912084in}}%
\pgfpathlineto{\pgfqpoint{3.651986in}{3.911968in}}%
\pgfpathlineto{\pgfqpoint{3.652289in}{3.912599in}}%
\pgfpathlineto{\pgfqpoint{3.654412in}{3.915842in}}%
\pgfpathlineto{\pgfqpoint{3.656333in}{3.916817in}}%
\pgfpathlineto{\pgfqpoint{3.658860in}{3.916926in}}%
\pgfpathlineto{\pgfqpoint{3.660377in}{3.917942in}}%
\pgfpathlineto{\pgfqpoint{3.661590in}{3.920880in}}%
\pgfpathlineto{\pgfqpoint{3.662297in}{3.920288in}}%
\pgfpathlineto{\pgfqpoint{3.664724in}{3.916989in}}%
\pgfpathlineto{\pgfqpoint{3.666139in}{3.912415in}}%
\pgfpathlineto{\pgfqpoint{3.666644in}{3.912588in}}%
\pgfpathlineto{\pgfqpoint{3.668565in}{3.911175in}}%
\pgfpathlineto{\pgfqpoint{3.670183in}{3.908143in}}%
\pgfpathlineto{\pgfqpoint{3.671800in}{3.904327in}}%
\pgfpathlineto{\pgfqpoint{3.672912in}{3.900725in}}%
\pgfpathlineto{\pgfqpoint{3.674530in}{3.899116in}}%
\pgfpathlineto{\pgfqpoint{3.676653in}{3.898037in}}%
\pgfpathlineto{\pgfqpoint{3.684942in}{3.893587in}}%
\pgfpathlineto{\pgfqpoint{3.688885in}{3.893509in}}%
\pgfpathlineto{\pgfqpoint{3.690199in}{3.894260in}}%
\pgfpathlineto{\pgfqpoint{3.692625in}{3.895837in}}%
\pgfpathlineto{\pgfqpoint{3.696063in}{3.898381in}}%
\pgfpathlineto{\pgfqpoint{3.698792in}{3.899539in}}%
\pgfpathlineto{\pgfqpoint{3.699904in}{3.901244in}}%
\pgfpathlineto{\pgfqpoint{3.700511in}{3.900988in}}%
\pgfpathlineto{\pgfqpoint{3.710519in}{3.900425in}}%
\pgfpathlineto{\pgfqpoint{3.712440in}{3.900284in}}%
\pgfpathlineto{\pgfqpoint{3.713349in}{3.899197in}}%
\pgfpathlineto{\pgfqpoint{3.715068in}{3.897072in}}%
\pgfpathlineto{\pgfqpoint{3.715270in}{3.897212in}}%
\pgfpathlineto{\pgfqpoint{3.716989in}{3.900505in}}%
\pgfpathlineto{\pgfqpoint{3.718202in}{3.901151in}}%
\pgfpathlineto{\pgfqpoint{3.718404in}{3.900971in}}%
\pgfpathlineto{\pgfqpoint{3.721639in}{3.897725in}}%
\pgfpathlineto{\pgfqpoint{3.721942in}{3.898020in}}%
\pgfpathlineto{\pgfqpoint{3.723661in}{3.898357in}}%
\pgfpathlineto{\pgfqpoint{3.726694in}{3.898388in}}%
\pgfpathlineto{\pgfqpoint{3.729322in}{3.899078in}}%
\pgfpathlineto{\pgfqpoint{3.731748in}{3.898366in}}%
\pgfpathlineto{\pgfqpoint{3.734579in}{3.897340in}}%
\pgfpathlineto{\pgfqpoint{3.735994in}{3.896405in}}%
\pgfpathlineto{\pgfqpoint{3.739027in}{3.892942in}}%
\pgfpathlineto{\pgfqpoint{3.749541in}{3.891920in}}%
\pgfpathlineto{\pgfqpoint{3.751765in}{3.892058in}}%
\pgfpathlineto{\pgfqpoint{3.754899in}{3.892589in}}%
\pgfpathlineto{\pgfqpoint{3.757325in}{3.894319in}}%
\pgfpathlineto{\pgfqpoint{3.759751in}{3.895205in}}%
\pgfpathlineto{\pgfqpoint{3.763997in}{3.896907in}}%
\pgfpathlineto{\pgfqpoint{3.766424in}{3.897686in}}%
\pgfpathlineto{\pgfqpoint{3.783913in}{3.896741in}}%
\pgfpathlineto{\pgfqpoint{3.785631in}{3.896250in}}%
\pgfpathlineto{\pgfqpoint{3.799481in}{3.893798in}}%
\pgfpathlineto{\pgfqpoint{3.800593in}{3.894193in}}%
\pgfpathlineto{\pgfqpoint{3.802817in}{3.894699in}}%
\pgfpathlineto{\pgfqpoint{3.817274in}{3.895298in}}%
\pgfpathlineto{\pgfqpoint{3.818183in}{3.895764in}}%
\pgfpathlineto{\pgfqpoint{3.818588in}{3.895413in}}%
\pgfpathlineto{\pgfqpoint{3.821924in}{3.893908in}}%
\pgfpathlineto{\pgfqpoint{3.832741in}{3.894228in}}%
\pgfpathlineto{\pgfqpoint{3.836279in}{3.893627in}}%
\pgfpathlineto{\pgfqpoint{3.837897in}{3.892587in}}%
\pgfpathlineto{\pgfqpoint{3.839716in}{3.892198in}}%
\pgfpathlineto{\pgfqpoint{3.843962in}{3.891172in}}%
\pgfpathlineto{\pgfqpoint{3.845984in}{3.890988in}}%
\pgfpathlineto{\pgfqpoint{3.852454in}{3.890192in}}%
\pgfpathlineto{\pgfqpoint{3.861654in}{3.892590in}}%
\pgfpathlineto{\pgfqpoint{3.870246in}{3.893462in}}%
\pgfpathlineto{\pgfqpoint{3.883085in}{3.894715in}}%
\pgfpathlineto{\pgfqpoint{3.886320in}{3.894628in}}%
\pgfpathlineto{\pgfqpoint{3.890667in}{3.895452in}}%
\pgfpathlineto{\pgfqpoint{3.892891in}{3.896561in}}%
\pgfpathlineto{\pgfqpoint{3.894307in}{3.895200in}}%
\pgfpathlineto{\pgfqpoint{3.895520in}{3.894191in}}%
\pgfpathlineto{\pgfqpoint{3.895924in}{3.894521in}}%
\pgfpathlineto{\pgfqpoint{3.897340in}{3.894444in}}%
\pgfpathlineto{\pgfqpoint{3.898856in}{3.894774in}}%
\pgfpathlineto{\pgfqpoint{3.901282in}{3.895291in}}%
\pgfpathlineto{\pgfqpoint{3.911493in}{3.893442in}}%
\pgfpathlineto{\pgfqpoint{3.913211in}{3.892782in}}%
\pgfpathlineto{\pgfqpoint{3.915031in}{3.893621in}}%
\pgfpathlineto{\pgfqpoint{3.915941in}{3.894254in}}%
\pgfpathlineto{\pgfqpoint{3.916345in}{3.893880in}}%
\pgfpathlineto{\pgfqpoint{3.918165in}{3.893439in}}%
\pgfpathlineto{\pgfqpoint{3.929588in}{3.895344in}}%
\pgfpathlineto{\pgfqpoint{3.931812in}{3.896059in}}%
\pgfpathlineto{\pgfqpoint{3.935856in}{3.894990in}}%
\pgfpathlineto{\pgfqpoint{3.939192in}{3.894287in}}%
\pgfpathlineto{\pgfqpoint{3.942225in}{3.893859in}}%
\pgfpathlineto{\pgfqpoint{3.946976in}{3.892829in}}%
\pgfpathlineto{\pgfqpoint{3.948796in}{3.892261in}}%
\pgfpathlineto{\pgfqpoint{3.950616in}{3.890668in}}%
\pgfpathlineto{\pgfqpoint{3.957591in}{3.889543in}}%
\pgfpathlineto{\pgfqpoint{3.967802in}{3.889872in}}%
\pgfpathlineto{\pgfqpoint{3.971643in}{3.889491in}}%
\pgfpathlineto{\pgfqpoint{3.981955in}{3.890698in}}%
\pgfpathlineto{\pgfqpoint{3.983673in}{3.891135in}}%
\pgfpathlineto{\pgfqpoint{3.989840in}{3.891566in}}%
\pgfpathlineto{\pgfqpoint{3.997119in}{3.891156in}}%
\pgfpathlineto{\pgfqpoint{3.999747in}{3.890659in}}%
\pgfpathlineto{\pgfqpoint{4.002072in}{3.889978in}}%
\pgfpathlineto{\pgfqpoint{4.003892in}{3.889295in}}%
\pgfpathlineto{\pgfqpoint{4.023201in}{3.889932in}}%
\pgfpathlineto{\pgfqpoint{4.025930in}{3.890244in}}%
\pgfpathlineto{\pgfqpoint{4.098920in}{3.890263in}}%
\pgfpathlineto{\pgfqpoint{4.110444in}{3.891087in}}%
\pgfpathlineto{\pgfqpoint{4.116712in}{3.890333in}}%
\pgfpathlineto{\pgfqpoint{4.227713in}{3.889460in}}%
\pgfpathlineto{\pgfqpoint{4.230746in}{3.889980in}}%
\pgfpathlineto{\pgfqpoint{4.247527in}{3.891230in}}%
\pgfpathlineto{\pgfqpoint{4.253087in}{3.890596in}}%
\pgfpathlineto{\pgfqpoint{4.262085in}{3.889236in}}%
\pgfpathlineto{\pgfqpoint{4.266027in}{3.889094in}}%
\pgfpathlineto{\pgfqpoint{4.277855in}{3.888917in}}%
\pgfpathlineto{\pgfqpoint{4.328200in}{3.889094in}}%
\pgfpathlineto{\pgfqpoint{4.332749in}{3.888856in}}%
\pgfpathlineto{\pgfqpoint{4.367727in}{3.889154in}}%
\pgfpathlineto{\pgfqpoint{4.371771in}{3.888821in}}%
\pgfpathlineto{\pgfqpoint{4.381779in}{3.888825in}}%
\pgfpathlineto{\pgfqpoint{4.409681in}{3.888512in}}%
\pgfpathlineto{\pgfqpoint{4.429596in}{3.889613in}}%
\pgfpathlineto{\pgfqpoint{4.435561in}{3.889307in}}%
\pgfpathlineto{\pgfqpoint{4.442638in}{3.889016in}}%
\pgfpathlineto{\pgfqpoint{4.454769in}{3.888509in}}%
\pgfpathlineto{\pgfqpoint{4.479233in}{3.888638in}}%
\pgfpathlineto{\pgfqpoint{4.573655in}{3.888171in}}%
\pgfpathlineto{\pgfqpoint{4.601759in}{3.888700in}}%
\pgfpathlineto{\pgfqpoint{4.608330in}{3.889044in}}%
\pgfpathlineto{\pgfqpoint{4.891998in}{3.888242in}}%
\pgfpathlineto{\pgfqpoint{4.975097in}{3.888102in}}%
\pgfpathlineto{\pgfqpoint{5.745833in}{3.888028in}}%
\pgfpathlineto{\pgfqpoint{5.745833in}{3.888028in}}%
\pgfusepath{stroke}%
\end{pgfscope}%
\begin{pgfscope}%
\pgfpathrectangle{\pgfqpoint{0.691161in}{3.729888in}}{\pgfqpoint{5.054672in}{0.911907in}}%
\pgfusepath{clip}%
\pgfsetbuttcap%
\pgfsetroundjoin%
\pgfsetlinewidth{2.007500pt}%
\definecolor{currentstroke}{rgb}{0.172549,0.627451,0.172549}%
\pgfsetstrokecolor{currentstroke}%
\pgfsetdash{{7.400000pt}{3.200000pt}}{0.000000pt}%
\pgfpathmoveto{\pgfqpoint{0.691161in}{4.226716in}}%
\pgfpathlineto{\pgfqpoint{5.745833in}{4.226716in}}%
\pgfusepath{stroke}%
\end{pgfscope}%
\begin{pgfscope}%
\pgfpathrectangle{\pgfqpoint{0.691161in}{3.729888in}}{\pgfqpoint{5.054672in}{0.911907in}}%
\pgfusepath{clip}%
\pgfsetrectcap%
\pgfsetroundjoin%
\pgfsetlinewidth{2.007500pt}%
\definecolor{currentstroke}{rgb}{0.172549,0.627451,0.172549}%
\pgfsetstrokecolor{currentstroke}%
\pgfsetdash{}{0pt}%
\pgfpathmoveto{\pgfqpoint{0.691060in}{3.901446in}}%
\pgfpathlineto{\pgfqpoint{3.336473in}{3.901446in}}%
\pgfpathlineto{\pgfqpoint{3.337585in}{3.836393in}}%
\pgfpathlineto{\pgfqpoint{3.377921in}{3.836393in}}%
\pgfpathlineto{\pgfqpoint{3.379033in}{3.901446in}}%
\pgfpathlineto{\pgfqpoint{3.420179in}{3.901446in}}%
\pgfpathlineto{\pgfqpoint{3.421291in}{3.836393in}}%
\pgfpathlineto{\pgfqpoint{3.454247in}{3.836393in}}%
\pgfpathlineto{\pgfqpoint{3.455359in}{3.901446in}}%
\pgfpathlineto{\pgfqpoint{5.745833in}{3.901446in}}%
\pgfpathlineto{\pgfqpoint{5.745833in}{3.901446in}}%
\pgfusepath{stroke}%
\end{pgfscope}%
\begin{pgfscope}%
\pgfpathrectangle{\pgfqpoint{0.691161in}{3.729888in}}{\pgfqpoint{5.054672in}{0.911907in}}%
\pgfusepath{clip}%
\pgfsetrectcap%
\pgfsetroundjoin%
\pgfsetlinewidth{3.011250pt}%
\definecolor{currentstroke}{rgb}{0.839216,0.152941,0.156863}%
\pgfsetstrokecolor{currentstroke}%
\pgfsetdash{}{0pt}%
\pgfpathmoveto{\pgfqpoint{0.691060in}{3.901446in}}%
\pgfpathlineto{\pgfqpoint{2.729913in}{3.901446in}}%
\pgfpathlineto{\pgfqpoint{2.731025in}{3.771339in}}%
\pgfpathlineto{\pgfqpoint{4.060808in}{3.771339in}}%
\pgfpathlineto{\pgfqpoint{4.061920in}{3.901446in}}%
\pgfpathlineto{\pgfqpoint{5.745833in}{3.901446in}}%
\pgfpathlineto{\pgfqpoint{5.745833in}{3.901446in}}%
\pgfusepath{stroke}%
\end{pgfscope}%
\begin{pgfscope}%
\pgfsetrectcap%
\pgfsetmiterjoin%
\pgfsetlinewidth{0.803000pt}%
\definecolor{currentstroke}{rgb}{0.737255,0.737255,0.737255}%
\pgfsetstrokecolor{currentstroke}%
\pgfsetdash{}{0pt}%
\pgfpathmoveto{\pgfqpoint{0.691161in}{3.729888in}}%
\pgfpathlineto{\pgfqpoint{0.691161in}{4.641796in}}%
\pgfusepath{stroke}%
\end{pgfscope}%
\begin{pgfscope}%
\pgfsetrectcap%
\pgfsetmiterjoin%
\pgfsetlinewidth{0.803000pt}%
\definecolor{currentstroke}{rgb}{0.737255,0.737255,0.737255}%
\pgfsetstrokecolor{currentstroke}%
\pgfsetdash{}{0pt}%
\pgfpathmoveto{\pgfqpoint{5.745833in}{3.729888in}}%
\pgfpathlineto{\pgfqpoint{5.745833in}{4.641796in}}%
\pgfusepath{stroke}%
\end{pgfscope}%
\begin{pgfscope}%
\pgfsetrectcap%
\pgfsetmiterjoin%
\pgfsetlinewidth{0.803000pt}%
\definecolor{currentstroke}{rgb}{0.737255,0.737255,0.737255}%
\pgfsetstrokecolor{currentstroke}%
\pgfsetdash{}{0pt}%
\pgfpathmoveto{\pgfqpoint{0.691161in}{3.729888in}}%
\pgfpathlineto{\pgfqpoint{5.745833in}{3.729888in}}%
\pgfusepath{stroke}%
\end{pgfscope}%
\begin{pgfscope}%
\pgfsetrectcap%
\pgfsetmiterjoin%
\pgfsetlinewidth{0.803000pt}%
\definecolor{currentstroke}{rgb}{0.737255,0.737255,0.737255}%
\pgfsetstrokecolor{currentstroke}%
\pgfsetdash{}{0pt}%
\pgfpathmoveto{\pgfqpoint{0.691161in}{4.641796in}}%
\pgfpathlineto{\pgfqpoint{5.745833in}{4.641796in}}%
\pgfusepath{stroke}%
\end{pgfscope}%
\begin{pgfscope}%
\pgfsetbuttcap%
\pgfsetmiterjoin%
\definecolor{currentfill}{rgb}{0.933333,0.933333,0.933333}%
\pgfsetfillcolor{currentfill}%
\pgfsetfillopacity{0.800000}%
\pgfsetlinewidth{0.501875pt}%
\definecolor{currentstroke}{rgb}{0.800000,0.800000,0.800000}%
\pgfsetstrokecolor{currentstroke}%
\pgfsetstrokeopacity{0.800000}%
\pgfsetdash{}{0pt}%
\pgfpathmoveto{\pgfqpoint{4.413888in}{3.949852in}}%
\pgfpathlineto{\pgfqpoint{5.648611in}{3.949852in}}%
\pgfpathquadraticcurveto{\pgfqpoint{5.676389in}{3.949852in}}{\pgfqpoint{5.676389in}{3.977630in}}%
\pgfpathlineto{\pgfqpoint{5.676389in}{4.544574in}}%
\pgfpathquadraticcurveto{\pgfqpoint{5.676389in}{4.572351in}}{\pgfqpoint{5.648611in}{4.572351in}}%
\pgfpathlineto{\pgfqpoint{4.413888in}{4.572351in}}%
\pgfpathquadraticcurveto{\pgfqpoint{4.386111in}{4.572351in}}{\pgfqpoint{4.386111in}{4.544574in}}%
\pgfpathlineto{\pgfqpoint{4.386111in}{3.977630in}}%
\pgfpathquadraticcurveto{\pgfqpoint{4.386111in}{3.949852in}}{\pgfqpoint{4.413888in}{3.949852in}}%
\pgfpathlineto{\pgfqpoint{4.413888in}{3.949852in}}%
\pgfpathclose%
\pgfusepath{stroke,fill}%
\end{pgfscope}%
\begin{pgfscope}%
\pgfsetbuttcap%
\pgfsetroundjoin%
\pgfsetlinewidth{2.007500pt}%
\definecolor{currentstroke}{rgb}{0.172549,0.627451,0.172549}%
\pgfsetstrokecolor{currentstroke}%
\pgfsetdash{{7.400000pt}{3.200000pt}}{0.000000pt}%
\pgfpathmoveto{\pgfqpoint{4.441666in}{4.468185in}}%
\pgfpathlineto{\pgfqpoint{4.719444in}{4.468185in}}%
\pgfusepath{stroke}%
\end{pgfscope}%
\begin{pgfscope}%
\definecolor{textcolor}{rgb}{0.000000,0.000000,0.000000}%
\pgfsetstrokecolor{textcolor}%
\pgfsetfillcolor{textcolor}%
\pgftext[x=4.830555in,y=4.419574in,left,base]{\color{textcolor}\rmfamily\fontsize{10.000000}{12.000000}\selectfont Seuil = 5}%
\end{pgfscope}%
\begin{pgfscope}%
\pgfsetrectcap%
\pgfsetroundjoin%
\pgfsetlinewidth{2.007500pt}%
\definecolor{currentstroke}{rgb}{0.172549,0.627451,0.172549}%
\pgfsetstrokecolor{currentstroke}%
\pgfsetdash{}{0pt}%
\pgfpathmoveto{\pgfqpoint{4.441666in}{4.274574in}}%
\pgfpathlineto{\pgfqpoint{4.580555in}{4.274574in}}%
\pgfpathlineto{\pgfqpoint{4.719444in}{4.274574in}}%
\pgfusepath{stroke}%
\end{pgfscope}%
\begin{pgfscope}%
\definecolor{textcolor}{rgb}{0.000000,0.000000,0.000000}%
\pgfsetstrokecolor{textcolor}%
\pgfsetfillcolor{textcolor}%
\pgftext[x=4.830555in,y=4.225963in,left,base]{\color{textcolor}\rmfamily\fontsize{10.000000}{12.000000}\selectfont Dépassement}%
\end{pgfscope}%
\begin{pgfscope}%
\pgfsetrectcap%
\pgfsetroundjoin%
\pgfsetlinewidth{3.011250pt}%
\definecolor{currentstroke}{rgb}{0.839216,0.152941,0.156863}%
\pgfsetstrokecolor{currentstroke}%
\pgfsetdash{}{0pt}%
\pgfpathmoveto{\pgfqpoint{4.441666in}{4.080963in}}%
\pgfpathlineto{\pgfqpoint{4.580555in}{4.080963in}}%
\pgfpathlineto{\pgfqpoint{4.719444in}{4.080963in}}%
\pgfusepath{stroke}%
\end{pgfscope}%
\begin{pgfscope}%
\definecolor{textcolor}{rgb}{0.000000,0.000000,0.000000}%
\pgfsetstrokecolor{textcolor}%
\pgfsetfillcolor{textcolor}%
\pgftext[x=4.830555in,y=4.032352in,left,base]{\color{textcolor}\rmfamily\fontsize{10.000000}{12.000000}\selectfont Secousse}%
\end{pgfscope}%
\begin{pgfscope}%
\pgfsetbuttcap%
\pgfsetmiterjoin%
\definecolor{currentfill}{rgb}{0.933333,0.933333,0.933333}%
\pgfsetfillcolor{currentfill}%
\pgfsetlinewidth{0.000000pt}%
\definecolor{currentstroke}{rgb}{0.000000,0.000000,0.000000}%
\pgfsetstrokecolor{currentstroke}%
\pgfsetstrokeopacity{0.000000}%
\pgfsetdash{}{0pt}%
\pgfpathmoveto{\pgfqpoint{0.691161in}{2.667981in}}%
\pgfpathlineto{\pgfqpoint{5.745833in}{2.667981in}}%
\pgfpathlineto{\pgfqpoint{5.745833in}{3.579888in}}%
\pgfpathlineto{\pgfqpoint{0.691161in}{3.579888in}}%
\pgfpathlineto{\pgfqpoint{0.691161in}{2.667981in}}%
\pgfpathclose%
\pgfusepath{fill}%
\end{pgfscope}%
\begin{pgfscope}%
\pgfpathrectangle{\pgfqpoint{0.691161in}{2.667981in}}{\pgfqpoint{5.054672in}{0.911907in}}%
\pgfusepath{clip}%
\pgfsetbuttcap%
\pgfsetroundjoin%
\pgfsetlinewidth{0.501875pt}%
\definecolor{currentstroke}{rgb}{0.698039,0.698039,0.698039}%
\pgfsetstrokecolor{currentstroke}%
\pgfsetdash{{1.850000pt}{0.800000pt}}{0.000000pt}%
\pgfpathmoveto{\pgfqpoint{0.691161in}{2.667981in}}%
\pgfpathlineto{\pgfqpoint{0.691161in}{3.579888in}}%
\pgfusepath{stroke}%
\end{pgfscope}%
\begin{pgfscope}%
\pgfsetbuttcap%
\pgfsetroundjoin%
\definecolor{currentfill}{rgb}{0.000000,0.000000,0.000000}%
\pgfsetfillcolor{currentfill}%
\pgfsetlinewidth{0.803000pt}%
\definecolor{currentstroke}{rgb}{0.000000,0.000000,0.000000}%
\pgfsetstrokecolor{currentstroke}%
\pgfsetdash{}{0pt}%
\pgfsys@defobject{currentmarker}{\pgfqpoint{0.000000in}{0.000000in}}{\pgfqpoint{0.000000in}{0.048611in}}{%
\pgfpathmoveto{\pgfqpoint{0.000000in}{0.000000in}}%
\pgfpathlineto{\pgfqpoint{0.000000in}{0.048611in}}%
\pgfusepath{stroke,fill}%
}%
\begin{pgfscope}%
\pgfsys@transformshift{0.691161in}{2.667981in}%
\pgfsys@useobject{currentmarker}{}%
\end{pgfscope}%
\end{pgfscope}%
\begin{pgfscope}%
\pgfpathrectangle{\pgfqpoint{0.691161in}{2.667981in}}{\pgfqpoint{5.054672in}{0.911907in}}%
\pgfusepath{clip}%
\pgfsetbuttcap%
\pgfsetroundjoin%
\pgfsetlinewidth{0.501875pt}%
\definecolor{currentstroke}{rgb}{0.698039,0.698039,0.698039}%
\pgfsetstrokecolor{currentstroke}%
\pgfsetdash{{1.850000pt}{0.800000pt}}{0.000000pt}%
\pgfpathmoveto{\pgfqpoint{1.702096in}{2.667981in}}%
\pgfpathlineto{\pgfqpoint{1.702096in}{3.579888in}}%
\pgfusepath{stroke}%
\end{pgfscope}%
\begin{pgfscope}%
\pgfsetbuttcap%
\pgfsetroundjoin%
\definecolor{currentfill}{rgb}{0.000000,0.000000,0.000000}%
\pgfsetfillcolor{currentfill}%
\pgfsetlinewidth{0.803000pt}%
\definecolor{currentstroke}{rgb}{0.000000,0.000000,0.000000}%
\pgfsetstrokecolor{currentstroke}%
\pgfsetdash{}{0pt}%
\pgfsys@defobject{currentmarker}{\pgfqpoint{0.000000in}{0.000000in}}{\pgfqpoint{0.000000in}{0.048611in}}{%
\pgfpathmoveto{\pgfqpoint{0.000000in}{0.000000in}}%
\pgfpathlineto{\pgfqpoint{0.000000in}{0.048611in}}%
\pgfusepath{stroke,fill}%
}%
\begin{pgfscope}%
\pgfsys@transformshift{1.702096in}{2.667981in}%
\pgfsys@useobject{currentmarker}{}%
\end{pgfscope}%
\end{pgfscope}%
\begin{pgfscope}%
\pgfpathrectangle{\pgfqpoint{0.691161in}{2.667981in}}{\pgfqpoint{5.054672in}{0.911907in}}%
\pgfusepath{clip}%
\pgfsetbuttcap%
\pgfsetroundjoin%
\pgfsetlinewidth{0.501875pt}%
\definecolor{currentstroke}{rgb}{0.698039,0.698039,0.698039}%
\pgfsetstrokecolor{currentstroke}%
\pgfsetdash{{1.850000pt}{0.800000pt}}{0.000000pt}%
\pgfpathmoveto{\pgfqpoint{2.713030in}{2.667981in}}%
\pgfpathlineto{\pgfqpoint{2.713030in}{3.579888in}}%
\pgfusepath{stroke}%
\end{pgfscope}%
\begin{pgfscope}%
\pgfsetbuttcap%
\pgfsetroundjoin%
\definecolor{currentfill}{rgb}{0.000000,0.000000,0.000000}%
\pgfsetfillcolor{currentfill}%
\pgfsetlinewidth{0.803000pt}%
\definecolor{currentstroke}{rgb}{0.000000,0.000000,0.000000}%
\pgfsetstrokecolor{currentstroke}%
\pgfsetdash{}{0pt}%
\pgfsys@defobject{currentmarker}{\pgfqpoint{0.000000in}{0.000000in}}{\pgfqpoint{0.000000in}{0.048611in}}{%
\pgfpathmoveto{\pgfqpoint{0.000000in}{0.000000in}}%
\pgfpathlineto{\pgfqpoint{0.000000in}{0.048611in}}%
\pgfusepath{stroke,fill}%
}%
\begin{pgfscope}%
\pgfsys@transformshift{2.713030in}{2.667981in}%
\pgfsys@useobject{currentmarker}{}%
\end{pgfscope}%
\end{pgfscope}%
\begin{pgfscope}%
\pgfpathrectangle{\pgfqpoint{0.691161in}{2.667981in}}{\pgfqpoint{5.054672in}{0.911907in}}%
\pgfusepath{clip}%
\pgfsetbuttcap%
\pgfsetroundjoin%
\pgfsetlinewidth{0.501875pt}%
\definecolor{currentstroke}{rgb}{0.698039,0.698039,0.698039}%
\pgfsetstrokecolor{currentstroke}%
\pgfsetdash{{1.850000pt}{0.800000pt}}{0.000000pt}%
\pgfpathmoveto{\pgfqpoint{3.723964in}{2.667981in}}%
\pgfpathlineto{\pgfqpoint{3.723964in}{3.579888in}}%
\pgfusepath{stroke}%
\end{pgfscope}%
\begin{pgfscope}%
\pgfsetbuttcap%
\pgfsetroundjoin%
\definecolor{currentfill}{rgb}{0.000000,0.000000,0.000000}%
\pgfsetfillcolor{currentfill}%
\pgfsetlinewidth{0.803000pt}%
\definecolor{currentstroke}{rgb}{0.000000,0.000000,0.000000}%
\pgfsetstrokecolor{currentstroke}%
\pgfsetdash{}{0pt}%
\pgfsys@defobject{currentmarker}{\pgfqpoint{0.000000in}{0.000000in}}{\pgfqpoint{0.000000in}{0.048611in}}{%
\pgfpathmoveto{\pgfqpoint{0.000000in}{0.000000in}}%
\pgfpathlineto{\pgfqpoint{0.000000in}{0.048611in}}%
\pgfusepath{stroke,fill}%
}%
\begin{pgfscope}%
\pgfsys@transformshift{3.723964in}{2.667981in}%
\pgfsys@useobject{currentmarker}{}%
\end{pgfscope}%
\end{pgfscope}%
\begin{pgfscope}%
\pgfpathrectangle{\pgfqpoint{0.691161in}{2.667981in}}{\pgfqpoint{5.054672in}{0.911907in}}%
\pgfusepath{clip}%
\pgfsetbuttcap%
\pgfsetroundjoin%
\pgfsetlinewidth{0.501875pt}%
\definecolor{currentstroke}{rgb}{0.698039,0.698039,0.698039}%
\pgfsetstrokecolor{currentstroke}%
\pgfsetdash{{1.850000pt}{0.800000pt}}{0.000000pt}%
\pgfpathmoveto{\pgfqpoint{4.734899in}{2.667981in}}%
\pgfpathlineto{\pgfqpoint{4.734899in}{3.579888in}}%
\pgfusepath{stroke}%
\end{pgfscope}%
\begin{pgfscope}%
\pgfsetbuttcap%
\pgfsetroundjoin%
\definecolor{currentfill}{rgb}{0.000000,0.000000,0.000000}%
\pgfsetfillcolor{currentfill}%
\pgfsetlinewidth{0.803000pt}%
\definecolor{currentstroke}{rgb}{0.000000,0.000000,0.000000}%
\pgfsetstrokecolor{currentstroke}%
\pgfsetdash{}{0pt}%
\pgfsys@defobject{currentmarker}{\pgfqpoint{0.000000in}{0.000000in}}{\pgfqpoint{0.000000in}{0.048611in}}{%
\pgfpathmoveto{\pgfqpoint{0.000000in}{0.000000in}}%
\pgfpathlineto{\pgfqpoint{0.000000in}{0.048611in}}%
\pgfusepath{stroke,fill}%
}%
\begin{pgfscope}%
\pgfsys@transformshift{4.734899in}{2.667981in}%
\pgfsys@useobject{currentmarker}{}%
\end{pgfscope}%
\end{pgfscope}%
\begin{pgfscope}%
\pgfpathrectangle{\pgfqpoint{0.691161in}{2.667981in}}{\pgfqpoint{5.054672in}{0.911907in}}%
\pgfusepath{clip}%
\pgfsetbuttcap%
\pgfsetroundjoin%
\pgfsetlinewidth{0.501875pt}%
\definecolor{currentstroke}{rgb}{0.698039,0.698039,0.698039}%
\pgfsetstrokecolor{currentstroke}%
\pgfsetdash{{1.850000pt}{0.800000pt}}{0.000000pt}%
\pgfpathmoveto{\pgfqpoint{5.745833in}{2.667981in}}%
\pgfpathlineto{\pgfqpoint{5.745833in}{3.579888in}}%
\pgfusepath{stroke}%
\end{pgfscope}%
\begin{pgfscope}%
\pgfsetbuttcap%
\pgfsetroundjoin%
\definecolor{currentfill}{rgb}{0.000000,0.000000,0.000000}%
\pgfsetfillcolor{currentfill}%
\pgfsetlinewidth{0.803000pt}%
\definecolor{currentstroke}{rgb}{0.000000,0.000000,0.000000}%
\pgfsetstrokecolor{currentstroke}%
\pgfsetdash{}{0pt}%
\pgfsys@defobject{currentmarker}{\pgfqpoint{0.000000in}{0.000000in}}{\pgfqpoint{0.000000in}{0.048611in}}{%
\pgfpathmoveto{\pgfqpoint{0.000000in}{0.000000in}}%
\pgfpathlineto{\pgfqpoint{0.000000in}{0.048611in}}%
\pgfusepath{stroke,fill}%
}%
\begin{pgfscope}%
\pgfsys@transformshift{5.745833in}{2.667981in}%
\pgfsys@useobject{currentmarker}{}%
\end{pgfscope}%
\end{pgfscope}%
\begin{pgfscope}%
\pgfpathrectangle{\pgfqpoint{0.691161in}{2.667981in}}{\pgfqpoint{5.054672in}{0.911907in}}%
\pgfusepath{clip}%
\pgfsetbuttcap%
\pgfsetroundjoin%
\pgfsetlinewidth{0.501875pt}%
\definecolor{currentstroke}{rgb}{0.698039,0.698039,0.698039}%
\pgfsetstrokecolor{currentstroke}%
\pgfsetdash{{1.850000pt}{0.800000pt}}{0.000000pt}%
\pgfpathmoveto{\pgfqpoint{0.691161in}{2.827861in}}%
\pgfpathlineto{\pgfqpoint{5.745833in}{2.827861in}}%
\pgfusepath{stroke}%
\end{pgfscope}%
\begin{pgfscope}%
\pgfsetbuttcap%
\pgfsetroundjoin%
\definecolor{currentfill}{rgb}{0.000000,0.000000,0.000000}%
\pgfsetfillcolor{currentfill}%
\pgfsetlinewidth{0.803000pt}%
\definecolor{currentstroke}{rgb}{0.000000,0.000000,0.000000}%
\pgfsetstrokecolor{currentstroke}%
\pgfsetdash{}{0pt}%
\pgfsys@defobject{currentmarker}{\pgfqpoint{0.000000in}{0.000000in}}{\pgfqpoint{0.048611in}{0.000000in}}{%
\pgfpathmoveto{\pgfqpoint{0.000000in}{0.000000in}}%
\pgfpathlineto{\pgfqpoint{0.048611in}{0.000000in}}%
\pgfusepath{stroke,fill}%
}%
\begin{pgfscope}%
\pgfsys@transformshift{0.691161in}{2.827861in}%
\pgfsys@useobject{currentmarker}{}%
\end{pgfscope}%
\end{pgfscope}%
\begin{pgfscope}%
\definecolor{textcolor}{rgb}{0.000000,0.000000,0.000000}%
\pgfsetstrokecolor{textcolor}%
\pgfsetfillcolor{textcolor}%
\pgftext[x=0.573105in, y=2.779666in, left, base]{\color{textcolor}\rmfamily\fontsize{10.000000}{12.000000}\selectfont \(\displaystyle {0}\)}%
\end{pgfscope}%
\begin{pgfscope}%
\pgfpathrectangle{\pgfqpoint{0.691161in}{2.667981in}}{\pgfqpoint{5.054672in}{0.911907in}}%
\pgfusepath{clip}%
\pgfsetbuttcap%
\pgfsetroundjoin%
\pgfsetlinewidth{0.501875pt}%
\definecolor{currentstroke}{rgb}{0.698039,0.698039,0.698039}%
\pgfsetstrokecolor{currentstroke}%
\pgfsetdash{{1.850000pt}{0.800000pt}}{0.000000pt}%
\pgfpathmoveto{\pgfqpoint{0.691161in}{3.430954in}}%
\pgfpathlineto{\pgfqpoint{5.745833in}{3.430954in}}%
\pgfusepath{stroke}%
\end{pgfscope}%
\begin{pgfscope}%
\pgfsetbuttcap%
\pgfsetroundjoin%
\definecolor{currentfill}{rgb}{0.000000,0.000000,0.000000}%
\pgfsetfillcolor{currentfill}%
\pgfsetlinewidth{0.803000pt}%
\definecolor{currentstroke}{rgb}{0.000000,0.000000,0.000000}%
\pgfsetstrokecolor{currentstroke}%
\pgfsetdash{}{0pt}%
\pgfsys@defobject{currentmarker}{\pgfqpoint{0.000000in}{0.000000in}}{\pgfqpoint{0.048611in}{0.000000in}}{%
\pgfpathmoveto{\pgfqpoint{0.000000in}{0.000000in}}%
\pgfpathlineto{\pgfqpoint{0.048611in}{0.000000in}}%
\pgfusepath{stroke,fill}%
}%
\begin{pgfscope}%
\pgfsys@transformshift{0.691161in}{3.430954in}%
\pgfsys@useobject{currentmarker}{}%
\end{pgfscope}%
\end{pgfscope}%
\begin{pgfscope}%
\definecolor{textcolor}{rgb}{0.000000,0.000000,0.000000}%
\pgfsetstrokecolor{textcolor}%
\pgfsetfillcolor{textcolor}%
\pgftext[x=0.503661in, y=3.382760in, left, base]{\color{textcolor}\rmfamily\fontsize{10.000000}{12.000000}\selectfont \(\displaystyle {20}\)}%
\end{pgfscope}%
\begin{pgfscope}%
\definecolor{textcolor}{rgb}{0.000000,0.000000,0.000000}%
\pgfsetstrokecolor{textcolor}%
\pgfsetfillcolor{textcolor}%
\pgftext[x=0.448105in,y=3.123935in,,bottom,rotate=90.000000]{\color{textcolor}\rmfamily\fontsize{12.000000}{14.400000}\selectfont STA/LTA}%
\end{pgfscope}%
\begin{pgfscope}%
\pgfpathrectangle{\pgfqpoint{0.691161in}{2.667981in}}{\pgfqpoint{5.054672in}{0.911907in}}%
\pgfusepath{clip}%
\pgfsetrectcap%
\pgfsetroundjoin%
\pgfsetlinewidth{1.505625pt}%
\definecolor{currentstroke}{rgb}{0.121569,0.466667,0.705882}%
\pgfsetstrokecolor{currentstroke}%
\pgfsetdash{}{0pt}%
\pgfpathmoveto{\pgfqpoint{0.691060in}{2.850275in}}%
\pgfpathlineto{\pgfqpoint{0.691667in}{2.849006in}}%
\pgfpathlineto{\pgfqpoint{0.692071in}{2.850120in}}%
\pgfpathlineto{\pgfqpoint{0.695205in}{2.861670in}}%
\pgfpathlineto{\pgfqpoint{0.697833in}{2.866539in}}%
\pgfpathlineto{\pgfqpoint{0.700159in}{2.876519in}}%
\pgfpathlineto{\pgfqpoint{0.701776in}{2.877503in}}%
\pgfpathlineto{\pgfqpoint{0.703899in}{2.875084in}}%
\pgfpathlineto{\pgfqpoint{0.706426in}{2.870289in}}%
\pgfpathlineto{\pgfqpoint{0.710571in}{2.866167in}}%
\pgfpathlineto{\pgfqpoint{0.712795in}{2.858982in}}%
\pgfpathlineto{\pgfqpoint{0.714514in}{2.853760in}}%
\pgfpathlineto{\pgfqpoint{0.715929in}{2.851562in}}%
\pgfpathlineto{\pgfqpoint{0.716435in}{2.852295in}}%
\pgfpathlineto{\pgfqpoint{0.717344in}{2.854062in}}%
\pgfpathlineto{\pgfqpoint{0.717749in}{2.853100in}}%
\pgfpathlineto{\pgfqpoint{0.720984in}{2.842938in}}%
\pgfpathlineto{\pgfqpoint{0.721186in}{2.843084in}}%
\pgfpathlineto{\pgfqpoint{0.721995in}{2.847060in}}%
\pgfpathlineto{\pgfqpoint{0.724219in}{2.853147in}}%
\pgfpathlineto{\pgfqpoint{0.727454in}{2.855746in}}%
\pgfpathlineto{\pgfqpoint{0.728263in}{2.856511in}}%
\pgfpathlineto{\pgfqpoint{0.730689in}{2.862450in}}%
\pgfpathlineto{\pgfqpoint{0.733115in}{2.864406in}}%
\pgfpathlineto{\pgfqpoint{0.735036in}{2.866539in}}%
\pgfpathlineto{\pgfqpoint{0.736855in}{2.864917in}}%
\pgfpathlineto{\pgfqpoint{0.739181in}{2.860202in}}%
\pgfpathlineto{\pgfqpoint{0.739686in}{2.861688in}}%
\pgfpathlineto{\pgfqpoint{0.741203in}{2.865047in}}%
\pgfpathlineto{\pgfqpoint{0.741506in}{2.864730in}}%
\pgfpathlineto{\pgfqpoint{0.744033in}{2.856555in}}%
\pgfpathlineto{\pgfqpoint{0.744943in}{2.857938in}}%
\pgfpathlineto{\pgfqpoint{0.746257in}{2.857383in}}%
\pgfpathlineto{\pgfqpoint{0.747470in}{2.857060in}}%
\pgfpathlineto{\pgfqpoint{0.747672in}{2.857248in}}%
\pgfpathlineto{\pgfqpoint{0.748380in}{2.857771in}}%
\pgfpathlineto{\pgfqpoint{0.748683in}{2.856736in}}%
\pgfpathlineto{\pgfqpoint{0.749897in}{2.852562in}}%
\pgfpathlineto{\pgfqpoint{0.750301in}{2.853325in}}%
\pgfpathlineto{\pgfqpoint{0.752828in}{2.858900in}}%
\pgfpathlineto{\pgfqpoint{0.752929in}{2.858840in}}%
\pgfpathlineto{\pgfqpoint{0.753839in}{2.857730in}}%
\pgfpathlineto{\pgfqpoint{0.754244in}{2.858914in}}%
\pgfpathlineto{\pgfqpoint{0.757175in}{2.867633in}}%
\pgfpathlineto{\pgfqpoint{0.757377in}{2.867554in}}%
\pgfpathlineto{\pgfqpoint{0.759703in}{2.865349in}}%
\pgfpathlineto{\pgfqpoint{0.761421in}{2.860266in}}%
\pgfpathlineto{\pgfqpoint{0.761724in}{2.860654in}}%
\pgfpathlineto{\pgfqpoint{0.768194in}{2.874108in}}%
\pgfpathlineto{\pgfqpoint{0.770216in}{2.874139in}}%
\pgfpathlineto{\pgfqpoint{0.771126in}{2.870545in}}%
\pgfpathlineto{\pgfqpoint{0.772744in}{2.868405in}}%
\pgfpathlineto{\pgfqpoint{0.772845in}{2.868475in}}%
\pgfpathlineto{\pgfqpoint{0.773957in}{2.870365in}}%
\pgfpathlineto{\pgfqpoint{0.774361in}{2.869178in}}%
\pgfpathlineto{\pgfqpoint{0.777293in}{2.858707in}}%
\pgfpathlineto{\pgfqpoint{0.777495in}{2.859068in}}%
\pgfpathlineto{\pgfqpoint{0.780022in}{2.863604in}}%
\pgfpathlineto{\pgfqpoint{0.782145in}{2.862748in}}%
\pgfpathlineto{\pgfqpoint{0.788919in}{2.842959in}}%
\pgfpathlineto{\pgfqpoint{0.789020in}{2.843009in}}%
\pgfpathlineto{\pgfqpoint{0.791850in}{2.844264in}}%
\pgfpathlineto{\pgfqpoint{0.792052in}{2.843999in}}%
\pgfpathlineto{\pgfqpoint{0.793872in}{2.840962in}}%
\pgfpathlineto{\pgfqpoint{0.794681in}{2.842607in}}%
\pgfpathlineto{\pgfqpoint{0.798421in}{2.850649in}}%
\pgfpathlineto{\pgfqpoint{0.799533in}{2.856739in}}%
\pgfpathlineto{\pgfqpoint{0.800140in}{2.856448in}}%
\pgfpathlineto{\pgfqpoint{0.801454in}{2.858547in}}%
\pgfpathlineto{\pgfqpoint{0.802971in}{2.858745in}}%
\pgfpathlineto{\pgfqpoint{0.804689in}{2.859455in}}%
\pgfpathlineto{\pgfqpoint{0.808025in}{2.860182in}}%
\pgfpathlineto{\pgfqpoint{0.810148in}{2.861257in}}%
\pgfpathlineto{\pgfqpoint{0.812574in}{2.868675in}}%
\pgfpathlineto{\pgfqpoint{0.813788in}{2.872518in}}%
\pgfpathlineto{\pgfqpoint{0.814394in}{2.871533in}}%
\pgfpathlineto{\pgfqpoint{0.816214in}{2.868777in}}%
\pgfpathlineto{\pgfqpoint{0.816517in}{2.869274in}}%
\pgfpathlineto{\pgfqpoint{0.817427in}{2.871294in}}%
\pgfpathlineto{\pgfqpoint{0.817932in}{2.869968in}}%
\pgfpathlineto{\pgfqpoint{0.819550in}{2.858534in}}%
\pgfpathlineto{\pgfqpoint{0.820561in}{2.861463in}}%
\pgfpathlineto{\pgfqpoint{0.822785in}{2.863575in}}%
\pgfpathlineto{\pgfqpoint{0.824200in}{2.866385in}}%
\pgfpathlineto{\pgfqpoint{0.825009in}{2.866024in}}%
\pgfpathlineto{\pgfqpoint{0.826525in}{2.866934in}}%
\pgfpathlineto{\pgfqpoint{0.828851in}{2.868930in}}%
\pgfpathlineto{\pgfqpoint{0.830064in}{2.868074in}}%
\pgfpathlineto{\pgfqpoint{0.834006in}{2.855069in}}%
\pgfpathlineto{\pgfqpoint{0.834310in}{2.855316in}}%
\pgfpathlineto{\pgfqpoint{0.836331in}{2.856458in}}%
\pgfpathlineto{\pgfqpoint{0.836534in}{2.856097in}}%
\pgfpathlineto{\pgfqpoint{0.838960in}{2.852512in}}%
\pgfpathlineto{\pgfqpoint{0.839870in}{2.851645in}}%
\pgfpathlineto{\pgfqpoint{0.843711in}{2.840949in}}%
\pgfpathlineto{\pgfqpoint{0.844520in}{2.840092in}}%
\pgfpathlineto{\pgfqpoint{0.845025in}{2.840336in}}%
\pgfpathlineto{\pgfqpoint{0.849069in}{2.841369in}}%
\pgfpathlineto{\pgfqpoint{0.851698in}{2.841996in}}%
\pgfpathlineto{\pgfqpoint{0.854225in}{2.846605in}}%
\pgfpathlineto{\pgfqpoint{0.854933in}{2.845662in}}%
\pgfpathlineto{\pgfqpoint{0.859987in}{2.842843in}}%
\pgfpathlineto{\pgfqpoint{0.866457in}{2.845174in}}%
\pgfpathlineto{\pgfqpoint{0.867064in}{2.844789in}}%
\pgfpathlineto{\pgfqpoint{0.867569in}{2.845431in}}%
\pgfpathlineto{\pgfqpoint{0.868681in}{2.844865in}}%
\pgfpathlineto{\pgfqpoint{0.869288in}{2.846288in}}%
\pgfpathlineto{\pgfqpoint{0.870602in}{2.850195in}}%
\pgfpathlineto{\pgfqpoint{0.871209in}{2.850041in}}%
\pgfpathlineto{\pgfqpoint{0.872826in}{2.849010in}}%
\pgfpathlineto{\pgfqpoint{0.873433in}{2.848697in}}%
\pgfpathlineto{\pgfqpoint{0.873837in}{2.849207in}}%
\pgfpathlineto{\pgfqpoint{0.875353in}{2.853959in}}%
\pgfpathlineto{\pgfqpoint{0.878588in}{2.875479in}}%
\pgfpathlineto{\pgfqpoint{0.882026in}{2.875693in}}%
\pgfpathlineto{\pgfqpoint{0.883845in}{2.876806in}}%
\pgfpathlineto{\pgfqpoint{0.885766in}{2.878725in}}%
\pgfpathlineto{\pgfqpoint{0.886878in}{2.878532in}}%
\pgfpathlineto{\pgfqpoint{0.886979in}{2.878359in}}%
\pgfpathlineto{\pgfqpoint{0.887889in}{2.876838in}}%
\pgfpathlineto{\pgfqpoint{0.888496in}{2.877663in}}%
\pgfpathlineto{\pgfqpoint{0.889001in}{2.877949in}}%
\pgfpathlineto{\pgfqpoint{0.889405in}{2.877100in}}%
\pgfpathlineto{\pgfqpoint{0.890619in}{2.873254in}}%
\pgfpathlineto{\pgfqpoint{0.891124in}{2.874265in}}%
\pgfpathlineto{\pgfqpoint{0.891933in}{2.875411in}}%
\pgfpathlineto{\pgfqpoint{0.892438in}{2.874751in}}%
\pgfpathlineto{\pgfqpoint{0.893045in}{2.874323in}}%
\pgfpathlineto{\pgfqpoint{0.893651in}{2.874844in}}%
\pgfpathlineto{\pgfqpoint{0.894157in}{2.873798in}}%
\pgfpathlineto{\pgfqpoint{0.900627in}{2.851771in}}%
\pgfpathlineto{\pgfqpoint{0.909321in}{2.849101in}}%
\pgfpathlineto{\pgfqpoint{0.911242in}{2.847935in}}%
\pgfpathlineto{\pgfqpoint{0.912354in}{2.846147in}}%
\pgfpathlineto{\pgfqpoint{0.912859in}{2.846523in}}%
\pgfpathlineto{\pgfqpoint{0.913971in}{2.845932in}}%
\pgfpathlineto{\pgfqpoint{0.917105in}{2.842500in}}%
\pgfpathlineto{\pgfqpoint{0.917509in}{2.843502in}}%
\pgfpathlineto{\pgfqpoint{0.920340in}{2.851125in}}%
\pgfpathlineto{\pgfqpoint{0.921250in}{2.854332in}}%
\pgfpathlineto{\pgfqpoint{0.922968in}{2.857354in}}%
\pgfpathlineto{\pgfqpoint{0.923777in}{2.858186in}}%
\pgfpathlineto{\pgfqpoint{0.924283in}{2.857689in}}%
\pgfpathlineto{\pgfqpoint{0.926203in}{2.857594in}}%
\pgfpathlineto{\pgfqpoint{0.928428in}{2.856956in}}%
\pgfpathlineto{\pgfqpoint{0.930348in}{2.857197in}}%
\pgfpathlineto{\pgfqpoint{0.932067in}{2.864570in}}%
\pgfpathlineto{\pgfqpoint{0.933179in}{2.871532in}}%
\pgfpathlineto{\pgfqpoint{0.933785in}{2.871227in}}%
\pgfpathlineto{\pgfqpoint{0.934392in}{2.872168in}}%
\pgfpathlineto{\pgfqpoint{0.935706in}{2.875244in}}%
\pgfpathlineto{\pgfqpoint{0.936212in}{2.874767in}}%
\pgfpathlineto{\pgfqpoint{0.938436in}{2.867986in}}%
\pgfpathlineto{\pgfqpoint{0.940357in}{2.864505in}}%
\pgfpathlineto{\pgfqpoint{0.941469in}{2.863510in}}%
\pgfpathlineto{\pgfqpoint{0.943996in}{2.859308in}}%
\pgfpathlineto{\pgfqpoint{0.946523in}{2.859797in}}%
\pgfpathlineto{\pgfqpoint{0.947433in}{2.862166in}}%
\pgfpathlineto{\pgfqpoint{0.948141in}{2.863194in}}%
\pgfpathlineto{\pgfqpoint{0.948747in}{2.862675in}}%
\pgfpathlineto{\pgfqpoint{0.951578in}{2.858553in}}%
\pgfpathlineto{\pgfqpoint{0.952286in}{2.854458in}}%
\pgfpathlineto{\pgfqpoint{0.954004in}{2.847324in}}%
\pgfpathlineto{\pgfqpoint{0.954813in}{2.845277in}}%
\pgfpathlineto{\pgfqpoint{0.956329in}{2.842962in}}%
\pgfpathlineto{\pgfqpoint{0.956430in}{2.842988in}}%
\pgfpathlineto{\pgfqpoint{0.959868in}{2.845156in}}%
\pgfpathlineto{\pgfqpoint{0.962193in}{2.854125in}}%
\pgfpathlineto{\pgfqpoint{0.965226in}{2.869870in}}%
\pgfpathlineto{\pgfqpoint{0.965832in}{2.869799in}}%
\pgfpathlineto{\pgfqpoint{0.966135in}{2.869333in}}%
\pgfpathlineto{\pgfqpoint{0.969977in}{2.865356in}}%
\pgfpathlineto{\pgfqpoint{0.975840in}{2.866760in}}%
\pgfpathlineto{\pgfqpoint{0.978267in}{2.869883in}}%
\pgfpathlineto{\pgfqpoint{0.980895in}{2.870435in}}%
\pgfpathlineto{\pgfqpoint{0.981502in}{2.872587in}}%
\pgfpathlineto{\pgfqpoint{0.982007in}{2.870722in}}%
\pgfpathlineto{\pgfqpoint{0.982614in}{2.868891in}}%
\pgfpathlineto{\pgfqpoint{0.983220in}{2.870016in}}%
\pgfpathlineto{\pgfqpoint{0.983523in}{2.870150in}}%
\pgfpathlineto{\pgfqpoint{0.983827in}{2.869424in}}%
\pgfpathlineto{\pgfqpoint{0.985444in}{2.861223in}}%
\pgfpathlineto{\pgfqpoint{0.986152in}{2.863253in}}%
\pgfpathlineto{\pgfqpoint{0.988679in}{2.872036in}}%
\pgfpathlineto{\pgfqpoint{0.988881in}{2.871976in}}%
\pgfpathlineto{\pgfqpoint{0.990701in}{2.872479in}}%
\pgfpathlineto{\pgfqpoint{0.995655in}{2.874887in}}%
\pgfpathlineto{\pgfqpoint{0.996160in}{2.873678in}}%
\pgfpathlineto{\pgfqpoint{0.998586in}{2.871362in}}%
\pgfpathlineto{\pgfqpoint{1.000709in}{2.870085in}}%
\pgfpathlineto{\pgfqpoint{1.004753in}{2.854853in}}%
\pgfpathlineto{\pgfqpoint{1.005056in}{2.855206in}}%
\pgfpathlineto{\pgfqpoint{1.007280in}{2.857597in}}%
\pgfpathlineto{\pgfqpoint{1.007887in}{2.855498in}}%
\pgfpathlineto{\pgfqpoint{1.009707in}{2.851374in}}%
\pgfpathlineto{\pgfqpoint{1.012739in}{2.850828in}}%
\pgfpathlineto{\pgfqpoint{1.013548in}{2.854552in}}%
\pgfpathlineto{\pgfqpoint{1.014256in}{2.856489in}}%
\pgfpathlineto{\pgfqpoint{1.014964in}{2.856232in}}%
\pgfpathlineto{\pgfqpoint{1.017592in}{2.857526in}}%
\pgfpathlineto{\pgfqpoint{1.019209in}{2.861380in}}%
\pgfpathlineto{\pgfqpoint{1.019513in}{2.861245in}}%
\pgfpathlineto{\pgfqpoint{1.020625in}{2.861728in}}%
\pgfpathlineto{\pgfqpoint{1.023051in}{2.862701in}}%
\pgfpathlineto{\pgfqpoint{1.024062in}{2.863520in}}%
\pgfpathlineto{\pgfqpoint{1.024567in}{2.864101in}}%
\pgfpathlineto{\pgfqpoint{1.025073in}{2.863092in}}%
\pgfpathlineto{\pgfqpoint{1.025781in}{2.861166in}}%
\pgfpathlineto{\pgfqpoint{1.026185in}{2.862377in}}%
\pgfpathlineto{\pgfqpoint{1.026791in}{2.864501in}}%
\pgfpathlineto{\pgfqpoint{1.027499in}{2.863588in}}%
\pgfpathlineto{\pgfqpoint{1.028510in}{2.864186in}}%
\pgfpathlineto{\pgfqpoint{1.030026in}{2.863983in}}%
\pgfpathlineto{\pgfqpoint{1.031341in}{2.864907in}}%
\pgfpathlineto{\pgfqpoint{1.032756in}{2.865428in}}%
\pgfpathlineto{\pgfqpoint{1.032857in}{2.865346in}}%
\pgfpathlineto{\pgfqpoint{1.033565in}{2.862622in}}%
\pgfpathlineto{\pgfqpoint{1.034373in}{2.859024in}}%
\pgfpathlineto{\pgfqpoint{1.034980in}{2.860416in}}%
\pgfpathlineto{\pgfqpoint{1.036699in}{2.861624in}}%
\pgfpathlineto{\pgfqpoint{1.037710in}{2.861499in}}%
\pgfpathlineto{\pgfqpoint{1.037811in}{2.861233in}}%
\pgfpathlineto{\pgfqpoint{1.038822in}{2.858047in}}%
\pgfpathlineto{\pgfqpoint{1.039327in}{2.859573in}}%
\pgfpathlineto{\pgfqpoint{1.040035in}{2.861191in}}%
\pgfpathlineto{\pgfqpoint{1.040641in}{2.860230in}}%
\pgfpathlineto{\pgfqpoint{1.042259in}{2.859270in}}%
\pgfpathlineto{\pgfqpoint{1.043169in}{2.859113in}}%
\pgfpathlineto{\pgfqpoint{1.043472in}{2.859548in}}%
\pgfpathlineto{\pgfqpoint{1.044078in}{2.859897in}}%
\pgfpathlineto{\pgfqpoint{1.044584in}{2.859280in}}%
\pgfpathlineto{\pgfqpoint{1.044988in}{2.859075in}}%
\pgfpathlineto{\pgfqpoint{1.045393in}{2.859821in}}%
\pgfpathlineto{\pgfqpoint{1.045898in}{2.860671in}}%
\pgfpathlineto{\pgfqpoint{1.046302in}{2.859735in}}%
\pgfpathlineto{\pgfqpoint{1.047516in}{2.853653in}}%
\pgfpathlineto{\pgfqpoint{1.048324in}{2.854959in}}%
\pgfpathlineto{\pgfqpoint{1.050650in}{2.856382in}}%
\pgfpathlineto{\pgfqpoint{1.052772in}{2.856041in}}%
\pgfpathlineto{\pgfqpoint{1.054491in}{2.855079in}}%
\pgfpathlineto{\pgfqpoint{1.055906in}{2.852646in}}%
\pgfpathlineto{\pgfqpoint{1.056311in}{2.852867in}}%
\pgfpathlineto{\pgfqpoint{1.058838in}{2.855817in}}%
\pgfpathlineto{\pgfqpoint{1.059141in}{2.856044in}}%
\pgfpathlineto{\pgfqpoint{1.059546in}{2.854839in}}%
\pgfpathlineto{\pgfqpoint{1.060759in}{2.853263in}}%
\pgfpathlineto{\pgfqpoint{1.061062in}{2.853455in}}%
\pgfpathlineto{\pgfqpoint{1.064297in}{2.858715in}}%
\pgfpathlineto{\pgfqpoint{1.066622in}{2.862768in}}%
\pgfpathlineto{\pgfqpoint{1.067431in}{2.862555in}}%
\pgfpathlineto{\pgfqpoint{1.067633in}{2.862346in}}%
\pgfpathlineto{\pgfqpoint{1.070767in}{2.859675in}}%
\pgfpathlineto{\pgfqpoint{1.073497in}{2.859782in}}%
\pgfpathlineto{\pgfqpoint{1.077035in}{2.861069in}}%
\pgfpathlineto{\pgfqpoint{1.080573in}{2.856119in}}%
\pgfpathlineto{\pgfqpoint{1.080978in}{2.856614in}}%
\pgfpathlineto{\pgfqpoint{1.081685in}{2.857049in}}%
\pgfpathlineto{\pgfqpoint{1.082191in}{2.856681in}}%
\pgfpathlineto{\pgfqpoint{1.084516in}{2.853686in}}%
\pgfpathlineto{\pgfqpoint{1.085122in}{2.852006in}}%
\pgfpathlineto{\pgfqpoint{1.085830in}{2.852737in}}%
\pgfpathlineto{\pgfqpoint{1.086942in}{2.849410in}}%
\pgfpathlineto{\pgfqpoint{1.088560in}{2.850659in}}%
\pgfpathlineto{\pgfqpoint{1.095737in}{2.851210in}}%
\pgfpathlineto{\pgfqpoint{1.096344in}{2.850786in}}%
\pgfpathlineto{\pgfqpoint{1.096748in}{2.851356in}}%
\pgfpathlineto{\pgfqpoint{1.098163in}{2.855620in}}%
\pgfpathlineto{\pgfqpoint{1.099073in}{2.859506in}}%
\pgfpathlineto{\pgfqpoint{1.099680in}{2.858959in}}%
\pgfpathlineto{\pgfqpoint{1.102612in}{2.859446in}}%
\pgfpathlineto{\pgfqpoint{1.103016in}{2.859808in}}%
\pgfpathlineto{\pgfqpoint{1.103420in}{2.858809in}}%
\pgfpathlineto{\pgfqpoint{1.104836in}{2.857526in}}%
\pgfpathlineto{\pgfqpoint{1.104937in}{2.857562in}}%
\pgfpathlineto{\pgfqpoint{1.105644in}{2.857507in}}%
\pgfpathlineto{\pgfqpoint{1.105847in}{2.857049in}}%
\pgfpathlineto{\pgfqpoint{1.107565in}{2.853605in}}%
\pgfpathlineto{\pgfqpoint{1.108071in}{2.854273in}}%
\pgfpathlineto{\pgfqpoint{1.111508in}{2.858099in}}%
\pgfpathlineto{\pgfqpoint{1.116967in}{2.860077in}}%
\pgfpathlineto{\pgfqpoint{1.118180in}{2.858009in}}%
\pgfpathlineto{\pgfqpoint{1.120000in}{2.853925in}}%
\pgfpathlineto{\pgfqpoint{1.121415in}{2.854278in}}%
\pgfpathlineto{\pgfqpoint{1.123133in}{2.858712in}}%
\pgfpathlineto{\pgfqpoint{1.125155in}{2.872943in}}%
\pgfpathlineto{\pgfqpoint{1.126470in}{2.872696in}}%
\pgfpathlineto{\pgfqpoint{1.129199in}{2.873759in}}%
\pgfpathlineto{\pgfqpoint{1.130210in}{2.877629in}}%
\pgfpathlineto{\pgfqpoint{1.130918in}{2.876747in}}%
\pgfpathlineto{\pgfqpoint{1.132636in}{2.877925in}}%
\pgfpathlineto{\pgfqpoint{1.138399in}{2.883021in}}%
\pgfpathlineto{\pgfqpoint{1.143150in}{2.877406in}}%
\pgfpathlineto{\pgfqpoint{1.145374in}{2.863480in}}%
\pgfpathlineto{\pgfqpoint{1.145576in}{2.863639in}}%
\pgfpathlineto{\pgfqpoint{1.147699in}{2.868677in}}%
\pgfpathlineto{\pgfqpoint{1.148407in}{2.867331in}}%
\pgfpathlineto{\pgfqpoint{1.149822in}{2.863262in}}%
\pgfpathlineto{\pgfqpoint{1.151642in}{2.859154in}}%
\pgfpathlineto{\pgfqpoint{1.154574in}{2.853808in}}%
\pgfpathlineto{\pgfqpoint{1.155281in}{2.852215in}}%
\pgfpathlineto{\pgfqpoint{1.155888in}{2.853019in}}%
\pgfpathlineto{\pgfqpoint{1.157000in}{2.852666in}}%
\pgfpathlineto{\pgfqpoint{1.160033in}{2.852439in}}%
\pgfpathlineto{\pgfqpoint{1.163268in}{2.856577in}}%
\pgfpathlineto{\pgfqpoint{1.163470in}{2.856372in}}%
\pgfpathlineto{\pgfqpoint{1.164177in}{2.855375in}}%
\pgfpathlineto{\pgfqpoint{1.164481in}{2.856186in}}%
\pgfpathlineto{\pgfqpoint{1.165492in}{2.860264in}}%
\pgfpathlineto{\pgfqpoint{1.166098in}{2.858765in}}%
\pgfpathlineto{\pgfqpoint{1.168120in}{2.854029in}}%
\pgfpathlineto{\pgfqpoint{1.168322in}{2.854249in}}%
\pgfpathlineto{\pgfqpoint{1.170243in}{2.855545in}}%
\pgfpathlineto{\pgfqpoint{1.170344in}{2.855478in}}%
\pgfpathlineto{\pgfqpoint{1.172770in}{2.855081in}}%
\pgfpathlineto{\pgfqpoint{1.175298in}{2.859449in}}%
\pgfpathlineto{\pgfqpoint{1.175904in}{2.857991in}}%
\pgfpathlineto{\pgfqpoint{1.176915in}{2.856063in}}%
\pgfpathlineto{\pgfqpoint{1.177421in}{2.856343in}}%
\pgfpathlineto{\pgfqpoint{1.183891in}{2.858263in}}%
\pgfpathlineto{\pgfqpoint{1.184497in}{2.857073in}}%
\pgfpathlineto{\pgfqpoint{1.186317in}{2.850905in}}%
\pgfpathlineto{\pgfqpoint{1.186721in}{2.851036in}}%
\pgfpathlineto{\pgfqpoint{1.188339in}{2.851657in}}%
\pgfpathlineto{\pgfqpoint{1.188642in}{2.851114in}}%
\pgfpathlineto{\pgfqpoint{1.190260in}{2.850243in}}%
\pgfpathlineto{\pgfqpoint{1.193393in}{2.849038in}}%
\pgfpathlineto{\pgfqpoint{1.195415in}{2.844195in}}%
\pgfpathlineto{\pgfqpoint{1.202391in}{2.842453in}}%
\pgfpathlineto{\pgfqpoint{1.203503in}{2.843399in}}%
\pgfpathlineto{\pgfqpoint{1.205727in}{2.848558in}}%
\pgfpathlineto{\pgfqpoint{1.208557in}{2.852941in}}%
\pgfpathlineto{\pgfqpoint{1.209669in}{2.855055in}}%
\pgfpathlineto{\pgfqpoint{1.210074in}{2.854932in}}%
\pgfpathlineto{\pgfqpoint{1.211691in}{2.855688in}}%
\pgfpathlineto{\pgfqpoint{1.220082in}{2.870511in}}%
\pgfpathlineto{\pgfqpoint{1.221194in}{2.871141in}}%
\pgfpathlineto{\pgfqpoint{1.221497in}{2.870424in}}%
\pgfpathlineto{\pgfqpoint{1.222508in}{2.868094in}}%
\pgfpathlineto{\pgfqpoint{1.223216in}{2.868408in}}%
\pgfpathlineto{\pgfqpoint{1.223823in}{2.866762in}}%
\pgfpathlineto{\pgfqpoint{1.225844in}{2.862254in}}%
\pgfpathlineto{\pgfqpoint{1.229383in}{2.856367in}}%
\pgfpathlineto{\pgfqpoint{1.230090in}{2.858221in}}%
\pgfpathlineto{\pgfqpoint{1.231910in}{2.861541in}}%
\pgfpathlineto{\pgfqpoint{1.232314in}{2.860804in}}%
\pgfpathlineto{\pgfqpoint{1.233426in}{2.858489in}}%
\pgfpathlineto{\pgfqpoint{1.233831in}{2.859480in}}%
\pgfpathlineto{\pgfqpoint{1.234640in}{2.861384in}}%
\pgfpathlineto{\pgfqpoint{1.235347in}{2.860994in}}%
\pgfpathlineto{\pgfqpoint{1.237470in}{2.861832in}}%
\pgfpathlineto{\pgfqpoint{1.237571in}{2.861664in}}%
\pgfpathlineto{\pgfqpoint{1.239189in}{2.857616in}}%
\pgfpathlineto{\pgfqpoint{1.239795in}{2.858582in}}%
\pgfpathlineto{\pgfqpoint{1.241110in}{2.864002in}}%
\pgfpathlineto{\pgfqpoint{1.243232in}{2.877531in}}%
\pgfpathlineto{\pgfqpoint{1.244142in}{2.876057in}}%
\pgfpathlineto{\pgfqpoint{1.244446in}{2.875939in}}%
\pgfpathlineto{\pgfqpoint{1.244749in}{2.876693in}}%
\pgfpathlineto{\pgfqpoint{1.246771in}{2.883317in}}%
\pgfpathlineto{\pgfqpoint{1.247175in}{2.883090in}}%
\pgfpathlineto{\pgfqpoint{1.247984in}{2.883312in}}%
\pgfpathlineto{\pgfqpoint{1.248186in}{2.883689in}}%
\pgfpathlineto{\pgfqpoint{1.249298in}{2.884943in}}%
\pgfpathlineto{\pgfqpoint{1.249601in}{2.884542in}}%
\pgfpathlineto{\pgfqpoint{1.250410in}{2.882733in}}%
\pgfpathlineto{\pgfqpoint{1.251118in}{2.883612in}}%
\pgfpathlineto{\pgfqpoint{1.252129in}{2.881432in}}%
\pgfpathlineto{\pgfqpoint{1.254757in}{2.875219in}}%
\pgfpathlineto{\pgfqpoint{1.255060in}{2.874552in}}%
\pgfpathlineto{\pgfqpoint{1.255667in}{2.875713in}}%
\pgfpathlineto{\pgfqpoint{1.257082in}{2.875897in}}%
\pgfpathlineto{\pgfqpoint{1.259711in}{2.874912in}}%
\pgfpathlineto{\pgfqpoint{1.261227in}{2.871297in}}%
\pgfpathlineto{\pgfqpoint{1.263249in}{2.859158in}}%
\pgfpathlineto{\pgfqpoint{1.263552in}{2.859994in}}%
\pgfpathlineto{\pgfqpoint{1.264664in}{2.863032in}}%
\pgfpathlineto{\pgfqpoint{1.265170in}{2.862299in}}%
\pgfpathlineto{\pgfqpoint{1.267899in}{2.857980in}}%
\pgfpathlineto{\pgfqpoint{1.270932in}{2.854795in}}%
\pgfpathlineto{\pgfqpoint{1.272347in}{2.853657in}}%
\pgfpathlineto{\pgfqpoint{1.272448in}{2.853696in}}%
\pgfpathlineto{\pgfqpoint{1.274976in}{2.853635in}}%
\pgfpathlineto{\pgfqpoint{1.275683in}{2.852080in}}%
\pgfpathlineto{\pgfqpoint{1.276189in}{2.851138in}}%
\pgfpathlineto{\pgfqpoint{1.276796in}{2.852025in}}%
\pgfpathlineto{\pgfqpoint{1.280940in}{2.855693in}}%
\pgfpathlineto{\pgfqpoint{1.281041in}{2.855582in}}%
\pgfpathlineto{\pgfqpoint{1.284074in}{2.846788in}}%
\pgfpathlineto{\pgfqpoint{1.286096in}{2.842880in}}%
\pgfpathlineto{\pgfqpoint{1.289028in}{2.842374in}}%
\pgfpathlineto{\pgfqpoint{1.290140in}{2.843278in}}%
\pgfpathlineto{\pgfqpoint{1.293072in}{2.844882in}}%
\pgfpathlineto{\pgfqpoint{1.296610in}{2.843965in}}%
\pgfpathlineto{\pgfqpoint{1.299137in}{2.841259in}}%
\pgfpathlineto{\pgfqpoint{1.302675in}{2.838571in}}%
\pgfpathlineto{\pgfqpoint{1.304697in}{2.837636in}}%
\pgfpathlineto{\pgfqpoint{1.307629in}{2.835725in}}%
\pgfpathlineto{\pgfqpoint{1.310460in}{2.835207in}}%
\pgfpathlineto{\pgfqpoint{1.314503in}{2.832817in}}%
\pgfpathlineto{\pgfqpoint{1.314807in}{2.833472in}}%
\pgfpathlineto{\pgfqpoint{1.317941in}{2.839058in}}%
\pgfpathlineto{\pgfqpoint{1.321681in}{2.841055in}}%
\pgfpathlineto{\pgfqpoint{1.329870in}{2.852657in}}%
\pgfpathlineto{\pgfqpoint{1.332902in}{2.862854in}}%
\pgfpathlineto{\pgfqpoint{1.334924in}{2.864642in}}%
\pgfpathlineto{\pgfqpoint{1.335430in}{2.863461in}}%
\pgfpathlineto{\pgfqpoint{1.338463in}{2.859204in}}%
\pgfpathlineto{\pgfqpoint{1.341293in}{2.860882in}}%
\pgfpathlineto{\pgfqpoint{1.342001in}{2.862155in}}%
\pgfpathlineto{\pgfqpoint{1.342607in}{2.861422in}}%
\pgfpathlineto{\pgfqpoint{1.344225in}{2.860512in}}%
\pgfpathlineto{\pgfqpoint{1.344326in}{2.860604in}}%
\pgfpathlineto{\pgfqpoint{1.347965in}{2.863518in}}%
\pgfpathlineto{\pgfqpoint{1.348875in}{2.862874in}}%
\pgfpathlineto{\pgfqpoint{1.350594in}{2.857484in}}%
\pgfpathlineto{\pgfqpoint{1.355244in}{2.848937in}}%
\pgfpathlineto{\pgfqpoint{1.356457in}{2.849418in}}%
\pgfpathlineto{\pgfqpoint{1.359692in}{2.852011in}}%
\pgfpathlineto{\pgfqpoint{1.361815in}{2.849513in}}%
\pgfpathlineto{\pgfqpoint{1.363433in}{2.848504in}}%
\pgfpathlineto{\pgfqpoint{1.367274in}{2.844323in}}%
\pgfpathlineto{\pgfqpoint{1.368487in}{2.841697in}}%
\pgfpathlineto{\pgfqpoint{1.368892in}{2.841999in}}%
\pgfpathlineto{\pgfqpoint{1.372531in}{2.843828in}}%
\pgfpathlineto{\pgfqpoint{1.378698in}{2.840507in}}%
\pgfpathlineto{\pgfqpoint{1.379304in}{2.839802in}}%
\pgfpathlineto{\pgfqpoint{1.379709in}{2.840479in}}%
\pgfpathlineto{\pgfqpoint{1.381629in}{2.842983in}}%
\pgfpathlineto{\pgfqpoint{1.382843in}{2.843996in}}%
\pgfpathlineto{\pgfqpoint{1.385067in}{2.848985in}}%
\pgfpathlineto{\pgfqpoint{1.388403in}{2.849163in}}%
\pgfpathlineto{\pgfqpoint{1.390020in}{2.850274in}}%
\pgfpathlineto{\pgfqpoint{1.391132in}{2.849762in}}%
\pgfpathlineto{\pgfqpoint{1.392851in}{2.850064in}}%
\pgfpathlineto{\pgfqpoint{1.398613in}{2.859291in}}%
\pgfpathlineto{\pgfqpoint{1.399826in}{2.858977in}}%
\pgfpathlineto{\pgfqpoint{1.400938in}{2.856567in}}%
\pgfpathlineto{\pgfqpoint{1.401545in}{2.858032in}}%
\pgfpathlineto{\pgfqpoint{1.402556in}{2.859957in}}%
\pgfpathlineto{\pgfqpoint{1.402960in}{2.859258in}}%
\pgfpathlineto{\pgfqpoint{1.403870in}{2.855474in}}%
\pgfpathlineto{\pgfqpoint{1.404477in}{2.857596in}}%
\pgfpathlineto{\pgfqpoint{1.406195in}{2.860751in}}%
\pgfpathlineto{\pgfqpoint{1.408419in}{2.860399in}}%
\pgfpathlineto{\pgfqpoint{1.409936in}{2.859400in}}%
\pgfpathlineto{\pgfqpoint{1.410845in}{2.860148in}}%
\pgfpathlineto{\pgfqpoint{1.414990in}{2.871840in}}%
\pgfpathlineto{\pgfqpoint{1.415091in}{2.871784in}}%
\pgfpathlineto{\pgfqpoint{1.416001in}{2.868818in}}%
\pgfpathlineto{\pgfqpoint{1.416709in}{2.867821in}}%
\pgfpathlineto{\pgfqpoint{1.417315in}{2.868324in}}%
\pgfpathlineto{\pgfqpoint{1.420955in}{2.869456in}}%
\pgfpathlineto{\pgfqpoint{1.422168in}{2.866850in}}%
\pgfpathlineto{\pgfqpoint{1.422876in}{2.868546in}}%
\pgfpathlineto{\pgfqpoint{1.423988in}{2.869614in}}%
\pgfpathlineto{\pgfqpoint{1.424291in}{2.868934in}}%
\pgfpathlineto{\pgfqpoint{1.425706in}{2.863066in}}%
\pgfpathlineto{\pgfqpoint{1.426515in}{2.863735in}}%
\pgfpathlineto{\pgfqpoint{1.431165in}{2.863236in}}%
\pgfpathlineto{\pgfqpoint{1.436321in}{2.845812in}}%
\pgfpathlineto{\pgfqpoint{1.436826in}{2.846779in}}%
\pgfpathlineto{\pgfqpoint{1.439455in}{2.862749in}}%
\pgfpathlineto{\pgfqpoint{1.442083in}{2.873813in}}%
\pgfpathlineto{\pgfqpoint{1.442184in}{2.873755in}}%
\pgfpathlineto{\pgfqpoint{1.443903in}{2.872133in}}%
\pgfpathlineto{\pgfqpoint{1.444408in}{2.872698in}}%
\pgfpathlineto{\pgfqpoint{1.447441in}{2.877178in}}%
\pgfpathlineto{\pgfqpoint{1.449665in}{2.882777in}}%
\pgfpathlineto{\pgfqpoint{1.455023in}{2.887674in}}%
\pgfpathlineto{\pgfqpoint{1.455630in}{2.887393in}}%
\pgfpathlineto{\pgfqpoint{1.456034in}{2.888165in}}%
\pgfpathlineto{\pgfqpoint{1.456944in}{2.890911in}}%
\pgfpathlineto{\pgfqpoint{1.457449in}{2.889345in}}%
\pgfpathlineto{\pgfqpoint{1.457652in}{2.889132in}}%
\pgfpathlineto{\pgfqpoint{1.457955in}{2.890639in}}%
\pgfpathlineto{\pgfqpoint{1.458359in}{2.892869in}}%
\pgfpathlineto{\pgfqpoint{1.458865in}{2.889702in}}%
\pgfpathlineto{\pgfqpoint{1.461190in}{2.878339in}}%
\pgfpathlineto{\pgfqpoint{1.463212in}{2.874669in}}%
\pgfpathlineto{\pgfqpoint{1.466144in}{2.873673in}}%
\pgfpathlineto{\pgfqpoint{1.469581in}{2.866632in}}%
\pgfpathlineto{\pgfqpoint{1.471704in}{2.868509in}}%
\pgfpathlineto{\pgfqpoint{1.475040in}{2.873113in}}%
\pgfpathlineto{\pgfqpoint{1.476556in}{2.872357in}}%
\pgfpathlineto{\pgfqpoint{1.478174in}{2.865983in}}%
\pgfpathlineto{\pgfqpoint{1.479286in}{2.857935in}}%
\pgfpathlineto{\pgfqpoint{1.479791in}{2.858569in}}%
\pgfpathlineto{\pgfqpoint{1.480297in}{2.859132in}}%
\pgfpathlineto{\pgfqpoint{1.480903in}{2.858523in}}%
\pgfpathlineto{\pgfqpoint{1.486868in}{2.857515in}}%
\pgfpathlineto{\pgfqpoint{1.489193in}{2.860930in}}%
\pgfpathlineto{\pgfqpoint{1.489698in}{2.860502in}}%
\pgfpathlineto{\pgfqpoint{1.489799in}{2.860232in}}%
\pgfpathlineto{\pgfqpoint{1.490608in}{2.858484in}}%
\pgfpathlineto{\pgfqpoint{1.491215in}{2.859317in}}%
\pgfpathlineto{\pgfqpoint{1.491720in}{2.858183in}}%
\pgfpathlineto{\pgfqpoint{1.493540in}{2.855049in}}%
\pgfpathlineto{\pgfqpoint{1.494349in}{2.853840in}}%
\pgfpathlineto{\pgfqpoint{1.494854in}{2.852949in}}%
\pgfpathlineto{\pgfqpoint{1.495258in}{2.853941in}}%
\pgfpathlineto{\pgfqpoint{1.496067in}{2.856298in}}%
\pgfpathlineto{\pgfqpoint{1.496674in}{2.855285in}}%
\pgfpathlineto{\pgfqpoint{1.498089in}{2.854705in}}%
\pgfpathlineto{\pgfqpoint{1.501728in}{2.855603in}}%
\pgfpathlineto{\pgfqpoint{1.505166in}{2.865065in}}%
\pgfpathlineto{\pgfqpoint{1.506278in}{2.864370in}}%
\pgfpathlineto{\pgfqpoint{1.507390in}{2.862342in}}%
\pgfpathlineto{\pgfqpoint{1.508299in}{2.861770in}}%
\pgfpathlineto{\pgfqpoint{1.508704in}{2.862011in}}%
\pgfpathlineto{\pgfqpoint{1.509816in}{2.861289in}}%
\pgfpathlineto{\pgfqpoint{1.510422in}{2.862037in}}%
\pgfpathlineto{\pgfqpoint{1.511636in}{2.866362in}}%
\pgfpathlineto{\pgfqpoint{1.512444in}{2.866092in}}%
\pgfpathlineto{\pgfqpoint{1.513455in}{2.867174in}}%
\pgfpathlineto{\pgfqpoint{1.514668in}{2.867212in}}%
\pgfpathlineto{\pgfqpoint{1.515376in}{2.865602in}}%
\pgfpathlineto{\pgfqpoint{1.517297in}{2.862027in}}%
\pgfpathlineto{\pgfqpoint{1.518510in}{2.861843in}}%
\pgfpathlineto{\pgfqpoint{1.518712in}{2.862170in}}%
\pgfpathlineto{\pgfqpoint{1.519521in}{2.862977in}}%
\pgfpathlineto{\pgfqpoint{1.519925in}{2.862319in}}%
\pgfpathlineto{\pgfqpoint{1.525283in}{2.849642in}}%
\pgfpathlineto{\pgfqpoint{1.527204in}{2.849474in}}%
\pgfpathlineto{\pgfqpoint{1.528720in}{2.851706in}}%
\pgfpathlineto{\pgfqpoint{1.529630in}{2.850536in}}%
\pgfpathlineto{\pgfqpoint{1.530843in}{2.848472in}}%
\pgfpathlineto{\pgfqpoint{1.532158in}{2.842215in}}%
\pgfpathlineto{\pgfqpoint{1.532764in}{2.842805in}}%
\pgfpathlineto{\pgfqpoint{1.541155in}{2.859442in}}%
\pgfpathlineto{\pgfqpoint{1.541660in}{2.858905in}}%
\pgfpathlineto{\pgfqpoint{1.543278in}{2.858262in}}%
\pgfpathlineto{\pgfqpoint{1.543379in}{2.858372in}}%
\pgfpathlineto{\pgfqpoint{1.546311in}{2.860230in}}%
\pgfpathlineto{\pgfqpoint{1.547220in}{2.861157in}}%
\pgfpathlineto{\pgfqpoint{1.547726in}{2.861965in}}%
\pgfpathlineto{\pgfqpoint{1.548231in}{2.861062in}}%
\pgfpathlineto{\pgfqpoint{1.548737in}{2.860537in}}%
\pgfpathlineto{\pgfqpoint{1.549242in}{2.861324in}}%
\pgfpathlineto{\pgfqpoint{1.550455in}{2.862347in}}%
\pgfpathlineto{\pgfqpoint{1.550759in}{2.862126in}}%
\pgfpathlineto{\pgfqpoint{1.553690in}{2.860199in}}%
\pgfpathlineto{\pgfqpoint{1.554196in}{2.859986in}}%
\pgfpathlineto{\pgfqpoint{1.554398in}{2.859466in}}%
\pgfpathlineto{\pgfqpoint{1.558442in}{2.850143in}}%
\pgfpathlineto{\pgfqpoint{1.558644in}{2.850281in}}%
\pgfpathlineto{\pgfqpoint{1.559251in}{2.849340in}}%
\pgfpathlineto{\pgfqpoint{1.559958in}{2.847980in}}%
\pgfpathlineto{\pgfqpoint{1.560565in}{2.848523in}}%
\pgfpathlineto{\pgfqpoint{1.563294in}{2.849139in}}%
\pgfpathlineto{\pgfqpoint{1.564609in}{2.847803in}}%
\pgfpathlineto{\pgfqpoint{1.565518in}{2.847215in}}%
\pgfpathlineto{\pgfqpoint{1.565923in}{2.847777in}}%
\pgfpathlineto{\pgfqpoint{1.567136in}{2.849566in}}%
\pgfpathlineto{\pgfqpoint{1.567540in}{2.848957in}}%
\pgfpathlineto{\pgfqpoint{1.570371in}{2.845178in}}%
\pgfpathlineto{\pgfqpoint{1.572089in}{2.846506in}}%
\pgfpathlineto{\pgfqpoint{1.573606in}{2.847061in}}%
\pgfpathlineto{\pgfqpoint{1.573808in}{2.846788in}}%
\pgfpathlineto{\pgfqpoint{1.575931in}{2.845091in}}%
\pgfpathlineto{\pgfqpoint{1.578863in}{2.843977in}}%
\pgfpathlineto{\pgfqpoint{1.580278in}{2.843229in}}%
\pgfpathlineto{\pgfqpoint{1.583008in}{2.842867in}}%
\pgfpathlineto{\pgfqpoint{1.584625in}{2.846877in}}%
\pgfpathlineto{\pgfqpoint{1.585232in}{2.848108in}}%
\pgfpathlineto{\pgfqpoint{1.585838in}{2.847564in}}%
\pgfpathlineto{\pgfqpoint{1.588365in}{2.844399in}}%
\pgfpathlineto{\pgfqpoint{1.588669in}{2.844977in}}%
\pgfpathlineto{\pgfqpoint{1.590994in}{2.852317in}}%
\pgfpathlineto{\pgfqpoint{1.591600in}{2.851867in}}%
\pgfpathlineto{\pgfqpoint{1.593420in}{2.852065in}}%
\pgfpathlineto{\pgfqpoint{1.595745in}{2.852588in}}%
\pgfpathlineto{\pgfqpoint{1.599081in}{2.854336in}}%
\pgfpathlineto{\pgfqpoint{1.601811in}{2.861628in}}%
\pgfpathlineto{\pgfqpoint{1.602923in}{2.862553in}}%
\pgfpathlineto{\pgfqpoint{1.603327in}{2.861982in}}%
\pgfpathlineto{\pgfqpoint{1.606562in}{2.856981in}}%
\pgfpathlineto{\pgfqpoint{1.606866in}{2.857556in}}%
\pgfpathlineto{\pgfqpoint{1.608786in}{2.861232in}}%
\pgfpathlineto{\pgfqpoint{1.608989in}{2.860934in}}%
\pgfpathlineto{\pgfqpoint{1.611112in}{2.854433in}}%
\pgfpathlineto{\pgfqpoint{1.611819in}{2.855053in}}%
\pgfpathlineto{\pgfqpoint{1.615863in}{2.858346in}}%
\pgfpathlineto{\pgfqpoint{1.618188in}{2.860220in}}%
\pgfpathlineto{\pgfqpoint{1.618491in}{2.859880in}}%
\pgfpathlineto{\pgfqpoint{1.619603in}{2.857280in}}%
\pgfpathlineto{\pgfqpoint{1.621120in}{2.854175in}}%
\pgfpathlineto{\pgfqpoint{1.622535in}{2.854392in}}%
\pgfpathlineto{\pgfqpoint{1.625871in}{2.854516in}}%
\pgfpathlineto{\pgfqpoint{1.626882in}{2.854877in}}%
\pgfpathlineto{\pgfqpoint{1.627185in}{2.854033in}}%
\pgfpathlineto{\pgfqpoint{1.629409in}{2.849845in}}%
\pgfpathlineto{\pgfqpoint{1.630926in}{2.850574in}}%
\pgfpathlineto{\pgfqpoint{1.632746in}{2.852118in}}%
\pgfpathlineto{\pgfqpoint{1.633150in}{2.851571in}}%
\pgfpathlineto{\pgfqpoint{1.634868in}{2.849011in}}%
\pgfpathlineto{\pgfqpoint{1.635172in}{2.849497in}}%
\pgfpathlineto{\pgfqpoint{1.637598in}{2.853448in}}%
\pgfpathlineto{\pgfqpoint{1.637699in}{2.853371in}}%
\pgfpathlineto{\pgfqpoint{1.639418in}{2.851735in}}%
\pgfpathlineto{\pgfqpoint{1.639721in}{2.852208in}}%
\pgfpathlineto{\pgfqpoint{1.640429in}{2.853543in}}%
\pgfpathlineto{\pgfqpoint{1.641237in}{2.853201in}}%
\pgfpathlineto{\pgfqpoint{1.641844in}{2.852241in}}%
\pgfpathlineto{\pgfqpoint{1.643461in}{2.850574in}}%
\pgfpathlineto{\pgfqpoint{1.651751in}{2.851524in}}%
\pgfpathlineto{\pgfqpoint{1.653571in}{2.849708in}}%
\pgfpathlineto{\pgfqpoint{1.656907in}{2.848222in}}%
\pgfpathlineto{\pgfqpoint{1.658322in}{2.845656in}}%
\pgfpathlineto{\pgfqpoint{1.658929in}{2.846603in}}%
\pgfpathlineto{\pgfqpoint{1.659636in}{2.847575in}}%
\pgfpathlineto{\pgfqpoint{1.660041in}{2.846628in}}%
\pgfpathlineto{\pgfqpoint{1.661860in}{2.844131in}}%
\pgfpathlineto{\pgfqpoint{1.664186in}{2.843966in}}%
\pgfpathlineto{\pgfqpoint{1.666005in}{2.844403in}}%
\pgfpathlineto{\pgfqpoint{1.666915in}{2.845636in}}%
\pgfpathlineto{\pgfqpoint{1.669645in}{2.850612in}}%
\pgfpathlineto{\pgfqpoint{1.671060in}{2.849818in}}%
\pgfpathlineto{\pgfqpoint{1.672677in}{2.848377in}}%
\pgfpathlineto{\pgfqpoint{1.672880in}{2.848547in}}%
\pgfpathlineto{\pgfqpoint{1.675104in}{2.849946in}}%
\pgfpathlineto{\pgfqpoint{1.675205in}{2.849816in}}%
\pgfpathlineto{\pgfqpoint{1.677732in}{2.847416in}}%
\pgfpathlineto{\pgfqpoint{1.678844in}{2.847368in}}%
\pgfpathlineto{\pgfqpoint{1.679046in}{2.846975in}}%
\pgfpathlineto{\pgfqpoint{1.680057in}{2.845777in}}%
\pgfpathlineto{\pgfqpoint{1.680462in}{2.846220in}}%
\pgfpathlineto{\pgfqpoint{1.682382in}{2.850289in}}%
\pgfpathlineto{\pgfqpoint{1.684404in}{2.857192in}}%
\pgfpathlineto{\pgfqpoint{1.687033in}{2.857744in}}%
\pgfpathlineto{\pgfqpoint{1.689863in}{2.854983in}}%
\pgfpathlineto{\pgfqpoint{1.690268in}{2.855550in}}%
\pgfpathlineto{\pgfqpoint{1.691582in}{2.859243in}}%
\pgfpathlineto{\pgfqpoint{1.693098in}{2.860926in}}%
\pgfpathlineto{\pgfqpoint{1.693604in}{2.861686in}}%
\pgfpathlineto{\pgfqpoint{1.697546in}{2.878048in}}%
\pgfpathlineto{\pgfqpoint{1.700984in}{2.877647in}}%
\pgfpathlineto{\pgfqpoint{1.703511in}{2.872473in}}%
\pgfpathlineto{\pgfqpoint{1.703713in}{2.873033in}}%
\pgfpathlineto{\pgfqpoint{1.707656in}{2.890566in}}%
\pgfpathlineto{\pgfqpoint{1.708262in}{2.889792in}}%
\pgfpathlineto{\pgfqpoint{1.709779in}{2.889695in}}%
\pgfpathlineto{\pgfqpoint{1.711194in}{2.888936in}}%
\pgfpathlineto{\pgfqpoint{1.713014in}{2.887028in}}%
\pgfpathlineto{\pgfqpoint{1.713822in}{2.885438in}}%
\pgfpathlineto{\pgfqpoint{1.716350in}{2.872277in}}%
\pgfpathlineto{\pgfqpoint{1.716653in}{2.872869in}}%
\pgfpathlineto{\pgfqpoint{1.718675in}{2.876479in}}%
\pgfpathlineto{\pgfqpoint{1.719989in}{2.876267in}}%
\pgfpathlineto{\pgfqpoint{1.721202in}{2.876955in}}%
\pgfpathlineto{\pgfqpoint{1.722112in}{2.876801in}}%
\pgfpathlineto{\pgfqpoint{1.722314in}{2.876232in}}%
\pgfpathlineto{\pgfqpoint{1.728077in}{2.850822in}}%
\pgfpathlineto{\pgfqpoint{1.729391in}{2.852142in}}%
\pgfpathlineto{\pgfqpoint{1.730705in}{2.851866in}}%
\pgfpathlineto{\pgfqpoint{1.733030in}{2.851031in}}%
\pgfpathlineto{\pgfqpoint{1.733131in}{2.851157in}}%
\pgfpathlineto{\pgfqpoint{1.736366in}{2.855035in}}%
\pgfpathlineto{\pgfqpoint{1.736973in}{2.853286in}}%
\pgfpathlineto{\pgfqpoint{1.737883in}{2.851071in}}%
\pgfpathlineto{\pgfqpoint{1.738489in}{2.851270in}}%
\pgfpathlineto{\pgfqpoint{1.740915in}{2.851435in}}%
\pgfpathlineto{\pgfqpoint{1.743544in}{2.854382in}}%
\pgfpathlineto{\pgfqpoint{1.746273in}{2.854499in}}%
\pgfpathlineto{\pgfqpoint{1.748902in}{2.853634in}}%
\pgfpathlineto{\pgfqpoint{1.753552in}{2.856060in}}%
\pgfpathlineto{\pgfqpoint{1.758506in}{2.851739in}}%
\pgfpathlineto{\pgfqpoint{1.759719in}{2.850230in}}%
\pgfpathlineto{\pgfqpoint{1.760123in}{2.850454in}}%
\pgfpathlineto{\pgfqpoint{1.760932in}{2.850185in}}%
\pgfpathlineto{\pgfqpoint{1.761134in}{2.849762in}}%
\pgfpathlineto{\pgfqpoint{1.763055in}{2.847970in}}%
\pgfpathlineto{\pgfqpoint{1.764672in}{2.847421in}}%
\pgfpathlineto{\pgfqpoint{1.766391in}{2.846143in}}%
\pgfpathlineto{\pgfqpoint{1.767907in}{2.846717in}}%
\pgfpathlineto{\pgfqpoint{1.768716in}{2.847701in}}%
\pgfpathlineto{\pgfqpoint{1.769222in}{2.846932in}}%
\pgfpathlineto{\pgfqpoint{1.771547in}{2.844977in}}%
\pgfpathlineto{\pgfqpoint{1.773771in}{2.841305in}}%
\pgfpathlineto{\pgfqpoint{1.774883in}{2.839093in}}%
\pgfpathlineto{\pgfqpoint{1.775388in}{2.840009in}}%
\pgfpathlineto{\pgfqpoint{1.776905in}{2.845701in}}%
\pgfpathlineto{\pgfqpoint{1.779432in}{2.859544in}}%
\pgfpathlineto{\pgfqpoint{1.781656in}{2.864451in}}%
\pgfpathlineto{\pgfqpoint{1.781858in}{2.864371in}}%
\pgfpathlineto{\pgfqpoint{1.786306in}{2.863758in}}%
\pgfpathlineto{\pgfqpoint{1.791058in}{2.868620in}}%
\pgfpathlineto{\pgfqpoint{1.792473in}{2.870264in}}%
\pgfpathlineto{\pgfqpoint{1.794192in}{2.870881in}}%
\pgfpathlineto{\pgfqpoint{1.795304in}{2.869589in}}%
\pgfpathlineto{\pgfqpoint{1.797225in}{2.862565in}}%
\pgfpathlineto{\pgfqpoint{1.799449in}{2.851074in}}%
\pgfpathlineto{\pgfqpoint{1.801470in}{2.845590in}}%
\pgfpathlineto{\pgfqpoint{1.802077in}{2.845244in}}%
\pgfpathlineto{\pgfqpoint{1.802481in}{2.845941in}}%
\pgfpathlineto{\pgfqpoint{1.806424in}{2.854570in}}%
\pgfpathlineto{\pgfqpoint{1.806626in}{2.854413in}}%
\pgfpathlineto{\pgfqpoint{1.807435in}{2.853651in}}%
\pgfpathlineto{\pgfqpoint{1.808042in}{2.854287in}}%
\pgfpathlineto{\pgfqpoint{1.808850in}{2.853312in}}%
\pgfpathlineto{\pgfqpoint{1.810973in}{2.850911in}}%
\pgfpathlineto{\pgfqpoint{1.811074in}{2.850959in}}%
\pgfpathlineto{\pgfqpoint{1.812287in}{2.850792in}}%
\pgfpathlineto{\pgfqpoint{1.812389in}{2.850631in}}%
\pgfpathlineto{\pgfqpoint{1.813905in}{2.850286in}}%
\pgfpathlineto{\pgfqpoint{1.821285in}{2.849839in}}%
\pgfpathlineto{\pgfqpoint{1.821689in}{2.849799in}}%
\pgfpathlineto{\pgfqpoint{1.821992in}{2.850575in}}%
\pgfpathlineto{\pgfqpoint{1.822902in}{2.853235in}}%
\pgfpathlineto{\pgfqpoint{1.823610in}{2.852718in}}%
\pgfpathlineto{\pgfqpoint{1.824318in}{2.850893in}}%
\pgfpathlineto{\pgfqpoint{1.826036in}{2.848158in}}%
\pgfpathlineto{\pgfqpoint{1.833820in}{2.845357in}}%
\pgfpathlineto{\pgfqpoint{1.836044in}{2.847444in}}%
\pgfpathlineto{\pgfqpoint{1.837460in}{2.848294in}}%
\pgfpathlineto{\pgfqpoint{1.839482in}{2.849322in}}%
\pgfpathlineto{\pgfqpoint{1.842312in}{2.846646in}}%
\pgfpathlineto{\pgfqpoint{1.844536in}{2.840981in}}%
\pgfpathlineto{\pgfqpoint{1.845850in}{2.841052in}}%
\pgfpathlineto{\pgfqpoint{1.848176in}{2.842583in}}%
\pgfpathlineto{\pgfqpoint{1.849692in}{2.843401in}}%
\pgfpathlineto{\pgfqpoint{1.850501in}{2.844742in}}%
\pgfpathlineto{\pgfqpoint{1.853938in}{2.850092in}}%
\pgfpathlineto{\pgfqpoint{1.855050in}{2.848363in}}%
\pgfpathlineto{\pgfqpoint{1.855859in}{2.849716in}}%
\pgfpathlineto{\pgfqpoint{1.857173in}{2.849972in}}%
\pgfpathlineto{\pgfqpoint{1.857375in}{2.849751in}}%
\pgfpathlineto{\pgfqpoint{1.858689in}{2.847990in}}%
\pgfpathlineto{\pgfqpoint{1.859498in}{2.848519in}}%
\pgfpathlineto{\pgfqpoint{1.860408in}{2.849867in}}%
\pgfpathlineto{\pgfqpoint{1.864452in}{2.858063in}}%
\pgfpathlineto{\pgfqpoint{1.867383in}{2.858334in}}%
\pgfpathlineto{\pgfqpoint{1.869607in}{2.854862in}}%
\pgfpathlineto{\pgfqpoint{1.870113in}{2.855892in}}%
\pgfpathlineto{\pgfqpoint{1.870821in}{2.857084in}}%
\pgfpathlineto{\pgfqpoint{1.871427in}{2.856615in}}%
\pgfpathlineto{\pgfqpoint{1.873853in}{2.855133in}}%
\pgfpathlineto{\pgfqpoint{1.880222in}{2.857707in}}%
\pgfpathlineto{\pgfqpoint{1.882952in}{2.851491in}}%
\pgfpathlineto{\pgfqpoint{1.883558in}{2.850760in}}%
\pgfpathlineto{\pgfqpoint{1.884064in}{2.851542in}}%
\pgfpathlineto{\pgfqpoint{1.886591in}{2.854202in}}%
\pgfpathlineto{\pgfqpoint{1.887299in}{2.857033in}}%
\pgfpathlineto{\pgfqpoint{1.888209in}{2.859890in}}%
\pgfpathlineto{\pgfqpoint{1.888815in}{2.859479in}}%
\pgfpathlineto{\pgfqpoint{1.889422in}{2.859440in}}%
\pgfpathlineto{\pgfqpoint{1.889624in}{2.859913in}}%
\pgfpathlineto{\pgfqpoint{1.893162in}{2.870015in}}%
\pgfpathlineto{\pgfqpoint{1.893465in}{2.869596in}}%
\pgfpathlineto{\pgfqpoint{1.896700in}{2.864368in}}%
\pgfpathlineto{\pgfqpoint{1.898419in}{2.864884in}}%
\pgfpathlineto{\pgfqpoint{1.904586in}{2.864284in}}%
\pgfpathlineto{\pgfqpoint{1.906405in}{2.861238in}}%
\pgfpathlineto{\pgfqpoint{1.906810in}{2.862776in}}%
\pgfpathlineto{\pgfqpoint{1.907416in}{2.865223in}}%
\pgfpathlineto{\pgfqpoint{1.908124in}{2.864033in}}%
\pgfpathlineto{\pgfqpoint{1.910348in}{2.862269in}}%
\pgfpathlineto{\pgfqpoint{1.914594in}{2.850463in}}%
\pgfpathlineto{\pgfqpoint{1.916515in}{2.849914in}}%
\pgfpathlineto{\pgfqpoint{1.919952in}{2.850951in}}%
\pgfpathlineto{\pgfqpoint{1.924400in}{2.876220in}}%
\pgfpathlineto{\pgfqpoint{1.926523in}{2.882505in}}%
\pgfpathlineto{\pgfqpoint{1.926826in}{2.881971in}}%
\pgfpathlineto{\pgfqpoint{1.929960in}{2.873436in}}%
\pgfpathlineto{\pgfqpoint{1.933600in}{2.872812in}}%
\pgfpathlineto{\pgfqpoint{1.937947in}{2.878230in}}%
\pgfpathlineto{\pgfqpoint{1.939968in}{2.878147in}}%
\pgfpathlineto{\pgfqpoint{1.940070in}{2.877754in}}%
\pgfpathlineto{\pgfqpoint{1.945630in}{2.853582in}}%
\pgfpathlineto{\pgfqpoint{1.946641in}{2.854908in}}%
\pgfpathlineto{\pgfqpoint{1.947348in}{2.854040in}}%
\pgfpathlineto{\pgfqpoint{1.948966in}{2.854303in}}%
\pgfpathlineto{\pgfqpoint{1.951089in}{2.854849in}}%
\pgfpathlineto{\pgfqpoint{1.954020in}{2.853557in}}%
\pgfpathlineto{\pgfqpoint{1.957255in}{2.848196in}}%
\pgfpathlineto{\pgfqpoint{1.959378in}{2.850319in}}%
\pgfpathlineto{\pgfqpoint{1.960794in}{2.852297in}}%
\pgfpathlineto{\pgfqpoint{1.960996in}{2.852148in}}%
\pgfpathlineto{\pgfqpoint{1.965848in}{2.843220in}}%
\pgfpathlineto{\pgfqpoint{1.967567in}{2.840487in}}%
\pgfpathlineto{\pgfqpoint{1.971611in}{2.840298in}}%
\pgfpathlineto{\pgfqpoint{1.975857in}{2.841975in}}%
\pgfpathlineto{\pgfqpoint{1.978081in}{2.844146in}}%
\pgfpathlineto{\pgfqpoint{1.980608in}{2.847211in}}%
\pgfpathlineto{\pgfqpoint{1.981316in}{2.848306in}}%
\pgfpathlineto{\pgfqpoint{1.983843in}{2.856093in}}%
\pgfpathlineto{\pgfqpoint{1.983944in}{2.856038in}}%
\pgfpathlineto{\pgfqpoint{1.985460in}{2.855507in}}%
\pgfpathlineto{\pgfqpoint{1.985663in}{2.855845in}}%
\pgfpathlineto{\pgfqpoint{1.988190in}{2.859222in}}%
\pgfpathlineto{\pgfqpoint{1.990111in}{2.861403in}}%
\pgfpathlineto{\pgfqpoint{1.991930in}{2.864333in}}%
\pgfpathlineto{\pgfqpoint{1.996075in}{2.865813in}}%
\pgfpathlineto{\pgfqpoint{1.997187in}{2.869689in}}%
\pgfpathlineto{\pgfqpoint{1.998299in}{2.868511in}}%
\pgfpathlineto{\pgfqpoint{2.003556in}{2.860346in}}%
\pgfpathlineto{\pgfqpoint{2.004062in}{2.860553in}}%
\pgfpathlineto{\pgfqpoint{2.004264in}{2.861143in}}%
\pgfpathlineto{\pgfqpoint{2.005679in}{2.865690in}}%
\pgfpathlineto{\pgfqpoint{2.006387in}{2.865267in}}%
\pgfpathlineto{\pgfqpoint{2.007398in}{2.867967in}}%
\pgfpathlineto{\pgfqpoint{2.009015in}{2.868768in}}%
\pgfpathlineto{\pgfqpoint{2.009723in}{2.869275in}}%
\pgfpathlineto{\pgfqpoint{2.010026in}{2.868568in}}%
\pgfpathlineto{\pgfqpoint{2.012250in}{2.863906in}}%
\pgfpathlineto{\pgfqpoint{2.015789in}{2.862648in}}%
\pgfpathlineto{\pgfqpoint{2.016597in}{2.861900in}}%
\pgfpathlineto{\pgfqpoint{2.017507in}{2.859253in}}%
\pgfpathlineto{\pgfqpoint{2.018013in}{2.860395in}}%
\pgfpathlineto{\pgfqpoint{2.020034in}{2.863093in}}%
\pgfpathlineto{\pgfqpoint{2.021349in}{2.862167in}}%
\pgfpathlineto{\pgfqpoint{2.023371in}{2.857126in}}%
\pgfpathlineto{\pgfqpoint{2.023573in}{2.857233in}}%
\pgfpathlineto{\pgfqpoint{2.024280in}{2.857622in}}%
\pgfpathlineto{\pgfqpoint{2.024584in}{2.856844in}}%
\pgfpathlineto{\pgfqpoint{2.025291in}{2.854984in}}%
\pgfpathlineto{\pgfqpoint{2.025696in}{2.856337in}}%
\pgfpathlineto{\pgfqpoint{2.026302in}{2.858418in}}%
\pgfpathlineto{\pgfqpoint{2.026909in}{2.856872in}}%
\pgfpathlineto{\pgfqpoint{2.028122in}{2.854521in}}%
\pgfpathlineto{\pgfqpoint{2.028526in}{2.854935in}}%
\pgfpathlineto{\pgfqpoint{2.029133in}{2.855080in}}%
\pgfpathlineto{\pgfqpoint{2.029537in}{2.854531in}}%
\pgfpathlineto{\pgfqpoint{2.030245in}{2.854158in}}%
\pgfpathlineto{\pgfqpoint{2.030548in}{2.854677in}}%
\pgfpathlineto{\pgfqpoint{2.032570in}{2.857138in}}%
\pgfpathlineto{\pgfqpoint{2.033480in}{2.857851in}}%
\pgfpathlineto{\pgfqpoint{2.036007in}{2.861290in}}%
\pgfpathlineto{\pgfqpoint{2.037726in}{2.859476in}}%
\pgfpathlineto{\pgfqpoint{2.040152in}{2.854581in}}%
\pgfpathlineto{\pgfqpoint{2.040860in}{2.855654in}}%
\pgfpathlineto{\pgfqpoint{2.042174in}{2.855675in}}%
\pgfpathlineto{\pgfqpoint{2.043994in}{2.854327in}}%
\pgfpathlineto{\pgfqpoint{2.049756in}{2.844188in}}%
\pgfpathlineto{\pgfqpoint{2.049958in}{2.844328in}}%
\pgfpathlineto{\pgfqpoint{2.053800in}{2.849558in}}%
\pgfpathlineto{\pgfqpoint{2.054811in}{2.848251in}}%
\pgfpathlineto{\pgfqpoint{2.058955in}{2.848317in}}%
\pgfpathlineto{\pgfqpoint{2.060775in}{2.848184in}}%
\pgfpathlineto{\pgfqpoint{2.061988in}{2.847628in}}%
\pgfpathlineto{\pgfqpoint{2.062190in}{2.847821in}}%
\pgfpathlineto{\pgfqpoint{2.067043in}{2.852810in}}%
\pgfpathlineto{\pgfqpoint{2.070581in}{2.852130in}}%
\pgfpathlineto{\pgfqpoint{2.074423in}{2.844635in}}%
\pgfpathlineto{\pgfqpoint{2.078972in}{2.843378in}}%
\pgfpathlineto{\pgfqpoint{2.082813in}{2.846345in}}%
\pgfpathlineto{\pgfqpoint{2.083218in}{2.845641in}}%
\pgfpathlineto{\pgfqpoint{2.083622in}{2.845403in}}%
\pgfpathlineto{\pgfqpoint{2.084229in}{2.846095in}}%
\pgfpathlineto{\pgfqpoint{2.084835in}{2.845082in}}%
\pgfpathlineto{\pgfqpoint{2.085745in}{2.843726in}}%
\pgfpathlineto{\pgfqpoint{2.086251in}{2.844069in}}%
\pgfpathlineto{\pgfqpoint{2.090598in}{2.845158in}}%
\pgfpathlineto{\pgfqpoint{2.092721in}{2.844289in}}%
\pgfpathlineto{\pgfqpoint{2.093630in}{2.842760in}}%
\pgfpathlineto{\pgfqpoint{2.094237in}{2.843400in}}%
\pgfpathlineto{\pgfqpoint{2.096158in}{2.846223in}}%
\pgfpathlineto{\pgfqpoint{2.096865in}{2.848263in}}%
\pgfpathlineto{\pgfqpoint{2.099999in}{2.860806in}}%
\pgfpathlineto{\pgfqpoint{2.100606in}{2.860969in}}%
\pgfpathlineto{\pgfqpoint{2.101111in}{2.860501in}}%
\pgfpathlineto{\pgfqpoint{2.107379in}{2.858458in}}%
\pgfpathlineto{\pgfqpoint{2.112838in}{2.869367in}}%
\pgfpathlineto{\pgfqpoint{2.114658in}{2.872927in}}%
\pgfpathlineto{\pgfqpoint{2.115568in}{2.874987in}}%
\pgfpathlineto{\pgfqpoint{2.116275in}{2.874319in}}%
\pgfpathlineto{\pgfqpoint{2.117084in}{2.872594in}}%
\pgfpathlineto{\pgfqpoint{2.120521in}{2.860247in}}%
\pgfpathlineto{\pgfqpoint{2.120622in}{2.860287in}}%
\pgfpathlineto{\pgfqpoint{2.123049in}{2.860841in}}%
\pgfpathlineto{\pgfqpoint{2.127295in}{2.860417in}}%
\pgfpathlineto{\pgfqpoint{2.130024in}{2.859062in}}%
\pgfpathlineto{\pgfqpoint{2.131035in}{2.857917in}}%
\pgfpathlineto{\pgfqpoint{2.134270in}{2.848026in}}%
\pgfpathlineto{\pgfqpoint{2.135079in}{2.846214in}}%
\pgfpathlineto{\pgfqpoint{2.136898in}{2.842855in}}%
\pgfpathlineto{\pgfqpoint{2.143267in}{2.840989in}}%
\pgfpathlineto{\pgfqpoint{2.143874in}{2.841469in}}%
\pgfpathlineto{\pgfqpoint{2.144480in}{2.840964in}}%
\pgfpathlineto{\pgfqpoint{2.144986in}{2.841580in}}%
\pgfpathlineto{\pgfqpoint{2.147918in}{2.852749in}}%
\pgfpathlineto{\pgfqpoint{2.148524in}{2.851596in}}%
\pgfpathlineto{\pgfqpoint{2.149535in}{2.851186in}}%
\pgfpathlineto{\pgfqpoint{2.149737in}{2.851381in}}%
\pgfpathlineto{\pgfqpoint{2.152366in}{2.853683in}}%
\pgfpathlineto{\pgfqpoint{2.155196in}{2.853204in}}%
\pgfpathlineto{\pgfqpoint{2.156005in}{2.857481in}}%
\pgfpathlineto{\pgfqpoint{2.157623in}{2.861016in}}%
\pgfpathlineto{\pgfqpoint{2.160655in}{2.863481in}}%
\pgfpathlineto{\pgfqpoint{2.163688in}{2.863296in}}%
\pgfpathlineto{\pgfqpoint{2.166013in}{2.855849in}}%
\pgfpathlineto{\pgfqpoint{2.167631in}{2.851486in}}%
\pgfpathlineto{\pgfqpoint{2.167833in}{2.851772in}}%
\pgfpathlineto{\pgfqpoint{2.169855in}{2.854766in}}%
\pgfpathlineto{\pgfqpoint{2.170057in}{2.854588in}}%
\pgfpathlineto{\pgfqpoint{2.172281in}{2.853180in}}%
\pgfpathlineto{\pgfqpoint{2.174910in}{2.853235in}}%
\pgfpathlineto{\pgfqpoint{2.175617in}{2.852422in}}%
\pgfpathlineto{\pgfqpoint{2.178145in}{2.846315in}}%
\pgfpathlineto{\pgfqpoint{2.179257in}{2.843813in}}%
\pgfpathlineto{\pgfqpoint{2.179762in}{2.844955in}}%
\pgfpathlineto{\pgfqpoint{2.181986in}{2.849920in}}%
\pgfpathlineto{\pgfqpoint{2.182289in}{2.849659in}}%
\pgfpathlineto{\pgfqpoint{2.184008in}{2.848856in}}%
\pgfpathlineto{\pgfqpoint{2.185727in}{2.850341in}}%
\pgfpathlineto{\pgfqpoint{2.187344in}{2.851247in}}%
\pgfpathlineto{\pgfqpoint{2.187951in}{2.849431in}}%
\pgfpathlineto{\pgfqpoint{2.189871in}{2.844273in}}%
\pgfpathlineto{\pgfqpoint{2.191792in}{2.844044in}}%
\pgfpathlineto{\pgfqpoint{2.191893in}{2.844231in}}%
\pgfpathlineto{\pgfqpoint{2.193915in}{2.845873in}}%
\pgfpathlineto{\pgfqpoint{2.195229in}{2.847656in}}%
\pgfpathlineto{\pgfqpoint{2.196139in}{2.848020in}}%
\pgfpathlineto{\pgfqpoint{2.196544in}{2.847715in}}%
\pgfpathlineto{\pgfqpoint{2.197150in}{2.848556in}}%
\pgfpathlineto{\pgfqpoint{2.199981in}{2.855285in}}%
\pgfpathlineto{\pgfqpoint{2.200284in}{2.855043in}}%
\pgfpathlineto{\pgfqpoint{2.201497in}{2.852286in}}%
\pgfpathlineto{\pgfqpoint{2.202104in}{2.851424in}}%
\pgfpathlineto{\pgfqpoint{2.202710in}{2.852024in}}%
\pgfpathlineto{\pgfqpoint{2.203620in}{2.854002in}}%
\pgfpathlineto{\pgfqpoint{2.206855in}{2.870890in}}%
\pgfpathlineto{\pgfqpoint{2.209686in}{2.874709in}}%
\pgfpathlineto{\pgfqpoint{2.211910in}{2.876893in}}%
\pgfpathlineto{\pgfqpoint{2.213628in}{2.877563in}}%
\pgfpathlineto{\pgfqpoint{2.214943in}{2.877010in}}%
\pgfpathlineto{\pgfqpoint{2.215347in}{2.876936in}}%
\pgfpathlineto{\pgfqpoint{2.215650in}{2.877792in}}%
\pgfpathlineto{\pgfqpoint{2.216762in}{2.880609in}}%
\pgfpathlineto{\pgfqpoint{2.217167in}{2.879829in}}%
\pgfpathlineto{\pgfqpoint{2.217874in}{2.877957in}}%
\pgfpathlineto{\pgfqpoint{2.218683in}{2.878426in}}%
\pgfpathlineto{\pgfqpoint{2.219593in}{2.875250in}}%
\pgfpathlineto{\pgfqpoint{2.220503in}{2.872112in}}%
\pgfpathlineto{\pgfqpoint{2.221109in}{2.873245in}}%
\pgfpathlineto{\pgfqpoint{2.222221in}{2.874592in}}%
\pgfpathlineto{\pgfqpoint{2.222626in}{2.874268in}}%
\pgfpathlineto{\pgfqpoint{2.223030in}{2.874291in}}%
\pgfpathlineto{\pgfqpoint{2.223333in}{2.874958in}}%
\pgfpathlineto{\pgfqpoint{2.223738in}{2.875638in}}%
\pgfpathlineto{\pgfqpoint{2.224142in}{2.874015in}}%
\pgfpathlineto{\pgfqpoint{2.226669in}{2.862466in}}%
\pgfpathlineto{\pgfqpoint{2.228287in}{2.858958in}}%
\pgfpathlineto{\pgfqpoint{2.228590in}{2.859519in}}%
\pgfpathlineto{\pgfqpoint{2.229601in}{2.860963in}}%
\pgfpathlineto{\pgfqpoint{2.230107in}{2.860560in}}%
\pgfpathlineto{\pgfqpoint{2.232937in}{2.859138in}}%
\pgfpathlineto{\pgfqpoint{2.234150in}{2.860086in}}%
\pgfpathlineto{\pgfqpoint{2.235363in}{2.865402in}}%
\pgfpathlineto{\pgfqpoint{2.235869in}{2.863843in}}%
\pgfpathlineto{\pgfqpoint{2.237385in}{2.862337in}}%
\pgfpathlineto{\pgfqpoint{2.239003in}{2.863874in}}%
\pgfpathlineto{\pgfqpoint{2.240721in}{2.868737in}}%
\pgfpathlineto{\pgfqpoint{2.240924in}{2.868608in}}%
\pgfpathlineto{\pgfqpoint{2.247292in}{2.860212in}}%
\pgfpathlineto{\pgfqpoint{2.248809in}{2.859590in}}%
\pgfpathlineto{\pgfqpoint{2.249314in}{2.859274in}}%
\pgfpathlineto{\pgfqpoint{2.249719in}{2.860116in}}%
\pgfpathlineto{\pgfqpoint{2.252954in}{2.871429in}}%
\pgfpathlineto{\pgfqpoint{2.253864in}{2.870640in}}%
\pgfpathlineto{\pgfqpoint{2.254571in}{2.867788in}}%
\pgfpathlineto{\pgfqpoint{2.256391in}{2.863118in}}%
\pgfpathlineto{\pgfqpoint{2.257604in}{2.862439in}}%
\pgfpathlineto{\pgfqpoint{2.257907in}{2.862966in}}%
\pgfpathlineto{\pgfqpoint{2.258817in}{2.866635in}}%
\pgfpathlineto{\pgfqpoint{2.259323in}{2.864656in}}%
\pgfpathlineto{\pgfqpoint{2.259929in}{2.863060in}}%
\pgfpathlineto{\pgfqpoint{2.260637in}{2.863340in}}%
\pgfpathlineto{\pgfqpoint{2.265489in}{2.862955in}}%
\pgfpathlineto{\pgfqpoint{2.266197in}{2.864864in}}%
\pgfpathlineto{\pgfqpoint{2.267410in}{2.867947in}}%
\pgfpathlineto{\pgfqpoint{2.267916in}{2.867818in}}%
\pgfpathlineto{\pgfqpoint{2.270443in}{2.867006in}}%
\pgfpathlineto{\pgfqpoint{2.272869in}{2.861028in}}%
\pgfpathlineto{\pgfqpoint{2.274284in}{2.860997in}}%
\pgfpathlineto{\pgfqpoint{2.277519in}{2.861165in}}%
\pgfpathlineto{\pgfqpoint{2.278227in}{2.859979in}}%
\pgfpathlineto{\pgfqpoint{2.281058in}{2.851379in}}%
\pgfpathlineto{\pgfqpoint{2.282473in}{2.851152in}}%
\pgfpathlineto{\pgfqpoint{2.284596in}{2.849956in}}%
\pgfpathlineto{\pgfqpoint{2.286315in}{2.848206in}}%
\pgfpathlineto{\pgfqpoint{2.287831in}{2.844275in}}%
\pgfpathlineto{\pgfqpoint{2.288235in}{2.844431in}}%
\pgfpathlineto{\pgfqpoint{2.289246in}{2.843444in}}%
\pgfpathlineto{\pgfqpoint{2.290156in}{2.841088in}}%
\pgfpathlineto{\pgfqpoint{2.292380in}{2.836251in}}%
\pgfpathlineto{\pgfqpoint{2.294402in}{2.834961in}}%
\pgfpathlineto{\pgfqpoint{2.297536in}{2.835604in}}%
\pgfpathlineto{\pgfqpoint{2.299760in}{2.837852in}}%
\pgfpathlineto{\pgfqpoint{2.302692in}{2.839103in}}%
\pgfpathlineto{\pgfqpoint{2.304714in}{2.841261in}}%
\pgfpathlineto{\pgfqpoint{2.309667in}{2.842225in}}%
\pgfpathlineto{\pgfqpoint{2.312902in}{2.843767in}}%
\pgfpathlineto{\pgfqpoint{2.315227in}{2.842916in}}%
\pgfpathlineto{\pgfqpoint{2.317350in}{2.842309in}}%
\pgfpathlineto{\pgfqpoint{2.319675in}{2.839359in}}%
\pgfpathlineto{\pgfqpoint{2.323618in}{2.837110in}}%
\pgfpathlineto{\pgfqpoint{2.325842in}{2.836107in}}%
\pgfpathlineto{\pgfqpoint{2.328268in}{2.836564in}}%
\pgfpathlineto{\pgfqpoint{2.329279in}{2.837119in}}%
\pgfpathlineto{\pgfqpoint{2.329684in}{2.836774in}}%
\pgfpathlineto{\pgfqpoint{2.331706in}{2.836286in}}%
\pgfpathlineto{\pgfqpoint{2.334233in}{2.835264in}}%
\pgfpathlineto{\pgfqpoint{2.334435in}{2.835758in}}%
\pgfpathlineto{\pgfqpoint{2.338074in}{2.844561in}}%
\pgfpathlineto{\pgfqpoint{2.339894in}{2.846417in}}%
\pgfpathlineto{\pgfqpoint{2.341714in}{2.848048in}}%
\pgfpathlineto{\pgfqpoint{2.346971in}{2.850123in}}%
\pgfpathlineto{\pgfqpoint{2.348588in}{2.850912in}}%
\pgfpathlineto{\pgfqpoint{2.353340in}{2.851014in}}%
\pgfpathlineto{\pgfqpoint{2.354856in}{2.849554in}}%
\pgfpathlineto{\pgfqpoint{2.357484in}{2.843130in}}%
\pgfpathlineto{\pgfqpoint{2.357687in}{2.843237in}}%
\pgfpathlineto{\pgfqpoint{2.358394in}{2.845555in}}%
\pgfpathlineto{\pgfqpoint{2.359708in}{2.850949in}}%
\pgfpathlineto{\pgfqpoint{2.360214in}{2.850103in}}%
\pgfpathlineto{\pgfqpoint{2.361629in}{2.848690in}}%
\pgfpathlineto{\pgfqpoint{2.361831in}{2.848935in}}%
\pgfpathlineto{\pgfqpoint{2.364561in}{2.852654in}}%
\pgfpathlineto{\pgfqpoint{2.368605in}{2.851323in}}%
\pgfpathlineto{\pgfqpoint{2.370121in}{2.851119in}}%
\pgfpathlineto{\pgfqpoint{2.372143in}{2.852149in}}%
\pgfpathlineto{\pgfqpoint{2.375378in}{2.853462in}}%
\pgfpathlineto{\pgfqpoint{2.377905in}{2.852141in}}%
\pgfpathlineto{\pgfqpoint{2.378613in}{2.849601in}}%
\pgfpathlineto{\pgfqpoint{2.380230in}{2.844193in}}%
\pgfpathlineto{\pgfqpoint{2.380534in}{2.844437in}}%
\pgfpathlineto{\pgfqpoint{2.381848in}{2.846584in}}%
\pgfpathlineto{\pgfqpoint{2.382252in}{2.845694in}}%
\pgfpathlineto{\pgfqpoint{2.383566in}{2.841165in}}%
\pgfpathlineto{\pgfqpoint{2.384274in}{2.841667in}}%
\pgfpathlineto{\pgfqpoint{2.385487in}{2.843399in}}%
\pgfpathlineto{\pgfqpoint{2.388419in}{2.860189in}}%
\pgfpathlineto{\pgfqpoint{2.390542in}{2.861771in}}%
\pgfpathlineto{\pgfqpoint{2.393271in}{2.862178in}}%
\pgfpathlineto{\pgfqpoint{2.396102in}{2.881673in}}%
\pgfpathlineto{\pgfqpoint{2.397922in}{2.889673in}}%
\pgfpathlineto{\pgfqpoint{2.399944in}{2.889377in}}%
\pgfpathlineto{\pgfqpoint{2.402572in}{2.889526in}}%
\pgfpathlineto{\pgfqpoint{2.403785in}{2.890206in}}%
\pgfpathlineto{\pgfqpoint{2.403987in}{2.890095in}}%
\pgfpathlineto{\pgfqpoint{2.405908in}{2.887595in}}%
\pgfpathlineto{\pgfqpoint{2.408435in}{2.874565in}}%
\pgfpathlineto{\pgfqpoint{2.409952in}{2.873326in}}%
\pgfpathlineto{\pgfqpoint{2.410154in}{2.873795in}}%
\pgfpathlineto{\pgfqpoint{2.411165in}{2.883157in}}%
\pgfpathlineto{\pgfqpoint{2.412782in}{2.887206in}}%
\pgfpathlineto{\pgfqpoint{2.413389in}{2.886021in}}%
\pgfpathlineto{\pgfqpoint{2.414703in}{2.875759in}}%
\pgfpathlineto{\pgfqpoint{2.418848in}{2.858459in}}%
\pgfpathlineto{\pgfqpoint{2.418949in}{2.858481in}}%
\pgfpathlineto{\pgfqpoint{2.421982in}{2.858696in}}%
\pgfpathlineto{\pgfqpoint{2.424509in}{2.858661in}}%
\pgfpathlineto{\pgfqpoint{2.424610in}{2.859017in}}%
\pgfpathlineto{\pgfqpoint{2.425823in}{2.861612in}}%
\pgfpathlineto{\pgfqpoint{2.426228in}{2.861242in}}%
\pgfpathlineto{\pgfqpoint{2.426834in}{2.860836in}}%
\pgfpathlineto{\pgfqpoint{2.427138in}{2.861626in}}%
\pgfpathlineto{\pgfqpoint{2.429766in}{2.869342in}}%
\pgfpathlineto{\pgfqpoint{2.430171in}{2.869447in}}%
\pgfpathlineto{\pgfqpoint{2.430474in}{2.868578in}}%
\pgfpathlineto{\pgfqpoint{2.431687in}{2.859751in}}%
\pgfpathlineto{\pgfqpoint{2.432799in}{2.860925in}}%
\pgfpathlineto{\pgfqpoint{2.434619in}{2.871197in}}%
\pgfpathlineto{\pgfqpoint{2.435630in}{2.870048in}}%
\pgfpathlineto{\pgfqpoint{2.437348in}{2.869548in}}%
\pgfpathlineto{\pgfqpoint{2.439067in}{2.870922in}}%
\pgfpathlineto{\pgfqpoint{2.439471in}{2.870494in}}%
\pgfpathlineto{\pgfqpoint{2.440988in}{2.870485in}}%
\pgfpathlineto{\pgfqpoint{2.443515in}{2.870630in}}%
\pgfpathlineto{\pgfqpoint{2.447255in}{2.865447in}}%
\pgfpathlineto{\pgfqpoint{2.448266in}{2.860123in}}%
\pgfpathlineto{\pgfqpoint{2.448873in}{2.862513in}}%
\pgfpathlineto{\pgfqpoint{2.450490in}{2.865302in}}%
\pgfpathlineto{\pgfqpoint{2.451299in}{2.865495in}}%
\pgfpathlineto{\pgfqpoint{2.451602in}{2.864916in}}%
\pgfpathlineto{\pgfqpoint{2.456050in}{2.854885in}}%
\pgfpathlineto{\pgfqpoint{2.459791in}{2.854524in}}%
\pgfpathlineto{\pgfqpoint{2.459892in}{2.854899in}}%
\pgfpathlineto{\pgfqpoint{2.461712in}{2.857413in}}%
\pgfpathlineto{\pgfqpoint{2.463127in}{2.857829in}}%
\pgfpathlineto{\pgfqpoint{2.466766in}{2.857858in}}%
\pgfpathlineto{\pgfqpoint{2.468889in}{2.868006in}}%
\pgfpathlineto{\pgfqpoint{2.469597in}{2.866105in}}%
\pgfpathlineto{\pgfqpoint{2.473742in}{2.865699in}}%
\pgfpathlineto{\pgfqpoint{2.474652in}{2.862970in}}%
\pgfpathlineto{\pgfqpoint{2.475359in}{2.863534in}}%
\pgfpathlineto{\pgfqpoint{2.479302in}{2.868727in}}%
\pgfpathlineto{\pgfqpoint{2.480313in}{2.868108in}}%
\pgfpathlineto{\pgfqpoint{2.482234in}{2.865523in}}%
\pgfpathlineto{\pgfqpoint{2.482537in}{2.866077in}}%
\pgfpathlineto{\pgfqpoint{2.484862in}{2.869130in}}%
\pgfpathlineto{\pgfqpoint{2.486985in}{2.866988in}}%
\pgfpathlineto{\pgfqpoint{2.491130in}{2.854923in}}%
\pgfpathlineto{\pgfqpoint{2.492141in}{2.854790in}}%
\pgfpathlineto{\pgfqpoint{2.492343in}{2.855549in}}%
\pgfpathlineto{\pgfqpoint{2.494466in}{2.862490in}}%
\pgfpathlineto{\pgfqpoint{2.494668in}{2.862391in}}%
\pgfpathlineto{\pgfqpoint{2.496892in}{2.859749in}}%
\pgfpathlineto{\pgfqpoint{2.497297in}{2.860645in}}%
\pgfpathlineto{\pgfqpoint{2.498409in}{2.862832in}}%
\pgfpathlineto{\pgfqpoint{2.499015in}{2.862538in}}%
\pgfpathlineto{\pgfqpoint{2.500936in}{2.862930in}}%
\pgfpathlineto{\pgfqpoint{2.502452in}{2.868162in}}%
\pgfpathlineto{\pgfqpoint{2.503665in}{2.866493in}}%
\pgfpathlineto{\pgfqpoint{2.506395in}{2.866928in}}%
\pgfpathlineto{\pgfqpoint{2.507810in}{2.869995in}}%
\pgfpathlineto{\pgfqpoint{2.508417in}{2.872403in}}%
\pgfpathlineto{\pgfqpoint{2.508922in}{2.870854in}}%
\pgfpathlineto{\pgfqpoint{2.509529in}{2.869510in}}%
\pgfpathlineto{\pgfqpoint{2.510237in}{2.869850in}}%
\pgfpathlineto{\pgfqpoint{2.512461in}{2.869083in}}%
\pgfpathlineto{\pgfqpoint{2.514988in}{2.860826in}}%
\pgfpathlineto{\pgfqpoint{2.515392in}{2.861138in}}%
\pgfpathlineto{\pgfqpoint{2.517313in}{2.860711in}}%
\pgfpathlineto{\pgfqpoint{2.518324in}{2.858492in}}%
\pgfpathlineto{\pgfqpoint{2.518931in}{2.859132in}}%
\pgfpathlineto{\pgfqpoint{2.519840in}{2.859128in}}%
\pgfpathlineto{\pgfqpoint{2.520043in}{2.858733in}}%
\pgfpathlineto{\pgfqpoint{2.524794in}{2.849345in}}%
\pgfpathlineto{\pgfqpoint{2.527018in}{2.849012in}}%
\pgfpathlineto{\pgfqpoint{2.527726in}{2.850163in}}%
\pgfpathlineto{\pgfqpoint{2.528130in}{2.848967in}}%
\pgfpathlineto{\pgfqpoint{2.530253in}{2.844182in}}%
\pgfpathlineto{\pgfqpoint{2.531466in}{2.844323in}}%
\pgfpathlineto{\pgfqpoint{2.531567in}{2.844454in}}%
\pgfpathlineto{\pgfqpoint{2.534903in}{2.849181in}}%
\pgfpathlineto{\pgfqpoint{2.535409in}{2.848530in}}%
\pgfpathlineto{\pgfqpoint{2.539250in}{2.844212in}}%
\pgfpathlineto{\pgfqpoint{2.539351in}{2.844278in}}%
\pgfpathlineto{\pgfqpoint{2.540160in}{2.846949in}}%
\pgfpathlineto{\pgfqpoint{2.541575in}{2.848402in}}%
\pgfpathlineto{\pgfqpoint{2.542586in}{2.850087in}}%
\pgfpathlineto{\pgfqpoint{2.544912in}{2.854003in}}%
\pgfpathlineto{\pgfqpoint{2.547439in}{2.853197in}}%
\pgfpathlineto{\pgfqpoint{2.548147in}{2.851644in}}%
\pgfpathlineto{\pgfqpoint{2.548551in}{2.852955in}}%
\pgfpathlineto{\pgfqpoint{2.550472in}{2.857092in}}%
\pgfpathlineto{\pgfqpoint{2.551584in}{2.858338in}}%
\pgfpathlineto{\pgfqpoint{2.551988in}{2.857829in}}%
\pgfpathlineto{\pgfqpoint{2.552595in}{2.857224in}}%
\pgfpathlineto{\pgfqpoint{2.553100in}{2.858031in}}%
\pgfpathlineto{\pgfqpoint{2.553606in}{2.858368in}}%
\pgfpathlineto{\pgfqpoint{2.554010in}{2.857666in}}%
\pgfpathlineto{\pgfqpoint{2.554617in}{2.856950in}}%
\pgfpathlineto{\pgfqpoint{2.555021in}{2.857778in}}%
\pgfpathlineto{\pgfqpoint{2.557043in}{2.860450in}}%
\pgfpathlineto{\pgfqpoint{2.558357in}{2.862707in}}%
\pgfpathlineto{\pgfqpoint{2.559570in}{2.863744in}}%
\pgfpathlineto{\pgfqpoint{2.559772in}{2.863594in}}%
\pgfpathlineto{\pgfqpoint{2.563007in}{2.860299in}}%
\pgfpathlineto{\pgfqpoint{2.564119in}{2.859323in}}%
\pgfpathlineto{\pgfqpoint{2.564524in}{2.860008in}}%
\pgfpathlineto{\pgfqpoint{2.567658in}{2.870540in}}%
\pgfpathlineto{\pgfqpoint{2.568466in}{2.868415in}}%
\pgfpathlineto{\pgfqpoint{2.569578in}{2.863664in}}%
\pgfpathlineto{\pgfqpoint{2.570185in}{2.864586in}}%
\pgfpathlineto{\pgfqpoint{2.570589in}{2.864560in}}%
\pgfpathlineto{\pgfqpoint{2.570994in}{2.863963in}}%
\pgfpathlineto{\pgfqpoint{2.572409in}{2.863822in}}%
\pgfpathlineto{\pgfqpoint{2.574431in}{2.864319in}}%
\pgfpathlineto{\pgfqpoint{2.577261in}{2.871545in}}%
\pgfpathlineto{\pgfqpoint{2.578171in}{2.872169in}}%
\pgfpathlineto{\pgfqpoint{2.578576in}{2.871374in}}%
\pgfpathlineto{\pgfqpoint{2.580395in}{2.870005in}}%
\pgfpathlineto{\pgfqpoint{2.584338in}{2.869893in}}%
\pgfpathlineto{\pgfqpoint{2.586158in}{2.865324in}}%
\pgfpathlineto{\pgfqpoint{2.587472in}{2.857981in}}%
\pgfpathlineto{\pgfqpoint{2.588584in}{2.859774in}}%
\pgfpathlineto{\pgfqpoint{2.591313in}{2.867265in}}%
\pgfpathlineto{\pgfqpoint{2.593639in}{2.866978in}}%
\pgfpathlineto{\pgfqpoint{2.595054in}{2.862632in}}%
\pgfpathlineto{\pgfqpoint{2.597076in}{2.853590in}}%
\pgfpathlineto{\pgfqpoint{2.598087in}{2.851904in}}%
\pgfpathlineto{\pgfqpoint{2.599098in}{2.851020in}}%
\pgfpathlineto{\pgfqpoint{2.599502in}{2.851235in}}%
\pgfpathlineto{\pgfqpoint{2.601726in}{2.850838in}}%
\pgfpathlineto{\pgfqpoint{2.603647in}{2.849292in}}%
\pgfpathlineto{\pgfqpoint{2.603748in}{2.849388in}}%
\pgfpathlineto{\pgfqpoint{2.608600in}{2.854453in}}%
\pgfpathlineto{\pgfqpoint{2.608702in}{2.854332in}}%
\pgfpathlineto{\pgfqpoint{2.609611in}{2.849417in}}%
\pgfpathlineto{\pgfqpoint{2.611734in}{2.845353in}}%
\pgfpathlineto{\pgfqpoint{2.612745in}{2.846530in}}%
\pgfpathlineto{\pgfqpoint{2.619013in}{2.854770in}}%
\pgfpathlineto{\pgfqpoint{2.624371in}{2.855014in}}%
\pgfpathlineto{\pgfqpoint{2.624775in}{2.854734in}}%
\pgfpathlineto{\pgfqpoint{2.625180in}{2.855636in}}%
\pgfpathlineto{\pgfqpoint{2.626999in}{2.857855in}}%
\pgfpathlineto{\pgfqpoint{2.628920in}{2.860464in}}%
\pgfpathlineto{\pgfqpoint{2.630841in}{2.867011in}}%
\pgfpathlineto{\pgfqpoint{2.632357in}{2.868108in}}%
\pgfpathlineto{\pgfqpoint{2.632458in}{2.868016in}}%
\pgfpathlineto{\pgfqpoint{2.635491in}{2.863454in}}%
\pgfpathlineto{\pgfqpoint{2.636401in}{2.864575in}}%
\pgfpathlineto{\pgfqpoint{2.637311in}{2.865412in}}%
\pgfpathlineto{\pgfqpoint{2.639737in}{2.871136in}}%
\pgfpathlineto{\pgfqpoint{2.640344in}{2.873357in}}%
\pgfpathlineto{\pgfqpoint{2.643275in}{2.886549in}}%
\pgfpathlineto{\pgfqpoint{2.644286in}{2.885875in}}%
\pgfpathlineto{\pgfqpoint{2.645196in}{2.884565in}}%
\pgfpathlineto{\pgfqpoint{2.646915in}{2.882369in}}%
\pgfpathlineto{\pgfqpoint{2.647319in}{2.882099in}}%
\pgfpathlineto{\pgfqpoint{2.647825in}{2.882892in}}%
\pgfpathlineto{\pgfqpoint{2.648330in}{2.883263in}}%
\pgfpathlineto{\pgfqpoint{2.648735in}{2.882592in}}%
\pgfpathlineto{\pgfqpoint{2.652374in}{2.874386in}}%
\pgfpathlineto{\pgfqpoint{2.652879in}{2.875219in}}%
\pgfpathlineto{\pgfqpoint{2.656822in}{2.882601in}}%
\pgfpathlineto{\pgfqpoint{2.657429in}{2.882542in}}%
\pgfpathlineto{\pgfqpoint{2.657631in}{2.882068in}}%
\pgfpathlineto{\pgfqpoint{2.658541in}{2.879724in}}%
\pgfpathlineto{\pgfqpoint{2.659147in}{2.880562in}}%
\pgfpathlineto{\pgfqpoint{2.659956in}{2.880899in}}%
\pgfpathlineto{\pgfqpoint{2.660259in}{2.880400in}}%
\pgfpathlineto{\pgfqpoint{2.665112in}{2.866588in}}%
\pgfpathlineto{\pgfqpoint{2.665718in}{2.867922in}}%
\pgfpathlineto{\pgfqpoint{2.667134in}{2.868963in}}%
\pgfpathlineto{\pgfqpoint{2.667639in}{2.869287in}}%
\pgfpathlineto{\pgfqpoint{2.668043in}{2.868637in}}%
\pgfpathlineto{\pgfqpoint{2.670166in}{2.864214in}}%
\pgfpathlineto{\pgfqpoint{2.670975in}{2.864981in}}%
\pgfpathlineto{\pgfqpoint{2.672491in}{2.869430in}}%
\pgfpathlineto{\pgfqpoint{2.673199in}{2.868186in}}%
\pgfpathlineto{\pgfqpoint{2.679669in}{2.854518in}}%
\pgfpathlineto{\pgfqpoint{2.679871in}{2.854868in}}%
\pgfpathlineto{\pgfqpoint{2.681287in}{2.858573in}}%
\pgfpathlineto{\pgfqpoint{2.682095in}{2.858230in}}%
\pgfpathlineto{\pgfqpoint{2.684825in}{2.861517in}}%
\pgfpathlineto{\pgfqpoint{2.685836in}{2.860516in}}%
\pgfpathlineto{\pgfqpoint{2.688060in}{2.854711in}}%
\pgfpathlineto{\pgfqpoint{2.688161in}{2.854776in}}%
\pgfpathlineto{\pgfqpoint{2.690992in}{2.857544in}}%
\pgfpathlineto{\pgfqpoint{2.691295in}{2.856992in}}%
\pgfpathlineto{\pgfqpoint{2.693115in}{2.854814in}}%
\pgfpathlineto{\pgfqpoint{2.695136in}{2.854403in}}%
\pgfpathlineto{\pgfqpoint{2.695237in}{2.854627in}}%
\pgfpathlineto{\pgfqpoint{2.699382in}{2.863038in}}%
\pgfpathlineto{\pgfqpoint{2.700090in}{2.863227in}}%
\pgfpathlineto{\pgfqpoint{2.700393in}{2.862498in}}%
\pgfpathlineto{\pgfqpoint{2.702314in}{2.859597in}}%
\pgfpathlineto{\pgfqpoint{2.702718in}{2.859428in}}%
\pgfpathlineto{\pgfqpoint{2.703123in}{2.860151in}}%
\pgfpathlineto{\pgfqpoint{2.707672in}{2.872195in}}%
\pgfpathlineto{\pgfqpoint{2.708380in}{2.874137in}}%
\pgfpathlineto{\pgfqpoint{2.709087in}{2.874004in}}%
\pgfpathlineto{\pgfqpoint{2.710907in}{2.874957in}}%
\pgfpathlineto{\pgfqpoint{2.711716in}{2.877012in}}%
\pgfpathlineto{\pgfqpoint{2.712423in}{2.876702in}}%
\pgfpathlineto{\pgfqpoint{2.713333in}{2.878932in}}%
\pgfpathlineto{\pgfqpoint{2.716063in}{2.882350in}}%
\pgfpathlineto{\pgfqpoint{2.717074in}{2.882544in}}%
\pgfpathlineto{\pgfqpoint{2.717377in}{2.881981in}}%
\pgfpathlineto{\pgfqpoint{2.721016in}{2.873878in}}%
\pgfpathlineto{\pgfqpoint{2.721724in}{2.874560in}}%
\pgfpathlineto{\pgfqpoint{2.722634in}{2.874633in}}%
\pgfpathlineto{\pgfqpoint{2.722836in}{2.874257in}}%
\pgfpathlineto{\pgfqpoint{2.725262in}{2.867504in}}%
\pgfpathlineto{\pgfqpoint{2.726172in}{2.868101in}}%
\pgfpathlineto{\pgfqpoint{2.726576in}{2.868311in}}%
\pgfpathlineto{\pgfqpoint{2.726779in}{2.868969in}}%
\pgfpathlineto{\pgfqpoint{2.727284in}{2.870374in}}%
\pgfpathlineto{\pgfqpoint{2.727688in}{2.868846in}}%
\pgfpathlineto{\pgfqpoint{2.728295in}{2.866783in}}%
\pgfpathlineto{\pgfqpoint{2.728902in}{2.867717in}}%
\pgfpathlineto{\pgfqpoint{2.729913in}{2.869589in}}%
\pgfpathlineto{\pgfqpoint{2.732238in}{2.880243in}}%
\pgfpathlineto{\pgfqpoint{2.732541in}{2.880877in}}%
\pgfpathlineto{\pgfqpoint{2.733046in}{2.879722in}}%
\pgfpathlineto{\pgfqpoint{2.735574in}{2.874357in}}%
\pgfpathlineto{\pgfqpoint{2.737596in}{2.871273in}}%
\pgfpathlineto{\pgfqpoint{2.737798in}{2.871674in}}%
\pgfpathlineto{\pgfqpoint{2.741336in}{2.880443in}}%
\pgfpathlineto{\pgfqpoint{2.742549in}{2.880969in}}%
\pgfpathlineto{\pgfqpoint{2.742853in}{2.880581in}}%
\pgfpathlineto{\pgfqpoint{2.743863in}{2.879158in}}%
\pgfpathlineto{\pgfqpoint{2.744470in}{2.879649in}}%
\pgfpathlineto{\pgfqpoint{2.747098in}{2.883050in}}%
\pgfpathlineto{\pgfqpoint{2.747402in}{2.882461in}}%
\pgfpathlineto{\pgfqpoint{2.750536in}{2.872364in}}%
\pgfpathlineto{\pgfqpoint{2.752659in}{2.859624in}}%
\pgfpathlineto{\pgfqpoint{2.753366in}{2.859158in}}%
\pgfpathlineto{\pgfqpoint{2.753771in}{2.859629in}}%
\pgfpathlineto{\pgfqpoint{2.756298in}{2.865044in}}%
\pgfpathlineto{\pgfqpoint{2.756904in}{2.864456in}}%
\pgfpathlineto{\pgfqpoint{2.758219in}{2.862471in}}%
\pgfpathlineto{\pgfqpoint{2.759533in}{2.855721in}}%
\pgfpathlineto{\pgfqpoint{2.760139in}{2.857271in}}%
\pgfpathlineto{\pgfqpoint{2.763476in}{2.865348in}}%
\pgfpathlineto{\pgfqpoint{2.764790in}{2.864270in}}%
\pgfpathlineto{\pgfqpoint{2.769238in}{2.855052in}}%
\pgfpathlineto{\pgfqpoint{2.769743in}{2.855591in}}%
\pgfpathlineto{\pgfqpoint{2.771563in}{2.856288in}}%
\pgfpathlineto{\pgfqpoint{2.772574in}{2.855079in}}%
\pgfpathlineto{\pgfqpoint{2.773181in}{2.854457in}}%
\pgfpathlineto{\pgfqpoint{2.773585in}{2.855108in}}%
\pgfpathlineto{\pgfqpoint{2.774191in}{2.856040in}}%
\pgfpathlineto{\pgfqpoint{2.774596in}{2.855173in}}%
\pgfpathlineto{\pgfqpoint{2.776112in}{2.854189in}}%
\pgfpathlineto{\pgfqpoint{2.777831in}{2.853397in}}%
\pgfpathlineto{\pgfqpoint{2.777932in}{2.853508in}}%
\pgfpathlineto{\pgfqpoint{2.780358in}{2.857691in}}%
\pgfpathlineto{\pgfqpoint{2.781066in}{2.856640in}}%
\pgfpathlineto{\pgfqpoint{2.782683in}{2.847724in}}%
\pgfpathlineto{\pgfqpoint{2.783896in}{2.848600in}}%
\pgfpathlineto{\pgfqpoint{2.787131in}{2.851661in}}%
\pgfpathlineto{\pgfqpoint{2.789052in}{2.853410in}}%
\pgfpathlineto{\pgfqpoint{2.791580in}{2.854739in}}%
\pgfpathlineto{\pgfqpoint{2.793399in}{2.857151in}}%
\pgfpathlineto{\pgfqpoint{2.793804in}{2.857573in}}%
\pgfpathlineto{\pgfqpoint{2.794309in}{2.856801in}}%
\pgfpathlineto{\pgfqpoint{2.794612in}{2.856500in}}%
\pgfpathlineto{\pgfqpoint{2.795017in}{2.857730in}}%
\pgfpathlineto{\pgfqpoint{2.797140in}{2.864524in}}%
\pgfpathlineto{\pgfqpoint{2.798050in}{2.865255in}}%
\pgfpathlineto{\pgfqpoint{2.798454in}{2.864554in}}%
\pgfpathlineto{\pgfqpoint{2.802194in}{2.858611in}}%
\pgfpathlineto{\pgfqpoint{2.804115in}{2.857700in}}%
\pgfpathlineto{\pgfqpoint{2.807249in}{2.854645in}}%
\pgfpathlineto{\pgfqpoint{2.809979in}{2.851171in}}%
\pgfpathlineto{\pgfqpoint{2.812203in}{2.849711in}}%
\pgfpathlineto{\pgfqpoint{2.813820in}{2.849177in}}%
\pgfpathlineto{\pgfqpoint{2.815741in}{2.843510in}}%
\pgfpathlineto{\pgfqpoint{2.817358in}{2.838053in}}%
\pgfpathlineto{\pgfqpoint{2.818976in}{2.837426in}}%
\pgfpathlineto{\pgfqpoint{2.821705in}{2.838842in}}%
\pgfpathlineto{\pgfqpoint{2.823727in}{2.840135in}}%
\pgfpathlineto{\pgfqpoint{2.832219in}{2.842779in}}%
\pgfpathlineto{\pgfqpoint{2.833736in}{2.842993in}}%
\pgfpathlineto{\pgfqpoint{2.836667in}{2.842650in}}%
\pgfpathlineto{\pgfqpoint{2.836768in}{2.842930in}}%
\pgfpathlineto{\pgfqpoint{2.839296in}{2.847464in}}%
\pgfpathlineto{\pgfqpoint{2.846979in}{2.850742in}}%
\pgfpathlineto{\pgfqpoint{2.851124in}{2.854391in}}%
\pgfpathlineto{\pgfqpoint{2.852337in}{2.854990in}}%
\pgfpathlineto{\pgfqpoint{2.852943in}{2.855040in}}%
\pgfpathlineto{\pgfqpoint{2.853348in}{2.854552in}}%
\pgfpathlineto{\pgfqpoint{2.854965in}{2.854685in}}%
\pgfpathlineto{\pgfqpoint{2.855572in}{2.856944in}}%
\pgfpathlineto{\pgfqpoint{2.860424in}{2.881540in}}%
\pgfpathlineto{\pgfqpoint{2.863356in}{2.882801in}}%
\pgfpathlineto{\pgfqpoint{2.866591in}{2.893450in}}%
\pgfpathlineto{\pgfqpoint{2.868107in}{2.892954in}}%
\pgfpathlineto{\pgfqpoint{2.869219in}{2.891798in}}%
\pgfpathlineto{\pgfqpoint{2.870432in}{2.889110in}}%
\pgfpathlineto{\pgfqpoint{2.870938in}{2.889436in}}%
\pgfpathlineto{\pgfqpoint{2.873162in}{2.894460in}}%
\pgfpathlineto{\pgfqpoint{2.874072in}{2.898016in}}%
\pgfpathlineto{\pgfqpoint{2.874678in}{2.897797in}}%
\pgfpathlineto{\pgfqpoint{2.875487in}{2.896639in}}%
\pgfpathlineto{\pgfqpoint{2.876296in}{2.893105in}}%
\pgfpathlineto{\pgfqpoint{2.876801in}{2.895191in}}%
\pgfpathlineto{\pgfqpoint{2.877105in}{2.895999in}}%
\pgfpathlineto{\pgfqpoint{2.877408in}{2.894091in}}%
\pgfpathlineto{\pgfqpoint{2.879733in}{2.878883in}}%
\pgfpathlineto{\pgfqpoint{2.880238in}{2.877468in}}%
\pgfpathlineto{\pgfqpoint{2.880744in}{2.878886in}}%
\pgfpathlineto{\pgfqpoint{2.881553in}{2.881694in}}%
\pgfpathlineto{\pgfqpoint{2.882159in}{2.880750in}}%
\pgfpathlineto{\pgfqpoint{2.882463in}{2.880523in}}%
\pgfpathlineto{\pgfqpoint{2.883069in}{2.881357in}}%
\pgfpathlineto{\pgfqpoint{2.883473in}{2.880711in}}%
\pgfpathlineto{\pgfqpoint{2.885495in}{2.871833in}}%
\pgfpathlineto{\pgfqpoint{2.886102in}{2.873299in}}%
\pgfpathlineto{\pgfqpoint{2.887315in}{2.877952in}}%
\pgfpathlineto{\pgfqpoint{2.889034in}{2.882206in}}%
\pgfpathlineto{\pgfqpoint{2.890045in}{2.884762in}}%
\pgfpathlineto{\pgfqpoint{2.892471in}{2.888638in}}%
\pgfpathlineto{\pgfqpoint{2.893077in}{2.887267in}}%
\pgfpathlineto{\pgfqpoint{2.894088in}{2.883049in}}%
\pgfpathlineto{\pgfqpoint{2.894695in}{2.884459in}}%
\pgfpathlineto{\pgfqpoint{2.896211in}{2.887581in}}%
\pgfpathlineto{\pgfqpoint{2.896616in}{2.886392in}}%
\pgfpathlineto{\pgfqpoint{2.898334in}{2.878237in}}%
\pgfpathlineto{\pgfqpoint{2.898941in}{2.878695in}}%
\pgfpathlineto{\pgfqpoint{2.900659in}{2.880273in}}%
\pgfpathlineto{\pgfqpoint{2.901064in}{2.879199in}}%
\pgfpathlineto{\pgfqpoint{2.902681in}{2.877625in}}%
\pgfpathlineto{\pgfqpoint{2.905714in}{2.875538in}}%
\pgfpathlineto{\pgfqpoint{2.908444in}{2.866730in}}%
\pgfpathlineto{\pgfqpoint{2.910971in}{2.867776in}}%
\pgfpathlineto{\pgfqpoint{2.911982in}{2.870868in}}%
\pgfpathlineto{\pgfqpoint{2.913195in}{2.872438in}}%
\pgfpathlineto{\pgfqpoint{2.913498in}{2.872291in}}%
\pgfpathlineto{\pgfqpoint{2.914509in}{2.870820in}}%
\pgfpathlineto{\pgfqpoint{2.916329in}{2.866594in}}%
\pgfpathlineto{\pgfqpoint{2.916632in}{2.866910in}}%
\pgfpathlineto{\pgfqpoint{2.918856in}{2.868411in}}%
\pgfpathlineto{\pgfqpoint{2.920979in}{2.865989in}}%
\pgfpathlineto{\pgfqpoint{2.921485in}{2.867128in}}%
\pgfpathlineto{\pgfqpoint{2.922799in}{2.868841in}}%
\pgfpathlineto{\pgfqpoint{2.923203in}{2.868571in}}%
\pgfpathlineto{\pgfqpoint{2.925528in}{2.866952in}}%
\pgfpathlineto{\pgfqpoint{2.925731in}{2.867307in}}%
\pgfpathlineto{\pgfqpoint{2.927045in}{2.870639in}}%
\pgfpathlineto{\pgfqpoint{2.927651in}{2.869446in}}%
\pgfpathlineto{\pgfqpoint{2.933919in}{2.853536in}}%
\pgfpathlineto{\pgfqpoint{2.934121in}{2.853747in}}%
\pgfpathlineto{\pgfqpoint{2.936143in}{2.855057in}}%
\pgfpathlineto{\pgfqpoint{2.940389in}{2.852611in}}%
\pgfpathlineto{\pgfqpoint{2.941198in}{2.851966in}}%
\pgfpathlineto{\pgfqpoint{2.943220in}{2.846429in}}%
\pgfpathlineto{\pgfqpoint{2.943927in}{2.847228in}}%
\pgfpathlineto{\pgfqpoint{2.945444in}{2.847712in}}%
\pgfpathlineto{\pgfqpoint{2.946252in}{2.845807in}}%
\pgfpathlineto{\pgfqpoint{2.947263in}{2.843013in}}%
\pgfpathlineto{\pgfqpoint{2.947769in}{2.843493in}}%
\pgfpathlineto{\pgfqpoint{2.949993in}{2.845452in}}%
\pgfpathlineto{\pgfqpoint{2.952925in}{2.845349in}}%
\pgfpathlineto{\pgfqpoint{2.953329in}{2.844856in}}%
\pgfpathlineto{\pgfqpoint{2.953733in}{2.845812in}}%
\pgfpathlineto{\pgfqpoint{2.954441in}{2.847392in}}%
\pgfpathlineto{\pgfqpoint{2.955048in}{2.846761in}}%
\pgfpathlineto{\pgfqpoint{2.955553in}{2.847734in}}%
\pgfpathlineto{\pgfqpoint{2.956968in}{2.850552in}}%
\pgfpathlineto{\pgfqpoint{2.957272in}{2.850342in}}%
\pgfpathlineto{\pgfqpoint{2.958889in}{2.850097in}}%
\pgfpathlineto{\pgfqpoint{2.960709in}{2.851727in}}%
\pgfpathlineto{\pgfqpoint{2.963438in}{2.853327in}}%
\pgfpathlineto{\pgfqpoint{2.963539in}{2.853216in}}%
\pgfpathlineto{\pgfqpoint{2.965865in}{2.851534in}}%
\pgfpathlineto{\pgfqpoint{2.968089in}{2.851003in}}%
\pgfpathlineto{\pgfqpoint{2.968999in}{2.850422in}}%
\pgfpathlineto{\pgfqpoint{2.969302in}{2.851067in}}%
\pgfpathlineto{\pgfqpoint{2.972335in}{2.861827in}}%
\pgfpathlineto{\pgfqpoint{2.972840in}{2.861686in}}%
\pgfpathlineto{\pgfqpoint{2.973750in}{2.860678in}}%
\pgfpathlineto{\pgfqpoint{2.976783in}{2.853026in}}%
\pgfpathlineto{\pgfqpoint{2.981130in}{2.852494in}}%
\pgfpathlineto{\pgfqpoint{2.982646in}{2.853210in}}%
\pgfpathlineto{\pgfqpoint{2.984061in}{2.852804in}}%
\pgfpathlineto{\pgfqpoint{2.989925in}{2.851242in}}%
\pgfpathlineto{\pgfqpoint{2.992250in}{2.846005in}}%
\pgfpathlineto{\pgfqpoint{2.994373in}{2.846698in}}%
\pgfpathlineto{\pgfqpoint{2.999630in}{2.857477in}}%
\pgfpathlineto{\pgfqpoint{2.999933in}{2.857294in}}%
\pgfpathlineto{\pgfqpoint{3.002359in}{2.856292in}}%
\pgfpathlineto{\pgfqpoint{3.002460in}{2.856542in}}%
\pgfpathlineto{\pgfqpoint{3.002764in}{2.861873in}}%
\pgfpathlineto{\pgfqpoint{3.003370in}{2.951905in}}%
\pgfpathlineto{\pgfqpoint{3.005999in}{3.318144in}}%
\pgfpathlineto{\pgfqpoint{3.007414in}{3.331158in}}%
\pgfpathlineto{\pgfqpoint{3.009335in}{3.350106in}}%
\pgfpathlineto{\pgfqpoint{3.011256in}{3.359025in}}%
\pgfpathlineto{\pgfqpoint{3.015097in}{3.362133in}}%
\pgfpathlineto{\pgfqpoint{3.017119in}{3.368253in}}%
\pgfpathlineto{\pgfqpoint{3.018231in}{3.368958in}}%
\pgfpathlineto{\pgfqpoint{3.022780in}{3.395958in}}%
\pgfpathlineto{\pgfqpoint{3.023387in}{3.402876in}}%
\pgfpathlineto{\pgfqpoint{3.023993in}{3.398849in}}%
\pgfpathlineto{\pgfqpoint{3.024499in}{3.401723in}}%
\pgfpathlineto{\pgfqpoint{3.024802in}{3.399016in}}%
\pgfpathlineto{\pgfqpoint{3.025813in}{3.373792in}}%
\pgfpathlineto{\pgfqpoint{3.026521in}{3.382121in}}%
\pgfpathlineto{\pgfqpoint{3.028543in}{3.391125in}}%
\pgfpathlineto{\pgfqpoint{3.028846in}{3.390870in}}%
\pgfpathlineto{\pgfqpoint{3.029452in}{3.391810in}}%
\pgfpathlineto{\pgfqpoint{3.029857in}{3.392002in}}%
\pgfpathlineto{\pgfqpoint{3.030362in}{3.391350in}}%
\pgfpathlineto{\pgfqpoint{3.031272in}{3.390648in}}%
\pgfpathlineto{\pgfqpoint{3.031676in}{3.391426in}}%
\pgfpathlineto{\pgfqpoint{3.034305in}{3.396242in}}%
\pgfpathlineto{\pgfqpoint{3.035215in}{3.395465in}}%
\pgfpathlineto{\pgfqpoint{3.036023in}{3.393407in}}%
\pgfpathlineto{\pgfqpoint{3.036327in}{3.394503in}}%
\pgfpathlineto{\pgfqpoint{3.038854in}{3.420009in}}%
\pgfpathlineto{\pgfqpoint{3.040067in}{3.419715in}}%
\pgfpathlineto{\pgfqpoint{3.042291in}{3.418375in}}%
\pgfpathlineto{\pgfqpoint{3.042898in}{3.416408in}}%
\pgfpathlineto{\pgfqpoint{3.047851in}{3.368027in}}%
\pgfpathlineto{\pgfqpoint{3.049469in}{3.357221in}}%
\pgfpathlineto{\pgfqpoint{3.050581in}{3.348268in}}%
\pgfpathlineto{\pgfqpoint{3.051086in}{3.350644in}}%
\pgfpathlineto{\pgfqpoint{3.052300in}{3.357146in}}%
\pgfpathlineto{\pgfqpoint{3.052805in}{3.357056in}}%
\pgfpathlineto{\pgfqpoint{3.053412in}{3.355301in}}%
\pgfpathlineto{\pgfqpoint{3.053917in}{3.353463in}}%
\pgfpathlineto{\pgfqpoint{3.054321in}{3.355426in}}%
\pgfpathlineto{\pgfqpoint{3.055737in}{3.364137in}}%
\pgfpathlineto{\pgfqpoint{3.056242in}{3.363511in}}%
\pgfpathlineto{\pgfqpoint{3.056748in}{3.355489in}}%
\pgfpathlineto{\pgfqpoint{3.058163in}{3.276592in}}%
\pgfpathlineto{\pgfqpoint{3.059780in}{3.167182in}}%
\pgfpathlineto{\pgfqpoint{3.059882in}{3.167537in}}%
\pgfpathlineto{\pgfqpoint{3.060892in}{3.182020in}}%
\pgfpathlineto{\pgfqpoint{3.062106in}{3.209406in}}%
\pgfpathlineto{\pgfqpoint{3.062712in}{3.209066in}}%
\pgfpathlineto{\pgfqpoint{3.063925in}{3.207009in}}%
\pgfpathlineto{\pgfqpoint{3.064633in}{3.201070in}}%
\pgfpathlineto{\pgfqpoint{3.064936in}{3.205689in}}%
\pgfpathlineto{\pgfqpoint{3.068171in}{3.278987in}}%
\pgfpathlineto{\pgfqpoint{3.069283in}{3.271231in}}%
\pgfpathlineto{\pgfqpoint{3.069991in}{3.275397in}}%
\pgfpathlineto{\pgfqpoint{3.070800in}{3.278928in}}%
\pgfpathlineto{\pgfqpoint{3.071204in}{3.277166in}}%
\pgfpathlineto{\pgfqpoint{3.075753in}{3.232045in}}%
\pgfpathlineto{\pgfqpoint{3.076360in}{3.235534in}}%
\pgfpathlineto{\pgfqpoint{3.079494in}{3.269846in}}%
\pgfpathlineto{\pgfqpoint{3.080201in}{3.266065in}}%
\pgfpathlineto{\pgfqpoint{3.081414in}{3.252186in}}%
\pgfpathlineto{\pgfqpoint{3.082425in}{3.228637in}}%
\pgfpathlineto{\pgfqpoint{3.083032in}{3.232108in}}%
\pgfpathlineto{\pgfqpoint{3.084548in}{3.239088in}}%
\pgfpathlineto{\pgfqpoint{3.084751in}{3.238669in}}%
\pgfpathlineto{\pgfqpoint{3.085256in}{3.229541in}}%
\pgfpathlineto{\pgfqpoint{3.088996in}{3.133313in}}%
\pgfpathlineto{\pgfqpoint{3.093546in}{3.104356in}}%
\pgfpathlineto{\pgfqpoint{3.094860in}{3.102694in}}%
\pgfpathlineto{\pgfqpoint{3.098095in}{3.049529in}}%
\pgfpathlineto{\pgfqpoint{3.099510in}{3.039963in}}%
\pgfpathlineto{\pgfqpoint{3.100117in}{3.038328in}}%
\pgfpathlineto{\pgfqpoint{3.100622in}{3.039542in}}%
\pgfpathlineto{\pgfqpoint{3.101027in}{3.040052in}}%
\pgfpathlineto{\pgfqpoint{3.101532in}{3.039151in}}%
\pgfpathlineto{\pgfqpoint{3.101835in}{3.039073in}}%
\pgfpathlineto{\pgfqpoint{3.102240in}{3.039775in}}%
\pgfpathlineto{\pgfqpoint{3.102442in}{3.039961in}}%
\pgfpathlineto{\pgfqpoint{3.102745in}{3.038604in}}%
\pgfpathlineto{\pgfqpoint{3.103857in}{3.019751in}}%
\pgfpathlineto{\pgfqpoint{3.104464in}{3.030295in}}%
\pgfpathlineto{\pgfqpoint{3.106283in}{3.050036in}}%
\pgfpathlineto{\pgfqpoint{3.106587in}{3.052195in}}%
\pgfpathlineto{\pgfqpoint{3.106991in}{3.047936in}}%
\pgfpathlineto{\pgfqpoint{3.108103in}{3.031523in}}%
\pgfpathlineto{\pgfqpoint{3.108710in}{3.034537in}}%
\pgfpathlineto{\pgfqpoint{3.112956in}{3.067426in}}%
\pgfpathlineto{\pgfqpoint{3.115382in}{3.068063in}}%
\pgfpathlineto{\pgfqpoint{3.115483in}{3.067869in}}%
\pgfpathlineto{\pgfqpoint{3.116191in}{3.065166in}}%
\pgfpathlineto{\pgfqpoint{3.116494in}{3.067760in}}%
\pgfpathlineto{\pgfqpoint{3.118010in}{3.088631in}}%
\pgfpathlineto{\pgfqpoint{3.118617in}{3.086505in}}%
\pgfpathlineto{\pgfqpoint{3.118920in}{3.087461in}}%
\pgfpathlineto{\pgfqpoint{3.119931in}{3.097142in}}%
\pgfpathlineto{\pgfqpoint{3.120538in}{3.094303in}}%
\pgfpathlineto{\pgfqpoint{3.122863in}{3.078055in}}%
\pgfpathlineto{\pgfqpoint{3.122964in}{3.078066in}}%
\pgfpathlineto{\pgfqpoint{3.123874in}{3.077632in}}%
\pgfpathlineto{\pgfqpoint{3.124581in}{3.070108in}}%
\pgfpathlineto{\pgfqpoint{3.127109in}{3.038836in}}%
\pgfpathlineto{\pgfqpoint{3.127311in}{3.038907in}}%
\pgfpathlineto{\pgfqpoint{3.128120in}{3.039218in}}%
\pgfpathlineto{\pgfqpoint{3.128322in}{3.038777in}}%
\pgfpathlineto{\pgfqpoint{3.129029in}{3.031727in}}%
\pgfpathlineto{\pgfqpoint{3.133478in}{2.976847in}}%
\pgfpathlineto{\pgfqpoint{3.133983in}{2.976979in}}%
\pgfpathlineto{\pgfqpoint{3.134185in}{2.977451in}}%
\pgfpathlineto{\pgfqpoint{3.135702in}{2.978626in}}%
\pgfpathlineto{\pgfqpoint{3.136106in}{2.978678in}}%
\pgfpathlineto{\pgfqpoint{3.136308in}{2.977915in}}%
\pgfpathlineto{\pgfqpoint{3.137016in}{2.963023in}}%
\pgfpathlineto{\pgfqpoint{3.139745in}{2.902685in}}%
\pgfpathlineto{\pgfqpoint{3.139846in}{2.901945in}}%
\pgfpathlineto{\pgfqpoint{3.140251in}{2.905693in}}%
\pgfpathlineto{\pgfqpoint{3.143385in}{2.943398in}}%
\pgfpathlineto{\pgfqpoint{3.143890in}{2.945262in}}%
\pgfpathlineto{\pgfqpoint{3.144396in}{2.943560in}}%
\pgfpathlineto{\pgfqpoint{3.146519in}{2.936065in}}%
\pgfpathlineto{\pgfqpoint{3.150562in}{2.921036in}}%
\pgfpathlineto{\pgfqpoint{3.151169in}{2.923224in}}%
\pgfpathlineto{\pgfqpoint{3.152281in}{2.933870in}}%
\pgfpathlineto{\pgfqpoint{3.156426in}{3.044101in}}%
\pgfpathlineto{\pgfqpoint{3.157740in}{3.045315in}}%
\pgfpathlineto{\pgfqpoint{3.158144in}{3.044518in}}%
\pgfpathlineto{\pgfqpoint{3.163098in}{3.026273in}}%
\pgfpathlineto{\pgfqpoint{3.163704in}{3.029832in}}%
\pgfpathlineto{\pgfqpoint{3.164715in}{3.034187in}}%
\pgfpathlineto{\pgfqpoint{3.165221in}{3.033497in}}%
\pgfpathlineto{\pgfqpoint{3.167243in}{3.028684in}}%
\pgfpathlineto{\pgfqpoint{3.167849in}{3.030044in}}%
\pgfpathlineto{\pgfqpoint{3.170781in}{3.038663in}}%
\pgfpathlineto{\pgfqpoint{3.171084in}{3.038560in}}%
\pgfpathlineto{\pgfqpoint{3.171489in}{3.039499in}}%
\pgfpathlineto{\pgfqpoint{3.172196in}{3.044732in}}%
\pgfpathlineto{\pgfqpoint{3.172601in}{3.039676in}}%
\pgfpathlineto{\pgfqpoint{3.176139in}{2.955958in}}%
\pgfpathlineto{\pgfqpoint{3.176442in}{2.957140in}}%
\pgfpathlineto{\pgfqpoint{3.179778in}{2.982902in}}%
\pgfpathlineto{\pgfqpoint{3.179981in}{2.982750in}}%
\pgfpathlineto{\pgfqpoint{3.180890in}{2.979294in}}%
\pgfpathlineto{\pgfqpoint{3.184934in}{2.951188in}}%
\pgfpathlineto{\pgfqpoint{3.185743in}{2.952059in}}%
\pgfpathlineto{\pgfqpoint{3.189180in}{2.961400in}}%
\pgfpathlineto{\pgfqpoint{3.189483in}{2.961081in}}%
\pgfpathlineto{\pgfqpoint{3.190494in}{2.958324in}}%
\pgfpathlineto{\pgfqpoint{3.191000in}{2.960075in}}%
\pgfpathlineto{\pgfqpoint{3.191404in}{2.961369in}}%
\pgfpathlineto{\pgfqpoint{3.191808in}{2.959703in}}%
\pgfpathlineto{\pgfqpoint{3.193224in}{2.950472in}}%
\pgfpathlineto{\pgfqpoint{3.193729in}{2.951455in}}%
\pgfpathlineto{\pgfqpoint{3.194740in}{2.956436in}}%
\pgfpathlineto{\pgfqpoint{3.195246in}{2.953385in}}%
\pgfpathlineto{\pgfqpoint{3.197773in}{2.937877in}}%
\pgfpathlineto{\pgfqpoint{3.199694in}{2.933213in}}%
\pgfpathlineto{\pgfqpoint{3.201008in}{2.930298in}}%
\pgfpathlineto{\pgfqpoint{3.203030in}{2.923791in}}%
\pgfpathlineto{\pgfqpoint{3.204647in}{2.921674in}}%
\pgfpathlineto{\pgfqpoint{3.206164in}{2.919747in}}%
\pgfpathlineto{\pgfqpoint{3.209702in}{2.906824in}}%
\pgfpathlineto{\pgfqpoint{3.210005in}{2.907139in}}%
\pgfpathlineto{\pgfqpoint{3.210814in}{2.908530in}}%
\pgfpathlineto{\pgfqpoint{3.211319in}{2.907312in}}%
\pgfpathlineto{\pgfqpoint{3.212735in}{2.901767in}}%
\pgfpathlineto{\pgfqpoint{3.213240in}{2.903427in}}%
\pgfpathlineto{\pgfqpoint{3.213746in}{2.905322in}}%
\pgfpathlineto{\pgfqpoint{3.214150in}{2.903339in}}%
\pgfpathlineto{\pgfqpoint{3.218699in}{2.865794in}}%
\pgfpathlineto{\pgfqpoint{3.219811in}{2.862003in}}%
\pgfpathlineto{\pgfqpoint{3.220418in}{2.863636in}}%
\pgfpathlineto{\pgfqpoint{3.223046in}{2.867340in}}%
\pgfpathlineto{\pgfqpoint{3.224057in}{2.868271in}}%
\pgfpathlineto{\pgfqpoint{3.224664in}{2.868885in}}%
\pgfpathlineto{\pgfqpoint{3.225270in}{2.868249in}}%
\pgfpathlineto{\pgfqpoint{3.226989in}{2.868202in}}%
\pgfpathlineto{\pgfqpoint{3.228404in}{2.867301in}}%
\pgfpathlineto{\pgfqpoint{3.236997in}{2.847353in}}%
\pgfpathlineto{\pgfqpoint{3.240232in}{2.846114in}}%
\pgfpathlineto{\pgfqpoint{3.241142in}{2.843526in}}%
\pgfpathlineto{\pgfqpoint{3.241749in}{2.844153in}}%
\pgfpathlineto{\pgfqpoint{3.242254in}{2.843263in}}%
\pgfpathlineto{\pgfqpoint{3.242962in}{2.841158in}}%
\pgfpathlineto{\pgfqpoint{3.243467in}{2.842349in}}%
\pgfpathlineto{\pgfqpoint{3.243973in}{2.843314in}}%
\pgfpathlineto{\pgfqpoint{3.244579in}{2.842411in}}%
\pgfpathlineto{\pgfqpoint{3.245085in}{2.842157in}}%
\pgfpathlineto{\pgfqpoint{3.245489in}{2.842746in}}%
\pgfpathlineto{\pgfqpoint{3.248522in}{2.847184in}}%
\pgfpathlineto{\pgfqpoint{3.251959in}{2.848398in}}%
\pgfpathlineto{\pgfqpoint{3.253273in}{2.850967in}}%
\pgfpathlineto{\pgfqpoint{3.254891in}{2.852464in}}%
\pgfpathlineto{\pgfqpoint{3.258328in}{2.853939in}}%
\pgfpathlineto{\pgfqpoint{3.261563in}{2.852664in}}%
\pgfpathlineto{\pgfqpoint{3.269448in}{2.841757in}}%
\pgfpathlineto{\pgfqpoint{3.271369in}{2.840885in}}%
\pgfpathlineto{\pgfqpoint{3.274705in}{2.835367in}}%
\pgfpathlineto{\pgfqpoint{3.278547in}{2.835094in}}%
\pgfpathlineto{\pgfqpoint{3.280265in}{2.835807in}}%
\pgfpathlineto{\pgfqpoint{3.281175in}{2.836498in}}%
\pgfpathlineto{\pgfqpoint{3.285623in}{2.848909in}}%
\pgfpathlineto{\pgfqpoint{3.286230in}{2.850424in}}%
\pgfpathlineto{\pgfqpoint{3.286937in}{2.849969in}}%
\pgfpathlineto{\pgfqpoint{3.288555in}{2.850321in}}%
\pgfpathlineto{\pgfqpoint{3.294014in}{2.850425in}}%
\pgfpathlineto{\pgfqpoint{3.297552in}{2.850204in}}%
\pgfpathlineto{\pgfqpoint{3.298866in}{2.849671in}}%
\pgfpathlineto{\pgfqpoint{3.300383in}{2.849224in}}%
\pgfpathlineto{\pgfqpoint{3.300484in}{2.849313in}}%
\pgfpathlineto{\pgfqpoint{3.303011in}{2.852446in}}%
\pgfpathlineto{\pgfqpoint{3.303416in}{2.851497in}}%
\pgfpathlineto{\pgfqpoint{3.306145in}{2.845346in}}%
\pgfpathlineto{\pgfqpoint{3.306246in}{2.845527in}}%
\pgfpathlineto{\pgfqpoint{3.307763in}{2.849820in}}%
\pgfpathlineto{\pgfqpoint{3.308571in}{2.849454in}}%
\pgfpathlineto{\pgfqpoint{3.311301in}{2.849548in}}%
\pgfpathlineto{\pgfqpoint{3.316659in}{2.861677in}}%
\pgfpathlineto{\pgfqpoint{3.319489in}{2.873214in}}%
\pgfpathlineto{\pgfqpoint{3.323331in}{2.899146in}}%
\pgfpathlineto{\pgfqpoint{3.327476in}{2.986549in}}%
\pgfpathlineto{\pgfqpoint{3.328183in}{3.014234in}}%
\pgfpathlineto{\pgfqpoint{3.330003in}{3.063794in}}%
\pgfpathlineto{\pgfqpoint{3.332935in}{3.097798in}}%
\pgfpathlineto{\pgfqpoint{3.333643in}{3.100901in}}%
\pgfpathlineto{\pgfqpoint{3.337585in}{3.138973in}}%
\pgfpathlineto{\pgfqpoint{3.337787in}{3.138737in}}%
\pgfpathlineto{\pgfqpoint{3.340214in}{3.135865in}}%
\pgfpathlineto{\pgfqpoint{3.340315in}{3.136011in}}%
\pgfpathlineto{\pgfqpoint{3.340820in}{3.141454in}}%
\pgfpathlineto{\pgfqpoint{3.343550in}{3.179723in}}%
\pgfpathlineto{\pgfqpoint{3.344156in}{3.176013in}}%
\pgfpathlineto{\pgfqpoint{3.348200in}{3.136485in}}%
\pgfpathlineto{\pgfqpoint{3.349110in}{3.107193in}}%
\pgfpathlineto{\pgfqpoint{3.349716in}{3.117152in}}%
\pgfpathlineto{\pgfqpoint{3.350323in}{3.126097in}}%
\pgfpathlineto{\pgfqpoint{3.350828in}{3.120107in}}%
\pgfpathlineto{\pgfqpoint{3.351435in}{3.114150in}}%
\pgfpathlineto{\pgfqpoint{3.351940in}{3.118468in}}%
\pgfpathlineto{\pgfqpoint{3.352547in}{3.121809in}}%
\pgfpathlineto{\pgfqpoint{3.353255in}{3.121343in}}%
\pgfpathlineto{\pgfqpoint{3.353760in}{3.120172in}}%
\pgfpathlineto{\pgfqpoint{3.355782in}{3.096562in}}%
\pgfpathlineto{\pgfqpoint{3.356995in}{3.088428in}}%
\pgfpathlineto{\pgfqpoint{3.357197in}{3.088930in}}%
\pgfpathlineto{\pgfqpoint{3.358107in}{3.103602in}}%
\pgfpathlineto{\pgfqpoint{3.360230in}{3.118149in}}%
\pgfpathlineto{\pgfqpoint{3.360432in}{3.118552in}}%
\pgfpathlineto{\pgfqpoint{3.360736in}{3.117447in}}%
\pgfpathlineto{\pgfqpoint{3.361443in}{3.099814in}}%
\pgfpathlineto{\pgfqpoint{3.363768in}{3.050014in}}%
\pgfpathlineto{\pgfqpoint{3.365487in}{3.038440in}}%
\pgfpathlineto{\pgfqpoint{3.367003in}{3.032525in}}%
\pgfpathlineto{\pgfqpoint{3.370036in}{2.990940in}}%
\pgfpathlineto{\pgfqpoint{3.371856in}{2.967965in}}%
\pgfpathlineto{\pgfqpoint{3.372564in}{2.962307in}}%
\pgfpathlineto{\pgfqpoint{3.373271in}{2.964288in}}%
\pgfpathlineto{\pgfqpoint{3.376708in}{2.968247in}}%
\pgfpathlineto{\pgfqpoint{3.377113in}{2.967036in}}%
\pgfpathlineto{\pgfqpoint{3.377820in}{2.953265in}}%
\pgfpathlineto{\pgfqpoint{3.379741in}{2.928051in}}%
\pgfpathlineto{\pgfqpoint{3.379943in}{2.927608in}}%
\pgfpathlineto{\pgfqpoint{3.380348in}{2.929647in}}%
\pgfpathlineto{\pgfqpoint{3.381460in}{2.935770in}}%
\pgfpathlineto{\pgfqpoint{3.381965in}{2.935061in}}%
\pgfpathlineto{\pgfqpoint{3.382268in}{2.935017in}}%
\pgfpathlineto{\pgfqpoint{3.382572in}{2.935990in}}%
\pgfpathlineto{\pgfqpoint{3.383279in}{2.938543in}}%
\pgfpathlineto{\pgfqpoint{3.384088in}{2.938261in}}%
\pgfpathlineto{\pgfqpoint{3.384594in}{2.938506in}}%
\pgfpathlineto{\pgfqpoint{3.384897in}{2.937830in}}%
\pgfpathlineto{\pgfqpoint{3.386615in}{2.935771in}}%
\pgfpathlineto{\pgfqpoint{3.387323in}{2.935999in}}%
\pgfpathlineto{\pgfqpoint{3.387525in}{2.936390in}}%
\pgfpathlineto{\pgfqpoint{3.389446in}{2.938270in}}%
\pgfpathlineto{\pgfqpoint{3.390154in}{2.936459in}}%
\pgfpathlineto{\pgfqpoint{3.393187in}{2.913730in}}%
\pgfpathlineto{\pgfqpoint{3.394602in}{2.909931in}}%
\pgfpathlineto{\pgfqpoint{3.400566in}{2.874297in}}%
\pgfpathlineto{\pgfqpoint{3.401274in}{2.866291in}}%
\pgfpathlineto{\pgfqpoint{3.401780in}{2.869776in}}%
\pgfpathlineto{\pgfqpoint{3.402689in}{2.877214in}}%
\pgfpathlineto{\pgfqpoint{3.403296in}{2.874907in}}%
\pgfpathlineto{\pgfqpoint{3.405116in}{2.872827in}}%
\pgfpathlineto{\pgfqpoint{3.405621in}{2.873809in}}%
\pgfpathlineto{\pgfqpoint{3.409968in}{2.889112in}}%
\pgfpathlineto{\pgfqpoint{3.411282in}{2.890018in}}%
\pgfpathlineto{\pgfqpoint{3.413304in}{2.898332in}}%
\pgfpathlineto{\pgfqpoint{3.413911in}{2.898127in}}%
\pgfpathlineto{\pgfqpoint{3.417954in}{2.906394in}}%
\pgfpathlineto{\pgfqpoint{3.421897in}{2.938588in}}%
\pgfpathlineto{\pgfqpoint{3.422301in}{2.936780in}}%
\pgfpathlineto{\pgfqpoint{3.423009in}{2.932250in}}%
\pgfpathlineto{\pgfqpoint{3.423515in}{2.935815in}}%
\pgfpathlineto{\pgfqpoint{3.425435in}{2.953084in}}%
\pgfpathlineto{\pgfqpoint{3.425840in}{2.952417in}}%
\pgfpathlineto{\pgfqpoint{3.427761in}{2.941698in}}%
\pgfpathlineto{\pgfqpoint{3.428468in}{2.946392in}}%
\pgfpathlineto{\pgfqpoint{3.429681in}{2.956531in}}%
\pgfpathlineto{\pgfqpoint{3.430288in}{2.956178in}}%
\pgfpathlineto{\pgfqpoint{3.431501in}{2.955502in}}%
\pgfpathlineto{\pgfqpoint{3.432714in}{2.949705in}}%
\pgfpathlineto{\pgfqpoint{3.433220in}{2.951417in}}%
\pgfpathlineto{\pgfqpoint{3.434129in}{2.957789in}}%
\pgfpathlineto{\pgfqpoint{3.434736in}{2.955702in}}%
\pgfpathlineto{\pgfqpoint{3.435039in}{2.954883in}}%
\pgfpathlineto{\pgfqpoint{3.435545in}{2.956565in}}%
\pgfpathlineto{\pgfqpoint{3.437364in}{2.969717in}}%
\pgfpathlineto{\pgfqpoint{3.438072in}{2.967061in}}%
\pgfpathlineto{\pgfqpoint{3.440397in}{2.957472in}}%
\pgfpathlineto{\pgfqpoint{3.441408in}{2.950429in}}%
\pgfpathlineto{\pgfqpoint{3.445149in}{2.918765in}}%
\pgfpathlineto{\pgfqpoint{3.445351in}{2.918573in}}%
\pgfpathlineto{\pgfqpoint{3.445856in}{2.919707in}}%
\pgfpathlineto{\pgfqpoint{3.447777in}{2.922721in}}%
\pgfpathlineto{\pgfqpoint{3.448080in}{2.922741in}}%
\pgfpathlineto{\pgfqpoint{3.448282in}{2.922184in}}%
\pgfpathlineto{\pgfqpoint{3.449293in}{2.911722in}}%
\pgfpathlineto{\pgfqpoint{3.450810in}{2.906948in}}%
\pgfpathlineto{\pgfqpoint{3.453135in}{2.903535in}}%
\pgfpathlineto{\pgfqpoint{3.458291in}{2.869582in}}%
\pgfpathlineto{\pgfqpoint{3.458392in}{2.869597in}}%
\pgfpathlineto{\pgfqpoint{3.459706in}{2.870977in}}%
\pgfpathlineto{\pgfqpoint{3.460110in}{2.870857in}}%
\pgfpathlineto{\pgfqpoint{3.460313in}{2.870346in}}%
\pgfpathlineto{\pgfqpoint{3.461526in}{2.861943in}}%
\pgfpathlineto{\pgfqpoint{3.463244in}{2.856643in}}%
\pgfpathlineto{\pgfqpoint{3.464862in}{2.854018in}}%
\pgfpathlineto{\pgfqpoint{3.465974in}{2.851441in}}%
\pgfpathlineto{\pgfqpoint{3.466985in}{2.847815in}}%
\pgfpathlineto{\pgfqpoint{3.467692in}{2.848390in}}%
\pgfpathlineto{\pgfqpoint{3.468299in}{2.846884in}}%
\pgfpathlineto{\pgfqpoint{3.470624in}{2.841186in}}%
\pgfpathlineto{\pgfqpoint{3.471534in}{2.841969in}}%
\pgfpathlineto{\pgfqpoint{3.473253in}{2.842603in}}%
\pgfpathlineto{\pgfqpoint{3.478509in}{2.844842in}}%
\pgfpathlineto{\pgfqpoint{3.479925in}{2.848473in}}%
\pgfpathlineto{\pgfqpoint{3.480632in}{2.847996in}}%
\pgfpathlineto{\pgfqpoint{3.481542in}{2.849063in}}%
\pgfpathlineto{\pgfqpoint{3.483766in}{2.851385in}}%
\pgfpathlineto{\pgfqpoint{3.485788in}{2.851054in}}%
\pgfpathlineto{\pgfqpoint{3.486900in}{2.852304in}}%
\pgfpathlineto{\pgfqpoint{3.489326in}{2.854744in}}%
\pgfpathlineto{\pgfqpoint{3.492056in}{2.854674in}}%
\pgfpathlineto{\pgfqpoint{3.493775in}{2.853705in}}%
\pgfpathlineto{\pgfqpoint{3.496807in}{2.852632in}}%
\pgfpathlineto{\pgfqpoint{3.498930in}{2.850784in}}%
\pgfpathlineto{\pgfqpoint{3.499941in}{2.847794in}}%
\pgfpathlineto{\pgfqpoint{3.500346in}{2.849039in}}%
\pgfpathlineto{\pgfqpoint{3.501660in}{2.854623in}}%
\pgfpathlineto{\pgfqpoint{3.502266in}{2.853795in}}%
\pgfpathlineto{\pgfqpoint{3.502671in}{2.853720in}}%
\pgfpathlineto{\pgfqpoint{3.502974in}{2.854328in}}%
\pgfpathlineto{\pgfqpoint{3.504592in}{2.858608in}}%
\pgfpathlineto{\pgfqpoint{3.505299in}{2.858322in}}%
\pgfpathlineto{\pgfqpoint{3.507018in}{2.856998in}}%
\pgfpathlineto{\pgfqpoint{3.510556in}{2.851105in}}%
\pgfpathlineto{\pgfqpoint{3.512275in}{2.850644in}}%
\pgfpathlineto{\pgfqpoint{3.514903in}{2.850692in}}%
\pgfpathlineto{\pgfqpoint{3.517633in}{2.849987in}}%
\pgfpathlineto{\pgfqpoint{3.519857in}{2.849133in}}%
\pgfpathlineto{\pgfqpoint{3.520665in}{2.846132in}}%
\pgfpathlineto{\pgfqpoint{3.522889in}{2.838943in}}%
\pgfpathlineto{\pgfqpoint{3.526327in}{2.833300in}}%
\pgfpathlineto{\pgfqpoint{3.532898in}{2.834800in}}%
\pgfpathlineto{\pgfqpoint{3.535829in}{2.835204in}}%
\pgfpathlineto{\pgfqpoint{3.537649in}{2.837085in}}%
\pgfpathlineto{\pgfqpoint{3.539469in}{2.837693in}}%
\pgfpathlineto{\pgfqpoint{3.545635in}{2.838828in}}%
\pgfpathlineto{\pgfqpoint{3.547354in}{2.839366in}}%
\pgfpathlineto{\pgfqpoint{3.552308in}{2.838224in}}%
\pgfpathlineto{\pgfqpoint{3.554734in}{2.839874in}}%
\pgfpathlineto{\pgfqpoint{3.557564in}{2.838977in}}%
\pgfpathlineto{\pgfqpoint{3.559485in}{2.837989in}}%
\pgfpathlineto{\pgfqpoint{3.562114in}{2.837902in}}%
\pgfpathlineto{\pgfqpoint{3.564439in}{2.837448in}}%
\pgfpathlineto{\pgfqpoint{3.566056in}{2.836385in}}%
\pgfpathlineto{\pgfqpoint{3.568179in}{2.835359in}}%
\pgfpathlineto{\pgfqpoint{3.571819in}{2.835608in}}%
\pgfpathlineto{\pgfqpoint{3.573032in}{2.834341in}}%
\pgfpathlineto{\pgfqpoint{3.575559in}{2.832825in}}%
\pgfpathlineto{\pgfqpoint{3.577177in}{2.833389in}}%
\pgfpathlineto{\pgfqpoint{3.581220in}{2.835744in}}%
\pgfpathlineto{\pgfqpoint{3.583242in}{2.839037in}}%
\pgfpathlineto{\pgfqpoint{3.586275in}{2.843406in}}%
\pgfpathlineto{\pgfqpoint{3.588297in}{2.844522in}}%
\pgfpathlineto{\pgfqpoint{3.589207in}{2.846555in}}%
\pgfpathlineto{\pgfqpoint{3.592138in}{2.853974in}}%
\pgfpathlineto{\pgfqpoint{3.593250in}{2.855184in}}%
\pgfpathlineto{\pgfqpoint{3.595980in}{2.863527in}}%
\pgfpathlineto{\pgfqpoint{3.597496in}{2.862874in}}%
\pgfpathlineto{\pgfqpoint{3.598406in}{2.864064in}}%
\pgfpathlineto{\pgfqpoint{3.600327in}{2.867617in}}%
\pgfpathlineto{\pgfqpoint{3.600731in}{2.867124in}}%
\pgfpathlineto{\pgfqpoint{3.603158in}{2.862188in}}%
\pgfpathlineto{\pgfqpoint{3.603865in}{2.862675in}}%
\pgfpathlineto{\pgfqpoint{3.605382in}{2.862542in}}%
\pgfpathlineto{\pgfqpoint{3.608516in}{2.860042in}}%
\pgfpathlineto{\pgfqpoint{3.609324in}{2.859172in}}%
\pgfpathlineto{\pgfqpoint{3.612054in}{2.851831in}}%
\pgfpathlineto{\pgfqpoint{3.612863in}{2.852119in}}%
\pgfpathlineto{\pgfqpoint{3.613671in}{2.850819in}}%
\pgfpathlineto{\pgfqpoint{3.617210in}{2.842453in}}%
\pgfpathlineto{\pgfqpoint{3.618423in}{2.841191in}}%
\pgfpathlineto{\pgfqpoint{3.621354in}{2.836751in}}%
\pgfpathlineto{\pgfqpoint{3.624792in}{2.835759in}}%
\pgfpathlineto{\pgfqpoint{3.626005in}{2.835326in}}%
\pgfpathlineto{\pgfqpoint{3.626207in}{2.835485in}}%
\pgfpathlineto{\pgfqpoint{3.628330in}{2.835854in}}%
\pgfpathlineto{\pgfqpoint{3.636518in}{2.832534in}}%
\pgfpathlineto{\pgfqpoint{3.644202in}{2.831648in}}%
\pgfpathlineto{\pgfqpoint{3.647538in}{2.831074in}}%
\pgfpathlineto{\pgfqpoint{3.650166in}{2.831913in}}%
\pgfpathlineto{\pgfqpoint{3.652795in}{2.832572in}}%
\pgfpathlineto{\pgfqpoint{3.655827in}{2.832857in}}%
\pgfpathlineto{\pgfqpoint{3.664016in}{2.833107in}}%
\pgfpathlineto{\pgfqpoint{3.672609in}{2.830099in}}%
\pgfpathlineto{\pgfqpoint{3.675439in}{2.829705in}}%
\pgfpathlineto{\pgfqpoint{3.692828in}{2.829272in}}%
\pgfpathlineto{\pgfqpoint{3.700005in}{2.830008in}}%
\pgfpathlineto{\pgfqpoint{3.717292in}{2.829973in}}%
\pgfpathlineto{\pgfqpoint{3.722246in}{2.829552in}}%
\pgfpathlineto{\pgfqpoint{3.734983in}{2.829532in}}%
\pgfpathlineto{\pgfqpoint{3.761773in}{2.829456in}}%
\pgfpathlineto{\pgfqpoint{3.785732in}{2.830206in}}%
\pgfpathlineto{\pgfqpoint{3.800795in}{2.829738in}}%
\pgfpathlineto{\pgfqpoint{3.810399in}{2.829735in}}%
\pgfpathlineto{\pgfqpoint{3.814645in}{2.830111in}}%
\pgfpathlineto{\pgfqpoint{3.843962in}{2.829550in}}%
\pgfpathlineto{\pgfqpoint{3.848815in}{2.829170in}}%
\pgfpathlineto{\pgfqpoint{3.854274in}{2.829976in}}%
\pgfpathlineto{\pgfqpoint{3.856599in}{2.830595in}}%
\pgfpathlineto{\pgfqpoint{3.862867in}{2.831268in}}%
\pgfpathlineto{\pgfqpoint{3.866506in}{2.831535in}}%
\pgfpathlineto{\pgfqpoint{3.873684in}{2.832327in}}%
\pgfpathlineto{\pgfqpoint{3.875402in}{2.832392in}}%
\pgfpathlineto{\pgfqpoint{3.878031in}{2.833257in}}%
\pgfpathlineto{\pgfqpoint{3.894307in}{2.833638in}}%
\pgfpathlineto{\pgfqpoint{3.896025in}{2.833320in}}%
\pgfpathlineto{\pgfqpoint{3.899968in}{2.833933in}}%
\pgfpathlineto{\pgfqpoint{3.910987in}{2.833913in}}%
\pgfpathlineto{\pgfqpoint{3.913110in}{2.832593in}}%
\pgfpathlineto{\pgfqpoint{3.914930in}{2.833503in}}%
\pgfpathlineto{\pgfqpoint{3.915941in}{2.834107in}}%
\pgfpathlineto{\pgfqpoint{3.916345in}{2.833767in}}%
\pgfpathlineto{\pgfqpoint{3.918367in}{2.833412in}}%
\pgfpathlineto{\pgfqpoint{3.926454in}{2.834185in}}%
\pgfpathlineto{\pgfqpoint{3.928678in}{2.835072in}}%
\pgfpathlineto{\pgfqpoint{3.933834in}{2.836540in}}%
\pgfpathlineto{\pgfqpoint{3.935250in}{2.835996in}}%
\pgfpathlineto{\pgfqpoint{3.937069in}{2.835164in}}%
\pgfpathlineto{\pgfqpoint{3.946572in}{2.833836in}}%
\pgfpathlineto{\pgfqpoint{3.948594in}{2.832808in}}%
\pgfpathlineto{\pgfqpoint{3.950211in}{2.831311in}}%
\pgfpathlineto{\pgfqpoint{3.952334in}{2.831054in}}%
\pgfpathlineto{\pgfqpoint{3.955064in}{2.829958in}}%
\pgfpathlineto{\pgfqpoint{3.957086in}{2.829697in}}%
\pgfpathlineto{\pgfqpoint{3.965375in}{2.830516in}}%
\pgfpathlineto{\pgfqpoint{3.976192in}{2.829858in}}%
\pgfpathlineto{\pgfqpoint{3.979933in}{2.830320in}}%
\pgfpathlineto{\pgfqpoint{3.981651in}{2.831043in}}%
\pgfpathlineto{\pgfqpoint{3.983572in}{2.832017in}}%
\pgfpathlineto{\pgfqpoint{3.986605in}{2.832715in}}%
\pgfpathlineto{\pgfqpoint{3.991963in}{2.833639in}}%
\pgfpathlineto{\pgfqpoint{4.001971in}{2.832549in}}%
\pgfpathlineto{\pgfqpoint{4.003589in}{2.831170in}}%
\pgfpathlineto{\pgfqpoint{4.006723in}{2.830728in}}%
\pgfpathlineto{\pgfqpoint{4.008138in}{2.830943in}}%
\pgfpathlineto{\pgfqpoint{4.011070in}{2.832065in}}%
\pgfpathlineto{\pgfqpoint{4.016225in}{2.833025in}}%
\pgfpathlineto{\pgfqpoint{4.022190in}{2.833929in}}%
\pgfpathlineto{\pgfqpoint{4.023302in}{2.835220in}}%
\pgfpathlineto{\pgfqpoint{4.025021in}{2.836207in}}%
\pgfpathlineto{\pgfqpoint{4.028256in}{2.836046in}}%
\pgfpathlineto{\pgfqpoint{4.030682in}{2.836596in}}%
\pgfpathlineto{\pgfqpoint{4.034726in}{2.837432in}}%
\pgfpathlineto{\pgfqpoint{4.037859in}{2.837656in}}%
\pgfpathlineto{\pgfqpoint{4.040589in}{2.836599in}}%
\pgfpathlineto{\pgfqpoint{4.042105in}{2.836510in}}%
\pgfpathlineto{\pgfqpoint{4.043217in}{2.835818in}}%
\pgfpathlineto{\pgfqpoint{4.045340in}{2.833800in}}%
\pgfpathlineto{\pgfqpoint{4.048777in}{2.833035in}}%
\pgfpathlineto{\pgfqpoint{4.050496in}{2.832290in}}%
\pgfpathlineto{\pgfqpoint{4.053125in}{2.832070in}}%
\pgfpathlineto{\pgfqpoint{4.056764in}{2.832240in}}%
\pgfpathlineto{\pgfqpoint{4.057977in}{2.833328in}}%
\pgfpathlineto{\pgfqpoint{4.060302in}{2.834895in}}%
\pgfpathlineto{\pgfqpoint{4.071928in}{2.836183in}}%
\pgfpathlineto{\pgfqpoint{4.077387in}{2.834702in}}%
\pgfpathlineto{\pgfqpoint{4.081330in}{2.832537in}}%
\pgfpathlineto{\pgfqpoint{4.081431in}{2.832670in}}%
\pgfpathlineto{\pgfqpoint{4.083453in}{2.833907in}}%
\pgfpathlineto{\pgfqpoint{4.085778in}{2.834204in}}%
\pgfpathlineto{\pgfqpoint{4.088002in}{2.835265in}}%
\pgfpathlineto{\pgfqpoint{4.089923in}{2.835514in}}%
\pgfpathlineto{\pgfqpoint{4.092045in}{2.836794in}}%
\pgfpathlineto{\pgfqpoint{4.093966in}{2.838446in}}%
\pgfpathlineto{\pgfqpoint{4.095988in}{2.840759in}}%
\pgfpathlineto{\pgfqpoint{4.099021in}{2.842216in}}%
\pgfpathlineto{\pgfqpoint{4.103065in}{2.848943in}}%
\pgfpathlineto{\pgfqpoint{4.105289in}{2.850456in}}%
\pgfpathlineto{\pgfqpoint{4.112669in}{2.846547in}}%
\pgfpathlineto{\pgfqpoint{4.114084in}{2.845204in}}%
\pgfpathlineto{\pgfqpoint{4.116005in}{2.843189in}}%
\pgfpathlineto{\pgfqpoint{4.119745in}{2.842023in}}%
\pgfpathlineto{\pgfqpoint{4.121363in}{2.836910in}}%
\pgfpathlineto{\pgfqpoint{4.121868in}{2.837690in}}%
\pgfpathlineto{\pgfqpoint{4.122980in}{2.839393in}}%
\pgfpathlineto{\pgfqpoint{4.123486in}{2.838963in}}%
\pgfpathlineto{\pgfqpoint{4.125002in}{2.838951in}}%
\pgfpathlineto{\pgfqpoint{4.132786in}{2.842757in}}%
\pgfpathlineto{\pgfqpoint{4.138346in}{2.842600in}}%
\pgfpathlineto{\pgfqpoint{4.142188in}{2.839637in}}%
\pgfpathlineto{\pgfqpoint{4.144311in}{2.835761in}}%
\pgfpathlineto{\pgfqpoint{4.145625in}{2.835984in}}%
\pgfpathlineto{\pgfqpoint{4.147141in}{2.836403in}}%
\pgfpathlineto{\pgfqpoint{4.147344in}{2.836186in}}%
\pgfpathlineto{\pgfqpoint{4.150579in}{2.833894in}}%
\pgfpathlineto{\pgfqpoint{4.165743in}{2.841716in}}%
\pgfpathlineto{\pgfqpoint{4.167663in}{2.841176in}}%
\pgfpathlineto{\pgfqpoint{4.170292in}{2.841836in}}%
\pgfpathlineto{\pgfqpoint{4.173426in}{2.841070in}}%
\pgfpathlineto{\pgfqpoint{4.174942in}{2.840306in}}%
\pgfpathlineto{\pgfqpoint{4.177672in}{2.838742in}}%
\pgfpathlineto{\pgfqpoint{4.179087in}{2.838315in}}%
\pgfpathlineto{\pgfqpoint{4.181412in}{2.836475in}}%
\pgfpathlineto{\pgfqpoint{4.183131in}{2.835998in}}%
\pgfpathlineto{\pgfqpoint{4.184950in}{2.835897in}}%
\pgfpathlineto{\pgfqpoint{4.186770in}{2.836118in}}%
\pgfpathlineto{\pgfqpoint{4.193139in}{2.838094in}}%
\pgfpathlineto{\pgfqpoint{4.195565in}{2.837738in}}%
\pgfpathlineto{\pgfqpoint{4.198194in}{2.837600in}}%
\pgfpathlineto{\pgfqpoint{4.202035in}{2.837712in}}%
\pgfpathlineto{\pgfqpoint{4.206888in}{2.836392in}}%
\pgfpathlineto{\pgfqpoint{4.209011in}{2.837031in}}%
\pgfpathlineto{\pgfqpoint{4.212043in}{2.836286in}}%
\pgfpathlineto{\pgfqpoint{4.213964in}{2.835958in}}%
\pgfpathlineto{\pgfqpoint{4.218716in}{2.838114in}}%
\pgfpathlineto{\pgfqpoint{4.221951in}{2.838686in}}%
\pgfpathlineto{\pgfqpoint{4.225287in}{2.840451in}}%
\pgfpathlineto{\pgfqpoint{4.228218in}{2.846608in}}%
\pgfpathlineto{\pgfqpoint{4.229634in}{2.847707in}}%
\pgfpathlineto{\pgfqpoint{4.240451in}{2.853967in}}%
\pgfpathlineto{\pgfqpoint{4.243585in}{2.864332in}}%
\pgfpathlineto{\pgfqpoint{4.245404in}{2.864282in}}%
\pgfpathlineto{\pgfqpoint{4.248033in}{2.858819in}}%
\pgfpathlineto{\pgfqpoint{4.249549in}{2.856464in}}%
\pgfpathlineto{\pgfqpoint{4.253896in}{2.854689in}}%
\pgfpathlineto{\pgfqpoint{4.255008in}{2.853844in}}%
\pgfpathlineto{\pgfqpoint{4.255817in}{2.853211in}}%
\pgfpathlineto{\pgfqpoint{4.256322in}{2.853676in}}%
\pgfpathlineto{\pgfqpoint{4.257839in}{2.854105in}}%
\pgfpathlineto{\pgfqpoint{4.257940in}{2.853993in}}%
\pgfpathlineto{\pgfqpoint{4.259962in}{2.853202in}}%
\pgfpathlineto{\pgfqpoint{4.260669in}{2.851640in}}%
\pgfpathlineto{\pgfqpoint{4.264005in}{2.839670in}}%
\pgfpathlineto{\pgfqpoint{4.265623in}{2.839592in}}%
\pgfpathlineto{\pgfqpoint{4.272194in}{2.842454in}}%
\pgfpathlineto{\pgfqpoint{4.276541in}{2.839122in}}%
\pgfpathlineto{\pgfqpoint{4.278563in}{2.839104in}}%
\pgfpathlineto{\pgfqpoint{4.282910in}{2.840280in}}%
\pgfpathlineto{\pgfqpoint{4.284527in}{2.840103in}}%
\pgfpathlineto{\pgfqpoint{4.287055in}{2.838679in}}%
\pgfpathlineto{\pgfqpoint{4.289380in}{2.837031in}}%
\pgfpathlineto{\pgfqpoint{4.291907in}{2.834008in}}%
\pgfpathlineto{\pgfqpoint{4.294839in}{2.834179in}}%
\pgfpathlineto{\pgfqpoint{4.296558in}{2.833779in}}%
\pgfpathlineto{\pgfqpoint{4.298377in}{2.833604in}}%
\pgfpathlineto{\pgfqpoint{4.302118in}{2.832896in}}%
\pgfpathlineto{\pgfqpoint{4.304140in}{2.832647in}}%
\pgfpathlineto{\pgfqpoint{4.306465in}{2.832556in}}%
\pgfpathlineto{\pgfqpoint{4.310306in}{2.834657in}}%
\pgfpathlineto{\pgfqpoint{4.313642in}{2.842147in}}%
\pgfpathlineto{\pgfqpoint{4.316473in}{2.842411in}}%
\pgfpathlineto{\pgfqpoint{4.320618in}{2.842973in}}%
\pgfpathlineto{\pgfqpoint{4.324257in}{2.841799in}}%
\pgfpathlineto{\pgfqpoint{4.327795in}{2.845068in}}%
\pgfpathlineto{\pgfqpoint{4.329110in}{2.847584in}}%
\pgfpathlineto{\pgfqpoint{4.329817in}{2.847179in}}%
\pgfpathlineto{\pgfqpoint{4.330727in}{2.845796in}}%
\pgfpathlineto{\pgfqpoint{4.332142in}{2.841822in}}%
\pgfpathlineto{\pgfqpoint{4.332749in}{2.842241in}}%
\pgfpathlineto{\pgfqpoint{4.336388in}{2.843175in}}%
\pgfpathlineto{\pgfqpoint{4.339017in}{2.843201in}}%
\pgfpathlineto{\pgfqpoint{4.343768in}{2.844224in}}%
\pgfpathlineto{\pgfqpoint{4.344981in}{2.845055in}}%
\pgfpathlineto{\pgfqpoint{4.348520in}{2.856270in}}%
\pgfpathlineto{\pgfqpoint{4.349126in}{2.855469in}}%
\pgfpathlineto{\pgfqpoint{4.351350in}{2.854176in}}%
\pgfpathlineto{\pgfqpoint{4.354787in}{2.851678in}}%
\pgfpathlineto{\pgfqpoint{4.363987in}{2.857524in}}%
\pgfpathlineto{\pgfqpoint{4.364796in}{2.858484in}}%
\pgfpathlineto{\pgfqpoint{4.365200in}{2.857559in}}%
\pgfpathlineto{\pgfqpoint{4.369345in}{2.842627in}}%
\pgfpathlineto{\pgfqpoint{4.371569in}{2.844045in}}%
\pgfpathlineto{\pgfqpoint{4.375006in}{2.847814in}}%
\pgfpathlineto{\pgfqpoint{4.381880in}{2.845002in}}%
\pgfpathlineto{\pgfqpoint{4.384913in}{2.842565in}}%
\pgfpathlineto{\pgfqpoint{4.385621in}{2.842362in}}%
\pgfpathlineto{\pgfqpoint{4.386025in}{2.842973in}}%
\pgfpathlineto{\pgfqpoint{4.386632in}{2.843513in}}%
\pgfpathlineto{\pgfqpoint{4.387036in}{2.842767in}}%
\pgfpathlineto{\pgfqpoint{4.389058in}{2.840478in}}%
\pgfpathlineto{\pgfqpoint{4.392293in}{2.838425in}}%
\pgfpathlineto{\pgfqpoint{4.395326in}{2.835292in}}%
\pgfpathlineto{\pgfqpoint{4.395427in}{2.835383in}}%
\pgfpathlineto{\pgfqpoint{4.398055in}{2.837058in}}%
\pgfpathlineto{\pgfqpoint{4.405536in}{2.839284in}}%
\pgfpathlineto{\pgfqpoint{4.406143in}{2.839463in}}%
\pgfpathlineto{\pgfqpoint{4.406648in}{2.838912in}}%
\pgfpathlineto{\pgfqpoint{4.408165in}{2.839002in}}%
\pgfpathlineto{\pgfqpoint{4.412309in}{2.841711in}}%
\pgfpathlineto{\pgfqpoint{4.417465in}{2.851397in}}%
\pgfpathlineto{\pgfqpoint{4.422924in}{2.856249in}}%
\pgfpathlineto{\pgfqpoint{4.424643in}{2.858644in}}%
\pgfpathlineto{\pgfqpoint{4.426159in}{2.858623in}}%
\pgfpathlineto{\pgfqpoint{4.427474in}{2.859171in}}%
\pgfpathlineto{\pgfqpoint{4.428687in}{2.860900in}}%
\pgfpathlineto{\pgfqpoint{4.431214in}{2.865153in}}%
\pgfpathlineto{\pgfqpoint{4.432124in}{2.865538in}}%
\pgfpathlineto{\pgfqpoint{4.432427in}{2.865086in}}%
\pgfpathlineto{\pgfqpoint{4.435460in}{2.857263in}}%
\pgfpathlineto{\pgfqpoint{4.437178in}{2.854689in}}%
\pgfpathlineto{\pgfqpoint{4.446782in}{2.847711in}}%
\pgfpathlineto{\pgfqpoint{4.449209in}{2.845540in}}%
\pgfpathlineto{\pgfqpoint{4.450725in}{2.842693in}}%
\pgfpathlineto{\pgfqpoint{4.451028in}{2.842823in}}%
\pgfpathlineto{\pgfqpoint{4.451938in}{2.841965in}}%
\pgfpathlineto{\pgfqpoint{4.454061in}{2.839303in}}%
\pgfpathlineto{\pgfqpoint{4.454263in}{2.839438in}}%
\pgfpathlineto{\pgfqpoint{4.455982in}{2.839580in}}%
\pgfpathlineto{\pgfqpoint{4.457195in}{2.839624in}}%
\pgfpathlineto{\pgfqpoint{4.457296in}{2.839800in}}%
\pgfpathlineto{\pgfqpoint{4.461542in}{2.846475in}}%
\pgfpathlineto{\pgfqpoint{4.466192in}{2.846017in}}%
\pgfpathlineto{\pgfqpoint{4.468517in}{2.844287in}}%
\pgfpathlineto{\pgfqpoint{4.469731in}{2.845170in}}%
\pgfpathlineto{\pgfqpoint{4.472561in}{2.847364in}}%
\pgfpathlineto{\pgfqpoint{4.474583in}{2.847989in}}%
\pgfpathlineto{\pgfqpoint{4.474684in}{2.847864in}}%
\pgfpathlineto{\pgfqpoint{4.476807in}{2.846807in}}%
\pgfpathlineto{\pgfqpoint{4.477717in}{2.845930in}}%
\pgfpathlineto{\pgfqpoint{4.479334in}{2.842935in}}%
\pgfpathlineto{\pgfqpoint{4.479739in}{2.843249in}}%
\pgfpathlineto{\pgfqpoint{4.482772in}{2.849583in}}%
\pgfpathlineto{\pgfqpoint{4.483479in}{2.851511in}}%
\pgfpathlineto{\pgfqpoint{4.484086in}{2.850541in}}%
\pgfpathlineto{\pgfqpoint{4.485400in}{2.850396in}}%
\pgfpathlineto{\pgfqpoint{4.487422in}{2.851820in}}%
\pgfpathlineto{\pgfqpoint{4.489039in}{2.854191in}}%
\pgfpathlineto{\pgfqpoint{4.489444in}{2.853921in}}%
\pgfpathlineto{\pgfqpoint{4.494903in}{2.850146in}}%
\pgfpathlineto{\pgfqpoint{4.495004in}{2.850263in}}%
\pgfpathlineto{\pgfqpoint{4.497127in}{2.851152in}}%
\pgfpathlineto{\pgfqpoint{4.500362in}{2.850111in}}%
\pgfpathlineto{\pgfqpoint{4.503698in}{2.840711in}}%
\pgfpathlineto{\pgfqpoint{4.503900in}{2.840910in}}%
\pgfpathlineto{\pgfqpoint{4.506124in}{2.842744in}}%
\pgfpathlineto{\pgfqpoint{4.507236in}{2.843563in}}%
\pgfpathlineto{\pgfqpoint{4.507641in}{2.843854in}}%
\pgfpathlineto{\pgfqpoint{4.508045in}{2.843165in}}%
\pgfpathlineto{\pgfqpoint{4.508652in}{2.842263in}}%
\pgfpathlineto{\pgfqpoint{4.509157in}{2.843067in}}%
\pgfpathlineto{\pgfqpoint{4.510471in}{2.844968in}}%
\pgfpathlineto{\pgfqpoint{4.510876in}{2.844703in}}%
\pgfpathlineto{\pgfqpoint{4.512493in}{2.844920in}}%
\pgfpathlineto{\pgfqpoint{4.515122in}{2.844983in}}%
\pgfpathlineto{\pgfqpoint{4.517447in}{2.843796in}}%
\pgfpathlineto{\pgfqpoint{4.523917in}{2.848167in}}%
\pgfpathlineto{\pgfqpoint{4.524321in}{2.847417in}}%
\pgfpathlineto{\pgfqpoint{4.525736in}{2.845441in}}%
\pgfpathlineto{\pgfqpoint{4.526141in}{2.845919in}}%
\pgfpathlineto{\pgfqpoint{4.527354in}{2.849112in}}%
\pgfpathlineto{\pgfqpoint{4.528061in}{2.848067in}}%
\pgfpathlineto{\pgfqpoint{4.532712in}{2.844543in}}%
\pgfpathlineto{\pgfqpoint{4.534734in}{2.846191in}}%
\pgfpathlineto{\pgfqpoint{4.536452in}{2.847862in}}%
\pgfpathlineto{\pgfqpoint{4.538171in}{2.847168in}}%
\pgfpathlineto{\pgfqpoint{4.539788in}{2.846391in}}%
\pgfpathlineto{\pgfqpoint{4.542922in}{2.844295in}}%
\pgfpathlineto{\pgfqpoint{4.544135in}{2.842098in}}%
\pgfpathlineto{\pgfqpoint{4.544641in}{2.842404in}}%
\pgfpathlineto{\pgfqpoint{4.546359in}{2.842696in}}%
\pgfpathlineto{\pgfqpoint{4.546460in}{2.842502in}}%
\pgfpathlineto{\pgfqpoint{4.548685in}{2.838503in}}%
\pgfpathlineto{\pgfqpoint{4.548988in}{2.838733in}}%
\pgfpathlineto{\pgfqpoint{4.554346in}{2.844334in}}%
\pgfpathlineto{\pgfqpoint{4.558288in}{2.841853in}}%
\pgfpathlineto{\pgfqpoint{4.561018in}{2.840489in}}%
\pgfpathlineto{\pgfqpoint{4.563545in}{2.840597in}}%
\pgfpathlineto{\pgfqpoint{4.565062in}{2.840047in}}%
\pgfpathlineto{\pgfqpoint{4.568802in}{2.839584in}}%
\pgfpathlineto{\pgfqpoint{4.569510in}{2.839328in}}%
\pgfpathlineto{\pgfqpoint{4.569712in}{2.838853in}}%
\pgfpathlineto{\pgfqpoint{4.572441in}{2.834288in}}%
\pgfpathlineto{\pgfqpoint{4.574160in}{2.832946in}}%
\pgfpathlineto{\pgfqpoint{4.574463in}{2.833189in}}%
\pgfpathlineto{\pgfqpoint{4.576890in}{2.833732in}}%
\pgfpathlineto{\pgfqpoint{4.578103in}{2.833964in}}%
\pgfpathlineto{\pgfqpoint{4.578204in}{2.834129in}}%
\pgfpathlineto{\pgfqpoint{4.580428in}{2.835891in}}%
\pgfpathlineto{\pgfqpoint{4.582551in}{2.836804in}}%
\pgfpathlineto{\pgfqpoint{4.589324in}{2.845699in}}%
\pgfpathlineto{\pgfqpoint{4.591144in}{2.845761in}}%
\pgfpathlineto{\pgfqpoint{4.595996in}{2.846753in}}%
\pgfpathlineto{\pgfqpoint{4.600748in}{2.845555in}}%
\pgfpathlineto{\pgfqpoint{4.601759in}{2.850131in}}%
\pgfpathlineto{\pgfqpoint{4.602871in}{2.854651in}}%
\pgfpathlineto{\pgfqpoint{4.603376in}{2.854024in}}%
\pgfpathlineto{\pgfqpoint{4.604084in}{2.853216in}}%
\pgfpathlineto{\pgfqpoint{4.604488in}{2.853832in}}%
\pgfpathlineto{\pgfqpoint{4.606914in}{2.861672in}}%
\pgfpathlineto{\pgfqpoint{4.608026in}{2.859900in}}%
\pgfpathlineto{\pgfqpoint{4.608835in}{2.860134in}}%
\pgfpathlineto{\pgfqpoint{4.609037in}{2.860508in}}%
\pgfpathlineto{\pgfqpoint{4.612980in}{2.867227in}}%
\pgfpathlineto{\pgfqpoint{4.615709in}{2.869656in}}%
\pgfpathlineto{\pgfqpoint{4.618338in}{2.870844in}}%
\pgfpathlineto{\pgfqpoint{4.619450in}{2.871751in}}%
\pgfpathlineto{\pgfqpoint{4.619955in}{2.871396in}}%
\pgfpathlineto{\pgfqpoint{4.621472in}{2.869777in}}%
\pgfpathlineto{\pgfqpoint{4.627032in}{2.848870in}}%
\pgfpathlineto{\pgfqpoint{4.633906in}{2.836270in}}%
\pgfpathlineto{\pgfqpoint{4.635524in}{2.835826in}}%
\pgfpathlineto{\pgfqpoint{4.635625in}{2.835910in}}%
\pgfpathlineto{\pgfqpoint{4.637950in}{2.836447in}}%
\pgfpathlineto{\pgfqpoint{4.644622in}{2.834959in}}%
\pgfpathlineto{\pgfqpoint{4.647453in}{2.838090in}}%
\pgfpathlineto{\pgfqpoint{4.650081in}{2.845083in}}%
\pgfpathlineto{\pgfqpoint{4.653114in}{2.848202in}}%
\pgfpathlineto{\pgfqpoint{4.658270in}{2.847241in}}%
\pgfpathlineto{\pgfqpoint{4.661505in}{2.847954in}}%
\pgfpathlineto{\pgfqpoint{4.663223in}{2.847867in}}%
\pgfpathlineto{\pgfqpoint{4.667672in}{2.844628in}}%
\pgfpathlineto{\pgfqpoint{4.670603in}{2.835901in}}%
\pgfpathlineto{\pgfqpoint{4.672625in}{2.833565in}}%
\pgfpathlineto{\pgfqpoint{4.675254in}{2.834993in}}%
\pgfpathlineto{\pgfqpoint{4.678590in}{2.836496in}}%
\pgfpathlineto{\pgfqpoint{4.684554in}{2.835659in}}%
\pgfpathlineto{\pgfqpoint{4.687991in}{2.836570in}}%
\pgfpathlineto{\pgfqpoint{4.696786in}{2.836016in}}%
\pgfpathlineto{\pgfqpoint{4.698909in}{2.834974in}}%
\pgfpathlineto{\pgfqpoint{4.710232in}{2.836176in}}%
\pgfpathlineto{\pgfqpoint{4.712254in}{2.839779in}}%
\pgfpathlineto{\pgfqpoint{4.715084in}{2.840442in}}%
\pgfpathlineto{\pgfqpoint{4.717409in}{2.840036in}}%
\pgfpathlineto{\pgfqpoint{4.722464in}{2.840887in}}%
\pgfpathlineto{\pgfqpoint{4.724688in}{2.840510in}}%
\pgfpathlineto{\pgfqpoint{4.726609in}{2.840532in}}%
\pgfpathlineto{\pgfqpoint{4.730046in}{2.843721in}}%
\pgfpathlineto{\pgfqpoint{4.730956in}{2.841562in}}%
\pgfpathlineto{\pgfqpoint{4.732573in}{2.840693in}}%
\pgfpathlineto{\pgfqpoint{4.734696in}{2.839827in}}%
\pgfpathlineto{\pgfqpoint{4.747232in}{2.838707in}}%
\pgfpathlineto{\pgfqpoint{4.749254in}{2.836998in}}%
\pgfpathlineto{\pgfqpoint{4.750871in}{2.836580in}}%
\pgfpathlineto{\pgfqpoint{4.752994in}{2.835525in}}%
\pgfpathlineto{\pgfqpoint{4.758049in}{2.837215in}}%
\pgfpathlineto{\pgfqpoint{4.759060in}{2.837875in}}%
\pgfpathlineto{\pgfqpoint{4.759565in}{2.837528in}}%
\pgfpathlineto{\pgfqpoint{4.763003in}{2.837498in}}%
\pgfpathlineto{\pgfqpoint{4.772303in}{2.842435in}}%
\pgfpathlineto{\pgfqpoint{4.778066in}{2.840662in}}%
\pgfpathlineto{\pgfqpoint{4.779481in}{2.840683in}}%
\pgfpathlineto{\pgfqpoint{4.781604in}{2.842041in}}%
\pgfpathlineto{\pgfqpoint{4.783019in}{2.843628in}}%
\pgfpathlineto{\pgfqpoint{4.783322in}{2.843401in}}%
\pgfpathlineto{\pgfqpoint{4.787669in}{2.840477in}}%
\pgfpathlineto{\pgfqpoint{4.789590in}{2.839963in}}%
\pgfpathlineto{\pgfqpoint{4.790601in}{2.838423in}}%
\pgfpathlineto{\pgfqpoint{4.791107in}{2.839151in}}%
\pgfpathlineto{\pgfqpoint{4.795049in}{2.844753in}}%
\pgfpathlineto{\pgfqpoint{4.798082in}{2.845639in}}%
\pgfpathlineto{\pgfqpoint{4.801418in}{2.847028in}}%
\pgfpathlineto{\pgfqpoint{4.803137in}{2.846934in}}%
\pgfpathlineto{\pgfqpoint{4.806372in}{2.847532in}}%
\pgfpathlineto{\pgfqpoint{4.809202in}{2.847267in}}%
\pgfpathlineto{\pgfqpoint{4.810921in}{2.846640in}}%
\pgfpathlineto{\pgfqpoint{4.813246in}{2.844179in}}%
\pgfpathlineto{\pgfqpoint{4.814864in}{2.841910in}}%
\pgfpathlineto{\pgfqpoint{4.815470in}{2.843072in}}%
\pgfpathlineto{\pgfqpoint{4.817492in}{2.845735in}}%
\pgfpathlineto{\pgfqpoint{4.819312in}{2.844037in}}%
\pgfpathlineto{\pgfqpoint{4.820120in}{2.843316in}}%
\pgfpathlineto{\pgfqpoint{4.820626in}{2.843842in}}%
\pgfpathlineto{\pgfqpoint{4.821940in}{2.844212in}}%
\pgfpathlineto{\pgfqpoint{4.822142in}{2.844046in}}%
\pgfpathlineto{\pgfqpoint{4.823760in}{2.844105in}}%
\pgfpathlineto{\pgfqpoint{4.825681in}{2.844019in}}%
\pgfpathlineto{\pgfqpoint{4.826590in}{2.843409in}}%
\pgfpathlineto{\pgfqpoint{4.826995in}{2.843896in}}%
\pgfpathlineto{\pgfqpoint{4.831443in}{2.850543in}}%
\pgfpathlineto{\pgfqpoint{4.833768in}{2.849607in}}%
\pgfpathlineto{\pgfqpoint{4.835790in}{2.847084in}}%
\pgfpathlineto{\pgfqpoint{4.837711in}{2.844100in}}%
\pgfpathlineto{\pgfqpoint{4.848224in}{2.835134in}}%
\pgfpathlineto{\pgfqpoint{4.850954in}{2.831005in}}%
\pgfpathlineto{\pgfqpoint{4.851965in}{2.832305in}}%
\pgfpathlineto{\pgfqpoint{4.853886in}{2.833268in}}%
\pgfpathlineto{\pgfqpoint{4.856716in}{2.835672in}}%
\pgfpathlineto{\pgfqpoint{4.864399in}{2.841743in}}%
\pgfpathlineto{\pgfqpoint{4.865511in}{2.842361in}}%
\pgfpathlineto{\pgfqpoint{4.868443in}{2.845828in}}%
\pgfpathlineto{\pgfqpoint{4.869858in}{2.847233in}}%
\pgfpathlineto{\pgfqpoint{4.871375in}{2.848626in}}%
\pgfpathlineto{\pgfqpoint{4.871678in}{2.848327in}}%
\pgfpathlineto{\pgfqpoint{4.873599in}{2.847521in}}%
\pgfpathlineto{\pgfqpoint{4.880372in}{2.846810in}}%
\pgfpathlineto{\pgfqpoint{4.884113in}{2.843872in}}%
\pgfpathlineto{\pgfqpoint{4.887954in}{2.842557in}}%
\pgfpathlineto{\pgfqpoint{4.888965in}{2.843163in}}%
\pgfpathlineto{\pgfqpoint{4.891189in}{2.845447in}}%
\pgfpathlineto{\pgfqpoint{4.892301in}{2.846172in}}%
\pgfpathlineto{\pgfqpoint{4.894222in}{2.847256in}}%
\pgfpathlineto{\pgfqpoint{4.896446in}{2.845968in}}%
\pgfpathlineto{\pgfqpoint{4.898569in}{2.844463in}}%
\pgfpathlineto{\pgfqpoint{4.900692in}{2.844094in}}%
\pgfpathlineto{\pgfqpoint{4.902512in}{2.842946in}}%
\pgfpathlineto{\pgfqpoint{4.907667in}{2.844618in}}%
\pgfpathlineto{\pgfqpoint{4.908577in}{2.847176in}}%
\pgfpathlineto{\pgfqpoint{4.909285in}{2.848767in}}%
\pgfpathlineto{\pgfqpoint{4.909790in}{2.847895in}}%
\pgfpathlineto{\pgfqpoint{4.912115in}{2.844720in}}%
\pgfpathlineto{\pgfqpoint{4.914845in}{2.842700in}}%
\pgfpathlineto{\pgfqpoint{4.915249in}{2.843231in}}%
\pgfpathlineto{\pgfqpoint{4.917069in}{2.844023in}}%
\pgfpathlineto{\pgfqpoint{4.924954in}{2.844668in}}%
\pgfpathlineto{\pgfqpoint{4.925763in}{2.844686in}}%
\pgfpathlineto{\pgfqpoint{4.926066in}{2.844200in}}%
\pgfpathlineto{\pgfqpoint{4.930312in}{2.835248in}}%
\pgfpathlineto{\pgfqpoint{4.931020in}{2.836060in}}%
\pgfpathlineto{\pgfqpoint{4.932637in}{2.836374in}}%
\pgfpathlineto{\pgfqpoint{4.938400in}{2.835512in}}%
\pgfpathlineto{\pgfqpoint{4.940624in}{2.836356in}}%
\pgfpathlineto{\pgfqpoint{4.943050in}{2.837776in}}%
\pgfpathlineto{\pgfqpoint{4.947700in}{2.839257in}}%
\pgfpathlineto{\pgfqpoint{4.950935in}{2.838131in}}%
\pgfpathlineto{\pgfqpoint{4.952957in}{2.837000in}}%
\pgfpathlineto{\pgfqpoint{4.958821in}{2.836483in}}%
\pgfpathlineto{\pgfqpoint{4.960944in}{2.835882in}}%
\pgfpathlineto{\pgfqpoint{4.962662in}{2.835440in}}%
\pgfpathlineto{\pgfqpoint{4.966200in}{2.831808in}}%
\pgfpathlineto{\pgfqpoint{4.967717in}{2.832135in}}%
\pgfpathlineto{\pgfqpoint{4.974894in}{2.836593in}}%
\pgfpathlineto{\pgfqpoint{4.980252in}{2.838126in}}%
\pgfpathlineto{\pgfqpoint{4.981668in}{2.838268in}}%
\pgfpathlineto{\pgfqpoint{4.983791in}{2.838385in}}%
\pgfpathlineto{\pgfqpoint{4.988744in}{2.840846in}}%
\pgfpathlineto{\pgfqpoint{4.991272in}{2.837768in}}%
\pgfpathlineto{\pgfqpoint{4.991878in}{2.837919in}}%
\pgfpathlineto{\pgfqpoint{4.992181in}{2.838358in}}%
\pgfpathlineto{\pgfqpoint{4.995315in}{2.840918in}}%
\pgfpathlineto{\pgfqpoint{4.999561in}{2.843450in}}%
\pgfpathlineto{\pgfqpoint{4.999865in}{2.842931in}}%
\pgfpathlineto{\pgfqpoint{5.001886in}{2.841195in}}%
\pgfpathlineto{\pgfqpoint{5.004818in}{2.840503in}}%
\pgfpathlineto{\pgfqpoint{5.006132in}{2.839264in}}%
\pgfpathlineto{\pgfqpoint{5.006436in}{2.839431in}}%
\pgfpathlineto{\pgfqpoint{5.008154in}{2.839148in}}%
\pgfpathlineto{\pgfqpoint{5.009974in}{2.839077in}}%
\pgfpathlineto{\pgfqpoint{5.012097in}{2.838610in}}%
\pgfpathlineto{\pgfqpoint{5.014220in}{2.836925in}}%
\pgfpathlineto{\pgfqpoint{5.019679in}{2.831198in}}%
\pgfpathlineto{\pgfqpoint{5.021397in}{2.831925in}}%
\pgfpathlineto{\pgfqpoint{5.026250in}{2.835485in}}%
\pgfpathlineto{\pgfqpoint{5.028070in}{2.836297in}}%
\pgfpathlineto{\pgfqpoint{5.031203in}{2.842663in}}%
\pgfpathlineto{\pgfqpoint{5.033023in}{2.845033in}}%
\pgfpathlineto{\pgfqpoint{5.037876in}{2.845872in}}%
\pgfpathlineto{\pgfqpoint{5.042223in}{2.846001in}}%
\pgfpathlineto{\pgfqpoint{5.043941in}{2.844329in}}%
\pgfpathlineto{\pgfqpoint{5.045862in}{2.841980in}}%
\pgfpathlineto{\pgfqpoint{5.049097in}{2.841458in}}%
\pgfpathlineto{\pgfqpoint{5.051220in}{2.838595in}}%
\pgfpathlineto{\pgfqpoint{5.053343in}{2.837107in}}%
\pgfpathlineto{\pgfqpoint{5.059813in}{2.839598in}}%
\pgfpathlineto{\pgfqpoint{5.062644in}{2.840457in}}%
\pgfpathlineto{\pgfqpoint{5.066081in}{2.840328in}}%
\pgfpathlineto{\pgfqpoint{5.072045in}{2.839002in}}%
\pgfpathlineto{\pgfqpoint{5.072146in}{2.839285in}}%
\pgfpathlineto{\pgfqpoint{5.073865in}{2.841032in}}%
\pgfpathlineto{\pgfqpoint{5.075280in}{2.842440in}}%
\pgfpathlineto{\pgfqpoint{5.076291in}{2.843038in}}%
\pgfpathlineto{\pgfqpoint{5.076696in}{2.842674in}}%
\pgfpathlineto{\pgfqpoint{5.080133in}{2.840802in}}%
\pgfpathlineto{\pgfqpoint{5.084682in}{2.840752in}}%
\pgfpathlineto{\pgfqpoint{5.086502in}{2.841136in}}%
\pgfpathlineto{\pgfqpoint{5.090748in}{2.841054in}}%
\pgfpathlineto{\pgfqpoint{5.091758in}{2.840134in}}%
\pgfpathlineto{\pgfqpoint{5.093376in}{2.836028in}}%
\pgfpathlineto{\pgfqpoint{5.094185in}{2.836670in}}%
\pgfpathlineto{\pgfqpoint{5.095297in}{2.836186in}}%
\pgfpathlineto{\pgfqpoint{5.096914in}{2.836064in}}%
\pgfpathlineto{\pgfqpoint{5.098431in}{2.837331in}}%
\pgfpathlineto{\pgfqpoint{5.106619in}{2.853052in}}%
\pgfpathlineto{\pgfqpoint{5.110056in}{2.854085in}}%
\pgfpathlineto{\pgfqpoint{5.113291in}{2.860066in}}%
\pgfpathlineto{\pgfqpoint{5.115718in}{2.860700in}}%
\pgfpathlineto{\pgfqpoint{5.117841in}{2.859929in}}%
\pgfpathlineto{\pgfqpoint{5.118750in}{2.858685in}}%
\pgfpathlineto{\pgfqpoint{5.126130in}{2.842947in}}%
\pgfpathlineto{\pgfqpoint{5.126434in}{2.843167in}}%
\pgfpathlineto{\pgfqpoint{5.129365in}{2.844359in}}%
\pgfpathlineto{\pgfqpoint{5.130174in}{2.843425in}}%
\pgfpathlineto{\pgfqpoint{5.133510in}{2.837279in}}%
\pgfpathlineto{\pgfqpoint{5.134319in}{2.838243in}}%
\pgfpathlineto{\pgfqpoint{5.136240in}{2.840059in}}%
\pgfpathlineto{\pgfqpoint{5.138868in}{2.841422in}}%
\pgfpathlineto{\pgfqpoint{5.140587in}{2.842127in}}%
\pgfpathlineto{\pgfqpoint{5.144125in}{2.841098in}}%
\pgfpathlineto{\pgfqpoint{5.144934in}{2.840894in}}%
\pgfpathlineto{\pgfqpoint{5.145338in}{2.841348in}}%
\pgfpathlineto{\pgfqpoint{5.146551in}{2.841352in}}%
\pgfpathlineto{\pgfqpoint{5.146652in}{2.841256in}}%
\pgfpathlineto{\pgfqpoint{5.148068in}{2.840643in}}%
\pgfpathlineto{\pgfqpoint{5.148371in}{2.841046in}}%
\pgfpathlineto{\pgfqpoint{5.150292in}{2.842064in}}%
\pgfpathlineto{\pgfqpoint{5.153931in}{2.844049in}}%
\pgfpathlineto{\pgfqpoint{5.154133in}{2.843647in}}%
\pgfpathlineto{\pgfqpoint{5.155346in}{2.840718in}}%
\pgfpathlineto{\pgfqpoint{5.155852in}{2.841757in}}%
\pgfpathlineto{\pgfqpoint{5.158783in}{2.848461in}}%
\pgfpathlineto{\pgfqpoint{5.159693in}{2.851114in}}%
\pgfpathlineto{\pgfqpoint{5.162322in}{2.858278in}}%
\pgfpathlineto{\pgfqpoint{5.164546in}{2.859646in}}%
\pgfpathlineto{\pgfqpoint{5.168589in}{2.870827in}}%
\pgfpathlineto{\pgfqpoint{5.170611in}{2.872225in}}%
\pgfpathlineto{\pgfqpoint{5.174554in}{2.870510in}}%
\pgfpathlineto{\pgfqpoint{5.175767in}{2.870015in}}%
\pgfpathlineto{\pgfqpoint{5.177284in}{2.864062in}}%
\pgfpathlineto{\pgfqpoint{5.177890in}{2.863091in}}%
\pgfpathlineto{\pgfqpoint{5.178497in}{2.863737in}}%
\pgfpathlineto{\pgfqpoint{5.178901in}{2.863806in}}%
\pgfpathlineto{\pgfqpoint{5.179204in}{2.863171in}}%
\pgfpathlineto{\pgfqpoint{5.183046in}{2.853675in}}%
\pgfpathlineto{\pgfqpoint{5.184461in}{2.852919in}}%
\pgfpathlineto{\pgfqpoint{5.185573in}{2.847239in}}%
\pgfpathlineto{\pgfqpoint{5.186988in}{2.841927in}}%
\pgfpathlineto{\pgfqpoint{5.187393in}{2.842020in}}%
\pgfpathlineto{\pgfqpoint{5.188303in}{2.840950in}}%
\pgfpathlineto{\pgfqpoint{5.190628in}{2.837032in}}%
\pgfpathlineto{\pgfqpoint{5.190729in}{2.837076in}}%
\pgfpathlineto{\pgfqpoint{5.193458in}{2.839827in}}%
\pgfpathlineto{\pgfqpoint{5.195581in}{2.847822in}}%
\pgfpathlineto{\pgfqpoint{5.196289in}{2.847492in}}%
\pgfpathlineto{\pgfqpoint{5.208218in}{2.845392in}}%
\pgfpathlineto{\pgfqpoint{5.211655in}{2.844818in}}%
\pgfpathlineto{\pgfqpoint{5.214183in}{2.838906in}}%
\pgfpathlineto{\pgfqpoint{5.215901in}{2.833278in}}%
\pgfpathlineto{\pgfqpoint{5.218125in}{2.833947in}}%
\pgfpathlineto{\pgfqpoint{5.220147in}{2.836238in}}%
\pgfpathlineto{\pgfqpoint{5.227830in}{2.836949in}}%
\pgfpathlineto{\pgfqpoint{5.230054in}{2.836679in}}%
\pgfpathlineto{\pgfqpoint{5.235008in}{2.838491in}}%
\pgfpathlineto{\pgfqpoint{5.237333in}{2.841203in}}%
\pgfpathlineto{\pgfqpoint{5.238445in}{2.840540in}}%
\pgfpathlineto{\pgfqpoint{5.239456in}{2.839713in}}%
\pgfpathlineto{\pgfqpoint{5.239759in}{2.840273in}}%
\pgfpathlineto{\pgfqpoint{5.241073in}{2.842600in}}%
\pgfpathlineto{\pgfqpoint{5.241579in}{2.842197in}}%
\pgfpathlineto{\pgfqpoint{5.242489in}{2.842050in}}%
\pgfpathlineto{\pgfqpoint{5.242792in}{2.842425in}}%
\pgfpathlineto{\pgfqpoint{5.244410in}{2.842769in}}%
\pgfpathlineto{\pgfqpoint{5.247038in}{2.843768in}}%
\pgfpathlineto{\pgfqpoint{5.248049in}{2.845089in}}%
\pgfpathlineto{\pgfqpoint{5.250778in}{2.850519in}}%
\pgfpathlineto{\pgfqpoint{5.255732in}{2.850097in}}%
\pgfpathlineto{\pgfqpoint{5.256440in}{2.849618in}}%
\pgfpathlineto{\pgfqpoint{5.256844in}{2.850155in}}%
\pgfpathlineto{\pgfqpoint{5.258259in}{2.852612in}}%
\pgfpathlineto{\pgfqpoint{5.258866in}{2.851873in}}%
\pgfpathlineto{\pgfqpoint{5.260686in}{2.850634in}}%
\pgfpathlineto{\pgfqpoint{5.260989in}{2.851033in}}%
\pgfpathlineto{\pgfqpoint{5.262303in}{2.851754in}}%
\pgfpathlineto{\pgfqpoint{5.262505in}{2.851607in}}%
\pgfpathlineto{\pgfqpoint{5.264932in}{2.850732in}}%
\pgfpathlineto{\pgfqpoint{5.267964in}{2.852373in}}%
\pgfpathlineto{\pgfqpoint{5.268167in}{2.852019in}}%
\pgfpathlineto{\pgfqpoint{5.269683in}{2.849001in}}%
\pgfpathlineto{\pgfqpoint{5.270289in}{2.849734in}}%
\pgfpathlineto{\pgfqpoint{5.274535in}{2.854542in}}%
\pgfpathlineto{\pgfqpoint{5.274636in}{2.854472in}}%
\pgfpathlineto{\pgfqpoint{5.278074in}{2.849860in}}%
\pgfpathlineto{\pgfqpoint{5.279691in}{2.849390in}}%
\pgfpathlineto{\pgfqpoint{5.281005in}{2.848062in}}%
\pgfpathlineto{\pgfqpoint{5.282825in}{2.846997in}}%
\pgfpathlineto{\pgfqpoint{5.285049in}{2.847686in}}%
\pgfpathlineto{\pgfqpoint{5.286060in}{2.848623in}}%
\pgfpathlineto{\pgfqpoint{5.286464in}{2.848088in}}%
\pgfpathlineto{\pgfqpoint{5.287576in}{2.847040in}}%
\pgfpathlineto{\pgfqpoint{5.287981in}{2.847480in}}%
\pgfpathlineto{\pgfqpoint{5.289497in}{2.850094in}}%
\pgfpathlineto{\pgfqpoint{5.290104in}{2.849207in}}%
\pgfpathlineto{\pgfqpoint{5.291721in}{2.848648in}}%
\pgfpathlineto{\pgfqpoint{5.292429in}{2.847878in}}%
\pgfpathlineto{\pgfqpoint{5.293541in}{2.846035in}}%
\pgfpathlineto{\pgfqpoint{5.294148in}{2.846428in}}%
\pgfpathlineto{\pgfqpoint{5.295563in}{2.848439in}}%
\pgfpathlineto{\pgfqpoint{5.297383in}{2.852660in}}%
\pgfpathlineto{\pgfqpoint{5.297787in}{2.852400in}}%
\pgfpathlineto{\pgfqpoint{5.298495in}{2.852411in}}%
\pgfpathlineto{\pgfqpoint{5.298697in}{2.852785in}}%
\pgfpathlineto{\pgfqpoint{5.300314in}{2.855426in}}%
\pgfpathlineto{\pgfqpoint{5.300719in}{2.855093in}}%
\pgfpathlineto{\pgfqpoint{5.301831in}{2.854347in}}%
\pgfpathlineto{\pgfqpoint{5.302134in}{2.854810in}}%
\pgfpathlineto{\pgfqpoint{5.307593in}{2.865599in}}%
\pgfpathlineto{\pgfqpoint{5.308604in}{2.864602in}}%
\pgfpathlineto{\pgfqpoint{5.310727in}{2.862409in}}%
\pgfpathlineto{\pgfqpoint{5.316186in}{2.860130in}}%
\pgfpathlineto{\pgfqpoint{5.317702in}{2.857957in}}%
\pgfpathlineto{\pgfqpoint{5.318006in}{2.858075in}}%
\pgfpathlineto{\pgfqpoint{5.318814in}{2.857416in}}%
\pgfpathlineto{\pgfqpoint{5.321038in}{2.854399in}}%
\pgfpathlineto{\pgfqpoint{5.322858in}{2.853378in}}%
\pgfpathlineto{\pgfqpoint{5.328115in}{2.841510in}}%
\pgfpathlineto{\pgfqpoint{5.334079in}{2.839414in}}%
\pgfpathlineto{\pgfqpoint{5.335191in}{2.840105in}}%
\pgfpathlineto{\pgfqpoint{5.336910in}{2.841397in}}%
\pgfpathlineto{\pgfqpoint{5.338730in}{2.840725in}}%
\pgfpathlineto{\pgfqpoint{5.340650in}{2.840482in}}%
\pgfpathlineto{\pgfqpoint{5.342571in}{2.840848in}}%
\pgfpathlineto{\pgfqpoint{5.342773in}{2.840387in}}%
\pgfpathlineto{\pgfqpoint{5.344795in}{2.838275in}}%
\pgfpathlineto{\pgfqpoint{5.346008in}{2.838782in}}%
\pgfpathlineto{\pgfqpoint{5.352377in}{2.849970in}}%
\pgfpathlineto{\pgfqpoint{5.353489in}{2.851360in}}%
\pgfpathlineto{\pgfqpoint{5.354804in}{2.853967in}}%
\pgfpathlineto{\pgfqpoint{5.355410in}{2.853424in}}%
\pgfpathlineto{\pgfqpoint{5.355815in}{2.853632in}}%
\pgfpathlineto{\pgfqpoint{5.356017in}{2.854237in}}%
\pgfpathlineto{\pgfqpoint{5.358443in}{2.859970in}}%
\pgfpathlineto{\pgfqpoint{5.360364in}{2.860807in}}%
\pgfpathlineto{\pgfqpoint{5.362386in}{2.862028in}}%
\pgfpathlineto{\pgfqpoint{5.364509in}{2.863739in}}%
\pgfpathlineto{\pgfqpoint{5.365823in}{2.865083in}}%
\pgfpathlineto{\pgfqpoint{5.366126in}{2.864777in}}%
\pgfpathlineto{\pgfqpoint{5.367238in}{2.860729in}}%
\pgfpathlineto{\pgfqpoint{5.368957in}{2.857299in}}%
\pgfpathlineto{\pgfqpoint{5.370877in}{2.856392in}}%
\pgfpathlineto{\pgfqpoint{5.372596in}{2.855641in}}%
\pgfpathlineto{\pgfqpoint{5.373506in}{2.854344in}}%
\pgfpathlineto{\pgfqpoint{5.374921in}{2.850803in}}%
\pgfpathlineto{\pgfqpoint{5.375528in}{2.851371in}}%
\pgfpathlineto{\pgfqpoint{5.375932in}{2.851203in}}%
\pgfpathlineto{\pgfqpoint{5.376134in}{2.850683in}}%
\pgfpathlineto{\pgfqpoint{5.378055in}{2.844555in}}%
\pgfpathlineto{\pgfqpoint{5.378763in}{2.845204in}}%
\pgfpathlineto{\pgfqpoint{5.380178in}{2.846626in}}%
\pgfpathlineto{\pgfqpoint{5.380582in}{2.846115in}}%
\pgfpathlineto{\pgfqpoint{5.383312in}{2.843280in}}%
\pgfpathlineto{\pgfqpoint{5.384424in}{2.842274in}}%
\pgfpathlineto{\pgfqpoint{5.385132in}{2.841473in}}%
\pgfpathlineto{\pgfqpoint{5.385536in}{2.842194in}}%
\pgfpathlineto{\pgfqpoint{5.389580in}{2.851050in}}%
\pgfpathlineto{\pgfqpoint{5.389681in}{2.850997in}}%
\pgfpathlineto{\pgfqpoint{5.392410in}{2.850145in}}%
\pgfpathlineto{\pgfqpoint{5.395140in}{2.851780in}}%
\pgfpathlineto{\pgfqpoint{5.395949in}{2.850839in}}%
\pgfpathlineto{\pgfqpoint{5.397364in}{2.848304in}}%
\pgfpathlineto{\pgfqpoint{5.397869in}{2.848695in}}%
\pgfpathlineto{\pgfqpoint{5.399285in}{2.848909in}}%
\pgfpathlineto{\pgfqpoint{5.399487in}{2.848662in}}%
\pgfpathlineto{\pgfqpoint{5.401711in}{2.847507in}}%
\pgfpathlineto{\pgfqpoint{5.403227in}{2.848395in}}%
\pgfpathlineto{\pgfqpoint{5.404845in}{2.848554in}}%
\pgfpathlineto{\pgfqpoint{5.405856in}{2.846023in}}%
\pgfpathlineto{\pgfqpoint{5.409091in}{2.837891in}}%
\pgfpathlineto{\pgfqpoint{5.410203in}{2.838859in}}%
\pgfpathlineto{\pgfqpoint{5.412124in}{2.839083in}}%
\pgfpathlineto{\pgfqpoint{5.413539in}{2.838500in}}%
\pgfpathlineto{\pgfqpoint{5.413842in}{2.839081in}}%
\pgfpathlineto{\pgfqpoint{5.415460in}{2.841628in}}%
\pgfpathlineto{\pgfqpoint{5.415965in}{2.841346in}}%
\pgfpathlineto{\pgfqpoint{5.417785in}{2.841483in}}%
\pgfpathlineto{\pgfqpoint{5.419908in}{2.841525in}}%
\pgfpathlineto{\pgfqpoint{5.423951in}{2.842160in}}%
\pgfpathlineto{\pgfqpoint{5.426378in}{2.843785in}}%
\pgfpathlineto{\pgfqpoint{5.433758in}{2.844117in}}%
\pgfpathlineto{\pgfqpoint{5.436993in}{2.840223in}}%
\pgfpathlineto{\pgfqpoint{5.438105in}{2.839070in}}%
\pgfpathlineto{\pgfqpoint{5.439722in}{2.837840in}}%
\pgfpathlineto{\pgfqpoint{5.439924in}{2.837959in}}%
\pgfpathlineto{\pgfqpoint{5.441845in}{2.837834in}}%
\pgfpathlineto{\pgfqpoint{5.443058in}{2.838339in}}%
\pgfpathlineto{\pgfqpoint{5.443766in}{2.838461in}}%
\pgfpathlineto{\pgfqpoint{5.444069in}{2.838044in}}%
\pgfpathlineto{\pgfqpoint{5.445383in}{2.836538in}}%
\pgfpathlineto{\pgfqpoint{5.445889in}{2.837031in}}%
\pgfpathlineto{\pgfqpoint{5.448113in}{2.837962in}}%
\pgfpathlineto{\pgfqpoint{5.452561in}{2.836995in}}%
\pgfpathlineto{\pgfqpoint{5.455392in}{2.839362in}}%
\pgfpathlineto{\pgfqpoint{5.466613in}{2.836616in}}%
\pgfpathlineto{\pgfqpoint{5.468129in}{2.837025in}}%
\pgfpathlineto{\pgfqpoint{5.470657in}{2.838998in}}%
\pgfpathlineto{\pgfqpoint{5.474397in}{2.838737in}}%
\pgfpathlineto{\pgfqpoint{5.477632in}{2.839975in}}%
\pgfpathlineto{\pgfqpoint{5.480058in}{2.843674in}}%
\pgfpathlineto{\pgfqpoint{5.481979in}{2.843611in}}%
\pgfpathlineto{\pgfqpoint{5.488247in}{2.842680in}}%
\pgfpathlineto{\pgfqpoint{5.490976in}{2.840329in}}%
\pgfpathlineto{\pgfqpoint{5.493200in}{2.839834in}}%
\pgfpathlineto{\pgfqpoint{5.495020in}{2.837983in}}%
\pgfpathlineto{\pgfqpoint{5.497649in}{2.837111in}}%
\pgfpathlineto{\pgfqpoint{5.500580in}{2.832542in}}%
\pgfpathlineto{\pgfqpoint{5.501996in}{2.832900in}}%
\pgfpathlineto{\pgfqpoint{5.505231in}{2.833767in}}%
\pgfpathlineto{\pgfqpoint{5.513824in}{2.834365in}}%
\pgfpathlineto{\pgfqpoint{5.515845in}{2.834978in}}%
\pgfpathlineto{\pgfqpoint{5.522012in}{2.835317in}}%
\pgfpathlineto{\pgfqpoint{5.524944in}{2.834386in}}%
\pgfpathlineto{\pgfqpoint{5.528078in}{2.835642in}}%
\pgfpathlineto{\pgfqpoint{5.534244in}{2.845856in}}%
\pgfpathlineto{\pgfqpoint{5.534952in}{2.845345in}}%
\pgfpathlineto{\pgfqpoint{5.535963in}{2.845966in}}%
\pgfpathlineto{\pgfqpoint{5.538794in}{2.849124in}}%
\pgfpathlineto{\pgfqpoint{5.543040in}{2.849940in}}%
\pgfpathlineto{\pgfqpoint{5.545365in}{2.851965in}}%
\pgfpathlineto{\pgfqpoint{5.548094in}{2.852185in}}%
\pgfpathlineto{\pgfqpoint{5.549105in}{2.850092in}}%
\pgfpathlineto{\pgfqpoint{5.551026in}{2.847973in}}%
\pgfpathlineto{\pgfqpoint{5.552138in}{2.846436in}}%
\pgfpathlineto{\pgfqpoint{5.554059in}{2.842585in}}%
\pgfpathlineto{\pgfqpoint{5.554564in}{2.843558in}}%
\pgfpathlineto{\pgfqpoint{5.555676in}{2.845621in}}%
\pgfpathlineto{\pgfqpoint{5.556182in}{2.844939in}}%
\pgfpathlineto{\pgfqpoint{5.558507in}{2.842706in}}%
\pgfpathlineto{\pgfqpoint{5.563359in}{2.841264in}}%
\pgfpathlineto{\pgfqpoint{5.566392in}{2.838689in}}%
\pgfpathlineto{\pgfqpoint{5.569425in}{2.841309in}}%
\pgfpathlineto{\pgfqpoint{5.571042in}{2.842053in}}%
\pgfpathlineto{\pgfqpoint{5.574277in}{2.843499in}}%
\pgfpathlineto{\pgfqpoint{5.574379in}{2.843303in}}%
\pgfpathlineto{\pgfqpoint{5.575895in}{2.839678in}}%
\pgfpathlineto{\pgfqpoint{5.576805in}{2.840186in}}%
\pgfpathlineto{\pgfqpoint{5.580444in}{2.840177in}}%
\pgfpathlineto{\pgfqpoint{5.583780in}{2.841857in}}%
\pgfpathlineto{\pgfqpoint{5.587420in}{2.843309in}}%
\pgfpathlineto{\pgfqpoint{5.588431in}{2.844752in}}%
\pgfpathlineto{\pgfqpoint{5.588936in}{2.843998in}}%
\pgfpathlineto{\pgfqpoint{5.591463in}{2.840782in}}%
\pgfpathlineto{\pgfqpoint{5.595103in}{2.838556in}}%
\pgfpathlineto{\pgfqpoint{5.602078in}{2.839100in}}%
\pgfpathlineto{\pgfqpoint{5.603797in}{2.838609in}}%
\pgfpathlineto{\pgfqpoint{5.607133in}{2.838127in}}%
\pgfpathlineto{\pgfqpoint{5.609357in}{2.836568in}}%
\pgfpathlineto{\pgfqpoint{5.609660in}{2.837000in}}%
\pgfpathlineto{\pgfqpoint{5.612491in}{2.840849in}}%
\pgfpathlineto{\pgfqpoint{5.616939in}{2.841076in}}%
\pgfpathlineto{\pgfqpoint{5.617849in}{2.840259in}}%
\pgfpathlineto{\pgfqpoint{5.618253in}{2.840907in}}%
\pgfpathlineto{\pgfqpoint{5.620376in}{2.843103in}}%
\pgfpathlineto{\pgfqpoint{5.621892in}{2.842701in}}%
\pgfpathlineto{\pgfqpoint{5.623106in}{2.843180in}}%
\pgfpathlineto{\pgfqpoint{5.624723in}{2.843370in}}%
\pgfpathlineto{\pgfqpoint{5.625835in}{2.844622in}}%
\pgfpathlineto{\pgfqpoint{5.627554in}{2.844963in}}%
\pgfpathlineto{\pgfqpoint{5.629677in}{2.843327in}}%
\pgfpathlineto{\pgfqpoint{5.634024in}{2.840630in}}%
\pgfpathlineto{\pgfqpoint{5.635338in}{2.841843in}}%
\pgfpathlineto{\pgfqpoint{5.637764in}{2.844126in}}%
\pgfpathlineto{\pgfqpoint{5.638775in}{2.842119in}}%
\pgfpathlineto{\pgfqpoint{5.639685in}{2.841252in}}%
\pgfpathlineto{\pgfqpoint{5.640190in}{2.841635in}}%
\pgfpathlineto{\pgfqpoint{5.644942in}{2.845879in}}%
\pgfpathlineto{\pgfqpoint{5.646761in}{2.844428in}}%
\pgfpathlineto{\pgfqpoint{5.647267in}{2.845273in}}%
\pgfpathlineto{\pgfqpoint{5.649188in}{2.846809in}}%
\pgfpathlineto{\pgfqpoint{5.652018in}{2.847186in}}%
\pgfpathlineto{\pgfqpoint{5.654445in}{2.849201in}}%
\pgfpathlineto{\pgfqpoint{5.659600in}{2.849042in}}%
\pgfpathlineto{\pgfqpoint{5.660510in}{2.848264in}}%
\pgfpathlineto{\pgfqpoint{5.662431in}{2.844219in}}%
\pgfpathlineto{\pgfqpoint{5.662936in}{2.844771in}}%
\pgfpathlineto{\pgfqpoint{5.664958in}{2.846159in}}%
\pgfpathlineto{\pgfqpoint{5.668294in}{2.845954in}}%
\pgfpathlineto{\pgfqpoint{5.669811in}{2.846113in}}%
\pgfpathlineto{\pgfqpoint{5.671630in}{2.845924in}}%
\pgfpathlineto{\pgfqpoint{5.675068in}{2.842794in}}%
\pgfpathlineto{\pgfqpoint{5.676180in}{2.842031in}}%
\pgfpathlineto{\pgfqpoint{5.677393in}{2.840680in}}%
\pgfpathlineto{\pgfqpoint{5.677797in}{2.841171in}}%
\pgfpathlineto{\pgfqpoint{5.682245in}{2.847586in}}%
\pgfpathlineto{\pgfqpoint{5.682852in}{2.846821in}}%
\pgfpathlineto{\pgfqpoint{5.684469in}{2.844118in}}%
\pgfpathlineto{\pgfqpoint{5.684975in}{2.844748in}}%
\pgfpathlineto{\pgfqpoint{5.686794in}{2.846315in}}%
\pgfpathlineto{\pgfqpoint{5.686997in}{2.846164in}}%
\pgfpathlineto{\pgfqpoint{5.689322in}{2.845537in}}%
\pgfpathlineto{\pgfqpoint{5.690434in}{2.846183in}}%
\pgfpathlineto{\pgfqpoint{5.693365in}{2.851306in}}%
\pgfpathlineto{\pgfqpoint{5.695084in}{2.851334in}}%
\pgfpathlineto{\pgfqpoint{5.696702in}{2.851498in}}%
\pgfpathlineto{\pgfqpoint{5.697713in}{2.852085in}}%
\pgfpathlineto{\pgfqpoint{5.698117in}{2.851625in}}%
\pgfpathlineto{\pgfqpoint{5.698622in}{2.851665in}}%
\pgfpathlineto{\pgfqpoint{5.698926in}{2.852186in}}%
\pgfpathlineto{\pgfqpoint{5.699330in}{2.852639in}}%
\pgfpathlineto{\pgfqpoint{5.699734in}{2.851707in}}%
\pgfpathlineto{\pgfqpoint{5.701554in}{2.845605in}}%
\pgfpathlineto{\pgfqpoint{5.702060in}{2.845811in}}%
\pgfpathlineto{\pgfqpoint{5.704991in}{2.846076in}}%
\pgfpathlineto{\pgfqpoint{5.707822in}{2.843189in}}%
\pgfpathlineto{\pgfqpoint{5.710551in}{2.843989in}}%
\pgfpathlineto{\pgfqpoint{5.710754in}{2.843374in}}%
\pgfpathlineto{\pgfqpoint{5.713079in}{2.838670in}}%
\pgfpathlineto{\pgfqpoint{5.714696in}{2.838188in}}%
\pgfpathlineto{\pgfqpoint{5.714898in}{2.838413in}}%
\pgfpathlineto{\pgfqpoint{5.717122in}{2.839926in}}%
\pgfpathlineto{\pgfqpoint{5.717325in}{2.839752in}}%
\pgfpathlineto{\pgfqpoint{5.719650in}{2.839065in}}%
\pgfpathlineto{\pgfqpoint{5.722986in}{2.839813in}}%
\pgfpathlineto{\pgfqpoint{5.724300in}{2.841133in}}%
\pgfpathlineto{\pgfqpoint{5.726120in}{2.841983in}}%
\pgfpathlineto{\pgfqpoint{5.728950in}{2.840846in}}%
\pgfpathlineto{\pgfqpoint{5.732286in}{2.840894in}}%
\pgfpathlineto{\pgfqpoint{5.734713in}{2.841293in}}%
\pgfpathlineto{\pgfqpoint{5.741688in}{2.836569in}}%
\pgfpathlineto{\pgfqpoint{5.741890in}{2.836935in}}%
\pgfpathlineto{\pgfqpoint{5.743508in}{2.840905in}}%
\pgfpathlineto{\pgfqpoint{5.744215in}{2.840173in}}%
\pgfpathlineto{\pgfqpoint{5.745833in}{2.838888in}}%
\pgfpathlineto{\pgfqpoint{5.745833in}{2.838888in}}%
\pgfusepath{stroke}%
\end{pgfscope}%
\begin{pgfscope}%
\pgfpathrectangle{\pgfqpoint{0.691161in}{2.667981in}}{\pgfqpoint{5.054672in}{0.911907in}}%
\pgfusepath{clip}%
\pgfsetbuttcap%
\pgfsetroundjoin%
\pgfsetlinewidth{2.007500pt}%
\definecolor{currentstroke}{rgb}{0.172549,0.627451,0.172549}%
\pgfsetstrokecolor{currentstroke}%
\pgfsetdash{{7.400000pt}{3.200000pt}}{0.000000pt}%
\pgfpathmoveto{\pgfqpoint{0.691161in}{2.978634in}}%
\pgfpathlineto{\pgfqpoint{5.745833in}{2.978634in}}%
\pgfusepath{stroke}%
\end{pgfscope}%
\begin{pgfscope}%
\pgfpathrectangle{\pgfqpoint{0.691161in}{2.667981in}}{\pgfqpoint{5.054672in}{0.911907in}}%
\pgfusepath{clip}%
\pgfsetbuttcap%
\pgfsetroundjoin%
\pgfsetlinewidth{2.007500pt}%
\definecolor{currentstroke}{rgb}{0.839216,0.152941,0.156863}%
\pgfsetstrokecolor{currentstroke}%
\pgfsetdash{{2.000000pt}{3.300000pt}}{0.000000pt}%
\pgfpathmoveto{\pgfqpoint{3.003572in}{2.709431in}}%
\pgfpathlineto{\pgfqpoint{3.003572in}{3.538438in}}%
\pgfusepath{stroke}%
\end{pgfscope}%
\begin{pgfscope}%
\pgfpathrectangle{\pgfqpoint{0.691161in}{2.667981in}}{\pgfqpoint{5.054672in}{0.911907in}}%
\pgfusepath{clip}%
\pgfsetbuttcap%
\pgfsetroundjoin%
\pgfsetlinewidth{2.007500pt}%
\definecolor{currentstroke}{rgb}{1.000000,0.498039,0.054902}%
\pgfsetstrokecolor{currentstroke}%
\pgfsetdash{{2.000000pt}{3.300000pt}}{0.000000pt}%
\pgfpathmoveto{\pgfqpoint{3.327071in}{2.709431in}}%
\pgfpathlineto{\pgfqpoint{3.327071in}{3.538438in}}%
\pgfusepath{stroke}%
\end{pgfscope}%
\begin{pgfscope}%
\pgfsetrectcap%
\pgfsetmiterjoin%
\pgfsetlinewidth{0.803000pt}%
\definecolor{currentstroke}{rgb}{0.737255,0.737255,0.737255}%
\pgfsetstrokecolor{currentstroke}%
\pgfsetdash{}{0pt}%
\pgfpathmoveto{\pgfqpoint{0.691161in}{2.667981in}}%
\pgfpathlineto{\pgfqpoint{0.691161in}{3.579888in}}%
\pgfusepath{stroke}%
\end{pgfscope}%
\begin{pgfscope}%
\pgfsetrectcap%
\pgfsetmiterjoin%
\pgfsetlinewidth{0.803000pt}%
\definecolor{currentstroke}{rgb}{0.737255,0.737255,0.737255}%
\pgfsetstrokecolor{currentstroke}%
\pgfsetdash{}{0pt}%
\pgfpathmoveto{\pgfqpoint{5.745833in}{2.667981in}}%
\pgfpathlineto{\pgfqpoint{5.745833in}{3.579888in}}%
\pgfusepath{stroke}%
\end{pgfscope}%
\begin{pgfscope}%
\pgfsetrectcap%
\pgfsetmiterjoin%
\pgfsetlinewidth{0.803000pt}%
\definecolor{currentstroke}{rgb}{0.737255,0.737255,0.737255}%
\pgfsetstrokecolor{currentstroke}%
\pgfsetdash{}{0pt}%
\pgfpathmoveto{\pgfqpoint{0.691161in}{2.667981in}}%
\pgfpathlineto{\pgfqpoint{5.745833in}{2.667981in}}%
\pgfusepath{stroke}%
\end{pgfscope}%
\begin{pgfscope}%
\pgfsetrectcap%
\pgfsetmiterjoin%
\pgfsetlinewidth{0.803000pt}%
\definecolor{currentstroke}{rgb}{0.737255,0.737255,0.737255}%
\pgfsetstrokecolor{currentstroke}%
\pgfsetdash{}{0pt}%
\pgfpathmoveto{\pgfqpoint{0.691161in}{3.579888in}}%
\pgfpathlineto{\pgfqpoint{5.745833in}{3.579888in}}%
\pgfusepath{stroke}%
\end{pgfscope}%
\begin{pgfscope}%
\pgfsetbuttcap%
\pgfsetmiterjoin%
\definecolor{currentfill}{rgb}{0.933333,0.933333,0.933333}%
\pgfsetfillcolor{currentfill}%
\pgfsetfillopacity{0.800000}%
\pgfsetlinewidth{0.501875pt}%
\definecolor{currentstroke}{rgb}{0.800000,0.800000,0.800000}%
\pgfsetstrokecolor{currentstroke}%
\pgfsetstrokeopacity{0.800000}%
\pgfsetdash{}{0pt}%
\pgfpathmoveto{\pgfqpoint{4.404343in}{2.887944in}}%
\pgfpathlineto{\pgfqpoint{5.648611in}{2.887944in}}%
\pgfpathquadraticcurveto{\pgfqpoint{5.676389in}{2.887944in}}{\pgfqpoint{5.676389in}{2.915722in}}%
\pgfpathlineto{\pgfqpoint{5.676389in}{3.482666in}}%
\pgfpathquadraticcurveto{\pgfqpoint{5.676389in}{3.510444in}}{\pgfqpoint{5.648611in}{3.510444in}}%
\pgfpathlineto{\pgfqpoint{4.404343in}{3.510444in}}%
\pgfpathquadraticcurveto{\pgfqpoint{4.376566in}{3.510444in}}{\pgfqpoint{4.376566in}{3.482666in}}%
\pgfpathlineto{\pgfqpoint{4.376566in}{2.915722in}}%
\pgfpathquadraticcurveto{\pgfqpoint{4.376566in}{2.887944in}}{\pgfqpoint{4.404343in}{2.887944in}}%
\pgfpathlineto{\pgfqpoint{4.404343in}{2.887944in}}%
\pgfpathclose%
\pgfusepath{stroke,fill}%
\end{pgfscope}%
\begin{pgfscope}%
\pgfsetbuttcap%
\pgfsetroundjoin%
\pgfsetlinewidth{2.007500pt}%
\definecolor{currentstroke}{rgb}{0.172549,0.627451,0.172549}%
\pgfsetstrokecolor{currentstroke}%
\pgfsetdash{{7.400000pt}{3.200000pt}}{0.000000pt}%
\pgfpathmoveto{\pgfqpoint{4.432121in}{3.406277in}}%
\pgfpathlineto{\pgfqpoint{4.709899in}{3.406277in}}%
\pgfusepath{stroke}%
\end{pgfscope}%
\begin{pgfscope}%
\definecolor{textcolor}{rgb}{0.000000,0.000000,0.000000}%
\pgfsetstrokecolor{textcolor}%
\pgfsetfillcolor{textcolor}%
\pgftext[x=4.821010in,y=3.357666in,left,base]{\color{textcolor}\rmfamily\fontsize{10.000000}{12.000000}\selectfont Seuil = 5}%
\end{pgfscope}%
\begin{pgfscope}%
\pgfsetbuttcap%
\pgfsetroundjoin%
\pgfsetlinewidth{2.007500pt}%
\definecolor{currentstroke}{rgb}{0.839216,0.152941,0.156863}%
\pgfsetstrokecolor{currentstroke}%
\pgfsetdash{{2.000000pt}{3.300000pt}}{0.000000pt}%
\pgfpathmoveto{\pgfqpoint{4.432121in}{3.212666in}}%
\pgfpathlineto{\pgfqpoint{4.709899in}{3.212666in}}%
\pgfusepath{stroke}%
\end{pgfscope}%
\begin{pgfscope}%
\definecolor{textcolor}{rgb}{0.000000,0.000000,0.000000}%
\pgfsetstrokecolor{textcolor}%
\pgfsetfillcolor{textcolor}%
\pgftext[x=4.821010in,y=3.164055in,left,base]{\color{textcolor}\rmfamily\fontsize{10.000000}{12.000000}\selectfont \(\displaystyle t_1\) = 164.37 s}%
\end{pgfscope}%
\begin{pgfscope}%
\pgfsetbuttcap%
\pgfsetroundjoin%
\pgfsetlinewidth{2.007500pt}%
\definecolor{currentstroke}{rgb}{1.000000,0.498039,0.054902}%
\pgfsetstrokecolor{currentstroke}%
\pgfsetdash{{2.000000pt}{3.300000pt}}{0.000000pt}%
\pgfpathmoveto{\pgfqpoint{4.432121in}{3.019055in}}%
\pgfpathlineto{\pgfqpoint{4.709899in}{3.019055in}}%
\pgfusepath{stroke}%
\end{pgfscope}%
\begin{pgfscope}%
\definecolor{textcolor}{rgb}{0.000000,0.000000,0.000000}%
\pgfsetstrokecolor{textcolor}%
\pgfsetfillcolor{textcolor}%
\pgftext[x=4.821010in,y=2.970444in,left,base]{\color{textcolor}\rmfamily\fontsize{10.000000}{12.000000}\selectfont \(\displaystyle t_2\) = 180.37 s}%
\end{pgfscope}%
\begin{pgfscope}%
\pgfsetbuttcap%
\pgfsetmiterjoin%
\definecolor{currentfill}{rgb}{0.933333,0.933333,0.933333}%
\pgfsetfillcolor{currentfill}%
\pgfsetlinewidth{0.000000pt}%
\definecolor{currentstroke}{rgb}{0.000000,0.000000,0.000000}%
\pgfsetstrokecolor{currentstroke}%
\pgfsetstrokeopacity{0.000000}%
\pgfsetdash{}{0pt}%
\pgfpathmoveto{\pgfqpoint{0.691161in}{1.606074in}}%
\pgfpathlineto{\pgfqpoint{5.745833in}{1.606074in}}%
\pgfpathlineto{\pgfqpoint{5.745833in}{2.517981in}}%
\pgfpathlineto{\pgfqpoint{0.691161in}{2.517981in}}%
\pgfpathlineto{\pgfqpoint{0.691161in}{1.606074in}}%
\pgfpathclose%
\pgfusepath{fill}%
\end{pgfscope}%
\begin{pgfscope}%
\pgfpathrectangle{\pgfqpoint{0.691161in}{1.606074in}}{\pgfqpoint{5.054672in}{0.911907in}}%
\pgfusepath{clip}%
\pgfsetbuttcap%
\pgfsetroundjoin%
\pgfsetlinewidth{0.501875pt}%
\definecolor{currentstroke}{rgb}{0.698039,0.698039,0.698039}%
\pgfsetstrokecolor{currentstroke}%
\pgfsetdash{{1.850000pt}{0.800000pt}}{0.000000pt}%
\pgfpathmoveto{\pgfqpoint{0.691161in}{1.606074in}}%
\pgfpathlineto{\pgfqpoint{0.691161in}{2.517981in}}%
\pgfusepath{stroke}%
\end{pgfscope}%
\begin{pgfscope}%
\pgfsetbuttcap%
\pgfsetroundjoin%
\definecolor{currentfill}{rgb}{0.000000,0.000000,0.000000}%
\pgfsetfillcolor{currentfill}%
\pgfsetlinewidth{0.803000pt}%
\definecolor{currentstroke}{rgb}{0.000000,0.000000,0.000000}%
\pgfsetstrokecolor{currentstroke}%
\pgfsetdash{}{0pt}%
\pgfsys@defobject{currentmarker}{\pgfqpoint{0.000000in}{0.000000in}}{\pgfqpoint{0.000000in}{0.048611in}}{%
\pgfpathmoveto{\pgfqpoint{0.000000in}{0.000000in}}%
\pgfpathlineto{\pgfqpoint{0.000000in}{0.048611in}}%
\pgfusepath{stroke,fill}%
}%
\begin{pgfscope}%
\pgfsys@transformshift{0.691161in}{1.606074in}%
\pgfsys@useobject{currentmarker}{}%
\end{pgfscope}%
\end{pgfscope}%
\begin{pgfscope}%
\pgfpathrectangle{\pgfqpoint{0.691161in}{1.606074in}}{\pgfqpoint{5.054672in}{0.911907in}}%
\pgfusepath{clip}%
\pgfsetbuttcap%
\pgfsetroundjoin%
\pgfsetlinewidth{0.501875pt}%
\definecolor{currentstroke}{rgb}{0.698039,0.698039,0.698039}%
\pgfsetstrokecolor{currentstroke}%
\pgfsetdash{{1.850000pt}{0.800000pt}}{0.000000pt}%
\pgfpathmoveto{\pgfqpoint{1.702096in}{1.606074in}}%
\pgfpathlineto{\pgfqpoint{1.702096in}{2.517981in}}%
\pgfusepath{stroke}%
\end{pgfscope}%
\begin{pgfscope}%
\pgfsetbuttcap%
\pgfsetroundjoin%
\definecolor{currentfill}{rgb}{0.000000,0.000000,0.000000}%
\pgfsetfillcolor{currentfill}%
\pgfsetlinewidth{0.803000pt}%
\definecolor{currentstroke}{rgb}{0.000000,0.000000,0.000000}%
\pgfsetstrokecolor{currentstroke}%
\pgfsetdash{}{0pt}%
\pgfsys@defobject{currentmarker}{\pgfqpoint{0.000000in}{0.000000in}}{\pgfqpoint{0.000000in}{0.048611in}}{%
\pgfpathmoveto{\pgfqpoint{0.000000in}{0.000000in}}%
\pgfpathlineto{\pgfqpoint{0.000000in}{0.048611in}}%
\pgfusepath{stroke,fill}%
}%
\begin{pgfscope}%
\pgfsys@transformshift{1.702096in}{1.606074in}%
\pgfsys@useobject{currentmarker}{}%
\end{pgfscope}%
\end{pgfscope}%
\begin{pgfscope}%
\pgfpathrectangle{\pgfqpoint{0.691161in}{1.606074in}}{\pgfqpoint{5.054672in}{0.911907in}}%
\pgfusepath{clip}%
\pgfsetbuttcap%
\pgfsetroundjoin%
\pgfsetlinewidth{0.501875pt}%
\definecolor{currentstroke}{rgb}{0.698039,0.698039,0.698039}%
\pgfsetstrokecolor{currentstroke}%
\pgfsetdash{{1.850000pt}{0.800000pt}}{0.000000pt}%
\pgfpathmoveto{\pgfqpoint{2.713030in}{1.606074in}}%
\pgfpathlineto{\pgfqpoint{2.713030in}{2.517981in}}%
\pgfusepath{stroke}%
\end{pgfscope}%
\begin{pgfscope}%
\pgfsetbuttcap%
\pgfsetroundjoin%
\definecolor{currentfill}{rgb}{0.000000,0.000000,0.000000}%
\pgfsetfillcolor{currentfill}%
\pgfsetlinewidth{0.803000pt}%
\definecolor{currentstroke}{rgb}{0.000000,0.000000,0.000000}%
\pgfsetstrokecolor{currentstroke}%
\pgfsetdash{}{0pt}%
\pgfsys@defobject{currentmarker}{\pgfqpoint{0.000000in}{0.000000in}}{\pgfqpoint{0.000000in}{0.048611in}}{%
\pgfpathmoveto{\pgfqpoint{0.000000in}{0.000000in}}%
\pgfpathlineto{\pgfqpoint{0.000000in}{0.048611in}}%
\pgfusepath{stroke,fill}%
}%
\begin{pgfscope}%
\pgfsys@transformshift{2.713030in}{1.606074in}%
\pgfsys@useobject{currentmarker}{}%
\end{pgfscope}%
\end{pgfscope}%
\begin{pgfscope}%
\pgfpathrectangle{\pgfqpoint{0.691161in}{1.606074in}}{\pgfqpoint{5.054672in}{0.911907in}}%
\pgfusepath{clip}%
\pgfsetbuttcap%
\pgfsetroundjoin%
\pgfsetlinewidth{0.501875pt}%
\definecolor{currentstroke}{rgb}{0.698039,0.698039,0.698039}%
\pgfsetstrokecolor{currentstroke}%
\pgfsetdash{{1.850000pt}{0.800000pt}}{0.000000pt}%
\pgfpathmoveto{\pgfqpoint{3.723964in}{1.606074in}}%
\pgfpathlineto{\pgfqpoint{3.723964in}{2.517981in}}%
\pgfusepath{stroke}%
\end{pgfscope}%
\begin{pgfscope}%
\pgfsetbuttcap%
\pgfsetroundjoin%
\definecolor{currentfill}{rgb}{0.000000,0.000000,0.000000}%
\pgfsetfillcolor{currentfill}%
\pgfsetlinewidth{0.803000pt}%
\definecolor{currentstroke}{rgb}{0.000000,0.000000,0.000000}%
\pgfsetstrokecolor{currentstroke}%
\pgfsetdash{}{0pt}%
\pgfsys@defobject{currentmarker}{\pgfqpoint{0.000000in}{0.000000in}}{\pgfqpoint{0.000000in}{0.048611in}}{%
\pgfpathmoveto{\pgfqpoint{0.000000in}{0.000000in}}%
\pgfpathlineto{\pgfqpoint{0.000000in}{0.048611in}}%
\pgfusepath{stroke,fill}%
}%
\begin{pgfscope}%
\pgfsys@transformshift{3.723964in}{1.606074in}%
\pgfsys@useobject{currentmarker}{}%
\end{pgfscope}%
\end{pgfscope}%
\begin{pgfscope}%
\pgfpathrectangle{\pgfqpoint{0.691161in}{1.606074in}}{\pgfqpoint{5.054672in}{0.911907in}}%
\pgfusepath{clip}%
\pgfsetbuttcap%
\pgfsetroundjoin%
\pgfsetlinewidth{0.501875pt}%
\definecolor{currentstroke}{rgb}{0.698039,0.698039,0.698039}%
\pgfsetstrokecolor{currentstroke}%
\pgfsetdash{{1.850000pt}{0.800000pt}}{0.000000pt}%
\pgfpathmoveto{\pgfqpoint{4.734899in}{1.606074in}}%
\pgfpathlineto{\pgfqpoint{4.734899in}{2.517981in}}%
\pgfusepath{stroke}%
\end{pgfscope}%
\begin{pgfscope}%
\pgfsetbuttcap%
\pgfsetroundjoin%
\definecolor{currentfill}{rgb}{0.000000,0.000000,0.000000}%
\pgfsetfillcolor{currentfill}%
\pgfsetlinewidth{0.803000pt}%
\definecolor{currentstroke}{rgb}{0.000000,0.000000,0.000000}%
\pgfsetstrokecolor{currentstroke}%
\pgfsetdash{}{0pt}%
\pgfsys@defobject{currentmarker}{\pgfqpoint{0.000000in}{0.000000in}}{\pgfqpoint{0.000000in}{0.048611in}}{%
\pgfpathmoveto{\pgfqpoint{0.000000in}{0.000000in}}%
\pgfpathlineto{\pgfqpoint{0.000000in}{0.048611in}}%
\pgfusepath{stroke,fill}%
}%
\begin{pgfscope}%
\pgfsys@transformshift{4.734899in}{1.606074in}%
\pgfsys@useobject{currentmarker}{}%
\end{pgfscope}%
\end{pgfscope}%
\begin{pgfscope}%
\pgfpathrectangle{\pgfqpoint{0.691161in}{1.606074in}}{\pgfqpoint{5.054672in}{0.911907in}}%
\pgfusepath{clip}%
\pgfsetbuttcap%
\pgfsetroundjoin%
\pgfsetlinewidth{0.501875pt}%
\definecolor{currentstroke}{rgb}{0.698039,0.698039,0.698039}%
\pgfsetstrokecolor{currentstroke}%
\pgfsetdash{{1.850000pt}{0.800000pt}}{0.000000pt}%
\pgfpathmoveto{\pgfqpoint{5.745833in}{1.606074in}}%
\pgfpathlineto{\pgfqpoint{5.745833in}{2.517981in}}%
\pgfusepath{stroke}%
\end{pgfscope}%
\begin{pgfscope}%
\pgfsetbuttcap%
\pgfsetroundjoin%
\definecolor{currentfill}{rgb}{0.000000,0.000000,0.000000}%
\pgfsetfillcolor{currentfill}%
\pgfsetlinewidth{0.803000pt}%
\definecolor{currentstroke}{rgb}{0.000000,0.000000,0.000000}%
\pgfsetstrokecolor{currentstroke}%
\pgfsetdash{}{0pt}%
\pgfsys@defobject{currentmarker}{\pgfqpoint{0.000000in}{0.000000in}}{\pgfqpoint{0.000000in}{0.048611in}}{%
\pgfpathmoveto{\pgfqpoint{0.000000in}{0.000000in}}%
\pgfpathlineto{\pgfqpoint{0.000000in}{0.048611in}}%
\pgfusepath{stroke,fill}%
}%
\begin{pgfscope}%
\pgfsys@transformshift{5.745833in}{1.606074in}%
\pgfsys@useobject{currentmarker}{}%
\end{pgfscope}%
\end{pgfscope}%
\begin{pgfscope}%
\pgfpathrectangle{\pgfqpoint{0.691161in}{1.606074in}}{\pgfqpoint{5.054672in}{0.911907in}}%
\pgfusepath{clip}%
\pgfsetbuttcap%
\pgfsetroundjoin%
\pgfsetlinewidth{0.501875pt}%
\definecolor{currentstroke}{rgb}{0.698039,0.698039,0.698039}%
\pgfsetstrokecolor{currentstroke}%
\pgfsetdash{{1.850000pt}{0.800000pt}}{0.000000pt}%
\pgfpathmoveto{\pgfqpoint{0.691161in}{1.765954in}}%
\pgfpathlineto{\pgfqpoint{5.745833in}{1.765954in}}%
\pgfusepath{stroke}%
\end{pgfscope}%
\begin{pgfscope}%
\pgfsetbuttcap%
\pgfsetroundjoin%
\definecolor{currentfill}{rgb}{0.000000,0.000000,0.000000}%
\pgfsetfillcolor{currentfill}%
\pgfsetlinewidth{0.803000pt}%
\definecolor{currentstroke}{rgb}{0.000000,0.000000,0.000000}%
\pgfsetstrokecolor{currentstroke}%
\pgfsetdash{}{0pt}%
\pgfsys@defobject{currentmarker}{\pgfqpoint{0.000000in}{0.000000in}}{\pgfqpoint{0.048611in}{0.000000in}}{%
\pgfpathmoveto{\pgfqpoint{0.000000in}{0.000000in}}%
\pgfpathlineto{\pgfqpoint{0.048611in}{0.000000in}}%
\pgfusepath{stroke,fill}%
}%
\begin{pgfscope}%
\pgfsys@transformshift{0.691161in}{1.765954in}%
\pgfsys@useobject{currentmarker}{}%
\end{pgfscope}%
\end{pgfscope}%
\begin{pgfscope}%
\definecolor{textcolor}{rgb}{0.000000,0.000000,0.000000}%
\pgfsetstrokecolor{textcolor}%
\pgfsetfillcolor{textcolor}%
\pgftext[x=0.573105in, y=1.717759in, left, base]{\color{textcolor}\rmfamily\fontsize{10.000000}{12.000000}\selectfont \(\displaystyle {0}\)}%
\end{pgfscope}%
\begin{pgfscope}%
\pgfpathrectangle{\pgfqpoint{0.691161in}{1.606074in}}{\pgfqpoint{5.054672in}{0.911907in}}%
\pgfusepath{clip}%
\pgfsetbuttcap%
\pgfsetroundjoin%
\pgfsetlinewidth{0.501875pt}%
\definecolor{currentstroke}{rgb}{0.698039,0.698039,0.698039}%
\pgfsetstrokecolor{currentstroke}%
\pgfsetdash{{1.850000pt}{0.800000pt}}{0.000000pt}%
\pgfpathmoveto{\pgfqpoint{0.691161in}{2.110077in}}%
\pgfpathlineto{\pgfqpoint{5.745833in}{2.110077in}}%
\pgfusepath{stroke}%
\end{pgfscope}%
\begin{pgfscope}%
\pgfsetbuttcap%
\pgfsetroundjoin%
\definecolor{currentfill}{rgb}{0.000000,0.000000,0.000000}%
\pgfsetfillcolor{currentfill}%
\pgfsetlinewidth{0.803000pt}%
\definecolor{currentstroke}{rgb}{0.000000,0.000000,0.000000}%
\pgfsetstrokecolor{currentstroke}%
\pgfsetdash{}{0pt}%
\pgfsys@defobject{currentmarker}{\pgfqpoint{0.000000in}{0.000000in}}{\pgfqpoint{0.048611in}{0.000000in}}{%
\pgfpathmoveto{\pgfqpoint{0.000000in}{0.000000in}}%
\pgfpathlineto{\pgfqpoint{0.048611in}{0.000000in}}%
\pgfusepath{stroke,fill}%
}%
\begin{pgfscope}%
\pgfsys@transformshift{0.691161in}{2.110077in}%
\pgfsys@useobject{currentmarker}{}%
\end{pgfscope}%
\end{pgfscope}%
\begin{pgfscope}%
\definecolor{textcolor}{rgb}{0.000000,0.000000,0.000000}%
\pgfsetstrokecolor{textcolor}%
\pgfsetfillcolor{textcolor}%
\pgftext[x=0.503661in, y=2.061883in, left, base]{\color{textcolor}\rmfamily\fontsize{10.000000}{12.000000}\selectfont \(\displaystyle {10}\)}%
\end{pgfscope}%
\begin{pgfscope}%
\pgfpathrectangle{\pgfqpoint{0.691161in}{1.606074in}}{\pgfqpoint{5.054672in}{0.911907in}}%
\pgfusepath{clip}%
\pgfsetbuttcap%
\pgfsetroundjoin%
\pgfsetlinewidth{0.501875pt}%
\definecolor{currentstroke}{rgb}{0.698039,0.698039,0.698039}%
\pgfsetstrokecolor{currentstroke}%
\pgfsetdash{{1.850000pt}{0.800000pt}}{0.000000pt}%
\pgfpathmoveto{\pgfqpoint{0.691161in}{2.454201in}}%
\pgfpathlineto{\pgfqpoint{5.745833in}{2.454201in}}%
\pgfusepath{stroke}%
\end{pgfscope}%
\begin{pgfscope}%
\pgfsetbuttcap%
\pgfsetroundjoin%
\definecolor{currentfill}{rgb}{0.000000,0.000000,0.000000}%
\pgfsetfillcolor{currentfill}%
\pgfsetlinewidth{0.803000pt}%
\definecolor{currentstroke}{rgb}{0.000000,0.000000,0.000000}%
\pgfsetstrokecolor{currentstroke}%
\pgfsetdash{}{0pt}%
\pgfsys@defobject{currentmarker}{\pgfqpoint{0.000000in}{0.000000in}}{\pgfqpoint{0.048611in}{0.000000in}}{%
\pgfpathmoveto{\pgfqpoint{0.000000in}{0.000000in}}%
\pgfpathlineto{\pgfqpoint{0.048611in}{0.000000in}}%
\pgfusepath{stroke,fill}%
}%
\begin{pgfscope}%
\pgfsys@transformshift{0.691161in}{2.454201in}%
\pgfsys@useobject{currentmarker}{}%
\end{pgfscope}%
\end{pgfscope}%
\begin{pgfscope}%
\definecolor{textcolor}{rgb}{0.000000,0.000000,0.000000}%
\pgfsetstrokecolor{textcolor}%
\pgfsetfillcolor{textcolor}%
\pgftext[x=0.503661in, y=2.406006in, left, base]{\color{textcolor}\rmfamily\fontsize{10.000000}{12.000000}\selectfont \(\displaystyle {20}\)}%
\end{pgfscope}%
\begin{pgfscope}%
\definecolor{textcolor}{rgb}{0.000000,0.000000,0.000000}%
\pgfsetstrokecolor{textcolor}%
\pgfsetfillcolor{textcolor}%
\pgftext[x=0.448105in,y=2.062027in,,bottom,rotate=90.000000]{\color{textcolor}\rmfamily\fontsize{12.000000}{14.400000}\selectfont Récursif}%
\end{pgfscope}%
\begin{pgfscope}%
\pgfpathrectangle{\pgfqpoint{0.691161in}{1.606074in}}{\pgfqpoint{5.054672in}{0.911907in}}%
\pgfusepath{clip}%
\pgfsetrectcap%
\pgfsetroundjoin%
\pgfsetlinewidth{1.505625pt}%
\definecolor{currentstroke}{rgb}{0.121569,0.466667,0.705882}%
\pgfsetstrokecolor{currentstroke}%
\pgfsetdash{}{0pt}%
\pgfpathmoveto{\pgfqpoint{0.691060in}{1.808871in}}%
\pgfpathlineto{\pgfqpoint{0.692273in}{1.810676in}}%
\pgfpathlineto{\pgfqpoint{0.693486in}{1.816207in}}%
\pgfpathlineto{\pgfqpoint{0.694093in}{1.815513in}}%
\pgfpathlineto{\pgfqpoint{0.694497in}{1.815537in}}%
\pgfpathlineto{\pgfqpoint{0.694902in}{1.816216in}}%
\pgfpathlineto{\pgfqpoint{0.695306in}{1.816312in}}%
\pgfpathlineto{\pgfqpoint{0.695710in}{1.815719in}}%
\pgfpathlineto{\pgfqpoint{0.697227in}{1.813461in}}%
\pgfpathlineto{\pgfqpoint{0.697530in}{1.814294in}}%
\pgfpathlineto{\pgfqpoint{0.700260in}{1.824209in}}%
\pgfpathlineto{\pgfqpoint{0.700361in}{1.824161in}}%
\pgfpathlineto{\pgfqpoint{0.701574in}{1.821583in}}%
\pgfpathlineto{\pgfqpoint{0.707437in}{1.809399in}}%
\pgfpathlineto{\pgfqpoint{0.713503in}{1.800106in}}%
\pgfpathlineto{\pgfqpoint{0.716131in}{1.798055in}}%
\pgfpathlineto{\pgfqpoint{0.718153in}{1.800847in}}%
\pgfpathlineto{\pgfqpoint{0.719670in}{1.802738in}}%
\pgfpathlineto{\pgfqpoint{0.720681in}{1.801751in}}%
\pgfpathlineto{\pgfqpoint{0.721186in}{1.801205in}}%
\pgfpathlineto{\pgfqpoint{0.721590in}{1.802152in}}%
\pgfpathlineto{\pgfqpoint{0.723612in}{1.806672in}}%
\pgfpathlineto{\pgfqpoint{0.724320in}{1.807037in}}%
\pgfpathlineto{\pgfqpoint{0.724724in}{1.806610in}}%
\pgfpathlineto{\pgfqpoint{0.728161in}{1.803864in}}%
\pgfpathlineto{\pgfqpoint{0.729476in}{1.807806in}}%
\pgfpathlineto{\pgfqpoint{0.730790in}{1.806473in}}%
\pgfpathlineto{\pgfqpoint{0.739181in}{1.800685in}}%
\pgfpathlineto{\pgfqpoint{0.739484in}{1.801507in}}%
\pgfpathlineto{\pgfqpoint{0.740697in}{1.805092in}}%
\pgfpathlineto{\pgfqpoint{0.741304in}{1.804544in}}%
\pgfpathlineto{\pgfqpoint{0.743831in}{1.801057in}}%
\pgfpathlineto{\pgfqpoint{0.744235in}{1.801874in}}%
\pgfpathlineto{\pgfqpoint{0.744943in}{1.802842in}}%
\pgfpathlineto{\pgfqpoint{0.745448in}{1.802210in}}%
\pgfpathlineto{\pgfqpoint{0.747672in}{1.800741in}}%
\pgfpathlineto{\pgfqpoint{0.750301in}{1.799720in}}%
\pgfpathlineto{\pgfqpoint{0.750604in}{1.800856in}}%
\pgfpathlineto{\pgfqpoint{0.752323in}{1.803449in}}%
\pgfpathlineto{\pgfqpoint{0.753536in}{1.802997in}}%
\pgfpathlineto{\pgfqpoint{0.753839in}{1.802979in}}%
\pgfpathlineto{\pgfqpoint{0.754142in}{1.803735in}}%
\pgfpathlineto{\pgfqpoint{0.756973in}{1.810687in}}%
\pgfpathlineto{\pgfqpoint{0.757984in}{1.809228in}}%
\pgfpathlineto{\pgfqpoint{0.761623in}{1.803683in}}%
\pgfpathlineto{\pgfqpoint{0.762028in}{1.804315in}}%
\pgfpathlineto{\pgfqpoint{0.763039in}{1.806132in}}%
\pgfpathlineto{\pgfqpoint{0.763746in}{1.805728in}}%
\pgfpathlineto{\pgfqpoint{0.764353in}{1.807956in}}%
\pgfpathlineto{\pgfqpoint{0.766274in}{1.813248in}}%
\pgfpathlineto{\pgfqpoint{0.766779in}{1.813918in}}%
\pgfpathlineto{\pgfqpoint{0.767386in}{1.813241in}}%
\pgfpathlineto{\pgfqpoint{0.770722in}{1.809256in}}%
\pgfpathlineto{\pgfqpoint{0.772744in}{1.807299in}}%
\pgfpathlineto{\pgfqpoint{0.772845in}{1.807387in}}%
\pgfpathlineto{\pgfqpoint{0.773856in}{1.808737in}}%
\pgfpathlineto{\pgfqpoint{0.774462in}{1.807866in}}%
\pgfpathlineto{\pgfqpoint{0.777293in}{1.803598in}}%
\pgfpathlineto{\pgfqpoint{0.777697in}{1.804369in}}%
\pgfpathlineto{\pgfqpoint{0.778405in}{1.805612in}}%
\pgfpathlineto{\pgfqpoint{0.779113in}{1.805270in}}%
\pgfpathlineto{\pgfqpoint{0.780427in}{1.804289in}}%
\pgfpathlineto{\pgfqpoint{0.787200in}{1.794954in}}%
\pgfpathlineto{\pgfqpoint{0.789828in}{1.793244in}}%
\pgfpathlineto{\pgfqpoint{0.791042in}{1.792682in}}%
\pgfpathlineto{\pgfqpoint{0.793569in}{1.790197in}}%
\pgfpathlineto{\pgfqpoint{0.793973in}{1.790861in}}%
\pgfpathlineto{\pgfqpoint{0.796400in}{1.793944in}}%
\pgfpathlineto{\pgfqpoint{0.797410in}{1.793628in}}%
\pgfpathlineto{\pgfqpoint{0.797613in}{1.794210in}}%
\pgfpathlineto{\pgfqpoint{0.798725in}{1.804190in}}%
\pgfpathlineto{\pgfqpoint{0.799533in}{1.807486in}}%
\pgfpathlineto{\pgfqpoint{0.800140in}{1.806886in}}%
\pgfpathlineto{\pgfqpoint{0.805397in}{1.801997in}}%
\pgfpathlineto{\pgfqpoint{0.809643in}{1.796778in}}%
\pgfpathlineto{\pgfqpoint{0.809744in}{1.796873in}}%
\pgfpathlineto{\pgfqpoint{0.810452in}{1.800659in}}%
\pgfpathlineto{\pgfqpoint{0.811260in}{1.804071in}}%
\pgfpathlineto{\pgfqpoint{0.811867in}{1.803489in}}%
\pgfpathlineto{\pgfqpoint{0.812372in}{1.803379in}}%
\pgfpathlineto{\pgfqpoint{0.812676in}{1.803926in}}%
\pgfpathlineto{\pgfqpoint{0.813889in}{1.806272in}}%
\pgfpathlineto{\pgfqpoint{0.814495in}{1.805708in}}%
\pgfpathlineto{\pgfqpoint{0.816214in}{1.804049in}}%
\pgfpathlineto{\pgfqpoint{0.816618in}{1.804795in}}%
\pgfpathlineto{\pgfqpoint{0.817427in}{1.805852in}}%
\pgfpathlineto{\pgfqpoint{0.817932in}{1.805337in}}%
\pgfpathlineto{\pgfqpoint{0.819550in}{1.805283in}}%
\pgfpathlineto{\pgfqpoint{0.823695in}{1.810300in}}%
\pgfpathlineto{\pgfqpoint{0.824605in}{1.809228in}}%
\pgfpathlineto{\pgfqpoint{0.827941in}{1.806250in}}%
\pgfpathlineto{\pgfqpoint{0.830670in}{1.802489in}}%
\pgfpathlineto{\pgfqpoint{0.835219in}{1.798076in}}%
\pgfpathlineto{\pgfqpoint{0.841588in}{1.794111in}}%
\pgfpathlineto{\pgfqpoint{0.846137in}{1.790531in}}%
\pgfpathlineto{\pgfqpoint{0.849372in}{1.791109in}}%
\pgfpathlineto{\pgfqpoint{0.851698in}{1.789562in}}%
\pgfpathlineto{\pgfqpoint{0.851900in}{1.789806in}}%
\pgfpathlineto{\pgfqpoint{0.854528in}{1.792501in}}%
\pgfpathlineto{\pgfqpoint{0.854629in}{1.792413in}}%
\pgfpathlineto{\pgfqpoint{0.860189in}{1.788089in}}%
\pgfpathlineto{\pgfqpoint{0.866963in}{1.787340in}}%
\pgfpathlineto{\pgfqpoint{0.867873in}{1.788993in}}%
\pgfpathlineto{\pgfqpoint{0.868479in}{1.788530in}}%
\pgfpathlineto{\pgfqpoint{0.868985in}{1.788645in}}%
\pgfpathlineto{\pgfqpoint{0.869187in}{1.789183in}}%
\pgfpathlineto{\pgfqpoint{0.870400in}{1.792740in}}%
\pgfpathlineto{\pgfqpoint{0.871006in}{1.792186in}}%
\pgfpathlineto{\pgfqpoint{0.872523in}{1.792191in}}%
\pgfpathlineto{\pgfqpoint{0.874039in}{1.793965in}}%
\pgfpathlineto{\pgfqpoint{0.876061in}{1.807052in}}%
\pgfpathlineto{\pgfqpoint{0.878083in}{1.816679in}}%
\pgfpathlineto{\pgfqpoint{0.878487in}{1.817125in}}%
\pgfpathlineto{\pgfqpoint{0.878993in}{1.816286in}}%
\pgfpathlineto{\pgfqpoint{0.883441in}{1.809910in}}%
\pgfpathlineto{\pgfqpoint{0.885665in}{1.810130in}}%
\pgfpathlineto{\pgfqpoint{0.886777in}{1.809033in}}%
\pgfpathlineto{\pgfqpoint{0.888496in}{1.807985in}}%
\pgfpathlineto{\pgfqpoint{0.889911in}{1.805971in}}%
\pgfpathlineto{\pgfqpoint{0.890821in}{1.805025in}}%
\pgfpathlineto{\pgfqpoint{0.891225in}{1.805629in}}%
\pgfpathlineto{\pgfqpoint{0.891731in}{1.805944in}}%
\pgfpathlineto{\pgfqpoint{0.892236in}{1.805210in}}%
\pgfpathlineto{\pgfqpoint{0.893854in}{1.804540in}}%
\pgfpathlineto{\pgfqpoint{0.895168in}{1.804508in}}%
\pgfpathlineto{\pgfqpoint{0.895370in}{1.804997in}}%
\pgfpathlineto{\pgfqpoint{0.896583in}{1.806656in}}%
\pgfpathlineto{\pgfqpoint{0.896987in}{1.806265in}}%
\pgfpathlineto{\pgfqpoint{0.903457in}{1.798374in}}%
\pgfpathlineto{\pgfqpoint{0.904873in}{1.797715in}}%
\pgfpathlineto{\pgfqpoint{0.908108in}{1.796709in}}%
\pgfpathlineto{\pgfqpoint{0.909321in}{1.797156in}}%
\pgfpathlineto{\pgfqpoint{0.909624in}{1.796755in}}%
\pgfpathlineto{\pgfqpoint{0.913264in}{1.793367in}}%
\pgfpathlineto{\pgfqpoint{0.914578in}{1.792635in}}%
\pgfpathlineto{\pgfqpoint{0.915892in}{1.791849in}}%
\pgfpathlineto{\pgfqpoint{0.916195in}{1.792302in}}%
\pgfpathlineto{\pgfqpoint{0.923171in}{1.804088in}}%
\pgfpathlineto{\pgfqpoint{0.923878in}{1.804240in}}%
\pgfpathlineto{\pgfqpoint{0.924283in}{1.803702in}}%
\pgfpathlineto{\pgfqpoint{0.927518in}{1.800680in}}%
\pgfpathlineto{\pgfqpoint{0.929236in}{1.800324in}}%
\pgfpathlineto{\pgfqpoint{0.930348in}{1.799594in}}%
\pgfpathlineto{\pgfqpoint{0.930652in}{1.800338in}}%
\pgfpathlineto{\pgfqpoint{0.932471in}{1.810378in}}%
\pgfpathlineto{\pgfqpoint{0.933078in}{1.812921in}}%
\pgfpathlineto{\pgfqpoint{0.933684in}{1.812112in}}%
\pgfpathlineto{\pgfqpoint{0.934190in}{1.811829in}}%
\pgfpathlineto{\pgfqpoint{0.934594in}{1.812612in}}%
\pgfpathlineto{\pgfqpoint{0.935302in}{1.813606in}}%
\pgfpathlineto{\pgfqpoint{0.935807in}{1.812919in}}%
\pgfpathlineto{\pgfqpoint{0.940761in}{1.805597in}}%
\pgfpathlineto{\pgfqpoint{0.940862in}{1.805693in}}%
\pgfpathlineto{\pgfqpoint{0.941772in}{1.806362in}}%
\pgfpathlineto{\pgfqpoint{0.942176in}{1.805783in}}%
\pgfpathlineto{\pgfqpoint{0.945613in}{1.801319in}}%
\pgfpathlineto{\pgfqpoint{0.945816in}{1.801445in}}%
\pgfpathlineto{\pgfqpoint{0.949253in}{1.802352in}}%
\pgfpathlineto{\pgfqpoint{0.955521in}{1.795379in}}%
\pgfpathlineto{\pgfqpoint{0.958857in}{1.793597in}}%
\pgfpathlineto{\pgfqpoint{0.960171in}{1.794983in}}%
\pgfpathlineto{\pgfqpoint{0.965023in}{1.818044in}}%
\pgfpathlineto{\pgfqpoint{0.966135in}{1.816393in}}%
\pgfpathlineto{\pgfqpoint{0.971493in}{1.807435in}}%
\pgfpathlineto{\pgfqpoint{0.975335in}{1.802424in}}%
\pgfpathlineto{\pgfqpoint{0.975638in}{1.802855in}}%
\pgfpathlineto{\pgfqpoint{0.977559in}{1.804195in}}%
\pgfpathlineto{\pgfqpoint{0.978772in}{1.804087in}}%
\pgfpathlineto{\pgfqpoint{0.980592in}{1.804946in}}%
\pgfpathlineto{\pgfqpoint{0.983422in}{1.816502in}}%
\pgfpathlineto{\pgfqpoint{0.983625in}{1.816314in}}%
\pgfpathlineto{\pgfqpoint{0.984231in}{1.816042in}}%
\pgfpathlineto{\pgfqpoint{0.984737in}{1.816584in}}%
\pgfpathlineto{\pgfqpoint{0.986354in}{1.818326in}}%
\pgfpathlineto{\pgfqpoint{0.988477in}{1.822323in}}%
\pgfpathlineto{\pgfqpoint{0.988679in}{1.822138in}}%
\pgfpathlineto{\pgfqpoint{0.992319in}{1.815759in}}%
\pgfpathlineto{\pgfqpoint{0.993532in}{1.813972in}}%
\pgfpathlineto{\pgfqpoint{0.994138in}{1.814411in}}%
\pgfpathlineto{\pgfqpoint{0.994947in}{1.813413in}}%
\pgfpathlineto{\pgfqpoint{0.998890in}{1.808150in}}%
\pgfpathlineto{\pgfqpoint{1.004450in}{1.801924in}}%
\pgfpathlineto{\pgfqpoint{1.004652in}{1.802181in}}%
\pgfpathlineto{\pgfqpoint{1.007280in}{1.810069in}}%
\pgfpathlineto{\pgfqpoint{1.008595in}{1.808007in}}%
\pgfpathlineto{\pgfqpoint{1.012739in}{1.802198in}}%
\pgfpathlineto{\pgfqpoint{1.013144in}{1.803188in}}%
\pgfpathlineto{\pgfqpoint{1.014357in}{1.808825in}}%
\pgfpathlineto{\pgfqpoint{1.015065in}{1.807951in}}%
\pgfpathlineto{\pgfqpoint{1.017188in}{1.805001in}}%
\pgfpathlineto{\pgfqpoint{1.017592in}{1.805955in}}%
\pgfpathlineto{\pgfqpoint{1.018704in}{1.809558in}}%
\pgfpathlineto{\pgfqpoint{1.019311in}{1.808872in}}%
\pgfpathlineto{\pgfqpoint{1.020119in}{1.807993in}}%
\pgfpathlineto{\pgfqpoint{1.020827in}{1.808448in}}%
\pgfpathlineto{\pgfqpoint{1.022546in}{1.806352in}}%
\pgfpathlineto{\pgfqpoint{1.025679in}{1.804731in}}%
\pgfpathlineto{\pgfqpoint{1.026185in}{1.806614in}}%
\pgfpathlineto{\pgfqpoint{1.027297in}{1.813394in}}%
\pgfpathlineto{\pgfqpoint{1.028005in}{1.812548in}}%
\pgfpathlineto{\pgfqpoint{1.032857in}{1.806359in}}%
\pgfpathlineto{\pgfqpoint{1.034070in}{1.804792in}}%
\pgfpathlineto{\pgfqpoint{1.034475in}{1.805563in}}%
\pgfpathlineto{\pgfqpoint{1.035283in}{1.806929in}}%
\pgfpathlineto{\pgfqpoint{1.035890in}{1.806505in}}%
\pgfpathlineto{\pgfqpoint{1.038822in}{1.805192in}}%
\pgfpathlineto{\pgfqpoint{1.039024in}{1.805671in}}%
\pgfpathlineto{\pgfqpoint{1.039934in}{1.807703in}}%
\pgfpathlineto{\pgfqpoint{1.040439in}{1.807019in}}%
\pgfpathlineto{\pgfqpoint{1.043674in}{1.803483in}}%
\pgfpathlineto{\pgfqpoint{1.048830in}{1.805134in}}%
\pgfpathlineto{\pgfqpoint{1.056412in}{1.798911in}}%
\pgfpathlineto{\pgfqpoint{1.058535in}{1.800865in}}%
\pgfpathlineto{\pgfqpoint{1.059445in}{1.802636in}}%
\pgfpathlineto{\pgfqpoint{1.060051in}{1.802092in}}%
\pgfpathlineto{\pgfqpoint{1.062275in}{1.800947in}}%
\pgfpathlineto{\pgfqpoint{1.064095in}{1.803655in}}%
\pgfpathlineto{\pgfqpoint{1.065611in}{1.805642in}}%
\pgfpathlineto{\pgfqpoint{1.067027in}{1.810221in}}%
\pgfpathlineto{\pgfqpoint{1.068038in}{1.808727in}}%
\pgfpathlineto{\pgfqpoint{1.074305in}{1.800621in}}%
\pgfpathlineto{\pgfqpoint{1.083303in}{1.798771in}}%
\pgfpathlineto{\pgfqpoint{1.083909in}{1.799303in}}%
\pgfpathlineto{\pgfqpoint{1.084516in}{1.798815in}}%
\pgfpathlineto{\pgfqpoint{1.085325in}{1.798447in}}%
\pgfpathlineto{\pgfqpoint{1.085628in}{1.798985in}}%
\pgfpathlineto{\pgfqpoint{1.087751in}{1.801761in}}%
\pgfpathlineto{\pgfqpoint{1.088054in}{1.801477in}}%
\pgfpathlineto{\pgfqpoint{1.092401in}{1.797755in}}%
\pgfpathlineto{\pgfqpoint{1.095636in}{1.796111in}}%
\pgfpathlineto{\pgfqpoint{1.096950in}{1.796118in}}%
\pgfpathlineto{\pgfqpoint{1.097860in}{1.797159in}}%
\pgfpathlineto{\pgfqpoint{1.099073in}{1.802843in}}%
\pgfpathlineto{\pgfqpoint{1.099882in}{1.802137in}}%
\pgfpathlineto{\pgfqpoint{1.102106in}{1.800506in}}%
\pgfpathlineto{\pgfqpoint{1.102612in}{1.801345in}}%
\pgfpathlineto{\pgfqpoint{1.103117in}{1.801473in}}%
\pgfpathlineto{\pgfqpoint{1.103521in}{1.800931in}}%
\pgfpathlineto{\pgfqpoint{1.105847in}{1.799216in}}%
\pgfpathlineto{\pgfqpoint{1.107363in}{1.797915in}}%
\pgfpathlineto{\pgfqpoint{1.107868in}{1.798899in}}%
\pgfpathlineto{\pgfqpoint{1.108677in}{1.799427in}}%
\pgfpathlineto{\pgfqpoint{1.109082in}{1.799044in}}%
\pgfpathlineto{\pgfqpoint{1.109688in}{1.798646in}}%
\pgfpathlineto{\pgfqpoint{1.110092in}{1.799427in}}%
\pgfpathlineto{\pgfqpoint{1.111407in}{1.800329in}}%
\pgfpathlineto{\pgfqpoint{1.111508in}{1.800285in}}%
\pgfpathlineto{\pgfqpoint{1.112620in}{1.800741in}}%
\pgfpathlineto{\pgfqpoint{1.113327in}{1.801269in}}%
\pgfpathlineto{\pgfqpoint{1.113833in}{1.800805in}}%
\pgfpathlineto{\pgfqpoint{1.114844in}{1.801546in}}%
\pgfpathlineto{\pgfqpoint{1.115956in}{1.800278in}}%
\pgfpathlineto{\pgfqpoint{1.118888in}{1.799304in}}%
\pgfpathlineto{\pgfqpoint{1.120101in}{1.799012in}}%
\pgfpathlineto{\pgfqpoint{1.121415in}{1.797958in}}%
\pgfpathlineto{\pgfqpoint{1.121819in}{1.798470in}}%
\pgfpathlineto{\pgfqpoint{1.122931in}{1.801328in}}%
\pgfpathlineto{\pgfqpoint{1.125155in}{1.817360in}}%
\pgfpathlineto{\pgfqpoint{1.125256in}{1.817342in}}%
\pgfpathlineto{\pgfqpoint{1.126368in}{1.815270in}}%
\pgfpathlineto{\pgfqpoint{1.127582in}{1.813391in}}%
\pgfpathlineto{\pgfqpoint{1.128087in}{1.813912in}}%
\pgfpathlineto{\pgfqpoint{1.129603in}{1.816036in}}%
\pgfpathlineto{\pgfqpoint{1.130412in}{1.818116in}}%
\pgfpathlineto{\pgfqpoint{1.131019in}{1.817526in}}%
\pgfpathlineto{\pgfqpoint{1.132535in}{1.817350in}}%
\pgfpathlineto{\pgfqpoint{1.132737in}{1.817714in}}%
\pgfpathlineto{\pgfqpoint{1.134759in}{1.819427in}}%
\pgfpathlineto{\pgfqpoint{1.136175in}{1.819344in}}%
\pgfpathlineto{\pgfqpoint{1.137489in}{1.817676in}}%
\pgfpathlineto{\pgfqpoint{1.139106in}{1.814764in}}%
\pgfpathlineto{\pgfqpoint{1.142948in}{1.809050in}}%
\pgfpathlineto{\pgfqpoint{1.143757in}{1.809913in}}%
\pgfpathlineto{\pgfqpoint{1.144464in}{1.810474in}}%
\pgfpathlineto{\pgfqpoint{1.144869in}{1.809958in}}%
\pgfpathlineto{\pgfqpoint{1.145475in}{1.809556in}}%
\pgfpathlineto{\pgfqpoint{1.145880in}{1.810380in}}%
\pgfpathlineto{\pgfqpoint{1.147699in}{1.812137in}}%
\pgfpathlineto{\pgfqpoint{1.149115in}{1.810169in}}%
\pgfpathlineto{\pgfqpoint{1.153259in}{1.805044in}}%
\pgfpathlineto{\pgfqpoint{1.154472in}{1.804059in}}%
\pgfpathlineto{\pgfqpoint{1.155180in}{1.803402in}}%
\pgfpathlineto{\pgfqpoint{1.155686in}{1.804207in}}%
\pgfpathlineto{\pgfqpoint{1.156393in}{1.805135in}}%
\pgfpathlineto{\pgfqpoint{1.156899in}{1.804523in}}%
\pgfpathlineto{\pgfqpoint{1.160033in}{1.800678in}}%
\pgfpathlineto{\pgfqpoint{1.160336in}{1.801043in}}%
\pgfpathlineto{\pgfqpoint{1.161044in}{1.801678in}}%
\pgfpathlineto{\pgfqpoint{1.161549in}{1.801036in}}%
\pgfpathlineto{\pgfqpoint{1.162156in}{1.800456in}}%
\pgfpathlineto{\pgfqpoint{1.162560in}{1.801116in}}%
\pgfpathlineto{\pgfqpoint{1.165593in}{1.807989in}}%
\pgfpathlineto{\pgfqpoint{1.166098in}{1.807291in}}%
\pgfpathlineto{\pgfqpoint{1.168322in}{1.804107in}}%
\pgfpathlineto{\pgfqpoint{1.168828in}{1.804434in}}%
\pgfpathlineto{\pgfqpoint{1.169738in}{1.803569in}}%
\pgfpathlineto{\pgfqpoint{1.172770in}{1.799533in}}%
\pgfpathlineto{\pgfqpoint{1.173074in}{1.800050in}}%
\pgfpathlineto{\pgfqpoint{1.174186in}{1.803471in}}%
\pgfpathlineto{\pgfqpoint{1.174994in}{1.803048in}}%
\pgfpathlineto{\pgfqpoint{1.176511in}{1.801228in}}%
\pgfpathlineto{\pgfqpoint{1.178836in}{1.799780in}}%
\pgfpathlineto{\pgfqpoint{1.182779in}{1.800565in}}%
\pgfpathlineto{\pgfqpoint{1.183587in}{1.801417in}}%
\pgfpathlineto{\pgfqpoint{1.184194in}{1.801025in}}%
\pgfpathlineto{\pgfqpoint{1.188238in}{1.797395in}}%
\pgfpathlineto{\pgfqpoint{1.192382in}{1.793452in}}%
\pgfpathlineto{\pgfqpoint{1.197842in}{1.788761in}}%
\pgfpathlineto{\pgfqpoint{1.200975in}{1.787096in}}%
\pgfpathlineto{\pgfqpoint{1.201279in}{1.787955in}}%
\pgfpathlineto{\pgfqpoint{1.204817in}{1.796870in}}%
\pgfpathlineto{\pgfqpoint{1.207344in}{1.798068in}}%
\pgfpathlineto{\pgfqpoint{1.208355in}{1.799572in}}%
\pgfpathlineto{\pgfqpoint{1.209366in}{1.801798in}}%
\pgfpathlineto{\pgfqpoint{1.209872in}{1.801290in}}%
\pgfpathlineto{\pgfqpoint{1.211489in}{1.799589in}}%
\pgfpathlineto{\pgfqpoint{1.211894in}{1.800164in}}%
\pgfpathlineto{\pgfqpoint{1.215027in}{1.805101in}}%
\pgfpathlineto{\pgfqpoint{1.215129in}{1.805019in}}%
\pgfpathlineto{\pgfqpoint{1.217251in}{1.802243in}}%
\pgfpathlineto{\pgfqpoint{1.217656in}{1.803493in}}%
\pgfpathlineto{\pgfqpoint{1.218465in}{1.805577in}}%
\pgfpathlineto{\pgfqpoint{1.219071in}{1.805018in}}%
\pgfpathlineto{\pgfqpoint{1.223014in}{1.802282in}}%
\pgfpathlineto{\pgfqpoint{1.223115in}{1.802405in}}%
\pgfpathlineto{\pgfqpoint{1.224227in}{1.803207in}}%
\pgfpathlineto{\pgfqpoint{1.224631in}{1.802762in}}%
\pgfpathlineto{\pgfqpoint{1.227361in}{1.800848in}}%
\pgfpathlineto{\pgfqpoint{1.229282in}{1.798334in}}%
\pgfpathlineto{\pgfqpoint{1.229787in}{1.799500in}}%
\pgfpathlineto{\pgfqpoint{1.231910in}{1.802495in}}%
\pgfpathlineto{\pgfqpoint{1.232921in}{1.801849in}}%
\pgfpathlineto{\pgfqpoint{1.233528in}{1.801414in}}%
\pgfpathlineto{\pgfqpoint{1.233831in}{1.802307in}}%
\pgfpathlineto{\pgfqpoint{1.235044in}{1.807439in}}%
\pgfpathlineto{\pgfqpoint{1.235853in}{1.807220in}}%
\pgfpathlineto{\pgfqpoint{1.238178in}{1.804785in}}%
\pgfpathlineto{\pgfqpoint{1.239391in}{1.803364in}}%
\pgfpathlineto{\pgfqpoint{1.239896in}{1.804227in}}%
\pgfpathlineto{\pgfqpoint{1.241008in}{1.808324in}}%
\pgfpathlineto{\pgfqpoint{1.243131in}{1.822380in}}%
\pgfpathlineto{\pgfqpoint{1.243637in}{1.821970in}}%
\pgfpathlineto{\pgfqpoint{1.243839in}{1.821549in}}%
\pgfpathlineto{\pgfqpoint{1.244547in}{1.820381in}}%
\pgfpathlineto{\pgfqpoint{1.244951in}{1.821489in}}%
\pgfpathlineto{\pgfqpoint{1.246670in}{1.824696in}}%
\pgfpathlineto{\pgfqpoint{1.246771in}{1.824649in}}%
\pgfpathlineto{\pgfqpoint{1.249500in}{1.820720in}}%
\pgfpathlineto{\pgfqpoint{1.250309in}{1.819408in}}%
\pgfpathlineto{\pgfqpoint{1.250916in}{1.820136in}}%
\pgfpathlineto{\pgfqpoint{1.251623in}{1.819432in}}%
\pgfpathlineto{\pgfqpoint{1.254959in}{1.813193in}}%
\pgfpathlineto{\pgfqpoint{1.255364in}{1.813824in}}%
\pgfpathlineto{\pgfqpoint{1.255970in}{1.814614in}}%
\pgfpathlineto{\pgfqpoint{1.256476in}{1.813815in}}%
\pgfpathlineto{\pgfqpoint{1.260519in}{1.807637in}}%
\pgfpathlineto{\pgfqpoint{1.260924in}{1.807806in}}%
\pgfpathlineto{\pgfqpoint{1.261126in}{1.808321in}}%
\pgfpathlineto{\pgfqpoint{1.261935in}{1.809968in}}%
\pgfpathlineto{\pgfqpoint{1.262541in}{1.809251in}}%
\pgfpathlineto{\pgfqpoint{1.262946in}{1.809067in}}%
\pgfpathlineto{\pgfqpoint{1.263249in}{1.809664in}}%
\pgfpathlineto{\pgfqpoint{1.265574in}{1.814429in}}%
\pgfpathlineto{\pgfqpoint{1.266181in}{1.814593in}}%
\pgfpathlineto{\pgfqpoint{1.266585in}{1.813996in}}%
\pgfpathlineto{\pgfqpoint{1.275279in}{1.800835in}}%
\pgfpathlineto{\pgfqpoint{1.278312in}{1.800162in}}%
\pgfpathlineto{\pgfqpoint{1.279323in}{1.800470in}}%
\pgfpathlineto{\pgfqpoint{1.279626in}{1.800140in}}%
\pgfpathlineto{\pgfqpoint{1.280940in}{1.800315in}}%
\pgfpathlineto{\pgfqpoint{1.282153in}{1.798849in}}%
\pgfpathlineto{\pgfqpoint{1.285085in}{1.796487in}}%
\pgfpathlineto{\pgfqpoint{1.291151in}{1.793180in}}%
\pgfpathlineto{\pgfqpoint{1.293981in}{1.792423in}}%
\pgfpathlineto{\pgfqpoint{1.301665in}{1.786157in}}%
\pgfpathlineto{\pgfqpoint{1.306921in}{1.782787in}}%
\pgfpathlineto{\pgfqpoint{1.309145in}{1.782100in}}%
\pgfpathlineto{\pgfqpoint{1.314503in}{1.779799in}}%
\pgfpathlineto{\pgfqpoint{1.314604in}{1.779981in}}%
\pgfpathlineto{\pgfqpoint{1.318143in}{1.785921in}}%
\pgfpathlineto{\pgfqpoint{1.321984in}{1.785986in}}%
\pgfpathlineto{\pgfqpoint{1.324309in}{1.787182in}}%
\pgfpathlineto{\pgfqpoint{1.326432in}{1.787361in}}%
\pgfpathlineto{\pgfqpoint{1.328555in}{1.790459in}}%
\pgfpathlineto{\pgfqpoint{1.330173in}{1.794567in}}%
\pgfpathlineto{\pgfqpoint{1.332599in}{1.801988in}}%
\pgfpathlineto{\pgfqpoint{1.333105in}{1.801603in}}%
\pgfpathlineto{\pgfqpoint{1.336137in}{1.800034in}}%
\pgfpathlineto{\pgfqpoint{1.339777in}{1.796480in}}%
\pgfpathlineto{\pgfqpoint{1.343416in}{1.797357in}}%
\pgfpathlineto{\pgfqpoint{1.344629in}{1.797240in}}%
\pgfpathlineto{\pgfqpoint{1.344831in}{1.797725in}}%
\pgfpathlineto{\pgfqpoint{1.345943in}{1.799438in}}%
\pgfpathlineto{\pgfqpoint{1.346449in}{1.799035in}}%
\pgfpathlineto{\pgfqpoint{1.346853in}{1.799265in}}%
\pgfpathlineto{\pgfqpoint{1.347055in}{1.799824in}}%
\pgfpathlineto{\pgfqpoint{1.347965in}{1.801635in}}%
\pgfpathlineto{\pgfqpoint{1.348471in}{1.801082in}}%
\pgfpathlineto{\pgfqpoint{1.353323in}{1.796187in}}%
\pgfpathlineto{\pgfqpoint{1.356862in}{1.794693in}}%
\pgfpathlineto{\pgfqpoint{1.358580in}{1.795347in}}%
\pgfpathlineto{\pgfqpoint{1.359187in}{1.795349in}}%
\pgfpathlineto{\pgfqpoint{1.359591in}{1.794915in}}%
\pgfpathlineto{\pgfqpoint{1.364646in}{1.791485in}}%
\pgfpathlineto{\pgfqpoint{1.366870in}{1.790974in}}%
\pgfpathlineto{\pgfqpoint{1.376575in}{1.787209in}}%
\pgfpathlineto{\pgfqpoint{1.380113in}{1.788092in}}%
\pgfpathlineto{\pgfqpoint{1.381023in}{1.789643in}}%
\pgfpathlineto{\pgfqpoint{1.381528in}{1.789428in}}%
\pgfpathlineto{\pgfqpoint{1.382640in}{1.789028in}}%
\pgfpathlineto{\pgfqpoint{1.382843in}{1.789695in}}%
\pgfpathlineto{\pgfqpoint{1.384864in}{1.793988in}}%
\pgfpathlineto{\pgfqpoint{1.386280in}{1.792732in}}%
\pgfpathlineto{\pgfqpoint{1.387998in}{1.791274in}}%
\pgfpathlineto{\pgfqpoint{1.388403in}{1.791940in}}%
\pgfpathlineto{\pgfqpoint{1.389818in}{1.792331in}}%
\pgfpathlineto{\pgfqpoint{1.391132in}{1.791581in}}%
\pgfpathlineto{\pgfqpoint{1.392750in}{1.790961in}}%
\pgfpathlineto{\pgfqpoint{1.392952in}{1.791216in}}%
\pgfpathlineto{\pgfqpoint{1.397602in}{1.797533in}}%
\pgfpathlineto{\pgfqpoint{1.399927in}{1.795995in}}%
\pgfpathlineto{\pgfqpoint{1.401242in}{1.796614in}}%
\pgfpathlineto{\pgfqpoint{1.402354in}{1.798733in}}%
\pgfpathlineto{\pgfqpoint{1.402960in}{1.798094in}}%
\pgfpathlineto{\pgfqpoint{1.403769in}{1.797277in}}%
\pgfpathlineto{\pgfqpoint{1.404173in}{1.798355in}}%
\pgfpathlineto{\pgfqpoint{1.405487in}{1.803392in}}%
\pgfpathlineto{\pgfqpoint{1.406094in}{1.802799in}}%
\pgfpathlineto{\pgfqpoint{1.410542in}{1.798445in}}%
\pgfpathlineto{\pgfqpoint{1.410643in}{1.798535in}}%
\pgfpathlineto{\pgfqpoint{1.412766in}{1.803594in}}%
\pgfpathlineto{\pgfqpoint{1.414788in}{1.810477in}}%
\pgfpathlineto{\pgfqpoint{1.415799in}{1.811084in}}%
\pgfpathlineto{\pgfqpoint{1.416203in}{1.810462in}}%
\pgfpathlineto{\pgfqpoint{1.418427in}{1.808611in}}%
\pgfpathlineto{\pgfqpoint{1.422067in}{1.805609in}}%
\pgfpathlineto{\pgfqpoint{1.422269in}{1.805990in}}%
\pgfpathlineto{\pgfqpoint{1.423078in}{1.807164in}}%
\pgfpathlineto{\pgfqpoint{1.423684in}{1.806844in}}%
\pgfpathlineto{\pgfqpoint{1.425100in}{1.805164in}}%
\pgfpathlineto{\pgfqpoint{1.429548in}{1.801319in}}%
\pgfpathlineto{\pgfqpoint{1.430862in}{1.800107in}}%
\pgfpathlineto{\pgfqpoint{1.435917in}{1.794768in}}%
\pgfpathlineto{\pgfqpoint{1.436624in}{1.794556in}}%
\pgfpathlineto{\pgfqpoint{1.436826in}{1.795079in}}%
\pgfpathlineto{\pgfqpoint{1.438646in}{1.807571in}}%
\pgfpathlineto{\pgfqpoint{1.440769in}{1.818893in}}%
\pgfpathlineto{\pgfqpoint{1.441780in}{1.821451in}}%
\pgfpathlineto{\pgfqpoint{1.442589in}{1.820598in}}%
\pgfpathlineto{\pgfqpoint{1.447138in}{1.818716in}}%
\pgfpathlineto{\pgfqpoint{1.449665in}{1.821440in}}%
\pgfpathlineto{\pgfqpoint{1.451990in}{1.820784in}}%
\pgfpathlineto{\pgfqpoint{1.455832in}{1.813820in}}%
\pgfpathlineto{\pgfqpoint{1.456135in}{1.814332in}}%
\pgfpathlineto{\pgfqpoint{1.458258in}{1.822812in}}%
\pgfpathlineto{\pgfqpoint{1.458865in}{1.826169in}}%
\pgfpathlineto{\pgfqpoint{1.459572in}{1.824917in}}%
\pgfpathlineto{\pgfqpoint{1.465436in}{1.818030in}}%
\pgfpathlineto{\pgfqpoint{1.467458in}{1.814395in}}%
\pgfpathlineto{\pgfqpoint{1.469075in}{1.811882in}}%
\pgfpathlineto{\pgfqpoint{1.469480in}{1.812344in}}%
\pgfpathlineto{\pgfqpoint{1.472209in}{1.815962in}}%
\pgfpathlineto{\pgfqpoint{1.472411in}{1.815774in}}%
\pgfpathlineto{\pgfqpoint{1.473827in}{1.814364in}}%
\pgfpathlineto{\pgfqpoint{1.474332in}{1.815098in}}%
\pgfpathlineto{\pgfqpoint{1.474838in}{1.815275in}}%
\pgfpathlineto{\pgfqpoint{1.475242in}{1.814678in}}%
\pgfpathlineto{\pgfqpoint{1.476152in}{1.813732in}}%
\pgfpathlineto{\pgfqpoint{1.476758in}{1.814049in}}%
\pgfpathlineto{\pgfqpoint{1.477668in}{1.812782in}}%
\pgfpathlineto{\pgfqpoint{1.479589in}{1.810185in}}%
\pgfpathlineto{\pgfqpoint{1.479892in}{1.810523in}}%
\pgfpathlineto{\pgfqpoint{1.480600in}{1.811059in}}%
\pgfpathlineto{\pgfqpoint{1.481105in}{1.810588in}}%
\pgfpathlineto{\pgfqpoint{1.486665in}{1.804806in}}%
\pgfpathlineto{\pgfqpoint{1.487070in}{1.805465in}}%
\pgfpathlineto{\pgfqpoint{1.487879in}{1.805596in}}%
\pgfpathlineto{\pgfqpoint{1.488182in}{1.805340in}}%
\pgfpathlineto{\pgfqpoint{1.488890in}{1.805101in}}%
\pgfpathlineto{\pgfqpoint{1.489294in}{1.805652in}}%
\pgfpathlineto{\pgfqpoint{1.490406in}{1.805202in}}%
\pgfpathlineto{\pgfqpoint{1.492731in}{1.804814in}}%
\pgfpathlineto{\pgfqpoint{1.494753in}{1.802788in}}%
\pgfpathlineto{\pgfqpoint{1.494955in}{1.803075in}}%
\pgfpathlineto{\pgfqpoint{1.496067in}{1.805794in}}%
\pgfpathlineto{\pgfqpoint{1.496775in}{1.804980in}}%
\pgfpathlineto{\pgfqpoint{1.499605in}{1.802163in}}%
\pgfpathlineto{\pgfqpoint{1.499909in}{1.802714in}}%
\pgfpathlineto{\pgfqpoint{1.500616in}{1.803897in}}%
\pgfpathlineto{\pgfqpoint{1.501223in}{1.803312in}}%
\pgfpathlineto{\pgfqpoint{1.501627in}{1.803518in}}%
\pgfpathlineto{\pgfqpoint{1.501830in}{1.804117in}}%
\pgfpathlineto{\pgfqpoint{1.504458in}{1.810229in}}%
\pgfpathlineto{\pgfqpoint{1.505469in}{1.809458in}}%
\pgfpathlineto{\pgfqpoint{1.510220in}{1.805247in}}%
\pgfpathlineto{\pgfqpoint{1.510422in}{1.805719in}}%
\pgfpathlineto{\pgfqpoint{1.511636in}{1.811896in}}%
\pgfpathlineto{\pgfqpoint{1.512545in}{1.810259in}}%
\pgfpathlineto{\pgfqpoint{1.515983in}{1.805880in}}%
\pgfpathlineto{\pgfqpoint{1.520633in}{1.800718in}}%
\pgfpathlineto{\pgfqpoint{1.523969in}{1.798478in}}%
\pgfpathlineto{\pgfqpoint{1.525789in}{1.796332in}}%
\pgfpathlineto{\pgfqpoint{1.527507in}{1.794930in}}%
\pgfpathlineto{\pgfqpoint{1.527912in}{1.795744in}}%
\pgfpathlineto{\pgfqpoint{1.528518in}{1.796145in}}%
\pgfpathlineto{\pgfqpoint{1.529024in}{1.795668in}}%
\pgfpathlineto{\pgfqpoint{1.531248in}{1.794950in}}%
\pgfpathlineto{\pgfqpoint{1.533977in}{1.794936in}}%
\pgfpathlineto{\pgfqpoint{1.536808in}{1.799184in}}%
\pgfpathlineto{\pgfqpoint{1.537414in}{1.800585in}}%
\pgfpathlineto{\pgfqpoint{1.539537in}{1.805591in}}%
\pgfpathlineto{\pgfqpoint{1.539638in}{1.805563in}}%
\pgfpathlineto{\pgfqpoint{1.542267in}{1.802300in}}%
\pgfpathlineto{\pgfqpoint{1.544289in}{1.801060in}}%
\pgfpathlineto{\pgfqpoint{1.546513in}{1.799684in}}%
\pgfpathlineto{\pgfqpoint{1.547018in}{1.799500in}}%
\pgfpathlineto{\pgfqpoint{1.547423in}{1.800083in}}%
\pgfpathlineto{\pgfqpoint{1.548029in}{1.800434in}}%
\pgfpathlineto{\pgfqpoint{1.548535in}{1.799902in}}%
\pgfpathlineto{\pgfqpoint{1.549040in}{1.799792in}}%
\pgfpathlineto{\pgfqpoint{1.549445in}{1.800361in}}%
\pgfpathlineto{\pgfqpoint{1.549950in}{1.800583in}}%
\pgfpathlineto{\pgfqpoint{1.550455in}{1.799950in}}%
\pgfpathlineto{\pgfqpoint{1.553589in}{1.796270in}}%
\pgfpathlineto{\pgfqpoint{1.553893in}{1.796505in}}%
\pgfpathlineto{\pgfqpoint{1.555005in}{1.796072in}}%
\pgfpathlineto{\pgfqpoint{1.558139in}{1.794452in}}%
\pgfpathlineto{\pgfqpoint{1.561475in}{1.795584in}}%
\pgfpathlineto{\pgfqpoint{1.565619in}{1.792775in}}%
\pgfpathlineto{\pgfqpoint{1.568248in}{1.793476in}}%
\pgfpathlineto{\pgfqpoint{1.571483in}{1.791228in}}%
\pgfpathlineto{\pgfqpoint{1.574920in}{1.789039in}}%
\pgfpathlineto{\pgfqpoint{1.578054in}{1.787543in}}%
\pgfpathlineto{\pgfqpoint{1.581491in}{1.787817in}}%
\pgfpathlineto{\pgfqpoint{1.583311in}{1.787581in}}%
\pgfpathlineto{\pgfqpoint{1.584221in}{1.789206in}}%
\pgfpathlineto{\pgfqpoint{1.585130in}{1.792045in}}%
\pgfpathlineto{\pgfqpoint{1.585737in}{1.791614in}}%
\pgfpathlineto{\pgfqpoint{1.588568in}{1.789369in}}%
\pgfpathlineto{\pgfqpoint{1.588871in}{1.790363in}}%
\pgfpathlineto{\pgfqpoint{1.590893in}{1.795698in}}%
\pgfpathlineto{\pgfqpoint{1.594431in}{1.793778in}}%
\pgfpathlineto{\pgfqpoint{1.597868in}{1.791339in}}%
\pgfpathlineto{\pgfqpoint{1.599284in}{1.794171in}}%
\pgfpathlineto{\pgfqpoint{1.601204in}{1.799594in}}%
\pgfpathlineto{\pgfqpoint{1.602620in}{1.800278in}}%
\pgfpathlineto{\pgfqpoint{1.602721in}{1.800223in}}%
\pgfpathlineto{\pgfqpoint{1.606764in}{1.796390in}}%
\pgfpathlineto{\pgfqpoint{1.607472in}{1.798105in}}%
\pgfpathlineto{\pgfqpoint{1.609090in}{1.798396in}}%
\pgfpathlineto{\pgfqpoint{1.610101in}{1.797454in}}%
\pgfpathlineto{\pgfqpoint{1.612325in}{1.796091in}}%
\pgfpathlineto{\pgfqpoint{1.613234in}{1.796837in}}%
\pgfpathlineto{\pgfqpoint{1.614144in}{1.797691in}}%
\pgfpathlineto{\pgfqpoint{1.614650in}{1.797386in}}%
\pgfpathlineto{\pgfqpoint{1.617784in}{1.796778in}}%
\pgfpathlineto{\pgfqpoint{1.619199in}{1.796971in}}%
\pgfpathlineto{\pgfqpoint{1.620614in}{1.797743in}}%
\pgfpathlineto{\pgfqpoint{1.623647in}{1.798365in}}%
\pgfpathlineto{\pgfqpoint{1.629207in}{1.792980in}}%
\pgfpathlineto{\pgfqpoint{1.631128in}{1.792431in}}%
\pgfpathlineto{\pgfqpoint{1.631330in}{1.792785in}}%
\pgfpathlineto{\pgfqpoint{1.632341in}{1.794061in}}%
\pgfpathlineto{\pgfqpoint{1.632847in}{1.793677in}}%
\pgfpathlineto{\pgfqpoint{1.635071in}{1.792257in}}%
\pgfpathlineto{\pgfqpoint{1.635374in}{1.792994in}}%
\pgfpathlineto{\pgfqpoint{1.637497in}{1.796138in}}%
\pgfpathlineto{\pgfqpoint{1.638508in}{1.795572in}}%
\pgfpathlineto{\pgfqpoint{1.639519in}{1.794957in}}%
\pgfpathlineto{\pgfqpoint{1.639822in}{1.795617in}}%
\pgfpathlineto{\pgfqpoint{1.641541in}{1.797585in}}%
\pgfpathlineto{\pgfqpoint{1.644776in}{1.795066in}}%
\pgfpathlineto{\pgfqpoint{1.651043in}{1.790541in}}%
\pgfpathlineto{\pgfqpoint{1.651246in}{1.790788in}}%
\pgfpathlineto{\pgfqpoint{1.652054in}{1.790904in}}%
\pgfpathlineto{\pgfqpoint{1.652459in}{1.790483in}}%
\pgfpathlineto{\pgfqpoint{1.654986in}{1.789106in}}%
\pgfpathlineto{\pgfqpoint{1.655087in}{1.789219in}}%
\pgfpathlineto{\pgfqpoint{1.656806in}{1.790290in}}%
\pgfpathlineto{\pgfqpoint{1.657008in}{1.790128in}}%
\pgfpathlineto{\pgfqpoint{1.658625in}{1.789187in}}%
\pgfpathlineto{\pgfqpoint{1.658929in}{1.789663in}}%
\pgfpathlineto{\pgfqpoint{1.660445in}{1.790159in}}%
\pgfpathlineto{\pgfqpoint{1.662770in}{1.790149in}}%
\pgfpathlineto{\pgfqpoint{1.666713in}{1.788176in}}%
\pgfpathlineto{\pgfqpoint{1.667016in}{1.788777in}}%
\pgfpathlineto{\pgfqpoint{1.669442in}{1.791739in}}%
\pgfpathlineto{\pgfqpoint{1.670453in}{1.790995in}}%
\pgfpathlineto{\pgfqpoint{1.673082in}{1.789144in}}%
\pgfpathlineto{\pgfqpoint{1.673284in}{1.789324in}}%
\pgfpathlineto{\pgfqpoint{1.675104in}{1.789374in}}%
\pgfpathlineto{\pgfqpoint{1.680361in}{1.787055in}}%
\pgfpathlineto{\pgfqpoint{1.682382in}{1.791967in}}%
\pgfpathlineto{\pgfqpoint{1.684101in}{1.797460in}}%
\pgfpathlineto{\pgfqpoint{1.689257in}{1.796267in}}%
\pgfpathlineto{\pgfqpoint{1.690773in}{1.796255in}}%
\pgfpathlineto{\pgfqpoint{1.691481in}{1.798090in}}%
\pgfpathlineto{\pgfqpoint{1.692795in}{1.798831in}}%
\pgfpathlineto{\pgfqpoint{1.693402in}{1.798910in}}%
\pgfpathlineto{\pgfqpoint{1.693604in}{1.799716in}}%
\pgfpathlineto{\pgfqpoint{1.696434in}{1.814128in}}%
\pgfpathlineto{\pgfqpoint{1.696839in}{1.813630in}}%
\pgfpathlineto{\pgfqpoint{1.701893in}{1.806878in}}%
\pgfpathlineto{\pgfqpoint{1.702601in}{1.808971in}}%
\pgfpathlineto{\pgfqpoint{1.706139in}{1.822674in}}%
\pgfpathlineto{\pgfqpoint{1.707656in}{1.825871in}}%
\pgfpathlineto{\pgfqpoint{1.708667in}{1.823837in}}%
\pgfpathlineto{\pgfqpoint{1.712003in}{1.818578in}}%
\pgfpathlineto{\pgfqpoint{1.713014in}{1.817778in}}%
\pgfpathlineto{\pgfqpoint{1.716046in}{1.814088in}}%
\pgfpathlineto{\pgfqpoint{1.716653in}{1.815646in}}%
\pgfpathlineto{\pgfqpoint{1.717361in}{1.817181in}}%
\pgfpathlineto{\pgfqpoint{1.717967in}{1.816697in}}%
\pgfpathlineto{\pgfqpoint{1.719585in}{1.815480in}}%
\pgfpathlineto{\pgfqpoint{1.724134in}{1.808058in}}%
\pgfpathlineto{\pgfqpoint{1.726055in}{1.806941in}}%
\pgfpathlineto{\pgfqpoint{1.727369in}{1.804940in}}%
\pgfpathlineto{\pgfqpoint{1.729088in}{1.804169in}}%
\pgfpathlineto{\pgfqpoint{1.733232in}{1.801377in}}%
\pgfpathlineto{\pgfqpoint{1.735760in}{1.803401in}}%
\pgfpathlineto{\pgfqpoint{1.738287in}{1.803197in}}%
\pgfpathlineto{\pgfqpoint{1.739298in}{1.803296in}}%
\pgfpathlineto{\pgfqpoint{1.739500in}{1.803094in}}%
\pgfpathlineto{\pgfqpoint{1.740713in}{1.801966in}}%
\pgfpathlineto{\pgfqpoint{1.741219in}{1.802820in}}%
\pgfpathlineto{\pgfqpoint{1.741724in}{1.803176in}}%
\pgfpathlineto{\pgfqpoint{1.742331in}{1.802520in}}%
\pgfpathlineto{\pgfqpoint{1.744353in}{1.801494in}}%
\pgfpathlineto{\pgfqpoint{1.744555in}{1.801714in}}%
\pgfpathlineto{\pgfqpoint{1.745667in}{1.801368in}}%
\pgfpathlineto{\pgfqpoint{1.751429in}{1.797701in}}%
\pgfpathlineto{\pgfqpoint{1.753957in}{1.799464in}}%
\pgfpathlineto{\pgfqpoint{1.754967in}{1.800531in}}%
\pgfpathlineto{\pgfqpoint{1.755372in}{1.800106in}}%
\pgfpathlineto{\pgfqpoint{1.760022in}{1.795431in}}%
\pgfpathlineto{\pgfqpoint{1.762549in}{1.793595in}}%
\pgfpathlineto{\pgfqpoint{1.765178in}{1.792371in}}%
\pgfpathlineto{\pgfqpoint{1.768514in}{1.791237in}}%
\pgfpathlineto{\pgfqpoint{1.770435in}{1.790126in}}%
\pgfpathlineto{\pgfqpoint{1.774782in}{1.787146in}}%
\pgfpathlineto{\pgfqpoint{1.775085in}{1.787773in}}%
\pgfpathlineto{\pgfqpoint{1.777006in}{1.795430in}}%
\pgfpathlineto{\pgfqpoint{1.778927in}{1.804317in}}%
\pgfpathlineto{\pgfqpoint{1.780645in}{1.808865in}}%
\pgfpathlineto{\pgfqpoint{1.781454in}{1.810419in}}%
\pgfpathlineto{\pgfqpoint{1.781959in}{1.809764in}}%
\pgfpathlineto{\pgfqpoint{1.786509in}{1.803541in}}%
\pgfpathlineto{\pgfqpoint{1.786711in}{1.803712in}}%
\pgfpathlineto{\pgfqpoint{1.787823in}{1.803325in}}%
\pgfpathlineto{\pgfqpoint{1.788328in}{1.803447in}}%
\pgfpathlineto{\pgfqpoint{1.788632in}{1.804110in}}%
\pgfpathlineto{\pgfqpoint{1.789541in}{1.805413in}}%
\pgfpathlineto{\pgfqpoint{1.790148in}{1.804987in}}%
\pgfpathlineto{\pgfqpoint{1.800965in}{1.795554in}}%
\pgfpathlineto{\pgfqpoint{1.803796in}{1.797601in}}%
\pgfpathlineto{\pgfqpoint{1.805716in}{1.801032in}}%
\pgfpathlineto{\pgfqpoint{1.808345in}{1.800713in}}%
\pgfpathlineto{\pgfqpoint{1.814613in}{1.795620in}}%
\pgfpathlineto{\pgfqpoint{1.820274in}{1.790750in}}%
\pgfpathlineto{\pgfqpoint{1.822195in}{1.792176in}}%
\pgfpathlineto{\pgfqpoint{1.824014in}{1.795666in}}%
\pgfpathlineto{\pgfqpoint{1.825430in}{1.795402in}}%
\pgfpathlineto{\pgfqpoint{1.827553in}{1.795615in}}%
\pgfpathlineto{\pgfqpoint{1.828462in}{1.795353in}}%
\pgfpathlineto{\pgfqpoint{1.828563in}{1.795227in}}%
\pgfpathlineto{\pgfqpoint{1.833719in}{1.790226in}}%
\pgfpathlineto{\pgfqpoint{1.835033in}{1.791579in}}%
\pgfpathlineto{\pgfqpoint{1.835640in}{1.790981in}}%
\pgfpathlineto{\pgfqpoint{1.837359in}{1.790040in}}%
\pgfpathlineto{\pgfqpoint{1.837662in}{1.790396in}}%
\pgfpathlineto{\pgfqpoint{1.838976in}{1.790438in}}%
\pgfpathlineto{\pgfqpoint{1.846255in}{1.786157in}}%
\pgfpathlineto{\pgfqpoint{1.847670in}{1.787222in}}%
\pgfpathlineto{\pgfqpoint{1.848580in}{1.789769in}}%
\pgfpathlineto{\pgfqpoint{1.849187in}{1.789262in}}%
\pgfpathlineto{\pgfqpoint{1.850096in}{1.788771in}}%
\pgfpathlineto{\pgfqpoint{1.850501in}{1.789405in}}%
\pgfpathlineto{\pgfqpoint{1.852624in}{1.790844in}}%
\pgfpathlineto{\pgfqpoint{1.853534in}{1.792015in}}%
\pgfpathlineto{\pgfqpoint{1.854241in}{1.792285in}}%
\pgfpathlineto{\pgfqpoint{1.854747in}{1.791900in}}%
\pgfpathlineto{\pgfqpoint{1.860509in}{1.791098in}}%
\pgfpathlineto{\pgfqpoint{1.863239in}{1.795570in}}%
\pgfpathlineto{\pgfqpoint{1.863744in}{1.795206in}}%
\pgfpathlineto{\pgfqpoint{1.868293in}{1.793080in}}%
\pgfpathlineto{\pgfqpoint{1.870416in}{1.794174in}}%
\pgfpathlineto{\pgfqpoint{1.871023in}{1.795252in}}%
\pgfpathlineto{\pgfqpoint{1.871629in}{1.794620in}}%
\pgfpathlineto{\pgfqpoint{1.872944in}{1.793367in}}%
\pgfpathlineto{\pgfqpoint{1.873348in}{1.794038in}}%
\pgfpathlineto{\pgfqpoint{1.875066in}{1.795408in}}%
\pgfpathlineto{\pgfqpoint{1.876280in}{1.797598in}}%
\pgfpathlineto{\pgfqpoint{1.877189in}{1.796776in}}%
\pgfpathlineto{\pgfqpoint{1.881435in}{1.792856in}}%
\pgfpathlineto{\pgfqpoint{1.883457in}{1.792136in}}%
\pgfpathlineto{\pgfqpoint{1.885075in}{1.793029in}}%
\pgfpathlineto{\pgfqpoint{1.886793in}{1.795201in}}%
\pgfpathlineto{\pgfqpoint{1.888108in}{1.799576in}}%
\pgfpathlineto{\pgfqpoint{1.889118in}{1.798539in}}%
\pgfpathlineto{\pgfqpoint{1.889523in}{1.798467in}}%
\pgfpathlineto{\pgfqpoint{1.889826in}{1.799269in}}%
\pgfpathlineto{\pgfqpoint{1.892455in}{1.809079in}}%
\pgfpathlineto{\pgfqpoint{1.893263in}{1.808461in}}%
\pgfpathlineto{\pgfqpoint{1.899329in}{1.801365in}}%
\pgfpathlineto{\pgfqpoint{1.902968in}{1.796879in}}%
\pgfpathlineto{\pgfqpoint{1.903170in}{1.797096in}}%
\pgfpathlineto{\pgfqpoint{1.903979in}{1.797885in}}%
\pgfpathlineto{\pgfqpoint{1.904485in}{1.797292in}}%
\pgfpathlineto{\pgfqpoint{1.906304in}{1.795197in}}%
\pgfpathlineto{\pgfqpoint{1.906608in}{1.795849in}}%
\pgfpathlineto{\pgfqpoint{1.908023in}{1.802759in}}%
\pgfpathlineto{\pgfqpoint{1.909034in}{1.802419in}}%
\pgfpathlineto{\pgfqpoint{1.919548in}{1.792328in}}%
\pgfpathlineto{\pgfqpoint{1.919851in}{1.793027in}}%
\pgfpathlineto{\pgfqpoint{1.925512in}{1.819279in}}%
\pgfpathlineto{\pgfqpoint{1.925815in}{1.818930in}}%
\pgfpathlineto{\pgfqpoint{1.933498in}{1.806944in}}%
\pgfpathlineto{\pgfqpoint{1.936733in}{1.807498in}}%
\pgfpathlineto{\pgfqpoint{1.942294in}{1.802605in}}%
\pgfpathlineto{\pgfqpoint{1.944012in}{1.802392in}}%
\pgfpathlineto{\pgfqpoint{1.944113in}{1.802622in}}%
\pgfpathlineto{\pgfqpoint{1.946337in}{1.806163in}}%
\pgfpathlineto{\pgfqpoint{1.946438in}{1.806109in}}%
\pgfpathlineto{\pgfqpoint{1.953414in}{1.797975in}}%
\pgfpathlineto{\pgfqpoint{1.958469in}{1.792963in}}%
\pgfpathlineto{\pgfqpoint{1.959075in}{1.793107in}}%
\pgfpathlineto{\pgfqpoint{1.959277in}{1.793529in}}%
\pgfpathlineto{\pgfqpoint{1.960187in}{1.794910in}}%
\pgfpathlineto{\pgfqpoint{1.960794in}{1.794486in}}%
\pgfpathlineto{\pgfqpoint{1.971712in}{1.787770in}}%
\pgfpathlineto{\pgfqpoint{1.974947in}{1.786737in}}%
\pgfpathlineto{\pgfqpoint{1.977373in}{1.786905in}}%
\pgfpathlineto{\pgfqpoint{1.981619in}{1.795480in}}%
\pgfpathlineto{\pgfqpoint{1.983236in}{1.799040in}}%
\pgfpathlineto{\pgfqpoint{1.984450in}{1.798693in}}%
\pgfpathlineto{\pgfqpoint{1.985562in}{1.797873in}}%
\pgfpathlineto{\pgfqpoint{1.985966in}{1.798551in}}%
\pgfpathlineto{\pgfqpoint{1.986775in}{1.799549in}}%
\pgfpathlineto{\pgfqpoint{1.987280in}{1.799077in}}%
\pgfpathlineto{\pgfqpoint{1.989504in}{1.797161in}}%
\pgfpathlineto{\pgfqpoint{1.989706in}{1.797495in}}%
\pgfpathlineto{\pgfqpoint{1.991122in}{1.801241in}}%
\pgfpathlineto{\pgfqpoint{1.991930in}{1.800433in}}%
\pgfpathlineto{\pgfqpoint{1.995469in}{1.797827in}}%
\pgfpathlineto{\pgfqpoint{1.996075in}{1.797751in}}%
\pgfpathlineto{\pgfqpoint{1.996277in}{1.798239in}}%
\pgfpathlineto{\pgfqpoint{1.997288in}{1.801200in}}%
\pgfpathlineto{\pgfqpoint{1.998097in}{1.800678in}}%
\pgfpathlineto{\pgfqpoint{1.999715in}{1.798925in}}%
\pgfpathlineto{\pgfqpoint{2.000928in}{1.797863in}}%
\pgfpathlineto{\pgfqpoint{2.001231in}{1.798675in}}%
\pgfpathlineto{\pgfqpoint{2.002242in}{1.802399in}}%
\pgfpathlineto{\pgfqpoint{2.002950in}{1.801952in}}%
\pgfpathlineto{\pgfqpoint{2.004264in}{1.802252in}}%
\pgfpathlineto{\pgfqpoint{2.004365in}{1.802558in}}%
\pgfpathlineto{\pgfqpoint{2.006589in}{1.806453in}}%
\pgfpathlineto{\pgfqpoint{2.007600in}{1.808695in}}%
\pgfpathlineto{\pgfqpoint{2.008105in}{1.808226in}}%
\pgfpathlineto{\pgfqpoint{2.012452in}{1.803370in}}%
\pgfpathlineto{\pgfqpoint{2.014980in}{1.800169in}}%
\pgfpathlineto{\pgfqpoint{2.016901in}{1.799132in}}%
\pgfpathlineto{\pgfqpoint{2.017911in}{1.799536in}}%
\pgfpathlineto{\pgfqpoint{2.018013in}{1.799794in}}%
\pgfpathlineto{\pgfqpoint{2.019529in}{1.800864in}}%
\pgfpathlineto{\pgfqpoint{2.020540in}{1.799886in}}%
\pgfpathlineto{\pgfqpoint{2.023371in}{1.797455in}}%
\pgfpathlineto{\pgfqpoint{2.025291in}{1.799042in}}%
\pgfpathlineto{\pgfqpoint{2.027414in}{1.803128in}}%
\pgfpathlineto{\pgfqpoint{2.031256in}{1.802657in}}%
\pgfpathlineto{\pgfqpoint{2.031761in}{1.802664in}}%
\pgfpathlineto{\pgfqpoint{2.032166in}{1.802093in}}%
\pgfpathlineto{\pgfqpoint{2.033076in}{1.801343in}}%
\pgfpathlineto{\pgfqpoint{2.033480in}{1.801840in}}%
\pgfpathlineto{\pgfqpoint{2.034491in}{1.802739in}}%
\pgfpathlineto{\pgfqpoint{2.034996in}{1.802371in}}%
\pgfpathlineto{\pgfqpoint{2.036816in}{1.800319in}}%
\pgfpathlineto{\pgfqpoint{2.045207in}{1.792361in}}%
\pgfpathlineto{\pgfqpoint{2.047026in}{1.791775in}}%
\pgfpathlineto{\pgfqpoint{2.050059in}{1.789264in}}%
\pgfpathlineto{\pgfqpoint{2.050261in}{1.789556in}}%
\pgfpathlineto{\pgfqpoint{2.054103in}{1.797718in}}%
\pgfpathlineto{\pgfqpoint{2.055316in}{1.796706in}}%
\pgfpathlineto{\pgfqpoint{2.061179in}{1.791584in}}%
\pgfpathlineto{\pgfqpoint{2.062797in}{1.791893in}}%
\pgfpathlineto{\pgfqpoint{2.065526in}{1.794762in}}%
\pgfpathlineto{\pgfqpoint{2.065628in}{1.794689in}}%
\pgfpathlineto{\pgfqpoint{2.072300in}{1.789378in}}%
\pgfpathlineto{\pgfqpoint{2.072603in}{1.789892in}}%
\pgfpathlineto{\pgfqpoint{2.073614in}{1.791001in}}%
\pgfpathlineto{\pgfqpoint{2.074119in}{1.790704in}}%
\pgfpathlineto{\pgfqpoint{2.080387in}{1.787383in}}%
\pgfpathlineto{\pgfqpoint{2.086453in}{1.789807in}}%
\pgfpathlineto{\pgfqpoint{2.094742in}{1.786675in}}%
\pgfpathlineto{\pgfqpoint{2.096663in}{1.788778in}}%
\pgfpathlineto{\pgfqpoint{2.100202in}{1.800796in}}%
\pgfpathlineto{\pgfqpoint{2.100808in}{1.800200in}}%
\pgfpathlineto{\pgfqpoint{2.104852in}{1.797709in}}%
\pgfpathlineto{\pgfqpoint{2.107278in}{1.796500in}}%
\pgfpathlineto{\pgfqpoint{2.107480in}{1.796848in}}%
\pgfpathlineto{\pgfqpoint{2.108592in}{1.798552in}}%
\pgfpathlineto{\pgfqpoint{2.109199in}{1.797985in}}%
\pgfpathlineto{\pgfqpoint{2.110614in}{1.796724in}}%
\pgfpathlineto{\pgfqpoint{2.110917in}{1.797465in}}%
\pgfpathlineto{\pgfqpoint{2.113647in}{1.803507in}}%
\pgfpathlineto{\pgfqpoint{2.115871in}{1.805812in}}%
\pgfpathlineto{\pgfqpoint{2.116478in}{1.804837in}}%
\pgfpathlineto{\pgfqpoint{2.119611in}{1.801641in}}%
\pgfpathlineto{\pgfqpoint{2.122442in}{1.800303in}}%
\pgfpathlineto{\pgfqpoint{2.126688in}{1.796848in}}%
\pgfpathlineto{\pgfqpoint{2.127800in}{1.796604in}}%
\pgfpathlineto{\pgfqpoint{2.128103in}{1.797180in}}%
\pgfpathlineto{\pgfqpoint{2.128710in}{1.797586in}}%
\pgfpathlineto{\pgfqpoint{2.129215in}{1.797119in}}%
\pgfpathlineto{\pgfqpoint{2.137909in}{1.788710in}}%
\pgfpathlineto{\pgfqpoint{2.142560in}{1.786430in}}%
\pgfpathlineto{\pgfqpoint{2.144986in}{1.788217in}}%
\pgfpathlineto{\pgfqpoint{2.148221in}{1.798050in}}%
\pgfpathlineto{\pgfqpoint{2.148726in}{1.797625in}}%
\pgfpathlineto{\pgfqpoint{2.149940in}{1.796942in}}%
\pgfpathlineto{\pgfqpoint{2.150344in}{1.797473in}}%
\pgfpathlineto{\pgfqpoint{2.150950in}{1.797516in}}%
\pgfpathlineto{\pgfqpoint{2.151355in}{1.797031in}}%
\pgfpathlineto{\pgfqpoint{2.155196in}{1.792968in}}%
\pgfpathlineto{\pgfqpoint{2.155500in}{1.793731in}}%
\pgfpathlineto{\pgfqpoint{2.156713in}{1.798799in}}%
\pgfpathlineto{\pgfqpoint{2.157522in}{1.798096in}}%
\pgfpathlineto{\pgfqpoint{2.158431in}{1.798908in}}%
\pgfpathlineto{\pgfqpoint{2.159038in}{1.799333in}}%
\pgfpathlineto{\pgfqpoint{2.159543in}{1.798784in}}%
\pgfpathlineto{\pgfqpoint{2.162475in}{1.796429in}}%
\pgfpathlineto{\pgfqpoint{2.163587in}{1.795776in}}%
\pgfpathlineto{\pgfqpoint{2.167226in}{1.791433in}}%
\pgfpathlineto{\pgfqpoint{2.167530in}{1.791977in}}%
\pgfpathlineto{\pgfqpoint{2.169653in}{1.796322in}}%
\pgfpathlineto{\pgfqpoint{2.169754in}{1.796295in}}%
\pgfpathlineto{\pgfqpoint{2.172888in}{1.793689in}}%
\pgfpathlineto{\pgfqpoint{2.176628in}{1.790853in}}%
\pgfpathlineto{\pgfqpoint{2.177841in}{1.790503in}}%
\pgfpathlineto{\pgfqpoint{2.179358in}{1.789349in}}%
\pgfpathlineto{\pgfqpoint{2.179762in}{1.790161in}}%
\pgfpathlineto{\pgfqpoint{2.181885in}{1.793077in}}%
\pgfpathlineto{\pgfqpoint{2.182087in}{1.792968in}}%
\pgfpathlineto{\pgfqpoint{2.188456in}{1.789254in}}%
\pgfpathlineto{\pgfqpoint{2.196847in}{1.788336in}}%
\pgfpathlineto{\pgfqpoint{2.200385in}{1.795768in}}%
\pgfpathlineto{\pgfqpoint{2.200587in}{1.795623in}}%
\pgfpathlineto{\pgfqpoint{2.203216in}{1.794190in}}%
\pgfpathlineto{\pgfqpoint{2.203721in}{1.796220in}}%
\pgfpathlineto{\pgfqpoint{2.206046in}{1.807756in}}%
\pgfpathlineto{\pgfqpoint{2.207866in}{1.810530in}}%
\pgfpathlineto{\pgfqpoint{2.207967in}{1.810500in}}%
\pgfpathlineto{\pgfqpoint{2.215246in}{1.804965in}}%
\pgfpathlineto{\pgfqpoint{2.215751in}{1.806449in}}%
\pgfpathlineto{\pgfqpoint{2.216459in}{1.807682in}}%
\pgfpathlineto{\pgfqpoint{2.217066in}{1.807028in}}%
\pgfpathlineto{\pgfqpoint{2.220604in}{1.803735in}}%
\pgfpathlineto{\pgfqpoint{2.222525in}{1.803649in}}%
\pgfpathlineto{\pgfqpoint{2.223030in}{1.803477in}}%
\pgfpathlineto{\pgfqpoint{2.223434in}{1.804296in}}%
\pgfpathlineto{\pgfqpoint{2.224749in}{1.807691in}}%
\pgfpathlineto{\pgfqpoint{2.225355in}{1.806990in}}%
\pgfpathlineto{\pgfqpoint{2.228388in}{1.803484in}}%
\pgfpathlineto{\pgfqpoint{2.228691in}{1.803965in}}%
\pgfpathlineto{\pgfqpoint{2.229298in}{1.804550in}}%
\pgfpathlineto{\pgfqpoint{2.229904in}{1.804098in}}%
\pgfpathlineto{\pgfqpoint{2.231724in}{1.802234in}}%
\pgfpathlineto{\pgfqpoint{2.233645in}{1.801485in}}%
\pgfpathlineto{\pgfqpoint{2.234150in}{1.802428in}}%
\pgfpathlineto{\pgfqpoint{2.235363in}{1.806538in}}%
\pgfpathlineto{\pgfqpoint{2.236071in}{1.806074in}}%
\pgfpathlineto{\pgfqpoint{2.237486in}{1.805441in}}%
\pgfpathlineto{\pgfqpoint{2.238396in}{1.804604in}}%
\pgfpathlineto{\pgfqpoint{2.238700in}{1.805356in}}%
\pgfpathlineto{\pgfqpoint{2.240620in}{1.809123in}}%
\pgfpathlineto{\pgfqpoint{2.241530in}{1.808138in}}%
\pgfpathlineto{\pgfqpoint{2.246686in}{1.802065in}}%
\pgfpathlineto{\pgfqpoint{2.249112in}{1.800174in}}%
\pgfpathlineto{\pgfqpoint{2.249618in}{1.801133in}}%
\pgfpathlineto{\pgfqpoint{2.252853in}{1.810601in}}%
\pgfpathlineto{\pgfqpoint{2.253864in}{1.809010in}}%
\pgfpathlineto{\pgfqpoint{2.257705in}{1.803917in}}%
\pgfpathlineto{\pgfqpoint{2.258211in}{1.805205in}}%
\pgfpathlineto{\pgfqpoint{2.259120in}{1.808291in}}%
\pgfpathlineto{\pgfqpoint{2.259828in}{1.807687in}}%
\pgfpathlineto{\pgfqpoint{2.262153in}{1.806557in}}%
\pgfpathlineto{\pgfqpoint{2.264782in}{1.803880in}}%
\pgfpathlineto{\pgfqpoint{2.265995in}{1.803711in}}%
\pgfpathlineto{\pgfqpoint{2.266197in}{1.804310in}}%
\pgfpathlineto{\pgfqpoint{2.267107in}{1.806205in}}%
\pgfpathlineto{\pgfqpoint{2.267612in}{1.805649in}}%
\pgfpathlineto{\pgfqpoint{2.269432in}{1.804210in}}%
\pgfpathlineto{\pgfqpoint{2.269634in}{1.804449in}}%
\pgfpathlineto{\pgfqpoint{2.271757in}{1.806495in}}%
\pgfpathlineto{\pgfqpoint{2.273274in}{1.806942in}}%
\pgfpathlineto{\pgfqpoint{2.273375in}{1.806841in}}%
\pgfpathlineto{\pgfqpoint{2.289550in}{1.788748in}}%
\pgfpathlineto{\pgfqpoint{2.292785in}{1.786836in}}%
\pgfpathlineto{\pgfqpoint{2.294200in}{1.786308in}}%
\pgfpathlineto{\pgfqpoint{2.296929in}{1.784546in}}%
\pgfpathlineto{\pgfqpoint{2.297233in}{1.785056in}}%
\pgfpathlineto{\pgfqpoint{2.298951in}{1.786227in}}%
\pgfpathlineto{\pgfqpoint{2.300265in}{1.785623in}}%
\pgfpathlineto{\pgfqpoint{2.302388in}{1.785169in}}%
\pgfpathlineto{\pgfqpoint{2.304714in}{1.786193in}}%
\pgfpathlineto{\pgfqpoint{2.310577in}{1.784593in}}%
\pgfpathlineto{\pgfqpoint{2.312296in}{1.784611in}}%
\pgfpathlineto{\pgfqpoint{2.313509in}{1.784308in}}%
\pgfpathlineto{\pgfqpoint{2.318968in}{1.781117in}}%
\pgfpathlineto{\pgfqpoint{2.322809in}{1.779980in}}%
\pgfpathlineto{\pgfqpoint{2.325539in}{1.778734in}}%
\pgfpathlineto{\pgfqpoint{2.327864in}{1.778642in}}%
\pgfpathlineto{\pgfqpoint{2.329481in}{1.779328in}}%
\pgfpathlineto{\pgfqpoint{2.334132in}{1.777955in}}%
\pgfpathlineto{\pgfqpoint{2.334940in}{1.780213in}}%
\pgfpathlineto{\pgfqpoint{2.337266in}{1.785676in}}%
\pgfpathlineto{\pgfqpoint{2.337872in}{1.785395in}}%
\pgfpathlineto{\pgfqpoint{2.339894in}{1.785941in}}%
\pgfpathlineto{\pgfqpoint{2.341410in}{1.786476in}}%
\pgfpathlineto{\pgfqpoint{2.342927in}{1.786840in}}%
\pgfpathlineto{\pgfqpoint{2.345050in}{1.785848in}}%
\pgfpathlineto{\pgfqpoint{2.346971in}{1.785079in}}%
\pgfpathlineto{\pgfqpoint{2.347274in}{1.785473in}}%
\pgfpathlineto{\pgfqpoint{2.349498in}{1.786281in}}%
\pgfpathlineto{\pgfqpoint{2.356574in}{1.782427in}}%
\pgfpathlineto{\pgfqpoint{2.358394in}{1.785125in}}%
\pgfpathlineto{\pgfqpoint{2.359911in}{1.790679in}}%
\pgfpathlineto{\pgfqpoint{2.360315in}{1.790403in}}%
\pgfpathlineto{\pgfqpoint{2.361831in}{1.789324in}}%
\pgfpathlineto{\pgfqpoint{2.362135in}{1.790139in}}%
\pgfpathlineto{\pgfqpoint{2.363146in}{1.793599in}}%
\pgfpathlineto{\pgfqpoint{2.363853in}{1.792929in}}%
\pgfpathlineto{\pgfqpoint{2.367493in}{1.790301in}}%
\pgfpathlineto{\pgfqpoint{2.369413in}{1.789664in}}%
\pgfpathlineto{\pgfqpoint{2.373760in}{1.787308in}}%
\pgfpathlineto{\pgfqpoint{2.378916in}{1.785285in}}%
\pgfpathlineto{\pgfqpoint{2.382454in}{1.786416in}}%
\pgfpathlineto{\pgfqpoint{2.385184in}{1.785825in}}%
\pgfpathlineto{\pgfqpoint{2.385285in}{1.786038in}}%
\pgfpathlineto{\pgfqpoint{2.385993in}{1.793119in}}%
\pgfpathlineto{\pgfqpoint{2.387610in}{1.800503in}}%
\pgfpathlineto{\pgfqpoint{2.389430in}{1.802518in}}%
\pgfpathlineto{\pgfqpoint{2.390744in}{1.800818in}}%
\pgfpathlineto{\pgfqpoint{2.392867in}{1.798183in}}%
\pgfpathlineto{\pgfqpoint{2.393170in}{1.798702in}}%
\pgfpathlineto{\pgfqpoint{2.394181in}{1.805322in}}%
\pgfpathlineto{\pgfqpoint{2.397113in}{1.821275in}}%
\pgfpathlineto{\pgfqpoint{2.397922in}{1.819987in}}%
\pgfpathlineto{\pgfqpoint{2.402875in}{1.812977in}}%
\pgfpathlineto{\pgfqpoint{2.403886in}{1.811399in}}%
\pgfpathlineto{\pgfqpoint{2.405605in}{1.808624in}}%
\pgfpathlineto{\pgfqpoint{2.406009in}{1.809205in}}%
\pgfpathlineto{\pgfqpoint{2.406515in}{1.809459in}}%
\pgfpathlineto{\pgfqpoint{2.407020in}{1.808863in}}%
\pgfpathlineto{\pgfqpoint{2.410053in}{1.805072in}}%
\pgfpathlineto{\pgfqpoint{2.410255in}{1.805544in}}%
\pgfpathlineto{\pgfqpoint{2.411670in}{1.814230in}}%
\pgfpathlineto{\pgfqpoint{2.412884in}{1.813177in}}%
\pgfpathlineto{\pgfqpoint{2.414905in}{1.810260in}}%
\pgfpathlineto{\pgfqpoint{2.415512in}{1.810610in}}%
\pgfpathlineto{\pgfqpoint{2.416523in}{1.809399in}}%
\pgfpathlineto{\pgfqpoint{2.423094in}{1.801012in}}%
\pgfpathlineto{\pgfqpoint{2.424610in}{1.800977in}}%
\pgfpathlineto{\pgfqpoint{2.424711in}{1.801226in}}%
\pgfpathlineto{\pgfqpoint{2.426430in}{1.802869in}}%
\pgfpathlineto{\pgfqpoint{2.427239in}{1.803767in}}%
\pgfpathlineto{\pgfqpoint{2.429160in}{1.807813in}}%
\pgfpathlineto{\pgfqpoint{2.430171in}{1.807012in}}%
\pgfpathlineto{\pgfqpoint{2.431181in}{1.805861in}}%
\pgfpathlineto{\pgfqpoint{2.431586in}{1.806463in}}%
\pgfpathlineto{\pgfqpoint{2.433608in}{1.813101in}}%
\pgfpathlineto{\pgfqpoint{2.434315in}{1.815309in}}%
\pgfpathlineto{\pgfqpoint{2.434922in}{1.814467in}}%
\pgfpathlineto{\pgfqpoint{2.444728in}{1.802759in}}%
\pgfpathlineto{\pgfqpoint{2.446244in}{1.801770in}}%
\pgfpathlineto{\pgfqpoint{2.448165in}{1.799111in}}%
\pgfpathlineto{\pgfqpoint{2.448468in}{1.799840in}}%
\pgfpathlineto{\pgfqpoint{2.450389in}{1.805019in}}%
\pgfpathlineto{\pgfqpoint{2.450591in}{1.804929in}}%
\pgfpathlineto{\pgfqpoint{2.453321in}{1.801804in}}%
\pgfpathlineto{\pgfqpoint{2.453826in}{1.803447in}}%
\pgfpathlineto{\pgfqpoint{2.454534in}{1.805303in}}%
\pgfpathlineto{\pgfqpoint{2.455141in}{1.804584in}}%
\pgfpathlineto{\pgfqpoint{2.459387in}{1.798751in}}%
\pgfpathlineto{\pgfqpoint{2.459690in}{1.799137in}}%
\pgfpathlineto{\pgfqpoint{2.461611in}{1.800817in}}%
\pgfpathlineto{\pgfqpoint{2.462217in}{1.800965in}}%
\pgfpathlineto{\pgfqpoint{2.462622in}{1.800437in}}%
\pgfpathlineto{\pgfqpoint{2.465452in}{1.797983in}}%
\pgfpathlineto{\pgfqpoint{2.466969in}{1.798449in}}%
\pgfpathlineto{\pgfqpoint{2.470001in}{1.809933in}}%
\pgfpathlineto{\pgfqpoint{2.470507in}{1.810121in}}%
\pgfpathlineto{\pgfqpoint{2.471012in}{1.809497in}}%
\pgfpathlineto{\pgfqpoint{2.475561in}{1.805855in}}%
\pgfpathlineto{\pgfqpoint{2.475663in}{1.805999in}}%
\pgfpathlineto{\pgfqpoint{2.476775in}{1.807731in}}%
\pgfpathlineto{\pgfqpoint{2.477280in}{1.807075in}}%
\pgfpathlineto{\pgfqpoint{2.479605in}{1.804386in}}%
\pgfpathlineto{\pgfqpoint{2.479807in}{1.804531in}}%
\pgfpathlineto{\pgfqpoint{2.480616in}{1.804676in}}%
\pgfpathlineto{\pgfqpoint{2.481021in}{1.804155in}}%
\pgfpathlineto{\pgfqpoint{2.482234in}{1.802925in}}%
\pgfpathlineto{\pgfqpoint{2.482739in}{1.803585in}}%
\pgfpathlineto{\pgfqpoint{2.483952in}{1.803260in}}%
\pgfpathlineto{\pgfqpoint{2.485368in}{1.802301in}}%
\pgfpathlineto{\pgfqpoint{2.488097in}{1.799114in}}%
\pgfpathlineto{\pgfqpoint{2.488400in}{1.799611in}}%
\pgfpathlineto{\pgfqpoint{2.489512in}{1.802663in}}%
\pgfpathlineto{\pgfqpoint{2.490220in}{1.802003in}}%
\pgfpathlineto{\pgfqpoint{2.492040in}{1.800018in}}%
\pgfpathlineto{\pgfqpoint{2.492343in}{1.800789in}}%
\pgfpathlineto{\pgfqpoint{2.494567in}{1.805579in}}%
\pgfpathlineto{\pgfqpoint{2.496387in}{1.805401in}}%
\pgfpathlineto{\pgfqpoint{2.496993in}{1.805454in}}%
\pgfpathlineto{\pgfqpoint{2.497297in}{1.806012in}}%
\pgfpathlineto{\pgfqpoint{2.498004in}{1.806846in}}%
\pgfpathlineto{\pgfqpoint{2.498510in}{1.806285in}}%
\pgfpathlineto{\pgfqpoint{2.499824in}{1.804821in}}%
\pgfpathlineto{\pgfqpoint{2.500329in}{1.805388in}}%
\pgfpathlineto{\pgfqpoint{2.501542in}{1.806942in}}%
\pgfpathlineto{\pgfqpoint{2.502351in}{1.808715in}}%
\pgfpathlineto{\pgfqpoint{2.502958in}{1.808055in}}%
\pgfpathlineto{\pgfqpoint{2.506900in}{1.803810in}}%
\pgfpathlineto{\pgfqpoint{2.507305in}{1.803804in}}%
\pgfpathlineto{\pgfqpoint{2.507507in}{1.804336in}}%
\pgfpathlineto{\pgfqpoint{2.509529in}{1.807978in}}%
\pgfpathlineto{\pgfqpoint{2.510338in}{1.807610in}}%
\pgfpathlineto{\pgfqpoint{2.510439in}{1.807448in}}%
\pgfpathlineto{\pgfqpoint{2.515291in}{1.801318in}}%
\pgfpathlineto{\pgfqpoint{2.517920in}{1.800504in}}%
\pgfpathlineto{\pgfqpoint{2.527018in}{1.791967in}}%
\pgfpathlineto{\pgfqpoint{2.527422in}{1.792922in}}%
\pgfpathlineto{\pgfqpoint{2.528433in}{1.795196in}}%
\pgfpathlineto{\pgfqpoint{2.528939in}{1.794717in}}%
\pgfpathlineto{\pgfqpoint{2.531769in}{1.792618in}}%
\pgfpathlineto{\pgfqpoint{2.532073in}{1.792905in}}%
\pgfpathlineto{\pgfqpoint{2.535611in}{1.793851in}}%
\pgfpathlineto{\pgfqpoint{2.539453in}{1.790372in}}%
\pgfpathlineto{\pgfqpoint{2.539857in}{1.791242in}}%
\pgfpathlineto{\pgfqpoint{2.540767in}{1.793331in}}%
\pgfpathlineto{\pgfqpoint{2.541373in}{1.792818in}}%
\pgfpathlineto{\pgfqpoint{2.542283in}{1.792786in}}%
\pgfpathlineto{\pgfqpoint{2.542485in}{1.793106in}}%
\pgfpathlineto{\pgfqpoint{2.544507in}{1.795008in}}%
\pgfpathlineto{\pgfqpoint{2.544608in}{1.794956in}}%
\pgfpathlineto{\pgfqpoint{2.547843in}{1.792865in}}%
\pgfpathlineto{\pgfqpoint{2.548147in}{1.793629in}}%
\pgfpathlineto{\pgfqpoint{2.549360in}{1.797887in}}%
\pgfpathlineto{\pgfqpoint{2.550168in}{1.797552in}}%
\pgfpathlineto{\pgfqpoint{2.551887in}{1.797203in}}%
\pgfpathlineto{\pgfqpoint{2.552696in}{1.796701in}}%
\pgfpathlineto{\pgfqpoint{2.553100in}{1.797189in}}%
\pgfpathlineto{\pgfqpoint{2.556032in}{1.799491in}}%
\pgfpathlineto{\pgfqpoint{2.557953in}{1.798805in}}%
\pgfpathlineto{\pgfqpoint{2.558155in}{1.799106in}}%
\pgfpathlineto{\pgfqpoint{2.559469in}{1.799210in}}%
\pgfpathlineto{\pgfqpoint{2.560985in}{1.797622in}}%
\pgfpathlineto{\pgfqpoint{2.562603in}{1.797416in}}%
\pgfpathlineto{\pgfqpoint{2.564625in}{1.798773in}}%
\pgfpathlineto{\pgfqpoint{2.567354in}{1.806766in}}%
\pgfpathlineto{\pgfqpoint{2.568163in}{1.805456in}}%
\pgfpathlineto{\pgfqpoint{2.569679in}{1.803397in}}%
\pgfpathlineto{\pgfqpoint{2.570185in}{1.803882in}}%
\pgfpathlineto{\pgfqpoint{2.571196in}{1.803257in}}%
\pgfpathlineto{\pgfqpoint{2.573218in}{1.800976in}}%
\pgfpathlineto{\pgfqpoint{2.573824in}{1.801755in}}%
\pgfpathlineto{\pgfqpoint{2.574532in}{1.802661in}}%
\pgfpathlineto{\pgfqpoint{2.576149in}{1.811730in}}%
\pgfpathlineto{\pgfqpoint{2.577160in}{1.810460in}}%
\pgfpathlineto{\pgfqpoint{2.579789in}{1.808085in}}%
\pgfpathlineto{\pgfqpoint{2.581507in}{1.805778in}}%
\pgfpathlineto{\pgfqpoint{2.582013in}{1.806335in}}%
\pgfpathlineto{\pgfqpoint{2.582619in}{1.806258in}}%
\pgfpathlineto{\pgfqpoint{2.583024in}{1.805783in}}%
\pgfpathlineto{\pgfqpoint{2.588483in}{1.800805in}}%
\pgfpathlineto{\pgfqpoint{2.588685in}{1.801354in}}%
\pgfpathlineto{\pgfqpoint{2.591111in}{1.807488in}}%
\pgfpathlineto{\pgfqpoint{2.592021in}{1.806437in}}%
\pgfpathlineto{\pgfqpoint{2.600007in}{1.797031in}}%
\pgfpathlineto{\pgfqpoint{2.603748in}{1.794099in}}%
\pgfpathlineto{\pgfqpoint{2.603849in}{1.794220in}}%
\pgfpathlineto{\pgfqpoint{2.606073in}{1.795609in}}%
\pgfpathlineto{\pgfqpoint{2.607488in}{1.795687in}}%
\pgfpathlineto{\pgfqpoint{2.610824in}{1.795492in}}%
\pgfpathlineto{\pgfqpoint{2.612947in}{1.794996in}}%
\pgfpathlineto{\pgfqpoint{2.614262in}{1.796340in}}%
\pgfpathlineto{\pgfqpoint{2.615171in}{1.798056in}}%
\pgfpathlineto{\pgfqpoint{2.615677in}{1.797594in}}%
\pgfpathlineto{\pgfqpoint{2.616486in}{1.797322in}}%
\pgfpathlineto{\pgfqpoint{2.616789in}{1.797764in}}%
\pgfpathlineto{\pgfqpoint{2.617699in}{1.798835in}}%
\pgfpathlineto{\pgfqpoint{2.618204in}{1.798255in}}%
\pgfpathlineto{\pgfqpoint{2.621439in}{1.795314in}}%
\pgfpathlineto{\pgfqpoint{2.625079in}{1.795433in}}%
\pgfpathlineto{\pgfqpoint{2.627101in}{1.798413in}}%
\pgfpathlineto{\pgfqpoint{2.628415in}{1.798268in}}%
\pgfpathlineto{\pgfqpoint{2.628617in}{1.799002in}}%
\pgfpathlineto{\pgfqpoint{2.631144in}{1.806697in}}%
\pgfpathlineto{\pgfqpoint{2.631245in}{1.806655in}}%
\pgfpathlineto{\pgfqpoint{2.636805in}{1.803103in}}%
\pgfpathlineto{\pgfqpoint{2.637008in}{1.803531in}}%
\pgfpathlineto{\pgfqpoint{2.638322in}{1.808797in}}%
\pgfpathlineto{\pgfqpoint{2.639333in}{1.808235in}}%
\pgfpathlineto{\pgfqpoint{2.639939in}{1.807782in}}%
\pgfpathlineto{\pgfqpoint{2.640243in}{1.808557in}}%
\pgfpathlineto{\pgfqpoint{2.642871in}{1.818934in}}%
\pgfpathlineto{\pgfqpoint{2.643275in}{1.818415in}}%
\pgfpathlineto{\pgfqpoint{2.645803in}{1.813959in}}%
\pgfpathlineto{\pgfqpoint{2.646409in}{1.814385in}}%
\pgfpathlineto{\pgfqpoint{2.651464in}{1.811977in}}%
\pgfpathlineto{\pgfqpoint{2.652677in}{1.810413in}}%
\pgfpathlineto{\pgfqpoint{2.653284in}{1.810963in}}%
\pgfpathlineto{\pgfqpoint{2.654295in}{1.811818in}}%
\pgfpathlineto{\pgfqpoint{2.656215in}{1.814078in}}%
\pgfpathlineto{\pgfqpoint{2.657530in}{1.814836in}}%
\pgfpathlineto{\pgfqpoint{2.659046in}{1.816060in}}%
\pgfpathlineto{\pgfqpoint{2.659552in}{1.816110in}}%
\pgfpathlineto{\pgfqpoint{2.659956in}{1.815517in}}%
\pgfpathlineto{\pgfqpoint{2.665415in}{1.808215in}}%
\pgfpathlineto{\pgfqpoint{2.665516in}{1.808343in}}%
\pgfpathlineto{\pgfqpoint{2.667538in}{1.810383in}}%
\pgfpathlineto{\pgfqpoint{2.667639in}{1.810325in}}%
\pgfpathlineto{\pgfqpoint{2.670975in}{1.806308in}}%
\pgfpathlineto{\pgfqpoint{2.671582in}{1.808074in}}%
\pgfpathlineto{\pgfqpoint{2.672087in}{1.808984in}}%
\pgfpathlineto{\pgfqpoint{2.672694in}{1.808251in}}%
\pgfpathlineto{\pgfqpoint{2.677748in}{1.801626in}}%
\pgfpathlineto{\pgfqpoint{2.678052in}{1.801947in}}%
\pgfpathlineto{\pgfqpoint{2.679669in}{1.802344in}}%
\pgfpathlineto{\pgfqpoint{2.683308in}{1.806856in}}%
\pgfpathlineto{\pgfqpoint{2.684117in}{1.806077in}}%
\pgfpathlineto{\pgfqpoint{2.689677in}{1.799893in}}%
\pgfpathlineto{\pgfqpoint{2.689778in}{1.799970in}}%
\pgfpathlineto{\pgfqpoint{2.691396in}{1.799965in}}%
\pgfpathlineto{\pgfqpoint{2.692407in}{1.800729in}}%
\pgfpathlineto{\pgfqpoint{2.693317in}{1.799886in}}%
\pgfpathlineto{\pgfqpoint{2.695035in}{1.797684in}}%
\pgfpathlineto{\pgfqpoint{2.695440in}{1.798316in}}%
\pgfpathlineto{\pgfqpoint{2.697765in}{1.801289in}}%
\pgfpathlineto{\pgfqpoint{2.698371in}{1.802310in}}%
\pgfpathlineto{\pgfqpoint{2.700191in}{1.805484in}}%
\pgfpathlineto{\pgfqpoint{2.701101in}{1.804851in}}%
\pgfpathlineto{\pgfqpoint{2.702314in}{1.803613in}}%
\pgfpathlineto{\pgfqpoint{2.702617in}{1.804419in}}%
\pgfpathlineto{\pgfqpoint{2.704841in}{1.809456in}}%
\pgfpathlineto{\pgfqpoint{2.707268in}{1.810139in}}%
\pgfpathlineto{\pgfqpoint{2.708177in}{1.813676in}}%
\pgfpathlineto{\pgfqpoint{2.708784in}{1.813289in}}%
\pgfpathlineto{\pgfqpoint{2.710300in}{1.811406in}}%
\pgfpathlineto{\pgfqpoint{2.710806in}{1.813198in}}%
\pgfpathlineto{\pgfqpoint{2.711514in}{1.814873in}}%
\pgfpathlineto{\pgfqpoint{2.712221in}{1.814417in}}%
\pgfpathlineto{\pgfqpoint{2.717680in}{1.816174in}}%
\pgfpathlineto{\pgfqpoint{2.722634in}{1.809634in}}%
\pgfpathlineto{\pgfqpoint{2.722735in}{1.809714in}}%
\pgfpathlineto{\pgfqpoint{2.723443in}{1.810091in}}%
\pgfpathlineto{\pgfqpoint{2.723847in}{1.809538in}}%
\pgfpathlineto{\pgfqpoint{2.726273in}{1.807343in}}%
\pgfpathlineto{\pgfqpoint{2.726779in}{1.808209in}}%
\pgfpathlineto{\pgfqpoint{2.728801in}{1.812102in}}%
\pgfpathlineto{\pgfqpoint{2.730115in}{1.814849in}}%
\pgfpathlineto{\pgfqpoint{2.732339in}{1.827455in}}%
\pgfpathlineto{\pgfqpoint{2.732642in}{1.827662in}}%
\pgfpathlineto{\pgfqpoint{2.733046in}{1.826872in}}%
\pgfpathlineto{\pgfqpoint{2.737596in}{1.817585in}}%
\pgfpathlineto{\pgfqpoint{2.738202in}{1.819497in}}%
\pgfpathlineto{\pgfqpoint{2.739213in}{1.823116in}}%
\pgfpathlineto{\pgfqpoint{2.739719in}{1.822331in}}%
\pgfpathlineto{\pgfqpoint{2.741639in}{1.820474in}}%
\pgfpathlineto{\pgfqpoint{2.742549in}{1.819512in}}%
\pgfpathlineto{\pgfqpoint{2.744672in}{1.816260in}}%
\pgfpathlineto{\pgfqpoint{2.745077in}{1.816770in}}%
\pgfpathlineto{\pgfqpoint{2.747705in}{1.819623in}}%
\pgfpathlineto{\pgfqpoint{2.747907in}{1.819369in}}%
\pgfpathlineto{\pgfqpoint{2.753771in}{1.810555in}}%
\pgfpathlineto{\pgfqpoint{2.754074in}{1.810950in}}%
\pgfpathlineto{\pgfqpoint{2.755590in}{1.811667in}}%
\pgfpathlineto{\pgfqpoint{2.756399in}{1.811865in}}%
\pgfpathlineto{\pgfqpoint{2.756803in}{1.811233in}}%
\pgfpathlineto{\pgfqpoint{2.759634in}{1.807068in}}%
\pgfpathlineto{\pgfqpoint{2.760038in}{1.807849in}}%
\pgfpathlineto{\pgfqpoint{2.762161in}{1.816798in}}%
\pgfpathlineto{\pgfqpoint{2.763476in}{1.814675in}}%
\pgfpathlineto{\pgfqpoint{2.769238in}{1.804845in}}%
\pgfpathlineto{\pgfqpoint{2.771967in}{1.801968in}}%
\pgfpathlineto{\pgfqpoint{2.773383in}{1.800154in}}%
\pgfpathlineto{\pgfqpoint{2.773787in}{1.800923in}}%
\pgfpathlineto{\pgfqpoint{2.774495in}{1.801937in}}%
\pgfpathlineto{\pgfqpoint{2.775101in}{1.801481in}}%
\pgfpathlineto{\pgfqpoint{2.779752in}{1.801919in}}%
\pgfpathlineto{\pgfqpoint{2.780763in}{1.804733in}}%
\pgfpathlineto{\pgfqpoint{2.781369in}{1.804085in}}%
\pgfpathlineto{\pgfqpoint{2.783998in}{1.801966in}}%
\pgfpathlineto{\pgfqpoint{2.784099in}{1.802039in}}%
\pgfpathlineto{\pgfqpoint{2.785110in}{1.803063in}}%
\pgfpathlineto{\pgfqpoint{2.785615in}{1.802388in}}%
\pgfpathlineto{\pgfqpoint{2.787839in}{1.800937in}}%
\pgfpathlineto{\pgfqpoint{2.790670in}{1.800722in}}%
\pgfpathlineto{\pgfqpoint{2.792995in}{1.799219in}}%
\pgfpathlineto{\pgfqpoint{2.793197in}{1.799484in}}%
\pgfpathlineto{\pgfqpoint{2.795320in}{1.804666in}}%
\pgfpathlineto{\pgfqpoint{2.796331in}{1.809948in}}%
\pgfpathlineto{\pgfqpoint{2.796937in}{1.809435in}}%
\pgfpathlineto{\pgfqpoint{2.798555in}{1.807702in}}%
\pgfpathlineto{\pgfqpoint{2.803306in}{1.800933in}}%
\pgfpathlineto{\pgfqpoint{2.805530in}{1.797976in}}%
\pgfpathlineto{\pgfqpoint{2.812506in}{1.790873in}}%
\pgfpathlineto{\pgfqpoint{2.812809in}{1.791305in}}%
\pgfpathlineto{\pgfqpoint{2.815134in}{1.792760in}}%
\pgfpathlineto{\pgfqpoint{2.816449in}{1.791561in}}%
\pgfpathlineto{\pgfqpoint{2.821099in}{1.788248in}}%
\pgfpathlineto{\pgfqpoint{2.823525in}{1.789391in}}%
\pgfpathlineto{\pgfqpoint{2.824940in}{1.788610in}}%
\pgfpathlineto{\pgfqpoint{2.828276in}{1.787059in}}%
\pgfpathlineto{\pgfqpoint{2.831815in}{1.785692in}}%
\pgfpathlineto{\pgfqpoint{2.832826in}{1.786639in}}%
\pgfpathlineto{\pgfqpoint{2.834746in}{1.787798in}}%
\pgfpathlineto{\pgfqpoint{2.837274in}{1.789487in}}%
\pgfpathlineto{\pgfqpoint{2.838689in}{1.790582in}}%
\pgfpathlineto{\pgfqpoint{2.840205in}{1.789732in}}%
\pgfpathlineto{\pgfqpoint{2.841216in}{1.789339in}}%
\pgfpathlineto{\pgfqpoint{2.841520in}{1.789874in}}%
\pgfpathlineto{\pgfqpoint{2.843744in}{1.792646in}}%
\pgfpathlineto{\pgfqpoint{2.847181in}{1.791951in}}%
\pgfpathlineto{\pgfqpoint{2.848394in}{1.791587in}}%
\pgfpathlineto{\pgfqpoint{2.848900in}{1.791951in}}%
\pgfpathlineto{\pgfqpoint{2.849001in}{1.792253in}}%
\pgfpathlineto{\pgfqpoint{2.850012in}{1.794936in}}%
\pgfpathlineto{\pgfqpoint{2.850618in}{1.794305in}}%
\pgfpathlineto{\pgfqpoint{2.852033in}{1.793110in}}%
\pgfpathlineto{\pgfqpoint{2.852438in}{1.793639in}}%
\pgfpathlineto{\pgfqpoint{2.853954in}{1.793846in}}%
\pgfpathlineto{\pgfqpoint{2.855572in}{1.796417in}}%
\pgfpathlineto{\pgfqpoint{2.859919in}{1.822826in}}%
\pgfpathlineto{\pgfqpoint{2.860424in}{1.822122in}}%
\pgfpathlineto{\pgfqpoint{2.861334in}{1.820382in}}%
\pgfpathlineto{\pgfqpoint{2.862042in}{1.820944in}}%
\pgfpathlineto{\pgfqpoint{2.863053in}{1.820690in}}%
\pgfpathlineto{\pgfqpoint{2.863255in}{1.821442in}}%
\pgfpathlineto{\pgfqpoint{2.865580in}{1.829602in}}%
\pgfpathlineto{\pgfqpoint{2.865782in}{1.829380in}}%
\pgfpathlineto{\pgfqpoint{2.872454in}{1.818236in}}%
\pgfpathlineto{\pgfqpoint{2.872656in}{1.818572in}}%
\pgfpathlineto{\pgfqpoint{2.873870in}{1.823729in}}%
\pgfpathlineto{\pgfqpoint{2.874779in}{1.822371in}}%
\pgfpathlineto{\pgfqpoint{2.876094in}{1.819772in}}%
\pgfpathlineto{\pgfqpoint{2.876498in}{1.821115in}}%
\pgfpathlineto{\pgfqpoint{2.877610in}{1.827274in}}%
\pgfpathlineto{\pgfqpoint{2.878318in}{1.826047in}}%
\pgfpathlineto{\pgfqpoint{2.880441in}{1.823182in}}%
\pgfpathlineto{\pgfqpoint{2.880845in}{1.824016in}}%
\pgfpathlineto{\pgfqpoint{2.881654in}{1.825369in}}%
\pgfpathlineto{\pgfqpoint{2.882159in}{1.824749in}}%
\pgfpathlineto{\pgfqpoint{2.882665in}{1.824305in}}%
\pgfpathlineto{\pgfqpoint{2.883271in}{1.824994in}}%
\pgfpathlineto{\pgfqpoint{2.883777in}{1.824792in}}%
\pgfpathlineto{\pgfqpoint{2.883979in}{1.824404in}}%
\pgfpathlineto{\pgfqpoint{2.885091in}{1.822109in}}%
\pgfpathlineto{\pgfqpoint{2.885495in}{1.823109in}}%
\pgfpathlineto{\pgfqpoint{2.887922in}{1.830740in}}%
\pgfpathlineto{\pgfqpoint{2.888326in}{1.830322in}}%
\pgfpathlineto{\pgfqpoint{2.889640in}{1.828849in}}%
\pgfpathlineto{\pgfqpoint{2.890146in}{1.829723in}}%
\pgfpathlineto{\pgfqpoint{2.890550in}{1.829687in}}%
\pgfpathlineto{\pgfqpoint{2.891055in}{1.829065in}}%
\pgfpathlineto{\pgfqpoint{2.891662in}{1.830146in}}%
\pgfpathlineto{\pgfqpoint{2.892168in}{1.830733in}}%
\pgfpathlineto{\pgfqpoint{2.892673in}{1.829859in}}%
\pgfpathlineto{\pgfqpoint{2.894088in}{1.828198in}}%
\pgfpathlineto{\pgfqpoint{2.894493in}{1.828735in}}%
\pgfpathlineto{\pgfqpoint{2.896110in}{1.829064in}}%
\pgfpathlineto{\pgfqpoint{2.897222in}{1.826554in}}%
\pgfpathlineto{\pgfqpoint{2.899952in}{1.823016in}}%
\pgfpathlineto{\pgfqpoint{2.902277in}{1.821953in}}%
\pgfpathlineto{\pgfqpoint{2.905107in}{1.819336in}}%
\pgfpathlineto{\pgfqpoint{2.907028in}{1.816264in}}%
\pgfpathlineto{\pgfqpoint{2.907635in}{1.817200in}}%
\pgfpathlineto{\pgfqpoint{2.909050in}{1.817273in}}%
\pgfpathlineto{\pgfqpoint{2.911577in}{1.820995in}}%
\pgfpathlineto{\pgfqpoint{2.912487in}{1.825840in}}%
\pgfpathlineto{\pgfqpoint{2.913195in}{1.824996in}}%
\pgfpathlineto{\pgfqpoint{2.915318in}{1.820480in}}%
\pgfpathlineto{\pgfqpoint{2.917239in}{1.818751in}}%
\pgfpathlineto{\pgfqpoint{2.920878in}{1.814424in}}%
\pgfpathlineto{\pgfqpoint{2.921586in}{1.815937in}}%
\pgfpathlineto{\pgfqpoint{2.922293in}{1.817384in}}%
\pgfpathlineto{\pgfqpoint{2.922900in}{1.816720in}}%
\pgfpathlineto{\pgfqpoint{2.925731in}{1.813084in}}%
\pgfpathlineto{\pgfqpoint{2.926236in}{1.814339in}}%
\pgfpathlineto{\pgfqpoint{2.926843in}{1.815145in}}%
\pgfpathlineto{\pgfqpoint{2.927449in}{1.814505in}}%
\pgfpathlineto{\pgfqpoint{2.930381in}{1.809504in}}%
\pgfpathlineto{\pgfqpoint{2.932302in}{1.806576in}}%
\pgfpathlineto{\pgfqpoint{2.932807in}{1.807138in}}%
\pgfpathlineto{\pgfqpoint{2.934020in}{1.806937in}}%
\pgfpathlineto{\pgfqpoint{2.935436in}{1.806405in}}%
\pgfpathlineto{\pgfqpoint{2.939378in}{1.801760in}}%
\pgfpathlineto{\pgfqpoint{2.943422in}{1.798230in}}%
\pgfpathlineto{\pgfqpoint{2.943725in}{1.798647in}}%
\pgfpathlineto{\pgfqpoint{2.944635in}{1.799111in}}%
\pgfpathlineto{\pgfqpoint{2.945039in}{1.798650in}}%
\pgfpathlineto{\pgfqpoint{2.947466in}{1.795927in}}%
\pgfpathlineto{\pgfqpoint{2.947870in}{1.796532in}}%
\pgfpathlineto{\pgfqpoint{2.948679in}{1.797521in}}%
\pgfpathlineto{\pgfqpoint{2.949285in}{1.797023in}}%
\pgfpathlineto{\pgfqpoint{2.953329in}{1.794714in}}%
\pgfpathlineto{\pgfqpoint{2.953531in}{1.795202in}}%
\pgfpathlineto{\pgfqpoint{2.956160in}{1.800568in}}%
\pgfpathlineto{\pgfqpoint{2.956564in}{1.800795in}}%
\pgfpathlineto{\pgfqpoint{2.957171in}{1.800117in}}%
\pgfpathlineto{\pgfqpoint{2.959799in}{1.797809in}}%
\pgfpathlineto{\pgfqpoint{2.960001in}{1.798015in}}%
\pgfpathlineto{\pgfqpoint{2.961821in}{1.799043in}}%
\pgfpathlineto{\pgfqpoint{2.963944in}{1.797384in}}%
\pgfpathlineto{\pgfqpoint{2.967886in}{1.793881in}}%
\pgfpathlineto{\pgfqpoint{2.969504in}{1.795983in}}%
\pgfpathlineto{\pgfqpoint{2.971728in}{1.804262in}}%
\pgfpathlineto{\pgfqpoint{2.972234in}{1.804935in}}%
\pgfpathlineto{\pgfqpoint{2.972840in}{1.804320in}}%
\pgfpathlineto{\pgfqpoint{2.980523in}{1.795540in}}%
\pgfpathlineto{\pgfqpoint{2.980826in}{1.796152in}}%
\pgfpathlineto{\pgfqpoint{2.981736in}{1.797694in}}%
\pgfpathlineto{\pgfqpoint{2.982343in}{1.797377in}}%
\pgfpathlineto{\pgfqpoint{2.983859in}{1.796115in}}%
\pgfpathlineto{\pgfqpoint{2.988004in}{1.792269in}}%
\pgfpathlineto{\pgfqpoint{2.989925in}{1.794655in}}%
\pgfpathlineto{\pgfqpoint{2.990835in}{1.796601in}}%
\pgfpathlineto{\pgfqpoint{2.991441in}{1.796214in}}%
\pgfpathlineto{\pgfqpoint{2.994777in}{1.796754in}}%
\pgfpathlineto{\pgfqpoint{2.997911in}{1.804628in}}%
\pgfpathlineto{\pgfqpoint{2.998316in}{1.804173in}}%
\pgfpathlineto{\pgfqpoint{3.001146in}{1.800403in}}%
\pgfpathlineto{\pgfqpoint{3.001551in}{1.800925in}}%
\pgfpathlineto{\pgfqpoint{3.002663in}{1.804548in}}%
\pgfpathlineto{\pgfqpoint{3.003168in}{1.863517in}}%
\pgfpathlineto{\pgfqpoint{3.005695in}{2.307858in}}%
\pgfpathlineto{\pgfqpoint{3.005999in}{2.304139in}}%
\pgfpathlineto{\pgfqpoint{3.007111in}{2.286940in}}%
\pgfpathlineto{\pgfqpoint{3.007616in}{2.292350in}}%
\pgfpathlineto{\pgfqpoint{3.007818in}{2.293426in}}%
\pgfpathlineto{\pgfqpoint{3.008324in}{2.289641in}}%
\pgfpathlineto{\pgfqpoint{3.011761in}{2.244120in}}%
\pgfpathlineto{\pgfqpoint{3.014895in}{2.186078in}}%
\pgfpathlineto{\pgfqpoint{3.015299in}{2.188630in}}%
\pgfpathlineto{\pgfqpoint{3.015805in}{2.191758in}}%
\pgfpathlineto{\pgfqpoint{3.016310in}{2.187938in}}%
\pgfpathlineto{\pgfqpoint{3.018231in}{2.158873in}}%
\pgfpathlineto{\pgfqpoint{3.018736in}{2.168384in}}%
\pgfpathlineto{\pgfqpoint{3.019444in}{2.180346in}}%
\pgfpathlineto{\pgfqpoint{3.019950in}{2.175428in}}%
\pgfpathlineto{\pgfqpoint{3.020657in}{2.166750in}}%
\pgfpathlineto{\pgfqpoint{3.021062in}{2.172917in}}%
\pgfpathlineto{\pgfqpoint{3.023185in}{2.268392in}}%
\pgfpathlineto{\pgfqpoint{3.025105in}{2.335298in}}%
\pgfpathlineto{\pgfqpoint{3.025308in}{2.334067in}}%
\pgfpathlineto{\pgfqpoint{3.025611in}{2.331718in}}%
\pgfpathlineto{\pgfqpoint{3.026116in}{2.336567in}}%
\pgfpathlineto{\pgfqpoint{3.026420in}{2.339426in}}%
\pgfpathlineto{\pgfqpoint{3.027026in}{2.334001in}}%
\pgfpathlineto{\pgfqpoint{3.027228in}{2.333192in}}%
\pgfpathlineto{\pgfqpoint{3.027532in}{2.336161in}}%
\pgfpathlineto{\pgfqpoint{3.028138in}{2.342628in}}%
\pgfpathlineto{\pgfqpoint{3.028644in}{2.336452in}}%
\pgfpathlineto{\pgfqpoint{3.033294in}{2.264759in}}%
\pgfpathlineto{\pgfqpoint{3.034608in}{2.242695in}}%
\pgfpathlineto{\pgfqpoint{3.036023in}{2.218485in}}%
\pgfpathlineto{\pgfqpoint{3.036327in}{2.223200in}}%
\pgfpathlineto{\pgfqpoint{3.038652in}{2.358101in}}%
\pgfpathlineto{\pgfqpoint{3.039764in}{2.340486in}}%
\pgfpathlineto{\pgfqpoint{3.045931in}{2.229876in}}%
\pgfpathlineto{\pgfqpoint{3.047952in}{2.200053in}}%
\pgfpathlineto{\pgfqpoint{3.048761in}{2.205569in}}%
\pgfpathlineto{\pgfqpoint{3.049166in}{2.202010in}}%
\pgfpathlineto{\pgfqpoint{3.050783in}{2.174485in}}%
\pgfpathlineto{\pgfqpoint{3.051390in}{2.181550in}}%
\pgfpathlineto{\pgfqpoint{3.052300in}{2.192985in}}%
\pgfpathlineto{\pgfqpoint{3.052805in}{2.189274in}}%
\pgfpathlineto{\pgfqpoint{3.054018in}{2.171639in}}%
\pgfpathlineto{\pgfqpoint{3.054524in}{2.179654in}}%
\pgfpathlineto{\pgfqpoint{3.055231in}{2.191041in}}%
\pgfpathlineto{\pgfqpoint{3.055737in}{2.185668in}}%
\pgfpathlineto{\pgfqpoint{3.059477in}{2.124662in}}%
\pgfpathlineto{\pgfqpoint{3.060488in}{2.131200in}}%
\pgfpathlineto{\pgfqpoint{3.060690in}{2.132013in}}%
\pgfpathlineto{\pgfqpoint{3.061398in}{2.160519in}}%
\pgfpathlineto{\pgfqpoint{3.061903in}{2.172203in}}%
\pgfpathlineto{\pgfqpoint{3.062510in}{2.165816in}}%
\pgfpathlineto{\pgfqpoint{3.064532in}{2.134797in}}%
\pgfpathlineto{\pgfqpoint{3.064936in}{2.141629in}}%
\pgfpathlineto{\pgfqpoint{3.067767in}{2.214589in}}%
\pgfpathlineto{\pgfqpoint{3.067969in}{2.216679in}}%
\pgfpathlineto{\pgfqpoint{3.068677in}{2.211761in}}%
\pgfpathlineto{\pgfqpoint{3.072013in}{2.178826in}}%
\pgfpathlineto{\pgfqpoint{3.076158in}{2.134432in}}%
\pgfpathlineto{\pgfqpoint{3.076562in}{2.136310in}}%
\pgfpathlineto{\pgfqpoint{3.078483in}{2.155473in}}%
\pgfpathlineto{\pgfqpoint{3.079190in}{2.148266in}}%
\pgfpathlineto{\pgfqpoint{3.081212in}{2.135847in}}%
\pgfpathlineto{\pgfqpoint{3.085155in}{2.103181in}}%
\pgfpathlineto{\pgfqpoint{3.086570in}{2.094174in}}%
\pgfpathlineto{\pgfqpoint{3.087278in}{2.098554in}}%
\pgfpathlineto{\pgfqpoint{3.087379in}{2.098785in}}%
\pgfpathlineto{\pgfqpoint{3.087682in}{2.097618in}}%
\pgfpathlineto{\pgfqpoint{3.097893in}{2.013026in}}%
\pgfpathlineto{\pgfqpoint{3.098095in}{2.013649in}}%
\pgfpathlineto{\pgfqpoint{3.099005in}{2.023108in}}%
\pgfpathlineto{\pgfqpoint{3.099712in}{2.018535in}}%
\pgfpathlineto{\pgfqpoint{3.100016in}{2.017411in}}%
\pgfpathlineto{\pgfqpoint{3.100420in}{2.019970in}}%
\pgfpathlineto{\pgfqpoint{3.101229in}{2.028722in}}%
\pgfpathlineto{\pgfqpoint{3.101835in}{2.025048in}}%
\pgfpathlineto{\pgfqpoint{3.103756in}{2.007137in}}%
\pgfpathlineto{\pgfqpoint{3.104262in}{2.012611in}}%
\pgfpathlineto{\pgfqpoint{3.106486in}{2.037122in}}%
\pgfpathlineto{\pgfqpoint{3.106587in}{2.037278in}}%
\pgfpathlineto{\pgfqpoint{3.106890in}{2.035864in}}%
\pgfpathlineto{\pgfqpoint{3.108406in}{2.022850in}}%
\pgfpathlineto{\pgfqpoint{3.108912in}{2.026584in}}%
\pgfpathlineto{\pgfqpoint{3.111540in}{2.049458in}}%
\pgfpathlineto{\pgfqpoint{3.111641in}{2.049371in}}%
\pgfpathlineto{\pgfqpoint{3.112551in}{2.041081in}}%
\pgfpathlineto{\pgfqpoint{3.116191in}{2.005015in}}%
\pgfpathlineto{\pgfqpoint{3.116595in}{2.009008in}}%
\pgfpathlineto{\pgfqpoint{3.118819in}{2.031795in}}%
\pgfpathlineto{\pgfqpoint{3.119628in}{2.043572in}}%
\pgfpathlineto{\pgfqpoint{3.120234in}{2.040080in}}%
\pgfpathlineto{\pgfqpoint{3.123368in}{2.008841in}}%
\pgfpathlineto{\pgfqpoint{3.125491in}{1.997830in}}%
\pgfpathlineto{\pgfqpoint{3.129434in}{1.974649in}}%
\pgfpathlineto{\pgfqpoint{3.130243in}{1.971294in}}%
\pgfpathlineto{\pgfqpoint{3.137825in}{1.919347in}}%
\pgfpathlineto{\pgfqpoint{3.139543in}{1.914321in}}%
\pgfpathlineto{\pgfqpoint{3.139948in}{1.916370in}}%
\pgfpathlineto{\pgfqpoint{3.143385in}{1.949516in}}%
\pgfpathlineto{\pgfqpoint{3.143486in}{1.949500in}}%
\pgfpathlineto{\pgfqpoint{3.144193in}{1.945576in}}%
\pgfpathlineto{\pgfqpoint{3.147226in}{1.930945in}}%
\pgfpathlineto{\pgfqpoint{3.151775in}{1.906116in}}%
\pgfpathlineto{\pgfqpoint{3.151877in}{1.906265in}}%
\pgfpathlineto{\pgfqpoint{3.152281in}{1.913025in}}%
\pgfpathlineto{\pgfqpoint{3.155718in}{2.019441in}}%
\pgfpathlineto{\pgfqpoint{3.156122in}{2.017514in}}%
\pgfpathlineto{\pgfqpoint{3.160267in}{1.979942in}}%
\pgfpathlineto{\pgfqpoint{3.160975in}{1.983120in}}%
\pgfpathlineto{\pgfqpoint{3.161480in}{1.984817in}}%
\pgfpathlineto{\pgfqpoint{3.161885in}{1.982997in}}%
\pgfpathlineto{\pgfqpoint{3.162694in}{1.978644in}}%
\pgfpathlineto{\pgfqpoint{3.163098in}{1.980906in}}%
\pgfpathlineto{\pgfqpoint{3.164008in}{1.988628in}}%
\pgfpathlineto{\pgfqpoint{3.164614in}{1.985596in}}%
\pgfpathlineto{\pgfqpoint{3.169669in}{1.949468in}}%
\pgfpathlineto{\pgfqpoint{3.170781in}{1.945967in}}%
\pgfpathlineto{\pgfqpoint{3.171286in}{1.943953in}}%
\pgfpathlineto{\pgfqpoint{3.171691in}{1.945944in}}%
\pgfpathlineto{\pgfqpoint{3.172500in}{1.951224in}}%
\pgfpathlineto{\pgfqpoint{3.173005in}{1.949088in}}%
\pgfpathlineto{\pgfqpoint{3.174016in}{1.944338in}}%
\pgfpathlineto{\pgfqpoint{3.174420in}{1.946112in}}%
\pgfpathlineto{\pgfqpoint{3.177655in}{1.967149in}}%
\pgfpathlineto{\pgfqpoint{3.177756in}{1.966941in}}%
\pgfpathlineto{\pgfqpoint{3.185237in}{1.932384in}}%
\pgfpathlineto{\pgfqpoint{3.187563in}{1.923400in}}%
\pgfpathlineto{\pgfqpoint{3.187664in}{1.923520in}}%
\pgfpathlineto{\pgfqpoint{3.188675in}{1.926040in}}%
\pgfpathlineto{\pgfqpoint{3.189180in}{1.924278in}}%
\pgfpathlineto{\pgfqpoint{3.190494in}{1.918296in}}%
\pgfpathlineto{\pgfqpoint{3.191101in}{1.919024in}}%
\pgfpathlineto{\pgfqpoint{3.191606in}{1.919142in}}%
\pgfpathlineto{\pgfqpoint{3.192112in}{1.918545in}}%
\pgfpathlineto{\pgfqpoint{3.193022in}{1.915457in}}%
\pgfpathlineto{\pgfqpoint{3.193628in}{1.913013in}}%
\pgfpathlineto{\pgfqpoint{3.194134in}{1.915515in}}%
\pgfpathlineto{\pgfqpoint{3.195347in}{1.926272in}}%
\pgfpathlineto{\pgfqpoint{3.195953in}{1.923887in}}%
\pgfpathlineto{\pgfqpoint{3.196257in}{1.923218in}}%
\pgfpathlineto{\pgfqpoint{3.196762in}{1.924863in}}%
\pgfpathlineto{\pgfqpoint{3.197571in}{1.928307in}}%
\pgfpathlineto{\pgfqpoint{3.198076in}{1.926712in}}%
\pgfpathlineto{\pgfqpoint{3.207781in}{1.879037in}}%
\pgfpathlineto{\pgfqpoint{3.210106in}{1.872879in}}%
\pgfpathlineto{\pgfqpoint{3.210612in}{1.874043in}}%
\pgfpathlineto{\pgfqpoint{3.211218in}{1.875446in}}%
\pgfpathlineto{\pgfqpoint{3.211724in}{1.874289in}}%
\pgfpathlineto{\pgfqpoint{3.212735in}{1.870915in}}%
\pgfpathlineto{\pgfqpoint{3.213240in}{1.872054in}}%
\pgfpathlineto{\pgfqpoint{3.213746in}{1.873329in}}%
\pgfpathlineto{\pgfqpoint{3.214251in}{1.872247in}}%
\pgfpathlineto{\pgfqpoint{3.219710in}{1.851675in}}%
\pgfpathlineto{\pgfqpoint{3.220519in}{1.852761in}}%
\pgfpathlineto{\pgfqpoint{3.221126in}{1.851715in}}%
\pgfpathlineto{\pgfqpoint{3.221732in}{1.850781in}}%
\pgfpathlineto{\pgfqpoint{3.222238in}{1.851576in}}%
\pgfpathlineto{\pgfqpoint{3.222743in}{1.852248in}}%
\pgfpathlineto{\pgfqpoint{3.223249in}{1.851360in}}%
\pgfpathlineto{\pgfqpoint{3.235986in}{1.818538in}}%
\pgfpathlineto{\pgfqpoint{3.240738in}{1.810059in}}%
\pgfpathlineto{\pgfqpoint{3.241344in}{1.810024in}}%
\pgfpathlineto{\pgfqpoint{3.241648in}{1.810470in}}%
\pgfpathlineto{\pgfqpoint{3.244074in}{1.813864in}}%
\pgfpathlineto{\pgfqpoint{3.244579in}{1.813133in}}%
\pgfpathlineto{\pgfqpoint{3.245388in}{1.812388in}}%
\pgfpathlineto{\pgfqpoint{3.245792in}{1.813018in}}%
\pgfpathlineto{\pgfqpoint{3.246702in}{1.814608in}}%
\pgfpathlineto{\pgfqpoint{3.247309in}{1.813965in}}%
\pgfpathlineto{\pgfqpoint{3.250139in}{1.810453in}}%
\pgfpathlineto{\pgfqpoint{3.251049in}{1.809494in}}%
\pgfpathlineto{\pgfqpoint{3.251454in}{1.810153in}}%
\pgfpathlineto{\pgfqpoint{3.253779in}{1.812530in}}%
\pgfpathlineto{\pgfqpoint{3.256003in}{1.810133in}}%
\pgfpathlineto{\pgfqpoint{3.265910in}{1.797530in}}%
\pgfpathlineto{\pgfqpoint{3.269145in}{1.794911in}}%
\pgfpathlineto{\pgfqpoint{3.274301in}{1.790442in}}%
\pgfpathlineto{\pgfqpoint{3.278344in}{1.788249in}}%
\pgfpathlineto{\pgfqpoint{3.278547in}{1.788529in}}%
\pgfpathlineto{\pgfqpoint{3.279659in}{1.788995in}}%
\pgfpathlineto{\pgfqpoint{3.279962in}{1.788740in}}%
\pgfpathlineto{\pgfqpoint{3.280771in}{1.788350in}}%
\pgfpathlineto{\pgfqpoint{3.281175in}{1.789103in}}%
\pgfpathlineto{\pgfqpoint{3.285421in}{1.803127in}}%
\pgfpathlineto{\pgfqpoint{3.286129in}{1.806396in}}%
\pgfpathlineto{\pgfqpoint{3.286735in}{1.805760in}}%
\pgfpathlineto{\pgfqpoint{3.298866in}{1.792545in}}%
\pgfpathlineto{\pgfqpoint{3.300484in}{1.791569in}}%
\pgfpathlineto{\pgfqpoint{3.300787in}{1.792171in}}%
\pgfpathlineto{\pgfqpoint{3.303213in}{1.798758in}}%
\pgfpathlineto{\pgfqpoint{3.303517in}{1.798526in}}%
\pgfpathlineto{\pgfqpoint{3.304325in}{1.798223in}}%
\pgfpathlineto{\pgfqpoint{3.304629in}{1.798792in}}%
\pgfpathlineto{\pgfqpoint{3.307358in}{1.808862in}}%
\pgfpathlineto{\pgfqpoint{3.308369in}{1.807294in}}%
\pgfpathlineto{\pgfqpoint{3.311099in}{1.803201in}}%
\pgfpathlineto{\pgfqpoint{3.311604in}{1.804042in}}%
\pgfpathlineto{\pgfqpoint{3.313323in}{1.805239in}}%
\pgfpathlineto{\pgfqpoint{3.313424in}{1.805181in}}%
\pgfpathlineto{\pgfqpoint{3.313929in}{1.805149in}}%
\pgfpathlineto{\pgfqpoint{3.314132in}{1.805627in}}%
\pgfpathlineto{\pgfqpoint{3.317468in}{1.822336in}}%
\pgfpathlineto{\pgfqpoint{3.318175in}{1.825187in}}%
\pgfpathlineto{\pgfqpoint{3.318782in}{1.824562in}}%
\pgfpathlineto{\pgfqpoint{3.319287in}{1.825577in}}%
\pgfpathlineto{\pgfqpoint{3.321916in}{1.844550in}}%
\pgfpathlineto{\pgfqpoint{3.327880in}{1.990497in}}%
\pgfpathlineto{\pgfqpoint{3.329296in}{2.055959in}}%
\pgfpathlineto{\pgfqpoint{3.330104in}{2.055613in}}%
\pgfpathlineto{\pgfqpoint{3.331115in}{2.071212in}}%
\pgfpathlineto{\pgfqpoint{3.331823in}{2.066803in}}%
\pgfpathlineto{\pgfqpoint{3.333541in}{2.058604in}}%
\pgfpathlineto{\pgfqpoint{3.334047in}{2.062206in}}%
\pgfpathlineto{\pgfqpoint{3.335867in}{2.073957in}}%
\pgfpathlineto{\pgfqpoint{3.336271in}{2.072641in}}%
\pgfpathlineto{\pgfqpoint{3.336473in}{2.072068in}}%
\pgfpathlineto{\pgfqpoint{3.336979in}{2.074171in}}%
\pgfpathlineto{\pgfqpoint{3.337282in}{2.074883in}}%
\pgfpathlineto{\pgfqpoint{3.337585in}{2.073221in}}%
\pgfpathlineto{\pgfqpoint{3.340416in}{2.039241in}}%
\pgfpathlineto{\pgfqpoint{3.341022in}{2.049697in}}%
\pgfpathlineto{\pgfqpoint{3.343044in}{2.087426in}}%
\pgfpathlineto{\pgfqpoint{3.343954in}{2.093269in}}%
\pgfpathlineto{\pgfqpoint{3.344460in}{2.091482in}}%
\pgfpathlineto{\pgfqpoint{3.348705in}{2.053435in}}%
\pgfpathlineto{\pgfqpoint{3.349211in}{2.059040in}}%
\pgfpathlineto{\pgfqpoint{3.350525in}{2.084886in}}%
\pgfpathlineto{\pgfqpoint{3.351132in}{2.080368in}}%
\pgfpathlineto{\pgfqpoint{3.351536in}{2.078199in}}%
\pgfpathlineto{\pgfqpoint{3.352042in}{2.080846in}}%
\pgfpathlineto{\pgfqpoint{3.352446in}{2.082459in}}%
\pgfpathlineto{\pgfqpoint{3.352951in}{2.079723in}}%
\pgfpathlineto{\pgfqpoint{3.356793in}{2.039684in}}%
\pgfpathlineto{\pgfqpoint{3.357399in}{2.046518in}}%
\pgfpathlineto{\pgfqpoint{3.358309in}{2.060596in}}%
\pgfpathlineto{\pgfqpoint{3.358916in}{2.056772in}}%
\pgfpathlineto{\pgfqpoint{3.363465in}{2.020838in}}%
\pgfpathlineto{\pgfqpoint{3.370542in}{1.962676in}}%
\pgfpathlineto{\pgfqpoint{3.371249in}{1.959668in}}%
\pgfpathlineto{\pgfqpoint{3.371856in}{1.956579in}}%
\pgfpathlineto{\pgfqpoint{3.372260in}{1.958822in}}%
\pgfpathlineto{\pgfqpoint{3.373271in}{1.967371in}}%
\pgfpathlineto{\pgfqpoint{3.373777in}{1.964903in}}%
\pgfpathlineto{\pgfqpoint{3.376708in}{1.952908in}}%
\pgfpathlineto{\pgfqpoint{3.376911in}{1.952821in}}%
\pgfpathlineto{\pgfqpoint{3.377214in}{1.953575in}}%
\pgfpathlineto{\pgfqpoint{3.377921in}{1.954958in}}%
\pgfpathlineto{\pgfqpoint{3.378326in}{1.953985in}}%
\pgfpathlineto{\pgfqpoint{3.379640in}{1.946894in}}%
\pgfpathlineto{\pgfqpoint{3.380146in}{1.950258in}}%
\pgfpathlineto{\pgfqpoint{3.380954in}{1.956562in}}%
\pgfpathlineto{\pgfqpoint{3.381561in}{1.953810in}}%
\pgfpathlineto{\pgfqpoint{3.384290in}{1.942008in}}%
\pgfpathlineto{\pgfqpoint{3.393692in}{1.893209in}}%
\pgfpathlineto{\pgfqpoint{3.395208in}{1.889758in}}%
\pgfpathlineto{\pgfqpoint{3.401072in}{1.864043in}}%
\pgfpathlineto{\pgfqpoint{3.401375in}{1.864898in}}%
\pgfpathlineto{\pgfqpoint{3.402892in}{1.878162in}}%
\pgfpathlineto{\pgfqpoint{3.404004in}{1.875218in}}%
\pgfpathlineto{\pgfqpoint{3.405520in}{1.869454in}}%
\pgfpathlineto{\pgfqpoint{3.406127in}{1.871792in}}%
\pgfpathlineto{\pgfqpoint{3.407744in}{1.878592in}}%
\pgfpathlineto{\pgfqpoint{3.408148in}{1.877737in}}%
\pgfpathlineto{\pgfqpoint{3.411080in}{1.870868in}}%
\pgfpathlineto{\pgfqpoint{3.411788in}{1.873573in}}%
\pgfpathlineto{\pgfqpoint{3.412900in}{1.877550in}}%
\pgfpathlineto{\pgfqpoint{3.413405in}{1.876500in}}%
\pgfpathlineto{\pgfqpoint{3.413810in}{1.875650in}}%
\pgfpathlineto{\pgfqpoint{3.414315in}{1.877095in}}%
\pgfpathlineto{\pgfqpoint{3.414821in}{1.878457in}}%
\pgfpathlineto{\pgfqpoint{3.415326in}{1.877080in}}%
\pgfpathlineto{\pgfqpoint{3.417348in}{1.870453in}}%
\pgfpathlineto{\pgfqpoint{3.417651in}{1.871057in}}%
\pgfpathlineto{\pgfqpoint{3.421088in}{1.891196in}}%
\pgfpathlineto{\pgfqpoint{3.422301in}{1.906539in}}%
\pgfpathlineto{\pgfqpoint{3.422807in}{1.904938in}}%
\pgfpathlineto{\pgfqpoint{3.423009in}{1.904587in}}%
\pgfpathlineto{\pgfqpoint{3.423312in}{1.905713in}}%
\pgfpathlineto{\pgfqpoint{3.424829in}{1.921884in}}%
\pgfpathlineto{\pgfqpoint{3.425739in}{1.917789in}}%
\pgfpathlineto{\pgfqpoint{3.427862in}{1.906914in}}%
\pgfpathlineto{\pgfqpoint{3.428266in}{1.908910in}}%
\pgfpathlineto{\pgfqpoint{3.429580in}{1.920269in}}%
\pgfpathlineto{\pgfqpoint{3.430187in}{1.917646in}}%
\pgfpathlineto{\pgfqpoint{3.432714in}{1.905552in}}%
\pgfpathlineto{\pgfqpoint{3.433017in}{1.906457in}}%
\pgfpathlineto{\pgfqpoint{3.434129in}{1.913166in}}%
\pgfpathlineto{\pgfqpoint{3.434837in}{1.910630in}}%
\pgfpathlineto{\pgfqpoint{3.435241in}{1.909602in}}%
\pgfpathlineto{\pgfqpoint{3.435747in}{1.910919in}}%
\pgfpathlineto{\pgfqpoint{3.437061in}{1.918706in}}%
\pgfpathlineto{\pgfqpoint{3.437769in}{1.916092in}}%
\pgfpathlineto{\pgfqpoint{3.440296in}{1.902459in}}%
\pgfpathlineto{\pgfqpoint{3.440802in}{1.903246in}}%
\pgfpathlineto{\pgfqpoint{3.441813in}{1.906361in}}%
\pgfpathlineto{\pgfqpoint{3.442318in}{1.905051in}}%
\pgfpathlineto{\pgfqpoint{3.445856in}{1.892161in}}%
\pgfpathlineto{\pgfqpoint{3.445957in}{1.892187in}}%
\pgfpathlineto{\pgfqpoint{3.446665in}{1.891635in}}%
\pgfpathlineto{\pgfqpoint{3.466783in}{1.828294in}}%
\pgfpathlineto{\pgfqpoint{3.469512in}{1.824093in}}%
\pgfpathlineto{\pgfqpoint{3.471433in}{1.821273in}}%
\pgfpathlineto{\pgfqpoint{3.471736in}{1.821437in}}%
\pgfpathlineto{\pgfqpoint{3.472646in}{1.820956in}}%
\pgfpathlineto{\pgfqpoint{3.472747in}{1.820758in}}%
\pgfpathlineto{\pgfqpoint{3.476589in}{1.815397in}}%
\pgfpathlineto{\pgfqpoint{3.478105in}{1.813248in}}%
\pgfpathlineto{\pgfqpoint{3.478509in}{1.813818in}}%
\pgfpathlineto{\pgfqpoint{3.479925in}{1.817968in}}%
\pgfpathlineto{\pgfqpoint{3.480733in}{1.816825in}}%
\pgfpathlineto{\pgfqpoint{3.481542in}{1.816673in}}%
\pgfpathlineto{\pgfqpoint{3.481947in}{1.817051in}}%
\pgfpathlineto{\pgfqpoint{3.482958in}{1.817145in}}%
\pgfpathlineto{\pgfqpoint{3.483160in}{1.816884in}}%
\pgfpathlineto{\pgfqpoint{3.486597in}{1.812036in}}%
\pgfpathlineto{\pgfqpoint{3.487001in}{1.812716in}}%
\pgfpathlineto{\pgfqpoint{3.488315in}{1.815948in}}%
\pgfpathlineto{\pgfqpoint{3.488922in}{1.815082in}}%
\pgfpathlineto{\pgfqpoint{3.490944in}{1.813448in}}%
\pgfpathlineto{\pgfqpoint{3.492258in}{1.811499in}}%
\pgfpathlineto{\pgfqpoint{3.497212in}{1.804564in}}%
\pgfpathlineto{\pgfqpoint{3.499638in}{1.802114in}}%
\pgfpathlineto{\pgfqpoint{3.499739in}{1.802218in}}%
\pgfpathlineto{\pgfqpoint{3.500447in}{1.805629in}}%
\pgfpathlineto{\pgfqpoint{3.501660in}{1.812552in}}%
\pgfpathlineto{\pgfqpoint{3.502266in}{1.812066in}}%
\pgfpathlineto{\pgfqpoint{3.502873in}{1.812890in}}%
\pgfpathlineto{\pgfqpoint{3.504288in}{1.817684in}}%
\pgfpathlineto{\pgfqpoint{3.504895in}{1.816702in}}%
\pgfpathlineto{\pgfqpoint{3.511365in}{1.804718in}}%
\pgfpathlineto{\pgfqpoint{3.515510in}{1.800184in}}%
\pgfpathlineto{\pgfqpoint{3.520665in}{1.794048in}}%
\pgfpathlineto{\pgfqpoint{3.527540in}{1.789771in}}%
\pgfpathlineto{\pgfqpoint{3.529865in}{1.788702in}}%
\pgfpathlineto{\pgfqpoint{3.531078in}{1.789185in}}%
\pgfpathlineto{\pgfqpoint{3.532392in}{1.788352in}}%
\pgfpathlineto{\pgfqpoint{3.535122in}{1.787369in}}%
\pgfpathlineto{\pgfqpoint{3.537245in}{1.787767in}}%
\pgfpathlineto{\pgfqpoint{3.538458in}{1.788859in}}%
\pgfpathlineto{\pgfqpoint{3.538862in}{1.788530in}}%
\pgfpathlineto{\pgfqpoint{3.541693in}{1.787321in}}%
\pgfpathlineto{\pgfqpoint{3.543816in}{1.786768in}}%
\pgfpathlineto{\pgfqpoint{3.545534in}{1.786781in}}%
\pgfpathlineto{\pgfqpoint{3.547253in}{1.787129in}}%
\pgfpathlineto{\pgfqpoint{3.552611in}{1.785644in}}%
\pgfpathlineto{\pgfqpoint{3.553622in}{1.786594in}}%
\pgfpathlineto{\pgfqpoint{3.554127in}{1.786325in}}%
\pgfpathlineto{\pgfqpoint{3.555442in}{1.787141in}}%
\pgfpathlineto{\pgfqpoint{3.556655in}{1.786576in}}%
\pgfpathlineto{\pgfqpoint{3.560193in}{1.784975in}}%
\pgfpathlineto{\pgfqpoint{3.561810in}{1.784542in}}%
\pgfpathlineto{\pgfqpoint{3.566157in}{1.782286in}}%
\pgfpathlineto{\pgfqpoint{3.569898in}{1.780967in}}%
\pgfpathlineto{\pgfqpoint{3.571920in}{1.780208in}}%
\pgfpathlineto{\pgfqpoint{3.575862in}{1.779418in}}%
\pgfpathlineto{\pgfqpoint{3.578491in}{1.780000in}}%
\pgfpathlineto{\pgfqpoint{3.580311in}{1.779855in}}%
\pgfpathlineto{\pgfqpoint{3.580412in}{1.780084in}}%
\pgfpathlineto{\pgfqpoint{3.585466in}{1.791250in}}%
\pgfpathlineto{\pgfqpoint{3.587084in}{1.793268in}}%
\pgfpathlineto{\pgfqpoint{3.587488in}{1.792974in}}%
\pgfpathlineto{\pgfqpoint{3.588499in}{1.792460in}}%
\pgfpathlineto{\pgfqpoint{3.588802in}{1.793105in}}%
\pgfpathlineto{\pgfqpoint{3.591532in}{1.804689in}}%
\pgfpathlineto{\pgfqpoint{3.592947in}{1.803804in}}%
\pgfpathlineto{\pgfqpoint{3.593756in}{1.806539in}}%
\pgfpathlineto{\pgfqpoint{3.595576in}{1.814531in}}%
\pgfpathlineto{\pgfqpoint{3.596081in}{1.813877in}}%
\pgfpathlineto{\pgfqpoint{3.597901in}{1.811123in}}%
\pgfpathlineto{\pgfqpoint{3.598305in}{1.811704in}}%
\pgfpathlineto{\pgfqpoint{3.599923in}{1.814560in}}%
\pgfpathlineto{\pgfqpoint{3.600428in}{1.813993in}}%
\pgfpathlineto{\pgfqpoint{3.603360in}{1.808870in}}%
\pgfpathlineto{\pgfqpoint{3.603966in}{1.809763in}}%
\pgfpathlineto{\pgfqpoint{3.604977in}{1.810678in}}%
\pgfpathlineto{\pgfqpoint{3.605382in}{1.810252in}}%
\pgfpathlineto{\pgfqpoint{3.609931in}{1.805317in}}%
\pgfpathlineto{\pgfqpoint{3.614278in}{1.800901in}}%
\pgfpathlineto{\pgfqpoint{3.622871in}{1.790404in}}%
\pgfpathlineto{\pgfqpoint{3.623882in}{1.791661in}}%
\pgfpathlineto{\pgfqpoint{3.624691in}{1.791784in}}%
\pgfpathlineto{\pgfqpoint{3.625095in}{1.791389in}}%
\pgfpathlineto{\pgfqpoint{3.626915in}{1.791150in}}%
\pgfpathlineto{\pgfqpoint{3.629038in}{1.789085in}}%
\pgfpathlineto{\pgfqpoint{3.636822in}{1.783126in}}%
\pgfpathlineto{\pgfqpoint{3.643292in}{1.779988in}}%
\pgfpathlineto{\pgfqpoint{3.645010in}{1.780472in}}%
\pgfpathlineto{\pgfqpoint{3.646729in}{1.780042in}}%
\pgfpathlineto{\pgfqpoint{3.652390in}{1.780386in}}%
\pgfpathlineto{\pgfqpoint{3.654109in}{1.780174in}}%
\pgfpathlineto{\pgfqpoint{3.659467in}{1.777929in}}%
\pgfpathlineto{\pgfqpoint{3.680797in}{1.772045in}}%
\pgfpathlineto{\pgfqpoint{3.693030in}{1.771200in}}%
\pgfpathlineto{\pgfqpoint{3.695355in}{1.771397in}}%
\pgfpathlineto{\pgfqpoint{3.699297in}{1.771529in}}%
\pgfpathlineto{\pgfqpoint{3.702330in}{1.771320in}}%
\pgfpathlineto{\pgfqpoint{3.712642in}{1.770338in}}%
\pgfpathlineto{\pgfqpoint{3.722044in}{1.770587in}}%
\pgfpathlineto{\pgfqpoint{3.725076in}{1.770141in}}%
\pgfpathlineto{\pgfqpoint{3.737612in}{1.769030in}}%
\pgfpathlineto{\pgfqpoint{3.748732in}{1.768585in}}%
\pgfpathlineto{\pgfqpoint{3.758740in}{1.769104in}}%
\pgfpathlineto{\pgfqpoint{3.765514in}{1.769315in}}%
\pgfpathlineto{\pgfqpoint{3.774006in}{1.769683in}}%
\pgfpathlineto{\pgfqpoint{3.798167in}{1.769164in}}%
\pgfpathlineto{\pgfqpoint{3.803727in}{1.769332in}}%
\pgfpathlineto{\pgfqpoint{3.835673in}{1.768666in}}%
\pgfpathlineto{\pgfqpoint{3.853061in}{1.767909in}}%
\pgfpathlineto{\pgfqpoint{3.873380in}{1.768603in}}%
\pgfpathlineto{\pgfqpoint{3.877323in}{1.769800in}}%
\pgfpathlineto{\pgfqpoint{3.887635in}{1.768987in}}%
\pgfpathlineto{\pgfqpoint{3.896126in}{1.769849in}}%
\pgfpathlineto{\pgfqpoint{3.900069in}{1.770032in}}%
\pgfpathlineto{\pgfqpoint{3.914222in}{1.768856in}}%
\pgfpathlineto{\pgfqpoint{3.918670in}{1.769724in}}%
\pgfpathlineto{\pgfqpoint{3.923219in}{1.769300in}}%
\pgfpathlineto{\pgfqpoint{3.927263in}{1.769816in}}%
\pgfpathlineto{\pgfqpoint{3.933430in}{1.770032in}}%
\pgfpathlineto{\pgfqpoint{3.958602in}{1.768158in}}%
\pgfpathlineto{\pgfqpoint{3.964769in}{1.768029in}}%
\pgfpathlineto{\pgfqpoint{3.981045in}{1.767759in}}%
\pgfpathlineto{\pgfqpoint{3.987111in}{1.768374in}}%
\pgfpathlineto{\pgfqpoint{4.227915in}{1.767852in}}%
\pgfpathlineto{\pgfqpoint{4.230948in}{1.768218in}}%
\pgfpathlineto{\pgfqpoint{4.279978in}{1.767602in}}%
\pgfpathlineto{\pgfqpoint{4.311418in}{1.767413in}}%
\pgfpathlineto{\pgfqpoint{4.315058in}{1.767410in}}%
\pgfpathlineto{\pgfqpoint{4.332041in}{1.767466in}}%
\pgfpathlineto{\pgfqpoint{4.339017in}{1.767234in}}%
\pgfpathlineto{\pgfqpoint{4.347509in}{1.767871in}}%
\pgfpathlineto{\pgfqpoint{4.350541in}{1.768157in}}%
\pgfpathlineto{\pgfqpoint{4.366817in}{1.767729in}}%
\pgfpathlineto{\pgfqpoint{4.375410in}{1.767866in}}%
\pgfpathlineto{\pgfqpoint{4.403919in}{1.767031in}}%
\pgfpathlineto{\pgfqpoint{4.423126in}{1.768085in}}%
\pgfpathlineto{\pgfqpoint{4.428181in}{1.768013in}}%
\pgfpathlineto{\pgfqpoint{4.442840in}{1.767729in}}%
\pgfpathlineto{\pgfqpoint{4.482569in}{1.768010in}}%
\pgfpathlineto{\pgfqpoint{4.485602in}{1.768163in}}%
\pgfpathlineto{\pgfqpoint{4.504608in}{1.767661in}}%
\pgfpathlineto{\pgfqpoint{4.516132in}{1.767536in}}%
\pgfpathlineto{\pgfqpoint{4.550100in}{1.767382in}}%
\pgfpathlineto{\pgfqpoint{4.561726in}{1.767170in}}%
\pgfpathlineto{\pgfqpoint{4.583360in}{1.767074in}}%
\pgfpathlineto{\pgfqpoint{4.591447in}{1.767379in}}%
\pgfpathlineto{\pgfqpoint{4.606005in}{1.768555in}}%
\pgfpathlineto{\pgfqpoint{4.611059in}{1.769183in}}%
\pgfpathlineto{\pgfqpoint{4.640275in}{1.767406in}}%
\pgfpathlineto{\pgfqpoint{4.661303in}{1.767440in}}%
\pgfpathlineto{\pgfqpoint{4.678590in}{1.767068in}}%
\pgfpathlineto{\pgfqpoint{4.710333in}{1.766780in}}%
\pgfpathlineto{\pgfqpoint{4.714579in}{1.767028in}}%
\pgfpathlineto{\pgfqpoint{4.742784in}{1.766890in}}%
\pgfpathlineto{\pgfqpoint{4.791713in}{1.767079in}}%
\pgfpathlineto{\pgfqpoint{4.923539in}{1.766951in}}%
\pgfpathlineto{\pgfqpoint{4.970244in}{1.766586in}}%
\pgfpathlineto{\pgfqpoint{4.978635in}{1.766543in}}%
\pgfpathlineto{\pgfqpoint{5.029687in}{1.766741in}}%
\pgfpathlineto{\pgfqpoint{5.079526in}{1.766652in}}%
\pgfpathlineto{\pgfqpoint{5.104597in}{1.767143in}}%
\pgfpathlineto{\pgfqpoint{5.113494in}{1.767270in}}%
\pgfpathlineto{\pgfqpoint{5.745833in}{1.766575in}}%
\pgfpathlineto{\pgfqpoint{5.745833in}{1.766575in}}%
\pgfusepath{stroke}%
\end{pgfscope}%
\begin{pgfscope}%
\pgfpathrectangle{\pgfqpoint{0.691161in}{1.606074in}}{\pgfqpoint{5.054672in}{0.911907in}}%
\pgfusepath{clip}%
\pgfsetbuttcap%
\pgfsetroundjoin%
\pgfsetlinewidth{2.007500pt}%
\definecolor{currentstroke}{rgb}{0.172549,0.627451,0.172549}%
\pgfsetstrokecolor{currentstroke}%
\pgfsetdash{{7.400000pt}{3.200000pt}}{0.000000pt}%
\pgfpathmoveto{\pgfqpoint{0.691161in}{1.938015in}}%
\pgfpathlineto{\pgfqpoint{5.745833in}{1.938015in}}%
\pgfusepath{stroke}%
\end{pgfscope}%
\begin{pgfscope}%
\pgfpathrectangle{\pgfqpoint{0.691161in}{1.606074in}}{\pgfqpoint{5.054672in}{0.911907in}}%
\pgfusepath{clip}%
\pgfsetbuttcap%
\pgfsetroundjoin%
\pgfsetlinewidth{2.007500pt}%
\definecolor{currentstroke}{rgb}{0.839216,0.152941,0.156863}%
\pgfsetstrokecolor{currentstroke}%
\pgfsetdash{{2.000000pt}{3.300000pt}}{0.000000pt}%
\pgfpathmoveto{\pgfqpoint{3.003471in}{1.647524in}}%
\pgfpathlineto{\pgfqpoint{3.003471in}{2.476531in}}%
\pgfusepath{stroke}%
\end{pgfscope}%
\begin{pgfscope}%
\pgfpathrectangle{\pgfqpoint{0.691161in}{1.606074in}}{\pgfqpoint{5.054672in}{0.911907in}}%
\pgfusepath{clip}%
\pgfsetbuttcap%
\pgfsetroundjoin%
\pgfsetlinewidth{2.007500pt}%
\definecolor{currentstroke}{rgb}{1.000000,0.498039,0.054902}%
\pgfsetstrokecolor{currentstroke}%
\pgfsetdash{{2.000000pt}{3.300000pt}}{0.000000pt}%
\pgfpathmoveto{\pgfqpoint{3.326364in}{1.647524in}}%
\pgfpathlineto{\pgfqpoint{3.326364in}{2.476531in}}%
\pgfusepath{stroke}%
\end{pgfscope}%
\begin{pgfscope}%
\pgfsetrectcap%
\pgfsetmiterjoin%
\pgfsetlinewidth{0.803000pt}%
\definecolor{currentstroke}{rgb}{0.737255,0.737255,0.737255}%
\pgfsetstrokecolor{currentstroke}%
\pgfsetdash{}{0pt}%
\pgfpathmoveto{\pgfqpoint{0.691161in}{1.606074in}}%
\pgfpathlineto{\pgfqpoint{0.691161in}{2.517981in}}%
\pgfusepath{stroke}%
\end{pgfscope}%
\begin{pgfscope}%
\pgfsetrectcap%
\pgfsetmiterjoin%
\pgfsetlinewidth{0.803000pt}%
\definecolor{currentstroke}{rgb}{0.737255,0.737255,0.737255}%
\pgfsetstrokecolor{currentstroke}%
\pgfsetdash{}{0pt}%
\pgfpathmoveto{\pgfqpoint{5.745833in}{1.606074in}}%
\pgfpathlineto{\pgfqpoint{5.745833in}{2.517981in}}%
\pgfusepath{stroke}%
\end{pgfscope}%
\begin{pgfscope}%
\pgfsetrectcap%
\pgfsetmiterjoin%
\pgfsetlinewidth{0.803000pt}%
\definecolor{currentstroke}{rgb}{0.737255,0.737255,0.737255}%
\pgfsetstrokecolor{currentstroke}%
\pgfsetdash{}{0pt}%
\pgfpathmoveto{\pgfqpoint{0.691161in}{1.606074in}}%
\pgfpathlineto{\pgfqpoint{5.745833in}{1.606074in}}%
\pgfusepath{stroke}%
\end{pgfscope}%
\begin{pgfscope}%
\pgfsetrectcap%
\pgfsetmiterjoin%
\pgfsetlinewidth{0.803000pt}%
\definecolor{currentstroke}{rgb}{0.737255,0.737255,0.737255}%
\pgfsetstrokecolor{currentstroke}%
\pgfsetdash{}{0pt}%
\pgfpathmoveto{\pgfqpoint{0.691161in}{2.517981in}}%
\pgfpathlineto{\pgfqpoint{5.745833in}{2.517981in}}%
\pgfusepath{stroke}%
\end{pgfscope}%
\begin{pgfscope}%
\pgfsetbuttcap%
\pgfsetmiterjoin%
\definecolor{currentfill}{rgb}{0.933333,0.933333,0.933333}%
\pgfsetfillcolor{currentfill}%
\pgfsetfillopacity{0.800000}%
\pgfsetlinewidth{0.501875pt}%
\definecolor{currentstroke}{rgb}{0.800000,0.800000,0.800000}%
\pgfsetstrokecolor{currentstroke}%
\pgfsetstrokeopacity{0.800000}%
\pgfsetdash{}{0pt}%
\pgfpathmoveto{\pgfqpoint{4.404343in}{1.826037in}}%
\pgfpathlineto{\pgfqpoint{5.648611in}{1.826037in}}%
\pgfpathquadraticcurveto{\pgfqpoint{5.676389in}{1.826037in}}{\pgfqpoint{5.676389in}{1.853815in}}%
\pgfpathlineto{\pgfqpoint{5.676389in}{2.420759in}}%
\pgfpathquadraticcurveto{\pgfqpoint{5.676389in}{2.448537in}}{\pgfqpoint{5.648611in}{2.448537in}}%
\pgfpathlineto{\pgfqpoint{4.404343in}{2.448537in}}%
\pgfpathquadraticcurveto{\pgfqpoint{4.376566in}{2.448537in}}{\pgfqpoint{4.376566in}{2.420759in}}%
\pgfpathlineto{\pgfqpoint{4.376566in}{1.853815in}}%
\pgfpathquadraticcurveto{\pgfqpoint{4.376566in}{1.826037in}}{\pgfqpoint{4.404343in}{1.826037in}}%
\pgfpathlineto{\pgfqpoint{4.404343in}{1.826037in}}%
\pgfpathclose%
\pgfusepath{stroke,fill}%
\end{pgfscope}%
\begin{pgfscope}%
\pgfsetbuttcap%
\pgfsetroundjoin%
\pgfsetlinewidth{2.007500pt}%
\definecolor{currentstroke}{rgb}{0.172549,0.627451,0.172549}%
\pgfsetstrokecolor{currentstroke}%
\pgfsetdash{{7.400000pt}{3.200000pt}}{0.000000pt}%
\pgfpathmoveto{\pgfqpoint{4.432121in}{2.344370in}}%
\pgfpathlineto{\pgfqpoint{4.709899in}{2.344370in}}%
\pgfusepath{stroke}%
\end{pgfscope}%
\begin{pgfscope}%
\definecolor{textcolor}{rgb}{0.000000,0.000000,0.000000}%
\pgfsetstrokecolor{textcolor}%
\pgfsetfillcolor{textcolor}%
\pgftext[x=4.821010in,y=2.295759in,left,base]{\color{textcolor}\rmfamily\fontsize{10.000000}{12.000000}\selectfont Seuil = 5}%
\end{pgfscope}%
\begin{pgfscope}%
\pgfsetbuttcap%
\pgfsetroundjoin%
\pgfsetlinewidth{2.007500pt}%
\definecolor{currentstroke}{rgb}{0.839216,0.152941,0.156863}%
\pgfsetstrokecolor{currentstroke}%
\pgfsetdash{{2.000000pt}{3.300000pt}}{0.000000pt}%
\pgfpathmoveto{\pgfqpoint{4.432121in}{2.150759in}}%
\pgfpathlineto{\pgfqpoint{4.709899in}{2.150759in}}%
\pgfusepath{stroke}%
\end{pgfscope}%
\begin{pgfscope}%
\definecolor{textcolor}{rgb}{0.000000,0.000000,0.000000}%
\pgfsetstrokecolor{textcolor}%
\pgfsetfillcolor{textcolor}%
\pgftext[x=4.821010in,y=2.102148in,left,base]{\color{textcolor}\rmfamily\fontsize{10.000000}{12.000000}\selectfont \(\displaystyle t_1\) = 164.37 s}%
\end{pgfscope}%
\begin{pgfscope}%
\pgfsetbuttcap%
\pgfsetroundjoin%
\pgfsetlinewidth{2.007500pt}%
\definecolor{currentstroke}{rgb}{1.000000,0.498039,0.054902}%
\pgfsetstrokecolor{currentstroke}%
\pgfsetdash{{2.000000pt}{3.300000pt}}{0.000000pt}%
\pgfpathmoveto{\pgfqpoint{4.432121in}{1.957148in}}%
\pgfpathlineto{\pgfqpoint{4.709899in}{1.957148in}}%
\pgfusepath{stroke}%
\end{pgfscope}%
\begin{pgfscope}%
\definecolor{textcolor}{rgb}{0.000000,0.000000,0.000000}%
\pgfsetstrokecolor{textcolor}%
\pgfsetfillcolor{textcolor}%
\pgftext[x=4.821010in,y=1.908537in,left,base]{\color{textcolor}\rmfamily\fontsize{10.000000}{12.000000}\selectfont \(\displaystyle t_2\) = 180.34 s}%
\end{pgfscope}%
\begin{pgfscope}%
\pgfsetbuttcap%
\pgfsetmiterjoin%
\definecolor{currentfill}{rgb}{0.933333,0.933333,0.933333}%
\pgfsetfillcolor{currentfill}%
\pgfsetlinewidth{0.000000pt}%
\definecolor{currentstroke}{rgb}{0.000000,0.000000,0.000000}%
\pgfsetstrokecolor{currentstroke}%
\pgfsetstrokeopacity{0.000000}%
\pgfsetdash{}{0pt}%
\pgfpathmoveto{\pgfqpoint{0.691161in}{0.544166in}}%
\pgfpathlineto{\pgfqpoint{5.745833in}{0.544166in}}%
\pgfpathlineto{\pgfqpoint{5.745833in}{1.456074in}}%
\pgfpathlineto{\pgfqpoint{0.691161in}{1.456074in}}%
\pgfpathlineto{\pgfqpoint{0.691161in}{0.544166in}}%
\pgfpathclose%
\pgfusepath{fill}%
\end{pgfscope}%
\begin{pgfscope}%
\pgfpathrectangle{\pgfqpoint{0.691161in}{0.544166in}}{\pgfqpoint{5.054672in}{0.911907in}}%
\pgfusepath{clip}%
\pgfsetbuttcap%
\pgfsetroundjoin%
\pgfsetlinewidth{0.501875pt}%
\definecolor{currentstroke}{rgb}{0.698039,0.698039,0.698039}%
\pgfsetstrokecolor{currentstroke}%
\pgfsetdash{{1.850000pt}{0.800000pt}}{0.000000pt}%
\pgfpathmoveto{\pgfqpoint{0.691161in}{0.544166in}}%
\pgfpathlineto{\pgfqpoint{0.691161in}{1.456074in}}%
\pgfusepath{stroke}%
\end{pgfscope}%
\begin{pgfscope}%
\pgfsetbuttcap%
\pgfsetroundjoin%
\definecolor{currentfill}{rgb}{0.000000,0.000000,0.000000}%
\pgfsetfillcolor{currentfill}%
\pgfsetlinewidth{0.803000pt}%
\definecolor{currentstroke}{rgb}{0.000000,0.000000,0.000000}%
\pgfsetstrokecolor{currentstroke}%
\pgfsetdash{}{0pt}%
\pgfsys@defobject{currentmarker}{\pgfqpoint{0.000000in}{0.000000in}}{\pgfqpoint{0.000000in}{0.048611in}}{%
\pgfpathmoveto{\pgfqpoint{0.000000in}{0.000000in}}%
\pgfpathlineto{\pgfqpoint{0.000000in}{0.048611in}}%
\pgfusepath{stroke,fill}%
}%
\begin{pgfscope}%
\pgfsys@transformshift{0.691161in}{0.544166in}%
\pgfsys@useobject{currentmarker}{}%
\end{pgfscope}%
\end{pgfscope}%
\begin{pgfscope}%
\definecolor{textcolor}{rgb}{0.000000,0.000000,0.000000}%
\pgfsetstrokecolor{textcolor}%
\pgfsetfillcolor{textcolor}%
\pgftext[x=0.691161in,y=0.495555in,,top]{\color{textcolor}\rmfamily\fontsize{10.000000}{12.000000}\selectfont \(\displaystyle {50}\)}%
\end{pgfscope}%
\begin{pgfscope}%
\pgfpathrectangle{\pgfqpoint{0.691161in}{0.544166in}}{\pgfqpoint{5.054672in}{0.911907in}}%
\pgfusepath{clip}%
\pgfsetbuttcap%
\pgfsetroundjoin%
\pgfsetlinewidth{0.501875pt}%
\definecolor{currentstroke}{rgb}{0.698039,0.698039,0.698039}%
\pgfsetstrokecolor{currentstroke}%
\pgfsetdash{{1.850000pt}{0.800000pt}}{0.000000pt}%
\pgfpathmoveto{\pgfqpoint{1.702096in}{0.544166in}}%
\pgfpathlineto{\pgfqpoint{1.702096in}{1.456074in}}%
\pgfusepath{stroke}%
\end{pgfscope}%
\begin{pgfscope}%
\pgfsetbuttcap%
\pgfsetroundjoin%
\definecolor{currentfill}{rgb}{0.000000,0.000000,0.000000}%
\pgfsetfillcolor{currentfill}%
\pgfsetlinewidth{0.803000pt}%
\definecolor{currentstroke}{rgb}{0.000000,0.000000,0.000000}%
\pgfsetstrokecolor{currentstroke}%
\pgfsetdash{}{0pt}%
\pgfsys@defobject{currentmarker}{\pgfqpoint{0.000000in}{0.000000in}}{\pgfqpoint{0.000000in}{0.048611in}}{%
\pgfpathmoveto{\pgfqpoint{0.000000in}{0.000000in}}%
\pgfpathlineto{\pgfqpoint{0.000000in}{0.048611in}}%
\pgfusepath{stroke,fill}%
}%
\begin{pgfscope}%
\pgfsys@transformshift{1.702096in}{0.544166in}%
\pgfsys@useobject{currentmarker}{}%
\end{pgfscope}%
\end{pgfscope}%
\begin{pgfscope}%
\definecolor{textcolor}{rgb}{0.000000,0.000000,0.000000}%
\pgfsetstrokecolor{textcolor}%
\pgfsetfillcolor{textcolor}%
\pgftext[x=1.702096in,y=0.495555in,,top]{\color{textcolor}\rmfamily\fontsize{10.000000}{12.000000}\selectfont \(\displaystyle {100}\)}%
\end{pgfscope}%
\begin{pgfscope}%
\pgfpathrectangle{\pgfqpoint{0.691161in}{0.544166in}}{\pgfqpoint{5.054672in}{0.911907in}}%
\pgfusepath{clip}%
\pgfsetbuttcap%
\pgfsetroundjoin%
\pgfsetlinewidth{0.501875pt}%
\definecolor{currentstroke}{rgb}{0.698039,0.698039,0.698039}%
\pgfsetstrokecolor{currentstroke}%
\pgfsetdash{{1.850000pt}{0.800000pt}}{0.000000pt}%
\pgfpathmoveto{\pgfqpoint{2.713030in}{0.544166in}}%
\pgfpathlineto{\pgfqpoint{2.713030in}{1.456074in}}%
\pgfusepath{stroke}%
\end{pgfscope}%
\begin{pgfscope}%
\pgfsetbuttcap%
\pgfsetroundjoin%
\definecolor{currentfill}{rgb}{0.000000,0.000000,0.000000}%
\pgfsetfillcolor{currentfill}%
\pgfsetlinewidth{0.803000pt}%
\definecolor{currentstroke}{rgb}{0.000000,0.000000,0.000000}%
\pgfsetstrokecolor{currentstroke}%
\pgfsetdash{}{0pt}%
\pgfsys@defobject{currentmarker}{\pgfqpoint{0.000000in}{0.000000in}}{\pgfqpoint{0.000000in}{0.048611in}}{%
\pgfpathmoveto{\pgfqpoint{0.000000in}{0.000000in}}%
\pgfpathlineto{\pgfqpoint{0.000000in}{0.048611in}}%
\pgfusepath{stroke,fill}%
}%
\begin{pgfscope}%
\pgfsys@transformshift{2.713030in}{0.544166in}%
\pgfsys@useobject{currentmarker}{}%
\end{pgfscope}%
\end{pgfscope}%
\begin{pgfscope}%
\definecolor{textcolor}{rgb}{0.000000,0.000000,0.000000}%
\pgfsetstrokecolor{textcolor}%
\pgfsetfillcolor{textcolor}%
\pgftext[x=2.713030in,y=0.495555in,,top]{\color{textcolor}\rmfamily\fontsize{10.000000}{12.000000}\selectfont \(\displaystyle {150}\)}%
\end{pgfscope}%
\begin{pgfscope}%
\pgfpathrectangle{\pgfqpoint{0.691161in}{0.544166in}}{\pgfqpoint{5.054672in}{0.911907in}}%
\pgfusepath{clip}%
\pgfsetbuttcap%
\pgfsetroundjoin%
\pgfsetlinewidth{0.501875pt}%
\definecolor{currentstroke}{rgb}{0.698039,0.698039,0.698039}%
\pgfsetstrokecolor{currentstroke}%
\pgfsetdash{{1.850000pt}{0.800000pt}}{0.000000pt}%
\pgfpathmoveto{\pgfqpoint{3.723964in}{0.544166in}}%
\pgfpathlineto{\pgfqpoint{3.723964in}{1.456074in}}%
\pgfusepath{stroke}%
\end{pgfscope}%
\begin{pgfscope}%
\pgfsetbuttcap%
\pgfsetroundjoin%
\definecolor{currentfill}{rgb}{0.000000,0.000000,0.000000}%
\pgfsetfillcolor{currentfill}%
\pgfsetlinewidth{0.803000pt}%
\definecolor{currentstroke}{rgb}{0.000000,0.000000,0.000000}%
\pgfsetstrokecolor{currentstroke}%
\pgfsetdash{}{0pt}%
\pgfsys@defobject{currentmarker}{\pgfqpoint{0.000000in}{0.000000in}}{\pgfqpoint{0.000000in}{0.048611in}}{%
\pgfpathmoveto{\pgfqpoint{0.000000in}{0.000000in}}%
\pgfpathlineto{\pgfqpoint{0.000000in}{0.048611in}}%
\pgfusepath{stroke,fill}%
}%
\begin{pgfscope}%
\pgfsys@transformshift{3.723964in}{0.544166in}%
\pgfsys@useobject{currentmarker}{}%
\end{pgfscope}%
\end{pgfscope}%
\begin{pgfscope}%
\definecolor{textcolor}{rgb}{0.000000,0.000000,0.000000}%
\pgfsetstrokecolor{textcolor}%
\pgfsetfillcolor{textcolor}%
\pgftext[x=3.723964in,y=0.495555in,,top]{\color{textcolor}\rmfamily\fontsize{10.000000}{12.000000}\selectfont \(\displaystyle {200}\)}%
\end{pgfscope}%
\begin{pgfscope}%
\pgfpathrectangle{\pgfqpoint{0.691161in}{0.544166in}}{\pgfqpoint{5.054672in}{0.911907in}}%
\pgfusepath{clip}%
\pgfsetbuttcap%
\pgfsetroundjoin%
\pgfsetlinewidth{0.501875pt}%
\definecolor{currentstroke}{rgb}{0.698039,0.698039,0.698039}%
\pgfsetstrokecolor{currentstroke}%
\pgfsetdash{{1.850000pt}{0.800000pt}}{0.000000pt}%
\pgfpathmoveto{\pgfqpoint{4.734899in}{0.544166in}}%
\pgfpathlineto{\pgfqpoint{4.734899in}{1.456074in}}%
\pgfusepath{stroke}%
\end{pgfscope}%
\begin{pgfscope}%
\pgfsetbuttcap%
\pgfsetroundjoin%
\definecolor{currentfill}{rgb}{0.000000,0.000000,0.000000}%
\pgfsetfillcolor{currentfill}%
\pgfsetlinewidth{0.803000pt}%
\definecolor{currentstroke}{rgb}{0.000000,0.000000,0.000000}%
\pgfsetstrokecolor{currentstroke}%
\pgfsetdash{}{0pt}%
\pgfsys@defobject{currentmarker}{\pgfqpoint{0.000000in}{0.000000in}}{\pgfqpoint{0.000000in}{0.048611in}}{%
\pgfpathmoveto{\pgfqpoint{0.000000in}{0.000000in}}%
\pgfpathlineto{\pgfqpoint{0.000000in}{0.048611in}}%
\pgfusepath{stroke,fill}%
}%
\begin{pgfscope}%
\pgfsys@transformshift{4.734899in}{0.544166in}%
\pgfsys@useobject{currentmarker}{}%
\end{pgfscope}%
\end{pgfscope}%
\begin{pgfscope}%
\definecolor{textcolor}{rgb}{0.000000,0.000000,0.000000}%
\pgfsetstrokecolor{textcolor}%
\pgfsetfillcolor{textcolor}%
\pgftext[x=4.734899in,y=0.495555in,,top]{\color{textcolor}\rmfamily\fontsize{10.000000}{12.000000}\selectfont \(\displaystyle {250}\)}%
\end{pgfscope}%
\begin{pgfscope}%
\pgfpathrectangle{\pgfqpoint{0.691161in}{0.544166in}}{\pgfqpoint{5.054672in}{0.911907in}}%
\pgfusepath{clip}%
\pgfsetbuttcap%
\pgfsetroundjoin%
\pgfsetlinewidth{0.501875pt}%
\definecolor{currentstroke}{rgb}{0.698039,0.698039,0.698039}%
\pgfsetstrokecolor{currentstroke}%
\pgfsetdash{{1.850000pt}{0.800000pt}}{0.000000pt}%
\pgfpathmoveto{\pgfqpoint{5.745833in}{0.544166in}}%
\pgfpathlineto{\pgfqpoint{5.745833in}{1.456074in}}%
\pgfusepath{stroke}%
\end{pgfscope}%
\begin{pgfscope}%
\pgfsetbuttcap%
\pgfsetroundjoin%
\definecolor{currentfill}{rgb}{0.000000,0.000000,0.000000}%
\pgfsetfillcolor{currentfill}%
\pgfsetlinewidth{0.803000pt}%
\definecolor{currentstroke}{rgb}{0.000000,0.000000,0.000000}%
\pgfsetstrokecolor{currentstroke}%
\pgfsetdash{}{0pt}%
\pgfsys@defobject{currentmarker}{\pgfqpoint{0.000000in}{0.000000in}}{\pgfqpoint{0.000000in}{0.048611in}}{%
\pgfpathmoveto{\pgfqpoint{0.000000in}{0.000000in}}%
\pgfpathlineto{\pgfqpoint{0.000000in}{0.048611in}}%
\pgfusepath{stroke,fill}%
}%
\begin{pgfscope}%
\pgfsys@transformshift{5.745833in}{0.544166in}%
\pgfsys@useobject{currentmarker}{}%
\end{pgfscope}%
\end{pgfscope}%
\begin{pgfscope}%
\definecolor{textcolor}{rgb}{0.000000,0.000000,0.000000}%
\pgfsetstrokecolor{textcolor}%
\pgfsetfillcolor{textcolor}%
\pgftext[x=5.745833in,y=0.495555in,,top]{\color{textcolor}\rmfamily\fontsize{10.000000}{12.000000}\selectfont \(\displaystyle {300}\)}%
\end{pgfscope}%
\begin{pgfscope}%
\definecolor{textcolor}{rgb}{0.000000,0.000000,0.000000}%
\pgfsetstrokecolor{textcolor}%
\pgfsetfillcolor{textcolor}%
\pgftext[x=3.218497in,y=0.316666in,,top]{\color{textcolor}\rmfamily\fontsize{12.000000}{14.400000}\selectfont Temps [s]}%
\end{pgfscope}%
\begin{pgfscope}%
\pgfpathrectangle{\pgfqpoint{0.691161in}{0.544166in}}{\pgfqpoint{5.054672in}{0.911907in}}%
\pgfusepath{clip}%
\pgfsetbuttcap%
\pgfsetroundjoin%
\pgfsetlinewidth{0.501875pt}%
\definecolor{currentstroke}{rgb}{0.698039,0.698039,0.698039}%
\pgfsetstrokecolor{currentstroke}%
\pgfsetdash{{1.850000pt}{0.800000pt}}{0.000000pt}%
\pgfpathmoveto{\pgfqpoint{0.691161in}{0.709870in}}%
\pgfpathlineto{\pgfqpoint{5.745833in}{0.709870in}}%
\pgfusepath{stroke}%
\end{pgfscope}%
\begin{pgfscope}%
\pgfsetbuttcap%
\pgfsetroundjoin%
\definecolor{currentfill}{rgb}{0.000000,0.000000,0.000000}%
\pgfsetfillcolor{currentfill}%
\pgfsetlinewidth{0.803000pt}%
\definecolor{currentstroke}{rgb}{0.000000,0.000000,0.000000}%
\pgfsetstrokecolor{currentstroke}%
\pgfsetdash{}{0pt}%
\pgfsys@defobject{currentmarker}{\pgfqpoint{0.000000in}{0.000000in}}{\pgfqpoint{0.048611in}{0.000000in}}{%
\pgfpathmoveto{\pgfqpoint{0.000000in}{0.000000in}}%
\pgfpathlineto{\pgfqpoint{0.048611in}{0.000000in}}%
\pgfusepath{stroke,fill}%
}%
\begin{pgfscope}%
\pgfsys@transformshift{0.691161in}{0.709870in}%
\pgfsys@useobject{currentmarker}{}%
\end{pgfscope}%
\end{pgfscope}%
\begin{pgfscope}%
\definecolor{textcolor}{rgb}{0.000000,0.000000,0.000000}%
\pgfsetstrokecolor{textcolor}%
\pgfsetfillcolor{textcolor}%
\pgftext[x=0.573105in, y=0.661676in, left, base]{\color{textcolor}\rmfamily\fontsize{10.000000}{12.000000}\selectfont \(\displaystyle {0}\)}%
\end{pgfscope}%
\begin{pgfscope}%
\pgfpathrectangle{\pgfqpoint{0.691161in}{0.544166in}}{\pgfqpoint{5.054672in}{0.911907in}}%
\pgfusepath{clip}%
\pgfsetbuttcap%
\pgfsetroundjoin%
\pgfsetlinewidth{0.501875pt}%
\definecolor{currentstroke}{rgb}{0.698039,0.698039,0.698039}%
\pgfsetstrokecolor{currentstroke}%
\pgfsetdash{{1.850000pt}{0.800000pt}}{0.000000pt}%
\pgfpathmoveto{\pgfqpoint{0.691161in}{1.045394in}}%
\pgfpathlineto{\pgfqpoint{5.745833in}{1.045394in}}%
\pgfusepath{stroke}%
\end{pgfscope}%
\begin{pgfscope}%
\pgfsetbuttcap%
\pgfsetroundjoin%
\definecolor{currentfill}{rgb}{0.000000,0.000000,0.000000}%
\pgfsetfillcolor{currentfill}%
\pgfsetlinewidth{0.803000pt}%
\definecolor{currentstroke}{rgb}{0.000000,0.000000,0.000000}%
\pgfsetstrokecolor{currentstroke}%
\pgfsetdash{}{0pt}%
\pgfsys@defobject{currentmarker}{\pgfqpoint{0.000000in}{0.000000in}}{\pgfqpoint{0.048611in}{0.000000in}}{%
\pgfpathmoveto{\pgfqpoint{0.000000in}{0.000000in}}%
\pgfpathlineto{\pgfqpoint{0.048611in}{0.000000in}}%
\pgfusepath{stroke,fill}%
}%
\begin{pgfscope}%
\pgfsys@transformshift{0.691161in}{1.045394in}%
\pgfsys@useobject{currentmarker}{}%
\end{pgfscope}%
\end{pgfscope}%
\begin{pgfscope}%
\definecolor{textcolor}{rgb}{0.000000,0.000000,0.000000}%
\pgfsetstrokecolor{textcolor}%
\pgfsetfillcolor{textcolor}%
\pgftext[x=0.434216in, y=0.997200in, left, base]{\color{textcolor}\rmfamily\fontsize{10.000000}{12.000000}\selectfont \(\displaystyle {250}\)}%
\end{pgfscope}%
\begin{pgfscope}%
\pgfpathrectangle{\pgfqpoint{0.691161in}{0.544166in}}{\pgfqpoint{5.054672in}{0.911907in}}%
\pgfusepath{clip}%
\pgfsetbuttcap%
\pgfsetroundjoin%
\pgfsetlinewidth{0.501875pt}%
\definecolor{currentstroke}{rgb}{0.698039,0.698039,0.698039}%
\pgfsetstrokecolor{currentstroke}%
\pgfsetdash{{1.850000pt}{0.800000pt}}{0.000000pt}%
\pgfpathmoveto{\pgfqpoint{0.691161in}{1.380918in}}%
\pgfpathlineto{\pgfqpoint{5.745833in}{1.380918in}}%
\pgfusepath{stroke}%
\end{pgfscope}%
\begin{pgfscope}%
\pgfsetbuttcap%
\pgfsetroundjoin%
\definecolor{currentfill}{rgb}{0.000000,0.000000,0.000000}%
\pgfsetfillcolor{currentfill}%
\pgfsetlinewidth{0.803000pt}%
\definecolor{currentstroke}{rgb}{0.000000,0.000000,0.000000}%
\pgfsetstrokecolor{currentstroke}%
\pgfsetdash{}{0pt}%
\pgfsys@defobject{currentmarker}{\pgfqpoint{0.000000in}{0.000000in}}{\pgfqpoint{0.048611in}{0.000000in}}{%
\pgfpathmoveto{\pgfqpoint{0.000000in}{0.000000in}}%
\pgfpathlineto{\pgfqpoint{0.048611in}{0.000000in}}%
\pgfusepath{stroke,fill}%
}%
\begin{pgfscope}%
\pgfsys@transformshift{0.691161in}{1.380918in}%
\pgfsys@useobject{currentmarker}{}%
\end{pgfscope}%
\end{pgfscope}%
\begin{pgfscope}%
\definecolor{textcolor}{rgb}{0.000000,0.000000,0.000000}%
\pgfsetstrokecolor{textcolor}%
\pgfsetfillcolor{textcolor}%
\pgftext[x=0.434216in, y=1.332724in, left, base]{\color{textcolor}\rmfamily\fontsize{10.000000}{12.000000}\selectfont \(\displaystyle {500}\)}%
\end{pgfscope}%
\begin{pgfscope}%
\definecolor{textcolor}{rgb}{0.000000,0.000000,0.000000}%
\pgfsetstrokecolor{textcolor}%
\pgfsetfillcolor{textcolor}%
\pgftext[x=0.378661in,y=1.000120in,,bottom,rotate=90.000000]{\color{textcolor}\rmfamily\fontsize{12.000000}{14.400000}\selectfont Baer}%
\end{pgfscope}%
\begin{pgfscope}%
\pgfpathrectangle{\pgfqpoint{0.691161in}{0.544166in}}{\pgfqpoint{5.054672in}{0.911907in}}%
\pgfusepath{clip}%
\pgfsetrectcap%
\pgfsetroundjoin%
\pgfsetlinewidth{1.505625pt}%
\definecolor{currentstroke}{rgb}{0.121569,0.466667,0.705882}%
\pgfsetstrokecolor{currentstroke}%
\pgfsetdash{}{0pt}%
\pgfpathmoveto{\pgfqpoint{0.691060in}{0.711896in}}%
\pgfpathlineto{\pgfqpoint{0.691869in}{0.744549in}}%
\pgfpathlineto{\pgfqpoint{0.692577in}{0.720058in}}%
\pgfpathlineto{\pgfqpoint{0.693689in}{0.711972in}}%
\pgfpathlineto{\pgfqpoint{0.694194in}{0.715949in}}%
\pgfpathlineto{\pgfqpoint{0.694295in}{0.716687in}}%
\pgfpathlineto{\pgfqpoint{0.694598in}{0.713050in}}%
\pgfpathlineto{\pgfqpoint{0.695306in}{0.715427in}}%
\pgfpathlineto{\pgfqpoint{0.695913in}{0.708211in}}%
\pgfpathlineto{\pgfqpoint{0.696216in}{0.707795in}}%
\pgfpathlineto{\pgfqpoint{0.696721in}{0.708734in}}%
\pgfpathlineto{\pgfqpoint{0.697126in}{0.712318in}}%
\pgfpathlineto{\pgfqpoint{0.698743in}{0.749357in}}%
\pgfpathlineto{\pgfqpoint{0.699047in}{0.738342in}}%
\pgfpathlineto{\pgfqpoint{0.700462in}{0.708296in}}%
\pgfpathlineto{\pgfqpoint{0.701877in}{0.707432in}}%
\pgfpathlineto{\pgfqpoint{0.714716in}{0.710465in}}%
\pgfpathlineto{\pgfqpoint{0.715828in}{0.711776in}}%
\pgfpathlineto{\pgfqpoint{0.715221in}{0.709368in}}%
\pgfpathlineto{\pgfqpoint{0.716030in}{0.711349in}}%
\pgfpathlineto{\pgfqpoint{0.716839in}{0.709954in}}%
\pgfpathlineto{\pgfqpoint{0.717142in}{0.710575in}}%
\pgfpathlineto{\pgfqpoint{0.717749in}{0.720637in}}%
\pgfpathlineto{\pgfqpoint{0.718254in}{0.711679in}}%
\pgfpathlineto{\pgfqpoint{0.718355in}{0.711212in}}%
\pgfpathlineto{\pgfqpoint{0.718558in}{0.712964in}}%
\pgfpathlineto{\pgfqpoint{0.718962in}{0.722229in}}%
\pgfpathlineto{\pgfqpoint{0.719467in}{0.710392in}}%
\pgfpathlineto{\pgfqpoint{0.720681in}{0.709159in}}%
\pgfpathlineto{\pgfqpoint{0.720984in}{0.709638in}}%
\pgfpathlineto{\pgfqpoint{0.722500in}{0.733444in}}%
\pgfpathlineto{\pgfqpoint{0.722905in}{0.719244in}}%
\pgfpathlineto{\pgfqpoint{0.724219in}{0.709442in}}%
\pgfpathlineto{\pgfqpoint{0.724522in}{0.709669in}}%
\pgfpathlineto{\pgfqpoint{0.724623in}{0.710059in}}%
\pgfpathlineto{\pgfqpoint{0.725937in}{0.713186in}}%
\pgfpathlineto{\pgfqpoint{0.725331in}{0.708775in}}%
\pgfpathlineto{\pgfqpoint{0.726038in}{0.712876in}}%
\pgfpathlineto{\pgfqpoint{0.727555in}{0.708543in}}%
\pgfpathlineto{\pgfqpoint{0.727858in}{0.708848in}}%
\pgfpathlineto{\pgfqpoint{0.729273in}{0.734683in}}%
\pgfpathlineto{\pgfqpoint{0.729476in}{0.721420in}}%
\pgfpathlineto{\pgfqpoint{0.729880in}{0.708311in}}%
\pgfpathlineto{\pgfqpoint{0.730689in}{0.708475in}}%
\pgfpathlineto{\pgfqpoint{0.732205in}{0.708245in}}%
\pgfpathlineto{\pgfqpoint{0.733519in}{0.710773in}}%
\pgfpathlineto{\pgfqpoint{0.733823in}{0.715289in}}%
\pgfpathlineto{\pgfqpoint{0.734429in}{0.707794in}}%
\pgfpathlineto{\pgfqpoint{0.734935in}{0.709622in}}%
\pgfpathlineto{\pgfqpoint{0.735845in}{0.707723in}}%
\pgfpathlineto{\pgfqpoint{0.736350in}{0.707815in}}%
\pgfpathlineto{\pgfqpoint{0.736451in}{0.708253in}}%
\pgfpathlineto{\pgfqpoint{0.736653in}{0.709026in}}%
\pgfpathlineto{\pgfqpoint{0.737058in}{0.707929in}}%
\pgfpathlineto{\pgfqpoint{0.737462in}{0.708297in}}%
\pgfpathlineto{\pgfqpoint{0.737866in}{0.707943in}}%
\pgfpathlineto{\pgfqpoint{0.738271in}{0.709161in}}%
\pgfpathlineto{\pgfqpoint{0.738574in}{0.708316in}}%
\pgfpathlineto{\pgfqpoint{0.738776in}{0.708959in}}%
\pgfpathlineto{\pgfqpoint{0.739282in}{0.734429in}}%
\pgfpathlineto{\pgfqpoint{0.739989in}{0.711455in}}%
\pgfpathlineto{\pgfqpoint{0.740090in}{0.711568in}}%
\pgfpathlineto{\pgfqpoint{0.740394in}{0.710306in}}%
\pgfpathlineto{\pgfqpoint{0.741809in}{0.708190in}}%
\pgfpathlineto{\pgfqpoint{0.743325in}{0.708498in}}%
\pgfpathlineto{\pgfqpoint{0.744842in}{0.712138in}}%
\pgfpathlineto{\pgfqpoint{0.745044in}{0.711003in}}%
\pgfpathlineto{\pgfqpoint{0.745752in}{0.708382in}}%
\pgfpathlineto{\pgfqpoint{0.746358in}{0.708858in}}%
\pgfpathlineto{\pgfqpoint{0.747470in}{0.714134in}}%
\pgfpathlineto{\pgfqpoint{0.747774in}{0.710960in}}%
\pgfpathlineto{\pgfqpoint{0.748987in}{0.708578in}}%
\pgfpathlineto{\pgfqpoint{0.749694in}{0.709336in}}%
\pgfpathlineto{\pgfqpoint{0.751110in}{0.723756in}}%
\pgfpathlineto{\pgfqpoint{0.751413in}{0.737238in}}%
\pgfpathlineto{\pgfqpoint{0.752019in}{0.709579in}}%
\pgfpathlineto{\pgfqpoint{0.753030in}{0.708501in}}%
\pgfpathlineto{\pgfqpoint{0.753233in}{0.708708in}}%
\pgfpathlineto{\pgfqpoint{0.753637in}{0.714644in}}%
\pgfpathlineto{\pgfqpoint{0.753940in}{0.722043in}}%
\pgfpathlineto{\pgfqpoint{0.754547in}{0.712368in}}%
\pgfpathlineto{\pgfqpoint{0.754648in}{0.712422in}}%
\pgfpathlineto{\pgfqpoint{0.755254in}{0.718910in}}%
\pgfpathlineto{\pgfqpoint{0.755558in}{0.725221in}}%
\pgfpathlineto{\pgfqpoint{0.756063in}{0.714277in}}%
\pgfpathlineto{\pgfqpoint{0.757479in}{0.708554in}}%
\pgfpathlineto{\pgfqpoint{0.758894in}{0.708068in}}%
\pgfpathlineto{\pgfqpoint{0.760107in}{0.708845in}}%
\pgfpathlineto{\pgfqpoint{0.760612in}{0.708160in}}%
\pgfpathlineto{\pgfqpoint{0.761421in}{0.708899in}}%
\pgfpathlineto{\pgfqpoint{0.761826in}{0.712349in}}%
\pgfpathlineto{\pgfqpoint{0.762634in}{0.710063in}}%
\pgfpathlineto{\pgfqpoint{0.763039in}{0.710085in}}%
\pgfpathlineto{\pgfqpoint{0.763241in}{0.710767in}}%
\pgfpathlineto{\pgfqpoint{0.765465in}{0.731461in}}%
\pgfpathlineto{\pgfqpoint{0.765768in}{0.723184in}}%
\pgfpathlineto{\pgfqpoint{0.766375in}{0.709513in}}%
\pgfpathlineto{\pgfqpoint{0.767082in}{0.711323in}}%
\pgfpathlineto{\pgfqpoint{0.769408in}{0.707401in}}%
\pgfpathlineto{\pgfqpoint{0.771227in}{0.707871in}}%
\pgfpathlineto{\pgfqpoint{0.772339in}{0.711872in}}%
\pgfpathlineto{\pgfqpoint{0.772541in}{0.713707in}}%
\pgfpathlineto{\pgfqpoint{0.773148in}{0.709694in}}%
\pgfpathlineto{\pgfqpoint{0.774361in}{0.707668in}}%
\pgfpathlineto{\pgfqpoint{0.774766in}{0.707826in}}%
\pgfpathlineto{\pgfqpoint{0.776990in}{0.708799in}}%
\pgfpathlineto{\pgfqpoint{0.778506in}{0.717283in}}%
\pgfpathlineto{\pgfqpoint{0.778607in}{0.716598in}}%
\pgfpathlineto{\pgfqpoint{0.780225in}{0.707906in}}%
\pgfpathlineto{\pgfqpoint{0.783358in}{0.708413in}}%
\pgfpathlineto{\pgfqpoint{0.784572in}{0.708507in}}%
\pgfpathlineto{\pgfqpoint{0.790435in}{0.710777in}}%
\pgfpathlineto{\pgfqpoint{0.790536in}{0.710898in}}%
\pgfpathlineto{\pgfqpoint{0.790738in}{0.710077in}}%
\pgfpathlineto{\pgfqpoint{0.792052in}{0.708913in}}%
\pgfpathlineto{\pgfqpoint{0.793165in}{0.709561in}}%
\pgfpathlineto{\pgfqpoint{0.793670in}{0.718385in}}%
\pgfpathlineto{\pgfqpoint{0.794479in}{0.712466in}}%
\pgfpathlineto{\pgfqpoint{0.794883in}{0.710094in}}%
\pgfpathlineto{\pgfqpoint{0.795793in}{0.711293in}}%
\pgfpathlineto{\pgfqpoint{0.796298in}{0.710170in}}%
\pgfpathlineto{\pgfqpoint{0.796602in}{0.709598in}}%
\pgfpathlineto{\pgfqpoint{0.796905in}{0.711256in}}%
\pgfpathlineto{\pgfqpoint{0.798118in}{0.743153in}}%
\pgfpathlineto{\pgfqpoint{0.798927in}{0.730854in}}%
\pgfpathlineto{\pgfqpoint{0.801859in}{0.709444in}}%
\pgfpathlineto{\pgfqpoint{0.809238in}{0.709181in}}%
\pgfpathlineto{\pgfqpoint{0.809643in}{0.717036in}}%
\pgfpathlineto{\pgfqpoint{0.809946in}{0.726746in}}%
\pgfpathlineto{\pgfqpoint{0.810654in}{0.716390in}}%
\pgfpathlineto{\pgfqpoint{0.810957in}{0.718105in}}%
\pgfpathlineto{\pgfqpoint{0.811260in}{0.715269in}}%
\pgfpathlineto{\pgfqpoint{0.811968in}{0.709580in}}%
\pgfpathlineto{\pgfqpoint{0.812574in}{0.711025in}}%
\pgfpathlineto{\pgfqpoint{0.812777in}{0.711200in}}%
\pgfpathlineto{\pgfqpoint{0.813484in}{0.710401in}}%
\pgfpathlineto{\pgfqpoint{0.815203in}{0.708601in}}%
\pgfpathlineto{\pgfqpoint{0.815708in}{0.709307in}}%
\pgfpathlineto{\pgfqpoint{0.816214in}{0.714197in}}%
\pgfpathlineto{\pgfqpoint{0.816820in}{0.710059in}}%
\pgfpathlineto{\pgfqpoint{0.817831in}{0.711005in}}%
\pgfpathlineto{\pgfqpoint{0.818034in}{0.711687in}}%
\pgfpathlineto{\pgfqpoint{0.818539in}{0.709204in}}%
\pgfpathlineto{\pgfqpoint{0.818640in}{0.709021in}}%
\pgfpathlineto{\pgfqpoint{0.818842in}{0.709884in}}%
\pgfpathlineto{\pgfqpoint{0.819550in}{0.724116in}}%
\pgfpathlineto{\pgfqpoint{0.820055in}{0.713315in}}%
\pgfpathlineto{\pgfqpoint{0.820156in}{0.712607in}}%
\pgfpathlineto{\pgfqpoint{0.820359in}{0.716149in}}%
\pgfpathlineto{\pgfqpoint{0.820763in}{0.730747in}}%
\pgfpathlineto{\pgfqpoint{0.821269in}{0.712679in}}%
\pgfpathlineto{\pgfqpoint{0.821875in}{0.709752in}}%
\pgfpathlineto{\pgfqpoint{0.822279in}{0.711782in}}%
\pgfpathlineto{\pgfqpoint{0.822785in}{0.722157in}}%
\pgfpathlineto{\pgfqpoint{0.823189in}{0.710832in}}%
\pgfpathlineto{\pgfqpoint{0.824503in}{0.708617in}}%
\pgfpathlineto{\pgfqpoint{0.825717in}{0.707714in}}%
\pgfpathlineto{\pgfqpoint{0.825818in}{0.707777in}}%
\pgfpathlineto{\pgfqpoint{0.827738in}{0.710420in}}%
\pgfpathlineto{\pgfqpoint{0.827840in}{0.710042in}}%
\pgfpathlineto{\pgfqpoint{0.829255in}{0.707004in}}%
\pgfpathlineto{\pgfqpoint{0.830468in}{0.707378in}}%
\pgfpathlineto{\pgfqpoint{0.833602in}{0.709759in}}%
\pgfpathlineto{\pgfqpoint{0.833804in}{0.711123in}}%
\pgfpathlineto{\pgfqpoint{0.834310in}{0.708834in}}%
\pgfpathlineto{\pgfqpoint{0.834613in}{0.709361in}}%
\pgfpathlineto{\pgfqpoint{0.835320in}{0.712101in}}%
\pgfpathlineto{\pgfqpoint{0.835624in}{0.709975in}}%
\pgfpathlineto{\pgfqpoint{0.836837in}{0.707948in}}%
\pgfpathlineto{\pgfqpoint{0.836331in}{0.712519in}}%
\pgfpathlineto{\pgfqpoint{0.836938in}{0.707970in}}%
\pgfpathlineto{\pgfqpoint{0.838151in}{0.709250in}}%
\pgfpathlineto{\pgfqpoint{0.838454in}{0.710778in}}%
\pgfpathlineto{\pgfqpoint{0.838859in}{0.709092in}}%
\pgfpathlineto{\pgfqpoint{0.839263in}{0.709815in}}%
\pgfpathlineto{\pgfqpoint{0.839769in}{0.708640in}}%
\pgfpathlineto{\pgfqpoint{0.841285in}{0.708348in}}%
\pgfpathlineto{\pgfqpoint{0.845329in}{0.709139in}}%
\pgfpathlineto{\pgfqpoint{0.845632in}{0.711316in}}%
\pgfpathlineto{\pgfqpoint{0.846137in}{0.726042in}}%
\pgfpathlineto{\pgfqpoint{0.846744in}{0.713216in}}%
\pgfpathlineto{\pgfqpoint{0.846845in}{0.713143in}}%
\pgfpathlineto{\pgfqpoint{0.847351in}{0.743218in}}%
\pgfpathlineto{\pgfqpoint{0.847755in}{0.714443in}}%
\pgfpathlineto{\pgfqpoint{0.848968in}{0.708719in}}%
\pgfpathlineto{\pgfqpoint{0.850282in}{0.708862in}}%
\pgfpathlineto{\pgfqpoint{0.851091in}{0.710280in}}%
\pgfpathlineto{\pgfqpoint{0.851495in}{0.715808in}}%
\pgfpathlineto{\pgfqpoint{0.852304in}{0.711859in}}%
\pgfpathlineto{\pgfqpoint{0.852506in}{0.711625in}}%
\pgfpathlineto{\pgfqpoint{0.852709in}{0.712345in}}%
\pgfpathlineto{\pgfqpoint{0.853315in}{0.737123in}}%
\pgfpathlineto{\pgfqpoint{0.853719in}{0.714076in}}%
\pgfpathlineto{\pgfqpoint{0.854326in}{0.715990in}}%
\pgfpathlineto{\pgfqpoint{0.854933in}{0.708254in}}%
\pgfpathlineto{\pgfqpoint{0.858976in}{0.710724in}}%
\pgfpathlineto{\pgfqpoint{0.860189in}{0.716284in}}%
\pgfpathlineto{\pgfqpoint{0.859684in}{0.709103in}}%
\pgfpathlineto{\pgfqpoint{0.860291in}{0.715798in}}%
\pgfpathlineto{\pgfqpoint{0.861807in}{0.708559in}}%
\pgfpathlineto{\pgfqpoint{0.862717in}{0.709677in}}%
\pgfpathlineto{\pgfqpoint{0.862919in}{0.712377in}}%
\pgfpathlineto{\pgfqpoint{0.863728in}{0.708537in}}%
\pgfpathlineto{\pgfqpoint{0.864739in}{0.709036in}}%
\pgfpathlineto{\pgfqpoint{0.865244in}{0.712171in}}%
\pgfpathlineto{\pgfqpoint{0.865750in}{0.709054in}}%
\pgfpathlineto{\pgfqpoint{0.865952in}{0.708681in}}%
\pgfpathlineto{\pgfqpoint{0.866255in}{0.710965in}}%
\pgfpathlineto{\pgfqpoint{0.866761in}{0.730738in}}%
\pgfpathlineto{\pgfqpoint{0.867670in}{0.719076in}}%
\pgfpathlineto{\pgfqpoint{0.867771in}{0.719015in}}%
\pgfpathlineto{\pgfqpoint{0.868378in}{0.708429in}}%
\pgfpathlineto{\pgfqpoint{0.868884in}{0.713853in}}%
\pgfpathlineto{\pgfqpoint{0.869187in}{0.722552in}}%
\pgfpathlineto{\pgfqpoint{0.870097in}{0.720232in}}%
\pgfpathlineto{\pgfqpoint{0.870198in}{0.720144in}}%
\pgfpathlineto{\pgfqpoint{0.871108in}{0.707944in}}%
\pgfpathlineto{\pgfqpoint{0.871916in}{0.708937in}}%
\pgfpathlineto{\pgfqpoint{0.872927in}{0.713058in}}%
\pgfpathlineto{\pgfqpoint{0.873736in}{0.721870in}}%
\pgfpathlineto{\pgfqpoint{0.874140in}{0.715470in}}%
\pgfpathlineto{\pgfqpoint{0.874241in}{0.715238in}}%
\pgfpathlineto{\pgfqpoint{0.874343in}{0.717547in}}%
\pgfpathlineto{\pgfqpoint{0.874848in}{0.783485in}}%
\pgfpathlineto{\pgfqpoint{0.875556in}{0.730601in}}%
\pgfpathlineto{\pgfqpoint{0.876061in}{0.721941in}}%
\pgfpathlineto{\pgfqpoint{0.876466in}{0.731159in}}%
\pgfpathlineto{\pgfqpoint{0.876769in}{0.738914in}}%
\pgfpathlineto{\pgfqpoint{0.877173in}{0.721258in}}%
\pgfpathlineto{\pgfqpoint{0.877780in}{0.711751in}}%
\pgfpathlineto{\pgfqpoint{0.878487in}{0.712025in}}%
\pgfpathlineto{\pgfqpoint{0.879903in}{0.708644in}}%
\pgfpathlineto{\pgfqpoint{0.884957in}{0.709711in}}%
\pgfpathlineto{\pgfqpoint{0.885160in}{0.709967in}}%
\pgfpathlineto{\pgfqpoint{0.885665in}{0.708681in}}%
\pgfpathlineto{\pgfqpoint{0.887181in}{0.708621in}}%
\pgfpathlineto{\pgfqpoint{0.888496in}{0.708743in}}%
\pgfpathlineto{\pgfqpoint{0.890619in}{0.709119in}}%
\pgfpathlineto{\pgfqpoint{0.891731in}{0.710054in}}%
\pgfpathlineto{\pgfqpoint{0.891832in}{0.709721in}}%
\pgfpathlineto{\pgfqpoint{0.892337in}{0.708696in}}%
\pgfpathlineto{\pgfqpoint{0.893045in}{0.709014in}}%
\pgfpathlineto{\pgfqpoint{0.894258in}{0.709471in}}%
\pgfpathlineto{\pgfqpoint{0.894763in}{0.710569in}}%
\pgfpathlineto{\pgfqpoint{0.895370in}{0.709605in}}%
\pgfpathlineto{\pgfqpoint{0.897291in}{0.708430in}}%
\pgfpathlineto{\pgfqpoint{0.898706in}{0.707210in}}%
\pgfpathlineto{\pgfqpoint{0.899212in}{0.707687in}}%
\pgfpathlineto{\pgfqpoint{0.900324in}{0.706704in}}%
\pgfpathlineto{\pgfqpoint{0.901537in}{0.707860in}}%
\pgfpathlineto{\pgfqpoint{0.903053in}{0.712304in}}%
\pgfpathlineto{\pgfqpoint{0.902244in}{0.707594in}}%
\pgfpathlineto{\pgfqpoint{0.903154in}{0.711958in}}%
\pgfpathlineto{\pgfqpoint{0.904671in}{0.707257in}}%
\pgfpathlineto{\pgfqpoint{0.904873in}{0.707783in}}%
\pgfpathlineto{\pgfqpoint{0.906086in}{0.714218in}}%
\pgfpathlineto{\pgfqpoint{0.906389in}{0.711460in}}%
\pgfpathlineto{\pgfqpoint{0.906996in}{0.708055in}}%
\pgfpathlineto{\pgfqpoint{0.907703in}{0.708520in}}%
\pgfpathlineto{\pgfqpoint{0.908815in}{0.712062in}}%
\pgfpathlineto{\pgfqpoint{0.909119in}{0.715788in}}%
\pgfpathlineto{\pgfqpoint{0.909624in}{0.707254in}}%
\pgfpathlineto{\pgfqpoint{0.911343in}{0.707728in}}%
\pgfpathlineto{\pgfqpoint{0.912354in}{0.712247in}}%
\pgfpathlineto{\pgfqpoint{0.912556in}{0.714695in}}%
\pgfpathlineto{\pgfqpoint{0.913264in}{0.709068in}}%
\pgfpathlineto{\pgfqpoint{0.913668in}{0.709259in}}%
\pgfpathlineto{\pgfqpoint{0.913769in}{0.709897in}}%
\pgfpathlineto{\pgfqpoint{0.914072in}{0.713422in}}%
\pgfpathlineto{\pgfqpoint{0.914578in}{0.707737in}}%
\pgfpathlineto{\pgfqpoint{0.914679in}{0.707780in}}%
\pgfpathlineto{\pgfqpoint{0.915690in}{0.711782in}}%
\pgfpathlineto{\pgfqpoint{0.916701in}{0.736282in}}%
\pgfpathlineto{\pgfqpoint{0.917004in}{0.810295in}}%
\pgfpathlineto{\pgfqpoint{0.917712in}{0.729244in}}%
\pgfpathlineto{\pgfqpoint{0.918116in}{0.720176in}}%
\pgfpathlineto{\pgfqpoint{0.918520in}{0.733910in}}%
\pgfpathlineto{\pgfqpoint{0.918824in}{0.750361in}}%
\pgfpathlineto{\pgfqpoint{0.919329in}{0.713153in}}%
\pgfpathlineto{\pgfqpoint{0.919632in}{0.710261in}}%
\pgfpathlineto{\pgfqpoint{0.920239in}{0.717214in}}%
\pgfpathlineto{\pgfqpoint{0.920542in}{0.719612in}}%
\pgfpathlineto{\pgfqpoint{0.921048in}{0.713595in}}%
\pgfpathlineto{\pgfqpoint{0.921856in}{0.711814in}}%
\pgfpathlineto{\pgfqpoint{0.922261in}{0.712506in}}%
\pgfpathlineto{\pgfqpoint{0.922564in}{0.712983in}}%
\pgfpathlineto{\pgfqpoint{0.922867in}{0.711425in}}%
\pgfpathlineto{\pgfqpoint{0.924283in}{0.708763in}}%
\pgfpathlineto{\pgfqpoint{0.924788in}{0.708324in}}%
\pgfpathlineto{\pgfqpoint{0.925091in}{0.709450in}}%
\pgfpathlineto{\pgfqpoint{0.925294in}{0.710127in}}%
\pgfpathlineto{\pgfqpoint{0.925799in}{0.708563in}}%
\pgfpathlineto{\pgfqpoint{0.926001in}{0.708615in}}%
\pgfpathlineto{\pgfqpoint{0.927720in}{0.709431in}}%
\pgfpathlineto{\pgfqpoint{0.927922in}{0.709560in}}%
\pgfpathlineto{\pgfqpoint{0.928326in}{0.708779in}}%
\pgfpathlineto{\pgfqpoint{0.928529in}{0.708686in}}%
\pgfpathlineto{\pgfqpoint{0.929944in}{0.709594in}}%
\pgfpathlineto{\pgfqpoint{0.931157in}{0.714893in}}%
\pgfpathlineto{\pgfqpoint{0.931663in}{0.787113in}}%
\pgfpathlineto{\pgfqpoint{0.932269in}{0.717391in}}%
\pgfpathlineto{\pgfqpoint{0.933785in}{0.708928in}}%
\pgfpathlineto{\pgfqpoint{0.932673in}{0.718754in}}%
\pgfpathlineto{\pgfqpoint{0.933988in}{0.709187in}}%
\pgfpathlineto{\pgfqpoint{0.935403in}{0.709837in}}%
\pgfpathlineto{\pgfqpoint{0.937930in}{0.708621in}}%
\pgfpathlineto{\pgfqpoint{0.940862in}{0.709863in}}%
\pgfpathlineto{\pgfqpoint{0.941064in}{0.709549in}}%
\pgfpathlineto{\pgfqpoint{0.943086in}{0.709089in}}%
\pgfpathlineto{\pgfqpoint{0.946220in}{0.710277in}}%
\pgfpathlineto{\pgfqpoint{0.946624in}{0.716010in}}%
\pgfpathlineto{\pgfqpoint{0.947231in}{0.710119in}}%
\pgfpathlineto{\pgfqpoint{0.949354in}{0.708996in}}%
\pgfpathlineto{\pgfqpoint{0.953094in}{0.709540in}}%
\pgfpathlineto{\pgfqpoint{0.953297in}{0.711133in}}%
\pgfpathlineto{\pgfqpoint{0.953903in}{0.708336in}}%
\pgfpathlineto{\pgfqpoint{0.954004in}{0.708358in}}%
\pgfpathlineto{\pgfqpoint{0.957138in}{0.709519in}}%
\pgfpathlineto{\pgfqpoint{0.957542in}{0.714941in}}%
\pgfpathlineto{\pgfqpoint{0.958048in}{0.709098in}}%
\pgfpathlineto{\pgfqpoint{0.958149in}{0.709106in}}%
\pgfpathlineto{\pgfqpoint{0.959261in}{0.709495in}}%
\pgfpathlineto{\pgfqpoint{0.959564in}{0.726755in}}%
\pgfpathlineto{\pgfqpoint{0.959868in}{0.758718in}}%
\pgfpathlineto{\pgfqpoint{0.960474in}{0.713381in}}%
\pgfpathlineto{\pgfqpoint{0.960575in}{0.714831in}}%
\pgfpathlineto{\pgfqpoint{0.961182in}{0.751527in}}%
\pgfpathlineto{\pgfqpoint{0.961788in}{0.722975in}}%
\pgfpathlineto{\pgfqpoint{0.963103in}{0.716399in}}%
\pgfpathlineto{\pgfqpoint{0.963204in}{0.715910in}}%
\pgfpathlineto{\pgfqpoint{0.963608in}{0.718895in}}%
\pgfpathlineto{\pgfqpoint{0.964821in}{0.727950in}}%
\pgfpathlineto{\pgfqpoint{0.964316in}{0.717226in}}%
\pgfpathlineto{\pgfqpoint{0.965023in}{0.724367in}}%
\pgfpathlineto{\pgfqpoint{0.965731in}{0.708691in}}%
\pgfpathlineto{\pgfqpoint{0.966439in}{0.708710in}}%
\pgfpathlineto{\pgfqpoint{0.975133in}{0.709827in}}%
\pgfpathlineto{\pgfqpoint{0.975840in}{0.709397in}}%
\pgfpathlineto{\pgfqpoint{0.976649in}{0.711663in}}%
\pgfpathlineto{\pgfqpoint{0.977761in}{0.709148in}}%
\pgfpathlineto{\pgfqpoint{0.978166in}{0.710347in}}%
\pgfpathlineto{\pgfqpoint{0.978368in}{0.711070in}}%
\pgfpathlineto{\pgfqpoint{0.978974in}{0.708799in}}%
\pgfpathlineto{\pgfqpoint{0.979783in}{0.708926in}}%
\pgfpathlineto{\pgfqpoint{0.979884in}{0.709352in}}%
\pgfpathlineto{\pgfqpoint{0.981805in}{0.728847in}}%
\pgfpathlineto{\pgfqpoint{0.982108in}{0.722343in}}%
\pgfpathlineto{\pgfqpoint{0.983422in}{0.711486in}}%
\pgfpathlineto{\pgfqpoint{0.983523in}{0.711522in}}%
\pgfpathlineto{\pgfqpoint{0.983827in}{0.712192in}}%
\pgfpathlineto{\pgfqpoint{0.984130in}{0.710942in}}%
\pgfpathlineto{\pgfqpoint{0.984635in}{0.709255in}}%
\pgfpathlineto{\pgfqpoint{0.985141in}{0.710919in}}%
\pgfpathlineto{\pgfqpoint{0.985748in}{0.716389in}}%
\pgfpathlineto{\pgfqpoint{0.986253in}{0.711181in}}%
\pgfpathlineto{\pgfqpoint{0.986354in}{0.710979in}}%
\pgfpathlineto{\pgfqpoint{0.986455in}{0.711679in}}%
\pgfpathlineto{\pgfqpoint{0.986961in}{0.726025in}}%
\pgfpathlineto{\pgfqpoint{0.987567in}{0.712561in}}%
\pgfpathlineto{\pgfqpoint{0.987972in}{0.710435in}}%
\pgfpathlineto{\pgfqpoint{0.988477in}{0.714417in}}%
\pgfpathlineto{\pgfqpoint{0.988578in}{0.715148in}}%
\pgfpathlineto{\pgfqpoint{0.988983in}{0.710390in}}%
\pgfpathlineto{\pgfqpoint{0.990196in}{0.707930in}}%
\pgfpathlineto{\pgfqpoint{0.993026in}{0.708404in}}%
\pgfpathlineto{\pgfqpoint{0.993330in}{0.709271in}}%
\pgfpathlineto{\pgfqpoint{0.994138in}{0.708289in}}%
\pgfpathlineto{\pgfqpoint{0.994543in}{0.708673in}}%
\pgfpathlineto{\pgfqpoint{0.994947in}{0.707652in}}%
\pgfpathlineto{\pgfqpoint{0.996160in}{0.707661in}}%
\pgfpathlineto{\pgfqpoint{0.996463in}{0.707795in}}%
\pgfpathlineto{\pgfqpoint{0.996666in}{0.708544in}}%
\pgfpathlineto{\pgfqpoint{0.996868in}{0.708840in}}%
\pgfpathlineto{\pgfqpoint{0.997373in}{0.707905in}}%
\pgfpathlineto{\pgfqpoint{0.997575in}{0.707915in}}%
\pgfpathlineto{\pgfqpoint{1.004450in}{0.709799in}}%
\pgfpathlineto{\pgfqpoint{1.005663in}{0.719301in}}%
\pgfpathlineto{\pgfqpoint{1.006067in}{0.744714in}}%
\pgfpathlineto{\pgfqpoint{1.006674in}{0.713285in}}%
\pgfpathlineto{\pgfqpoint{1.006775in}{0.713251in}}%
\pgfpathlineto{\pgfqpoint{1.007078in}{0.717210in}}%
\pgfpathlineto{\pgfqpoint{1.007483in}{0.710031in}}%
\pgfpathlineto{\pgfqpoint{1.007786in}{0.708825in}}%
\pgfpathlineto{\pgfqpoint{1.008696in}{0.709018in}}%
\pgfpathlineto{\pgfqpoint{1.012638in}{0.710846in}}%
\pgfpathlineto{\pgfqpoint{1.013245in}{0.718882in}}%
\pgfpathlineto{\pgfqpoint{1.014054in}{0.715955in}}%
\pgfpathlineto{\pgfqpoint{1.016581in}{0.709033in}}%
\pgfpathlineto{\pgfqpoint{1.016985in}{0.709835in}}%
\pgfpathlineto{\pgfqpoint{1.017491in}{0.717419in}}%
\pgfpathlineto{\pgfqpoint{1.018300in}{0.712554in}}%
\pgfpathlineto{\pgfqpoint{1.019311in}{0.709010in}}%
\pgfpathlineto{\pgfqpoint{1.019917in}{0.711003in}}%
\pgfpathlineto{\pgfqpoint{1.020524in}{0.709457in}}%
\pgfpathlineto{\pgfqpoint{1.021029in}{0.711417in}}%
\pgfpathlineto{\pgfqpoint{1.022444in}{0.708784in}}%
\pgfpathlineto{\pgfqpoint{1.022546in}{0.708807in}}%
\pgfpathlineto{\pgfqpoint{1.024871in}{0.710487in}}%
\pgfpathlineto{\pgfqpoint{1.025174in}{0.709398in}}%
\pgfpathlineto{\pgfqpoint{1.025376in}{0.709052in}}%
\pgfpathlineto{\pgfqpoint{1.025679in}{0.709992in}}%
\pgfpathlineto{\pgfqpoint{1.026994in}{0.731406in}}%
\pgfpathlineto{\pgfqpoint{1.027398in}{0.718943in}}%
\pgfpathlineto{\pgfqpoint{1.028510in}{0.708507in}}%
\pgfpathlineto{\pgfqpoint{1.028712in}{0.708564in}}%
\pgfpathlineto{\pgfqpoint{1.030128in}{0.708327in}}%
\pgfpathlineto{\pgfqpoint{1.033767in}{0.708885in}}%
\pgfpathlineto{\pgfqpoint{1.035384in}{0.718722in}}%
\pgfpathlineto{\pgfqpoint{1.035587in}{0.714615in}}%
\pgfpathlineto{\pgfqpoint{1.036193in}{0.708432in}}%
\pgfpathlineto{\pgfqpoint{1.036901in}{0.709111in}}%
\pgfpathlineto{\pgfqpoint{1.038619in}{0.719905in}}%
\pgfpathlineto{\pgfqpoint{1.038822in}{0.723372in}}%
\pgfpathlineto{\pgfqpoint{1.039327in}{0.711625in}}%
\pgfpathlineto{\pgfqpoint{1.039428in}{0.711650in}}%
\pgfpathlineto{\pgfqpoint{1.039833in}{0.723653in}}%
\pgfpathlineto{\pgfqpoint{1.040338in}{0.708173in}}%
\pgfpathlineto{\pgfqpoint{1.040540in}{0.708253in}}%
\pgfpathlineto{\pgfqpoint{1.040742in}{0.709154in}}%
\pgfpathlineto{\pgfqpoint{1.040843in}{0.709306in}}%
\pgfpathlineto{\pgfqpoint{1.041147in}{0.708318in}}%
\pgfpathlineto{\pgfqpoint{1.041551in}{0.708383in}}%
\pgfpathlineto{\pgfqpoint{1.044382in}{0.708982in}}%
\pgfpathlineto{\pgfqpoint{1.044786in}{0.712490in}}%
\pgfpathlineto{\pgfqpoint{1.046201in}{0.730418in}}%
\pgfpathlineto{\pgfqpoint{1.045595in}{0.709603in}}%
\pgfpathlineto{\pgfqpoint{1.046302in}{0.727808in}}%
\pgfpathlineto{\pgfqpoint{1.046808in}{0.709036in}}%
\pgfpathlineto{\pgfqpoint{1.047516in}{0.718376in}}%
\pgfpathlineto{\pgfqpoint{1.048021in}{0.709084in}}%
\pgfpathlineto{\pgfqpoint{1.048830in}{0.710735in}}%
\pgfpathlineto{\pgfqpoint{1.049335in}{0.708141in}}%
\pgfpathlineto{\pgfqpoint{1.049841in}{0.711337in}}%
\pgfpathlineto{\pgfqpoint{1.049942in}{0.711833in}}%
\pgfpathlineto{\pgfqpoint{1.050346in}{0.708485in}}%
\pgfpathlineto{\pgfqpoint{1.050548in}{0.708154in}}%
\pgfpathlineto{\pgfqpoint{1.050751in}{0.709353in}}%
\pgfpathlineto{\pgfqpoint{1.051054in}{0.712169in}}%
\pgfpathlineto{\pgfqpoint{1.051660in}{0.707875in}}%
\pgfpathlineto{\pgfqpoint{1.054794in}{0.708154in}}%
\pgfpathlineto{\pgfqpoint{1.055401in}{0.708818in}}%
\pgfpathlineto{\pgfqpoint{1.055805in}{0.715817in}}%
\pgfpathlineto{\pgfqpoint{1.056412in}{0.708516in}}%
\pgfpathlineto{\pgfqpoint{1.057524in}{0.709508in}}%
\pgfpathlineto{\pgfqpoint{1.058029in}{0.729875in}}%
\pgfpathlineto{\pgfqpoint{1.058636in}{0.711047in}}%
\pgfpathlineto{\pgfqpoint{1.058838in}{0.710266in}}%
\pgfpathlineto{\pgfqpoint{1.059242in}{0.713486in}}%
\pgfpathlineto{\pgfqpoint{1.059445in}{0.717159in}}%
\pgfpathlineto{\pgfqpoint{1.060051in}{0.708093in}}%
\pgfpathlineto{\pgfqpoint{1.060961in}{0.709036in}}%
\pgfpathlineto{\pgfqpoint{1.062073in}{0.719261in}}%
\pgfpathlineto{\pgfqpoint{1.061467in}{0.708616in}}%
\pgfpathlineto{\pgfqpoint{1.062477in}{0.711515in}}%
\pgfpathlineto{\pgfqpoint{1.062781in}{0.708784in}}%
\pgfpathlineto{\pgfqpoint{1.063185in}{0.717044in}}%
\pgfpathlineto{\pgfqpoint{1.063488in}{0.728838in}}%
\pgfpathlineto{\pgfqpoint{1.064095in}{0.713003in}}%
\pgfpathlineto{\pgfqpoint{1.064297in}{0.711665in}}%
\pgfpathlineto{\pgfqpoint{1.064701in}{0.714261in}}%
\pgfpathlineto{\pgfqpoint{1.065207in}{0.712947in}}%
\pgfpathlineto{\pgfqpoint{1.065712in}{0.717851in}}%
\pgfpathlineto{\pgfqpoint{1.066218in}{0.712374in}}%
\pgfpathlineto{\pgfqpoint{1.067027in}{0.709587in}}%
\pgfpathlineto{\pgfqpoint{1.068442in}{0.707579in}}%
\pgfpathlineto{\pgfqpoint{1.075215in}{0.708617in}}%
\pgfpathlineto{\pgfqpoint{1.079259in}{0.709286in}}%
\pgfpathlineto{\pgfqpoint{1.079764in}{0.713014in}}%
\pgfpathlineto{\pgfqpoint{1.080169in}{0.708834in}}%
\pgfpathlineto{\pgfqpoint{1.080270in}{0.708586in}}%
\pgfpathlineto{\pgfqpoint{1.080472in}{0.709387in}}%
\pgfpathlineto{\pgfqpoint{1.080876in}{0.719888in}}%
\pgfpathlineto{\pgfqpoint{1.081382in}{0.709229in}}%
\pgfpathlineto{\pgfqpoint{1.081584in}{0.710100in}}%
\pgfpathlineto{\pgfqpoint{1.082999in}{0.712457in}}%
\pgfpathlineto{\pgfqpoint{1.082292in}{0.709068in}}%
\pgfpathlineto{\pgfqpoint{1.083100in}{0.711905in}}%
\pgfpathlineto{\pgfqpoint{1.083303in}{0.710404in}}%
\pgfpathlineto{\pgfqpoint{1.083606in}{0.714239in}}%
\pgfpathlineto{\pgfqpoint{1.083909in}{0.720299in}}%
\pgfpathlineto{\pgfqpoint{1.084516in}{0.708217in}}%
\pgfpathlineto{\pgfqpoint{1.084718in}{0.708668in}}%
\pgfpathlineto{\pgfqpoint{1.086234in}{0.739995in}}%
\pgfpathlineto{\pgfqpoint{1.086437in}{0.750218in}}%
\pgfpathlineto{\pgfqpoint{1.086942in}{0.713997in}}%
\pgfpathlineto{\pgfqpoint{1.088054in}{0.709613in}}%
\pgfpathlineto{\pgfqpoint{1.088256in}{0.709829in}}%
\pgfpathlineto{\pgfqpoint{1.091087in}{0.708657in}}%
\pgfpathlineto{\pgfqpoint{1.091390in}{0.709483in}}%
\pgfpathlineto{\pgfqpoint{1.091592in}{0.709905in}}%
\pgfpathlineto{\pgfqpoint{1.092199in}{0.708476in}}%
\pgfpathlineto{\pgfqpoint{1.093412in}{0.708391in}}%
\pgfpathlineto{\pgfqpoint{1.093513in}{0.708753in}}%
\pgfpathlineto{\pgfqpoint{1.093917in}{0.718329in}}%
\pgfpathlineto{\pgfqpoint{1.094524in}{0.708086in}}%
\pgfpathlineto{\pgfqpoint{1.094827in}{0.708753in}}%
\pgfpathlineto{\pgfqpoint{1.095737in}{0.707951in}}%
\pgfpathlineto{\pgfqpoint{1.096950in}{0.710547in}}%
\pgfpathlineto{\pgfqpoint{1.097658in}{0.732014in}}%
\pgfpathlineto{\pgfqpoint{1.098265in}{0.718653in}}%
\pgfpathlineto{\pgfqpoint{1.100084in}{0.707668in}}%
\pgfpathlineto{\pgfqpoint{1.098871in}{0.721383in}}%
\pgfpathlineto{\pgfqpoint{1.100387in}{0.708246in}}%
\pgfpathlineto{\pgfqpoint{1.101601in}{0.709818in}}%
\pgfpathlineto{\pgfqpoint{1.100893in}{0.707644in}}%
\pgfpathlineto{\pgfqpoint{1.101904in}{0.709418in}}%
\pgfpathlineto{\pgfqpoint{1.102713in}{0.709494in}}%
\pgfpathlineto{\pgfqpoint{1.102814in}{0.710005in}}%
\pgfpathlineto{\pgfqpoint{1.103016in}{0.711096in}}%
\pgfpathlineto{\pgfqpoint{1.103521in}{0.707791in}}%
\pgfpathlineto{\pgfqpoint{1.104633in}{0.708756in}}%
\pgfpathlineto{\pgfqpoint{1.104937in}{0.712988in}}%
\pgfpathlineto{\pgfqpoint{1.105543in}{0.708323in}}%
\pgfpathlineto{\pgfqpoint{1.105745in}{0.708846in}}%
\pgfpathlineto{\pgfqpoint{1.105948in}{0.709377in}}%
\pgfpathlineto{\pgfqpoint{1.106655in}{0.708069in}}%
\pgfpathlineto{\pgfqpoint{1.106857in}{0.708485in}}%
\pgfpathlineto{\pgfqpoint{1.107262in}{0.717771in}}%
\pgfpathlineto{\pgfqpoint{1.108071in}{0.710322in}}%
\pgfpathlineto{\pgfqpoint{1.109082in}{0.708299in}}%
\pgfpathlineto{\pgfqpoint{1.109284in}{0.709230in}}%
\pgfpathlineto{\pgfqpoint{1.110901in}{0.729166in}}%
\pgfpathlineto{\pgfqpoint{1.111002in}{0.724486in}}%
\pgfpathlineto{\pgfqpoint{1.111407in}{0.707887in}}%
\pgfpathlineto{\pgfqpoint{1.112215in}{0.715661in}}%
\pgfpathlineto{\pgfqpoint{1.112822in}{0.709060in}}%
\pgfpathlineto{\pgfqpoint{1.113530in}{0.709503in}}%
\pgfpathlineto{\pgfqpoint{1.114035in}{0.713730in}}%
\pgfpathlineto{\pgfqpoint{1.114439in}{0.709018in}}%
\pgfpathlineto{\pgfqpoint{1.115551in}{0.707041in}}%
\pgfpathlineto{\pgfqpoint{1.114945in}{0.711043in}}%
\pgfpathlineto{\pgfqpoint{1.115855in}{0.707576in}}%
\pgfpathlineto{\pgfqpoint{1.117573in}{0.708609in}}%
\pgfpathlineto{\pgfqpoint{1.117978in}{0.710620in}}%
\pgfpathlineto{\pgfqpoint{1.118584in}{0.708306in}}%
\pgfpathlineto{\pgfqpoint{1.120909in}{0.707297in}}%
\pgfpathlineto{\pgfqpoint{1.121112in}{0.708237in}}%
\pgfpathlineto{\pgfqpoint{1.122224in}{0.715905in}}%
\pgfpathlineto{\pgfqpoint{1.122729in}{0.835742in}}%
\pgfpathlineto{\pgfqpoint{1.123336in}{0.731253in}}%
\pgfpathlineto{\pgfqpoint{1.123942in}{0.739061in}}%
\pgfpathlineto{\pgfqpoint{1.124852in}{0.709450in}}%
\pgfpathlineto{\pgfqpoint{1.126470in}{0.709046in}}%
\pgfpathlineto{\pgfqpoint{1.128390in}{0.709993in}}%
\pgfpathlineto{\pgfqpoint{1.128896in}{0.713837in}}%
\pgfpathlineto{\pgfqpoint{1.129502in}{0.710406in}}%
\pgfpathlineto{\pgfqpoint{1.130008in}{0.710169in}}%
\pgfpathlineto{\pgfqpoint{1.130311in}{0.710963in}}%
\pgfpathlineto{\pgfqpoint{1.130614in}{0.711763in}}%
\pgfpathlineto{\pgfqpoint{1.131019in}{0.709583in}}%
\pgfpathlineto{\pgfqpoint{1.132030in}{0.709050in}}%
\pgfpathlineto{\pgfqpoint{1.132232in}{0.709175in}}%
\pgfpathlineto{\pgfqpoint{1.133647in}{0.711625in}}%
\pgfpathlineto{\pgfqpoint{1.133950in}{0.713078in}}%
\pgfpathlineto{\pgfqpoint{1.134456in}{0.709546in}}%
\pgfpathlineto{\pgfqpoint{1.136073in}{0.709033in}}%
\pgfpathlineto{\pgfqpoint{1.144363in}{0.709724in}}%
\pgfpathlineto{\pgfqpoint{1.145576in}{0.713277in}}%
\pgfpathlineto{\pgfqpoint{1.146183in}{0.711981in}}%
\pgfpathlineto{\pgfqpoint{1.146688in}{0.738200in}}%
\pgfpathlineto{\pgfqpoint{1.147194in}{0.710699in}}%
\pgfpathlineto{\pgfqpoint{1.147800in}{0.715954in}}%
\pgfpathlineto{\pgfqpoint{1.148508in}{0.707471in}}%
\pgfpathlineto{\pgfqpoint{1.149721in}{0.708364in}}%
\pgfpathlineto{\pgfqpoint{1.149923in}{0.709244in}}%
\pgfpathlineto{\pgfqpoint{1.150328in}{0.707918in}}%
\pgfpathlineto{\pgfqpoint{1.150732in}{0.708089in}}%
\pgfpathlineto{\pgfqpoint{1.152046in}{0.708083in}}%
\pgfpathlineto{\pgfqpoint{1.152451in}{0.709134in}}%
\pgfpathlineto{\pgfqpoint{1.152754in}{0.711499in}}%
\pgfpathlineto{\pgfqpoint{1.153462in}{0.708524in}}%
\pgfpathlineto{\pgfqpoint{1.154776in}{0.709515in}}%
\pgfpathlineto{\pgfqpoint{1.156292in}{0.713534in}}%
\pgfpathlineto{\pgfqpoint{1.156494in}{0.712667in}}%
\pgfpathlineto{\pgfqpoint{1.157202in}{0.708022in}}%
\pgfpathlineto{\pgfqpoint{1.157910in}{0.708187in}}%
\pgfpathlineto{\pgfqpoint{1.159527in}{0.709257in}}%
\pgfpathlineto{\pgfqpoint{1.159932in}{0.721380in}}%
\pgfpathlineto{\pgfqpoint{1.160639in}{0.709876in}}%
\pgfpathlineto{\pgfqpoint{1.161044in}{0.712324in}}%
\pgfpathlineto{\pgfqpoint{1.161448in}{0.707944in}}%
\pgfpathlineto{\pgfqpoint{1.161650in}{0.707793in}}%
\pgfpathlineto{\pgfqpoint{1.161852in}{0.708430in}}%
\pgfpathlineto{\pgfqpoint{1.163874in}{0.720790in}}%
\pgfpathlineto{\pgfqpoint{1.164177in}{0.721789in}}%
\pgfpathlineto{\pgfqpoint{1.164481in}{0.719395in}}%
\pgfpathlineto{\pgfqpoint{1.166300in}{0.707841in}}%
\pgfpathlineto{\pgfqpoint{1.166401in}{0.707866in}}%
\pgfpathlineto{\pgfqpoint{1.167716in}{0.708534in}}%
\pgfpathlineto{\pgfqpoint{1.168120in}{0.713102in}}%
\pgfpathlineto{\pgfqpoint{1.168828in}{0.709241in}}%
\pgfpathlineto{\pgfqpoint{1.169030in}{0.709382in}}%
\pgfpathlineto{\pgfqpoint{1.169333in}{0.708445in}}%
\pgfpathlineto{\pgfqpoint{1.170647in}{0.707926in}}%
\pgfpathlineto{\pgfqpoint{1.172366in}{0.708714in}}%
\pgfpathlineto{\pgfqpoint{1.172871in}{0.724499in}}%
\pgfpathlineto{\pgfqpoint{1.173074in}{0.729188in}}%
\pgfpathlineto{\pgfqpoint{1.173579in}{0.714683in}}%
\pgfpathlineto{\pgfqpoint{1.173983in}{0.726008in}}%
\pgfpathlineto{\pgfqpoint{1.174186in}{0.722853in}}%
\pgfpathlineto{\pgfqpoint{1.174994in}{0.708286in}}%
\pgfpathlineto{\pgfqpoint{1.175601in}{0.709776in}}%
\pgfpathlineto{\pgfqpoint{1.176106in}{0.708170in}}%
\pgfpathlineto{\pgfqpoint{1.176915in}{0.708529in}}%
\pgfpathlineto{\pgfqpoint{1.177522in}{0.708983in}}%
\pgfpathlineto{\pgfqpoint{1.177926in}{0.711874in}}%
\pgfpathlineto{\pgfqpoint{1.178533in}{0.708542in}}%
\pgfpathlineto{\pgfqpoint{1.179746in}{0.708929in}}%
\pgfpathlineto{\pgfqpoint{1.180150in}{0.710422in}}%
\pgfpathlineto{\pgfqpoint{1.181060in}{0.710059in}}%
\pgfpathlineto{\pgfqpoint{1.181768in}{0.711476in}}%
\pgfpathlineto{\pgfqpoint{1.182273in}{0.718131in}}%
\pgfpathlineto{\pgfqpoint{1.182779in}{0.711696in}}%
\pgfpathlineto{\pgfqpoint{1.183082in}{0.709826in}}%
\pgfpathlineto{\pgfqpoint{1.183486in}{0.716340in}}%
\pgfpathlineto{\pgfqpoint{1.183587in}{0.717695in}}%
\pgfpathlineto{\pgfqpoint{1.183992in}{0.710192in}}%
\pgfpathlineto{\pgfqpoint{1.184295in}{0.708232in}}%
\pgfpathlineto{\pgfqpoint{1.185205in}{0.708831in}}%
\pgfpathlineto{\pgfqpoint{1.186115in}{0.708372in}}%
\pgfpathlineto{\pgfqpoint{1.186317in}{0.708477in}}%
\pgfpathlineto{\pgfqpoint{1.186822in}{0.711540in}}%
\pgfpathlineto{\pgfqpoint{1.187934in}{0.709184in}}%
\pgfpathlineto{\pgfqpoint{1.188238in}{0.708053in}}%
\pgfpathlineto{\pgfqpoint{1.189148in}{0.708490in}}%
\pgfpathlineto{\pgfqpoint{1.190765in}{0.708238in}}%
\pgfpathlineto{\pgfqpoint{1.193899in}{0.708400in}}%
\pgfpathlineto{\pgfqpoint{1.195719in}{0.708281in}}%
\pgfpathlineto{\pgfqpoint{1.198044in}{0.709260in}}%
\pgfpathlineto{\pgfqpoint{1.198246in}{0.710486in}}%
\pgfpathlineto{\pgfqpoint{1.198852in}{0.708462in}}%
\pgfpathlineto{\pgfqpoint{1.199055in}{0.708486in}}%
\pgfpathlineto{\pgfqpoint{1.200672in}{0.709507in}}%
\pgfpathlineto{\pgfqpoint{1.202087in}{0.726592in}}%
\pgfpathlineto{\pgfqpoint{1.202391in}{0.721630in}}%
\pgfpathlineto{\pgfqpoint{1.202694in}{0.716805in}}%
\pgfpathlineto{\pgfqpoint{1.203199in}{0.726423in}}%
\pgfpathlineto{\pgfqpoint{1.204615in}{0.735905in}}%
\pgfpathlineto{\pgfqpoint{1.203806in}{0.723640in}}%
\pgfpathlineto{\pgfqpoint{1.204716in}{0.734022in}}%
\pgfpathlineto{\pgfqpoint{1.205525in}{0.710017in}}%
\pgfpathlineto{\pgfqpoint{1.206232in}{0.713803in}}%
\pgfpathlineto{\pgfqpoint{1.206637in}{0.717461in}}%
\pgfpathlineto{\pgfqpoint{1.207041in}{0.710507in}}%
\pgfpathlineto{\pgfqpoint{1.207142in}{0.710074in}}%
\pgfpathlineto{\pgfqpoint{1.207243in}{0.710918in}}%
\pgfpathlineto{\pgfqpoint{1.207749in}{0.741382in}}%
\pgfpathlineto{\pgfqpoint{1.208355in}{0.712785in}}%
\pgfpathlineto{\pgfqpoint{1.210175in}{0.708504in}}%
\pgfpathlineto{\pgfqpoint{1.211489in}{0.709632in}}%
\pgfpathlineto{\pgfqpoint{1.213814in}{0.718675in}}%
\pgfpathlineto{\pgfqpoint{1.214118in}{0.715511in}}%
\pgfpathlineto{\pgfqpoint{1.214825in}{0.716373in}}%
\pgfpathlineto{\pgfqpoint{1.215634in}{0.707446in}}%
\pgfpathlineto{\pgfqpoint{1.216847in}{0.708393in}}%
\pgfpathlineto{\pgfqpoint{1.218364in}{0.725887in}}%
\pgfpathlineto{\pgfqpoint{1.218566in}{0.720234in}}%
\pgfpathlineto{\pgfqpoint{1.219374in}{0.706720in}}%
\pgfpathlineto{\pgfqpoint{1.219880in}{0.706899in}}%
\pgfpathlineto{\pgfqpoint{1.222508in}{0.707801in}}%
\pgfpathlineto{\pgfqpoint{1.223721in}{0.708471in}}%
\pgfpathlineto{\pgfqpoint{1.223924in}{0.708292in}}%
\pgfpathlineto{\pgfqpoint{1.225642in}{0.707658in}}%
\pgfpathlineto{\pgfqpoint{1.229079in}{0.709000in}}%
\pgfpathlineto{\pgfqpoint{1.230394in}{0.740251in}}%
\pgfpathlineto{\pgfqpoint{1.230798in}{0.718242in}}%
\pgfpathlineto{\pgfqpoint{1.232112in}{0.709030in}}%
\pgfpathlineto{\pgfqpoint{1.232820in}{0.708266in}}%
\pgfpathlineto{\pgfqpoint{1.233123in}{0.708915in}}%
\pgfpathlineto{\pgfqpoint{1.233629in}{0.723697in}}%
\pgfpathlineto{\pgfqpoint{1.233730in}{0.726107in}}%
\pgfpathlineto{\pgfqpoint{1.234336in}{0.716629in}}%
\pgfpathlineto{\pgfqpoint{1.234437in}{0.716939in}}%
\pgfpathlineto{\pgfqpoint{1.234943in}{0.729307in}}%
\pgfpathlineto{\pgfqpoint{1.235347in}{0.716380in}}%
\pgfpathlineto{\pgfqpoint{1.236661in}{0.708289in}}%
\pgfpathlineto{\pgfqpoint{1.237066in}{0.709063in}}%
\pgfpathlineto{\pgfqpoint{1.237268in}{0.709765in}}%
\pgfpathlineto{\pgfqpoint{1.237672in}{0.708593in}}%
\pgfpathlineto{\pgfqpoint{1.238178in}{0.709246in}}%
\pgfpathlineto{\pgfqpoint{1.238784in}{0.708528in}}%
\pgfpathlineto{\pgfqpoint{1.239189in}{0.709204in}}%
\pgfpathlineto{\pgfqpoint{1.240200in}{0.714450in}}%
\pgfpathlineto{\pgfqpoint{1.240705in}{0.796457in}}%
\pgfpathlineto{\pgfqpoint{1.241312in}{0.720317in}}%
\pgfpathlineto{\pgfqpoint{1.241413in}{0.719134in}}%
\pgfpathlineto{\pgfqpoint{1.241514in}{0.722479in}}%
\pgfpathlineto{\pgfqpoint{1.241918in}{0.759103in}}%
\pgfpathlineto{\pgfqpoint{1.242424in}{0.714454in}}%
\pgfpathlineto{\pgfqpoint{1.243637in}{0.709151in}}%
\pgfpathlineto{\pgfqpoint{1.244243in}{0.709398in}}%
\pgfpathlineto{\pgfqpoint{1.244345in}{0.709797in}}%
\pgfpathlineto{\pgfqpoint{1.245760in}{0.721524in}}%
\pgfpathlineto{\pgfqpoint{1.245962in}{0.717394in}}%
\pgfpathlineto{\pgfqpoint{1.247276in}{0.708777in}}%
\pgfpathlineto{\pgfqpoint{1.247681in}{0.709547in}}%
\pgfpathlineto{\pgfqpoint{1.247883in}{0.709886in}}%
\pgfpathlineto{\pgfqpoint{1.248388in}{0.708798in}}%
\pgfpathlineto{\pgfqpoint{1.248590in}{0.708809in}}%
\pgfpathlineto{\pgfqpoint{1.251320in}{0.710004in}}%
\pgfpathlineto{\pgfqpoint{1.251724in}{0.708761in}}%
\pgfpathlineto{\pgfqpoint{1.254555in}{0.708924in}}%
\pgfpathlineto{\pgfqpoint{1.254858in}{0.711027in}}%
\pgfpathlineto{\pgfqpoint{1.255060in}{0.712798in}}%
\pgfpathlineto{\pgfqpoint{1.255566in}{0.709272in}}%
\pgfpathlineto{\pgfqpoint{1.255768in}{0.709548in}}%
\pgfpathlineto{\pgfqpoint{1.255970in}{0.709605in}}%
\pgfpathlineto{\pgfqpoint{1.256274in}{0.708995in}}%
\pgfpathlineto{\pgfqpoint{1.256678in}{0.708915in}}%
\pgfpathlineto{\pgfqpoint{1.260418in}{0.709666in}}%
\pgfpathlineto{\pgfqpoint{1.260823in}{0.711459in}}%
\pgfpathlineto{\pgfqpoint{1.261530in}{0.709662in}}%
\pgfpathlineto{\pgfqpoint{1.261935in}{0.710782in}}%
\pgfpathlineto{\pgfqpoint{1.262440in}{0.709119in}}%
\pgfpathlineto{\pgfqpoint{1.262642in}{0.708841in}}%
\pgfpathlineto{\pgfqpoint{1.263047in}{0.710338in}}%
\pgfpathlineto{\pgfqpoint{1.264361in}{0.719185in}}%
\pgfpathlineto{\pgfqpoint{1.264563in}{0.716883in}}%
\pgfpathlineto{\pgfqpoint{1.266080in}{0.709142in}}%
\pgfpathlineto{\pgfqpoint{1.272347in}{0.708748in}}%
\pgfpathlineto{\pgfqpoint{1.275582in}{0.709237in}}%
\pgfpathlineto{\pgfqpoint{1.275987in}{0.716411in}}%
\pgfpathlineto{\pgfqpoint{1.276694in}{0.709685in}}%
\pgfpathlineto{\pgfqpoint{1.276998in}{0.710951in}}%
\pgfpathlineto{\pgfqpoint{1.277604in}{0.709448in}}%
\pgfpathlineto{\pgfqpoint{1.277705in}{0.709444in}}%
\pgfpathlineto{\pgfqpoint{1.280031in}{0.711269in}}%
\pgfpathlineto{\pgfqpoint{1.280334in}{0.709694in}}%
\pgfpathlineto{\pgfqpoint{1.281749in}{0.708363in}}%
\pgfpathlineto{\pgfqpoint{1.284782in}{0.709213in}}%
\pgfpathlineto{\pgfqpoint{1.285186in}{0.718438in}}%
\pgfpathlineto{\pgfqpoint{1.285388in}{0.722474in}}%
\pgfpathlineto{\pgfqpoint{1.285995in}{0.709765in}}%
\pgfpathlineto{\pgfqpoint{1.287916in}{0.707881in}}%
\pgfpathlineto{\pgfqpoint{1.289028in}{0.708729in}}%
\pgfpathlineto{\pgfqpoint{1.289432in}{0.712370in}}%
\pgfpathlineto{\pgfqpoint{1.290241in}{0.709882in}}%
\pgfpathlineto{\pgfqpoint{1.291151in}{0.717232in}}%
\pgfpathlineto{\pgfqpoint{1.291555in}{0.712047in}}%
\pgfpathlineto{\pgfqpoint{1.291757in}{0.710281in}}%
\pgfpathlineto{\pgfqpoint{1.292162in}{0.715839in}}%
\pgfpathlineto{\pgfqpoint{1.292364in}{0.717465in}}%
\pgfpathlineto{\pgfqpoint{1.292768in}{0.709019in}}%
\pgfpathlineto{\pgfqpoint{1.293375in}{0.707317in}}%
\pgfpathlineto{\pgfqpoint{1.293880in}{0.707916in}}%
\pgfpathlineto{\pgfqpoint{1.294285in}{0.708849in}}%
\pgfpathlineto{\pgfqpoint{1.294790in}{0.707226in}}%
\pgfpathlineto{\pgfqpoint{1.294992in}{0.707169in}}%
\pgfpathlineto{\pgfqpoint{1.295195in}{0.708179in}}%
\pgfpathlineto{\pgfqpoint{1.295397in}{0.709894in}}%
\pgfpathlineto{\pgfqpoint{1.295902in}{0.706981in}}%
\pgfpathlineto{\pgfqpoint{1.296205in}{0.707167in}}%
\pgfpathlineto{\pgfqpoint{1.297520in}{0.710523in}}%
\pgfpathlineto{\pgfqpoint{1.297722in}{0.715362in}}%
\pgfpathlineto{\pgfqpoint{1.298328in}{0.707018in}}%
\pgfpathlineto{\pgfqpoint{1.298430in}{0.707092in}}%
\pgfpathlineto{\pgfqpoint{1.300552in}{0.707530in}}%
\pgfpathlineto{\pgfqpoint{1.305304in}{0.708773in}}%
\pgfpathlineto{\pgfqpoint{1.306012in}{0.708413in}}%
\pgfpathlineto{\pgfqpoint{1.306416in}{0.718558in}}%
\pgfpathlineto{\pgfqpoint{1.306618in}{0.724049in}}%
\pgfpathlineto{\pgfqpoint{1.307022in}{0.709333in}}%
\pgfpathlineto{\pgfqpoint{1.307528in}{0.719258in}}%
\pgfpathlineto{\pgfqpoint{1.309145in}{0.708385in}}%
\pgfpathlineto{\pgfqpoint{1.309449in}{0.708940in}}%
\pgfpathlineto{\pgfqpoint{1.309853in}{0.711327in}}%
\pgfpathlineto{\pgfqpoint{1.310257in}{0.708727in}}%
\pgfpathlineto{\pgfqpoint{1.310561in}{0.709055in}}%
\pgfpathlineto{\pgfqpoint{1.311976in}{0.708594in}}%
\pgfpathlineto{\pgfqpoint{1.312178in}{0.709240in}}%
\pgfpathlineto{\pgfqpoint{1.312380in}{0.710038in}}%
\pgfpathlineto{\pgfqpoint{1.312987in}{0.708339in}}%
\pgfpathlineto{\pgfqpoint{1.313088in}{0.708359in}}%
\pgfpathlineto{\pgfqpoint{1.313998in}{0.709066in}}%
\pgfpathlineto{\pgfqpoint{1.314301in}{0.724842in}}%
\pgfpathlineto{\pgfqpoint{1.314604in}{0.763138in}}%
\pgfpathlineto{\pgfqpoint{1.315514in}{0.744354in}}%
\pgfpathlineto{\pgfqpoint{1.315615in}{0.746034in}}%
\pgfpathlineto{\pgfqpoint{1.315818in}{0.736867in}}%
\pgfpathlineto{\pgfqpoint{1.317233in}{0.716646in}}%
\pgfpathlineto{\pgfqpoint{1.317536in}{0.722728in}}%
\pgfpathlineto{\pgfqpoint{1.317839in}{0.732372in}}%
\pgfpathlineto{\pgfqpoint{1.318345in}{0.709778in}}%
\pgfpathlineto{\pgfqpoint{1.319255in}{0.709059in}}%
\pgfpathlineto{\pgfqpoint{1.319457in}{0.709229in}}%
\pgfpathlineto{\pgfqpoint{1.319962in}{0.716024in}}%
\pgfpathlineto{\pgfqpoint{1.321277in}{0.712712in}}%
\pgfpathlineto{\pgfqpoint{1.321984in}{0.711557in}}%
\pgfpathlineto{\pgfqpoint{1.323298in}{0.709607in}}%
\pgfpathlineto{\pgfqpoint{1.323703in}{0.711856in}}%
\pgfpathlineto{\pgfqpoint{1.323905in}{0.713656in}}%
\pgfpathlineto{\pgfqpoint{1.324411in}{0.708525in}}%
\pgfpathlineto{\pgfqpoint{1.324613in}{0.708185in}}%
\pgfpathlineto{\pgfqpoint{1.324815in}{0.709859in}}%
\pgfpathlineto{\pgfqpoint{1.326331in}{0.751954in}}%
\pgfpathlineto{\pgfqpoint{1.325624in}{0.708211in}}%
\pgfpathlineto{\pgfqpoint{1.326533in}{0.734522in}}%
\pgfpathlineto{\pgfqpoint{1.327848in}{0.710324in}}%
\pgfpathlineto{\pgfqpoint{1.328353in}{0.717170in}}%
\pgfpathlineto{\pgfqpoint{1.328758in}{0.709459in}}%
\pgfpathlineto{\pgfqpoint{1.328859in}{0.709500in}}%
\pgfpathlineto{\pgfqpoint{1.329162in}{0.736262in}}%
\pgfpathlineto{\pgfqpoint{1.329364in}{0.764846in}}%
\pgfpathlineto{\pgfqpoint{1.330072in}{0.713560in}}%
\pgfpathlineto{\pgfqpoint{1.330375in}{0.720262in}}%
\pgfpathlineto{\pgfqpoint{1.330678in}{0.729457in}}%
\pgfpathlineto{\pgfqpoint{1.331386in}{0.721197in}}%
\pgfpathlineto{\pgfqpoint{1.333307in}{0.709522in}}%
\pgfpathlineto{\pgfqpoint{1.333711in}{0.707872in}}%
\pgfpathlineto{\pgfqpoint{1.334014in}{0.710719in}}%
\pgfpathlineto{\pgfqpoint{1.334217in}{0.713847in}}%
\pgfpathlineto{\pgfqpoint{1.334924in}{0.707156in}}%
\pgfpathlineto{\pgfqpoint{1.335329in}{0.708163in}}%
\pgfpathlineto{\pgfqpoint{1.336340in}{0.707418in}}%
\pgfpathlineto{\pgfqpoint{1.339170in}{0.708916in}}%
\pgfpathlineto{\pgfqpoint{1.339473in}{0.712726in}}%
\pgfpathlineto{\pgfqpoint{1.340181in}{0.708232in}}%
\pgfpathlineto{\pgfqpoint{1.340585in}{0.711124in}}%
\pgfpathlineto{\pgfqpoint{1.341495in}{0.709092in}}%
\pgfpathlineto{\pgfqpoint{1.342001in}{0.719151in}}%
\pgfpathlineto{\pgfqpoint{1.342506in}{0.707609in}}%
\pgfpathlineto{\pgfqpoint{1.343416in}{0.707677in}}%
\pgfpathlineto{\pgfqpoint{1.343719in}{0.708068in}}%
\pgfpathlineto{\pgfqpoint{1.343820in}{0.708597in}}%
\pgfpathlineto{\pgfqpoint{1.344326in}{0.711134in}}%
\pgfpathlineto{\pgfqpoint{1.344932in}{0.709152in}}%
\pgfpathlineto{\pgfqpoint{1.346045in}{0.707932in}}%
\pgfpathlineto{\pgfqpoint{1.346247in}{0.708419in}}%
\pgfpathlineto{\pgfqpoint{1.346853in}{0.716490in}}%
\pgfpathlineto{\pgfqpoint{1.347561in}{0.710265in}}%
\pgfpathlineto{\pgfqpoint{1.347662in}{0.710383in}}%
\pgfpathlineto{\pgfqpoint{1.347864in}{0.709321in}}%
\pgfpathlineto{\pgfqpoint{1.349077in}{0.707440in}}%
\pgfpathlineto{\pgfqpoint{1.349178in}{0.707456in}}%
\pgfpathlineto{\pgfqpoint{1.352818in}{0.708179in}}%
\pgfpathlineto{\pgfqpoint{1.353020in}{0.708128in}}%
\pgfpathlineto{\pgfqpoint{1.353222in}{0.708687in}}%
\pgfpathlineto{\pgfqpoint{1.353627in}{0.710488in}}%
\pgfpathlineto{\pgfqpoint{1.354233in}{0.708426in}}%
\pgfpathlineto{\pgfqpoint{1.354840in}{0.708487in}}%
\pgfpathlineto{\pgfqpoint{1.355042in}{0.708989in}}%
\pgfpathlineto{\pgfqpoint{1.355244in}{0.709353in}}%
\pgfpathlineto{\pgfqpoint{1.355749in}{0.708523in}}%
\pgfpathlineto{\pgfqpoint{1.356053in}{0.708576in}}%
\pgfpathlineto{\pgfqpoint{1.357771in}{0.711065in}}%
\pgfpathlineto{\pgfqpoint{1.358883in}{0.714575in}}%
\pgfpathlineto{\pgfqpoint{1.358479in}{0.709692in}}%
\pgfpathlineto{\pgfqpoint{1.359086in}{0.712277in}}%
\pgfpathlineto{\pgfqpoint{1.359591in}{0.707405in}}%
\pgfpathlineto{\pgfqpoint{1.360400in}{0.708076in}}%
\pgfpathlineto{\pgfqpoint{1.360602in}{0.708016in}}%
\pgfpathlineto{\pgfqpoint{1.360804in}{0.708720in}}%
\pgfpathlineto{\pgfqpoint{1.361107in}{0.709929in}}%
\pgfpathlineto{\pgfqpoint{1.361613in}{0.707483in}}%
\pgfpathlineto{\pgfqpoint{1.361815in}{0.707272in}}%
\pgfpathlineto{\pgfqpoint{1.362118in}{0.708056in}}%
\pgfpathlineto{\pgfqpoint{1.362321in}{0.709606in}}%
\pgfpathlineto{\pgfqpoint{1.362826in}{0.706961in}}%
\pgfpathlineto{\pgfqpoint{1.363028in}{0.707029in}}%
\pgfpathlineto{\pgfqpoint{1.364140in}{0.708634in}}%
\pgfpathlineto{\pgfqpoint{1.365353in}{0.727614in}}%
\pgfpathlineto{\pgfqpoint{1.364848in}{0.708081in}}%
\pgfpathlineto{\pgfqpoint{1.365657in}{0.713962in}}%
\pgfpathlineto{\pgfqpoint{1.365960in}{0.708025in}}%
\pgfpathlineto{\pgfqpoint{1.366364in}{0.715289in}}%
\pgfpathlineto{\pgfqpoint{1.366870in}{0.708051in}}%
\pgfpathlineto{\pgfqpoint{1.367881in}{0.706972in}}%
\pgfpathlineto{\pgfqpoint{1.368083in}{0.707313in}}%
\pgfpathlineto{\pgfqpoint{1.368588in}{0.713169in}}%
\pgfpathlineto{\pgfqpoint{1.369296in}{0.708123in}}%
\pgfpathlineto{\pgfqpoint{1.369801in}{0.717688in}}%
\pgfpathlineto{\pgfqpoint{1.370004in}{0.721463in}}%
\pgfpathlineto{\pgfqpoint{1.370509in}{0.708062in}}%
\pgfpathlineto{\pgfqpoint{1.370711in}{0.710353in}}%
\pgfpathlineto{\pgfqpoint{1.371116in}{0.730286in}}%
\pgfpathlineto{\pgfqpoint{1.371722in}{0.707153in}}%
\pgfpathlineto{\pgfqpoint{1.372935in}{0.706529in}}%
\pgfpathlineto{\pgfqpoint{1.372026in}{0.707513in}}%
\pgfpathlineto{\pgfqpoint{1.373138in}{0.706842in}}%
\pgfpathlineto{\pgfqpoint{1.373542in}{0.708604in}}%
\pgfpathlineto{\pgfqpoint{1.374250in}{0.707001in}}%
\pgfpathlineto{\pgfqpoint{1.375463in}{0.708316in}}%
\pgfpathlineto{\pgfqpoint{1.375867in}{0.711834in}}%
\pgfpathlineto{\pgfqpoint{1.376474in}{0.707939in}}%
\pgfpathlineto{\pgfqpoint{1.378900in}{0.709473in}}%
\pgfpathlineto{\pgfqpoint{1.380821in}{0.742676in}}%
\pgfpathlineto{\pgfqpoint{1.381023in}{0.728086in}}%
\pgfpathlineto{\pgfqpoint{1.382135in}{0.708086in}}%
\pgfpathlineto{\pgfqpoint{1.382337in}{0.709262in}}%
\pgfpathlineto{\pgfqpoint{1.383853in}{0.746712in}}%
\pgfpathlineto{\pgfqpoint{1.383955in}{0.742911in}}%
\pgfpathlineto{\pgfqpoint{1.385471in}{0.708844in}}%
\pgfpathlineto{\pgfqpoint{1.386684in}{0.708558in}}%
\pgfpathlineto{\pgfqpoint{1.387695in}{0.709659in}}%
\pgfpathlineto{\pgfqpoint{1.389110in}{0.714405in}}%
\pgfpathlineto{\pgfqpoint{1.389211in}{0.714070in}}%
\pgfpathlineto{\pgfqpoint{1.390930in}{0.708463in}}%
\pgfpathlineto{\pgfqpoint{1.392143in}{0.709488in}}%
\pgfpathlineto{\pgfqpoint{1.392548in}{0.715114in}}%
\pgfpathlineto{\pgfqpoint{1.393255in}{0.710413in}}%
\pgfpathlineto{\pgfqpoint{1.393558in}{0.710177in}}%
\pgfpathlineto{\pgfqpoint{1.393963in}{0.711017in}}%
\pgfpathlineto{\pgfqpoint{1.394367in}{0.714711in}}%
\pgfpathlineto{\pgfqpoint{1.394772in}{0.722908in}}%
\pgfpathlineto{\pgfqpoint{1.395378in}{0.714399in}}%
\pgfpathlineto{\pgfqpoint{1.397400in}{0.708150in}}%
\pgfpathlineto{\pgfqpoint{1.397602in}{0.709253in}}%
\pgfpathlineto{\pgfqpoint{1.397905in}{0.717019in}}%
\pgfpathlineto{\pgfqpoint{1.398613in}{0.707404in}}%
\pgfpathlineto{\pgfqpoint{1.401039in}{0.709677in}}%
\pgfpathlineto{\pgfqpoint{1.402252in}{0.713619in}}%
\pgfpathlineto{\pgfqpoint{1.402354in}{0.713381in}}%
\pgfpathlineto{\pgfqpoint{1.403061in}{0.707193in}}%
\pgfpathlineto{\pgfqpoint{1.403668in}{0.708263in}}%
\pgfpathlineto{\pgfqpoint{1.404173in}{0.733241in}}%
\pgfpathlineto{\pgfqpoint{1.404982in}{0.717604in}}%
\pgfpathlineto{\pgfqpoint{1.406599in}{0.707055in}}%
\pgfpathlineto{\pgfqpoint{1.408217in}{0.706811in}}%
\pgfpathlineto{\pgfqpoint{1.410542in}{0.709187in}}%
\pgfpathlineto{\pgfqpoint{1.412059in}{0.741966in}}%
\pgfpathlineto{\pgfqpoint{1.412160in}{0.744761in}}%
\pgfpathlineto{\pgfqpoint{1.412564in}{0.727031in}}%
\pgfpathlineto{\pgfqpoint{1.413676in}{0.715682in}}%
\pgfpathlineto{\pgfqpoint{1.413878in}{0.716778in}}%
\pgfpathlineto{\pgfqpoint{1.414384in}{0.725299in}}%
\pgfpathlineto{\pgfqpoint{1.414788in}{0.713592in}}%
\pgfpathlineto{\pgfqpoint{1.416102in}{0.708288in}}%
\pgfpathlineto{\pgfqpoint{1.416507in}{0.708315in}}%
\pgfpathlineto{\pgfqpoint{1.416810in}{0.709136in}}%
\pgfpathlineto{\pgfqpoint{1.417922in}{0.710541in}}%
\pgfpathlineto{\pgfqpoint{1.417416in}{0.708607in}}%
\pgfpathlineto{\pgfqpoint{1.418124in}{0.709990in}}%
\pgfpathlineto{\pgfqpoint{1.419641in}{0.708295in}}%
\pgfpathlineto{\pgfqpoint{1.421460in}{0.708170in}}%
\pgfpathlineto{\pgfqpoint{1.421561in}{0.708438in}}%
\pgfpathlineto{\pgfqpoint{1.423078in}{0.720885in}}%
\pgfpathlineto{\pgfqpoint{1.423381in}{0.713872in}}%
\pgfpathlineto{\pgfqpoint{1.424594in}{0.708097in}}%
\pgfpathlineto{\pgfqpoint{1.425201in}{0.709153in}}%
\pgfpathlineto{\pgfqpoint{1.425403in}{0.709764in}}%
\pgfpathlineto{\pgfqpoint{1.425908in}{0.708691in}}%
\pgfpathlineto{\pgfqpoint{1.426212in}{0.708755in}}%
\pgfpathlineto{\pgfqpoint{1.429750in}{0.708963in}}%
\pgfpathlineto{\pgfqpoint{1.430053in}{0.711960in}}%
\pgfpathlineto{\pgfqpoint{1.430559in}{0.708079in}}%
\pgfpathlineto{\pgfqpoint{1.430761in}{0.708156in}}%
\pgfpathlineto{\pgfqpoint{1.434400in}{0.710770in}}%
\pgfpathlineto{\pgfqpoint{1.435613in}{0.718446in}}%
\pgfpathlineto{\pgfqpoint{1.435007in}{0.708992in}}%
\pgfpathlineto{\pgfqpoint{1.435815in}{0.713090in}}%
\pgfpathlineto{\pgfqpoint{1.436119in}{0.708486in}}%
\pgfpathlineto{\pgfqpoint{1.436422in}{0.722440in}}%
\pgfpathlineto{\pgfqpoint{1.437837in}{0.818118in}}%
\pgfpathlineto{\pgfqpoint{1.437332in}{0.716290in}}%
\pgfpathlineto{\pgfqpoint{1.437938in}{0.816614in}}%
\pgfpathlineto{\pgfqpoint{1.438646in}{0.716212in}}%
\pgfpathlineto{\pgfqpoint{1.439556in}{0.721918in}}%
\pgfpathlineto{\pgfqpoint{1.439859in}{0.710920in}}%
\pgfpathlineto{\pgfqpoint{1.440769in}{0.713283in}}%
\pgfpathlineto{\pgfqpoint{1.441376in}{0.709785in}}%
\pgfpathlineto{\pgfqpoint{1.441982in}{0.710687in}}%
\pgfpathlineto{\pgfqpoint{1.443195in}{0.709398in}}%
\pgfpathlineto{\pgfqpoint{1.443701in}{0.709964in}}%
\pgfpathlineto{\pgfqpoint{1.446430in}{0.709736in}}%
\pgfpathlineto{\pgfqpoint{1.446936in}{0.711753in}}%
\pgfpathlineto{\pgfqpoint{1.447441in}{0.709747in}}%
\pgfpathlineto{\pgfqpoint{1.447643in}{0.709487in}}%
\pgfpathlineto{\pgfqpoint{1.448048in}{0.710962in}}%
\pgfpathlineto{\pgfqpoint{1.448250in}{0.711766in}}%
\pgfpathlineto{\pgfqpoint{1.448857in}{0.709297in}}%
\pgfpathlineto{\pgfqpoint{1.450070in}{0.709161in}}%
\pgfpathlineto{\pgfqpoint{1.450171in}{0.709300in}}%
\pgfpathlineto{\pgfqpoint{1.450575in}{0.710614in}}%
\pgfpathlineto{\pgfqpoint{1.451182in}{0.709016in}}%
\pgfpathlineto{\pgfqpoint{1.457146in}{0.709804in}}%
\pgfpathlineto{\pgfqpoint{1.457753in}{0.722390in}}%
\pgfpathlineto{\pgfqpoint{1.458764in}{0.715018in}}%
\pgfpathlineto{\pgfqpoint{1.460280in}{0.708406in}}%
\pgfpathlineto{\pgfqpoint{1.460583in}{0.709258in}}%
\pgfpathlineto{\pgfqpoint{1.460887in}{0.711800in}}%
\pgfpathlineto{\pgfqpoint{1.461493in}{0.708063in}}%
\pgfpathlineto{\pgfqpoint{1.463010in}{0.708425in}}%
\pgfpathlineto{\pgfqpoint{1.463313in}{0.709386in}}%
\pgfpathlineto{\pgfqpoint{1.464122in}{0.708435in}}%
\pgfpathlineto{\pgfqpoint{1.465436in}{0.708811in}}%
\pgfpathlineto{\pgfqpoint{1.465840in}{0.710155in}}%
\pgfpathlineto{\pgfqpoint{1.466346in}{0.708457in}}%
\pgfpathlineto{\pgfqpoint{1.466447in}{0.708453in}}%
\pgfpathlineto{\pgfqpoint{1.470288in}{0.710274in}}%
\pgfpathlineto{\pgfqpoint{1.470895in}{0.715418in}}%
\pgfpathlineto{\pgfqpoint{1.471400in}{0.711096in}}%
\pgfpathlineto{\pgfqpoint{1.471704in}{0.710198in}}%
\pgfpathlineto{\pgfqpoint{1.472108in}{0.712786in}}%
\pgfpathlineto{\pgfqpoint{1.472310in}{0.713771in}}%
\pgfpathlineto{\pgfqpoint{1.472715in}{0.709405in}}%
\pgfpathlineto{\pgfqpoint{1.472917in}{0.708797in}}%
\pgfpathlineto{\pgfqpoint{1.473422in}{0.711302in}}%
\pgfpathlineto{\pgfqpoint{1.473523in}{0.711725in}}%
\pgfpathlineto{\pgfqpoint{1.474029in}{0.709701in}}%
\pgfpathlineto{\pgfqpoint{1.475343in}{0.708762in}}%
\pgfpathlineto{\pgfqpoint{1.475444in}{0.708837in}}%
\pgfpathlineto{\pgfqpoint{1.476859in}{0.708991in}}%
\pgfpathlineto{\pgfqpoint{1.478679in}{0.708608in}}%
\pgfpathlineto{\pgfqpoint{1.478982in}{0.707950in}}%
\pgfpathlineto{\pgfqpoint{1.479185in}{0.709067in}}%
\pgfpathlineto{\pgfqpoint{1.479589in}{0.731824in}}%
\pgfpathlineto{\pgfqpoint{1.480701in}{0.724511in}}%
\pgfpathlineto{\pgfqpoint{1.481206in}{0.707727in}}%
\pgfpathlineto{\pgfqpoint{1.482015in}{0.707754in}}%
\pgfpathlineto{\pgfqpoint{1.483228in}{0.708559in}}%
\pgfpathlineto{\pgfqpoint{1.483532in}{0.713182in}}%
\pgfpathlineto{\pgfqpoint{1.484138in}{0.707827in}}%
\pgfpathlineto{\pgfqpoint{1.484239in}{0.707944in}}%
\pgfpathlineto{\pgfqpoint{1.485351in}{0.708560in}}%
\pgfpathlineto{\pgfqpoint{1.485553in}{0.708282in}}%
\pgfpathlineto{\pgfqpoint{1.485756in}{0.708084in}}%
\pgfpathlineto{\pgfqpoint{1.486059in}{0.709414in}}%
\pgfpathlineto{\pgfqpoint{1.486564in}{0.722278in}}%
\pgfpathlineto{\pgfqpoint{1.487070in}{0.709319in}}%
\pgfpathlineto{\pgfqpoint{1.487171in}{0.709480in}}%
\pgfpathlineto{\pgfqpoint{1.487676in}{0.723697in}}%
\pgfpathlineto{\pgfqpoint{1.488081in}{0.707553in}}%
\pgfpathlineto{\pgfqpoint{1.488182in}{0.707265in}}%
\pgfpathlineto{\pgfqpoint{1.488384in}{0.709641in}}%
\pgfpathlineto{\pgfqpoint{1.488687in}{0.722945in}}%
\pgfpathlineto{\pgfqpoint{1.489294in}{0.708312in}}%
\pgfpathlineto{\pgfqpoint{1.489496in}{0.710608in}}%
\pgfpathlineto{\pgfqpoint{1.489698in}{0.713375in}}%
\pgfpathlineto{\pgfqpoint{1.490507in}{0.708952in}}%
\pgfpathlineto{\pgfqpoint{1.491619in}{0.709658in}}%
\pgfpathlineto{\pgfqpoint{1.491821in}{0.708971in}}%
\pgfpathlineto{\pgfqpoint{1.492630in}{0.706924in}}%
\pgfpathlineto{\pgfqpoint{1.493135in}{0.707384in}}%
\pgfpathlineto{\pgfqpoint{1.494450in}{0.713188in}}%
\pgfpathlineto{\pgfqpoint{1.494753in}{0.722475in}}%
\pgfpathlineto{\pgfqpoint{1.495461in}{0.711799in}}%
\pgfpathlineto{\pgfqpoint{1.495562in}{0.711727in}}%
\pgfpathlineto{\pgfqpoint{1.495764in}{0.712567in}}%
\pgfpathlineto{\pgfqpoint{1.495966in}{0.713387in}}%
\pgfpathlineto{\pgfqpoint{1.496370in}{0.710189in}}%
\pgfpathlineto{\pgfqpoint{1.497381in}{0.707482in}}%
\pgfpathlineto{\pgfqpoint{1.497685in}{0.707705in}}%
\pgfpathlineto{\pgfqpoint{1.499201in}{0.709153in}}%
\pgfpathlineto{\pgfqpoint{1.499605in}{0.728240in}}%
\pgfpathlineto{\pgfqpoint{1.500313in}{0.710779in}}%
\pgfpathlineto{\pgfqpoint{1.501627in}{0.719736in}}%
\pgfpathlineto{\pgfqpoint{1.501021in}{0.709865in}}%
\pgfpathlineto{\pgfqpoint{1.501931in}{0.716371in}}%
\pgfpathlineto{\pgfqpoint{1.502739in}{0.712142in}}%
\pgfpathlineto{\pgfqpoint{1.503043in}{0.715741in}}%
\pgfpathlineto{\pgfqpoint{1.503548in}{0.742677in}}%
\pgfpathlineto{\pgfqpoint{1.503952in}{0.714234in}}%
\pgfpathlineto{\pgfqpoint{1.504559in}{0.725438in}}%
\pgfpathlineto{\pgfqpoint{1.505267in}{0.707006in}}%
\pgfpathlineto{\pgfqpoint{1.506581in}{0.706950in}}%
\pgfpathlineto{\pgfqpoint{1.509715in}{0.709513in}}%
\pgfpathlineto{\pgfqpoint{1.510422in}{0.734168in}}%
\pgfpathlineto{\pgfqpoint{1.511534in}{0.722326in}}%
\pgfpathlineto{\pgfqpoint{1.512141in}{0.708007in}}%
\pgfpathlineto{\pgfqpoint{1.512950in}{0.710160in}}%
\pgfpathlineto{\pgfqpoint{1.513051in}{0.710859in}}%
\pgfpathlineto{\pgfqpoint{1.513455in}{0.708439in}}%
\pgfpathlineto{\pgfqpoint{1.513961in}{0.709752in}}%
\pgfpathlineto{\pgfqpoint{1.514365in}{0.707830in}}%
\pgfpathlineto{\pgfqpoint{1.515174in}{0.708194in}}%
\pgfpathlineto{\pgfqpoint{1.517196in}{0.708460in}}%
\pgfpathlineto{\pgfqpoint{1.518308in}{0.709167in}}%
\pgfpathlineto{\pgfqpoint{1.518611in}{0.713055in}}%
\pgfpathlineto{\pgfqpoint{1.519015in}{0.708885in}}%
\pgfpathlineto{\pgfqpoint{1.519420in}{0.710392in}}%
\pgfpathlineto{\pgfqpoint{1.519824in}{0.708382in}}%
\pgfpathlineto{\pgfqpoint{1.520633in}{0.708970in}}%
\pgfpathlineto{\pgfqpoint{1.520936in}{0.710051in}}%
\pgfpathlineto{\pgfqpoint{1.521442in}{0.708757in}}%
\pgfpathlineto{\pgfqpoint{1.521745in}{0.708883in}}%
\pgfpathlineto{\pgfqpoint{1.523059in}{0.709349in}}%
\pgfpathlineto{\pgfqpoint{1.523362in}{0.711613in}}%
\pgfpathlineto{\pgfqpoint{1.523868in}{0.708998in}}%
\pgfpathlineto{\pgfqpoint{1.524171in}{0.709408in}}%
\pgfpathlineto{\pgfqpoint{1.525991in}{0.709052in}}%
\pgfpathlineto{\pgfqpoint{1.527103in}{0.709685in}}%
\pgfpathlineto{\pgfqpoint{1.527507in}{0.715627in}}%
\pgfpathlineto{\pgfqpoint{1.528518in}{0.713733in}}%
\pgfpathlineto{\pgfqpoint{1.529125in}{0.708876in}}%
\pgfpathlineto{\pgfqpoint{1.529832in}{0.709027in}}%
\pgfpathlineto{\pgfqpoint{1.530136in}{0.709338in}}%
\pgfpathlineto{\pgfqpoint{1.530540in}{0.716193in}}%
\pgfpathlineto{\pgfqpoint{1.531046in}{0.709217in}}%
\pgfpathlineto{\pgfqpoint{1.531349in}{0.711243in}}%
\pgfpathlineto{\pgfqpoint{1.531551in}{0.710441in}}%
\pgfpathlineto{\pgfqpoint{1.532056in}{0.708464in}}%
\pgfpathlineto{\pgfqpoint{1.532461in}{0.710116in}}%
\pgfpathlineto{\pgfqpoint{1.533674in}{0.750971in}}%
\pgfpathlineto{\pgfqpoint{1.533067in}{0.709957in}}%
\pgfpathlineto{\pgfqpoint{1.533977in}{0.724696in}}%
\pgfpathlineto{\pgfqpoint{1.534280in}{0.711690in}}%
\pgfpathlineto{\pgfqpoint{1.534887in}{0.728672in}}%
\pgfpathlineto{\pgfqpoint{1.534988in}{0.728128in}}%
\pgfpathlineto{\pgfqpoint{1.535898in}{0.712073in}}%
\pgfpathlineto{\pgfqpoint{1.536403in}{0.719489in}}%
\pgfpathlineto{\pgfqpoint{1.536808in}{0.722625in}}%
\pgfpathlineto{\pgfqpoint{1.537313in}{0.717297in}}%
\pgfpathlineto{\pgfqpoint{1.537718in}{0.712736in}}%
\pgfpathlineto{\pgfqpoint{1.538021in}{0.719528in}}%
\pgfpathlineto{\pgfqpoint{1.538425in}{0.747359in}}%
\pgfpathlineto{\pgfqpoint{1.538931in}{0.712776in}}%
\pgfpathlineto{\pgfqpoint{1.539537in}{0.715273in}}%
\pgfpathlineto{\pgfqpoint{1.540245in}{0.708607in}}%
\pgfpathlineto{\pgfqpoint{1.541761in}{0.708190in}}%
\pgfpathlineto{\pgfqpoint{1.543076in}{0.709070in}}%
\pgfpathlineto{\pgfqpoint{1.544289in}{0.715882in}}%
\pgfpathlineto{\pgfqpoint{1.543682in}{0.708612in}}%
\pgfpathlineto{\pgfqpoint{1.544491in}{0.711962in}}%
\pgfpathlineto{\pgfqpoint{1.544895in}{0.708436in}}%
\pgfpathlineto{\pgfqpoint{1.545704in}{0.708756in}}%
\pgfpathlineto{\pgfqpoint{1.546311in}{0.707932in}}%
\pgfpathlineto{\pgfqpoint{1.546614in}{0.709096in}}%
\pgfpathlineto{\pgfqpoint{1.546917in}{0.712765in}}%
\pgfpathlineto{\pgfqpoint{1.547524in}{0.708552in}}%
\pgfpathlineto{\pgfqpoint{1.547625in}{0.708701in}}%
\pgfpathlineto{\pgfqpoint{1.549040in}{0.710031in}}%
\pgfpathlineto{\pgfqpoint{1.549141in}{0.709749in}}%
\pgfpathlineto{\pgfqpoint{1.550455in}{0.707445in}}%
\pgfpathlineto{\pgfqpoint{1.549950in}{0.710447in}}%
\pgfpathlineto{\pgfqpoint{1.550658in}{0.707526in}}%
\pgfpathlineto{\pgfqpoint{1.553589in}{0.709201in}}%
\pgfpathlineto{\pgfqpoint{1.553893in}{0.708525in}}%
\pgfpathlineto{\pgfqpoint{1.554095in}{0.708705in}}%
\pgfpathlineto{\pgfqpoint{1.554196in}{0.709272in}}%
\pgfpathlineto{\pgfqpoint{1.554499in}{0.711071in}}%
\pgfpathlineto{\pgfqpoint{1.555005in}{0.708136in}}%
\pgfpathlineto{\pgfqpoint{1.555106in}{0.708134in}}%
\pgfpathlineto{\pgfqpoint{1.556420in}{0.709281in}}%
\pgfpathlineto{\pgfqpoint{1.557633in}{0.712827in}}%
\pgfpathlineto{\pgfqpoint{1.557027in}{0.708881in}}%
\pgfpathlineto{\pgfqpoint{1.557835in}{0.711839in}}%
\pgfpathlineto{\pgfqpoint{1.558240in}{0.710059in}}%
\pgfpathlineto{\pgfqpoint{1.558947in}{0.710505in}}%
\pgfpathlineto{\pgfqpoint{1.559554in}{0.715837in}}%
\pgfpathlineto{\pgfqpoint{1.560059in}{0.711762in}}%
\pgfpathlineto{\pgfqpoint{1.560565in}{0.708661in}}%
\pgfpathlineto{\pgfqpoint{1.561171in}{0.711771in}}%
\pgfpathlineto{\pgfqpoint{1.563092in}{0.707103in}}%
\pgfpathlineto{\pgfqpoint{1.563193in}{0.707145in}}%
\pgfpathlineto{\pgfqpoint{1.565822in}{0.709772in}}%
\pgfpathlineto{\pgfqpoint{1.566630in}{0.720096in}}%
\pgfpathlineto{\pgfqpoint{1.566934in}{0.743238in}}%
\pgfpathlineto{\pgfqpoint{1.567540in}{0.707543in}}%
\pgfpathlineto{\pgfqpoint{1.567844in}{0.708665in}}%
\pgfpathlineto{\pgfqpoint{1.568753in}{0.707747in}}%
\pgfpathlineto{\pgfqpoint{1.570270in}{0.708204in}}%
\pgfpathlineto{\pgfqpoint{1.571584in}{0.724449in}}%
\pgfpathlineto{\pgfqpoint{1.571887in}{0.712857in}}%
\pgfpathlineto{\pgfqpoint{1.573201in}{0.707553in}}%
\pgfpathlineto{\pgfqpoint{1.576133in}{0.708337in}}%
\pgfpathlineto{\pgfqpoint{1.577245in}{0.710852in}}%
\pgfpathlineto{\pgfqpoint{1.577548in}{0.716951in}}%
\pgfpathlineto{\pgfqpoint{1.578256in}{0.708917in}}%
\pgfpathlineto{\pgfqpoint{1.578357in}{0.708852in}}%
\pgfpathlineto{\pgfqpoint{1.578559in}{0.709645in}}%
\pgfpathlineto{\pgfqpoint{1.579065in}{0.723034in}}%
\pgfpathlineto{\pgfqpoint{1.579570in}{0.709738in}}%
\pgfpathlineto{\pgfqpoint{1.579671in}{0.709327in}}%
\pgfpathlineto{\pgfqpoint{1.579874in}{0.711205in}}%
\pgfpathlineto{\pgfqpoint{1.580278in}{0.724259in}}%
\pgfpathlineto{\pgfqpoint{1.580885in}{0.708679in}}%
\pgfpathlineto{\pgfqpoint{1.581289in}{0.709655in}}%
\pgfpathlineto{\pgfqpoint{1.581997in}{0.708351in}}%
\pgfpathlineto{\pgfqpoint{1.582401in}{0.708779in}}%
\pgfpathlineto{\pgfqpoint{1.583513in}{0.721829in}}%
\pgfpathlineto{\pgfqpoint{1.583917in}{0.774852in}}%
\pgfpathlineto{\pgfqpoint{1.584524in}{0.714325in}}%
\pgfpathlineto{\pgfqpoint{1.584928in}{0.718218in}}%
\pgfpathlineto{\pgfqpoint{1.585232in}{0.713070in}}%
\pgfpathlineto{\pgfqpoint{1.586141in}{0.708701in}}%
\pgfpathlineto{\pgfqpoint{1.586546in}{0.708919in}}%
\pgfpathlineto{\pgfqpoint{1.588365in}{0.710305in}}%
\pgfpathlineto{\pgfqpoint{1.589781in}{0.747903in}}%
\pgfpathlineto{\pgfqpoint{1.590084in}{0.727776in}}%
\pgfpathlineto{\pgfqpoint{1.590590in}{0.709295in}}%
\pgfpathlineto{\pgfqpoint{1.591398in}{0.709455in}}%
\pgfpathlineto{\pgfqpoint{1.591904in}{0.708854in}}%
\pgfpathlineto{\pgfqpoint{1.592308in}{0.710017in}}%
\pgfpathlineto{\pgfqpoint{1.592510in}{0.710680in}}%
\pgfpathlineto{\pgfqpoint{1.593016in}{0.709152in}}%
\pgfpathlineto{\pgfqpoint{1.593218in}{0.709138in}}%
\pgfpathlineto{\pgfqpoint{1.597464in}{0.709525in}}%
\pgfpathlineto{\pgfqpoint{1.598980in}{0.740634in}}%
\pgfpathlineto{\pgfqpoint{1.598374in}{0.709303in}}%
\pgfpathlineto{\pgfqpoint{1.599284in}{0.723227in}}%
\pgfpathlineto{\pgfqpoint{1.599688in}{0.712943in}}%
\pgfpathlineto{\pgfqpoint{1.600497in}{0.714764in}}%
\pgfpathlineto{\pgfqpoint{1.602215in}{0.708695in}}%
\pgfpathlineto{\pgfqpoint{1.604844in}{0.709290in}}%
\pgfpathlineto{\pgfqpoint{1.605956in}{0.709287in}}%
\pgfpathlineto{\pgfqpoint{1.606360in}{0.708900in}}%
\pgfpathlineto{\pgfqpoint{1.606562in}{0.709874in}}%
\pgfpathlineto{\pgfqpoint{1.607775in}{0.713814in}}%
\pgfpathlineto{\pgfqpoint{1.607877in}{0.713568in}}%
\pgfpathlineto{\pgfqpoint{1.608887in}{0.708875in}}%
\pgfpathlineto{\pgfqpoint{1.609393in}{0.709120in}}%
\pgfpathlineto{\pgfqpoint{1.611516in}{0.709052in}}%
\pgfpathlineto{\pgfqpoint{1.613942in}{0.711445in}}%
\pgfpathlineto{\pgfqpoint{1.614144in}{0.712841in}}%
\pgfpathlineto{\pgfqpoint{1.614650in}{0.708857in}}%
\pgfpathlineto{\pgfqpoint{1.614852in}{0.708635in}}%
\pgfpathlineto{\pgfqpoint{1.615155in}{0.709814in}}%
\pgfpathlineto{\pgfqpoint{1.615459in}{0.711642in}}%
\pgfpathlineto{\pgfqpoint{1.615964in}{0.708618in}}%
\pgfpathlineto{\pgfqpoint{1.616166in}{0.709059in}}%
\pgfpathlineto{\pgfqpoint{1.616469in}{0.709785in}}%
\pgfpathlineto{\pgfqpoint{1.616975in}{0.708697in}}%
\pgfpathlineto{\pgfqpoint{1.617278in}{0.708970in}}%
\pgfpathlineto{\pgfqpoint{1.618592in}{0.713363in}}%
\pgfpathlineto{\pgfqpoint{1.618896in}{0.710396in}}%
\pgfpathlineto{\pgfqpoint{1.619199in}{0.708319in}}%
\pgfpathlineto{\pgfqpoint{1.619502in}{0.712292in}}%
\pgfpathlineto{\pgfqpoint{1.619806in}{0.722568in}}%
\pgfpathlineto{\pgfqpoint{1.620412in}{0.709222in}}%
\pgfpathlineto{\pgfqpoint{1.620513in}{0.709458in}}%
\pgfpathlineto{\pgfqpoint{1.620918in}{0.732755in}}%
\pgfpathlineto{\pgfqpoint{1.621120in}{0.743940in}}%
\pgfpathlineto{\pgfqpoint{1.621726in}{0.710975in}}%
\pgfpathlineto{\pgfqpoint{1.621827in}{0.710500in}}%
\pgfpathlineto{\pgfqpoint{1.621929in}{0.711806in}}%
\pgfpathlineto{\pgfqpoint{1.622333in}{0.724603in}}%
\pgfpathlineto{\pgfqpoint{1.622838in}{0.707661in}}%
\pgfpathlineto{\pgfqpoint{1.623445in}{0.708334in}}%
\pgfpathlineto{\pgfqpoint{1.624051in}{0.707021in}}%
\pgfpathlineto{\pgfqpoint{1.625265in}{0.707937in}}%
\pgfpathlineto{\pgfqpoint{1.625669in}{0.714080in}}%
\pgfpathlineto{\pgfqpoint{1.626174in}{0.707475in}}%
\pgfpathlineto{\pgfqpoint{1.626478in}{0.709952in}}%
\pgfpathlineto{\pgfqpoint{1.626579in}{0.710011in}}%
\pgfpathlineto{\pgfqpoint{1.626680in}{0.708989in}}%
\pgfpathlineto{\pgfqpoint{1.626983in}{0.706871in}}%
\pgfpathlineto{\pgfqpoint{1.627893in}{0.707350in}}%
\pgfpathlineto{\pgfqpoint{1.629207in}{0.708836in}}%
\pgfpathlineto{\pgfqpoint{1.629409in}{0.709693in}}%
\pgfpathlineto{\pgfqpoint{1.629814in}{0.708071in}}%
\pgfpathlineto{\pgfqpoint{1.630319in}{0.709248in}}%
\pgfpathlineto{\pgfqpoint{1.630825in}{0.710108in}}%
\pgfpathlineto{\pgfqpoint{1.631128in}{0.712414in}}%
\pgfpathlineto{\pgfqpoint{1.631633in}{0.709931in}}%
\pgfpathlineto{\pgfqpoint{1.632038in}{0.710994in}}%
\pgfpathlineto{\pgfqpoint{1.632341in}{0.711765in}}%
\pgfpathlineto{\pgfqpoint{1.632644in}{0.709905in}}%
\pgfpathlineto{\pgfqpoint{1.633251in}{0.707533in}}%
\pgfpathlineto{\pgfqpoint{1.633959in}{0.707785in}}%
\pgfpathlineto{\pgfqpoint{1.634464in}{0.707837in}}%
\pgfpathlineto{\pgfqpoint{1.634565in}{0.708121in}}%
\pgfpathlineto{\pgfqpoint{1.636688in}{0.726022in}}%
\pgfpathlineto{\pgfqpoint{1.636890in}{0.720058in}}%
\pgfpathlineto{\pgfqpoint{1.638205in}{0.708719in}}%
\pgfpathlineto{\pgfqpoint{1.638609in}{0.707812in}}%
\pgfpathlineto{\pgfqpoint{1.639114in}{0.709506in}}%
\pgfpathlineto{\pgfqpoint{1.640732in}{0.736572in}}%
\pgfpathlineto{\pgfqpoint{1.640934in}{0.722495in}}%
\pgfpathlineto{\pgfqpoint{1.642147in}{0.708042in}}%
\pgfpathlineto{\pgfqpoint{1.642653in}{0.709073in}}%
\pgfpathlineto{\pgfqpoint{1.643765in}{0.711773in}}%
\pgfpathlineto{\pgfqpoint{1.643158in}{0.708383in}}%
\pgfpathlineto{\pgfqpoint{1.643967in}{0.709853in}}%
\pgfpathlineto{\pgfqpoint{1.645079in}{0.708079in}}%
\pgfpathlineto{\pgfqpoint{1.645180in}{0.708107in}}%
\pgfpathlineto{\pgfqpoint{1.648415in}{0.708601in}}%
\pgfpathlineto{\pgfqpoint{1.649830in}{0.708441in}}%
\pgfpathlineto{\pgfqpoint{1.649931in}{0.708563in}}%
\pgfpathlineto{\pgfqpoint{1.651852in}{0.710788in}}%
\pgfpathlineto{\pgfqpoint{1.651953in}{0.710523in}}%
\pgfpathlineto{\pgfqpoint{1.652459in}{0.708338in}}%
\pgfpathlineto{\pgfqpoint{1.653267in}{0.708524in}}%
\pgfpathlineto{\pgfqpoint{1.654481in}{0.709308in}}%
\pgfpathlineto{\pgfqpoint{1.654885in}{0.717302in}}%
\pgfpathlineto{\pgfqpoint{1.655593in}{0.709787in}}%
\pgfpathlineto{\pgfqpoint{1.656199in}{0.708867in}}%
\pgfpathlineto{\pgfqpoint{1.657918in}{0.708823in}}%
\pgfpathlineto{\pgfqpoint{1.659737in}{0.714799in}}%
\pgfpathlineto{\pgfqpoint{1.659940in}{0.713037in}}%
\pgfpathlineto{\pgfqpoint{1.660748in}{0.709312in}}%
\pgfpathlineto{\pgfqpoint{1.661052in}{0.712672in}}%
\pgfpathlineto{\pgfqpoint{1.661355in}{0.726260in}}%
\pgfpathlineto{\pgfqpoint{1.662063in}{0.708677in}}%
\pgfpathlineto{\pgfqpoint{1.664590in}{0.708307in}}%
\pgfpathlineto{\pgfqpoint{1.666207in}{0.711949in}}%
\pgfpathlineto{\pgfqpoint{1.666612in}{0.719496in}}%
\pgfpathlineto{\pgfqpoint{1.667319in}{0.713331in}}%
\pgfpathlineto{\pgfqpoint{1.667623in}{0.714365in}}%
\pgfpathlineto{\pgfqpoint{1.668229in}{0.733809in}}%
\pgfpathlineto{\pgfqpoint{1.668735in}{0.716623in}}%
\pgfpathlineto{\pgfqpoint{1.670150in}{0.707219in}}%
\pgfpathlineto{\pgfqpoint{1.671565in}{0.708023in}}%
\pgfpathlineto{\pgfqpoint{1.672475in}{0.709111in}}%
\pgfpathlineto{\pgfqpoint{1.673891in}{0.728066in}}%
\pgfpathlineto{\pgfqpoint{1.673385in}{0.709053in}}%
\pgfpathlineto{\pgfqpoint{1.673992in}{0.723739in}}%
\pgfpathlineto{\pgfqpoint{1.675306in}{0.707836in}}%
\pgfpathlineto{\pgfqpoint{1.678541in}{0.708903in}}%
\pgfpathlineto{\pgfqpoint{1.680057in}{0.718968in}}%
\pgfpathlineto{\pgfqpoint{1.679350in}{0.707949in}}%
\pgfpathlineto{\pgfqpoint{1.680361in}{0.714454in}}%
\pgfpathlineto{\pgfqpoint{1.681068in}{0.711569in}}%
\pgfpathlineto{\pgfqpoint{1.681270in}{0.712959in}}%
\pgfpathlineto{\pgfqpoint{1.681675in}{0.754396in}}%
\pgfpathlineto{\pgfqpoint{1.681877in}{0.783807in}}%
\pgfpathlineto{\pgfqpoint{1.682585in}{0.721725in}}%
\pgfpathlineto{\pgfqpoint{1.682686in}{0.720236in}}%
\pgfpathlineto{\pgfqpoint{1.683292in}{0.726811in}}%
\pgfpathlineto{\pgfqpoint{1.683494in}{0.724690in}}%
\pgfpathlineto{\pgfqpoint{1.684202in}{0.708823in}}%
\pgfpathlineto{\pgfqpoint{1.685011in}{0.710302in}}%
\pgfpathlineto{\pgfqpoint{1.685718in}{0.708753in}}%
\pgfpathlineto{\pgfqpoint{1.686123in}{0.710063in}}%
\pgfpathlineto{\pgfqpoint{1.686325in}{0.710356in}}%
\pgfpathlineto{\pgfqpoint{1.686830in}{0.708978in}}%
\pgfpathlineto{\pgfqpoint{1.688044in}{0.709434in}}%
\pgfpathlineto{\pgfqpoint{1.688448in}{0.710269in}}%
\pgfpathlineto{\pgfqpoint{1.688852in}{0.708617in}}%
\pgfpathlineto{\pgfqpoint{1.689156in}{0.708602in}}%
\pgfpathlineto{\pgfqpoint{1.689358in}{0.709400in}}%
\pgfpathlineto{\pgfqpoint{1.690874in}{0.726698in}}%
\pgfpathlineto{\pgfqpoint{1.690167in}{0.708985in}}%
\pgfpathlineto{\pgfqpoint{1.691178in}{0.717240in}}%
\pgfpathlineto{\pgfqpoint{1.691481in}{0.710673in}}%
\pgfpathlineto{\pgfqpoint{1.692391in}{0.713357in}}%
\pgfpathlineto{\pgfqpoint{1.692795in}{0.708214in}}%
\pgfpathlineto{\pgfqpoint{1.693199in}{0.718955in}}%
\pgfpathlineto{\pgfqpoint{1.693503in}{0.746967in}}%
\pgfpathlineto{\pgfqpoint{1.694210in}{0.716400in}}%
\pgfpathlineto{\pgfqpoint{1.694817in}{0.740921in}}%
\pgfpathlineto{\pgfqpoint{1.695322in}{0.718643in}}%
\pgfpathlineto{\pgfqpoint{1.696738in}{0.709580in}}%
\pgfpathlineto{\pgfqpoint{1.697243in}{0.708118in}}%
\pgfpathlineto{\pgfqpoint{1.697951in}{0.708728in}}%
\pgfpathlineto{\pgfqpoint{1.698456in}{0.708157in}}%
\pgfpathlineto{\pgfqpoint{1.698962in}{0.709172in}}%
\pgfpathlineto{\pgfqpoint{1.699063in}{0.709262in}}%
\pgfpathlineto{\pgfqpoint{1.699467in}{0.708163in}}%
\pgfpathlineto{\pgfqpoint{1.700984in}{0.708071in}}%
\pgfpathlineto{\pgfqpoint{1.701691in}{0.709008in}}%
\pgfpathlineto{\pgfqpoint{1.703309in}{0.720324in}}%
\pgfpathlineto{\pgfqpoint{1.703612in}{0.717629in}}%
\pgfpathlineto{\pgfqpoint{1.704320in}{0.711464in}}%
\pgfpathlineto{\pgfqpoint{1.704825in}{0.715460in}}%
\pgfpathlineto{\pgfqpoint{1.705432in}{0.727522in}}%
\pgfpathlineto{\pgfqpoint{1.705836in}{0.717876in}}%
\pgfpathlineto{\pgfqpoint{1.706038in}{0.713792in}}%
\pgfpathlineto{\pgfqpoint{1.706443in}{0.730118in}}%
\pgfpathlineto{\pgfqpoint{1.706645in}{0.739923in}}%
\pgfpathlineto{\pgfqpoint{1.707251in}{0.708748in}}%
\pgfpathlineto{\pgfqpoint{1.708667in}{0.707926in}}%
\pgfpathlineto{\pgfqpoint{1.712609in}{0.708942in}}%
\pgfpathlineto{\pgfqpoint{1.712812in}{0.709080in}}%
\pgfpathlineto{\pgfqpoint{1.713216in}{0.708254in}}%
\pgfpathlineto{\pgfqpoint{1.713418in}{0.708224in}}%
\pgfpathlineto{\pgfqpoint{1.715642in}{0.709532in}}%
\pgfpathlineto{\pgfqpoint{1.716148in}{0.711713in}}%
\pgfpathlineto{\pgfqpoint{1.716754in}{0.709952in}}%
\pgfpathlineto{\pgfqpoint{1.716956in}{0.710082in}}%
\pgfpathlineto{\pgfqpoint{1.717057in}{0.710526in}}%
\pgfpathlineto{\pgfqpoint{1.717462in}{0.713388in}}%
\pgfpathlineto{\pgfqpoint{1.717967in}{0.709449in}}%
\pgfpathlineto{\pgfqpoint{1.719383in}{0.708517in}}%
\pgfpathlineto{\pgfqpoint{1.722011in}{0.708444in}}%
\pgfpathlineto{\pgfqpoint{1.727066in}{0.708408in}}%
\pgfpathlineto{\pgfqpoint{1.727369in}{0.707803in}}%
\pgfpathlineto{\pgfqpoint{1.727874in}{0.709236in}}%
\pgfpathlineto{\pgfqpoint{1.729189in}{0.716219in}}%
\pgfpathlineto{\pgfqpoint{1.728481in}{0.709098in}}%
\pgfpathlineto{\pgfqpoint{1.729492in}{0.712424in}}%
\pgfpathlineto{\pgfqpoint{1.730806in}{0.707729in}}%
\pgfpathlineto{\pgfqpoint{1.731211in}{0.708705in}}%
\pgfpathlineto{\pgfqpoint{1.733131in}{0.718734in}}%
\pgfpathlineto{\pgfqpoint{1.732019in}{0.708510in}}%
\pgfpathlineto{\pgfqpoint{1.733232in}{0.717903in}}%
\pgfpathlineto{\pgfqpoint{1.733839in}{0.711413in}}%
\pgfpathlineto{\pgfqpoint{1.734142in}{0.716087in}}%
\pgfpathlineto{\pgfqpoint{1.734547in}{0.727792in}}%
\pgfpathlineto{\pgfqpoint{1.735052in}{0.710361in}}%
\pgfpathlineto{\pgfqpoint{1.735254in}{0.708838in}}%
\pgfpathlineto{\pgfqpoint{1.735659in}{0.715346in}}%
\pgfpathlineto{\pgfqpoint{1.735962in}{0.722572in}}%
\pgfpathlineto{\pgfqpoint{1.736568in}{0.709067in}}%
\pgfpathlineto{\pgfqpoint{1.736771in}{0.709429in}}%
\pgfpathlineto{\pgfqpoint{1.736872in}{0.709968in}}%
\pgfpathlineto{\pgfqpoint{1.737984in}{0.713056in}}%
\pgfpathlineto{\pgfqpoint{1.737377in}{0.709696in}}%
\pgfpathlineto{\pgfqpoint{1.738186in}{0.711258in}}%
\pgfpathlineto{\pgfqpoint{1.739500in}{0.707556in}}%
\pgfpathlineto{\pgfqpoint{1.739601in}{0.707584in}}%
\pgfpathlineto{\pgfqpoint{1.740107in}{0.709087in}}%
\pgfpathlineto{\pgfqpoint{1.741724in}{0.719713in}}%
\pgfpathlineto{\pgfqpoint{1.741926in}{0.715186in}}%
\pgfpathlineto{\pgfqpoint{1.743140in}{0.706961in}}%
\pgfpathlineto{\pgfqpoint{1.743241in}{0.706972in}}%
\pgfpathlineto{\pgfqpoint{1.743544in}{0.708601in}}%
\pgfpathlineto{\pgfqpoint{1.744049in}{0.718956in}}%
\pgfpathlineto{\pgfqpoint{1.744555in}{0.708535in}}%
\pgfpathlineto{\pgfqpoint{1.744656in}{0.708232in}}%
\pgfpathlineto{\pgfqpoint{1.744858in}{0.709930in}}%
\pgfpathlineto{\pgfqpoint{1.745262in}{0.717006in}}%
\pgfpathlineto{\pgfqpoint{1.745768in}{0.706596in}}%
\pgfpathlineto{\pgfqpoint{1.746375in}{0.708080in}}%
\pgfpathlineto{\pgfqpoint{1.746981in}{0.706310in}}%
\pgfpathlineto{\pgfqpoint{1.748396in}{0.707386in}}%
\pgfpathlineto{\pgfqpoint{1.749610in}{0.732288in}}%
\pgfpathlineto{\pgfqpoint{1.749913in}{0.717805in}}%
\pgfpathlineto{\pgfqpoint{1.750317in}{0.707526in}}%
\pgfpathlineto{\pgfqpoint{1.751126in}{0.709238in}}%
\pgfpathlineto{\pgfqpoint{1.751530in}{0.710925in}}%
\pgfpathlineto{\pgfqpoint{1.751935in}{0.708677in}}%
\pgfpathlineto{\pgfqpoint{1.752036in}{0.708354in}}%
\pgfpathlineto{\pgfqpoint{1.752238in}{0.709551in}}%
\pgfpathlineto{\pgfqpoint{1.752642in}{0.718500in}}%
\pgfpathlineto{\pgfqpoint{1.753249in}{0.710015in}}%
\pgfpathlineto{\pgfqpoint{1.754563in}{0.709015in}}%
\pgfpathlineto{\pgfqpoint{1.754967in}{0.709267in}}%
\pgfpathlineto{\pgfqpoint{1.755170in}{0.708422in}}%
\pgfpathlineto{\pgfqpoint{1.756383in}{0.706313in}}%
\pgfpathlineto{\pgfqpoint{1.756585in}{0.706370in}}%
\pgfpathlineto{\pgfqpoint{1.760730in}{0.709161in}}%
\pgfpathlineto{\pgfqpoint{1.761033in}{0.710444in}}%
\pgfpathlineto{\pgfqpoint{1.761539in}{0.707913in}}%
\pgfpathlineto{\pgfqpoint{1.761842in}{0.709840in}}%
\pgfpathlineto{\pgfqpoint{1.762954in}{0.712090in}}%
\pgfpathlineto{\pgfqpoint{1.762347in}{0.707578in}}%
\pgfpathlineto{\pgfqpoint{1.763055in}{0.711782in}}%
\pgfpathlineto{\pgfqpoint{1.764369in}{0.707678in}}%
\pgfpathlineto{\pgfqpoint{1.764571in}{0.707721in}}%
\pgfpathlineto{\pgfqpoint{1.764976in}{0.708547in}}%
\pgfpathlineto{\pgfqpoint{1.765279in}{0.710101in}}%
\pgfpathlineto{\pgfqpoint{1.765683in}{0.708085in}}%
\pgfpathlineto{\pgfqpoint{1.766088in}{0.708649in}}%
\pgfpathlineto{\pgfqpoint{1.767705in}{0.711170in}}%
\pgfpathlineto{\pgfqpoint{1.766593in}{0.707995in}}%
\pgfpathlineto{\pgfqpoint{1.768110in}{0.709811in}}%
\pgfpathlineto{\pgfqpoint{1.768211in}{0.709651in}}%
\pgfpathlineto{\pgfqpoint{1.768413in}{0.710480in}}%
\pgfpathlineto{\pgfqpoint{1.768716in}{0.713399in}}%
\pgfpathlineto{\pgfqpoint{1.769323in}{0.708334in}}%
\pgfpathlineto{\pgfqpoint{1.770334in}{0.707379in}}%
\pgfpathlineto{\pgfqpoint{1.770637in}{0.707598in}}%
\pgfpathlineto{\pgfqpoint{1.771951in}{0.708984in}}%
\pgfpathlineto{\pgfqpoint{1.772153in}{0.708327in}}%
\pgfpathlineto{\pgfqpoint{1.773366in}{0.707144in}}%
\pgfpathlineto{\pgfqpoint{1.773468in}{0.707200in}}%
\pgfpathlineto{\pgfqpoint{1.774478in}{0.712455in}}%
\pgfpathlineto{\pgfqpoint{1.775692in}{0.755132in}}%
\pgfpathlineto{\pgfqpoint{1.776096in}{0.810482in}}%
\pgfpathlineto{\pgfqpoint{1.776703in}{0.752222in}}%
\pgfpathlineto{\pgfqpoint{1.777208in}{0.726983in}}%
\pgfpathlineto{\pgfqpoint{1.778017in}{0.728530in}}%
\pgfpathlineto{\pgfqpoint{1.779129in}{0.714142in}}%
\pgfpathlineto{\pgfqpoint{1.779432in}{0.716406in}}%
\pgfpathlineto{\pgfqpoint{1.779836in}{0.721088in}}%
\pgfpathlineto{\pgfqpoint{1.780342in}{0.714529in}}%
\pgfpathlineto{\pgfqpoint{1.780847in}{0.710439in}}%
\pgfpathlineto{\pgfqpoint{1.781656in}{0.710958in}}%
\pgfpathlineto{\pgfqpoint{1.783071in}{0.708794in}}%
\pgfpathlineto{\pgfqpoint{1.789440in}{0.709116in}}%
\pgfpathlineto{\pgfqpoint{1.790552in}{0.708717in}}%
\pgfpathlineto{\pgfqpoint{1.790755in}{0.708844in}}%
\pgfpathlineto{\pgfqpoint{1.792979in}{0.709130in}}%
\pgfpathlineto{\pgfqpoint{1.793080in}{0.708939in}}%
\pgfpathlineto{\pgfqpoint{1.794394in}{0.708416in}}%
\pgfpathlineto{\pgfqpoint{1.796011in}{0.709635in}}%
\pgfpathlineto{\pgfqpoint{1.796112in}{0.709876in}}%
\pgfpathlineto{\pgfqpoint{1.796618in}{0.708698in}}%
\pgfpathlineto{\pgfqpoint{1.796820in}{0.708690in}}%
\pgfpathlineto{\pgfqpoint{1.800358in}{0.710429in}}%
\pgfpathlineto{\pgfqpoint{1.800965in}{0.709225in}}%
\pgfpathlineto{\pgfqpoint{1.801976in}{0.722554in}}%
\pgfpathlineto{\pgfqpoint{1.803492in}{0.808356in}}%
\pgfpathlineto{\pgfqpoint{1.802785in}{0.715013in}}%
\pgfpathlineto{\pgfqpoint{1.803694in}{0.772398in}}%
\pgfpathlineto{\pgfqpoint{1.804200in}{0.715654in}}%
\pgfpathlineto{\pgfqpoint{1.805009in}{0.722722in}}%
\pgfpathlineto{\pgfqpoint{1.806222in}{0.708870in}}%
\pgfpathlineto{\pgfqpoint{1.806323in}{0.708886in}}%
\pgfpathlineto{\pgfqpoint{1.806727in}{0.710503in}}%
\pgfpathlineto{\pgfqpoint{1.807031in}{0.713421in}}%
\pgfpathlineto{\pgfqpoint{1.807637in}{0.709223in}}%
\pgfpathlineto{\pgfqpoint{1.807839in}{0.709706in}}%
\pgfpathlineto{\pgfqpoint{1.808244in}{0.714913in}}%
\pgfpathlineto{\pgfqpoint{1.808850in}{0.709107in}}%
\pgfpathlineto{\pgfqpoint{1.809457in}{0.709417in}}%
\pgfpathlineto{\pgfqpoint{1.810164in}{0.708575in}}%
\pgfpathlineto{\pgfqpoint{1.811883in}{0.709373in}}%
\pgfpathlineto{\pgfqpoint{1.812186in}{0.710103in}}%
\pgfpathlineto{\pgfqpoint{1.812793in}{0.708850in}}%
\pgfpathlineto{\pgfqpoint{1.816028in}{0.708683in}}%
\pgfpathlineto{\pgfqpoint{1.821386in}{0.710715in}}%
\pgfpathlineto{\pgfqpoint{1.821891in}{0.717538in}}%
\pgfpathlineto{\pgfqpoint{1.822599in}{0.712766in}}%
\pgfpathlineto{\pgfqpoint{1.823408in}{0.715534in}}%
\pgfpathlineto{\pgfqpoint{1.823711in}{0.713055in}}%
\pgfpathlineto{\pgfqpoint{1.824318in}{0.709191in}}%
\pgfpathlineto{\pgfqpoint{1.825025in}{0.709637in}}%
\pgfpathlineto{\pgfqpoint{1.825531in}{0.712191in}}%
\pgfpathlineto{\pgfqpoint{1.825935in}{0.708969in}}%
\pgfpathlineto{\pgfqpoint{1.826036in}{0.708775in}}%
\pgfpathlineto{\pgfqpoint{1.826238in}{0.709659in}}%
\pgfpathlineto{\pgfqpoint{1.826643in}{0.716981in}}%
\pgfpathlineto{\pgfqpoint{1.827350in}{0.710054in}}%
\pgfpathlineto{\pgfqpoint{1.827957in}{0.710358in}}%
\pgfpathlineto{\pgfqpoint{1.828968in}{0.708255in}}%
\pgfpathlineto{\pgfqpoint{1.833416in}{0.710079in}}%
\pgfpathlineto{\pgfqpoint{1.834730in}{0.715167in}}%
\pgfpathlineto{\pgfqpoint{1.834831in}{0.715025in}}%
\pgfpathlineto{\pgfqpoint{1.836247in}{0.708365in}}%
\pgfpathlineto{\pgfqpoint{1.836449in}{0.708402in}}%
\pgfpathlineto{\pgfqpoint{1.838268in}{0.710381in}}%
\pgfpathlineto{\pgfqpoint{1.838572in}{0.711760in}}%
\pgfpathlineto{\pgfqpoint{1.838976in}{0.708600in}}%
\pgfpathlineto{\pgfqpoint{1.840088in}{0.708135in}}%
\pgfpathlineto{\pgfqpoint{1.840189in}{0.708176in}}%
\pgfpathlineto{\pgfqpoint{1.841807in}{0.708328in}}%
\pgfpathlineto{\pgfqpoint{1.845547in}{0.709757in}}%
\pgfpathlineto{\pgfqpoint{1.846861in}{0.714980in}}%
\pgfpathlineto{\pgfqpoint{1.847468in}{0.744026in}}%
\pgfpathlineto{\pgfqpoint{1.848479in}{0.731189in}}%
\pgfpathlineto{\pgfqpoint{1.849490in}{0.708716in}}%
\pgfpathlineto{\pgfqpoint{1.849793in}{0.712443in}}%
\pgfpathlineto{\pgfqpoint{1.851208in}{0.737556in}}%
\pgfpathlineto{\pgfqpoint{1.850602in}{0.711318in}}%
\pgfpathlineto{\pgfqpoint{1.851310in}{0.735279in}}%
\pgfpathlineto{\pgfqpoint{1.851916in}{0.710144in}}%
\pgfpathlineto{\pgfqpoint{1.852624in}{0.722723in}}%
\pgfpathlineto{\pgfqpoint{1.853028in}{0.714872in}}%
\pgfpathlineto{\pgfqpoint{1.854342in}{0.709514in}}%
\pgfpathlineto{\pgfqpoint{1.854747in}{0.711937in}}%
\pgfpathlineto{\pgfqpoint{1.854949in}{0.712891in}}%
\pgfpathlineto{\pgfqpoint{1.855555in}{0.709566in}}%
\pgfpathlineto{\pgfqpoint{1.855960in}{0.710705in}}%
\pgfpathlineto{\pgfqpoint{1.856162in}{0.711447in}}%
\pgfpathlineto{\pgfqpoint{1.856566in}{0.708969in}}%
\pgfpathlineto{\pgfqpoint{1.857678in}{0.708145in}}%
\pgfpathlineto{\pgfqpoint{1.857779in}{0.708167in}}%
\pgfpathlineto{\pgfqpoint{1.859195in}{0.709224in}}%
\pgfpathlineto{\pgfqpoint{1.859801in}{0.715583in}}%
\pgfpathlineto{\pgfqpoint{1.860509in}{0.710652in}}%
\pgfpathlineto{\pgfqpoint{1.860711in}{0.710302in}}%
\pgfpathlineto{\pgfqpoint{1.861116in}{0.711704in}}%
\pgfpathlineto{\pgfqpoint{1.861823in}{0.729244in}}%
\pgfpathlineto{\pgfqpoint{1.862430in}{0.714454in}}%
\pgfpathlineto{\pgfqpoint{1.862531in}{0.713979in}}%
\pgfpathlineto{\pgfqpoint{1.862733in}{0.715849in}}%
\pgfpathlineto{\pgfqpoint{1.863137in}{0.722241in}}%
\pgfpathlineto{\pgfqpoint{1.863542in}{0.709743in}}%
\pgfpathlineto{\pgfqpoint{1.864249in}{0.707081in}}%
\pgfpathlineto{\pgfqpoint{1.864755in}{0.707701in}}%
\pgfpathlineto{\pgfqpoint{1.865867in}{0.710834in}}%
\pgfpathlineto{\pgfqpoint{1.866170in}{0.708810in}}%
\pgfpathlineto{\pgfqpoint{1.866777in}{0.706740in}}%
\pgfpathlineto{\pgfqpoint{1.867383in}{0.707535in}}%
\pgfpathlineto{\pgfqpoint{1.867889in}{0.708073in}}%
\pgfpathlineto{\pgfqpoint{1.868495in}{0.707560in}}%
\pgfpathlineto{\pgfqpoint{1.869405in}{0.708373in}}%
\pgfpathlineto{\pgfqpoint{1.869911in}{0.721597in}}%
\pgfpathlineto{\pgfqpoint{1.870012in}{0.723412in}}%
\pgfpathlineto{\pgfqpoint{1.870517in}{0.712722in}}%
\pgfpathlineto{\pgfqpoint{1.870922in}{0.720544in}}%
\pgfpathlineto{\pgfqpoint{1.871629in}{0.706935in}}%
\pgfpathlineto{\pgfqpoint{1.872438in}{0.707385in}}%
\pgfpathlineto{\pgfqpoint{1.872842in}{0.713745in}}%
\pgfpathlineto{\pgfqpoint{1.873045in}{0.716363in}}%
\pgfpathlineto{\pgfqpoint{1.873752in}{0.711025in}}%
\pgfpathlineto{\pgfqpoint{1.874460in}{0.712544in}}%
\pgfpathlineto{\pgfqpoint{1.875066in}{0.720485in}}%
\pgfpathlineto{\pgfqpoint{1.875471in}{0.712998in}}%
\pgfpathlineto{\pgfqpoint{1.876987in}{0.708344in}}%
\pgfpathlineto{\pgfqpoint{1.877493in}{0.707139in}}%
\pgfpathlineto{\pgfqpoint{1.878200in}{0.707909in}}%
\pgfpathlineto{\pgfqpoint{1.878807in}{0.706893in}}%
\pgfpathlineto{\pgfqpoint{1.879110in}{0.708369in}}%
\pgfpathlineto{\pgfqpoint{1.879312in}{0.710255in}}%
\pgfpathlineto{\pgfqpoint{1.879818in}{0.707008in}}%
\pgfpathlineto{\pgfqpoint{1.880121in}{0.707235in}}%
\pgfpathlineto{\pgfqpoint{1.883154in}{0.710171in}}%
\pgfpathlineto{\pgfqpoint{1.883659in}{0.723795in}}%
\pgfpathlineto{\pgfqpoint{1.884165in}{0.709701in}}%
\pgfpathlineto{\pgfqpoint{1.884367in}{0.711423in}}%
\pgfpathlineto{\pgfqpoint{1.884873in}{0.725418in}}%
\pgfpathlineto{\pgfqpoint{1.885479in}{0.713441in}}%
\pgfpathlineto{\pgfqpoint{1.885883in}{0.710952in}}%
\pgfpathlineto{\pgfqpoint{1.886086in}{0.714525in}}%
\pgfpathlineto{\pgfqpoint{1.886692in}{0.748762in}}%
\pgfpathlineto{\pgfqpoint{1.887198in}{0.721843in}}%
\pgfpathlineto{\pgfqpoint{1.887905in}{0.728876in}}%
\pgfpathlineto{\pgfqpoint{1.888613in}{0.707592in}}%
\pgfpathlineto{\pgfqpoint{1.888916in}{0.707716in}}%
\pgfpathlineto{\pgfqpoint{1.889017in}{0.708061in}}%
\pgfpathlineto{\pgfqpoint{1.889422in}{0.716293in}}%
\pgfpathlineto{\pgfqpoint{1.890736in}{0.753843in}}%
\pgfpathlineto{\pgfqpoint{1.890129in}{0.711664in}}%
\pgfpathlineto{\pgfqpoint{1.890938in}{0.739547in}}%
\pgfpathlineto{\pgfqpoint{1.891747in}{0.715039in}}%
\pgfpathlineto{\pgfqpoint{1.892252in}{0.717693in}}%
\pgfpathlineto{\pgfqpoint{1.892353in}{0.717809in}}%
\pgfpathlineto{\pgfqpoint{1.892556in}{0.716766in}}%
\pgfpathlineto{\pgfqpoint{1.893263in}{0.707809in}}%
\pgfpathlineto{\pgfqpoint{1.894274in}{0.708350in}}%
\pgfpathlineto{\pgfqpoint{1.894577in}{0.707953in}}%
\pgfpathlineto{\pgfqpoint{1.895083in}{0.709190in}}%
\pgfpathlineto{\pgfqpoint{1.895184in}{0.709331in}}%
\pgfpathlineto{\pgfqpoint{1.895588in}{0.708167in}}%
\pgfpathlineto{\pgfqpoint{1.897004in}{0.708253in}}%
\pgfpathlineto{\pgfqpoint{1.898823in}{0.708250in}}%
\pgfpathlineto{\pgfqpoint{1.902665in}{0.708712in}}%
\pgfpathlineto{\pgfqpoint{1.903878in}{0.710020in}}%
\pgfpathlineto{\pgfqpoint{1.903979in}{0.709898in}}%
\pgfpathlineto{\pgfqpoint{1.904687in}{0.708416in}}%
\pgfpathlineto{\pgfqpoint{1.905293in}{0.708568in}}%
\pgfpathlineto{\pgfqpoint{1.906102in}{0.709456in}}%
\pgfpathlineto{\pgfqpoint{1.907821in}{0.723919in}}%
\pgfpathlineto{\pgfqpoint{1.908023in}{0.719393in}}%
\pgfpathlineto{\pgfqpoint{1.909337in}{0.708763in}}%
\pgfpathlineto{\pgfqpoint{1.910348in}{0.709192in}}%
\pgfpathlineto{\pgfqpoint{1.910651in}{0.710807in}}%
\pgfpathlineto{\pgfqpoint{1.911359in}{0.708873in}}%
\pgfpathlineto{\pgfqpoint{1.917627in}{0.709676in}}%
\pgfpathlineto{\pgfqpoint{1.918840in}{0.709132in}}%
\pgfpathlineto{\pgfqpoint{1.919345in}{0.709885in}}%
\pgfpathlineto{\pgfqpoint{1.920761in}{0.740447in}}%
\pgfpathlineto{\pgfqpoint{1.921064in}{0.779586in}}%
\pgfpathlineto{\pgfqpoint{1.921671in}{0.716688in}}%
\pgfpathlineto{\pgfqpoint{1.921974in}{0.713000in}}%
\pgfpathlineto{\pgfqpoint{1.922580in}{0.721032in}}%
\pgfpathlineto{\pgfqpoint{1.922783in}{0.724288in}}%
\pgfpathlineto{\pgfqpoint{1.923288in}{0.713538in}}%
\pgfpathlineto{\pgfqpoint{1.923591in}{0.710707in}}%
\pgfpathlineto{\pgfqpoint{1.923996in}{0.718235in}}%
\pgfpathlineto{\pgfqpoint{1.924198in}{0.722517in}}%
\pgfpathlineto{\pgfqpoint{1.924804in}{0.710815in}}%
\pgfpathlineto{\pgfqpoint{1.925411in}{0.711086in}}%
\pgfpathlineto{\pgfqpoint{1.926220in}{0.708946in}}%
\pgfpathlineto{\pgfqpoint{1.929455in}{0.709075in}}%
\pgfpathlineto{\pgfqpoint{1.935217in}{0.709570in}}%
\pgfpathlineto{\pgfqpoint{1.937037in}{0.708959in}}%
\pgfpathlineto{\pgfqpoint{1.941687in}{0.709875in}}%
\pgfpathlineto{\pgfqpoint{1.942091in}{0.711481in}}%
\pgfpathlineto{\pgfqpoint{1.942597in}{0.709352in}}%
\pgfpathlineto{\pgfqpoint{1.943203in}{0.709148in}}%
\pgfpathlineto{\pgfqpoint{1.943406in}{0.709661in}}%
\pgfpathlineto{\pgfqpoint{1.944720in}{0.719748in}}%
\pgfpathlineto{\pgfqpoint{1.945023in}{0.751364in}}%
\pgfpathlineto{\pgfqpoint{1.945731in}{0.711388in}}%
\pgfpathlineto{\pgfqpoint{1.946236in}{0.712490in}}%
\pgfpathlineto{\pgfqpoint{1.947348in}{0.708246in}}%
\pgfpathlineto{\pgfqpoint{1.948460in}{0.709109in}}%
\pgfpathlineto{\pgfqpoint{1.949269in}{0.708533in}}%
\pgfpathlineto{\pgfqpoint{1.949876in}{0.710236in}}%
\pgfpathlineto{\pgfqpoint{1.950078in}{0.709970in}}%
\pgfpathlineto{\pgfqpoint{1.950179in}{0.709492in}}%
\pgfpathlineto{\pgfqpoint{1.950785in}{0.708254in}}%
\pgfpathlineto{\pgfqpoint{1.951392in}{0.708652in}}%
\pgfpathlineto{\pgfqpoint{1.952807in}{0.708272in}}%
\pgfpathlineto{\pgfqpoint{1.954324in}{0.708456in}}%
\pgfpathlineto{\pgfqpoint{1.956143in}{0.708732in}}%
\pgfpathlineto{\pgfqpoint{1.958367in}{0.709749in}}%
\pgfpathlineto{\pgfqpoint{1.958974in}{0.712701in}}%
\pgfpathlineto{\pgfqpoint{1.959581in}{0.710580in}}%
\pgfpathlineto{\pgfqpoint{1.959884in}{0.711799in}}%
\pgfpathlineto{\pgfqpoint{1.960187in}{0.713814in}}%
\pgfpathlineto{\pgfqpoint{1.960592in}{0.709798in}}%
\pgfpathlineto{\pgfqpoint{1.960996in}{0.708554in}}%
\pgfpathlineto{\pgfqpoint{1.961805in}{0.708673in}}%
\pgfpathlineto{\pgfqpoint{1.963220in}{0.709788in}}%
\pgfpathlineto{\pgfqpoint{1.964534in}{0.712957in}}%
\pgfpathlineto{\pgfqpoint{1.964029in}{0.709294in}}%
\pgfpathlineto{\pgfqpoint{1.964635in}{0.712760in}}%
\pgfpathlineto{\pgfqpoint{1.966152in}{0.708151in}}%
\pgfpathlineto{\pgfqpoint{1.967971in}{0.709177in}}%
\pgfpathlineto{\pgfqpoint{1.969286in}{0.724843in}}%
\pgfpathlineto{\pgfqpoint{1.968679in}{0.708660in}}%
\pgfpathlineto{\pgfqpoint{1.969589in}{0.714121in}}%
\pgfpathlineto{\pgfqpoint{1.970802in}{0.709515in}}%
\pgfpathlineto{\pgfqpoint{1.971307in}{0.708806in}}%
\pgfpathlineto{\pgfqpoint{1.972015in}{0.709000in}}%
\pgfpathlineto{\pgfqpoint{1.973127in}{0.710244in}}%
\pgfpathlineto{\pgfqpoint{1.974340in}{0.719824in}}%
\pgfpathlineto{\pgfqpoint{1.973734in}{0.708999in}}%
\pgfpathlineto{\pgfqpoint{1.974542in}{0.714023in}}%
\pgfpathlineto{\pgfqpoint{1.974846in}{0.708487in}}%
\pgfpathlineto{\pgfqpoint{1.975351in}{0.715141in}}%
\pgfpathlineto{\pgfqpoint{1.975756in}{0.708714in}}%
\pgfpathlineto{\pgfqpoint{1.975958in}{0.708290in}}%
\pgfpathlineto{\pgfqpoint{1.976362in}{0.709420in}}%
\pgfpathlineto{\pgfqpoint{1.977676in}{0.732811in}}%
\pgfpathlineto{\pgfqpoint{1.977980in}{0.781944in}}%
\pgfpathlineto{\pgfqpoint{1.978687in}{0.713342in}}%
\pgfpathlineto{\pgfqpoint{1.979092in}{0.740327in}}%
\pgfpathlineto{\pgfqpoint{1.979193in}{0.746904in}}%
\pgfpathlineto{\pgfqpoint{1.979698in}{0.715649in}}%
\pgfpathlineto{\pgfqpoint{1.980001in}{0.710372in}}%
\pgfpathlineto{\pgfqpoint{1.980810in}{0.715401in}}%
\pgfpathlineto{\pgfqpoint{1.981316in}{0.727753in}}%
\pgfpathlineto{\pgfqpoint{1.981821in}{0.714983in}}%
\pgfpathlineto{\pgfqpoint{1.981922in}{0.715403in}}%
\pgfpathlineto{\pgfqpoint{1.982327in}{0.724728in}}%
\pgfpathlineto{\pgfqpoint{1.982832in}{0.713248in}}%
\pgfpathlineto{\pgfqpoint{1.983439in}{0.708954in}}%
\pgfpathlineto{\pgfqpoint{1.984045in}{0.710260in}}%
\pgfpathlineto{\pgfqpoint{1.984854in}{0.708570in}}%
\pgfpathlineto{\pgfqpoint{1.985157in}{0.709873in}}%
\pgfpathlineto{\pgfqpoint{1.985460in}{0.712019in}}%
\pgfpathlineto{\pgfqpoint{1.986168in}{0.709564in}}%
\pgfpathlineto{\pgfqpoint{1.987280in}{0.708552in}}%
\pgfpathlineto{\pgfqpoint{1.987988in}{0.709052in}}%
\pgfpathlineto{\pgfqpoint{1.988493in}{0.708036in}}%
\pgfpathlineto{\pgfqpoint{1.989201in}{0.709538in}}%
\pgfpathlineto{\pgfqpoint{1.989909in}{0.712298in}}%
\pgfpathlineto{\pgfqpoint{1.990717in}{0.710833in}}%
\pgfpathlineto{\pgfqpoint{1.992638in}{0.707725in}}%
\pgfpathlineto{\pgfqpoint{1.995469in}{0.708324in}}%
\pgfpathlineto{\pgfqpoint{1.996075in}{0.717442in}}%
\pgfpathlineto{\pgfqpoint{1.997288in}{0.715111in}}%
\pgfpathlineto{\pgfqpoint{1.998400in}{0.707464in}}%
\pgfpathlineto{\pgfqpoint{1.999007in}{0.707551in}}%
\pgfpathlineto{\pgfqpoint{2.000523in}{0.708019in}}%
\pgfpathlineto{\pgfqpoint{2.000827in}{0.715907in}}%
\pgfpathlineto{\pgfqpoint{2.001130in}{0.729642in}}%
\pgfpathlineto{\pgfqpoint{2.001635in}{0.713708in}}%
\pgfpathlineto{\pgfqpoint{2.001939in}{0.718157in}}%
\pgfpathlineto{\pgfqpoint{2.002141in}{0.720195in}}%
\pgfpathlineto{\pgfqpoint{2.002646in}{0.712078in}}%
\pgfpathlineto{\pgfqpoint{2.003253in}{0.707876in}}%
\pgfpathlineto{\pgfqpoint{2.003859in}{0.708347in}}%
\pgfpathlineto{\pgfqpoint{2.005477in}{0.720049in}}%
\pgfpathlineto{\pgfqpoint{2.005881in}{0.714999in}}%
\pgfpathlineto{\pgfqpoint{2.007903in}{0.708062in}}%
\pgfpathlineto{\pgfqpoint{2.008207in}{0.707769in}}%
\pgfpathlineto{\pgfqpoint{2.008712in}{0.708900in}}%
\pgfpathlineto{\pgfqpoint{2.009217in}{0.707901in}}%
\pgfpathlineto{\pgfqpoint{2.009723in}{0.707885in}}%
\pgfpathlineto{\pgfqpoint{2.010127in}{0.708399in}}%
\pgfpathlineto{\pgfqpoint{2.010431in}{0.708555in}}%
\pgfpathlineto{\pgfqpoint{2.010936in}{0.707665in}}%
\pgfpathlineto{\pgfqpoint{2.013160in}{0.708251in}}%
\pgfpathlineto{\pgfqpoint{2.013362in}{0.708484in}}%
\pgfpathlineto{\pgfqpoint{2.014171in}{0.707834in}}%
\pgfpathlineto{\pgfqpoint{2.016193in}{0.708974in}}%
\pgfpathlineto{\pgfqpoint{2.016901in}{0.710242in}}%
\pgfpathlineto{\pgfqpoint{2.017406in}{0.717133in}}%
\pgfpathlineto{\pgfqpoint{2.018013in}{0.711404in}}%
\pgfpathlineto{\pgfqpoint{2.018215in}{0.710574in}}%
\pgfpathlineto{\pgfqpoint{2.018417in}{0.712706in}}%
\pgfpathlineto{\pgfqpoint{2.018821in}{0.724986in}}%
\pgfpathlineto{\pgfqpoint{2.019327in}{0.709503in}}%
\pgfpathlineto{\pgfqpoint{2.020540in}{0.707746in}}%
\pgfpathlineto{\pgfqpoint{2.021955in}{0.708925in}}%
\pgfpathlineto{\pgfqpoint{2.022056in}{0.709148in}}%
\pgfpathlineto{\pgfqpoint{2.022663in}{0.707999in}}%
\pgfpathlineto{\pgfqpoint{2.023371in}{0.708662in}}%
\pgfpathlineto{\pgfqpoint{2.024381in}{0.714968in}}%
\pgfpathlineto{\pgfqpoint{2.024786in}{0.726892in}}%
\pgfpathlineto{\pgfqpoint{2.025493in}{0.718335in}}%
\pgfpathlineto{\pgfqpoint{2.025797in}{0.716738in}}%
\pgfpathlineto{\pgfqpoint{2.025999in}{0.718646in}}%
\pgfpathlineto{\pgfqpoint{2.026403in}{0.735751in}}%
\pgfpathlineto{\pgfqpoint{2.026909in}{0.714774in}}%
\pgfpathlineto{\pgfqpoint{2.027212in}{0.708784in}}%
\pgfpathlineto{\pgfqpoint{2.027819in}{0.716262in}}%
\pgfpathlineto{\pgfqpoint{2.028021in}{0.714224in}}%
\pgfpathlineto{\pgfqpoint{2.028526in}{0.708541in}}%
\pgfpathlineto{\pgfqpoint{2.029335in}{0.708931in}}%
\pgfpathlineto{\pgfqpoint{2.029841in}{0.708311in}}%
\pgfpathlineto{\pgfqpoint{2.030346in}{0.709379in}}%
\pgfpathlineto{\pgfqpoint{2.031458in}{0.711383in}}%
\pgfpathlineto{\pgfqpoint{2.031559in}{0.711077in}}%
\pgfpathlineto{\pgfqpoint{2.032267in}{0.708097in}}%
\pgfpathlineto{\pgfqpoint{2.032873in}{0.709594in}}%
\pgfpathlineto{\pgfqpoint{2.033076in}{0.710004in}}%
\pgfpathlineto{\pgfqpoint{2.033985in}{0.709493in}}%
\pgfpathlineto{\pgfqpoint{2.034289in}{0.710330in}}%
\pgfpathlineto{\pgfqpoint{2.034693in}{0.715399in}}%
\pgfpathlineto{\pgfqpoint{2.035097in}{0.708931in}}%
\pgfpathlineto{\pgfqpoint{2.036209in}{0.707060in}}%
\pgfpathlineto{\pgfqpoint{2.036310in}{0.707095in}}%
\pgfpathlineto{\pgfqpoint{2.040354in}{0.710124in}}%
\pgfpathlineto{\pgfqpoint{2.040658in}{0.709036in}}%
\pgfpathlineto{\pgfqpoint{2.040759in}{0.708849in}}%
\pgfpathlineto{\pgfqpoint{2.041163in}{0.710318in}}%
\pgfpathlineto{\pgfqpoint{2.041264in}{0.710472in}}%
\pgfpathlineto{\pgfqpoint{2.041466in}{0.709502in}}%
\pgfpathlineto{\pgfqpoint{2.042073in}{0.708007in}}%
\pgfpathlineto{\pgfqpoint{2.042780in}{0.708191in}}%
\pgfpathlineto{\pgfqpoint{2.047026in}{0.707818in}}%
\pgfpathlineto{\pgfqpoint{2.048543in}{0.708095in}}%
\pgfpathlineto{\pgfqpoint{2.049554in}{0.710839in}}%
\pgfpathlineto{\pgfqpoint{2.049958in}{0.729342in}}%
\pgfpathlineto{\pgfqpoint{2.050666in}{0.713255in}}%
\pgfpathlineto{\pgfqpoint{2.051475in}{0.714973in}}%
\pgfpathlineto{\pgfqpoint{2.051778in}{0.741094in}}%
\pgfpathlineto{\pgfqpoint{2.051980in}{0.773942in}}%
\pgfpathlineto{\pgfqpoint{2.052688in}{0.715343in}}%
\pgfpathlineto{\pgfqpoint{2.053496in}{0.714049in}}%
\pgfpathlineto{\pgfqpoint{2.053699in}{0.715438in}}%
\pgfpathlineto{\pgfqpoint{2.054002in}{0.719889in}}%
\pgfpathlineto{\pgfqpoint{2.054507in}{0.710843in}}%
\pgfpathlineto{\pgfqpoint{2.055720in}{0.708882in}}%
\pgfpathlineto{\pgfqpoint{2.056024in}{0.708934in}}%
\pgfpathlineto{\pgfqpoint{2.056226in}{0.709696in}}%
\pgfpathlineto{\pgfqpoint{2.056428in}{0.710534in}}%
\pgfpathlineto{\pgfqpoint{2.056934in}{0.709067in}}%
\pgfpathlineto{\pgfqpoint{2.057439in}{0.710206in}}%
\pgfpathlineto{\pgfqpoint{2.057540in}{0.710276in}}%
\pgfpathlineto{\pgfqpoint{2.057742in}{0.709519in}}%
\pgfpathlineto{\pgfqpoint{2.058955in}{0.708891in}}%
\pgfpathlineto{\pgfqpoint{2.059360in}{0.708768in}}%
\pgfpathlineto{\pgfqpoint{2.059764in}{0.709612in}}%
\pgfpathlineto{\pgfqpoint{2.060270in}{0.708630in}}%
\pgfpathlineto{\pgfqpoint{2.060573in}{0.708883in}}%
\pgfpathlineto{\pgfqpoint{2.060674in}{0.709448in}}%
\pgfpathlineto{\pgfqpoint{2.061988in}{0.713408in}}%
\pgfpathlineto{\pgfqpoint{2.061382in}{0.708712in}}%
\pgfpathlineto{\pgfqpoint{2.062190in}{0.712770in}}%
\pgfpathlineto{\pgfqpoint{2.062898in}{0.710252in}}%
\pgfpathlineto{\pgfqpoint{2.063302in}{0.712242in}}%
\pgfpathlineto{\pgfqpoint{2.063808in}{0.723498in}}%
\pgfpathlineto{\pgfqpoint{2.064313in}{0.711871in}}%
\pgfpathlineto{\pgfqpoint{2.064920in}{0.710736in}}%
\pgfpathlineto{\pgfqpoint{2.065223in}{0.711979in}}%
\pgfpathlineto{\pgfqpoint{2.065425in}{0.713229in}}%
\pgfpathlineto{\pgfqpoint{2.065830in}{0.709587in}}%
\pgfpathlineto{\pgfqpoint{2.066234in}{0.708044in}}%
\pgfpathlineto{\pgfqpoint{2.067043in}{0.708135in}}%
\pgfpathlineto{\pgfqpoint{2.069166in}{0.709293in}}%
\pgfpathlineto{\pgfqpoint{2.069267in}{0.709493in}}%
\pgfpathlineto{\pgfqpoint{2.069671in}{0.707959in}}%
\pgfpathlineto{\pgfqpoint{2.070884in}{0.708678in}}%
\pgfpathlineto{\pgfqpoint{2.071794in}{0.710801in}}%
\pgfpathlineto{\pgfqpoint{2.073412in}{0.725435in}}%
\pgfpathlineto{\pgfqpoint{2.073614in}{0.720021in}}%
\pgfpathlineto{\pgfqpoint{2.074827in}{0.707934in}}%
\pgfpathlineto{\pgfqpoint{2.074928in}{0.707973in}}%
\pgfpathlineto{\pgfqpoint{2.076445in}{0.708119in}}%
\pgfpathlineto{\pgfqpoint{2.078568in}{0.709603in}}%
\pgfpathlineto{\pgfqpoint{2.080084in}{0.726204in}}%
\pgfpathlineto{\pgfqpoint{2.080286in}{0.720950in}}%
\pgfpathlineto{\pgfqpoint{2.080792in}{0.709068in}}%
\pgfpathlineto{\pgfqpoint{2.081499in}{0.713106in}}%
\pgfpathlineto{\pgfqpoint{2.081803in}{0.716473in}}%
\pgfpathlineto{\pgfqpoint{2.082409in}{0.710059in}}%
\pgfpathlineto{\pgfqpoint{2.082510in}{0.709973in}}%
\pgfpathlineto{\pgfqpoint{2.082813in}{0.710985in}}%
\pgfpathlineto{\pgfqpoint{2.083521in}{0.720118in}}%
\pgfpathlineto{\pgfqpoint{2.083926in}{0.712066in}}%
\pgfpathlineto{\pgfqpoint{2.084734in}{0.713971in}}%
\pgfpathlineto{\pgfqpoint{2.085240in}{0.709763in}}%
\pgfpathlineto{\pgfqpoint{2.086150in}{0.708677in}}%
\pgfpathlineto{\pgfqpoint{2.086453in}{0.709169in}}%
\pgfpathlineto{\pgfqpoint{2.086958in}{0.715763in}}%
\pgfpathlineto{\pgfqpoint{2.087464in}{0.709113in}}%
\pgfpathlineto{\pgfqpoint{2.087565in}{0.708793in}}%
\pgfpathlineto{\pgfqpoint{2.087767in}{0.710281in}}%
\pgfpathlineto{\pgfqpoint{2.087969in}{0.713390in}}%
\pgfpathlineto{\pgfqpoint{2.088576in}{0.707587in}}%
\pgfpathlineto{\pgfqpoint{2.088677in}{0.707607in}}%
\pgfpathlineto{\pgfqpoint{2.092620in}{0.708350in}}%
\pgfpathlineto{\pgfqpoint{2.093226in}{0.708500in}}%
\pgfpathlineto{\pgfqpoint{2.093428in}{0.709020in}}%
\pgfpathlineto{\pgfqpoint{2.094338in}{0.714816in}}%
\pgfpathlineto{\pgfqpoint{2.094742in}{0.729169in}}%
\pgfpathlineto{\pgfqpoint{2.095349in}{0.710863in}}%
\pgfpathlineto{\pgfqpoint{2.095450in}{0.710416in}}%
\pgfpathlineto{\pgfqpoint{2.095753in}{0.713032in}}%
\pgfpathlineto{\pgfqpoint{2.096360in}{0.730341in}}%
\pgfpathlineto{\pgfqpoint{2.097169in}{0.724155in}}%
\pgfpathlineto{\pgfqpoint{2.097371in}{0.721406in}}%
\pgfpathlineto{\pgfqpoint{2.097573in}{0.734091in}}%
\pgfpathlineto{\pgfqpoint{2.097876in}{0.773888in}}%
\pgfpathlineto{\pgfqpoint{2.098483in}{0.714886in}}%
\pgfpathlineto{\pgfqpoint{2.099797in}{0.710663in}}%
\pgfpathlineto{\pgfqpoint{2.100808in}{0.709109in}}%
\pgfpathlineto{\pgfqpoint{2.102122in}{0.708559in}}%
\pgfpathlineto{\pgfqpoint{2.103740in}{0.709588in}}%
\pgfpathlineto{\pgfqpoint{2.104043in}{0.712056in}}%
\pgfpathlineto{\pgfqpoint{2.104549in}{0.708670in}}%
\pgfpathlineto{\pgfqpoint{2.104751in}{0.708699in}}%
\pgfpathlineto{\pgfqpoint{2.106570in}{0.709866in}}%
\pgfpathlineto{\pgfqpoint{2.107076in}{0.711676in}}%
\pgfpathlineto{\pgfqpoint{2.107682in}{0.710240in}}%
\pgfpathlineto{\pgfqpoint{2.110311in}{0.708528in}}%
\pgfpathlineto{\pgfqpoint{2.110715in}{0.713906in}}%
\pgfpathlineto{\pgfqpoint{2.111726in}{0.715589in}}%
\pgfpathlineto{\pgfqpoint{2.111221in}{0.711134in}}%
\pgfpathlineto{\pgfqpoint{2.111928in}{0.714996in}}%
\pgfpathlineto{\pgfqpoint{2.113647in}{0.710965in}}%
\pgfpathlineto{\pgfqpoint{2.113849in}{0.710743in}}%
\pgfpathlineto{\pgfqpoint{2.114051in}{0.711781in}}%
\pgfpathlineto{\pgfqpoint{2.114557in}{0.719534in}}%
\pgfpathlineto{\pgfqpoint{2.115062in}{0.710376in}}%
\pgfpathlineto{\pgfqpoint{2.116376in}{0.707408in}}%
\pgfpathlineto{\pgfqpoint{2.115568in}{0.711253in}}%
\pgfpathlineto{\pgfqpoint{2.116882in}{0.708266in}}%
\pgfpathlineto{\pgfqpoint{2.118095in}{0.708466in}}%
\pgfpathlineto{\pgfqpoint{2.118196in}{0.708276in}}%
\pgfpathlineto{\pgfqpoint{2.118803in}{0.706681in}}%
\pgfpathlineto{\pgfqpoint{2.119106in}{0.708073in}}%
\pgfpathlineto{\pgfqpoint{2.120521in}{0.717074in}}%
\pgfpathlineto{\pgfqpoint{2.119915in}{0.707488in}}%
\pgfpathlineto{\pgfqpoint{2.120724in}{0.714937in}}%
\pgfpathlineto{\pgfqpoint{2.122038in}{0.707100in}}%
\pgfpathlineto{\pgfqpoint{2.122745in}{0.707222in}}%
\pgfpathlineto{\pgfqpoint{2.122846in}{0.707474in}}%
\pgfpathlineto{\pgfqpoint{2.124464in}{0.711479in}}%
\pgfpathlineto{\pgfqpoint{2.125071in}{0.707059in}}%
\pgfpathlineto{\pgfqpoint{2.125980in}{0.707250in}}%
\pgfpathlineto{\pgfqpoint{2.126991in}{0.708101in}}%
\pgfpathlineto{\pgfqpoint{2.128609in}{0.719204in}}%
\pgfpathlineto{\pgfqpoint{2.128811in}{0.715768in}}%
\pgfpathlineto{\pgfqpoint{2.129215in}{0.707282in}}%
\pgfpathlineto{\pgfqpoint{2.130125in}{0.707348in}}%
\pgfpathlineto{\pgfqpoint{2.132349in}{0.708254in}}%
\pgfpathlineto{\pgfqpoint{2.132754in}{0.709776in}}%
\pgfpathlineto{\pgfqpoint{2.133360in}{0.708047in}}%
\pgfpathlineto{\pgfqpoint{2.137505in}{0.708335in}}%
\pgfpathlineto{\pgfqpoint{2.138415in}{0.709579in}}%
\pgfpathlineto{\pgfqpoint{2.138718in}{0.715161in}}%
\pgfpathlineto{\pgfqpoint{2.139224in}{0.709225in}}%
\pgfpathlineto{\pgfqpoint{2.139527in}{0.710283in}}%
\pgfpathlineto{\pgfqpoint{2.139628in}{0.710694in}}%
\pgfpathlineto{\pgfqpoint{2.140133in}{0.708468in}}%
\pgfpathlineto{\pgfqpoint{2.141751in}{0.708863in}}%
\pgfpathlineto{\pgfqpoint{2.142358in}{0.712423in}}%
\pgfpathlineto{\pgfqpoint{2.143571in}{0.711020in}}%
\pgfpathlineto{\pgfqpoint{2.144076in}{0.711197in}}%
\pgfpathlineto{\pgfqpoint{2.144177in}{0.711618in}}%
\pgfpathlineto{\pgfqpoint{2.144582in}{0.723210in}}%
\pgfpathlineto{\pgfqpoint{2.144986in}{0.752874in}}%
\pgfpathlineto{\pgfqpoint{2.145694in}{0.723336in}}%
\pgfpathlineto{\pgfqpoint{2.145997in}{0.733175in}}%
\pgfpathlineto{\pgfqpoint{2.146300in}{0.747139in}}%
\pgfpathlineto{\pgfqpoint{2.146806in}{0.722190in}}%
\pgfpathlineto{\pgfqpoint{2.148120in}{0.710577in}}%
\pgfpathlineto{\pgfqpoint{2.148726in}{0.709017in}}%
\pgfpathlineto{\pgfqpoint{2.149333in}{0.709878in}}%
\pgfpathlineto{\pgfqpoint{2.149737in}{0.714037in}}%
\pgfpathlineto{\pgfqpoint{2.150445in}{0.710090in}}%
\pgfpathlineto{\pgfqpoint{2.150647in}{0.710620in}}%
\pgfpathlineto{\pgfqpoint{2.151153in}{0.708873in}}%
\pgfpathlineto{\pgfqpoint{2.154691in}{0.709033in}}%
\pgfpathlineto{\pgfqpoint{2.154994in}{0.710889in}}%
\pgfpathlineto{\pgfqpoint{2.155500in}{0.723728in}}%
\pgfpathlineto{\pgfqpoint{2.156308in}{0.716958in}}%
\pgfpathlineto{\pgfqpoint{2.156409in}{0.717499in}}%
\pgfpathlineto{\pgfqpoint{2.156713in}{0.714319in}}%
\pgfpathlineto{\pgfqpoint{2.157420in}{0.709114in}}%
\pgfpathlineto{\pgfqpoint{2.158027in}{0.709732in}}%
\pgfpathlineto{\pgfqpoint{2.158937in}{0.710929in}}%
\pgfpathlineto{\pgfqpoint{2.159038in}{0.711419in}}%
\pgfpathlineto{\pgfqpoint{2.159543in}{0.708448in}}%
\pgfpathlineto{\pgfqpoint{2.161565in}{0.708720in}}%
\pgfpathlineto{\pgfqpoint{2.161970in}{0.709313in}}%
\pgfpathlineto{\pgfqpoint{2.162778in}{0.708797in}}%
\pgfpathlineto{\pgfqpoint{2.167024in}{0.709162in}}%
\pgfpathlineto{\pgfqpoint{2.168642in}{0.736541in}}%
\pgfpathlineto{\pgfqpoint{2.168945in}{0.718652in}}%
\pgfpathlineto{\pgfqpoint{2.170158in}{0.708946in}}%
\pgfpathlineto{\pgfqpoint{2.174606in}{0.709006in}}%
\pgfpathlineto{\pgfqpoint{2.177033in}{0.709947in}}%
\pgfpathlineto{\pgfqpoint{2.177437in}{0.710457in}}%
\pgfpathlineto{\pgfqpoint{2.177841in}{0.709209in}}%
\pgfpathlineto{\pgfqpoint{2.178852in}{0.709175in}}%
\pgfpathlineto{\pgfqpoint{2.178953in}{0.709332in}}%
\pgfpathlineto{\pgfqpoint{2.180166in}{0.717811in}}%
\pgfpathlineto{\pgfqpoint{2.180470in}{0.738312in}}%
\pgfpathlineto{\pgfqpoint{2.181177in}{0.710761in}}%
\pgfpathlineto{\pgfqpoint{2.184817in}{0.709212in}}%
\pgfpathlineto{\pgfqpoint{2.185019in}{0.710195in}}%
\pgfpathlineto{\pgfqpoint{2.185322in}{0.711710in}}%
\pgfpathlineto{\pgfqpoint{2.185828in}{0.709565in}}%
\pgfpathlineto{\pgfqpoint{2.186030in}{0.709739in}}%
\pgfpathlineto{\pgfqpoint{2.188153in}{0.710972in}}%
\pgfpathlineto{\pgfqpoint{2.188254in}{0.711436in}}%
\pgfpathlineto{\pgfqpoint{2.188759in}{0.709092in}}%
\pgfpathlineto{\pgfqpoint{2.188860in}{0.709077in}}%
\pgfpathlineto{\pgfqpoint{2.189366in}{0.710035in}}%
\pgfpathlineto{\pgfqpoint{2.190276in}{0.709043in}}%
\pgfpathlineto{\pgfqpoint{2.191388in}{0.709150in}}%
\pgfpathlineto{\pgfqpoint{2.191489in}{0.709794in}}%
\pgfpathlineto{\pgfqpoint{2.191792in}{0.711431in}}%
\pgfpathlineto{\pgfqpoint{2.192702in}{0.710785in}}%
\pgfpathlineto{\pgfqpoint{2.194117in}{0.708764in}}%
\pgfpathlineto{\pgfqpoint{2.194320in}{0.708948in}}%
\pgfpathlineto{\pgfqpoint{2.195634in}{0.713384in}}%
\pgfpathlineto{\pgfqpoint{2.195836in}{0.710408in}}%
\pgfpathlineto{\pgfqpoint{2.196139in}{0.708438in}}%
\pgfpathlineto{\pgfqpoint{2.196544in}{0.714593in}}%
\pgfpathlineto{\pgfqpoint{2.196847in}{0.723457in}}%
\pgfpathlineto{\pgfqpoint{2.197352in}{0.713323in}}%
\pgfpathlineto{\pgfqpoint{2.197757in}{0.721414in}}%
\pgfpathlineto{\pgfqpoint{2.197959in}{0.723889in}}%
\pgfpathlineto{\pgfqpoint{2.198464in}{0.713965in}}%
\pgfpathlineto{\pgfqpoint{2.198768in}{0.712471in}}%
\pgfpathlineto{\pgfqpoint{2.199374in}{0.714097in}}%
\pgfpathlineto{\pgfqpoint{2.199576in}{0.713901in}}%
\pgfpathlineto{\pgfqpoint{2.200082in}{0.714667in}}%
\pgfpathlineto{\pgfqpoint{2.200486in}{0.710412in}}%
\pgfpathlineto{\pgfqpoint{2.201093in}{0.707597in}}%
\pgfpathlineto{\pgfqpoint{2.201699in}{0.707893in}}%
\pgfpathlineto{\pgfqpoint{2.202811in}{0.710168in}}%
\pgfpathlineto{\pgfqpoint{2.203216in}{0.752291in}}%
\pgfpathlineto{\pgfqpoint{2.203519in}{0.778779in}}%
\pgfpathlineto{\pgfqpoint{2.204126in}{0.731677in}}%
\pgfpathlineto{\pgfqpoint{2.205743in}{0.713896in}}%
\pgfpathlineto{\pgfqpoint{2.206451in}{0.711077in}}%
\pgfpathlineto{\pgfqpoint{2.206653in}{0.714076in}}%
\pgfpathlineto{\pgfqpoint{2.207057in}{0.730711in}}%
\pgfpathlineto{\pgfqpoint{2.207563in}{0.709385in}}%
\pgfpathlineto{\pgfqpoint{2.207664in}{0.709039in}}%
\pgfpathlineto{\pgfqpoint{2.207866in}{0.710684in}}%
\pgfpathlineto{\pgfqpoint{2.208068in}{0.713424in}}%
\pgfpathlineto{\pgfqpoint{2.208776in}{0.708423in}}%
\pgfpathlineto{\pgfqpoint{2.210494in}{0.709526in}}%
\pgfpathlineto{\pgfqpoint{2.210798in}{0.710359in}}%
\pgfpathlineto{\pgfqpoint{2.211303in}{0.708634in}}%
\pgfpathlineto{\pgfqpoint{2.212719in}{0.708332in}}%
\pgfpathlineto{\pgfqpoint{2.215145in}{0.709664in}}%
\pgfpathlineto{\pgfqpoint{2.215448in}{0.712224in}}%
\pgfpathlineto{\pgfqpoint{2.216257in}{0.710208in}}%
\pgfpathlineto{\pgfqpoint{2.216863in}{0.708034in}}%
\pgfpathlineto{\pgfqpoint{2.217369in}{0.709093in}}%
\pgfpathlineto{\pgfqpoint{2.217672in}{0.711372in}}%
\pgfpathlineto{\pgfqpoint{2.218178in}{0.708580in}}%
\pgfpathlineto{\pgfqpoint{2.218481in}{0.709171in}}%
\pgfpathlineto{\pgfqpoint{2.218582in}{0.709298in}}%
\pgfpathlineto{\pgfqpoint{2.218986in}{0.708303in}}%
\pgfpathlineto{\pgfqpoint{2.219087in}{0.708234in}}%
\pgfpathlineto{\pgfqpoint{2.220301in}{0.708961in}}%
\pgfpathlineto{\pgfqpoint{2.220604in}{0.709707in}}%
\pgfpathlineto{\pgfqpoint{2.221413in}{0.708887in}}%
\pgfpathlineto{\pgfqpoint{2.222322in}{0.708731in}}%
\pgfpathlineto{\pgfqpoint{2.222626in}{0.709435in}}%
\pgfpathlineto{\pgfqpoint{2.224142in}{0.710530in}}%
\pgfpathlineto{\pgfqpoint{2.225254in}{0.709310in}}%
\pgfpathlineto{\pgfqpoint{2.225861in}{0.707839in}}%
\pgfpathlineto{\pgfqpoint{2.226568in}{0.708069in}}%
\pgfpathlineto{\pgfqpoint{2.227276in}{0.708493in}}%
\pgfpathlineto{\pgfqpoint{2.227478in}{0.707404in}}%
\pgfpathlineto{\pgfqpoint{2.227680in}{0.706839in}}%
\pgfpathlineto{\pgfqpoint{2.227984in}{0.708069in}}%
\pgfpathlineto{\pgfqpoint{2.229298in}{0.732769in}}%
\pgfpathlineto{\pgfqpoint{2.229500in}{0.725808in}}%
\pgfpathlineto{\pgfqpoint{2.230309in}{0.707521in}}%
\pgfpathlineto{\pgfqpoint{2.230814in}{0.708893in}}%
\pgfpathlineto{\pgfqpoint{2.231016in}{0.708385in}}%
\pgfpathlineto{\pgfqpoint{2.231320in}{0.706960in}}%
\pgfpathlineto{\pgfqpoint{2.232027in}{0.708980in}}%
\pgfpathlineto{\pgfqpoint{2.232937in}{0.708055in}}%
\pgfpathlineto{\pgfqpoint{2.233139in}{0.708408in}}%
\pgfpathlineto{\pgfqpoint{2.233443in}{0.714665in}}%
\pgfpathlineto{\pgfqpoint{2.233847in}{0.746283in}}%
\pgfpathlineto{\pgfqpoint{2.234555in}{0.717978in}}%
\pgfpathlineto{\pgfqpoint{2.236475in}{0.708300in}}%
\pgfpathlineto{\pgfqpoint{2.234959in}{0.718119in}}%
\pgfpathlineto{\pgfqpoint{2.236779in}{0.712869in}}%
\pgfpathlineto{\pgfqpoint{2.236981in}{0.715380in}}%
\pgfpathlineto{\pgfqpoint{2.237588in}{0.707122in}}%
\pgfpathlineto{\pgfqpoint{2.237891in}{0.707743in}}%
\pgfpathlineto{\pgfqpoint{2.238295in}{0.722404in}}%
\pgfpathlineto{\pgfqpoint{2.239104in}{0.714979in}}%
\pgfpathlineto{\pgfqpoint{2.239609in}{0.741242in}}%
\pgfpathlineto{\pgfqpoint{2.239710in}{0.742528in}}%
\pgfpathlineto{\pgfqpoint{2.239913in}{0.732206in}}%
\pgfpathlineto{\pgfqpoint{2.241328in}{0.707983in}}%
\pgfpathlineto{\pgfqpoint{2.241732in}{0.707315in}}%
\pgfpathlineto{\pgfqpoint{2.242440in}{0.707640in}}%
\pgfpathlineto{\pgfqpoint{2.244260in}{0.708756in}}%
\pgfpathlineto{\pgfqpoint{2.244462in}{0.708557in}}%
\pgfpathlineto{\pgfqpoint{2.246079in}{0.707994in}}%
\pgfpathlineto{\pgfqpoint{2.248607in}{0.708979in}}%
\pgfpathlineto{\pgfqpoint{2.250123in}{0.713133in}}%
\pgfpathlineto{\pgfqpoint{2.250325in}{0.712700in}}%
\pgfpathlineto{\pgfqpoint{2.250426in}{0.712561in}}%
\pgfpathlineto{\pgfqpoint{2.250527in}{0.713053in}}%
\pgfpathlineto{\pgfqpoint{2.251134in}{0.730605in}}%
\pgfpathlineto{\pgfqpoint{2.251741in}{0.715621in}}%
\pgfpathlineto{\pgfqpoint{2.252246in}{0.712511in}}%
\pgfpathlineto{\pgfqpoint{2.252752in}{0.715931in}}%
\pgfpathlineto{\pgfqpoint{2.252853in}{0.715965in}}%
\pgfpathlineto{\pgfqpoint{2.253560in}{0.707604in}}%
\pgfpathlineto{\pgfqpoint{2.254571in}{0.707959in}}%
\pgfpathlineto{\pgfqpoint{2.257402in}{0.709734in}}%
\pgfpathlineto{\pgfqpoint{2.259019in}{0.715939in}}%
\pgfpathlineto{\pgfqpoint{2.259222in}{0.714633in}}%
\pgfpathlineto{\pgfqpoint{2.260536in}{0.709605in}}%
\pgfpathlineto{\pgfqpoint{2.260637in}{0.709715in}}%
\pgfpathlineto{\pgfqpoint{2.261041in}{0.711312in}}%
\pgfpathlineto{\pgfqpoint{2.261344in}{0.709105in}}%
\pgfpathlineto{\pgfqpoint{2.262254in}{0.708329in}}%
\pgfpathlineto{\pgfqpoint{2.262457in}{0.708496in}}%
\pgfpathlineto{\pgfqpoint{2.262760in}{0.708903in}}%
\pgfpathlineto{\pgfqpoint{2.263569in}{0.708509in}}%
\pgfpathlineto{\pgfqpoint{2.264984in}{0.708850in}}%
\pgfpathlineto{\pgfqpoint{2.265388in}{0.711774in}}%
\pgfpathlineto{\pgfqpoint{2.265793in}{0.717012in}}%
\pgfpathlineto{\pgfqpoint{2.266399in}{0.711200in}}%
\pgfpathlineto{\pgfqpoint{2.268825in}{0.708193in}}%
\pgfpathlineto{\pgfqpoint{2.269028in}{0.709481in}}%
\pgfpathlineto{\pgfqpoint{2.269533in}{0.725369in}}%
\pgfpathlineto{\pgfqpoint{2.270140in}{0.710348in}}%
\pgfpathlineto{\pgfqpoint{2.270645in}{0.718280in}}%
\pgfpathlineto{\pgfqpoint{2.271151in}{0.710703in}}%
\pgfpathlineto{\pgfqpoint{2.272465in}{0.708525in}}%
\pgfpathlineto{\pgfqpoint{2.279642in}{0.708439in}}%
\pgfpathlineto{\pgfqpoint{2.282069in}{0.708759in}}%
\pgfpathlineto{\pgfqpoint{2.284495in}{0.709503in}}%
\pgfpathlineto{\pgfqpoint{2.286112in}{0.709183in}}%
\pgfpathlineto{\pgfqpoint{2.287528in}{0.709137in}}%
\pgfpathlineto{\pgfqpoint{2.289752in}{0.709219in}}%
\pgfpathlineto{\pgfqpoint{2.290864in}{0.709506in}}%
\pgfpathlineto{\pgfqpoint{2.291673in}{0.709097in}}%
\pgfpathlineto{\pgfqpoint{2.292077in}{0.724378in}}%
\pgfpathlineto{\pgfqpoint{2.292279in}{0.734952in}}%
\pgfpathlineto{\pgfqpoint{2.292987in}{0.711713in}}%
\pgfpathlineto{\pgfqpoint{2.293492in}{0.725201in}}%
\pgfpathlineto{\pgfqpoint{2.293795in}{0.712032in}}%
\pgfpathlineto{\pgfqpoint{2.294099in}{0.708294in}}%
\pgfpathlineto{\pgfqpoint{2.295009in}{0.708382in}}%
\pgfpathlineto{\pgfqpoint{2.296323in}{0.709269in}}%
\pgfpathlineto{\pgfqpoint{2.296626in}{0.728077in}}%
\pgfpathlineto{\pgfqpoint{2.296929in}{0.759487in}}%
\pgfpathlineto{\pgfqpoint{2.297435in}{0.717487in}}%
\pgfpathlineto{\pgfqpoint{2.297839in}{0.744003in}}%
\pgfpathlineto{\pgfqpoint{2.297940in}{0.748410in}}%
\pgfpathlineto{\pgfqpoint{2.298345in}{0.721513in}}%
\pgfpathlineto{\pgfqpoint{2.299659in}{0.709664in}}%
\pgfpathlineto{\pgfqpoint{2.300265in}{0.708772in}}%
\pgfpathlineto{\pgfqpoint{2.300569in}{0.709764in}}%
\pgfpathlineto{\pgfqpoint{2.301782in}{0.714522in}}%
\pgfpathlineto{\pgfqpoint{2.301074in}{0.709201in}}%
\pgfpathlineto{\pgfqpoint{2.302186in}{0.713356in}}%
\pgfpathlineto{\pgfqpoint{2.303905in}{0.710902in}}%
\pgfpathlineto{\pgfqpoint{2.304208in}{0.712350in}}%
\pgfpathlineto{\pgfqpoint{2.304309in}{0.713001in}}%
\pgfpathlineto{\pgfqpoint{2.304714in}{0.709021in}}%
\pgfpathlineto{\pgfqpoint{2.304815in}{0.708597in}}%
\pgfpathlineto{\pgfqpoint{2.305017in}{0.709939in}}%
\pgfpathlineto{\pgfqpoint{2.305826in}{0.708312in}}%
\pgfpathlineto{\pgfqpoint{2.306432in}{0.718406in}}%
\pgfpathlineto{\pgfqpoint{2.308353in}{0.707838in}}%
\pgfpathlineto{\pgfqpoint{2.308555in}{0.708119in}}%
\pgfpathlineto{\pgfqpoint{2.308959in}{0.711426in}}%
\pgfpathlineto{\pgfqpoint{2.309667in}{0.708380in}}%
\pgfpathlineto{\pgfqpoint{2.310476in}{0.707545in}}%
\pgfpathlineto{\pgfqpoint{2.311082in}{0.707653in}}%
\pgfpathlineto{\pgfqpoint{2.311285in}{0.708282in}}%
\pgfpathlineto{\pgfqpoint{2.312700in}{0.712424in}}%
\pgfpathlineto{\pgfqpoint{2.312902in}{0.711085in}}%
\pgfpathlineto{\pgfqpoint{2.313408in}{0.707378in}}%
\pgfpathlineto{\pgfqpoint{2.314216in}{0.707703in}}%
\pgfpathlineto{\pgfqpoint{2.317552in}{0.708410in}}%
\pgfpathlineto{\pgfqpoint{2.317856in}{0.711837in}}%
\pgfpathlineto{\pgfqpoint{2.318361in}{0.707528in}}%
\pgfpathlineto{\pgfqpoint{2.318766in}{0.709837in}}%
\pgfpathlineto{\pgfqpoint{2.319170in}{0.707119in}}%
\pgfpathlineto{\pgfqpoint{2.320080in}{0.707542in}}%
\pgfpathlineto{\pgfqpoint{2.320282in}{0.707480in}}%
\pgfpathlineto{\pgfqpoint{2.320484in}{0.708135in}}%
\pgfpathlineto{\pgfqpoint{2.321899in}{0.717901in}}%
\pgfpathlineto{\pgfqpoint{2.321293in}{0.707500in}}%
\pgfpathlineto{\pgfqpoint{2.322001in}{0.715602in}}%
\pgfpathlineto{\pgfqpoint{2.323113in}{0.708023in}}%
\pgfpathlineto{\pgfqpoint{2.323315in}{0.708182in}}%
\pgfpathlineto{\pgfqpoint{2.323618in}{0.708661in}}%
\pgfpathlineto{\pgfqpoint{2.324022in}{0.707697in}}%
\pgfpathlineto{\pgfqpoint{2.324326in}{0.707736in}}%
\pgfpathlineto{\pgfqpoint{2.326853in}{0.708993in}}%
\pgfpathlineto{\pgfqpoint{2.327561in}{0.732712in}}%
\pgfpathlineto{\pgfqpoint{2.328572in}{0.720919in}}%
\pgfpathlineto{\pgfqpoint{2.328875in}{0.715350in}}%
\pgfpathlineto{\pgfqpoint{2.329684in}{0.708073in}}%
\pgfpathlineto{\pgfqpoint{2.330189in}{0.708526in}}%
\pgfpathlineto{\pgfqpoint{2.330897in}{0.708337in}}%
\pgfpathlineto{\pgfqpoint{2.331503in}{0.710133in}}%
\pgfpathlineto{\pgfqpoint{2.332413in}{0.708498in}}%
\pgfpathlineto{\pgfqpoint{2.333121in}{0.708740in}}%
\pgfpathlineto{\pgfqpoint{2.333930in}{0.709167in}}%
\pgfpathlineto{\pgfqpoint{2.334233in}{0.734308in}}%
\pgfpathlineto{\pgfqpoint{2.335345in}{0.835942in}}%
\pgfpathlineto{\pgfqpoint{2.334839in}{0.728833in}}%
\pgfpathlineto{\pgfqpoint{2.335547in}{0.797581in}}%
\pgfpathlineto{\pgfqpoint{2.336962in}{0.713017in}}%
\pgfpathlineto{\pgfqpoint{2.338782in}{0.709370in}}%
\pgfpathlineto{\pgfqpoint{2.338984in}{0.709674in}}%
\pgfpathlineto{\pgfqpoint{2.340400in}{0.725877in}}%
\pgfpathlineto{\pgfqpoint{2.340703in}{0.714239in}}%
\pgfpathlineto{\pgfqpoint{2.341006in}{0.709140in}}%
\pgfpathlineto{\pgfqpoint{2.341916in}{0.709400in}}%
\pgfpathlineto{\pgfqpoint{2.343230in}{0.711279in}}%
\pgfpathlineto{\pgfqpoint{2.343533in}{0.709381in}}%
\pgfpathlineto{\pgfqpoint{2.344848in}{0.708983in}}%
\pgfpathlineto{\pgfqpoint{2.347476in}{0.710644in}}%
\pgfpathlineto{\pgfqpoint{2.347880in}{0.720624in}}%
\pgfpathlineto{\pgfqpoint{2.348487in}{0.709276in}}%
\pgfpathlineto{\pgfqpoint{2.356271in}{0.709333in}}%
\pgfpathlineto{\pgfqpoint{2.357383in}{0.713272in}}%
\pgfpathlineto{\pgfqpoint{2.357889in}{0.728364in}}%
\pgfpathlineto{\pgfqpoint{2.358596in}{0.719353in}}%
\pgfpathlineto{\pgfqpoint{2.361023in}{0.708723in}}%
\pgfpathlineto{\pgfqpoint{2.361427in}{0.708867in}}%
\pgfpathlineto{\pgfqpoint{2.361730in}{0.711927in}}%
\pgfpathlineto{\pgfqpoint{2.362135in}{0.725424in}}%
\pgfpathlineto{\pgfqpoint{2.363044in}{0.721314in}}%
\pgfpathlineto{\pgfqpoint{2.363853in}{0.708839in}}%
\pgfpathlineto{\pgfqpoint{2.364460in}{0.708879in}}%
\pgfpathlineto{\pgfqpoint{2.366684in}{0.709182in}}%
\pgfpathlineto{\pgfqpoint{2.367088in}{0.712369in}}%
\pgfpathlineto{\pgfqpoint{2.367796in}{0.709227in}}%
\pgfpathlineto{\pgfqpoint{2.368301in}{0.710055in}}%
\pgfpathlineto{\pgfqpoint{2.368908in}{0.709336in}}%
\pgfpathlineto{\pgfqpoint{2.369818in}{0.709182in}}%
\pgfpathlineto{\pgfqpoint{2.370020in}{0.709390in}}%
\pgfpathlineto{\pgfqpoint{2.370424in}{0.709813in}}%
\pgfpathlineto{\pgfqpoint{2.371031in}{0.708929in}}%
\pgfpathlineto{\pgfqpoint{2.378411in}{0.709768in}}%
\pgfpathlineto{\pgfqpoint{2.378714in}{0.712419in}}%
\pgfpathlineto{\pgfqpoint{2.379321in}{0.709522in}}%
\pgfpathlineto{\pgfqpoint{2.379624in}{0.710910in}}%
\pgfpathlineto{\pgfqpoint{2.380534in}{0.714759in}}%
\pgfpathlineto{\pgfqpoint{2.380837in}{0.712901in}}%
\pgfpathlineto{\pgfqpoint{2.382353in}{0.708435in}}%
\pgfpathlineto{\pgfqpoint{2.381545in}{0.713607in}}%
\pgfpathlineto{\pgfqpoint{2.382454in}{0.708489in}}%
\pgfpathlineto{\pgfqpoint{2.383263in}{0.710500in}}%
\pgfpathlineto{\pgfqpoint{2.383566in}{0.724997in}}%
\pgfpathlineto{\pgfqpoint{2.384274in}{0.707645in}}%
\pgfpathlineto{\pgfqpoint{2.384577in}{0.707725in}}%
\pgfpathlineto{\pgfqpoint{2.384678in}{0.708206in}}%
\pgfpathlineto{\pgfqpoint{2.384982in}{0.730088in}}%
\pgfpathlineto{\pgfqpoint{2.385386in}{0.916134in}}%
\pgfpathlineto{\pgfqpoint{2.385993in}{0.732681in}}%
\pgfpathlineto{\pgfqpoint{2.386599in}{0.743650in}}%
\pgfpathlineto{\pgfqpoint{2.387610in}{0.710459in}}%
\pgfpathlineto{\pgfqpoint{2.389329in}{0.709620in}}%
\pgfpathlineto{\pgfqpoint{2.393271in}{0.711132in}}%
\pgfpathlineto{\pgfqpoint{2.394788in}{0.719730in}}%
\pgfpathlineto{\pgfqpoint{2.395192in}{0.729237in}}%
\pgfpathlineto{\pgfqpoint{2.395698in}{0.718254in}}%
\pgfpathlineto{\pgfqpoint{2.397214in}{0.710634in}}%
\pgfpathlineto{\pgfqpoint{2.398730in}{0.708718in}}%
\pgfpathlineto{\pgfqpoint{2.404594in}{0.708719in}}%
\pgfpathlineto{\pgfqpoint{2.406211in}{0.709590in}}%
\pgfpathlineto{\pgfqpoint{2.406616in}{0.712072in}}%
\pgfpathlineto{\pgfqpoint{2.407121in}{0.708724in}}%
\pgfpathlineto{\pgfqpoint{2.409649in}{0.709360in}}%
\pgfpathlineto{\pgfqpoint{2.410053in}{0.715044in}}%
\pgfpathlineto{\pgfqpoint{2.410457in}{0.722616in}}%
\pgfpathlineto{\pgfqpoint{2.411266in}{0.718429in}}%
\pgfpathlineto{\pgfqpoint{2.411569in}{0.721223in}}%
\pgfpathlineto{\pgfqpoint{2.411974in}{0.712441in}}%
\pgfpathlineto{\pgfqpoint{2.413187in}{0.708505in}}%
\pgfpathlineto{\pgfqpoint{2.414501in}{0.709079in}}%
\pgfpathlineto{\pgfqpoint{2.416017in}{0.708822in}}%
\pgfpathlineto{\pgfqpoint{2.418545in}{0.709230in}}%
\pgfpathlineto{\pgfqpoint{2.419758in}{0.712033in}}%
\pgfpathlineto{\pgfqpoint{2.419960in}{0.710692in}}%
\pgfpathlineto{\pgfqpoint{2.421173in}{0.708901in}}%
\pgfpathlineto{\pgfqpoint{2.422791in}{0.709015in}}%
\pgfpathlineto{\pgfqpoint{2.423903in}{0.710534in}}%
\pgfpathlineto{\pgfqpoint{2.424307in}{0.716011in}}%
\pgfpathlineto{\pgfqpoint{2.424914in}{0.710121in}}%
\pgfpathlineto{\pgfqpoint{2.425217in}{0.715217in}}%
\pgfpathlineto{\pgfqpoint{2.425520in}{0.723173in}}%
\pgfpathlineto{\pgfqpoint{2.426127in}{0.709277in}}%
\pgfpathlineto{\pgfqpoint{2.426228in}{0.709208in}}%
\pgfpathlineto{\pgfqpoint{2.426430in}{0.710208in}}%
\pgfpathlineto{\pgfqpoint{2.428149in}{0.728235in}}%
\pgfpathlineto{\pgfqpoint{2.428250in}{0.726113in}}%
\pgfpathlineto{\pgfqpoint{2.429665in}{0.708605in}}%
\pgfpathlineto{\pgfqpoint{2.430878in}{0.708773in}}%
\pgfpathlineto{\pgfqpoint{2.430979in}{0.709006in}}%
\pgfpathlineto{\pgfqpoint{2.432192in}{0.716307in}}%
\pgfpathlineto{\pgfqpoint{2.432597in}{0.729818in}}%
\pgfpathlineto{\pgfqpoint{2.433203in}{0.716066in}}%
\pgfpathlineto{\pgfqpoint{2.435427in}{0.708370in}}%
\pgfpathlineto{\pgfqpoint{2.436944in}{0.708976in}}%
\pgfpathlineto{\pgfqpoint{2.437247in}{0.710123in}}%
\pgfpathlineto{\pgfqpoint{2.437651in}{0.708568in}}%
\pgfpathlineto{\pgfqpoint{2.438056in}{0.709334in}}%
\pgfpathlineto{\pgfqpoint{2.438157in}{0.709440in}}%
\pgfpathlineto{\pgfqpoint{2.438561in}{0.708465in}}%
\pgfpathlineto{\pgfqpoint{2.439875in}{0.708356in}}%
\pgfpathlineto{\pgfqpoint{2.447356in}{0.708676in}}%
\pgfpathlineto{\pgfqpoint{2.447963in}{0.709566in}}%
\pgfpathlineto{\pgfqpoint{2.449580in}{0.732036in}}%
\pgfpathlineto{\pgfqpoint{2.449783in}{0.724099in}}%
\pgfpathlineto{\pgfqpoint{2.450288in}{0.708885in}}%
\pgfpathlineto{\pgfqpoint{2.451097in}{0.709526in}}%
\pgfpathlineto{\pgfqpoint{2.452310in}{0.708609in}}%
\pgfpathlineto{\pgfqpoint{2.453018in}{0.709284in}}%
\pgfpathlineto{\pgfqpoint{2.453523in}{0.717497in}}%
\pgfpathlineto{\pgfqpoint{2.454534in}{0.715539in}}%
\pgfpathlineto{\pgfqpoint{2.455343in}{0.708649in}}%
\pgfpathlineto{\pgfqpoint{2.456050in}{0.708706in}}%
\pgfpathlineto{\pgfqpoint{2.459285in}{0.709855in}}%
\pgfpathlineto{\pgfqpoint{2.460802in}{0.715138in}}%
\pgfpathlineto{\pgfqpoint{2.461004in}{0.714073in}}%
\pgfpathlineto{\pgfqpoint{2.462622in}{0.708735in}}%
\pgfpathlineto{\pgfqpoint{2.464846in}{0.709158in}}%
\pgfpathlineto{\pgfqpoint{2.466463in}{0.710557in}}%
\pgfpathlineto{\pgfqpoint{2.467979in}{0.742899in}}%
\pgfpathlineto{\pgfqpoint{2.468182in}{0.761075in}}%
\pgfpathlineto{\pgfqpoint{2.468788in}{0.712366in}}%
\pgfpathlineto{\pgfqpoint{2.468889in}{0.711715in}}%
\pgfpathlineto{\pgfqpoint{2.468990in}{0.713490in}}%
\pgfpathlineto{\pgfqpoint{2.469395in}{0.737274in}}%
\pgfpathlineto{\pgfqpoint{2.470001in}{0.709546in}}%
\pgfpathlineto{\pgfqpoint{2.470204in}{0.710140in}}%
\pgfpathlineto{\pgfqpoint{2.470507in}{0.711962in}}%
\pgfpathlineto{\pgfqpoint{2.471113in}{0.709142in}}%
\pgfpathlineto{\pgfqpoint{2.473742in}{0.709157in}}%
\pgfpathlineto{\pgfqpoint{2.475359in}{0.713941in}}%
\pgfpathlineto{\pgfqpoint{2.475663in}{0.711446in}}%
\pgfpathlineto{\pgfqpoint{2.477078in}{0.709073in}}%
\pgfpathlineto{\pgfqpoint{2.478695in}{0.708900in}}%
\pgfpathlineto{\pgfqpoint{2.480919in}{0.709322in}}%
\pgfpathlineto{\pgfqpoint{2.481526in}{0.708907in}}%
\pgfpathlineto{\pgfqpoint{2.481829in}{0.709526in}}%
\pgfpathlineto{\pgfqpoint{2.482739in}{0.709306in}}%
\pgfpathlineto{\pgfqpoint{2.483245in}{0.711181in}}%
\pgfpathlineto{\pgfqpoint{2.486985in}{0.708924in}}%
\pgfpathlineto{\pgfqpoint{2.487794in}{0.709514in}}%
\pgfpathlineto{\pgfqpoint{2.489411in}{0.721477in}}%
\pgfpathlineto{\pgfqpoint{2.489613in}{0.715717in}}%
\pgfpathlineto{\pgfqpoint{2.490018in}{0.708015in}}%
\pgfpathlineto{\pgfqpoint{2.490827in}{0.708318in}}%
\pgfpathlineto{\pgfqpoint{2.491635in}{0.708873in}}%
\pgfpathlineto{\pgfqpoint{2.492141in}{0.728205in}}%
\pgfpathlineto{\pgfqpoint{2.493758in}{0.725578in}}%
\pgfpathlineto{\pgfqpoint{2.494466in}{0.708977in}}%
\pgfpathlineto{\pgfqpoint{2.495072in}{0.717385in}}%
\pgfpathlineto{\pgfqpoint{2.495376in}{0.713232in}}%
\pgfpathlineto{\pgfqpoint{2.495780in}{0.708444in}}%
\pgfpathlineto{\pgfqpoint{2.496387in}{0.714938in}}%
\pgfpathlineto{\pgfqpoint{2.496488in}{0.715710in}}%
\pgfpathlineto{\pgfqpoint{2.496993in}{0.711123in}}%
\pgfpathlineto{\pgfqpoint{2.498409in}{0.709131in}}%
\pgfpathlineto{\pgfqpoint{2.499015in}{0.707332in}}%
\pgfpathlineto{\pgfqpoint{2.499420in}{0.709641in}}%
\pgfpathlineto{\pgfqpoint{2.500936in}{0.728526in}}%
\pgfpathlineto{\pgfqpoint{2.500228in}{0.708545in}}%
\pgfpathlineto{\pgfqpoint{2.501138in}{0.722465in}}%
\pgfpathlineto{\pgfqpoint{2.502553in}{0.707962in}}%
\pgfpathlineto{\pgfqpoint{2.503059in}{0.707337in}}%
\pgfpathlineto{\pgfqpoint{2.503564in}{0.707803in}}%
\pgfpathlineto{\pgfqpoint{2.503868in}{0.708596in}}%
\pgfpathlineto{\pgfqpoint{2.504676in}{0.707839in}}%
\pgfpathlineto{\pgfqpoint{2.506900in}{0.710293in}}%
\pgfpathlineto{\pgfqpoint{2.507406in}{0.722013in}}%
\pgfpathlineto{\pgfqpoint{2.508114in}{0.712875in}}%
\pgfpathlineto{\pgfqpoint{2.508619in}{0.720020in}}%
\pgfpathlineto{\pgfqpoint{2.509124in}{0.711845in}}%
\pgfpathlineto{\pgfqpoint{2.510439in}{0.707038in}}%
\pgfpathlineto{\pgfqpoint{2.512663in}{0.707421in}}%
\pgfpathlineto{\pgfqpoint{2.515493in}{0.709053in}}%
\pgfpathlineto{\pgfqpoint{2.515999in}{0.717427in}}%
\pgfpathlineto{\pgfqpoint{2.516504in}{0.709147in}}%
\pgfpathlineto{\pgfqpoint{2.517515in}{0.708250in}}%
\pgfpathlineto{\pgfqpoint{2.517717in}{0.708492in}}%
\pgfpathlineto{\pgfqpoint{2.518122in}{0.710916in}}%
\pgfpathlineto{\pgfqpoint{2.518931in}{0.709140in}}%
\pgfpathlineto{\pgfqpoint{2.526513in}{0.709700in}}%
\pgfpathlineto{\pgfqpoint{2.528231in}{0.717259in}}%
\pgfpathlineto{\pgfqpoint{2.528433in}{0.715032in}}%
\pgfpathlineto{\pgfqpoint{2.529343in}{0.708698in}}%
\pgfpathlineto{\pgfqpoint{2.529849in}{0.708776in}}%
\pgfpathlineto{\pgfqpoint{2.531466in}{0.710096in}}%
\pgfpathlineto{\pgfqpoint{2.532679in}{0.717196in}}%
\pgfpathlineto{\pgfqpoint{2.533084in}{0.714449in}}%
\pgfpathlineto{\pgfqpoint{2.535105in}{0.708599in}}%
\pgfpathlineto{\pgfqpoint{2.535207in}{0.708658in}}%
\pgfpathlineto{\pgfqpoint{2.535712in}{0.710528in}}%
\pgfpathlineto{\pgfqpoint{2.536319in}{0.708345in}}%
\pgfpathlineto{\pgfqpoint{2.539250in}{0.709921in}}%
\pgfpathlineto{\pgfqpoint{2.540565in}{0.724174in}}%
\pgfpathlineto{\pgfqpoint{2.540868in}{0.716718in}}%
\pgfpathlineto{\pgfqpoint{2.541474in}{0.708894in}}%
\pgfpathlineto{\pgfqpoint{2.542081in}{0.710482in}}%
\pgfpathlineto{\pgfqpoint{2.543597in}{0.722190in}}%
\pgfpathlineto{\pgfqpoint{2.543800in}{0.727219in}}%
\pgfpathlineto{\pgfqpoint{2.544305in}{0.709053in}}%
\pgfpathlineto{\pgfqpoint{2.544912in}{0.710181in}}%
\pgfpathlineto{\pgfqpoint{2.545518in}{0.708046in}}%
\pgfpathlineto{\pgfqpoint{2.547540in}{0.708591in}}%
\pgfpathlineto{\pgfqpoint{2.547944in}{0.719364in}}%
\pgfpathlineto{\pgfqpoint{2.548248in}{0.729177in}}%
\pgfpathlineto{\pgfqpoint{2.549157in}{0.725510in}}%
\pgfpathlineto{\pgfqpoint{2.551887in}{0.707187in}}%
\pgfpathlineto{\pgfqpoint{2.552190in}{0.708093in}}%
\pgfpathlineto{\pgfqpoint{2.552595in}{0.712028in}}%
\pgfpathlineto{\pgfqpoint{2.553403in}{0.709727in}}%
\pgfpathlineto{\pgfqpoint{2.553808in}{0.709556in}}%
\pgfpathlineto{\pgfqpoint{2.554010in}{0.710653in}}%
\pgfpathlineto{\pgfqpoint{2.554617in}{0.725895in}}%
\pgfpathlineto{\pgfqpoint{2.555223in}{0.712231in}}%
\pgfpathlineto{\pgfqpoint{2.555324in}{0.712103in}}%
\pgfpathlineto{\pgfqpoint{2.555526in}{0.712954in}}%
\pgfpathlineto{\pgfqpoint{2.556032in}{0.715635in}}%
\pgfpathlineto{\pgfqpoint{2.556436in}{0.712029in}}%
\pgfpathlineto{\pgfqpoint{2.556841in}{0.707338in}}%
\pgfpathlineto{\pgfqpoint{2.557245in}{0.712411in}}%
\pgfpathlineto{\pgfqpoint{2.557548in}{0.719830in}}%
\pgfpathlineto{\pgfqpoint{2.558155in}{0.709614in}}%
\pgfpathlineto{\pgfqpoint{2.558357in}{0.708853in}}%
\pgfpathlineto{\pgfqpoint{2.558660in}{0.713687in}}%
\pgfpathlineto{\pgfqpoint{2.558964in}{0.722105in}}%
\pgfpathlineto{\pgfqpoint{2.559469in}{0.706577in}}%
\pgfpathlineto{\pgfqpoint{2.560379in}{0.705594in}}%
\pgfpathlineto{\pgfqpoint{2.560581in}{0.705723in}}%
\pgfpathlineto{\pgfqpoint{2.561289in}{0.707754in}}%
\pgfpathlineto{\pgfqpoint{2.562401in}{0.724443in}}%
\pgfpathlineto{\pgfqpoint{2.561794in}{0.707454in}}%
\pgfpathlineto{\pgfqpoint{2.562805in}{0.714232in}}%
\pgfpathlineto{\pgfqpoint{2.563209in}{0.708014in}}%
\pgfpathlineto{\pgfqpoint{2.563715in}{0.713618in}}%
\pgfpathlineto{\pgfqpoint{2.564119in}{0.729680in}}%
\pgfpathlineto{\pgfqpoint{2.564726in}{0.712160in}}%
\pgfpathlineto{\pgfqpoint{2.564928in}{0.710970in}}%
\pgfpathlineto{\pgfqpoint{2.565231in}{0.714655in}}%
\pgfpathlineto{\pgfqpoint{2.565939in}{0.756365in}}%
\pgfpathlineto{\pgfqpoint{2.566444in}{0.723288in}}%
\pgfpathlineto{\pgfqpoint{2.567860in}{0.706633in}}%
\pgfpathlineto{\pgfqpoint{2.566950in}{0.723731in}}%
\pgfpathlineto{\pgfqpoint{2.567961in}{0.706653in}}%
\pgfpathlineto{\pgfqpoint{2.569275in}{0.708036in}}%
\pgfpathlineto{\pgfqpoint{2.569679in}{0.710440in}}%
\pgfpathlineto{\pgfqpoint{2.570387in}{0.708716in}}%
\pgfpathlineto{\pgfqpoint{2.572005in}{0.707884in}}%
\pgfpathlineto{\pgfqpoint{2.572813in}{0.708655in}}%
\pgfpathlineto{\pgfqpoint{2.574128in}{0.733117in}}%
\pgfpathlineto{\pgfqpoint{2.574330in}{0.741950in}}%
\pgfpathlineto{\pgfqpoint{2.574936in}{0.720028in}}%
\pgfpathlineto{\pgfqpoint{2.576756in}{0.708571in}}%
\pgfpathlineto{\pgfqpoint{2.578070in}{0.708651in}}%
\pgfpathlineto{\pgfqpoint{2.578475in}{0.709114in}}%
\pgfpathlineto{\pgfqpoint{2.579081in}{0.708309in}}%
\pgfpathlineto{\pgfqpoint{2.581305in}{0.708633in}}%
\pgfpathlineto{\pgfqpoint{2.582518in}{0.711546in}}%
\pgfpathlineto{\pgfqpoint{2.582721in}{0.710524in}}%
\pgfpathlineto{\pgfqpoint{2.583731in}{0.708405in}}%
\pgfpathlineto{\pgfqpoint{2.584035in}{0.708475in}}%
\pgfpathlineto{\pgfqpoint{2.586966in}{0.709897in}}%
\pgfpathlineto{\pgfqpoint{2.587775in}{0.709378in}}%
\pgfpathlineto{\pgfqpoint{2.588281in}{0.719736in}}%
\pgfpathlineto{\pgfqpoint{2.588584in}{0.736131in}}%
\pgfpathlineto{\pgfqpoint{2.589190in}{0.713419in}}%
\pgfpathlineto{\pgfqpoint{2.589292in}{0.714111in}}%
\pgfpathlineto{\pgfqpoint{2.589797in}{0.726824in}}%
\pgfpathlineto{\pgfqpoint{2.590303in}{0.712116in}}%
\pgfpathlineto{\pgfqpoint{2.591617in}{0.709298in}}%
\pgfpathlineto{\pgfqpoint{2.592729in}{0.708605in}}%
\pgfpathlineto{\pgfqpoint{2.592931in}{0.708861in}}%
\pgfpathlineto{\pgfqpoint{2.593234in}{0.709540in}}%
\pgfpathlineto{\pgfqpoint{2.594043in}{0.708715in}}%
\pgfpathlineto{\pgfqpoint{2.603445in}{0.711291in}}%
\pgfpathlineto{\pgfqpoint{2.603647in}{0.711519in}}%
\pgfpathlineto{\pgfqpoint{2.604152in}{0.710064in}}%
\pgfpathlineto{\pgfqpoint{2.605062in}{0.709855in}}%
\pgfpathlineto{\pgfqpoint{2.605264in}{0.710369in}}%
\pgfpathlineto{\pgfqpoint{2.605669in}{0.712399in}}%
\pgfpathlineto{\pgfqpoint{2.606174in}{0.709657in}}%
\pgfpathlineto{\pgfqpoint{2.606275in}{0.709534in}}%
\pgfpathlineto{\pgfqpoint{2.606579in}{0.710358in}}%
\pgfpathlineto{\pgfqpoint{2.606882in}{0.711427in}}%
\pgfpathlineto{\pgfqpoint{2.607488in}{0.709558in}}%
\pgfpathlineto{\pgfqpoint{2.608196in}{0.709335in}}%
\pgfpathlineto{\pgfqpoint{2.608398in}{0.709959in}}%
\pgfpathlineto{\pgfqpoint{2.608803in}{0.712181in}}%
\pgfpathlineto{\pgfqpoint{2.609308in}{0.709019in}}%
\pgfpathlineto{\pgfqpoint{2.610521in}{0.708538in}}%
\pgfpathlineto{\pgfqpoint{2.610723in}{0.709084in}}%
\pgfpathlineto{\pgfqpoint{2.610926in}{0.709290in}}%
\pgfpathlineto{\pgfqpoint{2.611330in}{0.708015in}}%
\pgfpathlineto{\pgfqpoint{2.611633in}{0.707829in}}%
\pgfpathlineto{\pgfqpoint{2.611937in}{0.708779in}}%
\pgfpathlineto{\pgfqpoint{2.613251in}{0.727111in}}%
\pgfpathlineto{\pgfqpoint{2.613554in}{0.738082in}}%
\pgfpathlineto{\pgfqpoint{2.614161in}{0.720764in}}%
\pgfpathlineto{\pgfqpoint{2.615677in}{0.709854in}}%
\pgfpathlineto{\pgfqpoint{2.616081in}{0.710060in}}%
\pgfpathlineto{\pgfqpoint{2.616284in}{0.710612in}}%
\pgfpathlineto{\pgfqpoint{2.617598in}{0.715447in}}%
\pgfpathlineto{\pgfqpoint{2.617800in}{0.713765in}}%
\pgfpathlineto{\pgfqpoint{2.619114in}{0.706447in}}%
\pgfpathlineto{\pgfqpoint{2.622450in}{0.707603in}}%
\pgfpathlineto{\pgfqpoint{2.622652in}{0.706516in}}%
\pgfpathlineto{\pgfqpoint{2.622855in}{0.706357in}}%
\pgfpathlineto{\pgfqpoint{2.623057in}{0.707141in}}%
\pgfpathlineto{\pgfqpoint{2.624472in}{0.740133in}}%
\pgfpathlineto{\pgfqpoint{2.624876in}{0.721748in}}%
\pgfpathlineto{\pgfqpoint{2.625988in}{0.708648in}}%
\pgfpathlineto{\pgfqpoint{2.626292in}{0.709544in}}%
\pgfpathlineto{\pgfqpoint{2.628314in}{0.717106in}}%
\pgfpathlineto{\pgfqpoint{2.630032in}{0.752917in}}%
\pgfpathlineto{\pgfqpoint{2.630133in}{0.748709in}}%
\pgfpathlineto{\pgfqpoint{2.630841in}{0.709306in}}%
\pgfpathlineto{\pgfqpoint{2.631650in}{0.710684in}}%
\pgfpathlineto{\pgfqpoint{2.632155in}{0.707746in}}%
\pgfpathlineto{\pgfqpoint{2.632762in}{0.709914in}}%
\pgfpathlineto{\pgfqpoint{2.632964in}{0.710648in}}%
\pgfpathlineto{\pgfqpoint{2.633469in}{0.708170in}}%
\pgfpathlineto{\pgfqpoint{2.634986in}{0.708174in}}%
\pgfpathlineto{\pgfqpoint{2.636300in}{0.709741in}}%
\pgfpathlineto{\pgfqpoint{2.636907in}{0.736226in}}%
\pgfpathlineto{\pgfqpoint{2.637614in}{0.714791in}}%
\pgfpathlineto{\pgfqpoint{2.637816in}{0.716069in}}%
\pgfpathlineto{\pgfqpoint{2.638221in}{0.724130in}}%
\pgfpathlineto{\pgfqpoint{2.638625in}{0.714388in}}%
\pgfpathlineto{\pgfqpoint{2.639030in}{0.708335in}}%
\pgfpathlineto{\pgfqpoint{2.639737in}{0.712746in}}%
\pgfpathlineto{\pgfqpoint{2.641254in}{0.754024in}}%
\pgfpathlineto{\pgfqpoint{2.641355in}{0.758855in}}%
\pgfpathlineto{\pgfqpoint{2.641658in}{0.733651in}}%
\pgfpathlineto{\pgfqpoint{2.642972in}{0.710019in}}%
\pgfpathlineto{\pgfqpoint{2.643781in}{0.707692in}}%
\pgfpathlineto{\pgfqpoint{2.644489in}{0.707909in}}%
\pgfpathlineto{\pgfqpoint{2.645196in}{0.708353in}}%
\pgfpathlineto{\pgfqpoint{2.645297in}{0.708589in}}%
\pgfpathlineto{\pgfqpoint{2.646814in}{0.711421in}}%
\pgfpathlineto{\pgfqpoint{2.646915in}{0.710844in}}%
\pgfpathlineto{\pgfqpoint{2.647926in}{0.708141in}}%
\pgfpathlineto{\pgfqpoint{2.648229in}{0.709076in}}%
\pgfpathlineto{\pgfqpoint{2.648633in}{0.715088in}}%
\pgfpathlineto{\pgfqpoint{2.649240in}{0.708911in}}%
\pgfpathlineto{\pgfqpoint{2.651262in}{0.708238in}}%
\pgfpathlineto{\pgfqpoint{2.649745in}{0.709618in}}%
\pgfpathlineto{\pgfqpoint{2.651363in}{0.708374in}}%
\pgfpathlineto{\pgfqpoint{2.653385in}{0.711412in}}%
\pgfpathlineto{\pgfqpoint{2.653789in}{0.718229in}}%
\pgfpathlineto{\pgfqpoint{2.654396in}{0.709082in}}%
\pgfpathlineto{\pgfqpoint{2.654598in}{0.709804in}}%
\pgfpathlineto{\pgfqpoint{2.655103in}{0.719297in}}%
\pgfpathlineto{\pgfqpoint{2.655609in}{0.709778in}}%
\pgfpathlineto{\pgfqpoint{2.655912in}{0.708435in}}%
\pgfpathlineto{\pgfqpoint{2.656620in}{0.710697in}}%
\pgfpathlineto{\pgfqpoint{2.657024in}{0.712410in}}%
\pgfpathlineto{\pgfqpoint{2.657429in}{0.709876in}}%
\pgfpathlineto{\pgfqpoint{2.657631in}{0.709009in}}%
\pgfpathlineto{\pgfqpoint{2.658035in}{0.712701in}}%
\pgfpathlineto{\pgfqpoint{2.658338in}{0.717721in}}%
\pgfpathlineto{\pgfqpoint{2.658844in}{0.709340in}}%
\pgfpathlineto{\pgfqpoint{2.659653in}{0.710116in}}%
\pgfpathlineto{\pgfqpoint{2.660158in}{0.707390in}}%
\pgfpathlineto{\pgfqpoint{2.660562in}{0.707386in}}%
\pgfpathlineto{\pgfqpoint{2.660866in}{0.707905in}}%
\pgfpathlineto{\pgfqpoint{2.661270in}{0.709182in}}%
\pgfpathlineto{\pgfqpoint{2.661776in}{0.707494in}}%
\pgfpathlineto{\pgfqpoint{2.662382in}{0.707943in}}%
\pgfpathlineto{\pgfqpoint{2.663292in}{0.706898in}}%
\pgfpathlineto{\pgfqpoint{2.665213in}{0.708510in}}%
\pgfpathlineto{\pgfqpoint{2.666527in}{0.731095in}}%
\pgfpathlineto{\pgfqpoint{2.666729in}{0.720828in}}%
\pgfpathlineto{\pgfqpoint{2.668043in}{0.707317in}}%
\pgfpathlineto{\pgfqpoint{2.669459in}{0.707723in}}%
\pgfpathlineto{\pgfqpoint{2.670571in}{0.708957in}}%
\pgfpathlineto{\pgfqpoint{2.671076in}{0.725001in}}%
\pgfpathlineto{\pgfqpoint{2.671784in}{0.711088in}}%
\pgfpathlineto{\pgfqpoint{2.672087in}{0.712982in}}%
\pgfpathlineto{\pgfqpoint{2.672593in}{0.708260in}}%
\pgfpathlineto{\pgfqpoint{2.673401in}{0.707797in}}%
\pgfpathlineto{\pgfqpoint{2.673705in}{0.708088in}}%
\pgfpathlineto{\pgfqpoint{2.675120in}{0.708211in}}%
\pgfpathlineto{\pgfqpoint{2.678961in}{0.711291in}}%
\pgfpathlineto{\pgfqpoint{2.679669in}{0.713952in}}%
\pgfpathlineto{\pgfqpoint{2.680175in}{0.712673in}}%
\pgfpathlineto{\pgfqpoint{2.681489in}{0.709640in}}%
\pgfpathlineto{\pgfqpoint{2.680882in}{0.714358in}}%
\pgfpathlineto{\pgfqpoint{2.681691in}{0.710154in}}%
\pgfpathlineto{\pgfqpoint{2.683106in}{0.722139in}}%
\pgfpathlineto{\pgfqpoint{2.683308in}{0.717915in}}%
\pgfpathlineto{\pgfqpoint{2.684623in}{0.708079in}}%
\pgfpathlineto{\pgfqpoint{2.687959in}{0.709164in}}%
\pgfpathlineto{\pgfqpoint{2.689172in}{0.716796in}}%
\pgfpathlineto{\pgfqpoint{2.688565in}{0.708924in}}%
\pgfpathlineto{\pgfqpoint{2.689475in}{0.711902in}}%
\pgfpathlineto{\pgfqpoint{2.690385in}{0.708977in}}%
\pgfpathlineto{\pgfqpoint{2.690688in}{0.709272in}}%
\pgfpathlineto{\pgfqpoint{2.691598in}{0.711623in}}%
\pgfpathlineto{\pgfqpoint{2.692003in}{0.709948in}}%
\pgfpathlineto{\pgfqpoint{2.692205in}{0.709475in}}%
\pgfpathlineto{\pgfqpoint{2.692508in}{0.711812in}}%
\pgfpathlineto{\pgfqpoint{2.692710in}{0.713054in}}%
\pgfpathlineto{\pgfqpoint{2.693216in}{0.707682in}}%
\pgfpathlineto{\pgfqpoint{2.694530in}{0.707840in}}%
\pgfpathlineto{\pgfqpoint{2.694631in}{0.708027in}}%
\pgfpathlineto{\pgfqpoint{2.696147in}{0.723255in}}%
\pgfpathlineto{\pgfqpoint{2.696451in}{0.728332in}}%
\pgfpathlineto{\pgfqpoint{2.696855in}{0.716950in}}%
\pgfpathlineto{\pgfqpoint{2.697259in}{0.710600in}}%
\pgfpathlineto{\pgfqpoint{2.697664in}{0.723201in}}%
\pgfpathlineto{\pgfqpoint{2.698068in}{0.747498in}}%
\pgfpathlineto{\pgfqpoint{2.698574in}{0.714475in}}%
\pgfpathlineto{\pgfqpoint{2.698776in}{0.712597in}}%
\pgfpathlineto{\pgfqpoint{2.699079in}{0.720295in}}%
\pgfpathlineto{\pgfqpoint{2.699382in}{0.728673in}}%
\pgfpathlineto{\pgfqpoint{2.699888in}{0.710736in}}%
\pgfpathlineto{\pgfqpoint{2.700697in}{0.711040in}}%
\pgfpathlineto{\pgfqpoint{2.701101in}{0.708453in}}%
\pgfpathlineto{\pgfqpoint{2.701404in}{0.707965in}}%
\pgfpathlineto{\pgfqpoint{2.701910in}{0.709015in}}%
\pgfpathlineto{\pgfqpoint{2.703426in}{0.732745in}}%
\pgfpathlineto{\pgfqpoint{2.703628in}{0.739751in}}%
\pgfpathlineto{\pgfqpoint{2.704134in}{0.712183in}}%
\pgfpathlineto{\pgfqpoint{2.704437in}{0.709791in}}%
\pgfpathlineto{\pgfqpoint{2.705347in}{0.710523in}}%
\pgfpathlineto{\pgfqpoint{2.705751in}{0.709537in}}%
\pgfpathlineto{\pgfqpoint{2.706156in}{0.708075in}}%
\pgfpathlineto{\pgfqpoint{2.706459in}{0.710749in}}%
\pgfpathlineto{\pgfqpoint{2.706964in}{0.731422in}}%
\pgfpathlineto{\pgfqpoint{2.707571in}{0.712827in}}%
\pgfpathlineto{\pgfqpoint{2.708076in}{0.726792in}}%
\pgfpathlineto{\pgfqpoint{2.708481in}{0.712345in}}%
\pgfpathlineto{\pgfqpoint{2.709694in}{0.707293in}}%
\pgfpathlineto{\pgfqpoint{2.709997in}{0.707909in}}%
\pgfpathlineto{\pgfqpoint{2.710402in}{0.715151in}}%
\pgfpathlineto{\pgfqpoint{2.711412in}{0.711992in}}%
\pgfpathlineto{\pgfqpoint{2.711716in}{0.713607in}}%
\pgfpathlineto{\pgfqpoint{2.712120in}{0.710710in}}%
\pgfpathlineto{\pgfqpoint{2.713131in}{0.708137in}}%
\pgfpathlineto{\pgfqpoint{2.713333in}{0.708468in}}%
\pgfpathlineto{\pgfqpoint{2.713940in}{0.713658in}}%
\pgfpathlineto{\pgfqpoint{2.714344in}{0.708187in}}%
\pgfpathlineto{\pgfqpoint{2.714647in}{0.706533in}}%
\pgfpathlineto{\pgfqpoint{2.714951in}{0.709413in}}%
\pgfpathlineto{\pgfqpoint{2.715355in}{0.722455in}}%
\pgfpathlineto{\pgfqpoint{2.715962in}{0.707201in}}%
\pgfpathlineto{\pgfqpoint{2.716366in}{0.716556in}}%
\pgfpathlineto{\pgfqpoint{2.716568in}{0.719241in}}%
\pgfpathlineto{\pgfqpoint{2.716973in}{0.707830in}}%
\pgfpathlineto{\pgfqpoint{2.717680in}{0.708247in}}%
\pgfpathlineto{\pgfqpoint{2.718186in}{0.705911in}}%
\pgfpathlineto{\pgfqpoint{2.719096in}{0.706710in}}%
\pgfpathlineto{\pgfqpoint{2.720410in}{0.707240in}}%
\pgfpathlineto{\pgfqpoint{2.720511in}{0.707164in}}%
\pgfpathlineto{\pgfqpoint{2.721825in}{0.707452in}}%
\pgfpathlineto{\pgfqpoint{2.722432in}{0.711207in}}%
\pgfpathlineto{\pgfqpoint{2.722937in}{0.707755in}}%
\pgfpathlineto{\pgfqpoint{2.723038in}{0.707616in}}%
\pgfpathlineto{\pgfqpoint{2.723341in}{0.709025in}}%
\pgfpathlineto{\pgfqpoint{2.723544in}{0.709619in}}%
\pgfpathlineto{\pgfqpoint{2.724049in}{0.706860in}}%
\pgfpathlineto{\pgfqpoint{2.724858in}{0.707564in}}%
\pgfpathlineto{\pgfqpoint{2.726172in}{0.714793in}}%
\pgfpathlineto{\pgfqpoint{2.726475in}{0.725487in}}%
\pgfpathlineto{\pgfqpoint{2.727183in}{0.710917in}}%
\pgfpathlineto{\pgfqpoint{2.727587in}{0.719222in}}%
\pgfpathlineto{\pgfqpoint{2.727891in}{0.725599in}}%
\pgfpathlineto{\pgfqpoint{2.728396in}{0.710570in}}%
\pgfpathlineto{\pgfqpoint{2.728801in}{0.708380in}}%
\pgfpathlineto{\pgfqpoint{2.729306in}{0.711981in}}%
\pgfpathlineto{\pgfqpoint{2.729913in}{0.758790in}}%
\pgfpathlineto{\pgfqpoint{2.731227in}{0.743213in}}%
\pgfpathlineto{\pgfqpoint{2.732036in}{0.710769in}}%
\pgfpathlineto{\pgfqpoint{2.732642in}{0.710891in}}%
\pgfpathlineto{\pgfqpoint{2.733350in}{0.708191in}}%
\pgfpathlineto{\pgfqpoint{2.733956in}{0.708316in}}%
\pgfpathlineto{\pgfqpoint{2.737393in}{0.710081in}}%
\pgfpathlineto{\pgfqpoint{2.738000in}{0.711359in}}%
\pgfpathlineto{\pgfqpoint{2.738708in}{0.710665in}}%
\pgfpathlineto{\pgfqpoint{2.739415in}{0.709384in}}%
\pgfpathlineto{\pgfqpoint{2.740831in}{0.708717in}}%
\pgfpathlineto{\pgfqpoint{2.744268in}{0.709866in}}%
\pgfpathlineto{\pgfqpoint{2.744571in}{0.712203in}}%
\pgfpathlineto{\pgfqpoint{2.745178in}{0.708996in}}%
\pgfpathlineto{\pgfqpoint{2.745279in}{0.709012in}}%
\pgfpathlineto{\pgfqpoint{2.745885in}{0.710906in}}%
\pgfpathlineto{\pgfqpoint{2.746189in}{0.711840in}}%
\pgfpathlineto{\pgfqpoint{2.746694in}{0.710022in}}%
\pgfpathlineto{\pgfqpoint{2.748008in}{0.709687in}}%
\pgfpathlineto{\pgfqpoint{2.750435in}{0.708752in}}%
\pgfpathlineto{\pgfqpoint{2.753366in}{0.709859in}}%
\pgfpathlineto{\pgfqpoint{2.753771in}{0.716331in}}%
\pgfpathlineto{\pgfqpoint{2.754478in}{0.710680in}}%
\pgfpathlineto{\pgfqpoint{2.755085in}{0.719642in}}%
\pgfpathlineto{\pgfqpoint{2.755287in}{0.722116in}}%
\pgfpathlineto{\pgfqpoint{2.755691in}{0.711047in}}%
\pgfpathlineto{\pgfqpoint{2.756399in}{0.712056in}}%
\pgfpathlineto{\pgfqpoint{2.757006in}{0.707301in}}%
\pgfpathlineto{\pgfqpoint{2.757612in}{0.707179in}}%
\pgfpathlineto{\pgfqpoint{2.757915in}{0.707696in}}%
\pgfpathlineto{\pgfqpoint{2.758522in}{0.707344in}}%
\pgfpathlineto{\pgfqpoint{2.759533in}{0.713350in}}%
\pgfpathlineto{\pgfqpoint{2.760241in}{0.711444in}}%
\pgfpathlineto{\pgfqpoint{2.760544in}{0.744894in}}%
\pgfpathlineto{\pgfqpoint{2.760847in}{0.811260in}}%
\pgfpathlineto{\pgfqpoint{2.761555in}{0.715015in}}%
\pgfpathlineto{\pgfqpoint{2.761959in}{0.723062in}}%
\pgfpathlineto{\pgfqpoint{2.762364in}{0.712505in}}%
\pgfpathlineto{\pgfqpoint{2.763577in}{0.708632in}}%
\pgfpathlineto{\pgfqpoint{2.770148in}{0.709482in}}%
\pgfpathlineto{\pgfqpoint{2.771361in}{0.709178in}}%
\pgfpathlineto{\pgfqpoint{2.773181in}{0.709884in}}%
\pgfpathlineto{\pgfqpoint{2.773484in}{0.711142in}}%
\pgfpathlineto{\pgfqpoint{2.774394in}{0.710592in}}%
\pgfpathlineto{\pgfqpoint{2.775101in}{0.710304in}}%
\pgfpathlineto{\pgfqpoint{2.775304in}{0.710023in}}%
\pgfpathlineto{\pgfqpoint{2.776921in}{0.709330in}}%
\pgfpathlineto{\pgfqpoint{2.778640in}{0.710095in}}%
\pgfpathlineto{\pgfqpoint{2.779246in}{0.713367in}}%
\pgfpathlineto{\pgfqpoint{2.779853in}{0.711166in}}%
\pgfpathlineto{\pgfqpoint{2.781773in}{0.708853in}}%
\pgfpathlineto{\pgfqpoint{2.783391in}{0.710206in}}%
\pgfpathlineto{\pgfqpoint{2.783896in}{0.726214in}}%
\pgfpathlineto{\pgfqpoint{2.784503in}{0.710906in}}%
\pgfpathlineto{\pgfqpoint{2.786323in}{0.708106in}}%
\pgfpathlineto{\pgfqpoint{2.784907in}{0.712208in}}%
\pgfpathlineto{\pgfqpoint{2.786525in}{0.709005in}}%
\pgfpathlineto{\pgfqpoint{2.786828in}{0.713353in}}%
\pgfpathlineto{\pgfqpoint{2.787536in}{0.708629in}}%
\pgfpathlineto{\pgfqpoint{2.788345in}{0.709249in}}%
\pgfpathlineto{\pgfqpoint{2.788850in}{0.719761in}}%
\pgfpathlineto{\pgfqpoint{2.789355in}{0.709267in}}%
\pgfpathlineto{\pgfqpoint{2.790366in}{0.707811in}}%
\pgfpathlineto{\pgfqpoint{2.790670in}{0.708306in}}%
\pgfpathlineto{\pgfqpoint{2.791074in}{0.710107in}}%
\pgfpathlineto{\pgfqpoint{2.791580in}{0.707527in}}%
\pgfpathlineto{\pgfqpoint{2.792489in}{0.707635in}}%
\pgfpathlineto{\pgfqpoint{2.792692in}{0.708107in}}%
\pgfpathlineto{\pgfqpoint{2.794208in}{0.715456in}}%
\pgfpathlineto{\pgfqpoint{2.794815in}{0.759742in}}%
\pgfpathlineto{\pgfqpoint{2.795421in}{0.724419in}}%
\pgfpathlineto{\pgfqpoint{2.797241in}{0.708339in}}%
\pgfpathlineto{\pgfqpoint{2.798858in}{0.708025in}}%
\pgfpathlineto{\pgfqpoint{2.803306in}{0.708788in}}%
\pgfpathlineto{\pgfqpoint{2.805429in}{0.709046in}}%
\pgfpathlineto{\pgfqpoint{2.811495in}{0.709270in}}%
\pgfpathlineto{\pgfqpoint{2.812203in}{0.710308in}}%
\pgfpathlineto{\pgfqpoint{2.813416in}{0.715727in}}%
\pgfpathlineto{\pgfqpoint{2.812910in}{0.709827in}}%
\pgfpathlineto{\pgfqpoint{2.813618in}{0.713895in}}%
\pgfpathlineto{\pgfqpoint{2.814326in}{0.709618in}}%
\pgfpathlineto{\pgfqpoint{2.814932in}{0.709694in}}%
\pgfpathlineto{\pgfqpoint{2.816145in}{0.709101in}}%
\pgfpathlineto{\pgfqpoint{2.818774in}{0.709528in}}%
\pgfpathlineto{\pgfqpoint{2.819077in}{0.711020in}}%
\pgfpathlineto{\pgfqpoint{2.819785in}{0.708936in}}%
\pgfpathlineto{\pgfqpoint{2.820290in}{0.709413in}}%
\pgfpathlineto{\pgfqpoint{2.822514in}{0.719837in}}%
\pgfpathlineto{\pgfqpoint{2.822716in}{0.717812in}}%
\pgfpathlineto{\pgfqpoint{2.823626in}{0.709603in}}%
\pgfpathlineto{\pgfqpoint{2.824031in}{0.710353in}}%
\pgfpathlineto{\pgfqpoint{2.824233in}{0.711115in}}%
\pgfpathlineto{\pgfqpoint{2.824738in}{0.708604in}}%
\pgfpathlineto{\pgfqpoint{2.825345in}{0.709396in}}%
\pgfpathlineto{\pgfqpoint{2.826153in}{0.708528in}}%
\pgfpathlineto{\pgfqpoint{2.827771in}{0.709688in}}%
\pgfpathlineto{\pgfqpoint{2.828276in}{0.717158in}}%
\pgfpathlineto{\pgfqpoint{2.828782in}{0.709406in}}%
\pgfpathlineto{\pgfqpoint{2.829894in}{0.708430in}}%
\pgfpathlineto{\pgfqpoint{2.829085in}{0.709677in}}%
\pgfpathlineto{\pgfqpoint{2.830197in}{0.709290in}}%
\pgfpathlineto{\pgfqpoint{2.830905in}{0.708350in}}%
\pgfpathlineto{\pgfqpoint{2.831511in}{0.710701in}}%
\pgfpathlineto{\pgfqpoint{2.831916in}{0.709978in}}%
\pgfpathlineto{\pgfqpoint{2.832219in}{0.711049in}}%
\pgfpathlineto{\pgfqpoint{2.833634in}{0.723121in}}%
\pgfpathlineto{\pgfqpoint{2.833938in}{0.734577in}}%
\pgfpathlineto{\pgfqpoint{2.834443in}{0.711498in}}%
\pgfpathlineto{\pgfqpoint{2.834544in}{0.710551in}}%
\pgfpathlineto{\pgfqpoint{2.834746in}{0.714701in}}%
\pgfpathlineto{\pgfqpoint{2.835050in}{0.726095in}}%
\pgfpathlineto{\pgfqpoint{2.835656in}{0.707499in}}%
\pgfpathlineto{\pgfqpoint{2.835858in}{0.707916in}}%
\pgfpathlineto{\pgfqpoint{2.836364in}{0.725146in}}%
\pgfpathlineto{\pgfqpoint{2.836667in}{0.735971in}}%
\pgfpathlineto{\pgfqpoint{2.837274in}{0.720566in}}%
\pgfpathlineto{\pgfqpoint{2.837375in}{0.720595in}}%
\pgfpathlineto{\pgfqpoint{2.837880in}{0.731666in}}%
\pgfpathlineto{\pgfqpoint{2.838285in}{0.719611in}}%
\pgfpathlineto{\pgfqpoint{2.838992in}{0.707799in}}%
\pgfpathlineto{\pgfqpoint{2.839599in}{0.709003in}}%
\pgfpathlineto{\pgfqpoint{2.840205in}{0.706975in}}%
\pgfpathlineto{\pgfqpoint{2.840610in}{0.708661in}}%
\pgfpathlineto{\pgfqpoint{2.842126in}{0.730127in}}%
\pgfpathlineto{\pgfqpoint{2.842430in}{0.768470in}}%
\pgfpathlineto{\pgfqpoint{2.843036in}{0.710700in}}%
\pgfpathlineto{\pgfqpoint{2.843137in}{0.709989in}}%
\pgfpathlineto{\pgfqpoint{2.843845in}{0.712485in}}%
\pgfpathlineto{\pgfqpoint{2.843946in}{0.712643in}}%
\pgfpathlineto{\pgfqpoint{2.844148in}{0.711614in}}%
\pgfpathlineto{\pgfqpoint{2.844654in}{0.707945in}}%
\pgfpathlineto{\pgfqpoint{2.845058in}{0.712942in}}%
\pgfpathlineto{\pgfqpoint{2.845159in}{0.713779in}}%
\pgfpathlineto{\pgfqpoint{2.845563in}{0.708064in}}%
\pgfpathlineto{\pgfqpoint{2.845665in}{0.707793in}}%
\pgfpathlineto{\pgfqpoint{2.845968in}{0.709685in}}%
\pgfpathlineto{\pgfqpoint{2.846271in}{0.714413in}}%
\pgfpathlineto{\pgfqpoint{2.846979in}{0.708716in}}%
\pgfpathlineto{\pgfqpoint{2.848192in}{0.708139in}}%
\pgfpathlineto{\pgfqpoint{2.848293in}{0.708276in}}%
\pgfpathlineto{\pgfqpoint{2.848697in}{0.715163in}}%
\pgfpathlineto{\pgfqpoint{2.849001in}{0.722868in}}%
\pgfpathlineto{\pgfqpoint{2.849405in}{0.714265in}}%
\pgfpathlineto{\pgfqpoint{2.849910in}{0.720573in}}%
\pgfpathlineto{\pgfqpoint{2.850618in}{0.706452in}}%
\pgfpathlineto{\pgfqpoint{2.851326in}{0.706670in}}%
\pgfpathlineto{\pgfqpoint{2.851629in}{0.707519in}}%
\pgfpathlineto{\pgfqpoint{2.853044in}{0.710321in}}%
\pgfpathlineto{\pgfqpoint{2.853449in}{0.712895in}}%
\pgfpathlineto{\pgfqpoint{2.853853in}{0.709926in}}%
\pgfpathlineto{\pgfqpoint{2.854156in}{0.707555in}}%
\pgfpathlineto{\pgfqpoint{2.854561in}{0.713974in}}%
\pgfpathlineto{\pgfqpoint{2.855066in}{0.735194in}}%
\pgfpathlineto{\pgfqpoint{2.855774in}{0.719393in}}%
\pgfpathlineto{\pgfqpoint{2.855875in}{0.718599in}}%
\pgfpathlineto{\pgfqpoint{2.856077in}{0.725187in}}%
\pgfpathlineto{\pgfqpoint{2.856583in}{0.800337in}}%
\pgfpathlineto{\pgfqpoint{2.857189in}{0.737499in}}%
\pgfpathlineto{\pgfqpoint{2.858402in}{0.720964in}}%
\pgfpathlineto{\pgfqpoint{2.858503in}{0.721024in}}%
\pgfpathlineto{\pgfqpoint{2.858807in}{0.718220in}}%
\pgfpathlineto{\pgfqpoint{2.860323in}{0.709178in}}%
\pgfpathlineto{\pgfqpoint{2.860829in}{0.708581in}}%
\pgfpathlineto{\pgfqpoint{2.861233in}{0.709891in}}%
\pgfpathlineto{\pgfqpoint{2.862446in}{0.711593in}}%
\pgfpathlineto{\pgfqpoint{2.861738in}{0.708984in}}%
\pgfpathlineto{\pgfqpoint{2.862648in}{0.710974in}}%
\pgfpathlineto{\pgfqpoint{2.862850in}{0.710552in}}%
\pgfpathlineto{\pgfqpoint{2.863356in}{0.711792in}}%
\pgfpathlineto{\pgfqpoint{2.863760in}{0.710999in}}%
\pgfpathlineto{\pgfqpoint{2.864064in}{0.720585in}}%
\pgfpathlineto{\pgfqpoint{2.864367in}{0.735155in}}%
\pgfpathlineto{\pgfqpoint{2.864973in}{0.710020in}}%
\pgfpathlineto{\pgfqpoint{2.865580in}{0.710738in}}%
\pgfpathlineto{\pgfqpoint{2.866793in}{0.708261in}}%
\pgfpathlineto{\pgfqpoint{2.872050in}{0.709485in}}%
\pgfpathlineto{\pgfqpoint{2.872656in}{0.712206in}}%
\pgfpathlineto{\pgfqpoint{2.873263in}{0.710027in}}%
\pgfpathlineto{\pgfqpoint{2.873667in}{0.710945in}}%
\pgfpathlineto{\pgfqpoint{2.873971in}{0.712030in}}%
\pgfpathlineto{\pgfqpoint{2.874375in}{0.708690in}}%
\pgfpathlineto{\pgfqpoint{2.875386in}{0.708066in}}%
\pgfpathlineto{\pgfqpoint{2.875588in}{0.708127in}}%
\pgfpathlineto{\pgfqpoint{2.875993in}{0.709432in}}%
\pgfpathlineto{\pgfqpoint{2.877408in}{0.713446in}}%
\pgfpathlineto{\pgfqpoint{2.877509in}{0.713429in}}%
\pgfpathlineto{\pgfqpoint{2.879025in}{0.708118in}}%
\pgfpathlineto{\pgfqpoint{2.879430in}{0.708289in}}%
\pgfpathlineto{\pgfqpoint{2.880036in}{0.708266in}}%
\pgfpathlineto{\pgfqpoint{2.880137in}{0.708577in}}%
\pgfpathlineto{\pgfqpoint{2.880542in}{0.710286in}}%
\pgfpathlineto{\pgfqpoint{2.881351in}{0.709257in}}%
\pgfpathlineto{\pgfqpoint{2.881856in}{0.708977in}}%
\pgfpathlineto{\pgfqpoint{2.882058in}{0.709622in}}%
\pgfpathlineto{\pgfqpoint{2.882463in}{0.712297in}}%
\pgfpathlineto{\pgfqpoint{2.883069in}{0.708913in}}%
\pgfpathlineto{\pgfqpoint{2.884788in}{0.708222in}}%
\pgfpathlineto{\pgfqpoint{2.884889in}{0.709073in}}%
\pgfpathlineto{\pgfqpoint{2.886506in}{0.740281in}}%
\pgfpathlineto{\pgfqpoint{2.886911in}{0.727324in}}%
\pgfpathlineto{\pgfqpoint{2.887416in}{0.713864in}}%
\pgfpathlineto{\pgfqpoint{2.888124in}{0.721619in}}%
\pgfpathlineto{\pgfqpoint{2.888831in}{0.707742in}}%
\pgfpathlineto{\pgfqpoint{2.889337in}{0.715968in}}%
\pgfpathlineto{\pgfqpoint{2.889438in}{0.717636in}}%
\pgfpathlineto{\pgfqpoint{2.889943in}{0.709003in}}%
\pgfpathlineto{\pgfqpoint{2.890045in}{0.708491in}}%
\pgfpathlineto{\pgfqpoint{2.890449in}{0.712037in}}%
\pgfpathlineto{\pgfqpoint{2.890752in}{0.714333in}}%
\pgfpathlineto{\pgfqpoint{2.891359in}{0.711015in}}%
\pgfpathlineto{\pgfqpoint{2.891763in}{0.708524in}}%
\pgfpathlineto{\pgfqpoint{2.892572in}{0.709623in}}%
\pgfpathlineto{\pgfqpoint{2.893381in}{0.706766in}}%
\pgfpathlineto{\pgfqpoint{2.893583in}{0.708368in}}%
\pgfpathlineto{\pgfqpoint{2.894594in}{0.708016in}}%
\pgfpathlineto{\pgfqpoint{2.895200in}{0.725022in}}%
\pgfpathlineto{\pgfqpoint{2.896919in}{0.706328in}}%
\pgfpathlineto{\pgfqpoint{2.898233in}{0.708015in}}%
\pgfpathlineto{\pgfqpoint{2.899750in}{0.714909in}}%
\pgfpathlineto{\pgfqpoint{2.899143in}{0.707446in}}%
\pgfpathlineto{\pgfqpoint{2.899952in}{0.713360in}}%
\pgfpathlineto{\pgfqpoint{2.900457in}{0.707074in}}%
\pgfpathlineto{\pgfqpoint{2.900862in}{0.714123in}}%
\pgfpathlineto{\pgfqpoint{2.901064in}{0.717583in}}%
\pgfpathlineto{\pgfqpoint{2.901670in}{0.706753in}}%
\pgfpathlineto{\pgfqpoint{2.902176in}{0.709777in}}%
\pgfpathlineto{\pgfqpoint{2.902985in}{0.707060in}}%
\pgfpathlineto{\pgfqpoint{2.903894in}{0.707260in}}%
\pgfpathlineto{\pgfqpoint{2.903995in}{0.707584in}}%
\pgfpathlineto{\pgfqpoint{2.904400in}{0.709220in}}%
\pgfpathlineto{\pgfqpoint{2.904804in}{0.706548in}}%
\pgfpathlineto{\pgfqpoint{2.905411in}{0.707448in}}%
\pgfpathlineto{\pgfqpoint{2.906017in}{0.706616in}}%
\pgfpathlineto{\pgfqpoint{2.906523in}{0.708094in}}%
\pgfpathlineto{\pgfqpoint{2.906826in}{0.709354in}}%
\pgfpathlineto{\pgfqpoint{2.907534in}{0.707988in}}%
\pgfpathlineto{\pgfqpoint{2.907837in}{0.708343in}}%
\pgfpathlineto{\pgfqpoint{2.907938in}{0.708833in}}%
\pgfpathlineto{\pgfqpoint{2.908545in}{0.719284in}}%
\pgfpathlineto{\pgfqpoint{2.908949in}{0.709043in}}%
\pgfpathlineto{\pgfqpoint{2.909151in}{0.707228in}}%
\pgfpathlineto{\pgfqpoint{2.909454in}{0.713723in}}%
\pgfpathlineto{\pgfqpoint{2.909859in}{0.729872in}}%
\pgfpathlineto{\pgfqpoint{2.910465in}{0.708695in}}%
\pgfpathlineto{\pgfqpoint{2.910668in}{0.712151in}}%
\pgfpathlineto{\pgfqpoint{2.911173in}{0.762665in}}%
\pgfpathlineto{\pgfqpoint{2.911780in}{0.715681in}}%
\pgfpathlineto{\pgfqpoint{2.912487in}{0.720713in}}%
\pgfpathlineto{\pgfqpoint{2.913397in}{0.707462in}}%
\pgfpathlineto{\pgfqpoint{2.915621in}{0.708363in}}%
\pgfpathlineto{\pgfqpoint{2.915924in}{0.709484in}}%
\pgfpathlineto{\pgfqpoint{2.916733in}{0.708503in}}%
\pgfpathlineto{\pgfqpoint{2.917441in}{0.709396in}}%
\pgfpathlineto{\pgfqpoint{2.917946in}{0.708230in}}%
\pgfpathlineto{\pgfqpoint{2.919463in}{0.708019in}}%
\pgfpathlineto{\pgfqpoint{2.920373in}{0.709105in}}%
\pgfpathlineto{\pgfqpoint{2.920777in}{0.711228in}}%
\pgfpathlineto{\pgfqpoint{2.921586in}{0.710128in}}%
\pgfpathlineto{\pgfqpoint{2.921990in}{0.710088in}}%
\pgfpathlineto{\pgfqpoint{2.922192in}{0.710971in}}%
\pgfpathlineto{\pgfqpoint{2.922496in}{0.712785in}}%
\pgfpathlineto{\pgfqpoint{2.923001in}{0.709026in}}%
\pgfpathlineto{\pgfqpoint{2.923405in}{0.708250in}}%
\pgfpathlineto{\pgfqpoint{2.924113in}{0.708717in}}%
\pgfpathlineto{\pgfqpoint{2.925326in}{0.710383in}}%
\pgfpathlineto{\pgfqpoint{2.925629in}{0.712101in}}%
\pgfpathlineto{\pgfqpoint{2.926337in}{0.709634in}}%
\pgfpathlineto{\pgfqpoint{2.926741in}{0.712080in}}%
\pgfpathlineto{\pgfqpoint{2.926944in}{0.713863in}}%
\pgfpathlineto{\pgfqpoint{2.927449in}{0.708434in}}%
\pgfpathlineto{\pgfqpoint{2.928157in}{0.708819in}}%
\pgfpathlineto{\pgfqpoint{2.928662in}{0.708162in}}%
\pgfpathlineto{\pgfqpoint{2.932099in}{0.708928in}}%
\pgfpathlineto{\pgfqpoint{2.933919in}{0.720942in}}%
\pgfpathlineto{\pgfqpoint{2.934121in}{0.715140in}}%
\pgfpathlineto{\pgfqpoint{2.935436in}{0.707571in}}%
\pgfpathlineto{\pgfqpoint{2.936548in}{0.708654in}}%
\pgfpathlineto{\pgfqpoint{2.936851in}{0.714473in}}%
\pgfpathlineto{\pgfqpoint{2.937457in}{0.707999in}}%
\pgfpathlineto{\pgfqpoint{2.937660in}{0.708901in}}%
\pgfpathlineto{\pgfqpoint{2.937862in}{0.709841in}}%
\pgfpathlineto{\pgfqpoint{2.938468in}{0.707574in}}%
\pgfpathlineto{\pgfqpoint{2.938569in}{0.707578in}}%
\pgfpathlineto{\pgfqpoint{2.940389in}{0.708830in}}%
\pgfpathlineto{\pgfqpoint{2.940591in}{0.709711in}}%
\pgfpathlineto{\pgfqpoint{2.941299in}{0.707689in}}%
\pgfpathlineto{\pgfqpoint{2.942815in}{0.708694in}}%
\pgfpathlineto{\pgfqpoint{2.943220in}{0.717004in}}%
\pgfpathlineto{\pgfqpoint{2.943927in}{0.710046in}}%
\pgfpathlineto{\pgfqpoint{2.944938in}{0.708575in}}%
\pgfpathlineto{\pgfqpoint{2.946252in}{0.708081in}}%
\pgfpathlineto{\pgfqpoint{2.946960in}{0.709280in}}%
\pgfpathlineto{\pgfqpoint{2.948477in}{0.718797in}}%
\pgfpathlineto{\pgfqpoint{2.948679in}{0.722957in}}%
\pgfpathlineto{\pgfqpoint{2.949184in}{0.710365in}}%
\pgfpathlineto{\pgfqpoint{2.949791in}{0.708009in}}%
\pgfpathlineto{\pgfqpoint{2.950296in}{0.709477in}}%
\pgfpathlineto{\pgfqpoint{2.950397in}{0.709941in}}%
\pgfpathlineto{\pgfqpoint{2.950903in}{0.707501in}}%
\pgfpathlineto{\pgfqpoint{2.951105in}{0.707590in}}%
\pgfpathlineto{\pgfqpoint{2.951206in}{0.708253in}}%
\pgfpathlineto{\pgfqpoint{2.951610in}{0.715172in}}%
\pgfpathlineto{\pgfqpoint{2.952116in}{0.708185in}}%
\pgfpathlineto{\pgfqpoint{2.952318in}{0.708872in}}%
\pgfpathlineto{\pgfqpoint{2.953026in}{0.716784in}}%
\pgfpathlineto{\pgfqpoint{2.953430in}{0.733600in}}%
\pgfpathlineto{\pgfqpoint{2.954340in}{0.727608in}}%
\pgfpathlineto{\pgfqpoint{2.954441in}{0.727741in}}%
\pgfpathlineto{\pgfqpoint{2.955957in}{0.716059in}}%
\pgfpathlineto{\pgfqpoint{2.955351in}{0.728328in}}%
\pgfpathlineto{\pgfqpoint{2.956059in}{0.718298in}}%
\pgfpathlineto{\pgfqpoint{2.956362in}{0.738298in}}%
\pgfpathlineto{\pgfqpoint{2.956968in}{0.707685in}}%
\pgfpathlineto{\pgfqpoint{2.957171in}{0.707662in}}%
\pgfpathlineto{\pgfqpoint{2.957373in}{0.708395in}}%
\pgfpathlineto{\pgfqpoint{2.957575in}{0.708883in}}%
\pgfpathlineto{\pgfqpoint{2.958182in}{0.707980in}}%
\pgfpathlineto{\pgfqpoint{2.958384in}{0.707965in}}%
\pgfpathlineto{\pgfqpoint{2.958990in}{0.708768in}}%
\pgfpathlineto{\pgfqpoint{2.960810in}{0.722326in}}%
\pgfpathlineto{\pgfqpoint{2.961012in}{0.728153in}}%
\pgfpathlineto{\pgfqpoint{2.961619in}{0.708211in}}%
\pgfpathlineto{\pgfqpoint{2.961922in}{0.707860in}}%
\pgfpathlineto{\pgfqpoint{2.962326in}{0.709027in}}%
\pgfpathlineto{\pgfqpoint{2.962630in}{0.710659in}}%
\pgfpathlineto{\pgfqpoint{2.963135in}{0.707467in}}%
\pgfpathlineto{\pgfqpoint{2.963641in}{0.708286in}}%
\pgfpathlineto{\pgfqpoint{2.964449in}{0.707509in}}%
\pgfpathlineto{\pgfqpoint{2.965966in}{0.708578in}}%
\pgfpathlineto{\pgfqpoint{2.966471in}{0.707382in}}%
\pgfpathlineto{\pgfqpoint{2.967583in}{0.708193in}}%
\pgfpathlineto{\pgfqpoint{2.968796in}{0.713809in}}%
\pgfpathlineto{\pgfqpoint{2.969302in}{0.766384in}}%
\pgfpathlineto{\pgfqpoint{2.969908in}{0.719294in}}%
\pgfpathlineto{\pgfqpoint{2.970009in}{0.718805in}}%
\pgfpathlineto{\pgfqpoint{2.970212in}{0.722482in}}%
\pgfpathlineto{\pgfqpoint{2.970515in}{0.728510in}}%
\pgfpathlineto{\pgfqpoint{2.971121in}{0.717461in}}%
\pgfpathlineto{\pgfqpoint{2.972638in}{0.709061in}}%
\pgfpathlineto{\pgfqpoint{2.973042in}{0.708103in}}%
\pgfpathlineto{\pgfqpoint{2.973851in}{0.708363in}}%
\pgfpathlineto{\pgfqpoint{2.977693in}{0.708888in}}%
\pgfpathlineto{\pgfqpoint{2.980119in}{0.709552in}}%
\pgfpathlineto{\pgfqpoint{2.980523in}{0.718977in}}%
\pgfpathlineto{\pgfqpoint{2.981231in}{0.710428in}}%
\pgfpathlineto{\pgfqpoint{2.981736in}{0.712745in}}%
\pgfpathlineto{\pgfqpoint{2.982141in}{0.710288in}}%
\pgfpathlineto{\pgfqpoint{2.983455in}{0.708992in}}%
\pgfpathlineto{\pgfqpoint{2.985275in}{0.708973in}}%
\pgfpathlineto{\pgfqpoint{2.988813in}{0.710700in}}%
\pgfpathlineto{\pgfqpoint{2.989318in}{0.714903in}}%
\pgfpathlineto{\pgfqpoint{2.989925in}{0.711710in}}%
\pgfpathlineto{\pgfqpoint{2.990936in}{0.711400in}}%
\pgfpathlineto{\pgfqpoint{2.991037in}{0.711574in}}%
\pgfpathlineto{\pgfqpoint{2.991340in}{0.712111in}}%
\pgfpathlineto{\pgfqpoint{2.991745in}{0.710679in}}%
\pgfpathlineto{\pgfqpoint{2.992452in}{0.709669in}}%
\pgfpathlineto{\pgfqpoint{2.992958in}{0.710210in}}%
\pgfpathlineto{\pgfqpoint{2.993261in}{0.709420in}}%
\pgfpathlineto{\pgfqpoint{2.993665in}{0.708429in}}%
\pgfpathlineto{\pgfqpoint{2.994070in}{0.710981in}}%
\pgfpathlineto{\pgfqpoint{2.995485in}{0.720078in}}%
\pgfpathlineto{\pgfqpoint{2.996092in}{0.713811in}}%
\pgfpathlineto{\pgfqpoint{2.996496in}{0.718887in}}%
\pgfpathlineto{\pgfqpoint{2.996799in}{0.722974in}}%
\pgfpathlineto{\pgfqpoint{2.997709in}{0.721453in}}%
\pgfpathlineto{\pgfqpoint{2.999428in}{0.707556in}}%
\pgfpathlineto{\pgfqpoint{2.999630in}{0.707599in}}%
\pgfpathlineto{\pgfqpoint{3.000944in}{0.709018in}}%
\pgfpathlineto{\pgfqpoint{3.002258in}{0.719451in}}%
\pgfpathlineto{\pgfqpoint{3.002562in}{1.134334in}}%
\pgfpathlineto{\pgfqpoint{3.002663in}{1.296194in}}%
\pgfpathlineto{\pgfqpoint{3.003370in}{0.797896in}}%
\pgfpathlineto{\pgfqpoint{3.003674in}{0.732792in}}%
\pgfpathlineto{\pgfqpoint{3.004078in}{0.923322in}}%
\pgfpathlineto{\pgfqpoint{3.004179in}{0.962622in}}%
\pgfpathlineto{\pgfqpoint{3.004685in}{0.766915in}}%
\pgfpathlineto{\pgfqpoint{3.005999in}{0.712391in}}%
\pgfpathlineto{\pgfqpoint{3.006706in}{0.709717in}}%
\pgfpathlineto{\pgfqpoint{3.007111in}{0.710916in}}%
\pgfpathlineto{\pgfqpoint{3.008223in}{0.721269in}}%
\pgfpathlineto{\pgfqpoint{3.007616in}{0.710715in}}%
\pgfpathlineto{\pgfqpoint{3.008425in}{0.714448in}}%
\pgfpathlineto{\pgfqpoint{3.008728in}{0.709683in}}%
\pgfpathlineto{\pgfqpoint{3.009234in}{0.719040in}}%
\pgfpathlineto{\pgfqpoint{3.009638in}{0.710518in}}%
\pgfpathlineto{\pgfqpoint{3.010245in}{0.713134in}}%
\pgfpathlineto{\pgfqpoint{3.010952in}{0.709428in}}%
\pgfpathlineto{\pgfqpoint{3.017220in}{0.709289in}}%
\pgfpathlineto{\pgfqpoint{3.018130in}{0.709504in}}%
\pgfpathlineto{\pgfqpoint{3.018231in}{0.710074in}}%
\pgfpathlineto{\pgfqpoint{3.018534in}{0.712451in}}%
\pgfpathlineto{\pgfqpoint{3.019343in}{0.710646in}}%
\pgfpathlineto{\pgfqpoint{3.020051in}{0.709163in}}%
\pgfpathlineto{\pgfqpoint{3.020455in}{0.710576in}}%
\pgfpathlineto{\pgfqpoint{3.022073in}{0.734429in}}%
\pgfpathlineto{\pgfqpoint{3.022578in}{0.789059in}}%
\pgfpathlineto{\pgfqpoint{3.023084in}{0.727518in}}%
\pgfpathlineto{\pgfqpoint{3.023185in}{0.725618in}}%
\pgfpathlineto{\pgfqpoint{3.023286in}{0.730377in}}%
\pgfpathlineto{\pgfqpoint{3.023690in}{0.811900in}}%
\pgfpathlineto{\pgfqpoint{3.024297in}{0.731476in}}%
\pgfpathlineto{\pgfqpoint{3.024701in}{0.717141in}}%
\pgfpathlineto{\pgfqpoint{3.025206in}{0.734777in}}%
\pgfpathlineto{\pgfqpoint{3.025409in}{0.743448in}}%
\pgfpathlineto{\pgfqpoint{3.025914in}{0.720912in}}%
\pgfpathlineto{\pgfqpoint{3.026318in}{0.712777in}}%
\pgfpathlineto{\pgfqpoint{3.026723in}{0.725463in}}%
\pgfpathlineto{\pgfqpoint{3.027026in}{0.738275in}}%
\pgfpathlineto{\pgfqpoint{3.027633in}{0.716299in}}%
\pgfpathlineto{\pgfqpoint{3.027936in}{0.713804in}}%
\pgfpathlineto{\pgfqpoint{3.028543in}{0.717935in}}%
\pgfpathlineto{\pgfqpoint{3.028644in}{0.718308in}}%
\pgfpathlineto{\pgfqpoint{3.028947in}{0.716485in}}%
\pgfpathlineto{\pgfqpoint{3.030564in}{0.709871in}}%
\pgfpathlineto{\pgfqpoint{3.030969in}{0.709042in}}%
\pgfpathlineto{\pgfqpoint{3.031474in}{0.710839in}}%
\pgfpathlineto{\pgfqpoint{3.032081in}{0.709453in}}%
\pgfpathlineto{\pgfqpoint{3.032485in}{0.711268in}}%
\pgfpathlineto{\pgfqpoint{3.032788in}{0.714276in}}%
\pgfpathlineto{\pgfqpoint{3.033395in}{0.709162in}}%
\pgfpathlineto{\pgfqpoint{3.033698in}{0.709871in}}%
\pgfpathlineto{\pgfqpoint{3.033901in}{0.710676in}}%
\pgfpathlineto{\pgfqpoint{3.034507in}{0.708528in}}%
\pgfpathlineto{\pgfqpoint{3.035922in}{0.708784in}}%
\pgfpathlineto{\pgfqpoint{3.036125in}{0.714283in}}%
\pgfpathlineto{\pgfqpoint{3.037338in}{0.966683in}}%
\pgfpathlineto{\pgfqpoint{3.037742in}{0.796631in}}%
\pgfpathlineto{\pgfqpoint{3.039056in}{0.718756in}}%
\pgfpathlineto{\pgfqpoint{3.040269in}{0.709543in}}%
\pgfpathlineto{\pgfqpoint{3.048054in}{0.710374in}}%
\pgfpathlineto{\pgfqpoint{3.048963in}{0.711005in}}%
\pgfpathlineto{\pgfqpoint{3.048458in}{0.709981in}}%
\pgfpathlineto{\pgfqpoint{3.049166in}{0.710412in}}%
\pgfpathlineto{\pgfqpoint{3.050379in}{0.709395in}}%
\pgfpathlineto{\pgfqpoint{3.050480in}{0.709450in}}%
\pgfpathlineto{\pgfqpoint{3.051895in}{0.710181in}}%
\pgfpathlineto{\pgfqpoint{3.052097in}{0.710026in}}%
\pgfpathlineto{\pgfqpoint{3.053614in}{0.709412in}}%
\pgfpathlineto{\pgfqpoint{3.053816in}{0.710028in}}%
\pgfpathlineto{\pgfqpoint{3.055332in}{0.714933in}}%
\pgfpathlineto{\pgfqpoint{3.055433in}{0.714745in}}%
\pgfpathlineto{\pgfqpoint{3.056849in}{0.708944in}}%
\pgfpathlineto{\pgfqpoint{3.056950in}{0.708955in}}%
\pgfpathlineto{\pgfqpoint{3.059174in}{0.709828in}}%
\pgfpathlineto{\pgfqpoint{3.060387in}{0.732347in}}%
\pgfpathlineto{\pgfqpoint{3.060690in}{0.805705in}}%
\pgfpathlineto{\pgfqpoint{3.061398in}{0.717321in}}%
\pgfpathlineto{\pgfqpoint{3.061499in}{0.717553in}}%
\pgfpathlineto{\pgfqpoint{3.061903in}{0.731273in}}%
\pgfpathlineto{\pgfqpoint{3.062308in}{0.712159in}}%
\pgfpathlineto{\pgfqpoint{3.062712in}{0.709031in}}%
\pgfpathlineto{\pgfqpoint{3.063521in}{0.709071in}}%
\pgfpathlineto{\pgfqpoint{3.064229in}{0.709270in}}%
\pgfpathlineto{\pgfqpoint{3.064633in}{0.719651in}}%
\pgfpathlineto{\pgfqpoint{3.064936in}{0.727952in}}%
\pgfpathlineto{\pgfqpoint{3.065543in}{0.717079in}}%
\pgfpathlineto{\pgfqpoint{3.065644in}{0.717935in}}%
\pgfpathlineto{\pgfqpoint{3.066857in}{0.728488in}}%
\pgfpathlineto{\pgfqpoint{3.067160in}{0.725146in}}%
\pgfpathlineto{\pgfqpoint{3.067565in}{0.719154in}}%
\pgfpathlineto{\pgfqpoint{3.067969in}{0.730872in}}%
\pgfpathlineto{\pgfqpoint{3.068171in}{0.736968in}}%
\pgfpathlineto{\pgfqpoint{3.068576in}{0.716484in}}%
\pgfpathlineto{\pgfqpoint{3.068879in}{0.709227in}}%
\pgfpathlineto{\pgfqpoint{3.069485in}{0.719212in}}%
\pgfpathlineto{\pgfqpoint{3.069688in}{0.715165in}}%
\pgfpathlineto{\pgfqpoint{3.070092in}{0.709141in}}%
\pgfpathlineto{\pgfqpoint{3.070901in}{0.711182in}}%
\pgfpathlineto{\pgfqpoint{3.072316in}{0.708531in}}%
\pgfpathlineto{\pgfqpoint{3.074540in}{0.709240in}}%
\pgfpathlineto{\pgfqpoint{3.075854in}{0.718481in}}%
\pgfpathlineto{\pgfqpoint{3.076158in}{0.712673in}}%
\pgfpathlineto{\pgfqpoint{3.076966in}{0.709483in}}%
\pgfpathlineto{\pgfqpoint{3.077371in}{0.710154in}}%
\pgfpathlineto{\pgfqpoint{3.078786in}{0.713411in}}%
\pgfpathlineto{\pgfqpoint{3.078887in}{0.713113in}}%
\pgfpathlineto{\pgfqpoint{3.079797in}{0.707726in}}%
\pgfpathlineto{\pgfqpoint{3.080606in}{0.708429in}}%
\pgfpathlineto{\pgfqpoint{3.080909in}{0.708508in}}%
\pgfpathlineto{\pgfqpoint{3.081212in}{0.709361in}}%
\pgfpathlineto{\pgfqpoint{3.081313in}{0.709510in}}%
\pgfpathlineto{\pgfqpoint{3.081617in}{0.708237in}}%
\pgfpathlineto{\pgfqpoint{3.081920in}{0.707342in}}%
\pgfpathlineto{\pgfqpoint{3.082324in}{0.708715in}}%
\pgfpathlineto{\pgfqpoint{3.082729in}{0.712576in}}%
\pgfpathlineto{\pgfqpoint{3.083436in}{0.709437in}}%
\pgfpathlineto{\pgfqpoint{3.083740in}{0.710051in}}%
\pgfpathlineto{\pgfqpoint{3.084245in}{0.708608in}}%
\pgfpathlineto{\pgfqpoint{3.085761in}{0.707334in}}%
\pgfpathlineto{\pgfqpoint{3.086166in}{0.707326in}}%
\pgfpathlineto{\pgfqpoint{3.086368in}{0.708076in}}%
\pgfpathlineto{\pgfqpoint{3.087783in}{0.716583in}}%
\pgfpathlineto{\pgfqpoint{3.088087in}{0.713127in}}%
\pgfpathlineto{\pgfqpoint{3.089300in}{0.707033in}}%
\pgfpathlineto{\pgfqpoint{3.089502in}{0.707104in}}%
\pgfpathlineto{\pgfqpoint{3.090614in}{0.708475in}}%
\pgfpathlineto{\pgfqpoint{3.091018in}{0.711346in}}%
\pgfpathlineto{\pgfqpoint{3.091524in}{0.706697in}}%
\pgfpathlineto{\pgfqpoint{3.092939in}{0.706421in}}%
\pgfpathlineto{\pgfqpoint{3.094961in}{0.707551in}}%
\pgfpathlineto{\pgfqpoint{3.096578in}{0.725194in}}%
\pgfpathlineto{\pgfqpoint{3.096781in}{0.717175in}}%
\pgfpathlineto{\pgfqpoint{3.097185in}{0.707359in}}%
\pgfpathlineto{\pgfqpoint{3.097589in}{0.718864in}}%
\pgfpathlineto{\pgfqpoint{3.097994in}{0.743553in}}%
\pgfpathlineto{\pgfqpoint{3.098600in}{0.713426in}}%
\pgfpathlineto{\pgfqpoint{3.098904in}{0.718190in}}%
\pgfpathlineto{\pgfqpoint{3.099106in}{0.721525in}}%
\pgfpathlineto{\pgfqpoint{3.099712in}{0.711478in}}%
\pgfpathlineto{\pgfqpoint{3.099813in}{0.711283in}}%
\pgfpathlineto{\pgfqpoint{3.100016in}{0.712471in}}%
\pgfpathlineto{\pgfqpoint{3.101431in}{0.724366in}}%
\pgfpathlineto{\pgfqpoint{3.101633in}{0.719707in}}%
\pgfpathlineto{\pgfqpoint{3.102240in}{0.706831in}}%
\pgfpathlineto{\pgfqpoint{3.102947in}{0.706924in}}%
\pgfpathlineto{\pgfqpoint{3.103352in}{0.707646in}}%
\pgfpathlineto{\pgfqpoint{3.105070in}{0.734764in}}%
\pgfpathlineto{\pgfqpoint{3.105374in}{0.761701in}}%
\pgfpathlineto{\pgfqpoint{3.105980in}{0.716864in}}%
\pgfpathlineto{\pgfqpoint{3.106688in}{0.720441in}}%
\pgfpathlineto{\pgfqpoint{3.107294in}{0.708205in}}%
\pgfpathlineto{\pgfqpoint{3.107497in}{0.708018in}}%
\pgfpathlineto{\pgfqpoint{3.107800in}{0.708842in}}%
\pgfpathlineto{\pgfqpoint{3.108710in}{0.712576in}}%
\pgfpathlineto{\pgfqpoint{3.109316in}{0.711617in}}%
\pgfpathlineto{\pgfqpoint{3.109619in}{0.713439in}}%
\pgfpathlineto{\pgfqpoint{3.110226in}{0.732297in}}%
\pgfpathlineto{\pgfqpoint{3.110833in}{0.716529in}}%
\pgfpathlineto{\pgfqpoint{3.111641in}{0.717447in}}%
\pgfpathlineto{\pgfqpoint{3.112450in}{0.707567in}}%
\pgfpathlineto{\pgfqpoint{3.114977in}{0.707360in}}%
\pgfpathlineto{\pgfqpoint{3.115786in}{0.708189in}}%
\pgfpathlineto{\pgfqpoint{3.118212in}{0.724295in}}%
\pgfpathlineto{\pgfqpoint{3.118516in}{0.721820in}}%
\pgfpathlineto{\pgfqpoint{3.119223in}{0.713545in}}%
\pgfpathlineto{\pgfqpoint{3.119729in}{0.720050in}}%
\pgfpathlineto{\pgfqpoint{3.119830in}{0.720754in}}%
\pgfpathlineto{\pgfqpoint{3.120133in}{0.715303in}}%
\pgfpathlineto{\pgfqpoint{3.120740in}{0.707373in}}%
\pgfpathlineto{\pgfqpoint{3.121447in}{0.707605in}}%
\pgfpathlineto{\pgfqpoint{3.122863in}{0.707531in}}%
\pgfpathlineto{\pgfqpoint{3.124986in}{0.708524in}}%
\pgfpathlineto{\pgfqpoint{3.125491in}{0.709860in}}%
\pgfpathlineto{\pgfqpoint{3.125997in}{0.708339in}}%
\pgfpathlineto{\pgfqpoint{3.127412in}{0.708085in}}%
\pgfpathlineto{\pgfqpoint{3.133579in}{0.708888in}}%
\pgfpathlineto{\pgfqpoint{3.135095in}{0.709592in}}%
\pgfpathlineto{\pgfqpoint{3.135297in}{0.709112in}}%
\pgfpathlineto{\pgfqpoint{3.136611in}{0.708790in}}%
\pgfpathlineto{\pgfqpoint{3.139038in}{0.710023in}}%
\pgfpathlineto{\pgfqpoint{3.139644in}{0.716857in}}%
\pgfpathlineto{\pgfqpoint{3.141262in}{0.741531in}}%
\pgfpathlineto{\pgfqpoint{3.141464in}{0.737626in}}%
\pgfpathlineto{\pgfqpoint{3.143183in}{0.714675in}}%
\pgfpathlineto{\pgfqpoint{3.145204in}{0.708569in}}%
\pgfpathlineto{\pgfqpoint{3.151472in}{0.710209in}}%
\pgfpathlineto{\pgfqpoint{3.151775in}{0.730487in}}%
\pgfpathlineto{\pgfqpoint{3.152180in}{0.777132in}}%
\pgfpathlineto{\pgfqpoint{3.152786in}{0.731997in}}%
\pgfpathlineto{\pgfqpoint{3.153191in}{0.726524in}}%
\pgfpathlineto{\pgfqpoint{3.153797in}{0.734881in}}%
\pgfpathlineto{\pgfqpoint{3.154101in}{0.738739in}}%
\pgfpathlineto{\pgfqpoint{3.154505in}{0.732196in}}%
\pgfpathlineto{\pgfqpoint{3.156224in}{0.710488in}}%
\pgfpathlineto{\pgfqpoint{3.157538in}{0.708975in}}%
\pgfpathlineto{\pgfqpoint{3.161379in}{0.709806in}}%
\pgfpathlineto{\pgfqpoint{3.164412in}{0.709911in}}%
\pgfpathlineto{\pgfqpoint{3.165929in}{0.708948in}}%
\pgfpathlineto{\pgfqpoint{3.170983in}{0.709959in}}%
\pgfpathlineto{\pgfqpoint{3.171388in}{0.710662in}}%
\pgfpathlineto{\pgfqpoint{3.171994in}{0.709662in}}%
\pgfpathlineto{\pgfqpoint{3.174319in}{0.710295in}}%
\pgfpathlineto{\pgfqpoint{3.175735in}{0.718286in}}%
\pgfpathlineto{\pgfqpoint{3.176341in}{0.736321in}}%
\pgfpathlineto{\pgfqpoint{3.176847in}{0.722475in}}%
\pgfpathlineto{\pgfqpoint{3.177150in}{0.718036in}}%
\pgfpathlineto{\pgfqpoint{3.177756in}{0.725107in}}%
\pgfpathlineto{\pgfqpoint{3.177858in}{0.724490in}}%
\pgfpathlineto{\pgfqpoint{3.179071in}{0.708108in}}%
\pgfpathlineto{\pgfqpoint{3.179879in}{0.708595in}}%
\pgfpathlineto{\pgfqpoint{3.180587in}{0.708804in}}%
\pgfpathlineto{\pgfqpoint{3.180890in}{0.707912in}}%
\pgfpathlineto{\pgfqpoint{3.181093in}{0.707695in}}%
\pgfpathlineto{\pgfqpoint{3.181396in}{0.708945in}}%
\pgfpathlineto{\pgfqpoint{3.181901in}{0.712389in}}%
\pgfpathlineto{\pgfqpoint{3.182407in}{0.708213in}}%
\pgfpathlineto{\pgfqpoint{3.182710in}{0.707273in}}%
\pgfpathlineto{\pgfqpoint{3.183620in}{0.707579in}}%
\pgfpathlineto{\pgfqpoint{3.184530in}{0.707102in}}%
\pgfpathlineto{\pgfqpoint{3.184833in}{0.707286in}}%
\pgfpathlineto{\pgfqpoint{3.186956in}{0.710783in}}%
\pgfpathlineto{\pgfqpoint{3.187158in}{0.711607in}}%
\pgfpathlineto{\pgfqpoint{3.187765in}{0.709332in}}%
\pgfpathlineto{\pgfqpoint{3.189787in}{0.706588in}}%
\pgfpathlineto{\pgfqpoint{3.189888in}{0.706653in}}%
\pgfpathlineto{\pgfqpoint{3.191303in}{0.707302in}}%
\pgfpathlineto{\pgfqpoint{3.191606in}{0.706930in}}%
\pgfpathlineto{\pgfqpoint{3.192516in}{0.706906in}}%
\pgfpathlineto{\pgfqpoint{3.192718in}{0.707212in}}%
\pgfpathlineto{\pgfqpoint{3.193527in}{0.712148in}}%
\pgfpathlineto{\pgfqpoint{3.194235in}{0.716995in}}%
\pgfpathlineto{\pgfqpoint{3.194841in}{0.715387in}}%
\pgfpathlineto{\pgfqpoint{3.195043in}{0.715194in}}%
\pgfpathlineto{\pgfqpoint{3.195448in}{0.716363in}}%
\pgfpathlineto{\pgfqpoint{3.195650in}{0.716604in}}%
\pgfpathlineto{\pgfqpoint{3.195953in}{0.715135in}}%
\pgfpathlineto{\pgfqpoint{3.197571in}{0.711381in}}%
\pgfpathlineto{\pgfqpoint{3.197874in}{0.712648in}}%
\pgfpathlineto{\pgfqpoint{3.198278in}{0.717387in}}%
\pgfpathlineto{\pgfqpoint{3.198683in}{0.711012in}}%
\pgfpathlineto{\pgfqpoint{3.199997in}{0.704046in}}%
\pgfpathlineto{\pgfqpoint{3.200098in}{0.704059in}}%
\pgfpathlineto{\pgfqpoint{3.201412in}{0.706701in}}%
\pgfpathlineto{\pgfqpoint{3.201918in}{0.706154in}}%
\pgfpathlineto{\pgfqpoint{3.202322in}{0.706086in}}%
\pgfpathlineto{\pgfqpoint{3.209904in}{0.710878in}}%
\pgfpathlineto{\pgfqpoint{3.210207in}{0.711644in}}%
\pgfpathlineto{\pgfqpoint{3.210814in}{0.710325in}}%
\pgfpathlineto{\pgfqpoint{3.211724in}{0.708929in}}%
\pgfpathlineto{\pgfqpoint{3.212128in}{0.708654in}}%
\pgfpathlineto{\pgfqpoint{3.212533in}{0.709590in}}%
\pgfpathlineto{\pgfqpoint{3.214049in}{0.712656in}}%
\pgfpathlineto{\pgfqpoint{3.214150in}{0.712509in}}%
\pgfpathlineto{\pgfqpoint{3.215970in}{0.707883in}}%
\pgfpathlineto{\pgfqpoint{3.219003in}{0.708466in}}%
\pgfpathlineto{\pgfqpoint{3.221024in}{0.719096in}}%
\pgfpathlineto{\pgfqpoint{3.221227in}{0.718058in}}%
\pgfpathlineto{\pgfqpoint{3.222541in}{0.711760in}}%
\pgfpathlineto{\pgfqpoint{3.222743in}{0.712095in}}%
\pgfpathlineto{\pgfqpoint{3.223350in}{0.716873in}}%
\pgfpathlineto{\pgfqpoint{3.223754in}{0.711664in}}%
\pgfpathlineto{\pgfqpoint{3.225068in}{0.707758in}}%
\pgfpathlineto{\pgfqpoint{3.226382in}{0.707343in}}%
\pgfpathlineto{\pgfqpoint{3.229415in}{0.708266in}}%
\pgfpathlineto{\pgfqpoint{3.230831in}{0.710194in}}%
\pgfpathlineto{\pgfqpoint{3.231134in}{0.708628in}}%
\pgfpathlineto{\pgfqpoint{3.232347in}{0.708169in}}%
\pgfpathlineto{\pgfqpoint{3.241041in}{0.710680in}}%
\pgfpathlineto{\pgfqpoint{3.242254in}{0.719621in}}%
\pgfpathlineto{\pgfqpoint{3.242760in}{0.741935in}}%
\pgfpathlineto{\pgfqpoint{3.243265in}{0.720098in}}%
\pgfpathlineto{\pgfqpoint{3.243973in}{0.721665in}}%
\pgfpathlineto{\pgfqpoint{3.244781in}{0.709992in}}%
\pgfpathlineto{\pgfqpoint{3.245186in}{0.711010in}}%
\pgfpathlineto{\pgfqpoint{3.246904in}{0.722352in}}%
\pgfpathlineto{\pgfqpoint{3.247208in}{0.716905in}}%
\pgfpathlineto{\pgfqpoint{3.248522in}{0.708979in}}%
\pgfpathlineto{\pgfqpoint{3.250240in}{0.708764in}}%
\pgfpathlineto{\pgfqpoint{3.250342in}{0.709115in}}%
\pgfpathlineto{\pgfqpoint{3.252465in}{0.725819in}}%
\pgfpathlineto{\pgfqpoint{3.252768in}{0.720098in}}%
\pgfpathlineto{\pgfqpoint{3.254183in}{0.711306in}}%
\pgfpathlineto{\pgfqpoint{3.255598in}{0.707979in}}%
\pgfpathlineto{\pgfqpoint{3.260754in}{0.707982in}}%
\pgfpathlineto{\pgfqpoint{3.261159in}{0.708660in}}%
\pgfpathlineto{\pgfqpoint{3.261866in}{0.707932in}}%
\pgfpathlineto{\pgfqpoint{3.265708in}{0.709421in}}%
\pgfpathlineto{\pgfqpoint{3.266011in}{0.710343in}}%
\pgfpathlineto{\pgfqpoint{3.266516in}{0.709135in}}%
\pgfpathlineto{\pgfqpoint{3.266719in}{0.709164in}}%
\pgfpathlineto{\pgfqpoint{3.267831in}{0.708567in}}%
\pgfpathlineto{\pgfqpoint{3.272986in}{0.710410in}}%
\pgfpathlineto{\pgfqpoint{3.274200in}{0.725168in}}%
\pgfpathlineto{\pgfqpoint{3.273694in}{0.710145in}}%
\pgfpathlineto{\pgfqpoint{3.274402in}{0.718618in}}%
\pgfpathlineto{\pgfqpoint{3.274806in}{0.707728in}}%
\pgfpathlineto{\pgfqpoint{3.275211in}{0.729015in}}%
\pgfpathlineto{\pgfqpoint{3.275312in}{0.732820in}}%
\pgfpathlineto{\pgfqpoint{3.275817in}{0.710333in}}%
\pgfpathlineto{\pgfqpoint{3.276120in}{0.715377in}}%
\pgfpathlineto{\pgfqpoint{3.276221in}{0.717148in}}%
\pgfpathlineto{\pgfqpoint{3.276727in}{0.707839in}}%
\pgfpathlineto{\pgfqpoint{3.277536in}{0.708138in}}%
\pgfpathlineto{\pgfqpoint{3.277839in}{0.717932in}}%
\pgfpathlineto{\pgfqpoint{3.278142in}{0.731638in}}%
\pgfpathlineto{\pgfqpoint{3.278749in}{0.716032in}}%
\pgfpathlineto{\pgfqpoint{3.278850in}{0.716105in}}%
\pgfpathlineto{\pgfqpoint{3.278951in}{0.716242in}}%
\pgfpathlineto{\pgfqpoint{3.279153in}{0.715054in}}%
\pgfpathlineto{\pgfqpoint{3.279962in}{0.708326in}}%
\pgfpathlineto{\pgfqpoint{3.280366in}{0.711677in}}%
\pgfpathlineto{\pgfqpoint{3.282085in}{0.748414in}}%
\pgfpathlineto{\pgfqpoint{3.282287in}{0.742341in}}%
\pgfpathlineto{\pgfqpoint{3.283601in}{0.720043in}}%
\pgfpathlineto{\pgfqpoint{3.283803in}{0.721040in}}%
\pgfpathlineto{\pgfqpoint{3.284208in}{0.744074in}}%
\pgfpathlineto{\pgfqpoint{3.284612in}{0.779394in}}%
\pgfpathlineto{\pgfqpoint{3.285118in}{0.732763in}}%
\pgfpathlineto{\pgfqpoint{3.286533in}{0.709837in}}%
\pgfpathlineto{\pgfqpoint{3.287746in}{0.709115in}}%
\pgfpathlineto{\pgfqpoint{3.300282in}{0.710453in}}%
\pgfpathlineto{\pgfqpoint{3.301899in}{0.728760in}}%
\pgfpathlineto{\pgfqpoint{3.302202in}{0.720093in}}%
\pgfpathlineto{\pgfqpoint{3.303517in}{0.709962in}}%
\pgfpathlineto{\pgfqpoint{3.303921in}{0.709316in}}%
\pgfpathlineto{\pgfqpoint{3.304528in}{0.710387in}}%
\pgfpathlineto{\pgfqpoint{3.305539in}{0.715377in}}%
\pgfpathlineto{\pgfqpoint{3.306044in}{0.742238in}}%
\pgfpathlineto{\pgfqpoint{3.306549in}{0.716732in}}%
\pgfpathlineto{\pgfqpoint{3.307965in}{0.709080in}}%
\pgfpathlineto{\pgfqpoint{3.310897in}{0.710249in}}%
\pgfpathlineto{\pgfqpoint{3.312211in}{0.719077in}}%
\pgfpathlineto{\pgfqpoint{3.311604in}{0.709844in}}%
\pgfpathlineto{\pgfqpoint{3.312413in}{0.716999in}}%
\pgfpathlineto{\pgfqpoint{3.312918in}{0.709330in}}%
\pgfpathlineto{\pgfqpoint{3.313626in}{0.713251in}}%
\pgfpathlineto{\pgfqpoint{3.314030in}{0.713734in}}%
\pgfpathlineto{\pgfqpoint{3.314536in}{0.712573in}}%
\pgfpathlineto{\pgfqpoint{3.315345in}{0.711477in}}%
\pgfpathlineto{\pgfqpoint{3.315648in}{0.712737in}}%
\pgfpathlineto{\pgfqpoint{3.316356in}{0.723211in}}%
\pgfpathlineto{\pgfqpoint{3.317063in}{0.716870in}}%
\pgfpathlineto{\pgfqpoint{3.318276in}{0.711376in}}%
\pgfpathlineto{\pgfqpoint{3.318782in}{0.714687in}}%
\pgfpathlineto{\pgfqpoint{3.319085in}{0.716773in}}%
\pgfpathlineto{\pgfqpoint{3.319591in}{0.712104in}}%
\pgfpathlineto{\pgfqpoint{3.319793in}{0.711483in}}%
\pgfpathlineto{\pgfqpoint{3.320399in}{0.713059in}}%
\pgfpathlineto{\pgfqpoint{3.320804in}{0.722368in}}%
\pgfpathlineto{\pgfqpoint{3.321309in}{0.760467in}}%
\pgfpathlineto{\pgfqpoint{3.322017in}{0.729698in}}%
\pgfpathlineto{\pgfqpoint{3.322522in}{0.777736in}}%
\pgfpathlineto{\pgfqpoint{3.322724in}{0.785939in}}%
\pgfpathlineto{\pgfqpoint{3.323230in}{0.749566in}}%
\pgfpathlineto{\pgfqpoint{3.324746in}{0.723165in}}%
\pgfpathlineto{\pgfqpoint{3.325454in}{0.714629in}}%
\pgfpathlineto{\pgfqpoint{3.325959in}{0.721291in}}%
\pgfpathlineto{\pgfqpoint{3.327476in}{0.794321in}}%
\pgfpathlineto{\pgfqpoint{3.327880in}{0.757372in}}%
\pgfpathlineto{\pgfqpoint{3.329296in}{0.716068in}}%
\pgfpathlineto{\pgfqpoint{3.330610in}{0.711574in}}%
\pgfpathlineto{\pgfqpoint{3.330812in}{0.711962in}}%
\pgfpathlineto{\pgfqpoint{3.331317in}{0.714905in}}%
\pgfpathlineto{\pgfqpoint{3.331722in}{0.711628in}}%
\pgfpathlineto{\pgfqpoint{3.332429in}{0.709222in}}%
\pgfpathlineto{\pgfqpoint{3.332935in}{0.709741in}}%
\pgfpathlineto{\pgfqpoint{3.334552in}{0.718481in}}%
\pgfpathlineto{\pgfqpoint{3.334856in}{0.728342in}}%
\pgfpathlineto{\pgfqpoint{3.335462in}{0.710955in}}%
\pgfpathlineto{\pgfqpoint{3.335563in}{0.710587in}}%
\pgfpathlineto{\pgfqpoint{3.335766in}{0.712429in}}%
\pgfpathlineto{\pgfqpoint{3.336271in}{0.722508in}}%
\pgfpathlineto{\pgfqpoint{3.336776in}{0.710499in}}%
\pgfpathlineto{\pgfqpoint{3.338091in}{0.707676in}}%
\pgfpathlineto{\pgfqpoint{3.340113in}{0.708067in}}%
\pgfpathlineto{\pgfqpoint{3.340618in}{0.714162in}}%
\pgfpathlineto{\pgfqpoint{3.342235in}{0.741789in}}%
\pgfpathlineto{\pgfqpoint{3.342438in}{0.735418in}}%
\pgfpathlineto{\pgfqpoint{3.343954in}{0.708370in}}%
\pgfpathlineto{\pgfqpoint{3.345774in}{0.707207in}}%
\pgfpathlineto{\pgfqpoint{3.345875in}{0.707413in}}%
\pgfpathlineto{\pgfqpoint{3.346380in}{0.710393in}}%
\pgfpathlineto{\pgfqpoint{3.346987in}{0.707623in}}%
\pgfpathlineto{\pgfqpoint{3.348402in}{0.708361in}}%
\pgfpathlineto{\pgfqpoint{3.349312in}{0.717489in}}%
\pgfpathlineto{\pgfqpoint{3.350222in}{0.714262in}}%
\pgfpathlineto{\pgfqpoint{3.350626in}{0.712770in}}%
\pgfpathlineto{\pgfqpoint{3.351132in}{0.715144in}}%
\pgfpathlineto{\pgfqpoint{3.351334in}{0.715751in}}%
\pgfpathlineto{\pgfqpoint{3.351637in}{0.713451in}}%
\pgfpathlineto{\pgfqpoint{3.353255in}{0.707105in}}%
\pgfpathlineto{\pgfqpoint{3.355378in}{0.707752in}}%
\pgfpathlineto{\pgfqpoint{3.356591in}{0.709099in}}%
\pgfpathlineto{\pgfqpoint{3.358613in}{0.720452in}}%
\pgfpathlineto{\pgfqpoint{3.358916in}{0.716313in}}%
\pgfpathlineto{\pgfqpoint{3.360432in}{0.707467in}}%
\pgfpathlineto{\pgfqpoint{3.361645in}{0.707600in}}%
\pgfpathlineto{\pgfqpoint{3.365891in}{0.708273in}}%
\pgfpathlineto{\pgfqpoint{3.367206in}{0.708226in}}%
\pgfpathlineto{\pgfqpoint{3.369531in}{0.708935in}}%
\pgfpathlineto{\pgfqpoint{3.371654in}{0.712896in}}%
\pgfpathlineto{\pgfqpoint{3.372058in}{0.716022in}}%
\pgfpathlineto{\pgfqpoint{3.372766in}{0.712998in}}%
\pgfpathlineto{\pgfqpoint{3.374181in}{0.711752in}}%
\pgfpathlineto{\pgfqpoint{3.374585in}{0.709201in}}%
\pgfpathlineto{\pgfqpoint{3.374990in}{0.711942in}}%
\pgfpathlineto{\pgfqpoint{3.375293in}{0.715216in}}%
\pgfpathlineto{\pgfqpoint{3.375900in}{0.708493in}}%
\pgfpathlineto{\pgfqpoint{3.376203in}{0.710575in}}%
\pgfpathlineto{\pgfqpoint{3.376506in}{0.714628in}}%
\pgfpathlineto{\pgfqpoint{3.377214in}{0.709838in}}%
\pgfpathlineto{\pgfqpoint{3.378831in}{0.708739in}}%
\pgfpathlineto{\pgfqpoint{3.379033in}{0.709307in}}%
\pgfpathlineto{\pgfqpoint{3.379842in}{0.720844in}}%
\pgfpathlineto{\pgfqpoint{3.381055in}{0.715860in}}%
\pgfpathlineto{\pgfqpoint{3.382875in}{0.708687in}}%
\pgfpathlineto{\pgfqpoint{3.383077in}{0.709296in}}%
\pgfpathlineto{\pgfqpoint{3.383583in}{0.712502in}}%
\pgfpathlineto{\pgfqpoint{3.383987in}{0.707971in}}%
\pgfpathlineto{\pgfqpoint{3.385200in}{0.706813in}}%
\pgfpathlineto{\pgfqpoint{3.386514in}{0.706950in}}%
\pgfpathlineto{\pgfqpoint{3.388334in}{0.706912in}}%
\pgfpathlineto{\pgfqpoint{3.390053in}{0.706631in}}%
\pgfpathlineto{\pgfqpoint{3.392580in}{0.708410in}}%
\pgfpathlineto{\pgfqpoint{3.392984in}{0.709544in}}%
\pgfpathlineto{\pgfqpoint{3.393591in}{0.708071in}}%
\pgfpathlineto{\pgfqpoint{3.395107in}{0.708401in}}%
\pgfpathlineto{\pgfqpoint{3.395310in}{0.708602in}}%
\pgfpathlineto{\pgfqpoint{3.395815in}{0.707696in}}%
\pgfpathlineto{\pgfqpoint{3.395916in}{0.707695in}}%
\pgfpathlineto{\pgfqpoint{3.400566in}{0.710073in}}%
\pgfpathlineto{\pgfqpoint{3.401173in}{0.724158in}}%
\pgfpathlineto{\pgfqpoint{3.401577in}{0.734085in}}%
\pgfpathlineto{\pgfqpoint{3.402285in}{0.726841in}}%
\pgfpathlineto{\pgfqpoint{3.404812in}{0.708511in}}%
\pgfpathlineto{\pgfqpoint{3.402993in}{0.728147in}}%
\pgfpathlineto{\pgfqpoint{3.405217in}{0.710617in}}%
\pgfpathlineto{\pgfqpoint{3.405621in}{0.717658in}}%
\pgfpathlineto{\pgfqpoint{3.406430in}{0.714733in}}%
\pgfpathlineto{\pgfqpoint{3.408047in}{0.711175in}}%
\pgfpathlineto{\pgfqpoint{3.408654in}{0.712655in}}%
\pgfpathlineto{\pgfqpoint{3.408755in}{0.712729in}}%
\pgfpathlineto{\pgfqpoint{3.408957in}{0.712009in}}%
\pgfpathlineto{\pgfqpoint{3.410474in}{0.707920in}}%
\pgfpathlineto{\pgfqpoint{3.410575in}{0.708009in}}%
\pgfpathlineto{\pgfqpoint{3.411282in}{0.713839in}}%
\pgfpathlineto{\pgfqpoint{3.412698in}{0.711347in}}%
\pgfpathlineto{\pgfqpoint{3.412900in}{0.711248in}}%
\pgfpathlineto{\pgfqpoint{3.413102in}{0.712431in}}%
\pgfpathlineto{\pgfqpoint{3.413709in}{0.724017in}}%
\pgfpathlineto{\pgfqpoint{3.414214in}{0.714327in}}%
\pgfpathlineto{\pgfqpoint{3.415528in}{0.707250in}}%
\pgfpathlineto{\pgfqpoint{3.415730in}{0.707301in}}%
\pgfpathlineto{\pgfqpoint{3.417348in}{0.709057in}}%
\pgfpathlineto{\pgfqpoint{3.418965in}{0.733100in}}%
\pgfpathlineto{\pgfqpoint{3.419370in}{0.721873in}}%
\pgfpathlineto{\pgfqpoint{3.419875in}{0.712131in}}%
\pgfpathlineto{\pgfqpoint{3.420381in}{0.718888in}}%
\pgfpathlineto{\pgfqpoint{3.421088in}{0.741224in}}%
\pgfpathlineto{\pgfqpoint{3.421695in}{0.730084in}}%
\pgfpathlineto{\pgfqpoint{3.422504in}{0.720713in}}%
\pgfpathlineto{\pgfqpoint{3.423009in}{0.725484in}}%
\pgfpathlineto{\pgfqpoint{3.423312in}{0.728763in}}%
\pgfpathlineto{\pgfqpoint{3.423919in}{0.722945in}}%
\pgfpathlineto{\pgfqpoint{3.425638in}{0.706709in}}%
\pgfpathlineto{\pgfqpoint{3.426952in}{0.707439in}}%
\pgfpathlineto{\pgfqpoint{3.427558in}{0.709169in}}%
\pgfpathlineto{\pgfqpoint{3.428367in}{0.718962in}}%
\pgfpathlineto{\pgfqpoint{3.429277in}{0.715111in}}%
\pgfpathlineto{\pgfqpoint{3.431097in}{0.706681in}}%
\pgfpathlineto{\pgfqpoint{3.431501in}{0.706773in}}%
\pgfpathlineto{\pgfqpoint{3.432411in}{0.707842in}}%
\pgfpathlineto{\pgfqpoint{3.434433in}{0.713621in}}%
\pgfpathlineto{\pgfqpoint{3.434534in}{0.713484in}}%
\pgfpathlineto{\pgfqpoint{3.435343in}{0.710421in}}%
\pgfpathlineto{\pgfqpoint{3.435949in}{0.712151in}}%
\pgfpathlineto{\pgfqpoint{3.436353in}{0.712868in}}%
\pgfpathlineto{\pgfqpoint{3.436859in}{0.711465in}}%
\pgfpathlineto{\pgfqpoint{3.438072in}{0.705085in}}%
\pgfpathlineto{\pgfqpoint{3.439487in}{0.706710in}}%
\pgfpathlineto{\pgfqpoint{3.441307in}{0.710011in}}%
\pgfpathlineto{\pgfqpoint{3.441914in}{0.709509in}}%
\pgfpathlineto{\pgfqpoint{3.444643in}{0.707307in}}%
\pgfpathlineto{\pgfqpoint{3.445452in}{0.708237in}}%
\pgfpathlineto{\pgfqpoint{3.447170in}{0.710213in}}%
\pgfpathlineto{\pgfqpoint{3.447272in}{0.710064in}}%
\pgfpathlineto{\pgfqpoint{3.448990in}{0.708442in}}%
\pgfpathlineto{\pgfqpoint{3.449597in}{0.709805in}}%
\pgfpathlineto{\pgfqpoint{3.449900in}{0.708360in}}%
\pgfpathlineto{\pgfqpoint{3.450405in}{0.706629in}}%
\pgfpathlineto{\pgfqpoint{3.451012in}{0.707856in}}%
\pgfpathlineto{\pgfqpoint{3.451214in}{0.707974in}}%
\pgfpathlineto{\pgfqpoint{3.451619in}{0.706988in}}%
\pgfpathlineto{\pgfqpoint{3.452124in}{0.706591in}}%
\pgfpathlineto{\pgfqpoint{3.452832in}{0.706997in}}%
\pgfpathlineto{\pgfqpoint{3.454550in}{0.707179in}}%
\pgfpathlineto{\pgfqpoint{3.456774in}{0.707311in}}%
\pgfpathlineto{\pgfqpoint{3.458493in}{0.708052in}}%
\pgfpathlineto{\pgfqpoint{3.458998in}{0.709063in}}%
\pgfpathlineto{\pgfqpoint{3.459605in}{0.708178in}}%
\pgfpathlineto{\pgfqpoint{3.463548in}{0.707718in}}%
\pgfpathlineto{\pgfqpoint{3.466479in}{0.709304in}}%
\pgfpathlineto{\pgfqpoint{3.466985in}{0.710995in}}%
\pgfpathlineto{\pgfqpoint{3.467895in}{0.710676in}}%
\pgfpathlineto{\pgfqpoint{3.469714in}{0.707853in}}%
\pgfpathlineto{\pgfqpoint{3.469916in}{0.708249in}}%
\pgfpathlineto{\pgfqpoint{3.472848in}{0.725092in}}%
\pgfpathlineto{\pgfqpoint{3.473151in}{0.720570in}}%
\pgfpathlineto{\pgfqpoint{3.474668in}{0.707259in}}%
\pgfpathlineto{\pgfqpoint{3.474870in}{0.707192in}}%
\pgfpathlineto{\pgfqpoint{3.475274in}{0.708091in}}%
\pgfpathlineto{\pgfqpoint{3.475881in}{0.708450in}}%
\pgfpathlineto{\pgfqpoint{3.476386in}{0.707940in}}%
\pgfpathlineto{\pgfqpoint{3.476791in}{0.707995in}}%
\pgfpathlineto{\pgfqpoint{3.476993in}{0.708588in}}%
\pgfpathlineto{\pgfqpoint{3.477903in}{0.717755in}}%
\pgfpathlineto{\pgfqpoint{3.478914in}{0.745374in}}%
\pgfpathlineto{\pgfqpoint{3.479621in}{0.732881in}}%
\pgfpathlineto{\pgfqpoint{3.480430in}{0.719408in}}%
\pgfpathlineto{\pgfqpoint{3.481239in}{0.712242in}}%
\pgfpathlineto{\pgfqpoint{3.481846in}{0.712938in}}%
\pgfpathlineto{\pgfqpoint{3.482452in}{0.711484in}}%
\pgfpathlineto{\pgfqpoint{3.483766in}{0.706933in}}%
\pgfpathlineto{\pgfqpoint{3.484575in}{0.707170in}}%
\pgfpathlineto{\pgfqpoint{3.485889in}{0.708397in}}%
\pgfpathlineto{\pgfqpoint{3.486395in}{0.711009in}}%
\pgfpathlineto{\pgfqpoint{3.487305in}{0.723892in}}%
\pgfpathlineto{\pgfqpoint{3.488113in}{0.718818in}}%
\pgfpathlineto{\pgfqpoint{3.489832in}{0.707767in}}%
\pgfpathlineto{\pgfqpoint{3.490438in}{0.707976in}}%
\pgfpathlineto{\pgfqpoint{3.494078in}{0.708525in}}%
\pgfpathlineto{\pgfqpoint{3.494482in}{0.710327in}}%
\pgfpathlineto{\pgfqpoint{3.494988in}{0.707937in}}%
\pgfpathlineto{\pgfqpoint{3.496706in}{0.706704in}}%
\pgfpathlineto{\pgfqpoint{3.498829in}{0.708225in}}%
\pgfpathlineto{\pgfqpoint{3.499335in}{0.713789in}}%
\pgfpathlineto{\pgfqpoint{3.500346in}{0.750172in}}%
\pgfpathlineto{\pgfqpoint{3.501053in}{0.733317in}}%
\pgfpathlineto{\pgfqpoint{3.502570in}{0.716227in}}%
\pgfpathlineto{\pgfqpoint{3.503075in}{0.717062in}}%
\pgfpathlineto{\pgfqpoint{3.503277in}{0.717074in}}%
\pgfpathlineto{\pgfqpoint{3.503581in}{0.716263in}}%
\pgfpathlineto{\pgfqpoint{3.507018in}{0.708291in}}%
\pgfpathlineto{\pgfqpoint{3.524709in}{0.708233in}}%
\pgfpathlineto{\pgfqpoint{3.525114in}{0.706927in}}%
\pgfpathlineto{\pgfqpoint{3.525720in}{0.707840in}}%
\pgfpathlineto{\pgfqpoint{3.527034in}{0.715256in}}%
\pgfpathlineto{\pgfqpoint{3.527742in}{0.712843in}}%
\pgfpathlineto{\pgfqpoint{3.529258in}{0.707376in}}%
\pgfpathlineto{\pgfqpoint{3.529461in}{0.708251in}}%
\pgfpathlineto{\pgfqpoint{3.529865in}{0.734911in}}%
\pgfpathlineto{\pgfqpoint{3.530168in}{0.768617in}}%
\pgfpathlineto{\pgfqpoint{3.530876in}{0.719554in}}%
\pgfpathlineto{\pgfqpoint{3.531179in}{0.721785in}}%
\pgfpathlineto{\pgfqpoint{3.531381in}{0.716475in}}%
\pgfpathlineto{\pgfqpoint{3.531988in}{0.707059in}}%
\pgfpathlineto{\pgfqpoint{3.532696in}{0.707798in}}%
\pgfpathlineto{\pgfqpoint{3.533403in}{0.707444in}}%
\pgfpathlineto{\pgfqpoint{3.533808in}{0.714753in}}%
\pgfpathlineto{\pgfqpoint{3.534111in}{0.733876in}}%
\pgfpathlineto{\pgfqpoint{3.534717in}{0.714730in}}%
\pgfpathlineto{\pgfqpoint{3.534920in}{0.717014in}}%
\pgfpathlineto{\pgfqpoint{3.535324in}{0.728068in}}%
\pgfpathlineto{\pgfqpoint{3.535931in}{0.713903in}}%
\pgfpathlineto{\pgfqpoint{3.536335in}{0.711369in}}%
\pgfpathlineto{\pgfqpoint{3.536840in}{0.715495in}}%
\pgfpathlineto{\pgfqpoint{3.537346in}{0.722501in}}%
\pgfpathlineto{\pgfqpoint{3.538155in}{0.719441in}}%
\pgfpathlineto{\pgfqpoint{3.539469in}{0.706924in}}%
\pgfpathlineto{\pgfqpoint{3.540379in}{0.707800in}}%
\pgfpathlineto{\pgfqpoint{3.542097in}{0.709751in}}%
\pgfpathlineto{\pgfqpoint{3.542400in}{0.709391in}}%
\pgfpathlineto{\pgfqpoint{3.544119in}{0.707815in}}%
\pgfpathlineto{\pgfqpoint{3.544321in}{0.708070in}}%
\pgfpathlineto{\pgfqpoint{3.545130in}{0.711816in}}%
\pgfpathlineto{\pgfqpoint{3.546646in}{0.720520in}}%
\pgfpathlineto{\pgfqpoint{3.546849in}{0.718036in}}%
\pgfpathlineto{\pgfqpoint{3.548365in}{0.706661in}}%
\pgfpathlineto{\pgfqpoint{3.549275in}{0.706776in}}%
\pgfpathlineto{\pgfqpoint{3.549376in}{0.706995in}}%
\pgfpathlineto{\pgfqpoint{3.549881in}{0.709359in}}%
\pgfpathlineto{\pgfqpoint{3.550589in}{0.707364in}}%
\pgfpathlineto{\pgfqpoint{3.551196in}{0.708535in}}%
\pgfpathlineto{\pgfqpoint{3.551802in}{0.718263in}}%
\pgfpathlineto{\pgfqpoint{3.552207in}{0.725733in}}%
\pgfpathlineto{\pgfqpoint{3.552914in}{0.719945in}}%
\pgfpathlineto{\pgfqpoint{3.553521in}{0.721087in}}%
\pgfpathlineto{\pgfqpoint{3.553723in}{0.719881in}}%
\pgfpathlineto{\pgfqpoint{3.555644in}{0.711487in}}%
\pgfpathlineto{\pgfqpoint{3.556452in}{0.705660in}}%
\pgfpathlineto{\pgfqpoint{3.557767in}{0.707258in}}%
\pgfpathlineto{\pgfqpoint{3.558070in}{0.708320in}}%
\pgfpathlineto{\pgfqpoint{3.558778in}{0.706973in}}%
\pgfpathlineto{\pgfqpoint{3.559586in}{0.708146in}}%
\pgfpathlineto{\pgfqpoint{3.560193in}{0.710083in}}%
\pgfpathlineto{\pgfqpoint{3.560901in}{0.709409in}}%
\pgfpathlineto{\pgfqpoint{3.561709in}{0.706388in}}%
\pgfpathlineto{\pgfqpoint{3.562518in}{0.706981in}}%
\pgfpathlineto{\pgfqpoint{3.564136in}{0.707473in}}%
\pgfpathlineto{\pgfqpoint{3.566764in}{0.707929in}}%
\pgfpathlineto{\pgfqpoint{3.571515in}{0.708302in}}%
\pgfpathlineto{\pgfqpoint{3.572122in}{0.708562in}}%
\pgfpathlineto{\pgfqpoint{3.572223in}{0.708840in}}%
\pgfpathlineto{\pgfqpoint{3.572729in}{0.710520in}}%
\pgfpathlineto{\pgfqpoint{3.573335in}{0.709063in}}%
\pgfpathlineto{\pgfqpoint{3.574548in}{0.709454in}}%
\pgfpathlineto{\pgfqpoint{3.575054in}{0.711498in}}%
\pgfpathlineto{\pgfqpoint{3.576065in}{0.723086in}}%
\pgfpathlineto{\pgfqpoint{3.576671in}{0.716694in}}%
\pgfpathlineto{\pgfqpoint{3.577076in}{0.712660in}}%
\pgfpathlineto{\pgfqpoint{3.577884in}{0.713306in}}%
\pgfpathlineto{\pgfqpoint{3.578289in}{0.716852in}}%
\pgfpathlineto{\pgfqpoint{3.578592in}{0.711989in}}%
\pgfpathlineto{\pgfqpoint{3.579198in}{0.705594in}}%
\pgfpathlineto{\pgfqpoint{3.579704in}{0.709858in}}%
\pgfpathlineto{\pgfqpoint{3.580209in}{0.790929in}}%
\pgfpathlineto{\pgfqpoint{3.580412in}{0.820966in}}%
\pgfpathlineto{\pgfqpoint{3.581119in}{0.761951in}}%
\pgfpathlineto{\pgfqpoint{3.582838in}{0.716704in}}%
\pgfpathlineto{\pgfqpoint{3.583647in}{0.712096in}}%
\pgfpathlineto{\pgfqpoint{3.584253in}{0.712147in}}%
\pgfpathlineto{\pgfqpoint{3.584961in}{0.711869in}}%
\pgfpathlineto{\pgfqpoint{3.585264in}{0.712507in}}%
\pgfpathlineto{\pgfqpoint{3.585770in}{0.713423in}}%
\pgfpathlineto{\pgfqpoint{3.586275in}{0.712434in}}%
\pgfpathlineto{\pgfqpoint{3.587589in}{0.708986in}}%
\pgfpathlineto{\pgfqpoint{3.587994in}{0.710155in}}%
\pgfpathlineto{\pgfqpoint{3.588701in}{0.722441in}}%
\pgfpathlineto{\pgfqpoint{3.589409in}{0.735284in}}%
\pgfpathlineto{\pgfqpoint{3.589914in}{0.729286in}}%
\pgfpathlineto{\pgfqpoint{3.591633in}{0.709997in}}%
\pgfpathlineto{\pgfqpoint{3.592037in}{0.710790in}}%
\pgfpathlineto{\pgfqpoint{3.593149in}{0.713997in}}%
\pgfpathlineto{\pgfqpoint{3.594059in}{0.726473in}}%
\pgfpathlineto{\pgfqpoint{3.594767in}{0.720888in}}%
\pgfpathlineto{\pgfqpoint{3.596688in}{0.706771in}}%
\pgfpathlineto{\pgfqpoint{3.596890in}{0.706807in}}%
\pgfpathlineto{\pgfqpoint{3.597597in}{0.708001in}}%
\pgfpathlineto{\pgfqpoint{3.598710in}{0.712274in}}%
\pgfpathlineto{\pgfqpoint{3.599215in}{0.711112in}}%
\pgfpathlineto{\pgfqpoint{3.600832in}{0.705324in}}%
\pgfpathlineto{\pgfqpoint{3.601439in}{0.705817in}}%
\pgfpathlineto{\pgfqpoint{3.603158in}{0.709014in}}%
\pgfpathlineto{\pgfqpoint{3.603865in}{0.710831in}}%
\pgfpathlineto{\pgfqpoint{3.604371in}{0.710195in}}%
\pgfpathlineto{\pgfqpoint{3.606595in}{0.707293in}}%
\pgfpathlineto{\pgfqpoint{3.613570in}{0.708564in}}%
\pgfpathlineto{\pgfqpoint{3.621557in}{0.708564in}}%
\pgfpathlineto{\pgfqpoint{3.622365in}{0.709022in}}%
\pgfpathlineto{\pgfqpoint{3.623174in}{0.720981in}}%
\pgfpathlineto{\pgfqpoint{3.624387in}{0.716321in}}%
\pgfpathlineto{\pgfqpoint{3.625095in}{0.711391in}}%
\pgfpathlineto{\pgfqpoint{3.625803in}{0.713116in}}%
\pgfpathlineto{\pgfqpoint{3.629846in}{0.708304in}}%
\pgfpathlineto{\pgfqpoint{3.629947in}{0.708396in}}%
\pgfpathlineto{\pgfqpoint{3.630655in}{0.710993in}}%
\pgfpathlineto{\pgfqpoint{3.631565in}{0.709166in}}%
\pgfpathlineto{\pgfqpoint{3.638338in}{0.710028in}}%
\pgfpathlineto{\pgfqpoint{3.638540in}{0.710455in}}%
\pgfpathlineto{\pgfqpoint{3.639046in}{0.709043in}}%
\pgfpathlineto{\pgfqpoint{3.640461in}{0.708901in}}%
\pgfpathlineto{\pgfqpoint{3.643393in}{0.711200in}}%
\pgfpathlineto{\pgfqpoint{3.643999in}{0.724180in}}%
\pgfpathlineto{\pgfqpoint{3.644808in}{0.717074in}}%
\pgfpathlineto{\pgfqpoint{3.645111in}{0.716592in}}%
\pgfpathlineto{\pgfqpoint{3.647032in}{0.709603in}}%
\pgfpathlineto{\pgfqpoint{3.647133in}{0.709970in}}%
\pgfpathlineto{\pgfqpoint{3.647740in}{0.718842in}}%
\pgfpathlineto{\pgfqpoint{3.648144in}{0.710638in}}%
\pgfpathlineto{\pgfqpoint{3.648447in}{0.708914in}}%
\pgfpathlineto{\pgfqpoint{3.648751in}{0.712353in}}%
\pgfpathlineto{\pgfqpoint{3.649357in}{0.737584in}}%
\pgfpathlineto{\pgfqpoint{3.650065in}{0.719662in}}%
\pgfpathlineto{\pgfqpoint{3.650166in}{0.719287in}}%
\pgfpathlineto{\pgfqpoint{3.650469in}{0.721437in}}%
\pgfpathlineto{\pgfqpoint{3.650672in}{0.722252in}}%
\pgfpathlineto{\pgfqpoint{3.650975in}{0.718769in}}%
\pgfpathlineto{\pgfqpoint{3.651682in}{0.711621in}}%
\pgfpathlineto{\pgfqpoint{3.652390in}{0.712556in}}%
\pgfpathlineto{\pgfqpoint{3.653401in}{0.710589in}}%
\pgfpathlineto{\pgfqpoint{3.654715in}{0.707830in}}%
\pgfpathlineto{\pgfqpoint{3.654816in}{0.707848in}}%
\pgfpathlineto{\pgfqpoint{3.656737in}{0.707599in}}%
\pgfpathlineto{\pgfqpoint{3.659669in}{0.708401in}}%
\pgfpathlineto{\pgfqpoint{3.661489in}{0.710694in}}%
\pgfpathlineto{\pgfqpoint{3.661792in}{0.709137in}}%
\pgfpathlineto{\pgfqpoint{3.662499in}{0.707090in}}%
\pgfpathlineto{\pgfqpoint{3.663106in}{0.707261in}}%
\pgfpathlineto{\pgfqpoint{3.673721in}{0.709459in}}%
\pgfpathlineto{\pgfqpoint{3.675136in}{0.708294in}}%
\pgfpathlineto{\pgfqpoint{3.677562in}{0.709428in}}%
\pgfpathlineto{\pgfqpoint{3.678978in}{0.720687in}}%
\pgfpathlineto{\pgfqpoint{3.679281in}{0.715067in}}%
\pgfpathlineto{\pgfqpoint{3.679685in}{0.709821in}}%
\pgfpathlineto{\pgfqpoint{3.680494in}{0.712428in}}%
\pgfpathlineto{\pgfqpoint{3.682112in}{0.708835in}}%
\pgfpathlineto{\pgfqpoint{3.682819in}{0.708656in}}%
\pgfpathlineto{\pgfqpoint{3.683123in}{0.709316in}}%
\pgfpathlineto{\pgfqpoint{3.683527in}{0.710364in}}%
\pgfpathlineto{\pgfqpoint{3.684133in}{0.708975in}}%
\pgfpathlineto{\pgfqpoint{3.685144in}{0.708714in}}%
\pgfpathlineto{\pgfqpoint{3.685448in}{0.709162in}}%
\pgfpathlineto{\pgfqpoint{3.686459in}{0.709077in}}%
\pgfpathlineto{\pgfqpoint{3.686560in}{0.708949in}}%
\pgfpathlineto{\pgfqpoint{3.688278in}{0.707834in}}%
\pgfpathlineto{\pgfqpoint{3.688885in}{0.709174in}}%
\pgfpathlineto{\pgfqpoint{3.690098in}{0.719804in}}%
\pgfpathlineto{\pgfqpoint{3.690401in}{0.722618in}}%
\pgfpathlineto{\pgfqpoint{3.691109in}{0.719773in}}%
\pgfpathlineto{\pgfqpoint{3.692019in}{0.720100in}}%
\pgfpathlineto{\pgfqpoint{3.692726in}{0.733066in}}%
\pgfpathlineto{\pgfqpoint{3.693434in}{0.725317in}}%
\pgfpathlineto{\pgfqpoint{3.695860in}{0.708011in}}%
\pgfpathlineto{\pgfqpoint{3.696164in}{0.708060in}}%
\pgfpathlineto{\pgfqpoint{3.697478in}{0.708864in}}%
\pgfpathlineto{\pgfqpoint{3.698388in}{0.723759in}}%
\pgfpathlineto{\pgfqpoint{3.699399in}{0.718053in}}%
\pgfpathlineto{\pgfqpoint{3.701420in}{0.709002in}}%
\pgfpathlineto{\pgfqpoint{3.703543in}{0.709588in}}%
\pgfpathlineto{\pgfqpoint{3.704655in}{0.710579in}}%
\pgfpathlineto{\pgfqpoint{3.704049in}{0.709120in}}%
\pgfpathlineto{\pgfqpoint{3.704858in}{0.709809in}}%
\pgfpathlineto{\pgfqpoint{3.706273in}{0.708054in}}%
\pgfpathlineto{\pgfqpoint{3.710721in}{0.709543in}}%
\pgfpathlineto{\pgfqpoint{3.711530in}{0.709732in}}%
\pgfpathlineto{\pgfqpoint{3.711833in}{0.709369in}}%
\pgfpathlineto{\pgfqpoint{3.713552in}{0.708455in}}%
\pgfpathlineto{\pgfqpoint{3.714360in}{0.709296in}}%
\pgfpathlineto{\pgfqpoint{3.716382in}{0.720071in}}%
\pgfpathlineto{\pgfqpoint{3.717090in}{0.717056in}}%
\pgfpathlineto{\pgfqpoint{3.717696in}{0.714809in}}%
\pgfpathlineto{\pgfqpoint{3.719213in}{0.707577in}}%
\pgfpathlineto{\pgfqpoint{3.720729in}{0.707909in}}%
\pgfpathlineto{\pgfqpoint{3.721134in}{0.719545in}}%
\pgfpathlineto{\pgfqpoint{3.721336in}{0.727096in}}%
\pgfpathlineto{\pgfqpoint{3.721942in}{0.716994in}}%
\pgfpathlineto{\pgfqpoint{3.722246in}{0.720236in}}%
\pgfpathlineto{\pgfqpoint{3.722347in}{0.720599in}}%
\pgfpathlineto{\pgfqpoint{3.722549in}{0.717551in}}%
\pgfpathlineto{\pgfqpoint{3.723358in}{0.707706in}}%
\pgfpathlineto{\pgfqpoint{3.723964in}{0.707873in}}%
\pgfpathlineto{\pgfqpoint{3.725481in}{0.709013in}}%
\pgfpathlineto{\pgfqpoint{3.727098in}{0.717261in}}%
\pgfpathlineto{\pgfqpoint{3.727401in}{0.714423in}}%
\pgfpathlineto{\pgfqpoint{3.728817in}{0.708757in}}%
\pgfpathlineto{\pgfqpoint{3.729322in}{0.708869in}}%
\pgfpathlineto{\pgfqpoint{3.729524in}{0.709323in}}%
\pgfpathlineto{\pgfqpoint{3.729828in}{0.709822in}}%
\pgfpathlineto{\pgfqpoint{3.730434in}{0.708650in}}%
\pgfpathlineto{\pgfqpoint{3.732052in}{0.708179in}}%
\pgfpathlineto{\pgfqpoint{3.733770in}{0.707980in}}%
\pgfpathlineto{\pgfqpoint{3.735388in}{0.708205in}}%
\pgfpathlineto{\pgfqpoint{3.742869in}{0.709474in}}%
\pgfpathlineto{\pgfqpoint{3.744688in}{0.708735in}}%
\pgfpathlineto{\pgfqpoint{3.744790in}{0.708943in}}%
\pgfpathlineto{\pgfqpoint{3.746003in}{0.714025in}}%
\pgfpathlineto{\pgfqpoint{3.746811in}{0.712597in}}%
\pgfpathlineto{\pgfqpoint{3.747014in}{0.712423in}}%
\pgfpathlineto{\pgfqpoint{3.747317in}{0.713550in}}%
\pgfpathlineto{\pgfqpoint{3.749137in}{0.719646in}}%
\pgfpathlineto{\pgfqpoint{3.749238in}{0.719727in}}%
\pgfpathlineto{\pgfqpoint{3.749541in}{0.718732in}}%
\pgfpathlineto{\pgfqpoint{3.751361in}{0.708284in}}%
\pgfpathlineto{\pgfqpoint{3.751967in}{0.709850in}}%
\pgfpathlineto{\pgfqpoint{3.752270in}{0.711804in}}%
\pgfpathlineto{\pgfqpoint{3.752877in}{0.708719in}}%
\pgfpathlineto{\pgfqpoint{3.753888in}{0.708114in}}%
\pgfpathlineto{\pgfqpoint{3.753989in}{0.708893in}}%
\pgfpathlineto{\pgfqpoint{3.754596in}{0.756783in}}%
\pgfpathlineto{\pgfqpoint{3.755708in}{0.729116in}}%
\pgfpathlineto{\pgfqpoint{3.757729in}{0.713575in}}%
\pgfpathlineto{\pgfqpoint{3.758134in}{0.712829in}}%
\pgfpathlineto{\pgfqpoint{3.759044in}{0.707555in}}%
\pgfpathlineto{\pgfqpoint{3.759852in}{0.707665in}}%
\pgfpathlineto{\pgfqpoint{3.761672in}{0.708577in}}%
\pgfpathlineto{\pgfqpoint{3.763189in}{0.719948in}}%
\pgfpathlineto{\pgfqpoint{3.763997in}{0.715438in}}%
\pgfpathlineto{\pgfqpoint{3.764705in}{0.713338in}}%
\pgfpathlineto{\pgfqpoint{3.765210in}{0.714888in}}%
\pgfpathlineto{\pgfqpoint{3.765615in}{0.716269in}}%
\pgfpathlineto{\pgfqpoint{3.766120in}{0.713984in}}%
\pgfpathlineto{\pgfqpoint{3.766221in}{0.713897in}}%
\pgfpathlineto{\pgfqpoint{3.766424in}{0.714929in}}%
\pgfpathlineto{\pgfqpoint{3.766626in}{0.716505in}}%
\pgfpathlineto{\pgfqpoint{3.767030in}{0.710545in}}%
\pgfpathlineto{\pgfqpoint{3.767434in}{0.706708in}}%
\pgfpathlineto{\pgfqpoint{3.768041in}{0.711262in}}%
\pgfpathlineto{\pgfqpoint{3.768243in}{0.712289in}}%
\pgfpathlineto{\pgfqpoint{3.769052in}{0.710420in}}%
\pgfpathlineto{\pgfqpoint{3.769355in}{0.710298in}}%
\pgfpathlineto{\pgfqpoint{3.769557in}{0.710782in}}%
\pgfpathlineto{\pgfqpoint{3.770164in}{0.720767in}}%
\pgfpathlineto{\pgfqpoint{3.770467in}{0.725170in}}%
\pgfpathlineto{\pgfqpoint{3.770973in}{0.717143in}}%
\pgfpathlineto{\pgfqpoint{3.771175in}{0.715708in}}%
\pgfpathlineto{\pgfqpoint{3.771579in}{0.721343in}}%
\pgfpathlineto{\pgfqpoint{3.771781in}{0.723790in}}%
\pgfpathlineto{\pgfqpoint{3.772186in}{0.716310in}}%
\pgfpathlineto{\pgfqpoint{3.772894in}{0.706992in}}%
\pgfpathlineto{\pgfqpoint{3.773500in}{0.709725in}}%
\pgfpathlineto{\pgfqpoint{3.773904in}{0.707665in}}%
\pgfpathlineto{\pgfqpoint{3.774511in}{0.705901in}}%
\pgfpathlineto{\pgfqpoint{3.775219in}{0.706055in}}%
\pgfpathlineto{\pgfqpoint{3.777139in}{0.707718in}}%
\pgfpathlineto{\pgfqpoint{3.777544in}{0.718658in}}%
\pgfpathlineto{\pgfqpoint{3.777948in}{0.731077in}}%
\pgfpathlineto{\pgfqpoint{3.778555in}{0.717320in}}%
\pgfpathlineto{\pgfqpoint{3.781689in}{0.705721in}}%
\pgfpathlineto{\pgfqpoint{3.783407in}{0.706749in}}%
\pgfpathlineto{\pgfqpoint{3.785732in}{0.708238in}}%
\pgfpathlineto{\pgfqpoint{3.787956in}{0.709541in}}%
\pgfpathlineto{\pgfqpoint{3.788260in}{0.710948in}}%
\pgfpathlineto{\pgfqpoint{3.789068in}{0.709712in}}%
\pgfpathlineto{\pgfqpoint{3.789877in}{0.712964in}}%
\pgfpathlineto{\pgfqpoint{3.790787in}{0.712110in}}%
\pgfpathlineto{\pgfqpoint{3.791495in}{0.707735in}}%
\pgfpathlineto{\pgfqpoint{3.792405in}{0.707910in}}%
\pgfpathlineto{\pgfqpoint{3.794730in}{0.708207in}}%
\pgfpathlineto{\pgfqpoint{3.796347in}{0.709543in}}%
\pgfpathlineto{\pgfqpoint{3.797055in}{0.715106in}}%
\pgfpathlineto{\pgfqpoint{3.798066in}{0.713410in}}%
\pgfpathlineto{\pgfqpoint{3.798672in}{0.714941in}}%
\pgfpathlineto{\pgfqpoint{3.799481in}{0.735592in}}%
\pgfpathlineto{\pgfqpoint{3.800088in}{0.719947in}}%
\pgfpathlineto{\pgfqpoint{3.801907in}{0.709144in}}%
\pgfpathlineto{\pgfqpoint{3.802211in}{0.708102in}}%
\pgfpathlineto{\pgfqpoint{3.802716in}{0.710228in}}%
\pgfpathlineto{\pgfqpoint{3.803019in}{0.710861in}}%
\pgfpathlineto{\pgfqpoint{3.803525in}{0.709027in}}%
\pgfpathlineto{\pgfqpoint{3.804030in}{0.707840in}}%
\pgfpathlineto{\pgfqpoint{3.804637in}{0.709154in}}%
\pgfpathlineto{\pgfqpoint{3.804839in}{0.709342in}}%
\pgfpathlineto{\pgfqpoint{3.805142in}{0.707982in}}%
\pgfpathlineto{\pgfqpoint{3.805648in}{0.706449in}}%
\pgfpathlineto{\pgfqpoint{3.806457in}{0.706663in}}%
\pgfpathlineto{\pgfqpoint{3.807164in}{0.706820in}}%
\pgfpathlineto{\pgfqpoint{3.807366in}{0.707172in}}%
\pgfpathlineto{\pgfqpoint{3.809085in}{0.710118in}}%
\pgfpathlineto{\pgfqpoint{3.810500in}{0.744297in}}%
\pgfpathlineto{\pgfqpoint{3.810905in}{0.722794in}}%
\pgfpathlineto{\pgfqpoint{3.812118in}{0.708130in}}%
\pgfpathlineto{\pgfqpoint{3.812219in}{0.708188in}}%
\pgfpathlineto{\pgfqpoint{3.812724in}{0.710083in}}%
\pgfpathlineto{\pgfqpoint{3.813533in}{0.708772in}}%
\pgfpathlineto{\pgfqpoint{3.814544in}{0.707459in}}%
\pgfpathlineto{\pgfqpoint{3.814847in}{0.707619in}}%
\pgfpathlineto{\pgfqpoint{3.816364in}{0.708910in}}%
\pgfpathlineto{\pgfqpoint{3.816970in}{0.722329in}}%
\pgfpathlineto{\pgfqpoint{3.817880in}{0.717503in}}%
\pgfpathlineto{\pgfqpoint{3.818891in}{0.707557in}}%
\pgfpathlineto{\pgfqpoint{3.819801in}{0.707902in}}%
\pgfpathlineto{\pgfqpoint{3.822834in}{0.709609in}}%
\pgfpathlineto{\pgfqpoint{3.824451in}{0.712239in}}%
\pgfpathlineto{\pgfqpoint{3.824552in}{0.712153in}}%
\pgfpathlineto{\pgfqpoint{3.825968in}{0.708895in}}%
\pgfpathlineto{\pgfqpoint{3.826473in}{0.709237in}}%
\pgfpathlineto{\pgfqpoint{3.827585in}{0.709216in}}%
\pgfpathlineto{\pgfqpoint{3.827787in}{0.708984in}}%
\pgfpathlineto{\pgfqpoint{3.829405in}{0.708334in}}%
\pgfpathlineto{\pgfqpoint{3.830011in}{0.710298in}}%
\pgfpathlineto{\pgfqpoint{3.830315in}{0.711109in}}%
\pgfpathlineto{\pgfqpoint{3.830921in}{0.709935in}}%
\pgfpathlineto{\pgfqpoint{3.831022in}{0.709955in}}%
\pgfpathlineto{\pgfqpoint{3.831427in}{0.710288in}}%
\pgfpathlineto{\pgfqpoint{3.831730in}{0.709352in}}%
\pgfpathlineto{\pgfqpoint{3.832438in}{0.707918in}}%
\pgfpathlineto{\pgfqpoint{3.833044in}{0.708116in}}%
\pgfpathlineto{\pgfqpoint{3.837795in}{0.710879in}}%
\pgfpathlineto{\pgfqpoint{3.838099in}{0.712559in}}%
\pgfpathlineto{\pgfqpoint{3.838503in}{0.708480in}}%
\pgfpathlineto{\pgfqpoint{3.839716in}{0.706958in}}%
\pgfpathlineto{\pgfqpoint{3.841030in}{0.707000in}}%
\pgfpathlineto{\pgfqpoint{3.841132in}{0.707240in}}%
\pgfpathlineto{\pgfqpoint{3.842345in}{0.707705in}}%
\pgfpathlineto{\pgfqpoint{3.843659in}{0.707616in}}%
\pgfpathlineto{\pgfqpoint{3.843962in}{0.708931in}}%
\pgfpathlineto{\pgfqpoint{3.844367in}{0.715733in}}%
\pgfpathlineto{\pgfqpoint{3.845479in}{0.715570in}}%
\pgfpathlineto{\pgfqpoint{3.845984in}{0.711192in}}%
\pgfpathlineto{\pgfqpoint{3.847399in}{0.707735in}}%
\pgfpathlineto{\pgfqpoint{3.848714in}{0.708077in}}%
\pgfpathlineto{\pgfqpoint{3.850331in}{0.709740in}}%
\pgfpathlineto{\pgfqpoint{3.850533in}{0.709353in}}%
\pgfpathlineto{\pgfqpoint{3.850735in}{0.709256in}}%
\pgfpathlineto{\pgfqpoint{3.850938in}{0.710150in}}%
\pgfpathlineto{\pgfqpoint{3.851443in}{0.728505in}}%
\pgfpathlineto{\pgfqpoint{3.852656in}{0.738778in}}%
\pgfpathlineto{\pgfqpoint{3.853162in}{0.729971in}}%
\pgfpathlineto{\pgfqpoint{3.853566in}{0.738927in}}%
\pgfpathlineto{\pgfqpoint{3.853970in}{0.753122in}}%
\pgfpathlineto{\pgfqpoint{3.854577in}{0.734257in}}%
\pgfpathlineto{\pgfqpoint{3.857408in}{0.708948in}}%
\pgfpathlineto{\pgfqpoint{3.858621in}{0.710061in}}%
\pgfpathlineto{\pgfqpoint{3.859429in}{0.710956in}}%
\pgfpathlineto{\pgfqpoint{3.859935in}{0.710535in}}%
\pgfpathlineto{\pgfqpoint{3.861249in}{0.709051in}}%
\pgfpathlineto{\pgfqpoint{3.861552in}{0.709717in}}%
\pgfpathlineto{\pgfqpoint{3.862159in}{0.713794in}}%
\pgfpathlineto{\pgfqpoint{3.862968in}{0.711650in}}%
\pgfpathlineto{\pgfqpoint{3.863675in}{0.711314in}}%
\pgfpathlineto{\pgfqpoint{3.863777in}{0.711016in}}%
\pgfpathlineto{\pgfqpoint{3.865495in}{0.708779in}}%
\pgfpathlineto{\pgfqpoint{3.870246in}{0.709259in}}%
\pgfpathlineto{\pgfqpoint{3.871662in}{0.711108in}}%
\pgfpathlineto{\pgfqpoint{3.872673in}{0.714115in}}%
\pgfpathlineto{\pgfqpoint{3.873785in}{0.728156in}}%
\pgfpathlineto{\pgfqpoint{3.874492in}{0.723007in}}%
\pgfpathlineto{\pgfqpoint{3.876919in}{0.710857in}}%
\pgfpathlineto{\pgfqpoint{3.879446in}{0.708085in}}%
\pgfpathlineto{\pgfqpoint{3.883894in}{0.710456in}}%
\pgfpathlineto{\pgfqpoint{3.884197in}{0.711382in}}%
\pgfpathlineto{\pgfqpoint{3.884804in}{0.709733in}}%
\pgfpathlineto{\pgfqpoint{3.887129in}{0.708916in}}%
\pgfpathlineto{\pgfqpoint{3.889353in}{0.708841in}}%
\pgfpathlineto{\pgfqpoint{3.889960in}{0.712596in}}%
\pgfpathlineto{\pgfqpoint{3.890667in}{0.722441in}}%
\pgfpathlineto{\pgfqpoint{3.891375in}{0.718798in}}%
\pgfpathlineto{\pgfqpoint{3.893094in}{0.708619in}}%
\pgfpathlineto{\pgfqpoint{3.893902in}{0.708784in}}%
\pgfpathlineto{\pgfqpoint{3.895115in}{0.709383in}}%
\pgfpathlineto{\pgfqpoint{3.896834in}{0.717555in}}%
\pgfpathlineto{\pgfqpoint{3.897137in}{0.715333in}}%
\pgfpathlineto{\pgfqpoint{3.897845in}{0.710401in}}%
\pgfpathlineto{\pgfqpoint{3.898452in}{0.711514in}}%
\pgfpathlineto{\pgfqpoint{3.898755in}{0.711591in}}%
\pgfpathlineto{\pgfqpoint{3.899260in}{0.710780in}}%
\pgfpathlineto{\pgfqpoint{3.901585in}{0.707722in}}%
\pgfpathlineto{\pgfqpoint{3.904214in}{0.707975in}}%
\pgfpathlineto{\pgfqpoint{3.907449in}{0.710730in}}%
\pgfpathlineto{\pgfqpoint{3.907853in}{0.713212in}}%
\pgfpathlineto{\pgfqpoint{3.908460in}{0.710669in}}%
\pgfpathlineto{\pgfqpoint{3.909875in}{0.708384in}}%
\pgfpathlineto{\pgfqpoint{3.910381in}{0.708884in}}%
\pgfpathlineto{\pgfqpoint{3.913211in}{0.707877in}}%
\pgfpathlineto{\pgfqpoint{3.913514in}{0.708975in}}%
\pgfpathlineto{\pgfqpoint{3.913919in}{0.730489in}}%
\pgfpathlineto{\pgfqpoint{3.914323in}{0.763473in}}%
\pgfpathlineto{\pgfqpoint{3.915031in}{0.730756in}}%
\pgfpathlineto{\pgfqpoint{3.916042in}{0.710614in}}%
\pgfpathlineto{\pgfqpoint{3.916547in}{0.708797in}}%
\pgfpathlineto{\pgfqpoint{3.917255in}{0.709300in}}%
\pgfpathlineto{\pgfqpoint{3.919277in}{0.710497in}}%
\pgfpathlineto{\pgfqpoint{3.920187in}{0.709635in}}%
\pgfpathlineto{\pgfqpoint{3.921501in}{0.708950in}}%
\pgfpathlineto{\pgfqpoint{3.925039in}{0.709483in}}%
\pgfpathlineto{\pgfqpoint{3.925848in}{0.718051in}}%
\pgfpathlineto{\pgfqpoint{3.926859in}{0.714711in}}%
\pgfpathlineto{\pgfqpoint{3.927769in}{0.711050in}}%
\pgfpathlineto{\pgfqpoint{3.928274in}{0.712661in}}%
\pgfpathlineto{\pgfqpoint{3.928982in}{0.714940in}}%
\pgfpathlineto{\pgfqpoint{3.929588in}{0.714065in}}%
\pgfpathlineto{\pgfqpoint{3.933026in}{0.708588in}}%
\pgfpathlineto{\pgfqpoint{3.933127in}{0.708720in}}%
\pgfpathlineto{\pgfqpoint{3.933632in}{0.712947in}}%
\pgfpathlineto{\pgfqpoint{3.933935in}{0.714994in}}%
\pgfpathlineto{\pgfqpoint{3.934542in}{0.711642in}}%
\pgfpathlineto{\pgfqpoint{3.935553in}{0.708985in}}%
\pgfpathlineto{\pgfqpoint{3.936867in}{0.707956in}}%
\pgfpathlineto{\pgfqpoint{3.946976in}{0.709518in}}%
\pgfpathlineto{\pgfqpoint{3.948392in}{0.711049in}}%
\pgfpathlineto{\pgfqpoint{3.948594in}{0.710564in}}%
\pgfpathlineto{\pgfqpoint{3.950009in}{0.708848in}}%
\pgfpathlineto{\pgfqpoint{3.952840in}{0.709859in}}%
\pgfpathlineto{\pgfqpoint{3.953143in}{0.710824in}}%
\pgfpathlineto{\pgfqpoint{3.953750in}{0.709185in}}%
\pgfpathlineto{\pgfqpoint{3.956176in}{0.708303in}}%
\pgfpathlineto{\pgfqpoint{3.956479in}{0.708239in}}%
\pgfpathlineto{\pgfqpoint{3.956681in}{0.709305in}}%
\pgfpathlineto{\pgfqpoint{3.957894in}{0.736277in}}%
\pgfpathlineto{\pgfqpoint{3.958198in}{0.757229in}}%
\pgfpathlineto{\pgfqpoint{3.958804in}{0.719004in}}%
\pgfpathlineto{\pgfqpoint{3.960523in}{0.709063in}}%
\pgfpathlineto{\pgfqpoint{3.961231in}{0.709546in}}%
\pgfpathlineto{\pgfqpoint{3.961635in}{0.708567in}}%
\pgfpathlineto{\pgfqpoint{3.962848in}{0.708368in}}%
\pgfpathlineto{\pgfqpoint{3.962949in}{0.708451in}}%
\pgfpathlineto{\pgfqpoint{3.964466in}{0.708460in}}%
\pgfpathlineto{\pgfqpoint{3.966690in}{0.708569in}}%
\pgfpathlineto{\pgfqpoint{3.971946in}{0.709060in}}%
\pgfpathlineto{\pgfqpoint{3.972452in}{0.710572in}}%
\pgfpathlineto{\pgfqpoint{3.972755in}{0.711686in}}%
\pgfpathlineto{\pgfqpoint{3.973463in}{0.710336in}}%
\pgfpathlineto{\pgfqpoint{3.974676in}{0.709064in}}%
\pgfpathlineto{\pgfqpoint{3.975586in}{0.709183in}}%
\pgfpathlineto{\pgfqpoint{3.975687in}{0.709630in}}%
\pgfpathlineto{\pgfqpoint{3.976192in}{0.727578in}}%
\pgfpathlineto{\pgfqpoint{3.976496in}{0.733855in}}%
\pgfpathlineto{\pgfqpoint{3.977102in}{0.724400in}}%
\pgfpathlineto{\pgfqpoint{3.979427in}{0.709254in}}%
\pgfpathlineto{\pgfqpoint{3.979630in}{0.709458in}}%
\pgfpathlineto{\pgfqpoint{3.980236in}{0.709018in}}%
\pgfpathlineto{\pgfqpoint{3.980641in}{0.711391in}}%
\pgfpathlineto{\pgfqpoint{3.981247in}{0.761180in}}%
\pgfpathlineto{\pgfqpoint{3.982056in}{0.728850in}}%
\pgfpathlineto{\pgfqpoint{3.982359in}{0.727336in}}%
\pgfpathlineto{\pgfqpoint{3.984078in}{0.709210in}}%
\pgfpathlineto{\pgfqpoint{3.984381in}{0.709245in}}%
\pgfpathlineto{\pgfqpoint{3.987818in}{0.709213in}}%
\pgfpathlineto{\pgfqpoint{3.989537in}{0.709165in}}%
\pgfpathlineto{\pgfqpoint{3.991154in}{0.709308in}}%
\pgfpathlineto{\pgfqpoint{3.991255in}{0.709136in}}%
\pgfpathlineto{\pgfqpoint{3.992772in}{0.708927in}}%
\pgfpathlineto{\pgfqpoint{4.002982in}{0.708821in}}%
\pgfpathlineto{\pgfqpoint{4.003488in}{0.708387in}}%
\pgfpathlineto{\pgfqpoint{4.003791in}{0.708970in}}%
\pgfpathlineto{\pgfqpoint{4.004397in}{0.715016in}}%
\pgfpathlineto{\pgfqpoint{4.005105in}{0.710583in}}%
\pgfpathlineto{\pgfqpoint{4.005408in}{0.711318in}}%
\pgfpathlineto{\pgfqpoint{4.005712in}{0.712595in}}%
\pgfpathlineto{\pgfqpoint{4.006116in}{0.709218in}}%
\pgfpathlineto{\pgfqpoint{4.006419in}{0.708080in}}%
\pgfpathlineto{\pgfqpoint{4.006824in}{0.710741in}}%
\pgfpathlineto{\pgfqpoint{4.007430in}{0.730951in}}%
\pgfpathlineto{\pgfqpoint{4.008239in}{0.720641in}}%
\pgfpathlineto{\pgfqpoint{4.008947in}{0.717467in}}%
\pgfpathlineto{\pgfqpoint{4.009149in}{0.719245in}}%
\pgfpathlineto{\pgfqpoint{4.009654in}{0.728219in}}%
\pgfpathlineto{\pgfqpoint{4.010160in}{0.718848in}}%
\pgfpathlineto{\pgfqpoint{4.011373in}{0.708204in}}%
\pgfpathlineto{\pgfqpoint{4.010665in}{0.719434in}}%
\pgfpathlineto{\pgfqpoint{4.012283in}{0.708459in}}%
\pgfpathlineto{\pgfqpoint{4.012788in}{0.709297in}}%
\pgfpathlineto{\pgfqpoint{4.013496in}{0.708241in}}%
\pgfpathlineto{\pgfqpoint{4.014810in}{0.708882in}}%
\pgfpathlineto{\pgfqpoint{4.015214in}{0.711040in}}%
\pgfpathlineto{\pgfqpoint{4.015821in}{0.708893in}}%
\pgfpathlineto{\pgfqpoint{4.017236in}{0.708573in}}%
\pgfpathlineto{\pgfqpoint{4.017742in}{0.708868in}}%
\pgfpathlineto{\pgfqpoint{4.018045in}{0.707984in}}%
\pgfpathlineto{\pgfqpoint{4.018753in}{0.707513in}}%
\pgfpathlineto{\pgfqpoint{4.019056in}{0.708109in}}%
\pgfpathlineto{\pgfqpoint{4.019460in}{0.719654in}}%
\pgfpathlineto{\pgfqpoint{4.019764in}{0.730448in}}%
\pgfpathlineto{\pgfqpoint{4.020572in}{0.721701in}}%
\pgfpathlineto{\pgfqpoint{4.021684in}{0.707885in}}%
\pgfpathlineto{\pgfqpoint{4.022089in}{0.710437in}}%
\pgfpathlineto{\pgfqpoint{4.022695in}{0.741647in}}%
\pgfpathlineto{\pgfqpoint{4.023605in}{0.725046in}}%
\pgfpathlineto{\pgfqpoint{4.024111in}{0.717697in}}%
\pgfpathlineto{\pgfqpoint{4.025021in}{0.707770in}}%
\pgfpathlineto{\pgfqpoint{4.025526in}{0.707880in}}%
\pgfpathlineto{\pgfqpoint{4.027548in}{0.709134in}}%
\pgfpathlineto{\pgfqpoint{4.028154in}{0.720232in}}%
\pgfpathlineto{\pgfqpoint{4.029064in}{0.715647in}}%
\pgfpathlineto{\pgfqpoint{4.029368in}{0.714787in}}%
\pgfpathlineto{\pgfqpoint{4.030581in}{0.709152in}}%
\pgfpathlineto{\pgfqpoint{4.031187in}{0.709247in}}%
\pgfpathlineto{\pgfqpoint{4.031895in}{0.710107in}}%
\pgfpathlineto{\pgfqpoint{4.032198in}{0.710870in}}%
\pgfpathlineto{\pgfqpoint{4.032603in}{0.709295in}}%
\pgfpathlineto{\pgfqpoint{4.033007in}{0.708433in}}%
\pgfpathlineto{\pgfqpoint{4.033512in}{0.709890in}}%
\pgfpathlineto{\pgfqpoint{4.033816in}{0.710449in}}%
\pgfpathlineto{\pgfqpoint{4.034422in}{0.709351in}}%
\pgfpathlineto{\pgfqpoint{4.035231in}{0.709026in}}%
\pgfpathlineto{\pgfqpoint{4.035534in}{0.709704in}}%
\pgfpathlineto{\pgfqpoint{4.035939in}{0.710902in}}%
\pgfpathlineto{\pgfqpoint{4.036343in}{0.708820in}}%
\pgfpathlineto{\pgfqpoint{4.037657in}{0.707755in}}%
\pgfpathlineto{\pgfqpoint{4.045644in}{0.708350in}}%
\pgfpathlineto{\pgfqpoint{4.047059in}{0.707833in}}%
\pgfpathlineto{\pgfqpoint{4.049890in}{0.708392in}}%
\pgfpathlineto{\pgfqpoint{4.051406in}{0.718523in}}%
\pgfpathlineto{\pgfqpoint{4.051608in}{0.716630in}}%
\pgfpathlineto{\pgfqpoint{4.052619in}{0.707662in}}%
\pgfpathlineto{\pgfqpoint{4.053125in}{0.707757in}}%
\pgfpathlineto{\pgfqpoint{4.054439in}{0.708457in}}%
\pgfpathlineto{\pgfqpoint{4.056258in}{0.722805in}}%
\pgfpathlineto{\pgfqpoint{4.056865in}{0.738360in}}%
\pgfpathlineto{\pgfqpoint{4.057472in}{0.727205in}}%
\pgfpathlineto{\pgfqpoint{4.058381in}{0.721750in}}%
\pgfpathlineto{\pgfqpoint{4.058685in}{0.725519in}}%
\pgfpathlineto{\pgfqpoint{4.059089in}{0.740541in}}%
\pgfpathlineto{\pgfqpoint{4.059594in}{0.720356in}}%
\pgfpathlineto{\pgfqpoint{4.060909in}{0.709365in}}%
\pgfpathlineto{\pgfqpoint{4.061313in}{0.708556in}}%
\pgfpathlineto{\pgfqpoint{4.061920in}{0.709615in}}%
\pgfpathlineto{\pgfqpoint{4.062728in}{0.712286in}}%
\pgfpathlineto{\pgfqpoint{4.063133in}{0.721650in}}%
\pgfpathlineto{\pgfqpoint{4.063739in}{0.710009in}}%
\pgfpathlineto{\pgfqpoint{4.063840in}{0.709809in}}%
\pgfpathlineto{\pgfqpoint{4.064043in}{0.710825in}}%
\pgfpathlineto{\pgfqpoint{4.064346in}{0.713337in}}%
\pgfpathlineto{\pgfqpoint{4.064851in}{0.709041in}}%
\pgfpathlineto{\pgfqpoint{4.066064in}{0.708051in}}%
\pgfpathlineto{\pgfqpoint{4.067075in}{0.708766in}}%
\pgfpathlineto{\pgfqpoint{4.068592in}{0.715593in}}%
\pgfpathlineto{\pgfqpoint{4.068895in}{0.712519in}}%
\pgfpathlineto{\pgfqpoint{4.070310in}{0.707180in}}%
\pgfpathlineto{\pgfqpoint{4.072534in}{0.707939in}}%
\pgfpathlineto{\pgfqpoint{4.077791in}{0.708709in}}%
\pgfpathlineto{\pgfqpoint{4.080521in}{0.711153in}}%
\pgfpathlineto{\pgfqpoint{4.081330in}{0.738526in}}%
\pgfpathlineto{\pgfqpoint{4.082239in}{0.721601in}}%
\pgfpathlineto{\pgfqpoint{4.083351in}{0.708873in}}%
\pgfpathlineto{\pgfqpoint{4.084362in}{0.709991in}}%
\pgfpathlineto{\pgfqpoint{4.085879in}{0.714969in}}%
\pgfpathlineto{\pgfqpoint{4.086081in}{0.714833in}}%
\pgfpathlineto{\pgfqpoint{4.087698in}{0.710372in}}%
\pgfpathlineto{\pgfqpoint{4.088709in}{0.708711in}}%
\pgfpathlineto{\pgfqpoint{4.088912in}{0.708801in}}%
\pgfpathlineto{\pgfqpoint{4.089215in}{0.712352in}}%
\pgfpathlineto{\pgfqpoint{4.089619in}{0.725247in}}%
\pgfpathlineto{\pgfqpoint{4.090428in}{0.718662in}}%
\pgfpathlineto{\pgfqpoint{4.090832in}{0.713255in}}%
\pgfpathlineto{\pgfqpoint{4.091439in}{0.708638in}}%
\pgfpathlineto{\pgfqpoint{4.092147in}{0.709055in}}%
\pgfpathlineto{\pgfqpoint{4.092551in}{0.709038in}}%
\pgfpathlineto{\pgfqpoint{4.092652in}{0.709455in}}%
\pgfpathlineto{\pgfqpoint{4.093056in}{0.721486in}}%
\pgfpathlineto{\pgfqpoint{4.093461in}{0.736907in}}%
\pgfpathlineto{\pgfqpoint{4.094067in}{0.720935in}}%
\pgfpathlineto{\pgfqpoint{4.096797in}{0.708151in}}%
\pgfpathlineto{\pgfqpoint{4.096898in}{0.708173in}}%
\pgfpathlineto{\pgfqpoint{4.098111in}{0.709107in}}%
\pgfpathlineto{\pgfqpoint{4.098617in}{0.725537in}}%
\pgfpathlineto{\pgfqpoint{4.099021in}{0.740951in}}%
\pgfpathlineto{\pgfqpoint{4.099830in}{0.731413in}}%
\pgfpathlineto{\pgfqpoint{4.102559in}{0.709593in}}%
\pgfpathlineto{\pgfqpoint{4.106199in}{0.708252in}}%
\pgfpathlineto{\pgfqpoint{4.112163in}{0.708712in}}%
\pgfpathlineto{\pgfqpoint{4.117117in}{0.709447in}}%
\pgfpathlineto{\pgfqpoint{4.118936in}{0.712937in}}%
\pgfpathlineto{\pgfqpoint{4.119341in}{0.711123in}}%
\pgfpathlineto{\pgfqpoint{4.120756in}{0.709383in}}%
\pgfpathlineto{\pgfqpoint{4.121059in}{0.711027in}}%
\pgfpathlineto{\pgfqpoint{4.121666in}{0.745592in}}%
\pgfpathlineto{\pgfqpoint{4.121767in}{0.748369in}}%
\pgfpathlineto{\pgfqpoint{4.122171in}{0.733978in}}%
\pgfpathlineto{\pgfqpoint{4.124092in}{0.710212in}}%
\pgfpathlineto{\pgfqpoint{4.126417in}{0.709016in}}%
\pgfpathlineto{\pgfqpoint{4.126619in}{0.709593in}}%
\pgfpathlineto{\pgfqpoint{4.128035in}{0.710776in}}%
\pgfpathlineto{\pgfqpoint{4.128540in}{0.710367in}}%
\pgfpathlineto{\pgfqpoint{4.128843in}{0.711495in}}%
\pgfpathlineto{\pgfqpoint{4.129248in}{0.713799in}}%
\pgfpathlineto{\pgfqpoint{4.130057in}{0.712560in}}%
\pgfpathlineto{\pgfqpoint{4.130360in}{0.711760in}}%
\pgfpathlineto{\pgfqpoint{4.131876in}{0.708276in}}%
\pgfpathlineto{\pgfqpoint{4.144311in}{0.709156in}}%
\pgfpathlineto{\pgfqpoint{4.144917in}{0.720841in}}%
\pgfpathlineto{\pgfqpoint{4.145726in}{0.715136in}}%
\pgfpathlineto{\pgfqpoint{4.147647in}{0.709058in}}%
\pgfpathlineto{\pgfqpoint{4.149264in}{0.708706in}}%
\pgfpathlineto{\pgfqpoint{4.149770in}{0.710549in}}%
\pgfpathlineto{\pgfqpoint{4.150680in}{0.716795in}}%
\pgfpathlineto{\pgfqpoint{4.151286in}{0.714386in}}%
\pgfpathlineto{\pgfqpoint{4.152095in}{0.708009in}}%
\pgfpathlineto{\pgfqpoint{4.152803in}{0.708191in}}%
\pgfpathlineto{\pgfqpoint{4.153207in}{0.708726in}}%
\pgfpathlineto{\pgfqpoint{4.153712in}{0.720228in}}%
\pgfpathlineto{\pgfqpoint{4.154016in}{0.726455in}}%
\pgfpathlineto{\pgfqpoint{4.154926in}{0.725523in}}%
\pgfpathlineto{\pgfqpoint{4.155128in}{0.725648in}}%
\pgfpathlineto{\pgfqpoint{4.155330in}{0.724759in}}%
\pgfpathlineto{\pgfqpoint{4.157453in}{0.707938in}}%
\pgfpathlineto{\pgfqpoint{4.158262in}{0.709396in}}%
\pgfpathlineto{\pgfqpoint{4.158868in}{0.722133in}}%
\pgfpathlineto{\pgfqpoint{4.159778in}{0.716734in}}%
\pgfpathlineto{\pgfqpoint{4.160890in}{0.708057in}}%
\pgfpathlineto{\pgfqpoint{4.161294in}{0.709145in}}%
\pgfpathlineto{\pgfqpoint{4.161901in}{0.721386in}}%
\pgfpathlineto{\pgfqpoint{4.162710in}{0.713549in}}%
\pgfpathlineto{\pgfqpoint{4.163417in}{0.708954in}}%
\pgfpathlineto{\pgfqpoint{4.164327in}{0.710604in}}%
\pgfpathlineto{\pgfqpoint{4.164833in}{0.723222in}}%
\pgfpathlineto{\pgfqpoint{4.165338in}{0.710056in}}%
\pgfpathlineto{\pgfqpoint{4.166046in}{0.711566in}}%
\pgfpathlineto{\pgfqpoint{4.166652in}{0.706693in}}%
\pgfpathlineto{\pgfqpoint{4.167057in}{0.707488in}}%
\pgfpathlineto{\pgfqpoint{4.167562in}{0.711477in}}%
\pgfpathlineto{\pgfqpoint{4.168270in}{0.708440in}}%
\pgfpathlineto{\pgfqpoint{4.169887in}{0.706642in}}%
\pgfpathlineto{\pgfqpoint{4.170393in}{0.708452in}}%
\pgfpathlineto{\pgfqpoint{4.170595in}{0.708581in}}%
\pgfpathlineto{\pgfqpoint{4.170999in}{0.707520in}}%
\pgfpathlineto{\pgfqpoint{4.172718in}{0.706889in}}%
\pgfpathlineto{\pgfqpoint{4.174032in}{0.707627in}}%
\pgfpathlineto{\pgfqpoint{4.174538in}{0.710695in}}%
\pgfpathlineto{\pgfqpoint{4.175144in}{0.708209in}}%
\pgfpathlineto{\pgfqpoint{4.176863in}{0.708832in}}%
\pgfpathlineto{\pgfqpoint{4.178379in}{0.707843in}}%
\pgfpathlineto{\pgfqpoint{4.178784in}{0.706624in}}%
\pgfpathlineto{\pgfqpoint{4.179592in}{0.706919in}}%
\pgfpathlineto{\pgfqpoint{4.180300in}{0.707779in}}%
\pgfpathlineto{\pgfqpoint{4.181614in}{0.708391in}}%
\pgfpathlineto{\pgfqpoint{4.182827in}{0.713791in}}%
\pgfpathlineto{\pgfqpoint{4.183333in}{0.723545in}}%
\pgfpathlineto{\pgfqpoint{4.183939in}{0.714217in}}%
\pgfpathlineto{\pgfqpoint{4.184142in}{0.713807in}}%
\pgfpathlineto{\pgfqpoint{4.184647in}{0.715590in}}%
\pgfpathlineto{\pgfqpoint{4.186062in}{0.721688in}}%
\pgfpathlineto{\pgfqpoint{4.186265in}{0.717510in}}%
\pgfpathlineto{\pgfqpoint{4.187579in}{0.705494in}}%
\pgfpathlineto{\pgfqpoint{4.187680in}{0.705518in}}%
\pgfpathlineto{\pgfqpoint{4.188185in}{0.707023in}}%
\pgfpathlineto{\pgfqpoint{4.188792in}{0.710488in}}%
\pgfpathlineto{\pgfqpoint{4.189500in}{0.709339in}}%
\pgfpathlineto{\pgfqpoint{4.189904in}{0.708267in}}%
\pgfpathlineto{\pgfqpoint{4.190207in}{0.710148in}}%
\pgfpathlineto{\pgfqpoint{4.190612in}{0.741107in}}%
\pgfpathlineto{\pgfqpoint{4.190915in}{0.774365in}}%
\pgfpathlineto{\pgfqpoint{4.191521in}{0.726447in}}%
\pgfpathlineto{\pgfqpoint{4.191623in}{0.729017in}}%
\pgfpathlineto{\pgfqpoint{4.192027in}{0.767163in}}%
\pgfpathlineto{\pgfqpoint{4.192532in}{0.717633in}}%
\pgfpathlineto{\pgfqpoint{4.193745in}{0.708387in}}%
\pgfpathlineto{\pgfqpoint{4.194959in}{0.708259in}}%
\pgfpathlineto{\pgfqpoint{4.196980in}{0.709600in}}%
\pgfpathlineto{\pgfqpoint{4.197385in}{0.712377in}}%
\pgfpathlineto{\pgfqpoint{4.197991in}{0.709490in}}%
\pgfpathlineto{\pgfqpoint{4.198194in}{0.709664in}}%
\pgfpathlineto{\pgfqpoint{4.198295in}{0.710067in}}%
\pgfpathlineto{\pgfqpoint{4.198598in}{0.711133in}}%
\pgfpathlineto{\pgfqpoint{4.199103in}{0.708944in}}%
\pgfpathlineto{\pgfqpoint{4.200418in}{0.708381in}}%
\pgfpathlineto{\pgfqpoint{4.202440in}{0.709751in}}%
\pgfpathlineto{\pgfqpoint{4.203046in}{0.708864in}}%
\pgfpathlineto{\pgfqpoint{4.204967in}{0.708786in}}%
\pgfpathlineto{\pgfqpoint{4.206180in}{0.710717in}}%
\pgfpathlineto{\pgfqpoint{4.206685in}{0.711156in}}%
\pgfpathlineto{\pgfqpoint{4.207292in}{0.710559in}}%
\pgfpathlineto{\pgfqpoint{4.208202in}{0.709041in}}%
\pgfpathlineto{\pgfqpoint{4.208606in}{0.709986in}}%
\pgfpathlineto{\pgfqpoint{4.209011in}{0.711837in}}%
\pgfpathlineto{\pgfqpoint{4.209718in}{0.710290in}}%
\pgfpathlineto{\pgfqpoint{4.210123in}{0.709606in}}%
\pgfpathlineto{\pgfqpoint{4.210527in}{0.708660in}}%
\pgfpathlineto{\pgfqpoint{4.211032in}{0.709877in}}%
\pgfpathlineto{\pgfqpoint{4.212347in}{0.714081in}}%
\pgfpathlineto{\pgfqpoint{4.212650in}{0.711547in}}%
\pgfpathlineto{\pgfqpoint{4.213257in}{0.707687in}}%
\pgfpathlineto{\pgfqpoint{4.213863in}{0.708992in}}%
\pgfpathlineto{\pgfqpoint{4.214672in}{0.720681in}}%
\pgfpathlineto{\pgfqpoint{4.215683in}{0.716081in}}%
\pgfpathlineto{\pgfqpoint{4.216492in}{0.705922in}}%
\pgfpathlineto{\pgfqpoint{4.217098in}{0.709210in}}%
\pgfpathlineto{\pgfqpoint{4.218513in}{0.714251in}}%
\pgfpathlineto{\pgfqpoint{4.218614in}{0.713615in}}%
\pgfpathlineto{\pgfqpoint{4.219423in}{0.704878in}}%
\pgfpathlineto{\pgfqpoint{4.219929in}{0.708267in}}%
\pgfpathlineto{\pgfqpoint{4.220333in}{0.713557in}}%
\pgfpathlineto{\pgfqpoint{4.220940in}{0.706814in}}%
\pgfpathlineto{\pgfqpoint{4.221142in}{0.706573in}}%
\pgfpathlineto{\pgfqpoint{4.221445in}{0.707662in}}%
\pgfpathlineto{\pgfqpoint{4.221951in}{0.718271in}}%
\pgfpathlineto{\pgfqpoint{4.222355in}{0.734674in}}%
\pgfpathlineto{\pgfqpoint{4.222961in}{0.718668in}}%
\pgfpathlineto{\pgfqpoint{4.223063in}{0.717876in}}%
\pgfpathlineto{\pgfqpoint{4.223568in}{0.722493in}}%
\pgfpathlineto{\pgfqpoint{4.223972in}{0.717546in}}%
\pgfpathlineto{\pgfqpoint{4.224579in}{0.711458in}}%
\pgfpathlineto{\pgfqpoint{4.225084in}{0.716492in}}%
\pgfpathlineto{\pgfqpoint{4.226399in}{0.755159in}}%
\pgfpathlineto{\pgfqpoint{4.226803in}{0.845794in}}%
\pgfpathlineto{\pgfqpoint{4.227410in}{0.760060in}}%
\pgfpathlineto{\pgfqpoint{4.228926in}{0.714734in}}%
\pgfpathlineto{\pgfqpoint{4.230240in}{0.710959in}}%
\pgfpathlineto{\pgfqpoint{4.231453in}{0.708789in}}%
\pgfpathlineto{\pgfqpoint{4.232060in}{0.708369in}}%
\pgfpathlineto{\pgfqpoint{4.232464in}{0.709241in}}%
\pgfpathlineto{\pgfqpoint{4.234183in}{0.712882in}}%
\pgfpathlineto{\pgfqpoint{4.234587in}{0.711894in}}%
\pgfpathlineto{\pgfqpoint{4.236811in}{0.708039in}}%
\pgfpathlineto{\pgfqpoint{4.237418in}{0.709142in}}%
\pgfpathlineto{\pgfqpoint{4.237721in}{0.709990in}}%
\pgfpathlineto{\pgfqpoint{4.238631in}{0.709549in}}%
\pgfpathlineto{\pgfqpoint{4.239541in}{0.708501in}}%
\pgfpathlineto{\pgfqpoint{4.239743in}{0.709240in}}%
\pgfpathlineto{\pgfqpoint{4.240248in}{0.721164in}}%
\pgfpathlineto{\pgfqpoint{4.240754in}{0.736838in}}%
\pgfpathlineto{\pgfqpoint{4.241462in}{0.727611in}}%
\pgfpathlineto{\pgfqpoint{4.244191in}{0.707975in}}%
\pgfpathlineto{\pgfqpoint{4.246314in}{0.708015in}}%
\pgfpathlineto{\pgfqpoint{4.252784in}{0.709110in}}%
\pgfpathlineto{\pgfqpoint{4.254604in}{0.709418in}}%
\pgfpathlineto{\pgfqpoint{4.255109in}{0.710787in}}%
\pgfpathlineto{\pgfqpoint{4.255817in}{0.709746in}}%
\pgfpathlineto{\pgfqpoint{4.261175in}{0.709279in}}%
\pgfpathlineto{\pgfqpoint{4.263096in}{0.711522in}}%
\pgfpathlineto{\pgfqpoint{4.263500in}{0.713017in}}%
\pgfpathlineto{\pgfqpoint{4.263904in}{0.710762in}}%
\pgfpathlineto{\pgfqpoint{4.265117in}{0.707626in}}%
\pgfpathlineto{\pgfqpoint{4.265320in}{0.707755in}}%
\pgfpathlineto{\pgfqpoint{4.266128in}{0.709469in}}%
\pgfpathlineto{\pgfqpoint{4.266836in}{0.708354in}}%
\pgfpathlineto{\pgfqpoint{4.267240in}{0.707938in}}%
\pgfpathlineto{\pgfqpoint{4.267544in}{0.708917in}}%
\pgfpathlineto{\pgfqpoint{4.268454in}{0.723841in}}%
\pgfpathlineto{\pgfqpoint{4.268757in}{0.726514in}}%
\pgfpathlineto{\pgfqpoint{4.269363in}{0.722111in}}%
\pgfpathlineto{\pgfqpoint{4.269464in}{0.722420in}}%
\pgfpathlineto{\pgfqpoint{4.270172in}{0.736135in}}%
\pgfpathlineto{\pgfqpoint{4.270576in}{0.725472in}}%
\pgfpathlineto{\pgfqpoint{4.271992in}{0.710242in}}%
\pgfpathlineto{\pgfqpoint{4.272497in}{0.706965in}}%
\pgfpathlineto{\pgfqpoint{4.273205in}{0.708057in}}%
\pgfpathlineto{\pgfqpoint{4.275429in}{0.708231in}}%
\pgfpathlineto{\pgfqpoint{4.277249in}{0.708800in}}%
\pgfpathlineto{\pgfqpoint{4.277350in}{0.708586in}}%
\pgfpathlineto{\pgfqpoint{4.278765in}{0.707887in}}%
\pgfpathlineto{\pgfqpoint{4.279169in}{0.708820in}}%
\pgfpathlineto{\pgfqpoint{4.279372in}{0.709492in}}%
\pgfpathlineto{\pgfqpoint{4.280079in}{0.707879in}}%
\pgfpathlineto{\pgfqpoint{4.280989in}{0.708536in}}%
\pgfpathlineto{\pgfqpoint{4.282607in}{0.713168in}}%
\pgfpathlineto{\pgfqpoint{4.282809in}{0.712712in}}%
\pgfpathlineto{\pgfqpoint{4.284628in}{0.708104in}}%
\pgfpathlineto{\pgfqpoint{4.287459in}{0.708307in}}%
\pgfpathlineto{\pgfqpoint{4.290593in}{0.708168in}}%
\pgfpathlineto{\pgfqpoint{4.292918in}{0.707526in}}%
\pgfpathlineto{\pgfqpoint{4.293525in}{0.708770in}}%
\pgfpathlineto{\pgfqpoint{4.294030in}{0.712016in}}%
\pgfpathlineto{\pgfqpoint{4.294940in}{0.711415in}}%
\pgfpathlineto{\pgfqpoint{4.296355in}{0.707904in}}%
\pgfpathlineto{\pgfqpoint{4.297063in}{0.709399in}}%
\pgfpathlineto{\pgfqpoint{4.298276in}{0.709111in}}%
\pgfpathlineto{\pgfqpoint{4.298883in}{0.708240in}}%
\pgfpathlineto{\pgfqpoint{4.299388in}{0.708198in}}%
\pgfpathlineto{\pgfqpoint{4.299590in}{0.708717in}}%
\pgfpathlineto{\pgfqpoint{4.302017in}{0.724852in}}%
\pgfpathlineto{\pgfqpoint{4.302421in}{0.716932in}}%
\pgfpathlineto{\pgfqpoint{4.303129in}{0.717212in}}%
\pgfpathlineto{\pgfqpoint{4.304038in}{0.708219in}}%
\pgfpathlineto{\pgfqpoint{4.304544in}{0.707685in}}%
\pgfpathlineto{\pgfqpoint{4.304948in}{0.708539in}}%
\pgfpathlineto{\pgfqpoint{4.305353in}{0.710181in}}%
\pgfpathlineto{\pgfqpoint{4.306060in}{0.708723in}}%
\pgfpathlineto{\pgfqpoint{4.307981in}{0.707565in}}%
\pgfpathlineto{\pgfqpoint{4.308082in}{0.707886in}}%
\pgfpathlineto{\pgfqpoint{4.308385in}{0.726637in}}%
\pgfpathlineto{\pgfqpoint{4.309700in}{0.804876in}}%
\pgfpathlineto{\pgfqpoint{4.309194in}{0.726465in}}%
\pgfpathlineto{\pgfqpoint{4.309902in}{0.795313in}}%
\pgfpathlineto{\pgfqpoint{4.311519in}{0.721313in}}%
\pgfpathlineto{\pgfqpoint{4.313440in}{0.709272in}}%
\pgfpathlineto{\pgfqpoint{4.324055in}{0.709597in}}%
\pgfpathlineto{\pgfqpoint{4.325470in}{0.710567in}}%
\pgfpathlineto{\pgfqpoint{4.325672in}{0.710423in}}%
\pgfpathlineto{\pgfqpoint{4.326886in}{0.710061in}}%
\pgfpathlineto{\pgfqpoint{4.327088in}{0.710413in}}%
\pgfpathlineto{\pgfqpoint{4.328200in}{0.712466in}}%
\pgfpathlineto{\pgfqpoint{4.328705in}{0.711753in}}%
\pgfpathlineto{\pgfqpoint{4.330424in}{0.709487in}}%
\pgfpathlineto{\pgfqpoint{4.330626in}{0.709629in}}%
\pgfpathlineto{\pgfqpoint{4.331334in}{0.711951in}}%
\pgfpathlineto{\pgfqpoint{4.333153in}{0.719474in}}%
\pgfpathlineto{\pgfqpoint{4.333356in}{0.717624in}}%
\pgfpathlineto{\pgfqpoint{4.334973in}{0.707983in}}%
\pgfpathlineto{\pgfqpoint{4.336287in}{0.708399in}}%
\pgfpathlineto{\pgfqpoint{4.344173in}{0.708293in}}%
\pgfpathlineto{\pgfqpoint{4.344476in}{0.719223in}}%
\pgfpathlineto{\pgfqpoint{4.344981in}{0.762102in}}%
\pgfpathlineto{\pgfqpoint{4.345689in}{0.732491in}}%
\pgfpathlineto{\pgfqpoint{4.346295in}{0.733258in}}%
\pgfpathlineto{\pgfqpoint{4.349126in}{0.709351in}}%
\pgfpathlineto{\pgfqpoint{4.350541in}{0.709074in}}%
\pgfpathlineto{\pgfqpoint{4.366110in}{0.711003in}}%
\pgfpathlineto{\pgfqpoint{4.367727in}{0.717400in}}%
\pgfpathlineto{\pgfqpoint{4.367929in}{0.715717in}}%
\pgfpathlineto{\pgfqpoint{4.368536in}{0.707947in}}%
\pgfpathlineto{\pgfqpoint{4.368940in}{0.713903in}}%
\pgfpathlineto{\pgfqpoint{4.369244in}{0.721693in}}%
\pgfpathlineto{\pgfqpoint{4.369850in}{0.709583in}}%
\pgfpathlineto{\pgfqpoint{4.369951in}{0.709258in}}%
\pgfpathlineto{\pgfqpoint{4.370457in}{0.710974in}}%
\pgfpathlineto{\pgfqpoint{4.371164in}{0.716308in}}%
\pgfpathlineto{\pgfqpoint{4.371569in}{0.720103in}}%
\pgfpathlineto{\pgfqpoint{4.372175in}{0.716365in}}%
\pgfpathlineto{\pgfqpoint{4.372276in}{0.716233in}}%
\pgfpathlineto{\pgfqpoint{4.372479in}{0.717053in}}%
\pgfpathlineto{\pgfqpoint{4.373287in}{0.724529in}}%
\pgfpathlineto{\pgfqpoint{4.373894in}{0.720206in}}%
\pgfpathlineto{\pgfqpoint{4.376219in}{0.704628in}}%
\pgfpathlineto{\pgfqpoint{4.378140in}{0.705705in}}%
\pgfpathlineto{\pgfqpoint{4.379656in}{0.706790in}}%
\pgfpathlineto{\pgfqpoint{4.381375in}{0.706303in}}%
\pgfpathlineto{\pgfqpoint{4.385520in}{0.710521in}}%
\pgfpathlineto{\pgfqpoint{4.385621in}{0.710670in}}%
\pgfpathlineto{\pgfqpoint{4.386126in}{0.709644in}}%
\pgfpathlineto{\pgfqpoint{4.386227in}{0.709636in}}%
\pgfpathlineto{\pgfqpoint{4.386632in}{0.709824in}}%
\pgfpathlineto{\pgfqpoint{4.386935in}{0.709024in}}%
\pgfpathlineto{\pgfqpoint{4.388249in}{0.707946in}}%
\pgfpathlineto{\pgfqpoint{4.388350in}{0.707979in}}%
\pgfpathlineto{\pgfqpoint{4.392798in}{0.710355in}}%
\pgfpathlineto{\pgfqpoint{4.393203in}{0.708714in}}%
\pgfpathlineto{\pgfqpoint{4.393506in}{0.708688in}}%
\pgfpathlineto{\pgfqpoint{4.393607in}{0.709100in}}%
\pgfpathlineto{\pgfqpoint{4.394719in}{0.723371in}}%
\pgfpathlineto{\pgfqpoint{4.395124in}{0.767998in}}%
\pgfpathlineto{\pgfqpoint{4.395730in}{0.720560in}}%
\pgfpathlineto{\pgfqpoint{4.397247in}{0.708823in}}%
\pgfpathlineto{\pgfqpoint{4.398561in}{0.708943in}}%
\pgfpathlineto{\pgfqpoint{4.400684in}{0.709275in}}%
\pgfpathlineto{\pgfqpoint{4.401290in}{0.710902in}}%
\pgfpathlineto{\pgfqpoint{4.402099in}{0.710270in}}%
\pgfpathlineto{\pgfqpoint{4.404121in}{0.709268in}}%
\pgfpathlineto{\pgfqpoint{4.404222in}{0.709678in}}%
\pgfpathlineto{\pgfqpoint{4.404930in}{0.721663in}}%
\pgfpathlineto{\pgfqpoint{4.406143in}{0.716787in}}%
\pgfpathlineto{\pgfqpoint{4.406446in}{0.713916in}}%
\pgfpathlineto{\pgfqpoint{4.407255in}{0.708616in}}%
\pgfpathlineto{\pgfqpoint{4.407861in}{0.708701in}}%
\pgfpathlineto{\pgfqpoint{4.410894in}{0.709454in}}%
\pgfpathlineto{\pgfqpoint{4.411602in}{0.716577in}}%
\pgfpathlineto{\pgfqpoint{4.413624in}{0.734660in}}%
\pgfpathlineto{\pgfqpoint{4.413725in}{0.734204in}}%
\pgfpathlineto{\pgfqpoint{4.418375in}{0.708680in}}%
\pgfpathlineto{\pgfqpoint{4.418476in}{0.708698in}}%
\pgfpathlineto{\pgfqpoint{4.421913in}{0.710685in}}%
\pgfpathlineto{\pgfqpoint{4.422520in}{0.715496in}}%
\pgfpathlineto{\pgfqpoint{4.423329in}{0.713982in}}%
\pgfpathlineto{\pgfqpoint{4.423733in}{0.713203in}}%
\pgfpathlineto{\pgfqpoint{4.425654in}{0.708451in}}%
\pgfpathlineto{\pgfqpoint{4.427271in}{0.708312in}}%
\pgfpathlineto{\pgfqpoint{4.427372in}{0.708469in}}%
\pgfpathlineto{\pgfqpoint{4.428889in}{0.712607in}}%
\pgfpathlineto{\pgfqpoint{4.429596in}{0.711085in}}%
\pgfpathlineto{\pgfqpoint{4.430304in}{0.709286in}}%
\pgfpathlineto{\pgfqpoint{4.430708in}{0.710561in}}%
\pgfpathlineto{\pgfqpoint{4.431012in}{0.711531in}}%
\pgfpathlineto{\pgfqpoint{4.431517in}{0.709574in}}%
\pgfpathlineto{\pgfqpoint{4.433438in}{0.707242in}}%
\pgfpathlineto{\pgfqpoint{4.435561in}{0.707312in}}%
\pgfpathlineto{\pgfqpoint{4.438695in}{0.706317in}}%
\pgfpathlineto{\pgfqpoint{4.440515in}{0.706849in}}%
\pgfpathlineto{\pgfqpoint{4.442233in}{0.708238in}}%
\pgfpathlineto{\pgfqpoint{4.442739in}{0.712629in}}%
\pgfpathlineto{\pgfqpoint{4.443345in}{0.709103in}}%
\pgfpathlineto{\pgfqpoint{4.443648in}{0.708713in}}%
\pgfpathlineto{\pgfqpoint{4.444154in}{0.709615in}}%
\pgfpathlineto{\pgfqpoint{4.445165in}{0.713270in}}%
\pgfpathlineto{\pgfqpoint{4.445974in}{0.711449in}}%
\pgfpathlineto{\pgfqpoint{4.446782in}{0.712890in}}%
\pgfpathlineto{\pgfqpoint{4.447187in}{0.711004in}}%
\pgfpathlineto{\pgfqpoint{4.447894in}{0.707735in}}%
\pgfpathlineto{\pgfqpoint{4.448602in}{0.708104in}}%
\pgfpathlineto{\pgfqpoint{4.449714in}{0.706464in}}%
\pgfpathlineto{\pgfqpoint{4.450017in}{0.707608in}}%
\pgfpathlineto{\pgfqpoint{4.450523in}{0.712042in}}%
\pgfpathlineto{\pgfqpoint{4.451230in}{0.709270in}}%
\pgfpathlineto{\pgfqpoint{4.452646in}{0.706258in}}%
\pgfpathlineto{\pgfqpoint{4.453050in}{0.705927in}}%
\pgfpathlineto{\pgfqpoint{4.453556in}{0.706745in}}%
\pgfpathlineto{\pgfqpoint{4.454567in}{0.712453in}}%
\pgfpathlineto{\pgfqpoint{4.455375in}{0.709874in}}%
\pgfpathlineto{\pgfqpoint{4.456184in}{0.706984in}}%
\pgfpathlineto{\pgfqpoint{4.456588in}{0.709250in}}%
\pgfpathlineto{\pgfqpoint{4.457296in}{0.752286in}}%
\pgfpathlineto{\pgfqpoint{4.458307in}{0.727773in}}%
\pgfpathlineto{\pgfqpoint{4.458610in}{0.723340in}}%
\pgfpathlineto{\pgfqpoint{4.459419in}{0.708489in}}%
\pgfpathlineto{\pgfqpoint{4.460026in}{0.713177in}}%
\pgfpathlineto{\pgfqpoint{4.461340in}{0.722735in}}%
\pgfpathlineto{\pgfqpoint{4.461441in}{0.721706in}}%
\pgfpathlineto{\pgfqpoint{4.462351in}{0.705651in}}%
\pgfpathlineto{\pgfqpoint{4.462957in}{0.708838in}}%
\pgfpathlineto{\pgfqpoint{4.464373in}{0.720467in}}%
\pgfpathlineto{\pgfqpoint{4.464474in}{0.719098in}}%
\pgfpathlineto{\pgfqpoint{4.465181in}{0.706016in}}%
\pgfpathlineto{\pgfqpoint{4.465990in}{0.706544in}}%
\pgfpathlineto{\pgfqpoint{4.469225in}{0.709616in}}%
\pgfpathlineto{\pgfqpoint{4.470843in}{0.717377in}}%
\pgfpathlineto{\pgfqpoint{4.471247in}{0.724407in}}%
\pgfpathlineto{\pgfqpoint{4.471752in}{0.713137in}}%
\pgfpathlineto{\pgfqpoint{4.473067in}{0.708106in}}%
\pgfpathlineto{\pgfqpoint{4.473269in}{0.708172in}}%
\pgfpathlineto{\pgfqpoint{4.479031in}{0.708440in}}%
\pgfpathlineto{\pgfqpoint{4.479436in}{0.711766in}}%
\pgfpathlineto{\pgfqpoint{4.481457in}{0.744060in}}%
\pgfpathlineto{\pgfqpoint{4.481761in}{0.738117in}}%
\pgfpathlineto{\pgfqpoint{4.483479in}{0.717128in}}%
\pgfpathlineto{\pgfqpoint{4.484895in}{0.708931in}}%
\pgfpathlineto{\pgfqpoint{4.485400in}{0.709252in}}%
\pgfpathlineto{\pgfqpoint{4.485804in}{0.709385in}}%
\pgfpathlineto{\pgfqpoint{4.486411in}{0.708699in}}%
\pgfpathlineto{\pgfqpoint{4.486714in}{0.709438in}}%
\pgfpathlineto{\pgfqpoint{4.487220in}{0.715921in}}%
\pgfpathlineto{\pgfqpoint{4.487927in}{0.710998in}}%
\pgfpathlineto{\pgfqpoint{4.488332in}{0.711909in}}%
\pgfpathlineto{\pgfqpoint{4.488635in}{0.709961in}}%
\pgfpathlineto{\pgfqpoint{4.489444in}{0.708129in}}%
\pgfpathlineto{\pgfqpoint{4.489949in}{0.708297in}}%
\pgfpathlineto{\pgfqpoint{4.492881in}{0.708655in}}%
\pgfpathlineto{\pgfqpoint{4.502687in}{0.710522in}}%
\pgfpathlineto{\pgfqpoint{4.504203in}{0.714105in}}%
\pgfpathlineto{\pgfqpoint{4.504305in}{0.714086in}}%
\pgfpathlineto{\pgfqpoint{4.505518in}{0.711969in}}%
\pgfpathlineto{\pgfqpoint{4.506023in}{0.708110in}}%
\pgfpathlineto{\pgfqpoint{4.506427in}{0.711299in}}%
\pgfpathlineto{\pgfqpoint{4.506832in}{0.720912in}}%
\pgfpathlineto{\pgfqpoint{4.507641in}{0.715051in}}%
\pgfpathlineto{\pgfqpoint{4.508449in}{0.730906in}}%
\pgfpathlineto{\pgfqpoint{4.508955in}{0.721216in}}%
\pgfpathlineto{\pgfqpoint{4.510471in}{0.711644in}}%
\pgfpathlineto{\pgfqpoint{4.511684in}{0.708132in}}%
\pgfpathlineto{\pgfqpoint{4.512089in}{0.710446in}}%
\pgfpathlineto{\pgfqpoint{4.512392in}{0.712150in}}%
\pgfpathlineto{\pgfqpoint{4.512897in}{0.708033in}}%
\pgfpathlineto{\pgfqpoint{4.514313in}{0.706695in}}%
\pgfpathlineto{\pgfqpoint{4.516941in}{0.708649in}}%
\pgfpathlineto{\pgfqpoint{4.517548in}{0.717430in}}%
\pgfpathlineto{\pgfqpoint{4.518357in}{0.711709in}}%
\pgfpathlineto{\pgfqpoint{4.518559in}{0.711550in}}%
\pgfpathlineto{\pgfqpoint{4.518660in}{0.711011in}}%
\pgfpathlineto{\pgfqpoint{4.519367in}{0.706972in}}%
\pgfpathlineto{\pgfqpoint{4.520176in}{0.707249in}}%
\pgfpathlineto{\pgfqpoint{4.521895in}{0.708829in}}%
\pgfpathlineto{\pgfqpoint{4.522704in}{0.717567in}}%
\pgfpathlineto{\pgfqpoint{4.523613in}{0.714532in}}%
\pgfpathlineto{\pgfqpoint{4.524422in}{0.706511in}}%
\pgfpathlineto{\pgfqpoint{4.525332in}{0.707106in}}%
\pgfpathlineto{\pgfqpoint{4.525635in}{0.708481in}}%
\pgfpathlineto{\pgfqpoint{4.526242in}{0.727193in}}%
\pgfpathlineto{\pgfqpoint{4.527152in}{0.718413in}}%
\pgfpathlineto{\pgfqpoint{4.529376in}{0.707602in}}%
\pgfpathlineto{\pgfqpoint{4.529780in}{0.707760in}}%
\pgfpathlineto{\pgfqpoint{4.533622in}{0.710018in}}%
\pgfpathlineto{\pgfqpoint{4.534329in}{0.725044in}}%
\pgfpathlineto{\pgfqpoint{4.535542in}{0.720096in}}%
\pgfpathlineto{\pgfqpoint{4.537261in}{0.707949in}}%
\pgfpathlineto{\pgfqpoint{4.537463in}{0.707995in}}%
\pgfpathlineto{\pgfqpoint{4.538272in}{0.709667in}}%
\pgfpathlineto{\pgfqpoint{4.539687in}{0.712025in}}%
\pgfpathlineto{\pgfqpoint{4.538980in}{0.709007in}}%
\pgfpathlineto{\pgfqpoint{4.539889in}{0.711032in}}%
\pgfpathlineto{\pgfqpoint{4.540698in}{0.708135in}}%
\pgfpathlineto{\pgfqpoint{4.541305in}{0.708355in}}%
\pgfpathlineto{\pgfqpoint{4.545247in}{0.710061in}}%
\pgfpathlineto{\pgfqpoint{4.545551in}{0.711270in}}%
\pgfpathlineto{\pgfqpoint{4.546056in}{0.708887in}}%
\pgfpathlineto{\pgfqpoint{4.546258in}{0.708638in}}%
\pgfpathlineto{\pgfqpoint{4.546764in}{0.709860in}}%
\pgfpathlineto{\pgfqpoint{4.547168in}{0.708651in}}%
\pgfpathlineto{\pgfqpoint{4.547775in}{0.708036in}}%
\pgfpathlineto{\pgfqpoint{4.548280in}{0.708469in}}%
\pgfpathlineto{\pgfqpoint{4.548685in}{0.711322in}}%
\pgfpathlineto{\pgfqpoint{4.550504in}{0.738579in}}%
\pgfpathlineto{\pgfqpoint{4.550706in}{0.732784in}}%
\pgfpathlineto{\pgfqpoint{4.552223in}{0.709849in}}%
\pgfpathlineto{\pgfqpoint{4.555660in}{0.708132in}}%
\pgfpathlineto{\pgfqpoint{4.555761in}{0.708253in}}%
\pgfpathlineto{\pgfqpoint{4.557075in}{0.708629in}}%
\pgfpathlineto{\pgfqpoint{4.557176in}{0.708531in}}%
\pgfpathlineto{\pgfqpoint{4.558794in}{0.708242in}}%
\pgfpathlineto{\pgfqpoint{4.564253in}{0.708939in}}%
\pgfpathlineto{\pgfqpoint{4.566174in}{0.709049in}}%
\pgfpathlineto{\pgfqpoint{4.572037in}{0.710027in}}%
\pgfpathlineto{\pgfqpoint{4.572644in}{0.720904in}}%
\pgfpathlineto{\pgfqpoint{4.573048in}{0.709799in}}%
\pgfpathlineto{\pgfqpoint{4.573250in}{0.708790in}}%
\pgfpathlineto{\pgfqpoint{4.573452in}{0.713458in}}%
\pgfpathlineto{\pgfqpoint{4.573857in}{0.749058in}}%
\pgfpathlineto{\pgfqpoint{4.574463in}{0.713314in}}%
\pgfpathlineto{\pgfqpoint{4.574564in}{0.713984in}}%
\pgfpathlineto{\pgfqpoint{4.575070in}{0.741090in}}%
\pgfpathlineto{\pgfqpoint{4.575676in}{0.714600in}}%
\pgfpathlineto{\pgfqpoint{4.576789in}{0.710377in}}%
\pgfpathlineto{\pgfqpoint{4.576890in}{0.710580in}}%
\pgfpathlineto{\pgfqpoint{4.577294in}{0.716687in}}%
\pgfpathlineto{\pgfqpoint{4.577799in}{0.731750in}}%
\pgfpathlineto{\pgfqpoint{4.578406in}{0.720695in}}%
\pgfpathlineto{\pgfqpoint{4.581136in}{0.708098in}}%
\pgfpathlineto{\pgfqpoint{4.581338in}{0.708174in}}%
\pgfpathlineto{\pgfqpoint{4.581944in}{0.710078in}}%
\pgfpathlineto{\pgfqpoint{4.583157in}{0.727477in}}%
\pgfpathlineto{\pgfqpoint{4.583966in}{0.719651in}}%
\pgfpathlineto{\pgfqpoint{4.585179in}{0.709638in}}%
\pgfpathlineto{\pgfqpoint{4.585584in}{0.711009in}}%
\pgfpathlineto{\pgfqpoint{4.586797in}{0.724502in}}%
\pgfpathlineto{\pgfqpoint{4.587504in}{0.719580in}}%
\pgfpathlineto{\pgfqpoint{4.588819in}{0.709622in}}%
\pgfpathlineto{\pgfqpoint{4.589223in}{0.710445in}}%
\pgfpathlineto{\pgfqpoint{4.589324in}{0.710629in}}%
\pgfpathlineto{\pgfqpoint{4.589627in}{0.709282in}}%
\pgfpathlineto{\pgfqpoint{4.590840in}{0.706058in}}%
\pgfpathlineto{\pgfqpoint{4.591043in}{0.706145in}}%
\pgfpathlineto{\pgfqpoint{4.591649in}{0.708614in}}%
\pgfpathlineto{\pgfqpoint{4.593166in}{0.711373in}}%
\pgfpathlineto{\pgfqpoint{4.593469in}{0.710356in}}%
\pgfpathlineto{\pgfqpoint{4.594278in}{0.705331in}}%
\pgfpathlineto{\pgfqpoint{4.594783in}{0.707134in}}%
\pgfpathlineto{\pgfqpoint{4.595086in}{0.708113in}}%
\pgfpathlineto{\pgfqpoint{4.595592in}{0.705491in}}%
\pgfpathlineto{\pgfqpoint{4.595794in}{0.705500in}}%
\pgfpathlineto{\pgfqpoint{4.596198in}{0.706345in}}%
\pgfpathlineto{\pgfqpoint{4.598321in}{0.708010in}}%
\pgfpathlineto{\pgfqpoint{4.601354in}{0.754855in}}%
\pgfpathlineto{\pgfqpoint{4.601961in}{0.740944in}}%
\pgfpathlineto{\pgfqpoint{4.603882in}{0.716486in}}%
\pgfpathlineto{\pgfqpoint{4.604488in}{0.714866in}}%
\pgfpathlineto{\pgfqpoint{4.605196in}{0.715343in}}%
\pgfpathlineto{\pgfqpoint{4.606005in}{0.718525in}}%
\pgfpathlineto{\pgfqpoint{4.606510in}{0.716166in}}%
\pgfpathlineto{\pgfqpoint{4.608330in}{0.709595in}}%
\pgfpathlineto{\pgfqpoint{4.608532in}{0.709885in}}%
\pgfpathlineto{\pgfqpoint{4.609745in}{0.715531in}}%
\pgfpathlineto{\pgfqpoint{4.610554in}{0.712317in}}%
\pgfpathlineto{\pgfqpoint{4.611666in}{0.708091in}}%
\pgfpathlineto{\pgfqpoint{4.612171in}{0.708861in}}%
\pgfpathlineto{\pgfqpoint{4.612778in}{0.709602in}}%
\pgfpathlineto{\pgfqpoint{4.613384in}{0.709165in}}%
\pgfpathlineto{\pgfqpoint{4.616619in}{0.706567in}}%
\pgfpathlineto{\pgfqpoint{4.620562in}{0.706511in}}%
\pgfpathlineto{\pgfqpoint{4.621977in}{0.706689in}}%
\pgfpathlineto{\pgfqpoint{4.628751in}{0.708789in}}%
\pgfpathlineto{\pgfqpoint{4.630166in}{0.709118in}}%
\pgfpathlineto{\pgfqpoint{4.632087in}{0.708640in}}%
\pgfpathlineto{\pgfqpoint{4.632693in}{0.710720in}}%
\pgfpathlineto{\pgfqpoint{4.634715in}{0.720587in}}%
\pgfpathlineto{\pgfqpoint{4.634917in}{0.718978in}}%
\pgfpathlineto{\pgfqpoint{4.636535in}{0.710768in}}%
\pgfpathlineto{\pgfqpoint{4.638051in}{0.709448in}}%
\pgfpathlineto{\pgfqpoint{4.638354in}{0.711421in}}%
\pgfpathlineto{\pgfqpoint{4.638759in}{0.720323in}}%
\pgfpathlineto{\pgfqpoint{4.639365in}{0.709878in}}%
\pgfpathlineto{\pgfqpoint{4.640578in}{0.707476in}}%
\pgfpathlineto{\pgfqpoint{4.640781in}{0.707689in}}%
\pgfpathlineto{\pgfqpoint{4.642095in}{0.712383in}}%
\pgfpathlineto{\pgfqpoint{4.642701in}{0.709382in}}%
\pgfpathlineto{\pgfqpoint{4.643106in}{0.709159in}}%
\pgfpathlineto{\pgfqpoint{4.643712in}{0.709614in}}%
\pgfpathlineto{\pgfqpoint{4.644925in}{0.709829in}}%
\pgfpathlineto{\pgfqpoint{4.645128in}{0.709569in}}%
\pgfpathlineto{\pgfqpoint{4.645835in}{0.708502in}}%
\pgfpathlineto{\pgfqpoint{4.646139in}{0.709987in}}%
\pgfpathlineto{\pgfqpoint{4.646543in}{0.734116in}}%
\pgfpathlineto{\pgfqpoint{4.647048in}{0.784841in}}%
\pgfpathlineto{\pgfqpoint{4.647756in}{0.750641in}}%
\pgfpathlineto{\pgfqpoint{4.649778in}{0.716392in}}%
\pgfpathlineto{\pgfqpoint{4.651395in}{0.709604in}}%
\pgfpathlineto{\pgfqpoint{4.653215in}{0.708244in}}%
\pgfpathlineto{\pgfqpoint{4.655136in}{0.709465in}}%
\pgfpathlineto{\pgfqpoint{4.655439in}{0.709877in}}%
\pgfpathlineto{\pgfqpoint{4.655945in}{0.708547in}}%
\pgfpathlineto{\pgfqpoint{4.657360in}{0.708897in}}%
\pgfpathlineto{\pgfqpoint{4.658573in}{0.709057in}}%
\pgfpathlineto{\pgfqpoint{4.658674in}{0.708961in}}%
\pgfpathlineto{\pgfqpoint{4.660089in}{0.708823in}}%
\pgfpathlineto{\pgfqpoint{4.661707in}{0.710020in}}%
\pgfpathlineto{\pgfqpoint{4.661909in}{0.709730in}}%
\pgfpathlineto{\pgfqpoint{4.663426in}{0.708606in}}%
\pgfpathlineto{\pgfqpoint{4.673535in}{0.710074in}}%
\pgfpathlineto{\pgfqpoint{4.674040in}{0.730760in}}%
\pgfpathlineto{\pgfqpoint{4.674344in}{0.741484in}}%
\pgfpathlineto{\pgfqpoint{4.674950in}{0.723978in}}%
\pgfpathlineto{\pgfqpoint{4.675051in}{0.723706in}}%
\pgfpathlineto{\pgfqpoint{4.675152in}{0.724750in}}%
\pgfpathlineto{\pgfqpoint{4.675658in}{0.734471in}}%
\pgfpathlineto{\pgfqpoint{4.676062in}{0.722962in}}%
\pgfpathlineto{\pgfqpoint{4.677478in}{0.711196in}}%
\pgfpathlineto{\pgfqpoint{4.679499in}{0.708888in}}%
\pgfpathlineto{\pgfqpoint{4.682128in}{0.709559in}}%
\pgfpathlineto{\pgfqpoint{4.682633in}{0.710500in}}%
\pgfpathlineto{\pgfqpoint{4.683139in}{0.709274in}}%
\pgfpathlineto{\pgfqpoint{4.684049in}{0.709180in}}%
\pgfpathlineto{\pgfqpoint{4.684251in}{0.709510in}}%
\pgfpathlineto{\pgfqpoint{4.685767in}{0.713090in}}%
\pgfpathlineto{\pgfqpoint{4.686071in}{0.711597in}}%
\pgfpathlineto{\pgfqpoint{4.687587in}{0.709051in}}%
\pgfpathlineto{\pgfqpoint{4.689103in}{0.708826in}}%
\pgfpathlineto{\pgfqpoint{4.689204in}{0.708989in}}%
\pgfpathlineto{\pgfqpoint{4.690519in}{0.710103in}}%
\pgfpathlineto{\pgfqpoint{4.690620in}{0.709857in}}%
\pgfpathlineto{\pgfqpoint{4.691125in}{0.708461in}}%
\pgfpathlineto{\pgfqpoint{4.691530in}{0.710266in}}%
\pgfpathlineto{\pgfqpoint{4.691833in}{0.711967in}}%
\pgfpathlineto{\pgfqpoint{4.692540in}{0.709654in}}%
\pgfpathlineto{\pgfqpoint{4.693653in}{0.708789in}}%
\pgfpathlineto{\pgfqpoint{4.694057in}{0.708620in}}%
\pgfpathlineto{\pgfqpoint{4.694360in}{0.709256in}}%
\pgfpathlineto{\pgfqpoint{4.696079in}{0.722941in}}%
\pgfpathlineto{\pgfqpoint{4.696281in}{0.717640in}}%
\pgfpathlineto{\pgfqpoint{4.696989in}{0.707201in}}%
\pgfpathlineto{\pgfqpoint{4.697595in}{0.707883in}}%
\pgfpathlineto{\pgfqpoint{4.698606in}{0.708566in}}%
\pgfpathlineto{\pgfqpoint{4.698808in}{0.708245in}}%
\pgfpathlineto{\pgfqpoint{4.700122in}{0.707715in}}%
\pgfpathlineto{\pgfqpoint{4.702043in}{0.708732in}}%
\pgfpathlineto{\pgfqpoint{4.703964in}{0.717440in}}%
\pgfpathlineto{\pgfqpoint{4.704368in}{0.713626in}}%
\pgfpathlineto{\pgfqpoint{4.705885in}{0.707250in}}%
\pgfpathlineto{\pgfqpoint{4.707603in}{0.708467in}}%
\pgfpathlineto{\pgfqpoint{4.708109in}{0.712512in}}%
\pgfpathlineto{\pgfqpoint{4.708715in}{0.708649in}}%
\pgfpathlineto{\pgfqpoint{4.708918in}{0.708374in}}%
\pgfpathlineto{\pgfqpoint{4.709120in}{0.710733in}}%
\pgfpathlineto{\pgfqpoint{4.709524in}{0.762980in}}%
\pgfpathlineto{\pgfqpoint{4.709827in}{0.802863in}}%
\pgfpathlineto{\pgfqpoint{4.710434in}{0.735896in}}%
\pgfpathlineto{\pgfqpoint{4.712052in}{0.711878in}}%
\pgfpathlineto{\pgfqpoint{4.713062in}{0.710566in}}%
\pgfpathlineto{\pgfqpoint{4.713366in}{0.710793in}}%
\pgfpathlineto{\pgfqpoint{4.713568in}{0.710832in}}%
\pgfpathlineto{\pgfqpoint{4.713871in}{0.709847in}}%
\pgfpathlineto{\pgfqpoint{4.714478in}{0.708829in}}%
\pgfpathlineto{\pgfqpoint{4.714983in}{0.709449in}}%
\pgfpathlineto{\pgfqpoint{4.716297in}{0.709219in}}%
\pgfpathlineto{\pgfqpoint{4.721352in}{0.709296in}}%
\pgfpathlineto{\pgfqpoint{4.723071in}{0.709421in}}%
\pgfpathlineto{\pgfqpoint{4.726002in}{0.709153in}}%
\pgfpathlineto{\pgfqpoint{4.726508in}{0.711362in}}%
\pgfpathlineto{\pgfqpoint{4.727013in}{0.713896in}}%
\pgfpathlineto{\pgfqpoint{4.727721in}{0.712879in}}%
\pgfpathlineto{\pgfqpoint{4.728530in}{0.710097in}}%
\pgfpathlineto{\pgfqpoint{4.729237in}{0.709288in}}%
\pgfpathlineto{\pgfqpoint{4.729743in}{0.709653in}}%
\pgfpathlineto{\pgfqpoint{4.732169in}{0.713094in}}%
\pgfpathlineto{\pgfqpoint{4.732472in}{0.711180in}}%
\pgfpathlineto{\pgfqpoint{4.733888in}{0.708013in}}%
\pgfpathlineto{\pgfqpoint{4.735808in}{0.708621in}}%
\pgfpathlineto{\pgfqpoint{4.736112in}{0.710011in}}%
\pgfpathlineto{\pgfqpoint{4.736819in}{0.708518in}}%
\pgfpathlineto{\pgfqpoint{4.738033in}{0.709309in}}%
\pgfpathlineto{\pgfqpoint{4.738538in}{0.717216in}}%
\pgfpathlineto{\pgfqpoint{4.739145in}{0.710079in}}%
\pgfpathlineto{\pgfqpoint{4.739246in}{0.709847in}}%
\pgfpathlineto{\pgfqpoint{4.739549in}{0.711516in}}%
\pgfpathlineto{\pgfqpoint{4.739751in}{0.712289in}}%
\pgfpathlineto{\pgfqpoint{4.740054in}{0.709264in}}%
\pgfpathlineto{\pgfqpoint{4.740459in}{0.707109in}}%
\pgfpathlineto{\pgfqpoint{4.741166in}{0.709277in}}%
\pgfpathlineto{\pgfqpoint{4.741369in}{0.709642in}}%
\pgfpathlineto{\pgfqpoint{4.741874in}{0.708051in}}%
\pgfpathlineto{\pgfqpoint{4.741975in}{0.708125in}}%
\pgfpathlineto{\pgfqpoint{4.742683in}{0.712864in}}%
\pgfpathlineto{\pgfqpoint{4.743390in}{0.709063in}}%
\pgfpathlineto{\pgfqpoint{4.745311in}{0.707061in}}%
\pgfpathlineto{\pgfqpoint{4.746524in}{0.707840in}}%
\pgfpathlineto{\pgfqpoint{4.748647in}{0.714005in}}%
\pgfpathlineto{\pgfqpoint{4.748850in}{0.713000in}}%
\pgfpathlineto{\pgfqpoint{4.750972in}{0.706180in}}%
\pgfpathlineto{\pgfqpoint{4.751478in}{0.705616in}}%
\pgfpathlineto{\pgfqpoint{4.751882in}{0.706544in}}%
\pgfpathlineto{\pgfqpoint{4.753500in}{0.712089in}}%
\pgfpathlineto{\pgfqpoint{4.753601in}{0.711145in}}%
\pgfpathlineto{\pgfqpoint{4.754106in}{0.705265in}}%
\pgfpathlineto{\pgfqpoint{4.754814in}{0.708025in}}%
\pgfpathlineto{\pgfqpoint{4.756937in}{0.743457in}}%
\pgfpathlineto{\pgfqpoint{4.757442in}{0.731235in}}%
\pgfpathlineto{\pgfqpoint{4.758352in}{0.719972in}}%
\pgfpathlineto{\pgfqpoint{4.758959in}{0.720951in}}%
\pgfpathlineto{\pgfqpoint{4.759464in}{0.713842in}}%
\pgfpathlineto{\pgfqpoint{4.760880in}{0.706879in}}%
\pgfpathlineto{\pgfqpoint{4.761183in}{0.706047in}}%
\pgfpathlineto{\pgfqpoint{4.761992in}{0.706923in}}%
\pgfpathlineto{\pgfqpoint{4.762699in}{0.709455in}}%
\pgfpathlineto{\pgfqpoint{4.764014in}{0.726121in}}%
\pgfpathlineto{\pgfqpoint{4.764317in}{0.735886in}}%
\pgfpathlineto{\pgfqpoint{4.765024in}{0.721357in}}%
\pgfpathlineto{\pgfqpoint{4.767552in}{0.706633in}}%
\pgfpathlineto{\pgfqpoint{4.768259in}{0.706988in}}%
\pgfpathlineto{\pgfqpoint{4.768664in}{0.708539in}}%
\pgfpathlineto{\pgfqpoint{4.769473in}{0.721709in}}%
\pgfpathlineto{\pgfqpoint{4.770382in}{0.716998in}}%
\pgfpathlineto{\pgfqpoint{4.771596in}{0.706938in}}%
\pgfpathlineto{\pgfqpoint{4.772505in}{0.707149in}}%
\pgfpathlineto{\pgfqpoint{4.774729in}{0.709113in}}%
\pgfpathlineto{\pgfqpoint{4.774932in}{0.709194in}}%
\pgfpathlineto{\pgfqpoint{4.775336in}{0.708474in}}%
\pgfpathlineto{\pgfqpoint{4.775538in}{0.708372in}}%
\pgfpathlineto{\pgfqpoint{4.778268in}{0.710310in}}%
\pgfpathlineto{\pgfqpoint{4.779784in}{0.713913in}}%
\pgfpathlineto{\pgfqpoint{4.780087in}{0.712197in}}%
\pgfpathlineto{\pgfqpoint{4.780593in}{0.710075in}}%
\pgfpathlineto{\pgfqpoint{4.781098in}{0.712612in}}%
\pgfpathlineto{\pgfqpoint{4.781705in}{0.716374in}}%
\pgfpathlineto{\pgfqpoint{4.782311in}{0.714257in}}%
\pgfpathlineto{\pgfqpoint{4.783525in}{0.707654in}}%
\pgfpathlineto{\pgfqpoint{4.784232in}{0.708150in}}%
\pgfpathlineto{\pgfqpoint{4.787164in}{0.710943in}}%
\pgfpathlineto{\pgfqpoint{4.787366in}{0.710360in}}%
\pgfpathlineto{\pgfqpoint{4.788984in}{0.707669in}}%
\pgfpathlineto{\pgfqpoint{4.789590in}{0.707830in}}%
\pgfpathlineto{\pgfqpoint{4.789691in}{0.708141in}}%
\pgfpathlineto{\pgfqpoint{4.790197in}{0.718704in}}%
\pgfpathlineto{\pgfqpoint{4.790702in}{0.730965in}}%
\pgfpathlineto{\pgfqpoint{4.791511in}{0.727626in}}%
\pgfpathlineto{\pgfqpoint{4.791915in}{0.733429in}}%
\pgfpathlineto{\pgfqpoint{4.792421in}{0.723801in}}%
\pgfpathlineto{\pgfqpoint{4.793937in}{0.717311in}}%
\pgfpathlineto{\pgfqpoint{4.795555in}{0.707281in}}%
\pgfpathlineto{\pgfqpoint{4.795959in}{0.707536in}}%
\pgfpathlineto{\pgfqpoint{4.796768in}{0.709924in}}%
\pgfpathlineto{\pgfqpoint{4.797779in}{0.708858in}}%
\pgfpathlineto{\pgfqpoint{4.798183in}{0.710281in}}%
\pgfpathlineto{\pgfqpoint{4.799700in}{0.719804in}}%
\pgfpathlineto{\pgfqpoint{4.799902in}{0.718749in}}%
\pgfpathlineto{\pgfqpoint{4.800913in}{0.706900in}}%
\pgfpathlineto{\pgfqpoint{4.801519in}{0.708500in}}%
\pgfpathlineto{\pgfqpoint{4.802328in}{0.713592in}}%
\pgfpathlineto{\pgfqpoint{4.802935in}{0.712010in}}%
\pgfpathlineto{\pgfqpoint{4.804047in}{0.706711in}}%
\pgfpathlineto{\pgfqpoint{4.804552in}{0.708290in}}%
\pgfpathlineto{\pgfqpoint{4.805057in}{0.711231in}}%
\pgfpathlineto{\pgfqpoint{4.805765in}{0.709324in}}%
\pgfpathlineto{\pgfqpoint{4.806574in}{0.706838in}}%
\pgfpathlineto{\pgfqpoint{4.808090in}{0.706523in}}%
\pgfpathlineto{\pgfqpoint{4.810112in}{0.706476in}}%
\pgfpathlineto{\pgfqpoint{4.811426in}{0.706863in}}%
\pgfpathlineto{\pgfqpoint{4.812235in}{0.708520in}}%
\pgfpathlineto{\pgfqpoint{4.812842in}{0.711693in}}%
\pgfpathlineto{\pgfqpoint{4.813752in}{0.711537in}}%
\pgfpathlineto{\pgfqpoint{4.814459in}{0.717611in}}%
\pgfpathlineto{\pgfqpoint{4.815167in}{0.735026in}}%
\pgfpathlineto{\pgfqpoint{4.815874in}{0.726872in}}%
\pgfpathlineto{\pgfqpoint{4.818604in}{0.707604in}}%
\pgfpathlineto{\pgfqpoint{4.819312in}{0.707221in}}%
\pgfpathlineto{\pgfqpoint{4.819615in}{0.707663in}}%
\pgfpathlineto{\pgfqpoint{4.821738in}{0.717329in}}%
\pgfpathlineto{\pgfqpoint{4.822041in}{0.720669in}}%
\pgfpathlineto{\pgfqpoint{4.822648in}{0.714906in}}%
\pgfpathlineto{\pgfqpoint{4.824771in}{0.707619in}}%
\pgfpathlineto{\pgfqpoint{4.826489in}{0.708851in}}%
\pgfpathlineto{\pgfqpoint{4.827298in}{0.718549in}}%
\pgfpathlineto{\pgfqpoint{4.828814in}{0.728446in}}%
\pgfpathlineto{\pgfqpoint{4.828006in}{0.718440in}}%
\pgfpathlineto{\pgfqpoint{4.829017in}{0.726040in}}%
\pgfpathlineto{\pgfqpoint{4.830836in}{0.708472in}}%
\pgfpathlineto{\pgfqpoint{4.832252in}{0.706114in}}%
\pgfpathlineto{\pgfqpoint{4.832454in}{0.706186in}}%
\pgfpathlineto{\pgfqpoint{4.842058in}{0.709170in}}%
\pgfpathlineto{\pgfqpoint{4.843170in}{0.710306in}}%
\pgfpathlineto{\pgfqpoint{4.844787in}{0.709813in}}%
\pgfpathlineto{\pgfqpoint{4.847719in}{0.709285in}}%
\pgfpathlineto{\pgfqpoint{4.850550in}{0.710189in}}%
\pgfpathlineto{\pgfqpoint{4.850954in}{0.719871in}}%
\pgfpathlineto{\pgfqpoint{4.851459in}{0.743720in}}%
\pgfpathlineto{\pgfqpoint{4.852167in}{0.729705in}}%
\pgfpathlineto{\pgfqpoint{4.852369in}{0.729751in}}%
\pgfpathlineto{\pgfqpoint{4.852470in}{0.729038in}}%
\pgfpathlineto{\pgfqpoint{4.853683in}{0.709793in}}%
\pgfpathlineto{\pgfqpoint{4.854492in}{0.712579in}}%
\pgfpathlineto{\pgfqpoint{4.854998in}{0.714676in}}%
\pgfpathlineto{\pgfqpoint{4.855604in}{0.713096in}}%
\pgfpathlineto{\pgfqpoint{4.856110in}{0.711175in}}%
\pgfpathlineto{\pgfqpoint{4.856817in}{0.712422in}}%
\pgfpathlineto{\pgfqpoint{4.857929in}{0.709358in}}%
\pgfpathlineto{\pgfqpoint{4.858435in}{0.708562in}}%
\pgfpathlineto{\pgfqpoint{4.858839in}{0.709557in}}%
\pgfpathlineto{\pgfqpoint{4.859446in}{0.718456in}}%
\pgfpathlineto{\pgfqpoint{4.860254in}{0.736079in}}%
\pgfpathlineto{\pgfqpoint{4.860861in}{0.728376in}}%
\pgfpathlineto{\pgfqpoint{4.862984in}{0.708572in}}%
\pgfpathlineto{\pgfqpoint{4.864399in}{0.708391in}}%
\pgfpathlineto{\pgfqpoint{4.864500in}{0.708525in}}%
\pgfpathlineto{\pgfqpoint{4.865006in}{0.712297in}}%
\pgfpathlineto{\pgfqpoint{4.865612in}{0.718518in}}%
\pgfpathlineto{\pgfqpoint{4.866320in}{0.715574in}}%
\pgfpathlineto{\pgfqpoint{4.868039in}{0.708087in}}%
\pgfpathlineto{\pgfqpoint{4.868140in}{0.708157in}}%
\pgfpathlineto{\pgfqpoint{4.868645in}{0.710555in}}%
\pgfpathlineto{\pgfqpoint{4.869757in}{0.715312in}}%
\pgfpathlineto{\pgfqpoint{4.870162in}{0.714450in}}%
\pgfpathlineto{\pgfqpoint{4.871779in}{0.705587in}}%
\pgfpathlineto{\pgfqpoint{4.872588in}{0.707598in}}%
\pgfpathlineto{\pgfqpoint{4.873194in}{0.708421in}}%
\pgfpathlineto{\pgfqpoint{4.873599in}{0.712033in}}%
\pgfpathlineto{\pgfqpoint{4.874003in}{0.707500in}}%
\pgfpathlineto{\pgfqpoint{4.874408in}{0.705576in}}%
\pgfpathlineto{\pgfqpoint{4.875115in}{0.706116in}}%
\pgfpathlineto{\pgfqpoint{4.875621in}{0.713924in}}%
\pgfpathlineto{\pgfqpoint{4.875924in}{0.716541in}}%
\pgfpathlineto{\pgfqpoint{4.876632in}{0.713708in}}%
\pgfpathlineto{\pgfqpoint{4.878654in}{0.706013in}}%
\pgfpathlineto{\pgfqpoint{4.879462in}{0.706434in}}%
\pgfpathlineto{\pgfqpoint{4.879563in}{0.706711in}}%
\pgfpathlineto{\pgfqpoint{4.881383in}{0.716006in}}%
\pgfpathlineto{\pgfqpoint{4.881686in}{0.712209in}}%
\pgfpathlineto{\pgfqpoint{4.882293in}{0.705376in}}%
\pgfpathlineto{\pgfqpoint{4.883001in}{0.706012in}}%
\pgfpathlineto{\pgfqpoint{4.884214in}{0.707712in}}%
\pgfpathlineto{\pgfqpoint{4.885225in}{0.716634in}}%
\pgfpathlineto{\pgfqpoint{4.886134in}{0.713936in}}%
\pgfpathlineto{\pgfqpoint{4.886943in}{0.706217in}}%
\pgfpathlineto{\pgfqpoint{4.887853in}{0.706824in}}%
\pgfpathlineto{\pgfqpoint{4.888156in}{0.707923in}}%
\pgfpathlineto{\pgfqpoint{4.888763in}{0.734731in}}%
\pgfpathlineto{\pgfqpoint{4.888965in}{0.739945in}}%
\pgfpathlineto{\pgfqpoint{4.889673in}{0.730220in}}%
\pgfpathlineto{\pgfqpoint{4.890785in}{0.714558in}}%
\pgfpathlineto{\pgfqpoint{4.891391in}{0.720592in}}%
\pgfpathlineto{\pgfqpoint{4.891492in}{0.721061in}}%
\pgfpathlineto{\pgfqpoint{4.891796in}{0.717668in}}%
\pgfpathlineto{\pgfqpoint{4.893312in}{0.709424in}}%
\pgfpathlineto{\pgfqpoint{4.895132in}{0.707551in}}%
\pgfpathlineto{\pgfqpoint{4.905241in}{0.709585in}}%
\pgfpathlineto{\pgfqpoint{4.907971in}{0.729480in}}%
\pgfpathlineto{\pgfqpoint{4.908476in}{0.723142in}}%
\pgfpathlineto{\pgfqpoint{4.910700in}{0.708649in}}%
\pgfpathlineto{\pgfqpoint{4.910902in}{0.708688in}}%
\pgfpathlineto{\pgfqpoint{4.914643in}{0.710626in}}%
\pgfpathlineto{\pgfqpoint{4.915957in}{0.714063in}}%
\pgfpathlineto{\pgfqpoint{4.916058in}{0.713930in}}%
\pgfpathlineto{\pgfqpoint{4.917777in}{0.708631in}}%
\pgfpathlineto{\pgfqpoint{4.921921in}{0.708839in}}%
\pgfpathlineto{\pgfqpoint{4.923943in}{0.710771in}}%
\pgfpathlineto{\pgfqpoint{4.924146in}{0.711015in}}%
\pgfpathlineto{\pgfqpoint{4.924651in}{0.709956in}}%
\pgfpathlineto{\pgfqpoint{4.926673in}{0.708714in}}%
\pgfpathlineto{\pgfqpoint{4.929504in}{0.710378in}}%
\pgfpathlineto{\pgfqpoint{4.930110in}{0.721684in}}%
\pgfpathlineto{\pgfqpoint{4.930818in}{0.715085in}}%
\pgfpathlineto{\pgfqpoint{4.931323in}{0.712104in}}%
\pgfpathlineto{\pgfqpoint{4.932435in}{0.708488in}}%
\pgfpathlineto{\pgfqpoint{4.932738in}{0.708557in}}%
\pgfpathlineto{\pgfqpoint{4.935569in}{0.709917in}}%
\pgfpathlineto{\pgfqpoint{4.936681in}{0.709991in}}%
\pgfpathlineto{\pgfqpoint{4.936883in}{0.709416in}}%
\pgfpathlineto{\pgfqpoint{4.937288in}{0.708818in}}%
\pgfpathlineto{\pgfqpoint{4.937692in}{0.709854in}}%
\pgfpathlineto{\pgfqpoint{4.939411in}{0.717101in}}%
\pgfpathlineto{\pgfqpoint{4.939512in}{0.716510in}}%
\pgfpathlineto{\pgfqpoint{4.941028in}{0.707844in}}%
\pgfpathlineto{\pgfqpoint{4.941230in}{0.707930in}}%
\pgfpathlineto{\pgfqpoint{4.941635in}{0.710364in}}%
\pgfpathlineto{\pgfqpoint{4.942545in}{0.718248in}}%
\pgfpathlineto{\pgfqpoint{4.943151in}{0.716018in}}%
\pgfpathlineto{\pgfqpoint{4.943353in}{0.715639in}}%
\pgfpathlineto{\pgfqpoint{4.943657in}{0.717429in}}%
\pgfpathlineto{\pgfqpoint{4.944364in}{0.726516in}}%
\pgfpathlineto{\pgfqpoint{4.944971in}{0.721015in}}%
\pgfpathlineto{\pgfqpoint{4.947094in}{0.707888in}}%
\pgfpathlineto{\pgfqpoint{4.947397in}{0.708251in}}%
\pgfpathlineto{\pgfqpoint{4.947903in}{0.710179in}}%
\pgfpathlineto{\pgfqpoint{4.948408in}{0.707853in}}%
\pgfpathlineto{\pgfqpoint{4.949924in}{0.707409in}}%
\pgfpathlineto{\pgfqpoint{4.954069in}{0.708198in}}%
\pgfpathlineto{\pgfqpoint{4.957304in}{0.708848in}}%
\pgfpathlineto{\pgfqpoint{4.959023in}{0.709310in}}%
\pgfpathlineto{\pgfqpoint{4.959124in}{0.709140in}}%
\pgfpathlineto{\pgfqpoint{4.960236in}{0.708792in}}%
\pgfpathlineto{\pgfqpoint{4.960438in}{0.708983in}}%
\pgfpathlineto{\pgfqpoint{4.961954in}{0.709315in}}%
\pgfpathlineto{\pgfqpoint{4.963673in}{0.709249in}}%
\pgfpathlineto{\pgfqpoint{4.965291in}{0.709634in}}%
\pgfpathlineto{\pgfqpoint{4.965392in}{0.709519in}}%
\pgfpathlineto{\pgfqpoint{4.965796in}{0.709221in}}%
\pgfpathlineto{\pgfqpoint{4.965998in}{0.709918in}}%
\pgfpathlineto{\pgfqpoint{4.966605in}{0.728265in}}%
\pgfpathlineto{\pgfqpoint{4.966908in}{0.734253in}}%
\pgfpathlineto{\pgfqpoint{4.967717in}{0.728289in}}%
\pgfpathlineto{\pgfqpoint{4.968829in}{0.723785in}}%
\pgfpathlineto{\pgfqpoint{4.968930in}{0.725076in}}%
\pgfpathlineto{\pgfqpoint{4.969334in}{0.776460in}}%
\pgfpathlineto{\pgfqpoint{4.969537in}{0.807259in}}%
\pgfpathlineto{\pgfqpoint{4.970244in}{0.737223in}}%
\pgfpathlineto{\pgfqpoint{4.971862in}{0.712849in}}%
\pgfpathlineto{\pgfqpoint{4.972873in}{0.708920in}}%
\pgfpathlineto{\pgfqpoint{4.973479in}{0.710174in}}%
\pgfpathlineto{\pgfqpoint{4.975804in}{0.708606in}}%
\pgfpathlineto{\pgfqpoint{4.978736in}{0.709663in}}%
\pgfpathlineto{\pgfqpoint{4.979545in}{0.720061in}}%
\pgfpathlineto{\pgfqpoint{4.980556in}{0.716966in}}%
\pgfpathlineto{\pgfqpoint{4.981668in}{0.708370in}}%
\pgfpathlineto{\pgfqpoint{4.982375in}{0.708472in}}%
\pgfpathlineto{\pgfqpoint{4.983690in}{0.709623in}}%
\pgfpathlineto{\pgfqpoint{4.984498in}{0.721955in}}%
\pgfpathlineto{\pgfqpoint{4.985610in}{0.717525in}}%
\pgfpathlineto{\pgfqpoint{4.986520in}{0.708193in}}%
\pgfpathlineto{\pgfqpoint{4.987026in}{0.708393in}}%
\pgfpathlineto{\pgfqpoint{4.988845in}{0.708988in}}%
\pgfpathlineto{\pgfqpoint{4.988946in}{0.708872in}}%
\pgfpathlineto{\pgfqpoint{4.990564in}{0.708522in}}%
\pgfpathlineto{\pgfqpoint{4.991069in}{0.710107in}}%
\pgfpathlineto{\pgfqpoint{4.991676in}{0.714706in}}%
\pgfpathlineto{\pgfqpoint{4.992485in}{0.712906in}}%
\pgfpathlineto{\pgfqpoint{4.992990in}{0.712605in}}%
\pgfpathlineto{\pgfqpoint{4.993192in}{0.713192in}}%
\pgfpathlineto{\pgfqpoint{4.993698in}{0.715172in}}%
\pgfpathlineto{\pgfqpoint{4.994203in}{0.713294in}}%
\pgfpathlineto{\pgfqpoint{4.996630in}{0.708716in}}%
\pgfpathlineto{\pgfqpoint{4.996731in}{0.708770in}}%
\pgfpathlineto{\pgfqpoint{4.997236in}{0.711650in}}%
\pgfpathlineto{\pgfqpoint{4.997640in}{0.715038in}}%
\pgfpathlineto{\pgfqpoint{4.998247in}{0.711960in}}%
\pgfpathlineto{\pgfqpoint{4.999966in}{0.707069in}}%
\pgfpathlineto{\pgfqpoint{5.000471in}{0.707313in}}%
\pgfpathlineto{\pgfqpoint{5.002594in}{0.709563in}}%
\pgfpathlineto{\pgfqpoint{5.002897in}{0.710357in}}%
\pgfpathlineto{\pgfqpoint{5.003504in}{0.708677in}}%
\pgfpathlineto{\pgfqpoint{5.003908in}{0.709869in}}%
\pgfpathlineto{\pgfqpoint{5.004414in}{0.712194in}}%
\pgfpathlineto{\pgfqpoint{5.004919in}{0.709130in}}%
\pgfpathlineto{\pgfqpoint{5.005222in}{0.708034in}}%
\pgfpathlineto{\pgfqpoint{5.005627in}{0.709960in}}%
\pgfpathlineto{\pgfqpoint{5.005930in}{0.712036in}}%
\pgfpathlineto{\pgfqpoint{5.006436in}{0.707341in}}%
\pgfpathlineto{\pgfqpoint{5.007750in}{0.706532in}}%
\pgfpathlineto{\pgfqpoint{5.018162in}{0.708987in}}%
\pgfpathlineto{\pgfqpoint{5.020285in}{0.709894in}}%
\pgfpathlineto{\pgfqpoint{5.022712in}{0.732848in}}%
\pgfpathlineto{\pgfqpoint{5.023015in}{0.741438in}}%
\pgfpathlineto{\pgfqpoint{5.023621in}{0.729348in}}%
\pgfpathlineto{\pgfqpoint{5.025846in}{0.709836in}}%
\pgfpathlineto{\pgfqpoint{5.027463in}{0.710484in}}%
\pgfpathlineto{\pgfqpoint{5.027969in}{0.711795in}}%
\pgfpathlineto{\pgfqpoint{5.030091in}{0.729911in}}%
\pgfpathlineto{\pgfqpoint{5.030496in}{0.725812in}}%
\pgfpathlineto{\pgfqpoint{5.032518in}{0.709586in}}%
\pgfpathlineto{\pgfqpoint{5.033933in}{0.708639in}}%
\pgfpathlineto{\pgfqpoint{5.040908in}{0.708606in}}%
\pgfpathlineto{\pgfqpoint{5.041313in}{0.709865in}}%
\pgfpathlineto{\pgfqpoint{5.041818in}{0.708158in}}%
\pgfpathlineto{\pgfqpoint{5.041919in}{0.708194in}}%
\pgfpathlineto{\pgfqpoint{5.042425in}{0.710856in}}%
\pgfpathlineto{\pgfqpoint{5.043031in}{0.708131in}}%
\pgfpathlineto{\pgfqpoint{5.046671in}{0.709492in}}%
\pgfpathlineto{\pgfqpoint{5.047277in}{0.712190in}}%
\pgfpathlineto{\pgfqpoint{5.048086in}{0.710889in}}%
\pgfpathlineto{\pgfqpoint{5.049805in}{0.708590in}}%
\pgfpathlineto{\pgfqpoint{5.050209in}{0.709635in}}%
\pgfpathlineto{\pgfqpoint{5.051827in}{0.716262in}}%
\pgfpathlineto{\pgfqpoint{5.052029in}{0.715289in}}%
\pgfpathlineto{\pgfqpoint{5.053444in}{0.706903in}}%
\pgfpathlineto{\pgfqpoint{5.053747in}{0.707684in}}%
\pgfpathlineto{\pgfqpoint{5.054253in}{0.714141in}}%
\pgfpathlineto{\pgfqpoint{5.054859in}{0.707902in}}%
\pgfpathlineto{\pgfqpoint{5.055062in}{0.709024in}}%
\pgfpathlineto{\pgfqpoint{5.055567in}{0.725202in}}%
\pgfpathlineto{\pgfqpoint{5.056174in}{0.711403in}}%
\pgfpathlineto{\pgfqpoint{5.057690in}{0.707264in}}%
\pgfpathlineto{\pgfqpoint{5.057993in}{0.708086in}}%
\pgfpathlineto{\pgfqpoint{5.058701in}{0.716414in}}%
\pgfpathlineto{\pgfqpoint{5.059307in}{0.711068in}}%
\pgfpathlineto{\pgfqpoint{5.060521in}{0.708058in}}%
\pgfpathlineto{\pgfqpoint{5.060824in}{0.708668in}}%
\pgfpathlineto{\pgfqpoint{5.061430in}{0.710770in}}%
\pgfpathlineto{\pgfqpoint{5.062037in}{0.709499in}}%
\pgfpathlineto{\pgfqpoint{5.065373in}{0.704236in}}%
\pgfpathlineto{\pgfqpoint{5.066788in}{0.705239in}}%
\pgfpathlineto{\pgfqpoint{5.068103in}{0.707970in}}%
\pgfpathlineto{\pgfqpoint{5.068810in}{0.709430in}}%
\pgfpathlineto{\pgfqpoint{5.069518in}{0.709145in}}%
\pgfpathlineto{\pgfqpoint{5.069922in}{0.708501in}}%
\pgfpathlineto{\pgfqpoint{5.070428in}{0.706774in}}%
\pgfpathlineto{\pgfqpoint{5.070832in}{0.708970in}}%
\pgfpathlineto{\pgfqpoint{5.071338in}{0.742686in}}%
\pgfpathlineto{\pgfqpoint{5.071641in}{0.768249in}}%
\pgfpathlineto{\pgfqpoint{5.072349in}{0.739827in}}%
\pgfpathlineto{\pgfqpoint{5.073865in}{0.708705in}}%
\pgfpathlineto{\pgfqpoint{5.075179in}{0.715346in}}%
\pgfpathlineto{\pgfqpoint{5.076291in}{0.708857in}}%
\pgfpathlineto{\pgfqpoint{5.076696in}{0.707614in}}%
\pgfpathlineto{\pgfqpoint{5.077504in}{0.707860in}}%
\pgfpathlineto{\pgfqpoint{5.079728in}{0.709016in}}%
\pgfpathlineto{\pgfqpoint{5.080234in}{0.712191in}}%
\pgfpathlineto{\pgfqpoint{5.081144in}{0.711213in}}%
\pgfpathlineto{\pgfqpoint{5.084581in}{0.708582in}}%
\pgfpathlineto{\pgfqpoint{5.085187in}{0.709507in}}%
\pgfpathlineto{\pgfqpoint{5.085390in}{0.710006in}}%
\pgfpathlineto{\pgfqpoint{5.086097in}{0.708951in}}%
\pgfpathlineto{\pgfqpoint{5.090748in}{0.708495in}}%
\pgfpathlineto{\pgfqpoint{5.092769in}{0.708782in}}%
\pgfpathlineto{\pgfqpoint{5.093376in}{0.713833in}}%
\pgfpathlineto{\pgfqpoint{5.093578in}{0.714516in}}%
\pgfpathlineto{\pgfqpoint{5.094185in}{0.712707in}}%
\pgfpathlineto{\pgfqpoint{5.094791in}{0.710930in}}%
\pgfpathlineto{\pgfqpoint{5.095095in}{0.712319in}}%
\pgfpathlineto{\pgfqpoint{5.095802in}{0.716115in}}%
\pgfpathlineto{\pgfqpoint{5.096308in}{0.714580in}}%
\pgfpathlineto{\pgfqpoint{5.097015in}{0.710681in}}%
\pgfpathlineto{\pgfqpoint{5.097319in}{0.713067in}}%
\pgfpathlineto{\pgfqpoint{5.099745in}{0.749502in}}%
\pgfpathlineto{\pgfqpoint{5.099846in}{0.748601in}}%
\pgfpathlineto{\pgfqpoint{5.101463in}{0.713541in}}%
\pgfpathlineto{\pgfqpoint{5.102171in}{0.723509in}}%
\pgfpathlineto{\pgfqpoint{5.102373in}{0.725168in}}%
\pgfpathlineto{\pgfqpoint{5.102879in}{0.720246in}}%
\pgfpathlineto{\pgfqpoint{5.103890in}{0.715050in}}%
\pgfpathlineto{\pgfqpoint{5.104294in}{0.717888in}}%
\pgfpathlineto{\pgfqpoint{5.104698in}{0.720516in}}%
\pgfpathlineto{\pgfqpoint{5.105406in}{0.718112in}}%
\pgfpathlineto{\pgfqpoint{5.105810in}{0.716044in}}%
\pgfpathlineto{\pgfqpoint{5.106821in}{0.707431in}}%
\pgfpathlineto{\pgfqpoint{5.107529in}{0.707561in}}%
\pgfpathlineto{\pgfqpoint{5.109147in}{0.708184in}}%
\pgfpathlineto{\pgfqpoint{5.109652in}{0.711986in}}%
\pgfpathlineto{\pgfqpoint{5.110461in}{0.723016in}}%
\pgfpathlineto{\pgfqpoint{5.111067in}{0.719534in}}%
\pgfpathlineto{\pgfqpoint{5.113696in}{0.706990in}}%
\pgfpathlineto{\pgfqpoint{5.113898in}{0.707070in}}%
\pgfpathlineto{\pgfqpoint{5.115313in}{0.706737in}}%
\pgfpathlineto{\pgfqpoint{5.116627in}{0.706512in}}%
\pgfpathlineto{\pgfqpoint{5.125018in}{0.708992in}}%
\pgfpathlineto{\pgfqpoint{5.126231in}{0.709435in}}%
\pgfpathlineto{\pgfqpoint{5.127546in}{0.709155in}}%
\pgfpathlineto{\pgfqpoint{5.128354in}{0.708416in}}%
\pgfpathlineto{\pgfqpoint{5.128658in}{0.709169in}}%
\pgfpathlineto{\pgfqpoint{5.129871in}{0.712152in}}%
\pgfpathlineto{\pgfqpoint{5.129163in}{0.708860in}}%
\pgfpathlineto{\pgfqpoint{5.130073in}{0.710862in}}%
\pgfpathlineto{\pgfqpoint{5.131488in}{0.708017in}}%
\pgfpathlineto{\pgfqpoint{5.132701in}{0.706676in}}%
\pgfpathlineto{\pgfqpoint{5.132802in}{0.707144in}}%
\pgfpathlineto{\pgfqpoint{5.133106in}{0.719887in}}%
\pgfpathlineto{\pgfqpoint{5.133611in}{0.786255in}}%
\pgfpathlineto{\pgfqpoint{5.134319in}{0.736605in}}%
\pgfpathlineto{\pgfqpoint{5.136341in}{0.708207in}}%
\pgfpathlineto{\pgfqpoint{5.137453in}{0.709261in}}%
\pgfpathlineto{\pgfqpoint{5.139576in}{0.711670in}}%
\pgfpathlineto{\pgfqpoint{5.139778in}{0.711824in}}%
\pgfpathlineto{\pgfqpoint{5.139980in}{0.711013in}}%
\pgfpathlineto{\pgfqpoint{5.140789in}{0.707628in}}%
\pgfpathlineto{\pgfqpoint{5.141496in}{0.707841in}}%
\pgfpathlineto{\pgfqpoint{5.143720in}{0.708860in}}%
\pgfpathlineto{\pgfqpoint{5.144630in}{0.712424in}}%
\pgfpathlineto{\pgfqpoint{5.145439in}{0.710429in}}%
\pgfpathlineto{\pgfqpoint{5.145945in}{0.711446in}}%
\pgfpathlineto{\pgfqpoint{5.146248in}{0.710132in}}%
\pgfpathlineto{\pgfqpoint{5.146854in}{0.707975in}}%
\pgfpathlineto{\pgfqpoint{5.147461in}{0.709103in}}%
\pgfpathlineto{\pgfqpoint{5.149685in}{0.715610in}}%
\pgfpathlineto{\pgfqpoint{5.153122in}{0.707235in}}%
\pgfpathlineto{\pgfqpoint{5.154537in}{0.708081in}}%
\pgfpathlineto{\pgfqpoint{5.155043in}{0.717899in}}%
\pgfpathlineto{\pgfqpoint{5.155852in}{0.748332in}}%
\pgfpathlineto{\pgfqpoint{5.156559in}{0.739060in}}%
\pgfpathlineto{\pgfqpoint{5.157975in}{0.718410in}}%
\pgfpathlineto{\pgfqpoint{5.158379in}{0.720792in}}%
\pgfpathlineto{\pgfqpoint{5.159390in}{0.728670in}}%
\pgfpathlineto{\pgfqpoint{5.159895in}{0.726437in}}%
\pgfpathlineto{\pgfqpoint{5.162524in}{0.707447in}}%
\pgfpathlineto{\pgfqpoint{5.162928in}{0.707500in}}%
\pgfpathlineto{\pgfqpoint{5.163535in}{0.708458in}}%
\pgfpathlineto{\pgfqpoint{5.164748in}{0.722578in}}%
\pgfpathlineto{\pgfqpoint{5.165253in}{0.731292in}}%
\pgfpathlineto{\pgfqpoint{5.165860in}{0.725460in}}%
\pgfpathlineto{\pgfqpoint{5.167376in}{0.708144in}}%
\pgfpathlineto{\pgfqpoint{5.168387in}{0.711360in}}%
\pgfpathlineto{\pgfqpoint{5.169196in}{0.709807in}}%
\pgfpathlineto{\pgfqpoint{5.170207in}{0.707032in}}%
\pgfpathlineto{\pgfqpoint{5.170814in}{0.707234in}}%
\pgfpathlineto{\pgfqpoint{5.175767in}{0.707995in}}%
\pgfpathlineto{\pgfqpoint{5.177688in}{0.709042in}}%
\pgfpathlineto{\pgfqpoint{5.179609in}{0.709062in}}%
\pgfpathlineto{\pgfqpoint{5.181934in}{0.708813in}}%
\pgfpathlineto{\pgfqpoint{5.185472in}{0.710057in}}%
\pgfpathlineto{\pgfqpoint{5.185674in}{0.710312in}}%
\pgfpathlineto{\pgfqpoint{5.186281in}{0.709223in}}%
\pgfpathlineto{\pgfqpoint{5.190325in}{0.707079in}}%
\pgfpathlineto{\pgfqpoint{5.190527in}{0.708412in}}%
\pgfpathlineto{\pgfqpoint{5.191133in}{0.721862in}}%
\pgfpathlineto{\pgfqpoint{5.192043in}{0.716105in}}%
\pgfpathlineto{\pgfqpoint{5.192346in}{0.713993in}}%
\pgfpathlineto{\pgfqpoint{5.192650in}{0.717450in}}%
\pgfpathlineto{\pgfqpoint{5.193155in}{0.791123in}}%
\pgfpathlineto{\pgfqpoint{5.193357in}{0.815368in}}%
\pgfpathlineto{\pgfqpoint{5.193964in}{0.746218in}}%
\pgfpathlineto{\pgfqpoint{5.195581in}{0.709038in}}%
\pgfpathlineto{\pgfqpoint{5.197199in}{0.709211in}}%
\pgfpathlineto{\pgfqpoint{5.208825in}{0.709364in}}%
\pgfpathlineto{\pgfqpoint{5.215497in}{0.707426in}}%
\pgfpathlineto{\pgfqpoint{5.215699in}{0.707031in}}%
\pgfpathlineto{\pgfqpoint{5.216508in}{0.707772in}}%
\pgfpathlineto{\pgfqpoint{5.216912in}{0.720592in}}%
\pgfpathlineto{\pgfqpoint{5.217519in}{0.770653in}}%
\pgfpathlineto{\pgfqpoint{5.218226in}{0.742600in}}%
\pgfpathlineto{\pgfqpoint{5.220248in}{0.709116in}}%
\pgfpathlineto{\pgfqpoint{5.220855in}{0.710035in}}%
\pgfpathlineto{\pgfqpoint{5.222472in}{0.715132in}}%
\pgfpathlineto{\pgfqpoint{5.222674in}{0.713854in}}%
\pgfpathlineto{\pgfqpoint{5.223382in}{0.709062in}}%
\pgfpathlineto{\pgfqpoint{5.224191in}{0.709416in}}%
\pgfpathlineto{\pgfqpoint{5.227224in}{0.708604in}}%
\pgfpathlineto{\pgfqpoint{5.233694in}{0.710234in}}%
\pgfpathlineto{\pgfqpoint{5.236322in}{0.722300in}}%
\pgfpathlineto{\pgfqpoint{5.236423in}{0.721803in}}%
\pgfpathlineto{\pgfqpoint{5.237333in}{0.710113in}}%
\pgfpathlineto{\pgfqpoint{5.238041in}{0.713274in}}%
\pgfpathlineto{\pgfqpoint{5.238849in}{0.722336in}}%
\pgfpathlineto{\pgfqpoint{5.239355in}{0.767800in}}%
\pgfpathlineto{\pgfqpoint{5.239860in}{0.727499in}}%
\pgfpathlineto{\pgfqpoint{5.241276in}{0.710003in}}%
\pgfpathlineto{\pgfqpoint{5.241680in}{0.709126in}}%
\pgfpathlineto{\pgfqpoint{5.242287in}{0.710204in}}%
\pgfpathlineto{\pgfqpoint{5.243500in}{0.710997in}}%
\pgfpathlineto{\pgfqpoint{5.243601in}{0.710820in}}%
\pgfpathlineto{\pgfqpoint{5.244511in}{0.708587in}}%
\pgfpathlineto{\pgfqpoint{5.245218in}{0.708861in}}%
\pgfpathlineto{\pgfqpoint{5.246836in}{0.710795in}}%
\pgfpathlineto{\pgfqpoint{5.248554in}{0.721453in}}%
\pgfpathlineto{\pgfqpoint{5.249060in}{0.719003in}}%
\pgfpathlineto{\pgfqpoint{5.251992in}{0.707322in}}%
\pgfpathlineto{\pgfqpoint{5.254013in}{0.707587in}}%
\pgfpathlineto{\pgfqpoint{5.255934in}{0.709462in}}%
\pgfpathlineto{\pgfqpoint{5.256440in}{0.711105in}}%
\pgfpathlineto{\pgfqpoint{5.257147in}{0.716363in}}%
\pgfpathlineto{\pgfqpoint{5.257855in}{0.714394in}}%
\pgfpathlineto{\pgfqpoint{5.259169in}{0.707717in}}%
\pgfpathlineto{\pgfqpoint{5.259877in}{0.710118in}}%
\pgfpathlineto{\pgfqpoint{5.260483in}{0.715153in}}%
\pgfpathlineto{\pgfqpoint{5.261393in}{0.714453in}}%
\pgfpathlineto{\pgfqpoint{5.262809in}{0.707012in}}%
\pgfpathlineto{\pgfqpoint{5.263921in}{0.707348in}}%
\pgfpathlineto{\pgfqpoint{5.264932in}{0.708322in}}%
\pgfpathlineto{\pgfqpoint{5.265942in}{0.710860in}}%
\pgfpathlineto{\pgfqpoint{5.266448in}{0.710359in}}%
\pgfpathlineto{\pgfqpoint{5.268065in}{0.707904in}}%
\pgfpathlineto{\pgfqpoint{5.268571in}{0.709380in}}%
\pgfpathlineto{\pgfqpoint{5.269481in}{0.712894in}}%
\pgfpathlineto{\pgfqpoint{5.269986in}{0.711635in}}%
\pgfpathlineto{\pgfqpoint{5.270896in}{0.708855in}}%
\pgfpathlineto{\pgfqpoint{5.271199in}{0.710459in}}%
\pgfpathlineto{\pgfqpoint{5.271907in}{0.735173in}}%
\pgfpathlineto{\pgfqpoint{5.272716in}{0.718226in}}%
\pgfpathlineto{\pgfqpoint{5.273019in}{0.717383in}}%
\pgfpathlineto{\pgfqpoint{5.274232in}{0.705446in}}%
\pgfpathlineto{\pgfqpoint{5.275041in}{0.706040in}}%
\pgfpathlineto{\pgfqpoint{5.277063in}{0.709070in}}%
\pgfpathlineto{\pgfqpoint{5.277366in}{0.710070in}}%
\pgfpathlineto{\pgfqpoint{5.278074in}{0.709019in}}%
\pgfpathlineto{\pgfqpoint{5.279590in}{0.707934in}}%
\pgfpathlineto{\pgfqpoint{5.279893in}{0.709070in}}%
\pgfpathlineto{\pgfqpoint{5.280298in}{0.710321in}}%
\pgfpathlineto{\pgfqpoint{5.281005in}{0.709185in}}%
\pgfpathlineto{\pgfqpoint{5.281612in}{0.708200in}}%
\pgfpathlineto{\pgfqpoint{5.282926in}{0.707918in}}%
\pgfpathlineto{\pgfqpoint{5.284645in}{0.709112in}}%
\pgfpathlineto{\pgfqpoint{5.285352in}{0.712779in}}%
\pgfpathlineto{\pgfqpoint{5.286161in}{0.711242in}}%
\pgfpathlineto{\pgfqpoint{5.286970in}{0.708561in}}%
\pgfpathlineto{\pgfqpoint{5.287374in}{0.710027in}}%
\pgfpathlineto{\pgfqpoint{5.288486in}{0.721697in}}%
\pgfpathlineto{\pgfqpoint{5.289295in}{0.717669in}}%
\pgfpathlineto{\pgfqpoint{5.292025in}{0.708168in}}%
\pgfpathlineto{\pgfqpoint{5.292530in}{0.708673in}}%
\pgfpathlineto{\pgfqpoint{5.292631in}{0.709014in}}%
\pgfpathlineto{\pgfqpoint{5.295462in}{0.723184in}}%
\pgfpathlineto{\pgfqpoint{5.295664in}{0.724516in}}%
\pgfpathlineto{\pgfqpoint{5.296270in}{0.721493in}}%
\pgfpathlineto{\pgfqpoint{5.297585in}{0.708814in}}%
\pgfpathlineto{\pgfqpoint{5.298191in}{0.713807in}}%
\pgfpathlineto{\pgfqpoint{5.298697in}{0.719857in}}%
\pgfpathlineto{\pgfqpoint{5.299404in}{0.716782in}}%
\pgfpathlineto{\pgfqpoint{5.300921in}{0.705845in}}%
\pgfpathlineto{\pgfqpoint{5.301730in}{0.708564in}}%
\pgfpathlineto{\pgfqpoint{5.304055in}{0.731462in}}%
\pgfpathlineto{\pgfqpoint{5.304156in}{0.731268in}}%
\pgfpathlineto{\pgfqpoint{5.307593in}{0.705245in}}%
\pgfpathlineto{\pgfqpoint{5.308098in}{0.705674in}}%
\pgfpathlineto{\pgfqpoint{5.309716in}{0.707142in}}%
\pgfpathlineto{\pgfqpoint{5.310019in}{0.706734in}}%
\pgfpathlineto{\pgfqpoint{5.310525in}{0.706189in}}%
\pgfpathlineto{\pgfqpoint{5.311232in}{0.706732in}}%
\pgfpathlineto{\pgfqpoint{5.315984in}{0.708145in}}%
\pgfpathlineto{\pgfqpoint{5.318612in}{0.708406in}}%
\pgfpathlineto{\pgfqpoint{5.321847in}{0.709893in}}%
\pgfpathlineto{\pgfqpoint{5.323566in}{0.718254in}}%
\pgfpathlineto{\pgfqpoint{5.323667in}{0.717148in}}%
\pgfpathlineto{\pgfqpoint{5.325183in}{0.708732in}}%
\pgfpathlineto{\pgfqpoint{5.325689in}{0.708025in}}%
\pgfpathlineto{\pgfqpoint{5.326194in}{0.708958in}}%
\pgfpathlineto{\pgfqpoint{5.330036in}{0.708108in}}%
\pgfpathlineto{\pgfqpoint{5.334079in}{0.711006in}}%
\pgfpathlineto{\pgfqpoint{5.334585in}{0.718739in}}%
\pgfpathlineto{\pgfqpoint{5.335090in}{0.731285in}}%
\pgfpathlineto{\pgfqpoint{5.335899in}{0.726560in}}%
\pgfpathlineto{\pgfqpoint{5.336202in}{0.731236in}}%
\pgfpathlineto{\pgfqpoint{5.336506in}{0.720714in}}%
\pgfpathlineto{\pgfqpoint{5.337112in}{0.708426in}}%
\pgfpathlineto{\pgfqpoint{5.337820in}{0.708656in}}%
\pgfpathlineto{\pgfqpoint{5.339235in}{0.708479in}}%
\pgfpathlineto{\pgfqpoint{5.345402in}{0.710743in}}%
\pgfpathlineto{\pgfqpoint{5.346413in}{0.726264in}}%
\pgfpathlineto{\pgfqpoint{5.346716in}{0.729694in}}%
\pgfpathlineto{\pgfqpoint{5.347323in}{0.724395in}}%
\pgfpathlineto{\pgfqpoint{5.349041in}{0.708914in}}%
\pgfpathlineto{\pgfqpoint{5.349648in}{0.710864in}}%
\pgfpathlineto{\pgfqpoint{5.351164in}{0.718419in}}%
\pgfpathlineto{\pgfqpoint{5.351265in}{0.717527in}}%
\pgfpathlineto{\pgfqpoint{5.351973in}{0.708085in}}%
\pgfpathlineto{\pgfqpoint{5.352478in}{0.712431in}}%
\pgfpathlineto{\pgfqpoint{5.352984in}{0.716551in}}%
\pgfpathlineto{\pgfqpoint{5.353692in}{0.715567in}}%
\pgfpathlineto{\pgfqpoint{5.354298in}{0.714310in}}%
\pgfpathlineto{\pgfqpoint{5.354601in}{0.715522in}}%
\pgfpathlineto{\pgfqpoint{5.355713in}{0.730496in}}%
\pgfpathlineto{\pgfqpoint{5.356522in}{0.724110in}}%
\pgfpathlineto{\pgfqpoint{5.359151in}{0.707962in}}%
\pgfpathlineto{\pgfqpoint{5.359959in}{0.709033in}}%
\pgfpathlineto{\pgfqpoint{5.360768in}{0.709041in}}%
\pgfpathlineto{\pgfqpoint{5.360970in}{0.708861in}}%
\pgfpathlineto{\pgfqpoint{5.362891in}{0.708059in}}%
\pgfpathlineto{\pgfqpoint{5.363801in}{0.709967in}}%
\pgfpathlineto{\pgfqpoint{5.364104in}{0.710187in}}%
\pgfpathlineto{\pgfqpoint{5.364711in}{0.709486in}}%
\pgfpathlineto{\pgfqpoint{5.367744in}{0.707598in}}%
\pgfpathlineto{\pgfqpoint{5.372091in}{0.708276in}}%
\pgfpathlineto{\pgfqpoint{5.374719in}{0.709250in}}%
\pgfpathlineto{\pgfqpoint{5.376438in}{0.713950in}}%
\pgfpathlineto{\pgfqpoint{5.376943in}{0.712661in}}%
\pgfpathlineto{\pgfqpoint{5.378055in}{0.709505in}}%
\pgfpathlineto{\pgfqpoint{5.378662in}{0.710208in}}%
\pgfpathlineto{\pgfqpoint{5.379369in}{0.710795in}}%
\pgfpathlineto{\pgfqpoint{5.379774in}{0.710141in}}%
\pgfpathlineto{\pgfqpoint{5.382099in}{0.707857in}}%
\pgfpathlineto{\pgfqpoint{5.384121in}{0.709495in}}%
\pgfpathlineto{\pgfqpoint{5.387153in}{0.735922in}}%
\pgfpathlineto{\pgfqpoint{5.387457in}{0.732512in}}%
\pgfpathlineto{\pgfqpoint{5.389479in}{0.708090in}}%
\pgfpathlineto{\pgfqpoint{5.398981in}{0.708957in}}%
\pgfpathlineto{\pgfqpoint{5.401610in}{0.708611in}}%
\pgfpathlineto{\pgfqpoint{5.408383in}{0.710277in}}%
\pgfpathlineto{\pgfqpoint{5.409192in}{0.718530in}}%
\pgfpathlineto{\pgfqpoint{5.410203in}{0.716693in}}%
\pgfpathlineto{\pgfqpoint{5.411719in}{0.706029in}}%
\pgfpathlineto{\pgfqpoint{5.412528in}{0.707455in}}%
\pgfpathlineto{\pgfqpoint{5.413033in}{0.722125in}}%
\pgfpathlineto{\pgfqpoint{5.413842in}{0.743576in}}%
\pgfpathlineto{\pgfqpoint{5.414348in}{0.737718in}}%
\pgfpathlineto{\pgfqpoint{5.416167in}{0.708337in}}%
\pgfpathlineto{\pgfqpoint{5.416673in}{0.709843in}}%
\pgfpathlineto{\pgfqpoint{5.417886in}{0.714743in}}%
\pgfpathlineto{\pgfqpoint{5.418290in}{0.713985in}}%
\pgfpathlineto{\pgfqpoint{5.420413in}{0.707742in}}%
\pgfpathlineto{\pgfqpoint{5.420717in}{0.708320in}}%
\pgfpathlineto{\pgfqpoint{5.421121in}{0.708928in}}%
\pgfpathlineto{\pgfqpoint{5.421829in}{0.708366in}}%
\pgfpathlineto{\pgfqpoint{5.422233in}{0.709219in}}%
\pgfpathlineto{\pgfqpoint{5.423951in}{0.716465in}}%
\pgfpathlineto{\pgfqpoint{5.424457in}{0.715305in}}%
\pgfpathlineto{\pgfqpoint{5.424962in}{0.716458in}}%
\pgfpathlineto{\pgfqpoint{5.425165in}{0.715253in}}%
\pgfpathlineto{\pgfqpoint{5.426883in}{0.706569in}}%
\pgfpathlineto{\pgfqpoint{5.427591in}{0.707669in}}%
\pgfpathlineto{\pgfqpoint{5.429613in}{0.717199in}}%
\pgfpathlineto{\pgfqpoint{5.429815in}{0.716268in}}%
\pgfpathlineto{\pgfqpoint{5.431635in}{0.707074in}}%
\pgfpathlineto{\pgfqpoint{5.433353in}{0.706513in}}%
\pgfpathlineto{\pgfqpoint{5.436184in}{0.707135in}}%
\pgfpathlineto{\pgfqpoint{5.437397in}{0.707417in}}%
\pgfpathlineto{\pgfqpoint{5.440329in}{0.709346in}}%
\pgfpathlineto{\pgfqpoint{5.440632in}{0.708570in}}%
\pgfpathlineto{\pgfqpoint{5.441340in}{0.707639in}}%
\pgfpathlineto{\pgfqpoint{5.441845in}{0.707977in}}%
\pgfpathlineto{\pgfqpoint{5.442350in}{0.711757in}}%
\pgfpathlineto{\pgfqpoint{5.442755in}{0.714923in}}%
\pgfpathlineto{\pgfqpoint{5.443463in}{0.713155in}}%
\pgfpathlineto{\pgfqpoint{5.444473in}{0.708398in}}%
\pgfpathlineto{\pgfqpoint{5.444878in}{0.710656in}}%
\pgfpathlineto{\pgfqpoint{5.445484in}{0.718659in}}%
\pgfpathlineto{\pgfqpoint{5.446293in}{0.715334in}}%
\pgfpathlineto{\pgfqpoint{5.447810in}{0.707717in}}%
\pgfpathlineto{\pgfqpoint{5.448820in}{0.708046in}}%
\pgfpathlineto{\pgfqpoint{5.450135in}{0.707298in}}%
\pgfpathlineto{\pgfqpoint{5.450236in}{0.707754in}}%
\pgfpathlineto{\pgfqpoint{5.451752in}{0.744332in}}%
\pgfpathlineto{\pgfqpoint{5.452055in}{0.768249in}}%
\pgfpathlineto{\pgfqpoint{5.452662in}{0.727529in}}%
\pgfpathlineto{\pgfqpoint{5.454077in}{0.713469in}}%
\pgfpathlineto{\pgfqpoint{5.455998in}{0.708215in}}%
\pgfpathlineto{\pgfqpoint{5.457515in}{0.708255in}}%
\pgfpathlineto{\pgfqpoint{5.459739in}{0.708802in}}%
\pgfpathlineto{\pgfqpoint{5.461154in}{0.708977in}}%
\pgfpathlineto{\pgfqpoint{5.463479in}{0.708502in}}%
\pgfpathlineto{\pgfqpoint{5.465602in}{0.709704in}}%
\pgfpathlineto{\pgfqpoint{5.467321in}{0.715627in}}%
\pgfpathlineto{\pgfqpoint{5.467624in}{0.714402in}}%
\pgfpathlineto{\pgfqpoint{5.469444in}{0.711315in}}%
\pgfpathlineto{\pgfqpoint{5.471465in}{0.708485in}}%
\pgfpathlineto{\pgfqpoint{5.472072in}{0.709419in}}%
\pgfpathlineto{\pgfqpoint{5.472577in}{0.724973in}}%
\pgfpathlineto{\pgfqpoint{5.472881in}{0.733590in}}%
\pgfpathlineto{\pgfqpoint{5.473689in}{0.726088in}}%
\pgfpathlineto{\pgfqpoint{5.475711in}{0.708670in}}%
\pgfpathlineto{\pgfqpoint{5.476621in}{0.709739in}}%
\pgfpathlineto{\pgfqpoint{5.477733in}{0.726188in}}%
\pgfpathlineto{\pgfqpoint{5.478946in}{0.719136in}}%
\pgfpathlineto{\pgfqpoint{5.481069in}{0.708864in}}%
\pgfpathlineto{\pgfqpoint{5.482080in}{0.708294in}}%
\pgfpathlineto{\pgfqpoint{5.482383in}{0.708523in}}%
\pgfpathlineto{\pgfqpoint{5.483900in}{0.709773in}}%
\pgfpathlineto{\pgfqpoint{5.484203in}{0.708762in}}%
\pgfpathlineto{\pgfqpoint{5.485618in}{0.707860in}}%
\pgfpathlineto{\pgfqpoint{5.490269in}{0.708926in}}%
\pgfpathlineto{\pgfqpoint{5.492796in}{0.710500in}}%
\pgfpathlineto{\pgfqpoint{5.493200in}{0.712245in}}%
\pgfpathlineto{\pgfqpoint{5.493807in}{0.710371in}}%
\pgfpathlineto{\pgfqpoint{5.495930in}{0.708737in}}%
\pgfpathlineto{\pgfqpoint{5.500884in}{0.709311in}}%
\pgfpathlineto{\pgfqpoint{5.502602in}{0.720692in}}%
\pgfpathlineto{\pgfqpoint{5.503007in}{0.715256in}}%
\pgfpathlineto{\pgfqpoint{5.503613in}{0.711154in}}%
\pgfpathlineto{\pgfqpoint{5.504321in}{0.712025in}}%
\pgfpathlineto{\pgfqpoint{5.505028in}{0.711215in}}%
\pgfpathlineto{\pgfqpoint{5.505736in}{0.708695in}}%
\pgfpathlineto{\pgfqpoint{5.507050in}{0.707916in}}%
\pgfpathlineto{\pgfqpoint{5.509173in}{0.708745in}}%
\pgfpathlineto{\pgfqpoint{5.510589in}{0.712602in}}%
\pgfpathlineto{\pgfqpoint{5.511296in}{0.710787in}}%
\pgfpathlineto{\pgfqpoint{5.512307in}{0.707887in}}%
\pgfpathlineto{\pgfqpoint{5.512712in}{0.708130in}}%
\pgfpathlineto{\pgfqpoint{5.513318in}{0.711727in}}%
\pgfpathlineto{\pgfqpoint{5.514531in}{0.717519in}}%
\pgfpathlineto{\pgfqpoint{5.514834in}{0.716879in}}%
\pgfpathlineto{\pgfqpoint{5.517867in}{0.706943in}}%
\pgfpathlineto{\pgfqpoint{5.518171in}{0.707540in}}%
\pgfpathlineto{\pgfqpoint{5.518777in}{0.716427in}}%
\pgfpathlineto{\pgfqpoint{5.519687in}{0.712151in}}%
\pgfpathlineto{\pgfqpoint{5.520496in}{0.708907in}}%
\pgfpathlineto{\pgfqpoint{5.520900in}{0.710804in}}%
\pgfpathlineto{\pgfqpoint{5.521102in}{0.712061in}}%
\pgfpathlineto{\pgfqpoint{5.521608in}{0.707634in}}%
\pgfpathlineto{\pgfqpoint{5.521810in}{0.707052in}}%
\pgfpathlineto{\pgfqpoint{5.522113in}{0.709490in}}%
\pgfpathlineto{\pgfqpoint{5.522315in}{0.712394in}}%
\pgfpathlineto{\pgfqpoint{5.522922in}{0.706667in}}%
\pgfpathlineto{\pgfqpoint{5.523023in}{0.706719in}}%
\pgfpathlineto{\pgfqpoint{5.525348in}{0.718651in}}%
\pgfpathlineto{\pgfqpoint{5.526056in}{0.714890in}}%
\pgfpathlineto{\pgfqpoint{5.526865in}{0.708854in}}%
\pgfpathlineto{\pgfqpoint{5.527269in}{0.711653in}}%
\pgfpathlineto{\pgfqpoint{5.527876in}{0.749816in}}%
\pgfpathlineto{\pgfqpoint{5.528381in}{0.774472in}}%
\pgfpathlineto{\pgfqpoint{5.528988in}{0.751874in}}%
\pgfpathlineto{\pgfqpoint{5.530403in}{0.716744in}}%
\pgfpathlineto{\pgfqpoint{5.530807in}{0.718110in}}%
\pgfpathlineto{\pgfqpoint{5.532324in}{0.726690in}}%
\pgfpathlineto{\pgfqpoint{5.532930in}{0.722366in}}%
\pgfpathlineto{\pgfqpoint{5.534851in}{0.709776in}}%
\pgfpathlineto{\pgfqpoint{5.535053in}{0.710106in}}%
\pgfpathlineto{\pgfqpoint{5.536266in}{0.715084in}}%
\pgfpathlineto{\pgfqpoint{5.536974in}{0.712675in}}%
\pgfpathlineto{\pgfqpoint{5.538591in}{0.706810in}}%
\pgfpathlineto{\pgfqpoint{5.538996in}{0.707023in}}%
\pgfpathlineto{\pgfqpoint{5.540816in}{0.707152in}}%
\pgfpathlineto{\pgfqpoint{5.541928in}{0.707981in}}%
\pgfpathlineto{\pgfqpoint{5.543949in}{0.719853in}}%
\pgfpathlineto{\pgfqpoint{5.544455in}{0.714627in}}%
\pgfpathlineto{\pgfqpoint{5.546072in}{0.708953in}}%
\pgfpathlineto{\pgfqpoint{5.548094in}{0.706067in}}%
\pgfpathlineto{\pgfqpoint{5.548802in}{0.707439in}}%
\pgfpathlineto{\pgfqpoint{5.549307in}{0.708828in}}%
\pgfpathlineto{\pgfqpoint{5.549914in}{0.707403in}}%
\pgfpathlineto{\pgfqpoint{5.551026in}{0.706524in}}%
\pgfpathlineto{\pgfqpoint{5.551228in}{0.706359in}}%
\pgfpathlineto{\pgfqpoint{5.551632in}{0.707072in}}%
\pgfpathlineto{\pgfqpoint{5.552138in}{0.710157in}}%
\pgfpathlineto{\pgfqpoint{5.552947in}{0.708634in}}%
\pgfpathlineto{\pgfqpoint{5.553452in}{0.713984in}}%
\pgfpathlineto{\pgfqpoint{5.554160in}{0.734903in}}%
\pgfpathlineto{\pgfqpoint{5.554867in}{0.724064in}}%
\pgfpathlineto{\pgfqpoint{5.558001in}{0.708121in}}%
\pgfpathlineto{\pgfqpoint{5.559417in}{0.707282in}}%
\pgfpathlineto{\pgfqpoint{5.563157in}{0.707858in}}%
\pgfpathlineto{\pgfqpoint{5.567403in}{0.710749in}}%
\pgfpathlineto{\pgfqpoint{5.568313in}{0.716910in}}%
\pgfpathlineto{\pgfqpoint{5.569122in}{0.715379in}}%
\pgfpathlineto{\pgfqpoint{5.572458in}{0.708801in}}%
\pgfpathlineto{\pgfqpoint{5.572862in}{0.708843in}}%
\pgfpathlineto{\pgfqpoint{5.573266in}{0.709456in}}%
\pgfpathlineto{\pgfqpoint{5.574480in}{0.709765in}}%
\pgfpathlineto{\pgfqpoint{5.574783in}{0.709163in}}%
\pgfpathlineto{\pgfqpoint{5.574985in}{0.709093in}}%
\pgfpathlineto{\pgfqpoint{5.575288in}{0.709974in}}%
\pgfpathlineto{\pgfqpoint{5.576704in}{0.716070in}}%
\pgfpathlineto{\pgfqpoint{5.576906in}{0.715027in}}%
\pgfpathlineto{\pgfqpoint{5.577816in}{0.708178in}}%
\pgfpathlineto{\pgfqpoint{5.578422in}{0.708308in}}%
\pgfpathlineto{\pgfqpoint{5.579939in}{0.709244in}}%
\pgfpathlineto{\pgfqpoint{5.581253in}{0.709428in}}%
\pgfpathlineto{\pgfqpoint{5.582365in}{0.708662in}}%
\pgfpathlineto{\pgfqpoint{5.584286in}{0.706484in}}%
\pgfpathlineto{\pgfqpoint{5.584387in}{0.706572in}}%
\pgfpathlineto{\pgfqpoint{5.586206in}{0.711497in}}%
\pgfpathlineto{\pgfqpoint{5.587116in}{0.728506in}}%
\pgfpathlineto{\pgfqpoint{5.587824in}{0.720570in}}%
\pgfpathlineto{\pgfqpoint{5.588431in}{0.713749in}}%
\pgfpathlineto{\pgfqpoint{5.589239in}{0.705789in}}%
\pgfpathlineto{\pgfqpoint{5.589846in}{0.706244in}}%
\pgfpathlineto{\pgfqpoint{5.591160in}{0.706265in}}%
\pgfpathlineto{\pgfqpoint{5.596215in}{0.710315in}}%
\pgfpathlineto{\pgfqpoint{5.596922in}{0.716790in}}%
\pgfpathlineto{\pgfqpoint{5.597731in}{0.714843in}}%
\pgfpathlineto{\pgfqpoint{5.599753in}{0.707977in}}%
\pgfpathlineto{\pgfqpoint{5.603595in}{0.709641in}}%
\pgfpathlineto{\pgfqpoint{5.605212in}{0.711304in}}%
\pgfpathlineto{\pgfqpoint{5.605313in}{0.711174in}}%
\pgfpathlineto{\pgfqpoint{5.607032in}{0.708159in}}%
\pgfpathlineto{\pgfqpoint{5.607234in}{0.708319in}}%
\pgfpathlineto{\pgfqpoint{5.607638in}{0.714321in}}%
\pgfpathlineto{\pgfqpoint{5.609256in}{0.766568in}}%
\pgfpathlineto{\pgfqpoint{5.609660in}{0.747356in}}%
\pgfpathlineto{\pgfqpoint{5.611278in}{0.716628in}}%
\pgfpathlineto{\pgfqpoint{5.612390in}{0.709100in}}%
\pgfpathlineto{\pgfqpoint{5.612794in}{0.710182in}}%
\pgfpathlineto{\pgfqpoint{5.612996in}{0.710610in}}%
\pgfpathlineto{\pgfqpoint{5.613401in}{0.709036in}}%
\pgfpathlineto{\pgfqpoint{5.613906in}{0.708129in}}%
\pgfpathlineto{\pgfqpoint{5.614513in}{0.708876in}}%
\pgfpathlineto{\pgfqpoint{5.616737in}{0.708568in}}%
\pgfpathlineto{\pgfqpoint{5.616838in}{0.708800in}}%
\pgfpathlineto{\pgfqpoint{5.617343in}{0.714005in}}%
\pgfpathlineto{\pgfqpoint{5.617849in}{0.721413in}}%
\pgfpathlineto{\pgfqpoint{5.618556in}{0.716747in}}%
\pgfpathlineto{\pgfqpoint{5.618860in}{0.717818in}}%
\pgfpathlineto{\pgfqpoint{5.619264in}{0.720135in}}%
\pgfpathlineto{\pgfqpoint{5.619668in}{0.715765in}}%
\pgfpathlineto{\pgfqpoint{5.621084in}{0.708335in}}%
\pgfpathlineto{\pgfqpoint{5.621690in}{0.707592in}}%
\pgfpathlineto{\pgfqpoint{5.622095in}{0.708267in}}%
\pgfpathlineto{\pgfqpoint{5.622600in}{0.709945in}}%
\pgfpathlineto{\pgfqpoint{5.623308in}{0.709034in}}%
\pgfpathlineto{\pgfqpoint{5.624319in}{0.710452in}}%
\pgfpathlineto{\pgfqpoint{5.624925in}{0.712236in}}%
\pgfpathlineto{\pgfqpoint{5.625633in}{0.711039in}}%
\pgfpathlineto{\pgfqpoint{5.626138in}{0.709520in}}%
\pgfpathlineto{\pgfqpoint{5.627554in}{0.706889in}}%
\pgfpathlineto{\pgfqpoint{5.627655in}{0.706910in}}%
\pgfpathlineto{\pgfqpoint{5.634529in}{0.710050in}}%
\pgfpathlineto{\pgfqpoint{5.635742in}{0.718382in}}%
\pgfpathlineto{\pgfqpoint{5.636450in}{0.714824in}}%
\pgfpathlineto{\pgfqpoint{5.637562in}{0.707372in}}%
\pgfpathlineto{\pgfqpoint{5.638270in}{0.707497in}}%
\pgfpathlineto{\pgfqpoint{5.639382in}{0.708777in}}%
\pgfpathlineto{\pgfqpoint{5.639887in}{0.717444in}}%
\pgfpathlineto{\pgfqpoint{5.640595in}{0.734758in}}%
\pgfpathlineto{\pgfqpoint{5.641201in}{0.727458in}}%
\pgfpathlineto{\pgfqpoint{5.642819in}{0.708490in}}%
\pgfpathlineto{\pgfqpoint{5.643223in}{0.709526in}}%
\pgfpathlineto{\pgfqpoint{5.643628in}{0.712109in}}%
\pgfpathlineto{\pgfqpoint{5.644133in}{0.708714in}}%
\pgfpathlineto{\pgfqpoint{5.645346in}{0.707674in}}%
\pgfpathlineto{\pgfqpoint{5.645548in}{0.707804in}}%
\pgfpathlineto{\pgfqpoint{5.646559in}{0.710134in}}%
\pgfpathlineto{\pgfqpoint{5.647267in}{0.714634in}}%
\pgfpathlineto{\pgfqpoint{5.648076in}{0.713314in}}%
\pgfpathlineto{\pgfqpoint{5.649188in}{0.707088in}}%
\pgfpathlineto{\pgfqpoint{5.650502in}{0.707919in}}%
\pgfpathlineto{\pgfqpoint{5.651513in}{0.712434in}}%
\pgfpathlineto{\pgfqpoint{5.652220in}{0.717505in}}%
\pgfpathlineto{\pgfqpoint{5.652827in}{0.714773in}}%
\pgfpathlineto{\pgfqpoint{5.654546in}{0.707548in}}%
\pgfpathlineto{\pgfqpoint{5.654748in}{0.707623in}}%
\pgfpathlineto{\pgfqpoint{5.655557in}{0.709465in}}%
\pgfpathlineto{\pgfqpoint{5.657275in}{0.712727in}}%
\pgfpathlineto{\pgfqpoint{5.657477in}{0.712890in}}%
\pgfpathlineto{\pgfqpoint{5.657882in}{0.711781in}}%
\pgfpathlineto{\pgfqpoint{5.660409in}{0.706428in}}%
\pgfpathlineto{\pgfqpoint{5.661622in}{0.707094in}}%
\pgfpathlineto{\pgfqpoint{5.662330in}{0.713078in}}%
\pgfpathlineto{\pgfqpoint{5.663037in}{0.723127in}}%
\pgfpathlineto{\pgfqpoint{5.663947in}{0.722237in}}%
\pgfpathlineto{\pgfqpoint{5.665059in}{0.706733in}}%
\pgfpathlineto{\pgfqpoint{5.666272in}{0.710474in}}%
\pgfpathlineto{\pgfqpoint{5.666677in}{0.712120in}}%
\pgfpathlineto{\pgfqpoint{5.667384in}{0.710572in}}%
\pgfpathlineto{\pgfqpoint{5.668395in}{0.706106in}}%
\pgfpathlineto{\pgfqpoint{5.669305in}{0.706562in}}%
\pgfpathlineto{\pgfqpoint{5.671226in}{0.705687in}}%
\pgfpathlineto{\pgfqpoint{5.671327in}{0.705771in}}%
\pgfpathlineto{\pgfqpoint{5.673956in}{0.709273in}}%
\pgfpathlineto{\pgfqpoint{5.674259in}{0.710515in}}%
\pgfpathlineto{\pgfqpoint{5.674966in}{0.708688in}}%
\pgfpathlineto{\pgfqpoint{5.676786in}{0.707630in}}%
\pgfpathlineto{\pgfqpoint{5.676887in}{0.708032in}}%
\pgfpathlineto{\pgfqpoint{5.678505in}{0.726657in}}%
\pgfpathlineto{\pgfqpoint{5.679212in}{0.772758in}}%
\pgfpathlineto{\pgfqpoint{5.679718in}{0.742649in}}%
\pgfpathlineto{\pgfqpoint{5.681436in}{0.708894in}}%
\pgfpathlineto{\pgfqpoint{5.683863in}{0.709141in}}%
\pgfpathlineto{\pgfqpoint{5.684570in}{0.710507in}}%
\pgfpathlineto{\pgfqpoint{5.685177in}{0.711167in}}%
\pgfpathlineto{\pgfqpoint{5.685783in}{0.710678in}}%
\pgfpathlineto{\pgfqpoint{5.687502in}{0.709695in}}%
\pgfpathlineto{\pgfqpoint{5.688816in}{0.709166in}}%
\pgfpathlineto{\pgfqpoint{5.689928in}{0.710231in}}%
\pgfpathlineto{\pgfqpoint{5.690535in}{0.715846in}}%
\pgfpathlineto{\pgfqpoint{5.691344in}{0.713655in}}%
\pgfpathlineto{\pgfqpoint{5.694882in}{0.708622in}}%
\pgfpathlineto{\pgfqpoint{5.697611in}{0.709873in}}%
\pgfpathlineto{\pgfqpoint{5.698218in}{0.712770in}}%
\pgfpathlineto{\pgfqpoint{5.698825in}{0.710865in}}%
\pgfpathlineto{\pgfqpoint{5.701453in}{0.708179in}}%
\pgfpathlineto{\pgfqpoint{5.703576in}{0.708019in}}%
\pgfpathlineto{\pgfqpoint{5.704991in}{0.707918in}}%
\pgfpathlineto{\pgfqpoint{5.708125in}{0.709985in}}%
\pgfpathlineto{\pgfqpoint{5.708631in}{0.710571in}}%
\pgfpathlineto{\pgfqpoint{5.709035in}{0.709705in}}%
\pgfpathlineto{\pgfqpoint{5.710855in}{0.708273in}}%
\pgfpathlineto{\pgfqpoint{5.713989in}{0.709461in}}%
\pgfpathlineto{\pgfqpoint{5.714494in}{0.713019in}}%
\pgfpathlineto{\pgfqpoint{5.715202in}{0.710963in}}%
\pgfpathlineto{\pgfqpoint{5.717224in}{0.710309in}}%
\pgfpathlineto{\pgfqpoint{5.717830in}{0.712966in}}%
\pgfpathlineto{\pgfqpoint{5.719448in}{0.722611in}}%
\pgfpathlineto{\pgfqpoint{5.719953in}{0.720014in}}%
\pgfpathlineto{\pgfqpoint{5.722076in}{0.707289in}}%
\pgfpathlineto{\pgfqpoint{5.722784in}{0.707356in}}%
\pgfpathlineto{\pgfqpoint{5.722986in}{0.707864in}}%
\pgfpathlineto{\pgfqpoint{5.723997in}{0.717336in}}%
\pgfpathlineto{\pgfqpoint{5.725008in}{0.713194in}}%
\pgfpathlineto{\pgfqpoint{5.725918in}{0.706578in}}%
\pgfpathlineto{\pgfqpoint{5.726524in}{0.706760in}}%
\pgfpathlineto{\pgfqpoint{5.728950in}{0.709286in}}%
\pgfpathlineto{\pgfqpoint{5.729254in}{0.710373in}}%
\pgfpathlineto{\pgfqpoint{5.730062in}{0.709493in}}%
\pgfpathlineto{\pgfqpoint{5.731377in}{0.708363in}}%
\pgfpathlineto{\pgfqpoint{5.731680in}{0.709086in}}%
\pgfpathlineto{\pgfqpoint{5.732084in}{0.710015in}}%
\pgfpathlineto{\pgfqpoint{5.732792in}{0.709212in}}%
\pgfpathlineto{\pgfqpoint{5.735420in}{0.707707in}}%
\pgfpathlineto{\pgfqpoint{5.737543in}{0.708246in}}%
\pgfpathlineto{\pgfqpoint{5.738453in}{0.709365in}}%
\pgfpathlineto{\pgfqpoint{5.740172in}{0.717756in}}%
\pgfpathlineto{\pgfqpoint{5.740475in}{0.716911in}}%
\pgfpathlineto{\pgfqpoint{5.740677in}{0.716603in}}%
\pgfpathlineto{\pgfqpoint{5.740879in}{0.718585in}}%
\pgfpathlineto{\pgfqpoint{5.741688in}{0.763967in}}%
\pgfpathlineto{\pgfqpoint{5.742396in}{0.731959in}}%
\pgfpathlineto{\pgfqpoint{5.744215in}{0.708936in}}%
\pgfpathlineto{\pgfqpoint{5.744317in}{0.708999in}}%
\pgfpathlineto{\pgfqpoint{5.745833in}{0.709188in}}%
\pgfpathlineto{\pgfqpoint{5.745833in}{0.709188in}}%
\pgfusepath{stroke}%
\end{pgfscope}%
\begin{pgfscope}%
\pgfpathrectangle{\pgfqpoint{0.691161in}{0.544166in}}{\pgfqpoint{5.054672in}{0.911907in}}%
\pgfusepath{clip}%
\pgfsetbuttcap%
\pgfsetroundjoin%
\pgfsetlinewidth{2.007500pt}%
\definecolor{currentstroke}{rgb}{0.172549,0.627451,0.172549}%
\pgfsetstrokecolor{currentstroke}%
\pgfsetdash{{7.400000pt}{3.200000pt}}{0.000000pt}%
\pgfpathmoveto{\pgfqpoint{0.691161in}{0.978290in}}%
\pgfpathlineto{\pgfqpoint{5.745833in}{0.978290in}}%
\pgfusepath{stroke}%
\end{pgfscope}%
\begin{pgfscope}%
\pgfpathrectangle{\pgfqpoint{0.691161in}{0.544166in}}{\pgfqpoint{5.054672in}{0.911907in}}%
\pgfusepath{clip}%
\pgfsetbuttcap%
\pgfsetroundjoin%
\pgfsetlinewidth{2.007500pt}%
\definecolor{currentstroke}{rgb}{0.839216,0.152941,0.156863}%
\pgfsetstrokecolor{currentstroke}%
\pgfsetdash{{2.000000pt}{3.300000pt}}{0.000000pt}%
\pgfpathmoveto{\pgfqpoint{3.002562in}{0.585617in}}%
\pgfpathlineto{\pgfqpoint{3.002562in}{1.414623in}}%
\pgfusepath{stroke}%
\end{pgfscope}%
\begin{pgfscope}%
\pgfsetrectcap%
\pgfsetmiterjoin%
\pgfsetlinewidth{0.803000pt}%
\definecolor{currentstroke}{rgb}{0.737255,0.737255,0.737255}%
\pgfsetstrokecolor{currentstroke}%
\pgfsetdash{}{0pt}%
\pgfpathmoveto{\pgfqpoint{0.691161in}{0.544166in}}%
\pgfpathlineto{\pgfqpoint{0.691161in}{1.456074in}}%
\pgfusepath{stroke}%
\end{pgfscope}%
\begin{pgfscope}%
\pgfsetrectcap%
\pgfsetmiterjoin%
\pgfsetlinewidth{0.803000pt}%
\definecolor{currentstroke}{rgb}{0.737255,0.737255,0.737255}%
\pgfsetstrokecolor{currentstroke}%
\pgfsetdash{}{0pt}%
\pgfpathmoveto{\pgfqpoint{5.745833in}{0.544166in}}%
\pgfpathlineto{\pgfqpoint{5.745833in}{1.456074in}}%
\pgfusepath{stroke}%
\end{pgfscope}%
\begin{pgfscope}%
\pgfsetrectcap%
\pgfsetmiterjoin%
\pgfsetlinewidth{0.803000pt}%
\definecolor{currentstroke}{rgb}{0.737255,0.737255,0.737255}%
\pgfsetstrokecolor{currentstroke}%
\pgfsetdash{}{0pt}%
\pgfpathmoveto{\pgfqpoint{0.691161in}{0.544166in}}%
\pgfpathlineto{\pgfqpoint{5.745833in}{0.544166in}}%
\pgfusepath{stroke}%
\end{pgfscope}%
\begin{pgfscope}%
\pgfsetrectcap%
\pgfsetmiterjoin%
\pgfsetlinewidth{0.803000pt}%
\definecolor{currentstroke}{rgb}{0.737255,0.737255,0.737255}%
\pgfsetstrokecolor{currentstroke}%
\pgfsetdash{}{0pt}%
\pgfpathmoveto{\pgfqpoint{0.691161in}{1.456074in}}%
\pgfpathlineto{\pgfqpoint{5.745833in}{1.456074in}}%
\pgfusepath{stroke}%
\end{pgfscope}%
\begin{pgfscope}%
\pgfsetbuttcap%
\pgfsetmiterjoin%
\definecolor{currentfill}{rgb}{0.933333,0.933333,0.933333}%
\pgfsetfillcolor{currentfill}%
\pgfsetfillopacity{0.800000}%
\pgfsetlinewidth{0.501875pt}%
\definecolor{currentstroke}{rgb}{0.800000,0.800000,0.800000}%
\pgfsetstrokecolor{currentstroke}%
\pgfsetstrokeopacity{0.800000}%
\pgfsetdash{}{0pt}%
\pgfpathmoveto{\pgfqpoint{4.404343in}{0.957741in}}%
\pgfpathlineto{\pgfqpoint{5.648611in}{0.957741in}}%
\pgfpathquadraticcurveto{\pgfqpoint{5.676389in}{0.957741in}}{\pgfqpoint{5.676389in}{0.985518in}}%
\pgfpathlineto{\pgfqpoint{5.676389in}{1.358851in}}%
\pgfpathquadraticcurveto{\pgfqpoint{5.676389in}{1.386629in}}{\pgfqpoint{5.648611in}{1.386629in}}%
\pgfpathlineto{\pgfqpoint{4.404343in}{1.386629in}}%
\pgfpathquadraticcurveto{\pgfqpoint{4.376566in}{1.386629in}}{\pgfqpoint{4.376566in}{1.358851in}}%
\pgfpathlineto{\pgfqpoint{4.376566in}{0.985518in}}%
\pgfpathquadraticcurveto{\pgfqpoint{4.376566in}{0.957741in}}{\pgfqpoint{4.404343in}{0.957741in}}%
\pgfpathlineto{\pgfqpoint{4.404343in}{0.957741in}}%
\pgfpathclose%
\pgfusepath{stroke,fill}%
\end{pgfscope}%
\begin{pgfscope}%
\pgfsetbuttcap%
\pgfsetroundjoin%
\pgfsetlinewidth{2.007500pt}%
\definecolor{currentstroke}{rgb}{0.172549,0.627451,0.172549}%
\pgfsetstrokecolor{currentstroke}%
\pgfsetdash{{7.400000pt}{3.200000pt}}{0.000000pt}%
\pgfpathmoveto{\pgfqpoint{4.432121in}{1.282463in}}%
\pgfpathlineto{\pgfqpoint{4.709899in}{1.282463in}}%
\pgfusepath{stroke}%
\end{pgfscope}%
\begin{pgfscope}%
\definecolor{textcolor}{rgb}{0.000000,0.000000,0.000000}%
\pgfsetstrokecolor{textcolor}%
\pgfsetfillcolor{textcolor}%
\pgftext[x=4.821010in,y=1.233851in,left,base]{\color{textcolor}\rmfamily\fontsize{10.000000}{12.000000}\selectfont Seuil = 200}%
\end{pgfscope}%
\begin{pgfscope}%
\pgfsetbuttcap%
\pgfsetroundjoin%
\pgfsetlinewidth{2.007500pt}%
\definecolor{currentstroke}{rgb}{0.839216,0.152941,0.156863}%
\pgfsetstrokecolor{currentstroke}%
\pgfsetdash{{2.000000pt}{3.300000pt}}{0.000000pt}%
\pgfpathmoveto{\pgfqpoint{4.432121in}{1.088852in}}%
\pgfpathlineto{\pgfqpoint{4.709899in}{1.088852in}}%
\pgfusepath{stroke}%
\end{pgfscope}%
\begin{pgfscope}%
\definecolor{textcolor}{rgb}{0.000000,0.000000,0.000000}%
\pgfsetstrokecolor{textcolor}%
\pgfsetfillcolor{textcolor}%
\pgftext[x=4.821010in,y=1.040240in,left,base]{\color{textcolor}\rmfamily\fontsize{10.000000}{12.000000}\selectfont \(\displaystyle t_1\) = 164.32 s}%
\end{pgfscope}%
\end{pgfpicture}%
\makeatother%
\endgroup%
}
    \caption{Visualisation de la détection et du pointage du séisme induit de Strasbourg du 26 juin 2021 de magnitude 4. Le signal d'entrée (gris) correspond à l'enregistrement de la vitesse de déplacement du sol au niveau de la station permanente d'Illfurth (ILLF) traité. Les fonctions caractéristiques de ce signal (bleu) sont représentés avec leurs seuils respectifs (vert pointillé).}
    \label{fig:pointage-strasbourg}
\end{figure}

\subsection{Détection}

Contrairement au calcul de fonctions caractéristiques vu précédemment, la détection d'un éventuel séisme implique l'introduction de différents paramètres qui dépendent de la nature et la localisation du séisme. Par exemple, nous pouvons considérer un premier seuil qui permet de déclencher des opérations de vérification. Mais ensuite il faut vérifier que le seuil est dépassé pendant une certaine durée. De la même façon nous pouvons introduire de nombreux paramètres.

Ces paramètres se déduisent de la littérature mais varient beaucoup d'une publication à une autre. Ils sont ajustées à partir de données de pointages manuels sur les échantillons de séismes qu'on cherche à étudier. Ces ajustements s'éloignant du "projet informatique" nous avons limité le nombre de paramètres. Le Z-détecteur présente l'avantage de s'adapter automatiquement à la variance du bruit fond [d'après \cite{wither1998}]. Ainsi les paramètres de détection varient peu et cette fonction caractéristique est intéressante à utiliser pour la détection de séismes plus que pour le pointage, même si les autres fonctions caractéristiques auraient également pu être utilisées pour jouer ce rôle de détection.

Notre algorithme se base d'abord sur la détection du séisme en vérifiant que le seuil est dépassé pendant une certaine période. Si c'est le cas, nous pouvons considérer qu'il y a un séisme et l'algorithme va sélectionner un intervalle en se basant sur un deuxième seuil plus faible et en élargissant cet intervalle. En sortie nous avons l'intervalle de temps correspondant à la secousse à partir duquel nous pourrons pointer le début des phases sismiques en se basant sur d'autres fonctions caractéristiques.

\subsection{Pointage}

Le pointage est dépendant de la détection, il a lieu que si un séisme est détecté et prend en compte l'intervalle de temps supposé correspondre à la secousse. Nous gardons tous les intervalles où le seuil d'amplitude est dépassé par une nouvelle fonction caractéristique et qui correspondent à l'intervalle supposé de la secousse précédent. Pour chacun de ces nouveaux intervalles, nous calculons le maximum local d'amplitude de l'intervalle et le début de l'intervalle. Nous trions les intervalles par ordre décroissante par rapport à leur maximum, pour ensuite ne garder que les temps de début d'intervalles. Nous obtenons ainsi plusieurs temps de pointages, normalement deux, correspondant à l'arrivée des ondes P puis S.

\begin{figure}[!ht]
    \centering
    \scalebox{.9}{%% Creator: Matplotlib, PGF backend
%%
%% To include the figure in your LaTeX document, write
%%   \input{<filename>.pgf}
%%
%% Make sure the required packages are loaded in your preamble
%%   \usepackage{pgf}
%%
%% Also ensure that all the required font packages are loaded; for instance,
%% the lmodern package is sometimes necessary when using math font.
%%   \usepackage{lmodern}
%%
%% Figures using additional raster images can only be included by \input if
%% they are in the same directory as the main LaTeX file. For loading figures
%% from other directories you can use the `import` package
%%   \usepackage{import}
%%
%% and then include the figures with
%%   \import{<path to file>}{<filename>.pgf}
%%
%% Matplotlib used the following preamble
%%   \usepackage{fontspec}
%%
\begingroup%
\makeatletter%
\begin{pgfpicture}%
\pgfpathrectangle{\pgfpointorigin}{\pgfqpoint{6.000000in}{6.000000in}}%
\pgfusepath{use as bounding box, clip}%
\begin{pgfscope}%
\pgfsetbuttcap%
\pgfsetmiterjoin%
\definecolor{currentfill}{rgb}{1.000000,1.000000,1.000000}%
\pgfsetfillcolor{currentfill}%
\pgfsetlinewidth{0.000000pt}%
\definecolor{currentstroke}{rgb}{1.000000,1.000000,1.000000}%
\pgfsetstrokecolor{currentstroke}%
\pgfsetdash{}{0pt}%
\pgfpathmoveto{\pgfqpoint{0.000000in}{0.000000in}}%
\pgfpathlineto{\pgfqpoint{6.000000in}{0.000000in}}%
\pgfpathlineto{\pgfqpoint{6.000000in}{6.000000in}}%
\pgfpathlineto{\pgfqpoint{0.000000in}{6.000000in}}%
\pgfpathlineto{\pgfqpoint{0.000000in}{0.000000in}}%
\pgfpathclose%
\pgfusepath{fill}%
\end{pgfscope}%
\begin{pgfscope}%
\pgfsetbuttcap%
\pgfsetmiterjoin%
\definecolor{currentfill}{rgb}{0.933333,0.933333,0.933333}%
\pgfsetfillcolor{currentfill}%
\pgfsetlinewidth{0.000000pt}%
\definecolor{currentstroke}{rgb}{0.000000,0.000000,0.000000}%
\pgfsetstrokecolor{currentstroke}%
\pgfsetstrokeopacity{0.000000}%
\pgfsetdash{}{0pt}%
\pgfpathmoveto{\pgfqpoint{0.691161in}{4.801386in}}%
\pgfpathlineto{\pgfqpoint{5.745833in}{4.801386in}}%
\pgfpathlineto{\pgfqpoint{5.745833in}{5.703703in}}%
\pgfpathlineto{\pgfqpoint{0.691161in}{5.703703in}}%
\pgfpathlineto{\pgfqpoint{0.691161in}{4.801386in}}%
\pgfpathclose%
\pgfusepath{fill}%
\end{pgfscope}%
\begin{pgfscope}%
\pgfpathrectangle{\pgfqpoint{0.691161in}{4.801386in}}{\pgfqpoint{5.054672in}{0.902317in}}%
\pgfusepath{clip}%
\pgfsetbuttcap%
\pgfsetroundjoin%
\pgfsetlinewidth{0.501875pt}%
\definecolor{currentstroke}{rgb}{0.698039,0.698039,0.698039}%
\pgfsetstrokecolor{currentstroke}%
\pgfsetdash{{1.850000pt}{0.800000pt}}{0.000000pt}%
\pgfpathmoveto{\pgfqpoint{0.691161in}{4.801386in}}%
\pgfpathlineto{\pgfqpoint{0.691161in}{5.703703in}}%
\pgfusepath{stroke}%
\end{pgfscope}%
\begin{pgfscope}%
\pgfsetbuttcap%
\pgfsetroundjoin%
\definecolor{currentfill}{rgb}{0.000000,0.000000,0.000000}%
\pgfsetfillcolor{currentfill}%
\pgfsetlinewidth{0.803000pt}%
\definecolor{currentstroke}{rgb}{0.000000,0.000000,0.000000}%
\pgfsetstrokecolor{currentstroke}%
\pgfsetdash{}{0pt}%
\pgfsys@defobject{currentmarker}{\pgfqpoint{0.000000in}{0.000000in}}{\pgfqpoint{0.000000in}{0.048611in}}{%
\pgfpathmoveto{\pgfqpoint{0.000000in}{0.000000in}}%
\pgfpathlineto{\pgfqpoint{0.000000in}{0.048611in}}%
\pgfusepath{stroke,fill}%
}%
\begin{pgfscope}%
\pgfsys@transformshift{0.691161in}{4.801386in}%
\pgfsys@useobject{currentmarker}{}%
\end{pgfscope}%
\end{pgfscope}%
\begin{pgfscope}%
\pgfpathrectangle{\pgfqpoint{0.691161in}{4.801386in}}{\pgfqpoint{5.054672in}{0.902317in}}%
\pgfusepath{clip}%
\pgfsetbuttcap%
\pgfsetroundjoin%
\pgfsetlinewidth{0.501875pt}%
\definecolor{currentstroke}{rgb}{0.698039,0.698039,0.698039}%
\pgfsetstrokecolor{currentstroke}%
\pgfsetdash{{1.850000pt}{0.800000pt}}{0.000000pt}%
\pgfpathmoveto{\pgfqpoint{1.533607in}{4.801386in}}%
\pgfpathlineto{\pgfqpoint{1.533607in}{5.703703in}}%
\pgfusepath{stroke}%
\end{pgfscope}%
\begin{pgfscope}%
\pgfsetbuttcap%
\pgfsetroundjoin%
\definecolor{currentfill}{rgb}{0.000000,0.000000,0.000000}%
\pgfsetfillcolor{currentfill}%
\pgfsetlinewidth{0.803000pt}%
\definecolor{currentstroke}{rgb}{0.000000,0.000000,0.000000}%
\pgfsetstrokecolor{currentstroke}%
\pgfsetdash{}{0pt}%
\pgfsys@defobject{currentmarker}{\pgfqpoint{0.000000in}{0.000000in}}{\pgfqpoint{0.000000in}{0.048611in}}{%
\pgfpathmoveto{\pgfqpoint{0.000000in}{0.000000in}}%
\pgfpathlineto{\pgfqpoint{0.000000in}{0.048611in}}%
\pgfusepath{stroke,fill}%
}%
\begin{pgfscope}%
\pgfsys@transformshift{1.533607in}{4.801386in}%
\pgfsys@useobject{currentmarker}{}%
\end{pgfscope}%
\end{pgfscope}%
\begin{pgfscope}%
\pgfpathrectangle{\pgfqpoint{0.691161in}{4.801386in}}{\pgfqpoint{5.054672in}{0.902317in}}%
\pgfusepath{clip}%
\pgfsetbuttcap%
\pgfsetroundjoin%
\pgfsetlinewidth{0.501875pt}%
\definecolor{currentstroke}{rgb}{0.698039,0.698039,0.698039}%
\pgfsetstrokecolor{currentstroke}%
\pgfsetdash{{1.850000pt}{0.800000pt}}{0.000000pt}%
\pgfpathmoveto{\pgfqpoint{2.376052in}{4.801386in}}%
\pgfpathlineto{\pgfqpoint{2.376052in}{5.703703in}}%
\pgfusepath{stroke}%
\end{pgfscope}%
\begin{pgfscope}%
\pgfsetbuttcap%
\pgfsetroundjoin%
\definecolor{currentfill}{rgb}{0.000000,0.000000,0.000000}%
\pgfsetfillcolor{currentfill}%
\pgfsetlinewidth{0.803000pt}%
\definecolor{currentstroke}{rgb}{0.000000,0.000000,0.000000}%
\pgfsetstrokecolor{currentstroke}%
\pgfsetdash{}{0pt}%
\pgfsys@defobject{currentmarker}{\pgfqpoint{0.000000in}{0.000000in}}{\pgfqpoint{0.000000in}{0.048611in}}{%
\pgfpathmoveto{\pgfqpoint{0.000000in}{0.000000in}}%
\pgfpathlineto{\pgfqpoint{0.000000in}{0.048611in}}%
\pgfusepath{stroke,fill}%
}%
\begin{pgfscope}%
\pgfsys@transformshift{2.376052in}{4.801386in}%
\pgfsys@useobject{currentmarker}{}%
\end{pgfscope}%
\end{pgfscope}%
\begin{pgfscope}%
\pgfpathrectangle{\pgfqpoint{0.691161in}{4.801386in}}{\pgfqpoint{5.054672in}{0.902317in}}%
\pgfusepath{clip}%
\pgfsetbuttcap%
\pgfsetroundjoin%
\pgfsetlinewidth{0.501875pt}%
\definecolor{currentstroke}{rgb}{0.698039,0.698039,0.698039}%
\pgfsetstrokecolor{currentstroke}%
\pgfsetdash{{1.850000pt}{0.800000pt}}{0.000000pt}%
\pgfpathmoveto{\pgfqpoint{3.218497in}{4.801386in}}%
\pgfpathlineto{\pgfqpoint{3.218497in}{5.703703in}}%
\pgfusepath{stroke}%
\end{pgfscope}%
\begin{pgfscope}%
\pgfsetbuttcap%
\pgfsetroundjoin%
\definecolor{currentfill}{rgb}{0.000000,0.000000,0.000000}%
\pgfsetfillcolor{currentfill}%
\pgfsetlinewidth{0.803000pt}%
\definecolor{currentstroke}{rgb}{0.000000,0.000000,0.000000}%
\pgfsetstrokecolor{currentstroke}%
\pgfsetdash{}{0pt}%
\pgfsys@defobject{currentmarker}{\pgfqpoint{0.000000in}{0.000000in}}{\pgfqpoint{0.000000in}{0.048611in}}{%
\pgfpathmoveto{\pgfqpoint{0.000000in}{0.000000in}}%
\pgfpathlineto{\pgfqpoint{0.000000in}{0.048611in}}%
\pgfusepath{stroke,fill}%
}%
\begin{pgfscope}%
\pgfsys@transformshift{3.218497in}{4.801386in}%
\pgfsys@useobject{currentmarker}{}%
\end{pgfscope}%
\end{pgfscope}%
\begin{pgfscope}%
\pgfpathrectangle{\pgfqpoint{0.691161in}{4.801386in}}{\pgfqpoint{5.054672in}{0.902317in}}%
\pgfusepath{clip}%
\pgfsetbuttcap%
\pgfsetroundjoin%
\pgfsetlinewidth{0.501875pt}%
\definecolor{currentstroke}{rgb}{0.698039,0.698039,0.698039}%
\pgfsetstrokecolor{currentstroke}%
\pgfsetdash{{1.850000pt}{0.800000pt}}{0.000000pt}%
\pgfpathmoveto{\pgfqpoint{4.060942in}{4.801386in}}%
\pgfpathlineto{\pgfqpoint{4.060942in}{5.703703in}}%
\pgfusepath{stroke}%
\end{pgfscope}%
\begin{pgfscope}%
\pgfsetbuttcap%
\pgfsetroundjoin%
\definecolor{currentfill}{rgb}{0.000000,0.000000,0.000000}%
\pgfsetfillcolor{currentfill}%
\pgfsetlinewidth{0.803000pt}%
\definecolor{currentstroke}{rgb}{0.000000,0.000000,0.000000}%
\pgfsetstrokecolor{currentstroke}%
\pgfsetdash{}{0pt}%
\pgfsys@defobject{currentmarker}{\pgfqpoint{0.000000in}{0.000000in}}{\pgfqpoint{0.000000in}{0.048611in}}{%
\pgfpathmoveto{\pgfqpoint{0.000000in}{0.000000in}}%
\pgfpathlineto{\pgfqpoint{0.000000in}{0.048611in}}%
\pgfusepath{stroke,fill}%
}%
\begin{pgfscope}%
\pgfsys@transformshift{4.060942in}{4.801386in}%
\pgfsys@useobject{currentmarker}{}%
\end{pgfscope}%
\end{pgfscope}%
\begin{pgfscope}%
\pgfpathrectangle{\pgfqpoint{0.691161in}{4.801386in}}{\pgfqpoint{5.054672in}{0.902317in}}%
\pgfusepath{clip}%
\pgfsetbuttcap%
\pgfsetroundjoin%
\pgfsetlinewidth{0.501875pt}%
\definecolor{currentstroke}{rgb}{0.698039,0.698039,0.698039}%
\pgfsetstrokecolor{currentstroke}%
\pgfsetdash{{1.850000pt}{0.800000pt}}{0.000000pt}%
\pgfpathmoveto{\pgfqpoint{4.903388in}{4.801386in}}%
\pgfpathlineto{\pgfqpoint{4.903388in}{5.703703in}}%
\pgfusepath{stroke}%
\end{pgfscope}%
\begin{pgfscope}%
\pgfsetbuttcap%
\pgfsetroundjoin%
\definecolor{currentfill}{rgb}{0.000000,0.000000,0.000000}%
\pgfsetfillcolor{currentfill}%
\pgfsetlinewidth{0.803000pt}%
\definecolor{currentstroke}{rgb}{0.000000,0.000000,0.000000}%
\pgfsetstrokecolor{currentstroke}%
\pgfsetdash{}{0pt}%
\pgfsys@defobject{currentmarker}{\pgfqpoint{0.000000in}{0.000000in}}{\pgfqpoint{0.000000in}{0.048611in}}{%
\pgfpathmoveto{\pgfqpoint{0.000000in}{0.000000in}}%
\pgfpathlineto{\pgfqpoint{0.000000in}{0.048611in}}%
\pgfusepath{stroke,fill}%
}%
\begin{pgfscope}%
\pgfsys@transformshift{4.903388in}{4.801386in}%
\pgfsys@useobject{currentmarker}{}%
\end{pgfscope}%
\end{pgfscope}%
\begin{pgfscope}%
\pgfpathrectangle{\pgfqpoint{0.691161in}{4.801386in}}{\pgfqpoint{5.054672in}{0.902317in}}%
\pgfusepath{clip}%
\pgfsetbuttcap%
\pgfsetroundjoin%
\pgfsetlinewidth{0.501875pt}%
\definecolor{currentstroke}{rgb}{0.698039,0.698039,0.698039}%
\pgfsetstrokecolor{currentstroke}%
\pgfsetdash{{1.850000pt}{0.800000pt}}{0.000000pt}%
\pgfpathmoveto{\pgfqpoint{5.745833in}{4.801386in}}%
\pgfpathlineto{\pgfqpoint{5.745833in}{5.703703in}}%
\pgfusepath{stroke}%
\end{pgfscope}%
\begin{pgfscope}%
\pgfsetbuttcap%
\pgfsetroundjoin%
\definecolor{currentfill}{rgb}{0.000000,0.000000,0.000000}%
\pgfsetfillcolor{currentfill}%
\pgfsetlinewidth{0.803000pt}%
\definecolor{currentstroke}{rgb}{0.000000,0.000000,0.000000}%
\pgfsetstrokecolor{currentstroke}%
\pgfsetdash{}{0pt}%
\pgfsys@defobject{currentmarker}{\pgfqpoint{0.000000in}{0.000000in}}{\pgfqpoint{0.000000in}{0.048611in}}{%
\pgfpathmoveto{\pgfqpoint{0.000000in}{0.000000in}}%
\pgfpathlineto{\pgfqpoint{0.000000in}{0.048611in}}%
\pgfusepath{stroke,fill}%
}%
\begin{pgfscope}%
\pgfsys@transformshift{5.745833in}{4.801386in}%
\pgfsys@useobject{currentmarker}{}%
\end{pgfscope}%
\end{pgfscope}%
\begin{pgfscope}%
\pgfpathrectangle{\pgfqpoint{0.691161in}{4.801386in}}{\pgfqpoint{5.054672in}{0.902317in}}%
\pgfusepath{clip}%
\pgfsetbuttcap%
\pgfsetroundjoin%
\pgfsetlinewidth{0.501875pt}%
\definecolor{currentstroke}{rgb}{0.698039,0.698039,0.698039}%
\pgfsetstrokecolor{currentstroke}%
\pgfsetdash{{1.850000pt}{0.800000pt}}{0.000000pt}%
\pgfpathmoveto{\pgfqpoint{0.691161in}{4.890344in}}%
\pgfpathlineto{\pgfqpoint{5.745833in}{4.890344in}}%
\pgfusepath{stroke}%
\end{pgfscope}%
\begin{pgfscope}%
\pgfsetbuttcap%
\pgfsetroundjoin%
\definecolor{currentfill}{rgb}{0.000000,0.000000,0.000000}%
\pgfsetfillcolor{currentfill}%
\pgfsetlinewidth{0.803000pt}%
\definecolor{currentstroke}{rgb}{0.000000,0.000000,0.000000}%
\pgfsetstrokecolor{currentstroke}%
\pgfsetdash{}{0pt}%
\pgfsys@defobject{currentmarker}{\pgfqpoint{0.000000in}{0.000000in}}{\pgfqpoint{0.048611in}{0.000000in}}{%
\pgfpathmoveto{\pgfqpoint{0.000000in}{0.000000in}}%
\pgfpathlineto{\pgfqpoint{0.048611in}{0.000000in}}%
\pgfusepath{stroke,fill}%
}%
\begin{pgfscope}%
\pgfsys@transformshift{0.691161in}{4.890344in}%
\pgfsys@useobject{currentmarker}{}%
\end{pgfscope}%
\end{pgfscope}%
\begin{pgfscope}%
\definecolor{textcolor}{rgb}{0.000000,0.000000,0.000000}%
\pgfsetstrokecolor{textcolor}%
\pgfsetfillcolor{textcolor}%
\pgftext[x=0.357055in, y=4.842149in, left, base]{\color{textcolor}\rmfamily\fontsize{10.000000}{12.000000}\selectfont \(\displaystyle {\ensuremath{-}2.5}\)}%
\end{pgfscope}%
\begin{pgfscope}%
\pgfpathrectangle{\pgfqpoint{0.691161in}{4.801386in}}{\pgfqpoint{5.054672in}{0.902317in}}%
\pgfusepath{clip}%
\pgfsetbuttcap%
\pgfsetroundjoin%
\pgfsetlinewidth{0.501875pt}%
\definecolor{currentstroke}{rgb}{0.698039,0.698039,0.698039}%
\pgfsetstrokecolor{currentstroke}%
\pgfsetdash{{1.850000pt}{0.800000pt}}{0.000000pt}%
\pgfpathmoveto{\pgfqpoint{0.691161in}{5.225310in}}%
\pgfpathlineto{\pgfqpoint{5.745833in}{5.225310in}}%
\pgfusepath{stroke}%
\end{pgfscope}%
\begin{pgfscope}%
\pgfsetbuttcap%
\pgfsetroundjoin%
\definecolor{currentfill}{rgb}{0.000000,0.000000,0.000000}%
\pgfsetfillcolor{currentfill}%
\pgfsetlinewidth{0.803000pt}%
\definecolor{currentstroke}{rgb}{0.000000,0.000000,0.000000}%
\pgfsetstrokecolor{currentstroke}%
\pgfsetdash{}{0pt}%
\pgfsys@defobject{currentmarker}{\pgfqpoint{0.000000in}{0.000000in}}{\pgfqpoint{0.048611in}{0.000000in}}{%
\pgfpathmoveto{\pgfqpoint{0.000000in}{0.000000in}}%
\pgfpathlineto{\pgfqpoint{0.048611in}{0.000000in}}%
\pgfusepath{stroke,fill}%
}%
\begin{pgfscope}%
\pgfsys@transformshift{0.691161in}{5.225310in}%
\pgfsys@useobject{currentmarker}{}%
\end{pgfscope}%
\end{pgfscope}%
\begin{pgfscope}%
\definecolor{textcolor}{rgb}{0.000000,0.000000,0.000000}%
\pgfsetstrokecolor{textcolor}%
\pgfsetfillcolor{textcolor}%
\pgftext[x=0.465080in, y=5.177116in, left, base]{\color{textcolor}\rmfamily\fontsize{10.000000}{12.000000}\selectfont \(\displaystyle {0.0}\)}%
\end{pgfscope}%
\begin{pgfscope}%
\pgfpathrectangle{\pgfqpoint{0.691161in}{4.801386in}}{\pgfqpoint{5.054672in}{0.902317in}}%
\pgfusepath{clip}%
\pgfsetbuttcap%
\pgfsetroundjoin%
\pgfsetlinewidth{0.501875pt}%
\definecolor{currentstroke}{rgb}{0.698039,0.698039,0.698039}%
\pgfsetstrokecolor{currentstroke}%
\pgfsetdash{{1.850000pt}{0.800000pt}}{0.000000pt}%
\pgfpathmoveto{\pgfqpoint{0.691161in}{5.560277in}}%
\pgfpathlineto{\pgfqpoint{5.745833in}{5.560277in}}%
\pgfusepath{stroke}%
\end{pgfscope}%
\begin{pgfscope}%
\pgfsetbuttcap%
\pgfsetroundjoin%
\definecolor{currentfill}{rgb}{0.000000,0.000000,0.000000}%
\pgfsetfillcolor{currentfill}%
\pgfsetlinewidth{0.803000pt}%
\definecolor{currentstroke}{rgb}{0.000000,0.000000,0.000000}%
\pgfsetstrokecolor{currentstroke}%
\pgfsetdash{}{0pt}%
\pgfsys@defobject{currentmarker}{\pgfqpoint{0.000000in}{0.000000in}}{\pgfqpoint{0.048611in}{0.000000in}}{%
\pgfpathmoveto{\pgfqpoint{0.000000in}{0.000000in}}%
\pgfpathlineto{\pgfqpoint{0.048611in}{0.000000in}}%
\pgfusepath{stroke,fill}%
}%
\begin{pgfscope}%
\pgfsys@transformshift{0.691161in}{5.560277in}%
\pgfsys@useobject{currentmarker}{}%
\end{pgfscope}%
\end{pgfscope}%
\begin{pgfscope}%
\definecolor{textcolor}{rgb}{0.000000,0.000000,0.000000}%
\pgfsetstrokecolor{textcolor}%
\pgfsetfillcolor{textcolor}%
\pgftext[x=0.465080in, y=5.512082in, left, base]{\color{textcolor}\rmfamily\fontsize{10.000000}{12.000000}\selectfont \(\displaystyle {2.5}\)}%
\end{pgfscope}%
\begin{pgfscope}%
\definecolor{textcolor}{rgb}{0.000000,0.000000,0.000000}%
\pgfsetstrokecolor{textcolor}%
\pgfsetfillcolor{textcolor}%
\pgftext[x=0.301500in,y=5.252545in,,bottom,rotate=90.000000]{\color{textcolor}\rmfamily\fontsize{12.000000}{14.400000}\selectfont Signal}%
\end{pgfscope}%
\begin{pgfscope}%
\definecolor{textcolor}{rgb}{0.000000,0.000000,0.000000}%
\pgfsetstrokecolor{textcolor}%
\pgfsetfillcolor{textcolor}%
\pgftext[x=0.691161in,y=5.745370in,left,base]{\color{textcolor}\rmfamily\fontsize{10.000000}{12.000000}\selectfont \(\displaystyle \times{10^{\ensuremath{-}6}}{}\)}%
\end{pgfscope}%
\begin{pgfscope}%
\pgfpathrectangle{\pgfqpoint{0.691161in}{4.801386in}}{\pgfqpoint{5.054672in}{0.902317in}}%
\pgfusepath{clip}%
\pgfsetrectcap%
\pgfsetroundjoin%
\pgfsetlinewidth{1.505625pt}%
\definecolor{currentstroke}{rgb}{0.498039,0.498039,0.498039}%
\pgfsetstrokecolor{currentstroke}%
\pgfsetdash{}{0pt}%
\pgfpathmoveto{\pgfqpoint{0.691161in}{5.225316in}}%
\pgfpathlineto{\pgfqpoint{0.692509in}{5.225598in}}%
\pgfpathlineto{\pgfqpoint{0.692678in}{5.224946in}}%
\pgfpathlineto{\pgfqpoint{0.693436in}{5.218541in}}%
\pgfpathlineto{\pgfqpoint{0.693773in}{5.223819in}}%
\pgfpathlineto{\pgfqpoint{0.694363in}{5.236465in}}%
\pgfpathlineto{\pgfqpoint{0.694868in}{5.227075in}}%
\pgfpathlineto{\pgfqpoint{0.695542in}{5.218135in}}%
\pgfpathlineto{\pgfqpoint{0.696047in}{5.224028in}}%
\pgfpathlineto{\pgfqpoint{0.696721in}{5.232525in}}%
\pgfpathlineto{\pgfqpoint{0.697227in}{5.226107in}}%
\pgfpathlineto{\pgfqpoint{0.698069in}{5.219970in}}%
\pgfpathlineto{\pgfqpoint{0.698491in}{5.223194in}}%
\pgfpathlineto{\pgfqpoint{0.698996in}{5.229493in}}%
\pgfpathlineto{\pgfqpoint{0.699586in}{5.223826in}}%
\pgfpathlineto{\pgfqpoint{0.699923in}{5.221220in}}%
\pgfpathlineto{\pgfqpoint{0.700428in}{5.226809in}}%
\pgfpathlineto{\pgfqpoint{0.700597in}{5.227679in}}%
\pgfpathlineto{\pgfqpoint{0.701018in}{5.224493in}}%
\pgfpathlineto{\pgfqpoint{0.701523in}{5.219517in}}%
\pgfpathlineto{\pgfqpoint{0.702029in}{5.225578in}}%
\pgfpathlineto{\pgfqpoint{0.702703in}{5.235407in}}%
\pgfpathlineto{\pgfqpoint{0.703124in}{5.228578in}}%
\pgfpathlineto{\pgfqpoint{0.703798in}{5.218297in}}%
\pgfpathlineto{\pgfqpoint{0.704388in}{5.224156in}}%
\pgfpathlineto{\pgfqpoint{0.704977in}{5.231008in}}%
\pgfpathlineto{\pgfqpoint{0.705399in}{5.224370in}}%
\pgfpathlineto{\pgfqpoint{0.705904in}{5.217206in}}%
\pgfpathlineto{\pgfqpoint{0.706410in}{5.226566in}}%
\pgfpathlineto{\pgfqpoint{0.706999in}{5.234860in}}%
\pgfpathlineto{\pgfqpoint{0.707420in}{5.227289in}}%
\pgfpathlineto{\pgfqpoint{0.708010in}{5.214311in}}%
\pgfpathlineto{\pgfqpoint{0.708600in}{5.225039in}}%
\pgfpathlineto{\pgfqpoint{0.709190in}{5.235815in}}%
\pgfpathlineto{\pgfqpoint{0.709695in}{5.228360in}}%
\pgfpathlineto{\pgfqpoint{0.710874in}{5.215875in}}%
\pgfpathlineto{\pgfqpoint{0.711127in}{5.218956in}}%
\pgfpathlineto{\pgfqpoint{0.711717in}{5.231612in}}%
\pgfpathlineto{\pgfqpoint{0.712475in}{5.225991in}}%
\pgfpathlineto{\pgfqpoint{0.712896in}{5.228173in}}%
\pgfpathlineto{\pgfqpoint{0.713065in}{5.228791in}}%
\pgfpathlineto{\pgfqpoint{0.713486in}{5.225619in}}%
\pgfpathlineto{\pgfqpoint{0.713907in}{5.220254in}}%
\pgfpathlineto{\pgfqpoint{0.714328in}{5.226637in}}%
\pgfpathlineto{\pgfqpoint{0.714750in}{5.233995in}}%
\pgfpathlineto{\pgfqpoint{0.715255in}{5.222304in}}%
\pgfpathlineto{\pgfqpoint{0.715592in}{5.216676in}}%
\pgfpathlineto{\pgfqpoint{0.716098in}{5.230356in}}%
\pgfpathlineto{\pgfqpoint{0.716350in}{5.233957in}}%
\pgfpathlineto{\pgfqpoint{0.716772in}{5.222427in}}%
\pgfpathlineto{\pgfqpoint{0.717361in}{5.208287in}}%
\pgfpathlineto{\pgfqpoint{0.717867in}{5.217853in}}%
\pgfpathlineto{\pgfqpoint{0.719130in}{5.237391in}}%
\pgfpathlineto{\pgfqpoint{0.719552in}{5.235624in}}%
\pgfpathlineto{\pgfqpoint{0.720563in}{5.224087in}}%
\pgfpathlineto{\pgfqpoint{0.721489in}{5.225901in}}%
\pgfpathlineto{\pgfqpoint{0.722079in}{5.229209in}}%
\pgfpathlineto{\pgfqpoint{0.722163in}{5.229553in}}%
\pgfpathlineto{\pgfqpoint{0.722416in}{5.227587in}}%
\pgfpathlineto{\pgfqpoint{0.723343in}{5.205702in}}%
\pgfpathlineto{\pgfqpoint{0.723764in}{5.218478in}}%
\pgfpathlineto{\pgfqpoint{0.724438in}{5.239336in}}%
\pgfpathlineto{\pgfqpoint{0.724943in}{5.225961in}}%
\pgfpathlineto{\pgfqpoint{0.725449in}{5.211646in}}%
\pgfpathlineto{\pgfqpoint{0.725954in}{5.225696in}}%
\pgfpathlineto{\pgfqpoint{0.726460in}{5.238245in}}%
\pgfpathlineto{\pgfqpoint{0.727049in}{5.227048in}}%
\pgfpathlineto{\pgfqpoint{0.727555in}{5.221382in}}%
\pgfpathlineto{\pgfqpoint{0.727976in}{5.227372in}}%
\pgfpathlineto{\pgfqpoint{0.728482in}{5.234178in}}%
\pgfpathlineto{\pgfqpoint{0.728987in}{5.225972in}}%
\pgfpathlineto{\pgfqpoint{0.729408in}{5.220667in}}%
\pgfpathlineto{\pgfqpoint{0.730082in}{5.225526in}}%
\pgfpathlineto{\pgfqpoint{0.730251in}{5.226234in}}%
\pgfpathlineto{\pgfqpoint{0.730588in}{5.222879in}}%
\pgfpathlineto{\pgfqpoint{0.730925in}{5.218800in}}%
\pgfpathlineto{\pgfqpoint{0.731599in}{5.224531in}}%
\pgfpathlineto{\pgfqpoint{0.731683in}{5.224616in}}%
\pgfpathlineto{\pgfqpoint{0.732020in}{5.223539in}}%
\pgfpathlineto{\pgfqpoint{0.732357in}{5.222802in}}%
\pgfpathlineto{\pgfqpoint{0.733031in}{5.223534in}}%
\pgfpathlineto{\pgfqpoint{0.733452in}{5.229379in}}%
\pgfpathlineto{\pgfqpoint{0.733873in}{5.235992in}}%
\pgfpathlineto{\pgfqpoint{0.734294in}{5.227260in}}%
\pgfpathlineto{\pgfqpoint{0.734884in}{5.214406in}}%
\pgfpathlineto{\pgfqpoint{0.735390in}{5.227026in}}%
\pgfpathlineto{\pgfqpoint{0.735811in}{5.235925in}}%
\pgfpathlineto{\pgfqpoint{0.736401in}{5.223982in}}%
\pgfpathlineto{\pgfqpoint{0.736569in}{5.222082in}}%
\pgfpathlineto{\pgfqpoint{0.737075in}{5.229865in}}%
\pgfpathlineto{\pgfqpoint{0.737327in}{5.232160in}}%
\pgfpathlineto{\pgfqpoint{0.737748in}{5.225297in}}%
\pgfpathlineto{\pgfqpoint{0.738507in}{5.212678in}}%
\pgfpathlineto{\pgfqpoint{0.739012in}{5.220453in}}%
\pgfpathlineto{\pgfqpoint{0.739686in}{5.237887in}}%
\pgfpathlineto{\pgfqpoint{0.740276in}{5.225431in}}%
\pgfpathlineto{\pgfqpoint{0.740866in}{5.213178in}}%
\pgfpathlineto{\pgfqpoint{0.741371in}{5.223877in}}%
\pgfpathlineto{\pgfqpoint{0.741961in}{5.239348in}}%
\pgfpathlineto{\pgfqpoint{0.742466in}{5.229055in}}%
\pgfpathlineto{\pgfqpoint{0.743056in}{5.214638in}}%
\pgfpathlineto{\pgfqpoint{0.743561in}{5.225883in}}%
\pgfpathlineto{\pgfqpoint{0.743898in}{5.231097in}}%
\pgfpathlineto{\pgfqpoint{0.744572in}{5.224251in}}%
\pgfpathlineto{\pgfqpoint{0.744657in}{5.224023in}}%
\pgfpathlineto{\pgfqpoint{0.744994in}{5.226283in}}%
\pgfpathlineto{\pgfqpoint{0.745246in}{5.227717in}}%
\pgfpathlineto{\pgfqpoint{0.745667in}{5.223419in}}%
\pgfpathlineto{\pgfqpoint{0.746173in}{5.216853in}}%
\pgfpathlineto{\pgfqpoint{0.746678in}{5.225155in}}%
\pgfpathlineto{\pgfqpoint{0.747268in}{5.233585in}}%
\pgfpathlineto{\pgfqpoint{0.747774in}{5.227388in}}%
\pgfpathlineto{\pgfqpoint{0.748195in}{5.223192in}}%
\pgfpathlineto{\pgfqpoint{0.749121in}{5.223428in}}%
\pgfpathlineto{\pgfqpoint{0.749627in}{5.219574in}}%
\pgfpathlineto{\pgfqpoint{0.750048in}{5.224198in}}%
\pgfpathlineto{\pgfqpoint{0.750554in}{5.230832in}}%
\pgfpathlineto{\pgfqpoint{0.751143in}{5.226032in}}%
\pgfpathlineto{\pgfqpoint{0.752407in}{5.220960in}}%
\pgfpathlineto{\pgfqpoint{0.752576in}{5.221656in}}%
\pgfpathlineto{\pgfqpoint{0.753586in}{5.237593in}}%
\pgfpathlineto{\pgfqpoint{0.754176in}{5.225832in}}%
\pgfpathlineto{\pgfqpoint{0.754682in}{5.219771in}}%
\pgfpathlineto{\pgfqpoint{0.755440in}{5.220653in}}%
\pgfpathlineto{\pgfqpoint{0.755777in}{5.220057in}}%
\pgfpathlineto{\pgfqpoint{0.756114in}{5.221634in}}%
\pgfpathlineto{\pgfqpoint{0.757209in}{5.228786in}}%
\pgfpathlineto{\pgfqpoint{0.757630in}{5.225480in}}%
\pgfpathlineto{\pgfqpoint{0.758051in}{5.221919in}}%
\pgfpathlineto{\pgfqpoint{0.758641in}{5.226256in}}%
\pgfpathlineto{\pgfqpoint{0.758978in}{5.225781in}}%
\pgfpathlineto{\pgfqpoint{0.759399in}{5.228334in}}%
\pgfpathlineto{\pgfqpoint{0.759652in}{5.229693in}}%
\pgfpathlineto{\pgfqpoint{0.760158in}{5.225568in}}%
\pgfpathlineto{\pgfqpoint{0.760326in}{5.224654in}}%
\pgfpathlineto{\pgfqpoint{0.760663in}{5.228303in}}%
\pgfpathlineto{\pgfqpoint{0.761084in}{5.236086in}}%
\pgfpathlineto{\pgfqpoint{0.761505in}{5.227170in}}%
\pgfpathlineto{\pgfqpoint{0.762011in}{5.211618in}}%
\pgfpathlineto{\pgfqpoint{0.762601in}{5.224412in}}%
\pgfpathlineto{\pgfqpoint{0.763022in}{5.232587in}}%
\pgfpathlineto{\pgfqpoint{0.763612in}{5.223334in}}%
\pgfpathlineto{\pgfqpoint{0.764117in}{5.214198in}}%
\pgfpathlineto{\pgfqpoint{0.764622in}{5.222876in}}%
\pgfpathlineto{\pgfqpoint{0.765128in}{5.232220in}}%
\pgfpathlineto{\pgfqpoint{0.765718in}{5.223007in}}%
\pgfpathlineto{\pgfqpoint{0.765886in}{5.221569in}}%
\pgfpathlineto{\pgfqpoint{0.766307in}{5.226430in}}%
\pgfpathlineto{\pgfqpoint{0.766897in}{5.233478in}}%
\pgfpathlineto{\pgfqpoint{0.767487in}{5.227879in}}%
\pgfpathlineto{\pgfqpoint{0.768329in}{5.217687in}}%
\pgfpathlineto{\pgfqpoint{0.768750in}{5.223264in}}%
\pgfpathlineto{\pgfqpoint{0.769424in}{5.232118in}}%
\pgfpathlineto{\pgfqpoint{0.770014in}{5.228381in}}%
\pgfpathlineto{\pgfqpoint{0.771025in}{5.220651in}}%
\pgfpathlineto{\pgfqpoint{0.771531in}{5.217679in}}%
\pgfpathlineto{\pgfqpoint{0.771952in}{5.222209in}}%
\pgfpathlineto{\pgfqpoint{0.772626in}{5.232252in}}%
\pgfpathlineto{\pgfqpoint{0.773131in}{5.224964in}}%
\pgfpathlineto{\pgfqpoint{0.773552in}{5.219940in}}%
\pgfpathlineto{\pgfqpoint{0.774311in}{5.222906in}}%
\pgfpathlineto{\pgfqpoint{0.776754in}{5.231270in}}%
\pgfpathlineto{\pgfqpoint{0.777175in}{5.228262in}}%
\pgfpathlineto{\pgfqpoint{0.777680in}{5.222499in}}%
\pgfpathlineto{\pgfqpoint{0.778186in}{5.228645in}}%
\pgfpathlineto{\pgfqpoint{0.778439in}{5.231416in}}%
\pgfpathlineto{\pgfqpoint{0.779028in}{5.226159in}}%
\pgfpathlineto{\pgfqpoint{0.780208in}{5.219403in}}%
\pgfpathlineto{\pgfqpoint{0.780460in}{5.221067in}}%
\pgfpathlineto{\pgfqpoint{0.781050in}{5.226755in}}%
\pgfpathlineto{\pgfqpoint{0.781640in}{5.222333in}}%
\pgfpathlineto{\pgfqpoint{0.781808in}{5.221734in}}%
\pgfpathlineto{\pgfqpoint{0.782145in}{5.224247in}}%
\pgfpathlineto{\pgfqpoint{0.782735in}{5.231823in}}%
\pgfpathlineto{\pgfqpoint{0.783241in}{5.225151in}}%
\pgfpathlineto{\pgfqpoint{0.783830in}{5.219043in}}%
\pgfpathlineto{\pgfqpoint{0.784336in}{5.224677in}}%
\pgfpathlineto{\pgfqpoint{0.784925in}{5.231502in}}%
\pgfpathlineto{\pgfqpoint{0.785431in}{5.226231in}}%
\pgfpathlineto{\pgfqpoint{0.785852in}{5.223418in}}%
\pgfpathlineto{\pgfqpoint{0.786442in}{5.226653in}}%
\pgfpathlineto{\pgfqpoint{0.786610in}{5.226939in}}%
\pgfpathlineto{\pgfqpoint{0.786947in}{5.225365in}}%
\pgfpathlineto{\pgfqpoint{0.787369in}{5.222188in}}%
\pgfpathlineto{\pgfqpoint{0.787705in}{5.226288in}}%
\pgfpathlineto{\pgfqpoint{0.788211in}{5.234559in}}%
\pgfpathlineto{\pgfqpoint{0.788632in}{5.225637in}}%
\pgfpathlineto{\pgfqpoint{0.789138in}{5.217847in}}%
\pgfpathlineto{\pgfqpoint{0.789727in}{5.224678in}}%
\pgfpathlineto{\pgfqpoint{0.790233in}{5.228565in}}%
\pgfpathlineto{\pgfqpoint{0.790654in}{5.222558in}}%
\pgfpathlineto{\pgfqpoint{0.791160in}{5.213363in}}%
\pgfpathlineto{\pgfqpoint{0.791581in}{5.223206in}}%
\pgfpathlineto{\pgfqpoint{0.792255in}{5.241422in}}%
\pgfpathlineto{\pgfqpoint{0.792760in}{5.228161in}}%
\pgfpathlineto{\pgfqpoint{0.793350in}{5.210145in}}%
\pgfpathlineto{\pgfqpoint{0.793855in}{5.225002in}}%
\pgfpathlineto{\pgfqpoint{0.794277in}{5.236513in}}%
\pgfpathlineto{\pgfqpoint{0.794866in}{5.222800in}}%
\pgfpathlineto{\pgfqpoint{0.795203in}{5.217863in}}%
\pgfpathlineto{\pgfqpoint{0.795709in}{5.227339in}}%
\pgfpathlineto{\pgfqpoint{0.796046in}{5.233909in}}%
\pgfpathlineto{\pgfqpoint{0.796551in}{5.224370in}}%
\pgfpathlineto{\pgfqpoint{0.796972in}{5.216355in}}%
\pgfpathlineto{\pgfqpoint{0.797562in}{5.227225in}}%
\pgfpathlineto{\pgfqpoint{0.797983in}{5.233206in}}%
\pgfpathlineto{\pgfqpoint{0.798489in}{5.225778in}}%
\pgfpathlineto{\pgfqpoint{0.798994in}{5.216302in}}%
\pgfpathlineto{\pgfqpoint{0.799500in}{5.225085in}}%
\pgfpathlineto{\pgfqpoint{0.800005in}{5.234842in}}%
\pgfpathlineto{\pgfqpoint{0.800511in}{5.224810in}}%
\pgfpathlineto{\pgfqpoint{0.800932in}{5.219150in}}%
\pgfpathlineto{\pgfqpoint{0.801437in}{5.226165in}}%
\pgfpathlineto{\pgfqpoint{0.801859in}{5.231351in}}%
\pgfpathlineto{\pgfqpoint{0.802364in}{5.223533in}}%
\pgfpathlineto{\pgfqpoint{0.802785in}{5.217959in}}%
\pgfpathlineto{\pgfqpoint{0.803291in}{5.226393in}}%
\pgfpathlineto{\pgfqpoint{0.803628in}{5.231944in}}%
\pgfpathlineto{\pgfqpoint{0.804133in}{5.221197in}}%
\pgfpathlineto{\pgfqpoint{0.804470in}{5.215684in}}%
\pgfpathlineto{\pgfqpoint{0.804976in}{5.226238in}}%
\pgfpathlineto{\pgfqpoint{0.805397in}{5.233058in}}%
\pgfpathlineto{\pgfqpoint{0.805987in}{5.226685in}}%
\pgfpathlineto{\pgfqpoint{0.806660in}{5.219610in}}%
\pgfpathlineto{\pgfqpoint{0.807082in}{5.224946in}}%
\pgfpathlineto{\pgfqpoint{0.807671in}{5.234654in}}%
\pgfpathlineto{\pgfqpoint{0.808177in}{5.225428in}}%
\pgfpathlineto{\pgfqpoint{0.808682in}{5.211984in}}%
\pgfpathlineto{\pgfqpoint{0.809104in}{5.227164in}}%
\pgfpathlineto{\pgfqpoint{0.809525in}{5.242883in}}%
\pgfpathlineto{\pgfqpoint{0.810115in}{5.223393in}}%
\pgfpathlineto{\pgfqpoint{0.810620in}{5.209479in}}%
\pgfpathlineto{\pgfqpoint{0.811125in}{5.223448in}}%
\pgfpathlineto{\pgfqpoint{0.811631in}{5.235439in}}%
\pgfpathlineto{\pgfqpoint{0.812136in}{5.220944in}}%
\pgfpathlineto{\pgfqpoint{0.812389in}{5.216264in}}%
\pgfpathlineto{\pgfqpoint{0.812979in}{5.225329in}}%
\pgfpathlineto{\pgfqpoint{0.813316in}{5.228589in}}%
\pgfpathlineto{\pgfqpoint{0.814158in}{5.227275in}}%
\pgfpathlineto{\pgfqpoint{0.814579in}{5.229513in}}%
\pgfpathlineto{\pgfqpoint{0.814916in}{5.225978in}}%
\pgfpathlineto{\pgfqpoint{0.815422in}{5.216501in}}%
\pgfpathlineto{\pgfqpoint{0.815927in}{5.226148in}}%
\pgfpathlineto{\pgfqpoint{0.817360in}{5.235059in}}%
\pgfpathlineto{\pgfqpoint{0.817528in}{5.235784in}}%
\pgfpathlineto{\pgfqpoint{0.817781in}{5.232118in}}%
\pgfpathlineto{\pgfqpoint{0.818539in}{5.209234in}}%
\pgfpathlineto{\pgfqpoint{0.819129in}{5.224476in}}%
\pgfpathlineto{\pgfqpoint{0.819550in}{5.230605in}}%
\pgfpathlineto{\pgfqpoint{0.820140in}{5.224538in}}%
\pgfpathlineto{\pgfqpoint{0.820561in}{5.222630in}}%
\pgfpathlineto{\pgfqpoint{0.821235in}{5.224226in}}%
\pgfpathlineto{\pgfqpoint{0.821488in}{5.223806in}}%
\pgfpathlineto{\pgfqpoint{0.821740in}{5.225205in}}%
\pgfpathlineto{\pgfqpoint{0.822414in}{5.236159in}}%
\pgfpathlineto{\pgfqpoint{0.822835in}{5.227996in}}%
\pgfpathlineto{\pgfqpoint{0.823341in}{5.216766in}}%
\pgfpathlineto{\pgfqpoint{0.823931in}{5.227287in}}%
\pgfpathlineto{\pgfqpoint{0.824268in}{5.230108in}}%
\pgfpathlineto{\pgfqpoint{0.824773in}{5.224010in}}%
\pgfpathlineto{\pgfqpoint{0.826121in}{5.218005in}}%
\pgfpathlineto{\pgfqpoint{0.825447in}{5.224340in}}%
\pgfpathlineto{\pgfqpoint{0.826205in}{5.218547in}}%
\pgfpathlineto{\pgfqpoint{0.827048in}{5.237122in}}%
\pgfpathlineto{\pgfqpoint{0.827637in}{5.225471in}}%
\pgfpathlineto{\pgfqpoint{0.828059in}{5.217865in}}%
\pgfpathlineto{\pgfqpoint{0.828648in}{5.228042in}}%
\pgfpathlineto{\pgfqpoint{0.828817in}{5.229296in}}%
\pgfpathlineto{\pgfqpoint{0.829322in}{5.223427in}}%
\pgfpathlineto{\pgfqpoint{0.829744in}{5.219376in}}%
\pgfpathlineto{\pgfqpoint{0.830165in}{5.224776in}}%
\pgfpathlineto{\pgfqpoint{0.830670in}{5.231416in}}%
\pgfpathlineto{\pgfqpoint{0.831091in}{5.224445in}}%
\pgfpathlineto{\pgfqpoint{0.831597in}{5.216855in}}%
\pgfpathlineto{\pgfqpoint{0.832102in}{5.226074in}}%
\pgfpathlineto{\pgfqpoint{0.832692in}{5.237020in}}%
\pgfpathlineto{\pgfqpoint{0.833198in}{5.226332in}}%
\pgfpathlineto{\pgfqpoint{0.833787in}{5.212001in}}%
\pgfpathlineto{\pgfqpoint{0.834293in}{5.222291in}}%
\pgfpathlineto{\pgfqpoint{0.834967in}{5.242316in}}%
\pgfpathlineto{\pgfqpoint{0.835388in}{5.228404in}}%
\pgfpathlineto{\pgfqpoint{0.835978in}{5.205529in}}%
\pgfpathlineto{\pgfqpoint{0.836483in}{5.222577in}}%
\pgfpathlineto{\pgfqpoint{0.837073in}{5.240995in}}%
\pgfpathlineto{\pgfqpoint{0.837578in}{5.227711in}}%
\pgfpathlineto{\pgfqpoint{0.837999in}{5.217539in}}%
\pgfpathlineto{\pgfqpoint{0.838589in}{5.229241in}}%
\pgfpathlineto{\pgfqpoint{0.838842in}{5.231730in}}%
\pgfpathlineto{\pgfqpoint{0.839347in}{5.223792in}}%
\pgfpathlineto{\pgfqpoint{0.839684in}{5.219825in}}%
\pgfpathlineto{\pgfqpoint{0.840190in}{5.225574in}}%
\pgfpathlineto{\pgfqpoint{0.840527in}{5.230397in}}%
\pgfpathlineto{\pgfqpoint{0.841032in}{5.220470in}}%
\pgfpathlineto{\pgfqpoint{0.841453in}{5.214009in}}%
\pgfpathlineto{\pgfqpoint{0.841959in}{5.224730in}}%
\pgfpathlineto{\pgfqpoint{0.842464in}{5.238009in}}%
\pgfpathlineto{\pgfqpoint{0.842970in}{5.225323in}}%
\pgfpathlineto{\pgfqpoint{0.843475in}{5.212658in}}%
\pgfpathlineto{\pgfqpoint{0.843981in}{5.223994in}}%
\pgfpathlineto{\pgfqpoint{0.844571in}{5.237775in}}%
\pgfpathlineto{\pgfqpoint{0.845076in}{5.225793in}}%
\pgfpathlineto{\pgfqpoint{0.845413in}{5.219653in}}%
\pgfpathlineto{\pgfqpoint{0.845918in}{5.230918in}}%
\pgfpathlineto{\pgfqpoint{0.846340in}{5.238756in}}%
\pgfpathlineto{\pgfqpoint{0.846761in}{5.227425in}}%
\pgfpathlineto{\pgfqpoint{0.847435in}{5.207933in}}%
\pgfpathlineto{\pgfqpoint{0.847940in}{5.222441in}}%
\pgfpathlineto{\pgfqpoint{0.848446in}{5.236281in}}%
\pgfpathlineto{\pgfqpoint{0.848951in}{5.218378in}}%
\pgfpathlineto{\pgfqpoint{0.849288in}{5.208310in}}%
\pgfpathlineto{\pgfqpoint{0.849794in}{5.228187in}}%
\pgfpathlineto{\pgfqpoint{0.850215in}{5.243551in}}%
\pgfpathlineto{\pgfqpoint{0.850805in}{5.226227in}}%
\pgfpathlineto{\pgfqpoint{0.851226in}{5.218218in}}%
\pgfpathlineto{\pgfqpoint{0.851816in}{5.229050in}}%
\pgfpathlineto{\pgfqpoint{0.852153in}{5.232750in}}%
\pgfpathlineto{\pgfqpoint{0.852658in}{5.224316in}}%
\pgfpathlineto{\pgfqpoint{0.852995in}{5.219714in}}%
\pgfpathlineto{\pgfqpoint{0.853669in}{5.224071in}}%
\pgfpathlineto{\pgfqpoint{0.854174in}{5.228528in}}%
\pgfpathlineto{\pgfqpoint{0.854680in}{5.223381in}}%
\pgfpathlineto{\pgfqpoint{0.855185in}{5.216776in}}%
\pgfpathlineto{\pgfqpoint{0.855691in}{5.222919in}}%
\pgfpathlineto{\pgfqpoint{0.856618in}{5.229224in}}%
\pgfpathlineto{\pgfqpoint{0.857039in}{5.227122in}}%
\pgfpathlineto{\pgfqpoint{0.857291in}{5.225221in}}%
\pgfpathlineto{\pgfqpoint{0.857797in}{5.229338in}}%
\pgfpathlineto{\pgfqpoint{0.858050in}{5.230844in}}%
\pgfpathlineto{\pgfqpoint{0.858555in}{5.225981in}}%
\pgfpathlineto{\pgfqpoint{0.858892in}{5.223578in}}%
\pgfpathlineto{\pgfqpoint{0.859313in}{5.229308in}}%
\pgfpathlineto{\pgfqpoint{0.859566in}{5.231833in}}%
\pgfpathlineto{\pgfqpoint{0.859987in}{5.222291in}}%
\pgfpathlineto{\pgfqpoint{0.860493in}{5.213034in}}%
\pgfpathlineto{\pgfqpoint{0.860914in}{5.222876in}}%
\pgfpathlineto{\pgfqpoint{0.861419in}{5.235545in}}%
\pgfpathlineto{\pgfqpoint{0.861925in}{5.224495in}}%
\pgfpathlineto{\pgfqpoint{0.862430in}{5.211096in}}%
\pgfpathlineto{\pgfqpoint{0.862936in}{5.226047in}}%
\pgfpathlineto{\pgfqpoint{0.863357in}{5.234657in}}%
\pgfpathlineto{\pgfqpoint{0.863947in}{5.226095in}}%
\pgfpathlineto{\pgfqpoint{0.864284in}{5.223086in}}%
\pgfpathlineto{\pgfqpoint{0.864873in}{5.228428in}}%
\pgfpathlineto{\pgfqpoint{0.865042in}{5.227983in}}%
\pgfpathlineto{\pgfqpoint{0.865716in}{5.221500in}}%
\pgfpathlineto{\pgfqpoint{0.866306in}{5.225625in}}%
\pgfpathlineto{\pgfqpoint{0.867906in}{5.231727in}}%
\pgfpathlineto{\pgfqpoint{0.866980in}{5.224213in}}%
\pgfpathlineto{\pgfqpoint{0.868159in}{5.228623in}}%
\pgfpathlineto{\pgfqpoint{0.868664in}{5.218692in}}%
\pgfpathlineto{\pgfqpoint{0.869086in}{5.228141in}}%
\pgfpathlineto{\pgfqpoint{0.869591in}{5.240686in}}%
\pgfpathlineto{\pgfqpoint{0.870012in}{5.222931in}}%
\pgfpathlineto{\pgfqpoint{0.870518in}{5.199394in}}%
\pgfpathlineto{\pgfqpoint{0.871023in}{5.224810in}}%
\pgfpathlineto{\pgfqpoint{0.871529in}{5.246443in}}%
\pgfpathlineto{\pgfqpoint{0.872034in}{5.224642in}}%
\pgfpathlineto{\pgfqpoint{0.872455in}{5.209053in}}%
\pgfpathlineto{\pgfqpoint{0.873045in}{5.227269in}}%
\pgfpathlineto{\pgfqpoint{0.873466in}{5.236108in}}%
\pgfpathlineto{\pgfqpoint{0.874056in}{5.227578in}}%
\pgfpathlineto{\pgfqpoint{0.874899in}{5.217596in}}%
\pgfpathlineto{\pgfqpoint{0.875320in}{5.222681in}}%
\pgfpathlineto{\pgfqpoint{0.876246in}{5.233037in}}%
\pgfpathlineto{\pgfqpoint{0.876668in}{5.228688in}}%
\pgfpathlineto{\pgfqpoint{0.877426in}{5.217899in}}%
\pgfpathlineto{\pgfqpoint{0.877931in}{5.223297in}}%
\pgfpathlineto{\pgfqpoint{0.879279in}{5.231756in}}%
\pgfpathlineto{\pgfqpoint{0.879448in}{5.230940in}}%
\pgfpathlineto{\pgfqpoint{0.880290in}{5.211834in}}%
\pgfpathlineto{\pgfqpoint{0.880796in}{5.224501in}}%
\pgfpathlineto{\pgfqpoint{0.881217in}{5.235784in}}%
\pgfpathlineto{\pgfqpoint{0.881722in}{5.221654in}}%
\pgfpathlineto{\pgfqpoint{0.882059in}{5.213178in}}%
\pgfpathlineto{\pgfqpoint{0.882565in}{5.227629in}}%
\pgfpathlineto{\pgfqpoint{0.882986in}{5.237546in}}%
\pgfpathlineto{\pgfqpoint{0.883492in}{5.225344in}}%
\pgfpathlineto{\pgfqpoint{0.883828in}{5.218867in}}%
\pgfpathlineto{\pgfqpoint{0.884334in}{5.230756in}}%
\pgfpathlineto{\pgfqpoint{0.884587in}{5.234039in}}%
\pgfpathlineto{\pgfqpoint{0.885092in}{5.223483in}}%
\pgfpathlineto{\pgfqpoint{0.885598in}{5.215134in}}%
\pgfpathlineto{\pgfqpoint{0.886187in}{5.223861in}}%
\pgfpathlineto{\pgfqpoint{0.887030in}{5.235413in}}%
\pgfpathlineto{\pgfqpoint{0.887451in}{5.228780in}}%
\pgfpathlineto{\pgfqpoint{0.888125in}{5.206708in}}%
\pgfpathlineto{\pgfqpoint{0.888630in}{5.222694in}}%
\pgfpathlineto{\pgfqpoint{0.889220in}{5.238742in}}%
\pgfpathlineto{\pgfqpoint{0.889726in}{5.227636in}}%
\pgfpathlineto{\pgfqpoint{0.890063in}{5.221875in}}%
\pgfpathlineto{\pgfqpoint{0.890652in}{5.229889in}}%
\pgfpathlineto{\pgfqpoint{0.890905in}{5.232664in}}%
\pgfpathlineto{\pgfqpoint{0.891495in}{5.226643in}}%
\pgfpathlineto{\pgfqpoint{0.892000in}{5.221872in}}%
\pgfpathlineto{\pgfqpoint{0.892590in}{5.225495in}}%
\pgfpathlineto{\pgfqpoint{0.892843in}{5.226165in}}%
\pgfpathlineto{\pgfqpoint{0.893264in}{5.224343in}}%
\pgfpathlineto{\pgfqpoint{0.894359in}{5.215516in}}%
\pgfpathlineto{\pgfqpoint{0.894780in}{5.219010in}}%
\pgfpathlineto{\pgfqpoint{0.895623in}{5.233941in}}%
\pgfpathlineto{\pgfqpoint{0.896212in}{5.225313in}}%
\pgfpathlineto{\pgfqpoint{0.896634in}{5.218475in}}%
\pgfpathlineto{\pgfqpoint{0.897139in}{5.229219in}}%
\pgfpathlineto{\pgfqpoint{0.897560in}{5.237692in}}%
\pgfpathlineto{\pgfqpoint{0.898066in}{5.227985in}}%
\pgfpathlineto{\pgfqpoint{0.898487in}{5.220355in}}%
\pgfpathlineto{\pgfqpoint{0.899077in}{5.227886in}}%
\pgfpathlineto{\pgfqpoint{0.899245in}{5.229205in}}%
\pgfpathlineto{\pgfqpoint{0.899751in}{5.224175in}}%
\pgfpathlineto{\pgfqpoint{0.900425in}{5.224418in}}%
\pgfpathlineto{\pgfqpoint{0.901183in}{5.220785in}}%
\pgfpathlineto{\pgfqpoint{0.901604in}{5.223976in}}%
\pgfpathlineto{\pgfqpoint{0.902447in}{5.234762in}}%
\pgfpathlineto{\pgfqpoint{0.902868in}{5.227979in}}%
\pgfpathlineto{\pgfqpoint{0.903626in}{5.208578in}}%
\pgfpathlineto{\pgfqpoint{0.904131in}{5.219205in}}%
\pgfpathlineto{\pgfqpoint{0.904890in}{5.237374in}}%
\pgfpathlineto{\pgfqpoint{0.905395in}{5.230354in}}%
\pgfpathlineto{\pgfqpoint{0.906153in}{5.215436in}}%
\pgfpathlineto{\pgfqpoint{0.906659in}{5.224555in}}%
\pgfpathlineto{\pgfqpoint{0.908259in}{5.234828in}}%
\pgfpathlineto{\pgfqpoint{0.908428in}{5.235707in}}%
\pgfpathlineto{\pgfqpoint{0.908765in}{5.230910in}}%
\pgfpathlineto{\pgfqpoint{0.909439in}{5.214274in}}%
\pgfpathlineto{\pgfqpoint{0.909944in}{5.224306in}}%
\pgfpathlineto{\pgfqpoint{0.910618in}{5.239189in}}%
\pgfpathlineto{\pgfqpoint{0.911039in}{5.229613in}}%
\pgfpathlineto{\pgfqpoint{0.911713in}{5.206491in}}%
\pgfpathlineto{\pgfqpoint{0.912219in}{5.219810in}}%
\pgfpathlineto{\pgfqpoint{0.912724in}{5.235269in}}%
\pgfpathlineto{\pgfqpoint{0.913398in}{5.223391in}}%
\pgfpathlineto{\pgfqpoint{0.913567in}{5.222297in}}%
\pgfpathlineto{\pgfqpoint{0.914157in}{5.226031in}}%
\pgfpathlineto{\pgfqpoint{0.914241in}{5.225960in}}%
\pgfpathlineto{\pgfqpoint{0.914746in}{5.222350in}}%
\pgfpathlineto{\pgfqpoint{0.914999in}{5.220753in}}%
\pgfpathlineto{\pgfqpoint{0.915420in}{5.225380in}}%
\pgfpathlineto{\pgfqpoint{0.916178in}{5.235490in}}%
\pgfpathlineto{\pgfqpoint{0.916600in}{5.228787in}}%
\pgfpathlineto{\pgfqpoint{0.917358in}{5.208983in}}%
\pgfpathlineto{\pgfqpoint{0.917779in}{5.223047in}}%
\pgfpathlineto{\pgfqpoint{0.918285in}{5.240366in}}%
\pgfpathlineto{\pgfqpoint{0.918958in}{5.228122in}}%
\pgfpathlineto{\pgfqpoint{0.919717in}{5.222276in}}%
\pgfpathlineto{\pgfqpoint{0.920306in}{5.224038in}}%
\pgfpathlineto{\pgfqpoint{0.920728in}{5.222657in}}%
\pgfpathlineto{\pgfqpoint{0.921065in}{5.224382in}}%
\pgfpathlineto{\pgfqpoint{0.921570in}{5.228642in}}%
\pgfpathlineto{\pgfqpoint{0.921991in}{5.223109in}}%
\pgfpathlineto{\pgfqpoint{0.922581in}{5.210736in}}%
\pgfpathlineto{\pgfqpoint{0.923086in}{5.222446in}}%
\pgfpathlineto{\pgfqpoint{0.923760in}{5.239146in}}%
\pgfpathlineto{\pgfqpoint{0.924266in}{5.230711in}}%
\pgfpathlineto{\pgfqpoint{0.924940in}{5.213614in}}%
\pgfpathlineto{\pgfqpoint{0.925445in}{5.224349in}}%
\pgfpathlineto{\pgfqpoint{0.926119in}{5.239734in}}%
\pgfpathlineto{\pgfqpoint{0.926540in}{5.229370in}}%
\pgfpathlineto{\pgfqpoint{0.927130in}{5.210071in}}%
\pgfpathlineto{\pgfqpoint{0.927636in}{5.226681in}}%
\pgfpathlineto{\pgfqpoint{0.928057in}{5.240721in}}%
\pgfpathlineto{\pgfqpoint{0.928647in}{5.223820in}}%
\pgfpathlineto{\pgfqpoint{0.929068in}{5.212114in}}%
\pgfpathlineto{\pgfqpoint{0.929573in}{5.226506in}}%
\pgfpathlineto{\pgfqpoint{0.929910in}{5.235578in}}%
\pgfpathlineto{\pgfqpoint{0.930416in}{5.220687in}}%
\pgfpathlineto{\pgfqpoint{0.930837in}{5.209757in}}%
\pgfpathlineto{\pgfqpoint{0.931342in}{5.224104in}}%
\pgfpathlineto{\pgfqpoint{0.931848in}{5.234305in}}%
\pgfpathlineto{\pgfqpoint{0.932438in}{5.225683in}}%
\pgfpathlineto{\pgfqpoint{0.932775in}{5.221669in}}%
\pgfpathlineto{\pgfqpoint{0.933280in}{5.228479in}}%
\pgfpathlineto{\pgfqpoint{0.933617in}{5.231205in}}%
\pgfpathlineto{\pgfqpoint{0.934207in}{5.226112in}}%
\pgfpathlineto{\pgfqpoint{0.935218in}{5.218346in}}%
\pgfpathlineto{\pgfqpoint{0.935555in}{5.223696in}}%
\pgfpathlineto{\pgfqpoint{0.936144in}{5.236368in}}%
\pgfpathlineto{\pgfqpoint{0.936650in}{5.225614in}}%
\pgfpathlineto{\pgfqpoint{0.936987in}{5.217818in}}%
\pgfpathlineto{\pgfqpoint{0.937576in}{5.228933in}}%
\pgfpathlineto{\pgfqpoint{0.937745in}{5.230683in}}%
\pgfpathlineto{\pgfqpoint{0.938250in}{5.222705in}}%
\pgfpathlineto{\pgfqpoint{0.938756in}{5.215864in}}%
\pgfpathlineto{\pgfqpoint{0.939261in}{5.222460in}}%
\pgfpathlineto{\pgfqpoint{0.939767in}{5.227963in}}%
\pgfpathlineto{\pgfqpoint{0.940525in}{5.225441in}}%
\pgfpathlineto{\pgfqpoint{0.940946in}{5.224650in}}%
\pgfpathlineto{\pgfqpoint{0.941368in}{5.225929in}}%
\pgfpathlineto{\pgfqpoint{0.941873in}{5.228662in}}%
\pgfpathlineto{\pgfqpoint{0.942294in}{5.225883in}}%
\pgfpathlineto{\pgfqpoint{0.942631in}{5.223967in}}%
\pgfpathlineto{\pgfqpoint{0.943052in}{5.227516in}}%
\pgfpathlineto{\pgfqpoint{0.943474in}{5.231159in}}%
\pgfpathlineto{\pgfqpoint{0.943895in}{5.224444in}}%
\pgfpathlineto{\pgfqpoint{0.944232in}{5.218164in}}%
\pgfpathlineto{\pgfqpoint{0.944653in}{5.227748in}}%
\pgfpathlineto{\pgfqpoint{0.945159in}{5.244778in}}%
\pgfpathlineto{\pgfqpoint{0.945664in}{5.224695in}}%
\pgfpathlineto{\pgfqpoint{0.946254in}{5.203503in}}%
\pgfpathlineto{\pgfqpoint{0.946759in}{5.219227in}}%
\pgfpathlineto{\pgfqpoint{0.947349in}{5.238760in}}%
\pgfpathlineto{\pgfqpoint{0.947854in}{5.221352in}}%
\pgfpathlineto{\pgfqpoint{0.948276in}{5.209358in}}%
\pgfpathlineto{\pgfqpoint{0.948781in}{5.225642in}}%
\pgfpathlineto{\pgfqpoint{0.949286in}{5.243401in}}%
\pgfpathlineto{\pgfqpoint{0.949792in}{5.227915in}}%
\pgfpathlineto{\pgfqpoint{0.950382in}{5.211433in}}%
\pgfpathlineto{\pgfqpoint{0.950887in}{5.221829in}}%
\pgfpathlineto{\pgfqpoint{0.951477in}{5.237467in}}%
\pgfpathlineto{\pgfqpoint{0.951982in}{5.226081in}}%
\pgfpathlineto{\pgfqpoint{0.952488in}{5.216551in}}%
\pgfpathlineto{\pgfqpoint{0.952993in}{5.227589in}}%
\pgfpathlineto{\pgfqpoint{0.953414in}{5.235929in}}%
\pgfpathlineto{\pgfqpoint{0.953836in}{5.224877in}}%
\pgfpathlineto{\pgfqpoint{0.954257in}{5.215241in}}%
\pgfpathlineto{\pgfqpoint{0.954847in}{5.225808in}}%
\pgfpathlineto{\pgfqpoint{0.955184in}{5.230222in}}%
\pgfpathlineto{\pgfqpoint{0.955689in}{5.222244in}}%
\pgfpathlineto{\pgfqpoint{0.955942in}{5.219813in}}%
\pgfpathlineto{\pgfqpoint{0.956363in}{5.229112in}}%
\pgfpathlineto{\pgfqpoint{0.956953in}{5.242810in}}%
\pgfpathlineto{\pgfqpoint{0.957458in}{5.231961in}}%
\pgfpathlineto{\pgfqpoint{0.958132in}{5.206571in}}%
\pgfpathlineto{\pgfqpoint{0.958722in}{5.224374in}}%
\pgfpathlineto{\pgfqpoint{0.959059in}{5.232860in}}%
\pgfpathlineto{\pgfqpoint{0.959649in}{5.218067in}}%
\pgfpathlineto{\pgfqpoint{0.959986in}{5.209443in}}%
\pgfpathlineto{\pgfqpoint{0.960491in}{5.223986in}}%
\pgfpathlineto{\pgfqpoint{0.960912in}{5.236312in}}%
\pgfpathlineto{\pgfqpoint{0.961586in}{5.224812in}}%
\pgfpathlineto{\pgfqpoint{0.962007in}{5.221383in}}%
\pgfpathlineto{\pgfqpoint{0.962429in}{5.227591in}}%
\pgfpathlineto{\pgfqpoint{0.962934in}{5.236753in}}%
\pgfpathlineto{\pgfqpoint{0.963440in}{5.227598in}}%
\pgfpathlineto{\pgfqpoint{0.963861in}{5.221356in}}%
\pgfpathlineto{\pgfqpoint{0.964366in}{5.231519in}}%
\pgfpathlineto{\pgfqpoint{0.964535in}{5.232883in}}%
\pgfpathlineto{\pgfqpoint{0.964872in}{5.224898in}}%
\pgfpathlineto{\pgfqpoint{0.965377in}{5.207670in}}%
\pgfpathlineto{\pgfqpoint{0.965883in}{5.223771in}}%
\pgfpathlineto{\pgfqpoint{0.966304in}{5.236520in}}%
\pgfpathlineto{\pgfqpoint{0.966894in}{5.222680in}}%
\pgfpathlineto{\pgfqpoint{0.967399in}{5.212038in}}%
\pgfpathlineto{\pgfqpoint{0.967905in}{5.226182in}}%
\pgfpathlineto{\pgfqpoint{0.968494in}{5.240435in}}%
\pgfpathlineto{\pgfqpoint{0.969000in}{5.230657in}}%
\pgfpathlineto{\pgfqpoint{0.969758in}{5.208579in}}%
\pgfpathlineto{\pgfqpoint{0.970263in}{5.222918in}}%
\pgfpathlineto{\pgfqpoint{0.970853in}{5.240373in}}%
\pgfpathlineto{\pgfqpoint{0.971359in}{5.224350in}}%
\pgfpathlineto{\pgfqpoint{0.971611in}{5.218645in}}%
\pgfpathlineto{\pgfqpoint{0.972117in}{5.230740in}}%
\pgfpathlineto{\pgfqpoint{0.972454in}{5.239266in}}%
\pgfpathlineto{\pgfqpoint{0.972959in}{5.224800in}}%
\pgfpathlineto{\pgfqpoint{0.973465in}{5.210698in}}%
\pgfpathlineto{\pgfqpoint{0.973970in}{5.225509in}}%
\pgfpathlineto{\pgfqpoint{0.974476in}{5.238069in}}%
\pgfpathlineto{\pgfqpoint{0.974981in}{5.227317in}}%
\pgfpathlineto{\pgfqpoint{0.975655in}{5.208387in}}%
\pgfpathlineto{\pgfqpoint{0.976245in}{5.219630in}}%
\pgfpathlineto{\pgfqpoint{0.976834in}{5.232932in}}%
\pgfpathlineto{\pgfqpoint{0.977424in}{5.222714in}}%
\pgfpathlineto{\pgfqpoint{0.977593in}{5.220790in}}%
\pgfpathlineto{\pgfqpoint{0.978098in}{5.229312in}}%
\pgfpathlineto{\pgfqpoint{0.978519in}{5.236842in}}%
\pgfpathlineto{\pgfqpoint{0.979025in}{5.225453in}}%
\pgfpathlineto{\pgfqpoint{0.979362in}{5.218380in}}%
\pgfpathlineto{\pgfqpoint{0.979867in}{5.231420in}}%
\pgfpathlineto{\pgfqpoint{0.980204in}{5.239730in}}%
\pgfpathlineto{\pgfqpoint{0.980710in}{5.225879in}}%
\pgfpathlineto{\pgfqpoint{0.981215in}{5.205823in}}%
\pgfpathlineto{\pgfqpoint{0.981721in}{5.224684in}}%
\pgfpathlineto{\pgfqpoint{0.982226in}{5.247083in}}%
\pgfpathlineto{\pgfqpoint{0.982732in}{5.222023in}}%
\pgfpathlineto{\pgfqpoint{0.983153in}{5.204265in}}%
\pgfpathlineto{\pgfqpoint{0.983742in}{5.220718in}}%
\pgfpathlineto{\pgfqpoint{0.985259in}{5.234767in}}%
\pgfpathlineto{\pgfqpoint{0.985343in}{5.234606in}}%
\pgfpathlineto{\pgfqpoint{0.986101in}{5.221920in}}%
\pgfpathlineto{\pgfqpoint{0.987534in}{5.223547in}}%
\pgfpathlineto{\pgfqpoint{0.988123in}{5.232511in}}%
\pgfpathlineto{\pgfqpoint{0.988544in}{5.224901in}}%
\pgfpathlineto{\pgfqpoint{0.989050in}{5.216402in}}%
\pgfpathlineto{\pgfqpoint{0.989555in}{5.226533in}}%
\pgfpathlineto{\pgfqpoint{0.989892in}{5.231415in}}%
\pgfpathlineto{\pgfqpoint{0.990398in}{5.222388in}}%
\pgfpathlineto{\pgfqpoint{0.990819in}{5.212698in}}%
\pgfpathlineto{\pgfqpoint{0.991325in}{5.225169in}}%
\pgfpathlineto{\pgfqpoint{0.991830in}{5.236408in}}%
\pgfpathlineto{\pgfqpoint{0.992420in}{5.225023in}}%
\pgfpathlineto{\pgfqpoint{0.992757in}{5.221363in}}%
\pgfpathlineto{\pgfqpoint{0.993346in}{5.227609in}}%
\pgfpathlineto{\pgfqpoint{0.993936in}{5.230902in}}%
\pgfpathlineto{\pgfqpoint{0.994357in}{5.227731in}}%
\pgfpathlineto{\pgfqpoint{0.995284in}{5.216201in}}%
\pgfpathlineto{\pgfqpoint{0.995705in}{5.223430in}}%
\pgfpathlineto{\pgfqpoint{0.996211in}{5.231792in}}%
\pgfpathlineto{\pgfqpoint{0.996716in}{5.222242in}}%
\pgfpathlineto{\pgfqpoint{0.996969in}{5.218080in}}%
\pgfpathlineto{\pgfqpoint{0.997559in}{5.228631in}}%
\pgfpathlineto{\pgfqpoint{0.997811in}{5.231402in}}%
\pgfpathlineto{\pgfqpoint{0.998233in}{5.224238in}}%
\pgfpathlineto{\pgfqpoint{0.998654in}{5.215624in}}%
\pgfpathlineto{\pgfqpoint{0.999075in}{5.227599in}}%
\pgfpathlineto{\pgfqpoint{0.999580in}{5.241080in}}%
\pgfpathlineto{\pgfqpoint{1.000086in}{5.226324in}}%
\pgfpathlineto{\pgfqpoint{1.000591in}{5.212173in}}%
\pgfpathlineto{\pgfqpoint{1.001097in}{5.224933in}}%
\pgfpathlineto{\pgfqpoint{1.001602in}{5.236352in}}%
\pgfpathlineto{\pgfqpoint{1.002108in}{5.223260in}}%
\pgfpathlineto{\pgfqpoint{1.002698in}{5.212921in}}%
\pgfpathlineto{\pgfqpoint{1.003119in}{5.221660in}}%
\pgfpathlineto{\pgfqpoint{1.003708in}{5.237492in}}%
\pgfpathlineto{\pgfqpoint{1.004298in}{5.224488in}}%
\pgfpathlineto{\pgfqpoint{1.004719in}{5.215414in}}%
\pgfpathlineto{\pgfqpoint{1.005225in}{5.226678in}}%
\pgfpathlineto{\pgfqpoint{1.005646in}{5.235165in}}%
\pgfpathlineto{\pgfqpoint{1.006236in}{5.224352in}}%
\pgfpathlineto{\pgfqpoint{1.006741in}{5.215211in}}%
\pgfpathlineto{\pgfqpoint{1.007247in}{5.224019in}}%
\pgfpathlineto{\pgfqpoint{1.007752in}{5.234107in}}%
\pgfpathlineto{\pgfqpoint{1.008258in}{5.224529in}}%
\pgfpathlineto{\pgfqpoint{1.008595in}{5.220228in}}%
\pgfpathlineto{\pgfqpoint{1.009100in}{5.229403in}}%
\pgfpathlineto{\pgfqpoint{1.009437in}{5.232633in}}%
\pgfpathlineto{\pgfqpoint{1.009858in}{5.223985in}}%
\pgfpathlineto{\pgfqpoint{1.010364in}{5.210641in}}%
\pgfpathlineto{\pgfqpoint{1.010869in}{5.223680in}}%
\pgfpathlineto{\pgfqpoint{1.011459in}{5.240200in}}%
\pgfpathlineto{\pgfqpoint{1.012049in}{5.229702in}}%
\pgfpathlineto{\pgfqpoint{1.012807in}{5.203642in}}%
\pgfpathlineto{\pgfqpoint{1.013312in}{5.222151in}}%
\pgfpathlineto{\pgfqpoint{1.013902in}{5.253048in}}%
\pgfpathlineto{\pgfqpoint{1.014408in}{5.225345in}}%
\pgfpathlineto{\pgfqpoint{1.014829in}{5.202703in}}%
\pgfpathlineto{\pgfqpoint{1.015418in}{5.230975in}}%
\pgfpathlineto{\pgfqpoint{1.015671in}{5.238657in}}%
\pgfpathlineto{\pgfqpoint{1.016261in}{5.221965in}}%
\pgfpathlineto{\pgfqpoint{1.016514in}{5.218215in}}%
\pgfpathlineto{\pgfqpoint{1.017103in}{5.227880in}}%
\pgfpathlineto{\pgfqpoint{1.017188in}{5.228487in}}%
\pgfpathlineto{\pgfqpoint{1.017525in}{5.225052in}}%
\pgfpathlineto{\pgfqpoint{1.017946in}{5.219430in}}%
\pgfpathlineto{\pgfqpoint{1.018367in}{5.227203in}}%
\pgfpathlineto{\pgfqpoint{1.018872in}{5.234724in}}%
\pgfpathlineto{\pgfqpoint{1.019378in}{5.228035in}}%
\pgfpathlineto{\pgfqpoint{1.020473in}{5.220545in}}%
\pgfpathlineto{\pgfqpoint{1.020810in}{5.221431in}}%
\pgfpathlineto{\pgfqpoint{1.021737in}{5.231708in}}%
\pgfpathlineto{\pgfqpoint{1.022158in}{5.224106in}}%
\pgfpathlineto{\pgfqpoint{1.022579in}{5.214264in}}%
\pgfpathlineto{\pgfqpoint{1.023169in}{5.226985in}}%
\pgfpathlineto{\pgfqpoint{1.023590in}{5.234418in}}%
\pgfpathlineto{\pgfqpoint{1.024096in}{5.223013in}}%
\pgfpathlineto{\pgfqpoint{1.024433in}{5.218212in}}%
\pgfpathlineto{\pgfqpoint{1.024938in}{5.228730in}}%
\pgfpathlineto{\pgfqpoint{1.025275in}{5.236075in}}%
\pgfpathlineto{\pgfqpoint{1.025781in}{5.225221in}}%
\pgfpathlineto{\pgfqpoint{1.026202in}{5.215112in}}%
\pgfpathlineto{\pgfqpoint{1.026707in}{5.227340in}}%
\pgfpathlineto{\pgfqpoint{1.027044in}{5.234012in}}%
\pgfpathlineto{\pgfqpoint{1.027550in}{5.222923in}}%
\pgfpathlineto{\pgfqpoint{1.027802in}{5.218841in}}%
\pgfpathlineto{\pgfqpoint{1.028308in}{5.230010in}}%
\pgfpathlineto{\pgfqpoint{1.028729in}{5.239262in}}%
\pgfpathlineto{\pgfqpoint{1.029235in}{5.223812in}}%
\pgfpathlineto{\pgfqpoint{1.029740in}{5.205525in}}%
\pgfpathlineto{\pgfqpoint{1.030330in}{5.224111in}}%
\pgfpathlineto{\pgfqpoint{1.030835in}{5.235801in}}%
\pgfpathlineto{\pgfqpoint{1.031425in}{5.227242in}}%
\pgfpathlineto{\pgfqpoint{1.032352in}{5.213087in}}%
\pgfpathlineto{\pgfqpoint{1.032773in}{5.220260in}}%
\pgfpathlineto{\pgfqpoint{1.033447in}{5.240375in}}%
\pgfpathlineto{\pgfqpoint{1.034121in}{5.229521in}}%
\pgfpathlineto{\pgfqpoint{1.035890in}{5.217827in}}%
\pgfpathlineto{\pgfqpoint{1.035974in}{5.218008in}}%
\pgfpathlineto{\pgfqpoint{1.036732in}{5.230934in}}%
\pgfpathlineto{\pgfqpoint{1.037238in}{5.221548in}}%
\pgfpathlineto{\pgfqpoint{1.037659in}{5.211029in}}%
\pgfpathlineto{\pgfqpoint{1.038164in}{5.227702in}}%
\pgfpathlineto{\pgfqpoint{1.038670in}{5.246227in}}%
\pgfpathlineto{\pgfqpoint{1.039175in}{5.225877in}}%
\pgfpathlineto{\pgfqpoint{1.039681in}{5.208136in}}%
\pgfpathlineto{\pgfqpoint{1.040271in}{5.222225in}}%
\pgfpathlineto{\pgfqpoint{1.040860in}{5.234747in}}%
\pgfpathlineto{\pgfqpoint{1.041534in}{5.229522in}}%
\pgfpathlineto{\pgfqpoint{1.041871in}{5.231407in}}%
\pgfpathlineto{\pgfqpoint{1.042124in}{5.228600in}}%
\pgfpathlineto{\pgfqpoint{1.042714in}{5.213433in}}%
\pgfpathlineto{\pgfqpoint{1.043303in}{5.225364in}}%
\pgfpathlineto{\pgfqpoint{1.043809in}{5.232434in}}%
\pgfpathlineto{\pgfqpoint{1.044483in}{5.227451in}}%
\pgfpathlineto{\pgfqpoint{1.045494in}{5.210217in}}%
\pgfpathlineto{\pgfqpoint{1.045831in}{5.220500in}}%
\pgfpathlineto{\pgfqpoint{1.046505in}{5.248536in}}%
\pgfpathlineto{\pgfqpoint{1.046926in}{5.228983in}}%
\pgfpathlineto{\pgfqpoint{1.047431in}{5.202256in}}%
\pgfpathlineto{\pgfqpoint{1.048021in}{5.229345in}}%
\pgfpathlineto{\pgfqpoint{1.048358in}{5.239782in}}%
\pgfpathlineto{\pgfqpoint{1.049032in}{5.224621in}}%
\pgfpathlineto{\pgfqpoint{1.049453in}{5.222943in}}%
\pgfpathlineto{\pgfqpoint{1.050127in}{5.224692in}}%
\pgfpathlineto{\pgfqpoint{1.050633in}{5.226096in}}%
\pgfpathlineto{\pgfqpoint{1.051307in}{5.224995in}}%
\pgfpathlineto{\pgfqpoint{1.051981in}{5.223625in}}%
\pgfpathlineto{\pgfqpoint{1.052402in}{5.220650in}}%
\pgfpathlineto{\pgfqpoint{1.052823in}{5.224638in}}%
\pgfpathlineto{\pgfqpoint{1.053076in}{5.226858in}}%
\pgfpathlineto{\pgfqpoint{1.053581in}{5.221638in}}%
\pgfpathlineto{\pgfqpoint{1.053665in}{5.221086in}}%
\pgfpathlineto{\pgfqpoint{1.053918in}{5.224402in}}%
\pgfpathlineto{\pgfqpoint{1.054424in}{5.237651in}}%
\pgfpathlineto{\pgfqpoint{1.054845in}{5.223627in}}%
\pgfpathlineto{\pgfqpoint{1.055182in}{5.212922in}}%
\pgfpathlineto{\pgfqpoint{1.055687in}{5.233065in}}%
\pgfpathlineto{\pgfqpoint{1.056024in}{5.244198in}}%
\pgfpathlineto{\pgfqpoint{1.056530in}{5.221846in}}%
\pgfpathlineto{\pgfqpoint{1.056951in}{5.208827in}}%
\pgfpathlineto{\pgfqpoint{1.057541in}{5.226636in}}%
\pgfpathlineto{\pgfqpoint{1.057878in}{5.232115in}}%
\pgfpathlineto{\pgfqpoint{1.058636in}{5.226166in}}%
\pgfpathlineto{\pgfqpoint{1.058973in}{5.224699in}}%
\pgfpathlineto{\pgfqpoint{1.059647in}{5.214547in}}%
\pgfpathlineto{\pgfqpoint{1.060237in}{5.222458in}}%
\pgfpathlineto{\pgfqpoint{1.061079in}{5.243932in}}%
\pgfpathlineto{\pgfqpoint{1.061500in}{5.231237in}}%
\pgfpathlineto{\pgfqpoint{1.062090in}{5.213527in}}%
\pgfpathlineto{\pgfqpoint{1.062680in}{5.223120in}}%
\pgfpathlineto{\pgfqpoint{1.063691in}{5.232175in}}%
\pgfpathlineto{\pgfqpoint{1.063943in}{5.229219in}}%
\pgfpathlineto{\pgfqpoint{1.064617in}{5.204595in}}%
\pgfpathlineto{\pgfqpoint{1.065123in}{5.226880in}}%
\pgfpathlineto{\pgfqpoint{1.065544in}{5.243335in}}%
\pgfpathlineto{\pgfqpoint{1.066134in}{5.224918in}}%
\pgfpathlineto{\pgfqpoint{1.066471in}{5.218686in}}%
\pgfpathlineto{\pgfqpoint{1.067145in}{5.226666in}}%
\pgfpathlineto{\pgfqpoint{1.067313in}{5.227142in}}%
\pgfpathlineto{\pgfqpoint{1.067819in}{5.224521in}}%
\pgfpathlineto{\pgfqpoint{1.068156in}{5.226337in}}%
\pgfpathlineto{\pgfqpoint{1.068998in}{5.233925in}}%
\pgfpathlineto{\pgfqpoint{1.069419in}{5.228569in}}%
\pgfpathlineto{\pgfqpoint{1.070093in}{5.209025in}}%
\pgfpathlineto{\pgfqpoint{1.070683in}{5.223499in}}%
\pgfpathlineto{\pgfqpoint{1.072031in}{5.229969in}}%
\pgfpathlineto{\pgfqpoint{1.072199in}{5.230902in}}%
\pgfpathlineto{\pgfqpoint{1.072620in}{5.226562in}}%
\pgfpathlineto{\pgfqpoint{1.073042in}{5.221089in}}%
\pgfpathlineto{\pgfqpoint{1.073547in}{5.229370in}}%
\pgfpathlineto{\pgfqpoint{1.073800in}{5.231017in}}%
\pgfpathlineto{\pgfqpoint{1.074137in}{5.224511in}}%
\pgfpathlineto{\pgfqpoint{1.074558in}{5.217160in}}%
\pgfpathlineto{\pgfqpoint{1.075148in}{5.225375in}}%
\pgfpathlineto{\pgfqpoint{1.075738in}{5.235173in}}%
\pgfpathlineto{\pgfqpoint{1.076243in}{5.226277in}}%
\pgfpathlineto{\pgfqpoint{1.076580in}{5.220256in}}%
\pgfpathlineto{\pgfqpoint{1.077170in}{5.229762in}}%
\pgfpathlineto{\pgfqpoint{1.077422in}{5.231589in}}%
\pgfpathlineto{\pgfqpoint{1.077844in}{5.226077in}}%
\pgfpathlineto{\pgfqpoint{1.078433in}{5.220585in}}%
\pgfpathlineto{\pgfqpoint{1.079023in}{5.224022in}}%
\pgfpathlineto{\pgfqpoint{1.079529in}{5.225672in}}%
\pgfpathlineto{\pgfqpoint{1.079950in}{5.223232in}}%
\pgfpathlineto{\pgfqpoint{1.080371in}{5.220943in}}%
\pgfpathlineto{\pgfqpoint{1.080792in}{5.224099in}}%
\pgfpathlineto{\pgfqpoint{1.081213in}{5.229019in}}%
\pgfpathlineto{\pgfqpoint{1.081635in}{5.222525in}}%
\pgfpathlineto{\pgfqpoint{1.082056in}{5.216081in}}%
\pgfpathlineto{\pgfqpoint{1.082477in}{5.224599in}}%
\pgfpathlineto{\pgfqpoint{1.083151in}{5.243531in}}%
\pgfpathlineto{\pgfqpoint{1.083657in}{5.231523in}}%
\pgfpathlineto{\pgfqpoint{1.084330in}{5.213939in}}%
\pgfpathlineto{\pgfqpoint{1.084920in}{5.220078in}}%
\pgfpathlineto{\pgfqpoint{1.085931in}{5.233881in}}%
\pgfpathlineto{\pgfqpoint{1.086352in}{5.229107in}}%
\pgfpathlineto{\pgfqpoint{1.087026in}{5.215970in}}%
\pgfpathlineto{\pgfqpoint{1.087532in}{5.225967in}}%
\pgfpathlineto{\pgfqpoint{1.087869in}{5.230761in}}%
\pgfpathlineto{\pgfqpoint{1.088374in}{5.222942in}}%
\pgfpathlineto{\pgfqpoint{1.088627in}{5.220112in}}%
\pgfpathlineto{\pgfqpoint{1.089132in}{5.227371in}}%
\pgfpathlineto{\pgfqpoint{1.089301in}{5.229156in}}%
\pgfpathlineto{\pgfqpoint{1.089722in}{5.221566in}}%
\pgfpathlineto{\pgfqpoint{1.090143in}{5.214525in}}%
\pgfpathlineto{\pgfqpoint{1.090565in}{5.227485in}}%
\pgfpathlineto{\pgfqpoint{1.091070in}{5.241855in}}%
\pgfpathlineto{\pgfqpoint{1.091575in}{5.228194in}}%
\pgfpathlineto{\pgfqpoint{1.092249in}{5.207651in}}%
\pgfpathlineto{\pgfqpoint{1.092671in}{5.221538in}}%
\pgfpathlineto{\pgfqpoint{1.093260in}{5.239927in}}%
\pgfpathlineto{\pgfqpoint{1.093850in}{5.226402in}}%
\pgfpathlineto{\pgfqpoint{1.094356in}{5.216906in}}%
\pgfpathlineto{\pgfqpoint{1.094945in}{5.224172in}}%
\pgfpathlineto{\pgfqpoint{1.095619in}{5.235920in}}%
\pgfpathlineto{\pgfqpoint{1.096209in}{5.228897in}}%
\pgfpathlineto{\pgfqpoint{1.096967in}{5.209848in}}%
\pgfpathlineto{\pgfqpoint{1.097473in}{5.220028in}}%
\pgfpathlineto{\pgfqpoint{1.098231in}{5.234292in}}%
\pgfpathlineto{\pgfqpoint{1.098736in}{5.226087in}}%
\pgfpathlineto{\pgfqpoint{1.099326in}{5.218022in}}%
\pgfpathlineto{\pgfqpoint{1.099747in}{5.225034in}}%
\pgfpathlineto{\pgfqpoint{1.100421in}{5.241679in}}%
\pgfpathlineto{\pgfqpoint{1.100842in}{5.228217in}}%
\pgfpathlineto{\pgfqpoint{1.101432in}{5.208658in}}%
\pgfpathlineto{\pgfqpoint{1.102022in}{5.219153in}}%
\pgfpathlineto{\pgfqpoint{1.102949in}{5.237252in}}%
\pgfpathlineto{\pgfqpoint{1.103454in}{5.228317in}}%
\pgfpathlineto{\pgfqpoint{1.104044in}{5.214320in}}%
\pgfpathlineto{\pgfqpoint{1.104549in}{5.223559in}}%
\pgfpathlineto{\pgfqpoint{1.105223in}{5.237248in}}%
\pgfpathlineto{\pgfqpoint{1.105729in}{5.227373in}}%
\pgfpathlineto{\pgfqpoint{1.106487in}{5.214473in}}%
\pgfpathlineto{\pgfqpoint{1.106908in}{5.221688in}}%
\pgfpathlineto{\pgfqpoint{1.107582in}{5.237105in}}%
\pgfpathlineto{\pgfqpoint{1.108087in}{5.228464in}}%
\pgfpathlineto{\pgfqpoint{1.108761in}{5.212997in}}%
\pgfpathlineto{\pgfqpoint{1.109267in}{5.223675in}}%
\pgfpathlineto{\pgfqpoint{1.109688in}{5.232875in}}%
\pgfpathlineto{\pgfqpoint{1.110362in}{5.225645in}}%
\pgfpathlineto{\pgfqpoint{1.111541in}{5.217407in}}%
\pgfpathlineto{\pgfqpoint{1.111963in}{5.223984in}}%
\pgfpathlineto{\pgfqpoint{1.112552in}{5.237905in}}%
\pgfpathlineto{\pgfqpoint{1.113142in}{5.229934in}}%
\pgfpathlineto{\pgfqpoint{1.114406in}{5.217196in}}%
\pgfpathlineto{\pgfqpoint{1.114658in}{5.219559in}}%
\pgfpathlineto{\pgfqpoint{1.115332in}{5.234973in}}%
\pgfpathlineto{\pgfqpoint{1.115922in}{5.225205in}}%
\pgfpathlineto{\pgfqpoint{1.116596in}{5.213534in}}%
\pgfpathlineto{\pgfqpoint{1.117102in}{5.222070in}}%
\pgfpathlineto{\pgfqpoint{1.118450in}{5.238870in}}%
\pgfpathlineto{\pgfqpoint{1.118618in}{5.236752in}}%
\pgfpathlineto{\pgfqpoint{1.119460in}{5.209040in}}%
\pgfpathlineto{\pgfqpoint{1.120050in}{5.223115in}}%
\pgfpathlineto{\pgfqpoint{1.120556in}{5.233422in}}%
\pgfpathlineto{\pgfqpoint{1.121061in}{5.222387in}}%
\pgfpathlineto{\pgfqpoint{1.121314in}{5.218663in}}%
\pgfpathlineto{\pgfqpoint{1.121904in}{5.227791in}}%
\pgfpathlineto{\pgfqpoint{1.122241in}{5.230378in}}%
\pgfpathlineto{\pgfqpoint{1.122746in}{5.224037in}}%
\pgfpathlineto{\pgfqpoint{1.123083in}{5.221247in}}%
\pgfpathlineto{\pgfqpoint{1.123588in}{5.226163in}}%
\pgfpathlineto{\pgfqpoint{1.125189in}{5.236532in}}%
\pgfpathlineto{\pgfqpoint{1.124347in}{5.225082in}}%
\pgfpathlineto{\pgfqpoint{1.125526in}{5.233355in}}%
\pgfpathlineto{\pgfqpoint{1.126284in}{5.216280in}}%
\pgfpathlineto{\pgfqpoint{1.127127in}{5.221554in}}%
\pgfpathlineto{\pgfqpoint{1.127632in}{5.225215in}}%
\pgfpathlineto{\pgfqpoint{1.128222in}{5.231872in}}%
\pgfpathlineto{\pgfqpoint{1.128643in}{5.223985in}}%
\pgfpathlineto{\pgfqpoint{1.129233in}{5.210453in}}%
\pgfpathlineto{\pgfqpoint{1.129654in}{5.222712in}}%
\pgfpathlineto{\pgfqpoint{1.130244in}{5.243310in}}%
\pgfpathlineto{\pgfqpoint{1.130749in}{5.225949in}}%
\pgfpathlineto{\pgfqpoint{1.131170in}{5.214562in}}%
\pgfpathlineto{\pgfqpoint{1.131760in}{5.226508in}}%
\pgfpathlineto{\pgfqpoint{1.132097in}{5.230435in}}%
\pgfpathlineto{\pgfqpoint{1.132771in}{5.225269in}}%
\pgfpathlineto{\pgfqpoint{1.132855in}{5.225148in}}%
\pgfpathlineto{\pgfqpoint{1.133192in}{5.226487in}}%
\pgfpathlineto{\pgfqpoint{1.134203in}{5.231859in}}%
\pgfpathlineto{\pgfqpoint{1.134540in}{5.228027in}}%
\pgfpathlineto{\pgfqpoint{1.135383in}{5.213843in}}%
\pgfpathlineto{\pgfqpoint{1.135972in}{5.219267in}}%
\pgfpathlineto{\pgfqpoint{1.137152in}{5.238228in}}%
\pgfpathlineto{\pgfqpoint{1.137657in}{5.227160in}}%
\pgfpathlineto{\pgfqpoint{1.138247in}{5.217162in}}%
\pgfpathlineto{\pgfqpoint{1.138837in}{5.222089in}}%
\pgfpathlineto{\pgfqpoint{1.139679in}{5.231009in}}%
\pgfpathlineto{\pgfqpoint{1.140353in}{5.228022in}}%
\pgfpathlineto{\pgfqpoint{1.142291in}{5.218021in}}%
\pgfpathlineto{\pgfqpoint{1.142543in}{5.220284in}}%
\pgfpathlineto{\pgfqpoint{1.143386in}{5.237444in}}%
\pgfpathlineto{\pgfqpoint{1.143891in}{5.227193in}}%
\pgfpathlineto{\pgfqpoint{1.144565in}{5.212760in}}%
\pgfpathlineto{\pgfqpoint{1.144987in}{5.222405in}}%
\pgfpathlineto{\pgfqpoint{1.145660in}{5.240590in}}%
\pgfpathlineto{\pgfqpoint{1.146166in}{5.226469in}}%
\pgfpathlineto{\pgfqpoint{1.146756in}{5.209289in}}%
\pgfpathlineto{\pgfqpoint{1.147261in}{5.225793in}}%
\pgfpathlineto{\pgfqpoint{1.147767in}{5.238844in}}%
\pgfpathlineto{\pgfqpoint{1.148272in}{5.227510in}}%
\pgfpathlineto{\pgfqpoint{1.148862in}{5.210931in}}%
\pgfpathlineto{\pgfqpoint{1.149367in}{5.225174in}}%
\pgfpathlineto{\pgfqpoint{1.149873in}{5.235507in}}%
\pgfpathlineto{\pgfqpoint{1.150378in}{5.224824in}}%
\pgfpathlineto{\pgfqpoint{1.150884in}{5.214901in}}%
\pgfpathlineto{\pgfqpoint{1.151389in}{5.225545in}}%
\pgfpathlineto{\pgfqpoint{1.151810in}{5.233085in}}%
\pgfpathlineto{\pgfqpoint{1.152400in}{5.223682in}}%
\pgfpathlineto{\pgfqpoint{1.152569in}{5.222140in}}%
\pgfpathlineto{\pgfqpoint{1.152990in}{5.228001in}}%
\pgfpathlineto{\pgfqpoint{1.153411in}{5.236248in}}%
\pgfpathlineto{\pgfqpoint{1.153916in}{5.225775in}}%
\pgfpathlineto{\pgfqpoint{1.154338in}{5.214886in}}%
\pgfpathlineto{\pgfqpoint{1.154927in}{5.227911in}}%
\pgfpathlineto{\pgfqpoint{1.155264in}{5.233366in}}%
\pgfpathlineto{\pgfqpoint{1.155770in}{5.222630in}}%
\pgfpathlineto{\pgfqpoint{1.156360in}{5.212107in}}%
\pgfpathlineto{\pgfqpoint{1.156865in}{5.222444in}}%
\pgfpathlineto{\pgfqpoint{1.157455in}{5.235825in}}%
\pgfpathlineto{\pgfqpoint{1.158044in}{5.227849in}}%
\pgfpathlineto{\pgfqpoint{1.158466in}{5.223456in}}%
\pgfpathlineto{\pgfqpoint{1.159140in}{5.227118in}}%
\pgfpathlineto{\pgfqpoint{1.159477in}{5.224614in}}%
\pgfpathlineto{\pgfqpoint{1.159982in}{5.218966in}}%
\pgfpathlineto{\pgfqpoint{1.160403in}{5.225114in}}%
\pgfpathlineto{\pgfqpoint{1.160909in}{5.236849in}}%
\pgfpathlineto{\pgfqpoint{1.161414in}{5.224036in}}%
\pgfpathlineto{\pgfqpoint{1.161835in}{5.215273in}}%
\pgfpathlineto{\pgfqpoint{1.162341in}{5.226196in}}%
\pgfpathlineto{\pgfqpoint{1.162594in}{5.230022in}}%
\pgfpathlineto{\pgfqpoint{1.163268in}{5.222348in}}%
\pgfpathlineto{\pgfqpoint{1.163520in}{5.223571in}}%
\pgfpathlineto{\pgfqpoint{1.165037in}{5.232194in}}%
\pgfpathlineto{\pgfqpoint{1.165205in}{5.231226in}}%
\pgfpathlineto{\pgfqpoint{1.165963in}{5.215809in}}%
\pgfpathlineto{\pgfqpoint{1.166553in}{5.224967in}}%
\pgfpathlineto{\pgfqpoint{1.167143in}{5.233489in}}%
\pgfpathlineto{\pgfqpoint{1.167564in}{5.224693in}}%
\pgfpathlineto{\pgfqpoint{1.168070in}{5.216283in}}%
\pgfpathlineto{\pgfqpoint{1.168659in}{5.225084in}}%
\pgfpathlineto{\pgfqpoint{1.169165in}{5.234033in}}%
\pgfpathlineto{\pgfqpoint{1.169670in}{5.225502in}}%
\pgfpathlineto{\pgfqpoint{1.170091in}{5.220074in}}%
\pgfpathlineto{\pgfqpoint{1.170597in}{5.226809in}}%
\pgfpathlineto{\pgfqpoint{1.171018in}{5.230775in}}%
\pgfpathlineto{\pgfqpoint{1.171608in}{5.225868in}}%
\pgfpathlineto{\pgfqpoint{1.171692in}{5.225540in}}%
\pgfpathlineto{\pgfqpoint{1.172029in}{5.228045in}}%
\pgfpathlineto{\pgfqpoint{1.172366in}{5.230812in}}%
\pgfpathlineto{\pgfqpoint{1.172871in}{5.226705in}}%
\pgfpathlineto{\pgfqpoint{1.173798in}{5.215321in}}%
\pgfpathlineto{\pgfqpoint{1.174219in}{5.221358in}}%
\pgfpathlineto{\pgfqpoint{1.174809in}{5.231915in}}%
\pgfpathlineto{\pgfqpoint{1.175399in}{5.223338in}}%
\pgfpathlineto{\pgfqpoint{1.175652in}{5.219306in}}%
\pgfpathlineto{\pgfqpoint{1.176241in}{5.229268in}}%
\pgfpathlineto{\pgfqpoint{1.176410in}{5.231108in}}%
\pgfpathlineto{\pgfqpoint{1.176915in}{5.223842in}}%
\pgfpathlineto{\pgfqpoint{1.177336in}{5.219204in}}%
\pgfpathlineto{\pgfqpoint{1.177842in}{5.226335in}}%
\pgfpathlineto{\pgfqpoint{1.178347in}{5.231750in}}%
\pgfpathlineto{\pgfqpoint{1.178853in}{5.226669in}}%
\pgfpathlineto{\pgfqpoint{1.179527in}{5.216467in}}%
\pgfpathlineto{\pgfqpoint{1.179948in}{5.226597in}}%
\pgfpathlineto{\pgfqpoint{1.180369in}{5.237950in}}%
\pgfpathlineto{\pgfqpoint{1.180875in}{5.224533in}}%
\pgfpathlineto{\pgfqpoint{1.181296in}{5.213351in}}%
\pgfpathlineto{\pgfqpoint{1.181886in}{5.226863in}}%
\pgfpathlineto{\pgfqpoint{1.182307in}{5.235072in}}%
\pgfpathlineto{\pgfqpoint{1.182897in}{5.225971in}}%
\pgfpathlineto{\pgfqpoint{1.183402in}{5.220480in}}%
\pgfpathlineto{\pgfqpoint{1.183992in}{5.226083in}}%
\pgfpathlineto{\pgfqpoint{1.184244in}{5.227343in}}%
\pgfpathlineto{\pgfqpoint{1.184666in}{5.223623in}}%
\pgfpathlineto{\pgfqpoint{1.185003in}{5.221920in}}%
\pgfpathlineto{\pgfqpoint{1.185424in}{5.223933in}}%
\pgfpathlineto{\pgfqpoint{1.185761in}{5.223213in}}%
\pgfpathlineto{\pgfqpoint{1.185929in}{5.223154in}}%
\pgfpathlineto{\pgfqpoint{1.186098in}{5.223952in}}%
\pgfpathlineto{\pgfqpoint{1.186688in}{5.229200in}}%
\pgfpathlineto{\pgfqpoint{1.187025in}{5.224575in}}%
\pgfpathlineto{\pgfqpoint{1.187530in}{5.215542in}}%
\pgfpathlineto{\pgfqpoint{1.187951in}{5.227387in}}%
\pgfpathlineto{\pgfqpoint{1.188457in}{5.242373in}}%
\pgfpathlineto{\pgfqpoint{1.188962in}{5.226319in}}%
\pgfpathlineto{\pgfqpoint{1.189468in}{5.218158in}}%
\pgfpathlineto{\pgfqpoint{1.190057in}{5.224464in}}%
\pgfpathlineto{\pgfqpoint{1.191153in}{5.231332in}}%
\pgfpathlineto{\pgfqpoint{1.191405in}{5.228854in}}%
\pgfpathlineto{\pgfqpoint{1.192248in}{5.209238in}}%
\pgfpathlineto{\pgfqpoint{1.192753in}{5.223202in}}%
\pgfpathlineto{\pgfqpoint{1.193259in}{5.233420in}}%
\pgfpathlineto{\pgfqpoint{1.193848in}{5.222858in}}%
\pgfpathlineto{\pgfqpoint{1.194101in}{5.220719in}}%
\pgfpathlineto{\pgfqpoint{1.194691in}{5.226698in}}%
\pgfpathlineto{\pgfqpoint{1.195281in}{5.220650in}}%
\pgfpathlineto{\pgfqpoint{1.195702in}{5.226954in}}%
\pgfpathlineto{\pgfqpoint{1.196291in}{5.244925in}}%
\pgfpathlineto{\pgfqpoint{1.196797in}{5.227097in}}%
\pgfpathlineto{\pgfqpoint{1.197302in}{5.211684in}}%
\pgfpathlineto{\pgfqpoint{1.197892in}{5.227276in}}%
\pgfpathlineto{\pgfqpoint{1.198145in}{5.231033in}}%
\pgfpathlineto{\pgfqpoint{1.198650in}{5.220267in}}%
\pgfpathlineto{\pgfqpoint{1.198903in}{5.216898in}}%
\pgfpathlineto{\pgfqpoint{1.199408in}{5.225368in}}%
\pgfpathlineto{\pgfqpoint{1.199830in}{5.232370in}}%
\pgfpathlineto{\pgfqpoint{1.200335in}{5.222258in}}%
\pgfpathlineto{\pgfqpoint{1.200672in}{5.218082in}}%
\pgfpathlineto{\pgfqpoint{1.201262in}{5.225347in}}%
\pgfpathlineto{\pgfqpoint{1.202526in}{5.234054in}}%
\pgfpathlineto{\pgfqpoint{1.202947in}{5.229669in}}%
\pgfpathlineto{\pgfqpoint{1.203621in}{5.218244in}}%
\pgfpathlineto{\pgfqpoint{1.204126in}{5.225953in}}%
\pgfpathlineto{\pgfqpoint{1.204295in}{5.227834in}}%
\pgfpathlineto{\pgfqpoint{1.204884in}{5.221271in}}%
\pgfpathlineto{\pgfqpoint{1.204969in}{5.220800in}}%
\pgfpathlineto{\pgfqpoint{1.205306in}{5.224128in}}%
\pgfpathlineto{\pgfqpoint{1.205811in}{5.232726in}}%
\pgfpathlineto{\pgfqpoint{1.206232in}{5.222031in}}%
\pgfpathlineto{\pgfqpoint{1.206738in}{5.204434in}}%
\pgfpathlineto{\pgfqpoint{1.207159in}{5.224418in}}%
\pgfpathlineto{\pgfqpoint{1.207664in}{5.252909in}}%
\pgfpathlineto{\pgfqpoint{1.208254in}{5.224277in}}%
\pgfpathlineto{\pgfqpoint{1.208675in}{5.207240in}}%
\pgfpathlineto{\pgfqpoint{1.209265in}{5.225934in}}%
\pgfpathlineto{\pgfqpoint{1.209686in}{5.235507in}}%
\pgfpathlineto{\pgfqpoint{1.210276in}{5.225353in}}%
\pgfpathlineto{\pgfqpoint{1.211708in}{5.212145in}}%
\pgfpathlineto{\pgfqpoint{1.211961in}{5.217252in}}%
\pgfpathlineto{\pgfqpoint{1.212635in}{5.242183in}}%
\pgfpathlineto{\pgfqpoint{1.213225in}{5.224635in}}%
\pgfpathlineto{\pgfqpoint{1.213562in}{5.220197in}}%
\pgfpathlineto{\pgfqpoint{1.214404in}{5.221754in}}%
\pgfpathlineto{\pgfqpoint{1.214488in}{5.221653in}}%
\pgfpathlineto{\pgfqpoint{1.214657in}{5.222508in}}%
\pgfpathlineto{\pgfqpoint{1.215331in}{5.235015in}}%
\pgfpathlineto{\pgfqpoint{1.215836in}{5.223687in}}%
\pgfpathlineto{\pgfqpoint{1.216342in}{5.207905in}}%
\pgfpathlineto{\pgfqpoint{1.216847in}{5.223222in}}%
\pgfpathlineto{\pgfqpoint{1.217437in}{5.241886in}}%
\pgfpathlineto{\pgfqpoint{1.218027in}{5.227407in}}%
\pgfpathlineto{\pgfqpoint{1.218448in}{5.221744in}}%
\pgfpathlineto{\pgfqpoint{1.219206in}{5.224488in}}%
\pgfpathlineto{\pgfqpoint{1.220133in}{5.219094in}}%
\pgfpathlineto{\pgfqpoint{1.220470in}{5.222182in}}%
\pgfpathlineto{\pgfqpoint{1.221059in}{5.232945in}}%
\pgfpathlineto{\pgfqpoint{1.221481in}{5.223467in}}%
\pgfpathlineto{\pgfqpoint{1.221902in}{5.215420in}}%
\pgfpathlineto{\pgfqpoint{1.222407in}{5.224701in}}%
\pgfpathlineto{\pgfqpoint{1.222997in}{5.231116in}}%
\pgfpathlineto{\pgfqpoint{1.223671in}{5.228089in}}%
\pgfpathlineto{\pgfqpoint{1.224008in}{5.227326in}}%
\pgfpathlineto{\pgfqpoint{1.224176in}{5.228325in}}%
\pgfpathlineto{\pgfqpoint{1.224598in}{5.233475in}}%
\pgfpathlineto{\pgfqpoint{1.224935in}{5.226071in}}%
\pgfpathlineto{\pgfqpoint{1.225524in}{5.210571in}}%
\pgfpathlineto{\pgfqpoint{1.226030in}{5.223852in}}%
\pgfpathlineto{\pgfqpoint{1.226451in}{5.234306in}}%
\pgfpathlineto{\pgfqpoint{1.227125in}{5.223216in}}%
\pgfpathlineto{\pgfqpoint{1.227209in}{5.223063in}}%
\pgfpathlineto{\pgfqpoint{1.227378in}{5.224625in}}%
\pgfpathlineto{\pgfqpoint{1.227967in}{5.234135in}}%
\pgfpathlineto{\pgfqpoint{1.228473in}{5.226031in}}%
\pgfpathlineto{\pgfqpoint{1.229315in}{5.215227in}}%
\pgfpathlineto{\pgfqpoint{1.229737in}{5.221418in}}%
\pgfpathlineto{\pgfqpoint{1.231421in}{5.237319in}}%
\pgfpathlineto{\pgfqpoint{1.231674in}{5.234596in}}%
\pgfpathlineto{\pgfqpoint{1.232432in}{5.213007in}}%
\pgfpathlineto{\pgfqpoint{1.233022in}{5.227926in}}%
\pgfpathlineto{\pgfqpoint{1.233275in}{5.231425in}}%
\pgfpathlineto{\pgfqpoint{1.233865in}{5.227757in}}%
\pgfpathlineto{\pgfqpoint{1.234117in}{5.228580in}}%
\pgfpathlineto{\pgfqpoint{1.234286in}{5.228986in}}%
\pgfpathlineto{\pgfqpoint{1.234623in}{5.226543in}}%
\pgfpathlineto{\pgfqpoint{1.235465in}{5.216312in}}%
\pgfpathlineto{\pgfqpoint{1.235971in}{5.221591in}}%
\pgfpathlineto{\pgfqpoint{1.236897in}{5.233577in}}%
\pgfpathlineto{\pgfqpoint{1.237319in}{5.227608in}}%
\pgfpathlineto{\pgfqpoint{1.237824in}{5.219778in}}%
\pgfpathlineto{\pgfqpoint{1.238414in}{5.226826in}}%
\pgfpathlineto{\pgfqpoint{1.238582in}{5.228073in}}%
\pgfpathlineto{\pgfqpoint{1.239088in}{5.224052in}}%
\pgfpathlineto{\pgfqpoint{1.239509in}{5.220315in}}%
\pgfpathlineto{\pgfqpoint{1.240183in}{5.224293in}}%
\pgfpathlineto{\pgfqpoint{1.240857in}{5.238343in}}%
\pgfpathlineto{\pgfqpoint{1.241194in}{5.246112in}}%
\pgfpathlineto{\pgfqpoint{1.241615in}{5.230932in}}%
\pgfpathlineto{\pgfqpoint{1.242120in}{5.210075in}}%
\pgfpathlineto{\pgfqpoint{1.242794in}{5.226182in}}%
\pgfpathlineto{\pgfqpoint{1.242963in}{5.227467in}}%
\pgfpathlineto{\pgfqpoint{1.243637in}{5.223381in}}%
\pgfpathlineto{\pgfqpoint{1.244058in}{5.221542in}}%
\pgfpathlineto{\pgfqpoint{1.244985in}{5.213054in}}%
\pgfpathlineto{\pgfqpoint{1.245322in}{5.218095in}}%
\pgfpathlineto{\pgfqpoint{1.246164in}{5.240834in}}%
\pgfpathlineto{\pgfqpoint{1.246754in}{5.230398in}}%
\pgfpathlineto{\pgfqpoint{1.247344in}{5.216080in}}%
\pgfpathlineto{\pgfqpoint{1.248018in}{5.224933in}}%
\pgfpathlineto{\pgfqpoint{1.248776in}{5.233795in}}%
\pgfpathlineto{\pgfqpoint{1.249281in}{5.226873in}}%
\pgfpathlineto{\pgfqpoint{1.249702in}{5.221036in}}%
\pgfpathlineto{\pgfqpoint{1.250208in}{5.227766in}}%
\pgfpathlineto{\pgfqpoint{1.250545in}{5.231503in}}%
\pgfpathlineto{\pgfqpoint{1.251050in}{5.223169in}}%
\pgfpathlineto{\pgfqpoint{1.251977in}{5.216146in}}%
\pgfpathlineto{\pgfqpoint{1.252314in}{5.219624in}}%
\pgfpathlineto{\pgfqpoint{1.253325in}{5.230867in}}%
\pgfpathlineto{\pgfqpoint{1.253746in}{5.226486in}}%
\pgfpathlineto{\pgfqpoint{1.254252in}{5.219957in}}%
\pgfpathlineto{\pgfqpoint{1.254841in}{5.226569in}}%
\pgfpathlineto{\pgfqpoint{1.255937in}{5.233539in}}%
\pgfpathlineto{\pgfqpoint{1.256189in}{5.231076in}}%
\pgfpathlineto{\pgfqpoint{1.257032in}{5.217217in}}%
\pgfpathlineto{\pgfqpoint{1.257621in}{5.224261in}}%
\pgfpathlineto{\pgfqpoint{1.258211in}{5.236611in}}%
\pgfpathlineto{\pgfqpoint{1.258717in}{5.224090in}}%
\pgfpathlineto{\pgfqpoint{1.259054in}{5.215883in}}%
\pgfpathlineto{\pgfqpoint{1.259643in}{5.227452in}}%
\pgfpathlineto{\pgfqpoint{1.259812in}{5.229102in}}%
\pgfpathlineto{\pgfqpoint{1.260317in}{5.223830in}}%
\pgfpathlineto{\pgfqpoint{1.260739in}{5.218889in}}%
\pgfpathlineto{\pgfqpoint{1.261244in}{5.225815in}}%
\pgfpathlineto{\pgfqpoint{1.261497in}{5.228553in}}%
\pgfpathlineto{\pgfqpoint{1.261918in}{5.221227in}}%
\pgfpathlineto{\pgfqpoint{1.262339in}{5.212887in}}%
\pgfpathlineto{\pgfqpoint{1.262760in}{5.223051in}}%
\pgfpathlineto{\pgfqpoint{1.263350in}{5.244520in}}%
\pgfpathlineto{\pgfqpoint{1.263940in}{5.228362in}}%
\pgfpathlineto{\pgfqpoint{1.265288in}{5.215402in}}%
\pgfpathlineto{\pgfqpoint{1.265540in}{5.218905in}}%
\pgfpathlineto{\pgfqpoint{1.266130in}{5.231481in}}%
\pgfpathlineto{\pgfqpoint{1.266720in}{5.221297in}}%
\pgfpathlineto{\pgfqpoint{1.266973in}{5.218656in}}%
\pgfpathlineto{\pgfqpoint{1.267562in}{5.224288in}}%
\pgfpathlineto{\pgfqpoint{1.268152in}{5.228604in}}%
\pgfpathlineto{\pgfqpoint{1.268742in}{5.225297in}}%
\pgfpathlineto{\pgfqpoint{1.269331in}{5.217767in}}%
\pgfpathlineto{\pgfqpoint{1.269668in}{5.224946in}}%
\pgfpathlineto{\pgfqpoint{1.270258in}{5.240830in}}%
\pgfpathlineto{\pgfqpoint{1.270764in}{5.226409in}}%
\pgfpathlineto{\pgfqpoint{1.271269in}{5.209119in}}%
\pgfpathlineto{\pgfqpoint{1.271775in}{5.223673in}}%
\pgfpathlineto{\pgfqpoint{1.272280in}{5.235458in}}%
\pgfpathlineto{\pgfqpoint{1.272870in}{5.224580in}}%
\pgfpathlineto{\pgfqpoint{1.273038in}{5.222856in}}%
\pgfpathlineto{\pgfqpoint{1.273544in}{5.230026in}}%
\pgfpathlineto{\pgfqpoint{1.273628in}{5.230353in}}%
\pgfpathlineto{\pgfqpoint{1.273881in}{5.227501in}}%
\pgfpathlineto{\pgfqpoint{1.274555in}{5.214703in}}%
\pgfpathlineto{\pgfqpoint{1.275060in}{5.222326in}}%
\pgfpathlineto{\pgfqpoint{1.275650in}{5.233182in}}%
\pgfpathlineto{\pgfqpoint{1.276240in}{5.226527in}}%
\pgfpathlineto{\pgfqpoint{1.276998in}{5.223204in}}%
\pgfpathlineto{\pgfqpoint{1.277419in}{5.224880in}}%
\pgfpathlineto{\pgfqpoint{1.277756in}{5.225837in}}%
\pgfpathlineto{\pgfqpoint{1.278093in}{5.222833in}}%
\pgfpathlineto{\pgfqpoint{1.278683in}{5.217760in}}%
\pgfpathlineto{\pgfqpoint{1.279104in}{5.222623in}}%
\pgfpathlineto{\pgfqpoint{1.280199in}{5.240825in}}%
\pgfpathlineto{\pgfqpoint{1.280620in}{5.230806in}}%
\pgfpathlineto{\pgfqpoint{1.281210in}{5.211816in}}%
\pgfpathlineto{\pgfqpoint{1.281715in}{5.228279in}}%
\pgfpathlineto{\pgfqpoint{1.281968in}{5.233575in}}%
\pgfpathlineto{\pgfqpoint{1.282558in}{5.221999in}}%
\pgfpathlineto{\pgfqpoint{1.282895in}{5.218902in}}%
\pgfpathlineto{\pgfqpoint{1.283569in}{5.222985in}}%
\pgfpathlineto{\pgfqpoint{1.283822in}{5.222035in}}%
\pgfpathlineto{\pgfqpoint{1.284158in}{5.220146in}}%
\pgfpathlineto{\pgfqpoint{1.284495in}{5.223168in}}%
\pgfpathlineto{\pgfqpoint{1.285254in}{5.243405in}}%
\pgfpathlineto{\pgfqpoint{1.285759in}{5.231373in}}%
\pgfpathlineto{\pgfqpoint{1.286517in}{5.208148in}}%
\pgfpathlineto{\pgfqpoint{1.287023in}{5.220669in}}%
\pgfpathlineto{\pgfqpoint{1.287697in}{5.235963in}}%
\pgfpathlineto{\pgfqpoint{1.288286in}{5.229074in}}%
\pgfpathlineto{\pgfqpoint{1.289045in}{5.213698in}}%
\pgfpathlineto{\pgfqpoint{1.289466in}{5.225100in}}%
\pgfpathlineto{\pgfqpoint{1.289803in}{5.231124in}}%
\pgfpathlineto{\pgfqpoint{1.290393in}{5.223452in}}%
\pgfpathlineto{\pgfqpoint{1.290730in}{5.219721in}}%
\pgfpathlineto{\pgfqpoint{1.291235in}{5.226038in}}%
\pgfpathlineto{\pgfqpoint{1.291825in}{5.228841in}}%
\pgfpathlineto{\pgfqpoint{1.292330in}{5.225907in}}%
\pgfpathlineto{\pgfqpoint{1.292920in}{5.218962in}}%
\pgfpathlineto{\pgfqpoint{1.293341in}{5.224930in}}%
\pgfpathlineto{\pgfqpoint{1.294015in}{5.234995in}}%
\pgfpathlineto{\pgfqpoint{1.294605in}{5.229332in}}%
\pgfpathlineto{\pgfqpoint{1.296290in}{5.209391in}}%
\pgfpathlineto{\pgfqpoint{1.296711in}{5.217953in}}%
\pgfpathlineto{\pgfqpoint{1.297216in}{5.225466in}}%
\pgfpathlineto{\pgfqpoint{1.297975in}{5.222891in}}%
\pgfpathlineto{\pgfqpoint{1.298312in}{5.227883in}}%
\pgfpathlineto{\pgfqpoint{1.298901in}{5.243972in}}%
\pgfpathlineto{\pgfqpoint{1.299407in}{5.232244in}}%
\pgfpathlineto{\pgfqpoint{1.299996in}{5.214341in}}%
\pgfpathlineto{\pgfqpoint{1.300502in}{5.228385in}}%
\pgfpathlineto{\pgfqpoint{1.300923in}{5.238777in}}%
\pgfpathlineto{\pgfqpoint{1.301429in}{5.224782in}}%
\pgfpathlineto{\pgfqpoint{1.302018in}{5.211211in}}%
\pgfpathlineto{\pgfqpoint{1.302524in}{5.223354in}}%
\pgfpathlineto{\pgfqpoint{1.302861in}{5.230005in}}%
\pgfpathlineto{\pgfqpoint{1.303450in}{5.219495in}}%
\pgfpathlineto{\pgfqpoint{1.303703in}{5.215891in}}%
\pgfpathlineto{\pgfqpoint{1.304293in}{5.223871in}}%
\pgfpathlineto{\pgfqpoint{1.304630in}{5.226428in}}%
\pgfpathlineto{\pgfqpoint{1.305220in}{5.222422in}}%
\pgfpathlineto{\pgfqpoint{1.305388in}{5.222178in}}%
\pgfpathlineto{\pgfqpoint{1.305641in}{5.223858in}}%
\pgfpathlineto{\pgfqpoint{1.306736in}{5.238850in}}%
\pgfpathlineto{\pgfqpoint{1.307241in}{5.231175in}}%
\pgfpathlineto{\pgfqpoint{1.307831in}{5.217521in}}%
\pgfpathlineto{\pgfqpoint{1.308337in}{5.229602in}}%
\pgfpathlineto{\pgfqpoint{1.308674in}{5.234720in}}%
\pgfpathlineto{\pgfqpoint{1.309179in}{5.222746in}}%
\pgfpathlineto{\pgfqpoint{1.310022in}{5.210739in}}%
\pgfpathlineto{\pgfqpoint{1.310359in}{5.217736in}}%
\pgfpathlineto{\pgfqpoint{1.311032in}{5.237530in}}%
\pgfpathlineto{\pgfqpoint{1.311538in}{5.226154in}}%
\pgfpathlineto{\pgfqpoint{1.312886in}{5.212662in}}%
\pgfpathlineto{\pgfqpoint{1.313054in}{5.213757in}}%
\pgfpathlineto{\pgfqpoint{1.313897in}{5.244654in}}%
\pgfpathlineto{\pgfqpoint{1.314487in}{5.224978in}}%
\pgfpathlineto{\pgfqpoint{1.314823in}{5.218146in}}%
\pgfpathlineto{\pgfqpoint{1.315497in}{5.228026in}}%
\pgfpathlineto{\pgfqpoint{1.315834in}{5.230458in}}%
\pgfpathlineto{\pgfqpoint{1.316340in}{5.226688in}}%
\pgfpathlineto{\pgfqpoint{1.317351in}{5.221660in}}%
\pgfpathlineto{\pgfqpoint{1.317688in}{5.222829in}}%
\pgfpathlineto{\pgfqpoint{1.318278in}{5.228148in}}%
\pgfpathlineto{\pgfqpoint{1.318699in}{5.223715in}}%
\pgfpathlineto{\pgfqpoint{1.319204in}{5.217633in}}%
\pgfpathlineto{\pgfqpoint{1.319710in}{5.224598in}}%
\pgfpathlineto{\pgfqpoint{1.320215in}{5.235646in}}%
\pgfpathlineto{\pgfqpoint{1.320805in}{5.224716in}}%
\pgfpathlineto{\pgfqpoint{1.321142in}{5.221062in}}%
\pgfpathlineto{\pgfqpoint{1.321647in}{5.228803in}}%
\pgfpathlineto{\pgfqpoint{1.321984in}{5.232088in}}%
\pgfpathlineto{\pgfqpoint{1.322490in}{5.225681in}}%
\pgfpathlineto{\pgfqpoint{1.323248in}{5.219420in}}%
\pgfpathlineto{\pgfqpoint{1.323838in}{5.221896in}}%
\pgfpathlineto{\pgfqpoint{1.324596in}{5.227504in}}%
\pgfpathlineto{\pgfqpoint{1.325186in}{5.223355in}}%
\pgfpathlineto{\pgfqpoint{1.325354in}{5.222872in}}%
\pgfpathlineto{\pgfqpoint{1.325607in}{5.225136in}}%
\pgfpathlineto{\pgfqpoint{1.326197in}{5.234378in}}%
\pgfpathlineto{\pgfqpoint{1.326702in}{5.227420in}}%
\pgfpathlineto{\pgfqpoint{1.327376in}{5.214833in}}%
\pgfpathlineto{\pgfqpoint{1.327881in}{5.224580in}}%
\pgfpathlineto{\pgfqpoint{1.328387in}{5.236867in}}%
\pgfpathlineto{\pgfqpoint{1.328892in}{5.223540in}}%
\pgfpathlineto{\pgfqpoint{1.329314in}{5.211437in}}%
\pgfpathlineto{\pgfqpoint{1.329988in}{5.223079in}}%
\pgfpathlineto{\pgfqpoint{1.330746in}{5.236665in}}%
\pgfpathlineto{\pgfqpoint{1.331335in}{5.227876in}}%
\pgfpathlineto{\pgfqpoint{1.331925in}{5.214174in}}%
\pgfpathlineto{\pgfqpoint{1.332346in}{5.226955in}}%
\pgfpathlineto{\pgfqpoint{1.332852in}{5.240825in}}%
\pgfpathlineto{\pgfqpoint{1.333357in}{5.226378in}}%
\pgfpathlineto{\pgfqpoint{1.334031in}{5.209973in}}%
\pgfpathlineto{\pgfqpoint{1.334621in}{5.219688in}}%
\pgfpathlineto{\pgfqpoint{1.335885in}{5.233114in}}%
\pgfpathlineto{\pgfqpoint{1.336222in}{5.231067in}}%
\pgfpathlineto{\pgfqpoint{1.336980in}{5.218113in}}%
\pgfpathlineto{\pgfqpoint{1.337485in}{5.228294in}}%
\pgfpathlineto{\pgfqpoint{1.337991in}{5.237611in}}%
\pgfpathlineto{\pgfqpoint{1.338496in}{5.225672in}}%
\pgfpathlineto{\pgfqpoint{1.339002in}{5.215675in}}%
\pgfpathlineto{\pgfqpoint{1.339676in}{5.223045in}}%
\pgfpathlineto{\pgfqpoint{1.340181in}{5.226842in}}%
\pgfpathlineto{\pgfqpoint{1.340602in}{5.221831in}}%
\pgfpathlineto{\pgfqpoint{1.341024in}{5.215627in}}%
\pgfpathlineto{\pgfqpoint{1.341445in}{5.222955in}}%
\pgfpathlineto{\pgfqpoint{1.342034in}{5.237445in}}%
\pgfpathlineto{\pgfqpoint{1.342708in}{5.230129in}}%
\pgfpathlineto{\pgfqpoint{1.343551in}{5.220338in}}%
\pgfpathlineto{\pgfqpoint{1.344056in}{5.226479in}}%
\pgfpathlineto{\pgfqpoint{1.344309in}{5.227871in}}%
\pgfpathlineto{\pgfqpoint{1.344983in}{5.224918in}}%
\pgfpathlineto{\pgfqpoint{1.345404in}{5.223028in}}%
\pgfpathlineto{\pgfqpoint{1.345825in}{5.225995in}}%
\pgfpathlineto{\pgfqpoint{1.345910in}{5.226259in}}%
\pgfpathlineto{\pgfqpoint{1.346162in}{5.224149in}}%
\pgfpathlineto{\pgfqpoint{1.346668in}{5.215757in}}%
\pgfpathlineto{\pgfqpoint{1.347089in}{5.223980in}}%
\pgfpathlineto{\pgfqpoint{1.347510in}{5.235346in}}%
\pgfpathlineto{\pgfqpoint{1.348100in}{5.221450in}}%
\pgfpathlineto{\pgfqpoint{1.348353in}{5.217222in}}%
\pgfpathlineto{\pgfqpoint{1.348943in}{5.227162in}}%
\pgfpathlineto{\pgfqpoint{1.349280in}{5.229434in}}%
\pgfpathlineto{\pgfqpoint{1.349953in}{5.226480in}}%
\pgfpathlineto{\pgfqpoint{1.350375in}{5.226148in}}%
\pgfpathlineto{\pgfqpoint{1.350627in}{5.227149in}}%
\pgfpathlineto{\pgfqpoint{1.351049in}{5.229768in}}%
\pgfpathlineto{\pgfqpoint{1.351554in}{5.225778in}}%
\pgfpathlineto{\pgfqpoint{1.351723in}{5.225057in}}%
\pgfpathlineto{\pgfqpoint{1.352144in}{5.228744in}}%
\pgfpathlineto{\pgfqpoint{1.352228in}{5.229055in}}%
\pgfpathlineto{\pgfqpoint{1.352481in}{5.226835in}}%
\pgfpathlineto{\pgfqpoint{1.353155in}{5.208678in}}%
\pgfpathlineto{\pgfqpoint{1.353576in}{5.224008in}}%
\pgfpathlineto{\pgfqpoint{1.354081in}{5.240856in}}%
\pgfpathlineto{\pgfqpoint{1.354587in}{5.226212in}}%
\pgfpathlineto{\pgfqpoint{1.355092in}{5.212275in}}%
\pgfpathlineto{\pgfqpoint{1.355682in}{5.222092in}}%
\pgfpathlineto{\pgfqpoint{1.356272in}{5.236779in}}%
\pgfpathlineto{\pgfqpoint{1.356862in}{5.225100in}}%
\pgfpathlineto{\pgfqpoint{1.357283in}{5.220379in}}%
\pgfpathlineto{\pgfqpoint{1.357872in}{5.225721in}}%
\pgfpathlineto{\pgfqpoint{1.359136in}{5.229392in}}%
\pgfpathlineto{\pgfqpoint{1.359473in}{5.227081in}}%
\pgfpathlineto{\pgfqpoint{1.360063in}{5.216705in}}%
\pgfpathlineto{\pgfqpoint{1.360568in}{5.225821in}}%
\pgfpathlineto{\pgfqpoint{1.360905in}{5.230309in}}%
\pgfpathlineto{\pgfqpoint{1.361748in}{5.227360in}}%
\pgfpathlineto{\pgfqpoint{1.362337in}{5.221903in}}%
\pgfpathlineto{\pgfqpoint{1.362506in}{5.220974in}}%
\pgfpathlineto{\pgfqpoint{1.363011in}{5.224475in}}%
\pgfpathlineto{\pgfqpoint{1.363854in}{5.234078in}}%
\pgfpathlineto{\pgfqpoint{1.364191in}{5.227120in}}%
\pgfpathlineto{\pgfqpoint{1.364696in}{5.218340in}}%
\pgfpathlineto{\pgfqpoint{1.365454in}{5.220515in}}%
\pgfpathlineto{\pgfqpoint{1.366297in}{5.228481in}}%
\pgfpathlineto{\pgfqpoint{1.366718in}{5.222923in}}%
\pgfpathlineto{\pgfqpoint{1.367055in}{5.219795in}}%
\pgfpathlineto{\pgfqpoint{1.367476in}{5.227312in}}%
\pgfpathlineto{\pgfqpoint{1.367982in}{5.240819in}}%
\pgfpathlineto{\pgfqpoint{1.368487in}{5.229247in}}%
\pgfpathlineto{\pgfqpoint{1.368908in}{5.220509in}}%
\pgfpathlineto{\pgfqpoint{1.369582in}{5.227409in}}%
\pgfpathlineto{\pgfqpoint{1.369919in}{5.221420in}}%
\pgfpathlineto{\pgfqpoint{1.370341in}{5.211226in}}%
\pgfpathlineto{\pgfqpoint{1.370846in}{5.226955in}}%
\pgfpathlineto{\pgfqpoint{1.371267in}{5.236763in}}%
\pgfpathlineto{\pgfqpoint{1.371773in}{5.221485in}}%
\pgfpathlineto{\pgfqpoint{1.372278in}{5.211709in}}%
\pgfpathlineto{\pgfqpoint{1.372868in}{5.221984in}}%
\pgfpathlineto{\pgfqpoint{1.373795in}{5.240922in}}%
\pgfpathlineto{\pgfqpoint{1.374300in}{5.231913in}}%
\pgfpathlineto{\pgfqpoint{1.374806in}{5.218032in}}%
\pgfpathlineto{\pgfqpoint{1.375480in}{5.229281in}}%
\pgfpathlineto{\pgfqpoint{1.375564in}{5.229770in}}%
\pgfpathlineto{\pgfqpoint{1.375901in}{5.226162in}}%
\pgfpathlineto{\pgfqpoint{1.376490in}{5.217323in}}%
\pgfpathlineto{\pgfqpoint{1.376996in}{5.224348in}}%
\pgfpathlineto{\pgfqpoint{1.377333in}{5.228594in}}%
\pgfpathlineto{\pgfqpoint{1.377838in}{5.222327in}}%
\pgfpathlineto{\pgfqpoint{1.378175in}{5.218383in}}%
\pgfpathlineto{\pgfqpoint{1.378765in}{5.223748in}}%
\pgfpathlineto{\pgfqpoint{1.380281in}{5.242111in}}%
\pgfpathlineto{\pgfqpoint{1.380618in}{5.233483in}}%
\pgfpathlineto{\pgfqpoint{1.381292in}{5.208354in}}%
\pgfpathlineto{\pgfqpoint{1.381882in}{5.222974in}}%
\pgfpathlineto{\pgfqpoint{1.382556in}{5.233331in}}%
\pgfpathlineto{\pgfqpoint{1.383146in}{5.228514in}}%
\pgfpathlineto{\pgfqpoint{1.383567in}{5.224792in}}%
\pgfpathlineto{\pgfqpoint{1.384157in}{5.214869in}}%
\pgfpathlineto{\pgfqpoint{1.384662in}{5.224141in}}%
\pgfpathlineto{\pgfqpoint{1.385083in}{5.232891in}}%
\pgfpathlineto{\pgfqpoint{1.385673in}{5.220894in}}%
\pgfpathlineto{\pgfqpoint{1.385842in}{5.218923in}}%
\pgfpathlineto{\pgfqpoint{1.386347in}{5.227712in}}%
\pgfpathlineto{\pgfqpoint{1.386768in}{5.234857in}}%
\pgfpathlineto{\pgfqpoint{1.387274in}{5.224872in}}%
\pgfpathlineto{\pgfqpoint{1.387695in}{5.217346in}}%
\pgfpathlineto{\pgfqpoint{1.388369in}{5.224406in}}%
\pgfpathlineto{\pgfqpoint{1.388790in}{5.226384in}}%
\pgfpathlineto{\pgfqpoint{1.389296in}{5.224077in}}%
\pgfpathlineto{\pgfqpoint{1.389633in}{5.222430in}}%
\pgfpathlineto{\pgfqpoint{1.389970in}{5.225201in}}%
\pgfpathlineto{\pgfqpoint{1.391149in}{5.235313in}}%
\pgfpathlineto{\pgfqpoint{1.391402in}{5.233002in}}%
\pgfpathlineto{\pgfqpoint{1.392160in}{5.209191in}}%
\pgfpathlineto{\pgfqpoint{1.392665in}{5.223363in}}%
\pgfpathlineto{\pgfqpoint{1.393171in}{5.234940in}}%
\pgfpathlineto{\pgfqpoint{1.393761in}{5.225545in}}%
\pgfpathlineto{\pgfqpoint{1.395277in}{5.214876in}}%
\pgfpathlineto{\pgfqpoint{1.395530in}{5.217842in}}%
\pgfpathlineto{\pgfqpoint{1.396372in}{5.238862in}}%
\pgfpathlineto{\pgfqpoint{1.396962in}{5.226651in}}%
\pgfpathlineto{\pgfqpoint{1.397299in}{5.219400in}}%
\pgfpathlineto{\pgfqpoint{1.398226in}{5.221027in}}%
\pgfpathlineto{\pgfqpoint{1.398478in}{5.218477in}}%
\pgfpathlineto{\pgfqpoint{1.398900in}{5.224758in}}%
\pgfpathlineto{\pgfqpoint{1.399405in}{5.235470in}}%
\pgfpathlineto{\pgfqpoint{1.399910in}{5.226614in}}%
\pgfpathlineto{\pgfqpoint{1.400332in}{5.218608in}}%
\pgfpathlineto{\pgfqpoint{1.400921in}{5.229340in}}%
\pgfpathlineto{\pgfqpoint{1.401174in}{5.232180in}}%
\pgfpathlineto{\pgfqpoint{1.401595in}{5.223952in}}%
\pgfpathlineto{\pgfqpoint{1.401932in}{5.218734in}}%
\pgfpathlineto{\pgfqpoint{1.402522in}{5.226903in}}%
\pgfpathlineto{\pgfqpoint{1.402606in}{5.227378in}}%
\pgfpathlineto{\pgfqpoint{1.402859in}{5.224653in}}%
\pgfpathlineto{\pgfqpoint{1.403280in}{5.216893in}}%
\pgfpathlineto{\pgfqpoint{1.403786in}{5.225981in}}%
\pgfpathlineto{\pgfqpoint{1.404207in}{5.231975in}}%
\pgfpathlineto{\pgfqpoint{1.404712in}{5.224467in}}%
\pgfpathlineto{\pgfqpoint{1.404965in}{5.222218in}}%
\pgfpathlineto{\pgfqpoint{1.405386in}{5.228942in}}%
\pgfpathlineto{\pgfqpoint{1.405723in}{5.234831in}}%
\pgfpathlineto{\pgfqpoint{1.406229in}{5.225709in}}%
\pgfpathlineto{\pgfqpoint{1.406734in}{5.215095in}}%
\pgfpathlineto{\pgfqpoint{1.407324in}{5.225074in}}%
\pgfpathlineto{\pgfqpoint{1.408503in}{5.234348in}}%
\pgfpathlineto{\pgfqpoint{1.408840in}{5.229639in}}%
\pgfpathlineto{\pgfqpoint{1.409514in}{5.212923in}}%
\pgfpathlineto{\pgfqpoint{1.409936in}{5.224018in}}%
\pgfpathlineto{\pgfqpoint{1.410357in}{5.236813in}}%
\pgfpathlineto{\pgfqpoint{1.410947in}{5.224596in}}%
\pgfpathlineto{\pgfqpoint{1.412631in}{5.213546in}}%
\pgfpathlineto{\pgfqpoint{1.412800in}{5.213001in}}%
\pgfpathlineto{\pgfqpoint{1.413053in}{5.216921in}}%
\pgfpathlineto{\pgfqpoint{1.413811in}{5.236084in}}%
\pgfpathlineto{\pgfqpoint{1.414316in}{5.221511in}}%
\pgfpathlineto{\pgfqpoint{1.414653in}{5.212504in}}%
\pgfpathlineto{\pgfqpoint{1.415159in}{5.230203in}}%
\pgfpathlineto{\pgfqpoint{1.415748in}{5.247192in}}%
\pgfpathlineto{\pgfqpoint{1.416254in}{5.233771in}}%
\pgfpathlineto{\pgfqpoint{1.417012in}{5.211065in}}%
\pgfpathlineto{\pgfqpoint{1.417518in}{5.221364in}}%
\pgfpathlineto{\pgfqpoint{1.417939in}{5.227537in}}%
\pgfpathlineto{\pgfqpoint{1.418781in}{5.225515in}}%
\pgfpathlineto{\pgfqpoint{1.419961in}{5.227374in}}%
\pgfpathlineto{\pgfqpoint{1.419287in}{5.224530in}}%
\pgfpathlineto{\pgfqpoint{1.420129in}{5.226619in}}%
\pgfpathlineto{\pgfqpoint{1.420466in}{5.224118in}}%
\pgfpathlineto{\pgfqpoint{1.420972in}{5.228467in}}%
\pgfpathlineto{\pgfqpoint{1.421056in}{5.228594in}}%
\pgfpathlineto{\pgfqpoint{1.421224in}{5.227406in}}%
\pgfpathlineto{\pgfqpoint{1.421983in}{5.214964in}}%
\pgfpathlineto{\pgfqpoint{1.422488in}{5.222516in}}%
\pgfpathlineto{\pgfqpoint{1.423415in}{5.240258in}}%
\pgfpathlineto{\pgfqpoint{1.423836in}{5.232438in}}%
\pgfpathlineto{\pgfqpoint{1.424426in}{5.216162in}}%
\pgfpathlineto{\pgfqpoint{1.425015in}{5.227925in}}%
\pgfpathlineto{\pgfqpoint{1.425184in}{5.229266in}}%
\pgfpathlineto{\pgfqpoint{1.425605in}{5.221347in}}%
\pgfpathlineto{\pgfqpoint{1.425942in}{5.217391in}}%
\pgfpathlineto{\pgfqpoint{1.426447in}{5.224897in}}%
\pgfpathlineto{\pgfqpoint{1.428385in}{5.235525in}}%
\pgfpathlineto{\pgfqpoint{1.428722in}{5.229683in}}%
\pgfpathlineto{\pgfqpoint{1.429480in}{5.201612in}}%
\pgfpathlineto{\pgfqpoint{1.429902in}{5.218999in}}%
\pgfpathlineto{\pgfqpoint{1.430407in}{5.237742in}}%
\pgfpathlineto{\pgfqpoint{1.431165in}{5.230081in}}%
\pgfpathlineto{\pgfqpoint{1.431502in}{5.234112in}}%
\pgfpathlineto{\pgfqpoint{1.432008in}{5.226983in}}%
\pgfpathlineto{\pgfqpoint{1.432429in}{5.220071in}}%
\pgfpathlineto{\pgfqpoint{1.432850in}{5.229209in}}%
\pgfpathlineto{\pgfqpoint{1.433187in}{5.237058in}}%
\pgfpathlineto{\pgfqpoint{1.433608in}{5.224955in}}%
\pgfpathlineto{\pgfqpoint{1.434114in}{5.207728in}}%
\pgfpathlineto{\pgfqpoint{1.434703in}{5.221862in}}%
\pgfpathlineto{\pgfqpoint{1.435125in}{5.229889in}}%
\pgfpathlineto{\pgfqpoint{1.435714in}{5.219700in}}%
\pgfpathlineto{\pgfqpoint{1.435883in}{5.218800in}}%
\pgfpathlineto{\pgfqpoint{1.436136in}{5.223241in}}%
\pgfpathlineto{\pgfqpoint{1.436725in}{5.240554in}}%
\pgfpathlineto{\pgfqpoint{1.437147in}{5.226207in}}%
\pgfpathlineto{\pgfqpoint{1.437484in}{5.216196in}}%
\pgfpathlineto{\pgfqpoint{1.438073in}{5.234268in}}%
\pgfpathlineto{\pgfqpoint{1.438242in}{5.236608in}}%
\pgfpathlineto{\pgfqpoint{1.438663in}{5.225746in}}%
\pgfpathlineto{\pgfqpoint{1.439168in}{5.214609in}}%
\pgfpathlineto{\pgfqpoint{1.439758in}{5.224281in}}%
\pgfpathlineto{\pgfqpoint{1.440264in}{5.229602in}}%
\pgfpathlineto{\pgfqpoint{1.440853in}{5.224977in}}%
\pgfpathlineto{\pgfqpoint{1.441275in}{5.222099in}}%
\pgfpathlineto{\pgfqpoint{1.441864in}{5.225454in}}%
\pgfpathlineto{\pgfqpoint{1.442454in}{5.223828in}}%
\pgfpathlineto{\pgfqpoint{1.442707in}{5.225666in}}%
\pgfpathlineto{\pgfqpoint{1.443296in}{5.232804in}}%
\pgfpathlineto{\pgfqpoint{1.443718in}{5.226726in}}%
\pgfpathlineto{\pgfqpoint{1.444981in}{5.218838in}}%
\pgfpathlineto{\pgfqpoint{1.445066in}{5.218954in}}%
\pgfpathlineto{\pgfqpoint{1.445824in}{5.232365in}}%
\pgfpathlineto{\pgfqpoint{1.446076in}{5.234537in}}%
\pgfpathlineto{\pgfqpoint{1.446666in}{5.229301in}}%
\pgfpathlineto{\pgfqpoint{1.447340in}{5.218056in}}%
\pgfpathlineto{\pgfqpoint{1.447930in}{5.225534in}}%
\pgfpathlineto{\pgfqpoint{1.448098in}{5.226555in}}%
\pgfpathlineto{\pgfqpoint{1.448604in}{5.221937in}}%
\pgfpathlineto{\pgfqpoint{1.448772in}{5.220927in}}%
\pgfpathlineto{\pgfqpoint{1.449194in}{5.225456in}}%
\pgfpathlineto{\pgfqpoint{1.449699in}{5.231564in}}%
\pgfpathlineto{\pgfqpoint{1.450457in}{5.229076in}}%
\pgfpathlineto{\pgfqpoint{1.451300in}{5.223493in}}%
\pgfpathlineto{\pgfqpoint{1.451805in}{5.220198in}}%
\pgfpathlineto{\pgfqpoint{1.452395in}{5.222623in}}%
\pgfpathlineto{\pgfqpoint{1.452816in}{5.224783in}}%
\pgfpathlineto{\pgfqpoint{1.453237in}{5.221746in}}%
\pgfpathlineto{\pgfqpoint{1.453490in}{5.220428in}}%
\pgfpathlineto{\pgfqpoint{1.453995in}{5.224218in}}%
\pgfpathlineto{\pgfqpoint{1.455259in}{5.239047in}}%
\pgfpathlineto{\pgfqpoint{1.455680in}{5.232083in}}%
\pgfpathlineto{\pgfqpoint{1.456354in}{5.209898in}}%
\pgfpathlineto{\pgfqpoint{1.456860in}{5.223531in}}%
\pgfpathlineto{\pgfqpoint{1.457197in}{5.231505in}}%
\pgfpathlineto{\pgfqpoint{1.457786in}{5.217563in}}%
\pgfpathlineto{\pgfqpoint{1.457955in}{5.215629in}}%
\pgfpathlineto{\pgfqpoint{1.458376in}{5.223125in}}%
\pgfpathlineto{\pgfqpoint{1.458882in}{5.231791in}}%
\pgfpathlineto{\pgfqpoint{1.459471in}{5.224247in}}%
\pgfpathlineto{\pgfqpoint{1.459640in}{5.223489in}}%
\pgfpathlineto{\pgfqpoint{1.460061in}{5.226551in}}%
\pgfpathlineto{\pgfqpoint{1.460482in}{5.230738in}}%
\pgfpathlineto{\pgfqpoint{1.461240in}{5.227535in}}%
\pgfpathlineto{\pgfqpoint{1.461409in}{5.227733in}}%
\pgfpathlineto{\pgfqpoint{1.461662in}{5.226618in}}%
\pgfpathlineto{\pgfqpoint{1.463262in}{5.216301in}}%
\pgfpathlineto{\pgfqpoint{1.463515in}{5.219645in}}%
\pgfpathlineto{\pgfqpoint{1.464105in}{5.232572in}}%
\pgfpathlineto{\pgfqpoint{1.464695in}{5.222106in}}%
\pgfpathlineto{\pgfqpoint{1.464863in}{5.220935in}}%
\pgfpathlineto{\pgfqpoint{1.465284in}{5.226335in}}%
\pgfpathlineto{\pgfqpoint{1.466042in}{5.224835in}}%
\pgfpathlineto{\pgfqpoint{1.466716in}{5.231803in}}%
\pgfpathlineto{\pgfqpoint{1.466801in}{5.231821in}}%
\pgfpathlineto{\pgfqpoint{1.466885in}{5.231283in}}%
\pgfpathlineto{\pgfqpoint{1.467812in}{5.215184in}}%
\pgfpathlineto{\pgfqpoint{1.468317in}{5.225167in}}%
\pgfpathlineto{\pgfqpoint{1.468738in}{5.231812in}}%
\pgfpathlineto{\pgfqpoint{1.469412in}{5.225575in}}%
\pgfpathlineto{\pgfqpoint{1.470086in}{5.232298in}}%
\pgfpathlineto{\pgfqpoint{1.470507in}{5.227050in}}%
\pgfpathlineto{\pgfqpoint{1.471097in}{5.217933in}}%
\pgfpathlineto{\pgfqpoint{1.471518in}{5.226373in}}%
\pgfpathlineto{\pgfqpoint{1.471855in}{5.233858in}}%
\pgfpathlineto{\pgfqpoint{1.472361in}{5.221352in}}%
\pgfpathlineto{\pgfqpoint{1.472782in}{5.213290in}}%
\pgfpathlineto{\pgfqpoint{1.473287in}{5.223985in}}%
\pgfpathlineto{\pgfqpoint{1.473709in}{5.231950in}}%
\pgfpathlineto{\pgfqpoint{1.474214in}{5.221608in}}%
\pgfpathlineto{\pgfqpoint{1.474635in}{5.211782in}}%
\pgfpathlineto{\pgfqpoint{1.475141in}{5.227964in}}%
\pgfpathlineto{\pgfqpoint{1.475646in}{5.241263in}}%
\pgfpathlineto{\pgfqpoint{1.476236in}{5.231548in}}%
\pgfpathlineto{\pgfqpoint{1.477921in}{5.218022in}}%
\pgfpathlineto{\pgfqpoint{1.478174in}{5.218726in}}%
\pgfpathlineto{\pgfqpoint{1.478932in}{5.232645in}}%
\pgfpathlineto{\pgfqpoint{1.479353in}{5.239076in}}%
\pgfpathlineto{\pgfqpoint{1.479859in}{5.230648in}}%
\pgfpathlineto{\pgfqpoint{1.480785in}{5.220241in}}%
\pgfpathlineto{\pgfqpoint{1.481206in}{5.222974in}}%
\pgfpathlineto{\pgfqpoint{1.481880in}{5.230300in}}%
\pgfpathlineto{\pgfqpoint{1.482386in}{5.225580in}}%
\pgfpathlineto{\pgfqpoint{1.483060in}{5.214114in}}%
\pgfpathlineto{\pgfqpoint{1.483650in}{5.220490in}}%
\pgfpathlineto{\pgfqpoint{1.485082in}{5.234383in}}%
\pgfpathlineto{\pgfqpoint{1.485503in}{5.229483in}}%
\pgfpathlineto{\pgfqpoint{1.486177in}{5.215385in}}%
\pgfpathlineto{\pgfqpoint{1.486598in}{5.225301in}}%
\pgfpathlineto{\pgfqpoint{1.487019in}{5.237980in}}%
\pgfpathlineto{\pgfqpoint{1.487525in}{5.223406in}}%
\pgfpathlineto{\pgfqpoint{1.487693in}{5.219426in}}%
\pgfpathlineto{\pgfqpoint{1.488114in}{5.233324in}}%
\pgfpathlineto{\pgfqpoint{1.488451in}{5.242465in}}%
\pgfpathlineto{\pgfqpoint{1.488957in}{5.220851in}}%
\pgfpathlineto{\pgfqpoint{1.489294in}{5.209233in}}%
\pgfpathlineto{\pgfqpoint{1.489884in}{5.224324in}}%
\pgfpathlineto{\pgfqpoint{1.490221in}{5.231409in}}%
\pgfpathlineto{\pgfqpoint{1.490726in}{5.217268in}}%
\pgfpathlineto{\pgfqpoint{1.490895in}{5.213893in}}%
\pgfpathlineto{\pgfqpoint{1.491400in}{5.226655in}}%
\pgfpathlineto{\pgfqpoint{1.491653in}{5.230166in}}%
\pgfpathlineto{\pgfqpoint{1.492074in}{5.218993in}}%
\pgfpathlineto{\pgfqpoint{1.492327in}{5.214887in}}%
\pgfpathlineto{\pgfqpoint{1.492748in}{5.227447in}}%
\pgfpathlineto{\pgfqpoint{1.493169in}{5.241144in}}%
\pgfpathlineto{\pgfqpoint{1.493675in}{5.225167in}}%
\pgfpathlineto{\pgfqpoint{1.494012in}{5.215664in}}%
\pgfpathlineto{\pgfqpoint{1.494517in}{5.235177in}}%
\pgfpathlineto{\pgfqpoint{1.494854in}{5.245814in}}%
\pgfpathlineto{\pgfqpoint{1.495275in}{5.225221in}}%
\pgfpathlineto{\pgfqpoint{1.495697in}{5.207071in}}%
\pgfpathlineto{\pgfqpoint{1.496286in}{5.227076in}}%
\pgfpathlineto{\pgfqpoint{1.496623in}{5.233094in}}%
\pgfpathlineto{\pgfqpoint{1.497129in}{5.220754in}}%
\pgfpathlineto{\pgfqpoint{1.497381in}{5.217032in}}%
\pgfpathlineto{\pgfqpoint{1.497887in}{5.225541in}}%
\pgfpathlineto{\pgfqpoint{1.498224in}{5.230005in}}%
\pgfpathlineto{\pgfqpoint{1.498645in}{5.220354in}}%
\pgfpathlineto{\pgfqpoint{1.498898in}{5.217662in}}%
\pgfpathlineto{\pgfqpoint{1.499319in}{5.227175in}}%
\pgfpathlineto{\pgfqpoint{1.499740in}{5.236124in}}%
\pgfpathlineto{\pgfqpoint{1.500246in}{5.225912in}}%
\pgfpathlineto{\pgfqpoint{1.500751in}{5.213503in}}%
\pgfpathlineto{\pgfqpoint{1.501257in}{5.227125in}}%
\pgfpathlineto{\pgfqpoint{1.502436in}{5.234355in}}%
\pgfpathlineto{\pgfqpoint{1.502605in}{5.233511in}}%
\pgfpathlineto{\pgfqpoint{1.503615in}{5.208446in}}%
\pgfpathlineto{\pgfqpoint{1.504121in}{5.223026in}}%
\pgfpathlineto{\pgfqpoint{1.504626in}{5.246088in}}%
\pgfpathlineto{\pgfqpoint{1.505132in}{5.224269in}}%
\pgfpathlineto{\pgfqpoint{1.505469in}{5.212350in}}%
\pgfpathlineto{\pgfqpoint{1.506059in}{5.231054in}}%
\pgfpathlineto{\pgfqpoint{1.506227in}{5.233551in}}%
\pgfpathlineto{\pgfqpoint{1.506648in}{5.223729in}}%
\pgfpathlineto{\pgfqpoint{1.506985in}{5.217091in}}%
\pgfpathlineto{\pgfqpoint{1.507491in}{5.231572in}}%
\pgfpathlineto{\pgfqpoint{1.507828in}{5.239834in}}%
\pgfpathlineto{\pgfqpoint{1.508333in}{5.225553in}}%
\pgfpathlineto{\pgfqpoint{1.508839in}{5.210956in}}%
\pgfpathlineto{\pgfqpoint{1.509428in}{5.223408in}}%
\pgfpathlineto{\pgfqpoint{1.509513in}{5.223765in}}%
\pgfpathlineto{\pgfqpoint{1.509850in}{5.220980in}}%
\pgfpathlineto{\pgfqpoint{1.510271in}{5.217014in}}%
\pgfpathlineto{\pgfqpoint{1.510776in}{5.222305in}}%
\pgfpathlineto{\pgfqpoint{1.512124in}{5.238376in}}%
\pgfpathlineto{\pgfqpoint{1.512461in}{5.231921in}}%
\pgfpathlineto{\pgfqpoint{1.513051in}{5.213043in}}%
\pgfpathlineto{\pgfqpoint{1.513556in}{5.228511in}}%
\pgfpathlineto{\pgfqpoint{1.514062in}{5.244192in}}%
\pgfpathlineto{\pgfqpoint{1.514567in}{5.225265in}}%
\pgfpathlineto{\pgfqpoint{1.514988in}{5.215909in}}%
\pgfpathlineto{\pgfqpoint{1.515494in}{5.228419in}}%
\pgfpathlineto{\pgfqpoint{1.515915in}{5.235742in}}%
\pgfpathlineto{\pgfqpoint{1.516421in}{5.226602in}}%
\pgfpathlineto{\pgfqpoint{1.517937in}{5.207823in}}%
\pgfpathlineto{\pgfqpoint{1.518190in}{5.211515in}}%
\pgfpathlineto{\pgfqpoint{1.518864in}{5.234642in}}%
\pgfpathlineto{\pgfqpoint{1.519369in}{5.217922in}}%
\pgfpathlineto{\pgfqpoint{1.519622in}{5.214214in}}%
\pgfpathlineto{\pgfqpoint{1.520127in}{5.225672in}}%
\pgfpathlineto{\pgfqpoint{1.521475in}{5.232987in}}%
\pgfpathlineto{\pgfqpoint{1.522065in}{5.236885in}}%
\pgfpathlineto{\pgfqpoint{1.522318in}{5.233394in}}%
\pgfpathlineto{\pgfqpoint{1.523076in}{5.214231in}}%
\pgfpathlineto{\pgfqpoint{1.523666in}{5.225321in}}%
\pgfpathlineto{\pgfqpoint{1.524171in}{5.236484in}}%
\pgfpathlineto{\pgfqpoint{1.524592in}{5.224998in}}%
\pgfpathlineto{\pgfqpoint{1.525014in}{5.214830in}}%
\pgfpathlineto{\pgfqpoint{1.525772in}{5.222122in}}%
\pgfpathlineto{\pgfqpoint{1.525940in}{5.222241in}}%
\pgfpathlineto{\pgfqpoint{1.526025in}{5.222673in}}%
\pgfpathlineto{\pgfqpoint{1.526698in}{5.232341in}}%
\pgfpathlineto{\pgfqpoint{1.527120in}{5.223451in}}%
\pgfpathlineto{\pgfqpoint{1.527625in}{5.212925in}}%
\pgfpathlineto{\pgfqpoint{1.528215in}{5.221291in}}%
\pgfpathlineto{\pgfqpoint{1.528973in}{5.237103in}}%
\pgfpathlineto{\pgfqpoint{1.529563in}{5.228394in}}%
\pgfpathlineto{\pgfqpoint{1.529984in}{5.223307in}}%
\pgfpathlineto{\pgfqpoint{1.530574in}{5.228575in}}%
\pgfpathlineto{\pgfqpoint{1.530658in}{5.228885in}}%
\pgfpathlineto{\pgfqpoint{1.530911in}{5.227017in}}%
\pgfpathlineto{\pgfqpoint{1.532680in}{5.217903in}}%
\pgfpathlineto{\pgfqpoint{1.532848in}{5.216959in}}%
\pgfpathlineto{\pgfqpoint{1.533185in}{5.221522in}}%
\pgfpathlineto{\pgfqpoint{1.533944in}{5.239209in}}%
\pgfpathlineto{\pgfqpoint{1.534449in}{5.226231in}}%
\pgfpathlineto{\pgfqpoint{1.534870in}{5.218761in}}%
\pgfpathlineto{\pgfqpoint{1.535544in}{5.224245in}}%
\pgfpathlineto{\pgfqpoint{1.536218in}{5.223382in}}%
\pgfpathlineto{\pgfqpoint{1.537145in}{5.226814in}}%
\pgfpathlineto{\pgfqpoint{1.537398in}{5.225081in}}%
\pgfpathlineto{\pgfqpoint{1.537903in}{5.216879in}}%
\pgfpathlineto{\pgfqpoint{1.538324in}{5.226602in}}%
\pgfpathlineto{\pgfqpoint{1.538745in}{5.238636in}}%
\pgfpathlineto{\pgfqpoint{1.539251in}{5.224385in}}%
\pgfpathlineto{\pgfqpoint{1.539672in}{5.214376in}}%
\pgfpathlineto{\pgfqpoint{1.540178in}{5.226706in}}%
\pgfpathlineto{\pgfqpoint{1.540599in}{5.239019in}}%
\pgfpathlineto{\pgfqpoint{1.541104in}{5.224761in}}%
\pgfpathlineto{\pgfqpoint{1.541357in}{5.218792in}}%
\pgfpathlineto{\pgfqpoint{1.542031in}{5.228734in}}%
\pgfpathlineto{\pgfqpoint{1.542284in}{5.227170in}}%
\pgfpathlineto{\pgfqpoint{1.542789in}{5.221854in}}%
\pgfpathlineto{\pgfqpoint{1.543295in}{5.226665in}}%
\pgfpathlineto{\pgfqpoint{1.543716in}{5.230646in}}%
\pgfpathlineto{\pgfqpoint{1.544221in}{5.225119in}}%
\pgfpathlineto{\pgfqpoint{1.545738in}{5.201276in}}%
\pgfpathlineto{\pgfqpoint{1.546075in}{5.214386in}}%
\pgfpathlineto{\pgfqpoint{1.546749in}{5.261747in}}%
\pgfpathlineto{\pgfqpoint{1.547254in}{5.229859in}}%
\pgfpathlineto{\pgfqpoint{1.547675in}{5.201050in}}%
\pgfpathlineto{\pgfqpoint{1.548349in}{5.230061in}}%
\pgfpathlineto{\pgfqpoint{1.548686in}{5.237870in}}%
\pgfpathlineto{\pgfqpoint{1.549276in}{5.222447in}}%
\pgfpathlineto{\pgfqpoint{1.549445in}{5.220550in}}%
\pgfpathlineto{\pgfqpoint{1.550034in}{5.226906in}}%
\pgfpathlineto{\pgfqpoint{1.550877in}{5.231608in}}%
\pgfpathlineto{\pgfqpoint{1.551466in}{5.230404in}}%
\pgfpathlineto{\pgfqpoint{1.552393in}{5.210560in}}%
\pgfpathlineto{\pgfqpoint{1.552814in}{5.224963in}}%
\pgfpathlineto{\pgfqpoint{1.553236in}{5.239062in}}%
\pgfpathlineto{\pgfqpoint{1.553825in}{5.221495in}}%
\pgfpathlineto{\pgfqpoint{1.554162in}{5.210046in}}%
\pgfpathlineto{\pgfqpoint{1.554752in}{5.228638in}}%
\pgfpathlineto{\pgfqpoint{1.555257in}{5.238977in}}%
\pgfpathlineto{\pgfqpoint{1.555931in}{5.231430in}}%
\pgfpathlineto{\pgfqpoint{1.557363in}{5.212936in}}%
\pgfpathlineto{\pgfqpoint{1.557616in}{5.217285in}}%
\pgfpathlineto{\pgfqpoint{1.558374in}{5.240419in}}%
\pgfpathlineto{\pgfqpoint{1.558964in}{5.226031in}}%
\pgfpathlineto{\pgfqpoint{1.559133in}{5.223595in}}%
\pgfpathlineto{\pgfqpoint{1.559722in}{5.232526in}}%
\pgfpathlineto{\pgfqpoint{1.559891in}{5.231658in}}%
\pgfpathlineto{\pgfqpoint{1.561828in}{5.210853in}}%
\pgfpathlineto{\pgfqpoint{1.561997in}{5.212268in}}%
\pgfpathlineto{\pgfqpoint{1.562755in}{5.238379in}}%
\pgfpathlineto{\pgfqpoint{1.563345in}{5.221063in}}%
\pgfpathlineto{\pgfqpoint{1.564187in}{5.205809in}}%
\pgfpathlineto{\pgfqpoint{1.564777in}{5.209074in}}%
\pgfpathlineto{\pgfqpoint{1.567726in}{5.246860in}}%
\pgfpathlineto{\pgfqpoint{1.568231in}{5.238372in}}%
\pgfpathlineto{\pgfqpoint{1.569663in}{5.213506in}}%
\pgfpathlineto{\pgfqpoint{1.570084in}{5.220207in}}%
\pgfpathlineto{\pgfqpoint{1.570337in}{5.224118in}}%
\pgfpathlineto{\pgfqpoint{1.570843in}{5.215556in}}%
\pgfpathlineto{\pgfqpoint{1.571601in}{5.195092in}}%
\pgfpathlineto{\pgfqpoint{1.572022in}{5.207301in}}%
\pgfpathlineto{\pgfqpoint{1.572864in}{5.237436in}}%
\pgfpathlineto{\pgfqpoint{1.573454in}{5.225334in}}%
\pgfpathlineto{\pgfqpoint{1.573707in}{5.221556in}}%
\pgfpathlineto{\pgfqpoint{1.574212in}{5.231924in}}%
\pgfpathlineto{\pgfqpoint{1.574549in}{5.239184in}}%
\pgfpathlineto{\pgfqpoint{1.575392in}{5.235178in}}%
\pgfpathlineto{\pgfqpoint{1.575729in}{5.237752in}}%
\pgfpathlineto{\pgfqpoint{1.576150in}{5.232909in}}%
\pgfpathlineto{\pgfqpoint{1.576403in}{5.230869in}}%
\pgfpathlineto{\pgfqpoint{1.576992in}{5.236068in}}%
\pgfpathlineto{\pgfqpoint{1.577582in}{5.243391in}}%
\pgfpathlineto{\pgfqpoint{1.577835in}{5.246169in}}%
\pgfpathlineto{\pgfqpoint{1.578425in}{5.239357in}}%
\pgfpathlineto{\pgfqpoint{1.579520in}{5.219887in}}%
\pgfpathlineto{\pgfqpoint{1.579941in}{5.231524in}}%
\pgfpathlineto{\pgfqpoint{1.580362in}{5.248297in}}%
\pgfpathlineto{\pgfqpoint{1.580868in}{5.228862in}}%
\pgfpathlineto{\pgfqpoint{1.581457in}{5.205654in}}%
\pgfpathlineto{\pgfqpoint{1.582131in}{5.217633in}}%
\pgfpathlineto{\pgfqpoint{1.582468in}{5.212261in}}%
\pgfpathlineto{\pgfqpoint{1.582974in}{5.199250in}}%
\pgfpathlineto{\pgfqpoint{1.583395in}{5.212332in}}%
\pgfpathlineto{\pgfqpoint{1.585080in}{5.239416in}}%
\pgfpathlineto{\pgfqpoint{1.585585in}{5.247935in}}%
\pgfpathlineto{\pgfqpoint{1.586007in}{5.238649in}}%
\pgfpathlineto{\pgfqpoint{1.586681in}{5.217013in}}%
\pgfpathlineto{\pgfqpoint{1.587186in}{5.231272in}}%
\pgfpathlineto{\pgfqpoint{1.587523in}{5.237608in}}%
\pgfpathlineto{\pgfqpoint{1.588365in}{5.233634in}}%
\pgfpathlineto{\pgfqpoint{1.588955in}{5.197767in}}%
\pgfpathlineto{\pgfqpoint{1.589292in}{5.181378in}}%
\pgfpathlineto{\pgfqpoint{1.589798in}{5.209982in}}%
\pgfpathlineto{\pgfqpoint{1.590135in}{5.224598in}}%
\pgfpathlineto{\pgfqpoint{1.590640in}{5.199405in}}%
\pgfpathlineto{\pgfqpoint{1.590977in}{5.185523in}}%
\pgfpathlineto{\pgfqpoint{1.591483in}{5.208003in}}%
\pgfpathlineto{\pgfqpoint{1.591988in}{5.226527in}}%
\pgfpathlineto{\pgfqpoint{1.592746in}{5.221548in}}%
\pgfpathlineto{\pgfqpoint{1.593420in}{5.228706in}}%
\pgfpathlineto{\pgfqpoint{1.593841in}{5.222041in}}%
\pgfpathlineto{\pgfqpoint{1.594431in}{5.211966in}}%
\pgfpathlineto{\pgfqpoint{1.594852in}{5.219426in}}%
\pgfpathlineto{\pgfqpoint{1.595442in}{5.236069in}}%
\pgfpathlineto{\pgfqpoint{1.596032in}{5.225251in}}%
\pgfpathlineto{\pgfqpoint{1.596116in}{5.224929in}}%
\pgfpathlineto{\pgfqpoint{1.596369in}{5.228022in}}%
\pgfpathlineto{\pgfqpoint{1.598306in}{5.258736in}}%
\pgfpathlineto{\pgfqpoint{1.598475in}{5.255980in}}%
\pgfpathlineto{\pgfqpoint{1.599149in}{5.233628in}}%
\pgfpathlineto{\pgfqpoint{1.599823in}{5.247450in}}%
\pgfpathlineto{\pgfqpoint{1.600075in}{5.244003in}}%
\pgfpathlineto{\pgfqpoint{1.602603in}{5.194356in}}%
\pgfpathlineto{\pgfqpoint{1.603529in}{5.187607in}}%
\pgfpathlineto{\pgfqpoint{1.603782in}{5.191412in}}%
\pgfpathlineto{\pgfqpoint{1.606478in}{5.290737in}}%
\pgfpathlineto{\pgfqpoint{1.606899in}{5.271908in}}%
\pgfpathlineto{\pgfqpoint{1.609258in}{5.194867in}}%
\pgfpathlineto{\pgfqpoint{1.609342in}{5.194974in}}%
\pgfpathlineto{\pgfqpoint{1.609932in}{5.206010in}}%
\pgfpathlineto{\pgfqpoint{1.610438in}{5.195639in}}%
\pgfpathlineto{\pgfqpoint{1.610859in}{5.189971in}}%
\pgfpathlineto{\pgfqpoint{1.611448in}{5.197069in}}%
\pgfpathlineto{\pgfqpoint{1.612291in}{5.207995in}}%
\pgfpathlineto{\pgfqpoint{1.613302in}{5.238792in}}%
\pgfpathlineto{\pgfqpoint{1.614229in}{5.235134in}}%
\pgfpathlineto{\pgfqpoint{1.614818in}{5.226329in}}%
\pgfpathlineto{\pgfqpoint{1.615239in}{5.232790in}}%
\pgfpathlineto{\pgfqpoint{1.615745in}{5.246348in}}%
\pgfpathlineto{\pgfqpoint{1.616166in}{5.232330in}}%
\pgfpathlineto{\pgfqpoint{1.616840in}{5.202996in}}%
\pgfpathlineto{\pgfqpoint{1.617430in}{5.216572in}}%
\pgfpathlineto{\pgfqpoint{1.617851in}{5.223705in}}%
\pgfpathlineto{\pgfqpoint{1.618357in}{5.212239in}}%
\pgfpathlineto{\pgfqpoint{1.618778in}{5.203522in}}%
\pgfpathlineto{\pgfqpoint{1.619367in}{5.214765in}}%
\pgfpathlineto{\pgfqpoint{1.620126in}{5.213481in}}%
\pgfpathlineto{\pgfqpoint{1.620884in}{5.221539in}}%
\pgfpathlineto{\pgfqpoint{1.621558in}{5.218095in}}%
\pgfpathlineto{\pgfqpoint{1.621811in}{5.220712in}}%
\pgfpathlineto{\pgfqpoint{1.624169in}{5.267126in}}%
\pgfpathlineto{\pgfqpoint{1.624506in}{5.261867in}}%
\pgfpathlineto{\pgfqpoint{1.625012in}{5.246158in}}%
\pgfpathlineto{\pgfqpoint{1.625602in}{5.262119in}}%
\pgfpathlineto{\pgfqpoint{1.625854in}{5.267101in}}%
\pgfpathlineto{\pgfqpoint{1.626276in}{5.254288in}}%
\pgfpathlineto{\pgfqpoint{1.628297in}{5.194382in}}%
\pgfpathlineto{\pgfqpoint{1.629477in}{5.153727in}}%
\pgfpathlineto{\pgfqpoint{1.629982in}{5.175358in}}%
\pgfpathlineto{\pgfqpoint{1.631920in}{5.240084in}}%
\pgfpathlineto{\pgfqpoint{1.632004in}{5.239201in}}%
\pgfpathlineto{\pgfqpoint{1.632594in}{5.227364in}}%
\pgfpathlineto{\pgfqpoint{1.633015in}{5.240688in}}%
\pgfpathlineto{\pgfqpoint{1.633689in}{5.262910in}}%
\pgfpathlineto{\pgfqpoint{1.634195in}{5.250031in}}%
\pgfpathlineto{\pgfqpoint{1.634616in}{5.240698in}}%
\pgfpathlineto{\pgfqpoint{1.635205in}{5.251886in}}%
\pgfpathlineto{\pgfqpoint{1.635964in}{5.250811in}}%
\pgfpathlineto{\pgfqpoint{1.636975in}{5.264851in}}%
\pgfpathlineto{\pgfqpoint{1.638575in}{5.271478in}}%
\pgfpathlineto{\pgfqpoint{1.638744in}{5.270474in}}%
\pgfpathlineto{\pgfqpoint{1.639418in}{5.247298in}}%
\pgfpathlineto{\pgfqpoint{1.641524in}{5.188160in}}%
\pgfpathlineto{\pgfqpoint{1.642956in}{5.154509in}}%
\pgfpathlineto{\pgfqpoint{1.643630in}{5.133322in}}%
\pgfpathlineto{\pgfqpoint{1.644135in}{5.148796in}}%
\pgfpathlineto{\pgfqpoint{1.648348in}{5.370869in}}%
\pgfpathlineto{\pgfqpoint{1.649274in}{5.329204in}}%
\pgfpathlineto{\pgfqpoint{1.652223in}{5.089923in}}%
\pgfpathlineto{\pgfqpoint{1.653234in}{5.134257in}}%
\pgfpathlineto{\pgfqpoint{1.654413in}{5.259265in}}%
\pgfpathlineto{\pgfqpoint{1.656182in}{5.321288in}}%
\pgfpathlineto{\pgfqpoint{1.656267in}{5.320629in}}%
\pgfpathlineto{\pgfqpoint{1.656856in}{5.297598in}}%
\pgfpathlineto{\pgfqpoint{1.658373in}{5.162194in}}%
\pgfpathlineto{\pgfqpoint{1.659636in}{5.166401in}}%
\pgfpathlineto{\pgfqpoint{1.659721in}{5.165295in}}%
\pgfpathlineto{\pgfqpoint{1.659973in}{5.171083in}}%
\pgfpathlineto{\pgfqpoint{1.662248in}{5.299688in}}%
\pgfpathlineto{\pgfqpoint{1.662838in}{5.287664in}}%
\pgfpathlineto{\pgfqpoint{1.665449in}{5.182942in}}%
\pgfpathlineto{\pgfqpoint{1.666544in}{5.200925in}}%
\pgfpathlineto{\pgfqpoint{1.667808in}{5.216830in}}%
\pgfpathlineto{\pgfqpoint{1.668398in}{5.210927in}}%
\pgfpathlineto{\pgfqpoint{1.669914in}{5.168895in}}%
\pgfpathlineto{\pgfqpoint{1.670672in}{5.111002in}}%
\pgfpathlineto{\pgfqpoint{1.671178in}{5.146005in}}%
\pgfpathlineto{\pgfqpoint{1.671515in}{5.164985in}}%
\pgfpathlineto{\pgfqpoint{1.672189in}{5.145847in}}%
\pgfpathlineto{\pgfqpoint{1.672273in}{5.146089in}}%
\pgfpathlineto{\pgfqpoint{1.672779in}{5.176849in}}%
\pgfpathlineto{\pgfqpoint{1.675811in}{5.431700in}}%
\pgfpathlineto{\pgfqpoint{1.676064in}{5.424958in}}%
\pgfpathlineto{\pgfqpoint{1.677075in}{5.301871in}}%
\pgfpathlineto{\pgfqpoint{1.680108in}{5.018212in}}%
\pgfpathlineto{\pgfqpoint{1.680192in}{5.019487in}}%
\pgfpathlineto{\pgfqpoint{1.680782in}{5.076055in}}%
\pgfpathlineto{\pgfqpoint{1.683815in}{5.413078in}}%
\pgfpathlineto{\pgfqpoint{1.683899in}{5.412841in}}%
\pgfpathlineto{\pgfqpoint{1.684320in}{5.401585in}}%
\pgfpathlineto{\pgfqpoint{1.685752in}{5.258755in}}%
\pgfpathlineto{\pgfqpoint{1.688027in}{5.070165in}}%
\pgfpathlineto{\pgfqpoint{1.688279in}{5.075994in}}%
\pgfpathlineto{\pgfqpoint{1.691481in}{5.301877in}}%
\pgfpathlineto{\pgfqpoint{1.692407in}{5.262353in}}%
\pgfpathlineto{\pgfqpoint{1.694092in}{5.179472in}}%
\pgfpathlineto{\pgfqpoint{1.694514in}{5.184856in}}%
\pgfpathlineto{\pgfqpoint{1.697041in}{5.257695in}}%
\pgfpathlineto{\pgfqpoint{1.698052in}{5.240385in}}%
\pgfpathlineto{\pgfqpoint{1.698642in}{5.230689in}}%
\pgfpathlineto{\pgfqpoint{1.699147in}{5.240169in}}%
\pgfpathlineto{\pgfqpoint{1.701590in}{5.343531in}}%
\pgfpathlineto{\pgfqpoint{1.702096in}{5.365157in}}%
\pgfpathlineto{\pgfqpoint{1.702685in}{5.343598in}}%
\pgfpathlineto{\pgfqpoint{1.703865in}{5.253554in}}%
\pgfpathlineto{\pgfqpoint{1.705971in}{4.953402in}}%
\pgfpathlineto{\pgfqpoint{1.706729in}{4.991750in}}%
\pgfpathlineto{\pgfqpoint{1.708161in}{5.204132in}}%
\pgfpathlineto{\pgfqpoint{1.710183in}{5.459578in}}%
\pgfpathlineto{\pgfqpoint{1.710352in}{5.458139in}}%
\pgfpathlineto{\pgfqpoint{1.710941in}{5.408535in}}%
\pgfpathlineto{\pgfqpoint{1.713300in}{5.151452in}}%
\pgfpathlineto{\pgfqpoint{1.713721in}{5.175470in}}%
\pgfpathlineto{\pgfqpoint{1.714817in}{5.257288in}}%
\pgfpathlineto{\pgfqpoint{1.715575in}{5.254195in}}%
\pgfpathlineto{\pgfqpoint{1.716164in}{5.266368in}}%
\pgfpathlineto{\pgfqpoint{1.716586in}{5.253465in}}%
\pgfpathlineto{\pgfqpoint{1.718776in}{5.018860in}}%
\pgfpathlineto{\pgfqpoint{1.719787in}{5.072871in}}%
\pgfpathlineto{\pgfqpoint{1.722567in}{5.464087in}}%
\pgfpathlineto{\pgfqpoint{1.723578in}{5.391263in}}%
\pgfpathlineto{\pgfqpoint{1.726695in}{5.097670in}}%
\pgfpathlineto{\pgfqpoint{1.726948in}{5.107342in}}%
\pgfpathlineto{\pgfqpoint{1.729138in}{5.227760in}}%
\pgfpathlineto{\pgfqpoint{1.729475in}{5.217987in}}%
\pgfpathlineto{\pgfqpoint{1.730991in}{5.110534in}}%
\pgfpathlineto{\pgfqpoint{1.731750in}{5.126000in}}%
\pgfpathlineto{\pgfqpoint{1.733182in}{5.238597in}}%
\pgfpathlineto{\pgfqpoint{1.735625in}{5.496597in}}%
\pgfpathlineto{\pgfqpoint{1.736046in}{5.469771in}}%
\pgfpathlineto{\pgfqpoint{1.740006in}{5.074803in}}%
\pgfpathlineto{\pgfqpoint{1.740427in}{5.091539in}}%
\pgfpathlineto{\pgfqpoint{1.741185in}{5.123132in}}%
\pgfpathlineto{\pgfqpoint{1.741691in}{5.108354in}}%
\pgfpathlineto{\pgfqpoint{1.741775in}{5.107054in}}%
\pgfpathlineto{\pgfqpoint{1.742112in}{5.115942in}}%
\pgfpathlineto{\pgfqpoint{1.745734in}{5.284692in}}%
\pgfpathlineto{\pgfqpoint{1.746577in}{5.286453in}}%
\pgfpathlineto{\pgfqpoint{1.746745in}{5.284857in}}%
\pgfpathlineto{\pgfqpoint{1.747419in}{5.243962in}}%
\pgfpathlineto{\pgfqpoint{1.748009in}{5.218438in}}%
\pgfpathlineto{\pgfqpoint{1.748683in}{5.231493in}}%
\pgfpathlineto{\pgfqpoint{1.749188in}{5.185409in}}%
\pgfpathlineto{\pgfqpoint{1.749694in}{5.138193in}}%
\pgfpathlineto{\pgfqpoint{1.750283in}{5.181718in}}%
\pgfpathlineto{\pgfqpoint{1.750789in}{5.202260in}}%
\pgfpathlineto{\pgfqpoint{1.751379in}{5.186206in}}%
\pgfpathlineto{\pgfqpoint{1.751716in}{5.180215in}}%
\pgfpathlineto{\pgfqpoint{1.752305in}{5.191243in}}%
\pgfpathlineto{\pgfqpoint{1.752642in}{5.190086in}}%
\pgfpathlineto{\pgfqpoint{1.752895in}{5.192810in}}%
\pgfpathlineto{\pgfqpoint{1.753653in}{5.227426in}}%
\pgfpathlineto{\pgfqpoint{1.754159in}{5.205533in}}%
\pgfpathlineto{\pgfqpoint{1.754664in}{5.184814in}}%
\pgfpathlineto{\pgfqpoint{1.755170in}{5.201890in}}%
\pgfpathlineto{\pgfqpoint{1.758539in}{5.385273in}}%
\pgfpathlineto{\pgfqpoint{1.759045in}{5.371789in}}%
\pgfpathlineto{\pgfqpoint{1.760730in}{5.254795in}}%
\pgfpathlineto{\pgfqpoint{1.762583in}{5.080186in}}%
\pgfpathlineto{\pgfqpoint{1.763257in}{5.086214in}}%
\pgfpathlineto{\pgfqpoint{1.764352in}{5.106167in}}%
\pgfpathlineto{\pgfqpoint{1.766543in}{5.291185in}}%
\pgfpathlineto{\pgfqpoint{1.767217in}{5.275441in}}%
\pgfpathlineto{\pgfqpoint{1.768986in}{5.244380in}}%
\pgfpathlineto{\pgfqpoint{1.769491in}{5.241037in}}%
\pgfpathlineto{\pgfqpoint{1.769912in}{5.245626in}}%
\pgfpathlineto{\pgfqpoint{1.771092in}{5.298090in}}%
\pgfpathlineto{\pgfqpoint{1.771682in}{5.268362in}}%
\pgfpathlineto{\pgfqpoint{1.771850in}{5.263599in}}%
\pgfpathlineto{\pgfqpoint{1.772271in}{5.281239in}}%
\pgfpathlineto{\pgfqpoint{1.772693in}{5.302825in}}%
\pgfpathlineto{\pgfqpoint{1.773114in}{5.272525in}}%
\pgfpathlineto{\pgfqpoint{1.774967in}{5.165888in}}%
\pgfpathlineto{\pgfqpoint{1.775978in}{5.146519in}}%
\pgfpathlineto{\pgfqpoint{1.776399in}{5.150007in}}%
\pgfpathlineto{\pgfqpoint{1.777157in}{5.178640in}}%
\pgfpathlineto{\pgfqpoint{1.779432in}{5.285779in}}%
\pgfpathlineto{\pgfqpoint{1.779853in}{5.274342in}}%
\pgfpathlineto{\pgfqpoint{1.781707in}{5.182691in}}%
\pgfpathlineto{\pgfqpoint{1.782718in}{5.201408in}}%
\pgfpathlineto{\pgfqpoint{1.783307in}{5.226069in}}%
\pgfpathlineto{\pgfqpoint{1.784824in}{5.250592in}}%
\pgfpathlineto{\pgfqpoint{1.785413in}{5.256633in}}%
\pgfpathlineto{\pgfqpoint{1.785750in}{5.251349in}}%
\pgfpathlineto{\pgfqpoint{1.786593in}{5.224425in}}%
\pgfpathlineto{\pgfqpoint{1.787183in}{5.238023in}}%
\pgfpathlineto{\pgfqpoint{1.787351in}{5.239282in}}%
\pgfpathlineto{\pgfqpoint{1.787604in}{5.230656in}}%
\pgfpathlineto{\pgfqpoint{1.788278in}{5.196095in}}%
\pgfpathlineto{\pgfqpoint{1.788867in}{5.215736in}}%
\pgfpathlineto{\pgfqpoint{1.790468in}{5.239291in}}%
\pgfpathlineto{\pgfqpoint{1.789626in}{5.215362in}}%
\pgfpathlineto{\pgfqpoint{1.790637in}{5.237932in}}%
\pgfpathlineto{\pgfqpoint{1.792827in}{5.198706in}}%
\pgfpathlineto{\pgfqpoint{1.793164in}{5.201768in}}%
\pgfpathlineto{\pgfqpoint{1.795102in}{5.254791in}}%
\pgfpathlineto{\pgfqpoint{1.796786in}{5.305550in}}%
\pgfpathlineto{\pgfqpoint{1.796955in}{5.306140in}}%
\pgfpathlineto{\pgfqpoint{1.797208in}{5.303389in}}%
\pgfpathlineto{\pgfqpoint{1.798556in}{5.253267in}}%
\pgfpathlineto{\pgfqpoint{1.801083in}{5.074021in}}%
\pgfpathlineto{\pgfqpoint{1.801167in}{5.074156in}}%
\pgfpathlineto{\pgfqpoint{1.801673in}{5.096211in}}%
\pgfpathlineto{\pgfqpoint{1.805211in}{5.345526in}}%
\pgfpathlineto{\pgfqpoint{1.805632in}{5.332432in}}%
\pgfpathlineto{\pgfqpoint{1.809844in}{5.127826in}}%
\pgfpathlineto{\pgfqpoint{1.810181in}{5.135319in}}%
\pgfpathlineto{\pgfqpoint{1.813383in}{5.282488in}}%
\pgfpathlineto{\pgfqpoint{1.814899in}{5.275715in}}%
\pgfpathlineto{\pgfqpoint{1.815068in}{5.276237in}}%
\pgfpathlineto{\pgfqpoint{1.815320in}{5.273975in}}%
\pgfpathlineto{\pgfqpoint{1.816752in}{5.225558in}}%
\pgfpathlineto{\pgfqpoint{1.817511in}{5.183519in}}%
\pgfpathlineto{\pgfqpoint{1.818100in}{5.202881in}}%
\pgfpathlineto{\pgfqpoint{1.818353in}{5.208903in}}%
\pgfpathlineto{\pgfqpoint{1.818859in}{5.194616in}}%
\pgfpathlineto{\pgfqpoint{1.819027in}{5.191815in}}%
\pgfpathlineto{\pgfqpoint{1.819448in}{5.206129in}}%
\pgfpathlineto{\pgfqpoint{1.820628in}{5.261758in}}%
\pgfpathlineto{\pgfqpoint{1.821133in}{5.247382in}}%
\pgfpathlineto{\pgfqpoint{1.822144in}{5.203205in}}%
\pgfpathlineto{\pgfqpoint{1.822902in}{5.209066in}}%
\pgfpathlineto{\pgfqpoint{1.823660in}{5.179357in}}%
\pgfpathlineto{\pgfqpoint{1.824166in}{5.203871in}}%
\pgfpathlineto{\pgfqpoint{1.826272in}{5.299973in}}%
\pgfpathlineto{\pgfqpoint{1.826441in}{5.297004in}}%
\pgfpathlineto{\pgfqpoint{1.827199in}{5.248172in}}%
\pgfpathlineto{\pgfqpoint{1.828041in}{5.266956in}}%
\pgfpathlineto{\pgfqpoint{1.830232in}{5.195905in}}%
\pgfpathlineto{\pgfqpoint{1.830821in}{5.200532in}}%
\pgfpathlineto{\pgfqpoint{1.831664in}{5.159389in}}%
\pgfpathlineto{\pgfqpoint{1.832253in}{5.185258in}}%
\pgfpathlineto{\pgfqpoint{1.832843in}{5.211691in}}%
\pgfpathlineto{\pgfqpoint{1.833517in}{5.197374in}}%
\pgfpathlineto{\pgfqpoint{1.833601in}{5.197136in}}%
\pgfpathlineto{\pgfqpoint{1.833770in}{5.199324in}}%
\pgfpathlineto{\pgfqpoint{1.835876in}{5.281676in}}%
\pgfpathlineto{\pgfqpoint{1.836718in}{5.251716in}}%
\pgfpathlineto{\pgfqpoint{1.836803in}{5.251325in}}%
\pgfpathlineto{\pgfqpoint{1.836971in}{5.254753in}}%
\pgfpathlineto{\pgfqpoint{1.837477in}{5.276320in}}%
\pgfpathlineto{\pgfqpoint{1.837898in}{5.252654in}}%
\pgfpathlineto{\pgfqpoint{1.839835in}{5.119831in}}%
\pgfpathlineto{\pgfqpoint{1.840004in}{5.123816in}}%
\pgfpathlineto{\pgfqpoint{1.844553in}{5.349217in}}%
\pgfpathlineto{\pgfqpoint{1.845311in}{5.323329in}}%
\pgfpathlineto{\pgfqpoint{1.847839in}{5.118600in}}%
\pgfpathlineto{\pgfqpoint{1.848681in}{5.138701in}}%
\pgfpathlineto{\pgfqpoint{1.849018in}{5.135405in}}%
\pgfpathlineto{\pgfqpoint{1.849271in}{5.143601in}}%
\pgfpathlineto{\pgfqpoint{1.852219in}{5.339239in}}%
\pgfpathlineto{\pgfqpoint{1.852641in}{5.323851in}}%
\pgfpathlineto{\pgfqpoint{1.856095in}{5.161464in}}%
\pgfpathlineto{\pgfqpoint{1.856179in}{5.162362in}}%
\pgfpathlineto{\pgfqpoint{1.856853in}{5.179606in}}%
\pgfpathlineto{\pgfqpoint{1.857443in}{5.167120in}}%
\pgfpathlineto{\pgfqpoint{1.857527in}{5.166987in}}%
\pgfpathlineto{\pgfqpoint{1.857611in}{5.168401in}}%
\pgfpathlineto{\pgfqpoint{1.858622in}{5.248223in}}%
\pgfpathlineto{\pgfqpoint{1.860054in}{5.293098in}}%
\pgfpathlineto{\pgfqpoint{1.860560in}{5.283165in}}%
\pgfpathlineto{\pgfqpoint{1.861655in}{5.248053in}}%
\pgfpathlineto{\pgfqpoint{1.863929in}{5.102735in}}%
\pgfpathlineto{\pgfqpoint{1.864182in}{5.109586in}}%
\pgfpathlineto{\pgfqpoint{1.867889in}{5.329359in}}%
\pgfpathlineto{\pgfqpoint{1.868310in}{5.307069in}}%
\pgfpathlineto{\pgfqpoint{1.870248in}{5.218654in}}%
\pgfpathlineto{\pgfqpoint{1.870922in}{5.189694in}}%
\pgfpathlineto{\pgfqpoint{1.871680in}{5.189993in}}%
\pgfpathlineto{\pgfqpoint{1.872017in}{5.182715in}}%
\pgfpathlineto{\pgfqpoint{1.872438in}{5.198287in}}%
\pgfpathlineto{\pgfqpoint{1.872944in}{5.222044in}}%
\pgfpathlineto{\pgfqpoint{1.873533in}{5.197766in}}%
\pgfpathlineto{\pgfqpoint{1.873786in}{5.194238in}}%
\pgfpathlineto{\pgfqpoint{1.874207in}{5.207473in}}%
\pgfpathlineto{\pgfqpoint{1.874628in}{5.216308in}}%
\pgfpathlineto{\pgfqpoint{1.875134in}{5.203950in}}%
\pgfpathlineto{\pgfqpoint{1.875302in}{5.201412in}}%
\pgfpathlineto{\pgfqpoint{1.875724in}{5.210355in}}%
\pgfpathlineto{\pgfqpoint{1.877914in}{5.315165in}}%
\pgfpathlineto{\pgfqpoint{1.878672in}{5.282791in}}%
\pgfpathlineto{\pgfqpoint{1.880273in}{5.198795in}}%
\pgfpathlineto{\pgfqpoint{1.881873in}{5.113137in}}%
\pgfpathlineto{\pgfqpoint{1.881958in}{5.113594in}}%
\pgfpathlineto{\pgfqpoint{1.882800in}{5.144429in}}%
\pgfpathlineto{\pgfqpoint{1.883811in}{5.207583in}}%
\pgfpathlineto{\pgfqpoint{1.885833in}{5.342339in}}%
\pgfpathlineto{\pgfqpoint{1.885917in}{5.342856in}}%
\pgfpathlineto{\pgfqpoint{1.886170in}{5.338682in}}%
\pgfpathlineto{\pgfqpoint{1.887771in}{5.235603in}}%
\pgfpathlineto{\pgfqpoint{1.889287in}{5.187722in}}%
\pgfpathlineto{\pgfqpoint{1.889877in}{5.172655in}}%
\pgfpathlineto{\pgfqpoint{1.890298in}{5.186343in}}%
\pgfpathlineto{\pgfqpoint{1.892320in}{5.279880in}}%
\pgfpathlineto{\pgfqpoint{1.892657in}{5.266388in}}%
\pgfpathlineto{\pgfqpoint{1.895774in}{5.094282in}}%
\pgfpathlineto{\pgfqpoint{1.896027in}{5.097343in}}%
\pgfpathlineto{\pgfqpoint{1.897206in}{5.162602in}}%
\pgfpathlineto{\pgfqpoint{1.898470in}{5.297221in}}%
\pgfpathlineto{\pgfqpoint{1.899228in}{5.357464in}}%
\pgfpathlineto{\pgfqpoint{1.899733in}{5.329950in}}%
\pgfpathlineto{\pgfqpoint{1.902850in}{5.094927in}}%
\pgfpathlineto{\pgfqpoint{1.903272in}{5.118387in}}%
\pgfpathlineto{\pgfqpoint{1.906641in}{5.438340in}}%
\pgfpathlineto{\pgfqpoint{1.906894in}{5.431267in}}%
\pgfpathlineto{\pgfqpoint{1.907736in}{5.302633in}}%
\pgfpathlineto{\pgfqpoint{1.910180in}{5.061376in}}%
\pgfpathlineto{\pgfqpoint{1.910517in}{5.055626in}}%
\pgfpathlineto{\pgfqpoint{1.910854in}{5.068824in}}%
\pgfpathlineto{\pgfqpoint{1.914139in}{5.297996in}}%
\pgfpathlineto{\pgfqpoint{1.914308in}{5.294650in}}%
\pgfpathlineto{\pgfqpoint{1.915571in}{5.224453in}}%
\pgfpathlineto{\pgfqpoint{1.916245in}{5.255604in}}%
\pgfpathlineto{\pgfqpoint{1.916414in}{5.259464in}}%
\pgfpathlineto{\pgfqpoint{1.916835in}{5.243917in}}%
\pgfpathlineto{\pgfqpoint{1.917256in}{5.224639in}}%
\pgfpathlineto{\pgfqpoint{1.917762in}{5.249441in}}%
\pgfpathlineto{\pgfqpoint{1.918267in}{5.270097in}}%
\pgfpathlineto{\pgfqpoint{1.918857in}{5.253090in}}%
\pgfpathlineto{\pgfqpoint{1.922395in}{5.152345in}}%
\pgfpathlineto{\pgfqpoint{1.922479in}{5.152303in}}%
\pgfpathlineto{\pgfqpoint{1.922648in}{5.153311in}}%
\pgfpathlineto{\pgfqpoint{1.924248in}{5.187205in}}%
\pgfpathlineto{\pgfqpoint{1.926776in}{5.239148in}}%
\pgfpathlineto{\pgfqpoint{1.927365in}{5.249415in}}%
\pgfpathlineto{\pgfqpoint{1.927787in}{5.238096in}}%
\pgfpathlineto{\pgfqpoint{1.928629in}{5.211460in}}%
\pgfpathlineto{\pgfqpoint{1.929135in}{5.224682in}}%
\pgfpathlineto{\pgfqpoint{1.929472in}{5.230802in}}%
\pgfpathlineto{\pgfqpoint{1.929893in}{5.217663in}}%
\pgfpathlineto{\pgfqpoint{1.930314in}{5.204001in}}%
\pgfpathlineto{\pgfqpoint{1.930819in}{5.221549in}}%
\pgfpathlineto{\pgfqpoint{1.931325in}{5.241438in}}%
\pgfpathlineto{\pgfqpoint{1.931830in}{5.223840in}}%
\pgfpathlineto{\pgfqpoint{1.932252in}{5.211774in}}%
\pgfpathlineto{\pgfqpoint{1.932757in}{5.230806in}}%
\pgfpathlineto{\pgfqpoint{1.934105in}{5.254088in}}%
\pgfpathlineto{\pgfqpoint{1.934274in}{5.252034in}}%
\pgfpathlineto{\pgfqpoint{1.934863in}{5.236528in}}%
\pgfpathlineto{\pgfqpoint{1.935369in}{5.248705in}}%
\pgfpathlineto{\pgfqpoint{1.936127in}{5.269328in}}%
\pgfpathlineto{\pgfqpoint{1.936801in}{5.262908in}}%
\pgfpathlineto{\pgfqpoint{1.937306in}{5.272402in}}%
\pgfpathlineto{\pgfqpoint{1.937896in}{5.285322in}}%
\pgfpathlineto{\pgfqpoint{1.938317in}{5.273452in}}%
\pgfpathlineto{\pgfqpoint{1.938907in}{5.255595in}}%
\pgfpathlineto{\pgfqpoint{1.939581in}{5.265088in}}%
\pgfpathlineto{\pgfqpoint{1.939834in}{5.262276in}}%
\pgfpathlineto{\pgfqpoint{1.943035in}{5.152176in}}%
\pgfpathlineto{\pgfqpoint{1.944636in}{5.110121in}}%
\pgfpathlineto{\pgfqpoint{1.944973in}{5.106399in}}%
\pgfpathlineto{\pgfqpoint{1.945310in}{5.114080in}}%
\pgfpathlineto{\pgfqpoint{1.948679in}{5.339656in}}%
\pgfpathlineto{\pgfqpoint{1.950027in}{5.307523in}}%
\pgfpathlineto{\pgfqpoint{1.950364in}{5.311047in}}%
\pgfpathlineto{\pgfqpoint{1.950701in}{5.301522in}}%
\pgfpathlineto{\pgfqpoint{1.954155in}{5.163860in}}%
\pgfpathlineto{\pgfqpoint{1.954239in}{5.163329in}}%
\pgfpathlineto{\pgfqpoint{1.954492in}{5.166672in}}%
\pgfpathlineto{\pgfqpoint{1.955250in}{5.185173in}}%
\pgfpathlineto{\pgfqpoint{1.955924in}{5.176086in}}%
\pgfpathlineto{\pgfqpoint{1.956177in}{5.173592in}}%
\pgfpathlineto{\pgfqpoint{1.956683in}{5.179083in}}%
\pgfpathlineto{\pgfqpoint{1.958030in}{5.215784in}}%
\pgfpathlineto{\pgfqpoint{1.960221in}{5.303901in}}%
\pgfpathlineto{\pgfqpoint{1.960642in}{5.311621in}}%
\pgfpathlineto{\pgfqpoint{1.961148in}{5.300821in}}%
\pgfpathlineto{\pgfqpoint{1.963675in}{5.180660in}}%
\pgfpathlineto{\pgfqpoint{1.964770in}{5.131146in}}%
\pgfpathlineto{\pgfqpoint{1.965360in}{5.146243in}}%
\pgfpathlineto{\pgfqpoint{1.967719in}{5.260393in}}%
\pgfpathlineto{\pgfqpoint{1.968561in}{5.297656in}}%
\pgfpathlineto{\pgfqpoint{1.969319in}{5.290104in}}%
\pgfpathlineto{\pgfqpoint{1.969740in}{5.297289in}}%
\pgfpathlineto{\pgfqpoint{1.970077in}{5.290211in}}%
\pgfpathlineto{\pgfqpoint{1.972268in}{5.228947in}}%
\pgfpathlineto{\pgfqpoint{1.973363in}{5.222374in}}%
\pgfpathlineto{\pgfqpoint{1.976227in}{5.170014in}}%
\pgfpathlineto{\pgfqpoint{1.976564in}{5.176907in}}%
\pgfpathlineto{\pgfqpoint{1.979850in}{5.293492in}}%
\pgfpathlineto{\pgfqpoint{1.980692in}{5.268405in}}%
\pgfpathlineto{\pgfqpoint{1.983978in}{5.070006in}}%
\pgfpathlineto{\pgfqpoint{1.984652in}{5.108917in}}%
\pgfpathlineto{\pgfqpoint{1.988022in}{5.402556in}}%
\pgfpathlineto{\pgfqpoint{1.989032in}{5.383863in}}%
\pgfpathlineto{\pgfqpoint{1.989706in}{5.348734in}}%
\pgfpathlineto{\pgfqpoint{1.992908in}{5.056798in}}%
\pgfpathlineto{\pgfqpoint{1.993497in}{5.075427in}}%
\pgfpathlineto{\pgfqpoint{1.995098in}{5.299211in}}%
\pgfpathlineto{\pgfqpoint{1.996362in}{5.444571in}}%
\pgfpathlineto{\pgfqpoint{1.996867in}{5.428426in}}%
\pgfpathlineto{\pgfqpoint{1.998047in}{5.280696in}}%
\pgfpathlineto{\pgfqpoint{2.000068in}{5.011341in}}%
\pgfpathlineto{\pgfqpoint{2.000742in}{5.033526in}}%
\pgfpathlineto{\pgfqpoint{2.001669in}{5.145382in}}%
\pgfpathlineto{\pgfqpoint{2.004028in}{5.426186in}}%
\pgfpathlineto{\pgfqpoint{2.004281in}{5.420039in}}%
\pgfpathlineto{\pgfqpoint{2.005123in}{5.310576in}}%
\pgfpathlineto{\pgfqpoint{2.007735in}{5.004546in}}%
\pgfpathlineto{\pgfqpoint{2.008072in}{5.019671in}}%
\pgfpathlineto{\pgfqpoint{2.011610in}{5.327658in}}%
\pgfpathlineto{\pgfqpoint{2.012200in}{5.307572in}}%
\pgfpathlineto{\pgfqpoint{2.013885in}{5.244740in}}%
\pgfpathlineto{\pgfqpoint{2.014137in}{5.246133in}}%
\pgfpathlineto{\pgfqpoint{2.014980in}{5.274063in}}%
\pgfpathlineto{\pgfqpoint{2.017086in}{5.335050in}}%
\pgfpathlineto{\pgfqpoint{2.017254in}{5.332058in}}%
\pgfpathlineto{\pgfqpoint{2.018181in}{5.249190in}}%
\pgfpathlineto{\pgfqpoint{2.020371in}{5.140511in}}%
\pgfpathlineto{\pgfqpoint{2.020624in}{5.144686in}}%
\pgfpathlineto{\pgfqpoint{2.024415in}{5.274095in}}%
\pgfpathlineto{\pgfqpoint{2.024836in}{5.263675in}}%
\pgfpathlineto{\pgfqpoint{2.028627in}{5.141409in}}%
\pgfpathlineto{\pgfqpoint{2.028712in}{5.141198in}}%
\pgfpathlineto{\pgfqpoint{2.029049in}{5.142760in}}%
\pgfpathlineto{\pgfqpoint{2.030565in}{5.160778in}}%
\pgfpathlineto{\pgfqpoint{2.034188in}{5.302047in}}%
\pgfpathlineto{\pgfqpoint{2.036294in}{5.387701in}}%
\pgfpathlineto{\pgfqpoint{2.036378in}{5.387817in}}%
\pgfpathlineto{\pgfqpoint{2.036462in}{5.386767in}}%
\pgfpathlineto{\pgfqpoint{2.037052in}{5.345632in}}%
\pgfpathlineto{\pgfqpoint{2.040337in}{5.040319in}}%
\pgfpathlineto{\pgfqpoint{2.040674in}{5.047726in}}%
\pgfpathlineto{\pgfqpoint{2.042191in}{5.203525in}}%
\pgfpathlineto{\pgfqpoint{2.044381in}{5.331516in}}%
\pgfpathlineto{\pgfqpoint{2.044718in}{5.328539in}}%
\pgfpathlineto{\pgfqpoint{2.045392in}{5.283745in}}%
\pgfpathlineto{\pgfqpoint{2.046656in}{5.201813in}}%
\pgfpathlineto{\pgfqpoint{2.047161in}{5.206874in}}%
\pgfpathlineto{\pgfqpoint{2.048509in}{5.230992in}}%
\pgfpathlineto{\pgfqpoint{2.050194in}{5.313901in}}%
\pgfpathlineto{\pgfqpoint{2.050615in}{5.306228in}}%
\pgfpathlineto{\pgfqpoint{2.052637in}{5.178984in}}%
\pgfpathlineto{\pgfqpoint{2.054153in}{5.109351in}}%
\pgfpathlineto{\pgfqpoint{2.054406in}{5.111853in}}%
\pgfpathlineto{\pgfqpoint{2.055586in}{5.155428in}}%
\pgfpathlineto{\pgfqpoint{2.058113in}{5.307343in}}%
\pgfpathlineto{\pgfqpoint{2.058787in}{5.283607in}}%
\pgfpathlineto{\pgfqpoint{2.060640in}{5.236076in}}%
\pgfpathlineto{\pgfqpoint{2.060725in}{5.235806in}}%
\pgfpathlineto{\pgfqpoint{2.060977in}{5.238683in}}%
\pgfpathlineto{\pgfqpoint{2.062831in}{5.277933in}}%
\pgfpathlineto{\pgfqpoint{2.063252in}{5.265227in}}%
\pgfpathlineto{\pgfqpoint{2.066622in}{5.124394in}}%
\pgfpathlineto{\pgfqpoint{2.066790in}{5.126108in}}%
\pgfpathlineto{\pgfqpoint{2.067801in}{5.178117in}}%
\pgfpathlineto{\pgfqpoint{2.070834in}{5.328924in}}%
\pgfpathlineto{\pgfqpoint{2.071424in}{5.308433in}}%
\pgfpathlineto{\pgfqpoint{2.074541in}{5.110573in}}%
\pgfpathlineto{\pgfqpoint{2.074962in}{5.118114in}}%
\pgfpathlineto{\pgfqpoint{2.076226in}{5.186837in}}%
\pgfpathlineto{\pgfqpoint{2.078416in}{5.306029in}}%
\pgfpathlineto{\pgfqpoint{2.078753in}{5.302854in}}%
\pgfpathlineto{\pgfqpoint{2.079595in}{5.256433in}}%
\pgfpathlineto{\pgfqpoint{2.081701in}{5.176053in}}%
\pgfpathlineto{\pgfqpoint{2.082038in}{5.181462in}}%
\pgfpathlineto{\pgfqpoint{2.083302in}{5.261269in}}%
\pgfpathlineto{\pgfqpoint{2.084650in}{5.317001in}}%
\pgfpathlineto{\pgfqpoint{2.084987in}{5.312832in}}%
\pgfpathlineto{\pgfqpoint{2.087177in}{5.223758in}}%
\pgfpathlineto{\pgfqpoint{2.089368in}{5.139690in}}%
\pgfpathlineto{\pgfqpoint{2.089873in}{5.129163in}}%
\pgfpathlineto{\pgfqpoint{2.090379in}{5.140601in}}%
\pgfpathlineto{\pgfqpoint{2.094338in}{5.293632in}}%
\pgfpathlineto{\pgfqpoint{2.094422in}{5.293888in}}%
\pgfpathlineto{\pgfqpoint{2.094675in}{5.292389in}}%
\pgfpathlineto{\pgfqpoint{2.096697in}{5.251075in}}%
\pgfpathlineto{\pgfqpoint{2.099561in}{5.159384in}}%
\pgfpathlineto{\pgfqpoint{2.099983in}{5.173032in}}%
\pgfpathlineto{\pgfqpoint{2.102173in}{5.268951in}}%
\pgfpathlineto{\pgfqpoint{2.102594in}{5.255652in}}%
\pgfpathlineto{\pgfqpoint{2.105290in}{5.148666in}}%
\pgfpathlineto{\pgfqpoint{2.105880in}{5.160716in}}%
\pgfpathlineto{\pgfqpoint{2.107649in}{5.252786in}}%
\pgfpathlineto{\pgfqpoint{2.108660in}{5.321812in}}%
\pgfpathlineto{\pgfqpoint{2.109249in}{5.311112in}}%
\pgfpathlineto{\pgfqpoint{2.110345in}{5.284413in}}%
\pgfpathlineto{\pgfqpoint{2.112872in}{5.167429in}}%
\pgfpathlineto{\pgfqpoint{2.113546in}{5.165372in}}%
\pgfpathlineto{\pgfqpoint{2.113799in}{5.167936in}}%
\pgfpathlineto{\pgfqpoint{2.120117in}{5.284253in}}%
\pgfpathlineto{\pgfqpoint{2.121465in}{5.289613in}}%
\pgfpathlineto{\pgfqpoint{2.121718in}{5.286721in}}%
\pgfpathlineto{\pgfqpoint{2.123487in}{5.220366in}}%
\pgfpathlineto{\pgfqpoint{2.125761in}{5.085599in}}%
\pgfpathlineto{\pgfqpoint{2.126014in}{5.088019in}}%
\pgfpathlineto{\pgfqpoint{2.126688in}{5.123750in}}%
\pgfpathlineto{\pgfqpoint{2.129468in}{5.449015in}}%
\pgfpathlineto{\pgfqpoint{2.130142in}{5.394914in}}%
\pgfpathlineto{\pgfqpoint{2.132922in}{4.977089in}}%
\pgfpathlineto{\pgfqpoint{2.133933in}{5.012036in}}%
\pgfpathlineto{\pgfqpoint{2.137134in}{5.430786in}}%
\pgfpathlineto{\pgfqpoint{2.137893in}{5.374621in}}%
\pgfpathlineto{\pgfqpoint{2.141431in}{5.039365in}}%
\pgfpathlineto{\pgfqpoint{2.141515in}{5.039374in}}%
\pgfpathlineto{\pgfqpoint{2.142021in}{5.062101in}}%
\pgfpathlineto{\pgfqpoint{2.143874in}{5.271316in}}%
\pgfpathlineto{\pgfqpoint{2.145812in}{5.435861in}}%
\pgfpathlineto{\pgfqpoint{2.145980in}{5.431823in}}%
\pgfpathlineto{\pgfqpoint{2.147665in}{5.305313in}}%
\pgfpathlineto{\pgfqpoint{2.149940in}{5.088032in}}%
\pgfpathlineto{\pgfqpoint{2.150529in}{5.089202in}}%
\pgfpathlineto{\pgfqpoint{2.151035in}{5.113977in}}%
\pgfpathlineto{\pgfqpoint{2.154573in}{5.352907in}}%
\pgfpathlineto{\pgfqpoint{2.154826in}{5.348739in}}%
\pgfpathlineto{\pgfqpoint{2.155668in}{5.254039in}}%
\pgfpathlineto{\pgfqpoint{2.157690in}{5.128785in}}%
\pgfpathlineto{\pgfqpoint{2.158027in}{5.121763in}}%
\pgfpathlineto{\pgfqpoint{2.158532in}{5.134905in}}%
\pgfpathlineto{\pgfqpoint{2.162323in}{5.293856in}}%
\pgfpathlineto{\pgfqpoint{2.162829in}{5.279403in}}%
\pgfpathlineto{\pgfqpoint{2.166788in}{5.116817in}}%
\pgfpathlineto{\pgfqpoint{2.166957in}{5.117957in}}%
\pgfpathlineto{\pgfqpoint{2.168052in}{5.153472in}}%
\pgfpathlineto{\pgfqpoint{2.169653in}{5.277029in}}%
\pgfpathlineto{\pgfqpoint{2.171590in}{5.346120in}}%
\pgfpathlineto{\pgfqpoint{2.172264in}{5.355440in}}%
\pgfpathlineto{\pgfqpoint{2.172601in}{5.346238in}}%
\pgfpathlineto{\pgfqpoint{2.176982in}{5.087370in}}%
\pgfpathlineto{\pgfqpoint{2.177572in}{5.119474in}}%
\pgfpathlineto{\pgfqpoint{2.181026in}{5.350507in}}%
\pgfpathlineto{\pgfqpoint{2.181363in}{5.355649in}}%
\pgfpathlineto{\pgfqpoint{2.181784in}{5.346876in}}%
\pgfpathlineto{\pgfqpoint{2.183890in}{5.210796in}}%
\pgfpathlineto{\pgfqpoint{2.185659in}{5.099984in}}%
\pgfpathlineto{\pgfqpoint{2.186249in}{5.115412in}}%
\pgfpathlineto{\pgfqpoint{2.188608in}{5.270330in}}%
\pgfpathlineto{\pgfqpoint{2.189619in}{5.329603in}}%
\pgfpathlineto{\pgfqpoint{2.190208in}{5.321407in}}%
\pgfpathlineto{\pgfqpoint{2.191051in}{5.304457in}}%
\pgfpathlineto{\pgfqpoint{2.194505in}{5.157094in}}%
\pgfpathlineto{\pgfqpoint{2.194673in}{5.158323in}}%
\pgfpathlineto{\pgfqpoint{2.196779in}{5.233442in}}%
\pgfpathlineto{\pgfqpoint{2.198633in}{5.268128in}}%
\pgfpathlineto{\pgfqpoint{2.198717in}{5.268551in}}%
\pgfpathlineto{\pgfqpoint{2.198970in}{5.266379in}}%
\pgfpathlineto{\pgfqpoint{2.200655in}{5.200900in}}%
\pgfpathlineto{\pgfqpoint{2.202592in}{5.120031in}}%
\pgfpathlineto{\pgfqpoint{2.202677in}{5.120557in}}%
\pgfpathlineto{\pgfqpoint{2.203182in}{5.145504in}}%
\pgfpathlineto{\pgfqpoint{2.205878in}{5.377718in}}%
\pgfpathlineto{\pgfqpoint{2.206383in}{5.360235in}}%
\pgfpathlineto{\pgfqpoint{2.209416in}{5.114115in}}%
\pgfpathlineto{\pgfqpoint{2.210343in}{5.134924in}}%
\pgfpathlineto{\pgfqpoint{2.213039in}{5.284029in}}%
\pgfpathlineto{\pgfqpoint{2.213881in}{5.257763in}}%
\pgfpathlineto{\pgfqpoint{2.216156in}{5.193557in}}%
\pgfpathlineto{\pgfqpoint{2.216240in}{5.193635in}}%
\pgfpathlineto{\pgfqpoint{2.216745in}{5.214248in}}%
\pgfpathlineto{\pgfqpoint{2.219104in}{5.304314in}}%
\pgfpathlineto{\pgfqpoint{2.219189in}{5.304044in}}%
\pgfpathlineto{\pgfqpoint{2.219694in}{5.284215in}}%
\pgfpathlineto{\pgfqpoint{2.222221in}{5.111582in}}%
\pgfpathlineto{\pgfqpoint{2.223064in}{5.126439in}}%
\pgfpathlineto{\pgfqpoint{2.223990in}{5.196622in}}%
\pgfpathlineto{\pgfqpoint{2.225423in}{5.288336in}}%
\pgfpathlineto{\pgfqpoint{2.225760in}{5.285401in}}%
\pgfpathlineto{\pgfqpoint{2.227023in}{5.247262in}}%
\pgfpathlineto{\pgfqpoint{2.228118in}{5.184784in}}%
\pgfpathlineto{\pgfqpoint{2.228792in}{5.192796in}}%
\pgfpathlineto{\pgfqpoint{2.229466in}{5.203908in}}%
\pgfpathlineto{\pgfqpoint{2.231741in}{5.331118in}}%
\pgfpathlineto{\pgfqpoint{2.232499in}{5.296530in}}%
\pgfpathlineto{\pgfqpoint{2.235363in}{5.122444in}}%
\pgfpathlineto{\pgfqpoint{2.235448in}{5.123142in}}%
\pgfpathlineto{\pgfqpoint{2.236122in}{5.163571in}}%
\pgfpathlineto{\pgfqpoint{2.238059in}{5.244125in}}%
\pgfpathlineto{\pgfqpoint{2.238228in}{5.243108in}}%
\pgfpathlineto{\pgfqpoint{2.238565in}{5.240860in}}%
\pgfpathlineto{\pgfqpoint{2.238986in}{5.245597in}}%
\pgfpathlineto{\pgfqpoint{2.239239in}{5.247988in}}%
\pgfpathlineto{\pgfqpoint{2.239576in}{5.239901in}}%
\pgfpathlineto{\pgfqpoint{2.240334in}{5.204238in}}%
\pgfpathlineto{\pgfqpoint{2.240924in}{5.225966in}}%
\pgfpathlineto{\pgfqpoint{2.243704in}{5.389215in}}%
\pgfpathlineto{\pgfqpoint{2.244293in}{5.361898in}}%
\pgfpathlineto{\pgfqpoint{2.247579in}{4.999751in}}%
\pgfpathlineto{\pgfqpoint{2.248421in}{5.059066in}}%
\pgfpathlineto{\pgfqpoint{2.251201in}{5.328352in}}%
\pgfpathlineto{\pgfqpoint{2.251370in}{5.326726in}}%
\pgfpathlineto{\pgfqpoint{2.252044in}{5.288810in}}%
\pgfpathlineto{\pgfqpoint{2.254066in}{5.167077in}}%
\pgfpathlineto{\pgfqpoint{2.254318in}{5.169322in}}%
\pgfpathlineto{\pgfqpoint{2.255077in}{5.211262in}}%
\pgfpathlineto{\pgfqpoint{2.257014in}{5.401533in}}%
\pgfpathlineto{\pgfqpoint{2.257857in}{5.371512in}}%
\pgfpathlineto{\pgfqpoint{2.259120in}{5.196681in}}%
\pgfpathlineto{\pgfqpoint{2.260974in}{5.014217in}}%
\pgfpathlineto{\pgfqpoint{2.261227in}{5.017415in}}%
\pgfpathlineto{\pgfqpoint{2.261985in}{5.084066in}}%
\pgfpathlineto{\pgfqpoint{2.264765in}{5.392377in}}%
\pgfpathlineto{\pgfqpoint{2.265186in}{5.375884in}}%
\pgfpathlineto{\pgfqpoint{2.268387in}{5.130753in}}%
\pgfpathlineto{\pgfqpoint{2.269061in}{5.149702in}}%
\pgfpathlineto{\pgfqpoint{2.272684in}{5.276826in}}%
\pgfpathlineto{\pgfqpoint{2.272852in}{5.275457in}}%
\pgfpathlineto{\pgfqpoint{2.273695in}{5.242105in}}%
\pgfpathlineto{\pgfqpoint{2.274453in}{5.226465in}}%
\pgfpathlineto{\pgfqpoint{2.275043in}{5.228674in}}%
\pgfpathlineto{\pgfqpoint{2.277654in}{5.267933in}}%
\pgfpathlineto{\pgfqpoint{2.278075in}{5.274319in}}%
\pgfpathlineto{\pgfqpoint{2.278412in}{5.264622in}}%
\pgfpathlineto{\pgfqpoint{2.281277in}{5.088051in}}%
\pgfpathlineto{\pgfqpoint{2.282288in}{5.104669in}}%
\pgfpathlineto{\pgfqpoint{2.285489in}{5.324583in}}%
\pgfpathlineto{\pgfqpoint{2.286247in}{5.293020in}}%
\pgfpathlineto{\pgfqpoint{2.288101in}{5.229290in}}%
\pgfpathlineto{\pgfqpoint{2.289027in}{5.203714in}}%
\pgfpathlineto{\pgfqpoint{2.289617in}{5.223263in}}%
\pgfpathlineto{\pgfqpoint{2.290207in}{5.244746in}}%
\pgfpathlineto{\pgfqpoint{2.290881in}{5.232303in}}%
\pgfpathlineto{\pgfqpoint{2.290965in}{5.232041in}}%
\pgfpathlineto{\pgfqpoint{2.291386in}{5.234187in}}%
\pgfpathlineto{\pgfqpoint{2.291555in}{5.234582in}}%
\pgfpathlineto{\pgfqpoint{2.291892in}{5.232060in}}%
\pgfpathlineto{\pgfqpoint{2.292902in}{5.196912in}}%
\pgfpathlineto{\pgfqpoint{2.293998in}{5.169445in}}%
\pgfpathlineto{\pgfqpoint{2.294335in}{5.175105in}}%
\pgfpathlineto{\pgfqpoint{2.297283in}{5.293203in}}%
\pgfpathlineto{\pgfqpoint{2.298126in}{5.272343in}}%
\pgfpathlineto{\pgfqpoint{2.300316in}{5.185482in}}%
\pgfpathlineto{\pgfqpoint{2.300906in}{5.201757in}}%
\pgfpathlineto{\pgfqpoint{2.302759in}{5.284436in}}%
\pgfpathlineto{\pgfqpoint{2.303096in}{5.291098in}}%
\pgfpathlineto{\pgfqpoint{2.303602in}{5.277399in}}%
\pgfpathlineto{\pgfqpoint{2.306719in}{5.113200in}}%
\pgfpathlineto{\pgfqpoint{2.307814in}{5.146983in}}%
\pgfpathlineto{\pgfqpoint{2.310847in}{5.325905in}}%
\pgfpathlineto{\pgfqpoint{2.311773in}{5.286929in}}%
\pgfpathlineto{\pgfqpoint{2.314469in}{5.153959in}}%
\pgfpathlineto{\pgfqpoint{2.314722in}{5.158150in}}%
\pgfpathlineto{\pgfqpoint{2.316407in}{5.235367in}}%
\pgfpathlineto{\pgfqpoint{2.318092in}{5.300314in}}%
\pgfpathlineto{\pgfqpoint{2.318429in}{5.305296in}}%
\pgfpathlineto{\pgfqpoint{2.318766in}{5.296848in}}%
\pgfpathlineto{\pgfqpoint{2.321377in}{5.118587in}}%
\pgfpathlineto{\pgfqpoint{2.322135in}{5.158499in}}%
\pgfpathlineto{\pgfqpoint{2.324915in}{5.404463in}}%
\pgfpathlineto{\pgfqpoint{2.325589in}{5.373236in}}%
\pgfpathlineto{\pgfqpoint{2.327864in}{5.102878in}}%
\pgfpathlineto{\pgfqpoint{2.328706in}{5.060874in}}%
\pgfpathlineto{\pgfqpoint{2.329212in}{5.073643in}}%
\pgfpathlineto{\pgfqpoint{2.332245in}{5.229675in}}%
\pgfpathlineto{\pgfqpoint{2.333003in}{5.208633in}}%
\pgfpathlineto{\pgfqpoint{2.333424in}{5.200004in}}%
\pgfpathlineto{\pgfqpoint{2.333845in}{5.211341in}}%
\pgfpathlineto{\pgfqpoint{2.336373in}{5.321271in}}%
\pgfpathlineto{\pgfqpoint{2.336794in}{5.313638in}}%
\pgfpathlineto{\pgfqpoint{2.338310in}{5.243143in}}%
\pgfpathlineto{\pgfqpoint{2.339405in}{5.126967in}}%
\pgfpathlineto{\pgfqpoint{2.340079in}{5.150449in}}%
\pgfpathlineto{\pgfqpoint{2.342775in}{5.310204in}}%
\pgfpathlineto{\pgfqpoint{2.344292in}{5.289614in}}%
\pgfpathlineto{\pgfqpoint{2.346987in}{5.141933in}}%
\pgfpathlineto{\pgfqpoint{2.347240in}{5.147268in}}%
\pgfpathlineto{\pgfqpoint{2.350020in}{5.252986in}}%
\pgfpathlineto{\pgfqpoint{2.350947in}{5.248965in}}%
\pgfpathlineto{\pgfqpoint{2.351874in}{5.263755in}}%
\pgfpathlineto{\pgfqpoint{2.352548in}{5.256478in}}%
\pgfpathlineto{\pgfqpoint{2.352800in}{5.254561in}}%
\pgfpathlineto{\pgfqpoint{2.353222in}{5.258806in}}%
\pgfpathlineto{\pgfqpoint{2.353896in}{5.271738in}}%
\pgfpathlineto{\pgfqpoint{2.354401in}{5.263854in}}%
\pgfpathlineto{\pgfqpoint{2.356676in}{5.179919in}}%
\pgfpathlineto{\pgfqpoint{2.358360in}{5.091599in}}%
\pgfpathlineto{\pgfqpoint{2.358782in}{5.103575in}}%
\pgfpathlineto{\pgfqpoint{2.361815in}{5.319431in}}%
\pgfpathlineto{\pgfqpoint{2.363247in}{5.297721in}}%
\pgfpathlineto{\pgfqpoint{2.366279in}{5.234298in}}%
\pgfpathlineto{\pgfqpoint{2.366448in}{5.235224in}}%
\pgfpathlineto{\pgfqpoint{2.367038in}{5.243641in}}%
\pgfpathlineto{\pgfqpoint{2.367459in}{5.235814in}}%
\pgfpathlineto{\pgfqpoint{2.370407in}{5.124468in}}%
\pgfpathlineto{\pgfqpoint{2.370744in}{5.131067in}}%
\pgfpathlineto{\pgfqpoint{2.374367in}{5.281715in}}%
\pgfpathlineto{\pgfqpoint{2.374957in}{5.270695in}}%
\pgfpathlineto{\pgfqpoint{2.376220in}{5.195766in}}%
\pgfpathlineto{\pgfqpoint{2.377147in}{5.217599in}}%
\pgfpathlineto{\pgfqpoint{2.377905in}{5.233180in}}%
\pgfpathlineto{\pgfqpoint{2.380180in}{5.297850in}}%
\pgfpathlineto{\pgfqpoint{2.380433in}{5.293910in}}%
\pgfpathlineto{\pgfqpoint{2.381359in}{5.211613in}}%
\pgfpathlineto{\pgfqpoint{2.383718in}{5.042056in}}%
\pgfpathlineto{\pgfqpoint{2.383802in}{5.042879in}}%
\pgfpathlineto{\pgfqpoint{2.384308in}{5.075558in}}%
\pgfpathlineto{\pgfqpoint{2.388015in}{5.405078in}}%
\pgfpathlineto{\pgfqpoint{2.388183in}{5.403163in}}%
\pgfpathlineto{\pgfqpoint{2.388773in}{5.367031in}}%
\pgfpathlineto{\pgfqpoint{2.391637in}{5.163974in}}%
\pgfpathlineto{\pgfqpoint{2.391806in}{5.164943in}}%
\pgfpathlineto{\pgfqpoint{2.392564in}{5.206437in}}%
\pgfpathlineto{\pgfqpoint{2.393069in}{5.223862in}}%
\pgfpathlineto{\pgfqpoint{2.393743in}{5.213590in}}%
\pgfpathlineto{\pgfqpoint{2.394754in}{5.174444in}}%
\pgfpathlineto{\pgfqpoint{2.395260in}{5.160113in}}%
\pgfpathlineto{\pgfqpoint{2.395849in}{5.171418in}}%
\pgfpathlineto{\pgfqpoint{2.397534in}{5.237144in}}%
\pgfpathlineto{\pgfqpoint{2.398966in}{5.356197in}}%
\pgfpathlineto{\pgfqpoint{2.399977in}{5.321227in}}%
\pgfpathlineto{\pgfqpoint{2.400820in}{5.265413in}}%
\pgfpathlineto{\pgfqpoint{2.403094in}{5.059828in}}%
\pgfpathlineto{\pgfqpoint{2.403431in}{5.068337in}}%
\pgfpathlineto{\pgfqpoint{2.404948in}{5.202704in}}%
\pgfpathlineto{\pgfqpoint{2.406633in}{5.349256in}}%
\pgfpathlineto{\pgfqpoint{2.407475in}{5.339479in}}%
\pgfpathlineto{\pgfqpoint{2.407812in}{5.340695in}}%
\pgfpathlineto{\pgfqpoint{2.408065in}{5.337686in}}%
\pgfpathlineto{\pgfqpoint{2.408739in}{5.283881in}}%
\pgfpathlineto{\pgfqpoint{2.410508in}{5.202551in}}%
\pgfpathlineto{\pgfqpoint{2.411772in}{5.155517in}}%
\pgfpathlineto{\pgfqpoint{2.412614in}{5.160934in}}%
\pgfpathlineto{\pgfqpoint{2.413541in}{5.150473in}}%
\pgfpathlineto{\pgfqpoint{2.413962in}{5.157326in}}%
\pgfpathlineto{\pgfqpoint{2.417584in}{5.292042in}}%
\pgfpathlineto{\pgfqpoint{2.418595in}{5.342723in}}%
\pgfpathlineto{\pgfqpoint{2.419185in}{5.336508in}}%
\pgfpathlineto{\pgfqpoint{2.419859in}{5.311441in}}%
\pgfpathlineto{\pgfqpoint{2.422808in}{5.120368in}}%
\pgfpathlineto{\pgfqpoint{2.423481in}{5.137678in}}%
\pgfpathlineto{\pgfqpoint{2.426093in}{5.314386in}}%
\pgfpathlineto{\pgfqpoint{2.427188in}{5.270375in}}%
\pgfpathlineto{\pgfqpoint{2.429968in}{5.172545in}}%
\pgfpathlineto{\pgfqpoint{2.430221in}{5.174439in}}%
\pgfpathlineto{\pgfqpoint{2.430811in}{5.212638in}}%
\pgfpathlineto{\pgfqpoint{2.431569in}{5.260234in}}%
\pgfpathlineto{\pgfqpoint{2.432327in}{5.253022in}}%
\pgfpathlineto{\pgfqpoint{2.433170in}{5.287429in}}%
\pgfpathlineto{\pgfqpoint{2.433759in}{5.266797in}}%
\pgfpathlineto{\pgfqpoint{2.437045in}{5.100393in}}%
\pgfpathlineto{\pgfqpoint{2.437298in}{5.105072in}}%
\pgfpathlineto{\pgfqpoint{2.438730in}{5.202622in}}%
\pgfpathlineto{\pgfqpoint{2.441089in}{5.355859in}}%
\pgfpathlineto{\pgfqpoint{2.441173in}{5.355664in}}%
\pgfpathlineto{\pgfqpoint{2.441594in}{5.340554in}}%
\pgfpathlineto{\pgfqpoint{2.444543in}{5.138210in}}%
\pgfpathlineto{\pgfqpoint{2.445217in}{5.165099in}}%
\pgfpathlineto{\pgfqpoint{2.447491in}{5.331159in}}%
\pgfpathlineto{\pgfqpoint{2.448671in}{5.320589in}}%
\pgfpathlineto{\pgfqpoint{2.449176in}{5.301405in}}%
\pgfpathlineto{\pgfqpoint{2.452125in}{5.167233in}}%
\pgfpathlineto{\pgfqpoint{2.452209in}{5.167663in}}%
\pgfpathlineto{\pgfqpoint{2.452967in}{5.175490in}}%
\pgfpathlineto{\pgfqpoint{2.453557in}{5.170569in}}%
\pgfpathlineto{\pgfqpoint{2.454147in}{5.161771in}}%
\pgfpathlineto{\pgfqpoint{2.454652in}{5.169967in}}%
\pgfpathlineto{\pgfqpoint{2.458022in}{5.282104in}}%
\pgfpathlineto{\pgfqpoint{2.458443in}{5.277607in}}%
\pgfpathlineto{\pgfqpoint{2.459370in}{5.212852in}}%
\pgfpathlineto{\pgfqpoint{2.461055in}{5.171709in}}%
\pgfpathlineto{\pgfqpoint{2.462150in}{5.153075in}}%
\pgfpathlineto{\pgfqpoint{2.462571in}{5.159195in}}%
\pgfpathlineto{\pgfqpoint{2.466446in}{5.302512in}}%
\pgfpathlineto{\pgfqpoint{2.467710in}{5.286676in}}%
\pgfpathlineto{\pgfqpoint{2.468215in}{5.270346in}}%
\pgfpathlineto{\pgfqpoint{2.470911in}{5.178454in}}%
\pgfpathlineto{\pgfqpoint{2.471164in}{5.173933in}}%
\pgfpathlineto{\pgfqpoint{2.471669in}{5.183802in}}%
\pgfpathlineto{\pgfqpoint{2.472680in}{5.203905in}}%
\pgfpathlineto{\pgfqpoint{2.473102in}{5.197365in}}%
\pgfpathlineto{\pgfqpoint{2.473607in}{5.188663in}}%
\pgfpathlineto{\pgfqpoint{2.474112in}{5.196921in}}%
\pgfpathlineto{\pgfqpoint{2.474449in}{5.201505in}}%
\pgfpathlineto{\pgfqpoint{2.475039in}{5.195101in}}%
\pgfpathlineto{\pgfqpoint{2.476977in}{5.164769in}}%
\pgfpathlineto{\pgfqpoint{2.477230in}{5.169202in}}%
\pgfpathlineto{\pgfqpoint{2.479504in}{5.287098in}}%
\pgfpathlineto{\pgfqpoint{2.481357in}{5.381540in}}%
\pgfpathlineto{\pgfqpoint{2.481863in}{5.391844in}}%
\pgfpathlineto{\pgfqpoint{2.482284in}{5.377733in}}%
\pgfpathlineto{\pgfqpoint{2.486075in}{5.097960in}}%
\pgfpathlineto{\pgfqpoint{2.486833in}{5.123294in}}%
\pgfpathlineto{\pgfqpoint{2.489361in}{5.240716in}}%
\pgfpathlineto{\pgfqpoint{2.489529in}{5.243027in}}%
\pgfpathlineto{\pgfqpoint{2.489950in}{5.233692in}}%
\pgfpathlineto{\pgfqpoint{2.490624in}{5.212845in}}%
\pgfpathlineto{\pgfqpoint{2.491046in}{5.226714in}}%
\pgfpathlineto{\pgfqpoint{2.493573in}{5.368750in}}%
\pgfpathlineto{\pgfqpoint{2.493994in}{5.353131in}}%
\pgfpathlineto{\pgfqpoint{2.495932in}{5.133043in}}%
\pgfpathlineto{\pgfqpoint{2.496943in}{5.041863in}}%
\pgfpathlineto{\pgfqpoint{2.497532in}{5.049826in}}%
\pgfpathlineto{\pgfqpoint{2.498291in}{5.074842in}}%
\pgfpathlineto{\pgfqpoint{2.500902in}{5.297688in}}%
\pgfpathlineto{\pgfqpoint{2.501576in}{5.272776in}}%
\pgfpathlineto{\pgfqpoint{2.503851in}{5.168210in}}%
\pgfpathlineto{\pgfqpoint{2.503935in}{5.169014in}}%
\pgfpathlineto{\pgfqpoint{2.504525in}{5.210174in}}%
\pgfpathlineto{\pgfqpoint{2.506968in}{5.368429in}}%
\pgfpathlineto{\pgfqpoint{2.507052in}{5.369050in}}%
\pgfpathlineto{\pgfqpoint{2.507221in}{5.364905in}}%
\pgfpathlineto{\pgfqpoint{2.508484in}{5.232144in}}%
\pgfpathlineto{\pgfqpoint{2.510253in}{5.084922in}}%
\pgfpathlineto{\pgfqpoint{2.510927in}{5.105792in}}%
\pgfpathlineto{\pgfqpoint{2.512275in}{5.204458in}}%
\pgfpathlineto{\pgfqpoint{2.514381in}{5.343757in}}%
\pgfpathlineto{\pgfqpoint{2.514550in}{5.341785in}}%
\pgfpathlineto{\pgfqpoint{2.515308in}{5.285551in}}%
\pgfpathlineto{\pgfqpoint{2.517246in}{5.197063in}}%
\pgfpathlineto{\pgfqpoint{2.517751in}{5.188989in}}%
\pgfpathlineto{\pgfqpoint{2.518172in}{5.196969in}}%
\pgfpathlineto{\pgfqpoint{2.519352in}{5.260370in}}%
\pgfpathlineto{\pgfqpoint{2.520363in}{5.256332in}}%
\pgfpathlineto{\pgfqpoint{2.520700in}{5.249337in}}%
\pgfpathlineto{\pgfqpoint{2.523480in}{5.128976in}}%
\pgfpathlineto{\pgfqpoint{2.524069in}{5.146726in}}%
\pgfpathlineto{\pgfqpoint{2.526681in}{5.258266in}}%
\pgfpathlineto{\pgfqpoint{2.527355in}{5.249353in}}%
\pgfpathlineto{\pgfqpoint{2.527439in}{5.249316in}}%
\pgfpathlineto{\pgfqpoint{2.527523in}{5.250042in}}%
\pgfpathlineto{\pgfqpoint{2.528450in}{5.275932in}}%
\pgfpathlineto{\pgfqpoint{2.528956in}{5.257483in}}%
\pgfpathlineto{\pgfqpoint{2.529545in}{5.237106in}}%
\pgfpathlineto{\pgfqpoint{2.530219in}{5.248161in}}%
\pgfpathlineto{\pgfqpoint{2.532157in}{5.282001in}}%
\pgfpathlineto{\pgfqpoint{2.532325in}{5.283617in}}%
\pgfpathlineto{\pgfqpoint{2.532662in}{5.276690in}}%
\pgfpathlineto{\pgfqpoint{2.535611in}{5.159779in}}%
\pgfpathlineto{\pgfqpoint{2.536369in}{5.169907in}}%
\pgfpathlineto{\pgfqpoint{2.537127in}{5.200671in}}%
\pgfpathlineto{\pgfqpoint{2.539739in}{5.306615in}}%
\pgfpathlineto{\pgfqpoint{2.540076in}{5.300000in}}%
\pgfpathlineto{\pgfqpoint{2.542856in}{5.153806in}}%
\pgfpathlineto{\pgfqpoint{2.544288in}{5.160944in}}%
\pgfpathlineto{\pgfqpoint{2.544625in}{5.156049in}}%
\pgfpathlineto{\pgfqpoint{2.545046in}{5.167213in}}%
\pgfpathlineto{\pgfqpoint{2.545805in}{5.198411in}}%
\pgfpathlineto{\pgfqpoint{2.546394in}{5.185180in}}%
\pgfpathlineto{\pgfqpoint{2.546731in}{5.179218in}}%
\pgfpathlineto{\pgfqpoint{2.547152in}{5.189096in}}%
\pgfpathlineto{\pgfqpoint{2.550270in}{5.316564in}}%
\pgfpathlineto{\pgfqpoint{2.550691in}{5.311918in}}%
\pgfpathlineto{\pgfqpoint{2.552207in}{5.265892in}}%
\pgfpathlineto{\pgfqpoint{2.554734in}{5.119725in}}%
\pgfpathlineto{\pgfqpoint{2.554819in}{5.120007in}}%
\pgfpathlineto{\pgfqpoint{2.555577in}{5.140334in}}%
\pgfpathlineto{\pgfqpoint{2.557178in}{5.199915in}}%
\pgfpathlineto{\pgfqpoint{2.559115in}{5.258242in}}%
\pgfpathlineto{\pgfqpoint{2.559199in}{5.258327in}}%
\pgfpathlineto{\pgfqpoint{2.559368in}{5.257667in}}%
\pgfpathlineto{\pgfqpoint{2.559873in}{5.242666in}}%
\pgfpathlineto{\pgfqpoint{2.561137in}{5.193130in}}%
\pgfpathlineto{\pgfqpoint{2.561727in}{5.198602in}}%
\pgfpathlineto{\pgfqpoint{2.562401in}{5.210322in}}%
\pgfpathlineto{\pgfqpoint{2.565012in}{5.340058in}}%
\pgfpathlineto{\pgfqpoint{2.565686in}{5.316947in}}%
\pgfpathlineto{\pgfqpoint{2.568551in}{5.186457in}}%
\pgfpathlineto{\pgfqpoint{2.568803in}{5.189996in}}%
\pgfpathlineto{\pgfqpoint{2.571331in}{5.260295in}}%
\pgfpathlineto{\pgfqpoint{2.571836in}{5.248252in}}%
\pgfpathlineto{\pgfqpoint{2.574195in}{5.173001in}}%
\pgfpathlineto{\pgfqpoint{2.574616in}{5.180555in}}%
\pgfpathlineto{\pgfqpoint{2.574953in}{5.186430in}}%
\pgfpathlineto{\pgfqpoint{2.575459in}{5.175935in}}%
\pgfpathlineto{\pgfqpoint{2.576048in}{5.162181in}}%
\pgfpathlineto{\pgfqpoint{2.576470in}{5.172152in}}%
\pgfpathlineto{\pgfqpoint{2.578407in}{5.217332in}}%
\pgfpathlineto{\pgfqpoint{2.579502in}{5.235697in}}%
\pgfpathlineto{\pgfqpoint{2.580261in}{5.230124in}}%
\pgfpathlineto{\pgfqpoint{2.581019in}{5.217850in}}%
\pgfpathlineto{\pgfqpoint{2.581608in}{5.224279in}}%
\pgfpathlineto{\pgfqpoint{2.582619in}{5.241458in}}%
\pgfpathlineto{\pgfqpoint{2.583125in}{5.234200in}}%
\pgfpathlineto{\pgfqpoint{2.583799in}{5.221226in}}%
\pgfpathlineto{\pgfqpoint{2.584304in}{5.229453in}}%
\pgfpathlineto{\pgfqpoint{2.584641in}{5.234824in}}%
\pgfpathlineto{\pgfqpoint{2.585063in}{5.222688in}}%
\pgfpathlineto{\pgfqpoint{2.585905in}{5.186527in}}%
\pgfpathlineto{\pgfqpoint{2.586579in}{5.198562in}}%
\pgfpathlineto{\pgfqpoint{2.586832in}{5.195757in}}%
\pgfpathlineto{\pgfqpoint{2.587337in}{5.185874in}}%
\pgfpathlineto{\pgfqpoint{2.587758in}{5.200147in}}%
\pgfpathlineto{\pgfqpoint{2.590370in}{5.301178in}}%
\pgfpathlineto{\pgfqpoint{2.590538in}{5.302007in}}%
\pgfpathlineto{\pgfqpoint{2.590791in}{5.298144in}}%
\pgfpathlineto{\pgfqpoint{2.593740in}{5.176637in}}%
\pgfpathlineto{\pgfqpoint{2.595593in}{5.182468in}}%
\pgfpathlineto{\pgfqpoint{2.596604in}{5.226397in}}%
\pgfpathlineto{\pgfqpoint{2.597699in}{5.269422in}}%
\pgfpathlineto{\pgfqpoint{2.598289in}{5.257225in}}%
\pgfpathlineto{\pgfqpoint{2.599553in}{5.176454in}}%
\pgfpathlineto{\pgfqpoint{2.601406in}{5.118321in}}%
\pgfpathlineto{\pgfqpoint{2.601490in}{5.119267in}}%
\pgfpathlineto{\pgfqpoint{2.602080in}{5.160510in}}%
\pgfpathlineto{\pgfqpoint{2.604355in}{5.359976in}}%
\pgfpathlineto{\pgfqpoint{2.604860in}{5.345540in}}%
\pgfpathlineto{\pgfqpoint{2.606713in}{5.227490in}}%
\pgfpathlineto{\pgfqpoint{2.607303in}{5.195457in}}%
\pgfpathlineto{\pgfqpoint{2.608061in}{5.198153in}}%
\pgfpathlineto{\pgfqpoint{2.608567in}{5.193672in}}%
\pgfpathlineto{\pgfqpoint{2.608904in}{5.199276in}}%
\pgfpathlineto{\pgfqpoint{2.609999in}{5.235794in}}%
\pgfpathlineto{\pgfqpoint{2.610504in}{5.221831in}}%
\pgfpathlineto{\pgfqpoint{2.613116in}{5.106896in}}%
\pgfpathlineto{\pgfqpoint{2.613453in}{5.115136in}}%
\pgfpathlineto{\pgfqpoint{2.615054in}{5.248625in}}%
\pgfpathlineto{\pgfqpoint{2.617075in}{5.358706in}}%
\pgfpathlineto{\pgfqpoint{2.617160in}{5.358409in}}%
\pgfpathlineto{\pgfqpoint{2.617581in}{5.348906in}}%
\pgfpathlineto{\pgfqpoint{2.618929in}{5.229740in}}%
\pgfpathlineto{\pgfqpoint{2.620529in}{5.146309in}}%
\pgfpathlineto{\pgfqpoint{2.620951in}{5.152560in}}%
\pgfpathlineto{\pgfqpoint{2.622551in}{5.228859in}}%
\pgfpathlineto{\pgfqpoint{2.623562in}{5.258680in}}%
\pgfpathlineto{\pgfqpoint{2.624152in}{5.254182in}}%
\pgfpathlineto{\pgfqpoint{2.626090in}{5.235023in}}%
\pgfpathlineto{\pgfqpoint{2.628617in}{5.185917in}}%
\pgfpathlineto{\pgfqpoint{2.628701in}{5.186181in}}%
\pgfpathlineto{\pgfqpoint{2.628954in}{5.187280in}}%
\pgfpathlineto{\pgfqpoint{2.629207in}{5.184600in}}%
\pgfpathlineto{\pgfqpoint{2.630386in}{5.133422in}}%
\pgfpathlineto{\pgfqpoint{2.631060in}{5.157739in}}%
\pgfpathlineto{\pgfqpoint{2.634683in}{5.339054in}}%
\pgfpathlineto{\pgfqpoint{2.635272in}{5.326415in}}%
\pgfpathlineto{\pgfqpoint{2.639737in}{5.127711in}}%
\pgfpathlineto{\pgfqpoint{2.639990in}{5.131958in}}%
\pgfpathlineto{\pgfqpoint{2.641759in}{5.237356in}}%
\pgfpathlineto{\pgfqpoint{2.643697in}{5.300324in}}%
\pgfpathlineto{\pgfqpoint{2.643865in}{5.298533in}}%
\pgfpathlineto{\pgfqpoint{2.645719in}{5.238083in}}%
\pgfpathlineto{\pgfqpoint{2.647740in}{5.165785in}}%
\pgfpathlineto{\pgfqpoint{2.648077in}{5.172395in}}%
\pgfpathlineto{\pgfqpoint{2.652458in}{5.307458in}}%
\pgfpathlineto{\pgfqpoint{2.652879in}{5.297696in}}%
\pgfpathlineto{\pgfqpoint{2.655744in}{5.125590in}}%
\pgfpathlineto{\pgfqpoint{2.656839in}{5.158177in}}%
\pgfpathlineto{\pgfqpoint{2.657934in}{5.228007in}}%
\pgfpathlineto{\pgfqpoint{2.659366in}{5.300192in}}%
\pgfpathlineto{\pgfqpoint{2.659787in}{5.295393in}}%
\pgfpathlineto{\pgfqpoint{2.661051in}{5.245840in}}%
\pgfpathlineto{\pgfqpoint{2.663241in}{5.140781in}}%
\pgfpathlineto{\pgfqpoint{2.663410in}{5.141569in}}%
\pgfpathlineto{\pgfqpoint{2.664168in}{5.165436in}}%
\pgfpathlineto{\pgfqpoint{2.665853in}{5.266495in}}%
\pgfpathlineto{\pgfqpoint{2.666358in}{5.287664in}}%
\pgfpathlineto{\pgfqpoint{2.666948in}{5.271223in}}%
\pgfpathlineto{\pgfqpoint{2.669644in}{5.212598in}}%
\pgfpathlineto{\pgfqpoint{2.669728in}{5.212791in}}%
\pgfpathlineto{\pgfqpoint{2.671750in}{5.227852in}}%
\pgfpathlineto{\pgfqpoint{2.672340in}{5.237456in}}%
\pgfpathlineto{\pgfqpoint{2.672845in}{5.229344in}}%
\pgfpathlineto{\pgfqpoint{2.674951in}{5.175766in}}%
\pgfpathlineto{\pgfqpoint{2.675204in}{5.179993in}}%
\pgfpathlineto{\pgfqpoint{2.678405in}{5.302748in}}%
\pgfpathlineto{\pgfqpoint{2.679753in}{5.284902in}}%
\pgfpathlineto{\pgfqpoint{2.680764in}{5.237246in}}%
\pgfpathlineto{\pgfqpoint{2.681944in}{5.187402in}}%
\pgfpathlineto{\pgfqpoint{2.682449in}{5.200659in}}%
\pgfpathlineto{\pgfqpoint{2.685313in}{5.269764in}}%
\pgfpathlineto{\pgfqpoint{2.685398in}{5.269650in}}%
\pgfpathlineto{\pgfqpoint{2.685987in}{5.257369in}}%
\pgfpathlineto{\pgfqpoint{2.688768in}{5.140155in}}%
\pgfpathlineto{\pgfqpoint{2.690115in}{5.155128in}}%
\pgfpathlineto{\pgfqpoint{2.690874in}{5.195213in}}%
\pgfpathlineto{\pgfqpoint{2.692896in}{5.262184in}}%
\pgfpathlineto{\pgfqpoint{2.693232in}{5.266436in}}%
\pgfpathlineto{\pgfqpoint{2.693822in}{5.259146in}}%
\pgfpathlineto{\pgfqpoint{2.695086in}{5.237109in}}%
\pgfpathlineto{\pgfqpoint{2.695676in}{5.241566in}}%
\pgfpathlineto{\pgfqpoint{2.696939in}{5.251290in}}%
\pgfpathlineto{\pgfqpoint{2.697445in}{5.259527in}}%
\pgfpathlineto{\pgfqpoint{2.698034in}{5.254051in}}%
\pgfpathlineto{\pgfqpoint{2.701910in}{5.170707in}}%
\pgfpathlineto{\pgfqpoint{2.702499in}{5.190736in}}%
\pgfpathlineto{\pgfqpoint{2.703426in}{5.211878in}}%
\pgfpathlineto{\pgfqpoint{2.703932in}{5.211073in}}%
\pgfpathlineto{\pgfqpoint{2.704269in}{5.213991in}}%
\pgfpathlineto{\pgfqpoint{2.705448in}{5.239716in}}%
\pgfpathlineto{\pgfqpoint{2.706459in}{5.237181in}}%
\pgfpathlineto{\pgfqpoint{2.706880in}{5.241282in}}%
\pgfpathlineto{\pgfqpoint{2.707217in}{5.244482in}}%
\pgfpathlineto{\pgfqpoint{2.707807in}{5.238940in}}%
\pgfpathlineto{\pgfqpoint{2.708060in}{5.237911in}}%
\pgfpathlineto{\pgfqpoint{2.708396in}{5.242180in}}%
\pgfpathlineto{\pgfqpoint{2.709070in}{5.249322in}}%
\pgfpathlineto{\pgfqpoint{2.709744in}{5.247417in}}%
\pgfpathlineto{\pgfqpoint{2.710671in}{5.261376in}}%
\pgfpathlineto{\pgfqpoint{2.711092in}{5.251609in}}%
\pgfpathlineto{\pgfqpoint{2.714125in}{5.157318in}}%
\pgfpathlineto{\pgfqpoint{2.714209in}{5.157557in}}%
\pgfpathlineto{\pgfqpoint{2.714715in}{5.166127in}}%
\pgfpathlineto{\pgfqpoint{2.717074in}{5.297470in}}%
\pgfpathlineto{\pgfqpoint{2.717832in}{5.324700in}}%
\pgfpathlineto{\pgfqpoint{2.718422in}{5.317868in}}%
\pgfpathlineto{\pgfqpoint{2.719264in}{5.279010in}}%
\pgfpathlineto{\pgfqpoint{2.721876in}{5.130541in}}%
\pgfpathlineto{\pgfqpoint{2.722044in}{5.132752in}}%
\pgfpathlineto{\pgfqpoint{2.723729in}{5.197334in}}%
\pgfpathlineto{\pgfqpoint{2.726172in}{5.279858in}}%
\pgfpathlineto{\pgfqpoint{2.726256in}{5.279919in}}%
\pgfpathlineto{\pgfqpoint{2.726341in}{5.279351in}}%
\pgfpathlineto{\pgfqpoint{2.728868in}{5.228010in}}%
\pgfpathlineto{\pgfqpoint{2.731479in}{5.166786in}}%
\pgfpathlineto{\pgfqpoint{2.731901in}{5.174453in}}%
\pgfpathlineto{\pgfqpoint{2.732406in}{5.183527in}}%
\pgfpathlineto{\pgfqpoint{2.733080in}{5.176689in}}%
\pgfpathlineto{\pgfqpoint{2.733501in}{5.195619in}}%
\pgfpathlineto{\pgfqpoint{2.736197in}{5.333020in}}%
\pgfpathlineto{\pgfqpoint{2.736534in}{5.325584in}}%
\pgfpathlineto{\pgfqpoint{2.738388in}{5.194591in}}%
\pgfpathlineto{\pgfqpoint{2.740072in}{5.102364in}}%
\pgfpathlineto{\pgfqpoint{2.740409in}{5.105387in}}%
\pgfpathlineto{\pgfqpoint{2.741589in}{5.148488in}}%
\pgfpathlineto{\pgfqpoint{2.744285in}{5.335199in}}%
\pgfpathlineto{\pgfqpoint{2.745127in}{5.320853in}}%
\pgfpathlineto{\pgfqpoint{2.745717in}{5.298353in}}%
\pgfpathlineto{\pgfqpoint{2.748328in}{5.121427in}}%
\pgfpathlineto{\pgfqpoint{2.748834in}{5.136511in}}%
\pgfpathlineto{\pgfqpoint{2.751614in}{5.356262in}}%
\pgfpathlineto{\pgfqpoint{2.752372in}{5.315758in}}%
\pgfpathlineto{\pgfqpoint{2.754899in}{5.128915in}}%
\pgfpathlineto{\pgfqpoint{2.755321in}{5.139078in}}%
\pgfpathlineto{\pgfqpoint{2.756921in}{5.263429in}}%
\pgfpathlineto{\pgfqpoint{2.758354in}{5.324474in}}%
\pgfpathlineto{\pgfqpoint{2.758522in}{5.323642in}}%
\pgfpathlineto{\pgfqpoint{2.759112in}{5.293928in}}%
\pgfpathlineto{\pgfqpoint{2.761555in}{5.094092in}}%
\pgfpathlineto{\pgfqpoint{2.762145in}{5.105612in}}%
\pgfpathlineto{\pgfqpoint{2.763324in}{5.228893in}}%
\pgfpathlineto{\pgfqpoint{2.765009in}{5.338159in}}%
\pgfpathlineto{\pgfqpoint{2.765346in}{5.325205in}}%
\pgfpathlineto{\pgfqpoint{2.768379in}{5.121744in}}%
\pgfpathlineto{\pgfqpoint{2.768800in}{5.135716in}}%
\pgfpathlineto{\pgfqpoint{2.771243in}{5.367743in}}%
\pgfpathlineto{\pgfqpoint{2.772085in}{5.318278in}}%
\pgfpathlineto{\pgfqpoint{2.775034in}{5.097629in}}%
\pgfpathlineto{\pgfqpoint{2.775287in}{5.102413in}}%
\pgfpathlineto{\pgfqpoint{2.776550in}{5.199706in}}%
\pgfpathlineto{\pgfqpoint{2.779162in}{5.367425in}}%
\pgfpathlineto{\pgfqpoint{2.779330in}{5.365241in}}%
\pgfpathlineto{\pgfqpoint{2.781437in}{5.245571in}}%
\pgfpathlineto{\pgfqpoint{2.783290in}{5.150819in}}%
\pgfpathlineto{\pgfqpoint{2.784048in}{5.151611in}}%
\pgfpathlineto{\pgfqpoint{2.784301in}{5.153388in}}%
\pgfpathlineto{\pgfqpoint{2.785312in}{5.191754in}}%
\pgfpathlineto{\pgfqpoint{2.786744in}{5.265862in}}%
\pgfpathlineto{\pgfqpoint{2.787334in}{5.245722in}}%
\pgfpathlineto{\pgfqpoint{2.789608in}{5.185133in}}%
\pgfpathlineto{\pgfqpoint{2.789692in}{5.185764in}}%
\pgfpathlineto{\pgfqpoint{2.793989in}{5.241515in}}%
\pgfpathlineto{\pgfqpoint{2.794494in}{5.247846in}}%
\pgfpathlineto{\pgfqpoint{2.794831in}{5.241369in}}%
\pgfpathlineto{\pgfqpoint{2.795674in}{5.206126in}}%
\pgfpathlineto{\pgfqpoint{2.796348in}{5.221101in}}%
\pgfpathlineto{\pgfqpoint{2.796601in}{5.224281in}}%
\pgfpathlineto{\pgfqpoint{2.797022in}{5.215453in}}%
\pgfpathlineto{\pgfqpoint{2.797274in}{5.211138in}}%
\pgfpathlineto{\pgfqpoint{2.797780in}{5.224074in}}%
\pgfpathlineto{\pgfqpoint{2.799802in}{5.311160in}}%
\pgfpathlineto{\pgfqpoint{2.800476in}{5.290506in}}%
\pgfpathlineto{\pgfqpoint{2.803340in}{5.100660in}}%
\pgfpathlineto{\pgfqpoint{2.804520in}{5.144091in}}%
\pgfpathlineto{\pgfqpoint{2.805025in}{5.164821in}}%
\pgfpathlineto{\pgfqpoint{2.806289in}{5.276192in}}%
\pgfpathlineto{\pgfqpoint{2.807131in}{5.266181in}}%
\pgfpathlineto{\pgfqpoint{2.807384in}{5.268456in}}%
\pgfpathlineto{\pgfqpoint{2.807721in}{5.262284in}}%
\pgfpathlineto{\pgfqpoint{2.810164in}{5.155611in}}%
\pgfpathlineto{\pgfqpoint{2.810669in}{5.166135in}}%
\pgfpathlineto{\pgfqpoint{2.813618in}{5.363151in}}%
\pgfpathlineto{\pgfqpoint{2.814208in}{5.321859in}}%
\pgfpathlineto{\pgfqpoint{2.816398in}{5.170887in}}%
\pgfpathlineto{\pgfqpoint{2.816651in}{5.165727in}}%
\pgfpathlineto{\pgfqpoint{2.817072in}{5.178837in}}%
\pgfpathlineto{\pgfqpoint{2.819178in}{5.265431in}}%
\pgfpathlineto{\pgfqpoint{2.819431in}{5.262958in}}%
\pgfpathlineto{\pgfqpoint{2.820694in}{5.200963in}}%
\pgfpathlineto{\pgfqpoint{2.822295in}{5.122220in}}%
\pgfpathlineto{\pgfqpoint{2.822885in}{5.141332in}}%
\pgfpathlineto{\pgfqpoint{2.825581in}{5.320926in}}%
\pgfpathlineto{\pgfqpoint{2.826339in}{5.306540in}}%
\pgfpathlineto{\pgfqpoint{2.827434in}{5.247422in}}%
\pgfpathlineto{\pgfqpoint{2.829456in}{5.131751in}}%
\pgfpathlineto{\pgfqpoint{2.829624in}{5.133279in}}%
\pgfpathlineto{\pgfqpoint{2.830383in}{5.176621in}}%
\pgfpathlineto{\pgfqpoint{2.832910in}{5.353039in}}%
\pgfpathlineto{\pgfqpoint{2.833247in}{5.341207in}}%
\pgfpathlineto{\pgfqpoint{2.836027in}{5.091124in}}%
\pgfpathlineto{\pgfqpoint{2.836869in}{5.128222in}}%
\pgfpathlineto{\pgfqpoint{2.838976in}{5.361724in}}%
\pgfpathlineto{\pgfqpoint{2.840155in}{5.309323in}}%
\pgfpathlineto{\pgfqpoint{2.842261in}{5.073937in}}%
\pgfpathlineto{\pgfqpoint{2.843272in}{5.127955in}}%
\pgfpathlineto{\pgfqpoint{2.846136in}{5.348183in}}%
\pgfpathlineto{\pgfqpoint{2.846810in}{5.313680in}}%
\pgfpathlineto{\pgfqpoint{2.849169in}{5.189858in}}%
\pgfpathlineto{\pgfqpoint{2.849338in}{5.188348in}}%
\pgfpathlineto{\pgfqpoint{2.849759in}{5.195686in}}%
\pgfpathlineto{\pgfqpoint{2.851191in}{5.270771in}}%
\pgfpathlineto{\pgfqpoint{2.852286in}{5.254590in}}%
\pgfpathlineto{\pgfqpoint{2.854055in}{5.129414in}}%
\pgfpathlineto{\pgfqpoint{2.855487in}{5.153657in}}%
\pgfpathlineto{\pgfqpoint{2.857931in}{5.340752in}}%
\pgfpathlineto{\pgfqpoint{2.858604in}{5.305718in}}%
\pgfpathlineto{\pgfqpoint{2.861300in}{5.127715in}}%
\pgfpathlineto{\pgfqpoint{2.861385in}{5.128640in}}%
\pgfpathlineto{\pgfqpoint{2.862311in}{5.180849in}}%
\pgfpathlineto{\pgfqpoint{2.864249in}{5.256219in}}%
\pgfpathlineto{\pgfqpoint{2.864754in}{5.247394in}}%
\pgfpathlineto{\pgfqpoint{2.866523in}{5.182717in}}%
\pgfpathlineto{\pgfqpoint{2.867029in}{5.189614in}}%
\pgfpathlineto{\pgfqpoint{2.868040in}{5.271935in}}%
\pgfpathlineto{\pgfqpoint{2.869893in}{5.365417in}}%
\pgfpathlineto{\pgfqpoint{2.870062in}{5.360831in}}%
\pgfpathlineto{\pgfqpoint{2.871241in}{5.229371in}}%
\pgfpathlineto{\pgfqpoint{2.873095in}{5.077378in}}%
\pgfpathlineto{\pgfqpoint{2.873769in}{5.097860in}}%
\pgfpathlineto{\pgfqpoint{2.875285in}{5.239546in}}%
\pgfpathlineto{\pgfqpoint{2.877223in}{5.336970in}}%
\pgfpathlineto{\pgfqpoint{2.877391in}{5.335631in}}%
\pgfpathlineto{\pgfqpoint{2.878065in}{5.302429in}}%
\pgfpathlineto{\pgfqpoint{2.880340in}{5.161492in}}%
\pgfpathlineto{\pgfqpoint{2.880845in}{5.173323in}}%
\pgfpathlineto{\pgfqpoint{2.882951in}{5.257011in}}%
\pgfpathlineto{\pgfqpoint{2.883457in}{5.246475in}}%
\pgfpathlineto{\pgfqpoint{2.884973in}{5.147006in}}%
\pgfpathlineto{\pgfqpoint{2.886405in}{5.067499in}}%
\pgfpathlineto{\pgfqpoint{2.886742in}{5.076740in}}%
\pgfpathlineto{\pgfqpoint{2.888006in}{5.218801in}}%
\pgfpathlineto{\pgfqpoint{2.890449in}{5.415683in}}%
\pgfpathlineto{\pgfqpoint{2.890617in}{5.414591in}}%
\pgfpathlineto{\pgfqpoint{2.891123in}{5.379980in}}%
\pgfpathlineto{\pgfqpoint{2.893734in}{5.069273in}}%
\pgfpathlineto{\pgfqpoint{2.894324in}{5.100343in}}%
\pgfpathlineto{\pgfqpoint{2.896936in}{5.343376in}}%
\pgfpathlineto{\pgfqpoint{2.897778in}{5.305965in}}%
\pgfpathlineto{\pgfqpoint{2.900137in}{5.112005in}}%
\pgfpathlineto{\pgfqpoint{2.901316in}{5.157891in}}%
\pgfpathlineto{\pgfqpoint{2.904097in}{5.347738in}}%
\pgfpathlineto{\pgfqpoint{2.904518in}{5.332608in}}%
\pgfpathlineto{\pgfqpoint{2.907972in}{5.096447in}}%
\pgfpathlineto{\pgfqpoint{2.908730in}{5.139753in}}%
\pgfpathlineto{\pgfqpoint{2.911931in}{5.301159in}}%
\pgfpathlineto{\pgfqpoint{2.912268in}{5.292996in}}%
\pgfpathlineto{\pgfqpoint{2.915470in}{5.175527in}}%
\pgfpathlineto{\pgfqpoint{2.915975in}{5.193562in}}%
\pgfpathlineto{\pgfqpoint{2.916565in}{5.212186in}}%
\pgfpathlineto{\pgfqpoint{2.917154in}{5.201345in}}%
\pgfpathlineto{\pgfqpoint{2.917323in}{5.200467in}}%
\pgfpathlineto{\pgfqpoint{2.917576in}{5.205528in}}%
\pgfpathlineto{\pgfqpoint{2.917913in}{5.213798in}}%
\pgfpathlineto{\pgfqpoint{2.918334in}{5.199766in}}%
\pgfpathlineto{\pgfqpoint{2.919176in}{5.142629in}}%
\pgfpathlineto{\pgfqpoint{2.919682in}{5.168956in}}%
\pgfpathlineto{\pgfqpoint{2.922546in}{5.341499in}}%
\pgfpathlineto{\pgfqpoint{2.923473in}{5.347395in}}%
\pgfpathlineto{\pgfqpoint{2.923726in}{5.343172in}}%
\pgfpathlineto{\pgfqpoint{2.924652in}{5.251825in}}%
\pgfpathlineto{\pgfqpoint{2.926674in}{5.130723in}}%
\pgfpathlineto{\pgfqpoint{2.926927in}{5.135304in}}%
\pgfpathlineto{\pgfqpoint{2.929875in}{5.271259in}}%
\pgfpathlineto{\pgfqpoint{2.930465in}{5.262173in}}%
\pgfpathlineto{\pgfqpoint{2.930634in}{5.261519in}}%
\pgfpathlineto{\pgfqpoint{2.931055in}{5.266133in}}%
\pgfpathlineto{\pgfqpoint{2.931308in}{5.267208in}}%
\pgfpathlineto{\pgfqpoint{2.931560in}{5.263689in}}%
\pgfpathlineto{\pgfqpoint{2.934340in}{5.157395in}}%
\pgfpathlineto{\pgfqpoint{2.934930in}{5.181359in}}%
\pgfpathlineto{\pgfqpoint{2.937373in}{5.293987in}}%
\pgfpathlineto{\pgfqpoint{2.937710in}{5.298867in}}%
\pgfpathlineto{\pgfqpoint{2.938131in}{5.290188in}}%
\pgfpathlineto{\pgfqpoint{2.941922in}{5.170717in}}%
\pgfpathlineto{\pgfqpoint{2.942091in}{5.169901in}}%
\pgfpathlineto{\pgfqpoint{2.942344in}{5.175162in}}%
\pgfpathlineto{\pgfqpoint{2.944955in}{5.244802in}}%
\pgfpathlineto{\pgfqpoint{2.945039in}{5.244209in}}%
\pgfpathlineto{\pgfqpoint{2.946219in}{5.208226in}}%
\pgfpathlineto{\pgfqpoint{2.947314in}{5.219022in}}%
\pgfpathlineto{\pgfqpoint{2.949420in}{5.277096in}}%
\pgfpathlineto{\pgfqpoint{2.949926in}{5.263902in}}%
\pgfpathlineto{\pgfqpoint{2.953632in}{5.120364in}}%
\pgfpathlineto{\pgfqpoint{2.953969in}{5.128588in}}%
\pgfpathlineto{\pgfqpoint{2.955738in}{5.278376in}}%
\pgfpathlineto{\pgfqpoint{2.957508in}{5.397060in}}%
\pgfpathlineto{\pgfqpoint{2.957929in}{5.389081in}}%
\pgfpathlineto{\pgfqpoint{2.959277in}{5.307048in}}%
\pgfpathlineto{\pgfqpoint{2.961888in}{5.056076in}}%
\pgfpathlineto{\pgfqpoint{2.962141in}{5.061512in}}%
\pgfpathlineto{\pgfqpoint{2.963994in}{5.165317in}}%
\pgfpathlineto{\pgfqpoint{2.966606in}{5.328549in}}%
\pgfpathlineto{\pgfqpoint{2.966859in}{5.325152in}}%
\pgfpathlineto{\pgfqpoint{2.968712in}{5.223620in}}%
\pgfpathlineto{\pgfqpoint{2.969807in}{5.147761in}}%
\pgfpathlineto{\pgfqpoint{2.970397in}{5.171076in}}%
\pgfpathlineto{\pgfqpoint{2.972587in}{5.247372in}}%
\pgfpathlineto{\pgfqpoint{2.972840in}{5.245267in}}%
\pgfpathlineto{\pgfqpoint{2.973093in}{5.243078in}}%
\pgfpathlineto{\pgfqpoint{2.973514in}{5.248493in}}%
\pgfpathlineto{\pgfqpoint{2.974272in}{5.266380in}}%
\pgfpathlineto{\pgfqpoint{2.974693in}{5.253647in}}%
\pgfpathlineto{\pgfqpoint{2.975367in}{5.226235in}}%
\pgfpathlineto{\pgfqpoint{2.976041in}{5.239524in}}%
\pgfpathlineto{\pgfqpoint{2.976126in}{5.239895in}}%
\pgfpathlineto{\pgfqpoint{2.976294in}{5.236987in}}%
\pgfpathlineto{\pgfqpoint{2.977221in}{5.191728in}}%
\pgfpathlineto{\pgfqpoint{2.977726in}{5.217636in}}%
\pgfpathlineto{\pgfqpoint{2.978316in}{5.247516in}}%
\pgfpathlineto{\pgfqpoint{2.978906in}{5.227560in}}%
\pgfpathlineto{\pgfqpoint{2.979411in}{5.215928in}}%
\pgfpathlineto{\pgfqpoint{2.980169in}{5.220325in}}%
\pgfpathlineto{\pgfqpoint{2.982528in}{5.201376in}}%
\pgfpathlineto{\pgfqpoint{2.982865in}{5.202658in}}%
\pgfpathlineto{\pgfqpoint{2.983455in}{5.215887in}}%
\pgfpathlineto{\pgfqpoint{2.986151in}{5.288706in}}%
\pgfpathlineto{\pgfqpoint{2.986235in}{5.288213in}}%
\pgfpathlineto{\pgfqpoint{2.986993in}{5.256550in}}%
\pgfpathlineto{\pgfqpoint{2.989099in}{5.153171in}}%
\pgfpathlineto{\pgfqpoint{2.989773in}{5.179789in}}%
\pgfpathlineto{\pgfqpoint{2.993311in}{5.286718in}}%
\pgfpathlineto{\pgfqpoint{2.993817in}{5.292294in}}%
\pgfpathlineto{\pgfqpoint{2.994154in}{5.286889in}}%
\pgfpathlineto{\pgfqpoint{2.997271in}{5.171845in}}%
\pgfpathlineto{\pgfqpoint{2.997776in}{5.176541in}}%
\pgfpathlineto{\pgfqpoint{2.998113in}{5.169133in}}%
\pgfpathlineto{\pgfqpoint{2.998872in}{5.140645in}}%
\pgfpathlineto{\pgfqpoint{2.999461in}{5.153304in}}%
\pgfpathlineto{\pgfqpoint{3.001483in}{5.235777in}}%
\pgfpathlineto{\pgfqpoint{3.003674in}{5.312112in}}%
\pgfpathlineto{\pgfqpoint{3.004095in}{5.302205in}}%
\pgfpathlineto{\pgfqpoint{3.008391in}{5.142396in}}%
\pgfpathlineto{\pgfqpoint{3.008812in}{5.151911in}}%
\pgfpathlineto{\pgfqpoint{3.009571in}{5.173784in}}%
\pgfpathlineto{\pgfqpoint{3.010076in}{5.161005in}}%
\pgfpathlineto{\pgfqpoint{3.010329in}{5.155410in}}%
\pgfpathlineto{\pgfqpoint{3.010750in}{5.170312in}}%
\pgfpathlineto{\pgfqpoint{3.013699in}{5.326123in}}%
\pgfpathlineto{\pgfqpoint{3.014878in}{5.345467in}}%
\pgfpathlineto{\pgfqpoint{3.015215in}{5.335792in}}%
\pgfpathlineto{\pgfqpoint{3.018248in}{5.100935in}}%
\pgfpathlineto{\pgfqpoint{3.019006in}{5.147664in}}%
\pgfpathlineto{\pgfqpoint{3.022292in}{5.323702in}}%
\pgfpathlineto{\pgfqpoint{3.022376in}{5.324027in}}%
\pgfpathlineto{\pgfqpoint{3.022629in}{5.321856in}}%
\pgfpathlineto{\pgfqpoint{3.023387in}{5.270903in}}%
\pgfpathlineto{\pgfqpoint{3.025830in}{5.120240in}}%
\pgfpathlineto{\pgfqpoint{3.026167in}{5.129211in}}%
\pgfpathlineto{\pgfqpoint{3.029200in}{5.298754in}}%
\pgfpathlineto{\pgfqpoint{3.030632in}{5.295185in}}%
\pgfpathlineto{\pgfqpoint{3.031306in}{5.242377in}}%
\pgfpathlineto{\pgfqpoint{3.033328in}{5.133649in}}%
\pgfpathlineto{\pgfqpoint{3.033580in}{5.128828in}}%
\pgfpathlineto{\pgfqpoint{3.034002in}{5.144183in}}%
\pgfpathlineto{\pgfqpoint{3.036782in}{5.282435in}}%
\pgfpathlineto{\pgfqpoint{3.037034in}{5.280878in}}%
\pgfpathlineto{\pgfqpoint{3.039141in}{5.233016in}}%
\pgfpathlineto{\pgfqpoint{3.040657in}{5.211038in}}%
\pgfpathlineto{\pgfqpoint{3.040994in}{5.205497in}}%
\pgfpathlineto{\pgfqpoint{3.041415in}{5.216878in}}%
\pgfpathlineto{\pgfqpoint{3.042426in}{5.273161in}}%
\pgfpathlineto{\pgfqpoint{3.043100in}{5.257872in}}%
\pgfpathlineto{\pgfqpoint{3.044532in}{5.197978in}}%
\pgfpathlineto{\pgfqpoint{3.046386in}{5.126605in}}%
\pgfpathlineto{\pgfqpoint{3.046470in}{5.126955in}}%
\pgfpathlineto{\pgfqpoint{3.046975in}{5.148007in}}%
\pgfpathlineto{\pgfqpoint{3.050177in}{5.420881in}}%
\pgfpathlineto{\pgfqpoint{3.050766in}{5.375688in}}%
\pgfpathlineto{\pgfqpoint{3.053378in}{5.047839in}}%
\pgfpathlineto{\pgfqpoint{3.054052in}{5.066813in}}%
\pgfpathlineto{\pgfqpoint{3.054978in}{5.139393in}}%
\pgfpathlineto{\pgfqpoint{3.056916in}{5.311775in}}%
\pgfpathlineto{\pgfqpoint{3.057253in}{5.308058in}}%
\pgfpathlineto{\pgfqpoint{3.058433in}{5.271901in}}%
\pgfpathlineto{\pgfqpoint{3.059528in}{5.186379in}}%
\pgfpathlineto{\pgfqpoint{3.060286in}{5.206946in}}%
\pgfpathlineto{\pgfqpoint{3.060370in}{5.207484in}}%
\pgfpathlineto{\pgfqpoint{3.060791in}{5.203863in}}%
\pgfpathlineto{\pgfqpoint{3.060876in}{5.203385in}}%
\pgfpathlineto{\pgfqpoint{3.061128in}{5.205636in}}%
\pgfpathlineto{\pgfqpoint{3.063487in}{5.268340in}}%
\pgfpathlineto{\pgfqpoint{3.063824in}{5.255148in}}%
\pgfpathlineto{\pgfqpoint{3.066520in}{5.134347in}}%
\pgfpathlineto{\pgfqpoint{3.066604in}{5.134589in}}%
\pgfpathlineto{\pgfqpoint{3.067194in}{5.172261in}}%
\pgfpathlineto{\pgfqpoint{3.069890in}{5.322081in}}%
\pgfpathlineto{\pgfqpoint{3.070058in}{5.321291in}}%
\pgfpathlineto{\pgfqpoint{3.071153in}{5.291182in}}%
\pgfpathlineto{\pgfqpoint{3.073765in}{5.151414in}}%
\pgfpathlineto{\pgfqpoint{3.074439in}{5.171194in}}%
\pgfpathlineto{\pgfqpoint{3.077303in}{5.282291in}}%
\pgfpathlineto{\pgfqpoint{3.077388in}{5.282987in}}%
\pgfpathlineto{\pgfqpoint{3.077640in}{5.277987in}}%
\pgfpathlineto{\pgfqpoint{3.080168in}{5.123718in}}%
\pgfpathlineto{\pgfqpoint{3.080757in}{5.157925in}}%
\pgfpathlineto{\pgfqpoint{3.083032in}{5.360702in}}%
\pgfpathlineto{\pgfqpoint{3.083622in}{5.342142in}}%
\pgfpathlineto{\pgfqpoint{3.086486in}{5.153747in}}%
\pgfpathlineto{\pgfqpoint{3.087750in}{5.179204in}}%
\pgfpathlineto{\pgfqpoint{3.089098in}{5.253151in}}%
\pgfpathlineto{\pgfqpoint{3.090698in}{5.242871in}}%
\pgfpathlineto{\pgfqpoint{3.092973in}{5.106968in}}%
\pgfpathlineto{\pgfqpoint{3.093394in}{5.120578in}}%
\pgfpathlineto{\pgfqpoint{3.096764in}{5.345748in}}%
\pgfpathlineto{\pgfqpoint{3.097690in}{5.315444in}}%
\pgfpathlineto{\pgfqpoint{3.100808in}{5.171139in}}%
\pgfpathlineto{\pgfqpoint{3.101650in}{5.187055in}}%
\pgfpathlineto{\pgfqpoint{3.104346in}{5.234686in}}%
\pgfpathlineto{\pgfqpoint{3.104430in}{5.235096in}}%
\pgfpathlineto{\pgfqpoint{3.104851in}{5.232184in}}%
\pgfpathlineto{\pgfqpoint{3.106031in}{5.221537in}}%
\pgfpathlineto{\pgfqpoint{3.106536in}{5.213673in}}%
\pgfpathlineto{\pgfqpoint{3.107042in}{5.220580in}}%
\pgfpathlineto{\pgfqpoint{3.108558in}{5.282165in}}%
\pgfpathlineto{\pgfqpoint{3.109232in}{5.263546in}}%
\pgfpathlineto{\pgfqpoint{3.109990in}{5.269256in}}%
\pgfpathlineto{\pgfqpoint{3.110833in}{5.208148in}}%
\pgfpathlineto{\pgfqpoint{3.112939in}{5.088552in}}%
\pgfpathlineto{\pgfqpoint{3.113023in}{5.088554in}}%
\pgfpathlineto{\pgfqpoint{3.113444in}{5.108757in}}%
\pgfpathlineto{\pgfqpoint{3.116477in}{5.396895in}}%
\pgfpathlineto{\pgfqpoint{3.117235in}{5.362143in}}%
\pgfpathlineto{\pgfqpoint{3.120436in}{5.060561in}}%
\pgfpathlineto{\pgfqpoint{3.121363in}{5.127234in}}%
\pgfpathlineto{\pgfqpoint{3.123806in}{5.335564in}}%
\pgfpathlineto{\pgfqpoint{3.124564in}{5.310939in}}%
\pgfpathlineto{\pgfqpoint{3.126418in}{5.230458in}}%
\pgfpathlineto{\pgfqpoint{3.127008in}{5.263794in}}%
\pgfpathlineto{\pgfqpoint{3.127513in}{5.278398in}}%
\pgfpathlineto{\pgfqpoint{3.128187in}{5.270968in}}%
\pgfpathlineto{\pgfqpoint{3.129114in}{5.246357in}}%
\pgfpathlineto{\pgfqpoint{3.131557in}{5.068752in}}%
\pgfpathlineto{\pgfqpoint{3.132315in}{5.080870in}}%
\pgfpathlineto{\pgfqpoint{3.133663in}{5.239229in}}%
\pgfpathlineto{\pgfqpoint{3.135179in}{5.391602in}}%
\pgfpathlineto{\pgfqpoint{3.135769in}{5.362562in}}%
\pgfpathlineto{\pgfqpoint{3.138128in}{5.125150in}}%
\pgfpathlineto{\pgfqpoint{3.138549in}{5.104232in}}%
\pgfpathlineto{\pgfqpoint{3.139055in}{5.131625in}}%
\pgfpathlineto{\pgfqpoint{3.142087in}{5.367126in}}%
\pgfpathlineto{\pgfqpoint{3.142172in}{5.367078in}}%
\pgfpathlineto{\pgfqpoint{3.142593in}{5.353346in}}%
\pgfpathlineto{\pgfqpoint{3.143941in}{5.203251in}}%
\pgfpathlineto{\pgfqpoint{3.145878in}{5.056398in}}%
\pgfpathlineto{\pgfqpoint{3.146131in}{5.059764in}}%
\pgfpathlineto{\pgfqpoint{3.146805in}{5.111877in}}%
\pgfpathlineto{\pgfqpoint{3.149332in}{5.325143in}}%
\pgfpathlineto{\pgfqpoint{3.149417in}{5.323824in}}%
\pgfpathlineto{\pgfqpoint{3.151017in}{5.243596in}}%
\pgfpathlineto{\pgfqpoint{3.152281in}{5.184572in}}%
\pgfpathlineto{\pgfqpoint{3.152871in}{5.198537in}}%
\pgfpathlineto{\pgfqpoint{3.154387in}{5.276365in}}%
\pgfpathlineto{\pgfqpoint{3.155145in}{5.316887in}}%
\pgfpathlineto{\pgfqpoint{3.155651in}{5.297088in}}%
\pgfpathlineto{\pgfqpoint{3.159357in}{5.148317in}}%
\pgfpathlineto{\pgfqpoint{3.159442in}{5.148623in}}%
\pgfpathlineto{\pgfqpoint{3.159863in}{5.163749in}}%
\pgfpathlineto{\pgfqpoint{3.162643in}{5.297391in}}%
\pgfpathlineto{\pgfqpoint{3.162896in}{5.292643in}}%
\pgfpathlineto{\pgfqpoint{3.167024in}{5.169160in}}%
\pgfpathlineto{\pgfqpoint{3.167276in}{5.173047in}}%
\pgfpathlineto{\pgfqpoint{3.170730in}{5.292978in}}%
\pgfpathlineto{\pgfqpoint{3.171489in}{5.290458in}}%
\pgfpathlineto{\pgfqpoint{3.171910in}{5.292056in}}%
\pgfpathlineto{\pgfqpoint{3.172247in}{5.289301in}}%
\pgfpathlineto{\pgfqpoint{3.172837in}{5.251630in}}%
\pgfpathlineto{\pgfqpoint{3.173932in}{5.162900in}}%
\pgfpathlineto{\pgfqpoint{3.174606in}{5.176347in}}%
\pgfpathlineto{\pgfqpoint{3.175027in}{5.162547in}}%
\pgfpathlineto{\pgfqpoint{3.175532in}{5.146062in}}%
\pgfpathlineto{\pgfqpoint{3.176122in}{5.163102in}}%
\pgfpathlineto{\pgfqpoint{3.178060in}{5.231663in}}%
\pgfpathlineto{\pgfqpoint{3.180250in}{5.284955in}}%
\pgfpathlineto{\pgfqpoint{3.178649in}{5.231432in}}%
\pgfpathlineto{\pgfqpoint{3.180587in}{5.281003in}}%
\pgfpathlineto{\pgfqpoint{3.183704in}{5.214858in}}%
\pgfpathlineto{\pgfqpoint{3.184884in}{5.215627in}}%
\pgfpathlineto{\pgfqpoint{3.185221in}{5.222531in}}%
\pgfpathlineto{\pgfqpoint{3.185726in}{5.237227in}}%
\pgfpathlineto{\pgfqpoint{3.186147in}{5.220237in}}%
\pgfpathlineto{\pgfqpoint{3.186821in}{5.188992in}}%
\pgfpathlineto{\pgfqpoint{3.187411in}{5.202654in}}%
\pgfpathlineto{\pgfqpoint{3.187579in}{5.204513in}}%
\pgfpathlineto{\pgfqpoint{3.187916in}{5.193078in}}%
\pgfpathlineto{\pgfqpoint{3.188506in}{5.158130in}}%
\pgfpathlineto{\pgfqpoint{3.189096in}{5.184995in}}%
\pgfpathlineto{\pgfqpoint{3.192213in}{5.317701in}}%
\pgfpathlineto{\pgfqpoint{3.192381in}{5.315257in}}%
\pgfpathlineto{\pgfqpoint{3.194319in}{5.223194in}}%
\pgfpathlineto{\pgfqpoint{3.195835in}{5.174604in}}%
\pgfpathlineto{\pgfqpoint{3.196004in}{5.172255in}}%
\pgfpathlineto{\pgfqpoint{3.196341in}{5.181263in}}%
\pgfpathlineto{\pgfqpoint{3.197267in}{5.234878in}}%
\pgfpathlineto{\pgfqpoint{3.198110in}{5.222214in}}%
\pgfpathlineto{\pgfqpoint{3.198784in}{5.247466in}}%
\pgfpathlineto{\pgfqpoint{3.199205in}{5.229436in}}%
\pgfpathlineto{\pgfqpoint{3.199963in}{5.182627in}}%
\pgfpathlineto{\pgfqpoint{3.200553in}{5.206614in}}%
\pgfpathlineto{\pgfqpoint{3.203165in}{5.261723in}}%
\pgfpathlineto{\pgfqpoint{3.203754in}{5.255385in}}%
\pgfpathlineto{\pgfqpoint{3.204007in}{5.252657in}}%
\pgfpathlineto{\pgfqpoint{3.204513in}{5.260592in}}%
\pgfpathlineto{\pgfqpoint{3.204681in}{5.262110in}}%
\pgfpathlineto{\pgfqpoint{3.205018in}{5.256331in}}%
\pgfpathlineto{\pgfqpoint{3.206450in}{5.169570in}}%
\pgfpathlineto{\pgfqpoint{3.207377in}{5.189706in}}%
\pgfpathlineto{\pgfqpoint{3.207798in}{5.188695in}}%
\pgfpathlineto{\pgfqpoint{3.208304in}{5.199113in}}%
\pgfpathlineto{\pgfqpoint{3.209483in}{5.256123in}}%
\pgfpathlineto{\pgfqpoint{3.210325in}{5.249398in}}%
\pgfpathlineto{\pgfqpoint{3.210831in}{5.259443in}}%
\pgfpathlineto{\pgfqpoint{3.211252in}{5.248245in}}%
\pgfpathlineto{\pgfqpoint{3.212516in}{5.189723in}}%
\pgfpathlineto{\pgfqpoint{3.213274in}{5.192362in}}%
\pgfpathlineto{\pgfqpoint{3.213611in}{5.189426in}}%
\pgfpathlineto{\pgfqpoint{3.213779in}{5.187699in}}%
\pgfpathlineto{\pgfqpoint{3.214201in}{5.196082in}}%
\pgfpathlineto{\pgfqpoint{3.215633in}{5.269748in}}%
\pgfpathlineto{\pgfqpoint{3.216391in}{5.254640in}}%
\pgfpathlineto{\pgfqpoint{3.216896in}{5.269300in}}%
\pgfpathlineto{\pgfqpoint{3.217065in}{5.272001in}}%
\pgfpathlineto{\pgfqpoint{3.217402in}{5.262175in}}%
\pgfpathlineto{\pgfqpoint{3.219845in}{5.139463in}}%
\pgfpathlineto{\pgfqpoint{3.220098in}{5.144566in}}%
\pgfpathlineto{\pgfqpoint{3.225068in}{5.318546in}}%
\pgfpathlineto{\pgfqpoint{3.225152in}{5.318526in}}%
\pgfpathlineto{\pgfqpoint{3.226922in}{5.296941in}}%
\pgfpathlineto{\pgfqpoint{3.229954in}{5.080586in}}%
\pgfpathlineto{\pgfqpoint{3.230460in}{5.097540in}}%
\pgfpathlineto{\pgfqpoint{3.231976in}{5.244808in}}%
\pgfpathlineto{\pgfqpoint{3.232819in}{5.325179in}}%
\pgfpathlineto{\pgfqpoint{3.233577in}{5.321198in}}%
\pgfpathlineto{\pgfqpoint{3.234167in}{5.338824in}}%
\pgfpathlineto{\pgfqpoint{3.234588in}{5.321736in}}%
\pgfpathlineto{\pgfqpoint{3.237284in}{5.087232in}}%
\pgfpathlineto{\pgfqpoint{3.237705in}{5.107152in}}%
\pgfpathlineto{\pgfqpoint{3.241075in}{5.318675in}}%
\pgfpathlineto{\pgfqpoint{3.241159in}{5.319545in}}%
\pgfpathlineto{\pgfqpoint{3.241496in}{5.311660in}}%
\pgfpathlineto{\pgfqpoint{3.242928in}{5.230853in}}%
\pgfpathlineto{\pgfqpoint{3.243770in}{5.239770in}}%
\pgfpathlineto{\pgfqpoint{3.247225in}{5.173978in}}%
\pgfpathlineto{\pgfqpoint{3.248488in}{5.180463in}}%
\pgfpathlineto{\pgfqpoint{3.251605in}{5.286219in}}%
\pgfpathlineto{\pgfqpoint{3.251942in}{5.281311in}}%
\pgfpathlineto{\pgfqpoint{3.255143in}{5.161579in}}%
\pgfpathlineto{\pgfqpoint{3.255817in}{5.181874in}}%
\pgfpathlineto{\pgfqpoint{3.258008in}{5.247632in}}%
\pgfpathlineto{\pgfqpoint{3.258513in}{5.220019in}}%
\pgfpathlineto{\pgfqpoint{3.258850in}{5.204570in}}%
\pgfpathlineto{\pgfqpoint{3.259356in}{5.226969in}}%
\pgfpathlineto{\pgfqpoint{3.261630in}{5.310318in}}%
\pgfpathlineto{\pgfqpoint{3.261715in}{5.310435in}}%
\pgfpathlineto{\pgfqpoint{3.261799in}{5.309351in}}%
\pgfpathlineto{\pgfqpoint{3.262220in}{5.299479in}}%
\pgfpathlineto{\pgfqpoint{3.262641in}{5.310147in}}%
\pgfpathlineto{\pgfqpoint{3.263147in}{5.330947in}}%
\pgfpathlineto{\pgfqpoint{3.263568in}{5.308146in}}%
\pgfpathlineto{\pgfqpoint{3.266180in}{5.082845in}}%
\pgfpathlineto{\pgfqpoint{3.266516in}{5.089271in}}%
\pgfpathlineto{\pgfqpoint{3.267780in}{5.158191in}}%
\pgfpathlineto{\pgfqpoint{3.270897in}{5.318590in}}%
\pgfpathlineto{\pgfqpoint{3.271150in}{5.322784in}}%
\pgfpathlineto{\pgfqpoint{3.271487in}{5.307250in}}%
\pgfpathlineto{\pgfqpoint{3.273930in}{5.170671in}}%
\pgfpathlineto{\pgfqpoint{3.274014in}{5.171371in}}%
\pgfpathlineto{\pgfqpoint{3.275531in}{5.235776in}}%
\pgfpathlineto{\pgfqpoint{3.276120in}{5.257013in}}%
\pgfpathlineto{\pgfqpoint{3.276879in}{5.250689in}}%
\pgfpathlineto{\pgfqpoint{3.277300in}{5.254944in}}%
\pgfpathlineto{\pgfqpoint{3.277553in}{5.248103in}}%
\pgfpathlineto{\pgfqpoint{3.278395in}{5.201627in}}%
\pgfpathlineto{\pgfqpoint{3.279069in}{5.222409in}}%
\pgfpathlineto{\pgfqpoint{3.279237in}{5.220685in}}%
\pgfpathlineto{\pgfqpoint{3.279911in}{5.192695in}}%
\pgfpathlineto{\pgfqpoint{3.280333in}{5.213320in}}%
\pgfpathlineto{\pgfqpoint{3.280754in}{5.232024in}}%
\pgfpathlineto{\pgfqpoint{3.281344in}{5.208013in}}%
\pgfpathlineto{\pgfqpoint{3.281512in}{5.204717in}}%
\pgfpathlineto{\pgfqpoint{3.281849in}{5.216881in}}%
\pgfpathlineto{\pgfqpoint{3.282607in}{5.262750in}}%
\pgfpathlineto{\pgfqpoint{3.283197in}{5.239804in}}%
\pgfpathlineto{\pgfqpoint{3.284713in}{5.186969in}}%
\pgfpathlineto{\pgfqpoint{3.285219in}{5.205891in}}%
\pgfpathlineto{\pgfqpoint{3.288673in}{5.326618in}}%
\pgfpathlineto{\pgfqpoint{3.286061in}{5.205766in}}%
\pgfpathlineto{\pgfqpoint{3.289178in}{5.303912in}}%
\pgfpathlineto{\pgfqpoint{3.293138in}{5.067098in}}%
\pgfpathlineto{\pgfqpoint{3.293222in}{5.068276in}}%
\pgfpathlineto{\pgfqpoint{3.294486in}{5.135587in}}%
\pgfpathlineto{\pgfqpoint{3.298108in}{5.340608in}}%
\pgfpathlineto{\pgfqpoint{3.298277in}{5.338976in}}%
\pgfpathlineto{\pgfqpoint{3.298951in}{5.273053in}}%
\pgfpathlineto{\pgfqpoint{3.300551in}{5.207266in}}%
\pgfpathlineto{\pgfqpoint{3.301983in}{5.188256in}}%
\pgfpathlineto{\pgfqpoint{3.302320in}{5.181205in}}%
\pgfpathlineto{\pgfqpoint{3.302742in}{5.196665in}}%
\pgfpathlineto{\pgfqpoint{3.303331in}{5.212169in}}%
\pgfpathlineto{\pgfqpoint{3.304090in}{5.211201in}}%
\pgfpathlineto{\pgfqpoint{3.304679in}{5.236729in}}%
\pgfpathlineto{\pgfqpoint{3.305185in}{5.260714in}}%
\pgfpathlineto{\pgfqpoint{3.305774in}{5.237591in}}%
\pgfpathlineto{\pgfqpoint{3.306280in}{5.220862in}}%
\pgfpathlineto{\pgfqpoint{3.306954in}{5.232894in}}%
\pgfpathlineto{\pgfqpoint{3.307207in}{5.229493in}}%
\pgfpathlineto{\pgfqpoint{3.307796in}{5.214491in}}%
\pgfpathlineto{\pgfqpoint{3.308302in}{5.224890in}}%
\pgfpathlineto{\pgfqpoint{3.310155in}{5.267904in}}%
\pgfpathlineto{\pgfqpoint{3.310492in}{5.264977in}}%
\pgfpathlineto{\pgfqpoint{3.311924in}{5.226635in}}%
\pgfpathlineto{\pgfqpoint{3.313778in}{5.177734in}}%
\pgfpathlineto{\pgfqpoint{3.314030in}{5.173833in}}%
\pgfpathlineto{\pgfqpoint{3.314452in}{5.189106in}}%
\pgfpathlineto{\pgfqpoint{3.316305in}{5.290427in}}%
\pgfpathlineto{\pgfqpoint{3.316810in}{5.268457in}}%
\pgfpathlineto{\pgfqpoint{3.317232in}{5.253182in}}%
\pgfpathlineto{\pgfqpoint{3.317821in}{5.270452in}}%
\pgfpathlineto{\pgfqpoint{3.317990in}{5.273242in}}%
\pgfpathlineto{\pgfqpoint{3.318327in}{5.262901in}}%
\pgfpathlineto{\pgfqpoint{3.319843in}{5.188607in}}%
\pgfpathlineto{\pgfqpoint{3.320349in}{5.194104in}}%
\pgfpathlineto{\pgfqpoint{3.320517in}{5.195292in}}%
\pgfpathlineto{\pgfqpoint{3.320938in}{5.190737in}}%
\pgfpathlineto{\pgfqpoint{3.321444in}{5.186530in}}%
\pgfpathlineto{\pgfqpoint{3.321781in}{5.192123in}}%
\pgfpathlineto{\pgfqpoint{3.324140in}{5.241369in}}%
\pgfpathlineto{\pgfqpoint{3.324308in}{5.238788in}}%
\pgfpathlineto{\pgfqpoint{3.326751in}{5.141363in}}%
\pgfpathlineto{\pgfqpoint{3.326920in}{5.146472in}}%
\pgfpathlineto{\pgfqpoint{3.329447in}{5.256688in}}%
\pgfpathlineto{\pgfqpoint{3.329616in}{5.254544in}}%
\pgfpathlineto{\pgfqpoint{3.329953in}{5.247604in}}%
\pgfpathlineto{\pgfqpoint{3.330374in}{5.263049in}}%
\pgfpathlineto{\pgfqpoint{3.330964in}{5.291745in}}%
\pgfpathlineto{\pgfqpoint{3.331469in}{5.270809in}}%
\pgfpathlineto{\pgfqpoint{3.333238in}{5.232203in}}%
\pgfpathlineto{\pgfqpoint{3.333407in}{5.232755in}}%
\pgfpathlineto{\pgfqpoint{3.333491in}{5.233133in}}%
\pgfpathlineto{\pgfqpoint{3.333744in}{5.230785in}}%
\pgfpathlineto{\pgfqpoint{3.334502in}{5.202193in}}%
\pgfpathlineto{\pgfqpoint{3.334923in}{5.222962in}}%
\pgfpathlineto{\pgfqpoint{3.335429in}{5.248025in}}%
\pgfpathlineto{\pgfqpoint{3.335934in}{5.220198in}}%
\pgfpathlineto{\pgfqpoint{3.336271in}{5.204432in}}%
\pgfpathlineto{\pgfqpoint{3.336776in}{5.238740in}}%
\pgfpathlineto{\pgfqpoint{3.337282in}{5.270603in}}%
\pgfpathlineto{\pgfqpoint{3.337787in}{5.232679in}}%
\pgfpathlineto{\pgfqpoint{3.338209in}{5.207816in}}%
\pgfpathlineto{\pgfqpoint{3.338798in}{5.233533in}}%
\pgfpathlineto{\pgfqpoint{3.339051in}{5.238749in}}%
\pgfpathlineto{\pgfqpoint{3.339388in}{5.223046in}}%
\pgfpathlineto{\pgfqpoint{3.339978in}{5.189804in}}%
\pgfpathlineto{\pgfqpoint{3.340652in}{5.209241in}}%
\pgfpathlineto{\pgfqpoint{3.343095in}{5.283893in}}%
\pgfpathlineto{\pgfqpoint{3.343853in}{5.262956in}}%
\pgfpathlineto{\pgfqpoint{3.346296in}{5.127647in}}%
\pgfpathlineto{\pgfqpoint{3.347139in}{5.147614in}}%
\pgfpathlineto{\pgfqpoint{3.347981in}{5.217682in}}%
\pgfpathlineto{\pgfqpoint{3.350087in}{5.363839in}}%
\pgfpathlineto{\pgfqpoint{3.350256in}{5.360779in}}%
\pgfpathlineto{\pgfqpoint{3.352362in}{5.191774in}}%
\pgfpathlineto{\pgfqpoint{3.354131in}{5.091018in}}%
\pgfpathlineto{\pgfqpoint{3.354552in}{5.105463in}}%
\pgfpathlineto{\pgfqpoint{3.356490in}{5.275416in}}%
\pgfpathlineto{\pgfqpoint{3.358259in}{5.371063in}}%
\pgfpathlineto{\pgfqpoint{3.358343in}{5.369566in}}%
\pgfpathlineto{\pgfqpoint{3.359775in}{5.271334in}}%
\pgfpathlineto{\pgfqpoint{3.362303in}{5.072821in}}%
\pgfpathlineto{\pgfqpoint{3.362808in}{5.103983in}}%
\pgfpathlineto{\pgfqpoint{3.366009in}{5.363616in}}%
\pgfpathlineto{\pgfqpoint{3.366262in}{5.358394in}}%
\pgfpathlineto{\pgfqpoint{3.367610in}{5.248516in}}%
\pgfpathlineto{\pgfqpoint{3.368958in}{5.109861in}}%
\pgfpathlineto{\pgfqpoint{3.369548in}{5.141128in}}%
\pgfpathlineto{\pgfqpoint{3.371906in}{5.315258in}}%
\pgfpathlineto{\pgfqpoint{3.372496in}{5.273313in}}%
\pgfpathlineto{\pgfqpoint{3.374939in}{5.138691in}}%
\pgfpathlineto{\pgfqpoint{3.375023in}{5.138699in}}%
\pgfpathlineto{\pgfqpoint{3.375529in}{5.160733in}}%
\pgfpathlineto{\pgfqpoint{3.378056in}{5.389657in}}%
\pgfpathlineto{\pgfqpoint{3.379067in}{5.343467in}}%
\pgfpathlineto{\pgfqpoint{3.379741in}{5.280545in}}%
\pgfpathlineto{\pgfqpoint{3.382184in}{5.009900in}}%
\pgfpathlineto{\pgfqpoint{3.382268in}{5.011210in}}%
\pgfpathlineto{\pgfqpoint{3.383027in}{5.087907in}}%
\pgfpathlineto{\pgfqpoint{3.385807in}{5.417294in}}%
\pgfpathlineto{\pgfqpoint{3.386396in}{5.377072in}}%
\pgfpathlineto{\pgfqpoint{3.388671in}{5.074057in}}%
\pgfpathlineto{\pgfqpoint{3.389261in}{5.106096in}}%
\pgfpathlineto{\pgfqpoint{3.392546in}{5.388124in}}%
\pgfpathlineto{\pgfqpoint{3.393220in}{5.335358in}}%
\pgfpathlineto{\pgfqpoint{3.395748in}{5.116076in}}%
\pgfpathlineto{\pgfqpoint{3.396169in}{5.099931in}}%
\pgfpathlineto{\pgfqpoint{3.396590in}{5.123254in}}%
\pgfpathlineto{\pgfqpoint{3.399202in}{5.334640in}}%
\pgfpathlineto{\pgfqpoint{3.399539in}{5.324475in}}%
\pgfpathlineto{\pgfqpoint{3.401729in}{5.158791in}}%
\pgfpathlineto{\pgfqpoint{3.402319in}{5.129255in}}%
\pgfpathlineto{\pgfqpoint{3.402908in}{5.142489in}}%
\pgfpathlineto{\pgfqpoint{3.404341in}{5.224326in}}%
\pgfpathlineto{\pgfqpoint{3.405520in}{5.309376in}}%
\pgfpathlineto{\pgfqpoint{3.406025in}{5.282223in}}%
\pgfpathlineto{\pgfqpoint{3.408637in}{5.095942in}}%
\pgfpathlineto{\pgfqpoint{3.408974in}{5.102484in}}%
\pgfpathlineto{\pgfqpoint{3.409648in}{5.189973in}}%
\pgfpathlineto{\pgfqpoint{3.411923in}{5.478351in}}%
\pgfpathlineto{\pgfqpoint{3.412007in}{5.476934in}}%
\pgfpathlineto{\pgfqpoint{3.412597in}{5.408606in}}%
\pgfpathlineto{\pgfqpoint{3.415461in}{4.976951in}}%
\pgfpathlineto{\pgfqpoint{3.415798in}{4.989305in}}%
\pgfpathlineto{\pgfqpoint{3.417230in}{5.201958in}}%
\pgfpathlineto{\pgfqpoint{3.419083in}{5.352171in}}%
\pgfpathlineto{\pgfqpoint{3.419336in}{5.349107in}}%
\pgfpathlineto{\pgfqpoint{3.420179in}{5.290006in}}%
\pgfpathlineto{\pgfqpoint{3.422285in}{5.167075in}}%
\pgfpathlineto{\pgfqpoint{3.422453in}{5.170243in}}%
\pgfpathlineto{\pgfqpoint{3.424812in}{5.222550in}}%
\pgfpathlineto{\pgfqpoint{3.426413in}{5.294435in}}%
\pgfpathlineto{\pgfqpoint{3.427592in}{5.330142in}}%
\pgfpathlineto{\pgfqpoint{3.427929in}{5.319857in}}%
\pgfpathlineto{\pgfqpoint{3.431383in}{5.011145in}}%
\pgfpathlineto{\pgfqpoint{3.432141in}{5.076261in}}%
\pgfpathlineto{\pgfqpoint{3.434921in}{5.502352in}}%
\pgfpathlineto{\pgfqpoint{3.435764in}{5.467939in}}%
\pgfpathlineto{\pgfqpoint{3.436859in}{5.231338in}}%
\pgfpathlineto{\pgfqpoint{3.438712in}{4.968873in}}%
\pgfpathlineto{\pgfqpoint{3.438965in}{4.975258in}}%
\pgfpathlineto{\pgfqpoint{3.439807in}{5.096654in}}%
\pgfpathlineto{\pgfqpoint{3.441914in}{5.394947in}}%
\pgfpathlineto{\pgfqpoint{3.442335in}{5.373243in}}%
\pgfpathlineto{\pgfqpoint{3.445199in}{5.059576in}}%
\pgfpathlineto{\pgfqpoint{3.445873in}{5.108763in}}%
\pgfpathlineto{\pgfqpoint{3.448232in}{5.437233in}}%
\pgfpathlineto{\pgfqpoint{3.448822in}{5.411692in}}%
\pgfpathlineto{\pgfqpoint{3.449833in}{5.195615in}}%
\pgfpathlineto{\pgfqpoint{3.451686in}{4.866293in}}%
\pgfpathlineto{\pgfqpoint{3.451939in}{4.880392in}}%
\pgfpathlineto{\pgfqpoint{3.452950in}{5.127480in}}%
\pgfpathlineto{\pgfqpoint{3.455224in}{5.644511in}}%
\pgfpathlineto{\pgfqpoint{3.455561in}{5.620810in}}%
\pgfpathlineto{\pgfqpoint{3.457078in}{5.220243in}}%
\pgfpathlineto{\pgfqpoint{3.458678in}{4.902436in}}%
\pgfpathlineto{\pgfqpoint{3.459099in}{4.932264in}}%
\pgfpathlineto{\pgfqpoint{3.460700in}{5.301238in}}%
\pgfpathlineto{\pgfqpoint{3.462048in}{5.518472in}}%
\pgfpathlineto{\pgfqpoint{3.462469in}{5.499828in}}%
\pgfpathlineto{\pgfqpoint{3.463649in}{5.282972in}}%
\pgfpathlineto{\pgfqpoint{3.465334in}{4.965595in}}%
\pgfpathlineto{\pgfqpoint{3.465839in}{4.982687in}}%
\pgfpathlineto{\pgfqpoint{3.467271in}{5.159219in}}%
\pgfpathlineto{\pgfqpoint{3.468619in}{5.304767in}}%
\pgfpathlineto{\pgfqpoint{3.469125in}{5.287452in}}%
\pgfpathlineto{\pgfqpoint{3.471483in}{5.120496in}}%
\pgfpathlineto{\pgfqpoint{3.471989in}{5.155516in}}%
\pgfpathlineto{\pgfqpoint{3.474432in}{5.491398in}}%
\pgfpathlineto{\pgfqpoint{3.475022in}{5.427792in}}%
\pgfpathlineto{\pgfqpoint{3.477633in}{4.897277in}}%
\pgfpathlineto{\pgfqpoint{3.478307in}{4.960176in}}%
\pgfpathlineto{\pgfqpoint{3.480329in}{5.529430in}}%
\pgfpathlineto{\pgfqpoint{3.481256in}{5.662689in}}%
\pgfpathlineto{\pgfqpoint{3.481761in}{5.627361in}}%
\pgfpathlineto{\pgfqpoint{3.483109in}{5.256029in}}%
\pgfpathlineto{\pgfqpoint{3.484878in}{4.905751in}}%
\pgfpathlineto{\pgfqpoint{3.485215in}{4.915434in}}%
\pgfpathlineto{\pgfqpoint{3.486310in}{5.059399in}}%
\pgfpathlineto{\pgfqpoint{3.488332in}{5.319099in}}%
\pgfpathlineto{\pgfqpoint{3.488585in}{5.313439in}}%
\pgfpathlineto{\pgfqpoint{3.489596in}{5.221021in}}%
\pgfpathlineto{\pgfqpoint{3.491028in}{5.049524in}}%
\pgfpathlineto{\pgfqpoint{3.491618in}{5.091177in}}%
\pgfpathlineto{\pgfqpoint{3.494398in}{5.448699in}}%
\pgfpathlineto{\pgfqpoint{3.494903in}{5.426170in}}%
\pgfpathlineto{\pgfqpoint{3.497094in}{5.128951in}}%
\pgfpathlineto{\pgfqpoint{3.498189in}{5.068944in}}%
\pgfpathlineto{\pgfqpoint{3.498610in}{5.078083in}}%
\pgfpathlineto{\pgfqpoint{3.499874in}{5.184547in}}%
\pgfpathlineto{\pgfqpoint{3.501643in}{5.277284in}}%
\pgfpathlineto{\pgfqpoint{3.501896in}{5.278336in}}%
\pgfpathlineto{\pgfqpoint{3.502317in}{5.274943in}}%
\pgfpathlineto{\pgfqpoint{3.503075in}{5.261472in}}%
\pgfpathlineto{\pgfqpoint{3.503496in}{5.273506in}}%
\pgfpathlineto{\pgfqpoint{3.505265in}{5.348309in}}%
\pgfpathlineto{\pgfqpoint{3.505771in}{5.340204in}}%
\pgfpathlineto{\pgfqpoint{3.506866in}{5.273949in}}%
\pgfpathlineto{\pgfqpoint{3.509478in}{4.992887in}}%
\pgfpathlineto{\pgfqpoint{3.509815in}{5.000103in}}%
\pgfpathlineto{\pgfqpoint{3.510741in}{5.111047in}}%
\pgfpathlineto{\pgfqpoint{3.513437in}{5.369133in}}%
\pgfpathlineto{\pgfqpoint{3.513606in}{5.370538in}}%
\pgfpathlineto{\pgfqpoint{3.513943in}{5.362681in}}%
\pgfpathlineto{\pgfqpoint{3.517565in}{5.246910in}}%
\pgfpathlineto{\pgfqpoint{3.518997in}{5.211376in}}%
\pgfpathlineto{\pgfqpoint{3.520093in}{5.091092in}}%
\pgfpathlineto{\pgfqpoint{3.521019in}{5.023735in}}%
\pgfpathlineto{\pgfqpoint{3.521525in}{5.036600in}}%
\pgfpathlineto{\pgfqpoint{3.522451in}{5.137237in}}%
\pgfpathlineto{\pgfqpoint{3.524979in}{5.535176in}}%
\pgfpathlineto{\pgfqpoint{3.525484in}{5.512193in}}%
\pgfpathlineto{\pgfqpoint{3.526495in}{5.310903in}}%
\pgfpathlineto{\pgfqpoint{3.528348in}{4.884348in}}%
\pgfpathlineto{\pgfqpoint{3.528854in}{4.926356in}}%
\pgfpathlineto{\pgfqpoint{3.531803in}{5.547213in}}%
\pgfpathlineto{\pgfqpoint{3.532645in}{5.418751in}}%
\pgfpathlineto{\pgfqpoint{3.535004in}{4.842400in}}%
\pgfpathlineto{\pgfqpoint{3.535594in}{4.887921in}}%
\pgfpathlineto{\pgfqpoint{3.536857in}{5.269169in}}%
\pgfpathlineto{\pgfqpoint{3.538374in}{5.580690in}}%
\pgfpathlineto{\pgfqpoint{3.538711in}{5.571048in}}%
\pgfpathlineto{\pgfqpoint{3.539469in}{5.447505in}}%
\pgfpathlineto{\pgfqpoint{3.542080in}{4.882332in}}%
\pgfpathlineto{\pgfqpoint{3.542502in}{4.917554in}}%
\pgfpathlineto{\pgfqpoint{3.545534in}{5.503602in}}%
\pgfpathlineto{\pgfqpoint{3.546545in}{5.374470in}}%
\pgfpathlineto{\pgfqpoint{3.548904in}{5.067938in}}%
\pgfpathlineto{\pgfqpoint{3.549241in}{5.080162in}}%
\pgfpathlineto{\pgfqpoint{3.550421in}{5.265568in}}%
\pgfpathlineto{\pgfqpoint{3.551853in}{5.403807in}}%
\pgfpathlineto{\pgfqpoint{3.552190in}{5.395659in}}%
\pgfpathlineto{\pgfqpoint{3.552948in}{5.283577in}}%
\pgfpathlineto{\pgfqpoint{3.555138in}{4.931484in}}%
\pgfpathlineto{\pgfqpoint{3.555475in}{4.948729in}}%
\pgfpathlineto{\pgfqpoint{3.556739in}{5.248693in}}%
\pgfpathlineto{\pgfqpoint{3.558424in}{5.587993in}}%
\pgfpathlineto{\pgfqpoint{3.558929in}{5.540802in}}%
\pgfpathlineto{\pgfqpoint{3.561878in}{4.899510in}}%
\pgfpathlineto{\pgfqpoint{3.562636in}{4.995168in}}%
\pgfpathlineto{\pgfqpoint{3.565079in}{5.518558in}}%
\pgfpathlineto{\pgfqpoint{3.565669in}{5.483704in}}%
\pgfpathlineto{\pgfqpoint{3.566764in}{5.278049in}}%
\pgfpathlineto{\pgfqpoint{3.568617in}{4.934098in}}%
\pgfpathlineto{\pgfqpoint{3.568954in}{4.950926in}}%
\pgfpathlineto{\pgfqpoint{3.570387in}{5.214426in}}%
\pgfpathlineto{\pgfqpoint{3.571903in}{5.445657in}}%
\pgfpathlineto{\pgfqpoint{3.572324in}{5.430169in}}%
\pgfpathlineto{\pgfqpoint{3.573588in}{5.241080in}}%
\pgfpathlineto{\pgfqpoint{3.575020in}{5.021170in}}%
\pgfpathlineto{\pgfqpoint{3.575610in}{5.065017in}}%
\pgfpathlineto{\pgfqpoint{3.578053in}{5.369736in}}%
\pgfpathlineto{\pgfqpoint{3.578811in}{5.335513in}}%
\pgfpathlineto{\pgfqpoint{3.580917in}{5.132176in}}%
\pgfpathlineto{\pgfqpoint{3.581507in}{5.164475in}}%
\pgfpathlineto{\pgfqpoint{3.584034in}{5.427543in}}%
\pgfpathlineto{\pgfqpoint{3.584708in}{5.385096in}}%
\pgfpathlineto{\pgfqpoint{3.587657in}{4.960880in}}%
\pgfpathlineto{\pgfqpoint{3.588668in}{5.049518in}}%
\pgfpathlineto{\pgfqpoint{3.591195in}{5.460915in}}%
\pgfpathlineto{\pgfqpoint{3.591869in}{5.423493in}}%
\pgfpathlineto{\pgfqpoint{3.593301in}{5.213646in}}%
\pgfpathlineto{\pgfqpoint{3.595323in}{4.938369in}}%
\pgfpathlineto{\pgfqpoint{3.595491in}{4.941455in}}%
\pgfpathlineto{\pgfqpoint{3.596250in}{5.020980in}}%
\pgfpathlineto{\pgfqpoint{3.599030in}{5.525415in}}%
\pgfpathlineto{\pgfqpoint{3.599704in}{5.454888in}}%
\pgfpathlineto{\pgfqpoint{3.602399in}{5.063497in}}%
\pgfpathlineto{\pgfqpoint{3.602905in}{5.078346in}}%
\pgfpathlineto{\pgfqpoint{3.605685in}{5.335597in}}%
\pgfpathlineto{\pgfqpoint{3.606864in}{5.282754in}}%
\pgfpathlineto{\pgfqpoint{3.608634in}{5.198919in}}%
\pgfpathlineto{\pgfqpoint{3.609307in}{5.215638in}}%
\pgfpathlineto{\pgfqpoint{3.610908in}{5.268194in}}%
\pgfpathlineto{\pgfqpoint{3.611329in}{5.259598in}}%
\pgfpathlineto{\pgfqpoint{3.613014in}{5.194838in}}%
\pgfpathlineto{\pgfqpoint{3.615036in}{5.061148in}}%
\pgfpathlineto{\pgfqpoint{3.615457in}{5.052920in}}%
\pgfpathlineto{\pgfqpoint{3.615879in}{5.065268in}}%
\pgfpathlineto{\pgfqpoint{3.617226in}{5.210834in}}%
\pgfpathlineto{\pgfqpoint{3.619670in}{5.456681in}}%
\pgfpathlineto{\pgfqpoint{3.619838in}{5.455556in}}%
\pgfpathlineto{\pgfqpoint{3.620344in}{5.431027in}}%
\pgfpathlineto{\pgfqpoint{3.623713in}{5.010785in}}%
\pgfpathlineto{\pgfqpoint{3.624724in}{5.073549in}}%
\pgfpathlineto{\pgfqpoint{3.627589in}{5.367536in}}%
\pgfpathlineto{\pgfqpoint{3.628852in}{5.341617in}}%
\pgfpathlineto{\pgfqpoint{3.632390in}{5.176086in}}%
\pgfpathlineto{\pgfqpoint{3.632643in}{5.179614in}}%
\pgfpathlineto{\pgfqpoint{3.633064in}{5.186033in}}%
\pgfpathlineto{\pgfqpoint{3.633570in}{5.177703in}}%
\pgfpathlineto{\pgfqpoint{3.634328in}{5.155812in}}%
\pgfpathlineto{\pgfqpoint{3.634834in}{5.168369in}}%
\pgfpathlineto{\pgfqpoint{3.638035in}{5.315580in}}%
\pgfpathlineto{\pgfqpoint{3.638540in}{5.288448in}}%
\pgfpathlineto{\pgfqpoint{3.641068in}{5.120249in}}%
\pgfpathlineto{\pgfqpoint{3.641236in}{5.121832in}}%
\pgfpathlineto{\pgfqpoint{3.642247in}{5.187698in}}%
\pgfpathlineto{\pgfqpoint{3.644437in}{5.320770in}}%
\pgfpathlineto{\pgfqpoint{3.644859in}{5.310081in}}%
\pgfpathlineto{\pgfqpoint{3.647807in}{5.093959in}}%
\pgfpathlineto{\pgfqpoint{3.648650in}{5.146417in}}%
\pgfpathlineto{\pgfqpoint{3.651261in}{5.416243in}}%
\pgfpathlineto{\pgfqpoint{3.651851in}{5.385548in}}%
\pgfpathlineto{\pgfqpoint{3.654631in}{5.052096in}}%
\pgfpathlineto{\pgfqpoint{3.655726in}{5.122711in}}%
\pgfpathlineto{\pgfqpoint{3.657832in}{5.340511in}}%
\pgfpathlineto{\pgfqpoint{3.658591in}{5.321558in}}%
\pgfpathlineto{\pgfqpoint{3.659433in}{5.309969in}}%
\pgfpathlineto{\pgfqpoint{3.661960in}{5.216791in}}%
\pgfpathlineto{\pgfqpoint{3.662550in}{5.205449in}}%
\pgfpathlineto{\pgfqpoint{3.663140in}{5.213751in}}%
\pgfpathlineto{\pgfqpoint{3.663392in}{5.217163in}}%
\pgfpathlineto{\pgfqpoint{3.663814in}{5.209796in}}%
\pgfpathlineto{\pgfqpoint{3.666088in}{5.124444in}}%
\pgfpathlineto{\pgfqpoint{3.666847in}{5.137678in}}%
\pgfpathlineto{\pgfqpoint{3.668784in}{5.266623in}}%
\pgfpathlineto{\pgfqpoint{3.670553in}{5.347137in}}%
\pgfpathlineto{\pgfqpoint{3.670638in}{5.346860in}}%
\pgfpathlineto{\pgfqpoint{3.671059in}{5.334459in}}%
\pgfpathlineto{\pgfqpoint{3.674597in}{5.116515in}}%
\pgfpathlineto{\pgfqpoint{3.675439in}{5.135526in}}%
\pgfpathlineto{\pgfqpoint{3.680915in}{5.340347in}}%
\pgfpathlineto{\pgfqpoint{3.681589in}{5.317434in}}%
\pgfpathlineto{\pgfqpoint{3.683274in}{5.220128in}}%
\pgfpathlineto{\pgfqpoint{3.684959in}{5.144518in}}%
\pgfpathlineto{\pgfqpoint{3.685212in}{5.143115in}}%
\pgfpathlineto{\pgfqpoint{3.685549in}{5.147043in}}%
\pgfpathlineto{\pgfqpoint{3.687065in}{5.231967in}}%
\pgfpathlineto{\pgfqpoint{3.688160in}{5.267573in}}%
\pgfpathlineto{\pgfqpoint{3.688666in}{5.252796in}}%
\pgfpathlineto{\pgfqpoint{3.691193in}{5.171288in}}%
\pgfpathlineto{\pgfqpoint{3.691446in}{5.180690in}}%
\pgfpathlineto{\pgfqpoint{3.692457in}{5.266049in}}%
\pgfpathlineto{\pgfqpoint{3.693552in}{5.259390in}}%
\pgfpathlineto{\pgfqpoint{3.693973in}{5.272526in}}%
\pgfpathlineto{\pgfqpoint{3.694563in}{5.255065in}}%
\pgfpathlineto{\pgfqpoint{3.697090in}{5.192993in}}%
\pgfpathlineto{\pgfqpoint{3.697175in}{5.193457in}}%
\pgfpathlineto{\pgfqpoint{3.697848in}{5.204828in}}%
\pgfpathlineto{\pgfqpoint{3.698270in}{5.194012in}}%
\pgfpathlineto{\pgfqpoint{3.698944in}{5.174352in}}%
\pgfpathlineto{\pgfqpoint{3.699449in}{5.187771in}}%
\pgfpathlineto{\pgfqpoint{3.699955in}{5.202731in}}%
\pgfpathlineto{\pgfqpoint{3.700544in}{5.187765in}}%
\pgfpathlineto{\pgfqpoint{3.700966in}{5.179416in}}%
\pgfpathlineto{\pgfqpoint{3.701471in}{5.190511in}}%
\pgfpathlineto{\pgfqpoint{3.703409in}{5.276754in}}%
\pgfpathlineto{\pgfqpoint{3.704251in}{5.263293in}}%
\pgfpathlineto{\pgfqpoint{3.706189in}{5.228064in}}%
\pgfpathlineto{\pgfqpoint{3.708548in}{5.181382in}}%
\pgfpathlineto{\pgfqpoint{3.708632in}{5.180839in}}%
\pgfpathlineto{\pgfqpoint{3.708800in}{5.183563in}}%
\pgfpathlineto{\pgfqpoint{3.709643in}{5.217434in}}%
\pgfpathlineto{\pgfqpoint{3.710317in}{5.200817in}}%
\pgfpathlineto{\pgfqpoint{3.710401in}{5.200363in}}%
\pgfpathlineto{\pgfqpoint{3.710654in}{5.203870in}}%
\pgfpathlineto{\pgfqpoint{3.712844in}{5.267320in}}%
\pgfpathlineto{\pgfqpoint{3.713518in}{5.255938in}}%
\pgfpathlineto{\pgfqpoint{3.716551in}{5.178367in}}%
\pgfpathlineto{\pgfqpoint{3.717056in}{5.193393in}}%
\pgfpathlineto{\pgfqpoint{3.717477in}{5.204430in}}%
\pgfpathlineto{\pgfqpoint{3.718236in}{5.196456in}}%
\pgfpathlineto{\pgfqpoint{3.718825in}{5.230221in}}%
\pgfpathlineto{\pgfqpoint{3.720510in}{5.313513in}}%
\pgfpathlineto{\pgfqpoint{3.720763in}{5.310629in}}%
\pgfpathlineto{\pgfqpoint{3.722448in}{5.240031in}}%
\pgfpathlineto{\pgfqpoint{3.723796in}{5.165430in}}%
\pgfpathlineto{\pgfqpoint{3.724554in}{5.174308in}}%
\pgfpathlineto{\pgfqpoint{3.726239in}{5.217214in}}%
\pgfpathlineto{\pgfqpoint{3.727587in}{5.260542in}}%
\pgfpathlineto{\pgfqpoint{3.728261in}{5.246803in}}%
\pgfpathlineto{\pgfqpoint{3.729609in}{5.217785in}}%
\pgfpathlineto{\pgfqpoint{3.730283in}{5.223829in}}%
\pgfpathlineto{\pgfqpoint{3.730367in}{5.224040in}}%
\pgfpathlineto{\pgfqpoint{3.730620in}{5.222847in}}%
\pgfpathlineto{\pgfqpoint{3.731799in}{5.190452in}}%
\pgfpathlineto{\pgfqpoint{3.732220in}{5.185021in}}%
\pgfpathlineto{\pgfqpoint{3.732641in}{5.191881in}}%
\pgfpathlineto{\pgfqpoint{3.733231in}{5.206953in}}%
\pgfpathlineto{\pgfqpoint{3.733905in}{5.197521in}}%
\pgfpathlineto{\pgfqpoint{3.733989in}{5.197500in}}%
\pgfpathlineto{\pgfqpoint{3.734074in}{5.197978in}}%
\pgfpathlineto{\pgfqpoint{3.734411in}{5.201763in}}%
\pgfpathlineto{\pgfqpoint{3.734748in}{5.196401in}}%
\pgfpathlineto{\pgfqpoint{3.735590in}{5.172177in}}%
\pgfpathlineto{\pgfqpoint{3.736096in}{5.187774in}}%
\pgfpathlineto{\pgfqpoint{3.738791in}{5.303909in}}%
\pgfpathlineto{\pgfqpoint{3.739128in}{5.296257in}}%
\pgfpathlineto{\pgfqpoint{3.742245in}{5.159043in}}%
\pgfpathlineto{\pgfqpoint{3.743172in}{5.181829in}}%
\pgfpathlineto{\pgfqpoint{3.744014in}{5.227572in}}%
\pgfpathlineto{\pgfqpoint{3.746373in}{5.346234in}}%
\pgfpathlineto{\pgfqpoint{3.746458in}{5.345797in}}%
\pgfpathlineto{\pgfqpoint{3.747132in}{5.311298in}}%
\pgfpathlineto{\pgfqpoint{3.750838in}{5.096373in}}%
\pgfpathlineto{\pgfqpoint{3.751512in}{5.121690in}}%
\pgfpathlineto{\pgfqpoint{3.754545in}{5.253847in}}%
\pgfpathlineto{\pgfqpoint{3.755135in}{5.247773in}}%
\pgfpathlineto{\pgfqpoint{3.755640in}{5.253353in}}%
\pgfpathlineto{\pgfqpoint{3.758589in}{5.304923in}}%
\pgfpathlineto{\pgfqpoint{3.759094in}{5.295691in}}%
\pgfpathlineto{\pgfqpoint{3.762043in}{5.157332in}}%
\pgfpathlineto{\pgfqpoint{3.762717in}{5.176843in}}%
\pgfpathlineto{\pgfqpoint{3.765328in}{5.303749in}}%
\pgfpathlineto{\pgfqpoint{3.765750in}{5.298939in}}%
\pgfpathlineto{\pgfqpoint{3.766592in}{5.233897in}}%
\pgfpathlineto{\pgfqpoint{3.768530in}{5.067538in}}%
\pgfpathlineto{\pgfqpoint{3.768867in}{5.080896in}}%
\pgfpathlineto{\pgfqpoint{3.771815in}{5.322624in}}%
\pgfpathlineto{\pgfqpoint{3.772658in}{5.281058in}}%
\pgfpathlineto{\pgfqpoint{3.774932in}{5.195126in}}%
\pgfpathlineto{\pgfqpoint{3.775016in}{5.195989in}}%
\pgfpathlineto{\pgfqpoint{3.777038in}{5.270250in}}%
\pgfpathlineto{\pgfqpoint{3.777712in}{5.295241in}}%
\pgfpathlineto{\pgfqpoint{3.778302in}{5.282708in}}%
\pgfpathlineto{\pgfqpoint{3.779987in}{5.214426in}}%
\pgfpathlineto{\pgfqpoint{3.781588in}{5.151794in}}%
\pgfpathlineto{\pgfqpoint{3.781756in}{5.153899in}}%
\pgfpathlineto{\pgfqpoint{3.784115in}{5.244971in}}%
\pgfpathlineto{\pgfqpoint{3.785294in}{5.232992in}}%
\pgfpathlineto{\pgfqpoint{3.785968in}{5.223095in}}%
\pgfpathlineto{\pgfqpoint{3.786389in}{5.231962in}}%
\pgfpathlineto{\pgfqpoint{3.787990in}{5.259037in}}%
\pgfpathlineto{\pgfqpoint{3.788243in}{5.256812in}}%
\pgfpathlineto{\pgfqpoint{3.793550in}{5.150925in}}%
\pgfpathlineto{\pgfqpoint{3.793719in}{5.151434in}}%
\pgfpathlineto{\pgfqpoint{3.794224in}{5.167001in}}%
\pgfpathlineto{\pgfqpoint{3.796920in}{5.345751in}}%
\pgfpathlineto{\pgfqpoint{3.798352in}{5.328676in}}%
\pgfpathlineto{\pgfqpoint{3.798773in}{5.311248in}}%
\pgfpathlineto{\pgfqpoint{3.802059in}{5.083129in}}%
\pgfpathlineto{\pgfqpoint{3.802396in}{5.095053in}}%
\pgfpathlineto{\pgfqpoint{3.806187in}{5.322764in}}%
\pgfpathlineto{\pgfqpoint{3.806440in}{5.320879in}}%
\pgfpathlineto{\pgfqpoint{3.809304in}{5.229059in}}%
\pgfpathlineto{\pgfqpoint{3.811663in}{5.141083in}}%
\pgfpathlineto{\pgfqpoint{3.811747in}{5.141018in}}%
\pgfpathlineto{\pgfqpoint{3.811831in}{5.141553in}}%
\pgfpathlineto{\pgfqpoint{3.812421in}{5.162527in}}%
\pgfpathlineto{\pgfqpoint{3.814527in}{5.231334in}}%
\pgfpathlineto{\pgfqpoint{3.814696in}{5.229637in}}%
\pgfpathlineto{\pgfqpoint{3.815117in}{5.224579in}}%
\pgfpathlineto{\pgfqpoint{3.815538in}{5.232040in}}%
\pgfpathlineto{\pgfqpoint{3.816044in}{5.241598in}}%
\pgfpathlineto{\pgfqpoint{3.816549in}{5.231441in}}%
\pgfpathlineto{\pgfqpoint{3.817223in}{5.205120in}}%
\pgfpathlineto{\pgfqpoint{3.817728in}{5.224295in}}%
\pgfpathlineto{\pgfqpoint{3.819750in}{5.273567in}}%
\pgfpathlineto{\pgfqpoint{3.819835in}{5.273622in}}%
\pgfpathlineto{\pgfqpoint{3.819919in}{5.272941in}}%
\pgfpathlineto{\pgfqpoint{3.821267in}{5.234520in}}%
\pgfpathlineto{\pgfqpoint{3.823036in}{5.190118in}}%
\pgfpathlineto{\pgfqpoint{3.823373in}{5.196553in}}%
\pgfpathlineto{\pgfqpoint{3.826069in}{5.263440in}}%
\pgfpathlineto{\pgfqpoint{3.826153in}{5.263408in}}%
\pgfpathlineto{\pgfqpoint{3.826321in}{5.263140in}}%
\pgfpathlineto{\pgfqpoint{3.826574in}{5.264397in}}%
\pgfpathlineto{\pgfqpoint{3.827248in}{5.277226in}}%
\pgfpathlineto{\pgfqpoint{3.827585in}{5.266003in}}%
\pgfpathlineto{\pgfqpoint{3.829607in}{5.198605in}}%
\pgfpathlineto{\pgfqpoint{3.829691in}{5.198028in}}%
\pgfpathlineto{\pgfqpoint{3.829944in}{5.202416in}}%
\pgfpathlineto{\pgfqpoint{3.830702in}{5.234444in}}%
\pgfpathlineto{\pgfqpoint{3.831797in}{5.234042in}}%
\pgfpathlineto{\pgfqpoint{3.832134in}{5.241721in}}%
\pgfpathlineto{\pgfqpoint{3.832640in}{5.225512in}}%
\pgfpathlineto{\pgfqpoint{3.834577in}{5.169750in}}%
\pgfpathlineto{\pgfqpoint{3.834746in}{5.171206in}}%
\pgfpathlineto{\pgfqpoint{3.836010in}{5.229997in}}%
\pgfpathlineto{\pgfqpoint{3.837273in}{5.292000in}}%
\pgfpathlineto{\pgfqpoint{3.838031in}{5.277312in}}%
\pgfpathlineto{\pgfqpoint{3.838874in}{5.238972in}}%
\pgfpathlineto{\pgfqpoint{3.841064in}{5.129883in}}%
\pgfpathlineto{\pgfqpoint{3.841485in}{5.147005in}}%
\pgfpathlineto{\pgfqpoint{3.843929in}{5.274566in}}%
\pgfpathlineto{\pgfqpoint{3.844265in}{5.274446in}}%
\pgfpathlineto{\pgfqpoint{3.845192in}{5.300866in}}%
\pgfpathlineto{\pgfqpoint{3.845698in}{5.281333in}}%
\pgfpathlineto{\pgfqpoint{3.847298in}{5.239193in}}%
\pgfpathlineto{\pgfqpoint{3.847467in}{5.239795in}}%
\pgfpathlineto{\pgfqpoint{3.848225in}{5.255483in}}%
\pgfpathlineto{\pgfqpoint{3.848730in}{5.243772in}}%
\pgfpathlineto{\pgfqpoint{3.849152in}{5.234688in}}%
\pgfpathlineto{\pgfqpoint{3.849741in}{5.245270in}}%
\pgfpathlineto{\pgfqpoint{3.849910in}{5.243522in}}%
\pgfpathlineto{\pgfqpoint{3.852521in}{5.125395in}}%
\pgfpathlineto{\pgfqpoint{3.853448in}{5.156349in}}%
\pgfpathlineto{\pgfqpoint{3.854459in}{5.234172in}}%
\pgfpathlineto{\pgfqpoint{3.856565in}{5.357264in}}%
\pgfpathlineto{\pgfqpoint{3.856818in}{5.353773in}}%
\pgfpathlineto{\pgfqpoint{3.858082in}{5.273838in}}%
\pgfpathlineto{\pgfqpoint{3.860356in}{5.136037in}}%
\pgfpathlineto{\pgfqpoint{3.860525in}{5.137670in}}%
\pgfpathlineto{\pgfqpoint{3.863220in}{5.237788in}}%
\pgfpathlineto{\pgfqpoint{3.864147in}{5.212286in}}%
\pgfpathlineto{\pgfqpoint{3.865074in}{5.195974in}}%
\pgfpathlineto{\pgfqpoint{3.865495in}{5.202084in}}%
\pgfpathlineto{\pgfqpoint{3.866927in}{5.278006in}}%
\pgfpathlineto{\pgfqpoint{3.867770in}{5.305180in}}%
\pgfpathlineto{\pgfqpoint{3.868275in}{5.297143in}}%
\pgfpathlineto{\pgfqpoint{3.869623in}{5.236311in}}%
\pgfpathlineto{\pgfqpoint{3.871982in}{5.105924in}}%
\pgfpathlineto{\pgfqpoint{3.872066in}{5.105232in}}%
\pgfpathlineto{\pgfqpoint{3.872319in}{5.109056in}}%
\pgfpathlineto{\pgfqpoint{3.873498in}{5.185741in}}%
\pgfpathlineto{\pgfqpoint{3.875520in}{5.317802in}}%
\pgfpathlineto{\pgfqpoint{3.876026in}{5.312274in}}%
\pgfpathlineto{\pgfqpoint{3.877205in}{5.265619in}}%
\pgfpathlineto{\pgfqpoint{3.878300in}{5.220465in}}%
\pgfpathlineto{\pgfqpoint{3.878890in}{5.235106in}}%
\pgfpathlineto{\pgfqpoint{3.879058in}{5.237592in}}%
\pgfpathlineto{\pgfqpoint{3.879480in}{5.228816in}}%
\pgfpathlineto{\pgfqpoint{3.881502in}{5.192805in}}%
\pgfpathlineto{\pgfqpoint{3.883355in}{5.160371in}}%
\pgfpathlineto{\pgfqpoint{3.883692in}{5.164950in}}%
\pgfpathlineto{\pgfqpoint{3.885208in}{5.250788in}}%
\pgfpathlineto{\pgfqpoint{3.886977in}{5.334475in}}%
\pgfpathlineto{\pgfqpoint{3.887146in}{5.333195in}}%
\pgfpathlineto{\pgfqpoint{3.887904in}{5.286763in}}%
\pgfpathlineto{\pgfqpoint{3.890347in}{5.128796in}}%
\pgfpathlineto{\pgfqpoint{3.890937in}{5.133706in}}%
\pgfpathlineto{\pgfqpoint{3.891695in}{5.171292in}}%
\pgfpathlineto{\pgfqpoint{3.893717in}{5.288608in}}%
\pgfpathlineto{\pgfqpoint{3.894138in}{5.282246in}}%
\pgfpathlineto{\pgfqpoint{3.897508in}{5.207649in}}%
\pgfpathlineto{\pgfqpoint{3.898013in}{5.220659in}}%
\pgfpathlineto{\pgfqpoint{3.898519in}{5.237375in}}%
\pgfpathlineto{\pgfqpoint{3.899109in}{5.221121in}}%
\pgfpathlineto{\pgfqpoint{3.901131in}{5.175849in}}%
\pgfpathlineto{\pgfqpoint{3.901552in}{5.181611in}}%
\pgfpathlineto{\pgfqpoint{3.901889in}{5.185080in}}%
\pgfpathlineto{\pgfqpoint{3.902478in}{5.179335in}}%
\pgfpathlineto{\pgfqpoint{3.902647in}{5.180131in}}%
\pgfpathlineto{\pgfqpoint{3.903237in}{5.204427in}}%
\pgfpathlineto{\pgfqpoint{3.906101in}{5.319195in}}%
\pgfpathlineto{\pgfqpoint{3.906185in}{5.319708in}}%
\pgfpathlineto{\pgfqpoint{3.906354in}{5.316923in}}%
\pgfpathlineto{\pgfqpoint{3.910229in}{5.156515in}}%
\pgfpathlineto{\pgfqpoint{3.910903in}{5.171232in}}%
\pgfpathlineto{\pgfqpoint{3.912082in}{5.215399in}}%
\pgfpathlineto{\pgfqpoint{3.912841in}{5.211376in}}%
\pgfpathlineto{\pgfqpoint{3.913178in}{5.209524in}}%
\pgfpathlineto{\pgfqpoint{3.913599in}{5.213595in}}%
\pgfpathlineto{\pgfqpoint{3.914020in}{5.219983in}}%
\pgfpathlineto{\pgfqpoint{3.914525in}{5.210168in}}%
\pgfpathlineto{\pgfqpoint{3.914862in}{5.203042in}}%
\pgfpathlineto{\pgfqpoint{3.915284in}{5.217219in}}%
\pgfpathlineto{\pgfqpoint{3.917727in}{5.304356in}}%
\pgfpathlineto{\pgfqpoint{3.917895in}{5.302780in}}%
\pgfpathlineto{\pgfqpoint{3.919917in}{5.220954in}}%
\pgfpathlineto{\pgfqpoint{3.922107in}{5.104295in}}%
\pgfpathlineto{\pgfqpoint{3.922360in}{5.110643in}}%
\pgfpathlineto{\pgfqpoint{3.925646in}{5.331438in}}%
\pgfpathlineto{\pgfqpoint{3.926320in}{5.289365in}}%
\pgfpathlineto{\pgfqpoint{3.928426in}{5.174848in}}%
\pgfpathlineto{\pgfqpoint{3.928510in}{5.175173in}}%
\pgfpathlineto{\pgfqpoint{3.929100in}{5.194390in}}%
\pgfpathlineto{\pgfqpoint{3.931037in}{5.294862in}}%
\pgfpathlineto{\pgfqpoint{3.931711in}{5.275081in}}%
\pgfpathlineto{\pgfqpoint{3.934491in}{5.140743in}}%
\pgfpathlineto{\pgfqpoint{3.935418in}{5.168712in}}%
\pgfpathlineto{\pgfqpoint{3.938282in}{5.268087in}}%
\pgfpathlineto{\pgfqpoint{3.938872in}{5.259070in}}%
\pgfpathlineto{\pgfqpoint{3.940388in}{5.218221in}}%
\pgfpathlineto{\pgfqpoint{3.941062in}{5.222288in}}%
\pgfpathlineto{\pgfqpoint{3.942158in}{5.199070in}}%
\pgfpathlineto{\pgfqpoint{3.942663in}{5.213347in}}%
\pgfpathlineto{\pgfqpoint{3.944685in}{5.256596in}}%
\pgfpathlineto{\pgfqpoint{3.945106in}{5.263655in}}%
\pgfpathlineto{\pgfqpoint{3.945527in}{5.255085in}}%
\pgfpathlineto{\pgfqpoint{3.947970in}{5.184719in}}%
\pgfpathlineto{\pgfqpoint{3.948139in}{5.185893in}}%
\pgfpathlineto{\pgfqpoint{3.948981in}{5.218894in}}%
\pgfpathlineto{\pgfqpoint{3.949487in}{5.234388in}}%
\pgfpathlineto{\pgfqpoint{3.950161in}{5.223184in}}%
\pgfpathlineto{\pgfqpoint{3.952014in}{5.203653in}}%
\pgfpathlineto{\pgfqpoint{3.952520in}{5.212005in}}%
\pgfpathlineto{\pgfqpoint{3.955047in}{5.264266in}}%
\pgfpathlineto{\pgfqpoint{3.955216in}{5.263223in}}%
\pgfpathlineto{\pgfqpoint{3.959344in}{5.170508in}}%
\pgfpathlineto{\pgfqpoint{3.960102in}{5.183362in}}%
\pgfpathlineto{\pgfqpoint{3.964314in}{5.308926in}}%
\pgfpathlineto{\pgfqpoint{3.964988in}{5.284270in}}%
\pgfpathlineto{\pgfqpoint{3.967431in}{5.206994in}}%
\pgfpathlineto{\pgfqpoint{3.969032in}{5.174632in}}%
\pgfpathlineto{\pgfqpoint{3.969284in}{5.176737in}}%
\pgfpathlineto{\pgfqpoint{3.972738in}{5.233844in}}%
\pgfpathlineto{\pgfqpoint{3.973244in}{5.224639in}}%
\pgfpathlineto{\pgfqpoint{3.974086in}{5.209120in}}%
\pgfpathlineto{\pgfqpoint{3.974592in}{5.213757in}}%
\pgfpathlineto{\pgfqpoint{3.975940in}{5.224204in}}%
\pgfpathlineto{\pgfqpoint{3.976277in}{5.222957in}}%
\pgfpathlineto{\pgfqpoint{3.976361in}{5.222853in}}%
\pgfpathlineto{\pgfqpoint{3.976529in}{5.223817in}}%
\pgfpathlineto{\pgfqpoint{3.977288in}{5.232638in}}%
\pgfpathlineto{\pgfqpoint{3.977793in}{5.226961in}}%
\pgfpathlineto{\pgfqpoint{3.978046in}{5.225364in}}%
\pgfpathlineto{\pgfqpoint{3.978467in}{5.231001in}}%
\pgfpathlineto{\pgfqpoint{3.980742in}{5.268625in}}%
\pgfpathlineto{\pgfqpoint{3.980994in}{5.270002in}}%
\pgfpathlineto{\pgfqpoint{3.981584in}{5.267227in}}%
\pgfpathlineto{\pgfqpoint{3.982427in}{5.256464in}}%
\pgfpathlineto{\pgfqpoint{3.984617in}{5.163267in}}%
\pgfpathlineto{\pgfqpoint{3.986302in}{5.122628in}}%
\pgfpathlineto{\pgfqpoint{3.986470in}{5.123906in}}%
\pgfpathlineto{\pgfqpoint{3.987228in}{5.164816in}}%
\pgfpathlineto{\pgfqpoint{3.989672in}{5.342025in}}%
\pgfpathlineto{\pgfqpoint{3.990093in}{5.331647in}}%
\pgfpathlineto{\pgfqpoint{3.992789in}{5.150596in}}%
\pgfpathlineto{\pgfqpoint{3.993631in}{5.189726in}}%
\pgfpathlineto{\pgfqpoint{3.995737in}{5.256184in}}%
\pgfpathlineto{\pgfqpoint{3.995906in}{5.254922in}}%
\pgfpathlineto{\pgfqpoint{3.996748in}{5.214563in}}%
\pgfpathlineto{\pgfqpoint{3.997927in}{5.166124in}}%
\pgfpathlineto{\pgfqpoint{3.998433in}{5.176271in}}%
\pgfpathlineto{\pgfqpoint{3.999949in}{5.275152in}}%
\pgfpathlineto{\pgfqpoint{4.001297in}{5.356924in}}%
\pgfpathlineto{\pgfqpoint{4.001803in}{5.334390in}}%
\pgfpathlineto{\pgfqpoint{4.005341in}{5.130609in}}%
\pgfpathlineto{\pgfqpoint{4.005762in}{5.141657in}}%
\pgfpathlineto{\pgfqpoint{4.007784in}{5.225726in}}%
\pgfpathlineto{\pgfqpoint{4.008458in}{5.223090in}}%
\pgfpathlineto{\pgfqpoint{4.008964in}{5.235078in}}%
\pgfpathlineto{\pgfqpoint{4.009469in}{5.247486in}}%
\pgfpathlineto{\pgfqpoint{4.009890in}{5.233177in}}%
\pgfpathlineto{\pgfqpoint{4.010648in}{5.197740in}}%
\pgfpathlineto{\pgfqpoint{4.011322in}{5.211442in}}%
\pgfpathlineto{\pgfqpoint{4.013681in}{5.292938in}}%
\pgfpathlineto{\pgfqpoint{4.014524in}{5.270901in}}%
\pgfpathlineto{\pgfqpoint{4.016630in}{5.189137in}}%
\pgfpathlineto{\pgfqpoint{4.018062in}{5.161595in}}%
\pgfpathlineto{\pgfqpoint{4.018399in}{5.159580in}}%
\pgfpathlineto{\pgfqpoint{4.018652in}{5.162892in}}%
\pgfpathlineto{\pgfqpoint{4.020421in}{5.250950in}}%
\pgfpathlineto{\pgfqpoint{4.021263in}{5.286303in}}%
\pgfpathlineto{\pgfqpoint{4.021937in}{5.282840in}}%
\pgfpathlineto{\pgfqpoint{4.022527in}{5.287258in}}%
\pgfpathlineto{\pgfqpoint{4.022948in}{5.282938in}}%
\pgfpathlineto{\pgfqpoint{4.024380in}{5.222375in}}%
\pgfpathlineto{\pgfqpoint{4.026065in}{5.161098in}}%
\pgfpathlineto{\pgfqpoint{4.026739in}{5.168461in}}%
\pgfpathlineto{\pgfqpoint{4.028087in}{5.219876in}}%
\pgfpathlineto{\pgfqpoint{4.028929in}{5.262538in}}%
\pgfpathlineto{\pgfqpoint{4.029519in}{5.247965in}}%
\pgfpathlineto{\pgfqpoint{4.029856in}{5.241419in}}%
\pgfpathlineto{\pgfqpoint{4.030530in}{5.250298in}}%
\pgfpathlineto{\pgfqpoint{4.030783in}{5.247103in}}%
\pgfpathlineto{\pgfqpoint{4.033479in}{5.188410in}}%
\pgfpathlineto{\pgfqpoint{4.033563in}{5.189125in}}%
\pgfpathlineto{\pgfqpoint{4.034237in}{5.217507in}}%
\pgfpathlineto{\pgfqpoint{4.035416in}{5.272496in}}%
\pgfpathlineto{\pgfqpoint{4.036006in}{5.259958in}}%
\pgfpathlineto{\pgfqpoint{4.038955in}{5.182668in}}%
\pgfpathlineto{\pgfqpoint{4.039629in}{5.197873in}}%
\pgfpathlineto{\pgfqpoint{4.040808in}{5.247937in}}%
\pgfpathlineto{\pgfqpoint{4.041819in}{5.247649in}}%
\pgfpathlineto{\pgfqpoint{4.041987in}{5.247915in}}%
\pgfpathlineto{\pgfqpoint{4.042240in}{5.245955in}}%
\pgfpathlineto{\pgfqpoint{4.043841in}{5.201262in}}%
\pgfpathlineto{\pgfqpoint{4.044767in}{5.165601in}}%
\pgfpathlineto{\pgfqpoint{4.045273in}{5.185145in}}%
\pgfpathlineto{\pgfqpoint{4.048053in}{5.330522in}}%
\pgfpathlineto{\pgfqpoint{4.048390in}{5.324155in}}%
\pgfpathlineto{\pgfqpoint{4.050412in}{5.184280in}}%
\pgfpathlineto{\pgfqpoint{4.051339in}{5.135660in}}%
\pgfpathlineto{\pgfqpoint{4.051844in}{5.151294in}}%
\pgfpathlineto{\pgfqpoint{4.054540in}{5.225737in}}%
\pgfpathlineto{\pgfqpoint{4.056477in}{5.216724in}}%
\pgfpathlineto{\pgfqpoint{4.056899in}{5.220984in}}%
\pgfpathlineto{\pgfqpoint{4.057994in}{5.245935in}}%
\pgfpathlineto{\pgfqpoint{4.058921in}{5.241936in}}%
\pgfpathlineto{\pgfqpoint{4.059594in}{5.250627in}}%
\pgfpathlineto{\pgfqpoint{4.060016in}{5.243281in}}%
\pgfpathlineto{\pgfqpoint{4.061701in}{5.215683in}}%
\pgfpathlineto{\pgfqpoint{4.061953in}{5.217729in}}%
\pgfpathlineto{\pgfqpoint{4.063807in}{5.264248in}}%
\pgfpathlineto{\pgfqpoint{4.064396in}{5.246240in}}%
\pgfpathlineto{\pgfqpoint{4.066671in}{5.198017in}}%
\pgfpathlineto{\pgfqpoint{4.066755in}{5.197877in}}%
\pgfpathlineto{\pgfqpoint{4.067008in}{5.199367in}}%
\pgfpathlineto{\pgfqpoint{4.068272in}{5.230530in}}%
\pgfpathlineto{\pgfqpoint{4.069367in}{5.264756in}}%
\pgfpathlineto{\pgfqpoint{4.070041in}{5.255920in}}%
\pgfpathlineto{\pgfqpoint{4.070968in}{5.239108in}}%
\pgfpathlineto{\pgfqpoint{4.072737in}{5.155337in}}%
\pgfpathlineto{\pgfqpoint{4.073326in}{5.169095in}}%
\pgfpathlineto{\pgfqpoint{4.075769in}{5.269229in}}%
\pgfpathlineto{\pgfqpoint{4.076612in}{5.257647in}}%
\pgfpathlineto{\pgfqpoint{4.077202in}{5.258526in}}%
\pgfpathlineto{\pgfqpoint{4.077876in}{5.247275in}}%
\pgfpathlineto{\pgfqpoint{4.080066in}{5.215017in}}%
\pgfpathlineto{\pgfqpoint{4.080150in}{5.214888in}}%
\pgfpathlineto{\pgfqpoint{4.080403in}{5.216474in}}%
\pgfpathlineto{\pgfqpoint{4.080993in}{5.220487in}}%
\pgfpathlineto{\pgfqpoint{4.081414in}{5.217307in}}%
\pgfpathlineto{\pgfqpoint{4.081919in}{5.212369in}}%
\pgfpathlineto{\pgfqpoint{4.082341in}{5.218529in}}%
\pgfpathlineto{\pgfqpoint{4.083014in}{5.228768in}}%
\pgfpathlineto{\pgfqpoint{4.083520in}{5.221144in}}%
\pgfpathlineto{\pgfqpoint{4.086047in}{5.195292in}}%
\pgfpathlineto{\pgfqpoint{4.086132in}{5.195680in}}%
\pgfpathlineto{\pgfqpoint{4.086637in}{5.215591in}}%
\pgfpathlineto{\pgfqpoint{4.088996in}{5.297713in}}%
\pgfpathlineto{\pgfqpoint{4.089164in}{5.299128in}}%
\pgfpathlineto{\pgfqpoint{4.089586in}{5.292119in}}%
\pgfpathlineto{\pgfqpoint{4.091102in}{5.199181in}}%
\pgfpathlineto{\pgfqpoint{4.092618in}{5.127605in}}%
\pgfpathlineto{\pgfqpoint{4.093040in}{5.135100in}}%
\pgfpathlineto{\pgfqpoint{4.094893in}{5.226661in}}%
\pgfpathlineto{\pgfqpoint{4.096662in}{5.270409in}}%
\pgfpathlineto{\pgfqpoint{4.096915in}{5.268914in}}%
\pgfpathlineto{\pgfqpoint{4.098431in}{5.230136in}}%
\pgfpathlineto{\pgfqpoint{4.099611in}{5.191081in}}%
\pgfpathlineto{\pgfqpoint{4.100369in}{5.194311in}}%
\pgfpathlineto{\pgfqpoint{4.100537in}{5.193926in}}%
\pgfpathlineto{\pgfqpoint{4.100790in}{5.195183in}}%
\pgfpathlineto{\pgfqpoint{4.101464in}{5.221489in}}%
\pgfpathlineto{\pgfqpoint{4.103823in}{5.295482in}}%
\pgfpathlineto{\pgfqpoint{4.104244in}{5.291535in}}%
\pgfpathlineto{\pgfqpoint{4.105339in}{5.262443in}}%
\pgfpathlineto{\pgfqpoint{4.109720in}{5.140001in}}%
\pgfpathlineto{\pgfqpoint{4.110057in}{5.142278in}}%
\pgfpathlineto{\pgfqpoint{4.111405in}{5.170122in}}%
\pgfpathlineto{\pgfqpoint{4.112753in}{5.217303in}}%
\pgfpathlineto{\pgfqpoint{4.115027in}{5.319064in}}%
\pgfpathlineto{\pgfqpoint{4.115364in}{5.318066in}}%
\pgfpathlineto{\pgfqpoint{4.115701in}{5.319987in}}%
\pgfpathlineto{\pgfqpoint{4.116038in}{5.322318in}}%
\pgfpathlineto{\pgfqpoint{4.116291in}{5.318533in}}%
\pgfpathlineto{\pgfqpoint{4.117302in}{5.240406in}}%
\pgfpathlineto{\pgfqpoint{4.120166in}{5.116387in}}%
\pgfpathlineto{\pgfqpoint{4.120588in}{5.123880in}}%
\pgfpathlineto{\pgfqpoint{4.122357in}{5.225237in}}%
\pgfpathlineto{\pgfqpoint{4.123873in}{5.309591in}}%
\pgfpathlineto{\pgfqpoint{4.124379in}{5.302668in}}%
\pgfpathlineto{\pgfqpoint{4.126316in}{5.238782in}}%
\pgfpathlineto{\pgfqpoint{4.127243in}{5.198841in}}%
\pgfpathlineto{\pgfqpoint{4.127917in}{5.199332in}}%
\pgfpathlineto{\pgfqpoint{4.128928in}{5.179390in}}%
\pgfpathlineto{\pgfqpoint{4.129517in}{5.191213in}}%
\pgfpathlineto{\pgfqpoint{4.131624in}{5.243372in}}%
\pgfpathlineto{\pgfqpoint{4.132045in}{5.238472in}}%
\pgfpathlineto{\pgfqpoint{4.134319in}{5.179932in}}%
\pgfpathlineto{\pgfqpoint{4.134993in}{5.162908in}}%
\pgfpathlineto{\pgfqpoint{4.135499in}{5.174348in}}%
\pgfpathlineto{\pgfqpoint{4.137099in}{5.253274in}}%
\pgfpathlineto{\pgfqpoint{4.138869in}{5.334003in}}%
\pgfpathlineto{\pgfqpoint{4.139121in}{5.330795in}}%
\pgfpathlineto{\pgfqpoint{4.140132in}{5.274429in}}%
\pgfpathlineto{\pgfqpoint{4.142407in}{5.166830in}}%
\pgfpathlineto{\pgfqpoint{4.142575in}{5.167399in}}%
\pgfpathlineto{\pgfqpoint{4.144513in}{5.202128in}}%
\pgfpathlineto{\pgfqpoint{4.146114in}{5.216709in}}%
\pgfpathlineto{\pgfqpoint{4.146451in}{5.215319in}}%
\pgfpathlineto{\pgfqpoint{4.147883in}{5.193621in}}%
\pgfpathlineto{\pgfqpoint{4.148472in}{5.205248in}}%
\pgfpathlineto{\pgfqpoint{4.151168in}{5.272556in}}%
\pgfpathlineto{\pgfqpoint{4.151253in}{5.271982in}}%
\pgfpathlineto{\pgfqpoint{4.155044in}{5.209058in}}%
\pgfpathlineto{\pgfqpoint{4.156307in}{5.172744in}}%
\pgfpathlineto{\pgfqpoint{4.156981in}{5.183036in}}%
\pgfpathlineto{\pgfqpoint{4.159509in}{5.253628in}}%
\pgfpathlineto{\pgfqpoint{4.160856in}{5.244419in}}%
\pgfpathlineto{\pgfqpoint{4.161783in}{5.224791in}}%
\pgfpathlineto{\pgfqpoint{4.162373in}{5.237881in}}%
\pgfpathlineto{\pgfqpoint{4.162710in}{5.242671in}}%
\pgfpathlineto{\pgfqpoint{4.163131in}{5.233656in}}%
\pgfpathlineto{\pgfqpoint{4.163552in}{5.223793in}}%
\pgfpathlineto{\pgfqpoint{4.164058in}{5.236754in}}%
\pgfpathlineto{\pgfqpoint{4.164816in}{5.257470in}}%
\pgfpathlineto{\pgfqpoint{4.165406in}{5.250108in}}%
\pgfpathlineto{\pgfqpoint{4.168944in}{5.193866in}}%
\pgfpathlineto{\pgfqpoint{4.169197in}{5.197195in}}%
\pgfpathlineto{\pgfqpoint{4.170797in}{5.253673in}}%
\pgfpathlineto{\pgfqpoint{4.171640in}{5.231385in}}%
\pgfpathlineto{\pgfqpoint{4.173746in}{5.210270in}}%
\pgfpathlineto{\pgfqpoint{4.174251in}{5.207303in}}%
\pgfpathlineto{\pgfqpoint{4.175178in}{5.190742in}}%
\pgfpathlineto{\pgfqpoint{4.175683in}{5.199809in}}%
\pgfpathlineto{\pgfqpoint{4.178464in}{5.248012in}}%
\pgfpathlineto{\pgfqpoint{4.178632in}{5.247255in}}%
\pgfpathlineto{\pgfqpoint{4.179306in}{5.229248in}}%
\pgfpathlineto{\pgfqpoint{4.180064in}{5.211117in}}%
\pgfpathlineto{\pgfqpoint{4.180822in}{5.211305in}}%
\pgfpathlineto{\pgfqpoint{4.180991in}{5.211158in}}%
\pgfpathlineto{\pgfqpoint{4.181159in}{5.212220in}}%
\pgfpathlineto{\pgfqpoint{4.182002in}{5.238761in}}%
\pgfpathlineto{\pgfqpoint{4.183687in}{5.287376in}}%
\pgfpathlineto{\pgfqpoint{4.184276in}{5.275071in}}%
\pgfpathlineto{\pgfqpoint{4.187983in}{5.174788in}}%
\pgfpathlineto{\pgfqpoint{4.188236in}{5.176410in}}%
\pgfpathlineto{\pgfqpoint{4.189668in}{5.218153in}}%
\pgfpathlineto{\pgfqpoint{4.190510in}{5.244157in}}%
\pgfpathlineto{\pgfqpoint{4.191269in}{5.243880in}}%
\pgfpathlineto{\pgfqpoint{4.192111in}{5.256433in}}%
\pgfpathlineto{\pgfqpoint{4.192701in}{5.249958in}}%
\pgfpathlineto{\pgfqpoint{4.195228in}{5.199382in}}%
\pgfpathlineto{\pgfqpoint{4.195986in}{5.207633in}}%
\pgfpathlineto{\pgfqpoint{4.196745in}{5.220007in}}%
\pgfpathlineto{\pgfqpoint{4.197250in}{5.212448in}}%
\pgfpathlineto{\pgfqpoint{4.197756in}{5.205185in}}%
\pgfpathlineto{\pgfqpoint{4.198261in}{5.211688in}}%
\pgfpathlineto{\pgfqpoint{4.198766in}{5.215665in}}%
\pgfpathlineto{\pgfqpoint{4.199440in}{5.212852in}}%
\pgfpathlineto{\pgfqpoint{4.200957in}{5.218695in}}%
\pgfpathlineto{\pgfqpoint{4.202052in}{5.247469in}}%
\pgfpathlineto{\pgfqpoint{4.203147in}{5.245189in}}%
\pgfpathlineto{\pgfqpoint{4.203484in}{5.244490in}}%
\pgfpathlineto{\pgfqpoint{4.203905in}{5.246147in}}%
\pgfpathlineto{\pgfqpoint{4.204748in}{5.249548in}}%
\pgfpathlineto{\pgfqpoint{4.205253in}{5.248159in}}%
\pgfpathlineto{\pgfqpoint{4.206348in}{5.236200in}}%
\pgfpathlineto{\pgfqpoint{4.207359in}{5.241024in}}%
\pgfpathlineto{\pgfqpoint{4.208455in}{5.237653in}}%
\pgfpathlineto{\pgfqpoint{4.208876in}{5.238983in}}%
\pgfpathlineto{\pgfqpoint{4.209213in}{5.239518in}}%
\pgfpathlineto{\pgfqpoint{4.209634in}{5.238173in}}%
\pgfpathlineto{\pgfqpoint{4.210392in}{5.229416in}}%
\pgfpathlineto{\pgfqpoint{4.212498in}{5.182663in}}%
\pgfpathlineto{\pgfqpoint{4.213425in}{5.191328in}}%
\pgfpathlineto{\pgfqpoint{4.214099in}{5.181993in}}%
\pgfpathlineto{\pgfqpoint{4.214520in}{5.190725in}}%
\pgfpathlineto{\pgfqpoint{4.217048in}{5.251622in}}%
\pgfpathlineto{\pgfqpoint{4.217890in}{5.275650in}}%
\pgfpathlineto{\pgfqpoint{4.218395in}{5.263477in}}%
\pgfpathlineto{\pgfqpoint{4.220417in}{5.214898in}}%
\pgfpathlineto{\pgfqpoint{4.222186in}{5.197217in}}%
\pgfpathlineto{\pgfqpoint{4.222776in}{5.192910in}}%
\pgfpathlineto{\pgfqpoint{4.223450in}{5.185012in}}%
\pgfpathlineto{\pgfqpoint{4.223787in}{5.192070in}}%
\pgfpathlineto{\pgfqpoint{4.226736in}{5.272200in}}%
\pgfpathlineto{\pgfqpoint{4.226820in}{5.271737in}}%
\pgfpathlineto{\pgfqpoint{4.230105in}{5.214372in}}%
\pgfpathlineto{\pgfqpoint{4.231538in}{5.197953in}}%
\pgfpathlineto{\pgfqpoint{4.231875in}{5.199133in}}%
\pgfpathlineto{\pgfqpoint{4.232717in}{5.218848in}}%
\pgfpathlineto{\pgfqpoint{4.234823in}{5.254875in}}%
\pgfpathlineto{\pgfqpoint{4.235160in}{5.251772in}}%
\pgfpathlineto{\pgfqpoint{4.236676in}{5.214940in}}%
\pgfpathlineto{\pgfqpoint{4.237772in}{5.220172in}}%
\pgfpathlineto{\pgfqpoint{4.238361in}{5.215466in}}%
\pgfpathlineto{\pgfqpoint{4.238783in}{5.219282in}}%
\pgfpathlineto{\pgfqpoint{4.239625in}{5.238939in}}%
\pgfpathlineto{\pgfqpoint{4.240131in}{5.226783in}}%
\pgfpathlineto{\pgfqpoint{4.241815in}{5.203723in}}%
\pgfpathlineto{\pgfqpoint{4.242826in}{5.183524in}}%
\pgfpathlineto{\pgfqpoint{4.243332in}{5.190281in}}%
\pgfpathlineto{\pgfqpoint{4.246196in}{5.278247in}}%
\pgfpathlineto{\pgfqpoint{4.246954in}{5.266957in}}%
\pgfpathlineto{\pgfqpoint{4.247544in}{5.268990in}}%
\pgfpathlineto{\pgfqpoint{4.248050in}{5.259281in}}%
\pgfpathlineto{\pgfqpoint{4.250998in}{5.178703in}}%
\pgfpathlineto{\pgfqpoint{4.251082in}{5.178781in}}%
\pgfpathlineto{\pgfqpoint{4.251588in}{5.186916in}}%
\pgfpathlineto{\pgfqpoint{4.254705in}{5.252291in}}%
\pgfpathlineto{\pgfqpoint{4.254873in}{5.251411in}}%
\pgfpathlineto{\pgfqpoint{4.258749in}{5.210553in}}%
\pgfpathlineto{\pgfqpoint{4.259507in}{5.221350in}}%
\pgfpathlineto{\pgfqpoint{4.259591in}{5.221586in}}%
\pgfpathlineto{\pgfqpoint{4.259844in}{5.219683in}}%
\pgfpathlineto{\pgfqpoint{4.260349in}{5.213289in}}%
\pgfpathlineto{\pgfqpoint{4.260770in}{5.220701in}}%
\pgfpathlineto{\pgfqpoint{4.261360in}{5.232671in}}%
\pgfpathlineto{\pgfqpoint{4.261950in}{5.224936in}}%
\pgfpathlineto{\pgfqpoint{4.262287in}{5.221149in}}%
\pgfpathlineto{\pgfqpoint{4.262792in}{5.227053in}}%
\pgfpathlineto{\pgfqpoint{4.263972in}{5.245771in}}%
\pgfpathlineto{\pgfqpoint{4.264477in}{5.239359in}}%
\pgfpathlineto{\pgfqpoint{4.264983in}{5.231938in}}%
\pgfpathlineto{\pgfqpoint{4.265572in}{5.239006in}}%
\pgfpathlineto{\pgfqpoint{4.265825in}{5.240788in}}%
\pgfpathlineto{\pgfqpoint{4.266246in}{5.235488in}}%
\pgfpathlineto{\pgfqpoint{4.268352in}{5.185398in}}%
\pgfpathlineto{\pgfqpoint{4.269026in}{5.197673in}}%
\pgfpathlineto{\pgfqpoint{4.271385in}{5.243927in}}%
\pgfpathlineto{\pgfqpoint{4.272228in}{5.237949in}}%
\pgfpathlineto{\pgfqpoint{4.274755in}{5.208645in}}%
\pgfpathlineto{\pgfqpoint{4.275092in}{5.213722in}}%
\pgfpathlineto{\pgfqpoint{4.277282in}{5.261285in}}%
\pgfpathlineto{\pgfqpoint{4.277451in}{5.261023in}}%
\pgfpathlineto{\pgfqpoint{4.278041in}{5.256113in}}%
\pgfpathlineto{\pgfqpoint{4.279725in}{5.208954in}}%
\pgfpathlineto{\pgfqpoint{4.281410in}{5.184867in}}%
\pgfpathlineto{\pgfqpoint{4.281916in}{5.181988in}}%
\pgfpathlineto{\pgfqpoint{4.282421in}{5.185896in}}%
\pgfpathlineto{\pgfqpoint{4.287055in}{5.268122in}}%
\pgfpathlineto{\pgfqpoint{4.287729in}{5.255195in}}%
\pgfpathlineto{\pgfqpoint{4.287897in}{5.253226in}}%
\pgfpathlineto{\pgfqpoint{4.288403in}{5.260637in}}%
\pgfpathlineto{\pgfqpoint{4.288740in}{5.266210in}}%
\pgfpathlineto{\pgfqpoint{4.289161in}{5.257860in}}%
\pgfpathlineto{\pgfqpoint{4.291857in}{5.191010in}}%
\pgfpathlineto{\pgfqpoint{4.292783in}{5.169218in}}%
\pgfpathlineto{\pgfqpoint{4.293289in}{5.184204in}}%
\pgfpathlineto{\pgfqpoint{4.294974in}{5.224933in}}%
\pgfpathlineto{\pgfqpoint{4.295142in}{5.223871in}}%
\pgfpathlineto{\pgfqpoint{4.295900in}{5.208317in}}%
\pgfpathlineto{\pgfqpoint{4.296322in}{5.219353in}}%
\pgfpathlineto{\pgfqpoint{4.296911in}{5.236237in}}%
\pgfpathlineto{\pgfqpoint{4.297501in}{5.226132in}}%
\pgfpathlineto{\pgfqpoint{4.297838in}{5.222665in}}%
\pgfpathlineto{\pgfqpoint{4.298343in}{5.229665in}}%
\pgfpathlineto{\pgfqpoint{4.300028in}{5.249488in}}%
\pgfpathlineto{\pgfqpoint{4.300365in}{5.248160in}}%
\pgfpathlineto{\pgfqpoint{4.300534in}{5.247726in}}%
\pgfpathlineto{\pgfqpoint{4.300871in}{5.249928in}}%
\pgfpathlineto{\pgfqpoint{4.301461in}{5.257211in}}%
\pgfpathlineto{\pgfqpoint{4.301966in}{5.250210in}}%
\pgfpathlineto{\pgfqpoint{4.305252in}{5.187850in}}%
\pgfpathlineto{\pgfqpoint{4.305673in}{5.196614in}}%
\pgfpathlineto{\pgfqpoint{4.307863in}{5.252702in}}%
\pgfpathlineto{\pgfqpoint{4.308032in}{5.252070in}}%
\pgfpathlineto{\pgfqpoint{4.308537in}{5.249177in}}%
\pgfpathlineto{\pgfqpoint{4.309211in}{5.251258in}}%
\pgfpathlineto{\pgfqpoint{4.309464in}{5.249167in}}%
\pgfpathlineto{\pgfqpoint{4.310390in}{5.226182in}}%
\pgfpathlineto{\pgfqpoint{4.311233in}{5.233400in}}%
\pgfpathlineto{\pgfqpoint{4.311317in}{5.233522in}}%
\pgfpathlineto{\pgfqpoint{4.311654in}{5.232189in}}%
\pgfpathlineto{\pgfqpoint{4.312834in}{5.221341in}}%
\pgfpathlineto{\pgfqpoint{4.313255in}{5.215870in}}%
\pgfpathlineto{\pgfqpoint{4.313844in}{5.221360in}}%
\pgfpathlineto{\pgfqpoint{4.313929in}{5.221827in}}%
\pgfpathlineto{\pgfqpoint{4.314181in}{5.219149in}}%
\pgfpathlineto{\pgfqpoint{4.315866in}{5.173502in}}%
\pgfpathlineto{\pgfqpoint{4.316372in}{5.185748in}}%
\pgfpathlineto{\pgfqpoint{4.318983in}{5.262857in}}%
\pgfpathlineto{\pgfqpoint{4.319573in}{5.270553in}}%
\pgfpathlineto{\pgfqpoint{4.320079in}{5.264963in}}%
\pgfpathlineto{\pgfqpoint{4.322774in}{5.200380in}}%
\pgfpathlineto{\pgfqpoint{4.323533in}{5.211807in}}%
\pgfpathlineto{\pgfqpoint{4.325470in}{5.245769in}}%
\pgfpathlineto{\pgfqpoint{4.326397in}{5.236592in}}%
\pgfpathlineto{\pgfqpoint{4.329261in}{5.197086in}}%
\pgfpathlineto{\pgfqpoint{4.330104in}{5.201996in}}%
\pgfpathlineto{\pgfqpoint{4.330272in}{5.201773in}}%
\pgfpathlineto{\pgfqpoint{4.330525in}{5.203180in}}%
\pgfpathlineto{\pgfqpoint{4.333389in}{5.266067in}}%
\pgfpathlineto{\pgfqpoint{4.334653in}{5.252775in}}%
\pgfpathlineto{\pgfqpoint{4.334990in}{5.253607in}}%
\pgfpathlineto{\pgfqpoint{4.335327in}{5.251170in}}%
\pgfpathlineto{\pgfqpoint{4.337433in}{5.203170in}}%
\pgfpathlineto{\pgfqpoint{4.338107in}{5.182135in}}%
\pgfpathlineto{\pgfqpoint{4.338697in}{5.189203in}}%
\pgfpathlineto{\pgfqpoint{4.340634in}{5.204784in}}%
\pgfpathlineto{\pgfqpoint{4.342066in}{5.246081in}}%
\pgfpathlineto{\pgfqpoint{4.343330in}{5.259601in}}%
\pgfpathlineto{\pgfqpoint{4.343667in}{5.255341in}}%
\pgfpathlineto{\pgfqpoint{4.346279in}{5.207500in}}%
\pgfpathlineto{\pgfqpoint{4.346447in}{5.208806in}}%
\pgfpathlineto{\pgfqpoint{4.347374in}{5.223142in}}%
\pgfpathlineto{\pgfqpoint{4.347879in}{5.214691in}}%
\pgfpathlineto{\pgfqpoint{4.348553in}{5.203348in}}%
\pgfpathlineto{\pgfqpoint{4.349059in}{5.210665in}}%
\pgfpathlineto{\pgfqpoint{4.349648in}{5.221661in}}%
\pgfpathlineto{\pgfqpoint{4.350154in}{5.210550in}}%
\pgfpathlineto{\pgfqpoint{4.350575in}{5.201630in}}%
\pgfpathlineto{\pgfqpoint{4.351081in}{5.212665in}}%
\pgfpathlineto{\pgfqpoint{4.352344in}{5.258283in}}%
\pgfpathlineto{\pgfqpoint{4.353102in}{5.257477in}}%
\pgfpathlineto{\pgfqpoint{4.354366in}{5.253517in}}%
\pgfpathlineto{\pgfqpoint{4.357062in}{5.195090in}}%
\pgfpathlineto{\pgfqpoint{4.357483in}{5.200607in}}%
\pgfpathlineto{\pgfqpoint{4.360516in}{5.249583in}}%
\pgfpathlineto{\pgfqpoint{4.360769in}{5.250532in}}%
\pgfpathlineto{\pgfqpoint{4.361106in}{5.247100in}}%
\pgfpathlineto{\pgfqpoint{4.364897in}{5.187206in}}%
\pgfpathlineto{\pgfqpoint{4.365149in}{5.189916in}}%
\pgfpathlineto{\pgfqpoint{4.367677in}{5.251280in}}%
\pgfpathlineto{\pgfqpoint{4.368435in}{5.246823in}}%
\pgfpathlineto{\pgfqpoint{4.371215in}{5.233274in}}%
\pgfpathlineto{\pgfqpoint{4.371383in}{5.234234in}}%
\pgfpathlineto{\pgfqpoint{4.372394in}{5.263237in}}%
\pgfpathlineto{\pgfqpoint{4.372984in}{5.242763in}}%
\pgfpathlineto{\pgfqpoint{4.375090in}{5.191164in}}%
\pgfpathlineto{\pgfqpoint{4.375343in}{5.191301in}}%
\pgfpathlineto{\pgfqpoint{4.375511in}{5.190580in}}%
\pgfpathlineto{\pgfqpoint{4.376270in}{5.179899in}}%
\pgfpathlineto{\pgfqpoint{4.376607in}{5.186745in}}%
\pgfpathlineto{\pgfqpoint{4.378292in}{5.276586in}}%
\pgfpathlineto{\pgfqpoint{4.379639in}{5.270689in}}%
\pgfpathlineto{\pgfqpoint{4.380566in}{5.220995in}}%
\pgfpathlineto{\pgfqpoint{4.382588in}{5.151789in}}%
\pgfpathlineto{\pgfqpoint{4.382672in}{5.151567in}}%
\pgfpathlineto{\pgfqpoint{4.382841in}{5.152727in}}%
\pgfpathlineto{\pgfqpoint{4.383767in}{5.188041in}}%
\pgfpathlineto{\pgfqpoint{4.386632in}{5.271683in}}%
\pgfpathlineto{\pgfqpoint{4.386800in}{5.270489in}}%
\pgfpathlineto{\pgfqpoint{4.389833in}{5.244686in}}%
\pgfpathlineto{\pgfqpoint{4.391265in}{5.233446in}}%
\pgfpathlineto{\pgfqpoint{4.394045in}{5.185770in}}%
\pgfpathlineto{\pgfqpoint{4.394719in}{5.193654in}}%
\pgfpathlineto{\pgfqpoint{4.398847in}{5.259076in}}%
\pgfpathlineto{\pgfqpoint{4.399016in}{5.257711in}}%
\pgfpathlineto{\pgfqpoint{4.399942in}{5.212936in}}%
\pgfpathlineto{\pgfqpoint{4.400701in}{5.193357in}}%
\pgfpathlineto{\pgfqpoint{4.401290in}{5.199666in}}%
\pgfpathlineto{\pgfqpoint{4.404829in}{5.273309in}}%
\pgfpathlineto{\pgfqpoint{4.405587in}{5.262892in}}%
\pgfpathlineto{\pgfqpoint{4.408957in}{5.176328in}}%
\pgfpathlineto{\pgfqpoint{4.409462in}{5.185819in}}%
\pgfpathlineto{\pgfqpoint{4.412326in}{5.262710in}}%
\pgfpathlineto{\pgfqpoint{4.412579in}{5.259799in}}%
\pgfpathlineto{\pgfqpoint{4.415022in}{5.204352in}}%
\pgfpathlineto{\pgfqpoint{4.415443in}{5.208365in}}%
\pgfpathlineto{\pgfqpoint{4.417802in}{5.257983in}}%
\pgfpathlineto{\pgfqpoint{4.418476in}{5.249489in}}%
\pgfpathlineto{\pgfqpoint{4.422183in}{5.188841in}}%
\pgfpathlineto{\pgfqpoint{4.423447in}{5.194893in}}%
\pgfpathlineto{\pgfqpoint{4.423699in}{5.197100in}}%
\pgfpathlineto{\pgfqpoint{4.424205in}{5.191703in}}%
\pgfpathlineto{\pgfqpoint{4.424458in}{5.190042in}}%
\pgfpathlineto{\pgfqpoint{4.424795in}{5.195906in}}%
\pgfpathlineto{\pgfqpoint{4.426479in}{5.213097in}}%
\pgfpathlineto{\pgfqpoint{4.429512in}{5.261278in}}%
\pgfpathlineto{\pgfqpoint{4.430018in}{5.257442in}}%
\pgfpathlineto{\pgfqpoint{4.430944in}{5.233604in}}%
\pgfpathlineto{\pgfqpoint{4.432629in}{5.177244in}}%
\pgfpathlineto{\pgfqpoint{4.433050in}{5.187163in}}%
\pgfpathlineto{\pgfqpoint{4.436168in}{5.303262in}}%
\pgfpathlineto{\pgfqpoint{4.436757in}{5.293114in}}%
\pgfpathlineto{\pgfqpoint{4.438442in}{5.213437in}}%
\pgfpathlineto{\pgfqpoint{4.440548in}{5.152957in}}%
\pgfpathlineto{\pgfqpoint{4.440717in}{5.152438in}}%
\pgfpathlineto{\pgfqpoint{4.440969in}{5.155090in}}%
\pgfpathlineto{\pgfqpoint{4.442149in}{5.203651in}}%
\pgfpathlineto{\pgfqpoint{4.444424in}{5.261697in}}%
\pgfpathlineto{\pgfqpoint{4.444592in}{5.261971in}}%
\pgfpathlineto{\pgfqpoint{4.445097in}{5.260291in}}%
\pgfpathlineto{\pgfqpoint{4.445856in}{5.255139in}}%
\pgfpathlineto{\pgfqpoint{4.446361in}{5.258206in}}%
\pgfpathlineto{\pgfqpoint{4.446782in}{5.260596in}}%
\pgfpathlineto{\pgfqpoint{4.447288in}{5.256928in}}%
\pgfpathlineto{\pgfqpoint{4.448636in}{5.233091in}}%
\pgfpathlineto{\pgfqpoint{4.450489in}{5.177307in}}%
\pgfpathlineto{\pgfqpoint{4.451163in}{5.180938in}}%
\pgfpathlineto{\pgfqpoint{4.452848in}{5.211373in}}%
\pgfpathlineto{\pgfqpoint{4.454196in}{5.270183in}}%
\pgfpathlineto{\pgfqpoint{4.455038in}{5.256728in}}%
\pgfpathlineto{\pgfqpoint{4.457987in}{5.186950in}}%
\pgfpathlineto{\pgfqpoint{4.458661in}{5.206287in}}%
\pgfpathlineto{\pgfqpoint{4.460430in}{5.247448in}}%
\pgfpathlineto{\pgfqpoint{4.461272in}{5.252258in}}%
\pgfpathlineto{\pgfqpoint{4.461609in}{5.250020in}}%
\pgfpathlineto{\pgfqpoint{4.462620in}{5.218677in}}%
\pgfpathlineto{\pgfqpoint{4.463379in}{5.199720in}}%
\pgfpathlineto{\pgfqpoint{4.463884in}{5.206902in}}%
\pgfpathlineto{\pgfqpoint{4.466917in}{5.264957in}}%
\pgfpathlineto{\pgfqpoint{4.467001in}{5.264526in}}%
\pgfpathlineto{\pgfqpoint{4.467675in}{5.248001in}}%
\pgfpathlineto{\pgfqpoint{4.469697in}{5.188637in}}%
\pgfpathlineto{\pgfqpoint{4.470287in}{5.192767in}}%
\pgfpathlineto{\pgfqpoint{4.472140in}{5.212242in}}%
\pgfpathlineto{\pgfqpoint{4.473151in}{5.243654in}}%
\pgfpathlineto{\pgfqpoint{4.473656in}{5.228085in}}%
\pgfpathlineto{\pgfqpoint{4.474415in}{5.200007in}}%
\pgfpathlineto{\pgfqpoint{4.475089in}{5.207682in}}%
\pgfpathlineto{\pgfqpoint{4.475257in}{5.208474in}}%
\pgfpathlineto{\pgfqpoint{4.475678in}{5.205353in}}%
\pgfpathlineto{\pgfqpoint{4.476184in}{5.201096in}}%
\pgfpathlineto{\pgfqpoint{4.476689in}{5.205646in}}%
\pgfpathlineto{\pgfqpoint{4.477532in}{5.225702in}}%
\pgfpathlineto{\pgfqpoint{4.478627in}{5.272614in}}%
\pgfpathlineto{\pgfqpoint{4.479385in}{5.264499in}}%
\pgfpathlineto{\pgfqpoint{4.479469in}{5.264405in}}%
\pgfpathlineto{\pgfqpoint{4.479638in}{5.265221in}}%
\pgfpathlineto{\pgfqpoint{4.480059in}{5.268050in}}%
\pgfpathlineto{\pgfqpoint{4.480396in}{5.263256in}}%
\pgfpathlineto{\pgfqpoint{4.481660in}{5.190888in}}%
\pgfpathlineto{\pgfqpoint{4.482334in}{5.169417in}}%
\pgfpathlineto{\pgfqpoint{4.482923in}{5.180701in}}%
\pgfpathlineto{\pgfqpoint{4.485198in}{5.249481in}}%
\pgfpathlineto{\pgfqpoint{4.486293in}{5.291348in}}%
\pgfpathlineto{\pgfqpoint{4.486799in}{5.282741in}}%
\pgfpathlineto{\pgfqpoint{4.490337in}{5.186111in}}%
\pgfpathlineto{\pgfqpoint{4.491600in}{5.198689in}}%
\pgfpathlineto{\pgfqpoint{4.494549in}{5.226087in}}%
\pgfpathlineto{\pgfqpoint{4.497666in}{5.262287in}}%
\pgfpathlineto{\pgfqpoint{4.498087in}{5.263023in}}%
\pgfpathlineto{\pgfqpoint{4.498340in}{5.261573in}}%
\pgfpathlineto{\pgfqpoint{4.499098in}{5.230881in}}%
\pgfpathlineto{\pgfqpoint{4.501541in}{5.166163in}}%
\pgfpathlineto{\pgfqpoint{4.501878in}{5.168928in}}%
\pgfpathlineto{\pgfqpoint{4.504153in}{5.221840in}}%
\pgfpathlineto{\pgfqpoint{4.506512in}{5.262083in}}%
\pgfpathlineto{\pgfqpoint{4.506596in}{5.261899in}}%
\pgfpathlineto{\pgfqpoint{4.507101in}{5.248665in}}%
\pgfpathlineto{\pgfqpoint{4.508618in}{5.210357in}}%
\pgfpathlineto{\pgfqpoint{4.509039in}{5.212219in}}%
\pgfpathlineto{\pgfqpoint{4.510218in}{5.233887in}}%
\pgfpathlineto{\pgfqpoint{4.512325in}{5.291718in}}%
\pgfpathlineto{\pgfqpoint{4.512830in}{5.281288in}}%
\pgfpathlineto{\pgfqpoint{4.515526in}{5.188984in}}%
\pgfpathlineto{\pgfqpoint{4.515947in}{5.189538in}}%
\pgfpathlineto{\pgfqpoint{4.516453in}{5.191065in}}%
\pgfpathlineto{\pgfqpoint{4.517632in}{5.201538in}}%
\pgfpathlineto{\pgfqpoint{4.519570in}{5.223827in}}%
\pgfpathlineto{\pgfqpoint{4.519991in}{5.220327in}}%
\pgfpathlineto{\pgfqpoint{4.520581in}{5.212583in}}%
\pgfpathlineto{\pgfqpoint{4.521002in}{5.220195in}}%
\pgfpathlineto{\pgfqpoint{4.521507in}{5.227686in}}%
\pgfpathlineto{\pgfqpoint{4.522013in}{5.220040in}}%
\pgfpathlineto{\pgfqpoint{4.522518in}{5.214498in}}%
\pgfpathlineto{\pgfqpoint{4.523024in}{5.221014in}}%
\pgfpathlineto{\pgfqpoint{4.524877in}{5.243402in}}%
\pgfpathlineto{\pgfqpoint{4.525214in}{5.240903in}}%
\pgfpathlineto{\pgfqpoint{4.527320in}{5.222635in}}%
\pgfpathlineto{\pgfqpoint{4.527573in}{5.223904in}}%
\pgfpathlineto{\pgfqpoint{4.527994in}{5.228646in}}%
\pgfpathlineto{\pgfqpoint{4.528415in}{5.222558in}}%
\pgfpathlineto{\pgfqpoint{4.529089in}{5.211513in}}%
\pgfpathlineto{\pgfqpoint{4.529847in}{5.212457in}}%
\pgfpathlineto{\pgfqpoint{4.530269in}{5.213527in}}%
\pgfpathlineto{\pgfqpoint{4.530690in}{5.211529in}}%
\pgfpathlineto{\pgfqpoint{4.530943in}{5.210696in}}%
\pgfpathlineto{\pgfqpoint{4.531448in}{5.213446in}}%
\pgfpathlineto{\pgfqpoint{4.533133in}{5.250817in}}%
\pgfpathlineto{\pgfqpoint{4.534312in}{5.242128in}}%
\pgfpathlineto{\pgfqpoint{4.537598in}{5.201916in}}%
\pgfpathlineto{\pgfqpoint{4.538019in}{5.210780in}}%
\pgfpathlineto{\pgfqpoint{4.538777in}{5.225831in}}%
\pgfpathlineto{\pgfqpoint{4.539283in}{5.218237in}}%
\pgfpathlineto{\pgfqpoint{4.539788in}{5.211654in}}%
\pgfpathlineto{\pgfqpoint{4.540210in}{5.218385in}}%
\pgfpathlineto{\pgfqpoint{4.541305in}{5.243995in}}%
\pgfpathlineto{\pgfqpoint{4.541810in}{5.232809in}}%
\pgfpathlineto{\pgfqpoint{4.542484in}{5.215633in}}%
\pgfpathlineto{\pgfqpoint{4.542990in}{5.226454in}}%
\pgfpathlineto{\pgfqpoint{4.543495in}{5.240885in}}%
\pgfpathlineto{\pgfqpoint{4.544085in}{5.227295in}}%
\pgfpathlineto{\pgfqpoint{4.545517in}{5.203553in}}%
\pgfpathlineto{\pgfqpoint{4.545770in}{5.204808in}}%
\pgfpathlineto{\pgfqpoint{4.548213in}{5.232053in}}%
\pgfpathlineto{\pgfqpoint{4.548802in}{5.227856in}}%
\pgfpathlineto{\pgfqpoint{4.550319in}{5.208930in}}%
\pgfpathlineto{\pgfqpoint{4.550909in}{5.216294in}}%
\pgfpathlineto{\pgfqpoint{4.554110in}{5.268699in}}%
\pgfpathlineto{\pgfqpoint{4.554447in}{5.265156in}}%
\pgfpathlineto{\pgfqpoint{4.557901in}{5.186240in}}%
\pgfpathlineto{\pgfqpoint{4.558491in}{5.198243in}}%
\pgfpathlineto{\pgfqpoint{4.560765in}{5.236919in}}%
\pgfpathlineto{\pgfqpoint{4.561608in}{5.244533in}}%
\pgfpathlineto{\pgfqpoint{4.562113in}{5.240258in}}%
\pgfpathlineto{\pgfqpoint{4.565314in}{5.200044in}}%
\pgfpathlineto{\pgfqpoint{4.565483in}{5.200020in}}%
\pgfpathlineto{\pgfqpoint{4.565988in}{5.200952in}}%
\pgfpathlineto{\pgfqpoint{4.566578in}{5.204773in}}%
\pgfpathlineto{\pgfqpoint{4.570032in}{5.246801in}}%
\pgfpathlineto{\pgfqpoint{4.570453in}{5.245030in}}%
\pgfpathlineto{\pgfqpoint{4.571464in}{5.227362in}}%
\pgfpathlineto{\pgfqpoint{4.571885in}{5.224001in}}%
\pgfpathlineto{\pgfqpoint{4.572559in}{5.226967in}}%
\pgfpathlineto{\pgfqpoint{4.572812in}{5.227701in}}%
\pgfpathlineto{\pgfqpoint{4.573149in}{5.224893in}}%
\pgfpathlineto{\pgfqpoint{4.574750in}{5.203751in}}%
\pgfpathlineto{\pgfqpoint{4.575171in}{5.209390in}}%
\pgfpathlineto{\pgfqpoint{4.578709in}{5.260931in}}%
\pgfpathlineto{\pgfqpoint{4.579131in}{5.264344in}}%
\pgfpathlineto{\pgfqpoint{4.579636in}{5.259721in}}%
\pgfpathlineto{\pgfqpoint{4.581489in}{5.209431in}}%
\pgfpathlineto{\pgfqpoint{4.583174in}{5.177945in}}%
\pgfpathlineto{\pgfqpoint{4.583764in}{5.183918in}}%
\pgfpathlineto{\pgfqpoint{4.586460in}{5.236112in}}%
\pgfpathlineto{\pgfqpoint{4.587471in}{5.229404in}}%
\pgfpathlineto{\pgfqpoint{4.588397in}{5.227241in}}%
\pgfpathlineto{\pgfqpoint{4.588819in}{5.225559in}}%
\pgfpathlineto{\pgfqpoint{4.589240in}{5.228557in}}%
\pgfpathlineto{\pgfqpoint{4.591683in}{5.254128in}}%
\pgfpathlineto{\pgfqpoint{4.592104in}{5.252928in}}%
\pgfpathlineto{\pgfqpoint{4.593199in}{5.248036in}}%
\pgfpathlineto{\pgfqpoint{4.595558in}{5.197589in}}%
\pgfpathlineto{\pgfqpoint{4.596232in}{5.206002in}}%
\pgfpathlineto{\pgfqpoint{4.596569in}{5.208402in}}%
\pgfpathlineto{\pgfqpoint{4.597075in}{5.203460in}}%
\pgfpathlineto{\pgfqpoint{4.597412in}{5.201130in}}%
\pgfpathlineto{\pgfqpoint{4.597917in}{5.205475in}}%
\pgfpathlineto{\pgfqpoint{4.599855in}{5.235620in}}%
\pgfpathlineto{\pgfqpoint{4.600360in}{5.230935in}}%
\pgfpathlineto{\pgfqpoint{4.600529in}{5.230121in}}%
\pgfpathlineto{\pgfqpoint{4.601034in}{5.233614in}}%
\pgfpathlineto{\pgfqpoint{4.601371in}{5.235009in}}%
\pgfpathlineto{\pgfqpoint{4.601792in}{5.231902in}}%
\pgfpathlineto{\pgfqpoint{4.603224in}{5.206178in}}%
\pgfpathlineto{\pgfqpoint{4.603983in}{5.214694in}}%
\pgfpathlineto{\pgfqpoint{4.605499in}{5.240400in}}%
\pgfpathlineto{\pgfqpoint{4.606257in}{5.238260in}}%
\pgfpathlineto{\pgfqpoint{4.606931in}{5.241642in}}%
\pgfpathlineto{\pgfqpoint{4.607437in}{5.238850in}}%
\pgfpathlineto{\pgfqpoint{4.608785in}{5.212128in}}%
\pgfpathlineto{\pgfqpoint{4.609459in}{5.225547in}}%
\pgfpathlineto{\pgfqpoint{4.609796in}{5.230299in}}%
\pgfpathlineto{\pgfqpoint{4.610385in}{5.222345in}}%
\pgfpathlineto{\pgfqpoint{4.610806in}{5.218602in}}%
\pgfpathlineto{\pgfqpoint{4.611312in}{5.225197in}}%
\pgfpathlineto{\pgfqpoint{4.612576in}{5.241671in}}%
\pgfpathlineto{\pgfqpoint{4.613081in}{5.238422in}}%
\pgfpathlineto{\pgfqpoint{4.615019in}{5.219181in}}%
\pgfpathlineto{\pgfqpoint{4.615440in}{5.211773in}}%
\pgfpathlineto{\pgfqpoint{4.616030in}{5.221092in}}%
\pgfpathlineto{\pgfqpoint{4.616619in}{5.230761in}}%
\pgfpathlineto{\pgfqpoint{4.617293in}{5.226395in}}%
\pgfpathlineto{\pgfqpoint{4.617883in}{5.224719in}}%
\pgfpathlineto{\pgfqpoint{4.618220in}{5.226871in}}%
\pgfpathlineto{\pgfqpoint{4.618894in}{5.235691in}}%
\pgfpathlineto{\pgfqpoint{4.619399in}{5.229543in}}%
\pgfpathlineto{\pgfqpoint{4.622348in}{5.195219in}}%
\pgfpathlineto{\pgfqpoint{4.622769in}{5.191382in}}%
\pgfpathlineto{\pgfqpoint{4.623359in}{5.196526in}}%
\pgfpathlineto{\pgfqpoint{4.628414in}{5.273741in}}%
\pgfpathlineto{\pgfqpoint{4.628582in}{5.274399in}}%
\pgfpathlineto{\pgfqpoint{4.628919in}{5.270264in}}%
\pgfpathlineto{\pgfqpoint{4.632710in}{5.181899in}}%
\pgfpathlineto{\pgfqpoint{4.633889in}{5.190144in}}%
\pgfpathlineto{\pgfqpoint{4.634058in}{5.189905in}}%
\pgfpathlineto{\pgfqpoint{4.634226in}{5.191115in}}%
\pgfpathlineto{\pgfqpoint{4.637512in}{5.247707in}}%
\pgfpathlineto{\pgfqpoint{4.638270in}{5.237546in}}%
\pgfpathlineto{\pgfqpoint{4.638944in}{5.232249in}}%
\pgfpathlineto{\pgfqpoint{4.639618in}{5.233226in}}%
\pgfpathlineto{\pgfqpoint{4.641134in}{5.226488in}}%
\pgfpathlineto{\pgfqpoint{4.641556in}{5.222517in}}%
\pgfpathlineto{\pgfqpoint{4.641977in}{5.229955in}}%
\pgfpathlineto{\pgfqpoint{4.642567in}{5.244787in}}%
\pgfpathlineto{\pgfqpoint{4.643072in}{5.234563in}}%
\pgfpathlineto{\pgfqpoint{4.643578in}{5.224837in}}%
\pgfpathlineto{\pgfqpoint{4.644336in}{5.229422in}}%
\pgfpathlineto{\pgfqpoint{4.645010in}{5.210499in}}%
\pgfpathlineto{\pgfqpoint{4.646021in}{5.185237in}}%
\pgfpathlineto{\pgfqpoint{4.646526in}{5.191395in}}%
\pgfpathlineto{\pgfqpoint{4.648716in}{5.233828in}}%
\pgfpathlineto{\pgfqpoint{4.650401in}{5.254672in}}%
\pgfpathlineto{\pgfqpoint{4.650570in}{5.255338in}}%
\pgfpathlineto{\pgfqpoint{4.651160in}{5.253306in}}%
\pgfpathlineto{\pgfqpoint{4.651749in}{5.245234in}}%
\pgfpathlineto{\pgfqpoint{4.653855in}{5.215336in}}%
\pgfpathlineto{\pgfqpoint{4.654192in}{5.214375in}}%
\pgfpathlineto{\pgfqpoint{4.654614in}{5.217133in}}%
\pgfpathlineto{\pgfqpoint{4.655962in}{5.233088in}}%
\pgfpathlineto{\pgfqpoint{4.656720in}{5.230817in}}%
\pgfpathlineto{\pgfqpoint{4.658152in}{5.212886in}}%
\pgfpathlineto{\pgfqpoint{4.658910in}{5.218922in}}%
\pgfpathlineto{\pgfqpoint{4.660679in}{5.241488in}}%
\pgfpathlineto{\pgfqpoint{4.662111in}{5.255952in}}%
\pgfpathlineto{\pgfqpoint{4.662280in}{5.255760in}}%
\pgfpathlineto{\pgfqpoint{4.662701in}{5.252002in}}%
\pgfpathlineto{\pgfqpoint{4.664891in}{5.201830in}}%
\pgfpathlineto{\pgfqpoint{4.665650in}{5.211629in}}%
\pgfpathlineto{\pgfqpoint{4.665818in}{5.212734in}}%
\pgfpathlineto{\pgfqpoint{4.666324in}{5.208529in}}%
\pgfpathlineto{\pgfqpoint{4.666492in}{5.207774in}}%
\pgfpathlineto{\pgfqpoint{4.666745in}{5.211482in}}%
\pgfpathlineto{\pgfqpoint{4.667840in}{5.251572in}}%
\pgfpathlineto{\pgfqpoint{4.668514in}{5.235786in}}%
\pgfpathlineto{\pgfqpoint{4.669693in}{5.214589in}}%
\pgfpathlineto{\pgfqpoint{4.670199in}{5.218943in}}%
\pgfpathlineto{\pgfqpoint{4.670452in}{5.220303in}}%
\pgfpathlineto{\pgfqpoint{4.670789in}{5.216007in}}%
\pgfpathlineto{\pgfqpoint{4.671631in}{5.195081in}}%
\pgfpathlineto{\pgfqpoint{4.672136in}{5.205545in}}%
\pgfpathlineto{\pgfqpoint{4.673063in}{5.230127in}}%
\pgfpathlineto{\pgfqpoint{4.673821in}{5.229314in}}%
\pgfpathlineto{\pgfqpoint{4.674917in}{5.234174in}}%
\pgfpathlineto{\pgfqpoint{4.675338in}{5.231623in}}%
\pgfpathlineto{\pgfqpoint{4.676517in}{5.205165in}}%
\pgfpathlineto{\pgfqpoint{4.677275in}{5.215742in}}%
\pgfpathlineto{\pgfqpoint{4.677360in}{5.216272in}}%
\pgfpathlineto{\pgfqpoint{4.677781in}{5.212328in}}%
\pgfpathlineto{\pgfqpoint{4.678118in}{5.208954in}}%
\pgfpathlineto{\pgfqpoint{4.678455in}{5.215006in}}%
\pgfpathlineto{\pgfqpoint{4.680814in}{5.266226in}}%
\pgfpathlineto{\pgfqpoint{4.681235in}{5.260045in}}%
\pgfpathlineto{\pgfqpoint{4.684099in}{5.190105in}}%
\pgfpathlineto{\pgfqpoint{4.684520in}{5.197856in}}%
\pgfpathlineto{\pgfqpoint{4.687469in}{5.249646in}}%
\pgfpathlineto{\pgfqpoint{4.687806in}{5.251664in}}%
\pgfpathlineto{\pgfqpoint{4.688227in}{5.247063in}}%
\pgfpathlineto{\pgfqpoint{4.690165in}{5.188232in}}%
\pgfpathlineto{\pgfqpoint{4.691091in}{5.197963in}}%
\pgfpathlineto{\pgfqpoint{4.692187in}{5.215444in}}%
\pgfpathlineto{\pgfqpoint{4.694209in}{5.276625in}}%
\pgfpathlineto{\pgfqpoint{4.694967in}{5.268359in}}%
\pgfpathlineto{\pgfqpoint{4.695556in}{5.264160in}}%
\pgfpathlineto{\pgfqpoint{4.698673in}{5.185239in}}%
\pgfpathlineto{\pgfqpoint{4.699347in}{5.189380in}}%
\pgfpathlineto{\pgfqpoint{4.700780in}{5.229086in}}%
\pgfpathlineto{\pgfqpoint{4.702633in}{5.225736in}}%
\pgfpathlineto{\pgfqpoint{4.702886in}{5.224987in}}%
\pgfpathlineto{\pgfqpoint{4.703223in}{5.227870in}}%
\pgfpathlineto{\pgfqpoint{4.703812in}{5.233997in}}%
\pgfpathlineto{\pgfqpoint{4.704318in}{5.228001in}}%
\pgfpathlineto{\pgfqpoint{4.704486in}{5.226949in}}%
\pgfpathlineto{\pgfqpoint{4.704823in}{5.231742in}}%
\pgfpathlineto{\pgfqpoint{4.705413in}{5.244154in}}%
\pgfpathlineto{\pgfqpoint{4.706087in}{5.236641in}}%
\pgfpathlineto{\pgfqpoint{4.706508in}{5.233474in}}%
\pgfpathlineto{\pgfqpoint{4.707014in}{5.237277in}}%
\pgfpathlineto{\pgfqpoint{4.707940in}{5.250441in}}%
\pgfpathlineto{\pgfqpoint{4.708446in}{5.243502in}}%
\pgfpathlineto{\pgfqpoint{4.709036in}{5.234705in}}%
\pgfpathlineto{\pgfqpoint{4.709541in}{5.242420in}}%
\pgfpathlineto{\pgfqpoint{4.710047in}{5.251118in}}%
\pgfpathlineto{\pgfqpoint{4.710468in}{5.241645in}}%
\pgfpathlineto{\pgfqpoint{4.711142in}{5.224582in}}%
\pgfpathlineto{\pgfqpoint{4.711816in}{5.230967in}}%
\pgfpathlineto{\pgfqpoint{4.711900in}{5.231059in}}%
\pgfpathlineto{\pgfqpoint{4.712068in}{5.230059in}}%
\pgfpathlineto{\pgfqpoint{4.713079in}{5.203405in}}%
\pgfpathlineto{\pgfqpoint{4.713838in}{5.187163in}}%
\pgfpathlineto{\pgfqpoint{4.714511in}{5.192306in}}%
\pgfpathlineto{\pgfqpoint{4.714764in}{5.193454in}}%
\pgfpathlineto{\pgfqpoint{4.715354in}{5.190734in}}%
\pgfpathlineto{\pgfqpoint{4.715607in}{5.191141in}}%
\pgfpathlineto{\pgfqpoint{4.715691in}{5.191693in}}%
\pgfpathlineto{\pgfqpoint{4.716365in}{5.209041in}}%
\pgfpathlineto{\pgfqpoint{4.718387in}{5.247979in}}%
\pgfpathlineto{\pgfqpoint{4.719145in}{5.262917in}}%
\pgfpathlineto{\pgfqpoint{4.719735in}{5.257684in}}%
\pgfpathlineto{\pgfqpoint{4.720830in}{5.228526in}}%
\pgfpathlineto{\pgfqpoint{4.722852in}{5.195185in}}%
\pgfpathlineto{\pgfqpoint{4.723189in}{5.198740in}}%
\pgfpathlineto{\pgfqpoint{4.725042in}{5.217742in}}%
\pgfpathlineto{\pgfqpoint{4.725211in}{5.216979in}}%
\pgfpathlineto{\pgfqpoint{4.725632in}{5.214647in}}%
\pgfpathlineto{\pgfqpoint{4.726053in}{5.217836in}}%
\pgfpathlineto{\pgfqpoint{4.726474in}{5.221930in}}%
\pgfpathlineto{\pgfqpoint{4.726980in}{5.216932in}}%
\pgfpathlineto{\pgfqpoint{4.727232in}{5.215366in}}%
\pgfpathlineto{\pgfqpoint{4.727654in}{5.219742in}}%
\pgfpathlineto{\pgfqpoint{4.728243in}{5.227748in}}%
\pgfpathlineto{\pgfqpoint{4.728749in}{5.221165in}}%
\pgfpathlineto{\pgfqpoint{4.729002in}{5.219038in}}%
\pgfpathlineto{\pgfqpoint{4.729423in}{5.225448in}}%
\pgfpathlineto{\pgfqpoint{4.730012in}{5.237183in}}%
\pgfpathlineto{\pgfqpoint{4.730602in}{5.228747in}}%
\pgfpathlineto{\pgfqpoint{4.730855in}{5.226246in}}%
\pgfpathlineto{\pgfqpoint{4.731360in}{5.232419in}}%
\pgfpathlineto{\pgfqpoint{4.733045in}{5.253480in}}%
\pgfpathlineto{\pgfqpoint{4.733214in}{5.251817in}}%
\pgfpathlineto{\pgfqpoint{4.733972in}{5.238506in}}%
\pgfpathlineto{\pgfqpoint{4.734646in}{5.244969in}}%
\pgfpathlineto{\pgfqpoint{4.735067in}{5.247087in}}%
\pgfpathlineto{\pgfqpoint{4.735404in}{5.243865in}}%
\pgfpathlineto{\pgfqpoint{4.737510in}{5.201337in}}%
\pgfpathlineto{\pgfqpoint{4.738353in}{5.212032in}}%
\pgfpathlineto{\pgfqpoint{4.740796in}{5.241049in}}%
\pgfpathlineto{\pgfqpoint{4.741133in}{5.237726in}}%
\pgfpathlineto{\pgfqpoint{4.744250in}{5.188763in}}%
\pgfpathlineto{\pgfqpoint{4.744334in}{5.188937in}}%
\pgfpathlineto{\pgfqpoint{4.745008in}{5.199319in}}%
\pgfpathlineto{\pgfqpoint{4.747535in}{5.259541in}}%
\pgfpathlineto{\pgfqpoint{4.748630in}{5.253334in}}%
\pgfpathlineto{\pgfqpoint{4.750568in}{5.201276in}}%
\pgfpathlineto{\pgfqpoint{4.752422in}{5.203992in}}%
\pgfpathlineto{\pgfqpoint{4.755286in}{5.222435in}}%
\pgfpathlineto{\pgfqpoint{4.755539in}{5.221731in}}%
\pgfpathlineto{\pgfqpoint{4.755876in}{5.220659in}}%
\pgfpathlineto{\pgfqpoint{4.756297in}{5.223009in}}%
\pgfpathlineto{\pgfqpoint{4.758993in}{5.263771in}}%
\pgfpathlineto{\pgfqpoint{4.759582in}{5.255372in}}%
\pgfpathlineto{\pgfqpoint{4.762868in}{5.186197in}}%
\pgfpathlineto{\pgfqpoint{4.763205in}{5.189743in}}%
\pgfpathlineto{\pgfqpoint{4.767164in}{5.261680in}}%
\pgfpathlineto{\pgfqpoint{4.767586in}{5.256094in}}%
\pgfpathlineto{\pgfqpoint{4.769692in}{5.201468in}}%
\pgfpathlineto{\pgfqpoint{4.770366in}{5.209699in}}%
\pgfpathlineto{\pgfqpoint{4.770534in}{5.210793in}}%
\pgfpathlineto{\pgfqpoint{4.770955in}{5.205002in}}%
\pgfpathlineto{\pgfqpoint{4.771882in}{5.194687in}}%
\pgfpathlineto{\pgfqpoint{4.772303in}{5.197058in}}%
\pgfpathlineto{\pgfqpoint{4.773820in}{5.226786in}}%
\pgfpathlineto{\pgfqpoint{4.776431in}{5.274559in}}%
\pgfpathlineto{\pgfqpoint{4.776515in}{5.274284in}}%
\pgfpathlineto{\pgfqpoint{4.777105in}{5.263138in}}%
\pgfpathlineto{\pgfqpoint{4.780138in}{5.177739in}}%
\pgfpathlineto{\pgfqpoint{4.780896in}{5.182886in}}%
\pgfpathlineto{\pgfqpoint{4.782750in}{5.221274in}}%
\pgfpathlineto{\pgfqpoint{4.784350in}{5.260724in}}%
\pgfpathlineto{\pgfqpoint{4.785024in}{5.257290in}}%
\pgfpathlineto{\pgfqpoint{4.786541in}{5.239160in}}%
\pgfpathlineto{\pgfqpoint{4.787973in}{5.215324in}}%
\pgfpathlineto{\pgfqpoint{4.788815in}{5.221634in}}%
\pgfpathlineto{\pgfqpoint{4.789489in}{5.219168in}}%
\pgfpathlineto{\pgfqpoint{4.790247in}{5.220155in}}%
\pgfpathlineto{\pgfqpoint{4.792016in}{5.230867in}}%
\pgfpathlineto{\pgfqpoint{4.792522in}{5.226248in}}%
\pgfpathlineto{\pgfqpoint{4.792775in}{5.224593in}}%
\pgfpathlineto{\pgfqpoint{4.793196in}{5.230104in}}%
\pgfpathlineto{\pgfqpoint{4.793870in}{5.241458in}}%
\pgfpathlineto{\pgfqpoint{4.794375in}{5.233143in}}%
\pgfpathlineto{\pgfqpoint{4.795218in}{5.218667in}}%
\pgfpathlineto{\pgfqpoint{4.795807in}{5.224122in}}%
\pgfpathlineto{\pgfqpoint{4.796060in}{5.224933in}}%
\pgfpathlineto{\pgfqpoint{4.796650in}{5.222531in}}%
\pgfpathlineto{\pgfqpoint{4.797240in}{5.223138in}}%
\pgfpathlineto{\pgfqpoint{4.797661in}{5.218853in}}%
\pgfpathlineto{\pgfqpoint{4.798588in}{5.201930in}}%
\pgfpathlineto{\pgfqpoint{4.799261in}{5.207650in}}%
\pgfpathlineto{\pgfqpoint{4.799598in}{5.208745in}}%
\pgfpathlineto{\pgfqpoint{4.799935in}{5.206157in}}%
\pgfpathlineto{\pgfqpoint{4.800609in}{5.195322in}}%
\pgfpathlineto{\pgfqpoint{4.801115in}{5.204570in}}%
\pgfpathlineto{\pgfqpoint{4.802126in}{5.228950in}}%
\pgfpathlineto{\pgfqpoint{4.802800in}{5.226442in}}%
\pgfpathlineto{\pgfqpoint{4.803305in}{5.232112in}}%
\pgfpathlineto{\pgfqpoint{4.804063in}{5.243321in}}%
\pgfpathlineto{\pgfqpoint{4.804569in}{5.236251in}}%
\pgfpathlineto{\pgfqpoint{4.804990in}{5.232171in}}%
\pgfpathlineto{\pgfqpoint{4.805748in}{5.234586in}}%
\pgfpathlineto{\pgfqpoint{4.807602in}{5.224351in}}%
\pgfpathlineto{\pgfqpoint{4.807939in}{5.227043in}}%
\pgfpathlineto{\pgfqpoint{4.808613in}{5.235888in}}%
\pgfpathlineto{\pgfqpoint{4.809118in}{5.229837in}}%
\pgfpathlineto{\pgfqpoint{4.809455in}{5.227829in}}%
\pgfpathlineto{\pgfqpoint{4.809876in}{5.232556in}}%
\pgfpathlineto{\pgfqpoint{4.810887in}{5.248653in}}%
\pgfpathlineto{\pgfqpoint{4.811477in}{5.242613in}}%
\pgfpathlineto{\pgfqpoint{4.814510in}{5.201976in}}%
\pgfpathlineto{\pgfqpoint{4.814678in}{5.202852in}}%
\pgfpathlineto{\pgfqpoint{4.818469in}{5.235874in}}%
\pgfpathlineto{\pgfqpoint{4.818722in}{5.234844in}}%
\pgfpathlineto{\pgfqpoint{4.821249in}{5.219704in}}%
\pgfpathlineto{\pgfqpoint{4.821418in}{5.220057in}}%
\pgfpathlineto{\pgfqpoint{4.822260in}{5.225787in}}%
\pgfpathlineto{\pgfqpoint{4.823271in}{5.224819in}}%
\pgfpathlineto{\pgfqpoint{4.825377in}{5.217675in}}%
\pgfpathlineto{\pgfqpoint{4.827483in}{5.204387in}}%
\pgfpathlineto{\pgfqpoint{4.827736in}{5.207412in}}%
\pgfpathlineto{\pgfqpoint{4.828747in}{5.230714in}}%
\pgfpathlineto{\pgfqpoint{4.829505in}{5.223917in}}%
\pgfpathlineto{\pgfqpoint{4.830600in}{5.223263in}}%
\pgfpathlineto{\pgfqpoint{4.830685in}{5.223397in}}%
\pgfpathlineto{\pgfqpoint{4.831696in}{5.231336in}}%
\pgfpathlineto{\pgfqpoint{4.832285in}{5.226005in}}%
\pgfpathlineto{\pgfqpoint{4.832875in}{5.222372in}}%
\pgfpathlineto{\pgfqpoint{4.833549in}{5.223956in}}%
\pgfpathlineto{\pgfqpoint{4.834476in}{5.227385in}}%
\pgfpathlineto{\pgfqpoint{4.834897in}{5.224823in}}%
\pgfpathlineto{\pgfqpoint{4.835234in}{5.222418in}}%
\pgfpathlineto{\pgfqpoint{4.835655in}{5.227474in}}%
\pgfpathlineto{\pgfqpoint{4.836750in}{5.255906in}}%
\pgfpathlineto{\pgfqpoint{4.837424in}{5.248395in}}%
\pgfpathlineto{\pgfqpoint{4.839446in}{5.235098in}}%
\pgfpathlineto{\pgfqpoint{4.839952in}{5.226054in}}%
\pgfpathlineto{\pgfqpoint{4.840878in}{5.198179in}}%
\pgfpathlineto{\pgfqpoint{4.841552in}{5.206635in}}%
\pgfpathlineto{\pgfqpoint{4.844164in}{5.242654in}}%
\pgfpathlineto{\pgfqpoint{4.844585in}{5.236304in}}%
\pgfpathlineto{\pgfqpoint{4.846775in}{5.208762in}}%
\pgfpathlineto{\pgfqpoint{4.847028in}{5.210161in}}%
\pgfpathlineto{\pgfqpoint{4.850229in}{5.240417in}}%
\pgfpathlineto{\pgfqpoint{4.850566in}{5.238887in}}%
\pgfpathlineto{\pgfqpoint{4.853262in}{5.216830in}}%
\pgfpathlineto{\pgfqpoint{4.853683in}{5.219822in}}%
\pgfpathlineto{\pgfqpoint{4.855200in}{5.239931in}}%
\pgfpathlineto{\pgfqpoint{4.855958in}{5.236419in}}%
\pgfpathlineto{\pgfqpoint{4.857727in}{5.222081in}}%
\pgfpathlineto{\pgfqpoint{4.858654in}{5.198517in}}%
\pgfpathlineto{\pgfqpoint{4.859244in}{5.203433in}}%
\pgfpathlineto{\pgfqpoint{4.859412in}{5.203858in}}%
\pgfpathlineto{\pgfqpoint{4.859833in}{5.201688in}}%
\pgfpathlineto{\pgfqpoint{4.860170in}{5.200323in}}%
\pgfpathlineto{\pgfqpoint{4.860591in}{5.203905in}}%
\pgfpathlineto{\pgfqpoint{4.864046in}{5.258607in}}%
\pgfpathlineto{\pgfqpoint{4.864467in}{5.250713in}}%
\pgfpathlineto{\pgfqpoint{4.866741in}{5.204840in}}%
\pgfpathlineto{\pgfqpoint{4.867500in}{5.196013in}}%
\pgfpathlineto{\pgfqpoint{4.868005in}{5.201255in}}%
\pgfpathlineto{\pgfqpoint{4.870027in}{5.238176in}}%
\pgfpathlineto{\pgfqpoint{4.870954in}{5.230895in}}%
\pgfpathlineto{\pgfqpoint{4.873481in}{5.202291in}}%
\pgfpathlineto{\pgfqpoint{4.873902in}{5.206858in}}%
\pgfpathlineto{\pgfqpoint{4.875166in}{5.233220in}}%
\pgfpathlineto{\pgfqpoint{4.875840in}{5.227466in}}%
\pgfpathlineto{\pgfqpoint{4.876008in}{5.227186in}}%
\pgfpathlineto{\pgfqpoint{4.876345in}{5.229186in}}%
\pgfpathlineto{\pgfqpoint{4.877609in}{5.242082in}}%
\pgfpathlineto{\pgfqpoint{4.878199in}{5.237938in}}%
\pgfpathlineto{\pgfqpoint{4.878957in}{5.231295in}}%
\pgfpathlineto{\pgfqpoint{4.879462in}{5.235471in}}%
\pgfpathlineto{\pgfqpoint{4.879968in}{5.240289in}}%
\pgfpathlineto{\pgfqpoint{4.880473in}{5.235989in}}%
\pgfpathlineto{\pgfqpoint{4.881653in}{5.226439in}}%
\pgfpathlineto{\pgfqpoint{4.882074in}{5.229725in}}%
\pgfpathlineto{\pgfqpoint{4.882664in}{5.235551in}}%
\pgfpathlineto{\pgfqpoint{4.883253in}{5.231784in}}%
\pgfpathlineto{\pgfqpoint{4.883674in}{5.230642in}}%
\pgfpathlineto{\pgfqpoint{4.884264in}{5.232190in}}%
\pgfpathlineto{\pgfqpoint{4.884601in}{5.233163in}}%
\pgfpathlineto{\pgfqpoint{4.884938in}{5.231405in}}%
\pgfpathlineto{\pgfqpoint{4.886286in}{5.217554in}}%
\pgfpathlineto{\pgfqpoint{4.886960in}{5.220147in}}%
\pgfpathlineto{\pgfqpoint{4.888139in}{5.226083in}}%
\pgfpathlineto{\pgfqpoint{4.888645in}{5.224473in}}%
\pgfpathlineto{\pgfqpoint{4.891256in}{5.204429in}}%
\pgfpathlineto{\pgfqpoint{4.891593in}{5.207309in}}%
\pgfpathlineto{\pgfqpoint{4.893952in}{5.227846in}}%
\pgfpathlineto{\pgfqpoint{4.894037in}{5.228140in}}%
\pgfpathlineto{\pgfqpoint{4.894374in}{5.225877in}}%
\pgfpathlineto{\pgfqpoint{4.894626in}{5.224356in}}%
\pgfpathlineto{\pgfqpoint{4.895047in}{5.228153in}}%
\pgfpathlineto{\pgfqpoint{4.896817in}{5.241660in}}%
\pgfpathlineto{\pgfqpoint{4.896985in}{5.241157in}}%
\pgfpathlineto{\pgfqpoint{4.898333in}{5.220316in}}%
\pgfpathlineto{\pgfqpoint{4.899344in}{5.229726in}}%
\pgfpathlineto{\pgfqpoint{4.900439in}{5.220963in}}%
\pgfpathlineto{\pgfqpoint{4.901113in}{5.225646in}}%
\pgfpathlineto{\pgfqpoint{4.901619in}{5.229622in}}%
\pgfpathlineto{\pgfqpoint{4.902040in}{5.225510in}}%
\pgfpathlineto{\pgfqpoint{4.902545in}{5.220303in}}%
\pgfpathlineto{\pgfqpoint{4.903051in}{5.226816in}}%
\pgfpathlineto{\pgfqpoint{4.903556in}{5.232367in}}%
\pgfpathlineto{\pgfqpoint{4.904146in}{5.228263in}}%
\pgfpathlineto{\pgfqpoint{4.904483in}{5.226338in}}%
\pgfpathlineto{\pgfqpoint{4.904988in}{5.230466in}}%
\pgfpathlineto{\pgfqpoint{4.905157in}{5.231202in}}%
\pgfpathlineto{\pgfqpoint{4.905410in}{5.227854in}}%
\pgfpathlineto{\pgfqpoint{4.906336in}{5.209611in}}%
\pgfpathlineto{\pgfqpoint{4.906926in}{5.215063in}}%
\pgfpathlineto{\pgfqpoint{4.909285in}{5.230589in}}%
\pgfpathlineto{\pgfqpoint{4.909538in}{5.230966in}}%
\pgfpathlineto{\pgfqpoint{4.909790in}{5.229560in}}%
\pgfpathlineto{\pgfqpoint{4.910633in}{5.219440in}}%
\pgfpathlineto{\pgfqpoint{4.911138in}{5.226495in}}%
\pgfpathlineto{\pgfqpoint{4.911559in}{5.230986in}}%
\pgfpathlineto{\pgfqpoint{4.911981in}{5.224528in}}%
\pgfpathlineto{\pgfqpoint{4.912570in}{5.214595in}}%
\pgfpathlineto{\pgfqpoint{4.913076in}{5.221786in}}%
\pgfpathlineto{\pgfqpoint{4.914592in}{5.248824in}}%
\pgfpathlineto{\pgfqpoint{4.915182in}{5.243640in}}%
\pgfpathlineto{\pgfqpoint{4.916024in}{5.243720in}}%
\pgfpathlineto{\pgfqpoint{4.917204in}{5.206836in}}%
\pgfpathlineto{\pgfqpoint{4.917709in}{5.201429in}}%
\pgfpathlineto{\pgfqpoint{4.918383in}{5.203660in}}%
\pgfpathlineto{\pgfqpoint{4.920742in}{5.233241in}}%
\pgfpathlineto{\pgfqpoint{4.921500in}{5.254625in}}%
\pgfpathlineto{\pgfqpoint{4.922090in}{5.246319in}}%
\pgfpathlineto{\pgfqpoint{4.925123in}{5.197643in}}%
\pgfpathlineto{\pgfqpoint{4.925544in}{5.200927in}}%
\pgfpathlineto{\pgfqpoint{4.925881in}{5.203307in}}%
\pgfpathlineto{\pgfqpoint{4.926302in}{5.198815in}}%
\pgfpathlineto{\pgfqpoint{4.926723in}{5.193473in}}%
\pgfpathlineto{\pgfqpoint{4.927229in}{5.201900in}}%
\pgfpathlineto{\pgfqpoint{4.929251in}{5.234586in}}%
\pgfpathlineto{\pgfqpoint{4.930262in}{5.257732in}}%
\pgfpathlineto{\pgfqpoint{4.930851in}{5.248389in}}%
\pgfpathlineto{\pgfqpoint{4.931357in}{5.242391in}}%
\pgfpathlineto{\pgfqpoint{4.932115in}{5.245226in}}%
\pgfpathlineto{\pgfqpoint{4.933716in}{5.216248in}}%
\pgfpathlineto{\pgfqpoint{4.935569in}{5.188698in}}%
\pgfpathlineto{\pgfqpoint{4.935906in}{5.189760in}}%
\pgfpathlineto{\pgfqpoint{4.936917in}{5.205114in}}%
\pgfpathlineto{\pgfqpoint{4.940203in}{5.254038in}}%
\pgfpathlineto{\pgfqpoint{4.940624in}{5.257038in}}%
\pgfpathlineto{\pgfqpoint{4.941045in}{5.250959in}}%
\pgfpathlineto{\pgfqpoint{4.943320in}{5.221945in}}%
\pgfpathlineto{\pgfqpoint{4.943404in}{5.221491in}}%
\pgfpathlineto{\pgfqpoint{4.943825in}{5.225121in}}%
\pgfpathlineto{\pgfqpoint{4.944078in}{5.226563in}}%
\pgfpathlineto{\pgfqpoint{4.944499in}{5.221779in}}%
\pgfpathlineto{\pgfqpoint{4.944920in}{5.218320in}}%
\pgfpathlineto{\pgfqpoint{4.945763in}{5.218898in}}%
\pgfpathlineto{\pgfqpoint{4.947448in}{5.216717in}}%
\pgfpathlineto{\pgfqpoint{4.947616in}{5.217148in}}%
\pgfpathlineto{\pgfqpoint{4.950228in}{5.237285in}}%
\pgfpathlineto{\pgfqpoint{4.950649in}{5.231084in}}%
\pgfpathlineto{\pgfqpoint{4.952587in}{5.206107in}}%
\pgfpathlineto{\pgfqpoint{4.952923in}{5.203069in}}%
\pgfpathlineto{\pgfqpoint{4.953429in}{5.209810in}}%
\pgfpathlineto{\pgfqpoint{4.955956in}{5.240995in}}%
\pgfpathlineto{\pgfqpoint{4.956714in}{5.252136in}}%
\pgfpathlineto{\pgfqpoint{4.957220in}{5.246101in}}%
\pgfpathlineto{\pgfqpoint{4.960253in}{5.212207in}}%
\pgfpathlineto{\pgfqpoint{4.960505in}{5.211924in}}%
\pgfpathlineto{\pgfqpoint{4.960842in}{5.213671in}}%
\pgfpathlineto{\pgfqpoint{4.961516in}{5.221313in}}%
\pgfpathlineto{\pgfqpoint{4.962022in}{5.214471in}}%
\pgfpathlineto{\pgfqpoint{4.962780in}{5.196479in}}%
\pgfpathlineto{\pgfqpoint{4.963286in}{5.206473in}}%
\pgfpathlineto{\pgfqpoint{4.965055in}{5.227555in}}%
\pgfpathlineto{\pgfqpoint{4.966908in}{5.241664in}}%
\pgfpathlineto{\pgfqpoint{4.967077in}{5.241333in}}%
\pgfpathlineto{\pgfqpoint{4.967245in}{5.240812in}}%
\pgfpathlineto{\pgfqpoint{4.967582in}{5.242954in}}%
\pgfpathlineto{\pgfqpoint{4.968003in}{5.247135in}}%
\pgfpathlineto{\pgfqpoint{4.968424in}{5.241794in}}%
\pgfpathlineto{\pgfqpoint{4.970362in}{5.221377in}}%
\pgfpathlineto{\pgfqpoint{4.970615in}{5.222274in}}%
\pgfpathlineto{\pgfqpoint{4.971457in}{5.232305in}}%
\pgfpathlineto{\pgfqpoint{4.972047in}{5.226394in}}%
\pgfpathlineto{\pgfqpoint{4.972889in}{5.226875in}}%
\pgfpathlineto{\pgfqpoint{4.973732in}{5.221875in}}%
\pgfpathlineto{\pgfqpoint{4.973985in}{5.221991in}}%
\pgfpathlineto{\pgfqpoint{4.974153in}{5.222580in}}%
\pgfpathlineto{\pgfqpoint{4.975333in}{5.230612in}}%
\pgfpathlineto{\pgfqpoint{4.975838in}{5.227028in}}%
\pgfpathlineto{\pgfqpoint{4.977776in}{5.214347in}}%
\pgfpathlineto{\pgfqpoint{4.979124in}{5.207646in}}%
\pgfpathlineto{\pgfqpoint{4.979797in}{5.202483in}}%
\pgfpathlineto{\pgfqpoint{4.980219in}{5.206707in}}%
\pgfpathlineto{\pgfqpoint{4.983504in}{5.257336in}}%
\pgfpathlineto{\pgfqpoint{4.983673in}{5.256388in}}%
\pgfpathlineto{\pgfqpoint{4.987464in}{5.211249in}}%
\pgfpathlineto{\pgfqpoint{4.987632in}{5.211451in}}%
\pgfpathlineto{\pgfqpoint{4.988138in}{5.214403in}}%
\pgfpathlineto{\pgfqpoint{4.989738in}{5.251597in}}%
\pgfpathlineto{\pgfqpoint{4.990749in}{5.237490in}}%
\pgfpathlineto{\pgfqpoint{4.993108in}{5.203583in}}%
\pgfpathlineto{\pgfqpoint{4.993529in}{5.201733in}}%
\pgfpathlineto{\pgfqpoint{4.994035in}{5.204800in}}%
\pgfpathlineto{\pgfqpoint{4.996309in}{5.233388in}}%
\pgfpathlineto{\pgfqpoint{4.997236in}{5.226931in}}%
\pgfpathlineto{\pgfqpoint{4.998163in}{5.222160in}}%
\pgfpathlineto{\pgfqpoint{4.998753in}{5.224252in}}%
\pgfpathlineto{\pgfqpoint{4.999005in}{5.224577in}}%
\pgfpathlineto{\pgfqpoint{4.999342in}{5.223159in}}%
\pgfpathlineto{\pgfqpoint{4.999932in}{5.219653in}}%
\pgfpathlineto{\pgfqpoint{5.000269in}{5.222825in}}%
\pgfpathlineto{\pgfqpoint{5.002038in}{5.244246in}}%
\pgfpathlineto{\pgfqpoint{5.002375in}{5.240454in}}%
\pgfpathlineto{\pgfqpoint{5.002965in}{5.232451in}}%
\pgfpathlineto{\pgfqpoint{5.003554in}{5.236993in}}%
\pgfpathlineto{\pgfqpoint{5.003723in}{5.237925in}}%
\pgfpathlineto{\pgfqpoint{5.004144in}{5.233597in}}%
\pgfpathlineto{\pgfqpoint{5.006587in}{5.200846in}}%
\pgfpathlineto{\pgfqpoint{5.006756in}{5.201704in}}%
\pgfpathlineto{\pgfqpoint{5.008609in}{5.233859in}}%
\pgfpathlineto{\pgfqpoint{5.009283in}{5.252750in}}%
\pgfpathlineto{\pgfqpoint{5.009873in}{5.242188in}}%
\pgfpathlineto{\pgfqpoint{5.010210in}{5.237651in}}%
\pgfpathlineto{\pgfqpoint{5.010884in}{5.244117in}}%
\pgfpathlineto{\pgfqpoint{5.011052in}{5.243490in}}%
\pgfpathlineto{\pgfqpoint{5.011810in}{5.221588in}}%
\pgfpathlineto{\pgfqpoint{5.012400in}{5.210440in}}%
\pgfpathlineto{\pgfqpoint{5.013158in}{5.212606in}}%
\pgfpathlineto{\pgfqpoint{5.014169in}{5.194790in}}%
\pgfpathlineto{\pgfqpoint{5.014675in}{5.205790in}}%
\pgfpathlineto{\pgfqpoint{5.016444in}{5.229550in}}%
\pgfpathlineto{\pgfqpoint{5.018550in}{5.256210in}}%
\pgfpathlineto{\pgfqpoint{5.019140in}{5.249599in}}%
\pgfpathlineto{\pgfqpoint{5.022341in}{5.203434in}}%
\pgfpathlineto{\pgfqpoint{5.022846in}{5.199148in}}%
\pgfpathlineto{\pgfqpoint{5.023352in}{5.203726in}}%
\pgfpathlineto{\pgfqpoint{5.025711in}{5.241684in}}%
\pgfpathlineto{\pgfqpoint{5.026132in}{5.236391in}}%
\pgfpathlineto{\pgfqpoint{5.028659in}{5.201069in}}%
\pgfpathlineto{\pgfqpoint{5.027059in}{5.237152in}}%
\pgfpathlineto{\pgfqpoint{5.029081in}{5.204393in}}%
\pgfpathlineto{\pgfqpoint{5.030428in}{5.215495in}}%
\pgfpathlineto{\pgfqpoint{5.031018in}{5.211126in}}%
\pgfpathlineto{\pgfqpoint{5.031187in}{5.210566in}}%
\pgfpathlineto{\pgfqpoint{5.031439in}{5.213040in}}%
\pgfpathlineto{\pgfqpoint{5.033293in}{5.246272in}}%
\pgfpathlineto{\pgfqpoint{5.033798in}{5.245958in}}%
\pgfpathlineto{\pgfqpoint{5.034219in}{5.248396in}}%
\pgfpathlineto{\pgfqpoint{5.034893in}{5.259739in}}%
\pgfpathlineto{\pgfqpoint{5.035315in}{5.252241in}}%
\pgfpathlineto{\pgfqpoint{5.036157in}{5.222370in}}%
\pgfpathlineto{\pgfqpoint{5.036831in}{5.231451in}}%
\pgfpathlineto{\pgfqpoint{5.037336in}{5.222453in}}%
\pgfpathlineto{\pgfqpoint{5.037842in}{5.212420in}}%
\pgfpathlineto{\pgfqpoint{5.038516in}{5.219949in}}%
\pgfpathlineto{\pgfqpoint{5.040117in}{5.228818in}}%
\pgfpathlineto{\pgfqpoint{5.039190in}{5.217978in}}%
\pgfpathlineto{\pgfqpoint{5.040369in}{5.226647in}}%
\pgfpathlineto{\pgfqpoint{5.042223in}{5.211652in}}%
\pgfpathlineto{\pgfqpoint{5.042391in}{5.211503in}}%
\pgfpathlineto{\pgfqpoint{5.042812in}{5.212754in}}%
\pgfpathlineto{\pgfqpoint{5.044329in}{5.221858in}}%
\pgfpathlineto{\pgfqpoint{5.045424in}{5.248101in}}%
\pgfpathlineto{\pgfqpoint{5.046014in}{5.242012in}}%
\pgfpathlineto{\pgfqpoint{5.046688in}{5.243147in}}%
\pgfpathlineto{\pgfqpoint{5.048036in}{5.227840in}}%
\pgfpathlineto{\pgfqpoint{5.050479in}{5.191180in}}%
\pgfpathlineto{\pgfqpoint{5.050984in}{5.198670in}}%
\pgfpathlineto{\pgfqpoint{5.053090in}{5.221280in}}%
\pgfpathlineto{\pgfqpoint{5.053511in}{5.223510in}}%
\pgfpathlineto{\pgfqpoint{5.054607in}{5.240133in}}%
\pgfpathlineto{\pgfqpoint{5.055533in}{5.235172in}}%
\pgfpathlineto{\pgfqpoint{5.055870in}{5.236076in}}%
\pgfpathlineto{\pgfqpoint{5.056460in}{5.241654in}}%
\pgfpathlineto{\pgfqpoint{5.056881in}{5.236324in}}%
\pgfpathlineto{\pgfqpoint{5.057639in}{5.223036in}}%
\pgfpathlineto{\pgfqpoint{5.058229in}{5.230935in}}%
\pgfpathlineto{\pgfqpoint{5.059072in}{5.237945in}}%
\pgfpathlineto{\pgfqpoint{5.059493in}{5.234962in}}%
\pgfpathlineto{\pgfqpoint{5.061430in}{5.225145in}}%
\pgfpathlineto{\pgfqpoint{5.061599in}{5.226190in}}%
\pgfpathlineto{\pgfqpoint{5.062357in}{5.239976in}}%
\pgfpathlineto{\pgfqpoint{5.062778in}{5.230235in}}%
\pgfpathlineto{\pgfqpoint{5.063368in}{5.212613in}}%
\pgfpathlineto{\pgfqpoint{5.063958in}{5.227335in}}%
\pgfpathlineto{\pgfqpoint{5.064379in}{5.236176in}}%
\pgfpathlineto{\pgfqpoint{5.064884in}{5.223045in}}%
\pgfpathlineto{\pgfqpoint{5.065306in}{5.213742in}}%
\pgfpathlineto{\pgfqpoint{5.065895in}{5.225638in}}%
\pgfpathlineto{\pgfqpoint{5.066148in}{5.228459in}}%
\pgfpathlineto{\pgfqpoint{5.066569in}{5.220424in}}%
\pgfpathlineto{\pgfqpoint{5.067243in}{5.205122in}}%
\pgfpathlineto{\pgfqpoint{5.067749in}{5.215026in}}%
\pgfpathlineto{\pgfqpoint{5.068254in}{5.223993in}}%
\pgfpathlineto{\pgfqpoint{5.068760in}{5.214034in}}%
\pgfpathlineto{\pgfqpoint{5.069265in}{5.205004in}}%
\pgfpathlineto{\pgfqpoint{5.069771in}{5.215964in}}%
\pgfpathlineto{\pgfqpoint{5.070192in}{5.223524in}}%
\pgfpathlineto{\pgfqpoint{5.070782in}{5.214982in}}%
\pgfpathlineto{\pgfqpoint{5.071034in}{5.212854in}}%
\pgfpathlineto{\pgfqpoint{5.071456in}{5.218009in}}%
\pgfpathlineto{\pgfqpoint{5.072382in}{5.238287in}}%
\pgfpathlineto{\pgfqpoint{5.072972in}{5.229585in}}%
\pgfpathlineto{\pgfqpoint{5.073477in}{5.224284in}}%
\pgfpathlineto{\pgfqpoint{5.073899in}{5.230513in}}%
\pgfpathlineto{\pgfqpoint{5.074404in}{5.240001in}}%
\pgfpathlineto{\pgfqpoint{5.074910in}{5.229121in}}%
\pgfpathlineto{\pgfqpoint{5.075415in}{5.219095in}}%
\pgfpathlineto{\pgfqpoint{5.076257in}{5.220582in}}%
\pgfpathlineto{\pgfqpoint{5.076847in}{5.215850in}}%
\pgfpathlineto{\pgfqpoint{5.077184in}{5.220019in}}%
\pgfpathlineto{\pgfqpoint{5.078027in}{5.236473in}}%
\pgfpathlineto{\pgfqpoint{5.078869in}{5.234481in}}%
\pgfpathlineto{\pgfqpoint{5.079880in}{5.246489in}}%
\pgfpathlineto{\pgfqpoint{5.080554in}{5.244467in}}%
\pgfpathlineto{\pgfqpoint{5.080975in}{5.242002in}}%
\pgfpathlineto{\pgfqpoint{5.082239in}{5.223681in}}%
\pgfpathlineto{\pgfqpoint{5.083081in}{5.205340in}}%
\pgfpathlineto{\pgfqpoint{5.083502in}{5.213826in}}%
\pgfpathlineto{\pgfqpoint{5.084008in}{5.223248in}}%
\pgfpathlineto{\pgfqpoint{5.084429in}{5.212456in}}%
\pgfpathlineto{\pgfqpoint{5.084935in}{5.201980in}}%
\pgfpathlineto{\pgfqpoint{5.085440in}{5.214407in}}%
\pgfpathlineto{\pgfqpoint{5.085861in}{5.223474in}}%
\pgfpathlineto{\pgfqpoint{5.086451in}{5.213629in}}%
\pgfpathlineto{\pgfqpoint{5.086620in}{5.212093in}}%
\pgfpathlineto{\pgfqpoint{5.087041in}{5.218696in}}%
\pgfpathlineto{\pgfqpoint{5.087799in}{5.233923in}}%
\pgfpathlineto{\pgfqpoint{5.088389in}{5.226017in}}%
\pgfpathlineto{\pgfqpoint{5.088557in}{5.224796in}}%
\pgfpathlineto{\pgfqpoint{5.088978in}{5.230650in}}%
\pgfpathlineto{\pgfqpoint{5.089821in}{5.251131in}}%
\pgfpathlineto{\pgfqpoint{5.090411in}{5.243228in}}%
\pgfpathlineto{\pgfqpoint{5.090916in}{5.237334in}}%
\pgfpathlineto{\pgfqpoint{5.091421in}{5.243615in}}%
\pgfpathlineto{\pgfqpoint{5.091590in}{5.245220in}}%
\pgfpathlineto{\pgfqpoint{5.092011in}{5.238202in}}%
\pgfpathlineto{\pgfqpoint{5.092938in}{5.205525in}}%
\pgfpathlineto{\pgfqpoint{5.093696in}{5.213209in}}%
\pgfpathlineto{\pgfqpoint{5.094033in}{5.208208in}}%
\pgfpathlineto{\pgfqpoint{5.094791in}{5.185935in}}%
\pgfpathlineto{\pgfqpoint{5.095297in}{5.198729in}}%
\pgfpathlineto{\pgfqpoint{5.096982in}{5.224337in}}%
\pgfpathlineto{\pgfqpoint{5.097403in}{5.231496in}}%
\pgfpathlineto{\pgfqpoint{5.097993in}{5.222593in}}%
\pgfpathlineto{\pgfqpoint{5.098330in}{5.218745in}}%
\pgfpathlineto{\pgfqpoint{5.098835in}{5.225400in}}%
\pgfpathlineto{\pgfqpoint{5.101110in}{5.244425in}}%
\pgfpathlineto{\pgfqpoint{5.101278in}{5.244204in}}%
\pgfpathlineto{\pgfqpoint{5.101362in}{5.243703in}}%
\pgfpathlineto{\pgfqpoint{5.101868in}{5.240019in}}%
\pgfpathlineto{\pgfqpoint{5.102205in}{5.244246in}}%
\pgfpathlineto{\pgfqpoint{5.102879in}{5.256282in}}%
\pgfpathlineto{\pgfqpoint{5.103300in}{5.246688in}}%
\pgfpathlineto{\pgfqpoint{5.104395in}{5.210114in}}%
\pgfpathlineto{\pgfqpoint{5.105238in}{5.211727in}}%
\pgfpathlineto{\pgfqpoint{5.105912in}{5.200036in}}%
\pgfpathlineto{\pgfqpoint{5.106333in}{5.210085in}}%
\pgfpathlineto{\pgfqpoint{5.106922in}{5.226931in}}%
\pgfpathlineto{\pgfqpoint{5.107428in}{5.214755in}}%
\pgfpathlineto{\pgfqpoint{5.107765in}{5.206970in}}%
\pgfpathlineto{\pgfqpoint{5.108270in}{5.221743in}}%
\pgfpathlineto{\pgfqpoint{5.109029in}{5.248141in}}%
\pgfpathlineto{\pgfqpoint{5.109618in}{5.233639in}}%
\pgfpathlineto{\pgfqpoint{5.110208in}{5.220470in}}%
\pgfpathlineto{\pgfqpoint{5.110798in}{5.228506in}}%
\pgfpathlineto{\pgfqpoint{5.111050in}{5.229998in}}%
\pgfpathlineto{\pgfqpoint{5.111640in}{5.226191in}}%
\pgfpathlineto{\pgfqpoint{5.111893in}{5.225044in}}%
\pgfpathlineto{\pgfqpoint{5.112398in}{5.228531in}}%
\pgfpathlineto{\pgfqpoint{5.113241in}{5.240451in}}%
\pgfpathlineto{\pgfqpoint{5.113915in}{5.234813in}}%
\pgfpathlineto{\pgfqpoint{5.114083in}{5.234411in}}%
\pgfpathlineto{\pgfqpoint{5.114336in}{5.236553in}}%
\pgfpathlineto{\pgfqpoint{5.115178in}{5.248730in}}%
\pgfpathlineto{\pgfqpoint{5.115684in}{5.243278in}}%
\pgfpathlineto{\pgfqpoint{5.118043in}{5.196940in}}%
\pgfpathlineto{\pgfqpoint{5.118380in}{5.200701in}}%
\pgfpathlineto{\pgfqpoint{5.119054in}{5.213324in}}%
\pgfpathlineto{\pgfqpoint{5.119559in}{5.203282in}}%
\pgfpathlineto{\pgfqpoint{5.119896in}{5.199191in}}%
\pgfpathlineto{\pgfqpoint{5.120317in}{5.208644in}}%
\pgfpathlineto{\pgfqpoint{5.122086in}{5.234665in}}%
\pgfpathlineto{\pgfqpoint{5.122255in}{5.235317in}}%
\pgfpathlineto{\pgfqpoint{5.122760in}{5.232430in}}%
\pgfpathlineto{\pgfqpoint{5.123097in}{5.230411in}}%
\pgfpathlineto{\pgfqpoint{5.123603in}{5.234467in}}%
\pgfpathlineto{\pgfqpoint{5.124024in}{5.238153in}}%
\pgfpathlineto{\pgfqpoint{5.124614in}{5.233148in}}%
\pgfpathlineto{\pgfqpoint{5.125709in}{5.222637in}}%
\pgfpathlineto{\pgfqpoint{5.126299in}{5.228986in}}%
\pgfpathlineto{\pgfqpoint{5.126467in}{5.230062in}}%
\pgfpathlineto{\pgfqpoint{5.126804in}{5.226384in}}%
\pgfpathlineto{\pgfqpoint{5.127647in}{5.202415in}}%
\pgfpathlineto{\pgfqpoint{5.128236in}{5.216680in}}%
\pgfpathlineto{\pgfqpoint{5.128573in}{5.222416in}}%
\pgfpathlineto{\pgfqpoint{5.129163in}{5.214628in}}%
\pgfpathlineto{\pgfqpoint{5.129247in}{5.214794in}}%
\pgfpathlineto{\pgfqpoint{5.129837in}{5.231985in}}%
\pgfpathlineto{\pgfqpoint{5.130258in}{5.241097in}}%
\pgfpathlineto{\pgfqpoint{5.130932in}{5.231537in}}%
\pgfpathlineto{\pgfqpoint{5.131185in}{5.229641in}}%
\pgfpathlineto{\pgfqpoint{5.131690in}{5.235195in}}%
\pgfpathlineto{\pgfqpoint{5.131943in}{5.236938in}}%
\pgfpathlineto{\pgfqpoint{5.132364in}{5.230494in}}%
\pgfpathlineto{\pgfqpoint{5.134302in}{5.196145in}}%
\pgfpathlineto{\pgfqpoint{5.134639in}{5.201688in}}%
\pgfpathlineto{\pgfqpoint{5.137082in}{5.240859in}}%
\pgfpathlineto{\pgfqpoint{5.137672in}{5.247235in}}%
\pgfpathlineto{\pgfqpoint{5.138093in}{5.253212in}}%
\pgfpathlineto{\pgfqpoint{5.138598in}{5.245796in}}%
\pgfpathlineto{\pgfqpoint{5.139104in}{5.239852in}}%
\pgfpathlineto{\pgfqpoint{5.139694in}{5.245140in}}%
\pgfpathlineto{\pgfqpoint{5.139862in}{5.245698in}}%
\pgfpathlineto{\pgfqpoint{5.140368in}{5.243330in}}%
\pgfpathlineto{\pgfqpoint{5.142474in}{5.222341in}}%
\pgfpathlineto{\pgfqpoint{5.143232in}{5.229680in}}%
\pgfpathlineto{\pgfqpoint{5.143400in}{5.230323in}}%
\pgfpathlineto{\pgfqpoint{5.143737in}{5.226654in}}%
\pgfpathlineto{\pgfqpoint{5.145001in}{5.203387in}}%
\pgfpathlineto{\pgfqpoint{5.145675in}{5.210008in}}%
\pgfpathlineto{\pgfqpoint{5.145759in}{5.210396in}}%
\pgfpathlineto{\pgfqpoint{5.146096in}{5.206985in}}%
\pgfpathlineto{\pgfqpoint{5.146854in}{5.187708in}}%
\pgfpathlineto{\pgfqpoint{5.147360in}{5.199339in}}%
\pgfpathlineto{\pgfqpoint{5.149213in}{5.233125in}}%
\pgfpathlineto{\pgfqpoint{5.149803in}{5.244363in}}%
\pgfpathlineto{\pgfqpoint{5.150477in}{5.238753in}}%
\pgfpathlineto{\pgfqpoint{5.151909in}{5.226091in}}%
\pgfpathlineto{\pgfqpoint{5.152499in}{5.218062in}}%
\pgfpathlineto{\pgfqpoint{5.153088in}{5.224414in}}%
\pgfpathlineto{\pgfqpoint{5.153257in}{5.225149in}}%
\pgfpathlineto{\pgfqpoint{5.153594in}{5.221886in}}%
\pgfpathlineto{\pgfqpoint{5.154436in}{5.212658in}}%
\pgfpathlineto{\pgfqpoint{5.154858in}{5.216979in}}%
\pgfpathlineto{\pgfqpoint{5.155784in}{5.237127in}}%
\pgfpathlineto{\pgfqpoint{5.156543in}{5.229876in}}%
\pgfpathlineto{\pgfqpoint{5.156795in}{5.229147in}}%
\pgfpathlineto{\pgfqpoint{5.157216in}{5.231331in}}%
\pgfpathlineto{\pgfqpoint{5.157975in}{5.238522in}}%
\pgfpathlineto{\pgfqpoint{5.158480in}{5.234115in}}%
\pgfpathlineto{\pgfqpoint{5.159828in}{5.212568in}}%
\pgfpathlineto{\pgfqpoint{5.160418in}{5.218690in}}%
\pgfpathlineto{\pgfqpoint{5.161007in}{5.224624in}}%
\pgfpathlineto{\pgfqpoint{5.161429in}{5.219657in}}%
\pgfpathlineto{\pgfqpoint{5.161934in}{5.212941in}}%
\pgfpathlineto{\pgfqpoint{5.162440in}{5.218502in}}%
\pgfpathlineto{\pgfqpoint{5.163366in}{5.234506in}}%
\pgfpathlineto{\pgfqpoint{5.164125in}{5.230149in}}%
\pgfpathlineto{\pgfqpoint{5.164630in}{5.226837in}}%
\pgfpathlineto{\pgfqpoint{5.165472in}{5.214932in}}%
\pgfpathlineto{\pgfqpoint{5.166231in}{5.218990in}}%
\pgfpathlineto{\pgfqpoint{5.166652in}{5.216485in}}%
\pgfpathlineto{\pgfqpoint{5.166989in}{5.214801in}}%
\pgfpathlineto{\pgfqpoint{5.167410in}{5.219263in}}%
\pgfpathlineto{\pgfqpoint{5.169348in}{5.251455in}}%
\pgfpathlineto{\pgfqpoint{5.169853in}{5.244282in}}%
\pgfpathlineto{\pgfqpoint{5.170443in}{5.235986in}}%
\pgfpathlineto{\pgfqpoint{5.171117in}{5.240798in}}%
\pgfpathlineto{\pgfqpoint{5.171201in}{5.240908in}}%
\pgfpathlineto{\pgfqpoint{5.171370in}{5.240117in}}%
\pgfpathlineto{\pgfqpoint{5.173644in}{5.218690in}}%
\pgfpathlineto{\pgfqpoint{5.174318in}{5.219882in}}%
\pgfpathlineto{\pgfqpoint{5.174992in}{5.215137in}}%
\pgfpathlineto{\pgfqpoint{5.175666in}{5.217740in}}%
\pgfpathlineto{\pgfqpoint{5.175835in}{5.217635in}}%
\pgfpathlineto{\pgfqpoint{5.175919in}{5.217262in}}%
\pgfpathlineto{\pgfqpoint{5.176593in}{5.208936in}}%
\pgfpathlineto{\pgfqpoint{5.177014in}{5.215442in}}%
\pgfpathlineto{\pgfqpoint{5.177604in}{5.224607in}}%
\pgfpathlineto{\pgfqpoint{5.178278in}{5.219679in}}%
\pgfpathlineto{\pgfqpoint{5.178530in}{5.221222in}}%
\pgfpathlineto{\pgfqpoint{5.179204in}{5.230910in}}%
\pgfpathlineto{\pgfqpoint{5.179794in}{5.223549in}}%
\pgfpathlineto{\pgfqpoint{5.180047in}{5.222365in}}%
\pgfpathlineto{\pgfqpoint{5.180636in}{5.225101in}}%
\pgfpathlineto{\pgfqpoint{5.183922in}{5.239547in}}%
\pgfpathlineto{\pgfqpoint{5.184175in}{5.241591in}}%
\pgfpathlineto{\pgfqpoint{5.184596in}{5.235724in}}%
\pgfpathlineto{\pgfqpoint{5.186449in}{5.214882in}}%
\pgfpathlineto{\pgfqpoint{5.186955in}{5.213914in}}%
\pgfpathlineto{\pgfqpoint{5.187797in}{5.205328in}}%
\pgfpathlineto{\pgfqpoint{5.188218in}{5.210115in}}%
\pgfpathlineto{\pgfqpoint{5.190409in}{5.232612in}}%
\pgfpathlineto{\pgfqpoint{5.190746in}{5.235083in}}%
\pgfpathlineto{\pgfqpoint{5.191335in}{5.231244in}}%
\pgfpathlineto{\pgfqpoint{5.192936in}{5.221205in}}%
\pgfpathlineto{\pgfqpoint{5.193357in}{5.222793in}}%
\pgfpathlineto{\pgfqpoint{5.193694in}{5.223969in}}%
\pgfpathlineto{\pgfqpoint{5.194031in}{5.221253in}}%
\pgfpathlineto{\pgfqpoint{5.194453in}{5.218742in}}%
\pgfpathlineto{\pgfqpoint{5.194958in}{5.222533in}}%
\pgfpathlineto{\pgfqpoint{5.196222in}{5.232589in}}%
\pgfpathlineto{\pgfqpoint{5.196643in}{5.229070in}}%
\pgfpathlineto{\pgfqpoint{5.197148in}{5.224199in}}%
\pgfpathlineto{\pgfqpoint{5.197654in}{5.227816in}}%
\pgfpathlineto{\pgfqpoint{5.198496in}{5.243548in}}%
\pgfpathlineto{\pgfqpoint{5.199002in}{5.235481in}}%
\pgfpathlineto{\pgfqpoint{5.200602in}{5.223842in}}%
\pgfpathlineto{\pgfqpoint{5.200855in}{5.222162in}}%
\pgfpathlineto{\pgfqpoint{5.201445in}{5.225925in}}%
\pgfpathlineto{\pgfqpoint{5.201782in}{5.226496in}}%
\pgfpathlineto{\pgfqpoint{5.202119in}{5.224603in}}%
\pgfpathlineto{\pgfqpoint{5.202877in}{5.209826in}}%
\pgfpathlineto{\pgfqpoint{5.203382in}{5.221893in}}%
\pgfpathlineto{\pgfqpoint{5.203888in}{5.231680in}}%
\pgfpathlineto{\pgfqpoint{5.204393in}{5.220738in}}%
\pgfpathlineto{\pgfqpoint{5.204899in}{5.211811in}}%
\pgfpathlineto{\pgfqpoint{5.205489in}{5.219747in}}%
\pgfpathlineto{\pgfqpoint{5.205657in}{5.220684in}}%
\pgfpathlineto{\pgfqpoint{5.206078in}{5.216460in}}%
\pgfpathlineto{\pgfqpoint{5.206331in}{5.214187in}}%
\pgfpathlineto{\pgfqpoint{5.206836in}{5.219706in}}%
\pgfpathlineto{\pgfqpoint{5.208690in}{5.232371in}}%
\pgfpathlineto{\pgfqpoint{5.209027in}{5.230865in}}%
\pgfpathlineto{\pgfqpoint{5.209280in}{5.229986in}}%
\pgfpathlineto{\pgfqpoint{5.209617in}{5.232779in}}%
\pgfpathlineto{\pgfqpoint{5.209954in}{5.235826in}}%
\pgfpathlineto{\pgfqpoint{5.210459in}{5.229806in}}%
\pgfpathlineto{\pgfqpoint{5.210796in}{5.226762in}}%
\pgfpathlineto{\pgfqpoint{5.211217in}{5.233824in}}%
\pgfpathlineto{\pgfqpoint{5.211723in}{5.240743in}}%
\pgfpathlineto{\pgfqpoint{5.212312in}{5.233810in}}%
\pgfpathlineto{\pgfqpoint{5.212986in}{5.223885in}}%
\pgfpathlineto{\pgfqpoint{5.213492in}{5.230484in}}%
\pgfpathlineto{\pgfqpoint{5.213829in}{5.234343in}}%
\pgfpathlineto{\pgfqpoint{5.214334in}{5.228122in}}%
\pgfpathlineto{\pgfqpoint{5.215177in}{5.208995in}}%
\pgfpathlineto{\pgfqpoint{5.215851in}{5.216178in}}%
\pgfpathlineto{\pgfqpoint{5.215935in}{5.216301in}}%
\pgfpathlineto{\pgfqpoint{5.216188in}{5.214792in}}%
\pgfpathlineto{\pgfqpoint{5.217199in}{5.202174in}}%
\pgfpathlineto{\pgfqpoint{5.217704in}{5.208448in}}%
\pgfpathlineto{\pgfqpoint{5.220484in}{5.253247in}}%
\pgfpathlineto{\pgfqpoint{5.220568in}{5.252587in}}%
\pgfpathlineto{\pgfqpoint{5.223096in}{5.206651in}}%
\pgfpathlineto{\pgfqpoint{5.223854in}{5.218206in}}%
\pgfpathlineto{\pgfqpoint{5.224275in}{5.222041in}}%
\pgfpathlineto{\pgfqpoint{5.224865in}{5.217169in}}%
\pgfpathlineto{\pgfqpoint{5.225286in}{5.214691in}}%
\pgfpathlineto{\pgfqpoint{5.225707in}{5.219039in}}%
\pgfpathlineto{\pgfqpoint{5.226887in}{5.240598in}}%
\pgfpathlineto{\pgfqpoint{5.227392in}{5.233160in}}%
\pgfpathlineto{\pgfqpoint{5.227982in}{5.225389in}}%
\pgfpathlineto{\pgfqpoint{5.228656in}{5.228777in}}%
\pgfpathlineto{\pgfqpoint{5.229330in}{5.224827in}}%
\pgfpathlineto{\pgfqpoint{5.230593in}{5.218683in}}%
\pgfpathlineto{\pgfqpoint{5.230846in}{5.220765in}}%
\pgfpathlineto{\pgfqpoint{5.231689in}{5.237326in}}%
\pgfpathlineto{\pgfqpoint{5.232363in}{5.229258in}}%
\pgfpathlineto{\pgfqpoint{5.232531in}{5.228626in}}%
\pgfpathlineto{\pgfqpoint{5.232952in}{5.232301in}}%
\pgfpathlineto{\pgfqpoint{5.233205in}{5.234159in}}%
\pgfpathlineto{\pgfqpoint{5.233626in}{5.228765in}}%
\pgfpathlineto{\pgfqpoint{5.234132in}{5.221078in}}%
\pgfpathlineto{\pgfqpoint{5.234637in}{5.228850in}}%
\pgfpathlineto{\pgfqpoint{5.235058in}{5.234772in}}%
\pgfpathlineto{\pgfqpoint{5.235480in}{5.226738in}}%
\pgfpathlineto{\pgfqpoint{5.236069in}{5.212508in}}%
\pgfpathlineto{\pgfqpoint{5.236575in}{5.224291in}}%
\pgfpathlineto{\pgfqpoint{5.236996in}{5.231017in}}%
\pgfpathlineto{\pgfqpoint{5.237586in}{5.223130in}}%
\pgfpathlineto{\pgfqpoint{5.237838in}{5.221242in}}%
\pgfpathlineto{\pgfqpoint{5.238344in}{5.225455in}}%
\pgfpathlineto{\pgfqpoint{5.239355in}{5.231028in}}%
\pgfpathlineto{\pgfqpoint{5.239692in}{5.228887in}}%
\pgfpathlineto{\pgfqpoint{5.240619in}{5.211475in}}%
\pgfpathlineto{\pgfqpoint{5.241124in}{5.223184in}}%
\pgfpathlineto{\pgfqpoint{5.241545in}{5.232786in}}%
\pgfpathlineto{\pgfqpoint{5.242135in}{5.220293in}}%
\pgfpathlineto{\pgfqpoint{5.242556in}{5.213874in}}%
\pgfpathlineto{\pgfqpoint{5.243062in}{5.223289in}}%
\pgfpathlineto{\pgfqpoint{5.243567in}{5.229135in}}%
\pgfpathlineto{\pgfqpoint{5.244157in}{5.223610in}}%
\pgfpathlineto{\pgfqpoint{5.245420in}{5.215778in}}%
\pgfpathlineto{\pgfqpoint{5.245757in}{5.218322in}}%
\pgfpathlineto{\pgfqpoint{5.246431in}{5.228378in}}%
\pgfpathlineto{\pgfqpoint{5.247021in}{5.222854in}}%
\pgfpathlineto{\pgfqpoint{5.247105in}{5.222573in}}%
\pgfpathlineto{\pgfqpoint{5.247442in}{5.224388in}}%
\pgfpathlineto{\pgfqpoint{5.248453in}{5.234845in}}%
\pgfpathlineto{\pgfqpoint{5.249043in}{5.230392in}}%
\pgfpathlineto{\pgfqpoint{5.249211in}{5.229638in}}%
\pgfpathlineto{\pgfqpoint{5.249717in}{5.232374in}}%
\pgfpathlineto{\pgfqpoint{5.250222in}{5.235208in}}%
\pgfpathlineto{\pgfqpoint{5.250644in}{5.232381in}}%
\pgfpathlineto{\pgfqpoint{5.251739in}{5.217638in}}%
\pgfpathlineto{\pgfqpoint{5.252244in}{5.225959in}}%
\pgfpathlineto{\pgfqpoint{5.252666in}{5.232078in}}%
\pgfpathlineto{\pgfqpoint{5.253087in}{5.222519in}}%
\pgfpathlineto{\pgfqpoint{5.253592in}{5.208324in}}%
\pgfpathlineto{\pgfqpoint{5.254182in}{5.223001in}}%
\pgfpathlineto{\pgfqpoint{5.254687in}{5.233634in}}%
\pgfpathlineto{\pgfqpoint{5.255277in}{5.223774in}}%
\pgfpathlineto{\pgfqpoint{5.255614in}{5.220299in}}%
\pgfpathlineto{\pgfqpoint{5.256204in}{5.225512in}}%
\pgfpathlineto{\pgfqpoint{5.257720in}{5.230831in}}%
\pgfpathlineto{\pgfqpoint{5.258394in}{5.233946in}}%
\pgfpathlineto{\pgfqpoint{5.258731in}{5.231519in}}%
\pgfpathlineto{\pgfqpoint{5.259405in}{5.224255in}}%
\pgfpathlineto{\pgfqpoint{5.259995in}{5.227789in}}%
\pgfpathlineto{\pgfqpoint{5.260500in}{5.231839in}}%
\pgfpathlineto{\pgfqpoint{5.261006in}{5.228140in}}%
\pgfpathlineto{\pgfqpoint{5.262017in}{5.211371in}}%
\pgfpathlineto{\pgfqpoint{5.262606in}{5.218471in}}%
\pgfpathlineto{\pgfqpoint{5.263533in}{5.222574in}}%
\pgfpathlineto{\pgfqpoint{5.263870in}{5.221454in}}%
\pgfpathlineto{\pgfqpoint{5.264712in}{5.214442in}}%
\pgfpathlineto{\pgfqpoint{5.265218in}{5.218481in}}%
\pgfpathlineto{\pgfqpoint{5.265976in}{5.229398in}}%
\pgfpathlineto{\pgfqpoint{5.266482in}{5.222713in}}%
\pgfpathlineto{\pgfqpoint{5.267071in}{5.214428in}}%
\pgfpathlineto{\pgfqpoint{5.267577in}{5.221209in}}%
\pgfpathlineto{\pgfqpoint{5.268082in}{5.228303in}}%
\pgfpathlineto{\pgfqpoint{5.268756in}{5.222884in}}%
\pgfpathlineto{\pgfqpoint{5.270104in}{5.212137in}}%
\pgfpathlineto{\pgfqpoint{5.270441in}{5.216329in}}%
\pgfpathlineto{\pgfqpoint{5.271368in}{5.241093in}}%
\pgfpathlineto{\pgfqpoint{5.272042in}{5.231915in}}%
\pgfpathlineto{\pgfqpoint{5.272210in}{5.230684in}}%
\pgfpathlineto{\pgfqpoint{5.272631in}{5.236510in}}%
\pgfpathlineto{\pgfqpoint{5.273811in}{5.256359in}}%
\pgfpathlineto{\pgfqpoint{5.274232in}{5.250961in}}%
\pgfpathlineto{\pgfqpoint{5.275159in}{5.222851in}}%
\pgfpathlineto{\pgfqpoint{5.276001in}{5.230201in}}%
\pgfpathlineto{\pgfqpoint{5.277012in}{5.199235in}}%
\pgfpathlineto{\pgfqpoint{5.277855in}{5.208888in}}%
\pgfpathlineto{\pgfqpoint{5.280719in}{5.239802in}}%
\pgfpathlineto{\pgfqpoint{5.281393in}{5.245919in}}%
\pgfpathlineto{\pgfqpoint{5.281983in}{5.243673in}}%
\pgfpathlineto{\pgfqpoint{5.283331in}{5.217195in}}%
\pgfpathlineto{\pgfqpoint{5.284257in}{5.227979in}}%
\pgfpathlineto{\pgfqpoint{5.284678in}{5.230398in}}%
\pgfpathlineto{\pgfqpoint{5.285100in}{5.226664in}}%
\pgfpathlineto{\pgfqpoint{5.285774in}{5.218821in}}%
\pgfpathlineto{\pgfqpoint{5.286195in}{5.223639in}}%
\pgfpathlineto{\pgfqpoint{5.287037in}{5.240907in}}%
\pgfpathlineto{\pgfqpoint{5.287543in}{5.230358in}}%
\pgfpathlineto{\pgfqpoint{5.288385in}{5.210683in}}%
\pgfpathlineto{\pgfqpoint{5.289059in}{5.212591in}}%
\pgfpathlineto{\pgfqpoint{5.289986in}{5.204283in}}%
\pgfpathlineto{\pgfqpoint{5.290407in}{5.209903in}}%
\pgfpathlineto{\pgfqpoint{5.290744in}{5.214205in}}%
\pgfpathlineto{\pgfqpoint{5.291250in}{5.205785in}}%
\pgfpathlineto{\pgfqpoint{5.291586in}{5.201494in}}%
\pgfpathlineto{\pgfqpoint{5.292008in}{5.211345in}}%
\pgfpathlineto{\pgfqpoint{5.293187in}{5.251391in}}%
\pgfpathlineto{\pgfqpoint{5.293777in}{5.245114in}}%
\pgfpathlineto{\pgfqpoint{5.294535in}{5.228397in}}%
\pgfpathlineto{\pgfqpoint{5.295041in}{5.239712in}}%
\pgfpathlineto{\pgfqpoint{5.295377in}{5.247345in}}%
\pgfpathlineto{\pgfqpoint{5.295883in}{5.233710in}}%
\pgfpathlineto{\pgfqpoint{5.296641in}{5.207177in}}%
\pgfpathlineto{\pgfqpoint{5.297231in}{5.220421in}}%
\pgfpathlineto{\pgfqpoint{5.298579in}{5.231843in}}%
\pgfpathlineto{\pgfqpoint{5.298747in}{5.231615in}}%
\pgfpathlineto{\pgfqpoint{5.299505in}{5.229252in}}%
\pgfpathlineto{\pgfqpoint{5.299842in}{5.231361in}}%
\pgfpathlineto{\pgfqpoint{5.300348in}{5.234503in}}%
\pgfpathlineto{\pgfqpoint{5.300685in}{5.229934in}}%
\pgfpathlineto{\pgfqpoint{5.301443in}{5.214368in}}%
\pgfpathlineto{\pgfqpoint{5.302117in}{5.219866in}}%
\pgfpathlineto{\pgfqpoint{5.302454in}{5.215512in}}%
\pgfpathlineto{\pgfqpoint{5.303212in}{5.197979in}}%
\pgfpathlineto{\pgfqpoint{5.303718in}{5.206909in}}%
\pgfpathlineto{\pgfqpoint{5.305992in}{5.242392in}}%
\pgfpathlineto{\pgfqpoint{5.306414in}{5.244414in}}%
\pgfpathlineto{\pgfqpoint{5.306835in}{5.241340in}}%
\pgfpathlineto{\pgfqpoint{5.307761in}{5.224624in}}%
\pgfpathlineto{\pgfqpoint{5.308520in}{5.231402in}}%
\pgfpathlineto{\pgfqpoint{5.309109in}{5.222615in}}%
\pgfpathlineto{\pgfqpoint{5.309699in}{5.212797in}}%
\pgfpathlineto{\pgfqpoint{5.310373in}{5.218557in}}%
\pgfpathlineto{\pgfqpoint{5.310710in}{5.220945in}}%
\pgfpathlineto{\pgfqpoint{5.311384in}{5.217688in}}%
\pgfpathlineto{\pgfqpoint{5.311805in}{5.223627in}}%
\pgfpathlineto{\pgfqpoint{5.312395in}{5.233021in}}%
\pgfpathlineto{\pgfqpoint{5.313069in}{5.229059in}}%
\pgfpathlineto{\pgfqpoint{5.314417in}{5.226051in}}%
\pgfpathlineto{\pgfqpoint{5.314669in}{5.227522in}}%
\pgfpathlineto{\pgfqpoint{5.315259in}{5.232765in}}%
\pgfpathlineto{\pgfqpoint{5.315765in}{5.228922in}}%
\pgfpathlineto{\pgfqpoint{5.316270in}{5.225284in}}%
\pgfpathlineto{\pgfqpoint{5.316607in}{5.229530in}}%
\pgfpathlineto{\pgfqpoint{5.317365in}{5.244816in}}%
\pgfpathlineto{\pgfqpoint{5.317955in}{5.237939in}}%
\pgfpathlineto{\pgfqpoint{5.320735in}{5.214450in}}%
\pgfpathlineto{\pgfqpoint{5.321241in}{5.215042in}}%
\pgfpathlineto{\pgfqpoint{5.321493in}{5.213560in}}%
\pgfpathlineto{\pgfqpoint{5.322673in}{5.197448in}}%
\pgfpathlineto{\pgfqpoint{5.323347in}{5.204672in}}%
\pgfpathlineto{\pgfqpoint{5.323515in}{5.205108in}}%
\pgfpathlineto{\pgfqpoint{5.323852in}{5.202553in}}%
\pgfpathlineto{\pgfqpoint{5.324021in}{5.201630in}}%
\pgfpathlineto{\pgfqpoint{5.324358in}{5.205344in}}%
\pgfpathlineto{\pgfqpoint{5.325453in}{5.252525in}}%
\pgfpathlineto{\pgfqpoint{5.326295in}{5.231308in}}%
\pgfpathlineto{\pgfqpoint{5.326548in}{5.234847in}}%
\pgfpathlineto{\pgfqpoint{5.327475in}{5.257732in}}%
\pgfpathlineto{\pgfqpoint{5.328064in}{5.248653in}}%
\pgfpathlineto{\pgfqpoint{5.329244in}{5.217542in}}%
\pgfpathlineto{\pgfqpoint{5.329834in}{5.229194in}}%
\pgfpathlineto{\pgfqpoint{5.329918in}{5.229991in}}%
\pgfpathlineto{\pgfqpoint{5.330170in}{5.225953in}}%
\pgfpathlineto{\pgfqpoint{5.331097in}{5.184746in}}%
\pgfpathlineto{\pgfqpoint{5.331771in}{5.205009in}}%
\pgfpathlineto{\pgfqpoint{5.332108in}{5.209309in}}%
\pgfpathlineto{\pgfqpoint{5.332782in}{5.202755in}}%
\pgfpathlineto{\pgfqpoint{5.333203in}{5.210135in}}%
\pgfpathlineto{\pgfqpoint{5.335646in}{5.253278in}}%
\pgfpathlineto{\pgfqpoint{5.335815in}{5.254452in}}%
\pgfpathlineto{\pgfqpoint{5.336236in}{5.248400in}}%
\pgfpathlineto{\pgfqpoint{5.337500in}{5.222489in}}%
\pgfpathlineto{\pgfqpoint{5.338089in}{5.229080in}}%
\pgfpathlineto{\pgfqpoint{5.338258in}{5.229702in}}%
\pgfpathlineto{\pgfqpoint{5.338511in}{5.226058in}}%
\pgfpathlineto{\pgfqpoint{5.339353in}{5.210964in}}%
\pgfpathlineto{\pgfqpoint{5.339774in}{5.219125in}}%
\pgfpathlineto{\pgfqpoint{5.340364in}{5.234095in}}%
\pgfpathlineto{\pgfqpoint{5.341122in}{5.227460in}}%
\pgfpathlineto{\pgfqpoint{5.341965in}{5.239800in}}%
\pgfpathlineto{\pgfqpoint{5.342639in}{5.230854in}}%
\pgfpathlineto{\pgfqpoint{5.344492in}{5.216053in}}%
\pgfpathlineto{\pgfqpoint{5.344576in}{5.216064in}}%
\pgfpathlineto{\pgfqpoint{5.344998in}{5.219774in}}%
\pgfpathlineto{\pgfqpoint{5.345503in}{5.224523in}}%
\pgfpathlineto{\pgfqpoint{5.346008in}{5.219464in}}%
\pgfpathlineto{\pgfqpoint{5.346177in}{5.218539in}}%
\pgfpathlineto{\pgfqpoint{5.346514in}{5.221945in}}%
\pgfpathlineto{\pgfqpoint{5.347862in}{5.236697in}}%
\pgfpathlineto{\pgfqpoint{5.348283in}{5.235226in}}%
\pgfpathlineto{\pgfqpoint{5.348957in}{5.225467in}}%
\pgfpathlineto{\pgfqpoint{5.350810in}{5.202053in}}%
\pgfpathlineto{\pgfqpoint{5.350979in}{5.203111in}}%
\pgfpathlineto{\pgfqpoint{5.353422in}{5.246256in}}%
\pgfpathlineto{\pgfqpoint{5.354601in}{5.236885in}}%
\pgfpathlineto{\pgfqpoint{5.354938in}{5.238846in}}%
\pgfpathlineto{\pgfqpoint{5.355360in}{5.233413in}}%
\pgfpathlineto{\pgfqpoint{5.357297in}{5.209228in}}%
\pgfpathlineto{\pgfqpoint{5.357466in}{5.211058in}}%
\pgfpathlineto{\pgfqpoint{5.358055in}{5.220401in}}%
\pgfpathlineto{\pgfqpoint{5.358645in}{5.212403in}}%
\pgfpathlineto{\pgfqpoint{5.358898in}{5.210993in}}%
\pgfpathlineto{\pgfqpoint{5.359235in}{5.215541in}}%
\pgfpathlineto{\pgfqpoint{5.359825in}{5.224542in}}%
\pgfpathlineto{\pgfqpoint{5.360330in}{5.216919in}}%
\pgfpathlineto{\pgfqpoint{5.360583in}{5.213572in}}%
\pgfpathlineto{\pgfqpoint{5.361004in}{5.220989in}}%
\pgfpathlineto{\pgfqpoint{5.361678in}{5.240669in}}%
\pgfpathlineto{\pgfqpoint{5.362268in}{5.231648in}}%
\pgfpathlineto{\pgfqpoint{5.362773in}{5.224527in}}%
\pgfpathlineto{\pgfqpoint{5.363363in}{5.231469in}}%
\pgfpathlineto{\pgfqpoint{5.364121in}{5.246458in}}%
\pgfpathlineto{\pgfqpoint{5.364626in}{5.236868in}}%
\pgfpathlineto{\pgfqpoint{5.365216in}{5.225750in}}%
\pgfpathlineto{\pgfqpoint{5.365722in}{5.233751in}}%
\pgfpathlineto{\pgfqpoint{5.366311in}{5.238777in}}%
\pgfpathlineto{\pgfqpoint{5.366733in}{5.233759in}}%
\pgfpathlineto{\pgfqpoint{5.368502in}{5.217254in}}%
\pgfpathlineto{\pgfqpoint{5.368586in}{5.217290in}}%
\pgfpathlineto{\pgfqpoint{5.368839in}{5.217441in}}%
\pgfpathlineto{\pgfqpoint{5.369091in}{5.216483in}}%
\pgfpathlineto{\pgfqpoint{5.370018in}{5.205526in}}%
\pgfpathlineto{\pgfqpoint{5.370524in}{5.213063in}}%
\pgfpathlineto{\pgfqpoint{5.372293in}{5.231477in}}%
\pgfpathlineto{\pgfqpoint{5.373893in}{5.237530in}}%
\pgfpathlineto{\pgfqpoint{5.373978in}{5.237550in}}%
\pgfpathlineto{\pgfqpoint{5.374062in}{5.237163in}}%
\pgfpathlineto{\pgfqpoint{5.374736in}{5.221071in}}%
\pgfpathlineto{\pgfqpoint{5.375494in}{5.208494in}}%
\pgfpathlineto{\pgfqpoint{5.376084in}{5.213246in}}%
\pgfpathlineto{\pgfqpoint{5.376673in}{5.219252in}}%
\pgfpathlineto{\pgfqpoint{5.377095in}{5.214139in}}%
\pgfpathlineto{\pgfqpoint{5.377937in}{5.193610in}}%
\pgfpathlineto{\pgfqpoint{5.378358in}{5.204798in}}%
\pgfpathlineto{\pgfqpoint{5.379959in}{5.249246in}}%
\pgfpathlineto{\pgfqpoint{5.380380in}{5.246078in}}%
\pgfpathlineto{\pgfqpoint{5.380886in}{5.240940in}}%
\pgfpathlineto{\pgfqpoint{5.381391in}{5.245841in}}%
\pgfpathlineto{\pgfqpoint{5.381897in}{5.250786in}}%
\pgfpathlineto{\pgfqpoint{5.382402in}{5.246554in}}%
\pgfpathlineto{\pgfqpoint{5.383413in}{5.219386in}}%
\pgfpathlineto{\pgfqpoint{5.384929in}{5.202363in}}%
\pgfpathlineto{\pgfqpoint{5.385098in}{5.203330in}}%
\pgfpathlineto{\pgfqpoint{5.386867in}{5.225278in}}%
\pgfpathlineto{\pgfqpoint{5.387794in}{5.237764in}}%
\pgfpathlineto{\pgfqpoint{5.388383in}{5.235512in}}%
\pgfpathlineto{\pgfqpoint{5.388552in}{5.235107in}}%
\pgfpathlineto{\pgfqpoint{5.388889in}{5.237637in}}%
\pgfpathlineto{\pgfqpoint{5.389563in}{5.244518in}}%
\pgfpathlineto{\pgfqpoint{5.390153in}{5.241465in}}%
\pgfpathlineto{\pgfqpoint{5.391416in}{5.216033in}}%
\pgfpathlineto{\pgfqpoint{5.392174in}{5.228131in}}%
\pgfpathlineto{\pgfqpoint{5.392343in}{5.226932in}}%
\pgfpathlineto{\pgfqpoint{5.393185in}{5.205241in}}%
\pgfpathlineto{\pgfqpoint{5.393691in}{5.217621in}}%
\pgfpathlineto{\pgfqpoint{5.394028in}{5.223938in}}%
\pgfpathlineto{\pgfqpoint{5.394702in}{5.215660in}}%
\pgfpathlineto{\pgfqpoint{5.395292in}{5.227429in}}%
\pgfpathlineto{\pgfqpoint{5.395460in}{5.228742in}}%
\pgfpathlineto{\pgfqpoint{5.395881in}{5.224211in}}%
\pgfpathlineto{\pgfqpoint{5.396302in}{5.218197in}}%
\pgfpathlineto{\pgfqpoint{5.396976in}{5.223058in}}%
\pgfpathlineto{\pgfqpoint{5.397482in}{5.222177in}}%
\pgfpathlineto{\pgfqpoint{5.398156in}{5.230649in}}%
\pgfpathlineto{\pgfqpoint{5.398493in}{5.236668in}}%
\pgfpathlineto{\pgfqpoint{5.398998in}{5.224072in}}%
\pgfpathlineto{\pgfqpoint{5.399419in}{5.210593in}}%
\pgfpathlineto{\pgfqpoint{5.399925in}{5.225464in}}%
\pgfpathlineto{\pgfqpoint{5.400346in}{5.236278in}}%
\pgfpathlineto{\pgfqpoint{5.400936in}{5.224711in}}%
\pgfpathlineto{\pgfqpoint{5.401778in}{5.226555in}}%
\pgfpathlineto{\pgfqpoint{5.402705in}{5.213234in}}%
\pgfpathlineto{\pgfqpoint{5.402789in}{5.213121in}}%
\pgfpathlineto{\pgfqpoint{5.402958in}{5.214034in}}%
\pgfpathlineto{\pgfqpoint{5.403379in}{5.216921in}}%
\pgfpathlineto{\pgfqpoint{5.403716in}{5.212907in}}%
\pgfpathlineto{\pgfqpoint{5.404221in}{5.202319in}}%
\pgfpathlineto{\pgfqpoint{5.404643in}{5.213003in}}%
\pgfpathlineto{\pgfqpoint{5.405232in}{5.232791in}}%
\pgfpathlineto{\pgfqpoint{5.405906in}{5.223109in}}%
\pgfpathlineto{\pgfqpoint{5.406328in}{5.233163in}}%
\pgfpathlineto{\pgfqpoint{5.407086in}{5.250374in}}%
\pgfpathlineto{\pgfqpoint{5.407675in}{5.244304in}}%
\pgfpathlineto{\pgfqpoint{5.408771in}{5.226593in}}%
\pgfpathlineto{\pgfqpoint{5.409276in}{5.233091in}}%
\pgfpathlineto{\pgfqpoint{5.409697in}{5.240210in}}%
\pgfpathlineto{\pgfqpoint{5.410119in}{5.229521in}}%
\pgfpathlineto{\pgfqpoint{5.410792in}{5.204292in}}%
\pgfpathlineto{\pgfqpoint{5.411466in}{5.214283in}}%
\pgfpathlineto{\pgfqpoint{5.411635in}{5.215038in}}%
\pgfpathlineto{\pgfqpoint{5.412056in}{5.210957in}}%
\pgfpathlineto{\pgfqpoint{5.412225in}{5.210060in}}%
\pgfpathlineto{\pgfqpoint{5.412562in}{5.213479in}}%
\pgfpathlineto{\pgfqpoint{5.414415in}{5.230590in}}%
\pgfpathlineto{\pgfqpoint{5.415679in}{5.245320in}}%
\pgfpathlineto{\pgfqpoint{5.416100in}{5.238897in}}%
\pgfpathlineto{\pgfqpoint{5.416690in}{5.228639in}}%
\pgfpathlineto{\pgfqpoint{5.417195in}{5.236959in}}%
\pgfpathlineto{\pgfqpoint{5.417616in}{5.242695in}}%
\pgfpathlineto{\pgfqpoint{5.418038in}{5.234955in}}%
\pgfpathlineto{\pgfqpoint{5.418796in}{5.215336in}}%
\pgfpathlineto{\pgfqpoint{5.419385in}{5.225713in}}%
\pgfpathlineto{\pgfqpoint{5.419638in}{5.227612in}}%
\pgfpathlineto{\pgfqpoint{5.420059in}{5.220883in}}%
\pgfpathlineto{\pgfqpoint{5.421407in}{5.209837in}}%
\pgfpathlineto{\pgfqpoint{5.421576in}{5.210531in}}%
\pgfpathlineto{\pgfqpoint{5.423345in}{5.218827in}}%
\pgfpathlineto{\pgfqpoint{5.423766in}{5.217909in}}%
\pgfpathlineto{\pgfqpoint{5.424187in}{5.215629in}}%
\pgfpathlineto{\pgfqpoint{5.424524in}{5.219585in}}%
\pgfpathlineto{\pgfqpoint{5.425620in}{5.246630in}}%
\pgfpathlineto{\pgfqpoint{5.426715in}{5.244680in}}%
\pgfpathlineto{\pgfqpoint{5.427304in}{5.249220in}}%
\pgfpathlineto{\pgfqpoint{5.427473in}{5.249989in}}%
\pgfpathlineto{\pgfqpoint{5.427810in}{5.247030in}}%
\pgfpathlineto{\pgfqpoint{5.429832in}{5.199520in}}%
\pgfpathlineto{\pgfqpoint{5.430758in}{5.213727in}}%
\pgfpathlineto{\pgfqpoint{5.430927in}{5.214407in}}%
\pgfpathlineto{\pgfqpoint{5.431264in}{5.211460in}}%
\pgfpathlineto{\pgfqpoint{5.431854in}{5.200895in}}%
\pgfpathlineto{\pgfqpoint{5.432359in}{5.209597in}}%
\pgfpathlineto{\pgfqpoint{5.432949in}{5.221473in}}%
\pgfpathlineto{\pgfqpoint{5.433539in}{5.214903in}}%
\pgfpathlineto{\pgfqpoint{5.433623in}{5.214600in}}%
\pgfpathlineto{\pgfqpoint{5.433875in}{5.216535in}}%
\pgfpathlineto{\pgfqpoint{5.434634in}{5.231758in}}%
\pgfpathlineto{\pgfqpoint{5.435223in}{5.222739in}}%
\pgfpathlineto{\pgfqpoint{5.435560in}{5.220957in}}%
\pgfpathlineto{\pgfqpoint{5.435982in}{5.225397in}}%
\pgfpathlineto{\pgfqpoint{5.438088in}{5.247906in}}%
\pgfpathlineto{\pgfqpoint{5.438509in}{5.246938in}}%
\pgfpathlineto{\pgfqpoint{5.439436in}{5.237337in}}%
\pgfpathlineto{\pgfqpoint{5.441710in}{5.197136in}}%
\pgfpathlineto{\pgfqpoint{5.442468in}{5.210303in}}%
\pgfpathlineto{\pgfqpoint{5.442805in}{5.215773in}}%
\pgfpathlineto{\pgfqpoint{5.443395in}{5.205911in}}%
\pgfpathlineto{\pgfqpoint{5.443564in}{5.204099in}}%
\pgfpathlineto{\pgfqpoint{5.443901in}{5.211355in}}%
\pgfpathlineto{\pgfqpoint{5.444827in}{5.246476in}}%
\pgfpathlineto{\pgfqpoint{5.445501in}{5.235641in}}%
\pgfpathlineto{\pgfqpoint{5.446175in}{5.229744in}}%
\pgfpathlineto{\pgfqpoint{5.446596in}{5.234314in}}%
\pgfpathlineto{\pgfqpoint{5.447186in}{5.242903in}}%
\pgfpathlineto{\pgfqpoint{5.447607in}{5.233982in}}%
\pgfpathlineto{\pgfqpoint{5.448197in}{5.219957in}}%
\pgfpathlineto{\pgfqpoint{5.448871in}{5.226369in}}%
\pgfpathlineto{\pgfqpoint{5.449208in}{5.222808in}}%
\pgfpathlineto{\pgfqpoint{5.449629in}{5.216037in}}%
\pgfpathlineto{\pgfqpoint{5.450135in}{5.224982in}}%
\pgfpathlineto{\pgfqpoint{5.450556in}{5.232539in}}%
\pgfpathlineto{\pgfqpoint{5.451230in}{5.224544in}}%
\pgfpathlineto{\pgfqpoint{5.451314in}{5.224230in}}%
\pgfpathlineto{\pgfqpoint{5.451651in}{5.226625in}}%
\pgfpathlineto{\pgfqpoint{5.451904in}{5.228351in}}%
\pgfpathlineto{\pgfqpoint{5.452409in}{5.223755in}}%
\pgfpathlineto{\pgfqpoint{5.452831in}{5.219548in}}%
\pgfpathlineto{\pgfqpoint{5.453252in}{5.227470in}}%
\pgfpathlineto{\pgfqpoint{5.453673in}{5.236019in}}%
\pgfpathlineto{\pgfqpoint{5.454263in}{5.225386in}}%
\pgfpathlineto{\pgfqpoint{5.454600in}{5.221460in}}%
\pgfpathlineto{\pgfqpoint{5.455442in}{5.223031in}}%
\pgfpathlineto{\pgfqpoint{5.455695in}{5.222492in}}%
\pgfpathlineto{\pgfqpoint{5.455948in}{5.224035in}}%
\pgfpathlineto{\pgfqpoint{5.456537in}{5.232706in}}%
\pgfpathlineto{\pgfqpoint{5.457043in}{5.224557in}}%
\pgfpathlineto{\pgfqpoint{5.457380in}{5.219710in}}%
\pgfpathlineto{\pgfqpoint{5.457801in}{5.229236in}}%
\pgfpathlineto{\pgfqpoint{5.458222in}{5.238847in}}%
\pgfpathlineto{\pgfqpoint{5.458812in}{5.227515in}}%
\pgfpathlineto{\pgfqpoint{5.459570in}{5.212750in}}%
\pgfpathlineto{\pgfqpoint{5.460328in}{5.215409in}}%
\pgfpathlineto{\pgfqpoint{5.461255in}{5.202792in}}%
\pgfpathlineto{\pgfqpoint{5.461676in}{5.212019in}}%
\pgfpathlineto{\pgfqpoint{5.462434in}{5.233273in}}%
\pgfpathlineto{\pgfqpoint{5.463277in}{5.231973in}}%
\pgfpathlineto{\pgfqpoint{5.464288in}{5.246866in}}%
\pgfpathlineto{\pgfqpoint{5.464793in}{5.239359in}}%
\pgfpathlineto{\pgfqpoint{5.467152in}{5.213684in}}%
\pgfpathlineto{\pgfqpoint{5.467995in}{5.206658in}}%
\pgfpathlineto{\pgfqpoint{5.468584in}{5.208631in}}%
\pgfpathlineto{\pgfqpoint{5.469005in}{5.212528in}}%
\pgfpathlineto{\pgfqpoint{5.469764in}{5.228102in}}%
\pgfpathlineto{\pgfqpoint{5.470353in}{5.219474in}}%
\pgfpathlineto{\pgfqpoint{5.470859in}{5.210332in}}%
\pgfpathlineto{\pgfqpoint{5.471364in}{5.221196in}}%
\pgfpathlineto{\pgfqpoint{5.471786in}{5.228881in}}%
\pgfpathlineto{\pgfqpoint{5.472291in}{5.217339in}}%
\pgfpathlineto{\pgfqpoint{5.472628in}{5.211920in}}%
\pgfpathlineto{\pgfqpoint{5.473049in}{5.221798in}}%
\pgfpathlineto{\pgfqpoint{5.473555in}{5.235728in}}%
\pgfpathlineto{\pgfqpoint{5.474144in}{5.221902in}}%
\pgfpathlineto{\pgfqpoint{5.474566in}{5.216308in}}%
\pgfpathlineto{\pgfqpoint{5.474987in}{5.224491in}}%
\pgfpathlineto{\pgfqpoint{5.476082in}{5.250897in}}%
\pgfpathlineto{\pgfqpoint{5.476672in}{5.245799in}}%
\pgfpathlineto{\pgfqpoint{5.478609in}{5.227611in}}%
\pgfpathlineto{\pgfqpoint{5.478862in}{5.231015in}}%
\pgfpathlineto{\pgfqpoint{5.479368in}{5.242624in}}%
\pgfpathlineto{\pgfqpoint{5.479789in}{5.231621in}}%
\pgfpathlineto{\pgfqpoint{5.480547in}{5.196877in}}%
\pgfpathlineto{\pgfqpoint{5.481052in}{5.218713in}}%
\pgfpathlineto{\pgfqpoint{5.481558in}{5.233317in}}%
\pgfpathlineto{\pgfqpoint{5.482148in}{5.219606in}}%
\pgfpathlineto{\pgfqpoint{5.482653in}{5.214433in}}%
\pgfpathlineto{\pgfqpoint{5.483243in}{5.218779in}}%
\pgfpathlineto{\pgfqpoint{5.484085in}{5.233199in}}%
\pgfpathlineto{\pgfqpoint{5.484591in}{5.226407in}}%
\pgfpathlineto{\pgfqpoint{5.485180in}{5.216351in}}%
\pgfpathlineto{\pgfqpoint{5.485686in}{5.223884in}}%
\pgfpathlineto{\pgfqpoint{5.487202in}{5.230955in}}%
\pgfpathlineto{\pgfqpoint{5.487371in}{5.232149in}}%
\pgfpathlineto{\pgfqpoint{5.487792in}{5.226938in}}%
\pgfpathlineto{\pgfqpoint{5.488129in}{5.223218in}}%
\pgfpathlineto{\pgfqpoint{5.488550in}{5.231598in}}%
\pgfpathlineto{\pgfqpoint{5.489140in}{5.243566in}}%
\pgfpathlineto{\pgfqpoint{5.489645in}{5.233313in}}%
\pgfpathlineto{\pgfqpoint{5.491415in}{5.207049in}}%
\pgfpathlineto{\pgfqpoint{5.491583in}{5.208072in}}%
\pgfpathlineto{\pgfqpoint{5.492173in}{5.213340in}}%
\pgfpathlineto{\pgfqpoint{5.492847in}{5.210553in}}%
\pgfpathlineto{\pgfqpoint{5.493521in}{5.203587in}}%
\pgfpathlineto{\pgfqpoint{5.493858in}{5.208378in}}%
\pgfpathlineto{\pgfqpoint{5.494953in}{5.241732in}}%
\pgfpathlineto{\pgfqpoint{5.495627in}{5.230630in}}%
\pgfpathlineto{\pgfqpoint{5.495879in}{5.228312in}}%
\pgfpathlineto{\pgfqpoint{5.496301in}{5.236702in}}%
\pgfpathlineto{\pgfqpoint{5.496806in}{5.246882in}}%
\pgfpathlineto{\pgfqpoint{5.497312in}{5.234738in}}%
\pgfpathlineto{\pgfqpoint{5.498154in}{5.213694in}}%
\pgfpathlineto{\pgfqpoint{5.498744in}{5.220656in}}%
\pgfpathlineto{\pgfqpoint{5.498997in}{5.222127in}}%
\pgfpathlineto{\pgfqpoint{5.499418in}{5.216873in}}%
\pgfpathlineto{\pgfqpoint{5.500092in}{5.201990in}}%
\pgfpathlineto{\pgfqpoint{5.500513in}{5.214060in}}%
\pgfpathlineto{\pgfqpoint{5.501355in}{5.244839in}}%
\pgfpathlineto{\pgfqpoint{5.501945in}{5.233968in}}%
\pgfpathlineto{\pgfqpoint{5.502282in}{5.229381in}}%
\pgfpathlineto{\pgfqpoint{5.502872in}{5.236383in}}%
\pgfpathlineto{\pgfqpoint{5.503209in}{5.239723in}}%
\pgfpathlineto{\pgfqpoint{5.503714in}{5.234632in}}%
\pgfpathlineto{\pgfqpoint{5.505568in}{5.220717in}}%
\pgfpathlineto{\pgfqpoint{5.505736in}{5.220504in}}%
\pgfpathlineto{\pgfqpoint{5.505989in}{5.222154in}}%
\pgfpathlineto{\pgfqpoint{5.506831in}{5.229327in}}%
\pgfpathlineto{\pgfqpoint{5.507252in}{5.223995in}}%
\pgfpathlineto{\pgfqpoint{5.507926in}{5.211825in}}%
\pgfpathlineto{\pgfqpoint{5.508432in}{5.221114in}}%
\pgfpathlineto{\pgfqpoint{5.508937in}{5.228250in}}%
\pgfpathlineto{\pgfqpoint{5.509527in}{5.222105in}}%
\pgfpathlineto{\pgfqpoint{5.509864in}{5.219355in}}%
\pgfpathlineto{\pgfqpoint{5.510538in}{5.223277in}}%
\pgfpathlineto{\pgfqpoint{5.511043in}{5.222933in}}%
\pgfpathlineto{\pgfqpoint{5.511717in}{5.230708in}}%
\pgfpathlineto{\pgfqpoint{5.511970in}{5.232675in}}%
\pgfpathlineto{\pgfqpoint{5.512307in}{5.225586in}}%
\pgfpathlineto{\pgfqpoint{5.512897in}{5.210373in}}%
\pgfpathlineto{\pgfqpoint{5.513487in}{5.222124in}}%
\pgfpathlineto{\pgfqpoint{5.514161in}{5.242112in}}%
\pgfpathlineto{\pgfqpoint{5.514666in}{5.231512in}}%
\pgfpathlineto{\pgfqpoint{5.515340in}{5.212346in}}%
\pgfpathlineto{\pgfqpoint{5.515845in}{5.223710in}}%
\pgfpathlineto{\pgfqpoint{5.516519in}{5.235030in}}%
\pgfpathlineto{\pgfqpoint{5.517025in}{5.228673in}}%
\pgfpathlineto{\pgfqpoint{5.517699in}{5.218300in}}%
\pgfpathlineto{\pgfqpoint{5.518204in}{5.224860in}}%
\pgfpathlineto{\pgfqpoint{5.519047in}{5.242938in}}%
\pgfpathlineto{\pgfqpoint{5.519721in}{5.234357in}}%
\pgfpathlineto{\pgfqpoint{5.521490in}{5.220019in}}%
\pgfpathlineto{\pgfqpoint{5.521743in}{5.222678in}}%
\pgfpathlineto{\pgfqpoint{5.522248in}{5.229886in}}%
\pgfpathlineto{\pgfqpoint{5.522753in}{5.223526in}}%
\pgfpathlineto{\pgfqpoint{5.523090in}{5.220200in}}%
\pgfpathlineto{\pgfqpoint{5.523680in}{5.225277in}}%
\pgfpathlineto{\pgfqpoint{5.523933in}{5.226938in}}%
\pgfpathlineto{\pgfqpoint{5.524438in}{5.222442in}}%
\pgfpathlineto{\pgfqpoint{5.524860in}{5.219492in}}%
\pgfpathlineto{\pgfqpoint{5.525449in}{5.223546in}}%
\pgfpathlineto{\pgfqpoint{5.526881in}{5.230566in}}%
\pgfpathlineto{\pgfqpoint{5.527303in}{5.229278in}}%
\pgfpathlineto{\pgfqpoint{5.529409in}{5.206438in}}%
\pgfpathlineto{\pgfqpoint{5.530504in}{5.208591in}}%
\pgfpathlineto{\pgfqpoint{5.531515in}{5.230245in}}%
\pgfpathlineto{\pgfqpoint{5.532105in}{5.218156in}}%
\pgfpathlineto{\pgfqpoint{5.532357in}{5.215037in}}%
\pgfpathlineto{\pgfqpoint{5.532694in}{5.222748in}}%
\pgfpathlineto{\pgfqpoint{5.533453in}{5.252984in}}%
\pgfpathlineto{\pgfqpoint{5.534042in}{5.238778in}}%
\pgfpathlineto{\pgfqpoint{5.534463in}{5.229872in}}%
\pgfpathlineto{\pgfqpoint{5.535222in}{5.235971in}}%
\pgfpathlineto{\pgfqpoint{5.535474in}{5.235368in}}%
\pgfpathlineto{\pgfqpoint{5.536654in}{5.221922in}}%
\pgfpathlineto{\pgfqpoint{5.537244in}{5.209666in}}%
\pgfpathlineto{\pgfqpoint{5.537917in}{5.216823in}}%
\pgfpathlineto{\pgfqpoint{5.538423in}{5.218754in}}%
\pgfpathlineto{\pgfqpoint{5.538928in}{5.216372in}}%
\pgfpathlineto{\pgfqpoint{5.539434in}{5.214081in}}%
\pgfpathlineto{\pgfqpoint{5.539771in}{5.216553in}}%
\pgfpathlineto{\pgfqpoint{5.540613in}{5.234950in}}%
\pgfpathlineto{\pgfqpoint{5.541119in}{5.224848in}}%
\pgfpathlineto{\pgfqpoint{5.541624in}{5.212029in}}%
\pgfpathlineto{\pgfqpoint{5.542130in}{5.225764in}}%
\pgfpathlineto{\pgfqpoint{5.542635in}{5.238855in}}%
\pgfpathlineto{\pgfqpoint{5.543225in}{5.226390in}}%
\pgfpathlineto{\pgfqpoint{5.543899in}{5.215941in}}%
\pgfpathlineto{\pgfqpoint{5.544404in}{5.221252in}}%
\pgfpathlineto{\pgfqpoint{5.545668in}{5.246199in}}%
\pgfpathlineto{\pgfqpoint{5.546258in}{5.233509in}}%
\pgfpathlineto{\pgfqpoint{5.546679in}{5.224748in}}%
\pgfpathlineto{\pgfqpoint{5.547269in}{5.233550in}}%
\pgfpathlineto{\pgfqpoint{5.547690in}{5.238286in}}%
\pgfpathlineto{\pgfqpoint{5.548280in}{5.234001in}}%
\pgfpathlineto{\pgfqpoint{5.550386in}{5.207718in}}%
\pgfpathlineto{\pgfqpoint{5.550891in}{5.212426in}}%
\pgfpathlineto{\pgfqpoint{5.553250in}{5.236879in}}%
\pgfpathlineto{\pgfqpoint{5.553418in}{5.237292in}}%
\pgfpathlineto{\pgfqpoint{5.553840in}{5.235306in}}%
\pgfpathlineto{\pgfqpoint{5.554598in}{5.232231in}}%
\pgfpathlineto{\pgfqpoint{5.555019in}{5.234454in}}%
\pgfpathlineto{\pgfqpoint{5.555609in}{5.240939in}}%
\pgfpathlineto{\pgfqpoint{5.556030in}{5.233895in}}%
\pgfpathlineto{\pgfqpoint{5.556536in}{5.224352in}}%
\pgfpathlineto{\pgfqpoint{5.557041in}{5.234137in}}%
\pgfpathlineto{\pgfqpoint{5.557378in}{5.239105in}}%
\pgfpathlineto{\pgfqpoint{5.557883in}{5.230607in}}%
\pgfpathlineto{\pgfqpoint{5.558642in}{5.216784in}}%
\pgfpathlineto{\pgfqpoint{5.559231in}{5.221853in}}%
\pgfpathlineto{\pgfqpoint{5.559316in}{5.222288in}}%
\pgfpathlineto{\pgfqpoint{5.559653in}{5.219225in}}%
\pgfpathlineto{\pgfqpoint{5.561337in}{5.206908in}}%
\pgfpathlineto{\pgfqpoint{5.561590in}{5.206322in}}%
\pgfpathlineto{\pgfqpoint{5.561927in}{5.208194in}}%
\pgfpathlineto{\pgfqpoint{5.562938in}{5.227805in}}%
\pgfpathlineto{\pgfqpoint{5.563781in}{5.219156in}}%
\pgfpathlineto{\pgfqpoint{5.564033in}{5.217531in}}%
\pgfpathlineto{\pgfqpoint{5.564370in}{5.222948in}}%
\pgfpathlineto{\pgfqpoint{5.565465in}{5.242339in}}%
\pgfpathlineto{\pgfqpoint{5.565887in}{5.238469in}}%
\pgfpathlineto{\pgfqpoint{5.566898in}{5.204953in}}%
\pgfpathlineto{\pgfqpoint{5.567572in}{5.221300in}}%
\pgfpathlineto{\pgfqpoint{5.567824in}{5.224457in}}%
\pgfpathlineto{\pgfqpoint{5.568330in}{5.214697in}}%
\pgfpathlineto{\pgfqpoint{5.568751in}{5.207258in}}%
\pgfpathlineto{\pgfqpoint{5.569256in}{5.219336in}}%
\pgfpathlineto{\pgfqpoint{5.570099in}{5.244614in}}%
\pgfpathlineto{\pgfqpoint{5.570689in}{5.232594in}}%
\pgfpathlineto{\pgfqpoint{5.570857in}{5.231321in}}%
\pgfpathlineto{\pgfqpoint{5.571194in}{5.238248in}}%
\pgfpathlineto{\pgfqpoint{5.571952in}{5.254458in}}%
\pgfpathlineto{\pgfqpoint{5.572458in}{5.247372in}}%
\pgfpathlineto{\pgfqpoint{5.573637in}{5.215336in}}%
\pgfpathlineto{\pgfqpoint{5.574480in}{5.224284in}}%
\pgfpathlineto{\pgfqpoint{5.575406in}{5.204427in}}%
\pgfpathlineto{\pgfqpoint{5.576501in}{5.212950in}}%
\pgfpathlineto{\pgfqpoint{5.576754in}{5.215089in}}%
\pgfpathlineto{\pgfqpoint{5.577681in}{5.242656in}}%
\pgfpathlineto{\pgfqpoint{5.578439in}{5.228731in}}%
\pgfpathlineto{\pgfqpoint{5.579282in}{5.249155in}}%
\pgfpathlineto{\pgfqpoint{5.579787in}{5.233020in}}%
\pgfpathlineto{\pgfqpoint{5.581388in}{5.204732in}}%
\pgfpathlineto{\pgfqpoint{5.581472in}{5.204865in}}%
\pgfpathlineto{\pgfqpoint{5.581977in}{5.212396in}}%
\pgfpathlineto{\pgfqpoint{5.582483in}{5.220777in}}%
\pgfpathlineto{\pgfqpoint{5.583073in}{5.213263in}}%
\pgfpathlineto{\pgfqpoint{5.583578in}{5.207498in}}%
\pgfpathlineto{\pgfqpoint{5.583999in}{5.215828in}}%
\pgfpathlineto{\pgfqpoint{5.585263in}{5.245487in}}%
\pgfpathlineto{\pgfqpoint{5.585684in}{5.238122in}}%
\pgfpathlineto{\pgfqpoint{5.586358in}{5.216334in}}%
\pgfpathlineto{\pgfqpoint{5.586948in}{5.231853in}}%
\pgfpathlineto{\pgfqpoint{5.587201in}{5.234897in}}%
\pgfpathlineto{\pgfqpoint{5.587706in}{5.225671in}}%
\pgfpathlineto{\pgfqpoint{5.587874in}{5.224604in}}%
\pgfpathlineto{\pgfqpoint{5.588464in}{5.228580in}}%
\pgfpathlineto{\pgfqpoint{5.588633in}{5.228468in}}%
\pgfpathlineto{\pgfqpoint{5.588801in}{5.227806in}}%
\pgfpathlineto{\pgfqpoint{5.589222in}{5.225924in}}%
\pgfpathlineto{\pgfqpoint{5.589559in}{5.228916in}}%
\pgfpathlineto{\pgfqpoint{5.590655in}{5.250689in}}%
\pgfpathlineto{\pgfqpoint{5.591244in}{5.237649in}}%
\pgfpathlineto{\pgfqpoint{5.592929in}{5.218222in}}%
\pgfpathlineto{\pgfqpoint{5.593772in}{5.209313in}}%
\pgfpathlineto{\pgfqpoint{5.594361in}{5.214508in}}%
\pgfpathlineto{\pgfqpoint{5.595120in}{5.218673in}}%
\pgfpathlineto{\pgfqpoint{5.595457in}{5.215800in}}%
\pgfpathlineto{\pgfqpoint{5.595878in}{5.210252in}}%
\pgfpathlineto{\pgfqpoint{5.596299in}{5.217434in}}%
\pgfpathlineto{\pgfqpoint{5.597900in}{5.243474in}}%
\pgfpathlineto{\pgfqpoint{5.598152in}{5.240934in}}%
\pgfpathlineto{\pgfqpoint{5.598995in}{5.221033in}}%
\pgfpathlineto{\pgfqpoint{5.599753in}{5.226615in}}%
\pgfpathlineto{\pgfqpoint{5.600090in}{5.228531in}}%
\pgfpathlineto{\pgfqpoint{5.600511in}{5.224956in}}%
\pgfpathlineto{\pgfqpoint{5.601017in}{5.218146in}}%
\pgfpathlineto{\pgfqpoint{5.601522in}{5.225561in}}%
\pgfpathlineto{\pgfqpoint{5.601859in}{5.229129in}}%
\pgfpathlineto{\pgfqpoint{5.602280in}{5.222168in}}%
\pgfpathlineto{\pgfqpoint{5.602702in}{5.215152in}}%
\pgfpathlineto{\pgfqpoint{5.603375in}{5.221517in}}%
\pgfpathlineto{\pgfqpoint{5.604302in}{5.229230in}}%
\pgfpathlineto{\pgfqpoint{5.604892in}{5.223979in}}%
\pgfpathlineto{\pgfqpoint{5.605313in}{5.220490in}}%
\pgfpathlineto{\pgfqpoint{5.605819in}{5.225806in}}%
\pgfpathlineto{\pgfqpoint{5.606071in}{5.228146in}}%
\pgfpathlineto{\pgfqpoint{5.606493in}{5.221580in}}%
\pgfpathlineto{\pgfqpoint{5.606745in}{5.218910in}}%
\pgfpathlineto{\pgfqpoint{5.607166in}{5.228196in}}%
\pgfpathlineto{\pgfqpoint{5.607588in}{5.236415in}}%
\pgfpathlineto{\pgfqpoint{5.608093in}{5.222529in}}%
\pgfpathlineto{\pgfqpoint{5.608599in}{5.212043in}}%
\pgfpathlineto{\pgfqpoint{5.609020in}{5.223584in}}%
\pgfpathlineto{\pgfqpoint{5.609525in}{5.237037in}}%
\pgfpathlineto{\pgfqpoint{5.610199in}{5.228358in}}%
\pgfpathlineto{\pgfqpoint{5.610957in}{5.222770in}}%
\pgfpathlineto{\pgfqpoint{5.611294in}{5.227827in}}%
\pgfpathlineto{\pgfqpoint{5.611884in}{5.241478in}}%
\pgfpathlineto{\pgfqpoint{5.612390in}{5.231939in}}%
\pgfpathlineto{\pgfqpoint{5.613064in}{5.219501in}}%
\pgfpathlineto{\pgfqpoint{5.613822in}{5.221088in}}%
\pgfpathlineto{\pgfqpoint{5.614412in}{5.219972in}}%
\pgfpathlineto{\pgfqpoint{5.614833in}{5.221394in}}%
\pgfpathlineto{\pgfqpoint{5.615422in}{5.226422in}}%
\pgfpathlineto{\pgfqpoint{5.616096in}{5.222887in}}%
\pgfpathlineto{\pgfqpoint{5.616518in}{5.226249in}}%
\pgfpathlineto{\pgfqpoint{5.617192in}{5.239407in}}%
\pgfpathlineto{\pgfqpoint{5.617697in}{5.230510in}}%
\pgfpathlineto{\pgfqpoint{5.618203in}{5.222525in}}%
\pgfpathlineto{\pgfqpoint{5.618876in}{5.227556in}}%
\pgfpathlineto{\pgfqpoint{5.619972in}{5.238673in}}%
\pgfpathlineto{\pgfqpoint{5.620309in}{5.232541in}}%
\pgfpathlineto{\pgfqpoint{5.621067in}{5.205738in}}%
\pgfpathlineto{\pgfqpoint{5.621572in}{5.220546in}}%
\pgfpathlineto{\pgfqpoint{5.621909in}{5.227457in}}%
\pgfpathlineto{\pgfqpoint{5.622415in}{5.215338in}}%
\pgfpathlineto{\pgfqpoint{5.622752in}{5.209752in}}%
\pgfpathlineto{\pgfqpoint{5.623257in}{5.222833in}}%
\pgfpathlineto{\pgfqpoint{5.623763in}{5.230629in}}%
\pgfpathlineto{\pgfqpoint{5.624352in}{5.224710in}}%
\pgfpathlineto{\pgfqpoint{5.624605in}{5.222892in}}%
\pgfpathlineto{\pgfqpoint{5.625111in}{5.227932in}}%
\pgfpathlineto{\pgfqpoint{5.625363in}{5.229197in}}%
\pgfpathlineto{\pgfqpoint{5.626037in}{5.226829in}}%
\pgfpathlineto{\pgfqpoint{5.626206in}{5.226587in}}%
\pgfpathlineto{\pgfqpoint{5.626458in}{5.228068in}}%
\pgfpathlineto{\pgfqpoint{5.626627in}{5.228969in}}%
\pgfpathlineto{\pgfqpoint{5.627132in}{5.225646in}}%
\pgfpathlineto{\pgfqpoint{5.628817in}{5.210438in}}%
\pgfpathlineto{\pgfqpoint{5.629070in}{5.214424in}}%
\pgfpathlineto{\pgfqpoint{5.629660in}{5.228079in}}%
\pgfpathlineto{\pgfqpoint{5.630165in}{5.215915in}}%
\pgfpathlineto{\pgfqpoint{5.630334in}{5.214444in}}%
\pgfpathlineto{\pgfqpoint{5.630671in}{5.223152in}}%
\pgfpathlineto{\pgfqpoint{5.631260in}{5.242945in}}%
\pgfpathlineto{\pgfqpoint{5.631682in}{5.225970in}}%
\pgfpathlineto{\pgfqpoint{5.632103in}{5.207572in}}%
\pgfpathlineto{\pgfqpoint{5.632693in}{5.231422in}}%
\pgfpathlineto{\pgfqpoint{5.633114in}{5.243672in}}%
\pgfpathlineto{\pgfqpoint{5.633619in}{5.226188in}}%
\pgfpathlineto{\pgfqpoint{5.634125in}{5.212795in}}%
\pgfpathlineto{\pgfqpoint{5.634799in}{5.221957in}}%
\pgfpathlineto{\pgfqpoint{5.634883in}{5.222070in}}%
\pgfpathlineto{\pgfqpoint{5.635051in}{5.221024in}}%
\pgfpathlineto{\pgfqpoint{5.635388in}{5.218475in}}%
\pgfpathlineto{\pgfqpoint{5.635810in}{5.224008in}}%
\pgfpathlineto{\pgfqpoint{5.636399in}{5.234097in}}%
\pgfpathlineto{\pgfqpoint{5.636821in}{5.225496in}}%
\pgfpathlineto{\pgfqpoint{5.637410in}{5.212520in}}%
\pgfpathlineto{\pgfqpoint{5.637916in}{5.222651in}}%
\pgfpathlineto{\pgfqpoint{5.638674in}{5.237751in}}%
\pgfpathlineto{\pgfqpoint{5.639348in}{5.235172in}}%
\pgfpathlineto{\pgfqpoint{5.639938in}{5.232617in}}%
\pgfpathlineto{\pgfqpoint{5.640359in}{5.235512in}}%
\pgfpathlineto{\pgfqpoint{5.640696in}{5.239294in}}%
\pgfpathlineto{\pgfqpoint{5.641201in}{5.232526in}}%
\pgfpathlineto{\pgfqpoint{5.641791in}{5.222040in}}%
\pgfpathlineto{\pgfqpoint{5.642465in}{5.228012in}}%
\pgfpathlineto{\pgfqpoint{5.642633in}{5.228555in}}%
\pgfpathlineto{\pgfqpoint{5.642970in}{5.225349in}}%
\pgfpathlineto{\pgfqpoint{5.643476in}{5.220199in}}%
\pgfpathlineto{\pgfqpoint{5.644234in}{5.223007in}}%
\pgfpathlineto{\pgfqpoint{5.644824in}{5.216767in}}%
\pgfpathlineto{\pgfqpoint{5.645245in}{5.223991in}}%
\pgfpathlineto{\pgfqpoint{5.645835in}{5.237297in}}%
\pgfpathlineto{\pgfqpoint{5.646340in}{5.228154in}}%
\pgfpathlineto{\pgfqpoint{5.646846in}{5.219880in}}%
\pgfpathlineto{\pgfqpoint{5.647351in}{5.228149in}}%
\pgfpathlineto{\pgfqpoint{5.647688in}{5.231871in}}%
\pgfpathlineto{\pgfqpoint{5.648109in}{5.225150in}}%
\pgfpathlineto{\pgfqpoint{5.648783in}{5.214879in}}%
\pgfpathlineto{\pgfqpoint{5.649457in}{5.219062in}}%
\pgfpathlineto{\pgfqpoint{5.649710in}{5.220700in}}%
\pgfpathlineto{\pgfqpoint{5.650131in}{5.216976in}}%
\pgfpathlineto{\pgfqpoint{5.650637in}{5.212281in}}%
\pgfpathlineto{\pgfqpoint{5.651058in}{5.216972in}}%
\pgfpathlineto{\pgfqpoint{5.652069in}{5.236312in}}%
\pgfpathlineto{\pgfqpoint{5.652659in}{5.227162in}}%
\pgfpathlineto{\pgfqpoint{5.652996in}{5.223253in}}%
\pgfpathlineto{\pgfqpoint{5.653585in}{5.229534in}}%
\pgfpathlineto{\pgfqpoint{5.654512in}{5.241429in}}%
\pgfpathlineto{\pgfqpoint{5.654933in}{5.234106in}}%
\pgfpathlineto{\pgfqpoint{5.655607in}{5.220001in}}%
\pgfpathlineto{\pgfqpoint{5.656197in}{5.227533in}}%
\pgfpathlineto{\pgfqpoint{5.656534in}{5.230977in}}%
\pgfpathlineto{\pgfqpoint{5.656955in}{5.224566in}}%
\pgfpathlineto{\pgfqpoint{5.657629in}{5.209460in}}%
\pgfpathlineto{\pgfqpoint{5.658050in}{5.221178in}}%
\pgfpathlineto{\pgfqpoint{5.658640in}{5.234649in}}%
\pgfpathlineto{\pgfqpoint{5.659145in}{5.225117in}}%
\pgfpathlineto{\pgfqpoint{5.659735in}{5.212233in}}%
\pgfpathlineto{\pgfqpoint{5.660241in}{5.222942in}}%
\pgfpathlineto{\pgfqpoint{5.660746in}{5.238034in}}%
\pgfpathlineto{\pgfqpoint{5.661251in}{5.221594in}}%
\pgfpathlineto{\pgfqpoint{5.661673in}{5.209316in}}%
\pgfpathlineto{\pgfqpoint{5.662178in}{5.225679in}}%
\pgfpathlineto{\pgfqpoint{5.662599in}{5.234257in}}%
\pgfpathlineto{\pgfqpoint{5.663189in}{5.225114in}}%
\pgfpathlineto{\pgfqpoint{5.663695in}{5.216784in}}%
\pgfpathlineto{\pgfqpoint{5.664200in}{5.226278in}}%
\pgfpathlineto{\pgfqpoint{5.664874in}{5.236375in}}%
\pgfpathlineto{\pgfqpoint{5.665379in}{5.231833in}}%
\pgfpathlineto{\pgfqpoint{5.665885in}{5.224140in}}%
\pgfpathlineto{\pgfqpoint{5.666559in}{5.229273in}}%
\pgfpathlineto{\pgfqpoint{5.666812in}{5.230078in}}%
\pgfpathlineto{\pgfqpoint{5.667233in}{5.227318in}}%
\pgfpathlineto{\pgfqpoint{5.670518in}{5.210400in}}%
\pgfpathlineto{\pgfqpoint{5.667991in}{5.227882in}}%
\pgfpathlineto{\pgfqpoint{5.670603in}{5.210695in}}%
\pgfpathlineto{\pgfqpoint{5.671192in}{5.230674in}}%
\pgfpathlineto{\pgfqpoint{5.671698in}{5.242989in}}%
\pgfpathlineto{\pgfqpoint{5.672203in}{5.227113in}}%
\pgfpathlineto{\pgfqpoint{5.672624in}{5.215013in}}%
\pgfpathlineto{\pgfqpoint{5.673214in}{5.230283in}}%
\pgfpathlineto{\pgfqpoint{5.673720in}{5.239577in}}%
\pgfpathlineto{\pgfqpoint{5.674225in}{5.231580in}}%
\pgfpathlineto{\pgfqpoint{5.675320in}{5.209628in}}%
\pgfpathlineto{\pgfqpoint{5.675742in}{5.214899in}}%
\pgfpathlineto{\pgfqpoint{5.676247in}{5.224071in}}%
\pgfpathlineto{\pgfqpoint{5.676837in}{5.215194in}}%
\pgfpathlineto{\pgfqpoint{5.677089in}{5.213453in}}%
\pgfpathlineto{\pgfqpoint{5.677511in}{5.219371in}}%
\pgfpathlineto{\pgfqpoint{5.679280in}{5.245816in}}%
\pgfpathlineto{\pgfqpoint{5.679533in}{5.241987in}}%
\pgfpathlineto{\pgfqpoint{5.680291in}{5.224891in}}%
\pgfpathlineto{\pgfqpoint{5.680796in}{5.233110in}}%
\pgfpathlineto{\pgfqpoint{5.681133in}{5.238294in}}%
\pgfpathlineto{\pgfqpoint{5.681639in}{5.228059in}}%
\pgfpathlineto{\pgfqpoint{5.682313in}{5.211586in}}%
\pgfpathlineto{\pgfqpoint{5.682902in}{5.220851in}}%
\pgfpathlineto{\pgfqpoint{5.683829in}{5.230186in}}%
\pgfpathlineto{\pgfqpoint{5.684250in}{5.226124in}}%
\pgfpathlineto{\pgfqpoint{5.685177in}{5.211024in}}%
\pgfpathlineto{\pgfqpoint{5.685682in}{5.214998in}}%
\pgfpathlineto{\pgfqpoint{5.687115in}{5.232927in}}%
\pgfpathlineto{\pgfqpoint{5.687536in}{5.228335in}}%
\pgfpathlineto{\pgfqpoint{5.687873in}{5.224988in}}%
\pgfpathlineto{\pgfqpoint{5.688378in}{5.230870in}}%
\pgfpathlineto{\pgfqpoint{5.690063in}{5.241462in}}%
\pgfpathlineto{\pgfqpoint{5.689136in}{5.230403in}}%
\pgfpathlineto{\pgfqpoint{5.690147in}{5.241154in}}%
\pgfpathlineto{\pgfqpoint{5.694444in}{5.206976in}}%
\pgfpathlineto{\pgfqpoint{5.694781in}{5.214378in}}%
\pgfpathlineto{\pgfqpoint{5.695455in}{5.234879in}}%
\pgfpathlineto{\pgfqpoint{5.696044in}{5.221145in}}%
\pgfpathlineto{\pgfqpoint{5.696381in}{5.216832in}}%
\pgfpathlineto{\pgfqpoint{5.696887in}{5.223652in}}%
\pgfpathlineto{\pgfqpoint{5.697477in}{5.235567in}}%
\pgfpathlineto{\pgfqpoint{5.697898in}{5.224692in}}%
\pgfpathlineto{\pgfqpoint{5.698572in}{5.204243in}}%
\pgfpathlineto{\pgfqpoint{5.699077in}{5.218978in}}%
\pgfpathlineto{\pgfqpoint{5.699583in}{5.232790in}}%
\pgfpathlineto{\pgfqpoint{5.700088in}{5.219932in}}%
\pgfpathlineto{\pgfqpoint{5.700762in}{5.206988in}}%
\pgfpathlineto{\pgfqpoint{5.701183in}{5.215931in}}%
\pgfpathlineto{\pgfqpoint{5.701773in}{5.230263in}}%
\pgfpathlineto{\pgfqpoint{5.702363in}{5.221596in}}%
\pgfpathlineto{\pgfqpoint{5.702953in}{5.216143in}}%
\pgfpathlineto{\pgfqpoint{5.703458in}{5.220211in}}%
\pgfpathlineto{\pgfqpoint{5.705311in}{5.230132in}}%
\pgfpathlineto{\pgfqpoint{5.705480in}{5.229154in}}%
\pgfpathlineto{\pgfqpoint{5.706070in}{5.223916in}}%
\pgfpathlineto{\pgfqpoint{5.706491in}{5.229470in}}%
\pgfpathlineto{\pgfqpoint{5.707333in}{5.251939in}}%
\pgfpathlineto{\pgfqpoint{5.707839in}{5.237907in}}%
\pgfpathlineto{\pgfqpoint{5.708428in}{5.223894in}}%
\pgfpathlineto{\pgfqpoint{5.709102in}{5.231218in}}%
\pgfpathlineto{\pgfqpoint{5.709439in}{5.233974in}}%
\pgfpathlineto{\pgfqpoint{5.709945in}{5.227866in}}%
\pgfpathlineto{\pgfqpoint{5.711293in}{5.207778in}}%
\pgfpathlineto{\pgfqpoint{5.711798in}{5.213741in}}%
\pgfpathlineto{\pgfqpoint{5.712641in}{5.230046in}}%
\pgfpathlineto{\pgfqpoint{5.713652in}{5.229864in}}%
\pgfpathlineto{\pgfqpoint{5.715084in}{5.235389in}}%
\pgfpathlineto{\pgfqpoint{5.714241in}{5.229484in}}%
\pgfpathlineto{\pgfqpoint{5.715336in}{5.233190in}}%
\pgfpathlineto{\pgfqpoint{5.716010in}{5.224692in}}%
\pgfpathlineto{\pgfqpoint{5.716516in}{5.230116in}}%
\pgfpathlineto{\pgfqpoint{5.716769in}{5.232347in}}%
\pgfpathlineto{\pgfqpoint{5.717190in}{5.225498in}}%
\pgfpathlineto{\pgfqpoint{5.717695in}{5.216393in}}%
\pgfpathlineto{\pgfqpoint{5.718285in}{5.223099in}}%
\pgfpathlineto{\pgfqpoint{5.718622in}{5.225655in}}%
\pgfpathlineto{\pgfqpoint{5.719127in}{5.220969in}}%
\pgfpathlineto{\pgfqpoint{5.719801in}{5.212119in}}%
\pgfpathlineto{\pgfqpoint{5.720391in}{5.216707in}}%
\pgfpathlineto{\pgfqpoint{5.721234in}{5.229786in}}%
\pgfpathlineto{\pgfqpoint{5.721739in}{5.221383in}}%
\pgfpathlineto{\pgfqpoint{5.722076in}{5.217122in}}%
\pgfpathlineto{\pgfqpoint{5.722497in}{5.225096in}}%
\pgfpathlineto{\pgfqpoint{5.723003in}{5.236942in}}%
\pgfpathlineto{\pgfqpoint{5.723592in}{5.226082in}}%
\pgfpathlineto{\pgfqpoint{5.723761in}{5.225369in}}%
\pgfpathlineto{\pgfqpoint{5.724014in}{5.229312in}}%
\pgfpathlineto{\pgfqpoint{5.724603in}{5.243565in}}%
\pgfpathlineto{\pgfqpoint{5.725362in}{5.237951in}}%
\pgfpathlineto{\pgfqpoint{5.725783in}{5.240321in}}%
\pgfpathlineto{\pgfqpoint{5.726036in}{5.237239in}}%
\pgfpathlineto{\pgfqpoint{5.727889in}{5.215162in}}%
\pgfpathlineto{\pgfqpoint{5.727973in}{5.215206in}}%
\pgfpathlineto{\pgfqpoint{5.728394in}{5.215880in}}%
\pgfpathlineto{\pgfqpoint{5.728731in}{5.214426in}}%
\pgfpathlineto{\pgfqpoint{5.729827in}{5.210067in}}%
\pgfpathlineto{\pgfqpoint{5.730164in}{5.212349in}}%
\pgfpathlineto{\pgfqpoint{5.731343in}{5.229886in}}%
\pgfpathlineto{\pgfqpoint{5.732101in}{5.226859in}}%
\pgfpathlineto{\pgfqpoint{5.732270in}{5.226725in}}%
\pgfpathlineto{\pgfqpoint{5.732607in}{5.227632in}}%
\pgfpathlineto{\pgfqpoint{5.733281in}{5.236624in}}%
\pgfpathlineto{\pgfqpoint{5.733786in}{5.245102in}}%
\pgfpathlineto{\pgfqpoint{5.734207in}{5.233788in}}%
\pgfpathlineto{\pgfqpoint{5.734965in}{5.202370in}}%
\pgfpathlineto{\pgfqpoint{5.735555in}{5.219068in}}%
\pgfpathlineto{\pgfqpoint{5.736145in}{5.235640in}}%
\pgfpathlineto{\pgfqpoint{5.736735in}{5.221992in}}%
\pgfpathlineto{\pgfqpoint{5.737072in}{5.214777in}}%
\pgfpathlineto{\pgfqpoint{5.737577in}{5.226969in}}%
\pgfpathlineto{\pgfqpoint{5.738167in}{5.246924in}}%
\pgfpathlineto{\pgfqpoint{5.738672in}{5.233070in}}%
\pgfpathlineto{\pgfqpoint{5.739515in}{5.205872in}}%
\pgfpathlineto{\pgfqpoint{5.740020in}{5.216904in}}%
\pgfpathlineto{\pgfqpoint{5.740526in}{5.231246in}}%
\pgfpathlineto{\pgfqpoint{5.741115in}{5.217298in}}%
\pgfpathlineto{\pgfqpoint{5.741368in}{5.214252in}}%
\pgfpathlineto{\pgfqpoint{5.741873in}{5.224064in}}%
\pgfpathlineto{\pgfqpoint{5.742463in}{5.234294in}}%
\pgfpathlineto{\pgfqpoint{5.743137in}{5.228494in}}%
\pgfpathlineto{\pgfqpoint{5.743811in}{5.246180in}}%
\pgfpathlineto{\pgfqpoint{5.744064in}{5.250091in}}%
\pgfpathlineto{\pgfqpoint{5.744569in}{5.242056in}}%
\pgfpathlineto{\pgfqpoint{5.745664in}{5.223581in}}%
\pgfpathlineto{\pgfqpoint{5.745833in}{5.223888in}}%
\pgfusepath{stroke}%
\end{pgfscope}%
\begin{pgfscope}%
\pgfsetrectcap%
\pgfsetmiterjoin%
\pgfsetlinewidth{0.803000pt}%
\definecolor{currentstroke}{rgb}{0.737255,0.737255,0.737255}%
\pgfsetstrokecolor{currentstroke}%
\pgfsetdash{}{0pt}%
\pgfpathmoveto{\pgfqpoint{0.691161in}{4.801386in}}%
\pgfpathlineto{\pgfqpoint{0.691161in}{5.703703in}}%
\pgfusepath{stroke}%
\end{pgfscope}%
\begin{pgfscope}%
\pgfsetrectcap%
\pgfsetmiterjoin%
\pgfsetlinewidth{0.803000pt}%
\definecolor{currentstroke}{rgb}{0.737255,0.737255,0.737255}%
\pgfsetstrokecolor{currentstroke}%
\pgfsetdash{}{0pt}%
\pgfpathmoveto{\pgfqpoint{5.745833in}{4.801386in}}%
\pgfpathlineto{\pgfqpoint{5.745833in}{5.703703in}}%
\pgfusepath{stroke}%
\end{pgfscope}%
\begin{pgfscope}%
\pgfsetrectcap%
\pgfsetmiterjoin%
\pgfsetlinewidth{0.803000pt}%
\definecolor{currentstroke}{rgb}{0.737255,0.737255,0.737255}%
\pgfsetstrokecolor{currentstroke}%
\pgfsetdash{}{0pt}%
\pgfpathmoveto{\pgfqpoint{0.691161in}{4.801386in}}%
\pgfpathlineto{\pgfqpoint{5.745833in}{4.801386in}}%
\pgfusepath{stroke}%
\end{pgfscope}%
\begin{pgfscope}%
\pgfsetrectcap%
\pgfsetmiterjoin%
\pgfsetlinewidth{0.803000pt}%
\definecolor{currentstroke}{rgb}{0.737255,0.737255,0.737255}%
\pgfsetstrokecolor{currentstroke}%
\pgfsetdash{}{0pt}%
\pgfpathmoveto{\pgfqpoint{0.691161in}{5.703703in}}%
\pgfpathlineto{\pgfqpoint{5.745833in}{5.703703in}}%
\pgfusepath{stroke}%
\end{pgfscope}%
\begin{pgfscope}%
\pgfsetbuttcap%
\pgfsetmiterjoin%
\definecolor{currentfill}{rgb}{0.933333,0.933333,0.933333}%
\pgfsetfillcolor{currentfill}%
\pgfsetlinewidth{0.000000pt}%
\definecolor{currentstroke}{rgb}{0.000000,0.000000,0.000000}%
\pgfsetstrokecolor{currentstroke}%
\pgfsetstrokeopacity{0.000000}%
\pgfsetdash{}{0pt}%
\pgfpathmoveto{\pgfqpoint{0.691161in}{3.737081in}}%
\pgfpathlineto{\pgfqpoint{5.745833in}{3.737081in}}%
\pgfpathlineto{\pgfqpoint{5.745833in}{4.639398in}}%
\pgfpathlineto{\pgfqpoint{0.691161in}{4.639398in}}%
\pgfpathlineto{\pgfqpoint{0.691161in}{3.737081in}}%
\pgfpathclose%
\pgfusepath{fill}%
\end{pgfscope}%
\begin{pgfscope}%
\pgfpathrectangle{\pgfqpoint{0.691161in}{3.737081in}}{\pgfqpoint{5.054672in}{0.902317in}}%
\pgfusepath{clip}%
\pgfsetbuttcap%
\pgfsetroundjoin%
\pgfsetlinewidth{0.501875pt}%
\definecolor{currentstroke}{rgb}{0.698039,0.698039,0.698039}%
\pgfsetstrokecolor{currentstroke}%
\pgfsetdash{{1.850000pt}{0.800000pt}}{0.000000pt}%
\pgfpathmoveto{\pgfqpoint{0.691161in}{3.737081in}}%
\pgfpathlineto{\pgfqpoint{0.691161in}{4.639398in}}%
\pgfusepath{stroke}%
\end{pgfscope}%
\begin{pgfscope}%
\pgfsetbuttcap%
\pgfsetroundjoin%
\definecolor{currentfill}{rgb}{0.000000,0.000000,0.000000}%
\pgfsetfillcolor{currentfill}%
\pgfsetlinewidth{0.803000pt}%
\definecolor{currentstroke}{rgb}{0.000000,0.000000,0.000000}%
\pgfsetstrokecolor{currentstroke}%
\pgfsetdash{}{0pt}%
\pgfsys@defobject{currentmarker}{\pgfqpoint{0.000000in}{0.000000in}}{\pgfqpoint{0.000000in}{0.048611in}}{%
\pgfpathmoveto{\pgfqpoint{0.000000in}{0.000000in}}%
\pgfpathlineto{\pgfqpoint{0.000000in}{0.048611in}}%
\pgfusepath{stroke,fill}%
}%
\begin{pgfscope}%
\pgfsys@transformshift{0.691161in}{3.737081in}%
\pgfsys@useobject{currentmarker}{}%
\end{pgfscope}%
\end{pgfscope}%
\begin{pgfscope}%
\pgfpathrectangle{\pgfqpoint{0.691161in}{3.737081in}}{\pgfqpoint{5.054672in}{0.902317in}}%
\pgfusepath{clip}%
\pgfsetbuttcap%
\pgfsetroundjoin%
\pgfsetlinewidth{0.501875pt}%
\definecolor{currentstroke}{rgb}{0.698039,0.698039,0.698039}%
\pgfsetstrokecolor{currentstroke}%
\pgfsetdash{{1.850000pt}{0.800000pt}}{0.000000pt}%
\pgfpathmoveto{\pgfqpoint{1.533607in}{3.737081in}}%
\pgfpathlineto{\pgfqpoint{1.533607in}{4.639398in}}%
\pgfusepath{stroke}%
\end{pgfscope}%
\begin{pgfscope}%
\pgfsetbuttcap%
\pgfsetroundjoin%
\definecolor{currentfill}{rgb}{0.000000,0.000000,0.000000}%
\pgfsetfillcolor{currentfill}%
\pgfsetlinewidth{0.803000pt}%
\definecolor{currentstroke}{rgb}{0.000000,0.000000,0.000000}%
\pgfsetstrokecolor{currentstroke}%
\pgfsetdash{}{0pt}%
\pgfsys@defobject{currentmarker}{\pgfqpoint{0.000000in}{0.000000in}}{\pgfqpoint{0.000000in}{0.048611in}}{%
\pgfpathmoveto{\pgfqpoint{0.000000in}{0.000000in}}%
\pgfpathlineto{\pgfqpoint{0.000000in}{0.048611in}}%
\pgfusepath{stroke,fill}%
}%
\begin{pgfscope}%
\pgfsys@transformshift{1.533607in}{3.737081in}%
\pgfsys@useobject{currentmarker}{}%
\end{pgfscope}%
\end{pgfscope}%
\begin{pgfscope}%
\pgfpathrectangle{\pgfqpoint{0.691161in}{3.737081in}}{\pgfqpoint{5.054672in}{0.902317in}}%
\pgfusepath{clip}%
\pgfsetbuttcap%
\pgfsetroundjoin%
\pgfsetlinewidth{0.501875pt}%
\definecolor{currentstroke}{rgb}{0.698039,0.698039,0.698039}%
\pgfsetstrokecolor{currentstroke}%
\pgfsetdash{{1.850000pt}{0.800000pt}}{0.000000pt}%
\pgfpathmoveto{\pgfqpoint{2.376052in}{3.737081in}}%
\pgfpathlineto{\pgfqpoint{2.376052in}{4.639398in}}%
\pgfusepath{stroke}%
\end{pgfscope}%
\begin{pgfscope}%
\pgfsetbuttcap%
\pgfsetroundjoin%
\definecolor{currentfill}{rgb}{0.000000,0.000000,0.000000}%
\pgfsetfillcolor{currentfill}%
\pgfsetlinewidth{0.803000pt}%
\definecolor{currentstroke}{rgb}{0.000000,0.000000,0.000000}%
\pgfsetstrokecolor{currentstroke}%
\pgfsetdash{}{0pt}%
\pgfsys@defobject{currentmarker}{\pgfqpoint{0.000000in}{0.000000in}}{\pgfqpoint{0.000000in}{0.048611in}}{%
\pgfpathmoveto{\pgfqpoint{0.000000in}{0.000000in}}%
\pgfpathlineto{\pgfqpoint{0.000000in}{0.048611in}}%
\pgfusepath{stroke,fill}%
}%
\begin{pgfscope}%
\pgfsys@transformshift{2.376052in}{3.737081in}%
\pgfsys@useobject{currentmarker}{}%
\end{pgfscope}%
\end{pgfscope}%
\begin{pgfscope}%
\pgfpathrectangle{\pgfqpoint{0.691161in}{3.737081in}}{\pgfqpoint{5.054672in}{0.902317in}}%
\pgfusepath{clip}%
\pgfsetbuttcap%
\pgfsetroundjoin%
\pgfsetlinewidth{0.501875pt}%
\definecolor{currentstroke}{rgb}{0.698039,0.698039,0.698039}%
\pgfsetstrokecolor{currentstroke}%
\pgfsetdash{{1.850000pt}{0.800000pt}}{0.000000pt}%
\pgfpathmoveto{\pgfqpoint{3.218497in}{3.737081in}}%
\pgfpathlineto{\pgfqpoint{3.218497in}{4.639398in}}%
\pgfusepath{stroke}%
\end{pgfscope}%
\begin{pgfscope}%
\pgfsetbuttcap%
\pgfsetroundjoin%
\definecolor{currentfill}{rgb}{0.000000,0.000000,0.000000}%
\pgfsetfillcolor{currentfill}%
\pgfsetlinewidth{0.803000pt}%
\definecolor{currentstroke}{rgb}{0.000000,0.000000,0.000000}%
\pgfsetstrokecolor{currentstroke}%
\pgfsetdash{}{0pt}%
\pgfsys@defobject{currentmarker}{\pgfqpoint{0.000000in}{0.000000in}}{\pgfqpoint{0.000000in}{0.048611in}}{%
\pgfpathmoveto{\pgfqpoint{0.000000in}{0.000000in}}%
\pgfpathlineto{\pgfqpoint{0.000000in}{0.048611in}}%
\pgfusepath{stroke,fill}%
}%
\begin{pgfscope}%
\pgfsys@transformshift{3.218497in}{3.737081in}%
\pgfsys@useobject{currentmarker}{}%
\end{pgfscope}%
\end{pgfscope}%
\begin{pgfscope}%
\pgfpathrectangle{\pgfqpoint{0.691161in}{3.737081in}}{\pgfqpoint{5.054672in}{0.902317in}}%
\pgfusepath{clip}%
\pgfsetbuttcap%
\pgfsetroundjoin%
\pgfsetlinewidth{0.501875pt}%
\definecolor{currentstroke}{rgb}{0.698039,0.698039,0.698039}%
\pgfsetstrokecolor{currentstroke}%
\pgfsetdash{{1.850000pt}{0.800000pt}}{0.000000pt}%
\pgfpathmoveto{\pgfqpoint{4.060942in}{3.737081in}}%
\pgfpathlineto{\pgfqpoint{4.060942in}{4.639398in}}%
\pgfusepath{stroke}%
\end{pgfscope}%
\begin{pgfscope}%
\pgfsetbuttcap%
\pgfsetroundjoin%
\definecolor{currentfill}{rgb}{0.000000,0.000000,0.000000}%
\pgfsetfillcolor{currentfill}%
\pgfsetlinewidth{0.803000pt}%
\definecolor{currentstroke}{rgb}{0.000000,0.000000,0.000000}%
\pgfsetstrokecolor{currentstroke}%
\pgfsetdash{}{0pt}%
\pgfsys@defobject{currentmarker}{\pgfqpoint{0.000000in}{0.000000in}}{\pgfqpoint{0.000000in}{0.048611in}}{%
\pgfpathmoveto{\pgfqpoint{0.000000in}{0.000000in}}%
\pgfpathlineto{\pgfqpoint{0.000000in}{0.048611in}}%
\pgfusepath{stroke,fill}%
}%
\begin{pgfscope}%
\pgfsys@transformshift{4.060942in}{3.737081in}%
\pgfsys@useobject{currentmarker}{}%
\end{pgfscope}%
\end{pgfscope}%
\begin{pgfscope}%
\pgfpathrectangle{\pgfqpoint{0.691161in}{3.737081in}}{\pgfqpoint{5.054672in}{0.902317in}}%
\pgfusepath{clip}%
\pgfsetbuttcap%
\pgfsetroundjoin%
\pgfsetlinewidth{0.501875pt}%
\definecolor{currentstroke}{rgb}{0.698039,0.698039,0.698039}%
\pgfsetstrokecolor{currentstroke}%
\pgfsetdash{{1.850000pt}{0.800000pt}}{0.000000pt}%
\pgfpathmoveto{\pgfqpoint{4.903388in}{3.737081in}}%
\pgfpathlineto{\pgfqpoint{4.903388in}{4.639398in}}%
\pgfusepath{stroke}%
\end{pgfscope}%
\begin{pgfscope}%
\pgfsetbuttcap%
\pgfsetroundjoin%
\definecolor{currentfill}{rgb}{0.000000,0.000000,0.000000}%
\pgfsetfillcolor{currentfill}%
\pgfsetlinewidth{0.803000pt}%
\definecolor{currentstroke}{rgb}{0.000000,0.000000,0.000000}%
\pgfsetstrokecolor{currentstroke}%
\pgfsetdash{}{0pt}%
\pgfsys@defobject{currentmarker}{\pgfqpoint{0.000000in}{0.000000in}}{\pgfqpoint{0.000000in}{0.048611in}}{%
\pgfpathmoveto{\pgfqpoint{0.000000in}{0.000000in}}%
\pgfpathlineto{\pgfqpoint{0.000000in}{0.048611in}}%
\pgfusepath{stroke,fill}%
}%
\begin{pgfscope}%
\pgfsys@transformshift{4.903388in}{3.737081in}%
\pgfsys@useobject{currentmarker}{}%
\end{pgfscope}%
\end{pgfscope}%
\begin{pgfscope}%
\pgfpathrectangle{\pgfqpoint{0.691161in}{3.737081in}}{\pgfqpoint{5.054672in}{0.902317in}}%
\pgfusepath{clip}%
\pgfsetbuttcap%
\pgfsetroundjoin%
\pgfsetlinewidth{0.501875pt}%
\definecolor{currentstroke}{rgb}{0.698039,0.698039,0.698039}%
\pgfsetstrokecolor{currentstroke}%
\pgfsetdash{{1.850000pt}{0.800000pt}}{0.000000pt}%
\pgfpathmoveto{\pgfqpoint{5.745833in}{3.737081in}}%
\pgfpathlineto{\pgfqpoint{5.745833in}{4.639398in}}%
\pgfusepath{stroke}%
\end{pgfscope}%
\begin{pgfscope}%
\pgfsetbuttcap%
\pgfsetroundjoin%
\definecolor{currentfill}{rgb}{0.000000,0.000000,0.000000}%
\pgfsetfillcolor{currentfill}%
\pgfsetlinewidth{0.803000pt}%
\definecolor{currentstroke}{rgb}{0.000000,0.000000,0.000000}%
\pgfsetstrokecolor{currentstroke}%
\pgfsetdash{}{0pt}%
\pgfsys@defobject{currentmarker}{\pgfqpoint{0.000000in}{0.000000in}}{\pgfqpoint{0.000000in}{0.048611in}}{%
\pgfpathmoveto{\pgfqpoint{0.000000in}{0.000000in}}%
\pgfpathlineto{\pgfqpoint{0.000000in}{0.048611in}}%
\pgfusepath{stroke,fill}%
}%
\begin{pgfscope}%
\pgfsys@transformshift{5.745833in}{3.737081in}%
\pgfsys@useobject{currentmarker}{}%
\end{pgfscope}%
\end{pgfscope}%
\begin{pgfscope}%
\pgfpathrectangle{\pgfqpoint{0.691161in}{3.737081in}}{\pgfqpoint{5.054672in}{0.902317in}}%
\pgfusepath{clip}%
\pgfsetbuttcap%
\pgfsetroundjoin%
\pgfsetlinewidth{0.501875pt}%
\definecolor{currentstroke}{rgb}{0.698039,0.698039,0.698039}%
\pgfsetstrokecolor{currentstroke}%
\pgfsetdash{{1.850000pt}{0.800000pt}}{0.000000pt}%
\pgfpathmoveto{\pgfqpoint{0.691161in}{3.941520in}}%
\pgfpathlineto{\pgfqpoint{5.745833in}{3.941520in}}%
\pgfusepath{stroke}%
\end{pgfscope}%
\begin{pgfscope}%
\pgfsetbuttcap%
\pgfsetroundjoin%
\definecolor{currentfill}{rgb}{0.000000,0.000000,0.000000}%
\pgfsetfillcolor{currentfill}%
\pgfsetlinewidth{0.803000pt}%
\definecolor{currentstroke}{rgb}{0.000000,0.000000,0.000000}%
\pgfsetstrokecolor{currentstroke}%
\pgfsetdash{}{0pt}%
\pgfsys@defobject{currentmarker}{\pgfqpoint{0.000000in}{0.000000in}}{\pgfqpoint{0.048611in}{0.000000in}}{%
\pgfpathmoveto{\pgfqpoint{0.000000in}{0.000000in}}%
\pgfpathlineto{\pgfqpoint{0.048611in}{0.000000in}}%
\pgfusepath{stroke,fill}%
}%
\begin{pgfscope}%
\pgfsys@transformshift{0.691161in}{3.941520in}%
\pgfsys@useobject{currentmarker}{}%
\end{pgfscope}%
\end{pgfscope}%
\begin{pgfscope}%
\definecolor{textcolor}{rgb}{0.000000,0.000000,0.000000}%
\pgfsetstrokecolor{textcolor}%
\pgfsetfillcolor{textcolor}%
\pgftext[x=0.573105in, y=3.893325in, left, base]{\color{textcolor}\rmfamily\fontsize{10.000000}{12.000000}\selectfont \(\displaystyle {0}\)}%
\end{pgfscope}%
\begin{pgfscope}%
\pgfpathrectangle{\pgfqpoint{0.691161in}{3.737081in}}{\pgfqpoint{5.054672in}{0.902317in}}%
\pgfusepath{clip}%
\pgfsetbuttcap%
\pgfsetroundjoin%
\pgfsetlinewidth{0.501875pt}%
\definecolor{currentstroke}{rgb}{0.698039,0.698039,0.698039}%
\pgfsetstrokecolor{currentstroke}%
\pgfsetdash{{1.850000pt}{0.800000pt}}{0.000000pt}%
\pgfpathmoveto{\pgfqpoint{0.691161in}{4.350081in}}%
\pgfpathlineto{\pgfqpoint{5.745833in}{4.350081in}}%
\pgfusepath{stroke}%
\end{pgfscope}%
\begin{pgfscope}%
\pgfsetbuttcap%
\pgfsetroundjoin%
\definecolor{currentfill}{rgb}{0.000000,0.000000,0.000000}%
\pgfsetfillcolor{currentfill}%
\pgfsetlinewidth{0.803000pt}%
\definecolor{currentstroke}{rgb}{0.000000,0.000000,0.000000}%
\pgfsetstrokecolor{currentstroke}%
\pgfsetdash{}{0pt}%
\pgfsys@defobject{currentmarker}{\pgfqpoint{0.000000in}{0.000000in}}{\pgfqpoint{0.048611in}{0.000000in}}{%
\pgfpathmoveto{\pgfqpoint{0.000000in}{0.000000in}}%
\pgfpathlineto{\pgfqpoint{0.048611in}{0.000000in}}%
\pgfusepath{stroke,fill}%
}%
\begin{pgfscope}%
\pgfsys@transformshift{0.691161in}{4.350081in}%
\pgfsys@useobject{currentmarker}{}%
\end{pgfscope}%
\end{pgfscope}%
\begin{pgfscope}%
\definecolor{textcolor}{rgb}{0.000000,0.000000,0.000000}%
\pgfsetstrokecolor{textcolor}%
\pgfsetfillcolor{textcolor}%
\pgftext[x=0.573105in, y=4.301886in, left, base]{\color{textcolor}\rmfamily\fontsize{10.000000}{12.000000}\selectfont \(\displaystyle {5}\)}%
\end{pgfscope}%
\begin{pgfscope}%
\definecolor{textcolor}{rgb}{0.000000,0.000000,0.000000}%
\pgfsetstrokecolor{textcolor}%
\pgfsetfillcolor{textcolor}%
\pgftext[x=0.517550in,y=4.188240in,,bottom,rotate=90.000000]{\color{textcolor}\rmfamily\fontsize{12.000000}{14.400000}\selectfont Z-detect}%
\end{pgfscope}%
\begin{pgfscope}%
\pgfpathrectangle{\pgfqpoint{0.691161in}{3.737081in}}{\pgfqpoint{5.054672in}{0.902317in}}%
\pgfusepath{clip}%
\pgfsetrectcap%
\pgfsetroundjoin%
\pgfsetlinewidth{1.505625pt}%
\definecolor{currentstroke}{rgb}{0.121569,0.466667,0.705882}%
\pgfsetstrokecolor{currentstroke}%
\pgfsetdash{}{0pt}%
\pgfpathmoveto{\pgfqpoint{0.691161in}{3.941520in}}%
\pgfpathlineto{\pgfqpoint{0.707926in}{3.941520in}}%
\pgfpathlineto{\pgfqpoint{0.708010in}{3.900861in}}%
\pgfpathlineto{\pgfqpoint{0.709105in}{3.900902in}}%
\pgfpathlineto{\pgfqpoint{0.730503in}{3.901299in}}%
\pgfpathlineto{\pgfqpoint{0.958385in}{3.901655in}}%
\pgfpathlineto{\pgfqpoint{0.965124in}{3.901450in}}%
\pgfpathlineto{\pgfqpoint{0.976245in}{3.901604in}}%
\pgfpathlineto{\pgfqpoint{1.030582in}{3.901485in}}%
\pgfpathlineto{\pgfqpoint{1.034542in}{3.901249in}}%
\pgfpathlineto{\pgfqpoint{1.196291in}{3.901109in}}%
\pgfpathlineto{\pgfqpoint{1.201936in}{3.901153in}}%
\pgfpathlineto{\pgfqpoint{1.553994in}{3.901893in}}%
\pgfpathlineto{\pgfqpoint{1.563766in}{3.901630in}}%
\pgfpathlineto{\pgfqpoint{1.566041in}{3.901546in}}%
\pgfpathlineto{\pgfqpoint{1.590977in}{3.903237in}}%
\pgfpathlineto{\pgfqpoint{1.592746in}{3.903524in}}%
\pgfpathlineto{\pgfqpoint{1.601845in}{3.903921in}}%
\pgfpathlineto{\pgfqpoint{1.606478in}{3.906204in}}%
\pgfpathlineto{\pgfqpoint{1.608247in}{3.907583in}}%
\pgfpathlineto{\pgfqpoint{1.611785in}{3.909075in}}%
\pgfpathlineto{\pgfqpoint{1.614650in}{3.909219in}}%
\pgfpathlineto{\pgfqpoint{1.623158in}{3.905932in}}%
\pgfpathlineto{\pgfqpoint{1.624759in}{3.904705in}}%
\pgfpathlineto{\pgfqpoint{1.628466in}{3.904547in}}%
\pgfpathlineto{\pgfqpoint{1.629393in}{3.906012in}}%
\pgfpathlineto{\pgfqpoint{1.631246in}{3.908473in}}%
\pgfpathlineto{\pgfqpoint{1.636975in}{3.909996in}}%
\pgfpathlineto{\pgfqpoint{1.641355in}{3.912652in}}%
\pgfpathlineto{\pgfqpoint{1.642450in}{3.913615in}}%
\pgfpathlineto{\pgfqpoint{1.643546in}{3.916762in}}%
\pgfpathlineto{\pgfqpoint{1.645652in}{3.923010in}}%
\pgfpathlineto{\pgfqpoint{1.645736in}{3.922988in}}%
\pgfpathlineto{\pgfqpoint{1.646915in}{3.920867in}}%
\pgfpathlineto{\pgfqpoint{1.647421in}{3.922561in}}%
\pgfpathlineto{\pgfqpoint{1.648600in}{3.935667in}}%
\pgfpathlineto{\pgfqpoint{1.650201in}{3.946360in}}%
\pgfpathlineto{\pgfqpoint{1.650454in}{3.946205in}}%
\pgfpathlineto{\pgfqpoint{1.650791in}{3.946115in}}%
\pgfpathlineto{\pgfqpoint{1.651128in}{3.947029in}}%
\pgfpathlineto{\pgfqpoint{1.652391in}{3.958571in}}%
\pgfpathlineto{\pgfqpoint{1.653908in}{3.967293in}}%
\pgfpathlineto{\pgfqpoint{1.654245in}{3.966989in}}%
\pgfpathlineto{\pgfqpoint{1.655003in}{3.967164in}}%
\pgfpathlineto{\pgfqpoint{1.655256in}{3.967519in}}%
\pgfpathlineto{\pgfqpoint{1.656267in}{3.971422in}}%
\pgfpathlineto{\pgfqpoint{1.657362in}{3.975339in}}%
\pgfpathlineto{\pgfqpoint{1.657951in}{3.975052in}}%
\pgfpathlineto{\pgfqpoint{1.660058in}{3.975383in}}%
\pgfpathlineto{\pgfqpoint{1.660226in}{3.974830in}}%
\pgfpathlineto{\pgfqpoint{1.661827in}{3.970133in}}%
\pgfpathlineto{\pgfqpoint{1.662332in}{3.970548in}}%
\pgfpathlineto{\pgfqpoint{1.663512in}{3.972600in}}%
\pgfpathlineto{\pgfqpoint{1.664017in}{3.971533in}}%
\pgfpathlineto{\pgfqpoint{1.665028in}{3.962944in}}%
\pgfpathlineto{\pgfqpoint{1.667218in}{3.947646in}}%
\pgfpathlineto{\pgfqpoint{1.667303in}{3.947657in}}%
\pgfpathlineto{\pgfqpoint{1.667892in}{3.946747in}}%
\pgfpathlineto{\pgfqpoint{1.669072in}{3.936688in}}%
\pgfpathlineto{\pgfqpoint{1.670251in}{3.928044in}}%
\pgfpathlineto{\pgfqpoint{1.670672in}{3.929764in}}%
\pgfpathlineto{\pgfqpoint{1.672357in}{3.934632in}}%
\pgfpathlineto{\pgfqpoint{1.672863in}{3.933523in}}%
\pgfpathlineto{\pgfqpoint{1.673789in}{3.929481in}}%
\pgfpathlineto{\pgfqpoint{1.674295in}{3.931940in}}%
\pgfpathlineto{\pgfqpoint{1.675643in}{3.955103in}}%
\pgfpathlineto{\pgfqpoint{1.677580in}{3.981570in}}%
\pgfpathlineto{\pgfqpoint{1.677749in}{3.981530in}}%
\pgfpathlineto{\pgfqpoint{1.678254in}{3.981328in}}%
\pgfpathlineto{\pgfqpoint{1.678507in}{3.982171in}}%
\pgfpathlineto{\pgfqpoint{1.679350in}{3.994901in}}%
\pgfpathlineto{\pgfqpoint{1.681708in}{4.034076in}}%
\pgfpathlineto{\pgfqpoint{1.682298in}{4.033673in}}%
\pgfpathlineto{\pgfqpoint{1.682719in}{4.034548in}}%
\pgfpathlineto{\pgfqpoint{1.683562in}{4.047202in}}%
\pgfpathlineto{\pgfqpoint{1.685921in}{4.079293in}}%
\pgfpathlineto{\pgfqpoint{1.686089in}{4.079233in}}%
\pgfpathlineto{\pgfqpoint{1.686932in}{4.079905in}}%
\pgfpathlineto{\pgfqpoint{1.687943in}{4.083785in}}%
\pgfpathlineto{\pgfqpoint{1.689206in}{4.093441in}}%
\pgfpathlineto{\pgfqpoint{1.689880in}{4.093112in}}%
\pgfpathlineto{\pgfqpoint{1.690807in}{4.092221in}}%
\pgfpathlineto{\pgfqpoint{1.691734in}{4.085148in}}%
\pgfpathlineto{\pgfqpoint{1.701759in}{3.947627in}}%
\pgfpathlineto{\pgfqpoint{1.702264in}{3.949965in}}%
\pgfpathlineto{\pgfqpoint{1.703612in}{3.958992in}}%
\pgfpathlineto{\pgfqpoint{1.704117in}{3.956452in}}%
\pgfpathlineto{\pgfqpoint{1.704960in}{3.950296in}}%
\pgfpathlineto{\pgfqpoint{1.705381in}{3.955001in}}%
\pgfpathlineto{\pgfqpoint{1.712205in}{4.101903in}}%
\pgfpathlineto{\pgfqpoint{1.712373in}{4.101847in}}%
\pgfpathlineto{\pgfqpoint{1.712963in}{4.101864in}}%
\pgfpathlineto{\pgfqpoint{1.713216in}{4.102516in}}%
\pgfpathlineto{\pgfqpoint{1.713974in}{4.104131in}}%
\pgfpathlineto{\pgfqpoint{1.714732in}{4.103905in}}%
\pgfpathlineto{\pgfqpoint{1.717428in}{4.105838in}}%
\pgfpathlineto{\pgfqpoint{1.718355in}{4.118154in}}%
\pgfpathlineto{\pgfqpoint{1.720377in}{4.140875in}}%
\pgfpathlineto{\pgfqpoint{1.720714in}{4.140667in}}%
\pgfpathlineto{\pgfqpoint{1.721051in}{4.140830in}}%
\pgfpathlineto{\pgfqpoint{1.721219in}{4.141403in}}%
\pgfpathlineto{\pgfqpoint{1.722230in}{4.148044in}}%
\pgfpathlineto{\pgfqpoint{1.722820in}{4.144184in}}%
\pgfpathlineto{\pgfqpoint{1.730065in}{4.059963in}}%
\pgfpathlineto{\pgfqpoint{1.730318in}{4.060041in}}%
\pgfpathlineto{\pgfqpoint{1.730486in}{4.060549in}}%
\pgfpathlineto{\pgfqpoint{1.734193in}{4.083646in}}%
\pgfpathlineto{\pgfqpoint{1.736973in}{4.126137in}}%
\pgfpathlineto{\pgfqpoint{1.737731in}{4.127607in}}%
\pgfpathlineto{\pgfqpoint{1.738068in}{4.126707in}}%
\pgfpathlineto{\pgfqpoint{1.738826in}{4.114296in}}%
\pgfpathlineto{\pgfqpoint{1.740848in}{4.077383in}}%
\pgfpathlineto{\pgfqpoint{1.741269in}{4.078614in}}%
\pgfpathlineto{\pgfqpoint{1.742280in}{4.084988in}}%
\pgfpathlineto{\pgfqpoint{1.742954in}{4.082426in}}%
\pgfpathlineto{\pgfqpoint{1.745229in}{4.069159in}}%
\pgfpathlineto{\pgfqpoint{1.745819in}{4.070035in}}%
\pgfpathlineto{\pgfqpoint{1.746914in}{4.072199in}}%
\pgfpathlineto{\pgfqpoint{1.747251in}{4.071252in}}%
\pgfpathlineto{\pgfqpoint{1.751210in}{4.044230in}}%
\pgfpathlineto{\pgfqpoint{1.754496in}{3.949537in}}%
\pgfpathlineto{\pgfqpoint{1.755254in}{3.949803in}}%
\pgfpathlineto{\pgfqpoint{1.756096in}{3.946088in}}%
\pgfpathlineto{\pgfqpoint{1.757781in}{3.929975in}}%
\pgfpathlineto{\pgfqpoint{1.758624in}{3.934972in}}%
\pgfpathlineto{\pgfqpoint{1.761404in}{3.947732in}}%
\pgfpathlineto{\pgfqpoint{1.762078in}{3.951682in}}%
\pgfpathlineto{\pgfqpoint{1.765953in}{3.980467in}}%
\pgfpathlineto{\pgfqpoint{1.766374in}{3.980572in}}%
\pgfpathlineto{\pgfqpoint{1.766795in}{3.979774in}}%
\pgfpathlineto{\pgfqpoint{1.767891in}{3.980066in}}%
\pgfpathlineto{\pgfqpoint{1.769491in}{3.979241in}}%
\pgfpathlineto{\pgfqpoint{1.770586in}{3.979407in}}%
\pgfpathlineto{\pgfqpoint{1.770671in}{3.979561in}}%
\pgfpathlineto{\pgfqpoint{1.773535in}{3.985072in}}%
\pgfpathlineto{\pgfqpoint{1.774040in}{3.984607in}}%
\pgfpathlineto{\pgfqpoint{1.774799in}{3.981343in}}%
\pgfpathlineto{\pgfqpoint{1.778421in}{3.958720in}}%
\pgfpathlineto{\pgfqpoint{1.779348in}{3.950222in}}%
\pgfpathlineto{\pgfqpoint{1.781875in}{3.925349in}}%
\pgfpathlineto{\pgfqpoint{1.782549in}{3.925634in}}%
\pgfpathlineto{\pgfqpoint{1.783307in}{3.924477in}}%
\pgfpathlineto{\pgfqpoint{1.785245in}{3.922039in}}%
\pgfpathlineto{\pgfqpoint{1.787688in}{3.921020in}}%
\pgfpathlineto{\pgfqpoint{1.790721in}{3.915335in}}%
\pgfpathlineto{\pgfqpoint{1.792153in}{3.913121in}}%
\pgfpathlineto{\pgfqpoint{1.795017in}{3.906841in}}%
\pgfpathlineto{\pgfqpoint{1.795186in}{3.906916in}}%
\pgfpathlineto{\pgfqpoint{1.797039in}{3.910009in}}%
\pgfpathlineto{\pgfqpoint{1.798556in}{3.913495in}}%
\pgfpathlineto{\pgfqpoint{1.799061in}{3.913104in}}%
\pgfpathlineto{\pgfqpoint{1.799651in}{3.914559in}}%
\pgfpathlineto{\pgfqpoint{1.800999in}{3.925078in}}%
\pgfpathlineto{\pgfqpoint{1.803021in}{3.939523in}}%
\pgfpathlineto{\pgfqpoint{1.804031in}{3.940468in}}%
\pgfpathlineto{\pgfqpoint{1.805042in}{3.945210in}}%
\pgfpathlineto{\pgfqpoint{1.807570in}{3.957387in}}%
\pgfpathlineto{\pgfqpoint{1.809002in}{3.958612in}}%
\pgfpathlineto{\pgfqpoint{1.810013in}{3.962740in}}%
\pgfpathlineto{\pgfqpoint{1.811866in}{3.966951in}}%
\pgfpathlineto{\pgfqpoint{1.812624in}{3.965805in}}%
\pgfpathlineto{\pgfqpoint{1.815236in}{3.961197in}}%
\pgfpathlineto{\pgfqpoint{1.815489in}{3.961375in}}%
\pgfpathlineto{\pgfqpoint{1.815994in}{3.961716in}}%
\pgfpathlineto{\pgfqpoint{1.816331in}{3.960945in}}%
\pgfpathlineto{\pgfqpoint{1.817763in}{3.951371in}}%
\pgfpathlineto{\pgfqpoint{1.819954in}{3.935720in}}%
\pgfpathlineto{\pgfqpoint{1.821133in}{3.934837in}}%
\pgfpathlineto{\pgfqpoint{1.822144in}{3.928415in}}%
\pgfpathlineto{\pgfqpoint{1.824166in}{3.919650in}}%
\pgfpathlineto{\pgfqpoint{1.826778in}{3.916850in}}%
\pgfpathlineto{\pgfqpoint{1.827957in}{3.913007in}}%
\pgfpathlineto{\pgfqpoint{1.828631in}{3.913182in}}%
\pgfpathlineto{\pgfqpoint{1.830484in}{3.912075in}}%
\pgfpathlineto{\pgfqpoint{1.831411in}{3.911527in}}%
\pgfpathlineto{\pgfqpoint{1.831832in}{3.911973in}}%
\pgfpathlineto{\pgfqpoint{1.833096in}{3.911708in}}%
\pgfpathlineto{\pgfqpoint{1.836213in}{3.912400in}}%
\pgfpathlineto{\pgfqpoint{1.839751in}{3.917855in}}%
\pgfpathlineto{\pgfqpoint{1.842026in}{3.926554in}}%
\pgfpathlineto{\pgfqpoint{1.843289in}{3.927998in}}%
\pgfpathlineto{\pgfqpoint{1.843963in}{3.930445in}}%
\pgfpathlineto{\pgfqpoint{1.847165in}{3.944823in}}%
\pgfpathlineto{\pgfqpoint{1.849271in}{3.954348in}}%
\pgfpathlineto{\pgfqpoint{1.850029in}{3.955599in}}%
\pgfpathlineto{\pgfqpoint{1.850619in}{3.955428in}}%
\pgfpathlineto{\pgfqpoint{1.851630in}{3.956415in}}%
\pgfpathlineto{\pgfqpoint{1.853988in}{3.964844in}}%
\pgfpathlineto{\pgfqpoint{1.855252in}{3.964075in}}%
\pgfpathlineto{\pgfqpoint{1.856347in}{3.963309in}}%
\pgfpathlineto{\pgfqpoint{1.859717in}{3.954723in}}%
\pgfpathlineto{\pgfqpoint{1.861149in}{3.949954in}}%
\pgfpathlineto{\pgfqpoint{1.862834in}{3.939435in}}%
\pgfpathlineto{\pgfqpoint{1.863340in}{3.940453in}}%
\pgfpathlineto{\pgfqpoint{1.864435in}{3.946874in}}%
\pgfpathlineto{\pgfqpoint{1.865109in}{3.944242in}}%
\pgfpathlineto{\pgfqpoint{1.866625in}{3.938676in}}%
\pgfpathlineto{\pgfqpoint{1.867131in}{3.939900in}}%
\pgfpathlineto{\pgfqpoint{1.868479in}{3.946527in}}%
\pgfpathlineto{\pgfqpoint{1.869068in}{3.943643in}}%
\pgfpathlineto{\pgfqpoint{1.870837in}{3.936935in}}%
\pgfpathlineto{\pgfqpoint{1.870922in}{3.936963in}}%
\pgfpathlineto{\pgfqpoint{1.872607in}{3.937296in}}%
\pgfpathlineto{\pgfqpoint{1.872775in}{3.936903in}}%
\pgfpathlineto{\pgfqpoint{1.875302in}{3.933344in}}%
\pgfpathlineto{\pgfqpoint{1.876482in}{3.932560in}}%
\pgfpathlineto{\pgfqpoint{1.877493in}{3.930526in}}%
\pgfpathlineto{\pgfqpoint{1.877914in}{3.931629in}}%
\pgfpathlineto{\pgfqpoint{1.879599in}{3.934861in}}%
\pgfpathlineto{\pgfqpoint{1.879852in}{3.934603in}}%
\pgfpathlineto{\pgfqpoint{1.880778in}{3.930074in}}%
\pgfpathlineto{\pgfqpoint{1.881284in}{3.928485in}}%
\pgfpathlineto{\pgfqpoint{1.881789in}{3.930324in}}%
\pgfpathlineto{\pgfqpoint{1.883558in}{3.938185in}}%
\pgfpathlineto{\pgfqpoint{1.883980in}{3.937284in}}%
\pgfpathlineto{\pgfqpoint{1.884906in}{3.934734in}}%
\pgfpathlineto{\pgfqpoint{1.885412in}{3.936053in}}%
\pgfpathlineto{\pgfqpoint{1.887602in}{3.948163in}}%
\pgfpathlineto{\pgfqpoint{1.888444in}{3.947800in}}%
\pgfpathlineto{\pgfqpoint{1.892151in}{3.949321in}}%
\pgfpathlineto{\pgfqpoint{1.894089in}{3.950260in}}%
\pgfpathlineto{\pgfqpoint{1.894931in}{3.950467in}}%
\pgfpathlineto{\pgfqpoint{1.895016in}{3.950731in}}%
\pgfpathlineto{\pgfqpoint{1.896111in}{3.959474in}}%
\pgfpathlineto{\pgfqpoint{1.897374in}{3.966107in}}%
\pgfpathlineto{\pgfqpoint{1.897796in}{3.964949in}}%
\pgfpathlineto{\pgfqpoint{1.898891in}{3.959887in}}%
\pgfpathlineto{\pgfqpoint{1.899396in}{3.961890in}}%
\pgfpathlineto{\pgfqpoint{1.900997in}{3.965442in}}%
\pgfpathlineto{\pgfqpoint{1.901081in}{3.965413in}}%
\pgfpathlineto{\pgfqpoint{1.901671in}{3.963070in}}%
\pgfpathlineto{\pgfqpoint{1.902682in}{3.958849in}}%
\pgfpathlineto{\pgfqpoint{1.903272in}{3.959953in}}%
\pgfpathlineto{\pgfqpoint{1.903524in}{3.959933in}}%
\pgfpathlineto{\pgfqpoint{1.903861in}{3.959167in}}%
\pgfpathlineto{\pgfqpoint{1.904451in}{3.957992in}}%
\pgfpathlineto{\pgfqpoint{1.904956in}{3.959374in}}%
\pgfpathlineto{\pgfqpoint{1.905883in}{3.971621in}}%
\pgfpathlineto{\pgfqpoint{1.908916in}{4.013863in}}%
\pgfpathlineto{\pgfqpoint{1.909590in}{4.019835in}}%
\pgfpathlineto{\pgfqpoint{1.911443in}{4.047181in}}%
\pgfpathlineto{\pgfqpoint{1.912117in}{4.042749in}}%
\pgfpathlineto{\pgfqpoint{1.913971in}{4.028708in}}%
\pgfpathlineto{\pgfqpoint{1.914476in}{4.029357in}}%
\pgfpathlineto{\pgfqpoint{1.915066in}{4.029991in}}%
\pgfpathlineto{\pgfqpoint{1.915487in}{4.028925in}}%
\pgfpathlineto{\pgfqpoint{1.917846in}{4.014452in}}%
\pgfpathlineto{\pgfqpoint{1.918941in}{4.014997in}}%
\pgfpathlineto{\pgfqpoint{1.919531in}{4.011863in}}%
\pgfpathlineto{\pgfqpoint{1.920626in}{4.003833in}}%
\pgfpathlineto{\pgfqpoint{1.921300in}{4.004359in}}%
\pgfpathlineto{\pgfqpoint{1.921805in}{4.003488in}}%
\pgfpathlineto{\pgfqpoint{1.922648in}{3.994908in}}%
\pgfpathlineto{\pgfqpoint{1.928124in}{3.917741in}}%
\pgfpathlineto{\pgfqpoint{1.929472in}{3.916008in}}%
\pgfpathlineto{\pgfqpoint{1.930567in}{3.914623in}}%
\pgfpathlineto{\pgfqpoint{1.932926in}{3.911471in}}%
\pgfpathlineto{\pgfqpoint{1.936801in}{3.911768in}}%
\pgfpathlineto{\pgfqpoint{1.938738in}{3.912885in}}%
\pgfpathlineto{\pgfqpoint{1.938991in}{3.912387in}}%
\pgfpathlineto{\pgfqpoint{1.941940in}{3.907833in}}%
\pgfpathlineto{\pgfqpoint{1.943035in}{3.908863in}}%
\pgfpathlineto{\pgfqpoint{1.944299in}{3.917379in}}%
\pgfpathlineto{\pgfqpoint{1.947079in}{3.931606in}}%
\pgfpathlineto{\pgfqpoint{1.947163in}{3.931591in}}%
\pgfpathlineto{\pgfqpoint{1.947921in}{3.932319in}}%
\pgfpathlineto{\pgfqpoint{1.949269in}{3.942085in}}%
\pgfpathlineto{\pgfqpoint{1.951628in}{3.950189in}}%
\pgfpathlineto{\pgfqpoint{1.953229in}{3.949212in}}%
\pgfpathlineto{\pgfqpoint{1.954408in}{3.949633in}}%
\pgfpathlineto{\pgfqpoint{1.960221in}{3.952097in}}%
\pgfpathlineto{\pgfqpoint{1.960474in}{3.951481in}}%
\pgfpathlineto{\pgfqpoint{1.961990in}{3.943906in}}%
\pgfpathlineto{\pgfqpoint{1.963591in}{3.938022in}}%
\pgfpathlineto{\pgfqpoint{1.963759in}{3.938099in}}%
\pgfpathlineto{\pgfqpoint{1.964602in}{3.940751in}}%
\pgfpathlineto{\pgfqpoint{1.965023in}{3.941525in}}%
\pgfpathlineto{\pgfqpoint{1.965528in}{3.940001in}}%
\pgfpathlineto{\pgfqpoint{1.967971in}{3.929955in}}%
\pgfpathlineto{\pgfqpoint{1.968393in}{3.930619in}}%
\pgfpathlineto{\pgfqpoint{1.970583in}{3.936454in}}%
\pgfpathlineto{\pgfqpoint{1.971088in}{3.935613in}}%
\pgfpathlineto{\pgfqpoint{1.974205in}{3.931070in}}%
\pgfpathlineto{\pgfqpoint{1.976312in}{3.930918in}}%
\pgfpathlineto{\pgfqpoint{1.977070in}{3.929814in}}%
\pgfpathlineto{\pgfqpoint{1.979007in}{3.923997in}}%
\pgfpathlineto{\pgfqpoint{1.979597in}{3.925001in}}%
\pgfpathlineto{\pgfqpoint{1.980692in}{3.927058in}}%
\pgfpathlineto{\pgfqpoint{1.981113in}{3.926144in}}%
\pgfpathlineto{\pgfqpoint{1.982293in}{3.921573in}}%
\pgfpathlineto{\pgfqpoint{1.982798in}{3.922980in}}%
\pgfpathlineto{\pgfqpoint{1.983894in}{3.934859in}}%
\pgfpathlineto{\pgfqpoint{1.985326in}{3.947787in}}%
\pgfpathlineto{\pgfqpoint{1.985747in}{3.947081in}}%
\pgfpathlineto{\pgfqpoint{1.986842in}{3.944700in}}%
\pgfpathlineto{\pgfqpoint{1.987179in}{3.946077in}}%
\pgfpathlineto{\pgfqpoint{1.988274in}{3.965093in}}%
\pgfpathlineto{\pgfqpoint{1.990717in}{3.993305in}}%
\pgfpathlineto{\pgfqpoint{1.991476in}{3.994236in}}%
\pgfpathlineto{\pgfqpoint{1.992402in}{4.004927in}}%
\pgfpathlineto{\pgfqpoint{1.994930in}{4.029751in}}%
\pgfpathlineto{\pgfqpoint{1.995351in}{4.031295in}}%
\pgfpathlineto{\pgfqpoint{1.996193in}{4.047934in}}%
\pgfpathlineto{\pgfqpoint{1.998299in}{4.083597in}}%
\pgfpathlineto{\pgfqpoint{1.998468in}{4.083541in}}%
\pgfpathlineto{\pgfqpoint{1.998889in}{4.083379in}}%
\pgfpathlineto{\pgfqpoint{1.999142in}{4.084209in}}%
\pgfpathlineto{\pgfqpoint{1.999984in}{4.096724in}}%
\pgfpathlineto{\pgfqpoint{2.001416in}{4.111723in}}%
\pgfpathlineto{\pgfqpoint{2.001753in}{4.111222in}}%
\pgfpathlineto{\pgfqpoint{2.002427in}{4.109268in}}%
\pgfpathlineto{\pgfqpoint{2.002849in}{4.110908in}}%
\pgfpathlineto{\pgfqpoint{2.004112in}{4.131390in}}%
\pgfpathlineto{\pgfqpoint{2.004618in}{4.135489in}}%
\pgfpathlineto{\pgfqpoint{2.005123in}{4.131378in}}%
\pgfpathlineto{\pgfqpoint{2.006471in}{4.113560in}}%
\pgfpathlineto{\pgfqpoint{2.006977in}{4.118039in}}%
\pgfpathlineto{\pgfqpoint{2.008746in}{4.157729in}}%
\pgfpathlineto{\pgfqpoint{2.009588in}{4.147975in}}%
\pgfpathlineto{\pgfqpoint{2.010936in}{4.130854in}}%
\pgfpathlineto{\pgfqpoint{2.011442in}{4.132777in}}%
\pgfpathlineto{\pgfqpoint{2.012115in}{4.136120in}}%
\pgfpathlineto{\pgfqpoint{2.012452in}{4.133760in}}%
\pgfpathlineto{\pgfqpoint{2.013800in}{4.097078in}}%
\pgfpathlineto{\pgfqpoint{2.014980in}{4.081816in}}%
\pgfpathlineto{\pgfqpoint{2.015401in}{4.082542in}}%
\pgfpathlineto{\pgfqpoint{2.015738in}{4.083163in}}%
\pgfpathlineto{\pgfqpoint{2.016075in}{4.081962in}}%
\pgfpathlineto{\pgfqpoint{2.017086in}{4.062773in}}%
\pgfpathlineto{\pgfqpoint{2.019276in}{4.037965in}}%
\pgfpathlineto{\pgfqpoint{2.019866in}{4.034963in}}%
\pgfpathlineto{\pgfqpoint{2.021045in}{4.014341in}}%
\pgfpathlineto{\pgfqpoint{2.022815in}{3.997630in}}%
\pgfpathlineto{\pgfqpoint{2.023488in}{3.991456in}}%
\pgfpathlineto{\pgfqpoint{2.028122in}{3.935464in}}%
\pgfpathlineto{\pgfqpoint{2.029133in}{3.933645in}}%
\pgfpathlineto{\pgfqpoint{2.029638in}{3.934491in}}%
\pgfpathlineto{\pgfqpoint{2.031660in}{3.938712in}}%
\pgfpathlineto{\pgfqpoint{2.032081in}{3.938203in}}%
\pgfpathlineto{\pgfqpoint{2.033429in}{3.932599in}}%
\pgfpathlineto{\pgfqpoint{2.034440in}{3.927551in}}%
\pgfpathlineto{\pgfqpoint{2.034946in}{3.929812in}}%
\pgfpathlineto{\pgfqpoint{2.036631in}{3.952365in}}%
\pgfpathlineto{\pgfqpoint{2.037389in}{3.957906in}}%
\pgfpathlineto{\pgfqpoint{2.037979in}{3.956715in}}%
\pgfpathlineto{\pgfqpoint{2.038316in}{3.956364in}}%
\pgfpathlineto{\pgfqpoint{2.038652in}{3.957398in}}%
\pgfpathlineto{\pgfqpoint{2.039748in}{3.971336in}}%
\pgfpathlineto{\pgfqpoint{2.042528in}{4.003066in}}%
\pgfpathlineto{\pgfqpoint{2.043370in}{4.004169in}}%
\pgfpathlineto{\pgfqpoint{2.045224in}{4.012557in}}%
\pgfpathlineto{\pgfqpoint{2.045982in}{4.010135in}}%
\pgfpathlineto{\pgfqpoint{2.048678in}{4.003074in}}%
\pgfpathlineto{\pgfqpoint{2.049352in}{4.003916in}}%
\pgfpathlineto{\pgfqpoint{2.050952in}{4.010297in}}%
\pgfpathlineto{\pgfqpoint{2.051626in}{4.007564in}}%
\pgfpathlineto{\pgfqpoint{2.053985in}{3.983810in}}%
\pgfpathlineto{\pgfqpoint{2.054996in}{3.989466in}}%
\pgfpathlineto{\pgfqpoint{2.055417in}{3.990806in}}%
\pgfpathlineto{\pgfqpoint{2.055838in}{3.988889in}}%
\pgfpathlineto{\pgfqpoint{2.057102in}{3.967957in}}%
\pgfpathlineto{\pgfqpoint{2.058534in}{3.950309in}}%
\pgfpathlineto{\pgfqpoint{2.058955in}{3.950608in}}%
\pgfpathlineto{\pgfqpoint{2.059882in}{3.951130in}}%
\pgfpathlineto{\pgfqpoint{2.060219in}{3.950424in}}%
\pgfpathlineto{\pgfqpoint{2.061904in}{3.939489in}}%
\pgfpathlineto{\pgfqpoint{2.062494in}{3.938024in}}%
\pgfpathlineto{\pgfqpoint{2.063083in}{3.938896in}}%
\pgfpathlineto{\pgfqpoint{2.064600in}{3.939114in}}%
\pgfpathlineto{\pgfqpoint{2.065526in}{3.940455in}}%
\pgfpathlineto{\pgfqpoint{2.067211in}{3.944130in}}%
\pgfpathlineto{\pgfqpoint{2.067717in}{3.943499in}}%
\pgfpathlineto{\pgfqpoint{2.068981in}{3.942877in}}%
\pgfpathlineto{\pgfqpoint{2.069149in}{3.943124in}}%
\pgfpathlineto{\pgfqpoint{2.070328in}{3.945993in}}%
\pgfpathlineto{\pgfqpoint{2.071087in}{3.945034in}}%
\pgfpathlineto{\pgfqpoint{2.072940in}{3.938820in}}%
\pgfpathlineto{\pgfqpoint{2.073698in}{3.941277in}}%
\pgfpathlineto{\pgfqpoint{2.076394in}{3.950487in}}%
\pgfpathlineto{\pgfqpoint{2.076478in}{3.950459in}}%
\pgfpathlineto{\pgfqpoint{2.077405in}{3.950639in}}%
\pgfpathlineto{\pgfqpoint{2.077573in}{3.951014in}}%
\pgfpathlineto{\pgfqpoint{2.079511in}{3.956781in}}%
\pgfpathlineto{\pgfqpoint{2.080522in}{3.955745in}}%
\pgfpathlineto{\pgfqpoint{2.082797in}{3.954637in}}%
\pgfpathlineto{\pgfqpoint{2.084145in}{3.948890in}}%
\pgfpathlineto{\pgfqpoint{2.084734in}{3.950449in}}%
\pgfpathlineto{\pgfqpoint{2.086166in}{3.954957in}}%
\pgfpathlineto{\pgfqpoint{2.086588in}{3.953909in}}%
\pgfpathlineto{\pgfqpoint{2.088609in}{3.943813in}}%
\pgfpathlineto{\pgfqpoint{2.089368in}{3.946019in}}%
\pgfpathlineto{\pgfqpoint{2.090463in}{3.949765in}}%
\pgfpathlineto{\pgfqpoint{2.090884in}{3.948483in}}%
\pgfpathlineto{\pgfqpoint{2.093159in}{3.935787in}}%
\pgfpathlineto{\pgfqpoint{2.093917in}{3.936938in}}%
\pgfpathlineto{\pgfqpoint{2.094759in}{3.937556in}}%
\pgfpathlineto{\pgfqpoint{2.095181in}{3.937003in}}%
\pgfpathlineto{\pgfqpoint{2.097202in}{3.935364in}}%
\pgfpathlineto{\pgfqpoint{2.099056in}{3.936292in}}%
\pgfpathlineto{\pgfqpoint{2.100404in}{3.938512in}}%
\pgfpathlineto{\pgfqpoint{2.100993in}{3.936891in}}%
\pgfpathlineto{\pgfqpoint{2.104195in}{3.927863in}}%
\pgfpathlineto{\pgfqpoint{2.105206in}{3.928973in}}%
\pgfpathlineto{\pgfqpoint{2.105880in}{3.929615in}}%
\pgfpathlineto{\pgfqpoint{2.106301in}{3.928798in}}%
\pgfpathlineto{\pgfqpoint{2.107901in}{3.922521in}}%
\pgfpathlineto{\pgfqpoint{2.108491in}{3.924614in}}%
\pgfpathlineto{\pgfqpoint{2.110513in}{3.932880in}}%
\pgfpathlineto{\pgfqpoint{2.110850in}{3.932368in}}%
\pgfpathlineto{\pgfqpoint{2.112703in}{3.929200in}}%
\pgfpathlineto{\pgfqpoint{2.113125in}{3.929496in}}%
\pgfpathlineto{\pgfqpoint{2.114810in}{3.931653in}}%
\pgfpathlineto{\pgfqpoint{2.115484in}{3.930742in}}%
\pgfpathlineto{\pgfqpoint{2.117927in}{3.927200in}}%
\pgfpathlineto{\pgfqpoint{2.118685in}{3.928292in}}%
\pgfpathlineto{\pgfqpoint{2.120959in}{3.931245in}}%
\pgfpathlineto{\pgfqpoint{2.121718in}{3.932371in}}%
\pgfpathlineto{\pgfqpoint{2.122223in}{3.931573in}}%
\pgfpathlineto{\pgfqpoint{2.123824in}{3.928861in}}%
\pgfpathlineto{\pgfqpoint{2.124161in}{3.929432in}}%
\pgfpathlineto{\pgfqpoint{2.125846in}{3.939960in}}%
\pgfpathlineto{\pgfqpoint{2.126941in}{3.945605in}}%
\pgfpathlineto{\pgfqpoint{2.127615in}{3.944867in}}%
\pgfpathlineto{\pgfqpoint{2.128120in}{3.947013in}}%
\pgfpathlineto{\pgfqpoint{2.129215in}{3.965663in}}%
\pgfpathlineto{\pgfqpoint{2.130984in}{3.997466in}}%
\pgfpathlineto{\pgfqpoint{2.131406in}{3.997127in}}%
\pgfpathlineto{\pgfqpoint{2.131743in}{3.998306in}}%
\pgfpathlineto{\pgfqpoint{2.132501in}{4.015938in}}%
\pgfpathlineto{\pgfqpoint{2.134860in}{4.085974in}}%
\pgfpathlineto{\pgfqpoint{2.135449in}{4.085742in}}%
\pgfpathlineto{\pgfqpoint{2.136123in}{4.092774in}}%
\pgfpathlineto{\pgfqpoint{2.138651in}{4.135866in}}%
\pgfpathlineto{\pgfqpoint{2.139577in}{4.135144in}}%
\pgfpathlineto{\pgfqpoint{2.140083in}{4.137633in}}%
\pgfpathlineto{\pgfqpoint{2.142442in}{4.165567in}}%
\pgfpathlineto{\pgfqpoint{2.143453in}{4.160206in}}%
\pgfpathlineto{\pgfqpoint{2.143958in}{4.157880in}}%
\pgfpathlineto{\pgfqpoint{2.144464in}{4.160264in}}%
\pgfpathlineto{\pgfqpoint{2.145980in}{4.172239in}}%
\pgfpathlineto{\pgfqpoint{2.146570in}{4.168122in}}%
\pgfpathlineto{\pgfqpoint{2.147159in}{4.164784in}}%
\pgfpathlineto{\pgfqpoint{2.147833in}{4.165780in}}%
\pgfpathlineto{\pgfqpoint{2.148339in}{4.165738in}}%
\pgfpathlineto{\pgfqpoint{2.148507in}{4.165234in}}%
\pgfpathlineto{\pgfqpoint{2.149181in}{4.155563in}}%
\pgfpathlineto{\pgfqpoint{2.152298in}{4.103400in}}%
\pgfpathlineto{\pgfqpoint{2.153225in}{4.092578in}}%
\pgfpathlineto{\pgfqpoint{2.155078in}{4.067154in}}%
\pgfpathlineto{\pgfqpoint{2.155837in}{4.067265in}}%
\pgfpathlineto{\pgfqpoint{2.156763in}{4.066233in}}%
\pgfpathlineto{\pgfqpoint{2.158111in}{4.052410in}}%
\pgfpathlineto{\pgfqpoint{2.160723in}{4.028894in}}%
\pgfpathlineto{\pgfqpoint{2.161313in}{4.026552in}}%
\pgfpathlineto{\pgfqpoint{2.162408in}{4.010187in}}%
\pgfpathlineto{\pgfqpoint{2.164767in}{3.968185in}}%
\pgfpathlineto{\pgfqpoint{2.165104in}{3.968650in}}%
\pgfpathlineto{\pgfqpoint{2.165862in}{3.969857in}}%
\pgfpathlineto{\pgfqpoint{2.166283in}{3.968960in}}%
\pgfpathlineto{\pgfqpoint{2.169905in}{3.959066in}}%
\pgfpathlineto{\pgfqpoint{2.171843in}{3.958017in}}%
\pgfpathlineto{\pgfqpoint{2.172012in}{3.958460in}}%
\pgfpathlineto{\pgfqpoint{2.173612in}{3.968007in}}%
\pgfpathlineto{\pgfqpoint{2.174539in}{3.965246in}}%
\pgfpathlineto{\pgfqpoint{2.175466in}{3.961046in}}%
\pgfpathlineto{\pgfqpoint{2.176055in}{3.962832in}}%
\pgfpathlineto{\pgfqpoint{2.177487in}{3.977113in}}%
\pgfpathlineto{\pgfqpoint{2.178414in}{3.980315in}}%
\pgfpathlineto{\pgfqpoint{2.178835in}{3.979809in}}%
\pgfpathlineto{\pgfqpoint{2.179762in}{3.977686in}}%
\pgfpathlineto{\pgfqpoint{2.180268in}{3.979303in}}%
\pgfpathlineto{\pgfqpoint{2.182458in}{3.996202in}}%
\pgfpathlineto{\pgfqpoint{2.183469in}{3.991662in}}%
\pgfpathlineto{\pgfqpoint{2.184817in}{3.984395in}}%
\pgfpathlineto{\pgfqpoint{2.185238in}{3.985572in}}%
\pgfpathlineto{\pgfqpoint{2.186923in}{3.997490in}}%
\pgfpathlineto{\pgfqpoint{2.187681in}{3.994005in}}%
\pgfpathlineto{\pgfqpoint{2.189787in}{3.979433in}}%
\pgfpathlineto{\pgfqpoint{2.190377in}{3.980909in}}%
\pgfpathlineto{\pgfqpoint{2.191725in}{3.985633in}}%
\pgfpathlineto{\pgfqpoint{2.192230in}{3.984365in}}%
\pgfpathlineto{\pgfqpoint{2.193999in}{3.970165in}}%
\pgfpathlineto{\pgfqpoint{2.195516in}{3.965478in}}%
\pgfpathlineto{\pgfqpoint{2.196948in}{3.963839in}}%
\pgfpathlineto{\pgfqpoint{2.198549in}{3.950256in}}%
\pgfpathlineto{\pgfqpoint{2.200402in}{3.943153in}}%
\pgfpathlineto{\pgfqpoint{2.201413in}{3.944121in}}%
\pgfpathlineto{\pgfqpoint{2.202003in}{3.944851in}}%
\pgfpathlineto{\pgfqpoint{2.202424in}{3.944023in}}%
\pgfpathlineto{\pgfqpoint{2.204277in}{3.938167in}}%
\pgfpathlineto{\pgfqpoint{2.204867in}{3.939947in}}%
\pgfpathlineto{\pgfqpoint{2.206805in}{3.955498in}}%
\pgfpathlineto{\pgfqpoint{2.207816in}{3.951575in}}%
\pgfpathlineto{\pgfqpoint{2.208321in}{3.950379in}}%
\pgfpathlineto{\pgfqpoint{2.208742in}{3.951787in}}%
\pgfpathlineto{\pgfqpoint{2.210933in}{3.964234in}}%
\pgfpathlineto{\pgfqpoint{2.211607in}{3.962788in}}%
\pgfpathlineto{\pgfqpoint{2.212280in}{3.961863in}}%
\pgfpathlineto{\pgfqpoint{2.212786in}{3.962486in}}%
\pgfpathlineto{\pgfqpoint{2.214387in}{3.964409in}}%
\pgfpathlineto{\pgfqpoint{2.214808in}{3.964153in}}%
\pgfpathlineto{\pgfqpoint{2.219273in}{3.959937in}}%
\pgfpathlineto{\pgfqpoint{2.220705in}{3.955874in}}%
\pgfpathlineto{\pgfqpoint{2.221463in}{3.956495in}}%
\pgfpathlineto{\pgfqpoint{2.222053in}{3.957048in}}%
\pgfpathlineto{\pgfqpoint{2.222390in}{3.956169in}}%
\pgfpathlineto{\pgfqpoint{2.224496in}{3.946556in}}%
\pgfpathlineto{\pgfqpoint{2.225338in}{3.946905in}}%
\pgfpathlineto{\pgfqpoint{2.226097in}{3.943820in}}%
\pgfpathlineto{\pgfqpoint{2.228034in}{3.934377in}}%
\pgfpathlineto{\pgfqpoint{2.228624in}{3.934838in}}%
\pgfpathlineto{\pgfqpoint{2.229551in}{3.934495in}}%
\pgfpathlineto{\pgfqpoint{2.229635in}{3.934349in}}%
\pgfpathlineto{\pgfqpoint{2.230562in}{3.932985in}}%
\pgfpathlineto{\pgfqpoint{2.230983in}{3.933865in}}%
\pgfpathlineto{\pgfqpoint{2.235953in}{3.949057in}}%
\pgfpathlineto{\pgfqpoint{2.236037in}{3.949020in}}%
\pgfpathlineto{\pgfqpoint{2.238481in}{3.944950in}}%
\pgfpathlineto{\pgfqpoint{2.241766in}{3.930163in}}%
\pgfpathlineto{\pgfqpoint{2.242440in}{3.930496in}}%
\pgfpathlineto{\pgfqpoint{2.242524in}{3.930731in}}%
\pgfpathlineto{\pgfqpoint{2.243367in}{3.937734in}}%
\pgfpathlineto{\pgfqpoint{2.245220in}{3.957545in}}%
\pgfpathlineto{\pgfqpoint{2.245810in}{3.957354in}}%
\pgfpathlineto{\pgfqpoint{2.246315in}{3.960483in}}%
\pgfpathlineto{\pgfqpoint{2.247579in}{3.991443in}}%
\pgfpathlineto{\pgfqpoint{2.248927in}{4.015086in}}%
\pgfpathlineto{\pgfqpoint{2.249348in}{4.014489in}}%
\pgfpathlineto{\pgfqpoint{2.250022in}{4.013834in}}%
\pgfpathlineto{\pgfqpoint{2.250443in}{4.014674in}}%
\pgfpathlineto{\pgfqpoint{2.251791in}{4.019661in}}%
\pgfpathlineto{\pgfqpoint{2.252381in}{4.017439in}}%
\pgfpathlineto{\pgfqpoint{2.253308in}{4.014165in}}%
\pgfpathlineto{\pgfqpoint{2.253897in}{4.014884in}}%
\pgfpathlineto{\pgfqpoint{2.256425in}{4.022021in}}%
\pgfpathlineto{\pgfqpoint{2.258699in}{4.055201in}}%
\pgfpathlineto{\pgfqpoint{2.259542in}{4.052957in}}%
\pgfpathlineto{\pgfqpoint{2.259879in}{4.052717in}}%
\pgfpathlineto{\pgfqpoint{2.260216in}{4.053607in}}%
\pgfpathlineto{\pgfqpoint{2.261142in}{4.063057in}}%
\pgfpathlineto{\pgfqpoint{2.262659in}{4.082028in}}%
\pgfpathlineto{\pgfqpoint{2.263164in}{4.078821in}}%
\pgfpathlineto{\pgfqpoint{2.265691in}{4.044203in}}%
\pgfpathlineto{\pgfqpoint{2.266787in}{4.045776in}}%
\pgfpathlineto{\pgfqpoint{2.267376in}{4.044600in}}%
\pgfpathlineto{\pgfqpoint{2.268556in}{4.041969in}}%
\pgfpathlineto{\pgfqpoint{2.269061in}{4.042652in}}%
\pgfpathlineto{\pgfqpoint{2.270156in}{4.044355in}}%
\pgfpathlineto{\pgfqpoint{2.270662in}{4.043662in}}%
\pgfpathlineto{\pgfqpoint{2.271673in}{4.042439in}}%
\pgfpathlineto{\pgfqpoint{2.272178in}{4.043003in}}%
\pgfpathlineto{\pgfqpoint{2.272600in}{4.043329in}}%
\pgfpathlineto{\pgfqpoint{2.272937in}{4.042317in}}%
\pgfpathlineto{\pgfqpoint{2.273779in}{4.030018in}}%
\pgfpathlineto{\pgfqpoint{2.276054in}{4.005381in}}%
\pgfpathlineto{\pgfqpoint{2.276559in}{4.003466in}}%
\pgfpathlineto{\pgfqpoint{2.277486in}{3.986619in}}%
\pgfpathlineto{\pgfqpoint{2.279592in}{3.948603in}}%
\pgfpathlineto{\pgfqpoint{2.279929in}{3.948863in}}%
\pgfpathlineto{\pgfqpoint{2.281108in}{3.950449in}}%
\pgfpathlineto{\pgfqpoint{2.281445in}{3.949518in}}%
\pgfpathlineto{\pgfqpoint{2.283636in}{3.943229in}}%
\pgfpathlineto{\pgfqpoint{2.283720in}{3.943250in}}%
\pgfpathlineto{\pgfqpoint{2.288185in}{3.945352in}}%
\pgfpathlineto{\pgfqpoint{2.292229in}{3.942609in}}%
\pgfpathlineto{\pgfqpoint{2.293829in}{3.943584in}}%
\pgfpathlineto{\pgfqpoint{2.294756in}{3.944331in}}%
\pgfpathlineto{\pgfqpoint{2.295177in}{3.943861in}}%
\pgfpathlineto{\pgfqpoint{2.296693in}{3.943719in}}%
\pgfpathlineto{\pgfqpoint{2.297199in}{3.943754in}}%
\pgfpathlineto{\pgfqpoint{2.297452in}{3.943042in}}%
\pgfpathlineto{\pgfqpoint{2.298631in}{3.931585in}}%
\pgfpathlineto{\pgfqpoint{2.299979in}{3.921646in}}%
\pgfpathlineto{\pgfqpoint{2.300484in}{3.922034in}}%
\pgfpathlineto{\pgfqpoint{2.301074in}{3.921491in}}%
\pgfpathlineto{\pgfqpoint{2.301158in}{3.921275in}}%
\pgfpathlineto{\pgfqpoint{2.303265in}{3.912706in}}%
\pgfpathlineto{\pgfqpoint{2.304865in}{3.913735in}}%
\pgfpathlineto{\pgfqpoint{2.305960in}{3.914653in}}%
\pgfpathlineto{\pgfqpoint{2.306971in}{3.921284in}}%
\pgfpathlineto{\pgfqpoint{2.308993in}{3.928320in}}%
\pgfpathlineto{\pgfqpoint{2.310425in}{3.929292in}}%
\pgfpathlineto{\pgfqpoint{2.313121in}{3.935181in}}%
\pgfpathlineto{\pgfqpoint{2.313374in}{3.935045in}}%
\pgfpathlineto{\pgfqpoint{2.315143in}{3.934905in}}%
\pgfpathlineto{\pgfqpoint{2.318260in}{3.939045in}}%
\pgfpathlineto{\pgfqpoint{2.319355in}{3.942052in}}%
\pgfpathlineto{\pgfqpoint{2.319861in}{3.941040in}}%
\pgfpathlineto{\pgfqpoint{2.320535in}{3.939957in}}%
\pgfpathlineto{\pgfqpoint{2.320956in}{3.941017in}}%
\pgfpathlineto{\pgfqpoint{2.322472in}{3.948679in}}%
\pgfpathlineto{\pgfqpoint{2.323146in}{3.946659in}}%
\pgfpathlineto{\pgfqpoint{2.323989in}{3.942623in}}%
\pgfpathlineto{\pgfqpoint{2.324494in}{3.945279in}}%
\pgfpathlineto{\pgfqpoint{2.326685in}{3.971685in}}%
\pgfpathlineto{\pgfqpoint{2.327611in}{3.969580in}}%
\pgfpathlineto{\pgfqpoint{2.327780in}{3.969587in}}%
\pgfpathlineto{\pgfqpoint{2.328032in}{3.970481in}}%
\pgfpathlineto{\pgfqpoint{2.329128in}{3.985142in}}%
\pgfpathlineto{\pgfqpoint{2.330728in}{3.997413in}}%
\pgfpathlineto{\pgfqpoint{2.330981in}{3.997255in}}%
\pgfpathlineto{\pgfqpoint{2.335699in}{3.989211in}}%
\pgfpathlineto{\pgfqpoint{2.336120in}{3.990452in}}%
\pgfpathlineto{\pgfqpoint{2.337552in}{3.996539in}}%
\pgfpathlineto{\pgfqpoint{2.338058in}{3.995186in}}%
\pgfpathlineto{\pgfqpoint{2.338984in}{3.990177in}}%
\pgfpathlineto{\pgfqpoint{2.339574in}{3.992585in}}%
\pgfpathlineto{\pgfqpoint{2.340416in}{3.995732in}}%
\pgfpathlineto{\pgfqpoint{2.340922in}{3.994180in}}%
\pgfpathlineto{\pgfqpoint{2.341933in}{3.977760in}}%
\pgfpathlineto{\pgfqpoint{2.343196in}{3.965314in}}%
\pgfpathlineto{\pgfqpoint{2.343533in}{3.965969in}}%
\pgfpathlineto{\pgfqpoint{2.344292in}{3.968404in}}%
\pgfpathlineto{\pgfqpoint{2.344713in}{3.966541in}}%
\pgfpathlineto{\pgfqpoint{2.347072in}{3.939057in}}%
\pgfpathlineto{\pgfqpoint{2.348588in}{3.939871in}}%
\pgfpathlineto{\pgfqpoint{2.352379in}{3.939734in}}%
\pgfpathlineto{\pgfqpoint{2.357181in}{3.925232in}}%
\pgfpathlineto{\pgfqpoint{2.357939in}{3.928305in}}%
\pgfpathlineto{\pgfqpoint{2.359371in}{3.940616in}}%
\pgfpathlineto{\pgfqpoint{2.360130in}{3.938499in}}%
\pgfpathlineto{\pgfqpoint{2.361141in}{3.935058in}}%
\pgfpathlineto{\pgfqpoint{2.361646in}{3.936500in}}%
\pgfpathlineto{\pgfqpoint{2.363415in}{3.943810in}}%
\pgfpathlineto{\pgfqpoint{2.363921in}{3.942940in}}%
\pgfpathlineto{\pgfqpoint{2.365942in}{3.940719in}}%
\pgfpathlineto{\pgfqpoint{2.370070in}{3.942037in}}%
\pgfpathlineto{\pgfqpoint{2.371924in}{3.947910in}}%
\pgfpathlineto{\pgfqpoint{2.372345in}{3.947684in}}%
\pgfpathlineto{\pgfqpoint{2.374704in}{3.944793in}}%
\pgfpathlineto{\pgfqpoint{2.377989in}{3.929406in}}%
\pgfpathlineto{\pgfqpoint{2.381275in}{3.922945in}}%
\pgfpathlineto{\pgfqpoint{2.381696in}{3.922952in}}%
\pgfpathlineto{\pgfqpoint{2.381949in}{3.923643in}}%
\pgfpathlineto{\pgfqpoint{2.382791in}{3.932582in}}%
\pgfpathlineto{\pgfqpoint{2.385403in}{3.969352in}}%
\pgfpathlineto{\pgfqpoint{2.386077in}{3.969155in}}%
\pgfpathlineto{\pgfqpoint{2.386582in}{3.972137in}}%
\pgfpathlineto{\pgfqpoint{2.389784in}{4.010420in}}%
\pgfpathlineto{\pgfqpoint{2.391047in}{4.009739in}}%
\pgfpathlineto{\pgfqpoint{2.392227in}{4.010283in}}%
\pgfpathlineto{\pgfqpoint{2.393659in}{4.010164in}}%
\pgfpathlineto{\pgfqpoint{2.395007in}{4.011124in}}%
\pgfpathlineto{\pgfqpoint{2.395849in}{4.012056in}}%
\pgfpathlineto{\pgfqpoint{2.396355in}{4.011559in}}%
\pgfpathlineto{\pgfqpoint{2.397955in}{4.008158in}}%
\pgfpathlineto{\pgfqpoint{2.398545in}{4.009938in}}%
\pgfpathlineto{\pgfqpoint{2.399219in}{4.012992in}}%
\pgfpathlineto{\pgfqpoint{2.399725in}{4.010415in}}%
\pgfpathlineto{\pgfqpoint{2.401746in}{3.981759in}}%
\pgfpathlineto{\pgfqpoint{2.402673in}{3.989852in}}%
\pgfpathlineto{\pgfqpoint{2.403516in}{3.999192in}}%
\pgfpathlineto{\pgfqpoint{2.404105in}{3.995675in}}%
\pgfpathlineto{\pgfqpoint{2.406043in}{3.965757in}}%
\pgfpathlineto{\pgfqpoint{2.406970in}{3.972978in}}%
\pgfpathlineto{\pgfqpoint{2.408823in}{3.984307in}}%
\pgfpathlineto{\pgfqpoint{2.409244in}{3.983996in}}%
\pgfpathlineto{\pgfqpoint{2.411013in}{3.984252in}}%
\pgfpathlineto{\pgfqpoint{2.415141in}{3.990688in}}%
\pgfpathlineto{\pgfqpoint{2.415563in}{3.988799in}}%
\pgfpathlineto{\pgfqpoint{2.417500in}{3.975049in}}%
\pgfpathlineto{\pgfqpoint{2.418006in}{3.976621in}}%
\pgfpathlineto{\pgfqpoint{2.419017in}{3.981791in}}%
\pgfpathlineto{\pgfqpoint{2.419522in}{3.979422in}}%
\pgfpathlineto{\pgfqpoint{2.421544in}{3.959394in}}%
\pgfpathlineto{\pgfqpoint{2.422386in}{3.962226in}}%
\pgfpathlineto{\pgfqpoint{2.423060in}{3.964352in}}%
\pgfpathlineto{\pgfqpoint{2.423566in}{3.962677in}}%
\pgfpathlineto{\pgfqpoint{2.425588in}{3.949193in}}%
\pgfpathlineto{\pgfqpoint{2.426430in}{3.952508in}}%
\pgfpathlineto{\pgfqpoint{2.427609in}{3.954964in}}%
\pgfpathlineto{\pgfqpoint{2.427946in}{3.954665in}}%
\pgfpathlineto{\pgfqpoint{2.432580in}{3.945835in}}%
\pgfpathlineto{\pgfqpoint{2.433338in}{3.947147in}}%
\pgfpathlineto{\pgfqpoint{2.434012in}{3.947633in}}%
\pgfpathlineto{\pgfqpoint{2.434433in}{3.946989in}}%
\pgfpathlineto{\pgfqpoint{2.436455in}{3.938492in}}%
\pgfpathlineto{\pgfqpoint{2.437213in}{3.942209in}}%
\pgfpathlineto{\pgfqpoint{2.438477in}{3.948961in}}%
\pgfpathlineto{\pgfqpoint{2.438898in}{3.947981in}}%
\pgfpathlineto{\pgfqpoint{2.440162in}{3.942790in}}%
\pgfpathlineto{\pgfqpoint{2.440667in}{3.944214in}}%
\pgfpathlineto{\pgfqpoint{2.442352in}{3.956483in}}%
\pgfpathlineto{\pgfqpoint{2.443110in}{3.953658in}}%
\pgfpathlineto{\pgfqpoint{2.443784in}{3.951853in}}%
\pgfpathlineto{\pgfqpoint{2.444290in}{3.952914in}}%
\pgfpathlineto{\pgfqpoint{2.445722in}{3.957287in}}%
\pgfpathlineto{\pgfqpoint{2.446312in}{3.956609in}}%
\pgfpathlineto{\pgfqpoint{2.446817in}{3.956377in}}%
\pgfpathlineto{\pgfqpoint{2.447154in}{3.957291in}}%
\pgfpathlineto{\pgfqpoint{2.449597in}{3.969926in}}%
\pgfpathlineto{\pgfqpoint{2.450524in}{3.968204in}}%
\pgfpathlineto{\pgfqpoint{2.453725in}{3.961639in}}%
\pgfpathlineto{\pgfqpoint{2.455831in}{3.953473in}}%
\pgfpathlineto{\pgfqpoint{2.456590in}{3.952518in}}%
\pgfpathlineto{\pgfqpoint{2.457853in}{3.944825in}}%
\pgfpathlineto{\pgfqpoint{2.459538in}{3.935949in}}%
\pgfpathlineto{\pgfqpoint{2.459875in}{3.936095in}}%
\pgfpathlineto{\pgfqpoint{2.460718in}{3.935794in}}%
\pgfpathlineto{\pgfqpoint{2.460886in}{3.935486in}}%
\pgfpathlineto{\pgfqpoint{2.461981in}{3.933390in}}%
\pgfpathlineto{\pgfqpoint{2.462571in}{3.934394in}}%
\pgfpathlineto{\pgfqpoint{2.463413in}{3.935731in}}%
\pgfpathlineto{\pgfqpoint{2.463835in}{3.934883in}}%
\pgfpathlineto{\pgfqpoint{2.466193in}{3.922454in}}%
\pgfpathlineto{\pgfqpoint{2.467120in}{3.924910in}}%
\pgfpathlineto{\pgfqpoint{2.468468in}{3.927567in}}%
\pgfpathlineto{\pgfqpoint{2.468974in}{3.926793in}}%
\pgfpathlineto{\pgfqpoint{2.471922in}{3.922717in}}%
\pgfpathlineto{\pgfqpoint{2.475123in}{3.921515in}}%
\pgfpathlineto{\pgfqpoint{2.475713in}{3.921175in}}%
\pgfpathlineto{\pgfqpoint{2.476219in}{3.921803in}}%
\pgfpathlineto{\pgfqpoint{2.477735in}{3.923506in}}%
\pgfpathlineto{\pgfqpoint{2.478156in}{3.922845in}}%
\pgfpathlineto{\pgfqpoint{2.479588in}{3.919031in}}%
\pgfpathlineto{\pgfqpoint{2.480010in}{3.920612in}}%
\pgfpathlineto{\pgfqpoint{2.481610in}{3.941609in}}%
\pgfpathlineto{\pgfqpoint{2.483127in}{3.959414in}}%
\pgfpathlineto{\pgfqpoint{2.483632in}{3.958075in}}%
\pgfpathlineto{\pgfqpoint{2.484390in}{3.956530in}}%
\pgfpathlineto{\pgfqpoint{2.484812in}{3.957512in}}%
\pgfpathlineto{\pgfqpoint{2.486159in}{3.969756in}}%
\pgfpathlineto{\pgfqpoint{2.487592in}{3.978836in}}%
\pgfpathlineto{\pgfqpoint{2.488013in}{3.978406in}}%
\pgfpathlineto{\pgfqpoint{2.490287in}{3.976829in}}%
\pgfpathlineto{\pgfqpoint{2.491720in}{3.976388in}}%
\pgfpathlineto{\pgfqpoint{2.491888in}{3.976698in}}%
\pgfpathlineto{\pgfqpoint{2.493152in}{3.982695in}}%
\pgfpathlineto{\pgfqpoint{2.495763in}{3.996733in}}%
\pgfpathlineto{\pgfqpoint{2.496437in}{4.001931in}}%
\pgfpathlineto{\pgfqpoint{2.498291in}{4.018275in}}%
\pgfpathlineto{\pgfqpoint{2.498628in}{4.016668in}}%
\pgfpathlineto{\pgfqpoint{2.500060in}{4.002466in}}%
\pgfpathlineto{\pgfqpoint{2.500902in}{4.004521in}}%
\pgfpathlineto{\pgfqpoint{2.501323in}{4.005111in}}%
\pgfpathlineto{\pgfqpoint{2.501660in}{4.004070in}}%
\pgfpathlineto{\pgfqpoint{2.504862in}{3.983774in}}%
\pgfpathlineto{\pgfqpoint{2.505704in}{3.985639in}}%
\pgfpathlineto{\pgfqpoint{2.506968in}{3.995516in}}%
\pgfpathlineto{\pgfqpoint{2.508316in}{4.005755in}}%
\pgfpathlineto{\pgfqpoint{2.508737in}{4.005103in}}%
\pgfpathlineto{\pgfqpoint{2.509748in}{4.002917in}}%
\pgfpathlineto{\pgfqpoint{2.510506in}{4.003235in}}%
\pgfpathlineto{\pgfqpoint{2.511180in}{4.002767in}}%
\pgfpathlineto{\pgfqpoint{2.511601in}{4.003590in}}%
\pgfpathlineto{\pgfqpoint{2.512359in}{4.004828in}}%
\pgfpathlineto{\pgfqpoint{2.512781in}{4.003999in}}%
\pgfpathlineto{\pgfqpoint{2.513623in}{3.994618in}}%
\pgfpathlineto{\pgfqpoint{2.516235in}{3.969832in}}%
\pgfpathlineto{\pgfqpoint{2.517077in}{3.968831in}}%
\pgfpathlineto{\pgfqpoint{2.519015in}{3.965327in}}%
\pgfpathlineto{\pgfqpoint{2.519605in}{3.965751in}}%
\pgfpathlineto{\pgfqpoint{2.520615in}{3.965464in}}%
\pgfpathlineto{\pgfqpoint{2.520784in}{3.965172in}}%
\pgfpathlineto{\pgfqpoint{2.524154in}{3.956991in}}%
\pgfpathlineto{\pgfqpoint{2.524996in}{3.953510in}}%
\pgfpathlineto{\pgfqpoint{2.525586in}{3.954069in}}%
\pgfpathlineto{\pgfqpoint{2.526091in}{3.953728in}}%
\pgfpathlineto{\pgfqpoint{2.526260in}{3.953241in}}%
\pgfpathlineto{\pgfqpoint{2.527187in}{3.944853in}}%
\pgfpathlineto{\pgfqpoint{2.529208in}{3.932748in}}%
\pgfpathlineto{\pgfqpoint{2.530135in}{3.931958in}}%
\pgfpathlineto{\pgfqpoint{2.531230in}{3.925138in}}%
\pgfpathlineto{\pgfqpoint{2.532241in}{3.919667in}}%
\pgfpathlineto{\pgfqpoint{2.532831in}{3.920453in}}%
\pgfpathlineto{\pgfqpoint{2.534347in}{3.920663in}}%
\pgfpathlineto{\pgfqpoint{2.535190in}{3.920434in}}%
\pgfpathlineto{\pgfqpoint{2.535442in}{3.920973in}}%
\pgfpathlineto{\pgfqpoint{2.537212in}{3.923499in}}%
\pgfpathlineto{\pgfqpoint{2.537464in}{3.923375in}}%
\pgfpathlineto{\pgfqpoint{2.538307in}{3.923920in}}%
\pgfpathlineto{\pgfqpoint{2.538391in}{3.924107in}}%
\pgfpathlineto{\pgfqpoint{2.539823in}{3.926759in}}%
\pgfpathlineto{\pgfqpoint{2.540244in}{3.926119in}}%
\pgfpathlineto{\pgfqpoint{2.542266in}{3.920542in}}%
\pgfpathlineto{\pgfqpoint{2.542772in}{3.921531in}}%
\pgfpathlineto{\pgfqpoint{2.545299in}{3.926781in}}%
\pgfpathlineto{\pgfqpoint{2.545383in}{3.926745in}}%
\pgfpathlineto{\pgfqpoint{2.546731in}{3.927082in}}%
\pgfpathlineto{\pgfqpoint{2.548332in}{3.927337in}}%
\pgfpathlineto{\pgfqpoint{2.549764in}{3.927228in}}%
\pgfpathlineto{\pgfqpoint{2.549933in}{3.927827in}}%
\pgfpathlineto{\pgfqpoint{2.552123in}{3.936442in}}%
\pgfpathlineto{\pgfqpoint{2.552713in}{3.935184in}}%
\pgfpathlineto{\pgfqpoint{2.553639in}{3.933825in}}%
\pgfpathlineto{\pgfqpoint{2.554061in}{3.934578in}}%
\pgfpathlineto{\pgfqpoint{2.556251in}{3.942707in}}%
\pgfpathlineto{\pgfqpoint{2.556925in}{3.941707in}}%
\pgfpathlineto{\pgfqpoint{2.558525in}{3.940642in}}%
\pgfpathlineto{\pgfqpoint{2.559536in}{3.940004in}}%
\pgfpathlineto{\pgfqpoint{2.563243in}{3.933365in}}%
\pgfpathlineto{\pgfqpoint{2.564001in}{3.934255in}}%
\pgfpathlineto{\pgfqpoint{2.565097in}{3.941212in}}%
\pgfpathlineto{\pgfqpoint{2.566276in}{3.947006in}}%
\pgfpathlineto{\pgfqpoint{2.566697in}{3.946313in}}%
\pgfpathlineto{\pgfqpoint{2.570657in}{3.936788in}}%
\pgfpathlineto{\pgfqpoint{2.571752in}{3.931249in}}%
\pgfpathlineto{\pgfqpoint{2.573858in}{3.922532in}}%
\pgfpathlineto{\pgfqpoint{2.574111in}{3.922745in}}%
\pgfpathlineto{\pgfqpoint{2.577733in}{3.926352in}}%
\pgfpathlineto{\pgfqpoint{2.580850in}{3.923652in}}%
\pgfpathlineto{\pgfqpoint{2.582367in}{3.912995in}}%
\pgfpathlineto{\pgfqpoint{2.584136in}{3.908332in}}%
\pgfpathlineto{\pgfqpoint{2.587253in}{3.908566in}}%
\pgfpathlineto{\pgfqpoint{2.588685in}{3.908504in}}%
\pgfpathlineto{\pgfqpoint{2.589780in}{3.909480in}}%
\pgfpathlineto{\pgfqpoint{2.591634in}{3.913010in}}%
\pgfpathlineto{\pgfqpoint{2.592308in}{3.912593in}}%
\pgfpathlineto{\pgfqpoint{2.594835in}{3.911454in}}%
\pgfpathlineto{\pgfqpoint{2.600227in}{3.916127in}}%
\pgfpathlineto{\pgfqpoint{2.604607in}{3.937048in}}%
\pgfpathlineto{\pgfqpoint{2.606376in}{3.948524in}}%
\pgfpathlineto{\pgfqpoint{2.606713in}{3.948054in}}%
\pgfpathlineto{\pgfqpoint{2.609578in}{3.943547in}}%
\pgfpathlineto{\pgfqpoint{2.612021in}{3.943527in}}%
\pgfpathlineto{\pgfqpoint{2.612863in}{3.947267in}}%
\pgfpathlineto{\pgfqpoint{2.614464in}{3.956460in}}%
\pgfpathlineto{\pgfqpoint{2.614969in}{3.955990in}}%
\pgfpathlineto{\pgfqpoint{2.615391in}{3.956039in}}%
\pgfpathlineto{\pgfqpoint{2.615643in}{3.956731in}}%
\pgfpathlineto{\pgfqpoint{2.617075in}{3.967791in}}%
\pgfpathlineto{\pgfqpoint{2.618002in}{3.972570in}}%
\pgfpathlineto{\pgfqpoint{2.618508in}{3.970789in}}%
\pgfpathlineto{\pgfqpoint{2.619350in}{3.968229in}}%
\pgfpathlineto{\pgfqpoint{2.619940in}{3.968863in}}%
\pgfpathlineto{\pgfqpoint{2.620445in}{3.969323in}}%
\pgfpathlineto{\pgfqpoint{2.620782in}{3.968366in}}%
\pgfpathlineto{\pgfqpoint{2.623394in}{3.951790in}}%
\pgfpathlineto{\pgfqpoint{2.624742in}{3.952108in}}%
\pgfpathlineto{\pgfqpoint{2.628785in}{3.950981in}}%
\pgfpathlineto{\pgfqpoint{2.629712in}{3.947320in}}%
\pgfpathlineto{\pgfqpoint{2.630723in}{3.943836in}}%
\pgfpathlineto{\pgfqpoint{2.631313in}{3.944305in}}%
\pgfpathlineto{\pgfqpoint{2.632071in}{3.944803in}}%
\pgfpathlineto{\pgfqpoint{2.632408in}{3.944160in}}%
\pgfpathlineto{\pgfqpoint{2.634683in}{3.932527in}}%
\pgfpathlineto{\pgfqpoint{2.635693in}{3.936226in}}%
\pgfpathlineto{\pgfqpoint{2.636367in}{3.937359in}}%
\pgfpathlineto{\pgfqpoint{2.636873in}{3.936671in}}%
\pgfpathlineto{\pgfqpoint{2.638052in}{3.933153in}}%
\pgfpathlineto{\pgfqpoint{2.638726in}{3.934622in}}%
\pgfpathlineto{\pgfqpoint{2.641254in}{3.944142in}}%
\pgfpathlineto{\pgfqpoint{2.642096in}{3.943894in}}%
\pgfpathlineto{\pgfqpoint{2.643023in}{3.945097in}}%
\pgfpathlineto{\pgfqpoint{2.645466in}{3.950672in}}%
\pgfpathlineto{\pgfqpoint{2.646140in}{3.950085in}}%
\pgfpathlineto{\pgfqpoint{2.647151in}{3.947106in}}%
\pgfpathlineto{\pgfqpoint{2.648583in}{3.944627in}}%
\pgfpathlineto{\pgfqpoint{2.648751in}{3.944673in}}%
\pgfpathlineto{\pgfqpoint{2.650099in}{3.944486in}}%
\pgfpathlineto{\pgfqpoint{2.650184in}{3.944222in}}%
\pgfpathlineto{\pgfqpoint{2.652458in}{3.932728in}}%
\pgfpathlineto{\pgfqpoint{2.653975in}{3.934233in}}%
\pgfpathlineto{\pgfqpoint{2.655154in}{3.933142in}}%
\pgfpathlineto{\pgfqpoint{2.655575in}{3.933039in}}%
\pgfpathlineto{\pgfqpoint{2.656165in}{3.933596in}}%
\pgfpathlineto{\pgfqpoint{2.656839in}{3.932308in}}%
\pgfpathlineto{\pgfqpoint{2.658271in}{3.930522in}}%
\pgfpathlineto{\pgfqpoint{2.658524in}{3.930736in}}%
\pgfpathlineto{\pgfqpoint{2.660209in}{3.933371in}}%
\pgfpathlineto{\pgfqpoint{2.660798in}{3.932346in}}%
\pgfpathlineto{\pgfqpoint{2.661978in}{3.930212in}}%
\pgfpathlineto{\pgfqpoint{2.662399in}{3.931044in}}%
\pgfpathlineto{\pgfqpoint{2.664421in}{3.938176in}}%
\pgfpathlineto{\pgfqpoint{2.665263in}{3.937404in}}%
\pgfpathlineto{\pgfqpoint{2.666021in}{3.937137in}}%
\pgfpathlineto{\pgfqpoint{2.666358in}{3.937721in}}%
\pgfpathlineto{\pgfqpoint{2.668043in}{3.939482in}}%
\pgfpathlineto{\pgfqpoint{2.668212in}{3.939379in}}%
\pgfpathlineto{\pgfqpoint{2.669139in}{3.936924in}}%
\pgfpathlineto{\pgfqpoint{2.671160in}{3.932929in}}%
\pgfpathlineto{\pgfqpoint{2.672087in}{3.931987in}}%
\pgfpathlineto{\pgfqpoint{2.674362in}{3.922879in}}%
\pgfpathlineto{\pgfqpoint{2.675288in}{3.923905in}}%
\pgfpathlineto{\pgfqpoint{2.675962in}{3.922853in}}%
\pgfpathlineto{\pgfqpoint{2.677647in}{3.918491in}}%
\pgfpathlineto{\pgfqpoint{2.678068in}{3.919318in}}%
\pgfpathlineto{\pgfqpoint{2.679164in}{3.922071in}}%
\pgfpathlineto{\pgfqpoint{2.679753in}{3.921262in}}%
\pgfpathlineto{\pgfqpoint{2.684134in}{3.914092in}}%
\pgfpathlineto{\pgfqpoint{2.685903in}{3.914535in}}%
\pgfpathlineto{\pgfqpoint{2.688599in}{3.917900in}}%
\pgfpathlineto{\pgfqpoint{2.690958in}{3.925681in}}%
\pgfpathlineto{\pgfqpoint{2.691295in}{3.925473in}}%
\pgfpathlineto{\pgfqpoint{2.693064in}{3.924150in}}%
\pgfpathlineto{\pgfqpoint{2.693485in}{3.924606in}}%
\pgfpathlineto{\pgfqpoint{2.694412in}{3.924961in}}%
\pgfpathlineto{\pgfqpoint{2.694749in}{3.924381in}}%
\pgfpathlineto{\pgfqpoint{2.697445in}{3.917615in}}%
\pgfpathlineto{\pgfqpoint{2.698203in}{3.918037in}}%
\pgfpathlineto{\pgfqpoint{2.704521in}{3.917643in}}%
\pgfpathlineto{\pgfqpoint{2.705532in}{3.914541in}}%
\pgfpathlineto{\pgfqpoint{2.708228in}{3.907095in}}%
\pgfpathlineto{\pgfqpoint{2.713788in}{3.908824in}}%
\pgfpathlineto{\pgfqpoint{2.716063in}{3.911513in}}%
\pgfpathlineto{\pgfqpoint{2.717074in}{3.912825in}}%
\pgfpathlineto{\pgfqpoint{2.724150in}{3.932537in}}%
\pgfpathlineto{\pgfqpoint{2.725835in}{3.933236in}}%
\pgfpathlineto{\pgfqpoint{2.728194in}{3.934437in}}%
\pgfpathlineto{\pgfqpoint{2.729963in}{3.933649in}}%
\pgfpathlineto{\pgfqpoint{2.732069in}{3.931222in}}%
\pgfpathlineto{\pgfqpoint{2.732659in}{3.931849in}}%
\pgfpathlineto{\pgfqpoint{2.733501in}{3.932556in}}%
\pgfpathlineto{\pgfqpoint{2.733923in}{3.931718in}}%
\pgfpathlineto{\pgfqpoint{2.735607in}{3.926671in}}%
\pgfpathlineto{\pgfqpoint{2.736113in}{3.928202in}}%
\pgfpathlineto{\pgfqpoint{2.737629in}{3.934656in}}%
\pgfpathlineto{\pgfqpoint{2.738135in}{3.933690in}}%
\pgfpathlineto{\pgfqpoint{2.739146in}{3.930236in}}%
\pgfpathlineto{\pgfqpoint{2.739735in}{3.931903in}}%
\pgfpathlineto{\pgfqpoint{2.742179in}{3.946387in}}%
\pgfpathlineto{\pgfqpoint{2.743021in}{3.945353in}}%
\pgfpathlineto{\pgfqpoint{2.743442in}{3.945133in}}%
\pgfpathlineto{\pgfqpoint{2.743779in}{3.945901in}}%
\pgfpathlineto{\pgfqpoint{2.749424in}{3.967845in}}%
\pgfpathlineto{\pgfqpoint{2.749592in}{3.967735in}}%
\pgfpathlineto{\pgfqpoint{2.750350in}{3.967013in}}%
\pgfpathlineto{\pgfqpoint{2.750687in}{3.967824in}}%
\pgfpathlineto{\pgfqpoint{2.752456in}{3.978582in}}%
\pgfpathlineto{\pgfqpoint{2.753215in}{3.974692in}}%
\pgfpathlineto{\pgfqpoint{2.754057in}{3.970937in}}%
\pgfpathlineto{\pgfqpoint{2.754562in}{3.972098in}}%
\pgfpathlineto{\pgfqpoint{2.755742in}{3.977255in}}%
\pgfpathlineto{\pgfqpoint{2.756247in}{3.975580in}}%
\pgfpathlineto{\pgfqpoint{2.758101in}{3.964495in}}%
\pgfpathlineto{\pgfqpoint{2.758775in}{3.965982in}}%
\pgfpathlineto{\pgfqpoint{2.759701in}{3.968002in}}%
\pgfpathlineto{\pgfqpoint{2.760291in}{3.967589in}}%
\pgfpathlineto{\pgfqpoint{2.761134in}{3.967699in}}%
\pgfpathlineto{\pgfqpoint{2.761302in}{3.968094in}}%
\pgfpathlineto{\pgfqpoint{2.764082in}{3.973367in}}%
\pgfpathlineto{\pgfqpoint{2.767368in}{3.973985in}}%
\pgfpathlineto{\pgfqpoint{2.767873in}{3.974807in}}%
\pgfpathlineto{\pgfqpoint{2.768379in}{3.973567in}}%
\pgfpathlineto{\pgfqpoint{2.769811in}{3.968215in}}%
\pgfpathlineto{\pgfqpoint{2.770400in}{3.969582in}}%
\pgfpathlineto{\pgfqpoint{2.775118in}{3.986930in}}%
\pgfpathlineto{\pgfqpoint{2.776803in}{3.989398in}}%
\pgfpathlineto{\pgfqpoint{2.777561in}{3.988272in}}%
\pgfpathlineto{\pgfqpoint{2.778825in}{3.982147in}}%
\pgfpathlineto{\pgfqpoint{2.779499in}{3.985179in}}%
\pgfpathlineto{\pgfqpoint{2.780931in}{3.992324in}}%
\pgfpathlineto{\pgfqpoint{2.781352in}{3.991176in}}%
\pgfpathlineto{\pgfqpoint{2.782700in}{3.983231in}}%
\pgfpathlineto{\pgfqpoint{2.783458in}{3.985435in}}%
\pgfpathlineto{\pgfqpoint{2.784048in}{3.986487in}}%
\pgfpathlineto{\pgfqpoint{2.784554in}{3.985356in}}%
\pgfpathlineto{\pgfqpoint{2.788176in}{3.967690in}}%
\pgfpathlineto{\pgfqpoint{2.789524in}{3.957984in}}%
\pgfpathlineto{\pgfqpoint{2.789945in}{3.958313in}}%
\pgfpathlineto{\pgfqpoint{2.790366in}{3.958114in}}%
\pgfpathlineto{\pgfqpoint{2.790535in}{3.957610in}}%
\pgfpathlineto{\pgfqpoint{2.791714in}{3.948099in}}%
\pgfpathlineto{\pgfqpoint{2.793652in}{3.937282in}}%
\pgfpathlineto{\pgfqpoint{2.793820in}{3.937300in}}%
\pgfpathlineto{\pgfqpoint{2.794579in}{3.936401in}}%
\pgfpathlineto{\pgfqpoint{2.795674in}{3.930404in}}%
\pgfpathlineto{\pgfqpoint{2.798370in}{3.912531in}}%
\pgfpathlineto{\pgfqpoint{2.799549in}{3.913597in}}%
\pgfpathlineto{\pgfqpoint{2.800476in}{3.914465in}}%
\pgfpathlineto{\pgfqpoint{2.800897in}{3.913877in}}%
\pgfpathlineto{\pgfqpoint{2.802245in}{3.911302in}}%
\pgfpathlineto{\pgfqpoint{2.802666in}{3.912264in}}%
\pgfpathlineto{\pgfqpoint{2.807047in}{3.928691in}}%
\pgfpathlineto{\pgfqpoint{2.809658in}{3.931614in}}%
\pgfpathlineto{\pgfqpoint{2.813112in}{3.945049in}}%
\pgfpathlineto{\pgfqpoint{2.815050in}{3.958943in}}%
\pgfpathlineto{\pgfqpoint{2.815556in}{3.958823in}}%
\pgfpathlineto{\pgfqpoint{2.816651in}{3.957232in}}%
\pgfpathlineto{\pgfqpoint{2.818588in}{3.954276in}}%
\pgfpathlineto{\pgfqpoint{2.818925in}{3.954433in}}%
\pgfpathlineto{\pgfqpoint{2.819431in}{3.954083in}}%
\pgfpathlineto{\pgfqpoint{2.819515in}{3.953819in}}%
\pgfpathlineto{\pgfqpoint{2.820610in}{3.944554in}}%
\pgfpathlineto{\pgfqpoint{2.821621in}{3.940532in}}%
\pgfpathlineto{\pgfqpoint{2.822042in}{3.941465in}}%
\pgfpathlineto{\pgfqpoint{2.823475in}{3.947418in}}%
\pgfpathlineto{\pgfqpoint{2.824148in}{3.946977in}}%
\pgfpathlineto{\pgfqpoint{2.824822in}{3.946503in}}%
\pgfpathlineto{\pgfqpoint{2.825159in}{3.947334in}}%
\pgfpathlineto{\pgfqpoint{2.826844in}{3.952125in}}%
\pgfpathlineto{\pgfqpoint{2.827266in}{3.951591in}}%
\pgfpathlineto{\pgfqpoint{2.828024in}{3.950474in}}%
\pgfpathlineto{\pgfqpoint{2.828529in}{3.951358in}}%
\pgfpathlineto{\pgfqpoint{2.828866in}{3.951517in}}%
\pgfpathlineto{\pgfqpoint{2.829372in}{3.950719in}}%
\pgfpathlineto{\pgfqpoint{2.830298in}{3.946347in}}%
\pgfpathlineto{\pgfqpoint{2.831478in}{3.937202in}}%
\pgfpathlineto{\pgfqpoint{2.832067in}{3.938989in}}%
\pgfpathlineto{\pgfqpoint{2.840913in}{3.978223in}}%
\pgfpathlineto{\pgfqpoint{2.841503in}{3.979306in}}%
\pgfpathlineto{\pgfqpoint{2.845125in}{3.993281in}}%
\pgfpathlineto{\pgfqpoint{2.847990in}{3.998740in}}%
\pgfpathlineto{\pgfqpoint{2.848495in}{3.997943in}}%
\pgfpathlineto{\pgfqpoint{2.849590in}{3.992060in}}%
\pgfpathlineto{\pgfqpoint{2.850770in}{3.983714in}}%
\pgfpathlineto{\pgfqpoint{2.851444in}{3.984307in}}%
\pgfpathlineto{\pgfqpoint{2.852033in}{3.983372in}}%
\pgfpathlineto{\pgfqpoint{2.853044in}{3.974156in}}%
\pgfpathlineto{\pgfqpoint{2.853803in}{3.969238in}}%
\pgfpathlineto{\pgfqpoint{2.854392in}{3.971008in}}%
\pgfpathlineto{\pgfqpoint{2.855150in}{3.974114in}}%
\pgfpathlineto{\pgfqpoint{2.855572in}{3.972423in}}%
\pgfpathlineto{\pgfqpoint{2.857172in}{3.959136in}}%
\pgfpathlineto{\pgfqpoint{2.857846in}{3.962461in}}%
\pgfpathlineto{\pgfqpoint{2.858436in}{3.965702in}}%
\pgfpathlineto{\pgfqpoint{2.858857in}{3.963069in}}%
\pgfpathlineto{\pgfqpoint{2.860626in}{3.946384in}}%
\pgfpathlineto{\pgfqpoint{2.861132in}{3.948141in}}%
\pgfpathlineto{\pgfqpoint{2.861890in}{3.951155in}}%
\pgfpathlineto{\pgfqpoint{2.862395in}{3.949516in}}%
\pgfpathlineto{\pgfqpoint{2.865428in}{3.937585in}}%
\pgfpathlineto{\pgfqpoint{2.868461in}{3.938193in}}%
\pgfpathlineto{\pgfqpoint{2.869472in}{3.944176in}}%
\pgfpathlineto{\pgfqpoint{2.870567in}{3.954595in}}%
\pgfpathlineto{\pgfqpoint{2.871241in}{3.951915in}}%
\pgfpathlineto{\pgfqpoint{2.872168in}{3.947998in}}%
\pgfpathlineto{\pgfqpoint{2.872589in}{3.949728in}}%
\pgfpathlineto{\pgfqpoint{2.874358in}{3.967413in}}%
\pgfpathlineto{\pgfqpoint{2.875032in}{3.963413in}}%
\pgfpathlineto{\pgfqpoint{2.875790in}{3.959910in}}%
\pgfpathlineto{\pgfqpoint{2.876296in}{3.961489in}}%
\pgfpathlineto{\pgfqpoint{2.877896in}{3.970961in}}%
\pgfpathlineto{\pgfqpoint{2.878655in}{3.968811in}}%
\pgfpathlineto{\pgfqpoint{2.879666in}{3.967333in}}%
\pgfpathlineto{\pgfqpoint{2.880087in}{3.967917in}}%
\pgfpathlineto{\pgfqpoint{2.882109in}{3.969857in}}%
\pgfpathlineto{\pgfqpoint{2.885394in}{3.970937in}}%
\pgfpathlineto{\pgfqpoint{2.886995in}{3.978614in}}%
\pgfpathlineto{\pgfqpoint{2.888090in}{3.978265in}}%
\pgfpathlineto{\pgfqpoint{2.888680in}{3.979131in}}%
\pgfpathlineto{\pgfqpoint{2.889943in}{3.991178in}}%
\pgfpathlineto{\pgfqpoint{2.893061in}{4.009224in}}%
\pgfpathlineto{\pgfqpoint{2.893903in}{4.014512in}}%
\pgfpathlineto{\pgfqpoint{2.895419in}{4.017521in}}%
\pgfpathlineto{\pgfqpoint{2.896430in}{4.018544in}}%
\pgfpathlineto{\pgfqpoint{2.899379in}{4.027708in}}%
\pgfpathlineto{\pgfqpoint{2.899969in}{4.030088in}}%
\pgfpathlineto{\pgfqpoint{2.901569in}{4.039198in}}%
\pgfpathlineto{\pgfqpoint{2.902075in}{4.037533in}}%
\pgfpathlineto{\pgfqpoint{2.904012in}{4.020419in}}%
\pgfpathlineto{\pgfqpoint{2.904939in}{4.024957in}}%
\pgfpathlineto{\pgfqpoint{2.905360in}{4.026084in}}%
\pgfpathlineto{\pgfqpoint{2.905781in}{4.023700in}}%
\pgfpathlineto{\pgfqpoint{2.908225in}{3.985113in}}%
\pgfpathlineto{\pgfqpoint{2.909657in}{3.987694in}}%
\pgfpathlineto{\pgfqpoint{2.910499in}{3.979936in}}%
\pgfpathlineto{\pgfqpoint{2.911847in}{3.969567in}}%
\pgfpathlineto{\pgfqpoint{2.912268in}{3.970083in}}%
\pgfpathlineto{\pgfqpoint{2.912858in}{3.970958in}}%
\pgfpathlineto{\pgfqpoint{2.913279in}{3.969915in}}%
\pgfpathlineto{\pgfqpoint{2.915217in}{3.960019in}}%
\pgfpathlineto{\pgfqpoint{2.915975in}{3.960740in}}%
\pgfpathlineto{\pgfqpoint{2.916396in}{3.960368in}}%
\pgfpathlineto{\pgfqpoint{2.916480in}{3.960082in}}%
\pgfpathlineto{\pgfqpoint{2.918587in}{3.948773in}}%
\pgfpathlineto{\pgfqpoint{2.919513in}{3.951459in}}%
\pgfpathlineto{\pgfqpoint{2.919766in}{3.951482in}}%
\pgfpathlineto{\pgfqpoint{2.920019in}{3.950796in}}%
\pgfpathlineto{\pgfqpoint{2.921282in}{3.940542in}}%
\pgfpathlineto{\pgfqpoint{2.921956in}{3.937513in}}%
\pgfpathlineto{\pgfqpoint{2.922462in}{3.939743in}}%
\pgfpathlineto{\pgfqpoint{2.924147in}{3.951135in}}%
\pgfpathlineto{\pgfqpoint{2.924568in}{3.948951in}}%
\pgfpathlineto{\pgfqpoint{2.926000in}{3.938767in}}%
\pgfpathlineto{\pgfqpoint{2.926506in}{3.940057in}}%
\pgfpathlineto{\pgfqpoint{2.927432in}{3.943676in}}%
\pgfpathlineto{\pgfqpoint{2.928022in}{3.942493in}}%
\pgfpathlineto{\pgfqpoint{2.929623in}{3.938230in}}%
\pgfpathlineto{\pgfqpoint{2.930465in}{3.939114in}}%
\pgfpathlineto{\pgfqpoint{2.931981in}{3.938620in}}%
\pgfpathlineto{\pgfqpoint{2.933329in}{3.937792in}}%
\pgfpathlineto{\pgfqpoint{2.933666in}{3.938145in}}%
\pgfpathlineto{\pgfqpoint{2.935351in}{3.941130in}}%
\pgfpathlineto{\pgfqpoint{2.935941in}{3.939732in}}%
\pgfpathlineto{\pgfqpoint{2.936699in}{3.937728in}}%
\pgfpathlineto{\pgfqpoint{2.937289in}{3.938737in}}%
\pgfpathlineto{\pgfqpoint{2.938553in}{3.942151in}}%
\pgfpathlineto{\pgfqpoint{2.938974in}{3.940756in}}%
\pgfpathlineto{\pgfqpoint{2.941417in}{3.922897in}}%
\pgfpathlineto{\pgfqpoint{2.942596in}{3.924515in}}%
\pgfpathlineto{\pgfqpoint{2.943186in}{3.923390in}}%
\pgfpathlineto{\pgfqpoint{2.945629in}{3.917550in}}%
\pgfpathlineto{\pgfqpoint{2.947061in}{3.916491in}}%
\pgfpathlineto{\pgfqpoint{2.948999in}{3.915150in}}%
\pgfpathlineto{\pgfqpoint{2.949336in}{3.915599in}}%
\pgfpathlineto{\pgfqpoint{2.950263in}{3.916195in}}%
\pgfpathlineto{\pgfqpoint{2.950684in}{3.915679in}}%
\pgfpathlineto{\pgfqpoint{2.951779in}{3.913210in}}%
\pgfpathlineto{\pgfqpoint{2.952284in}{3.914763in}}%
\pgfpathlineto{\pgfqpoint{2.954391in}{3.924073in}}%
\pgfpathlineto{\pgfqpoint{2.955064in}{3.922806in}}%
\pgfpathlineto{\pgfqpoint{2.955654in}{3.922065in}}%
\pgfpathlineto{\pgfqpoint{2.956075in}{3.923087in}}%
\pgfpathlineto{\pgfqpoint{2.957002in}{3.932677in}}%
\pgfpathlineto{\pgfqpoint{2.959782in}{3.965123in}}%
\pgfpathlineto{\pgfqpoint{2.960035in}{3.965096in}}%
\pgfpathlineto{\pgfqpoint{2.960877in}{3.966161in}}%
\pgfpathlineto{\pgfqpoint{2.961636in}{3.973805in}}%
\pgfpathlineto{\pgfqpoint{2.964416in}{4.004050in}}%
\pgfpathlineto{\pgfqpoint{2.965932in}{4.005350in}}%
\pgfpathlineto{\pgfqpoint{2.968375in}{4.013302in}}%
\pgfpathlineto{\pgfqpoint{2.969133in}{4.011283in}}%
\pgfpathlineto{\pgfqpoint{2.972419in}{4.002891in}}%
\pgfpathlineto{\pgfqpoint{2.973009in}{4.001467in}}%
\pgfpathlineto{\pgfqpoint{2.974019in}{3.989703in}}%
\pgfpathlineto{\pgfqpoint{2.976715in}{3.957772in}}%
\pgfpathlineto{\pgfqpoint{2.976884in}{3.957798in}}%
\pgfpathlineto{\pgfqpoint{2.977726in}{3.957158in}}%
\pgfpathlineto{\pgfqpoint{2.978484in}{3.949600in}}%
\pgfpathlineto{\pgfqpoint{2.981265in}{3.919158in}}%
\pgfpathlineto{\pgfqpoint{2.982781in}{3.917745in}}%
\pgfpathlineto{\pgfqpoint{2.985140in}{3.908551in}}%
\pgfpathlineto{\pgfqpoint{2.985982in}{3.909561in}}%
\pgfpathlineto{\pgfqpoint{2.986403in}{3.909865in}}%
\pgfpathlineto{\pgfqpoint{2.986825in}{3.909034in}}%
\pgfpathlineto{\pgfqpoint{2.988341in}{3.907190in}}%
\pgfpathlineto{\pgfqpoint{2.988510in}{3.907332in}}%
\pgfpathlineto{\pgfqpoint{2.995755in}{3.915858in}}%
\pgfpathlineto{\pgfqpoint{2.997355in}{3.916867in}}%
\pgfpathlineto{\pgfqpoint{2.999377in}{3.922851in}}%
\pgfpathlineto{\pgfqpoint{3.001483in}{3.926274in}}%
\pgfpathlineto{\pgfqpoint{3.002663in}{3.927097in}}%
\pgfpathlineto{\pgfqpoint{3.004011in}{3.930517in}}%
\pgfpathlineto{\pgfqpoint{3.005358in}{3.933622in}}%
\pgfpathlineto{\pgfqpoint{3.005695in}{3.933134in}}%
\pgfpathlineto{\pgfqpoint{3.006959in}{3.931141in}}%
\pgfpathlineto{\pgfqpoint{3.007465in}{3.931408in}}%
\pgfpathlineto{\pgfqpoint{3.008307in}{3.933772in}}%
\pgfpathlineto{\pgfqpoint{3.009908in}{3.935962in}}%
\pgfpathlineto{\pgfqpoint{3.011593in}{3.935039in}}%
\pgfpathlineto{\pgfqpoint{3.012856in}{3.936026in}}%
\pgfpathlineto{\pgfqpoint{3.014288in}{3.943474in}}%
\pgfpathlineto{\pgfqpoint{3.015552in}{3.951282in}}%
\pgfpathlineto{\pgfqpoint{3.016226in}{3.949461in}}%
\pgfpathlineto{\pgfqpoint{3.017321in}{3.946961in}}%
\pgfpathlineto{\pgfqpoint{3.017742in}{3.947983in}}%
\pgfpathlineto{\pgfqpoint{3.019259in}{3.958210in}}%
\pgfpathlineto{\pgfqpoint{3.020185in}{3.955850in}}%
\pgfpathlineto{\pgfqpoint{3.021196in}{3.952288in}}%
\pgfpathlineto{\pgfqpoint{3.021786in}{3.953587in}}%
\pgfpathlineto{\pgfqpoint{3.023471in}{3.960697in}}%
\pgfpathlineto{\pgfqpoint{3.024482in}{3.959616in}}%
\pgfpathlineto{\pgfqpoint{3.025409in}{3.958199in}}%
\pgfpathlineto{\pgfqpoint{3.025830in}{3.959287in}}%
\pgfpathlineto{\pgfqpoint{3.026925in}{3.962572in}}%
\pgfpathlineto{\pgfqpoint{3.027515in}{3.961752in}}%
\pgfpathlineto{\pgfqpoint{3.028526in}{3.961338in}}%
\pgfpathlineto{\pgfqpoint{3.028778in}{3.961660in}}%
\pgfpathlineto{\pgfqpoint{3.029874in}{3.963791in}}%
\pgfpathlineto{\pgfqpoint{3.030379in}{3.962861in}}%
\pgfpathlineto{\pgfqpoint{3.031474in}{3.956616in}}%
\pgfpathlineto{\pgfqpoint{3.032569in}{3.949652in}}%
\pgfpathlineto{\pgfqpoint{3.033159in}{3.951275in}}%
\pgfpathlineto{\pgfqpoint{3.034254in}{3.956218in}}%
\pgfpathlineto{\pgfqpoint{3.034760in}{3.954532in}}%
\pgfpathlineto{\pgfqpoint{3.036445in}{3.944425in}}%
\pgfpathlineto{\pgfqpoint{3.037119in}{3.945358in}}%
\pgfpathlineto{\pgfqpoint{3.037793in}{3.946143in}}%
\pgfpathlineto{\pgfqpoint{3.038214in}{3.945458in}}%
\pgfpathlineto{\pgfqpoint{3.039646in}{3.937675in}}%
\pgfpathlineto{\pgfqpoint{3.040825in}{3.935239in}}%
\pgfpathlineto{\pgfqpoint{3.041162in}{3.935316in}}%
\pgfpathlineto{\pgfqpoint{3.042005in}{3.934479in}}%
\pgfpathlineto{\pgfqpoint{3.044616in}{3.925663in}}%
\pgfpathlineto{\pgfqpoint{3.045543in}{3.927821in}}%
\pgfpathlineto{\pgfqpoint{3.047144in}{3.931474in}}%
\pgfpathlineto{\pgfqpoint{3.047649in}{3.930448in}}%
\pgfpathlineto{\pgfqpoint{3.048155in}{3.929760in}}%
\pgfpathlineto{\pgfqpoint{3.048660in}{3.930589in}}%
\pgfpathlineto{\pgfqpoint{3.049503in}{3.935208in}}%
\pgfpathlineto{\pgfqpoint{3.055315in}{3.995075in}}%
\pgfpathlineto{\pgfqpoint{3.055568in}{3.994961in}}%
\pgfpathlineto{\pgfqpoint{3.056158in}{3.995142in}}%
\pgfpathlineto{\pgfqpoint{3.056326in}{3.995515in}}%
\pgfpathlineto{\pgfqpoint{3.058769in}{4.003792in}}%
\pgfpathlineto{\pgfqpoint{3.060033in}{4.003063in}}%
\pgfpathlineto{\pgfqpoint{3.062139in}{4.001605in}}%
\pgfpathlineto{\pgfqpoint{3.066857in}{3.980160in}}%
\pgfpathlineto{\pgfqpoint{3.068626in}{3.958213in}}%
\pgfpathlineto{\pgfqpoint{3.068963in}{3.958526in}}%
\pgfpathlineto{\pgfqpoint{3.069384in}{3.958921in}}%
\pgfpathlineto{\pgfqpoint{3.069637in}{3.958132in}}%
\pgfpathlineto{\pgfqpoint{3.070732in}{3.943806in}}%
\pgfpathlineto{\pgfqpoint{3.072249in}{3.930964in}}%
\pgfpathlineto{\pgfqpoint{3.072670in}{3.931640in}}%
\pgfpathlineto{\pgfqpoint{3.073344in}{3.932561in}}%
\pgfpathlineto{\pgfqpoint{3.073934in}{3.931875in}}%
\pgfpathlineto{\pgfqpoint{3.076798in}{3.929181in}}%
\pgfpathlineto{\pgfqpoint{3.077051in}{3.929462in}}%
\pgfpathlineto{\pgfqpoint{3.080168in}{3.935998in}}%
\pgfpathlineto{\pgfqpoint{3.081852in}{3.939400in}}%
\pgfpathlineto{\pgfqpoint{3.082442in}{3.940630in}}%
\pgfpathlineto{\pgfqpoint{3.085307in}{3.953538in}}%
\pgfpathlineto{\pgfqpoint{3.085896in}{3.953183in}}%
\pgfpathlineto{\pgfqpoint{3.086991in}{3.950917in}}%
\pgfpathlineto{\pgfqpoint{3.089182in}{3.947288in}}%
\pgfpathlineto{\pgfqpoint{3.090024in}{3.945196in}}%
\pgfpathlineto{\pgfqpoint{3.091456in}{3.941333in}}%
\pgfpathlineto{\pgfqpoint{3.091878in}{3.942517in}}%
\pgfpathlineto{\pgfqpoint{3.095669in}{3.955634in}}%
\pgfpathlineto{\pgfqpoint{3.098701in}{3.964599in}}%
\pgfpathlineto{\pgfqpoint{3.099375in}{3.962391in}}%
\pgfpathlineto{\pgfqpoint{3.102492in}{3.946403in}}%
\pgfpathlineto{\pgfqpoint{3.103251in}{3.944992in}}%
\pgfpathlineto{\pgfqpoint{3.105525in}{3.941135in}}%
\pgfpathlineto{\pgfqpoint{3.108811in}{3.939959in}}%
\pgfpathlineto{\pgfqpoint{3.110074in}{3.931446in}}%
\pgfpathlineto{\pgfqpoint{3.111085in}{3.927183in}}%
\pgfpathlineto{\pgfqpoint{3.111507in}{3.928224in}}%
\pgfpathlineto{\pgfqpoint{3.113444in}{3.937854in}}%
\pgfpathlineto{\pgfqpoint{3.114118in}{3.935032in}}%
\pgfpathlineto{\pgfqpoint{3.115129in}{3.930511in}}%
\pgfpathlineto{\pgfqpoint{3.115550in}{3.932193in}}%
\pgfpathlineto{\pgfqpoint{3.116730in}{3.950804in}}%
\pgfpathlineto{\pgfqpoint{3.118078in}{3.963395in}}%
\pgfpathlineto{\pgfqpoint{3.118499in}{3.963022in}}%
\pgfpathlineto{\pgfqpoint{3.118752in}{3.963025in}}%
\pgfpathlineto{\pgfqpoint{3.119004in}{3.963670in}}%
\pgfpathlineto{\pgfqpoint{3.120015in}{3.972742in}}%
\pgfpathlineto{\pgfqpoint{3.122543in}{3.994871in}}%
\pgfpathlineto{\pgfqpoint{3.123132in}{3.995776in}}%
\pgfpathlineto{\pgfqpoint{3.125744in}{4.007585in}}%
\pgfpathlineto{\pgfqpoint{3.126839in}{4.006739in}}%
\pgfpathlineto{\pgfqpoint{3.127260in}{4.006704in}}%
\pgfpathlineto{\pgfqpoint{3.127682in}{4.007400in}}%
\pgfpathlineto{\pgfqpoint{3.128103in}{4.007688in}}%
\pgfpathlineto{\pgfqpoint{3.128524in}{4.006664in}}%
\pgfpathlineto{\pgfqpoint{3.129703in}{3.997814in}}%
\pgfpathlineto{\pgfqpoint{3.130462in}{3.992419in}}%
\pgfpathlineto{\pgfqpoint{3.130967in}{3.995722in}}%
\pgfpathlineto{\pgfqpoint{3.132568in}{4.015681in}}%
\pgfpathlineto{\pgfqpoint{3.133157in}{4.010271in}}%
\pgfpathlineto{\pgfqpoint{3.134505in}{3.991948in}}%
\pgfpathlineto{\pgfqpoint{3.135011in}{3.996498in}}%
\pgfpathlineto{\pgfqpoint{3.136190in}{4.010733in}}%
\pgfpathlineto{\pgfqpoint{3.136696in}{4.008263in}}%
\pgfpathlineto{\pgfqpoint{3.138212in}{3.989262in}}%
\pgfpathlineto{\pgfqpoint{3.138970in}{3.993514in}}%
\pgfpathlineto{\pgfqpoint{3.139644in}{3.995673in}}%
\pgfpathlineto{\pgfqpoint{3.140150in}{3.994177in}}%
\pgfpathlineto{\pgfqpoint{3.141161in}{3.989106in}}%
\pgfpathlineto{\pgfqpoint{3.141666in}{3.991043in}}%
\pgfpathlineto{\pgfqpoint{3.147058in}{4.037279in}}%
\pgfpathlineto{\pgfqpoint{3.147900in}{4.030377in}}%
\pgfpathlineto{\pgfqpoint{3.149754in}{4.010387in}}%
\pgfpathlineto{\pgfqpoint{3.150259in}{4.011108in}}%
\pgfpathlineto{\pgfqpoint{3.150849in}{4.011683in}}%
\pgfpathlineto{\pgfqpoint{3.151186in}{4.010797in}}%
\pgfpathlineto{\pgfqpoint{3.152028in}{4.000426in}}%
\pgfpathlineto{\pgfqpoint{3.153966in}{3.984432in}}%
\pgfpathlineto{\pgfqpoint{3.155398in}{3.983535in}}%
\pgfpathlineto{\pgfqpoint{3.156409in}{3.980118in}}%
\pgfpathlineto{\pgfqpoint{3.157083in}{3.980405in}}%
\pgfpathlineto{\pgfqpoint{3.157841in}{3.979368in}}%
\pgfpathlineto{\pgfqpoint{3.158852in}{3.973061in}}%
\pgfpathlineto{\pgfqpoint{3.160874in}{3.964102in}}%
\pgfpathlineto{\pgfqpoint{3.161464in}{3.962342in}}%
\pgfpathlineto{\pgfqpoint{3.162727in}{3.946184in}}%
\pgfpathlineto{\pgfqpoint{3.164328in}{3.931320in}}%
\pgfpathlineto{\pgfqpoint{3.164749in}{3.931428in}}%
\pgfpathlineto{\pgfqpoint{3.165676in}{3.930764in}}%
\pgfpathlineto{\pgfqpoint{3.168372in}{3.925265in}}%
\pgfpathlineto{\pgfqpoint{3.170730in}{3.926103in}}%
\pgfpathlineto{\pgfqpoint{3.171489in}{3.927477in}}%
\pgfpathlineto{\pgfqpoint{3.171994in}{3.926553in}}%
\pgfpathlineto{\pgfqpoint{3.173511in}{3.924304in}}%
\pgfpathlineto{\pgfqpoint{3.173848in}{3.924778in}}%
\pgfpathlineto{\pgfqpoint{3.174690in}{3.926177in}}%
\pgfpathlineto{\pgfqpoint{3.175448in}{3.925807in}}%
\pgfpathlineto{\pgfqpoint{3.176459in}{3.925236in}}%
\pgfpathlineto{\pgfqpoint{3.176965in}{3.924965in}}%
\pgfpathlineto{\pgfqpoint{3.177470in}{3.925571in}}%
\pgfpathlineto{\pgfqpoint{3.178986in}{3.925598in}}%
\pgfpathlineto{\pgfqpoint{3.179071in}{3.925492in}}%
\pgfpathlineto{\pgfqpoint{3.180166in}{3.923958in}}%
\pgfpathlineto{\pgfqpoint{3.180924in}{3.924553in}}%
\pgfpathlineto{\pgfqpoint{3.183283in}{3.924145in}}%
\pgfpathlineto{\pgfqpoint{3.191539in}{3.916540in}}%
\pgfpathlineto{\pgfqpoint{3.191960in}{3.917322in}}%
\pgfpathlineto{\pgfqpoint{3.193813in}{3.918825in}}%
\pgfpathlineto{\pgfqpoint{3.195583in}{3.919227in}}%
\pgfpathlineto{\pgfqpoint{3.196509in}{3.920190in}}%
\pgfpathlineto{\pgfqpoint{3.197015in}{3.919296in}}%
\pgfpathlineto{\pgfqpoint{3.199037in}{3.917247in}}%
\pgfpathlineto{\pgfqpoint{3.205608in}{3.917386in}}%
\pgfpathlineto{\pgfqpoint{3.206029in}{3.916871in}}%
\pgfpathlineto{\pgfqpoint{3.206619in}{3.917703in}}%
\pgfpathlineto{\pgfqpoint{3.208051in}{3.919025in}}%
\pgfpathlineto{\pgfqpoint{3.208304in}{3.918761in}}%
\pgfpathlineto{\pgfqpoint{3.209399in}{3.913910in}}%
\pgfpathlineto{\pgfqpoint{3.210915in}{3.910580in}}%
\pgfpathlineto{\pgfqpoint{3.211252in}{3.910739in}}%
\pgfpathlineto{\pgfqpoint{3.212600in}{3.910024in}}%
\pgfpathlineto{\pgfqpoint{3.214369in}{3.910030in}}%
\pgfpathlineto{\pgfqpoint{3.218497in}{3.912095in}}%
\pgfpathlineto{\pgfqpoint{3.223131in}{3.918658in}}%
\pgfpathlineto{\pgfqpoint{3.224142in}{3.919709in}}%
\pgfpathlineto{\pgfqpoint{3.225742in}{3.926565in}}%
\pgfpathlineto{\pgfqpoint{3.227427in}{3.931732in}}%
\pgfpathlineto{\pgfqpoint{3.227933in}{3.931431in}}%
\pgfpathlineto{\pgfqpoint{3.228859in}{3.932299in}}%
\pgfpathlineto{\pgfqpoint{3.229617in}{3.937488in}}%
\pgfpathlineto{\pgfqpoint{3.231808in}{3.954377in}}%
\pgfpathlineto{\pgfqpoint{3.232145in}{3.954262in}}%
\pgfpathlineto{\pgfqpoint{3.232650in}{3.955399in}}%
\pgfpathlineto{\pgfqpoint{3.235093in}{3.968786in}}%
\pgfpathlineto{\pgfqpoint{3.236188in}{3.967484in}}%
\pgfpathlineto{\pgfqpoint{3.236694in}{3.967748in}}%
\pgfpathlineto{\pgfqpoint{3.236862in}{3.968297in}}%
\pgfpathlineto{\pgfqpoint{3.238969in}{3.979975in}}%
\pgfpathlineto{\pgfqpoint{3.240232in}{3.979512in}}%
\pgfpathlineto{\pgfqpoint{3.241749in}{3.981266in}}%
\pgfpathlineto{\pgfqpoint{3.242170in}{3.979690in}}%
\pgfpathlineto{\pgfqpoint{3.244866in}{3.971881in}}%
\pgfpathlineto{\pgfqpoint{3.245877in}{3.970191in}}%
\pgfpathlineto{\pgfqpoint{3.247056in}{3.958109in}}%
\pgfpathlineto{\pgfqpoint{3.248404in}{3.950174in}}%
\pgfpathlineto{\pgfqpoint{3.248825in}{3.950510in}}%
\pgfpathlineto{\pgfqpoint{3.249246in}{3.950039in}}%
\pgfpathlineto{\pgfqpoint{3.249331in}{3.949715in}}%
\pgfpathlineto{\pgfqpoint{3.255396in}{3.920663in}}%
\pgfpathlineto{\pgfqpoint{3.256323in}{3.921349in}}%
\pgfpathlineto{\pgfqpoint{3.257165in}{3.920014in}}%
\pgfpathlineto{\pgfqpoint{3.260619in}{3.912628in}}%
\pgfpathlineto{\pgfqpoint{3.261378in}{3.913720in}}%
\pgfpathlineto{\pgfqpoint{3.266516in}{3.937294in}}%
\pgfpathlineto{\pgfqpoint{3.268033in}{3.947514in}}%
\pgfpathlineto{\pgfqpoint{3.268370in}{3.947064in}}%
\pgfpathlineto{\pgfqpoint{3.269212in}{3.945641in}}%
\pgfpathlineto{\pgfqpoint{3.269718in}{3.946704in}}%
\pgfpathlineto{\pgfqpoint{3.271740in}{3.954744in}}%
\pgfpathlineto{\pgfqpoint{3.272582in}{3.953090in}}%
\pgfpathlineto{\pgfqpoint{3.273509in}{3.952678in}}%
\pgfpathlineto{\pgfqpoint{3.273846in}{3.953078in}}%
\pgfpathlineto{\pgfqpoint{3.275699in}{3.954055in}}%
\pgfpathlineto{\pgfqpoint{3.278395in}{3.952968in}}%
\pgfpathlineto{\pgfqpoint{3.283450in}{3.926569in}}%
\pgfpathlineto{\pgfqpoint{3.285303in}{3.916818in}}%
\pgfpathlineto{\pgfqpoint{3.286398in}{3.916045in}}%
\pgfpathlineto{\pgfqpoint{3.288336in}{3.910421in}}%
\pgfpathlineto{\pgfqpoint{3.288757in}{3.911847in}}%
\pgfpathlineto{\pgfqpoint{3.290526in}{3.916897in}}%
\pgfpathlineto{\pgfqpoint{3.290610in}{3.916847in}}%
\pgfpathlineto{\pgfqpoint{3.291200in}{3.916065in}}%
\pgfpathlineto{\pgfqpoint{3.291621in}{3.917243in}}%
\pgfpathlineto{\pgfqpoint{3.293054in}{3.929145in}}%
\pgfpathlineto{\pgfqpoint{3.295328in}{3.947992in}}%
\pgfpathlineto{\pgfqpoint{3.296592in}{3.949463in}}%
\pgfpathlineto{\pgfqpoint{3.299203in}{3.965038in}}%
\pgfpathlineto{\pgfqpoint{3.300888in}{3.964635in}}%
\pgfpathlineto{\pgfqpoint{3.304090in}{3.965411in}}%
\pgfpathlineto{\pgfqpoint{3.305100in}{3.963002in}}%
\pgfpathlineto{\pgfqpoint{3.309987in}{3.940023in}}%
\pgfpathlineto{\pgfqpoint{3.312177in}{3.923091in}}%
\pgfpathlineto{\pgfqpoint{3.313609in}{3.921092in}}%
\pgfpathlineto{\pgfqpoint{3.315884in}{3.907923in}}%
\pgfpathlineto{\pgfqpoint{3.316810in}{3.909928in}}%
\pgfpathlineto{\pgfqpoint{3.318748in}{3.910718in}}%
\pgfpathlineto{\pgfqpoint{3.320517in}{3.910636in}}%
\pgfpathlineto{\pgfqpoint{3.324645in}{3.911307in}}%
\pgfpathlineto{\pgfqpoint{3.326162in}{3.912550in}}%
\pgfpathlineto{\pgfqpoint{3.326836in}{3.914205in}}%
\pgfpathlineto{\pgfqpoint{3.327594in}{3.915246in}}%
\pgfpathlineto{\pgfqpoint{3.328183in}{3.915059in}}%
\pgfpathlineto{\pgfqpoint{3.331216in}{3.915823in}}%
\pgfpathlineto{\pgfqpoint{3.332648in}{3.916833in}}%
\pgfpathlineto{\pgfqpoint{3.332985in}{3.916281in}}%
\pgfpathlineto{\pgfqpoint{3.335344in}{3.913251in}}%
\pgfpathlineto{\pgfqpoint{3.343853in}{3.911481in}}%
\pgfpathlineto{\pgfqpoint{3.345117in}{3.910455in}}%
\pgfpathlineto{\pgfqpoint{3.345285in}{3.910565in}}%
\pgfpathlineto{\pgfqpoint{3.346043in}{3.913192in}}%
\pgfpathlineto{\pgfqpoint{3.347560in}{3.919718in}}%
\pgfpathlineto{\pgfqpoint{3.348065in}{3.918748in}}%
\pgfpathlineto{\pgfqpoint{3.348571in}{3.918246in}}%
\pgfpathlineto{\pgfqpoint{3.349076in}{3.919201in}}%
\pgfpathlineto{\pgfqpoint{3.349919in}{3.924859in}}%
\pgfpathlineto{\pgfqpoint{3.351940in}{3.935826in}}%
\pgfpathlineto{\pgfqpoint{3.352783in}{3.936715in}}%
\pgfpathlineto{\pgfqpoint{3.354215in}{3.950294in}}%
\pgfpathlineto{\pgfqpoint{3.356405in}{3.961987in}}%
\pgfpathlineto{\pgfqpoint{3.357164in}{3.963947in}}%
\pgfpathlineto{\pgfqpoint{3.358175in}{3.972049in}}%
\pgfpathlineto{\pgfqpoint{3.359691in}{3.983413in}}%
\pgfpathlineto{\pgfqpoint{3.360028in}{3.983067in}}%
\pgfpathlineto{\pgfqpoint{3.360786in}{3.982042in}}%
\pgfpathlineto{\pgfqpoint{3.361207in}{3.983087in}}%
\pgfpathlineto{\pgfqpoint{3.362218in}{3.993117in}}%
\pgfpathlineto{\pgfqpoint{3.363061in}{3.999984in}}%
\pgfpathlineto{\pgfqpoint{3.363650in}{3.998474in}}%
\pgfpathlineto{\pgfqpoint{3.364493in}{3.995856in}}%
\pgfpathlineto{\pgfqpoint{3.364914in}{3.997677in}}%
\pgfpathlineto{\pgfqpoint{3.366599in}{4.009975in}}%
\pgfpathlineto{\pgfqpoint{3.367104in}{4.007156in}}%
\pgfpathlineto{\pgfqpoint{3.368200in}{4.001796in}}%
\pgfpathlineto{\pgfqpoint{3.368621in}{4.003216in}}%
\pgfpathlineto{\pgfqpoint{3.369632in}{4.009706in}}%
\pgfpathlineto{\pgfqpoint{3.370222in}{4.006803in}}%
\pgfpathlineto{\pgfqpoint{3.373254in}{3.991045in}}%
\pgfpathlineto{\pgfqpoint{3.374013in}{3.988838in}}%
\pgfpathlineto{\pgfqpoint{3.375445in}{3.979691in}}%
\pgfpathlineto{\pgfqpoint{3.376961in}{3.973877in}}%
\pgfpathlineto{\pgfqpoint{3.377214in}{3.974217in}}%
\pgfpathlineto{\pgfqpoint{3.377888in}{3.979872in}}%
\pgfpathlineto{\pgfqpoint{3.378730in}{3.987173in}}%
\pgfpathlineto{\pgfqpoint{3.379320in}{3.984323in}}%
\pgfpathlineto{\pgfqpoint{3.380331in}{3.978736in}}%
\pgfpathlineto{\pgfqpoint{3.380752in}{3.980653in}}%
\pgfpathlineto{\pgfqpoint{3.386986in}{4.042930in}}%
\pgfpathlineto{\pgfqpoint{3.387660in}{4.042338in}}%
\pgfpathlineto{\pgfqpoint{3.388250in}{4.044705in}}%
\pgfpathlineto{\pgfqpoint{3.391367in}{4.060287in}}%
\pgfpathlineto{\pgfqpoint{3.392378in}{4.067748in}}%
\pgfpathlineto{\pgfqpoint{3.393810in}{4.081796in}}%
\pgfpathlineto{\pgfqpoint{3.394315in}{4.080306in}}%
\pgfpathlineto{\pgfqpoint{3.395916in}{4.064738in}}%
\pgfpathlineto{\pgfqpoint{3.396927in}{4.068290in}}%
\pgfpathlineto{\pgfqpoint{3.397180in}{4.068674in}}%
\pgfpathlineto{\pgfqpoint{3.397517in}{4.067231in}}%
\pgfpathlineto{\pgfqpoint{3.398696in}{4.047280in}}%
\pgfpathlineto{\pgfqpoint{3.400465in}{4.022314in}}%
\pgfpathlineto{\pgfqpoint{3.400802in}{4.022675in}}%
\pgfpathlineto{\pgfqpoint{3.401055in}{4.022699in}}%
\pgfpathlineto{\pgfqpoint{3.401308in}{4.021697in}}%
\pgfpathlineto{\pgfqpoint{3.402571in}{4.004777in}}%
\pgfpathlineto{\pgfqpoint{3.404509in}{3.988846in}}%
\pgfpathlineto{\pgfqpoint{3.405099in}{3.986792in}}%
\pgfpathlineto{\pgfqpoint{3.407289in}{3.971451in}}%
\pgfpathlineto{\pgfqpoint{3.407963in}{3.973798in}}%
\pgfpathlineto{\pgfqpoint{3.408721in}{3.975783in}}%
\pgfpathlineto{\pgfqpoint{3.409142in}{3.974286in}}%
\pgfpathlineto{\pgfqpoint{3.410238in}{3.962178in}}%
\pgfpathlineto{\pgfqpoint{3.410827in}{3.968943in}}%
\pgfpathlineto{\pgfqpoint{3.416977in}{4.088989in}}%
\pgfpathlineto{\pgfqpoint{3.417567in}{4.087830in}}%
\pgfpathlineto{\pgfqpoint{3.417820in}{4.087750in}}%
\pgfpathlineto{\pgfqpoint{3.418157in}{4.088702in}}%
\pgfpathlineto{\pgfqpoint{3.420263in}{4.097034in}}%
\pgfpathlineto{\pgfqpoint{3.420852in}{4.096638in}}%
\pgfpathlineto{\pgfqpoint{3.422622in}{4.094844in}}%
\pgfpathlineto{\pgfqpoint{3.424138in}{4.093972in}}%
\pgfpathlineto{\pgfqpoint{3.424728in}{4.091826in}}%
\pgfpathlineto{\pgfqpoint{3.428434in}{4.057757in}}%
\pgfpathlineto{\pgfqpoint{3.429867in}{4.014324in}}%
\pgfpathlineto{\pgfqpoint{3.430456in}{4.019566in}}%
\pgfpathlineto{\pgfqpoint{3.431552in}{4.035561in}}%
\pgfpathlineto{\pgfqpoint{3.432057in}{4.030543in}}%
\pgfpathlineto{\pgfqpoint{3.433573in}{3.993363in}}%
\pgfpathlineto{\pgfqpoint{3.434163in}{4.004531in}}%
\pgfpathlineto{\pgfqpoint{3.436522in}{4.083747in}}%
\pgfpathlineto{\pgfqpoint{3.437364in}{4.082039in}}%
\pgfpathlineto{\pgfqpoint{3.438038in}{4.092650in}}%
\pgfpathlineto{\pgfqpoint{3.441324in}{4.156951in}}%
\pgfpathlineto{\pgfqpoint{3.443093in}{4.179784in}}%
\pgfpathlineto{\pgfqpoint{3.444104in}{4.176359in}}%
\pgfpathlineto{\pgfqpoint{3.444357in}{4.175992in}}%
\pgfpathlineto{\pgfqpoint{3.444694in}{4.177018in}}%
\pgfpathlineto{\pgfqpoint{3.446294in}{4.195320in}}%
\pgfpathlineto{\pgfqpoint{3.447895in}{4.192547in}}%
\pgfpathlineto{\pgfqpoint{3.448737in}{4.192497in}}%
\pgfpathlineto{\pgfqpoint{3.448906in}{4.192985in}}%
\pgfpathlineto{\pgfqpoint{3.450675in}{4.202702in}}%
\pgfpathlineto{\pgfqpoint{3.452444in}{4.250129in}}%
\pgfpathlineto{\pgfqpoint{3.453287in}{4.238354in}}%
\pgfpathlineto{\pgfqpoint{3.453539in}{4.237154in}}%
\pgfpathlineto{\pgfqpoint{3.453876in}{4.240849in}}%
\pgfpathlineto{\pgfqpoint{3.454803in}{4.285050in}}%
\pgfpathlineto{\pgfqpoint{3.456572in}{4.369266in}}%
\pgfpathlineto{\pgfqpoint{3.456993in}{4.368403in}}%
\pgfpathlineto{\pgfqpoint{3.457246in}{4.368046in}}%
\pgfpathlineto{\pgfqpoint{3.457499in}{4.369220in}}%
\pgfpathlineto{\pgfqpoint{3.458173in}{4.386855in}}%
\pgfpathlineto{\pgfqpoint{3.460869in}{4.456808in}}%
\pgfpathlineto{\pgfqpoint{3.461458in}{4.466488in}}%
\pgfpathlineto{\pgfqpoint{3.463649in}{4.529701in}}%
\pgfpathlineto{\pgfqpoint{3.464407in}{4.522482in}}%
\pgfpathlineto{\pgfqpoint{3.464744in}{4.520572in}}%
\pgfpathlineto{\pgfqpoint{3.465249in}{4.524578in}}%
\pgfpathlineto{\pgfqpoint{3.467018in}{4.556621in}}%
\pgfpathlineto{\pgfqpoint{3.467524in}{4.546160in}}%
\pgfpathlineto{\pgfqpoint{3.472326in}{4.287338in}}%
\pgfpathlineto{\pgfqpoint{3.473421in}{4.221758in}}%
\pgfpathlineto{\pgfqpoint{3.473927in}{4.232041in}}%
\pgfpathlineto{\pgfqpoint{3.474853in}{4.258941in}}%
\pgfpathlineto{\pgfqpoint{3.475274in}{4.247485in}}%
\pgfpathlineto{\pgfqpoint{3.476622in}{4.182340in}}%
\pgfpathlineto{\pgfqpoint{3.477212in}{4.198378in}}%
\pgfpathlineto{\pgfqpoint{3.478391in}{4.255414in}}%
\pgfpathlineto{\pgfqpoint{3.478981in}{4.239646in}}%
\pgfpathlineto{\pgfqpoint{3.479992in}{4.200672in}}%
\pgfpathlineto{\pgfqpoint{3.480498in}{4.218850in}}%
\pgfpathlineto{\pgfqpoint{3.482435in}{4.375455in}}%
\pgfpathlineto{\pgfqpoint{3.483278in}{4.357841in}}%
\pgfpathlineto{\pgfqpoint{3.483699in}{4.351782in}}%
\pgfpathlineto{\pgfqpoint{3.484204in}{4.361486in}}%
\pgfpathlineto{\pgfqpoint{3.487069in}{4.473663in}}%
\pgfpathlineto{\pgfqpoint{3.487911in}{4.472879in}}%
\pgfpathlineto{\pgfqpoint{3.490186in}{4.469852in}}%
\pgfpathlineto{\pgfqpoint{3.491197in}{4.452941in}}%
\pgfpathlineto{\pgfqpoint{3.492629in}{4.426244in}}%
\pgfpathlineto{\pgfqpoint{3.493219in}{4.427989in}}%
\pgfpathlineto{\pgfqpoint{3.493471in}{4.428310in}}%
\pgfpathlineto{\pgfqpoint{3.493724in}{4.426932in}}%
\pgfpathlineto{\pgfqpoint{3.494566in}{4.403770in}}%
\pgfpathlineto{\pgfqpoint{3.496083in}{4.371494in}}%
\pgfpathlineto{\pgfqpoint{3.496420in}{4.371612in}}%
\pgfpathlineto{\pgfqpoint{3.496757in}{4.369684in}}%
\pgfpathlineto{\pgfqpoint{3.497431in}{4.340918in}}%
\pgfpathlineto{\pgfqpoint{3.500801in}{4.157537in}}%
\pgfpathlineto{\pgfqpoint{3.501896in}{4.099890in}}%
\pgfpathlineto{\pgfqpoint{3.503749in}{4.036570in}}%
\pgfpathlineto{\pgfqpoint{3.503918in}{4.036897in}}%
\pgfpathlineto{\pgfqpoint{3.506866in}{4.050272in}}%
\pgfpathlineto{\pgfqpoint{3.507456in}{4.045792in}}%
\pgfpathlineto{\pgfqpoint{3.508298in}{4.037044in}}%
\pgfpathlineto{\pgfqpoint{3.508720in}{4.041480in}}%
\pgfpathlineto{\pgfqpoint{3.510404in}{4.086427in}}%
\pgfpathlineto{\pgfqpoint{3.511163in}{4.073273in}}%
\pgfpathlineto{\pgfqpoint{3.512679in}{4.042056in}}%
\pgfpathlineto{\pgfqpoint{3.513100in}{4.044494in}}%
\pgfpathlineto{\pgfqpoint{3.514280in}{4.055624in}}%
\pgfpathlineto{\pgfqpoint{3.514869in}{4.052091in}}%
\pgfpathlineto{\pgfqpoint{3.516891in}{4.035513in}}%
\pgfpathlineto{\pgfqpoint{3.517481in}{4.035584in}}%
\pgfpathlineto{\pgfqpoint{3.519587in}{4.032878in}}%
\pgfpathlineto{\pgfqpoint{3.520008in}{4.034682in}}%
\pgfpathlineto{\pgfqpoint{3.521103in}{4.054713in}}%
\pgfpathlineto{\pgfqpoint{3.522283in}{4.069525in}}%
\pgfpathlineto{\pgfqpoint{3.522788in}{4.067174in}}%
\pgfpathlineto{\pgfqpoint{3.523378in}{4.064152in}}%
\pgfpathlineto{\pgfqpoint{3.523715in}{4.066741in}}%
\pgfpathlineto{\pgfqpoint{3.524642in}{4.102785in}}%
\pgfpathlineto{\pgfqpoint{3.525990in}{4.149881in}}%
\pgfpathlineto{\pgfqpoint{3.526411in}{4.143705in}}%
\pgfpathlineto{\pgfqpoint{3.527338in}{4.120891in}}%
\pgfpathlineto{\pgfqpoint{3.527843in}{4.134028in}}%
\pgfpathlineto{\pgfqpoint{3.529696in}{4.228921in}}%
\pgfpathlineto{\pgfqpoint{3.530539in}{4.221731in}}%
\pgfpathlineto{\pgfqpoint{3.530792in}{4.223308in}}%
\pgfpathlineto{\pgfqpoint{3.531550in}{4.251592in}}%
\pgfpathlineto{\pgfqpoint{3.533740in}{4.318529in}}%
\pgfpathlineto{\pgfqpoint{3.534414in}{4.344980in}}%
\pgfpathlineto{\pgfqpoint{3.536604in}{4.488631in}}%
\pgfpathlineto{\pgfqpoint{3.537447in}{4.485744in}}%
\pgfpathlineto{\pgfqpoint{3.538121in}{4.508671in}}%
\pgfpathlineto{\pgfqpoint{3.540058in}{4.592191in}}%
\pgfpathlineto{\pgfqpoint{3.540648in}{4.587462in}}%
\pgfpathlineto{\pgfqpoint{3.541575in}{4.575746in}}%
\pgfpathlineto{\pgfqpoint{3.542165in}{4.581499in}}%
\pgfpathlineto{\pgfqpoint{3.543765in}{4.598384in}}%
\pgfpathlineto{\pgfqpoint{3.544102in}{4.596733in}}%
\pgfpathlineto{\pgfqpoint{3.545029in}{4.571485in}}%
\pgfpathlineto{\pgfqpoint{3.546040in}{4.552354in}}%
\pgfpathlineto{\pgfqpoint{3.546545in}{4.555383in}}%
\pgfpathlineto{\pgfqpoint{3.547051in}{4.558994in}}%
\pgfpathlineto{\pgfqpoint{3.547472in}{4.555764in}}%
\pgfpathlineto{\pgfqpoint{3.548483in}{4.516935in}}%
\pgfpathlineto{\pgfqpoint{3.550421in}{4.470332in}}%
\pgfpathlineto{\pgfqpoint{3.550926in}{4.462539in}}%
\pgfpathlineto{\pgfqpoint{3.552105in}{4.388901in}}%
\pgfpathlineto{\pgfqpoint{3.553453in}{4.331447in}}%
\pgfpathlineto{\pgfqpoint{3.553959in}{4.333519in}}%
\pgfpathlineto{\pgfqpoint{3.554127in}{4.333705in}}%
\pgfpathlineto{\pgfqpoint{3.554380in}{4.331914in}}%
\pgfpathlineto{\pgfqpoint{3.555728in}{4.297411in}}%
\pgfpathlineto{\pgfqpoint{3.556907in}{4.267593in}}%
\pgfpathlineto{\pgfqpoint{3.557413in}{4.272272in}}%
\pgfpathlineto{\pgfqpoint{3.558845in}{4.308119in}}%
\pgfpathlineto{\pgfqpoint{3.559519in}{4.292616in}}%
\pgfpathlineto{\pgfqpoint{3.560446in}{4.272916in}}%
\pgfpathlineto{\pgfqpoint{3.560951in}{4.281184in}}%
\pgfpathlineto{\pgfqpoint{3.562468in}{4.327841in}}%
\pgfpathlineto{\pgfqpoint{3.563141in}{4.318407in}}%
\pgfpathlineto{\pgfqpoint{3.563815in}{4.310202in}}%
\pgfpathlineto{\pgfqpoint{3.564321in}{4.316144in}}%
\pgfpathlineto{\pgfqpoint{3.569797in}{4.445633in}}%
\pgfpathlineto{\pgfqpoint{3.570302in}{4.448865in}}%
\pgfpathlineto{\pgfqpoint{3.570892in}{4.445497in}}%
\pgfpathlineto{\pgfqpoint{3.572156in}{4.425293in}}%
\pgfpathlineto{\pgfqpoint{3.573167in}{4.411247in}}%
\pgfpathlineto{\pgfqpoint{3.573756in}{4.411526in}}%
\pgfpathlineto{\pgfqpoint{3.574262in}{4.407591in}}%
\pgfpathlineto{\pgfqpoint{3.575188in}{4.374634in}}%
\pgfpathlineto{\pgfqpoint{3.576958in}{4.311883in}}%
\pgfpathlineto{\pgfqpoint{3.577295in}{4.312274in}}%
\pgfpathlineto{\pgfqpoint{3.577547in}{4.311629in}}%
\pgfpathlineto{\pgfqpoint{3.578137in}{4.298053in}}%
\pgfpathlineto{\pgfqpoint{3.583276in}{4.125967in}}%
\pgfpathlineto{\pgfqpoint{3.583444in}{4.126565in}}%
\pgfpathlineto{\pgfqpoint{3.584540in}{4.146668in}}%
\pgfpathlineto{\pgfqpoint{3.585129in}{4.133524in}}%
\pgfpathlineto{\pgfqpoint{3.586561in}{4.076349in}}%
\pgfpathlineto{\pgfqpoint{3.587151in}{4.088195in}}%
\pgfpathlineto{\pgfqpoint{3.588499in}{4.128304in}}%
\pgfpathlineto{\pgfqpoint{3.589005in}{4.121845in}}%
\pgfpathlineto{\pgfqpoint{3.590100in}{4.104047in}}%
\pgfpathlineto{\pgfqpoint{3.590605in}{4.110690in}}%
\pgfpathlineto{\pgfqpoint{3.591869in}{4.136024in}}%
\pgfpathlineto{\pgfqpoint{3.592543in}{4.133405in}}%
\pgfpathlineto{\pgfqpoint{3.593470in}{4.130596in}}%
\pgfpathlineto{\pgfqpoint{3.593806in}{4.132153in}}%
\pgfpathlineto{\pgfqpoint{3.594817in}{4.153442in}}%
\pgfpathlineto{\pgfqpoint{3.597934in}{4.222111in}}%
\pgfpathlineto{\pgfqpoint{3.598861in}{4.258082in}}%
\pgfpathlineto{\pgfqpoint{3.600209in}{4.310561in}}%
\pgfpathlineto{\pgfqpoint{3.600715in}{4.303721in}}%
\pgfpathlineto{\pgfqpoint{3.601810in}{4.282070in}}%
\pgfpathlineto{\pgfqpoint{3.602399in}{4.287393in}}%
\pgfpathlineto{\pgfqpoint{3.603242in}{4.297702in}}%
\pgfpathlineto{\pgfqpoint{3.603663in}{4.292374in}}%
\pgfpathlineto{\pgfqpoint{3.609897in}{4.158658in}}%
\pgfpathlineto{\pgfqpoint{3.610740in}{4.156278in}}%
\pgfpathlineto{\pgfqpoint{3.611582in}{4.132686in}}%
\pgfpathlineto{\pgfqpoint{3.613604in}{4.052836in}}%
\pgfpathlineto{\pgfqpoint{3.614278in}{4.059000in}}%
\pgfpathlineto{\pgfqpoint{3.614615in}{4.060928in}}%
\pgfpathlineto{\pgfqpoint{3.615036in}{4.057544in}}%
\pgfpathlineto{\pgfqpoint{3.616131in}{4.021395in}}%
\pgfpathlineto{\pgfqpoint{3.617648in}{3.991096in}}%
\pgfpathlineto{\pgfqpoint{3.617900in}{3.991743in}}%
\pgfpathlineto{\pgfqpoint{3.618911in}{4.004313in}}%
\pgfpathlineto{\pgfqpoint{3.621523in}{4.046086in}}%
\pgfpathlineto{\pgfqpoint{3.622450in}{4.044824in}}%
\pgfpathlineto{\pgfqpoint{3.622871in}{4.047278in}}%
\pgfpathlineto{\pgfqpoint{3.624135in}{4.077544in}}%
\pgfpathlineto{\pgfqpoint{3.625988in}{4.098865in}}%
\pgfpathlineto{\pgfqpoint{3.626746in}{4.099810in}}%
\pgfpathlineto{\pgfqpoint{3.627841in}{4.111087in}}%
\pgfpathlineto{\pgfqpoint{3.630032in}{4.127211in}}%
\pgfpathlineto{\pgfqpoint{3.630200in}{4.126931in}}%
\pgfpathlineto{\pgfqpoint{3.630958in}{4.120415in}}%
\pgfpathlineto{\pgfqpoint{3.633823in}{4.082209in}}%
\pgfpathlineto{\pgfqpoint{3.634834in}{4.083355in}}%
\pgfpathlineto{\pgfqpoint{3.635508in}{4.073557in}}%
\pgfpathlineto{\pgfqpoint{3.638035in}{4.007412in}}%
\pgfpathlineto{\pgfqpoint{3.639046in}{4.008977in}}%
\pgfpathlineto{\pgfqpoint{3.639804in}{4.000127in}}%
\pgfpathlineto{\pgfqpoint{3.642921in}{3.955159in}}%
\pgfpathlineto{\pgfqpoint{3.643090in}{3.955210in}}%
\pgfpathlineto{\pgfqpoint{3.643679in}{3.955024in}}%
\pgfpathlineto{\pgfqpoint{3.643848in}{3.954499in}}%
\pgfpathlineto{\pgfqpoint{3.645364in}{3.943092in}}%
\pgfpathlineto{\pgfqpoint{3.646628in}{3.935421in}}%
\pgfpathlineto{\pgfqpoint{3.647049in}{3.936808in}}%
\pgfpathlineto{\pgfqpoint{3.651346in}{3.971542in}}%
\pgfpathlineto{\pgfqpoint{3.653283in}{3.988926in}}%
\pgfpathlineto{\pgfqpoint{3.653789in}{3.991101in}}%
\pgfpathlineto{\pgfqpoint{3.658001in}{4.019214in}}%
\pgfpathlineto{\pgfqpoint{3.659096in}{4.021710in}}%
\pgfpathlineto{\pgfqpoint{3.660023in}{4.025035in}}%
\pgfpathlineto{\pgfqpoint{3.660612in}{4.023689in}}%
\pgfpathlineto{\pgfqpoint{3.665499in}{4.001529in}}%
\pgfpathlineto{\pgfqpoint{3.666173in}{4.004650in}}%
\pgfpathlineto{\pgfqpoint{3.666847in}{4.007741in}}%
\pgfpathlineto{\pgfqpoint{3.667352in}{4.005050in}}%
\pgfpathlineto{\pgfqpoint{3.669374in}{3.971908in}}%
\pgfpathlineto{\pgfqpoint{3.670553in}{3.978587in}}%
\pgfpathlineto{\pgfqpoint{3.670722in}{3.978594in}}%
\pgfpathlineto{\pgfqpoint{3.670974in}{3.977543in}}%
\pgfpathlineto{\pgfqpoint{3.672238in}{3.959559in}}%
\pgfpathlineto{\pgfqpoint{3.673165in}{3.953993in}}%
\pgfpathlineto{\pgfqpoint{3.673670in}{3.954926in}}%
\pgfpathlineto{\pgfqpoint{3.674429in}{3.956439in}}%
\pgfpathlineto{\pgfqpoint{3.675102in}{3.955926in}}%
\pgfpathlineto{\pgfqpoint{3.676198in}{3.954201in}}%
\pgfpathlineto{\pgfqpoint{3.677798in}{3.952771in}}%
\pgfpathlineto{\pgfqpoint{3.679652in}{3.953819in}}%
\pgfpathlineto{\pgfqpoint{3.680410in}{3.956037in}}%
\pgfpathlineto{\pgfqpoint{3.682179in}{3.965717in}}%
\pgfpathlineto{\pgfqpoint{3.682769in}{3.964402in}}%
\pgfpathlineto{\pgfqpoint{3.684117in}{3.958283in}}%
\pgfpathlineto{\pgfqpoint{3.684791in}{3.959642in}}%
\pgfpathlineto{\pgfqpoint{3.685970in}{3.963485in}}%
\pgfpathlineto{\pgfqpoint{3.686475in}{3.961669in}}%
\pgfpathlineto{\pgfqpoint{3.690014in}{3.945808in}}%
\pgfpathlineto{\pgfqpoint{3.690856in}{3.943426in}}%
\pgfpathlineto{\pgfqpoint{3.693552in}{3.931793in}}%
\pgfpathlineto{\pgfqpoint{3.694226in}{3.932547in}}%
\pgfpathlineto{\pgfqpoint{3.695742in}{3.932944in}}%
\pgfpathlineto{\pgfqpoint{3.695827in}{3.932878in}}%
\pgfpathlineto{\pgfqpoint{3.697259in}{3.930291in}}%
\pgfpathlineto{\pgfqpoint{3.699955in}{3.918171in}}%
\pgfpathlineto{\pgfqpoint{3.700376in}{3.918286in}}%
\pgfpathlineto{\pgfqpoint{3.701218in}{3.917141in}}%
\pgfpathlineto{\pgfqpoint{3.703156in}{3.910963in}}%
\pgfpathlineto{\pgfqpoint{3.704083in}{3.912195in}}%
\pgfpathlineto{\pgfqpoint{3.705599in}{3.912199in}}%
\pgfpathlineto{\pgfqpoint{3.719921in}{3.911081in}}%
\pgfpathlineto{\pgfqpoint{3.723038in}{3.916283in}}%
\pgfpathlineto{\pgfqpoint{3.724133in}{3.917996in}}%
\pgfpathlineto{\pgfqpoint{3.725312in}{3.919292in}}%
\pgfpathlineto{\pgfqpoint{3.725818in}{3.918858in}}%
\pgfpathlineto{\pgfqpoint{3.727671in}{3.918881in}}%
\pgfpathlineto{\pgfqpoint{3.729356in}{3.918725in}}%
\pgfpathlineto{\pgfqpoint{3.731883in}{3.917625in}}%
\pgfpathlineto{\pgfqpoint{3.733231in}{3.918137in}}%
\pgfpathlineto{\pgfqpoint{3.733568in}{3.917733in}}%
\pgfpathlineto{\pgfqpoint{3.735169in}{3.917669in}}%
\pgfpathlineto{\pgfqpoint{3.736264in}{3.918837in}}%
\pgfpathlineto{\pgfqpoint{3.736769in}{3.917597in}}%
\pgfpathlineto{\pgfqpoint{3.738370in}{3.911495in}}%
\pgfpathlineto{\pgfqpoint{3.739044in}{3.913073in}}%
\pgfpathlineto{\pgfqpoint{3.740139in}{3.914996in}}%
\pgfpathlineto{\pgfqpoint{3.740560in}{3.914347in}}%
\pgfpathlineto{\pgfqpoint{3.741740in}{3.912660in}}%
\pgfpathlineto{\pgfqpoint{3.742161in}{3.913201in}}%
\pgfpathlineto{\pgfqpoint{3.744014in}{3.915657in}}%
\pgfpathlineto{\pgfqpoint{3.744351in}{3.915493in}}%
\pgfpathlineto{\pgfqpoint{3.745025in}{3.916369in}}%
\pgfpathlineto{\pgfqpoint{3.746289in}{3.923636in}}%
\pgfpathlineto{\pgfqpoint{3.748227in}{3.933046in}}%
\pgfpathlineto{\pgfqpoint{3.748311in}{3.933021in}}%
\pgfpathlineto{\pgfqpoint{3.749575in}{3.933287in}}%
\pgfpathlineto{\pgfqpoint{3.749659in}{3.933508in}}%
\pgfpathlineto{\pgfqpoint{3.750670in}{3.939914in}}%
\pgfpathlineto{\pgfqpoint{3.752270in}{3.950588in}}%
\pgfpathlineto{\pgfqpoint{3.752776in}{3.950158in}}%
\pgfpathlineto{\pgfqpoint{3.754461in}{3.950017in}}%
\pgfpathlineto{\pgfqpoint{3.755303in}{3.949456in}}%
\pgfpathlineto{\pgfqpoint{3.755388in}{3.949216in}}%
\pgfpathlineto{\pgfqpoint{3.756904in}{3.946042in}}%
\pgfpathlineto{\pgfqpoint{3.757325in}{3.946412in}}%
\pgfpathlineto{\pgfqpoint{3.759431in}{3.950682in}}%
\pgfpathlineto{\pgfqpoint{3.760442in}{3.949677in}}%
\pgfpathlineto{\pgfqpoint{3.761285in}{3.950906in}}%
\pgfpathlineto{\pgfqpoint{3.762127in}{3.951024in}}%
\pgfpathlineto{\pgfqpoint{3.762380in}{3.950686in}}%
\pgfpathlineto{\pgfqpoint{3.763138in}{3.946463in}}%
\pgfpathlineto{\pgfqpoint{3.764317in}{3.939322in}}%
\pgfpathlineto{\pgfqpoint{3.764823in}{3.940211in}}%
\pgfpathlineto{\pgfqpoint{3.766171in}{3.944361in}}%
\pgfpathlineto{\pgfqpoint{3.766761in}{3.942799in}}%
\pgfpathlineto{\pgfqpoint{3.768024in}{3.937028in}}%
\pgfpathlineto{\pgfqpoint{3.768445in}{3.939348in}}%
\pgfpathlineto{\pgfqpoint{3.770720in}{3.951630in}}%
\pgfpathlineto{\pgfqpoint{3.771394in}{3.952526in}}%
\pgfpathlineto{\pgfqpoint{3.773247in}{3.958557in}}%
\pgfpathlineto{\pgfqpoint{3.773921in}{3.958077in}}%
\pgfpathlineto{\pgfqpoint{3.775269in}{3.954633in}}%
\pgfpathlineto{\pgfqpoint{3.776617in}{3.950883in}}%
\pgfpathlineto{\pgfqpoint{3.777207in}{3.951396in}}%
\pgfpathlineto{\pgfqpoint{3.778471in}{3.952497in}}%
\pgfpathlineto{\pgfqpoint{3.778892in}{3.951988in}}%
\pgfpathlineto{\pgfqpoint{3.780829in}{3.950825in}}%
\pgfpathlineto{\pgfqpoint{3.782262in}{3.949718in}}%
\pgfpathlineto{\pgfqpoint{3.785042in}{3.940174in}}%
\pgfpathlineto{\pgfqpoint{3.787232in}{3.920915in}}%
\pgfpathlineto{\pgfqpoint{3.787485in}{3.921006in}}%
\pgfpathlineto{\pgfqpoint{3.788074in}{3.920846in}}%
\pgfpathlineto{\pgfqpoint{3.788243in}{3.920399in}}%
\pgfpathlineto{\pgfqpoint{3.790517in}{3.913586in}}%
\pgfpathlineto{\pgfqpoint{3.791023in}{3.913966in}}%
\pgfpathlineto{\pgfqpoint{3.793213in}{3.918144in}}%
\pgfpathlineto{\pgfqpoint{3.794224in}{3.920535in}}%
\pgfpathlineto{\pgfqpoint{3.794814in}{3.919507in}}%
\pgfpathlineto{\pgfqpoint{3.795656in}{3.918117in}}%
\pgfpathlineto{\pgfqpoint{3.796078in}{3.919040in}}%
\pgfpathlineto{\pgfqpoint{3.797510in}{3.930503in}}%
\pgfpathlineto{\pgfqpoint{3.799279in}{3.936537in}}%
\pgfpathlineto{\pgfqpoint{3.799447in}{3.936462in}}%
\pgfpathlineto{\pgfqpoint{3.800458in}{3.936622in}}%
\pgfpathlineto{\pgfqpoint{3.800627in}{3.936935in}}%
\pgfpathlineto{\pgfqpoint{3.801469in}{3.941162in}}%
\pgfpathlineto{\pgfqpoint{3.804081in}{3.959394in}}%
\pgfpathlineto{\pgfqpoint{3.804418in}{3.959254in}}%
\pgfpathlineto{\pgfqpoint{3.805176in}{3.960154in}}%
\pgfpathlineto{\pgfqpoint{3.806692in}{3.968102in}}%
\pgfpathlineto{\pgfqpoint{3.808462in}{3.971644in}}%
\pgfpathlineto{\pgfqpoint{3.809304in}{3.970519in}}%
\pgfpathlineto{\pgfqpoint{3.810905in}{3.967254in}}%
\pgfpathlineto{\pgfqpoint{3.811326in}{3.967849in}}%
\pgfpathlineto{\pgfqpoint{3.812590in}{3.972000in}}%
\pgfpathlineto{\pgfqpoint{3.813179in}{3.969915in}}%
\pgfpathlineto{\pgfqpoint{3.817391in}{3.948436in}}%
\pgfpathlineto{\pgfqpoint{3.818234in}{3.944673in}}%
\pgfpathlineto{\pgfqpoint{3.821267in}{3.927425in}}%
\pgfpathlineto{\pgfqpoint{3.821941in}{3.926307in}}%
\pgfpathlineto{\pgfqpoint{3.825732in}{3.914398in}}%
\pgfpathlineto{\pgfqpoint{3.826574in}{3.915066in}}%
\pgfpathlineto{\pgfqpoint{3.827585in}{3.914871in}}%
\pgfpathlineto{\pgfqpoint{3.827754in}{3.914519in}}%
\pgfpathlineto{\pgfqpoint{3.831376in}{3.907559in}}%
\pgfpathlineto{\pgfqpoint{3.834409in}{3.908590in}}%
\pgfpathlineto{\pgfqpoint{3.836010in}{3.909974in}}%
\pgfpathlineto{\pgfqpoint{3.836262in}{3.909791in}}%
\pgfpathlineto{\pgfqpoint{3.836936in}{3.909577in}}%
\pgfpathlineto{\pgfqpoint{3.837273in}{3.910132in}}%
\pgfpathlineto{\pgfqpoint{3.839632in}{3.912763in}}%
\pgfpathlineto{\pgfqpoint{3.840727in}{3.915682in}}%
\pgfpathlineto{\pgfqpoint{3.842665in}{3.921326in}}%
\pgfpathlineto{\pgfqpoint{3.842833in}{3.921267in}}%
\pgfpathlineto{\pgfqpoint{3.844602in}{3.920925in}}%
\pgfpathlineto{\pgfqpoint{3.844771in}{3.921262in}}%
\pgfpathlineto{\pgfqpoint{3.847130in}{3.924689in}}%
\pgfpathlineto{\pgfqpoint{3.851847in}{3.926284in}}%
\pgfpathlineto{\pgfqpoint{3.853785in}{3.934736in}}%
\pgfpathlineto{\pgfqpoint{3.854965in}{3.932748in}}%
\pgfpathlineto{\pgfqpoint{3.855638in}{3.934642in}}%
\pgfpathlineto{\pgfqpoint{3.857660in}{3.948115in}}%
\pgfpathlineto{\pgfqpoint{3.858671in}{3.944914in}}%
\pgfpathlineto{\pgfqpoint{3.859429in}{3.944555in}}%
\pgfpathlineto{\pgfqpoint{3.859766in}{3.945005in}}%
\pgfpathlineto{\pgfqpoint{3.861283in}{3.949427in}}%
\pgfpathlineto{\pgfqpoint{3.861957in}{3.948082in}}%
\pgfpathlineto{\pgfqpoint{3.863810in}{3.945306in}}%
\pgfpathlineto{\pgfqpoint{3.867012in}{3.945803in}}%
\pgfpathlineto{\pgfqpoint{3.868444in}{3.948728in}}%
\pgfpathlineto{\pgfqpoint{3.868949in}{3.947761in}}%
\pgfpathlineto{\pgfqpoint{3.870466in}{3.941368in}}%
\pgfpathlineto{\pgfqpoint{3.871139in}{3.943091in}}%
\pgfpathlineto{\pgfqpoint{3.872740in}{3.950907in}}%
\pgfpathlineto{\pgfqpoint{3.873246in}{3.948649in}}%
\pgfpathlineto{\pgfqpoint{3.874846in}{3.937258in}}%
\pgfpathlineto{\pgfqpoint{3.875352in}{3.938953in}}%
\pgfpathlineto{\pgfqpoint{3.876700in}{3.944135in}}%
\pgfpathlineto{\pgfqpoint{3.877205in}{3.942972in}}%
\pgfpathlineto{\pgfqpoint{3.879143in}{3.939291in}}%
\pgfpathlineto{\pgfqpoint{3.882849in}{3.940581in}}%
\pgfpathlineto{\pgfqpoint{3.883945in}{3.942457in}}%
\pgfpathlineto{\pgfqpoint{3.884450in}{3.941334in}}%
\pgfpathlineto{\pgfqpoint{3.885714in}{3.937717in}}%
\pgfpathlineto{\pgfqpoint{3.886135in}{3.938720in}}%
\pgfpathlineto{\pgfqpoint{3.887736in}{3.946803in}}%
\pgfpathlineto{\pgfqpoint{3.888494in}{3.944052in}}%
\pgfpathlineto{\pgfqpoint{3.889842in}{3.935258in}}%
\pgfpathlineto{\pgfqpoint{3.890431in}{3.937275in}}%
\pgfpathlineto{\pgfqpoint{3.891611in}{3.942563in}}%
\pgfpathlineto{\pgfqpoint{3.892201in}{3.940844in}}%
\pgfpathlineto{\pgfqpoint{3.893633in}{3.935465in}}%
\pgfpathlineto{\pgfqpoint{3.894222in}{3.935988in}}%
\pgfpathlineto{\pgfqpoint{3.897171in}{3.938131in}}%
\pgfpathlineto{\pgfqpoint{3.899361in}{3.936870in}}%
\pgfpathlineto{\pgfqpoint{3.901131in}{3.934168in}}%
\pgfpathlineto{\pgfqpoint{3.902057in}{3.935163in}}%
\pgfpathlineto{\pgfqpoint{3.902563in}{3.935112in}}%
\pgfpathlineto{\pgfqpoint{3.902815in}{3.934539in}}%
\pgfpathlineto{\pgfqpoint{3.904163in}{3.925996in}}%
\pgfpathlineto{\pgfqpoint{3.904922in}{3.923623in}}%
\pgfpathlineto{\pgfqpoint{3.905427in}{3.924421in}}%
\pgfpathlineto{\pgfqpoint{3.906606in}{3.928572in}}%
\pgfpathlineto{\pgfqpoint{3.907196in}{3.926518in}}%
\pgfpathlineto{\pgfqpoint{3.908797in}{3.919660in}}%
\pgfpathlineto{\pgfqpoint{3.909302in}{3.920409in}}%
\pgfpathlineto{\pgfqpoint{3.910987in}{3.922334in}}%
\pgfpathlineto{\pgfqpoint{3.911240in}{3.922179in}}%
\pgfpathlineto{\pgfqpoint{3.914694in}{3.920505in}}%
\pgfpathlineto{\pgfqpoint{3.917558in}{3.921996in}}%
\pgfpathlineto{\pgfqpoint{3.919243in}{3.923960in}}%
\pgfpathlineto{\pgfqpoint{3.919327in}{3.923925in}}%
\pgfpathlineto{\pgfqpoint{3.920507in}{3.923732in}}%
\pgfpathlineto{\pgfqpoint{3.920760in}{3.924253in}}%
\pgfpathlineto{\pgfqpoint{3.922107in}{3.929848in}}%
\pgfpathlineto{\pgfqpoint{3.922781in}{3.932744in}}%
\pgfpathlineto{\pgfqpoint{3.923287in}{3.931110in}}%
\pgfpathlineto{\pgfqpoint{3.924045in}{3.929135in}}%
\pgfpathlineto{\pgfqpoint{3.924551in}{3.930043in}}%
\pgfpathlineto{\pgfqpoint{3.926404in}{3.938715in}}%
\pgfpathlineto{\pgfqpoint{3.927415in}{3.936244in}}%
\pgfpathlineto{\pgfqpoint{3.928089in}{3.935470in}}%
\pgfpathlineto{\pgfqpoint{3.928594in}{3.936033in}}%
\pgfpathlineto{\pgfqpoint{3.933228in}{3.940915in}}%
\pgfpathlineto{\pgfqpoint{3.934491in}{3.941668in}}%
\pgfpathlineto{\pgfqpoint{3.937019in}{3.942173in}}%
\pgfpathlineto{\pgfqpoint{3.938030in}{3.940443in}}%
\pgfpathlineto{\pgfqpoint{3.938956in}{3.935220in}}%
\pgfpathlineto{\pgfqpoint{3.940725in}{3.928616in}}%
\pgfpathlineto{\pgfqpoint{3.941484in}{3.927191in}}%
\pgfpathlineto{\pgfqpoint{3.945359in}{3.916876in}}%
\pgfpathlineto{\pgfqpoint{3.951256in}{3.910546in}}%
\pgfpathlineto{\pgfqpoint{3.953362in}{3.905751in}}%
\pgfpathlineto{\pgfqpoint{3.959091in}{3.906469in}}%
\pgfpathlineto{\pgfqpoint{3.961618in}{3.908240in}}%
\pgfpathlineto{\pgfqpoint{3.963387in}{3.908236in}}%
\pgfpathlineto{\pgfqpoint{3.964398in}{3.911159in}}%
\pgfpathlineto{\pgfqpoint{3.966167in}{3.913686in}}%
\pgfpathlineto{\pgfqpoint{3.972233in}{3.916502in}}%
\pgfpathlineto{\pgfqpoint{3.974423in}{3.916209in}}%
\pgfpathlineto{\pgfqpoint{3.975855in}{3.915094in}}%
\pgfpathlineto{\pgfqpoint{3.978551in}{3.912427in}}%
\pgfpathlineto{\pgfqpoint{3.980910in}{3.911437in}}%
\pgfpathlineto{\pgfqpoint{3.983185in}{3.908257in}}%
\pgfpathlineto{\pgfqpoint{3.984701in}{3.908687in}}%
\pgfpathlineto{\pgfqpoint{3.986554in}{3.916321in}}%
\pgfpathlineto{\pgfqpoint{3.988155in}{3.919112in}}%
\pgfpathlineto{\pgfqpoint{3.988829in}{3.920737in}}%
\pgfpathlineto{\pgfqpoint{3.993715in}{3.938566in}}%
\pgfpathlineto{\pgfqpoint{3.996327in}{3.938956in}}%
\pgfpathlineto{\pgfqpoint{3.998180in}{3.939271in}}%
\pgfpathlineto{\pgfqpoint{3.999612in}{3.939087in}}%
\pgfpathlineto{\pgfqpoint{4.000286in}{3.939985in}}%
\pgfpathlineto{\pgfqpoint{4.001297in}{3.947398in}}%
\pgfpathlineto{\pgfqpoint{4.002140in}{3.952041in}}%
\pgfpathlineto{\pgfqpoint{4.002645in}{3.950879in}}%
\pgfpathlineto{\pgfqpoint{4.004077in}{3.943896in}}%
\pgfpathlineto{\pgfqpoint{4.004751in}{3.946239in}}%
\pgfpathlineto{\pgfqpoint{4.005762in}{3.949765in}}%
\pgfpathlineto{\pgfqpoint{4.006183in}{3.948622in}}%
\pgfpathlineto{\pgfqpoint{4.009048in}{3.938427in}}%
\pgfpathlineto{\pgfqpoint{4.012165in}{3.934764in}}%
\pgfpathlineto{\pgfqpoint{4.013428in}{3.935533in}}%
\pgfpathlineto{\pgfqpoint{4.014439in}{3.937480in}}%
\pgfpathlineto{\pgfqpoint{4.015029in}{3.936871in}}%
\pgfpathlineto{\pgfqpoint{4.016883in}{3.936250in}}%
\pgfpathlineto{\pgfqpoint{4.017556in}{3.935100in}}%
\pgfpathlineto{\pgfqpoint{4.021179in}{3.923689in}}%
\pgfpathlineto{\pgfqpoint{4.024717in}{3.919233in}}%
\pgfpathlineto{\pgfqpoint{4.025728in}{3.920545in}}%
\pgfpathlineto{\pgfqpoint{4.028256in}{3.924032in}}%
\pgfpathlineto{\pgfqpoint{4.030614in}{3.922784in}}%
\pgfpathlineto{\pgfqpoint{4.032636in}{3.920554in}}%
\pgfpathlineto{\pgfqpoint{4.034321in}{3.919653in}}%
\pgfpathlineto{\pgfqpoint{4.036427in}{3.916968in}}%
\pgfpathlineto{\pgfqpoint{4.036511in}{3.917005in}}%
\pgfpathlineto{\pgfqpoint{4.037775in}{3.916943in}}%
\pgfpathlineto{\pgfqpoint{4.037944in}{3.916583in}}%
\pgfpathlineto{\pgfqpoint{4.040808in}{3.912504in}}%
\pgfpathlineto{\pgfqpoint{4.042493in}{3.911866in}}%
\pgfpathlineto{\pgfqpoint{4.044346in}{3.908351in}}%
\pgfpathlineto{\pgfqpoint{4.044852in}{3.909160in}}%
\pgfpathlineto{\pgfqpoint{4.045610in}{3.910022in}}%
\pgfpathlineto{\pgfqpoint{4.046200in}{3.909573in}}%
\pgfpathlineto{\pgfqpoint{4.047126in}{3.910441in}}%
\pgfpathlineto{\pgfqpoint{4.048221in}{3.916478in}}%
\pgfpathlineto{\pgfqpoint{4.049569in}{3.920702in}}%
\pgfpathlineto{\pgfqpoint{4.049906in}{3.920584in}}%
\pgfpathlineto{\pgfqpoint{4.050665in}{3.920529in}}%
\pgfpathlineto{\pgfqpoint{4.050917in}{3.921181in}}%
\pgfpathlineto{\pgfqpoint{4.053697in}{3.927558in}}%
\pgfpathlineto{\pgfqpoint{4.061448in}{3.925010in}}%
\pgfpathlineto{\pgfqpoint{4.063385in}{3.923362in}}%
\pgfpathlineto{\pgfqpoint{4.064228in}{3.922231in}}%
\pgfpathlineto{\pgfqpoint{4.069451in}{3.905610in}}%
\pgfpathlineto{\pgfqpoint{4.071557in}{3.904660in}}%
\pgfpathlineto{\pgfqpoint{4.072484in}{3.906187in}}%
\pgfpathlineto{\pgfqpoint{4.074759in}{3.909791in}}%
\pgfpathlineto{\pgfqpoint{4.076191in}{3.910489in}}%
\pgfpathlineto{\pgfqpoint{4.079645in}{3.911505in}}%
\pgfpathlineto{\pgfqpoint{4.086300in}{3.909990in}}%
\pgfpathlineto{\pgfqpoint{4.088238in}{3.910019in}}%
\pgfpathlineto{\pgfqpoint{4.089838in}{3.911422in}}%
\pgfpathlineto{\pgfqpoint{4.090260in}{3.910998in}}%
\pgfpathlineto{\pgfqpoint{4.091355in}{3.910404in}}%
\pgfpathlineto{\pgfqpoint{4.091692in}{3.910961in}}%
\pgfpathlineto{\pgfqpoint{4.093545in}{3.918754in}}%
\pgfpathlineto{\pgfqpoint{4.095146in}{3.920032in}}%
\pgfpathlineto{\pgfqpoint{4.096662in}{3.921137in}}%
\pgfpathlineto{\pgfqpoint{4.098768in}{3.922197in}}%
\pgfpathlineto{\pgfqpoint{4.100622in}{3.923246in}}%
\pgfpathlineto{\pgfqpoint{4.103991in}{3.926476in}}%
\pgfpathlineto{\pgfqpoint{4.105171in}{3.928432in}}%
\pgfpathlineto{\pgfqpoint{4.105592in}{3.927925in}}%
\pgfpathlineto{\pgfqpoint{4.108456in}{3.923995in}}%
\pgfpathlineto{\pgfqpoint{4.110478in}{3.921642in}}%
\pgfpathlineto{\pgfqpoint{4.111068in}{3.922252in}}%
\pgfpathlineto{\pgfqpoint{4.112753in}{3.923945in}}%
\pgfpathlineto{\pgfqpoint{4.113343in}{3.923470in}}%
\pgfpathlineto{\pgfqpoint{4.114101in}{3.924196in}}%
\pgfpathlineto{\pgfqpoint{4.115533in}{3.931258in}}%
\pgfpathlineto{\pgfqpoint{4.117218in}{3.937849in}}%
\pgfpathlineto{\pgfqpoint{4.117723in}{3.937569in}}%
\pgfpathlineto{\pgfqpoint{4.118903in}{3.938088in}}%
\pgfpathlineto{\pgfqpoint{4.119914in}{3.942292in}}%
\pgfpathlineto{\pgfqpoint{4.121430in}{3.947165in}}%
\pgfpathlineto{\pgfqpoint{4.121767in}{3.946949in}}%
\pgfpathlineto{\pgfqpoint{4.122946in}{3.946382in}}%
\pgfpathlineto{\pgfqpoint{4.123283in}{3.947014in}}%
\pgfpathlineto{\pgfqpoint{4.125726in}{3.954397in}}%
\pgfpathlineto{\pgfqpoint{4.126316in}{3.952890in}}%
\pgfpathlineto{\pgfqpoint{4.129012in}{3.945596in}}%
\pgfpathlineto{\pgfqpoint{4.130697in}{3.945078in}}%
\pgfpathlineto{\pgfqpoint{4.132298in}{3.937481in}}%
\pgfpathlineto{\pgfqpoint{4.134067in}{3.929620in}}%
\pgfpathlineto{\pgfqpoint{4.134656in}{3.930387in}}%
\pgfpathlineto{\pgfqpoint{4.135667in}{3.932075in}}%
\pgfpathlineto{\pgfqpoint{4.136173in}{3.931079in}}%
\pgfpathlineto{\pgfqpoint{4.137942in}{3.921248in}}%
\pgfpathlineto{\pgfqpoint{4.138784in}{3.924645in}}%
\pgfpathlineto{\pgfqpoint{4.140048in}{3.931171in}}%
\pgfpathlineto{\pgfqpoint{4.140553in}{3.929818in}}%
\pgfpathlineto{\pgfqpoint{4.142491in}{3.924901in}}%
\pgfpathlineto{\pgfqpoint{4.142744in}{3.925020in}}%
\pgfpathlineto{\pgfqpoint{4.145524in}{3.925832in}}%
\pgfpathlineto{\pgfqpoint{4.147714in}{3.924870in}}%
\pgfpathlineto{\pgfqpoint{4.150242in}{3.925197in}}%
\pgfpathlineto{\pgfqpoint{4.152179in}{3.924466in}}%
\pgfpathlineto{\pgfqpoint{4.153611in}{3.924218in}}%
\pgfpathlineto{\pgfqpoint{4.154454in}{3.923568in}}%
\pgfpathlineto{\pgfqpoint{4.155633in}{3.917515in}}%
\pgfpathlineto{\pgfqpoint{4.157234in}{3.911915in}}%
\pgfpathlineto{\pgfqpoint{4.157487in}{3.912026in}}%
\pgfpathlineto{\pgfqpoint{4.158835in}{3.911544in}}%
\pgfpathlineto{\pgfqpoint{4.161783in}{3.908696in}}%
\pgfpathlineto{\pgfqpoint{4.172903in}{3.906448in}}%
\pgfpathlineto{\pgfqpoint{4.175009in}{3.904630in}}%
\pgfpathlineto{\pgfqpoint{4.175178in}{3.904741in}}%
\pgfpathlineto{\pgfqpoint{4.176947in}{3.904710in}}%
\pgfpathlineto{\pgfqpoint{4.179980in}{3.904631in}}%
\pgfpathlineto{\pgfqpoint{4.183855in}{3.905968in}}%
\pgfpathlineto{\pgfqpoint{4.185624in}{3.907336in}}%
\pgfpathlineto{\pgfqpoint{4.186972in}{3.907765in}}%
\pgfpathlineto{\pgfqpoint{4.190258in}{3.910375in}}%
\pgfpathlineto{\pgfqpoint{4.200283in}{3.909763in}}%
\pgfpathlineto{\pgfqpoint{4.202726in}{3.907066in}}%
\pgfpathlineto{\pgfqpoint{4.203990in}{3.906304in}}%
\pgfpathlineto{\pgfqpoint{4.207444in}{3.904202in}}%
\pgfpathlineto{\pgfqpoint{4.213593in}{3.904773in}}%
\pgfpathlineto{\pgfqpoint{4.216458in}{3.905766in}}%
\pgfpathlineto{\pgfqpoint{4.218058in}{3.907012in}}%
\pgfpathlineto{\pgfqpoint{4.219575in}{3.907435in}}%
\pgfpathlineto{\pgfqpoint{4.222945in}{3.907615in}}%
\pgfpathlineto{\pgfqpoint{4.225809in}{3.908335in}}%
\pgfpathlineto{\pgfqpoint{4.227073in}{3.909390in}}%
\pgfpathlineto{\pgfqpoint{4.229010in}{3.909985in}}%
\pgfpathlineto{\pgfqpoint{4.232380in}{3.908112in}}%
\pgfpathlineto{\pgfqpoint{4.234739in}{3.907402in}}%
\pgfpathlineto{\pgfqpoint{4.236761in}{3.906783in}}%
\pgfpathlineto{\pgfqpoint{4.239541in}{3.905883in}}%
\pgfpathlineto{\pgfqpoint{4.242574in}{3.905402in}}%
\pgfpathlineto{\pgfqpoint{4.243669in}{3.905599in}}%
\pgfpathlineto{\pgfqpoint{4.243922in}{3.905325in}}%
\pgfpathlineto{\pgfqpoint{4.245775in}{3.904615in}}%
\pgfpathlineto{\pgfqpoint{4.245943in}{3.904838in}}%
\pgfpathlineto{\pgfqpoint{4.248723in}{3.907330in}}%
\pgfpathlineto{\pgfqpoint{4.250745in}{3.907609in}}%
\pgfpathlineto{\pgfqpoint{4.253020in}{3.908468in}}%
\pgfpathlineto{\pgfqpoint{4.260349in}{3.907706in}}%
\pgfpathlineto{\pgfqpoint{4.270964in}{3.903700in}}%
\pgfpathlineto{\pgfqpoint{4.277956in}{3.904392in}}%
\pgfpathlineto{\pgfqpoint{4.284275in}{3.907337in}}%
\pgfpathlineto{\pgfqpoint{4.288487in}{3.907770in}}%
\pgfpathlineto{\pgfqpoint{4.294300in}{3.910625in}}%
\pgfpathlineto{\pgfqpoint{4.300197in}{3.907467in}}%
\pgfpathlineto{\pgfqpoint{4.309127in}{3.906886in}}%
\pgfpathlineto{\pgfqpoint{4.313086in}{3.904817in}}%
\pgfpathlineto{\pgfqpoint{4.315614in}{3.905888in}}%
\pgfpathlineto{\pgfqpoint{4.317467in}{3.906911in}}%
\pgfpathlineto{\pgfqpoint{4.319742in}{3.907242in}}%
\pgfpathlineto{\pgfqpoint{4.321342in}{3.907602in}}%
\pgfpathlineto{\pgfqpoint{4.325554in}{3.906488in}}%
\pgfpathlineto{\pgfqpoint{4.330862in}{3.906996in}}%
\pgfpathlineto{\pgfqpoint{4.332294in}{3.906184in}}%
\pgfpathlineto{\pgfqpoint{4.333473in}{3.905087in}}%
\pgfpathlineto{\pgfqpoint{4.333979in}{3.905480in}}%
\pgfpathlineto{\pgfqpoint{4.336169in}{3.905721in}}%
\pgfpathlineto{\pgfqpoint{4.337770in}{3.904548in}}%
\pgfpathlineto{\pgfqpoint{4.338107in}{3.904922in}}%
\pgfpathlineto{\pgfqpoint{4.340297in}{3.905931in}}%
\pgfpathlineto{\pgfqpoint{4.349564in}{3.906575in}}%
\pgfpathlineto{\pgfqpoint{4.354113in}{3.906425in}}%
\pgfpathlineto{\pgfqpoint{4.355209in}{3.905845in}}%
\pgfpathlineto{\pgfqpoint{4.357230in}{3.905159in}}%
\pgfpathlineto{\pgfqpoint{4.360263in}{3.904793in}}%
\pgfpathlineto{\pgfqpoint{4.362791in}{3.904819in}}%
\pgfpathlineto{\pgfqpoint{4.364644in}{3.906032in}}%
\pgfpathlineto{\pgfqpoint{4.367003in}{3.906555in}}%
\pgfpathlineto{\pgfqpoint{4.370710in}{3.905705in}}%
\pgfpathlineto{\pgfqpoint{4.372647in}{3.905852in}}%
\pgfpathlineto{\pgfqpoint{4.374332in}{3.905679in}}%
\pgfpathlineto{\pgfqpoint{4.376017in}{3.906764in}}%
\pgfpathlineto{\pgfqpoint{4.380482in}{3.910347in}}%
\pgfpathlineto{\pgfqpoint{4.380566in}{3.910270in}}%
\pgfpathlineto{\pgfqpoint{4.381661in}{3.910337in}}%
\pgfpathlineto{\pgfqpoint{4.381830in}{3.910540in}}%
\pgfpathlineto{\pgfqpoint{4.383178in}{3.914346in}}%
\pgfpathlineto{\pgfqpoint{4.384526in}{3.915510in}}%
\pgfpathlineto{\pgfqpoint{4.384694in}{3.915440in}}%
\pgfpathlineto{\pgfqpoint{4.386548in}{3.915795in}}%
\pgfpathlineto{\pgfqpoint{4.389075in}{3.917480in}}%
\pgfpathlineto{\pgfqpoint{4.394466in}{3.916011in}}%
\pgfpathlineto{\pgfqpoint{4.395477in}{3.915101in}}%
\pgfpathlineto{\pgfqpoint{4.402133in}{3.908026in}}%
\pgfpathlineto{\pgfqpoint{4.403649in}{3.907015in}}%
\pgfpathlineto{\pgfqpoint{4.404660in}{3.906828in}}%
\pgfpathlineto{\pgfqpoint{4.404913in}{3.907045in}}%
\pgfpathlineto{\pgfqpoint{4.407187in}{3.907527in}}%
\pgfpathlineto{\pgfqpoint{4.408704in}{3.908241in}}%
\pgfpathlineto{\pgfqpoint{4.410389in}{3.909400in}}%
\pgfpathlineto{\pgfqpoint{4.410641in}{3.909199in}}%
\pgfpathlineto{\pgfqpoint{4.412579in}{3.908977in}}%
\pgfpathlineto{\pgfqpoint{4.415106in}{3.908875in}}%
\pgfpathlineto{\pgfqpoint{4.432629in}{3.907358in}}%
\pgfpathlineto{\pgfqpoint{4.434398in}{3.908058in}}%
\pgfpathlineto{\pgfqpoint{4.435494in}{3.908058in}}%
\pgfpathlineto{\pgfqpoint{4.435662in}{3.908389in}}%
\pgfpathlineto{\pgfqpoint{4.438948in}{3.913671in}}%
\pgfpathlineto{\pgfqpoint{4.439959in}{3.915052in}}%
\pgfpathlineto{\pgfqpoint{4.443076in}{3.919123in}}%
\pgfpathlineto{\pgfqpoint{4.445350in}{3.920250in}}%
\pgfpathlineto{\pgfqpoint{4.449225in}{3.920216in}}%
\pgfpathlineto{\pgfqpoint{4.450152in}{3.919687in}}%
\pgfpathlineto{\pgfqpoint{4.450573in}{3.920171in}}%
\pgfpathlineto{\pgfqpoint{4.452258in}{3.921651in}}%
\pgfpathlineto{\pgfqpoint{4.452511in}{3.921269in}}%
\pgfpathlineto{\pgfqpoint{4.454449in}{3.916691in}}%
\pgfpathlineto{\pgfqpoint{4.455375in}{3.917279in}}%
\pgfpathlineto{\pgfqpoint{4.456302in}{3.916366in}}%
\pgfpathlineto{\pgfqpoint{4.459756in}{3.910788in}}%
\pgfpathlineto{\pgfqpoint{4.462283in}{3.910140in}}%
\pgfpathlineto{\pgfqpoint{4.465906in}{3.908996in}}%
\pgfpathlineto{\pgfqpoint{4.467675in}{3.908821in}}%
\pgfpathlineto{\pgfqpoint{4.470034in}{3.907808in}}%
\pgfpathlineto{\pgfqpoint{4.471466in}{3.907616in}}%
\pgfpathlineto{\pgfqpoint{4.475847in}{3.906540in}}%
\pgfpathlineto{\pgfqpoint{4.482334in}{3.909788in}}%
\pgfpathlineto{\pgfqpoint{4.484103in}{3.910671in}}%
\pgfpathlineto{\pgfqpoint{4.485956in}{3.911440in}}%
\pgfpathlineto{\pgfqpoint{4.488568in}{3.914430in}}%
\pgfpathlineto{\pgfqpoint{4.490421in}{3.915091in}}%
\pgfpathlineto{\pgfqpoint{4.492696in}{3.915878in}}%
\pgfpathlineto{\pgfqpoint{4.495813in}{3.914568in}}%
\pgfpathlineto{\pgfqpoint{4.497835in}{3.914260in}}%
\pgfpathlineto{\pgfqpoint{4.498761in}{3.914654in}}%
\pgfpathlineto{\pgfqpoint{4.499098in}{3.914112in}}%
\pgfpathlineto{\pgfqpoint{4.500783in}{3.912445in}}%
\pgfpathlineto{\pgfqpoint{4.501036in}{3.912642in}}%
\pgfpathlineto{\pgfqpoint{4.502721in}{3.914939in}}%
\pgfpathlineto{\pgfqpoint{4.503310in}{3.913933in}}%
\pgfpathlineto{\pgfqpoint{4.505838in}{3.911096in}}%
\pgfpathlineto{\pgfqpoint{4.507775in}{3.910696in}}%
\pgfpathlineto{\pgfqpoint{4.510471in}{3.909923in}}%
\pgfpathlineto{\pgfqpoint{4.511651in}{3.911157in}}%
\pgfpathlineto{\pgfqpoint{4.513420in}{3.914421in}}%
\pgfpathlineto{\pgfqpoint{4.514009in}{3.913999in}}%
\pgfpathlineto{\pgfqpoint{4.516031in}{3.913566in}}%
\pgfpathlineto{\pgfqpoint{4.517716in}{3.913921in}}%
\pgfpathlineto{\pgfqpoint{4.517885in}{3.913700in}}%
\pgfpathlineto{\pgfqpoint{4.521591in}{3.909818in}}%
\pgfpathlineto{\pgfqpoint{4.527994in}{3.908338in}}%
\pgfpathlineto{\pgfqpoint{4.529595in}{3.904580in}}%
\pgfpathlineto{\pgfqpoint{4.531195in}{3.903708in}}%
\pgfpathlineto{\pgfqpoint{4.533217in}{3.902755in}}%
\pgfpathlineto{\pgfqpoint{4.537598in}{3.902255in}}%
\pgfpathlineto{\pgfqpoint{4.542147in}{3.902423in}}%
\pgfpathlineto{\pgfqpoint{4.552425in}{3.902204in}}%
\pgfpathlineto{\pgfqpoint{4.567168in}{3.905774in}}%
\pgfpathlineto{\pgfqpoint{4.571548in}{3.904168in}}%
\pgfpathlineto{\pgfqpoint{4.574160in}{3.903587in}}%
\pgfpathlineto{\pgfqpoint{4.578288in}{3.902961in}}%
\pgfpathlineto{\pgfqpoint{4.589830in}{3.906033in}}%
\pgfpathlineto{\pgfqpoint{4.599265in}{3.905798in}}%
\pgfpathlineto{\pgfqpoint{4.603983in}{3.903545in}}%
\pgfpathlineto{\pgfqpoint{4.609880in}{3.902779in}}%
\pgfpathlineto{\pgfqpoint{4.614850in}{3.901848in}}%
\pgfpathlineto{\pgfqpoint{4.621000in}{3.901575in}}%
\pgfpathlineto{\pgfqpoint{4.627487in}{3.903498in}}%
\pgfpathlineto{\pgfqpoint{4.631699in}{3.905909in}}%
\pgfpathlineto{\pgfqpoint{4.633300in}{3.907039in}}%
\pgfpathlineto{\pgfqpoint{4.636417in}{3.908154in}}%
\pgfpathlineto{\pgfqpoint{4.640376in}{3.907410in}}%
\pgfpathlineto{\pgfqpoint{4.647537in}{3.905597in}}%
\pgfpathlineto{\pgfqpoint{4.649390in}{3.905716in}}%
\pgfpathlineto{\pgfqpoint{4.651918in}{3.904854in}}%
\pgfpathlineto{\pgfqpoint{4.663628in}{3.904278in}}%
\pgfpathlineto{\pgfqpoint{4.666576in}{3.903659in}}%
\pgfpathlineto{\pgfqpoint{4.676938in}{3.903687in}}%
\pgfpathlineto{\pgfqpoint{4.679381in}{3.902961in}}%
\pgfpathlineto{\pgfqpoint{4.680814in}{3.903674in}}%
\pgfpathlineto{\pgfqpoint{4.682583in}{3.904056in}}%
\pgfpathlineto{\pgfqpoint{4.687300in}{3.904720in}}%
\pgfpathlineto{\pgfqpoint{4.695219in}{3.908489in}}%
\pgfpathlineto{\pgfqpoint{4.697073in}{3.908906in}}%
\pgfpathlineto{\pgfqpoint{4.699263in}{3.909508in}}%
\pgfpathlineto{\pgfqpoint{4.700864in}{3.909512in}}%
\pgfpathlineto{\pgfqpoint{4.703307in}{3.908916in}}%
\pgfpathlineto{\pgfqpoint{4.710805in}{3.906991in}}%
\pgfpathlineto{\pgfqpoint{4.713501in}{3.904614in}}%
\pgfpathlineto{\pgfqpoint{4.715775in}{3.904717in}}%
\pgfpathlineto{\pgfqpoint{4.719061in}{3.904774in}}%
\pgfpathlineto{\pgfqpoint{4.721672in}{3.905703in}}%
\pgfpathlineto{\pgfqpoint{4.731360in}{3.904361in}}%
\pgfpathlineto{\pgfqpoint{4.734056in}{3.903824in}}%
\pgfpathlineto{\pgfqpoint{4.736921in}{3.902930in}}%
\pgfpathlineto{\pgfqpoint{4.740543in}{3.902593in}}%
\pgfpathlineto{\pgfqpoint{4.744418in}{3.903228in}}%
\pgfpathlineto{\pgfqpoint{4.746946in}{3.903885in}}%
\pgfpathlineto{\pgfqpoint{4.753601in}{3.904864in}}%
\pgfpathlineto{\pgfqpoint{4.764890in}{3.905448in}}%
\pgfpathlineto{\pgfqpoint{4.766575in}{3.905602in}}%
\pgfpathlineto{\pgfqpoint{4.769186in}{3.906125in}}%
\pgfpathlineto{\pgfqpoint{4.772387in}{3.906964in}}%
\pgfpathlineto{\pgfqpoint{4.775083in}{3.907074in}}%
\pgfpathlineto{\pgfqpoint{4.776684in}{3.907610in}}%
\pgfpathlineto{\pgfqpoint{4.778706in}{3.908370in}}%
\pgfpathlineto{\pgfqpoint{4.780728in}{3.909003in}}%
\pgfpathlineto{\pgfqpoint{4.782918in}{3.909789in}}%
\pgfpathlineto{\pgfqpoint{4.785108in}{3.909540in}}%
\pgfpathlineto{\pgfqpoint{4.787467in}{3.909727in}}%
\pgfpathlineto{\pgfqpoint{4.792775in}{3.907868in}}%
\pgfpathlineto{\pgfqpoint{4.795555in}{3.905867in}}%
\pgfpathlineto{\pgfqpoint{4.796987in}{3.904761in}}%
\pgfpathlineto{\pgfqpoint{4.799430in}{3.903049in}}%
\pgfpathlineto{\pgfqpoint{4.802379in}{3.902355in}}%
\pgfpathlineto{\pgfqpoint{4.805159in}{3.902134in}}%
\pgfpathlineto{\pgfqpoint{4.831527in}{3.901291in}}%
\pgfpathlineto{\pgfqpoint{4.835992in}{3.901028in}}%
\pgfpathlineto{\pgfqpoint{4.856379in}{3.902032in}}%
\pgfpathlineto{\pgfqpoint{4.864130in}{3.903251in}}%
\pgfpathlineto{\pgfqpoint{4.866741in}{3.903588in}}%
\pgfpathlineto{\pgfqpoint{4.876345in}{3.903902in}}%
\pgfpathlineto{\pgfqpoint{4.887887in}{3.901563in}}%
\pgfpathlineto{\pgfqpoint{4.921248in}{3.902221in}}%
\pgfpathlineto{\pgfqpoint{4.928071in}{3.904070in}}%
\pgfpathlineto{\pgfqpoint{4.936159in}{3.905358in}}%
\pgfpathlineto{\pgfqpoint{4.938349in}{3.905632in}}%
\pgfpathlineto{\pgfqpoint{4.940540in}{3.905654in}}%
\pgfpathlineto{\pgfqpoint{4.942561in}{3.905672in}}%
\pgfpathlineto{\pgfqpoint{4.961095in}{3.902032in}}%
\pgfpathlineto{\pgfqpoint{4.979797in}{3.901760in}}%
\pgfpathlineto{\pgfqpoint{4.983336in}{3.902278in}}%
\pgfpathlineto{\pgfqpoint{4.985695in}{3.902345in}}%
\pgfpathlineto{\pgfqpoint{4.993361in}{3.903409in}}%
\pgfpathlineto{\pgfqpoint{4.996815in}{3.903429in}}%
\pgfpathlineto{\pgfqpoint{5.003639in}{3.902452in}}%
\pgfpathlineto{\pgfqpoint{5.019224in}{3.903717in}}%
\pgfpathlineto{\pgfqpoint{5.025879in}{3.903822in}}%
\pgfpathlineto{\pgfqpoint{5.032703in}{3.903349in}}%
\pgfpathlineto{\pgfqpoint{5.038179in}{3.903571in}}%
\pgfpathlineto{\pgfqpoint{5.048120in}{3.903006in}}%
\pgfpathlineto{\pgfqpoint{5.056460in}{3.902944in}}%
\pgfpathlineto{\pgfqpoint{5.062020in}{3.902760in}}%
\pgfpathlineto{\pgfqpoint{5.071203in}{3.901741in}}%
\pgfpathlineto{\pgfqpoint{5.089905in}{3.902230in}}%
\pgfpathlineto{\pgfqpoint{5.093949in}{3.902807in}}%
\pgfpathlineto{\pgfqpoint{5.110292in}{3.903605in}}%
\pgfpathlineto{\pgfqpoint{5.127478in}{3.902638in}}%
\pgfpathlineto{\pgfqpoint{5.131775in}{3.902749in}}%
\pgfpathlineto{\pgfqpoint{5.138346in}{3.902269in}}%
\pgfpathlineto{\pgfqpoint{5.142305in}{3.902764in}}%
\pgfpathlineto{\pgfqpoint{5.163872in}{3.901795in}}%
\pgfpathlineto{\pgfqpoint{5.165894in}{3.901489in}}%
\pgfpathlineto{\pgfqpoint{5.197317in}{3.901302in}}%
\pgfpathlineto{\pgfqpoint{5.278444in}{3.902396in}}%
\pgfpathlineto{\pgfqpoint{5.308435in}{3.902597in}}%
\pgfpathlineto{\pgfqpoint{5.314164in}{3.901725in}}%
\pgfpathlineto{\pgfqpoint{5.327475in}{3.902754in}}%
\pgfpathlineto{\pgfqpoint{5.330002in}{3.903177in}}%
\pgfpathlineto{\pgfqpoint{5.332698in}{3.904284in}}%
\pgfpathlineto{\pgfqpoint{5.337752in}{3.904817in}}%
\pgfpathlineto{\pgfqpoint{5.344492in}{3.903202in}}%
\pgfpathlineto{\pgfqpoint{5.347104in}{3.902888in}}%
\pgfpathlineto{\pgfqpoint{5.349547in}{3.901869in}}%
\pgfpathlineto{\pgfqpoint{5.354012in}{3.901707in}}%
\pgfpathlineto{\pgfqpoint{5.367828in}{3.901902in}}%
\pgfpathlineto{\pgfqpoint{5.372209in}{3.901663in}}%
\pgfpathlineto{\pgfqpoint{5.380296in}{3.902561in}}%
\pgfpathlineto{\pgfqpoint{5.387120in}{3.903314in}}%
\pgfpathlineto{\pgfqpoint{5.391753in}{3.903573in}}%
\pgfpathlineto{\pgfqpoint{5.398914in}{3.902205in}}%
\pgfpathlineto{\pgfqpoint{5.403463in}{3.901542in}}%
\pgfpathlineto{\pgfqpoint{5.457969in}{3.901954in}}%
\pgfpathlineto{\pgfqpoint{5.463867in}{3.901375in}}%
\pgfpathlineto{\pgfqpoint{5.509780in}{3.901960in}}%
\pgfpathlineto{\pgfqpoint{5.523427in}{3.901247in}}%
\pgfpathlineto{\pgfqpoint{5.572205in}{3.902350in}}%
\pgfpathlineto{\pgfqpoint{5.576165in}{3.902626in}}%
\pgfpathlineto{\pgfqpoint{5.588885in}{3.902488in}}%
\pgfpathlineto{\pgfqpoint{5.591076in}{3.902488in}}%
\pgfpathlineto{\pgfqpoint{5.595457in}{3.902325in}}%
\pgfpathlineto{\pgfqpoint{5.605229in}{3.901547in}}%
\pgfpathlineto{\pgfqpoint{5.717527in}{3.901421in}}%
\pgfpathlineto{\pgfqpoint{5.723424in}{3.901464in}}%
\pgfpathlineto{\pgfqpoint{5.745833in}{3.902079in}}%
\pgfpathlineto{\pgfqpoint{5.745833in}{3.902079in}}%
\pgfusepath{stroke}%
\end{pgfscope}%
\begin{pgfscope}%
\pgfpathrectangle{\pgfqpoint{0.691161in}{3.737081in}}{\pgfqpoint{5.054672in}{0.902317in}}%
\pgfusepath{clip}%
\pgfsetbuttcap%
\pgfsetroundjoin%
\pgfsetlinewidth{2.007500pt}%
\definecolor{currentstroke}{rgb}{0.172549,0.627451,0.172549}%
\pgfsetstrokecolor{currentstroke}%
\pgfsetdash{{7.400000pt}{3.200000pt}}{0.000000pt}%
\pgfpathmoveto{\pgfqpoint{0.691161in}{4.350081in}}%
\pgfpathlineto{\pgfqpoint{5.745833in}{4.350081in}}%
\pgfusepath{stroke}%
\end{pgfscope}%
\begin{pgfscope}%
\pgfpathrectangle{\pgfqpoint{0.691161in}{3.737081in}}{\pgfqpoint{5.054672in}{0.902317in}}%
\pgfusepath{clip}%
\pgfsetrectcap%
\pgfsetroundjoin%
\pgfsetlinewidth{2.007500pt}%
\definecolor{currentstroke}{rgb}{0.172549,0.627451,0.172549}%
\pgfsetstrokecolor{currentstroke}%
\pgfsetdash{}{0pt}%
\pgfpathmoveto{\pgfqpoint{0.691161in}{3.941520in}}%
\pgfpathlineto{\pgfqpoint{3.455645in}{3.941520in}}%
\pgfpathlineto{\pgfqpoint{3.456825in}{3.859808in}}%
\pgfpathlineto{\pgfqpoint{3.471736in}{3.859808in}}%
\pgfpathlineto{\pgfqpoint{3.472916in}{3.941520in}}%
\pgfpathlineto{\pgfqpoint{3.481677in}{3.941520in}}%
\pgfpathlineto{\pgfqpoint{3.482856in}{3.859808in}}%
\pgfpathlineto{\pgfqpoint{3.497262in}{3.859808in}}%
\pgfpathlineto{\pgfqpoint{3.498442in}{3.941520in}}%
\pgfpathlineto{\pgfqpoint{3.534414in}{3.941520in}}%
\pgfpathlineto{\pgfqpoint{3.535594in}{3.859808in}}%
\pgfpathlineto{\pgfqpoint{3.552695in}{3.859808in}}%
\pgfpathlineto{\pgfqpoint{3.553875in}{3.941520in}}%
\pgfpathlineto{\pgfqpoint{3.565079in}{3.941520in}}%
\pgfpathlineto{\pgfqpoint{3.566259in}{3.859808in}}%
\pgfpathlineto{\pgfqpoint{3.575525in}{3.859808in}}%
\pgfpathlineto{\pgfqpoint{3.576705in}{3.941520in}}%
\pgfpathlineto{\pgfqpoint{5.745833in}{3.941520in}}%
\pgfpathlineto{\pgfqpoint{5.745833in}{3.941520in}}%
\pgfusepath{stroke}%
\end{pgfscope}%
\begin{pgfscope}%
\pgfpathrectangle{\pgfqpoint{0.691161in}{3.737081in}}{\pgfqpoint{5.054672in}{0.902317in}}%
\pgfusepath{clip}%
\pgfsetrectcap%
\pgfsetroundjoin%
\pgfsetlinewidth{3.011250pt}%
\definecolor{currentstroke}{rgb}{0.839216,0.152941,0.156863}%
\pgfsetstrokecolor{currentstroke}%
\pgfsetdash{}{0pt}%
\pgfpathmoveto{\pgfqpoint{0.691161in}{3.941520in}}%
\pgfpathlineto{\pgfqpoint{1.326449in}{3.941520in}}%
\pgfpathlineto{\pgfqpoint{1.327629in}{3.778095in}}%
\pgfpathlineto{\pgfqpoint{4.305841in}{3.778095in}}%
\pgfpathlineto{\pgfqpoint{4.307021in}{3.941520in}}%
\pgfpathlineto{\pgfqpoint{5.745833in}{3.941520in}}%
\pgfpathlineto{\pgfqpoint{5.745833in}{3.941520in}}%
\pgfusepath{stroke}%
\end{pgfscope}%
\begin{pgfscope}%
\pgfsetrectcap%
\pgfsetmiterjoin%
\pgfsetlinewidth{0.803000pt}%
\definecolor{currentstroke}{rgb}{0.737255,0.737255,0.737255}%
\pgfsetstrokecolor{currentstroke}%
\pgfsetdash{}{0pt}%
\pgfpathmoveto{\pgfqpoint{0.691161in}{3.737081in}}%
\pgfpathlineto{\pgfqpoint{0.691161in}{4.639398in}}%
\pgfusepath{stroke}%
\end{pgfscope}%
\begin{pgfscope}%
\pgfsetrectcap%
\pgfsetmiterjoin%
\pgfsetlinewidth{0.803000pt}%
\definecolor{currentstroke}{rgb}{0.737255,0.737255,0.737255}%
\pgfsetstrokecolor{currentstroke}%
\pgfsetdash{}{0pt}%
\pgfpathmoveto{\pgfqpoint{5.745833in}{3.737081in}}%
\pgfpathlineto{\pgfqpoint{5.745833in}{4.639398in}}%
\pgfusepath{stroke}%
\end{pgfscope}%
\begin{pgfscope}%
\pgfsetrectcap%
\pgfsetmiterjoin%
\pgfsetlinewidth{0.803000pt}%
\definecolor{currentstroke}{rgb}{0.737255,0.737255,0.737255}%
\pgfsetstrokecolor{currentstroke}%
\pgfsetdash{}{0pt}%
\pgfpathmoveto{\pgfqpoint{0.691161in}{3.737081in}}%
\pgfpathlineto{\pgfqpoint{5.745833in}{3.737081in}}%
\pgfusepath{stroke}%
\end{pgfscope}%
\begin{pgfscope}%
\pgfsetrectcap%
\pgfsetmiterjoin%
\pgfsetlinewidth{0.803000pt}%
\definecolor{currentstroke}{rgb}{0.737255,0.737255,0.737255}%
\pgfsetstrokecolor{currentstroke}%
\pgfsetdash{}{0pt}%
\pgfpathmoveto{\pgfqpoint{0.691161in}{4.639398in}}%
\pgfpathlineto{\pgfqpoint{5.745833in}{4.639398in}}%
\pgfusepath{stroke}%
\end{pgfscope}%
\begin{pgfscope}%
\pgfsetbuttcap%
\pgfsetmiterjoin%
\definecolor{currentfill}{rgb}{0.933333,0.933333,0.933333}%
\pgfsetfillcolor{currentfill}%
\pgfsetfillopacity{0.800000}%
\pgfsetlinewidth{0.501875pt}%
\definecolor{currentstroke}{rgb}{0.800000,0.800000,0.800000}%
\pgfsetstrokecolor{currentstroke}%
\pgfsetstrokeopacity{0.800000}%
\pgfsetdash{}{0pt}%
\pgfpathmoveto{\pgfqpoint{4.413888in}{3.947454in}}%
\pgfpathlineto{\pgfqpoint{5.648611in}{3.947454in}}%
\pgfpathquadraticcurveto{\pgfqpoint{5.676389in}{3.947454in}}{\pgfqpoint{5.676389in}{3.975232in}}%
\pgfpathlineto{\pgfqpoint{5.676389in}{4.542176in}}%
\pgfpathquadraticcurveto{\pgfqpoint{5.676389in}{4.569954in}}{\pgfqpoint{5.648611in}{4.569954in}}%
\pgfpathlineto{\pgfqpoint{4.413888in}{4.569954in}}%
\pgfpathquadraticcurveto{\pgfqpoint{4.386111in}{4.569954in}}{\pgfqpoint{4.386111in}{4.542176in}}%
\pgfpathlineto{\pgfqpoint{4.386111in}{3.975232in}}%
\pgfpathquadraticcurveto{\pgfqpoint{4.386111in}{3.947454in}}{\pgfqpoint{4.413888in}{3.947454in}}%
\pgfpathlineto{\pgfqpoint{4.413888in}{3.947454in}}%
\pgfpathclose%
\pgfusepath{stroke,fill}%
\end{pgfscope}%
\begin{pgfscope}%
\pgfsetbuttcap%
\pgfsetroundjoin%
\pgfsetlinewidth{2.007500pt}%
\definecolor{currentstroke}{rgb}{0.172549,0.627451,0.172549}%
\pgfsetstrokecolor{currentstroke}%
\pgfsetdash{{7.400000pt}{3.200000pt}}{0.000000pt}%
\pgfpathmoveto{\pgfqpoint{4.441666in}{4.465787in}}%
\pgfpathlineto{\pgfqpoint{4.719444in}{4.465787in}}%
\pgfusepath{stroke}%
\end{pgfscope}%
\begin{pgfscope}%
\definecolor{textcolor}{rgb}{0.000000,0.000000,0.000000}%
\pgfsetstrokecolor{textcolor}%
\pgfsetfillcolor{textcolor}%
\pgftext[x=4.830555in,y=4.417176in,left,base]{\color{textcolor}\rmfamily\fontsize{10.000000}{12.000000}\selectfont Seuil = 5}%
\end{pgfscope}%
\begin{pgfscope}%
\pgfsetrectcap%
\pgfsetroundjoin%
\pgfsetlinewidth{2.007500pt}%
\definecolor{currentstroke}{rgb}{0.172549,0.627451,0.172549}%
\pgfsetstrokecolor{currentstroke}%
\pgfsetdash{}{0pt}%
\pgfpathmoveto{\pgfqpoint{4.441666in}{4.272176in}}%
\pgfpathlineto{\pgfqpoint{4.580555in}{4.272176in}}%
\pgfpathlineto{\pgfqpoint{4.719444in}{4.272176in}}%
\pgfusepath{stroke}%
\end{pgfscope}%
\begin{pgfscope}%
\definecolor{textcolor}{rgb}{0.000000,0.000000,0.000000}%
\pgfsetstrokecolor{textcolor}%
\pgfsetfillcolor{textcolor}%
\pgftext[x=4.830555in,y=4.223565in,left,base]{\color{textcolor}\rmfamily\fontsize{10.000000}{12.000000}\selectfont Dépassement}%
\end{pgfscope}%
\begin{pgfscope}%
\pgfsetrectcap%
\pgfsetroundjoin%
\pgfsetlinewidth{3.011250pt}%
\definecolor{currentstroke}{rgb}{0.839216,0.152941,0.156863}%
\pgfsetstrokecolor{currentstroke}%
\pgfsetdash{}{0pt}%
\pgfpathmoveto{\pgfqpoint{4.441666in}{4.078565in}}%
\pgfpathlineto{\pgfqpoint{4.580555in}{4.078565in}}%
\pgfpathlineto{\pgfqpoint{4.719444in}{4.078565in}}%
\pgfusepath{stroke}%
\end{pgfscope}%
\begin{pgfscope}%
\definecolor{textcolor}{rgb}{0.000000,0.000000,0.000000}%
\pgfsetstrokecolor{textcolor}%
\pgfsetfillcolor{textcolor}%
\pgftext[x=4.830555in,y=4.029954in,left,base]{\color{textcolor}\rmfamily\fontsize{10.000000}{12.000000}\selectfont Secousse}%
\end{pgfscope}%
\begin{pgfscope}%
\pgfsetbuttcap%
\pgfsetmiterjoin%
\definecolor{currentfill}{rgb}{0.933333,0.933333,0.933333}%
\pgfsetfillcolor{currentfill}%
\pgfsetlinewidth{0.000000pt}%
\definecolor{currentstroke}{rgb}{0.000000,0.000000,0.000000}%
\pgfsetstrokecolor{currentstroke}%
\pgfsetstrokeopacity{0.000000}%
\pgfsetdash{}{0pt}%
\pgfpathmoveto{\pgfqpoint{0.691161in}{2.672776in}}%
\pgfpathlineto{\pgfqpoint{5.745833in}{2.672776in}}%
\pgfpathlineto{\pgfqpoint{5.745833in}{3.575093in}}%
\pgfpathlineto{\pgfqpoint{0.691161in}{3.575093in}}%
\pgfpathlineto{\pgfqpoint{0.691161in}{2.672776in}}%
\pgfpathclose%
\pgfusepath{fill}%
\end{pgfscope}%
\begin{pgfscope}%
\pgfpathrectangle{\pgfqpoint{0.691161in}{2.672776in}}{\pgfqpoint{5.054672in}{0.902317in}}%
\pgfusepath{clip}%
\pgfsetbuttcap%
\pgfsetroundjoin%
\pgfsetlinewidth{0.501875pt}%
\definecolor{currentstroke}{rgb}{0.698039,0.698039,0.698039}%
\pgfsetstrokecolor{currentstroke}%
\pgfsetdash{{1.850000pt}{0.800000pt}}{0.000000pt}%
\pgfpathmoveto{\pgfqpoint{0.691161in}{2.672776in}}%
\pgfpathlineto{\pgfqpoint{0.691161in}{3.575093in}}%
\pgfusepath{stroke}%
\end{pgfscope}%
\begin{pgfscope}%
\pgfsetbuttcap%
\pgfsetroundjoin%
\definecolor{currentfill}{rgb}{0.000000,0.000000,0.000000}%
\pgfsetfillcolor{currentfill}%
\pgfsetlinewidth{0.803000pt}%
\definecolor{currentstroke}{rgb}{0.000000,0.000000,0.000000}%
\pgfsetstrokecolor{currentstroke}%
\pgfsetdash{}{0pt}%
\pgfsys@defobject{currentmarker}{\pgfqpoint{0.000000in}{0.000000in}}{\pgfqpoint{0.000000in}{0.048611in}}{%
\pgfpathmoveto{\pgfqpoint{0.000000in}{0.000000in}}%
\pgfpathlineto{\pgfqpoint{0.000000in}{0.048611in}}%
\pgfusepath{stroke,fill}%
}%
\begin{pgfscope}%
\pgfsys@transformshift{0.691161in}{2.672776in}%
\pgfsys@useobject{currentmarker}{}%
\end{pgfscope}%
\end{pgfscope}%
\begin{pgfscope}%
\pgfpathrectangle{\pgfqpoint{0.691161in}{2.672776in}}{\pgfqpoint{5.054672in}{0.902317in}}%
\pgfusepath{clip}%
\pgfsetbuttcap%
\pgfsetroundjoin%
\pgfsetlinewidth{0.501875pt}%
\definecolor{currentstroke}{rgb}{0.698039,0.698039,0.698039}%
\pgfsetstrokecolor{currentstroke}%
\pgfsetdash{{1.850000pt}{0.800000pt}}{0.000000pt}%
\pgfpathmoveto{\pgfqpoint{1.533607in}{2.672776in}}%
\pgfpathlineto{\pgfqpoint{1.533607in}{3.575093in}}%
\pgfusepath{stroke}%
\end{pgfscope}%
\begin{pgfscope}%
\pgfsetbuttcap%
\pgfsetroundjoin%
\definecolor{currentfill}{rgb}{0.000000,0.000000,0.000000}%
\pgfsetfillcolor{currentfill}%
\pgfsetlinewidth{0.803000pt}%
\definecolor{currentstroke}{rgb}{0.000000,0.000000,0.000000}%
\pgfsetstrokecolor{currentstroke}%
\pgfsetdash{}{0pt}%
\pgfsys@defobject{currentmarker}{\pgfqpoint{0.000000in}{0.000000in}}{\pgfqpoint{0.000000in}{0.048611in}}{%
\pgfpathmoveto{\pgfqpoint{0.000000in}{0.000000in}}%
\pgfpathlineto{\pgfqpoint{0.000000in}{0.048611in}}%
\pgfusepath{stroke,fill}%
}%
\begin{pgfscope}%
\pgfsys@transformshift{1.533607in}{2.672776in}%
\pgfsys@useobject{currentmarker}{}%
\end{pgfscope}%
\end{pgfscope}%
\begin{pgfscope}%
\pgfpathrectangle{\pgfqpoint{0.691161in}{2.672776in}}{\pgfqpoint{5.054672in}{0.902317in}}%
\pgfusepath{clip}%
\pgfsetbuttcap%
\pgfsetroundjoin%
\pgfsetlinewidth{0.501875pt}%
\definecolor{currentstroke}{rgb}{0.698039,0.698039,0.698039}%
\pgfsetstrokecolor{currentstroke}%
\pgfsetdash{{1.850000pt}{0.800000pt}}{0.000000pt}%
\pgfpathmoveto{\pgfqpoint{2.376052in}{2.672776in}}%
\pgfpathlineto{\pgfqpoint{2.376052in}{3.575093in}}%
\pgfusepath{stroke}%
\end{pgfscope}%
\begin{pgfscope}%
\pgfsetbuttcap%
\pgfsetroundjoin%
\definecolor{currentfill}{rgb}{0.000000,0.000000,0.000000}%
\pgfsetfillcolor{currentfill}%
\pgfsetlinewidth{0.803000pt}%
\definecolor{currentstroke}{rgb}{0.000000,0.000000,0.000000}%
\pgfsetstrokecolor{currentstroke}%
\pgfsetdash{}{0pt}%
\pgfsys@defobject{currentmarker}{\pgfqpoint{0.000000in}{0.000000in}}{\pgfqpoint{0.000000in}{0.048611in}}{%
\pgfpathmoveto{\pgfqpoint{0.000000in}{0.000000in}}%
\pgfpathlineto{\pgfqpoint{0.000000in}{0.048611in}}%
\pgfusepath{stroke,fill}%
}%
\begin{pgfscope}%
\pgfsys@transformshift{2.376052in}{2.672776in}%
\pgfsys@useobject{currentmarker}{}%
\end{pgfscope}%
\end{pgfscope}%
\begin{pgfscope}%
\pgfpathrectangle{\pgfqpoint{0.691161in}{2.672776in}}{\pgfqpoint{5.054672in}{0.902317in}}%
\pgfusepath{clip}%
\pgfsetbuttcap%
\pgfsetroundjoin%
\pgfsetlinewidth{0.501875pt}%
\definecolor{currentstroke}{rgb}{0.698039,0.698039,0.698039}%
\pgfsetstrokecolor{currentstroke}%
\pgfsetdash{{1.850000pt}{0.800000pt}}{0.000000pt}%
\pgfpathmoveto{\pgfqpoint{3.218497in}{2.672776in}}%
\pgfpathlineto{\pgfqpoint{3.218497in}{3.575093in}}%
\pgfusepath{stroke}%
\end{pgfscope}%
\begin{pgfscope}%
\pgfsetbuttcap%
\pgfsetroundjoin%
\definecolor{currentfill}{rgb}{0.000000,0.000000,0.000000}%
\pgfsetfillcolor{currentfill}%
\pgfsetlinewidth{0.803000pt}%
\definecolor{currentstroke}{rgb}{0.000000,0.000000,0.000000}%
\pgfsetstrokecolor{currentstroke}%
\pgfsetdash{}{0pt}%
\pgfsys@defobject{currentmarker}{\pgfqpoint{0.000000in}{0.000000in}}{\pgfqpoint{0.000000in}{0.048611in}}{%
\pgfpathmoveto{\pgfqpoint{0.000000in}{0.000000in}}%
\pgfpathlineto{\pgfqpoint{0.000000in}{0.048611in}}%
\pgfusepath{stroke,fill}%
}%
\begin{pgfscope}%
\pgfsys@transformshift{3.218497in}{2.672776in}%
\pgfsys@useobject{currentmarker}{}%
\end{pgfscope}%
\end{pgfscope}%
\begin{pgfscope}%
\pgfpathrectangle{\pgfqpoint{0.691161in}{2.672776in}}{\pgfqpoint{5.054672in}{0.902317in}}%
\pgfusepath{clip}%
\pgfsetbuttcap%
\pgfsetroundjoin%
\pgfsetlinewidth{0.501875pt}%
\definecolor{currentstroke}{rgb}{0.698039,0.698039,0.698039}%
\pgfsetstrokecolor{currentstroke}%
\pgfsetdash{{1.850000pt}{0.800000pt}}{0.000000pt}%
\pgfpathmoveto{\pgfqpoint{4.060942in}{2.672776in}}%
\pgfpathlineto{\pgfqpoint{4.060942in}{3.575093in}}%
\pgfusepath{stroke}%
\end{pgfscope}%
\begin{pgfscope}%
\pgfsetbuttcap%
\pgfsetroundjoin%
\definecolor{currentfill}{rgb}{0.000000,0.000000,0.000000}%
\pgfsetfillcolor{currentfill}%
\pgfsetlinewidth{0.803000pt}%
\definecolor{currentstroke}{rgb}{0.000000,0.000000,0.000000}%
\pgfsetstrokecolor{currentstroke}%
\pgfsetdash{}{0pt}%
\pgfsys@defobject{currentmarker}{\pgfqpoint{0.000000in}{0.000000in}}{\pgfqpoint{0.000000in}{0.048611in}}{%
\pgfpathmoveto{\pgfqpoint{0.000000in}{0.000000in}}%
\pgfpathlineto{\pgfqpoint{0.000000in}{0.048611in}}%
\pgfusepath{stroke,fill}%
}%
\begin{pgfscope}%
\pgfsys@transformshift{4.060942in}{2.672776in}%
\pgfsys@useobject{currentmarker}{}%
\end{pgfscope}%
\end{pgfscope}%
\begin{pgfscope}%
\pgfpathrectangle{\pgfqpoint{0.691161in}{2.672776in}}{\pgfqpoint{5.054672in}{0.902317in}}%
\pgfusepath{clip}%
\pgfsetbuttcap%
\pgfsetroundjoin%
\pgfsetlinewidth{0.501875pt}%
\definecolor{currentstroke}{rgb}{0.698039,0.698039,0.698039}%
\pgfsetstrokecolor{currentstroke}%
\pgfsetdash{{1.850000pt}{0.800000pt}}{0.000000pt}%
\pgfpathmoveto{\pgfqpoint{4.903388in}{2.672776in}}%
\pgfpathlineto{\pgfqpoint{4.903388in}{3.575093in}}%
\pgfusepath{stroke}%
\end{pgfscope}%
\begin{pgfscope}%
\pgfsetbuttcap%
\pgfsetroundjoin%
\definecolor{currentfill}{rgb}{0.000000,0.000000,0.000000}%
\pgfsetfillcolor{currentfill}%
\pgfsetlinewidth{0.803000pt}%
\definecolor{currentstroke}{rgb}{0.000000,0.000000,0.000000}%
\pgfsetstrokecolor{currentstroke}%
\pgfsetdash{}{0pt}%
\pgfsys@defobject{currentmarker}{\pgfqpoint{0.000000in}{0.000000in}}{\pgfqpoint{0.000000in}{0.048611in}}{%
\pgfpathmoveto{\pgfqpoint{0.000000in}{0.000000in}}%
\pgfpathlineto{\pgfqpoint{0.000000in}{0.048611in}}%
\pgfusepath{stroke,fill}%
}%
\begin{pgfscope}%
\pgfsys@transformshift{4.903388in}{2.672776in}%
\pgfsys@useobject{currentmarker}{}%
\end{pgfscope}%
\end{pgfscope}%
\begin{pgfscope}%
\pgfpathrectangle{\pgfqpoint{0.691161in}{2.672776in}}{\pgfqpoint{5.054672in}{0.902317in}}%
\pgfusepath{clip}%
\pgfsetbuttcap%
\pgfsetroundjoin%
\pgfsetlinewidth{0.501875pt}%
\definecolor{currentstroke}{rgb}{0.698039,0.698039,0.698039}%
\pgfsetstrokecolor{currentstroke}%
\pgfsetdash{{1.850000pt}{0.800000pt}}{0.000000pt}%
\pgfpathmoveto{\pgfqpoint{5.745833in}{2.672776in}}%
\pgfpathlineto{\pgfqpoint{5.745833in}{3.575093in}}%
\pgfusepath{stroke}%
\end{pgfscope}%
\begin{pgfscope}%
\pgfsetbuttcap%
\pgfsetroundjoin%
\definecolor{currentfill}{rgb}{0.000000,0.000000,0.000000}%
\pgfsetfillcolor{currentfill}%
\pgfsetlinewidth{0.803000pt}%
\definecolor{currentstroke}{rgb}{0.000000,0.000000,0.000000}%
\pgfsetstrokecolor{currentstroke}%
\pgfsetdash{}{0pt}%
\pgfsys@defobject{currentmarker}{\pgfqpoint{0.000000in}{0.000000in}}{\pgfqpoint{0.000000in}{0.048611in}}{%
\pgfpathmoveto{\pgfqpoint{0.000000in}{0.000000in}}%
\pgfpathlineto{\pgfqpoint{0.000000in}{0.048611in}}%
\pgfusepath{stroke,fill}%
}%
\begin{pgfscope}%
\pgfsys@transformshift{5.745833in}{2.672776in}%
\pgfsys@useobject{currentmarker}{}%
\end{pgfscope}%
\end{pgfscope}%
\begin{pgfscope}%
\pgfpathrectangle{\pgfqpoint{0.691161in}{2.672776in}}{\pgfqpoint{5.054672in}{0.902317in}}%
\pgfusepath{clip}%
\pgfsetbuttcap%
\pgfsetroundjoin%
\pgfsetlinewidth{0.501875pt}%
\definecolor{currentstroke}{rgb}{0.698039,0.698039,0.698039}%
\pgfsetstrokecolor{currentstroke}%
\pgfsetdash{{1.850000pt}{0.800000pt}}{0.000000pt}%
\pgfpathmoveto{\pgfqpoint{0.691161in}{2.830975in}}%
\pgfpathlineto{\pgfqpoint{5.745833in}{2.830975in}}%
\pgfusepath{stroke}%
\end{pgfscope}%
\begin{pgfscope}%
\pgfsetbuttcap%
\pgfsetroundjoin%
\definecolor{currentfill}{rgb}{0.000000,0.000000,0.000000}%
\pgfsetfillcolor{currentfill}%
\pgfsetlinewidth{0.803000pt}%
\definecolor{currentstroke}{rgb}{0.000000,0.000000,0.000000}%
\pgfsetstrokecolor{currentstroke}%
\pgfsetdash{}{0pt}%
\pgfsys@defobject{currentmarker}{\pgfqpoint{0.000000in}{0.000000in}}{\pgfqpoint{0.048611in}{0.000000in}}{%
\pgfpathmoveto{\pgfqpoint{0.000000in}{0.000000in}}%
\pgfpathlineto{\pgfqpoint{0.048611in}{0.000000in}}%
\pgfusepath{stroke,fill}%
}%
\begin{pgfscope}%
\pgfsys@transformshift{0.691161in}{2.830975in}%
\pgfsys@useobject{currentmarker}{}%
\end{pgfscope}%
\end{pgfscope}%
\begin{pgfscope}%
\definecolor{textcolor}{rgb}{0.000000,0.000000,0.000000}%
\pgfsetstrokecolor{textcolor}%
\pgfsetfillcolor{textcolor}%
\pgftext[x=0.573105in, y=2.782780in, left, base]{\color{textcolor}\rmfamily\fontsize{10.000000}{12.000000}\selectfont \(\displaystyle {0}\)}%
\end{pgfscope}%
\begin{pgfscope}%
\pgfpathrectangle{\pgfqpoint{0.691161in}{2.672776in}}{\pgfqpoint{5.054672in}{0.902317in}}%
\pgfusepath{clip}%
\pgfsetbuttcap%
\pgfsetroundjoin%
\pgfsetlinewidth{0.501875pt}%
\definecolor{currentstroke}{rgb}{0.698039,0.698039,0.698039}%
\pgfsetstrokecolor{currentstroke}%
\pgfsetdash{{1.850000pt}{0.800000pt}}{0.000000pt}%
\pgfpathmoveto{\pgfqpoint{0.691161in}{3.296744in}}%
\pgfpathlineto{\pgfqpoint{5.745833in}{3.296744in}}%
\pgfusepath{stroke}%
\end{pgfscope}%
\begin{pgfscope}%
\pgfsetbuttcap%
\pgfsetroundjoin%
\definecolor{currentfill}{rgb}{0.000000,0.000000,0.000000}%
\pgfsetfillcolor{currentfill}%
\pgfsetlinewidth{0.803000pt}%
\definecolor{currentstroke}{rgb}{0.000000,0.000000,0.000000}%
\pgfsetstrokecolor{currentstroke}%
\pgfsetdash{}{0pt}%
\pgfsys@defobject{currentmarker}{\pgfqpoint{0.000000in}{0.000000in}}{\pgfqpoint{0.048611in}{0.000000in}}{%
\pgfpathmoveto{\pgfqpoint{0.000000in}{0.000000in}}%
\pgfpathlineto{\pgfqpoint{0.048611in}{0.000000in}}%
\pgfusepath{stroke,fill}%
}%
\begin{pgfscope}%
\pgfsys@transformshift{0.691161in}{3.296744in}%
\pgfsys@useobject{currentmarker}{}%
\end{pgfscope}%
\end{pgfscope}%
\begin{pgfscope}%
\definecolor{textcolor}{rgb}{0.000000,0.000000,0.000000}%
\pgfsetstrokecolor{textcolor}%
\pgfsetfillcolor{textcolor}%
\pgftext[x=0.503661in, y=3.248550in, left, base]{\color{textcolor}\rmfamily\fontsize{10.000000}{12.000000}\selectfont \(\displaystyle {10}\)}%
\end{pgfscope}%
\begin{pgfscope}%
\definecolor{textcolor}{rgb}{0.000000,0.000000,0.000000}%
\pgfsetstrokecolor{textcolor}%
\pgfsetfillcolor{textcolor}%
\pgftext[x=0.448105in,y=3.123935in,,bottom,rotate=90.000000]{\color{textcolor}\rmfamily\fontsize{12.000000}{14.400000}\selectfont STA/LTA}%
\end{pgfscope}%
\begin{pgfscope}%
\pgfpathrectangle{\pgfqpoint{0.691161in}{2.672776in}}{\pgfqpoint{5.054672in}{0.902317in}}%
\pgfusepath{clip}%
\pgfsetrectcap%
\pgfsetroundjoin%
\pgfsetlinewidth{1.505625pt}%
\definecolor{currentstroke}{rgb}{0.121569,0.466667,0.705882}%
\pgfsetstrokecolor{currentstroke}%
\pgfsetdash{}{0pt}%
\pgfpathmoveto{\pgfqpoint{0.691161in}{2.830975in}}%
\pgfpathlineto{\pgfqpoint{1.028055in}{2.830975in}}%
\pgfpathlineto{\pgfqpoint{1.028139in}{2.907693in}}%
\pgfpathlineto{\pgfqpoint{1.029235in}{2.907031in}}%
\pgfpathlineto{\pgfqpoint{1.030414in}{2.901820in}}%
\pgfpathlineto{\pgfqpoint{1.032941in}{2.875466in}}%
\pgfpathlineto{\pgfqpoint{1.033026in}{2.875533in}}%
\pgfpathlineto{\pgfqpoint{1.034289in}{2.879002in}}%
\pgfpathlineto{\pgfqpoint{1.035216in}{2.877987in}}%
\pgfpathlineto{\pgfqpoint{1.036311in}{2.877069in}}%
\pgfpathlineto{\pgfqpoint{1.036732in}{2.877345in}}%
\pgfpathlineto{\pgfqpoint{1.037406in}{2.878441in}}%
\pgfpathlineto{\pgfqpoint{1.041282in}{2.895013in}}%
\pgfpathlineto{\pgfqpoint{1.041871in}{2.894348in}}%
\pgfpathlineto{\pgfqpoint{1.043303in}{2.893715in}}%
\pgfpathlineto{\pgfqpoint{1.045578in}{2.892459in}}%
\pgfpathlineto{\pgfqpoint{1.045999in}{2.893359in}}%
\pgfpathlineto{\pgfqpoint{1.048779in}{2.908434in}}%
\pgfpathlineto{\pgfqpoint{1.049537in}{2.906210in}}%
\pgfpathlineto{\pgfqpoint{1.052823in}{2.900106in}}%
\pgfpathlineto{\pgfqpoint{1.055266in}{2.897353in}}%
\pgfpathlineto{\pgfqpoint{1.055687in}{2.895320in}}%
\pgfpathlineto{\pgfqpoint{1.056530in}{2.896005in}}%
\pgfpathlineto{\pgfqpoint{1.059478in}{2.895812in}}%
\pgfpathlineto{\pgfqpoint{1.060321in}{2.895963in}}%
\pgfpathlineto{\pgfqpoint{1.060489in}{2.896265in}}%
\pgfpathlineto{\pgfqpoint{1.061500in}{2.902602in}}%
\pgfpathlineto{\pgfqpoint{1.061921in}{2.903803in}}%
\pgfpathlineto{\pgfqpoint{1.062511in}{2.902418in}}%
\pgfpathlineto{\pgfqpoint{1.063185in}{2.896184in}}%
\pgfpathlineto{\pgfqpoint{1.064870in}{2.883377in}}%
\pgfpathlineto{\pgfqpoint{1.065123in}{2.883896in}}%
\pgfpathlineto{\pgfqpoint{1.067650in}{2.889535in}}%
\pgfpathlineto{\pgfqpoint{1.069335in}{2.890787in}}%
\pgfpathlineto{\pgfqpoint{1.069925in}{2.892827in}}%
\pgfpathlineto{\pgfqpoint{1.070767in}{2.896264in}}%
\pgfpathlineto{\pgfqpoint{1.071273in}{2.894566in}}%
\pgfpathlineto{\pgfqpoint{1.072789in}{2.886384in}}%
\pgfpathlineto{\pgfqpoint{1.074558in}{2.877103in}}%
\pgfpathlineto{\pgfqpoint{1.074979in}{2.877200in}}%
\pgfpathlineto{\pgfqpoint{1.075822in}{2.878758in}}%
\pgfpathlineto{\pgfqpoint{1.076075in}{2.878974in}}%
\pgfpathlineto{\pgfqpoint{1.076664in}{2.878113in}}%
\pgfpathlineto{\pgfqpoint{1.077844in}{2.874472in}}%
\pgfpathlineto{\pgfqpoint{1.079781in}{2.867662in}}%
\pgfpathlineto{\pgfqpoint{1.081045in}{2.866375in}}%
\pgfpathlineto{\pgfqpoint{1.082814in}{2.853616in}}%
\pgfpathlineto{\pgfqpoint{1.083404in}{2.856812in}}%
\pgfpathlineto{\pgfqpoint{1.085426in}{2.861619in}}%
\pgfpathlineto{\pgfqpoint{1.086689in}{2.860549in}}%
\pgfpathlineto{\pgfqpoint{1.087616in}{2.857691in}}%
\pgfpathlineto{\pgfqpoint{1.088374in}{2.858127in}}%
\pgfpathlineto{\pgfqpoint{1.090733in}{2.861384in}}%
\pgfpathlineto{\pgfqpoint{1.093934in}{2.873629in}}%
\pgfpathlineto{\pgfqpoint{1.096377in}{2.876791in}}%
\pgfpathlineto{\pgfqpoint{1.098484in}{2.882647in}}%
\pgfpathlineto{\pgfqpoint{1.099326in}{2.882083in}}%
\pgfpathlineto{\pgfqpoint{1.100337in}{2.879119in}}%
\pgfpathlineto{\pgfqpoint{1.101264in}{2.881069in}}%
\pgfpathlineto{\pgfqpoint{1.103370in}{2.885106in}}%
\pgfpathlineto{\pgfqpoint{1.104970in}{2.886774in}}%
\pgfpathlineto{\pgfqpoint{1.106655in}{2.889641in}}%
\pgfpathlineto{\pgfqpoint{1.107076in}{2.889261in}}%
\pgfpathlineto{\pgfqpoint{1.107666in}{2.888252in}}%
\pgfpathlineto{\pgfqpoint{1.111457in}{2.877096in}}%
\pgfpathlineto{\pgfqpoint{1.111541in}{2.877118in}}%
\pgfpathlineto{\pgfqpoint{1.113311in}{2.877704in}}%
\pgfpathlineto{\pgfqpoint{1.113563in}{2.876671in}}%
\pgfpathlineto{\pgfqpoint{1.114911in}{2.873540in}}%
\pgfpathlineto{\pgfqpoint{1.115248in}{2.873727in}}%
\pgfpathlineto{\pgfqpoint{1.116765in}{2.874948in}}%
\pgfpathlineto{\pgfqpoint{1.117017in}{2.874187in}}%
\pgfpathlineto{\pgfqpoint{1.118702in}{2.867989in}}%
\pgfpathlineto{\pgfqpoint{1.119123in}{2.868817in}}%
\pgfpathlineto{\pgfqpoint{1.120134in}{2.871045in}}%
\pgfpathlineto{\pgfqpoint{1.120640in}{2.870601in}}%
\pgfpathlineto{\pgfqpoint{1.124515in}{2.862789in}}%
\pgfpathlineto{\pgfqpoint{1.126200in}{2.861328in}}%
\pgfpathlineto{\pgfqpoint{1.128727in}{2.861613in}}%
\pgfpathlineto{\pgfqpoint{1.130328in}{2.866855in}}%
\pgfpathlineto{\pgfqpoint{1.131676in}{2.870489in}}%
\pgfpathlineto{\pgfqpoint{1.131929in}{2.870201in}}%
\pgfpathlineto{\pgfqpoint{1.136815in}{2.861022in}}%
\pgfpathlineto{\pgfqpoint{1.137741in}{2.862173in}}%
\pgfpathlineto{\pgfqpoint{1.142291in}{2.861243in}}%
\pgfpathlineto{\pgfqpoint{1.143049in}{2.860942in}}%
\pgfpathlineto{\pgfqpoint{1.143470in}{2.861615in}}%
\pgfpathlineto{\pgfqpoint{1.146671in}{2.867234in}}%
\pgfpathlineto{\pgfqpoint{1.147093in}{2.866247in}}%
\pgfpathlineto{\pgfqpoint{1.147682in}{2.865244in}}%
\pgfpathlineto{\pgfqpoint{1.148356in}{2.865829in}}%
\pgfpathlineto{\pgfqpoint{1.151726in}{2.873178in}}%
\pgfpathlineto{\pgfqpoint{1.152063in}{2.872722in}}%
\pgfpathlineto{\pgfqpoint{1.152906in}{2.871372in}}%
\pgfpathlineto{\pgfqpoint{1.153495in}{2.872307in}}%
\pgfpathlineto{\pgfqpoint{1.155770in}{2.872641in}}%
\pgfpathlineto{\pgfqpoint{1.158550in}{2.877811in}}%
\pgfpathlineto{\pgfqpoint{1.159140in}{2.877310in}}%
\pgfpathlineto{\pgfqpoint{1.162762in}{2.872257in}}%
\pgfpathlineto{\pgfqpoint{1.167733in}{2.858417in}}%
\pgfpathlineto{\pgfqpoint{1.169417in}{2.858742in}}%
\pgfpathlineto{\pgfqpoint{1.169839in}{2.859012in}}%
\pgfpathlineto{\pgfqpoint{1.170260in}{2.858183in}}%
\pgfpathlineto{\pgfqpoint{1.174641in}{2.851283in}}%
\pgfpathlineto{\pgfqpoint{1.175820in}{2.852069in}}%
\pgfpathlineto{\pgfqpoint{1.177336in}{2.852503in}}%
\pgfpathlineto{\pgfqpoint{1.177505in}{2.852130in}}%
\pgfpathlineto{\pgfqpoint{1.179190in}{2.849190in}}%
\pgfpathlineto{\pgfqpoint{1.179527in}{2.849579in}}%
\pgfpathlineto{\pgfqpoint{1.180622in}{2.852632in}}%
\pgfpathlineto{\pgfqpoint{1.182391in}{2.856916in}}%
\pgfpathlineto{\pgfqpoint{1.182897in}{2.856532in}}%
\pgfpathlineto{\pgfqpoint{1.187109in}{2.852264in}}%
\pgfpathlineto{\pgfqpoint{1.187867in}{2.853603in}}%
\pgfpathlineto{\pgfqpoint{1.189889in}{2.860465in}}%
\pgfpathlineto{\pgfqpoint{1.190479in}{2.859813in}}%
\pgfpathlineto{\pgfqpoint{1.191574in}{2.858782in}}%
\pgfpathlineto{\pgfqpoint{1.191826in}{2.859266in}}%
\pgfpathlineto{\pgfqpoint{1.193848in}{2.863756in}}%
\pgfpathlineto{\pgfqpoint{1.194017in}{2.863698in}}%
\pgfpathlineto{\pgfqpoint{1.195702in}{2.863148in}}%
\pgfpathlineto{\pgfqpoint{1.195870in}{2.863683in}}%
\pgfpathlineto{\pgfqpoint{1.197555in}{2.869918in}}%
\pgfpathlineto{\pgfqpoint{1.197976in}{2.869380in}}%
\pgfpathlineto{\pgfqpoint{1.199661in}{2.867667in}}%
\pgfpathlineto{\pgfqpoint{1.199745in}{2.867705in}}%
\pgfpathlineto{\pgfqpoint{1.203452in}{2.870476in}}%
\pgfpathlineto{\pgfqpoint{1.204042in}{2.870601in}}%
\pgfpathlineto{\pgfqpoint{1.204379in}{2.869936in}}%
\pgfpathlineto{\pgfqpoint{1.205221in}{2.866028in}}%
\pgfpathlineto{\pgfqpoint{1.205980in}{2.863483in}}%
\pgfpathlineto{\pgfqpoint{1.206401in}{2.864950in}}%
\pgfpathlineto{\pgfqpoint{1.208928in}{2.892546in}}%
\pgfpathlineto{\pgfqpoint{1.210445in}{2.891918in}}%
\pgfpathlineto{\pgfqpoint{1.211203in}{2.891547in}}%
\pgfpathlineto{\pgfqpoint{1.211455in}{2.892076in}}%
\pgfpathlineto{\pgfqpoint{1.212719in}{2.897602in}}%
\pgfpathlineto{\pgfqpoint{1.213225in}{2.895090in}}%
\pgfpathlineto{\pgfqpoint{1.214909in}{2.888091in}}%
\pgfpathlineto{\pgfqpoint{1.215246in}{2.888472in}}%
\pgfpathlineto{\pgfqpoint{1.216173in}{2.890150in}}%
\pgfpathlineto{\pgfqpoint{1.218700in}{2.898779in}}%
\pgfpathlineto{\pgfqpoint{1.219796in}{2.897852in}}%
\pgfpathlineto{\pgfqpoint{1.221396in}{2.898464in}}%
\pgfpathlineto{\pgfqpoint{1.222828in}{2.899638in}}%
\pgfpathlineto{\pgfqpoint{1.223081in}{2.899090in}}%
\pgfpathlineto{\pgfqpoint{1.224261in}{2.886259in}}%
\pgfpathlineto{\pgfqpoint{1.226114in}{2.871814in}}%
\pgfpathlineto{\pgfqpoint{1.227715in}{2.871600in}}%
\pgfpathlineto{\pgfqpoint{1.227883in}{2.871916in}}%
\pgfpathlineto{\pgfqpoint{1.228136in}{2.872118in}}%
\pgfpathlineto{\pgfqpoint{1.228557in}{2.871088in}}%
\pgfpathlineto{\pgfqpoint{1.230832in}{2.865024in}}%
\pgfpathlineto{\pgfqpoint{1.231421in}{2.866493in}}%
\pgfpathlineto{\pgfqpoint{1.232601in}{2.868421in}}%
\pgfpathlineto{\pgfqpoint{1.232938in}{2.867607in}}%
\pgfpathlineto{\pgfqpoint{1.235044in}{2.857171in}}%
\pgfpathlineto{\pgfqpoint{1.235718in}{2.858116in}}%
\pgfpathlineto{\pgfqpoint{1.237571in}{2.859105in}}%
\pgfpathlineto{\pgfqpoint{1.237740in}{2.858960in}}%
\pgfpathlineto{\pgfqpoint{1.240520in}{2.856640in}}%
\pgfpathlineto{\pgfqpoint{1.240773in}{2.857425in}}%
\pgfpathlineto{\pgfqpoint{1.242205in}{2.865918in}}%
\pgfpathlineto{\pgfqpoint{1.242879in}{2.864700in}}%
\pgfpathlineto{\pgfqpoint{1.244479in}{2.863003in}}%
\pgfpathlineto{\pgfqpoint{1.245238in}{2.863941in}}%
\pgfpathlineto{\pgfqpoint{1.247765in}{2.869554in}}%
\pgfpathlineto{\pgfqpoint{1.248860in}{2.867479in}}%
\pgfpathlineto{\pgfqpoint{1.250376in}{2.865206in}}%
\pgfpathlineto{\pgfqpoint{1.250966in}{2.865617in}}%
\pgfpathlineto{\pgfqpoint{1.253241in}{2.866178in}}%
\pgfpathlineto{\pgfqpoint{1.256274in}{2.866436in}}%
\pgfpathlineto{\pgfqpoint{1.257537in}{2.867250in}}%
\pgfpathlineto{\pgfqpoint{1.257790in}{2.865823in}}%
\pgfpathlineto{\pgfqpoint{1.259475in}{2.858087in}}%
\pgfpathlineto{\pgfqpoint{1.259812in}{2.858197in}}%
\pgfpathlineto{\pgfqpoint{1.261834in}{2.857784in}}%
\pgfpathlineto{\pgfqpoint{1.262929in}{2.857358in}}%
\pgfpathlineto{\pgfqpoint{1.263097in}{2.857773in}}%
\pgfpathlineto{\pgfqpoint{1.263856in}{2.860514in}}%
\pgfpathlineto{\pgfqpoint{1.264614in}{2.859450in}}%
\pgfpathlineto{\pgfqpoint{1.265372in}{2.860463in}}%
\pgfpathlineto{\pgfqpoint{1.267984in}{2.860689in}}%
\pgfpathlineto{\pgfqpoint{1.269584in}{2.859711in}}%
\pgfpathlineto{\pgfqpoint{1.269837in}{2.860479in}}%
\pgfpathlineto{\pgfqpoint{1.272701in}{2.871465in}}%
\pgfpathlineto{\pgfqpoint{1.273207in}{2.871158in}}%
\pgfpathlineto{\pgfqpoint{1.276155in}{2.869549in}}%
\pgfpathlineto{\pgfqpoint{1.278261in}{2.868035in}}%
\pgfpathlineto{\pgfqpoint{1.278346in}{2.868102in}}%
\pgfpathlineto{\pgfqpoint{1.278767in}{2.868250in}}%
\pgfpathlineto{\pgfqpoint{1.279104in}{2.867381in}}%
\pgfpathlineto{\pgfqpoint{1.280620in}{2.862717in}}%
\pgfpathlineto{\pgfqpoint{1.281041in}{2.864497in}}%
\pgfpathlineto{\pgfqpoint{1.281884in}{2.867144in}}%
\pgfpathlineto{\pgfqpoint{1.282642in}{2.866915in}}%
\pgfpathlineto{\pgfqpoint{1.284748in}{2.867059in}}%
\pgfpathlineto{\pgfqpoint{1.284832in}{2.867392in}}%
\pgfpathlineto{\pgfqpoint{1.286770in}{2.877014in}}%
\pgfpathlineto{\pgfqpoint{1.287444in}{2.875815in}}%
\pgfpathlineto{\pgfqpoint{1.288202in}{2.873108in}}%
\pgfpathlineto{\pgfqpoint{1.288792in}{2.871219in}}%
\pgfpathlineto{\pgfqpoint{1.289550in}{2.871732in}}%
\pgfpathlineto{\pgfqpoint{1.291404in}{2.870929in}}%
\pgfpathlineto{\pgfqpoint{1.293004in}{2.869066in}}%
\pgfpathlineto{\pgfqpoint{1.293257in}{2.869244in}}%
\pgfpathlineto{\pgfqpoint{1.296205in}{2.873749in}}%
\pgfpathlineto{\pgfqpoint{1.296711in}{2.875540in}}%
\pgfpathlineto{\pgfqpoint{1.297216in}{2.873170in}}%
\pgfpathlineto{\pgfqpoint{1.298396in}{2.867865in}}%
\pgfpathlineto{\pgfqpoint{1.298901in}{2.870167in}}%
\pgfpathlineto{\pgfqpoint{1.301260in}{2.878806in}}%
\pgfpathlineto{\pgfqpoint{1.301681in}{2.879552in}}%
\pgfpathlineto{\pgfqpoint{1.302187in}{2.878176in}}%
\pgfpathlineto{\pgfqpoint{1.306231in}{2.867526in}}%
\pgfpathlineto{\pgfqpoint{1.306736in}{2.868879in}}%
\pgfpathlineto{\pgfqpoint{1.309011in}{2.873581in}}%
\pgfpathlineto{\pgfqpoint{1.312802in}{2.880839in}}%
\pgfpathlineto{\pgfqpoint{1.313223in}{2.879872in}}%
\pgfpathlineto{\pgfqpoint{1.313391in}{2.879820in}}%
\pgfpathlineto{\pgfqpoint{1.313644in}{2.880948in}}%
\pgfpathlineto{\pgfqpoint{1.315076in}{2.887925in}}%
\pgfpathlineto{\pgfqpoint{1.315413in}{2.886938in}}%
\pgfpathlineto{\pgfqpoint{1.318615in}{2.872494in}}%
\pgfpathlineto{\pgfqpoint{1.319541in}{2.870589in}}%
\pgfpathlineto{\pgfqpoint{1.319962in}{2.871284in}}%
\pgfpathlineto{\pgfqpoint{1.321310in}{2.871681in}}%
\pgfpathlineto{\pgfqpoint{1.323079in}{2.872202in}}%
\pgfpathlineto{\pgfqpoint{1.323248in}{2.871758in}}%
\pgfpathlineto{\pgfqpoint{1.326281in}{2.865303in}}%
\pgfpathlineto{\pgfqpoint{1.326870in}{2.863431in}}%
\pgfpathlineto{\pgfqpoint{1.328050in}{2.861046in}}%
\pgfpathlineto{\pgfqpoint{1.328471in}{2.861684in}}%
\pgfpathlineto{\pgfqpoint{1.329566in}{2.865119in}}%
\pgfpathlineto{\pgfqpoint{1.330240in}{2.862964in}}%
\pgfpathlineto{\pgfqpoint{1.331420in}{2.856722in}}%
\pgfpathlineto{\pgfqpoint{1.332178in}{2.858227in}}%
\pgfpathlineto{\pgfqpoint{1.336643in}{2.870559in}}%
\pgfpathlineto{\pgfqpoint{1.336980in}{2.870157in}}%
\pgfpathlineto{\pgfqpoint{1.337485in}{2.869733in}}%
\pgfpathlineto{\pgfqpoint{1.337906in}{2.870754in}}%
\pgfpathlineto{\pgfqpoint{1.339760in}{2.873620in}}%
\pgfpathlineto{\pgfqpoint{1.341024in}{2.874275in}}%
\pgfpathlineto{\pgfqpoint{1.342624in}{2.878484in}}%
\pgfpathlineto{\pgfqpoint{1.343635in}{2.877353in}}%
\pgfpathlineto{\pgfqpoint{1.344730in}{2.875120in}}%
\pgfpathlineto{\pgfqpoint{1.346584in}{2.868365in}}%
\pgfpathlineto{\pgfqpoint{1.348269in}{2.869446in}}%
\pgfpathlineto{\pgfqpoint{1.349111in}{2.867766in}}%
\pgfpathlineto{\pgfqpoint{1.352565in}{2.856154in}}%
\pgfpathlineto{\pgfqpoint{1.352734in}{2.856488in}}%
\pgfpathlineto{\pgfqpoint{1.356103in}{2.866593in}}%
\pgfpathlineto{\pgfqpoint{1.357367in}{2.868657in}}%
\pgfpathlineto{\pgfqpoint{1.357788in}{2.867931in}}%
\pgfpathlineto{\pgfqpoint{1.359557in}{2.863184in}}%
\pgfpathlineto{\pgfqpoint{1.360316in}{2.864182in}}%
\pgfpathlineto{\pgfqpoint{1.362674in}{2.865084in}}%
\pgfpathlineto{\pgfqpoint{1.363854in}{2.864421in}}%
\pgfpathlineto{\pgfqpoint{1.366550in}{2.862350in}}%
\pgfpathlineto{\pgfqpoint{1.366887in}{2.862733in}}%
\pgfpathlineto{\pgfqpoint{1.367898in}{2.865727in}}%
\pgfpathlineto{\pgfqpoint{1.369330in}{2.869623in}}%
\pgfpathlineto{\pgfqpoint{1.369498in}{2.869345in}}%
\pgfpathlineto{\pgfqpoint{1.371941in}{2.863277in}}%
\pgfpathlineto{\pgfqpoint{1.372026in}{2.863301in}}%
\pgfpathlineto{\pgfqpoint{1.373879in}{2.865776in}}%
\pgfpathlineto{\pgfqpoint{1.375817in}{2.870234in}}%
\pgfpathlineto{\pgfqpoint{1.377586in}{2.870260in}}%
\pgfpathlineto{\pgfqpoint{1.379608in}{2.870950in}}%
\pgfpathlineto{\pgfqpoint{1.380113in}{2.873353in}}%
\pgfpathlineto{\pgfqpoint{1.382472in}{2.882920in}}%
\pgfpathlineto{\pgfqpoint{1.384409in}{2.885552in}}%
\pgfpathlineto{\pgfqpoint{1.384662in}{2.884852in}}%
\pgfpathlineto{\pgfqpoint{1.385505in}{2.882242in}}%
\pgfpathlineto{\pgfqpoint{1.386263in}{2.882824in}}%
\pgfpathlineto{\pgfqpoint{1.386768in}{2.883616in}}%
\pgfpathlineto{\pgfqpoint{1.387105in}{2.882519in}}%
\pgfpathlineto{\pgfqpoint{1.391486in}{2.867311in}}%
\pgfpathlineto{\pgfqpoint{1.391823in}{2.868256in}}%
\pgfpathlineto{\pgfqpoint{1.393676in}{2.874660in}}%
\pgfpathlineto{\pgfqpoint{1.394013in}{2.874561in}}%
\pgfpathlineto{\pgfqpoint{1.395530in}{2.875486in}}%
\pgfpathlineto{\pgfqpoint{1.396709in}{2.879511in}}%
\pgfpathlineto{\pgfqpoint{1.397215in}{2.876959in}}%
\pgfpathlineto{\pgfqpoint{1.398900in}{2.867742in}}%
\pgfpathlineto{\pgfqpoint{1.399658in}{2.868638in}}%
\pgfpathlineto{\pgfqpoint{1.400669in}{2.869659in}}%
\pgfpathlineto{\pgfqpoint{1.401174in}{2.869025in}}%
\pgfpathlineto{\pgfqpoint{1.403196in}{2.868398in}}%
\pgfpathlineto{\pgfqpoint{1.404544in}{2.867331in}}%
\pgfpathlineto{\pgfqpoint{1.405218in}{2.866929in}}%
\pgfpathlineto{\pgfqpoint{1.405555in}{2.867711in}}%
\pgfpathlineto{\pgfqpoint{1.407240in}{2.871788in}}%
\pgfpathlineto{\pgfqpoint{1.407745in}{2.871044in}}%
\pgfpathlineto{\pgfqpoint{1.409767in}{2.866175in}}%
\pgfpathlineto{\pgfqpoint{1.410441in}{2.867885in}}%
\pgfpathlineto{\pgfqpoint{1.411536in}{2.869340in}}%
\pgfpathlineto{\pgfqpoint{1.411873in}{2.868810in}}%
\pgfpathlineto{\pgfqpoint{1.413558in}{2.867064in}}%
\pgfpathlineto{\pgfqpoint{1.414232in}{2.867858in}}%
\pgfpathlineto{\pgfqpoint{1.415833in}{2.879816in}}%
\pgfpathlineto{\pgfqpoint{1.417602in}{2.886364in}}%
\pgfpathlineto{\pgfqpoint{1.421477in}{2.882153in}}%
\pgfpathlineto{\pgfqpoint{1.421983in}{2.883346in}}%
\pgfpathlineto{\pgfqpoint{1.422235in}{2.883668in}}%
\pgfpathlineto{\pgfqpoint{1.423078in}{2.883099in}}%
\pgfpathlineto{\pgfqpoint{1.425015in}{2.887159in}}%
\pgfpathlineto{\pgfqpoint{1.425942in}{2.886325in}}%
\pgfpathlineto{\pgfqpoint{1.427037in}{2.881808in}}%
\pgfpathlineto{\pgfqpoint{1.427795in}{2.879849in}}%
\pgfpathlineto{\pgfqpoint{1.428301in}{2.880833in}}%
\pgfpathlineto{\pgfqpoint{1.429228in}{2.886249in}}%
\pgfpathlineto{\pgfqpoint{1.430660in}{2.895338in}}%
\pgfpathlineto{\pgfqpoint{1.430912in}{2.895053in}}%
\pgfpathlineto{\pgfqpoint{1.432429in}{2.886465in}}%
\pgfpathlineto{\pgfqpoint{1.433524in}{2.879788in}}%
\pgfpathlineto{\pgfqpoint{1.433945in}{2.880762in}}%
\pgfpathlineto{\pgfqpoint{1.439084in}{2.897669in}}%
\pgfpathlineto{\pgfqpoint{1.439758in}{2.898184in}}%
\pgfpathlineto{\pgfqpoint{1.440011in}{2.897451in}}%
\pgfpathlineto{\pgfqpoint{1.442370in}{2.890738in}}%
\pgfpathlineto{\pgfqpoint{1.443128in}{2.889994in}}%
\pgfpathlineto{\pgfqpoint{1.443633in}{2.890657in}}%
\pgfpathlineto{\pgfqpoint{1.444813in}{2.891545in}}%
\pgfpathlineto{\pgfqpoint{1.445150in}{2.891076in}}%
\pgfpathlineto{\pgfqpoint{1.445992in}{2.887859in}}%
\pgfpathlineto{\pgfqpoint{1.448098in}{2.873607in}}%
\pgfpathlineto{\pgfqpoint{1.450878in}{2.865002in}}%
\pgfpathlineto{\pgfqpoint{1.452479in}{2.859696in}}%
\pgfpathlineto{\pgfqpoint{1.453406in}{2.856801in}}%
\pgfpathlineto{\pgfqpoint{1.454922in}{2.850368in}}%
\pgfpathlineto{\pgfqpoint{1.455259in}{2.850889in}}%
\pgfpathlineto{\pgfqpoint{1.459556in}{2.860749in}}%
\pgfpathlineto{\pgfqpoint{1.460819in}{2.859638in}}%
\pgfpathlineto{\pgfqpoint{1.464442in}{2.858049in}}%
\pgfpathlineto{\pgfqpoint{1.467812in}{2.860250in}}%
\pgfpathlineto{\pgfqpoint{1.469581in}{2.862163in}}%
\pgfpathlineto{\pgfqpoint{1.471687in}{2.864236in}}%
\pgfpathlineto{\pgfqpoint{1.472024in}{2.863282in}}%
\pgfpathlineto{\pgfqpoint{1.473961in}{2.858256in}}%
\pgfpathlineto{\pgfqpoint{1.474298in}{2.858956in}}%
\pgfpathlineto{\pgfqpoint{1.478258in}{2.868483in}}%
\pgfpathlineto{\pgfqpoint{1.479185in}{2.870786in}}%
\pgfpathlineto{\pgfqpoint{1.479943in}{2.873672in}}%
\pgfpathlineto{\pgfqpoint{1.480617in}{2.872627in}}%
\pgfpathlineto{\pgfqpoint{1.482302in}{2.871918in}}%
\pgfpathlineto{\pgfqpoint{1.483060in}{2.873908in}}%
\pgfpathlineto{\pgfqpoint{1.483734in}{2.874930in}}%
\pgfpathlineto{\pgfqpoint{1.484239in}{2.874392in}}%
\pgfpathlineto{\pgfqpoint{1.485503in}{2.873599in}}%
\pgfpathlineto{\pgfqpoint{1.485671in}{2.873727in}}%
\pgfpathlineto{\pgfqpoint{1.486682in}{2.876599in}}%
\pgfpathlineto{\pgfqpoint{1.490052in}{2.889348in}}%
\pgfpathlineto{\pgfqpoint{1.490305in}{2.889116in}}%
\pgfpathlineto{\pgfqpoint{1.490558in}{2.889355in}}%
\pgfpathlineto{\pgfqpoint{1.490726in}{2.890013in}}%
\pgfpathlineto{\pgfqpoint{1.491063in}{2.890685in}}%
\pgfpathlineto{\pgfqpoint{1.491569in}{2.889276in}}%
\pgfpathlineto{\pgfqpoint{1.492748in}{2.885180in}}%
\pgfpathlineto{\pgfqpoint{1.493169in}{2.887133in}}%
\pgfpathlineto{\pgfqpoint{1.497550in}{2.907657in}}%
\pgfpathlineto{\pgfqpoint{1.502942in}{2.912519in}}%
\pgfpathlineto{\pgfqpoint{1.503447in}{2.913531in}}%
\pgfpathlineto{\pgfqpoint{1.504542in}{2.918820in}}%
\pgfpathlineto{\pgfqpoint{1.504963in}{2.921577in}}%
\pgfpathlineto{\pgfqpoint{1.505722in}{2.919811in}}%
\pgfpathlineto{\pgfqpoint{1.506901in}{2.915647in}}%
\pgfpathlineto{\pgfqpoint{1.507575in}{2.916137in}}%
\pgfpathlineto{\pgfqpoint{1.509260in}{2.919806in}}%
\pgfpathlineto{\pgfqpoint{1.509765in}{2.917215in}}%
\pgfpathlineto{\pgfqpoint{1.513135in}{2.896816in}}%
\pgfpathlineto{\pgfqpoint{1.513472in}{2.897480in}}%
\pgfpathlineto{\pgfqpoint{1.516084in}{2.906280in}}%
\pgfpathlineto{\pgfqpoint{1.516252in}{2.906165in}}%
\pgfpathlineto{\pgfqpoint{1.517516in}{2.902943in}}%
\pgfpathlineto{\pgfqpoint{1.518021in}{2.905051in}}%
\pgfpathlineto{\pgfqpoint{1.519790in}{2.910196in}}%
\pgfpathlineto{\pgfqpoint{1.519959in}{2.910053in}}%
\pgfpathlineto{\pgfqpoint{1.520970in}{2.904027in}}%
\pgfpathlineto{\pgfqpoint{1.522907in}{2.891790in}}%
\pgfpathlineto{\pgfqpoint{1.523329in}{2.892212in}}%
\pgfpathlineto{\pgfqpoint{1.525182in}{2.890801in}}%
\pgfpathlineto{\pgfqpoint{1.526193in}{2.886604in}}%
\pgfpathlineto{\pgfqpoint{1.527372in}{2.887661in}}%
\pgfpathlineto{\pgfqpoint{1.528720in}{2.890543in}}%
\pgfpathlineto{\pgfqpoint{1.529310in}{2.890066in}}%
\pgfpathlineto{\pgfqpoint{1.530489in}{2.884821in}}%
\pgfpathlineto{\pgfqpoint{1.532848in}{2.873202in}}%
\pgfpathlineto{\pgfqpoint{1.533354in}{2.873452in}}%
\pgfpathlineto{\pgfqpoint{1.533522in}{2.873879in}}%
\pgfpathlineto{\pgfqpoint{1.534365in}{2.877390in}}%
\pgfpathlineto{\pgfqpoint{1.534786in}{2.875069in}}%
\pgfpathlineto{\pgfqpoint{1.537145in}{2.865390in}}%
\pgfpathlineto{\pgfqpoint{1.538156in}{2.865494in}}%
\pgfpathlineto{\pgfqpoint{1.538240in}{2.865613in}}%
\pgfpathlineto{\pgfqpoint{1.541104in}{2.869469in}}%
\pgfpathlineto{\pgfqpoint{1.541526in}{2.868599in}}%
\pgfpathlineto{\pgfqpoint{1.543042in}{2.867052in}}%
\pgfpathlineto{\pgfqpoint{1.544306in}{2.865444in}}%
\pgfpathlineto{\pgfqpoint{1.545064in}{2.863518in}}%
\pgfpathlineto{\pgfqpoint{1.545401in}{2.865527in}}%
\pgfpathlineto{\pgfqpoint{1.546749in}{2.892960in}}%
\pgfpathlineto{\pgfqpoint{1.548771in}{2.926691in}}%
\pgfpathlineto{\pgfqpoint{1.549445in}{2.928318in}}%
\pgfpathlineto{\pgfqpoint{1.550118in}{2.927779in}}%
\pgfpathlineto{\pgfqpoint{1.551045in}{2.924497in}}%
\pgfpathlineto{\pgfqpoint{1.551719in}{2.923371in}}%
\pgfpathlineto{\pgfqpoint{1.552140in}{2.924707in}}%
\pgfpathlineto{\pgfqpoint{1.555173in}{2.940792in}}%
\pgfpathlineto{\pgfqpoint{1.555847in}{2.939841in}}%
\pgfpathlineto{\pgfqpoint{1.557869in}{2.934981in}}%
\pgfpathlineto{\pgfqpoint{1.558206in}{2.935915in}}%
\pgfpathlineto{\pgfqpoint{1.559975in}{2.939398in}}%
\pgfpathlineto{\pgfqpoint{1.561997in}{2.942788in}}%
\pgfpathlineto{\pgfqpoint{1.562334in}{2.941113in}}%
\pgfpathlineto{\pgfqpoint{1.563513in}{2.920834in}}%
\pgfpathlineto{\pgfqpoint{1.565282in}{2.896163in}}%
\pgfpathlineto{\pgfqpoint{1.566883in}{2.894582in}}%
\pgfpathlineto{\pgfqpoint{1.567304in}{2.895930in}}%
\pgfpathlineto{\pgfqpoint{1.568568in}{2.905240in}}%
\pgfpathlineto{\pgfqpoint{1.569242in}{2.903047in}}%
\pgfpathlineto{\pgfqpoint{1.570927in}{2.897408in}}%
\pgfpathlineto{\pgfqpoint{1.571348in}{2.900757in}}%
\pgfpathlineto{\pgfqpoint{1.573201in}{2.913381in}}%
\pgfpathlineto{\pgfqpoint{1.573707in}{2.914052in}}%
\pgfpathlineto{\pgfqpoint{1.574297in}{2.913139in}}%
\pgfpathlineto{\pgfqpoint{1.574718in}{2.913607in}}%
\pgfpathlineto{\pgfqpoint{1.575055in}{2.912621in}}%
\pgfpathlineto{\pgfqpoint{1.575560in}{2.911165in}}%
\pgfpathlineto{\pgfqpoint{1.576234in}{2.912328in}}%
\pgfpathlineto{\pgfqpoint{1.577245in}{2.912924in}}%
\pgfpathlineto{\pgfqpoint{1.578003in}{2.919185in}}%
\pgfpathlineto{\pgfqpoint{1.578509in}{2.921649in}}%
\pgfpathlineto{\pgfqpoint{1.579183in}{2.919826in}}%
\pgfpathlineto{\pgfqpoint{1.579857in}{2.917641in}}%
\pgfpathlineto{\pgfqpoint{1.580194in}{2.919843in}}%
\pgfpathlineto{\pgfqpoint{1.581626in}{2.923069in}}%
\pgfpathlineto{\pgfqpoint{1.582637in}{2.924644in}}%
\pgfpathlineto{\pgfqpoint{1.583901in}{2.936309in}}%
\pgfpathlineto{\pgfqpoint{1.584406in}{2.933033in}}%
\pgfpathlineto{\pgfqpoint{1.584996in}{2.927866in}}%
\pgfpathlineto{\pgfqpoint{1.585585in}{2.932235in}}%
\pgfpathlineto{\pgfqpoint{1.586259in}{2.936178in}}%
\pgfpathlineto{\pgfqpoint{1.587018in}{2.935754in}}%
\pgfpathlineto{\pgfqpoint{1.587607in}{2.937391in}}%
\pgfpathlineto{\pgfqpoint{1.587944in}{2.935647in}}%
\pgfpathlineto{\pgfqpoint{1.588702in}{2.922930in}}%
\pgfpathlineto{\pgfqpoint{1.589124in}{2.931209in}}%
\pgfpathlineto{\pgfqpoint{1.591314in}{2.975101in}}%
\pgfpathlineto{\pgfqpoint{1.591735in}{2.977813in}}%
\pgfpathlineto{\pgfqpoint{1.592493in}{2.976393in}}%
\pgfpathlineto{\pgfqpoint{1.596116in}{2.968568in}}%
\pgfpathlineto{\pgfqpoint{1.597211in}{2.966183in}}%
\pgfpathlineto{\pgfqpoint{1.597717in}{2.963821in}}%
\pgfpathlineto{\pgfqpoint{1.598138in}{2.966758in}}%
\pgfpathlineto{\pgfqpoint{1.599065in}{2.974855in}}%
\pgfpathlineto{\pgfqpoint{1.599654in}{2.973268in}}%
\pgfpathlineto{\pgfqpoint{1.600244in}{2.971161in}}%
\pgfpathlineto{\pgfqpoint{1.600749in}{2.973212in}}%
\pgfpathlineto{\pgfqpoint{1.603614in}{3.004453in}}%
\pgfpathlineto{\pgfqpoint{1.605551in}{3.026141in}}%
\pgfpathlineto{\pgfqpoint{1.605636in}{3.026090in}}%
\pgfpathlineto{\pgfqpoint{1.605973in}{3.024314in}}%
\pgfpathlineto{\pgfqpoint{1.606225in}{3.027672in}}%
\pgfpathlineto{\pgfqpoint{1.607321in}{3.062209in}}%
\pgfpathlineto{\pgfqpoint{1.608163in}{3.054963in}}%
\pgfpathlineto{\pgfqpoint{1.608331in}{3.054637in}}%
\pgfpathlineto{\pgfqpoint{1.608668in}{3.056119in}}%
\pgfpathlineto{\pgfqpoint{1.613386in}{3.092767in}}%
\pgfpathlineto{\pgfqpoint{1.613807in}{3.091842in}}%
\pgfpathlineto{\pgfqpoint{1.614903in}{3.084917in}}%
\pgfpathlineto{\pgfqpoint{1.615661in}{3.075060in}}%
\pgfpathlineto{\pgfqpoint{1.616419in}{3.075994in}}%
\pgfpathlineto{\pgfqpoint{1.618104in}{3.070372in}}%
\pgfpathlineto{\pgfqpoint{1.620041in}{3.050837in}}%
\pgfpathlineto{\pgfqpoint{1.625265in}{2.950785in}}%
\pgfpathlineto{\pgfqpoint{1.625686in}{2.954408in}}%
\pgfpathlineto{\pgfqpoint{1.626360in}{2.959124in}}%
\pgfpathlineto{\pgfqpoint{1.626949in}{2.957584in}}%
\pgfpathlineto{\pgfqpoint{1.628213in}{2.944631in}}%
\pgfpathlineto{\pgfqpoint{1.628803in}{2.951893in}}%
\pgfpathlineto{\pgfqpoint{1.632088in}{3.024826in}}%
\pgfpathlineto{\pgfqpoint{1.632257in}{3.024536in}}%
\pgfpathlineto{\pgfqpoint{1.632931in}{3.021453in}}%
\pgfpathlineto{\pgfqpoint{1.633436in}{3.023913in}}%
\pgfpathlineto{\pgfqpoint{1.635879in}{3.035837in}}%
\pgfpathlineto{\pgfqpoint{1.636722in}{3.043340in}}%
\pgfpathlineto{\pgfqpoint{1.640344in}{3.090582in}}%
\pgfpathlineto{\pgfqpoint{1.640934in}{3.087075in}}%
\pgfpathlineto{\pgfqpoint{1.641524in}{3.082857in}}%
\pgfpathlineto{\pgfqpoint{1.641945in}{3.086824in}}%
\pgfpathlineto{\pgfqpoint{1.643461in}{3.127504in}}%
\pgfpathlineto{\pgfqpoint{1.645483in}{3.189273in}}%
\pgfpathlineto{\pgfqpoint{1.645820in}{3.184981in}}%
\pgfpathlineto{\pgfqpoint{1.646831in}{3.154604in}}%
\pgfpathlineto{\pgfqpoint{1.647337in}{3.168835in}}%
\pgfpathlineto{\pgfqpoint{1.652391in}{3.382781in}}%
\pgfpathlineto{\pgfqpoint{1.653655in}{3.405927in}}%
\pgfpathlineto{\pgfqpoint{1.654076in}{3.404439in}}%
\pgfpathlineto{\pgfqpoint{1.655340in}{3.400581in}}%
\pgfpathlineto{\pgfqpoint{1.655677in}{3.401607in}}%
\pgfpathlineto{\pgfqpoint{1.657193in}{3.416895in}}%
\pgfpathlineto{\pgfqpoint{1.658710in}{3.414812in}}%
\pgfpathlineto{\pgfqpoint{1.659721in}{3.403765in}}%
\pgfpathlineto{\pgfqpoint{1.664775in}{3.295630in}}%
\pgfpathlineto{\pgfqpoint{1.667387in}{3.168579in}}%
\pgfpathlineto{\pgfqpoint{1.668145in}{3.147157in}}%
\pgfpathlineto{\pgfqpoint{1.670167in}{3.031768in}}%
\pgfpathlineto{\pgfqpoint{1.671094in}{3.052721in}}%
\pgfpathlineto{\pgfqpoint{1.671515in}{3.055657in}}%
\pgfpathlineto{\pgfqpoint{1.672273in}{3.054726in}}%
\pgfpathlineto{\pgfqpoint{1.672947in}{3.041293in}}%
\pgfpathlineto{\pgfqpoint{1.673705in}{3.024529in}}%
\pgfpathlineto{\pgfqpoint{1.674211in}{3.034467in}}%
\pgfpathlineto{\pgfqpoint{1.675811in}{3.144220in}}%
\pgfpathlineto{\pgfqpoint{1.677496in}{3.199598in}}%
\pgfpathlineto{\pgfqpoint{1.677580in}{3.199521in}}%
\pgfpathlineto{\pgfqpoint{1.678086in}{3.198619in}}%
\pgfpathlineto{\pgfqpoint{1.678339in}{3.200121in}}%
\pgfpathlineto{\pgfqpoint{1.679265in}{3.229142in}}%
\pgfpathlineto{\pgfqpoint{1.681624in}{3.302540in}}%
\pgfpathlineto{\pgfqpoint{1.682045in}{3.301722in}}%
\pgfpathlineto{\pgfqpoint{1.682298in}{3.301364in}}%
\pgfpathlineto{\pgfqpoint{1.682635in}{3.302921in}}%
\pgfpathlineto{\pgfqpoint{1.683562in}{3.325685in}}%
\pgfpathlineto{\pgfqpoint{1.685836in}{3.366962in}}%
\pgfpathlineto{\pgfqpoint{1.686847in}{3.366371in}}%
\pgfpathlineto{\pgfqpoint{1.687774in}{3.359420in}}%
\pgfpathlineto{\pgfqpoint{1.688448in}{3.364118in}}%
\pgfpathlineto{\pgfqpoint{1.688701in}{3.365195in}}%
\pgfpathlineto{\pgfqpoint{1.689290in}{3.362747in}}%
\pgfpathlineto{\pgfqpoint{1.691228in}{3.346298in}}%
\pgfpathlineto{\pgfqpoint{1.692492in}{3.283465in}}%
\pgfpathlineto{\pgfqpoint{1.694514in}{3.213281in}}%
\pgfpathlineto{\pgfqpoint{1.694935in}{3.212261in}}%
\pgfpathlineto{\pgfqpoint{1.695525in}{3.198694in}}%
\pgfpathlineto{\pgfqpoint{1.701759in}{2.958908in}}%
\pgfpathlineto{\pgfqpoint{1.702011in}{2.960058in}}%
\pgfpathlineto{\pgfqpoint{1.703444in}{2.978114in}}%
\pgfpathlineto{\pgfqpoint{1.704033in}{2.972381in}}%
\pgfpathlineto{\pgfqpoint{1.704876in}{2.959202in}}%
\pgfpathlineto{\pgfqpoint{1.705297in}{2.967682in}}%
\pgfpathlineto{\pgfqpoint{1.710520in}{3.153712in}}%
\pgfpathlineto{\pgfqpoint{1.712205in}{3.174410in}}%
\pgfpathlineto{\pgfqpoint{1.713132in}{3.175268in}}%
\pgfpathlineto{\pgfqpoint{1.714648in}{3.176579in}}%
\pgfpathlineto{\pgfqpoint{1.717091in}{3.178060in}}%
\pgfpathlineto{\pgfqpoint{1.717849in}{3.184709in}}%
\pgfpathlineto{\pgfqpoint{1.720040in}{3.199963in}}%
\pgfpathlineto{\pgfqpoint{1.720124in}{3.199924in}}%
\pgfpathlineto{\pgfqpoint{1.720882in}{3.199462in}}%
\pgfpathlineto{\pgfqpoint{1.721219in}{3.200141in}}%
\pgfpathlineto{\pgfqpoint{1.721556in}{3.200621in}}%
\pgfpathlineto{\pgfqpoint{1.721893in}{3.199035in}}%
\pgfpathlineto{\pgfqpoint{1.722736in}{3.178091in}}%
\pgfpathlineto{\pgfqpoint{1.725347in}{3.121498in}}%
\pgfpathlineto{\pgfqpoint{1.725937in}{3.114813in}}%
\pgfpathlineto{\pgfqpoint{1.730149in}{3.042740in}}%
\pgfpathlineto{\pgfqpoint{1.730486in}{3.043502in}}%
\pgfpathlineto{\pgfqpoint{1.734193in}{3.066543in}}%
\pgfpathlineto{\pgfqpoint{1.736215in}{3.083678in}}%
\pgfpathlineto{\pgfqpoint{1.737563in}{3.083192in}}%
\pgfpathlineto{\pgfqpoint{1.738068in}{3.080527in}}%
\pgfpathlineto{\pgfqpoint{1.739079in}{3.056195in}}%
\pgfpathlineto{\pgfqpoint{1.741017in}{3.022253in}}%
\pgfpathlineto{\pgfqpoint{1.741185in}{3.022523in}}%
\pgfpathlineto{\pgfqpoint{1.742112in}{3.026068in}}%
\pgfpathlineto{\pgfqpoint{1.742701in}{3.024107in}}%
\pgfpathlineto{\pgfqpoint{1.745229in}{3.009295in}}%
\pgfpathlineto{\pgfqpoint{1.745987in}{3.010341in}}%
\pgfpathlineto{\pgfqpoint{1.746745in}{3.011295in}}%
\pgfpathlineto{\pgfqpoint{1.747082in}{3.010421in}}%
\pgfpathlineto{\pgfqpoint{1.750873in}{2.987378in}}%
\pgfpathlineto{\pgfqpoint{1.755759in}{2.880591in}}%
\pgfpathlineto{\pgfqpoint{1.757107in}{2.866177in}}%
\pgfpathlineto{\pgfqpoint{1.757697in}{2.862818in}}%
\pgfpathlineto{\pgfqpoint{1.758202in}{2.865073in}}%
\pgfpathlineto{\pgfqpoint{1.761151in}{2.879398in}}%
\pgfpathlineto{\pgfqpoint{1.761825in}{2.882294in}}%
\pgfpathlineto{\pgfqpoint{1.765953in}{2.909189in}}%
\pgfpathlineto{\pgfqpoint{1.766206in}{2.909025in}}%
\pgfpathlineto{\pgfqpoint{1.768228in}{2.907882in}}%
\pgfpathlineto{\pgfqpoint{1.770586in}{2.907345in}}%
\pgfpathlineto{\pgfqpoint{1.770755in}{2.907714in}}%
\pgfpathlineto{\pgfqpoint{1.773535in}{2.912336in}}%
\pgfpathlineto{\pgfqpoint{1.773703in}{2.912242in}}%
\pgfpathlineto{\pgfqpoint{1.774546in}{2.909829in}}%
\pgfpathlineto{\pgfqpoint{1.781117in}{2.858763in}}%
\pgfpathlineto{\pgfqpoint{1.782549in}{2.855536in}}%
\pgfpathlineto{\pgfqpoint{1.783644in}{2.853416in}}%
\pgfpathlineto{\pgfqpoint{1.785413in}{2.852206in}}%
\pgfpathlineto{\pgfqpoint{1.787520in}{2.851439in}}%
\pgfpathlineto{\pgfqpoint{1.791563in}{2.844531in}}%
\pgfpathlineto{\pgfqpoint{1.793248in}{2.839667in}}%
\pgfpathlineto{\pgfqpoint{1.794849in}{2.837669in}}%
\pgfpathlineto{\pgfqpoint{1.795017in}{2.837760in}}%
\pgfpathlineto{\pgfqpoint{1.797123in}{2.840994in}}%
\pgfpathlineto{\pgfqpoint{1.800999in}{2.855285in}}%
\pgfpathlineto{\pgfqpoint{1.803021in}{2.867025in}}%
\pgfpathlineto{\pgfqpoint{1.804200in}{2.868508in}}%
\pgfpathlineto{\pgfqpoint{1.805295in}{2.874379in}}%
\pgfpathlineto{\pgfqpoint{1.807401in}{2.881905in}}%
\pgfpathlineto{\pgfqpoint{1.809423in}{2.884344in}}%
\pgfpathlineto{\pgfqpoint{1.811698in}{2.890027in}}%
\pgfpathlineto{\pgfqpoint{1.811950in}{2.889875in}}%
\pgfpathlineto{\pgfqpoint{1.814225in}{2.885817in}}%
\pgfpathlineto{\pgfqpoint{1.817595in}{2.876839in}}%
\pgfpathlineto{\pgfqpoint{1.820038in}{2.862264in}}%
\pgfpathlineto{\pgfqpoint{1.821217in}{2.860954in}}%
\pgfpathlineto{\pgfqpoint{1.822481in}{2.852955in}}%
\pgfpathlineto{\pgfqpoint{1.824334in}{2.848495in}}%
\pgfpathlineto{\pgfqpoint{1.826609in}{2.846614in}}%
\pgfpathlineto{\pgfqpoint{1.828715in}{2.843269in}}%
\pgfpathlineto{\pgfqpoint{1.830400in}{2.842197in}}%
\pgfpathlineto{\pgfqpoint{1.831411in}{2.841855in}}%
\pgfpathlineto{\pgfqpoint{1.831748in}{2.842182in}}%
\pgfpathlineto{\pgfqpoint{1.833349in}{2.842075in}}%
\pgfpathlineto{\pgfqpoint{1.839077in}{2.844729in}}%
\pgfpathlineto{\pgfqpoint{1.851293in}{2.878127in}}%
\pgfpathlineto{\pgfqpoint{1.852051in}{2.880657in}}%
\pgfpathlineto{\pgfqpoint{1.853904in}{2.884916in}}%
\pgfpathlineto{\pgfqpoint{1.856263in}{2.883233in}}%
\pgfpathlineto{\pgfqpoint{1.859801in}{2.875466in}}%
\pgfpathlineto{\pgfqpoint{1.860981in}{2.872338in}}%
\pgfpathlineto{\pgfqpoint{1.862750in}{2.863780in}}%
\pgfpathlineto{\pgfqpoint{1.863340in}{2.864834in}}%
\pgfpathlineto{\pgfqpoint{1.864351in}{2.868711in}}%
\pgfpathlineto{\pgfqpoint{1.864940in}{2.867133in}}%
\pgfpathlineto{\pgfqpoint{1.866541in}{2.862602in}}%
\pgfpathlineto{\pgfqpoint{1.867046in}{2.863403in}}%
\pgfpathlineto{\pgfqpoint{1.868394in}{2.868261in}}%
\pgfpathlineto{\pgfqpoint{1.868984in}{2.866074in}}%
\pgfpathlineto{\pgfqpoint{1.870837in}{2.861021in}}%
\pgfpathlineto{\pgfqpoint{1.872944in}{2.860329in}}%
\pgfpathlineto{\pgfqpoint{1.875387in}{2.857943in}}%
\pgfpathlineto{\pgfqpoint{1.876650in}{2.856806in}}%
\pgfpathlineto{\pgfqpoint{1.877324in}{2.855878in}}%
\pgfpathlineto{\pgfqpoint{1.877830in}{2.856810in}}%
\pgfpathlineto{\pgfqpoint{1.879515in}{2.858857in}}%
\pgfpathlineto{\pgfqpoint{1.879683in}{2.858770in}}%
\pgfpathlineto{\pgfqpoint{1.880441in}{2.856718in}}%
\pgfpathlineto{\pgfqpoint{1.881284in}{2.854392in}}%
\pgfpathlineto{\pgfqpoint{1.881705in}{2.855503in}}%
\pgfpathlineto{\pgfqpoint{1.883474in}{2.860887in}}%
\pgfpathlineto{\pgfqpoint{1.883895in}{2.860390in}}%
\pgfpathlineto{\pgfqpoint{1.884906in}{2.858987in}}%
\pgfpathlineto{\pgfqpoint{1.885327in}{2.859718in}}%
\pgfpathlineto{\pgfqpoint{1.888697in}{2.868014in}}%
\pgfpathlineto{\pgfqpoint{1.890129in}{2.868717in}}%
\pgfpathlineto{\pgfqpoint{1.893836in}{2.869857in}}%
\pgfpathlineto{\pgfqpoint{1.895100in}{2.870637in}}%
\pgfpathlineto{\pgfqpoint{1.897290in}{2.880159in}}%
\pgfpathlineto{\pgfqpoint{1.898048in}{2.877939in}}%
\pgfpathlineto{\pgfqpoint{1.898807in}{2.876222in}}%
\pgfpathlineto{\pgfqpoint{1.899312in}{2.877469in}}%
\pgfpathlineto{\pgfqpoint{1.900828in}{2.879519in}}%
\pgfpathlineto{\pgfqpoint{1.900997in}{2.879393in}}%
\pgfpathlineto{\pgfqpoint{1.902008in}{2.875823in}}%
\pgfpathlineto{\pgfqpoint{1.902598in}{2.874862in}}%
\pgfpathlineto{\pgfqpoint{1.903187in}{2.875479in}}%
\pgfpathlineto{\pgfqpoint{1.904030in}{2.874791in}}%
\pgfpathlineto{\pgfqpoint{1.904451in}{2.874583in}}%
\pgfpathlineto{\pgfqpoint{1.904788in}{2.875224in}}%
\pgfpathlineto{\pgfqpoint{1.905715in}{2.882677in}}%
\pgfpathlineto{\pgfqpoint{1.909000in}{2.912163in}}%
\pgfpathlineto{\pgfqpoint{1.910180in}{2.922150in}}%
\pgfpathlineto{\pgfqpoint{1.911359in}{2.931443in}}%
\pgfpathlineto{\pgfqpoint{1.911949in}{2.929332in}}%
\pgfpathlineto{\pgfqpoint{1.913971in}{2.919572in}}%
\pgfpathlineto{\pgfqpoint{1.914560in}{2.919959in}}%
\pgfpathlineto{\pgfqpoint{1.915150in}{2.919799in}}%
\pgfpathlineto{\pgfqpoint{1.915318in}{2.919379in}}%
\pgfpathlineto{\pgfqpoint{1.921300in}{2.901783in}}%
\pgfpathlineto{\pgfqpoint{1.921974in}{2.899913in}}%
\pgfpathlineto{\pgfqpoint{1.923069in}{2.888073in}}%
\pgfpathlineto{\pgfqpoint{1.925428in}{2.866780in}}%
\pgfpathlineto{\pgfqpoint{1.926102in}{2.863648in}}%
\pgfpathlineto{\pgfqpoint{1.929724in}{2.841574in}}%
\pgfpathlineto{\pgfqpoint{1.931156in}{2.839452in}}%
\pgfpathlineto{\pgfqpoint{1.932757in}{2.838724in}}%
\pgfpathlineto{\pgfqpoint{1.937812in}{2.839100in}}%
\pgfpathlineto{\pgfqpoint{1.938738in}{2.839260in}}%
\pgfpathlineto{\pgfqpoint{1.938991in}{2.838873in}}%
\pgfpathlineto{\pgfqpoint{1.941856in}{2.836084in}}%
\pgfpathlineto{\pgfqpoint{1.943035in}{2.837064in}}%
\pgfpathlineto{\pgfqpoint{1.944804in}{2.845476in}}%
\pgfpathlineto{\pgfqpoint{1.946910in}{2.851292in}}%
\pgfpathlineto{\pgfqpoint{1.948005in}{2.852535in}}%
\pgfpathlineto{\pgfqpoint{1.952049in}{2.863094in}}%
\pgfpathlineto{\pgfqpoint{1.957104in}{2.863170in}}%
\pgfpathlineto{\pgfqpoint{1.960474in}{2.863523in}}%
\pgfpathlineto{\pgfqpoint{1.962074in}{2.858164in}}%
\pgfpathlineto{\pgfqpoint{1.963591in}{2.855377in}}%
\pgfpathlineto{\pgfqpoint{1.963759in}{2.855462in}}%
\pgfpathlineto{\pgfqpoint{1.965023in}{2.857013in}}%
\pgfpathlineto{\pgfqpoint{1.965528in}{2.855823in}}%
\pgfpathlineto{\pgfqpoint{1.967971in}{2.850080in}}%
\pgfpathlineto{\pgfqpoint{1.968308in}{2.850441in}}%
\pgfpathlineto{\pgfqpoint{1.970583in}{2.853882in}}%
\pgfpathlineto{\pgfqpoint{1.971088in}{2.853289in}}%
\pgfpathlineto{\pgfqpoint{1.974290in}{2.850585in}}%
\pgfpathlineto{\pgfqpoint{1.977407in}{2.848654in}}%
\pgfpathlineto{\pgfqpoint{1.979007in}{2.846233in}}%
\pgfpathlineto{\pgfqpoint{1.979260in}{2.846409in}}%
\pgfpathlineto{\pgfqpoint{1.980776in}{2.847838in}}%
\pgfpathlineto{\pgfqpoint{1.981198in}{2.846971in}}%
\pgfpathlineto{\pgfqpoint{1.982209in}{2.844869in}}%
\pgfpathlineto{\pgfqpoint{1.982714in}{2.845765in}}%
\pgfpathlineto{\pgfqpoint{1.983894in}{2.853586in}}%
\pgfpathlineto{\pgfqpoint{1.985326in}{2.860682in}}%
\pgfpathlineto{\pgfqpoint{1.985663in}{2.860446in}}%
\pgfpathlineto{\pgfqpoint{1.986758in}{2.859508in}}%
\pgfpathlineto{\pgfqpoint{1.987095in}{2.860478in}}%
\pgfpathlineto{\pgfqpoint{1.988359in}{2.873740in}}%
\pgfpathlineto{\pgfqpoint{1.990802in}{2.889062in}}%
\pgfpathlineto{\pgfqpoint{1.991560in}{2.890641in}}%
\pgfpathlineto{\pgfqpoint{1.993160in}{2.904232in}}%
\pgfpathlineto{\pgfqpoint{1.995014in}{2.911134in}}%
\pgfpathlineto{\pgfqpoint{1.995772in}{2.916694in}}%
\pgfpathlineto{\pgfqpoint{1.998636in}{2.940386in}}%
\pgfpathlineto{\pgfqpoint{1.999226in}{2.941956in}}%
\pgfpathlineto{\pgfqpoint{2.001332in}{2.953526in}}%
\pgfpathlineto{\pgfqpoint{2.002090in}{2.952730in}}%
\pgfpathlineto{\pgfqpoint{2.002427in}{2.952759in}}%
\pgfpathlineto{\pgfqpoint{2.002680in}{2.953335in}}%
\pgfpathlineto{\pgfqpoint{2.003859in}{2.961629in}}%
\pgfpathlineto{\pgfqpoint{2.004449in}{2.964028in}}%
\pgfpathlineto{\pgfqpoint{2.004955in}{2.962161in}}%
\pgfpathlineto{\pgfqpoint{2.006387in}{2.952784in}}%
\pgfpathlineto{\pgfqpoint{2.006977in}{2.955231in}}%
\pgfpathlineto{\pgfqpoint{2.008661in}{2.972251in}}%
\pgfpathlineto{\pgfqpoint{2.009420in}{2.968097in}}%
\pgfpathlineto{\pgfqpoint{2.010936in}{2.958601in}}%
\pgfpathlineto{\pgfqpoint{2.011442in}{2.959716in}}%
\pgfpathlineto{\pgfqpoint{2.012031in}{2.961049in}}%
\pgfpathlineto{\pgfqpoint{2.012452in}{2.959439in}}%
\pgfpathlineto{\pgfqpoint{2.019782in}{2.908775in}}%
\pgfpathlineto{\pgfqpoint{2.021045in}{2.897045in}}%
\pgfpathlineto{\pgfqpoint{2.022899in}{2.887559in}}%
\pgfpathlineto{\pgfqpoint{2.023994in}{2.877650in}}%
\pgfpathlineto{\pgfqpoint{2.026774in}{2.853413in}}%
\pgfpathlineto{\pgfqpoint{2.028459in}{2.851383in}}%
\pgfpathlineto{\pgfqpoint{2.029301in}{2.851018in}}%
\pgfpathlineto{\pgfqpoint{2.029638in}{2.851423in}}%
\pgfpathlineto{\pgfqpoint{2.031744in}{2.853581in}}%
\pgfpathlineto{\pgfqpoint{2.031997in}{2.853357in}}%
\pgfpathlineto{\pgfqpoint{2.033514in}{2.849578in}}%
\pgfpathlineto{\pgfqpoint{2.034356in}{2.847421in}}%
\pgfpathlineto{\pgfqpoint{2.034861in}{2.848530in}}%
\pgfpathlineto{\pgfqpoint{2.036378in}{2.859351in}}%
\pgfpathlineto{\pgfqpoint{2.037389in}{2.863876in}}%
\pgfpathlineto{\pgfqpoint{2.037894in}{2.863542in}}%
\pgfpathlineto{\pgfqpoint{2.038400in}{2.863838in}}%
\pgfpathlineto{\pgfqpoint{2.038568in}{2.864335in}}%
\pgfpathlineto{\pgfqpoint{2.039748in}{2.873080in}}%
\pgfpathlineto{\pgfqpoint{2.042443in}{2.890305in}}%
\pgfpathlineto{\pgfqpoint{2.043791in}{2.894705in}}%
\pgfpathlineto{\pgfqpoint{2.045224in}{2.898595in}}%
\pgfpathlineto{\pgfqpoint{2.045645in}{2.898112in}}%
\pgfpathlineto{\pgfqpoint{2.047919in}{2.896424in}}%
\pgfpathlineto{\pgfqpoint{2.049520in}{2.897606in}}%
\pgfpathlineto{\pgfqpoint{2.050868in}{2.900695in}}%
\pgfpathlineto{\pgfqpoint{2.051373in}{2.899657in}}%
\pgfpathlineto{\pgfqpoint{2.053058in}{2.888903in}}%
\pgfpathlineto{\pgfqpoint{2.053985in}{2.884477in}}%
\pgfpathlineto{\pgfqpoint{2.054490in}{2.885652in}}%
\pgfpathlineto{\pgfqpoint{2.055333in}{2.888106in}}%
\pgfpathlineto{\pgfqpoint{2.055838in}{2.886804in}}%
\pgfpathlineto{\pgfqpoint{2.057102in}{2.874212in}}%
\pgfpathlineto{\pgfqpoint{2.058450in}{2.864129in}}%
\pgfpathlineto{\pgfqpoint{2.058955in}{2.864504in}}%
\pgfpathlineto{\pgfqpoint{2.059966in}{2.865198in}}%
\pgfpathlineto{\pgfqpoint{2.060303in}{2.864591in}}%
\pgfpathlineto{\pgfqpoint{2.062494in}{2.857102in}}%
\pgfpathlineto{\pgfqpoint{2.063673in}{2.857958in}}%
\pgfpathlineto{\pgfqpoint{2.065611in}{2.859279in}}%
\pgfpathlineto{\pgfqpoint{2.067296in}{2.861076in}}%
\pgfpathlineto{\pgfqpoint{2.067717in}{2.860749in}}%
\pgfpathlineto{\pgfqpoint{2.068981in}{2.860802in}}%
\pgfpathlineto{\pgfqpoint{2.069149in}{2.861089in}}%
\pgfpathlineto{\pgfqpoint{2.070328in}{2.862613in}}%
\pgfpathlineto{\pgfqpoint{2.070834in}{2.862264in}}%
\pgfpathlineto{\pgfqpoint{2.072856in}{2.859723in}}%
\pgfpathlineto{\pgfqpoint{2.073445in}{2.861214in}}%
\pgfpathlineto{\pgfqpoint{2.076563in}{2.868447in}}%
\pgfpathlineto{\pgfqpoint{2.077658in}{2.869841in}}%
\pgfpathlineto{\pgfqpoint{2.079511in}{2.873942in}}%
\pgfpathlineto{\pgfqpoint{2.080269in}{2.873383in}}%
\pgfpathlineto{\pgfqpoint{2.082965in}{2.871557in}}%
\pgfpathlineto{\pgfqpoint{2.084060in}{2.868500in}}%
\pgfpathlineto{\pgfqpoint{2.084566in}{2.869249in}}%
\pgfpathlineto{\pgfqpoint{2.085998in}{2.872189in}}%
\pgfpathlineto{\pgfqpoint{2.086503in}{2.871342in}}%
\pgfpathlineto{\pgfqpoint{2.088609in}{2.864309in}}%
\pgfpathlineto{\pgfqpoint{2.089452in}{2.866128in}}%
\pgfpathlineto{\pgfqpoint{2.090379in}{2.867941in}}%
\pgfpathlineto{\pgfqpoint{2.090800in}{2.867030in}}%
\pgfpathlineto{\pgfqpoint{2.093159in}{2.857966in}}%
\pgfpathlineto{\pgfqpoint{2.094001in}{2.858665in}}%
\pgfpathlineto{\pgfqpoint{2.095265in}{2.858366in}}%
\pgfpathlineto{\pgfqpoint{2.097118in}{2.857723in}}%
\pgfpathlineto{\pgfqpoint{2.099309in}{2.859089in}}%
\pgfpathlineto{\pgfqpoint{2.100319in}{2.860419in}}%
\pgfpathlineto{\pgfqpoint{2.100741in}{2.859759in}}%
\pgfpathlineto{\pgfqpoint{2.104279in}{2.852771in}}%
\pgfpathlineto{\pgfqpoint{2.104616in}{2.853009in}}%
\pgfpathlineto{\pgfqpoint{2.105880in}{2.854053in}}%
\pgfpathlineto{\pgfqpoint{2.106301in}{2.853256in}}%
\pgfpathlineto{\pgfqpoint{2.107733in}{2.848900in}}%
\pgfpathlineto{\pgfqpoint{2.108323in}{2.850260in}}%
\pgfpathlineto{\pgfqpoint{2.110429in}{2.856669in}}%
\pgfpathlineto{\pgfqpoint{2.110850in}{2.856247in}}%
\pgfpathlineto{\pgfqpoint{2.112872in}{2.854331in}}%
\pgfpathlineto{\pgfqpoint{2.113040in}{2.854446in}}%
\pgfpathlineto{\pgfqpoint{2.115399in}{2.855164in}}%
\pgfpathlineto{\pgfqpoint{2.118685in}{2.853363in}}%
\pgfpathlineto{\pgfqpoint{2.120538in}{2.854875in}}%
\pgfpathlineto{\pgfqpoint{2.121718in}{2.855916in}}%
\pgfpathlineto{\pgfqpoint{2.122139in}{2.855349in}}%
\pgfpathlineto{\pgfqpoint{2.123739in}{2.853463in}}%
\pgfpathlineto{\pgfqpoint{2.124076in}{2.854005in}}%
\pgfpathlineto{\pgfqpoint{2.128457in}{2.870615in}}%
\pgfpathlineto{\pgfqpoint{2.133006in}{2.930554in}}%
\pgfpathlineto{\pgfqpoint{2.135281in}{2.959760in}}%
\pgfpathlineto{\pgfqpoint{2.136039in}{2.964268in}}%
\pgfpathlineto{\pgfqpoint{2.139746in}{2.991026in}}%
\pgfpathlineto{\pgfqpoint{2.141010in}{2.999131in}}%
\pgfpathlineto{\pgfqpoint{2.142273in}{3.005921in}}%
\pgfpathlineto{\pgfqpoint{2.142779in}{3.005072in}}%
\pgfpathlineto{\pgfqpoint{2.143874in}{3.001677in}}%
\pgfpathlineto{\pgfqpoint{2.144464in}{3.002872in}}%
\pgfpathlineto{\pgfqpoint{2.145727in}{3.006369in}}%
\pgfpathlineto{\pgfqpoint{2.146149in}{3.004932in}}%
\pgfpathlineto{\pgfqpoint{2.147328in}{2.998960in}}%
\pgfpathlineto{\pgfqpoint{2.148002in}{2.999209in}}%
\pgfpathlineto{\pgfqpoint{2.148507in}{2.998005in}}%
\pgfpathlineto{\pgfqpoint{2.149434in}{2.985627in}}%
\pgfpathlineto{\pgfqpoint{2.152046in}{2.957128in}}%
\pgfpathlineto{\pgfqpoint{2.152888in}{2.952331in}}%
\pgfpathlineto{\pgfqpoint{2.156005in}{2.933484in}}%
\pgfpathlineto{\pgfqpoint{2.156679in}{2.932511in}}%
\pgfpathlineto{\pgfqpoint{2.158195in}{2.922387in}}%
\pgfpathlineto{\pgfqpoint{2.160554in}{2.909395in}}%
\pgfpathlineto{\pgfqpoint{2.161397in}{2.906753in}}%
\pgfpathlineto{\pgfqpoint{2.162576in}{2.893889in}}%
\pgfpathlineto{\pgfqpoint{2.164851in}{2.872854in}}%
\pgfpathlineto{\pgfqpoint{2.164935in}{2.872871in}}%
\pgfpathlineto{\pgfqpoint{2.166030in}{2.872955in}}%
\pgfpathlineto{\pgfqpoint{2.166283in}{2.872390in}}%
\pgfpathlineto{\pgfqpoint{2.169568in}{2.866509in}}%
\pgfpathlineto{\pgfqpoint{2.171927in}{2.865641in}}%
\pgfpathlineto{\pgfqpoint{2.172180in}{2.866346in}}%
\pgfpathlineto{\pgfqpoint{2.173612in}{2.870458in}}%
\pgfpathlineto{\pgfqpoint{2.174118in}{2.869862in}}%
\pgfpathlineto{\pgfqpoint{2.175466in}{2.866774in}}%
\pgfpathlineto{\pgfqpoint{2.176055in}{2.867937in}}%
\pgfpathlineto{\pgfqpoint{2.178414in}{2.877396in}}%
\pgfpathlineto{\pgfqpoint{2.179678in}{2.876262in}}%
\pgfpathlineto{\pgfqpoint{2.180436in}{2.878159in}}%
\pgfpathlineto{\pgfqpoint{2.182374in}{2.886544in}}%
\pgfpathlineto{\pgfqpoint{2.183132in}{2.885082in}}%
\pgfpathlineto{\pgfqpoint{2.184733in}{2.880345in}}%
\pgfpathlineto{\pgfqpoint{2.185238in}{2.881350in}}%
\pgfpathlineto{\pgfqpoint{2.186839in}{2.887248in}}%
\pgfpathlineto{\pgfqpoint{2.187428in}{2.885977in}}%
\pgfpathlineto{\pgfqpoint{2.189787in}{2.877511in}}%
\pgfpathlineto{\pgfqpoint{2.190461in}{2.878534in}}%
\pgfpathlineto{\pgfqpoint{2.191641in}{2.880584in}}%
\pgfpathlineto{\pgfqpoint{2.192146in}{2.879820in}}%
\pgfpathlineto{\pgfqpoint{2.193999in}{2.871447in}}%
\pgfpathlineto{\pgfqpoint{2.195516in}{2.869092in}}%
\pgfpathlineto{\pgfqpoint{2.196948in}{2.867662in}}%
\pgfpathlineto{\pgfqpoint{2.200570in}{2.856208in}}%
\pgfpathlineto{\pgfqpoint{2.201497in}{2.857045in}}%
\pgfpathlineto{\pgfqpoint{2.202087in}{2.857027in}}%
\pgfpathlineto{\pgfqpoint{2.202424in}{2.856480in}}%
\pgfpathlineto{\pgfqpoint{2.204277in}{2.853781in}}%
\pgfpathlineto{\pgfqpoint{2.204698in}{2.854507in}}%
\pgfpathlineto{\pgfqpoint{2.206805in}{2.863098in}}%
\pgfpathlineto{\pgfqpoint{2.207900in}{2.860930in}}%
\pgfpathlineto{\pgfqpoint{2.208237in}{2.860705in}}%
\pgfpathlineto{\pgfqpoint{2.208658in}{2.861467in}}%
\pgfpathlineto{\pgfqpoint{2.210933in}{2.868038in}}%
\pgfpathlineto{\pgfqpoint{2.211607in}{2.867282in}}%
\pgfpathlineto{\pgfqpoint{2.212533in}{2.867076in}}%
\pgfpathlineto{\pgfqpoint{2.212870in}{2.867409in}}%
\pgfpathlineto{\pgfqpoint{2.214892in}{2.868116in}}%
\pgfpathlineto{\pgfqpoint{2.219441in}{2.865575in}}%
\pgfpathlineto{\pgfqpoint{2.221042in}{2.863780in}}%
\pgfpathlineto{\pgfqpoint{2.221295in}{2.863937in}}%
\pgfpathlineto{\pgfqpoint{2.222053in}{2.864173in}}%
\pgfpathlineto{\pgfqpoint{2.222390in}{2.863451in}}%
\pgfpathlineto{\pgfqpoint{2.225001in}{2.858465in}}%
\pgfpathlineto{\pgfqpoint{2.225760in}{2.857335in}}%
\pgfpathlineto{\pgfqpoint{2.228877in}{2.851493in}}%
\pgfpathlineto{\pgfqpoint{2.231320in}{2.852170in}}%
\pgfpathlineto{\pgfqpoint{2.233847in}{2.857124in}}%
\pgfpathlineto{\pgfqpoint{2.237807in}{2.859008in}}%
\pgfpathlineto{\pgfqpoint{2.237891in}{2.858887in}}%
\pgfpathlineto{\pgfqpoint{2.239070in}{2.854590in}}%
\pgfpathlineto{\pgfqpoint{2.241345in}{2.849202in}}%
\pgfpathlineto{\pgfqpoint{2.242524in}{2.849776in}}%
\pgfpathlineto{\pgfqpoint{2.243451in}{2.855329in}}%
\pgfpathlineto{\pgfqpoint{2.245641in}{2.865854in}}%
\pgfpathlineto{\pgfqpoint{2.246231in}{2.868238in}}%
\pgfpathlineto{\pgfqpoint{2.247663in}{2.888645in}}%
\pgfpathlineto{\pgfqpoint{2.248927in}{2.898865in}}%
\pgfpathlineto{\pgfqpoint{2.249348in}{2.898523in}}%
\pgfpathlineto{\pgfqpoint{2.250106in}{2.898459in}}%
\pgfpathlineto{\pgfqpoint{2.250443in}{2.898970in}}%
\pgfpathlineto{\pgfqpoint{2.251707in}{2.901414in}}%
\pgfpathlineto{\pgfqpoint{2.252297in}{2.900235in}}%
\pgfpathlineto{\pgfqpoint{2.253392in}{2.898478in}}%
\pgfpathlineto{\pgfqpoint{2.253813in}{2.898797in}}%
\pgfpathlineto{\pgfqpoint{2.256256in}{2.902566in}}%
\pgfpathlineto{\pgfqpoint{2.258783in}{2.920296in}}%
\pgfpathlineto{\pgfqpoint{2.259626in}{2.919568in}}%
\pgfpathlineto{\pgfqpoint{2.260384in}{2.920500in}}%
\pgfpathlineto{\pgfqpoint{2.261564in}{2.927893in}}%
\pgfpathlineto{\pgfqpoint{2.262574in}{2.932367in}}%
\pgfpathlineto{\pgfqpoint{2.262996in}{2.931060in}}%
\pgfpathlineto{\pgfqpoint{2.267545in}{2.910051in}}%
\pgfpathlineto{\pgfqpoint{2.268809in}{2.909069in}}%
\pgfpathlineto{\pgfqpoint{2.269146in}{2.909315in}}%
\pgfpathlineto{\pgfqpoint{2.270746in}{2.909292in}}%
\pgfpathlineto{\pgfqpoint{2.273274in}{2.905742in}}%
\pgfpathlineto{\pgfqpoint{2.279845in}{2.857602in}}%
\pgfpathlineto{\pgfqpoint{2.280013in}{2.857645in}}%
\pgfpathlineto{\pgfqpoint{2.281361in}{2.857655in}}%
\pgfpathlineto{\pgfqpoint{2.281529in}{2.857142in}}%
\pgfpathlineto{\pgfqpoint{2.283383in}{2.854602in}}%
\pgfpathlineto{\pgfqpoint{2.285236in}{2.855063in}}%
\pgfpathlineto{\pgfqpoint{2.288859in}{2.855035in}}%
\pgfpathlineto{\pgfqpoint{2.292397in}{2.854149in}}%
\pgfpathlineto{\pgfqpoint{2.297620in}{2.853425in}}%
\pgfpathlineto{\pgfqpoint{2.301327in}{2.842022in}}%
\pgfpathlineto{\pgfqpoint{2.304191in}{2.838531in}}%
\pgfpathlineto{\pgfqpoint{2.306213in}{2.839994in}}%
\pgfpathlineto{\pgfqpoint{2.309330in}{2.846538in}}%
\pgfpathlineto{\pgfqpoint{2.310594in}{2.847572in}}%
\pgfpathlineto{\pgfqpoint{2.312953in}{2.850225in}}%
\pgfpathlineto{\pgfqpoint{2.316828in}{2.850665in}}%
\pgfpathlineto{\pgfqpoint{2.318513in}{2.853009in}}%
\pgfpathlineto{\pgfqpoint{2.319355in}{2.853833in}}%
\pgfpathlineto{\pgfqpoint{2.319861in}{2.853322in}}%
\pgfpathlineto{\pgfqpoint{2.320535in}{2.853069in}}%
\pgfpathlineto{\pgfqpoint{2.320872in}{2.853679in}}%
\pgfpathlineto{\pgfqpoint{2.322472in}{2.857807in}}%
\pgfpathlineto{\pgfqpoint{2.323062in}{2.856841in}}%
\pgfpathlineto{\pgfqpoint{2.323904in}{2.854910in}}%
\pgfpathlineto{\pgfqpoint{2.324326in}{2.855979in}}%
\pgfpathlineto{\pgfqpoint{2.330644in}{2.884930in}}%
\pgfpathlineto{\pgfqpoint{2.331571in}{2.884080in}}%
\pgfpathlineto{\pgfqpoint{2.333003in}{2.883592in}}%
\pgfpathlineto{\pgfqpoint{2.333171in}{2.883732in}}%
\pgfpathlineto{\pgfqpoint{2.334519in}{2.883407in}}%
\pgfpathlineto{\pgfqpoint{2.335699in}{2.882446in}}%
\pgfpathlineto{\pgfqpoint{2.336120in}{2.883275in}}%
\pgfpathlineto{\pgfqpoint{2.337636in}{2.887138in}}%
\pgfpathlineto{\pgfqpoint{2.338142in}{2.886206in}}%
\pgfpathlineto{\pgfqpoint{2.338900in}{2.884537in}}%
\pgfpathlineto{\pgfqpoint{2.339405in}{2.885502in}}%
\pgfpathlineto{\pgfqpoint{2.340416in}{2.887569in}}%
\pgfpathlineto{\pgfqpoint{2.340838in}{2.886900in}}%
\pgfpathlineto{\pgfqpoint{2.341764in}{2.879125in}}%
\pgfpathlineto{\pgfqpoint{2.343196in}{2.870962in}}%
\pgfpathlineto{\pgfqpoint{2.343533in}{2.871366in}}%
\pgfpathlineto{\pgfqpoint{2.344292in}{2.872432in}}%
\pgfpathlineto{\pgfqpoint{2.344629in}{2.871579in}}%
\pgfpathlineto{\pgfqpoint{2.346061in}{2.859285in}}%
\pgfpathlineto{\pgfqpoint{2.347072in}{2.855804in}}%
\pgfpathlineto{\pgfqpoint{2.347577in}{2.856079in}}%
\pgfpathlineto{\pgfqpoint{2.351452in}{2.856639in}}%
\pgfpathlineto{\pgfqpoint{2.352632in}{2.855537in}}%
\pgfpathlineto{\pgfqpoint{2.357013in}{2.847122in}}%
\pgfpathlineto{\pgfqpoint{2.357687in}{2.848314in}}%
\pgfpathlineto{\pgfqpoint{2.359371in}{2.856057in}}%
\pgfpathlineto{\pgfqpoint{2.360298in}{2.854344in}}%
\pgfpathlineto{\pgfqpoint{2.361141in}{2.853116in}}%
\pgfpathlineto{\pgfqpoint{2.361562in}{2.853877in}}%
\pgfpathlineto{\pgfqpoint{2.363331in}{2.857915in}}%
\pgfpathlineto{\pgfqpoint{2.363836in}{2.857373in}}%
\pgfpathlineto{\pgfqpoint{2.365942in}{2.856091in}}%
\pgfpathlineto{\pgfqpoint{2.370070in}{2.857333in}}%
\pgfpathlineto{\pgfqpoint{2.372261in}{2.860701in}}%
\pgfpathlineto{\pgfqpoint{2.374535in}{2.859495in}}%
\pgfpathlineto{\pgfqpoint{2.378411in}{2.848797in}}%
\pgfpathlineto{\pgfqpoint{2.380517in}{2.846223in}}%
\pgfpathlineto{\pgfqpoint{2.382033in}{2.847669in}}%
\pgfpathlineto{\pgfqpoint{2.383297in}{2.859618in}}%
\pgfpathlineto{\pgfqpoint{2.385571in}{2.875362in}}%
\pgfpathlineto{\pgfqpoint{2.386245in}{2.876447in}}%
\pgfpathlineto{\pgfqpoint{2.388015in}{2.889879in}}%
\pgfpathlineto{\pgfqpoint{2.390036in}{2.899952in}}%
\pgfpathlineto{\pgfqpoint{2.394923in}{2.901198in}}%
\pgfpathlineto{\pgfqpoint{2.396186in}{2.901500in}}%
\pgfpathlineto{\pgfqpoint{2.396355in}{2.901361in}}%
\pgfpathlineto{\pgfqpoint{2.398040in}{2.899488in}}%
\pgfpathlineto{\pgfqpoint{2.398545in}{2.900563in}}%
\pgfpathlineto{\pgfqpoint{2.399135in}{2.901613in}}%
\pgfpathlineto{\pgfqpoint{2.399556in}{2.900309in}}%
\pgfpathlineto{\pgfqpoint{2.401746in}{2.882786in}}%
\pgfpathlineto{\pgfqpoint{2.402673in}{2.887909in}}%
\pgfpathlineto{\pgfqpoint{2.403431in}{2.891617in}}%
\pgfpathlineto{\pgfqpoint{2.403937in}{2.890005in}}%
\pgfpathlineto{\pgfqpoint{2.405959in}{2.872281in}}%
\pgfpathlineto{\pgfqpoint{2.407054in}{2.876878in}}%
\pgfpathlineto{\pgfqpoint{2.408907in}{2.882641in}}%
\pgfpathlineto{\pgfqpoint{2.409328in}{2.882511in}}%
\pgfpathlineto{\pgfqpoint{2.411182in}{2.883159in}}%
\pgfpathlineto{\pgfqpoint{2.415057in}{2.886548in}}%
\pgfpathlineto{\pgfqpoint{2.415141in}{2.886428in}}%
\pgfpathlineto{\pgfqpoint{2.416068in}{2.881820in}}%
\pgfpathlineto{\pgfqpoint{2.417500in}{2.877378in}}%
\pgfpathlineto{\pgfqpoint{2.417837in}{2.877880in}}%
\pgfpathlineto{\pgfqpoint{2.418848in}{2.880503in}}%
\pgfpathlineto{\pgfqpoint{2.419354in}{2.879287in}}%
\pgfpathlineto{\pgfqpoint{2.421544in}{2.867227in}}%
\pgfpathlineto{\pgfqpoint{2.422471in}{2.869128in}}%
\pgfpathlineto{\pgfqpoint{2.422976in}{2.869618in}}%
\pgfpathlineto{\pgfqpoint{2.423397in}{2.868836in}}%
\pgfpathlineto{\pgfqpoint{2.425588in}{2.860948in}}%
\pgfpathlineto{\pgfqpoint{2.426430in}{2.862928in}}%
\pgfpathlineto{\pgfqpoint{2.427694in}{2.864227in}}%
\pgfpathlineto{\pgfqpoint{2.427946in}{2.864019in}}%
\pgfpathlineto{\pgfqpoint{2.432748in}{2.859335in}}%
\pgfpathlineto{\pgfqpoint{2.433001in}{2.859578in}}%
\pgfpathlineto{\pgfqpoint{2.434181in}{2.860048in}}%
\pgfpathlineto{\pgfqpoint{2.434433in}{2.859708in}}%
\pgfpathlineto{\pgfqpoint{2.436455in}{2.854923in}}%
\pgfpathlineto{\pgfqpoint{2.437045in}{2.856543in}}%
\pgfpathlineto{\pgfqpoint{2.438393in}{2.860543in}}%
\pgfpathlineto{\pgfqpoint{2.438898in}{2.859854in}}%
\pgfpathlineto{\pgfqpoint{2.440162in}{2.857303in}}%
\pgfpathlineto{\pgfqpoint{2.440667in}{2.858341in}}%
\pgfpathlineto{\pgfqpoint{2.442268in}{2.864764in}}%
\pgfpathlineto{\pgfqpoint{2.443026in}{2.863315in}}%
\pgfpathlineto{\pgfqpoint{2.443784in}{2.862376in}}%
\pgfpathlineto{\pgfqpoint{2.444290in}{2.863104in}}%
\pgfpathlineto{\pgfqpoint{2.445891in}{2.865357in}}%
\pgfpathlineto{\pgfqpoint{2.446312in}{2.865157in}}%
\pgfpathlineto{\pgfqpoint{2.447070in}{2.865852in}}%
\pgfpathlineto{\pgfqpoint{2.449597in}{2.872668in}}%
\pgfpathlineto{\pgfqpoint{2.450356in}{2.871874in}}%
\pgfpathlineto{\pgfqpoint{2.453810in}{2.866979in}}%
\pgfpathlineto{\pgfqpoint{2.455831in}{2.862810in}}%
\pgfpathlineto{\pgfqpoint{2.456927in}{2.861112in}}%
\pgfpathlineto{\pgfqpoint{2.461392in}{2.851210in}}%
\pgfpathlineto{\pgfqpoint{2.462065in}{2.850960in}}%
\pgfpathlineto{\pgfqpoint{2.462487in}{2.851436in}}%
\pgfpathlineto{\pgfqpoint{2.463413in}{2.852128in}}%
\pgfpathlineto{\pgfqpoint{2.463750in}{2.851588in}}%
\pgfpathlineto{\pgfqpoint{2.466109in}{2.844641in}}%
\pgfpathlineto{\pgfqpoint{2.466867in}{2.845859in}}%
\pgfpathlineto{\pgfqpoint{2.468468in}{2.847786in}}%
\pgfpathlineto{\pgfqpoint{2.468889in}{2.847428in}}%
\pgfpathlineto{\pgfqpoint{2.471922in}{2.845852in}}%
\pgfpathlineto{\pgfqpoint{2.477314in}{2.846760in}}%
\pgfpathlineto{\pgfqpoint{2.478325in}{2.846156in}}%
\pgfpathlineto{\pgfqpoint{2.479504in}{2.844410in}}%
\pgfpathlineto{\pgfqpoint{2.479841in}{2.845286in}}%
\pgfpathlineto{\pgfqpoint{2.481357in}{2.857829in}}%
\pgfpathlineto{\pgfqpoint{2.483127in}{2.872511in}}%
\pgfpathlineto{\pgfqpoint{2.483716in}{2.871901in}}%
\pgfpathlineto{\pgfqpoint{2.484306in}{2.871732in}}%
\pgfpathlineto{\pgfqpoint{2.484643in}{2.872322in}}%
\pgfpathlineto{\pgfqpoint{2.485907in}{2.879474in}}%
\pgfpathlineto{\pgfqpoint{2.487676in}{2.887562in}}%
\pgfpathlineto{\pgfqpoint{2.488097in}{2.887314in}}%
\pgfpathlineto{\pgfqpoint{2.490793in}{2.886540in}}%
\pgfpathlineto{\pgfqpoint{2.491888in}{2.887548in}}%
\pgfpathlineto{\pgfqpoint{2.493404in}{2.893928in}}%
\pgfpathlineto{\pgfqpoint{2.495511in}{2.901566in}}%
\pgfpathlineto{\pgfqpoint{2.496269in}{2.905000in}}%
\pgfpathlineto{\pgfqpoint{2.498122in}{2.914714in}}%
\pgfpathlineto{\pgfqpoint{2.498459in}{2.913887in}}%
\pgfpathlineto{\pgfqpoint{2.500060in}{2.904344in}}%
\pgfpathlineto{\pgfqpoint{2.501071in}{2.905499in}}%
\pgfpathlineto{\pgfqpoint{2.501576in}{2.904652in}}%
\pgfpathlineto{\pgfqpoint{2.504440in}{2.891549in}}%
\pgfpathlineto{\pgfqpoint{2.505536in}{2.892971in}}%
\pgfpathlineto{\pgfqpoint{2.506884in}{2.899615in}}%
\pgfpathlineto{\pgfqpoint{2.508316in}{2.906914in}}%
\pgfpathlineto{\pgfqpoint{2.508737in}{2.906575in}}%
\pgfpathlineto{\pgfqpoint{2.511012in}{2.905148in}}%
\pgfpathlineto{\pgfqpoint{2.512022in}{2.906042in}}%
\pgfpathlineto{\pgfqpoint{2.512444in}{2.906051in}}%
\pgfpathlineto{\pgfqpoint{2.512696in}{2.905518in}}%
\pgfpathlineto{\pgfqpoint{2.513623in}{2.898164in}}%
\pgfpathlineto{\pgfqpoint{2.516235in}{2.881637in}}%
\pgfpathlineto{\pgfqpoint{2.523732in}{2.875336in}}%
\pgfpathlineto{\pgfqpoint{2.525080in}{2.870432in}}%
\pgfpathlineto{\pgfqpoint{2.525586in}{2.870611in}}%
\pgfpathlineto{\pgfqpoint{2.526260in}{2.869732in}}%
\pgfpathlineto{\pgfqpoint{2.527439in}{2.861208in}}%
\pgfpathlineto{\pgfqpoint{2.529293in}{2.855283in}}%
\pgfpathlineto{\pgfqpoint{2.530304in}{2.853810in}}%
\pgfpathlineto{\pgfqpoint{2.532241in}{2.845574in}}%
\pgfpathlineto{\pgfqpoint{2.533673in}{2.846239in}}%
\pgfpathlineto{\pgfqpoint{2.535695in}{2.847009in}}%
\pgfpathlineto{\pgfqpoint{2.537801in}{2.848175in}}%
\pgfpathlineto{\pgfqpoint{2.540076in}{2.850282in}}%
\pgfpathlineto{\pgfqpoint{2.540666in}{2.848767in}}%
\pgfpathlineto{\pgfqpoint{2.542182in}{2.846478in}}%
\pgfpathlineto{\pgfqpoint{2.542435in}{2.846732in}}%
\pgfpathlineto{\pgfqpoint{2.546142in}{2.850926in}}%
\pgfpathlineto{\pgfqpoint{2.548922in}{2.851416in}}%
\pgfpathlineto{\pgfqpoint{2.549511in}{2.851438in}}%
\pgfpathlineto{\pgfqpoint{2.549764in}{2.851920in}}%
\pgfpathlineto{\pgfqpoint{2.552039in}{2.858070in}}%
\pgfpathlineto{\pgfqpoint{2.552797in}{2.857081in}}%
\pgfpathlineto{\pgfqpoint{2.553639in}{2.856483in}}%
\pgfpathlineto{\pgfqpoint{2.553976in}{2.857001in}}%
\pgfpathlineto{\pgfqpoint{2.556251in}{2.862480in}}%
\pgfpathlineto{\pgfqpoint{2.556841in}{2.861837in}}%
\pgfpathlineto{\pgfqpoint{2.558525in}{2.861089in}}%
\pgfpathlineto{\pgfqpoint{2.559621in}{2.860342in}}%
\pgfpathlineto{\pgfqpoint{2.562906in}{2.856210in}}%
\pgfpathlineto{\pgfqpoint{2.564001in}{2.857115in}}%
\pgfpathlineto{\pgfqpoint{2.565349in}{2.863667in}}%
\pgfpathlineto{\pgfqpoint{2.566192in}{2.865840in}}%
\pgfpathlineto{\pgfqpoint{2.566697in}{2.865091in}}%
\pgfpathlineto{\pgfqpoint{2.569983in}{2.859060in}}%
\pgfpathlineto{\pgfqpoint{2.570994in}{2.857436in}}%
\pgfpathlineto{\pgfqpoint{2.573858in}{2.848280in}}%
\pgfpathlineto{\pgfqpoint{2.574448in}{2.848820in}}%
\pgfpathlineto{\pgfqpoint{2.577986in}{2.850729in}}%
\pgfpathlineto{\pgfqpoint{2.580935in}{2.848471in}}%
\pgfpathlineto{\pgfqpoint{2.585315in}{2.837411in}}%
\pgfpathlineto{\pgfqpoint{2.587253in}{2.838054in}}%
\pgfpathlineto{\pgfqpoint{2.592560in}{2.840941in}}%
\pgfpathlineto{\pgfqpoint{2.592645in}{2.840817in}}%
\pgfpathlineto{\pgfqpoint{2.593571in}{2.839869in}}%
\pgfpathlineto{\pgfqpoint{2.594161in}{2.840436in}}%
\pgfpathlineto{\pgfqpoint{2.600563in}{2.847575in}}%
\pgfpathlineto{\pgfqpoint{2.606376in}{2.875680in}}%
\pgfpathlineto{\pgfqpoint{2.607556in}{2.873077in}}%
\pgfpathlineto{\pgfqpoint{2.609156in}{2.871779in}}%
\pgfpathlineto{\pgfqpoint{2.612189in}{2.871944in}}%
\pgfpathlineto{\pgfqpoint{2.613200in}{2.877739in}}%
\pgfpathlineto{\pgfqpoint{2.614464in}{2.882337in}}%
\pgfpathlineto{\pgfqpoint{2.614885in}{2.882169in}}%
\pgfpathlineto{\pgfqpoint{2.615559in}{2.883088in}}%
\pgfpathlineto{\pgfqpoint{2.617412in}{2.894376in}}%
\pgfpathlineto{\pgfqpoint{2.617918in}{2.895837in}}%
\pgfpathlineto{\pgfqpoint{2.618508in}{2.894478in}}%
\pgfpathlineto{\pgfqpoint{2.619266in}{2.893072in}}%
\pgfpathlineto{\pgfqpoint{2.619855in}{2.893618in}}%
\pgfpathlineto{\pgfqpoint{2.620277in}{2.893824in}}%
\pgfpathlineto{\pgfqpoint{2.620614in}{2.893063in}}%
\pgfpathlineto{\pgfqpoint{2.624994in}{2.878485in}}%
\pgfpathlineto{\pgfqpoint{2.628954in}{2.876839in}}%
\pgfpathlineto{\pgfqpoint{2.634683in}{2.860436in}}%
\pgfpathlineto{\pgfqpoint{2.635356in}{2.862226in}}%
\pgfpathlineto{\pgfqpoint{2.636283in}{2.863961in}}%
\pgfpathlineto{\pgfqpoint{2.636789in}{2.863296in}}%
\pgfpathlineto{\pgfqpoint{2.637968in}{2.860554in}}%
\pgfpathlineto{\pgfqpoint{2.638558in}{2.861498in}}%
\pgfpathlineto{\pgfqpoint{2.642770in}{2.870303in}}%
\pgfpathlineto{\pgfqpoint{2.645634in}{2.875574in}}%
\pgfpathlineto{\pgfqpoint{2.646393in}{2.874792in}}%
\pgfpathlineto{\pgfqpoint{2.649425in}{2.871013in}}%
\pgfpathlineto{\pgfqpoint{2.650099in}{2.870104in}}%
\pgfpathlineto{\pgfqpoint{2.652458in}{2.860195in}}%
\pgfpathlineto{\pgfqpoint{2.653806in}{2.861421in}}%
\pgfpathlineto{\pgfqpoint{2.655070in}{2.860460in}}%
\pgfpathlineto{\pgfqpoint{2.658524in}{2.858773in}}%
\pgfpathlineto{\pgfqpoint{2.658776in}{2.859221in}}%
\pgfpathlineto{\pgfqpoint{2.660040in}{2.860899in}}%
\pgfpathlineto{\pgfqpoint{2.660546in}{2.860318in}}%
\pgfpathlineto{\pgfqpoint{2.661809in}{2.858866in}}%
\pgfpathlineto{\pgfqpoint{2.662146in}{2.859404in}}%
\pgfpathlineto{\pgfqpoint{2.665853in}{2.866167in}}%
\pgfpathlineto{\pgfqpoint{2.668043in}{2.868719in}}%
\pgfpathlineto{\pgfqpoint{2.668380in}{2.868354in}}%
\pgfpathlineto{\pgfqpoint{2.669813in}{2.863674in}}%
\pgfpathlineto{\pgfqpoint{2.671413in}{2.862528in}}%
\pgfpathlineto{\pgfqpoint{2.672087in}{2.861194in}}%
\pgfpathlineto{\pgfqpoint{2.674362in}{2.853259in}}%
\pgfpathlineto{\pgfqpoint{2.674867in}{2.853706in}}%
\pgfpathlineto{\pgfqpoint{2.675457in}{2.853789in}}%
\pgfpathlineto{\pgfqpoint{2.675794in}{2.853140in}}%
\pgfpathlineto{\pgfqpoint{2.677563in}{2.849130in}}%
\pgfpathlineto{\pgfqpoint{2.678068in}{2.850088in}}%
\pgfpathlineto{\pgfqpoint{2.679079in}{2.852005in}}%
\pgfpathlineto{\pgfqpoint{2.679585in}{2.851411in}}%
\pgfpathlineto{\pgfqpoint{2.684555in}{2.844470in}}%
\pgfpathlineto{\pgfqpoint{2.686914in}{2.845234in}}%
\pgfpathlineto{\pgfqpoint{2.688094in}{2.846702in}}%
\pgfpathlineto{\pgfqpoint{2.691042in}{2.855447in}}%
\pgfpathlineto{\pgfqpoint{2.691716in}{2.854685in}}%
\pgfpathlineto{\pgfqpoint{2.692896in}{2.854063in}}%
\pgfpathlineto{\pgfqpoint{2.693232in}{2.854364in}}%
\pgfpathlineto{\pgfqpoint{2.694328in}{2.854813in}}%
\pgfpathlineto{\pgfqpoint{2.694665in}{2.854261in}}%
\pgfpathlineto{\pgfqpoint{2.698203in}{2.848403in}}%
\pgfpathlineto{\pgfqpoint{2.705027in}{2.846696in}}%
\pgfpathlineto{\pgfqpoint{2.708481in}{2.837738in}}%
\pgfpathlineto{\pgfqpoint{2.708649in}{2.837789in}}%
\pgfpathlineto{\pgfqpoint{2.711261in}{2.837478in}}%
\pgfpathlineto{\pgfqpoint{2.713198in}{2.838550in}}%
\pgfpathlineto{\pgfqpoint{2.717663in}{2.846863in}}%
\pgfpathlineto{\pgfqpoint{2.719854in}{2.852952in}}%
\pgfpathlineto{\pgfqpoint{2.721117in}{2.854826in}}%
\pgfpathlineto{\pgfqpoint{2.725245in}{2.866507in}}%
\pgfpathlineto{\pgfqpoint{2.727941in}{2.868800in}}%
\pgfpathlineto{\pgfqpoint{2.730132in}{2.867572in}}%
\pgfpathlineto{\pgfqpoint{2.731985in}{2.865347in}}%
\pgfpathlineto{\pgfqpoint{2.732322in}{2.865601in}}%
\pgfpathlineto{\pgfqpoint{2.733501in}{2.866494in}}%
\pgfpathlineto{\pgfqpoint{2.733838in}{2.865811in}}%
\pgfpathlineto{\pgfqpoint{2.735523in}{2.860660in}}%
\pgfpathlineto{\pgfqpoint{2.736029in}{2.862321in}}%
\pgfpathlineto{\pgfqpoint{2.737545in}{2.869094in}}%
\pgfpathlineto{\pgfqpoint{2.738051in}{2.868167in}}%
\pgfpathlineto{\pgfqpoint{2.739062in}{2.864937in}}%
\pgfpathlineto{\pgfqpoint{2.739567in}{2.866437in}}%
\pgfpathlineto{\pgfqpoint{2.742179in}{2.882978in}}%
\pgfpathlineto{\pgfqpoint{2.743274in}{2.882206in}}%
\pgfpathlineto{\pgfqpoint{2.743863in}{2.884389in}}%
\pgfpathlineto{\pgfqpoint{2.746980in}{2.899041in}}%
\pgfpathlineto{\pgfqpoint{2.748328in}{2.904573in}}%
\pgfpathlineto{\pgfqpoint{2.749508in}{2.908047in}}%
\pgfpathlineto{\pgfqpoint{2.750013in}{2.907697in}}%
\pgfpathlineto{\pgfqpoint{2.750603in}{2.908671in}}%
\pgfpathlineto{\pgfqpoint{2.752372in}{2.919199in}}%
\pgfpathlineto{\pgfqpoint{2.753130in}{2.915409in}}%
\pgfpathlineto{\pgfqpoint{2.754057in}{2.911586in}}%
\pgfpathlineto{\pgfqpoint{2.754562in}{2.913082in}}%
\pgfpathlineto{\pgfqpoint{2.755658in}{2.917943in}}%
\pgfpathlineto{\pgfqpoint{2.756163in}{2.916509in}}%
\pgfpathlineto{\pgfqpoint{2.758101in}{2.905250in}}%
\pgfpathlineto{\pgfqpoint{2.758859in}{2.907042in}}%
\pgfpathlineto{\pgfqpoint{2.760544in}{2.908955in}}%
\pgfpathlineto{\pgfqpoint{2.761386in}{2.909908in}}%
\pgfpathlineto{\pgfqpoint{2.763829in}{2.914575in}}%
\pgfpathlineto{\pgfqpoint{2.767536in}{2.915018in}}%
\pgfpathlineto{\pgfqpoint{2.767789in}{2.915006in}}%
\pgfpathlineto{\pgfqpoint{2.768042in}{2.914393in}}%
\pgfpathlineto{\pgfqpoint{2.769811in}{2.907767in}}%
\pgfpathlineto{\pgfqpoint{2.770400in}{2.909479in}}%
\pgfpathlineto{\pgfqpoint{2.772170in}{2.919029in}}%
\pgfpathlineto{\pgfqpoint{2.773181in}{2.918701in}}%
\pgfpathlineto{\pgfqpoint{2.774023in}{2.921980in}}%
\pgfpathlineto{\pgfqpoint{2.776382in}{2.928226in}}%
\pgfpathlineto{\pgfqpoint{2.777309in}{2.927930in}}%
\pgfpathlineto{\pgfqpoint{2.777393in}{2.927739in}}%
\pgfpathlineto{\pgfqpoint{2.778825in}{2.921925in}}%
\pgfpathlineto{\pgfqpoint{2.779499in}{2.924828in}}%
\pgfpathlineto{\pgfqpoint{2.780847in}{2.930176in}}%
\pgfpathlineto{\pgfqpoint{2.781268in}{2.929101in}}%
\pgfpathlineto{\pgfqpoint{2.782616in}{2.921354in}}%
\pgfpathlineto{\pgfqpoint{2.783458in}{2.923199in}}%
\pgfpathlineto{\pgfqpoint{2.783964in}{2.923677in}}%
\pgfpathlineto{\pgfqpoint{2.784385in}{2.922883in}}%
\pgfpathlineto{\pgfqpoint{2.787671in}{2.909770in}}%
\pgfpathlineto{\pgfqpoint{2.790029in}{2.893887in}}%
\pgfpathlineto{\pgfqpoint{2.790619in}{2.891903in}}%
\pgfpathlineto{\pgfqpoint{2.798454in}{2.844757in}}%
\pgfpathlineto{\pgfqpoint{2.798959in}{2.845048in}}%
\pgfpathlineto{\pgfqpoint{2.800476in}{2.846612in}}%
\pgfpathlineto{\pgfqpoint{2.800981in}{2.845706in}}%
\pgfpathlineto{\pgfqpoint{2.802076in}{2.843628in}}%
\pgfpathlineto{\pgfqpoint{2.802498in}{2.844457in}}%
\pgfpathlineto{\pgfqpoint{2.804267in}{2.857624in}}%
\pgfpathlineto{\pgfqpoint{2.806289in}{2.861935in}}%
\pgfpathlineto{\pgfqpoint{2.809321in}{2.864867in}}%
\pgfpathlineto{\pgfqpoint{2.812860in}{2.878236in}}%
\pgfpathlineto{\pgfqpoint{2.815134in}{2.893837in}}%
\pgfpathlineto{\pgfqpoint{2.815724in}{2.893601in}}%
\pgfpathlineto{\pgfqpoint{2.817072in}{2.890980in}}%
\pgfpathlineto{\pgfqpoint{2.818167in}{2.890267in}}%
\pgfpathlineto{\pgfqpoint{2.818504in}{2.890586in}}%
\pgfpathlineto{\pgfqpoint{2.819178in}{2.891181in}}%
\pgfpathlineto{\pgfqpoint{2.819515in}{2.890315in}}%
\pgfpathlineto{\pgfqpoint{2.821537in}{2.877143in}}%
\pgfpathlineto{\pgfqpoint{2.822379in}{2.880643in}}%
\pgfpathlineto{\pgfqpoint{2.823727in}{2.884430in}}%
\pgfpathlineto{\pgfqpoint{2.823980in}{2.884336in}}%
\pgfpathlineto{\pgfqpoint{2.824738in}{2.884038in}}%
\pgfpathlineto{\pgfqpoint{2.825075in}{2.884899in}}%
\pgfpathlineto{\pgfqpoint{2.826760in}{2.889289in}}%
\pgfpathlineto{\pgfqpoint{2.827097in}{2.888912in}}%
\pgfpathlineto{\pgfqpoint{2.828782in}{2.888106in}}%
\pgfpathlineto{\pgfqpoint{2.829877in}{2.885281in}}%
\pgfpathlineto{\pgfqpoint{2.831394in}{2.873837in}}%
\pgfpathlineto{\pgfqpoint{2.832320in}{2.877594in}}%
\pgfpathlineto{\pgfqpoint{2.837712in}{2.911522in}}%
\pgfpathlineto{\pgfqpoint{2.838470in}{2.913352in}}%
\pgfpathlineto{\pgfqpoint{2.844283in}{2.936471in}}%
\pgfpathlineto{\pgfqpoint{2.848074in}{2.943971in}}%
\pgfpathlineto{\pgfqpoint{2.848832in}{2.942147in}}%
\pgfpathlineto{\pgfqpoint{2.850096in}{2.931137in}}%
\pgfpathlineto{\pgfqpoint{2.850770in}{2.928333in}}%
\pgfpathlineto{\pgfqpoint{2.851444in}{2.928994in}}%
\pgfpathlineto{\pgfqpoint{2.851949in}{2.928161in}}%
\pgfpathlineto{\pgfqpoint{2.852960in}{2.918309in}}%
\pgfpathlineto{\pgfqpoint{2.853803in}{2.912914in}}%
\pgfpathlineto{\pgfqpoint{2.854392in}{2.914499in}}%
\pgfpathlineto{\pgfqpoint{2.854982in}{2.916103in}}%
\pgfpathlineto{\pgfqpoint{2.855403in}{2.914478in}}%
\pgfpathlineto{\pgfqpoint{2.857172in}{2.899871in}}%
\pgfpathlineto{\pgfqpoint{2.857846in}{2.903179in}}%
\pgfpathlineto{\pgfqpoint{2.858268in}{2.904975in}}%
\pgfpathlineto{\pgfqpoint{2.858689in}{2.902823in}}%
\pgfpathlineto{\pgfqpoint{2.860542in}{2.884566in}}%
\pgfpathlineto{\pgfqpoint{2.861132in}{2.886648in}}%
\pgfpathlineto{\pgfqpoint{2.861806in}{2.888834in}}%
\pgfpathlineto{\pgfqpoint{2.862311in}{2.887098in}}%
\pgfpathlineto{\pgfqpoint{2.864417in}{2.873910in}}%
\pgfpathlineto{\pgfqpoint{2.865428in}{2.874069in}}%
\pgfpathlineto{\pgfqpoint{2.868377in}{2.874938in}}%
\pgfpathlineto{\pgfqpoint{2.869388in}{2.881111in}}%
\pgfpathlineto{\pgfqpoint{2.870567in}{2.891136in}}%
\pgfpathlineto{\pgfqpoint{2.871157in}{2.888822in}}%
\pgfpathlineto{\pgfqpoint{2.872084in}{2.885186in}}%
\pgfpathlineto{\pgfqpoint{2.872505in}{2.887053in}}%
\pgfpathlineto{\pgfqpoint{2.874190in}{2.903527in}}%
\pgfpathlineto{\pgfqpoint{2.874948in}{2.899701in}}%
\pgfpathlineto{\pgfqpoint{2.875706in}{2.896306in}}%
\pgfpathlineto{\pgfqpoint{2.876212in}{2.897807in}}%
\pgfpathlineto{\pgfqpoint{2.877812in}{2.906631in}}%
\pgfpathlineto{\pgfqpoint{2.878486in}{2.904744in}}%
\pgfpathlineto{\pgfqpoint{2.879497in}{2.903074in}}%
\pgfpathlineto{\pgfqpoint{2.879918in}{2.903599in}}%
\pgfpathlineto{\pgfqpoint{2.882193in}{2.906010in}}%
\pgfpathlineto{\pgfqpoint{2.885394in}{2.907281in}}%
\pgfpathlineto{\pgfqpoint{2.886911in}{2.913033in}}%
\pgfpathlineto{\pgfqpoint{2.887669in}{2.912623in}}%
\pgfpathlineto{\pgfqpoint{2.888511in}{2.913826in}}%
\pgfpathlineto{\pgfqpoint{2.890280in}{2.927141in}}%
\pgfpathlineto{\pgfqpoint{2.892808in}{2.939702in}}%
\pgfpathlineto{\pgfqpoint{2.893566in}{2.942668in}}%
\pgfpathlineto{\pgfqpoint{2.894577in}{2.946104in}}%
\pgfpathlineto{\pgfqpoint{2.895167in}{2.945940in}}%
\pgfpathlineto{\pgfqpoint{2.896262in}{2.947008in}}%
\pgfpathlineto{\pgfqpoint{2.901569in}{2.964758in}}%
\pgfpathlineto{\pgfqpoint{2.901906in}{2.964085in}}%
\pgfpathlineto{\pgfqpoint{2.903001in}{2.955049in}}%
\pgfpathlineto{\pgfqpoint{2.904012in}{2.948144in}}%
\pgfpathlineto{\pgfqpoint{2.904518in}{2.949563in}}%
\pgfpathlineto{\pgfqpoint{2.905192in}{2.951299in}}%
\pgfpathlineto{\pgfqpoint{2.905613in}{2.949444in}}%
\pgfpathlineto{\pgfqpoint{2.907214in}{2.922371in}}%
\pgfpathlineto{\pgfqpoint{2.908309in}{2.912642in}}%
\pgfpathlineto{\pgfqpoint{2.908814in}{2.913532in}}%
\pgfpathlineto{\pgfqpoint{2.909320in}{2.914033in}}%
\pgfpathlineto{\pgfqpoint{2.909657in}{2.913220in}}%
\pgfpathlineto{\pgfqpoint{2.910752in}{2.902512in}}%
\pgfpathlineto{\pgfqpoint{2.911847in}{2.896883in}}%
\pgfpathlineto{\pgfqpoint{2.912268in}{2.897272in}}%
\pgfpathlineto{\pgfqpoint{2.912774in}{2.897595in}}%
\pgfpathlineto{\pgfqpoint{2.913111in}{2.896830in}}%
\pgfpathlineto{\pgfqpoint{2.918502in}{2.876551in}}%
\pgfpathlineto{\pgfqpoint{2.918755in}{2.876855in}}%
\pgfpathlineto{\pgfqpoint{2.919598in}{2.878567in}}%
\pgfpathlineto{\pgfqpoint{2.920103in}{2.877318in}}%
\pgfpathlineto{\pgfqpoint{2.921872in}{2.866534in}}%
\pgfpathlineto{\pgfqpoint{2.922630in}{2.870050in}}%
\pgfpathlineto{\pgfqpoint{2.924062in}{2.877174in}}%
\pgfpathlineto{\pgfqpoint{2.924484in}{2.875453in}}%
\pgfpathlineto{\pgfqpoint{2.926000in}{2.866777in}}%
\pgfpathlineto{\pgfqpoint{2.926506in}{2.868154in}}%
\pgfpathlineto{\pgfqpoint{2.927348in}{2.870592in}}%
\pgfpathlineto{\pgfqpoint{2.927938in}{2.869720in}}%
\pgfpathlineto{\pgfqpoint{2.929623in}{2.866157in}}%
\pgfpathlineto{\pgfqpoint{2.930465in}{2.866678in}}%
\pgfpathlineto{\pgfqpoint{2.932571in}{2.865575in}}%
\pgfpathlineto{\pgfqpoint{2.933751in}{2.866164in}}%
\pgfpathlineto{\pgfqpoint{2.935267in}{2.868577in}}%
\pgfpathlineto{\pgfqpoint{2.935772in}{2.867603in}}%
\pgfpathlineto{\pgfqpoint{2.936699in}{2.865495in}}%
\pgfpathlineto{\pgfqpoint{2.937205in}{2.866270in}}%
\pgfpathlineto{\pgfqpoint{2.938468in}{2.869143in}}%
\pgfpathlineto{\pgfqpoint{2.938890in}{2.867881in}}%
\pgfpathlineto{\pgfqpoint{2.941417in}{2.852298in}}%
\pgfpathlineto{\pgfqpoint{2.942512in}{2.853634in}}%
\pgfpathlineto{\pgfqpoint{2.943102in}{2.852642in}}%
\pgfpathlineto{\pgfqpoint{2.945545in}{2.847274in}}%
\pgfpathlineto{\pgfqpoint{2.947145in}{2.846086in}}%
\pgfpathlineto{\pgfqpoint{2.949420in}{2.845609in}}%
\pgfpathlineto{\pgfqpoint{2.950347in}{2.845926in}}%
\pgfpathlineto{\pgfqpoint{2.950684in}{2.845444in}}%
\pgfpathlineto{\pgfqpoint{2.951695in}{2.843633in}}%
\pgfpathlineto{\pgfqpoint{2.952116in}{2.844608in}}%
\pgfpathlineto{\pgfqpoint{2.954391in}{2.853321in}}%
\pgfpathlineto{\pgfqpoint{2.955233in}{2.852096in}}%
\pgfpathlineto{\pgfqpoint{2.955570in}{2.852027in}}%
\pgfpathlineto{\pgfqpoint{2.955907in}{2.852758in}}%
\pgfpathlineto{\pgfqpoint{2.956834in}{2.860621in}}%
\pgfpathlineto{\pgfqpoint{2.960035in}{2.890005in}}%
\pgfpathlineto{\pgfqpoint{2.960877in}{2.891418in}}%
\pgfpathlineto{\pgfqpoint{2.961804in}{2.901489in}}%
\pgfpathlineto{\pgfqpoint{2.964247in}{2.921722in}}%
\pgfpathlineto{\pgfqpoint{2.966353in}{2.924498in}}%
\pgfpathlineto{\pgfqpoint{2.968207in}{2.929694in}}%
\pgfpathlineto{\pgfqpoint{2.968628in}{2.929165in}}%
\pgfpathlineto{\pgfqpoint{2.973683in}{2.913611in}}%
\pgfpathlineto{\pgfqpoint{2.976715in}{2.882759in}}%
\pgfpathlineto{\pgfqpoint{2.977052in}{2.882838in}}%
\pgfpathlineto{\pgfqpoint{2.977726in}{2.881905in}}%
\pgfpathlineto{\pgfqpoint{2.978569in}{2.872691in}}%
\pgfpathlineto{\pgfqpoint{2.981096in}{2.848840in}}%
\pgfpathlineto{\pgfqpoint{2.982949in}{2.846267in}}%
\pgfpathlineto{\pgfqpoint{2.985056in}{2.839070in}}%
\pgfpathlineto{\pgfqpoint{2.985477in}{2.839327in}}%
\pgfpathlineto{\pgfqpoint{2.986319in}{2.839934in}}%
\pgfpathlineto{\pgfqpoint{2.986740in}{2.839222in}}%
\pgfpathlineto{\pgfqpoint{2.988173in}{2.837663in}}%
\pgfpathlineto{\pgfqpoint{2.988425in}{2.837853in}}%
\pgfpathlineto{\pgfqpoint{2.995333in}{2.845547in}}%
\pgfpathlineto{\pgfqpoint{2.997102in}{2.846317in}}%
\pgfpathlineto{\pgfqpoint{2.999461in}{2.852322in}}%
\pgfpathlineto{\pgfqpoint{3.001736in}{2.855147in}}%
\pgfpathlineto{\pgfqpoint{3.003421in}{2.857089in}}%
\pgfpathlineto{\pgfqpoint{3.005274in}{2.861244in}}%
\pgfpathlineto{\pgfqpoint{3.005780in}{2.860470in}}%
\pgfpathlineto{\pgfqpoint{3.007128in}{2.859345in}}%
\pgfpathlineto{\pgfqpoint{3.007296in}{2.859424in}}%
\pgfpathlineto{\pgfqpoint{3.008223in}{2.861396in}}%
\pgfpathlineto{\pgfqpoint{3.009908in}{2.863101in}}%
\pgfpathlineto{\pgfqpoint{3.013193in}{2.864734in}}%
\pgfpathlineto{\pgfqpoint{3.015552in}{2.875772in}}%
\pgfpathlineto{\pgfqpoint{3.016394in}{2.873961in}}%
\pgfpathlineto{\pgfqpoint{3.017237in}{2.872758in}}%
\pgfpathlineto{\pgfqpoint{3.017658in}{2.873688in}}%
\pgfpathlineto{\pgfqpoint{3.019175in}{2.881897in}}%
\pgfpathlineto{\pgfqpoint{3.020017in}{2.880252in}}%
\pgfpathlineto{\pgfqpoint{3.021112in}{2.877249in}}%
\pgfpathlineto{\pgfqpoint{3.021702in}{2.878278in}}%
\pgfpathlineto{\pgfqpoint{3.023555in}{2.883669in}}%
\pgfpathlineto{\pgfqpoint{3.024398in}{2.882974in}}%
\pgfpathlineto{\pgfqpoint{3.025324in}{2.881962in}}%
\pgfpathlineto{\pgfqpoint{3.025746in}{2.882865in}}%
\pgfpathlineto{\pgfqpoint{3.026925in}{2.885572in}}%
\pgfpathlineto{\pgfqpoint{3.027431in}{2.884980in}}%
\pgfpathlineto{\pgfqpoint{3.028357in}{2.884508in}}%
\pgfpathlineto{\pgfqpoint{3.028694in}{2.884902in}}%
\pgfpathlineto{\pgfqpoint{3.029705in}{2.886460in}}%
\pgfpathlineto{\pgfqpoint{3.030295in}{2.885532in}}%
\pgfpathlineto{\pgfqpoint{3.031559in}{2.879324in}}%
\pgfpathlineto{\pgfqpoint{3.032569in}{2.874775in}}%
\pgfpathlineto{\pgfqpoint{3.033075in}{2.875759in}}%
\pgfpathlineto{\pgfqpoint{3.034170in}{2.879477in}}%
\pgfpathlineto{\pgfqpoint{3.034676in}{2.877954in}}%
\pgfpathlineto{\pgfqpoint{3.036360in}{2.869556in}}%
\pgfpathlineto{\pgfqpoint{3.037034in}{2.870216in}}%
\pgfpathlineto{\pgfqpoint{3.037708in}{2.870572in}}%
\pgfpathlineto{\pgfqpoint{3.038130in}{2.869909in}}%
\pgfpathlineto{\pgfqpoint{3.044532in}{2.853350in}}%
\pgfpathlineto{\pgfqpoint{3.045206in}{2.854426in}}%
\pgfpathlineto{\pgfqpoint{3.047144in}{2.857712in}}%
\pgfpathlineto{\pgfqpoint{3.047565in}{2.857129in}}%
\pgfpathlineto{\pgfqpoint{3.048239in}{2.856712in}}%
\pgfpathlineto{\pgfqpoint{3.048660in}{2.857290in}}%
\pgfpathlineto{\pgfqpoint{3.049503in}{2.861758in}}%
\pgfpathlineto{\pgfqpoint{3.052367in}{2.881442in}}%
\pgfpathlineto{\pgfqpoint{3.053209in}{2.889752in}}%
\pgfpathlineto{\pgfqpoint{3.055568in}{2.907924in}}%
\pgfpathlineto{\pgfqpoint{3.056495in}{2.909354in}}%
\pgfpathlineto{\pgfqpoint{3.059359in}{2.914772in}}%
\pgfpathlineto{\pgfqpoint{3.062055in}{2.913554in}}%
\pgfpathlineto{\pgfqpoint{3.066688in}{2.897371in}}%
\pgfpathlineto{\pgfqpoint{3.068626in}{2.878757in}}%
\pgfpathlineto{\pgfqpoint{3.069047in}{2.878898in}}%
\pgfpathlineto{\pgfqpoint{3.069553in}{2.877736in}}%
\pgfpathlineto{\pgfqpoint{3.071322in}{2.860036in}}%
\pgfpathlineto{\pgfqpoint{3.072249in}{2.856454in}}%
\pgfpathlineto{\pgfqpoint{3.072754in}{2.857096in}}%
\pgfpathlineto{\pgfqpoint{3.073344in}{2.857340in}}%
\pgfpathlineto{\pgfqpoint{3.073849in}{2.856874in}}%
\pgfpathlineto{\pgfqpoint{3.076882in}{2.854926in}}%
\pgfpathlineto{\pgfqpoint{3.077051in}{2.855134in}}%
\pgfpathlineto{\pgfqpoint{3.080505in}{2.862265in}}%
\pgfpathlineto{\pgfqpoint{3.085728in}{2.874757in}}%
\pgfpathlineto{\pgfqpoint{3.086739in}{2.873363in}}%
\pgfpathlineto{\pgfqpoint{3.089098in}{2.870278in}}%
\pgfpathlineto{\pgfqpoint{3.090277in}{2.867881in}}%
\pgfpathlineto{\pgfqpoint{3.091288in}{2.865958in}}%
\pgfpathlineto{\pgfqpoint{3.091793in}{2.867001in}}%
\pgfpathlineto{\pgfqpoint{3.095669in}{2.877552in}}%
\pgfpathlineto{\pgfqpoint{3.098701in}{2.883946in}}%
\pgfpathlineto{\pgfqpoint{3.099038in}{2.883404in}}%
\pgfpathlineto{\pgfqpoint{3.100049in}{2.875950in}}%
\pgfpathlineto{\pgfqpoint{3.101903in}{2.869639in}}%
\pgfpathlineto{\pgfqpoint{3.103082in}{2.868560in}}%
\pgfpathlineto{\pgfqpoint{3.105441in}{2.865418in}}%
\pgfpathlineto{\pgfqpoint{3.108895in}{2.864131in}}%
\pgfpathlineto{\pgfqpoint{3.111001in}{2.854404in}}%
\pgfpathlineto{\pgfqpoint{3.111844in}{2.856947in}}%
\pgfpathlineto{\pgfqpoint{3.113360in}{2.862940in}}%
\pgfpathlineto{\pgfqpoint{3.113950in}{2.861329in}}%
\pgfpathlineto{\pgfqpoint{3.114961in}{2.857697in}}%
\pgfpathlineto{\pgfqpoint{3.115466in}{2.859188in}}%
\pgfpathlineto{\pgfqpoint{3.116730in}{2.875569in}}%
\pgfpathlineto{\pgfqpoint{3.118499in}{2.884845in}}%
\pgfpathlineto{\pgfqpoint{3.119173in}{2.886718in}}%
\pgfpathlineto{\pgfqpoint{3.120436in}{2.899289in}}%
\pgfpathlineto{\pgfqpoint{3.122290in}{2.910093in}}%
\pgfpathlineto{\pgfqpoint{3.123217in}{2.911740in}}%
\pgfpathlineto{\pgfqpoint{3.125660in}{2.919577in}}%
\pgfpathlineto{\pgfqpoint{3.126081in}{2.919448in}}%
\pgfpathlineto{\pgfqpoint{3.128692in}{2.917558in}}%
\pgfpathlineto{\pgfqpoint{3.130377in}{2.907484in}}%
\pgfpathlineto{\pgfqpoint{3.131136in}{2.911822in}}%
\pgfpathlineto{\pgfqpoint{3.132483in}{2.922340in}}%
\pgfpathlineto{\pgfqpoint{3.132905in}{2.920107in}}%
\pgfpathlineto{\pgfqpoint{3.134421in}{2.904800in}}%
\pgfpathlineto{\pgfqpoint{3.135011in}{2.908647in}}%
\pgfpathlineto{\pgfqpoint{3.136106in}{2.916579in}}%
\pgfpathlineto{\pgfqpoint{3.136611in}{2.914840in}}%
\pgfpathlineto{\pgfqpoint{3.138128in}{2.901301in}}%
\pgfpathlineto{\pgfqpoint{3.138970in}{2.904231in}}%
\pgfpathlineto{\pgfqpoint{3.139560in}{2.905421in}}%
\pgfpathlineto{\pgfqpoint{3.140065in}{2.904452in}}%
\pgfpathlineto{\pgfqpoint{3.141076in}{2.901323in}}%
\pgfpathlineto{\pgfqpoint{3.141582in}{2.902683in}}%
\pgfpathlineto{\pgfqpoint{3.146974in}{2.933984in}}%
\pgfpathlineto{\pgfqpoint{3.147816in}{2.929262in}}%
\pgfpathlineto{\pgfqpoint{3.149754in}{2.915248in}}%
\pgfpathlineto{\pgfqpoint{3.150175in}{2.915659in}}%
\pgfpathlineto{\pgfqpoint{3.150849in}{2.916182in}}%
\pgfpathlineto{\pgfqpoint{3.151186in}{2.915324in}}%
\pgfpathlineto{\pgfqpoint{3.152197in}{2.904857in}}%
\pgfpathlineto{\pgfqpoint{3.154050in}{2.895952in}}%
\pgfpathlineto{\pgfqpoint{3.155229in}{2.895148in}}%
\pgfpathlineto{\pgfqpoint{3.157251in}{2.891990in}}%
\pgfpathlineto{\pgfqpoint{3.158178in}{2.890014in}}%
\pgfpathlineto{\pgfqpoint{3.162811in}{2.864755in}}%
\pgfpathlineto{\pgfqpoint{3.164412in}{2.855424in}}%
\pgfpathlineto{\pgfqpoint{3.164665in}{2.855462in}}%
\pgfpathlineto{\pgfqpoint{3.165676in}{2.854558in}}%
\pgfpathlineto{\pgfqpoint{3.167950in}{2.850787in}}%
\pgfpathlineto{\pgfqpoint{3.170646in}{2.851725in}}%
\pgfpathlineto{\pgfqpoint{3.171404in}{2.852483in}}%
\pgfpathlineto{\pgfqpoint{3.171826in}{2.851895in}}%
\pgfpathlineto{\pgfqpoint{3.173342in}{2.850530in}}%
\pgfpathlineto{\pgfqpoint{3.173595in}{2.850760in}}%
\pgfpathlineto{\pgfqpoint{3.175532in}{2.851626in}}%
\pgfpathlineto{\pgfqpoint{3.176880in}{2.851035in}}%
\pgfpathlineto{\pgfqpoint{3.179408in}{2.850837in}}%
\pgfpathlineto{\pgfqpoint{3.180166in}{2.850502in}}%
\pgfpathlineto{\pgfqpoint{3.180671in}{2.850868in}}%
\pgfpathlineto{\pgfqpoint{3.183114in}{2.850792in}}%
\pgfpathlineto{\pgfqpoint{3.188085in}{2.847016in}}%
\pgfpathlineto{\pgfqpoint{3.192297in}{2.845803in}}%
\pgfpathlineto{\pgfqpoint{3.194319in}{2.846207in}}%
\pgfpathlineto{\pgfqpoint{3.195751in}{2.847305in}}%
\pgfpathlineto{\pgfqpoint{3.196509in}{2.847681in}}%
\pgfpathlineto{\pgfqpoint{3.196931in}{2.847130in}}%
\pgfpathlineto{\pgfqpoint{3.199037in}{2.845629in}}%
\pgfpathlineto{\pgfqpoint{3.205355in}{2.846237in}}%
\pgfpathlineto{\pgfqpoint{3.206029in}{2.845618in}}%
\pgfpathlineto{\pgfqpoint{3.206534in}{2.846276in}}%
\pgfpathlineto{\pgfqpoint{3.207967in}{2.847399in}}%
\pgfpathlineto{\pgfqpoint{3.208219in}{2.847145in}}%
\pgfpathlineto{\pgfqpoint{3.213611in}{2.839601in}}%
\pgfpathlineto{\pgfqpoint{3.219255in}{2.843684in}}%
\pgfpathlineto{\pgfqpoint{3.221867in}{2.847257in}}%
\pgfpathlineto{\pgfqpoint{3.223889in}{2.848434in}}%
\pgfpathlineto{\pgfqpoint{3.225405in}{2.853488in}}%
\pgfpathlineto{\pgfqpoint{3.228185in}{2.860321in}}%
\pgfpathlineto{\pgfqpoint{3.229028in}{2.862070in}}%
\pgfpathlineto{\pgfqpoint{3.230460in}{2.875777in}}%
\pgfpathlineto{\pgfqpoint{3.232313in}{2.882180in}}%
\pgfpathlineto{\pgfqpoint{3.234082in}{2.890575in}}%
\pgfpathlineto{\pgfqpoint{3.235178in}{2.894907in}}%
\pgfpathlineto{\pgfqpoint{3.235767in}{2.894419in}}%
\pgfpathlineto{\pgfqpoint{3.236525in}{2.894364in}}%
\pgfpathlineto{\pgfqpoint{3.236778in}{2.895045in}}%
\pgfpathlineto{\pgfqpoint{3.240148in}{2.905270in}}%
\pgfpathlineto{\pgfqpoint{3.241664in}{2.907269in}}%
\pgfpathlineto{\pgfqpoint{3.242170in}{2.905778in}}%
\pgfpathlineto{\pgfqpoint{3.244866in}{2.899416in}}%
\pgfpathlineto{\pgfqpoint{3.245708in}{2.898593in}}%
\pgfpathlineto{\pgfqpoint{3.246719in}{2.890038in}}%
\pgfpathlineto{\pgfqpoint{3.248488in}{2.879647in}}%
\pgfpathlineto{\pgfqpoint{3.248657in}{2.879666in}}%
\pgfpathlineto{\pgfqpoint{3.249246in}{2.878542in}}%
\pgfpathlineto{\pgfqpoint{3.255396in}{2.850776in}}%
\pgfpathlineto{\pgfqpoint{3.256239in}{2.851133in}}%
\pgfpathlineto{\pgfqpoint{3.257165in}{2.849577in}}%
\pgfpathlineto{\pgfqpoint{3.260367in}{2.843271in}}%
\pgfpathlineto{\pgfqpoint{3.261209in}{2.844273in}}%
\pgfpathlineto{\pgfqpoint{3.266264in}{2.865050in}}%
\pgfpathlineto{\pgfqpoint{3.267949in}{2.876032in}}%
\pgfpathlineto{\pgfqpoint{3.268286in}{2.875743in}}%
\pgfpathlineto{\pgfqpoint{3.269128in}{2.874699in}}%
\pgfpathlineto{\pgfqpoint{3.269634in}{2.875580in}}%
\pgfpathlineto{\pgfqpoint{3.271740in}{2.882594in}}%
\pgfpathlineto{\pgfqpoint{3.272666in}{2.881185in}}%
\pgfpathlineto{\pgfqpoint{3.273425in}{2.881082in}}%
\pgfpathlineto{\pgfqpoint{3.273762in}{2.881490in}}%
\pgfpathlineto{\pgfqpoint{3.275699in}{2.882465in}}%
\pgfpathlineto{\pgfqpoint{3.278142in}{2.881930in}}%
\pgfpathlineto{\pgfqpoint{3.282860in}{2.862669in}}%
\pgfpathlineto{\pgfqpoint{3.285219in}{2.847722in}}%
\pgfpathlineto{\pgfqpoint{3.286398in}{2.846508in}}%
\pgfpathlineto{\pgfqpoint{3.288252in}{2.841837in}}%
\pgfpathlineto{\pgfqpoint{3.288673in}{2.842886in}}%
\pgfpathlineto{\pgfqpoint{3.290526in}{2.847307in}}%
\pgfpathlineto{\pgfqpoint{3.290695in}{2.847221in}}%
\pgfpathlineto{\pgfqpoint{3.291200in}{2.847376in}}%
\pgfpathlineto{\pgfqpoint{3.291369in}{2.847821in}}%
\pgfpathlineto{\pgfqpoint{3.292801in}{2.856815in}}%
\pgfpathlineto{\pgfqpoint{3.295412in}{2.876951in}}%
\pgfpathlineto{\pgfqpoint{3.296592in}{2.879439in}}%
\pgfpathlineto{\pgfqpoint{3.300973in}{2.894778in}}%
\pgfpathlineto{\pgfqpoint{3.304427in}{2.895199in}}%
\pgfpathlineto{\pgfqpoint{3.305690in}{2.890064in}}%
\pgfpathlineto{\pgfqpoint{3.307544in}{2.885760in}}%
\pgfpathlineto{\pgfqpoint{3.308386in}{2.883544in}}%
\pgfpathlineto{\pgfqpoint{3.309818in}{2.872077in}}%
\pgfpathlineto{\pgfqpoint{3.312093in}{2.854429in}}%
\pgfpathlineto{\pgfqpoint{3.313356in}{2.852687in}}%
\pgfpathlineto{\pgfqpoint{3.315884in}{2.839286in}}%
\pgfpathlineto{\pgfqpoint{3.316895in}{2.841217in}}%
\pgfpathlineto{\pgfqpoint{3.319001in}{2.841658in}}%
\pgfpathlineto{\pgfqpoint{3.320433in}{2.841857in}}%
\pgfpathlineto{\pgfqpoint{3.326667in}{2.845563in}}%
\pgfpathlineto{\pgfqpoint{3.328183in}{2.846853in}}%
\pgfpathlineto{\pgfqpoint{3.331132in}{2.847747in}}%
\pgfpathlineto{\pgfqpoint{3.332564in}{2.848483in}}%
\pgfpathlineto{\pgfqpoint{3.332817in}{2.848121in}}%
\pgfpathlineto{\pgfqpoint{3.335429in}{2.845101in}}%
\pgfpathlineto{\pgfqpoint{3.344106in}{2.842947in}}%
\pgfpathlineto{\pgfqpoint{3.345117in}{2.842767in}}%
\pgfpathlineto{\pgfqpoint{3.345285in}{2.842998in}}%
\pgfpathlineto{\pgfqpoint{3.346380in}{2.847717in}}%
\pgfpathlineto{\pgfqpoint{3.347475in}{2.851619in}}%
\pgfpathlineto{\pgfqpoint{3.348065in}{2.850851in}}%
\pgfpathlineto{\pgfqpoint{3.348486in}{2.850751in}}%
\pgfpathlineto{\pgfqpoint{3.348908in}{2.851507in}}%
\pgfpathlineto{\pgfqpoint{3.349750in}{2.856289in}}%
\pgfpathlineto{\pgfqpoint{3.351940in}{2.868228in}}%
\pgfpathlineto{\pgfqpoint{3.352614in}{2.869379in}}%
\pgfpathlineto{\pgfqpoint{3.354047in}{2.881169in}}%
\pgfpathlineto{\pgfqpoint{3.356490in}{2.894073in}}%
\pgfpathlineto{\pgfqpoint{3.357753in}{2.899393in}}%
\pgfpathlineto{\pgfqpoint{3.359691in}{2.913709in}}%
\pgfpathlineto{\pgfqpoint{3.360281in}{2.913194in}}%
\pgfpathlineto{\pgfqpoint{3.360702in}{2.913049in}}%
\pgfpathlineto{\pgfqpoint{3.361039in}{2.913714in}}%
\pgfpathlineto{\pgfqpoint{3.361966in}{2.920758in}}%
\pgfpathlineto{\pgfqpoint{3.363061in}{2.928650in}}%
\pgfpathlineto{\pgfqpoint{3.363650in}{2.927296in}}%
\pgfpathlineto{\pgfqpoint{3.364324in}{2.925747in}}%
\pgfpathlineto{\pgfqpoint{3.364746in}{2.926861in}}%
\pgfpathlineto{\pgfqpoint{3.366431in}{2.936762in}}%
\pgfpathlineto{\pgfqpoint{3.367020in}{2.934059in}}%
\pgfpathlineto{\pgfqpoint{3.368031in}{2.929501in}}%
\pgfpathlineto{\pgfqpoint{3.368537in}{2.931012in}}%
\pgfpathlineto{\pgfqpoint{3.369463in}{2.935872in}}%
\pgfpathlineto{\pgfqpoint{3.370053in}{2.933719in}}%
\pgfpathlineto{\pgfqpoint{3.373423in}{2.918513in}}%
\pgfpathlineto{\pgfqpoint{3.375108in}{2.910460in}}%
\pgfpathlineto{\pgfqpoint{3.376877in}{2.902899in}}%
\pgfpathlineto{\pgfqpoint{3.377045in}{2.903090in}}%
\pgfpathlineto{\pgfqpoint{3.377719in}{2.907413in}}%
\pgfpathlineto{\pgfqpoint{3.378562in}{2.913630in}}%
\pgfpathlineto{\pgfqpoint{3.379151in}{2.911235in}}%
\pgfpathlineto{\pgfqpoint{3.380162in}{2.906448in}}%
\pgfpathlineto{\pgfqpoint{3.380668in}{2.908419in}}%
\pgfpathlineto{\pgfqpoint{3.387576in}{2.958149in}}%
\pgfpathlineto{\pgfqpoint{3.389092in}{2.966355in}}%
\pgfpathlineto{\pgfqpoint{3.393810in}{2.992479in}}%
\pgfpathlineto{\pgfqpoint{3.394484in}{2.989255in}}%
\pgfpathlineto{\pgfqpoint{3.396000in}{2.977602in}}%
\pgfpathlineto{\pgfqpoint{3.396590in}{2.978630in}}%
\pgfpathlineto{\pgfqpoint{3.396927in}{2.979107in}}%
\pgfpathlineto{\pgfqpoint{3.397348in}{2.977856in}}%
\pgfpathlineto{\pgfqpoint{3.398528in}{2.962364in}}%
\pgfpathlineto{\pgfqpoint{3.400887in}{2.938266in}}%
\pgfpathlineto{\pgfqpoint{3.401560in}{2.934352in}}%
\pgfpathlineto{\pgfqpoint{3.406615in}{2.895735in}}%
\pgfpathlineto{\pgfqpoint{3.407289in}{2.894435in}}%
\pgfpathlineto{\pgfqpoint{3.407795in}{2.895469in}}%
\pgfpathlineto{\pgfqpoint{3.408553in}{2.896775in}}%
\pgfpathlineto{\pgfqpoint{3.408974in}{2.895722in}}%
\pgfpathlineto{\pgfqpoint{3.410153in}{2.886856in}}%
\pgfpathlineto{\pgfqpoint{3.410659in}{2.890738in}}%
\pgfpathlineto{\pgfqpoint{3.417061in}{2.974674in}}%
\pgfpathlineto{\pgfqpoint{3.417567in}{2.974573in}}%
\pgfpathlineto{\pgfqpoint{3.418325in}{2.976027in}}%
\pgfpathlineto{\pgfqpoint{3.421105in}{2.981004in}}%
\pgfpathlineto{\pgfqpoint{3.422285in}{2.980259in}}%
\pgfpathlineto{\pgfqpoint{3.424054in}{2.979162in}}%
\pgfpathlineto{\pgfqpoint{3.424728in}{2.977085in}}%
\pgfpathlineto{\pgfqpoint{3.428350in}{2.949974in}}%
\pgfpathlineto{\pgfqpoint{3.429867in}{2.918891in}}%
\pgfpathlineto{\pgfqpoint{3.430456in}{2.922906in}}%
\pgfpathlineto{\pgfqpoint{3.431383in}{2.930183in}}%
\pgfpathlineto{\pgfqpoint{3.431889in}{2.926856in}}%
\pgfpathlineto{\pgfqpoint{3.433405in}{2.901103in}}%
\pgfpathlineto{\pgfqpoint{3.434079in}{2.908592in}}%
\pgfpathlineto{\pgfqpoint{3.436522in}{2.955668in}}%
\pgfpathlineto{\pgfqpoint{3.437196in}{2.955497in}}%
\pgfpathlineto{\pgfqpoint{3.437870in}{2.960969in}}%
\pgfpathlineto{\pgfqpoint{3.441492in}{2.999503in}}%
\pgfpathlineto{\pgfqpoint{3.443093in}{3.009338in}}%
\pgfpathlineto{\pgfqpoint{3.443598in}{3.008487in}}%
\pgfpathlineto{\pgfqpoint{3.444357in}{3.007323in}}%
\pgfpathlineto{\pgfqpoint{3.444778in}{3.008264in}}%
\pgfpathlineto{\pgfqpoint{3.446379in}{3.016525in}}%
\pgfpathlineto{\pgfqpoint{3.447221in}{3.015095in}}%
\pgfpathlineto{\pgfqpoint{3.448906in}{3.010917in}}%
\pgfpathlineto{\pgfqpoint{3.449748in}{3.012107in}}%
\pgfpathlineto{\pgfqpoint{3.450675in}{3.017440in}}%
\pgfpathlineto{\pgfqpoint{3.452276in}{3.029724in}}%
\pgfpathlineto{\pgfqpoint{3.452781in}{3.026939in}}%
\pgfpathlineto{\pgfqpoint{3.453371in}{3.024129in}}%
\pgfpathlineto{\pgfqpoint{3.453792in}{3.026562in}}%
\pgfpathlineto{\pgfqpoint{3.455056in}{3.054088in}}%
\pgfpathlineto{\pgfqpoint{3.456741in}{3.070938in}}%
\pgfpathlineto{\pgfqpoint{3.457330in}{3.072375in}}%
\pgfpathlineto{\pgfqpoint{3.458426in}{3.085598in}}%
\pgfpathlineto{\pgfqpoint{3.460616in}{3.102822in}}%
\pgfpathlineto{\pgfqpoint{3.461458in}{3.107570in}}%
\pgfpathlineto{\pgfqpoint{3.463564in}{3.123643in}}%
\pgfpathlineto{\pgfqpoint{3.464070in}{3.122057in}}%
\pgfpathlineto{\pgfqpoint{3.465081in}{3.118742in}}%
\pgfpathlineto{\pgfqpoint{3.465586in}{3.119413in}}%
\pgfpathlineto{\pgfqpoint{3.466850in}{3.123945in}}%
\pgfpathlineto{\pgfqpoint{3.467271in}{3.121327in}}%
\pgfpathlineto{\pgfqpoint{3.468451in}{3.091310in}}%
\pgfpathlineto{\pgfqpoint{3.470388in}{3.062490in}}%
\pgfpathlineto{\pgfqpoint{3.471062in}{3.054300in}}%
\pgfpathlineto{\pgfqpoint{3.472579in}{2.994472in}}%
\pgfpathlineto{\pgfqpoint{3.473421in}{2.980234in}}%
\pgfpathlineto{\pgfqpoint{3.473927in}{2.984031in}}%
\pgfpathlineto{\pgfqpoint{3.474685in}{2.990187in}}%
\pgfpathlineto{\pgfqpoint{3.475106in}{2.986109in}}%
\pgfpathlineto{\pgfqpoint{3.476622in}{2.958497in}}%
\pgfpathlineto{\pgfqpoint{3.477212in}{2.965268in}}%
\pgfpathlineto{\pgfqpoint{3.478223in}{2.981012in}}%
\pgfpathlineto{\pgfqpoint{3.478813in}{2.975910in}}%
\pgfpathlineto{\pgfqpoint{3.479908in}{2.960972in}}%
\pgfpathlineto{\pgfqpoint{3.480413in}{2.967388in}}%
\pgfpathlineto{\pgfqpoint{3.482351in}{3.013701in}}%
\pgfpathlineto{\pgfqpoint{3.483109in}{3.009465in}}%
\pgfpathlineto{\pgfqpoint{3.483615in}{3.007463in}}%
\pgfpathlineto{\pgfqpoint{3.484036in}{3.009683in}}%
\pgfpathlineto{\pgfqpoint{3.487321in}{3.042784in}}%
\pgfpathlineto{\pgfqpoint{3.488080in}{3.042430in}}%
\pgfpathlineto{\pgfqpoint{3.490438in}{3.039208in}}%
\pgfpathlineto{\pgfqpoint{3.495072in}{3.004239in}}%
\pgfpathlineto{\pgfqpoint{3.498020in}{2.963654in}}%
\pgfpathlineto{\pgfqpoint{3.500127in}{2.923980in}}%
\pgfpathlineto{\pgfqpoint{3.500801in}{2.920375in}}%
\pgfpathlineto{\pgfqpoint{3.503749in}{2.879649in}}%
\pgfpathlineto{\pgfqpoint{3.505265in}{2.881221in}}%
\pgfpathlineto{\pgfqpoint{3.506866in}{2.883188in}}%
\pgfpathlineto{\pgfqpoint{3.507287in}{2.882141in}}%
\pgfpathlineto{\pgfqpoint{3.508214in}{2.879106in}}%
\pgfpathlineto{\pgfqpoint{3.508720in}{2.880896in}}%
\pgfpathlineto{\pgfqpoint{3.510320in}{2.893286in}}%
\pgfpathlineto{\pgfqpoint{3.510994in}{2.889930in}}%
\pgfpathlineto{\pgfqpoint{3.512595in}{2.879207in}}%
\pgfpathlineto{\pgfqpoint{3.513100in}{2.880067in}}%
\pgfpathlineto{\pgfqpoint{3.514195in}{2.882742in}}%
\pgfpathlineto{\pgfqpoint{3.514701in}{2.881813in}}%
\pgfpathlineto{\pgfqpoint{3.517649in}{2.875850in}}%
\pgfpathlineto{\pgfqpoint{3.519924in}{2.875818in}}%
\pgfpathlineto{\pgfqpoint{3.520008in}{2.876081in}}%
\pgfpathlineto{\pgfqpoint{3.524557in}{2.896574in}}%
\pgfpathlineto{\pgfqpoint{3.525905in}{2.909147in}}%
\pgfpathlineto{\pgfqpoint{3.526327in}{2.907435in}}%
\pgfpathlineto{\pgfqpoint{3.527253in}{2.901766in}}%
\pgfpathlineto{\pgfqpoint{3.527675in}{2.904500in}}%
\pgfpathlineto{\pgfqpoint{3.529696in}{2.931404in}}%
\pgfpathlineto{\pgfqpoint{3.530792in}{2.931267in}}%
\pgfpathlineto{\pgfqpoint{3.531803in}{2.943366in}}%
\pgfpathlineto{\pgfqpoint{3.533740in}{2.956776in}}%
\pgfpathlineto{\pgfqpoint{3.534667in}{2.969874in}}%
\pgfpathlineto{\pgfqpoint{3.536604in}{2.997051in}}%
\pgfpathlineto{\pgfqpoint{3.536941in}{2.996845in}}%
\pgfpathlineto{\pgfqpoint{3.537278in}{2.996855in}}%
\pgfpathlineto{\pgfqpoint{3.537531in}{2.997593in}}%
\pgfpathlineto{\pgfqpoint{3.538711in}{3.009130in}}%
\pgfpathlineto{\pgfqpoint{3.540058in}{3.017869in}}%
\pgfpathlineto{\pgfqpoint{3.540480in}{3.017121in}}%
\pgfpathlineto{\pgfqpoint{3.541996in}{3.012030in}}%
\pgfpathlineto{\pgfqpoint{3.542923in}{3.012643in}}%
\pgfpathlineto{\pgfqpoint{3.543765in}{3.012972in}}%
\pgfpathlineto{\pgfqpoint{3.544018in}{3.012436in}}%
\pgfpathlineto{\pgfqpoint{3.545029in}{3.004517in}}%
\pgfpathlineto{\pgfqpoint{3.546293in}{2.998258in}}%
\pgfpathlineto{\pgfqpoint{3.546714in}{2.998524in}}%
\pgfpathlineto{\pgfqpoint{3.547135in}{2.998326in}}%
\pgfpathlineto{\pgfqpoint{3.547304in}{2.997845in}}%
\pgfpathlineto{\pgfqpoint{3.548314in}{2.988993in}}%
\pgfpathlineto{\pgfqpoint{3.550505in}{2.974587in}}%
\pgfpathlineto{\pgfqpoint{3.551347in}{2.966418in}}%
\pgfpathlineto{\pgfqpoint{3.554127in}{2.939134in}}%
\pgfpathlineto{\pgfqpoint{3.555223in}{2.932303in}}%
\pgfpathlineto{\pgfqpoint{3.556907in}{2.921740in}}%
\pgfpathlineto{\pgfqpoint{3.557497in}{2.923232in}}%
\pgfpathlineto{\pgfqpoint{3.558677in}{2.928406in}}%
\pgfpathlineto{\pgfqpoint{3.559182in}{2.926280in}}%
\pgfpathlineto{\pgfqpoint{3.560361in}{2.920286in}}%
\pgfpathlineto{\pgfqpoint{3.560867in}{2.921993in}}%
\pgfpathlineto{\pgfqpoint{3.562299in}{2.930007in}}%
\pgfpathlineto{\pgfqpoint{3.562889in}{2.928546in}}%
\pgfpathlineto{\pgfqpoint{3.563731in}{2.926435in}}%
\pgfpathlineto{\pgfqpoint{3.564152in}{2.927376in}}%
\pgfpathlineto{\pgfqpoint{3.570302in}{2.952538in}}%
\pgfpathlineto{\pgfqpoint{3.571229in}{2.951216in}}%
\pgfpathlineto{\pgfqpoint{3.574851in}{2.938076in}}%
\pgfpathlineto{\pgfqpoint{3.577632in}{2.920356in}}%
\pgfpathlineto{\pgfqpoint{3.578642in}{2.911284in}}%
\pgfpathlineto{\pgfqpoint{3.580664in}{2.899633in}}%
\pgfpathlineto{\pgfqpoint{3.581254in}{2.897391in}}%
\pgfpathlineto{\pgfqpoint{3.583276in}{2.881031in}}%
\pgfpathlineto{\pgfqpoint{3.584118in}{2.883577in}}%
\pgfpathlineto{\pgfqpoint{3.584455in}{2.883966in}}%
\pgfpathlineto{\pgfqpoint{3.584792in}{2.882905in}}%
\pgfpathlineto{\pgfqpoint{3.586477in}{2.869970in}}%
\pgfpathlineto{\pgfqpoint{3.587320in}{2.873887in}}%
\pgfpathlineto{\pgfqpoint{3.588415in}{2.878898in}}%
\pgfpathlineto{\pgfqpoint{3.588920in}{2.877682in}}%
\pgfpathlineto{\pgfqpoint{3.590015in}{2.874453in}}%
\pgfpathlineto{\pgfqpoint{3.590521in}{2.875682in}}%
\pgfpathlineto{\pgfqpoint{3.591785in}{2.879913in}}%
\pgfpathlineto{\pgfqpoint{3.592374in}{2.879382in}}%
\pgfpathlineto{\pgfqpoint{3.593554in}{2.879050in}}%
\pgfpathlineto{\pgfqpoint{3.593806in}{2.879460in}}%
\pgfpathlineto{\pgfqpoint{3.594986in}{2.884774in}}%
\pgfpathlineto{\pgfqpoint{3.597597in}{2.895438in}}%
\pgfpathlineto{\pgfqpoint{3.598440in}{2.899368in}}%
\pgfpathlineto{\pgfqpoint{3.600125in}{2.911473in}}%
\pgfpathlineto{\pgfqpoint{3.600715in}{2.909909in}}%
\pgfpathlineto{\pgfqpoint{3.601810in}{2.906472in}}%
\pgfpathlineto{\pgfqpoint{3.602315in}{2.907257in}}%
\pgfpathlineto{\pgfqpoint{3.603158in}{2.908764in}}%
\pgfpathlineto{\pgfqpoint{3.603579in}{2.907733in}}%
\pgfpathlineto{\pgfqpoint{3.610150in}{2.881892in}}%
\pgfpathlineto{\pgfqpoint{3.610908in}{2.880311in}}%
\pgfpathlineto{\pgfqpoint{3.612256in}{2.869333in}}%
\pgfpathlineto{\pgfqpoint{3.613604in}{2.861573in}}%
\pgfpathlineto{\pgfqpoint{3.614025in}{2.861993in}}%
\pgfpathlineto{\pgfqpoint{3.614531in}{2.862321in}}%
\pgfpathlineto{\pgfqpoint{3.614952in}{2.861514in}}%
\pgfpathlineto{\pgfqpoint{3.616216in}{2.853351in}}%
\pgfpathlineto{\pgfqpoint{3.617648in}{2.848900in}}%
\pgfpathlineto{\pgfqpoint{3.617900in}{2.849080in}}%
\pgfpathlineto{\pgfqpoint{3.619333in}{2.852671in}}%
\pgfpathlineto{\pgfqpoint{3.622281in}{2.858452in}}%
\pgfpathlineto{\pgfqpoint{3.623039in}{2.859621in}}%
\pgfpathlineto{\pgfqpoint{3.626746in}{2.868460in}}%
\pgfpathlineto{\pgfqpoint{3.630200in}{2.873068in}}%
\pgfpathlineto{\pgfqpoint{3.630874in}{2.871989in}}%
\pgfpathlineto{\pgfqpoint{3.636603in}{2.856227in}}%
\pgfpathlineto{\pgfqpoint{3.638035in}{2.851187in}}%
\pgfpathlineto{\pgfqpoint{3.638456in}{2.851351in}}%
\pgfpathlineto{\pgfqpoint{3.639383in}{2.850595in}}%
\pgfpathlineto{\pgfqpoint{3.644859in}{2.839779in}}%
\pgfpathlineto{\pgfqpoint{3.646796in}{2.837911in}}%
\pgfpathlineto{\pgfqpoint{3.647133in}{2.838279in}}%
\pgfpathlineto{\pgfqpoint{3.658506in}{2.853380in}}%
\pgfpathlineto{\pgfqpoint{3.661286in}{2.853239in}}%
\pgfpathlineto{\pgfqpoint{3.665751in}{2.850037in}}%
\pgfpathlineto{\pgfqpoint{3.667099in}{2.850509in}}%
\pgfpathlineto{\pgfqpoint{3.667352in}{2.850028in}}%
\pgfpathlineto{\pgfqpoint{3.669458in}{2.844431in}}%
\pgfpathlineto{\pgfqpoint{3.670385in}{2.845213in}}%
\pgfpathlineto{\pgfqpoint{3.671227in}{2.844394in}}%
\pgfpathlineto{\pgfqpoint{3.673839in}{2.841046in}}%
\pgfpathlineto{\pgfqpoint{3.676198in}{2.840680in}}%
\pgfpathlineto{\pgfqpoint{3.678978in}{2.840552in}}%
\pgfpathlineto{\pgfqpoint{3.681000in}{2.841978in}}%
\pgfpathlineto{\pgfqpoint{3.682684in}{2.842483in}}%
\pgfpathlineto{\pgfqpoint{3.685043in}{2.841981in}}%
\pgfpathlineto{\pgfqpoint{3.686644in}{2.841736in}}%
\pgfpathlineto{\pgfqpoint{3.689761in}{2.839265in}}%
\pgfpathlineto{\pgfqpoint{3.691699in}{2.837909in}}%
\pgfpathlineto{\pgfqpoint{3.693973in}{2.836962in}}%
\pgfpathlineto{\pgfqpoint{3.697090in}{2.836693in}}%
\pgfpathlineto{\pgfqpoint{3.701639in}{2.833896in}}%
\pgfpathlineto{\pgfqpoint{3.704420in}{2.833362in}}%
\pgfpathlineto{\pgfqpoint{3.737612in}{2.833490in}}%
\pgfpathlineto{\pgfqpoint{3.739128in}{2.833585in}}%
\pgfpathlineto{\pgfqpoint{3.741234in}{2.833453in}}%
\pgfpathlineto{\pgfqpoint{3.744436in}{2.834000in}}%
\pgfpathlineto{\pgfqpoint{3.746121in}{2.835394in}}%
\pgfpathlineto{\pgfqpoint{3.748479in}{2.837374in}}%
\pgfpathlineto{\pgfqpoint{3.750333in}{2.838365in}}%
\pgfpathlineto{\pgfqpoint{3.753113in}{2.840842in}}%
\pgfpathlineto{\pgfqpoint{3.763306in}{2.839695in}}%
\pgfpathlineto{\pgfqpoint{3.764823in}{2.839038in}}%
\pgfpathlineto{\pgfqpoint{3.764907in}{2.839088in}}%
\pgfpathlineto{\pgfqpoint{3.767013in}{2.839178in}}%
\pgfpathlineto{\pgfqpoint{3.768108in}{2.838609in}}%
\pgfpathlineto{\pgfqpoint{3.768445in}{2.839133in}}%
\pgfpathlineto{\pgfqpoint{3.770720in}{2.841446in}}%
\pgfpathlineto{\pgfqpoint{3.772405in}{2.842907in}}%
\pgfpathlineto{\pgfqpoint{3.774343in}{2.842893in}}%
\pgfpathlineto{\pgfqpoint{3.779566in}{2.841799in}}%
\pgfpathlineto{\pgfqpoint{3.784873in}{2.839759in}}%
\pgfpathlineto{\pgfqpoint{3.787569in}{2.835544in}}%
\pgfpathlineto{\pgfqpoint{3.789085in}{2.834386in}}%
\pgfpathlineto{\pgfqpoint{3.791023in}{2.834103in}}%
\pgfpathlineto{\pgfqpoint{3.796667in}{2.836659in}}%
\pgfpathlineto{\pgfqpoint{3.799279in}{2.839682in}}%
\pgfpathlineto{\pgfqpoint{3.801385in}{2.840976in}}%
\pgfpathlineto{\pgfqpoint{3.804502in}{2.845639in}}%
\pgfpathlineto{\pgfqpoint{3.806103in}{2.847048in}}%
\pgfpathlineto{\pgfqpoint{3.808462in}{2.848606in}}%
\pgfpathlineto{\pgfqpoint{3.813685in}{2.847278in}}%
\pgfpathlineto{\pgfqpoint{3.816549in}{2.843644in}}%
\pgfpathlineto{\pgfqpoint{3.818318in}{2.842543in}}%
\pgfpathlineto{\pgfqpoint{3.821098in}{2.838598in}}%
\pgfpathlineto{\pgfqpoint{3.822783in}{2.837342in}}%
\pgfpathlineto{\pgfqpoint{3.825647in}{2.835153in}}%
\pgfpathlineto{\pgfqpoint{3.828428in}{2.834395in}}%
\pgfpathlineto{\pgfqpoint{3.830702in}{2.833256in}}%
\pgfpathlineto{\pgfqpoint{3.837442in}{2.834299in}}%
\pgfpathlineto{\pgfqpoint{3.844265in}{2.837508in}}%
\pgfpathlineto{\pgfqpoint{3.848141in}{2.838868in}}%
\pgfpathlineto{\pgfqpoint{3.852606in}{2.840744in}}%
\pgfpathlineto{\pgfqpoint{3.854038in}{2.841746in}}%
\pgfpathlineto{\pgfqpoint{3.854291in}{2.841592in}}%
\pgfpathlineto{\pgfqpoint{3.855554in}{2.842022in}}%
\pgfpathlineto{\pgfqpoint{3.858924in}{2.845166in}}%
\pgfpathlineto{\pgfqpoint{3.860440in}{2.846188in}}%
\pgfpathlineto{\pgfqpoint{3.862041in}{2.846431in}}%
\pgfpathlineto{\pgfqpoint{3.864231in}{2.846047in}}%
\pgfpathlineto{\pgfqpoint{3.867854in}{2.847797in}}%
\pgfpathlineto{\pgfqpoint{3.869118in}{2.847664in}}%
\pgfpathlineto{\pgfqpoint{3.869202in}{2.847542in}}%
\pgfpathlineto{\pgfqpoint{3.870550in}{2.846262in}}%
\pgfpathlineto{\pgfqpoint{3.870971in}{2.846696in}}%
\pgfpathlineto{\pgfqpoint{3.872824in}{2.850219in}}%
\pgfpathlineto{\pgfqpoint{3.873414in}{2.848787in}}%
\pgfpathlineto{\pgfqpoint{3.874678in}{2.845770in}}%
\pgfpathlineto{\pgfqpoint{3.875183in}{2.846407in}}%
\pgfpathlineto{\pgfqpoint{3.876784in}{2.848998in}}%
\pgfpathlineto{\pgfqpoint{3.877205in}{2.848483in}}%
\pgfpathlineto{\pgfqpoint{3.878974in}{2.847643in}}%
\pgfpathlineto{\pgfqpoint{3.884450in}{2.849508in}}%
\pgfpathlineto{\pgfqpoint{3.885630in}{2.848308in}}%
\pgfpathlineto{\pgfqpoint{3.886051in}{2.848811in}}%
\pgfpathlineto{\pgfqpoint{3.887820in}{2.852094in}}%
\pgfpathlineto{\pgfqpoint{3.888410in}{2.851080in}}%
\pgfpathlineto{\pgfqpoint{3.889758in}{2.847565in}}%
\pgfpathlineto{\pgfqpoint{3.890347in}{2.848365in}}%
\pgfpathlineto{\pgfqpoint{3.891695in}{2.850703in}}%
\pgfpathlineto{\pgfqpoint{3.892201in}{2.850105in}}%
\pgfpathlineto{\pgfqpoint{3.893970in}{2.848408in}}%
\pgfpathlineto{\pgfqpoint{3.894222in}{2.848575in}}%
\pgfpathlineto{\pgfqpoint{3.899530in}{2.851378in}}%
\pgfpathlineto{\pgfqpoint{3.902226in}{2.851402in}}%
\pgfpathlineto{\pgfqpoint{3.903068in}{2.850518in}}%
\pgfpathlineto{\pgfqpoint{3.904837in}{2.845378in}}%
\pgfpathlineto{\pgfqpoint{3.905595in}{2.846595in}}%
\pgfpathlineto{\pgfqpoint{3.906606in}{2.848664in}}%
\pgfpathlineto{\pgfqpoint{3.907112in}{2.847634in}}%
\pgfpathlineto{\pgfqpoint{3.908713in}{2.843707in}}%
\pgfpathlineto{\pgfqpoint{3.909218in}{2.844234in}}%
\pgfpathlineto{\pgfqpoint{3.911240in}{2.845359in}}%
\pgfpathlineto{\pgfqpoint{3.915199in}{2.844865in}}%
\pgfpathlineto{\pgfqpoint{3.917727in}{2.846473in}}%
\pgfpathlineto{\pgfqpoint{3.919412in}{2.847269in}}%
\pgfpathlineto{\pgfqpoint{3.920675in}{2.847868in}}%
\pgfpathlineto{\pgfqpoint{3.922107in}{2.852355in}}%
\pgfpathlineto{\pgfqpoint{3.922697in}{2.853909in}}%
\pgfpathlineto{\pgfqpoint{3.923287in}{2.852801in}}%
\pgfpathlineto{\pgfqpoint{3.923877in}{2.852014in}}%
\pgfpathlineto{\pgfqpoint{3.924382in}{2.852809in}}%
\pgfpathlineto{\pgfqpoint{3.926404in}{2.859982in}}%
\pgfpathlineto{\pgfqpoint{3.927499in}{2.858472in}}%
\pgfpathlineto{\pgfqpoint{3.928257in}{2.858822in}}%
\pgfpathlineto{\pgfqpoint{3.928426in}{2.859102in}}%
\pgfpathlineto{\pgfqpoint{3.934828in}{2.868102in}}%
\pgfpathlineto{\pgfqpoint{3.937187in}{2.871430in}}%
\pgfpathlineto{\pgfqpoint{3.937945in}{2.870190in}}%
\pgfpathlineto{\pgfqpoint{3.938956in}{2.864795in}}%
\pgfpathlineto{\pgfqpoint{3.940725in}{2.859361in}}%
\pgfpathlineto{\pgfqpoint{3.941652in}{2.856934in}}%
\pgfpathlineto{\pgfqpoint{3.944853in}{2.848090in}}%
\pgfpathlineto{\pgfqpoint{3.947886in}{2.845269in}}%
\pgfpathlineto{\pgfqpoint{3.949908in}{2.843647in}}%
\pgfpathlineto{\pgfqpoint{3.950835in}{2.842627in}}%
\pgfpathlineto{\pgfqpoint{3.953783in}{2.836921in}}%
\pgfpathlineto{\pgfqpoint{3.959091in}{2.838382in}}%
\pgfpathlineto{\pgfqpoint{3.961618in}{2.840710in}}%
\pgfpathlineto{\pgfqpoint{3.963303in}{2.840843in}}%
\pgfpathlineto{\pgfqpoint{3.964314in}{2.844517in}}%
\pgfpathlineto{\pgfqpoint{3.966167in}{2.848010in}}%
\pgfpathlineto{\pgfqpoint{3.972064in}{2.851760in}}%
\pgfpathlineto{\pgfqpoint{3.974002in}{2.851301in}}%
\pgfpathlineto{\pgfqpoint{3.975518in}{2.850537in}}%
\pgfpathlineto{\pgfqpoint{3.978720in}{2.846809in}}%
\pgfpathlineto{\pgfqpoint{3.980657in}{2.845967in}}%
\pgfpathlineto{\pgfqpoint{3.983522in}{2.841168in}}%
\pgfpathlineto{\pgfqpoint{3.984533in}{2.842000in}}%
\pgfpathlineto{\pgfqpoint{3.986386in}{2.851658in}}%
\pgfpathlineto{\pgfqpoint{3.988492in}{2.858849in}}%
\pgfpathlineto{\pgfqpoint{3.989840in}{2.871677in}}%
\pgfpathlineto{\pgfqpoint{3.991693in}{2.880516in}}%
\pgfpathlineto{\pgfqpoint{3.993210in}{2.887188in}}%
\pgfpathlineto{\pgfqpoint{3.994979in}{2.889235in}}%
\pgfpathlineto{\pgfqpoint{3.998770in}{2.891364in}}%
\pgfpathlineto{\pgfqpoint{3.999865in}{2.891409in}}%
\pgfpathlineto{\pgfqpoint{4.000034in}{2.891872in}}%
\pgfpathlineto{\pgfqpoint{4.000960in}{2.899097in}}%
\pgfpathlineto{\pgfqpoint{4.002055in}{2.908291in}}%
\pgfpathlineto{\pgfqpoint{4.002645in}{2.906577in}}%
\pgfpathlineto{\pgfqpoint{4.003993in}{2.898440in}}%
\pgfpathlineto{\pgfqpoint{4.004583in}{2.900612in}}%
\pgfpathlineto{\pgfqpoint{4.005678in}{2.905018in}}%
\pgfpathlineto{\pgfqpoint{4.006099in}{2.903375in}}%
\pgfpathlineto{\pgfqpoint{4.008964in}{2.889876in}}%
\pgfpathlineto{\pgfqpoint{4.010648in}{2.884480in}}%
\pgfpathlineto{\pgfqpoint{4.011744in}{2.885244in}}%
\pgfpathlineto{\pgfqpoint{4.013428in}{2.887261in}}%
\pgfpathlineto{\pgfqpoint{4.014355in}{2.889413in}}%
\pgfpathlineto{\pgfqpoint{4.014861in}{2.888628in}}%
\pgfpathlineto{\pgfqpoint{4.016630in}{2.887450in}}%
\pgfpathlineto{\pgfqpoint{4.017304in}{2.886440in}}%
\pgfpathlineto{\pgfqpoint{4.018230in}{2.877498in}}%
\pgfpathlineto{\pgfqpoint{4.019915in}{2.869807in}}%
\pgfpathlineto{\pgfqpoint{4.021011in}{2.868397in}}%
\pgfpathlineto{\pgfqpoint{4.024549in}{2.861343in}}%
\pgfpathlineto{\pgfqpoint{4.024802in}{2.861497in}}%
\pgfpathlineto{\pgfqpoint{4.025812in}{2.864153in}}%
\pgfpathlineto{\pgfqpoint{4.028087in}{2.868552in}}%
\pgfpathlineto{\pgfqpoint{4.030109in}{2.868338in}}%
\pgfpathlineto{\pgfqpoint{4.030193in}{2.868036in}}%
\pgfpathlineto{\pgfqpoint{4.032468in}{2.863542in}}%
\pgfpathlineto{\pgfqpoint{4.034068in}{2.862411in}}%
\pgfpathlineto{\pgfqpoint{4.036259in}{2.857838in}}%
\pgfpathlineto{\pgfqpoint{4.036511in}{2.857956in}}%
\pgfpathlineto{\pgfqpoint{4.037438in}{2.858133in}}%
\pgfpathlineto{\pgfqpoint{4.037691in}{2.857574in}}%
\pgfpathlineto{\pgfqpoint{4.040976in}{2.851231in}}%
\pgfpathlineto{\pgfqpoint{4.042156in}{2.850644in}}%
\pgfpathlineto{\pgfqpoint{4.044178in}{2.844649in}}%
\pgfpathlineto{\pgfqpoint{4.044936in}{2.846373in}}%
\pgfpathlineto{\pgfqpoint{4.045526in}{2.847080in}}%
\pgfpathlineto{\pgfqpoint{4.046200in}{2.846580in}}%
\pgfpathlineto{\pgfqpoint{4.046874in}{2.847482in}}%
\pgfpathlineto{\pgfqpoint{4.047800in}{2.853489in}}%
\pgfpathlineto{\pgfqpoint{4.049654in}{2.863516in}}%
\pgfpathlineto{\pgfqpoint{4.049822in}{2.863453in}}%
\pgfpathlineto{\pgfqpoint{4.050496in}{2.863493in}}%
\pgfpathlineto{\pgfqpoint{4.050665in}{2.863954in}}%
\pgfpathlineto{\pgfqpoint{4.053782in}{2.873587in}}%
\pgfpathlineto{\pgfqpoint{4.054119in}{2.873447in}}%
\pgfpathlineto{\pgfqpoint{4.056309in}{2.871269in}}%
\pgfpathlineto{\pgfqpoint{4.058078in}{2.871169in}}%
\pgfpathlineto{\pgfqpoint{4.061111in}{2.870831in}}%
\pgfpathlineto{\pgfqpoint{4.064818in}{2.859362in}}%
\pgfpathlineto{\pgfqpoint{4.066418in}{2.850173in}}%
\pgfpathlineto{\pgfqpoint{4.066671in}{2.850258in}}%
\pgfpathlineto{\pgfqpoint{4.067345in}{2.849794in}}%
\pgfpathlineto{\pgfqpoint{4.067429in}{2.849524in}}%
\pgfpathlineto{\pgfqpoint{4.070968in}{2.837732in}}%
\pgfpathlineto{\pgfqpoint{4.071389in}{2.837904in}}%
\pgfpathlineto{\pgfqpoint{4.072231in}{2.839762in}}%
\pgfpathlineto{\pgfqpoint{4.074843in}{2.845946in}}%
\pgfpathlineto{\pgfqpoint{4.075854in}{2.846966in}}%
\pgfpathlineto{\pgfqpoint{4.078297in}{2.848808in}}%
\pgfpathlineto{\pgfqpoint{4.080740in}{2.848000in}}%
\pgfpathlineto{\pgfqpoint{4.082341in}{2.847584in}}%
\pgfpathlineto{\pgfqpoint{4.085963in}{2.847036in}}%
\pgfpathlineto{\pgfqpoint{4.087985in}{2.847110in}}%
\pgfpathlineto{\pgfqpoint{4.089586in}{2.849685in}}%
\pgfpathlineto{\pgfqpoint{4.090175in}{2.848890in}}%
\pgfpathlineto{\pgfqpoint{4.091102in}{2.848124in}}%
\pgfpathlineto{\pgfqpoint{4.091439in}{2.848698in}}%
\pgfpathlineto{\pgfqpoint{4.092534in}{2.855065in}}%
\pgfpathlineto{\pgfqpoint{4.094724in}{2.863683in}}%
\pgfpathlineto{\pgfqpoint{4.096157in}{2.864978in}}%
\pgfpathlineto{\pgfqpoint{4.099021in}{2.867934in}}%
\pgfpathlineto{\pgfqpoint{4.103486in}{2.873355in}}%
\pgfpathlineto{\pgfqpoint{4.105255in}{2.878878in}}%
\pgfpathlineto{\pgfqpoint{4.105592in}{2.878310in}}%
\pgfpathlineto{\pgfqpoint{4.109804in}{2.868734in}}%
\pgfpathlineto{\pgfqpoint{4.110394in}{2.868516in}}%
\pgfpathlineto{\pgfqpoint{4.110815in}{2.869058in}}%
\pgfpathlineto{\pgfqpoint{4.113511in}{2.871421in}}%
\pgfpathlineto{\pgfqpoint{4.114269in}{2.873941in}}%
\pgfpathlineto{\pgfqpoint{4.118734in}{2.895415in}}%
\pgfpathlineto{\pgfqpoint{4.119745in}{2.901632in}}%
\pgfpathlineto{\pgfqpoint{4.121346in}{2.909281in}}%
\pgfpathlineto{\pgfqpoint{4.121683in}{2.908943in}}%
\pgfpathlineto{\pgfqpoint{4.122609in}{2.908148in}}%
\pgfpathlineto{\pgfqpoint{4.123031in}{2.908868in}}%
\pgfpathlineto{\pgfqpoint{4.125558in}{2.920034in}}%
\pgfpathlineto{\pgfqpoint{4.126400in}{2.916389in}}%
\pgfpathlineto{\pgfqpoint{4.128843in}{2.906188in}}%
\pgfpathlineto{\pgfqpoint{4.130528in}{2.905691in}}%
\pgfpathlineto{\pgfqpoint{4.131876in}{2.896543in}}%
\pgfpathlineto{\pgfqpoint{4.133898in}{2.881315in}}%
\pgfpathlineto{\pgfqpoint{4.134488in}{2.882554in}}%
\pgfpathlineto{\pgfqpoint{4.135583in}{2.886034in}}%
\pgfpathlineto{\pgfqpoint{4.136089in}{2.884359in}}%
\pgfpathlineto{\pgfqpoint{4.137858in}{2.868439in}}%
\pgfpathlineto{\pgfqpoint{4.138616in}{2.873123in}}%
\pgfpathlineto{\pgfqpoint{4.139964in}{2.885130in}}%
\pgfpathlineto{\pgfqpoint{4.140469in}{2.883114in}}%
\pgfpathlineto{\pgfqpoint{4.142323in}{2.875241in}}%
\pgfpathlineto{\pgfqpoint{4.142660in}{2.875504in}}%
\pgfpathlineto{\pgfqpoint{4.144934in}{2.877838in}}%
\pgfpathlineto{\pgfqpoint{4.145271in}{2.877457in}}%
\pgfpathlineto{\pgfqpoint{4.147209in}{2.875149in}}%
\pgfpathlineto{\pgfqpoint{4.147630in}{2.875559in}}%
\pgfpathlineto{\pgfqpoint{4.150073in}{2.876819in}}%
\pgfpathlineto{\pgfqpoint{4.151674in}{2.876650in}}%
\pgfpathlineto{\pgfqpoint{4.151758in}{2.876391in}}%
\pgfpathlineto{\pgfqpoint{4.153022in}{2.874397in}}%
\pgfpathlineto{\pgfqpoint{4.153443in}{2.874582in}}%
\pgfpathlineto{\pgfqpoint{4.154201in}{2.873769in}}%
\pgfpathlineto{\pgfqpoint{4.155296in}{2.865367in}}%
\pgfpathlineto{\pgfqpoint{4.157234in}{2.852116in}}%
\pgfpathlineto{\pgfqpoint{4.157487in}{2.852223in}}%
\pgfpathlineto{\pgfqpoint{4.158413in}{2.851907in}}%
\pgfpathlineto{\pgfqpoint{4.158582in}{2.851562in}}%
\pgfpathlineto{\pgfqpoint{4.161952in}{2.846129in}}%
\pgfpathlineto{\pgfqpoint{4.168354in}{2.844728in}}%
\pgfpathlineto{\pgfqpoint{4.170797in}{2.843481in}}%
\pgfpathlineto{\pgfqpoint{4.172314in}{2.843439in}}%
\pgfpathlineto{\pgfqpoint{4.172398in}{2.843316in}}%
\pgfpathlineto{\pgfqpoint{4.174757in}{2.839053in}}%
\pgfpathlineto{\pgfqpoint{4.175683in}{2.839875in}}%
\pgfpathlineto{\pgfqpoint{4.177368in}{2.838919in}}%
\pgfpathlineto{\pgfqpoint{4.179306in}{2.839361in}}%
\pgfpathlineto{\pgfqpoint{4.183265in}{2.839696in}}%
\pgfpathlineto{\pgfqpoint{4.185372in}{2.844405in}}%
\pgfpathlineto{\pgfqpoint{4.185624in}{2.844278in}}%
\pgfpathlineto{\pgfqpoint{4.186383in}{2.844301in}}%
\pgfpathlineto{\pgfqpoint{4.186635in}{2.844782in}}%
\pgfpathlineto{\pgfqpoint{4.190342in}{2.850529in}}%
\pgfpathlineto{\pgfqpoint{4.199356in}{2.852034in}}%
\pgfpathlineto{\pgfqpoint{4.200199in}{2.850316in}}%
\pgfpathlineto{\pgfqpoint{4.202557in}{2.844819in}}%
\pgfpathlineto{\pgfqpoint{4.203484in}{2.843924in}}%
\pgfpathlineto{\pgfqpoint{4.206938in}{2.838728in}}%
\pgfpathlineto{\pgfqpoint{4.209802in}{2.837547in}}%
\pgfpathlineto{\pgfqpoint{4.212161in}{2.838000in}}%
\pgfpathlineto{\pgfqpoint{4.216374in}{2.842681in}}%
\pgfpathlineto{\pgfqpoint{4.217553in}{2.844005in}}%
\pgfpathlineto{\pgfqpoint{4.218901in}{2.846796in}}%
\pgfpathlineto{\pgfqpoint{4.219491in}{2.846496in}}%
\pgfpathlineto{\pgfqpoint{4.221934in}{2.846369in}}%
\pgfpathlineto{\pgfqpoint{4.223282in}{2.847963in}}%
\pgfpathlineto{\pgfqpoint{4.225051in}{2.849498in}}%
\pgfpathlineto{\pgfqpoint{4.226399in}{2.850258in}}%
\pgfpathlineto{\pgfqpoint{4.228505in}{2.854293in}}%
\pgfpathlineto{\pgfqpoint{4.228842in}{2.853944in}}%
\pgfpathlineto{\pgfqpoint{4.231790in}{2.849229in}}%
\pgfpathlineto{\pgfqpoint{4.232296in}{2.849585in}}%
\pgfpathlineto{\pgfqpoint{4.233981in}{2.849190in}}%
\pgfpathlineto{\pgfqpoint{4.241815in}{2.842180in}}%
\pgfpathlineto{\pgfqpoint{4.241900in}{2.842264in}}%
\pgfpathlineto{\pgfqpoint{4.243248in}{2.844292in}}%
\pgfpathlineto{\pgfqpoint{4.243922in}{2.842990in}}%
\pgfpathlineto{\pgfqpoint{4.245354in}{2.841226in}}%
\pgfpathlineto{\pgfqpoint{4.245691in}{2.841759in}}%
\pgfpathlineto{\pgfqpoint{4.248386in}{2.848695in}}%
\pgfpathlineto{\pgfqpoint{4.248976in}{2.848294in}}%
\pgfpathlineto{\pgfqpoint{4.250408in}{2.848741in}}%
\pgfpathlineto{\pgfqpoint{4.252936in}{2.851280in}}%
\pgfpathlineto{\pgfqpoint{4.255042in}{2.852475in}}%
\pgfpathlineto{\pgfqpoint{4.257485in}{2.853270in}}%
\pgfpathlineto{\pgfqpoint{4.259675in}{2.851746in}}%
\pgfpathlineto{\pgfqpoint{4.261444in}{2.850066in}}%
\pgfpathlineto{\pgfqpoint{4.262540in}{2.849254in}}%
\pgfpathlineto{\pgfqpoint{4.265994in}{2.841403in}}%
\pgfpathlineto{\pgfqpoint{4.267426in}{2.840459in}}%
\pgfpathlineto{\pgfqpoint{4.268605in}{2.839053in}}%
\pgfpathlineto{\pgfqpoint{4.269026in}{2.839290in}}%
\pgfpathlineto{\pgfqpoint{4.271722in}{2.839787in}}%
\pgfpathlineto{\pgfqpoint{4.275176in}{2.839028in}}%
\pgfpathlineto{\pgfqpoint{4.276777in}{2.840161in}}%
\pgfpathlineto{\pgfqpoint{4.280568in}{2.844605in}}%
\pgfpathlineto{\pgfqpoint{4.281832in}{2.847301in}}%
\pgfpathlineto{\pgfqpoint{4.283938in}{2.850572in}}%
\pgfpathlineto{\pgfqpoint{4.284949in}{2.849443in}}%
\pgfpathlineto{\pgfqpoint{4.286044in}{2.847602in}}%
\pgfpathlineto{\pgfqpoint{4.286549in}{2.848098in}}%
\pgfpathlineto{\pgfqpoint{4.291267in}{2.854199in}}%
\pgfpathlineto{\pgfqpoint{4.292446in}{2.856779in}}%
\pgfpathlineto{\pgfqpoint{4.293542in}{2.860281in}}%
\pgfpathlineto{\pgfqpoint{4.294131in}{2.859514in}}%
\pgfpathlineto{\pgfqpoint{4.296237in}{2.857676in}}%
\pgfpathlineto{\pgfqpoint{4.297333in}{2.856758in}}%
\pgfpathlineto{\pgfqpoint{4.300534in}{2.851225in}}%
\pgfpathlineto{\pgfqpoint{4.301376in}{2.852442in}}%
\pgfpathlineto{\pgfqpoint{4.302640in}{2.853764in}}%
\pgfpathlineto{\pgfqpoint{4.302977in}{2.853437in}}%
\pgfpathlineto{\pgfqpoint{4.306515in}{2.850100in}}%
\pgfpathlineto{\pgfqpoint{4.309295in}{2.849012in}}%
\pgfpathlineto{\pgfqpoint{4.311570in}{2.844094in}}%
\pgfpathlineto{\pgfqpoint{4.315108in}{2.845120in}}%
\pgfpathlineto{\pgfqpoint{4.316793in}{2.850349in}}%
\pgfpathlineto{\pgfqpoint{4.317551in}{2.849800in}}%
\pgfpathlineto{\pgfqpoint{4.318815in}{2.848742in}}%
\pgfpathlineto{\pgfqpoint{4.319236in}{2.849640in}}%
\pgfpathlineto{\pgfqpoint{4.320584in}{2.852871in}}%
\pgfpathlineto{\pgfqpoint{4.321174in}{2.852164in}}%
\pgfpathlineto{\pgfqpoint{4.322774in}{2.850030in}}%
\pgfpathlineto{\pgfqpoint{4.323448in}{2.850676in}}%
\pgfpathlineto{\pgfqpoint{4.324712in}{2.850210in}}%
\pgfpathlineto{\pgfqpoint{4.326818in}{2.849490in}}%
\pgfpathlineto{\pgfqpoint{4.330272in}{2.852190in}}%
\pgfpathlineto{\pgfqpoint{4.331367in}{2.852616in}}%
\pgfpathlineto{\pgfqpoint{4.331620in}{2.852188in}}%
\pgfpathlineto{\pgfqpoint{4.333305in}{2.846665in}}%
\pgfpathlineto{\pgfqpoint{4.334232in}{2.848331in}}%
\pgfpathlineto{\pgfqpoint{4.335495in}{2.849813in}}%
\pgfpathlineto{\pgfqpoint{4.335917in}{2.848990in}}%
\pgfpathlineto{\pgfqpoint{4.337349in}{2.844521in}}%
\pgfpathlineto{\pgfqpoint{4.337938in}{2.846003in}}%
\pgfpathlineto{\pgfqpoint{4.339539in}{2.850233in}}%
\pgfpathlineto{\pgfqpoint{4.339960in}{2.850134in}}%
\pgfpathlineto{\pgfqpoint{4.342319in}{2.851344in}}%
\pgfpathlineto{\pgfqpoint{4.345605in}{2.854358in}}%
\pgfpathlineto{\pgfqpoint{4.346194in}{2.853701in}}%
\pgfpathlineto{\pgfqpoint{4.348385in}{2.852585in}}%
\pgfpathlineto{\pgfqpoint{4.348553in}{2.852764in}}%
\pgfpathlineto{\pgfqpoint{4.349396in}{2.853167in}}%
\pgfpathlineto{\pgfqpoint{4.349733in}{2.852604in}}%
\pgfpathlineto{\pgfqpoint{4.351923in}{2.849365in}}%
\pgfpathlineto{\pgfqpoint{4.352091in}{2.849468in}}%
\pgfpathlineto{\pgfqpoint{4.353692in}{2.852426in}}%
\pgfpathlineto{\pgfqpoint{4.354366in}{2.853370in}}%
\pgfpathlineto{\pgfqpoint{4.354787in}{2.852469in}}%
\pgfpathlineto{\pgfqpoint{4.356641in}{2.848663in}}%
\pgfpathlineto{\pgfqpoint{4.356893in}{2.848843in}}%
\pgfpathlineto{\pgfqpoint{4.359084in}{2.849388in}}%
\pgfpathlineto{\pgfqpoint{4.362791in}{2.848394in}}%
\pgfpathlineto{\pgfqpoint{4.363801in}{2.850763in}}%
\pgfpathlineto{\pgfqpoint{4.366245in}{2.855009in}}%
\pgfpathlineto{\pgfqpoint{4.369362in}{2.854159in}}%
\pgfpathlineto{\pgfqpoint{4.371720in}{2.849938in}}%
\pgfpathlineto{\pgfqpoint{4.372057in}{2.850470in}}%
\pgfpathlineto{\pgfqpoint{4.373237in}{2.852939in}}%
\pgfpathlineto{\pgfqpoint{4.373827in}{2.852217in}}%
\pgfpathlineto{\pgfqpoint{4.374416in}{2.851842in}}%
\pgfpathlineto{\pgfqpoint{4.374838in}{2.852529in}}%
\pgfpathlineto{\pgfqpoint{4.378544in}{2.863659in}}%
\pgfpathlineto{\pgfqpoint{4.379976in}{2.870313in}}%
\pgfpathlineto{\pgfqpoint{4.380566in}{2.869378in}}%
\pgfpathlineto{\pgfqpoint{4.380987in}{2.868965in}}%
\pgfpathlineto{\pgfqpoint{4.381409in}{2.869844in}}%
\pgfpathlineto{\pgfqpoint{4.382504in}{2.877626in}}%
\pgfpathlineto{\pgfqpoint{4.384610in}{2.889356in}}%
\pgfpathlineto{\pgfqpoint{4.385874in}{2.890530in}}%
\pgfpathlineto{\pgfqpoint{4.390507in}{2.899978in}}%
\pgfpathlineto{\pgfqpoint{4.391349in}{2.899281in}}%
\pgfpathlineto{\pgfqpoint{4.396404in}{2.884637in}}%
\pgfpathlineto{\pgfqpoint{4.400785in}{2.860587in}}%
\pgfpathlineto{\pgfqpoint{4.402385in}{2.861444in}}%
\pgfpathlineto{\pgfqpoint{4.403059in}{2.860005in}}%
\pgfpathlineto{\pgfqpoint{4.404155in}{2.856231in}}%
\pgfpathlineto{\pgfqpoint{4.404744in}{2.857592in}}%
\pgfpathlineto{\pgfqpoint{4.406513in}{2.860210in}}%
\pgfpathlineto{\pgfqpoint{4.406682in}{2.860140in}}%
\pgfpathlineto{\pgfqpoint{4.407946in}{2.860326in}}%
\pgfpathlineto{\pgfqpoint{4.408030in}{2.860525in}}%
\pgfpathlineto{\pgfqpoint{4.409967in}{2.868491in}}%
\pgfpathlineto{\pgfqpoint{4.411147in}{2.865908in}}%
\pgfpathlineto{\pgfqpoint{4.411737in}{2.865423in}}%
\pgfpathlineto{\pgfqpoint{4.412242in}{2.866136in}}%
\pgfpathlineto{\pgfqpoint{4.413758in}{2.867070in}}%
\pgfpathlineto{\pgfqpoint{4.413843in}{2.867027in}}%
\pgfpathlineto{\pgfqpoint{4.416033in}{2.864567in}}%
\pgfpathlineto{\pgfqpoint{4.417886in}{2.863489in}}%
\pgfpathlineto{\pgfqpoint{4.420667in}{2.863747in}}%
\pgfpathlineto{\pgfqpoint{4.422604in}{2.858982in}}%
\pgfpathlineto{\pgfqpoint{4.423531in}{2.860488in}}%
\pgfpathlineto{\pgfqpoint{4.424795in}{2.862856in}}%
\pgfpathlineto{\pgfqpoint{4.425300in}{2.861742in}}%
\pgfpathlineto{\pgfqpoint{4.427827in}{2.855685in}}%
\pgfpathlineto{\pgfqpoint{4.429596in}{2.855268in}}%
\pgfpathlineto{\pgfqpoint{4.429765in}{2.855601in}}%
\pgfpathlineto{\pgfqpoint{4.433809in}{2.864740in}}%
\pgfpathlineto{\pgfqpoint{4.434398in}{2.863861in}}%
\pgfpathlineto{\pgfqpoint{4.434988in}{2.863019in}}%
\pgfpathlineto{\pgfqpoint{4.435325in}{2.864030in}}%
\pgfpathlineto{\pgfqpoint{4.436420in}{2.877827in}}%
\pgfpathlineto{\pgfqpoint{4.437937in}{2.888537in}}%
\pgfpathlineto{\pgfqpoint{4.438274in}{2.888295in}}%
\pgfpathlineto{\pgfqpoint{4.438863in}{2.887850in}}%
\pgfpathlineto{\pgfqpoint{4.439200in}{2.888835in}}%
\pgfpathlineto{\pgfqpoint{4.440632in}{2.901189in}}%
\pgfpathlineto{\pgfqpoint{4.442486in}{2.910769in}}%
\pgfpathlineto{\pgfqpoint{4.443918in}{2.911814in}}%
\pgfpathlineto{\pgfqpoint{4.448383in}{2.919946in}}%
\pgfpathlineto{\pgfqpoint{4.448973in}{2.918990in}}%
\pgfpathlineto{\pgfqpoint{4.449899in}{2.915854in}}%
\pgfpathlineto{\pgfqpoint{4.450405in}{2.917556in}}%
\pgfpathlineto{\pgfqpoint{4.452090in}{2.926116in}}%
\pgfpathlineto{\pgfqpoint{4.452511in}{2.924043in}}%
\pgfpathlineto{\pgfqpoint{4.454280in}{2.907455in}}%
\pgfpathlineto{\pgfqpoint{4.455038in}{2.909002in}}%
\pgfpathlineto{\pgfqpoint{4.455460in}{2.909391in}}%
\pgfpathlineto{\pgfqpoint{4.455797in}{2.908466in}}%
\pgfpathlineto{\pgfqpoint{4.457313in}{2.894371in}}%
\pgfpathlineto{\pgfqpoint{4.459251in}{2.883309in}}%
\pgfpathlineto{\pgfqpoint{4.459335in}{2.883336in}}%
\pgfpathlineto{\pgfqpoint{4.461694in}{2.883622in}}%
\pgfpathlineto{\pgfqpoint{4.461778in}{2.883471in}}%
\pgfpathlineto{\pgfqpoint{4.465232in}{2.876669in}}%
\pgfpathlineto{\pgfqpoint{4.465822in}{2.877867in}}%
\pgfpathlineto{\pgfqpoint{4.466664in}{2.879323in}}%
\pgfpathlineto{\pgfqpoint{4.467085in}{2.878666in}}%
\pgfpathlineto{\pgfqpoint{4.469191in}{2.869475in}}%
\pgfpathlineto{\pgfqpoint{4.470287in}{2.872210in}}%
\pgfpathlineto{\pgfqpoint{4.470539in}{2.872158in}}%
\pgfpathlineto{\pgfqpoint{4.470876in}{2.871280in}}%
\pgfpathlineto{\pgfqpoint{4.472730in}{2.868025in}}%
\pgfpathlineto{\pgfqpoint{4.472898in}{2.868108in}}%
\pgfpathlineto{\pgfqpoint{4.474330in}{2.867903in}}%
\pgfpathlineto{\pgfqpoint{4.475510in}{2.865795in}}%
\pgfpathlineto{\pgfqpoint{4.476184in}{2.867243in}}%
\pgfpathlineto{\pgfqpoint{4.477110in}{2.868112in}}%
\pgfpathlineto{\pgfqpoint{4.477532in}{2.867592in}}%
\pgfpathlineto{\pgfqpoint{4.477869in}{2.867314in}}%
\pgfpathlineto{\pgfqpoint{4.478290in}{2.868371in}}%
\pgfpathlineto{\pgfqpoint{4.483429in}{2.892040in}}%
\pgfpathlineto{\pgfqpoint{4.484018in}{2.891677in}}%
\pgfpathlineto{\pgfqpoint{4.484945in}{2.891302in}}%
\pgfpathlineto{\pgfqpoint{4.485198in}{2.891860in}}%
\pgfpathlineto{\pgfqpoint{4.486125in}{2.899278in}}%
\pgfpathlineto{\pgfqpoint{4.488399in}{2.911901in}}%
\pgfpathlineto{\pgfqpoint{4.489831in}{2.913974in}}%
\pgfpathlineto{\pgfqpoint{4.492527in}{2.919453in}}%
\pgfpathlineto{\pgfqpoint{4.494633in}{2.918864in}}%
\pgfpathlineto{\pgfqpoint{4.495476in}{2.914373in}}%
\pgfpathlineto{\pgfqpoint{4.497245in}{2.909773in}}%
\pgfpathlineto{\pgfqpoint{4.497835in}{2.910763in}}%
\pgfpathlineto{\pgfqpoint{4.498340in}{2.912089in}}%
\pgfpathlineto{\pgfqpoint{4.498761in}{2.910624in}}%
\pgfpathlineto{\pgfqpoint{4.500530in}{2.899340in}}%
\pgfpathlineto{\pgfqpoint{4.501036in}{2.900952in}}%
\pgfpathlineto{\pgfqpoint{4.502384in}{2.911245in}}%
\pgfpathlineto{\pgfqpoint{4.502973in}{2.907532in}}%
\pgfpathlineto{\pgfqpoint{4.505417in}{2.888912in}}%
\pgfpathlineto{\pgfqpoint{4.505754in}{2.889173in}}%
\pgfpathlineto{\pgfqpoint{4.506764in}{2.890314in}}%
\pgfpathlineto{\pgfqpoint{4.507101in}{2.889589in}}%
\pgfpathlineto{\pgfqpoint{4.510050in}{2.883191in}}%
\pgfpathlineto{\pgfqpoint{4.510808in}{2.884136in}}%
\pgfpathlineto{\pgfqpoint{4.511735in}{2.891897in}}%
\pgfpathlineto{\pgfqpoint{4.513251in}{2.906227in}}%
\pgfpathlineto{\pgfqpoint{4.513841in}{2.904620in}}%
\pgfpathlineto{\pgfqpoint{4.515273in}{2.900360in}}%
\pgfpathlineto{\pgfqpoint{4.515779in}{2.901164in}}%
\pgfpathlineto{\pgfqpoint{4.517127in}{2.904299in}}%
\pgfpathlineto{\pgfqpoint{4.517548in}{2.903164in}}%
\pgfpathlineto{\pgfqpoint{4.519233in}{2.887519in}}%
\pgfpathlineto{\pgfqpoint{4.520749in}{2.881919in}}%
\pgfpathlineto{\pgfqpoint{4.521086in}{2.882085in}}%
\pgfpathlineto{\pgfqpoint{4.522434in}{2.881515in}}%
\pgfpathlineto{\pgfqpoint{4.523613in}{2.877472in}}%
\pgfpathlineto{\pgfqpoint{4.524287in}{2.876285in}}%
\pgfpathlineto{\pgfqpoint{4.524877in}{2.876830in}}%
\pgfpathlineto{\pgfqpoint{4.526730in}{2.877201in}}%
\pgfpathlineto{\pgfqpoint{4.527573in}{2.876351in}}%
\pgfpathlineto{\pgfqpoint{4.528415in}{2.869937in}}%
\pgfpathlineto{\pgfqpoint{4.530606in}{2.848682in}}%
\pgfpathlineto{\pgfqpoint{4.531195in}{2.848852in}}%
\pgfpathlineto{\pgfqpoint{4.531869in}{2.847698in}}%
\pgfpathlineto{\pgfqpoint{4.534902in}{2.840868in}}%
\pgfpathlineto{\pgfqpoint{4.537008in}{2.840646in}}%
\pgfpathlineto{\pgfqpoint{4.542231in}{2.842471in}}%
\pgfpathlineto{\pgfqpoint{4.544338in}{2.843111in}}%
\pgfpathlineto{\pgfqpoint{4.548381in}{2.844223in}}%
\pgfpathlineto{\pgfqpoint{4.549982in}{2.842676in}}%
\pgfpathlineto{\pgfqpoint{4.551835in}{2.841504in}}%
\pgfpathlineto{\pgfqpoint{4.552004in}{2.841652in}}%
\pgfpathlineto{\pgfqpoint{4.553352in}{2.844780in}}%
\pgfpathlineto{\pgfqpoint{4.556300in}{2.852627in}}%
\pgfpathlineto{\pgfqpoint{4.557143in}{2.853545in}}%
\pgfpathlineto{\pgfqpoint{4.560007in}{2.858931in}}%
\pgfpathlineto{\pgfqpoint{4.560260in}{2.858795in}}%
\pgfpathlineto{\pgfqpoint{4.564472in}{2.857722in}}%
\pgfpathlineto{\pgfqpoint{4.565820in}{2.860289in}}%
\pgfpathlineto{\pgfqpoint{4.567084in}{2.861763in}}%
\pgfpathlineto{\pgfqpoint{4.567505in}{2.861509in}}%
\pgfpathlineto{\pgfqpoint{4.570285in}{2.859031in}}%
\pgfpathlineto{\pgfqpoint{4.575676in}{2.844309in}}%
\pgfpathlineto{\pgfqpoint{4.577361in}{2.844304in}}%
\pgfpathlineto{\pgfqpoint{4.578625in}{2.846451in}}%
\pgfpathlineto{\pgfqpoint{4.581826in}{2.853662in}}%
\pgfpathlineto{\pgfqpoint{4.582585in}{2.854681in}}%
\pgfpathlineto{\pgfqpoint{4.585533in}{2.865623in}}%
\pgfpathlineto{\pgfqpoint{4.586207in}{2.865297in}}%
\pgfpathlineto{\pgfqpoint{4.589830in}{2.863830in}}%
\pgfpathlineto{\pgfqpoint{4.592273in}{2.866523in}}%
\pgfpathlineto{\pgfqpoint{4.593958in}{2.869221in}}%
\pgfpathlineto{\pgfqpoint{4.594379in}{2.868859in}}%
\pgfpathlineto{\pgfqpoint{4.596316in}{2.864386in}}%
\pgfpathlineto{\pgfqpoint{4.597159in}{2.862711in}}%
\pgfpathlineto{\pgfqpoint{4.597664in}{2.863264in}}%
\pgfpathlineto{\pgfqpoint{4.598338in}{2.863629in}}%
\pgfpathlineto{\pgfqpoint{4.598759in}{2.863053in}}%
\pgfpathlineto{\pgfqpoint{4.599770in}{2.858141in}}%
\pgfpathlineto{\pgfqpoint{4.602129in}{2.847805in}}%
\pgfpathlineto{\pgfqpoint{4.603393in}{2.848465in}}%
\pgfpathlineto{\pgfqpoint{4.607268in}{2.850343in}}%
\pgfpathlineto{\pgfqpoint{4.608279in}{2.848851in}}%
\pgfpathlineto{\pgfqpoint{4.611312in}{2.843809in}}%
\pgfpathlineto{\pgfqpoint{4.612323in}{2.842640in}}%
\pgfpathlineto{\pgfqpoint{4.615271in}{2.839278in}}%
\pgfpathlineto{\pgfqpoint{4.618388in}{2.839120in}}%
\pgfpathlineto{\pgfqpoint{4.622264in}{2.839976in}}%
\pgfpathlineto{\pgfqpoint{4.625212in}{2.844910in}}%
\pgfpathlineto{\pgfqpoint{4.626560in}{2.845264in}}%
\pgfpathlineto{\pgfqpoint{4.626644in}{2.845548in}}%
\pgfpathlineto{\pgfqpoint{4.628582in}{2.857703in}}%
\pgfpathlineto{\pgfqpoint{4.630435in}{2.864489in}}%
\pgfpathlineto{\pgfqpoint{4.632289in}{2.865490in}}%
\pgfpathlineto{\pgfqpoint{4.635490in}{2.877595in}}%
\pgfpathlineto{\pgfqpoint{4.635743in}{2.877528in}}%
\pgfpathlineto{\pgfqpoint{4.637006in}{2.878250in}}%
\pgfpathlineto{\pgfqpoint{4.638270in}{2.879381in}}%
\pgfpathlineto{\pgfqpoint{4.638691in}{2.878903in}}%
\pgfpathlineto{\pgfqpoint{4.640039in}{2.874188in}}%
\pgfpathlineto{\pgfqpoint{4.641893in}{2.871605in}}%
\pgfpathlineto{\pgfqpoint{4.643325in}{2.872443in}}%
\pgfpathlineto{\pgfqpoint{4.643662in}{2.871176in}}%
\pgfpathlineto{\pgfqpoint{4.646021in}{2.859531in}}%
\pgfpathlineto{\pgfqpoint{4.646779in}{2.860468in}}%
\pgfpathlineto{\pgfqpoint{4.648716in}{2.865217in}}%
\pgfpathlineto{\pgfqpoint{4.649306in}{2.863377in}}%
\pgfpathlineto{\pgfqpoint{4.651328in}{2.857954in}}%
\pgfpathlineto{\pgfqpoint{4.653013in}{2.858494in}}%
\pgfpathlineto{\pgfqpoint{4.654192in}{2.857551in}}%
\pgfpathlineto{\pgfqpoint{4.656214in}{2.856385in}}%
\pgfpathlineto{\pgfqpoint{4.660089in}{2.855897in}}%
\pgfpathlineto{\pgfqpoint{4.661016in}{2.856900in}}%
\pgfpathlineto{\pgfqpoint{4.662196in}{2.859628in}}%
\pgfpathlineto{\pgfqpoint{4.662701in}{2.858611in}}%
\pgfpathlineto{\pgfqpoint{4.664976in}{2.850930in}}%
\pgfpathlineto{\pgfqpoint{4.665734in}{2.851147in}}%
\pgfpathlineto{\pgfqpoint{4.667503in}{2.849186in}}%
\pgfpathlineto{\pgfqpoint{4.670199in}{2.847920in}}%
\pgfpathlineto{\pgfqpoint{4.671378in}{2.848880in}}%
\pgfpathlineto{\pgfqpoint{4.673316in}{2.851464in}}%
\pgfpathlineto{\pgfqpoint{4.676433in}{2.852040in}}%
\pgfpathlineto{\pgfqpoint{4.677360in}{2.852841in}}%
\pgfpathlineto{\pgfqpoint{4.677781in}{2.852387in}}%
\pgfpathlineto{\pgfqpoint{4.679718in}{2.848315in}}%
\pgfpathlineto{\pgfqpoint{4.680308in}{2.850116in}}%
\pgfpathlineto{\pgfqpoint{4.681656in}{2.856079in}}%
\pgfpathlineto{\pgfqpoint{4.682330in}{2.855345in}}%
\pgfpathlineto{\pgfqpoint{4.683425in}{2.855479in}}%
\pgfpathlineto{\pgfqpoint{4.683509in}{2.855630in}}%
\pgfpathlineto{\pgfqpoint{4.684605in}{2.858346in}}%
\pgfpathlineto{\pgfqpoint{4.685616in}{2.857760in}}%
\pgfpathlineto{\pgfqpoint{4.687132in}{2.858654in}}%
\pgfpathlineto{\pgfqpoint{4.688059in}{2.860168in}}%
\pgfpathlineto{\pgfqpoint{4.688564in}{2.859262in}}%
\pgfpathlineto{\pgfqpoint{4.689070in}{2.858431in}}%
\pgfpathlineto{\pgfqpoint{4.689575in}{2.859421in}}%
\pgfpathlineto{\pgfqpoint{4.692945in}{2.868119in}}%
\pgfpathlineto{\pgfqpoint{4.693619in}{2.869799in}}%
\pgfpathlineto{\pgfqpoint{4.696146in}{2.885559in}}%
\pgfpathlineto{\pgfqpoint{4.697073in}{2.884551in}}%
\pgfpathlineto{\pgfqpoint{4.698084in}{2.882834in}}%
\pgfpathlineto{\pgfqpoint{4.698505in}{2.883933in}}%
\pgfpathlineto{\pgfqpoint{4.700021in}{2.889663in}}%
\pgfpathlineto{\pgfqpoint{4.700611in}{2.888395in}}%
\pgfpathlineto{\pgfqpoint{4.702633in}{2.884525in}}%
\pgfpathlineto{\pgfqpoint{4.703981in}{2.883429in}}%
\pgfpathlineto{\pgfqpoint{4.710973in}{2.869572in}}%
\pgfpathlineto{\pgfqpoint{4.712911in}{2.856572in}}%
\pgfpathlineto{\pgfqpoint{4.713248in}{2.857041in}}%
\pgfpathlineto{\pgfqpoint{4.714511in}{2.859666in}}%
\pgfpathlineto{\pgfqpoint{4.715101in}{2.859013in}}%
\pgfpathlineto{\pgfqpoint{4.717965in}{2.855741in}}%
\pgfpathlineto{\pgfqpoint{4.718302in}{2.856251in}}%
\pgfpathlineto{\pgfqpoint{4.719482in}{2.862423in}}%
\pgfpathlineto{\pgfqpoint{4.721251in}{2.866395in}}%
\pgfpathlineto{\pgfqpoint{4.722936in}{2.867415in}}%
\pgfpathlineto{\pgfqpoint{4.724200in}{2.869530in}}%
\pgfpathlineto{\pgfqpoint{4.724537in}{2.869146in}}%
\pgfpathlineto{\pgfqpoint{4.727822in}{2.865213in}}%
\pgfpathlineto{\pgfqpoint{4.730012in}{2.864542in}}%
\pgfpathlineto{\pgfqpoint{4.732961in}{2.853221in}}%
\pgfpathlineto{\pgfqpoint{4.734393in}{2.854513in}}%
\pgfpathlineto{\pgfqpoint{4.735236in}{2.855215in}}%
\pgfpathlineto{\pgfqpoint{4.735573in}{2.854356in}}%
\pgfpathlineto{\pgfqpoint{4.737173in}{2.847895in}}%
\pgfpathlineto{\pgfqpoint{4.737847in}{2.849022in}}%
\pgfpathlineto{\pgfqpoint{4.738942in}{2.849825in}}%
\pgfpathlineto{\pgfqpoint{4.739195in}{2.849463in}}%
\pgfpathlineto{\pgfqpoint{4.741807in}{2.845707in}}%
\pgfpathlineto{\pgfqpoint{4.743576in}{2.847032in}}%
\pgfpathlineto{\pgfqpoint{4.749052in}{2.863355in}}%
\pgfpathlineto{\pgfqpoint{4.749389in}{2.862991in}}%
\pgfpathlineto{\pgfqpoint{4.750315in}{2.861060in}}%
\pgfpathlineto{\pgfqpoint{4.751074in}{2.861844in}}%
\pgfpathlineto{\pgfqpoint{4.755117in}{2.860610in}}%
\pgfpathlineto{\pgfqpoint{4.757308in}{2.860369in}}%
\pgfpathlineto{\pgfqpoint{4.758066in}{2.861559in}}%
\pgfpathlineto{\pgfqpoint{4.760088in}{2.870436in}}%
\pgfpathlineto{\pgfqpoint{4.760846in}{2.868021in}}%
\pgfpathlineto{\pgfqpoint{4.762025in}{2.864090in}}%
\pgfpathlineto{\pgfqpoint{4.762447in}{2.865054in}}%
\pgfpathlineto{\pgfqpoint{4.763710in}{2.870722in}}%
\pgfpathlineto{\pgfqpoint{4.764300in}{2.869055in}}%
\pgfpathlineto{\pgfqpoint{4.765564in}{2.864760in}}%
\pgfpathlineto{\pgfqpoint{4.766069in}{2.865952in}}%
\pgfpathlineto{\pgfqpoint{4.768259in}{2.872308in}}%
\pgfpathlineto{\pgfqpoint{4.768681in}{2.871959in}}%
\pgfpathlineto{\pgfqpoint{4.769439in}{2.871726in}}%
\pgfpathlineto{\pgfqpoint{4.769860in}{2.872318in}}%
\pgfpathlineto{\pgfqpoint{4.772050in}{2.876814in}}%
\pgfpathlineto{\pgfqpoint{4.774325in}{2.882096in}}%
\pgfpathlineto{\pgfqpoint{4.774409in}{2.882069in}}%
\pgfpathlineto{\pgfqpoint{4.775673in}{2.880377in}}%
\pgfpathlineto{\pgfqpoint{4.776178in}{2.882008in}}%
\pgfpathlineto{\pgfqpoint{4.778369in}{2.893270in}}%
\pgfpathlineto{\pgfqpoint{4.779043in}{2.892874in}}%
\pgfpathlineto{\pgfqpoint{4.779717in}{2.892898in}}%
\pgfpathlineto{\pgfqpoint{4.779969in}{2.893588in}}%
\pgfpathlineto{\pgfqpoint{4.782160in}{2.905154in}}%
\pgfpathlineto{\pgfqpoint{4.783087in}{2.902189in}}%
\pgfpathlineto{\pgfqpoint{4.784266in}{2.899900in}}%
\pgfpathlineto{\pgfqpoint{4.784603in}{2.900360in}}%
\pgfpathlineto{\pgfqpoint{4.786204in}{2.904288in}}%
\pgfpathlineto{\pgfqpoint{4.786962in}{2.903357in}}%
\pgfpathlineto{\pgfqpoint{4.788478in}{2.901301in}}%
\pgfpathlineto{\pgfqpoint{4.790753in}{2.896572in}}%
\pgfpathlineto{\pgfqpoint{4.791932in}{2.895612in}}%
\pgfpathlineto{\pgfqpoint{4.792690in}{2.890686in}}%
\pgfpathlineto{\pgfqpoint{4.794965in}{2.875020in}}%
\pgfpathlineto{\pgfqpoint{4.795218in}{2.875078in}}%
\pgfpathlineto{\pgfqpoint{4.795892in}{2.874340in}}%
\pgfpathlineto{\pgfqpoint{4.796818in}{2.866889in}}%
\pgfpathlineto{\pgfqpoint{4.799093in}{2.852247in}}%
\pgfpathlineto{\pgfqpoint{4.800272in}{2.853424in}}%
\pgfpathlineto{\pgfqpoint{4.800694in}{2.853709in}}%
\pgfpathlineto{\pgfqpoint{4.801031in}{2.852991in}}%
\pgfpathlineto{\pgfqpoint{4.803558in}{2.843951in}}%
\pgfpathlineto{\pgfqpoint{4.804232in}{2.844800in}}%
\pgfpathlineto{\pgfqpoint{4.806085in}{2.845304in}}%
\pgfpathlineto{\pgfqpoint{4.810887in}{2.846936in}}%
\pgfpathlineto{\pgfqpoint{4.812993in}{2.848582in}}%
\pgfpathlineto{\pgfqpoint{4.814257in}{2.849911in}}%
\pgfpathlineto{\pgfqpoint{4.815015in}{2.851136in}}%
\pgfpathlineto{\pgfqpoint{4.815521in}{2.850358in}}%
\pgfpathlineto{\pgfqpoint{4.819817in}{2.844098in}}%
\pgfpathlineto{\pgfqpoint{4.820912in}{2.843185in}}%
\pgfpathlineto{\pgfqpoint{4.822850in}{2.842134in}}%
\pgfpathlineto{\pgfqpoint{4.827231in}{2.843101in}}%
\pgfpathlineto{\pgfqpoint{4.828073in}{2.842340in}}%
\pgfpathlineto{\pgfqpoint{4.829926in}{2.841240in}}%
\pgfpathlineto{\pgfqpoint{4.831022in}{2.840004in}}%
\pgfpathlineto{\pgfqpoint{4.833212in}{2.836723in}}%
\pgfpathlineto{\pgfqpoint{4.836245in}{2.837240in}}%
\pgfpathlineto{\pgfqpoint{4.839699in}{2.847431in}}%
\pgfpathlineto{\pgfqpoint{4.840541in}{2.849016in}}%
\pgfpathlineto{\pgfqpoint{4.842479in}{2.853859in}}%
\pgfpathlineto{\pgfqpoint{4.842732in}{2.853760in}}%
\pgfpathlineto{\pgfqpoint{4.846691in}{2.852990in}}%
\pgfpathlineto{\pgfqpoint{4.850903in}{2.858480in}}%
\pgfpathlineto{\pgfqpoint{4.852841in}{2.858581in}}%
\pgfpathlineto{\pgfqpoint{4.853768in}{2.853988in}}%
\pgfpathlineto{\pgfqpoint{4.855537in}{2.849850in}}%
\pgfpathlineto{\pgfqpoint{4.857474in}{2.848468in}}%
\pgfpathlineto{\pgfqpoint{4.858148in}{2.846215in}}%
\pgfpathlineto{\pgfqpoint{4.858738in}{2.847652in}}%
\pgfpathlineto{\pgfqpoint{4.860928in}{2.854367in}}%
\pgfpathlineto{\pgfqpoint{4.861602in}{2.853924in}}%
\pgfpathlineto{\pgfqpoint{4.862361in}{2.855119in}}%
\pgfpathlineto{\pgfqpoint{4.869016in}{2.874080in}}%
\pgfpathlineto{\pgfqpoint{4.871880in}{2.874625in}}%
\pgfpathlineto{\pgfqpoint{4.872807in}{2.873794in}}%
\pgfpathlineto{\pgfqpoint{4.873228in}{2.874540in}}%
\pgfpathlineto{\pgfqpoint{4.874576in}{2.876725in}}%
\pgfpathlineto{\pgfqpoint{4.874913in}{2.876237in}}%
\pgfpathlineto{\pgfqpoint{4.887297in}{2.843460in}}%
\pgfpathlineto{\pgfqpoint{4.889235in}{2.843310in}}%
\pgfpathlineto{\pgfqpoint{4.889909in}{2.843372in}}%
\pgfpathlineto{\pgfqpoint{4.890161in}{2.842806in}}%
\pgfpathlineto{\pgfqpoint{4.890835in}{2.841315in}}%
\pgfpathlineto{\pgfqpoint{4.891341in}{2.842328in}}%
\pgfpathlineto{\pgfqpoint{4.893194in}{2.844253in}}%
\pgfpathlineto{\pgfqpoint{4.894205in}{2.843576in}}%
\pgfpathlineto{\pgfqpoint{4.895553in}{2.842127in}}%
\pgfpathlineto{\pgfqpoint{4.895974in}{2.842472in}}%
\pgfpathlineto{\pgfqpoint{4.898333in}{2.843048in}}%
\pgfpathlineto{\pgfqpoint{4.901029in}{2.842416in}}%
\pgfpathlineto{\pgfqpoint{4.905494in}{2.842397in}}%
\pgfpathlineto{\pgfqpoint{4.907684in}{2.842510in}}%
\pgfpathlineto{\pgfqpoint{4.910464in}{2.838806in}}%
\pgfpathlineto{\pgfqpoint{4.913750in}{2.837833in}}%
\pgfpathlineto{\pgfqpoint{4.914087in}{2.837953in}}%
\pgfpathlineto{\pgfqpoint{4.914339in}{2.838635in}}%
\pgfpathlineto{\pgfqpoint{4.919226in}{2.852350in}}%
\pgfpathlineto{\pgfqpoint{4.921079in}{2.854106in}}%
\pgfpathlineto{\pgfqpoint{4.923438in}{2.861103in}}%
\pgfpathlineto{\pgfqpoint{4.924533in}{2.861913in}}%
\pgfpathlineto{\pgfqpoint{4.926976in}{2.875086in}}%
\pgfpathlineto{\pgfqpoint{4.928577in}{2.878138in}}%
\pgfpathlineto{\pgfqpoint{4.929756in}{2.879566in}}%
\pgfpathlineto{\pgfqpoint{4.931020in}{2.886353in}}%
\pgfpathlineto{\pgfqpoint{4.931778in}{2.885154in}}%
\pgfpathlineto{\pgfqpoint{4.934053in}{2.883740in}}%
\pgfpathlineto{\pgfqpoint{4.934642in}{2.883752in}}%
\pgfpathlineto{\pgfqpoint{4.934895in}{2.884407in}}%
\pgfpathlineto{\pgfqpoint{4.936412in}{2.895038in}}%
\pgfpathlineto{\pgfqpoint{4.937507in}{2.898159in}}%
\pgfpathlineto{\pgfqpoint{4.937844in}{2.897491in}}%
\pgfpathlineto{\pgfqpoint{4.939192in}{2.891620in}}%
\pgfpathlineto{\pgfqpoint{4.940034in}{2.893066in}}%
\pgfpathlineto{\pgfqpoint{4.941298in}{2.899078in}}%
\pgfpathlineto{\pgfqpoint{4.941972in}{2.896646in}}%
\pgfpathlineto{\pgfqpoint{4.945510in}{2.884390in}}%
\pgfpathlineto{\pgfqpoint{4.946521in}{2.883358in}}%
\pgfpathlineto{\pgfqpoint{4.949806in}{2.871255in}}%
\pgfpathlineto{\pgfqpoint{4.950059in}{2.871404in}}%
\pgfpathlineto{\pgfqpoint{4.950733in}{2.871486in}}%
\pgfpathlineto{\pgfqpoint{4.951070in}{2.870825in}}%
\pgfpathlineto{\pgfqpoint{4.952081in}{2.864501in}}%
\pgfpathlineto{\pgfqpoint{4.954271in}{2.854042in}}%
\pgfpathlineto{\pgfqpoint{4.956293in}{2.854359in}}%
\pgfpathlineto{\pgfqpoint{4.956883in}{2.854875in}}%
\pgfpathlineto{\pgfqpoint{4.957220in}{2.854009in}}%
\pgfpathlineto{\pgfqpoint{4.959495in}{2.848576in}}%
\pgfpathlineto{\pgfqpoint{4.960505in}{2.849425in}}%
\pgfpathlineto{\pgfqpoint{4.962949in}{2.854404in}}%
\pgfpathlineto{\pgfqpoint{4.964465in}{2.856643in}}%
\pgfpathlineto{\pgfqpoint{4.967582in}{2.859128in}}%
\pgfpathlineto{\pgfqpoint{4.968930in}{2.861763in}}%
\pgfpathlineto{\pgfqpoint{4.969435in}{2.860893in}}%
\pgfpathlineto{\pgfqpoint{4.971205in}{2.858038in}}%
\pgfpathlineto{\pgfqpoint{4.971457in}{2.858129in}}%
\pgfpathlineto{\pgfqpoint{4.972637in}{2.857483in}}%
\pgfpathlineto{\pgfqpoint{4.973563in}{2.853619in}}%
\pgfpathlineto{\pgfqpoint{4.975417in}{2.849231in}}%
\pgfpathlineto{\pgfqpoint{4.979713in}{2.847757in}}%
\pgfpathlineto{\pgfqpoint{4.980303in}{2.847287in}}%
\pgfpathlineto{\pgfqpoint{4.980977in}{2.847600in}}%
\pgfpathlineto{\pgfqpoint{4.981735in}{2.848820in}}%
\pgfpathlineto{\pgfqpoint{4.984347in}{2.859581in}}%
\pgfpathlineto{\pgfqpoint{4.985189in}{2.857029in}}%
\pgfpathlineto{\pgfqpoint{4.985947in}{2.856388in}}%
\pgfpathlineto{\pgfqpoint{4.986453in}{2.856656in}}%
\pgfpathlineto{\pgfqpoint{4.989064in}{2.860058in}}%
\pgfpathlineto{\pgfqpoint{4.994709in}{2.875390in}}%
\pgfpathlineto{\pgfqpoint{4.995804in}{2.874894in}}%
\pgfpathlineto{\pgfqpoint{4.999763in}{2.864212in}}%
\pgfpathlineto{\pgfqpoint{5.001280in}{2.855322in}}%
\pgfpathlineto{\pgfqpoint{5.002207in}{2.857503in}}%
\pgfpathlineto{\pgfqpoint{5.003723in}{2.858318in}}%
\pgfpathlineto{\pgfqpoint{5.007093in}{2.856989in}}%
\pgfpathlineto{\pgfqpoint{5.008441in}{2.856026in}}%
\pgfpathlineto{\pgfqpoint{5.008693in}{2.856359in}}%
\pgfpathlineto{\pgfqpoint{5.009789in}{2.859710in}}%
\pgfpathlineto{\pgfqpoint{5.010378in}{2.858074in}}%
\pgfpathlineto{\pgfqpoint{5.011895in}{2.856672in}}%
\pgfpathlineto{\pgfqpoint{5.013327in}{2.858183in}}%
\pgfpathlineto{\pgfqpoint{5.014085in}{2.862402in}}%
\pgfpathlineto{\pgfqpoint{5.015770in}{2.867072in}}%
\pgfpathlineto{\pgfqpoint{5.017371in}{2.868238in}}%
\pgfpathlineto{\pgfqpoint{5.021667in}{2.878120in}}%
\pgfpathlineto{\pgfqpoint{5.023268in}{2.880947in}}%
\pgfpathlineto{\pgfqpoint{5.024026in}{2.879412in}}%
\pgfpathlineto{\pgfqpoint{5.025037in}{2.879193in}}%
\pgfpathlineto{\pgfqpoint{5.025290in}{2.879416in}}%
\pgfpathlineto{\pgfqpoint{5.025627in}{2.879386in}}%
\pgfpathlineto{\pgfqpoint{5.025879in}{2.878646in}}%
\pgfpathlineto{\pgfqpoint{5.027985in}{2.874194in}}%
\pgfpathlineto{\pgfqpoint{5.028575in}{2.875885in}}%
\pgfpathlineto{\pgfqpoint{5.030091in}{2.878396in}}%
\pgfpathlineto{\pgfqpoint{5.030260in}{2.878233in}}%
\pgfpathlineto{\pgfqpoint{5.031187in}{2.873788in}}%
\pgfpathlineto{\pgfqpoint{5.031776in}{2.872519in}}%
\pgfpathlineto{\pgfqpoint{5.032282in}{2.873554in}}%
\pgfpathlineto{\pgfqpoint{5.035230in}{2.883828in}}%
\pgfpathlineto{\pgfqpoint{5.035736in}{2.882220in}}%
\pgfpathlineto{\pgfqpoint{5.037168in}{2.879505in}}%
\pgfpathlineto{\pgfqpoint{5.037336in}{2.879559in}}%
\pgfpathlineto{\pgfqpoint{5.038516in}{2.880400in}}%
\pgfpathlineto{\pgfqpoint{5.038937in}{2.879399in}}%
\pgfpathlineto{\pgfqpoint{5.040791in}{2.871313in}}%
\pgfpathlineto{\pgfqpoint{5.041970in}{2.872695in}}%
\pgfpathlineto{\pgfqpoint{5.044329in}{2.872767in}}%
\pgfpathlineto{\pgfqpoint{5.046098in}{2.871667in}}%
\pgfpathlineto{\pgfqpoint{5.046603in}{2.871523in}}%
\pgfpathlineto{\pgfqpoint{5.047109in}{2.872175in}}%
\pgfpathlineto{\pgfqpoint{5.047614in}{2.872297in}}%
\pgfpathlineto{\pgfqpoint{5.048036in}{2.871610in}}%
\pgfpathlineto{\pgfqpoint{5.049636in}{2.869985in}}%
\pgfpathlineto{\pgfqpoint{5.049889in}{2.870413in}}%
\pgfpathlineto{\pgfqpoint{5.051153in}{2.876565in}}%
\pgfpathlineto{\pgfqpoint{5.051742in}{2.872719in}}%
\pgfpathlineto{\pgfqpoint{5.052753in}{2.867602in}}%
\pgfpathlineto{\pgfqpoint{5.053343in}{2.867709in}}%
\pgfpathlineto{\pgfqpoint{5.055112in}{2.868327in}}%
\pgfpathlineto{\pgfqpoint{5.056460in}{2.870430in}}%
\pgfpathlineto{\pgfqpoint{5.057471in}{2.871718in}}%
\pgfpathlineto{\pgfqpoint{5.057892in}{2.871234in}}%
\pgfpathlineto{\pgfqpoint{5.060251in}{2.869898in}}%
\pgfpathlineto{\pgfqpoint{5.061936in}{2.867586in}}%
\pgfpathlineto{\pgfqpoint{5.064211in}{2.863510in}}%
\pgfpathlineto{\pgfqpoint{5.066485in}{2.862952in}}%
\pgfpathlineto{\pgfqpoint{5.068675in}{2.851260in}}%
\pgfpathlineto{\pgfqpoint{5.069771in}{2.853995in}}%
\pgfpathlineto{\pgfqpoint{5.071203in}{2.854507in}}%
\pgfpathlineto{\pgfqpoint{5.071371in}{2.854340in}}%
\pgfpathlineto{\pgfqpoint{5.073983in}{2.852169in}}%
\pgfpathlineto{\pgfqpoint{5.074151in}{2.852377in}}%
\pgfpathlineto{\pgfqpoint{5.075331in}{2.853780in}}%
\pgfpathlineto{\pgfqpoint{5.075836in}{2.853234in}}%
\pgfpathlineto{\pgfqpoint{5.077437in}{2.852790in}}%
\pgfpathlineto{\pgfqpoint{5.079711in}{2.854881in}}%
\pgfpathlineto{\pgfqpoint{5.081986in}{2.859192in}}%
\pgfpathlineto{\pgfqpoint{5.082155in}{2.859133in}}%
\pgfpathlineto{\pgfqpoint{5.082829in}{2.860139in}}%
\pgfpathlineto{\pgfqpoint{5.083502in}{2.862167in}}%
\pgfpathlineto{\pgfqpoint{5.084008in}{2.860869in}}%
\pgfpathlineto{\pgfqpoint{5.084345in}{2.859949in}}%
\pgfpathlineto{\pgfqpoint{5.084850in}{2.861619in}}%
\pgfpathlineto{\pgfqpoint{5.085524in}{2.864218in}}%
\pgfpathlineto{\pgfqpoint{5.086030in}{2.862706in}}%
\pgfpathlineto{\pgfqpoint{5.086620in}{2.861548in}}%
\pgfpathlineto{\pgfqpoint{5.087294in}{2.862157in}}%
\pgfpathlineto{\pgfqpoint{5.088557in}{2.861415in}}%
\pgfpathlineto{\pgfqpoint{5.089147in}{2.861017in}}%
\pgfpathlineto{\pgfqpoint{5.089484in}{2.861912in}}%
\pgfpathlineto{\pgfqpoint{5.092180in}{2.869926in}}%
\pgfpathlineto{\pgfqpoint{5.094202in}{2.875140in}}%
\pgfpathlineto{\pgfqpoint{5.095886in}{2.890373in}}%
\pgfpathlineto{\pgfqpoint{5.096897in}{2.887690in}}%
\pgfpathlineto{\pgfqpoint{5.099425in}{2.882409in}}%
\pgfpathlineto{\pgfqpoint{5.100436in}{2.880372in}}%
\pgfpathlineto{\pgfqpoint{5.100857in}{2.881475in}}%
\pgfpathlineto{\pgfqpoint{5.101278in}{2.882529in}}%
\pgfpathlineto{\pgfqpoint{5.101868in}{2.880968in}}%
\pgfpathlineto{\pgfqpoint{5.102121in}{2.880525in}}%
\pgfpathlineto{\pgfqpoint{5.102458in}{2.881804in}}%
\pgfpathlineto{\pgfqpoint{5.104816in}{2.892393in}}%
\pgfpathlineto{\pgfqpoint{5.105743in}{2.895521in}}%
\pgfpathlineto{\pgfqpoint{5.106249in}{2.897266in}}%
\pgfpathlineto{\pgfqpoint{5.106754in}{2.894971in}}%
\pgfpathlineto{\pgfqpoint{5.107428in}{2.892205in}}%
\pgfpathlineto{\pgfqpoint{5.108102in}{2.893172in}}%
\pgfpathlineto{\pgfqpoint{5.108692in}{2.892786in}}%
\pgfpathlineto{\pgfqpoint{5.109029in}{2.893872in}}%
\pgfpathlineto{\pgfqpoint{5.109450in}{2.894939in}}%
\pgfpathlineto{\pgfqpoint{5.110040in}{2.893472in}}%
\pgfpathlineto{\pgfqpoint{5.111387in}{2.886540in}}%
\pgfpathlineto{\pgfqpoint{5.112567in}{2.873241in}}%
\pgfpathlineto{\pgfqpoint{5.113157in}{2.874420in}}%
\pgfpathlineto{\pgfqpoint{5.117200in}{2.888388in}}%
\pgfpathlineto{\pgfqpoint{5.118801in}{2.896434in}}%
\pgfpathlineto{\pgfqpoint{5.119306in}{2.894852in}}%
\pgfpathlineto{\pgfqpoint{5.120991in}{2.891493in}}%
\pgfpathlineto{\pgfqpoint{5.122760in}{2.887068in}}%
\pgfpathlineto{\pgfqpoint{5.123519in}{2.884115in}}%
\pgfpathlineto{\pgfqpoint{5.124108in}{2.884947in}}%
\pgfpathlineto{\pgfqpoint{5.124614in}{2.883757in}}%
\pgfpathlineto{\pgfqpoint{5.126804in}{2.875444in}}%
\pgfpathlineto{\pgfqpoint{5.127478in}{2.877481in}}%
\pgfpathlineto{\pgfqpoint{5.129247in}{2.881877in}}%
\pgfpathlineto{\pgfqpoint{5.131101in}{2.882057in}}%
\pgfpathlineto{\pgfqpoint{5.131943in}{2.879004in}}%
\pgfpathlineto{\pgfqpoint{5.133796in}{2.873201in}}%
\pgfpathlineto{\pgfqpoint{5.133965in}{2.873412in}}%
\pgfpathlineto{\pgfqpoint{5.134470in}{2.874751in}}%
\pgfpathlineto{\pgfqpoint{5.134892in}{2.873303in}}%
\pgfpathlineto{\pgfqpoint{5.137251in}{2.863252in}}%
\pgfpathlineto{\pgfqpoint{5.137419in}{2.863509in}}%
\pgfpathlineto{\pgfqpoint{5.138261in}{2.870090in}}%
\pgfpathlineto{\pgfqpoint{5.140283in}{2.877923in}}%
\pgfpathlineto{\pgfqpoint{5.141968in}{2.879670in}}%
\pgfpathlineto{\pgfqpoint{5.144159in}{2.878763in}}%
\pgfpathlineto{\pgfqpoint{5.144748in}{2.877363in}}%
\pgfpathlineto{\pgfqpoint{5.145254in}{2.878859in}}%
\pgfpathlineto{\pgfqpoint{5.147023in}{2.891566in}}%
\pgfpathlineto{\pgfqpoint{5.148202in}{2.896589in}}%
\pgfpathlineto{\pgfqpoint{5.148539in}{2.896338in}}%
\pgfpathlineto{\pgfqpoint{5.149213in}{2.895705in}}%
\pgfpathlineto{\pgfqpoint{5.149550in}{2.896528in}}%
\pgfpathlineto{\pgfqpoint{5.150477in}{2.899403in}}%
\pgfpathlineto{\pgfqpoint{5.150898in}{2.897628in}}%
\pgfpathlineto{\pgfqpoint{5.152752in}{2.891346in}}%
\pgfpathlineto{\pgfqpoint{5.153762in}{2.890289in}}%
\pgfpathlineto{\pgfqpoint{5.154773in}{2.886630in}}%
\pgfpathlineto{\pgfqpoint{5.157048in}{2.876395in}}%
\pgfpathlineto{\pgfqpoint{5.158564in}{2.874545in}}%
\pgfpathlineto{\pgfqpoint{5.159828in}{2.875629in}}%
\pgfpathlineto{\pgfqpoint{5.160755in}{2.876258in}}%
\pgfpathlineto{\pgfqpoint{5.161092in}{2.875786in}}%
\pgfpathlineto{\pgfqpoint{5.163282in}{2.866827in}}%
\pgfpathlineto{\pgfqpoint{5.164798in}{2.850972in}}%
\pgfpathlineto{\pgfqpoint{5.165557in}{2.851739in}}%
\pgfpathlineto{\pgfqpoint{5.165978in}{2.851960in}}%
\pgfpathlineto{\pgfqpoint{5.166315in}{2.851279in}}%
\pgfpathlineto{\pgfqpoint{5.168505in}{2.846902in}}%
\pgfpathlineto{\pgfqpoint{5.168674in}{2.847044in}}%
\pgfpathlineto{\pgfqpoint{5.169348in}{2.850796in}}%
\pgfpathlineto{\pgfqpoint{5.171117in}{2.855995in}}%
\pgfpathlineto{\pgfqpoint{5.172128in}{2.856236in}}%
\pgfpathlineto{\pgfqpoint{5.172380in}{2.855857in}}%
\pgfpathlineto{\pgfqpoint{5.173897in}{2.855323in}}%
\pgfpathlineto{\pgfqpoint{5.178783in}{2.855198in}}%
\pgfpathlineto{\pgfqpoint{5.181226in}{2.853627in}}%
\pgfpathlineto{\pgfqpoint{5.184343in}{2.854658in}}%
\pgfpathlineto{\pgfqpoint{5.185354in}{2.855679in}}%
\pgfpathlineto{\pgfqpoint{5.185691in}{2.855062in}}%
\pgfpathlineto{\pgfqpoint{5.187039in}{2.848319in}}%
\pgfpathlineto{\pgfqpoint{5.187881in}{2.850166in}}%
\pgfpathlineto{\pgfqpoint{5.189314in}{2.850459in}}%
\pgfpathlineto{\pgfqpoint{5.193020in}{2.849468in}}%
\pgfpathlineto{\pgfqpoint{5.195379in}{2.847067in}}%
\pgfpathlineto{\pgfqpoint{5.197822in}{2.847655in}}%
\pgfpathlineto{\pgfqpoint{5.199423in}{2.851361in}}%
\pgfpathlineto{\pgfqpoint{5.200602in}{2.850220in}}%
\pgfpathlineto{\pgfqpoint{5.202287in}{2.847364in}}%
\pgfpathlineto{\pgfqpoint{5.202793in}{2.848145in}}%
\pgfpathlineto{\pgfqpoint{5.203635in}{2.848807in}}%
\pgfpathlineto{\pgfqpoint{5.203972in}{2.848345in}}%
\pgfpathlineto{\pgfqpoint{5.205826in}{2.844964in}}%
\pgfpathlineto{\pgfqpoint{5.206247in}{2.845450in}}%
\pgfpathlineto{\pgfqpoint{5.207089in}{2.846229in}}%
\pgfpathlineto{\pgfqpoint{5.207595in}{2.845695in}}%
\pgfpathlineto{\pgfqpoint{5.209111in}{2.845545in}}%
\pgfpathlineto{\pgfqpoint{5.211301in}{2.847045in}}%
\pgfpathlineto{\pgfqpoint{5.213660in}{2.849501in}}%
\pgfpathlineto{\pgfqpoint{5.215177in}{2.850558in}}%
\pgfpathlineto{\pgfqpoint{5.215598in}{2.849939in}}%
\pgfpathlineto{\pgfqpoint{5.216103in}{2.849634in}}%
\pgfpathlineto{\pgfqpoint{5.216440in}{2.850328in}}%
\pgfpathlineto{\pgfqpoint{5.218968in}{2.858754in}}%
\pgfpathlineto{\pgfqpoint{5.219642in}{2.857863in}}%
\pgfpathlineto{\pgfqpoint{5.219810in}{2.857861in}}%
\pgfpathlineto{\pgfqpoint{5.220063in}{2.858708in}}%
\pgfpathlineto{\pgfqpoint{5.221327in}{2.866459in}}%
\pgfpathlineto{\pgfqpoint{5.222169in}{2.865117in}}%
\pgfpathlineto{\pgfqpoint{5.222506in}{2.865259in}}%
\pgfpathlineto{\pgfqpoint{5.222759in}{2.865883in}}%
\pgfpathlineto{\pgfqpoint{5.224781in}{2.868530in}}%
\pgfpathlineto{\pgfqpoint{5.227982in}{2.871005in}}%
\pgfpathlineto{\pgfqpoint{5.228403in}{2.870063in}}%
\pgfpathlineto{\pgfqpoint{5.230256in}{2.868060in}}%
\pgfpathlineto{\pgfqpoint{5.232278in}{2.866403in}}%
\pgfpathlineto{\pgfqpoint{5.234384in}{2.859185in}}%
\pgfpathlineto{\pgfqpoint{5.234974in}{2.857896in}}%
\pgfpathlineto{\pgfqpoint{5.235564in}{2.858456in}}%
\pgfpathlineto{\pgfqpoint{5.236406in}{2.859918in}}%
\pgfpathlineto{\pgfqpoint{5.236828in}{2.858691in}}%
\pgfpathlineto{\pgfqpoint{5.239355in}{2.848720in}}%
\pgfpathlineto{\pgfqpoint{5.240450in}{2.845703in}}%
\pgfpathlineto{\pgfqpoint{5.241461in}{2.846861in}}%
\pgfpathlineto{\pgfqpoint{5.243651in}{2.846083in}}%
\pgfpathlineto{\pgfqpoint{5.244747in}{2.844565in}}%
\pgfpathlineto{\pgfqpoint{5.245168in}{2.844817in}}%
\pgfpathlineto{\pgfqpoint{5.247442in}{2.845533in}}%
\pgfpathlineto{\pgfqpoint{5.251065in}{2.845177in}}%
\pgfpathlineto{\pgfqpoint{5.252666in}{2.844512in}}%
\pgfpathlineto{\pgfqpoint{5.253087in}{2.844017in}}%
\pgfpathlineto{\pgfqpoint{5.253592in}{2.845130in}}%
\pgfpathlineto{\pgfqpoint{5.255446in}{2.847581in}}%
\pgfpathlineto{\pgfqpoint{5.257215in}{2.846797in}}%
\pgfpathlineto{\pgfqpoint{5.259068in}{2.845234in}}%
\pgfpathlineto{\pgfqpoint{5.260921in}{2.844280in}}%
\pgfpathlineto{\pgfqpoint{5.261932in}{2.845436in}}%
\pgfpathlineto{\pgfqpoint{5.263449in}{2.846018in}}%
\pgfpathlineto{\pgfqpoint{5.270441in}{2.846191in}}%
\pgfpathlineto{\pgfqpoint{5.270947in}{2.845359in}}%
\pgfpathlineto{\pgfqpoint{5.271452in}{2.846400in}}%
\pgfpathlineto{\pgfqpoint{5.273474in}{2.854661in}}%
\pgfpathlineto{\pgfqpoint{5.274906in}{2.867606in}}%
\pgfpathlineto{\pgfqpoint{5.275496in}{2.867172in}}%
\pgfpathlineto{\pgfqpoint{5.276338in}{2.868127in}}%
\pgfpathlineto{\pgfqpoint{5.278529in}{2.880961in}}%
\pgfpathlineto{\pgfqpoint{5.279876in}{2.880525in}}%
\pgfpathlineto{\pgfqpoint{5.280803in}{2.882124in}}%
\pgfpathlineto{\pgfqpoint{5.283499in}{2.889174in}}%
\pgfpathlineto{\pgfqpoint{5.283583in}{2.889128in}}%
\pgfpathlineto{\pgfqpoint{5.285858in}{2.888523in}}%
\pgfpathlineto{\pgfqpoint{5.289312in}{2.890209in}}%
\pgfpathlineto{\pgfqpoint{5.289902in}{2.890950in}}%
\pgfpathlineto{\pgfqpoint{5.290239in}{2.889698in}}%
\pgfpathlineto{\pgfqpoint{5.291250in}{2.880312in}}%
\pgfpathlineto{\pgfqpoint{5.292008in}{2.883917in}}%
\pgfpathlineto{\pgfqpoint{5.293524in}{2.891476in}}%
\pgfpathlineto{\pgfqpoint{5.294198in}{2.888707in}}%
\pgfpathlineto{\pgfqpoint{5.294956in}{2.886072in}}%
\pgfpathlineto{\pgfqpoint{5.295377in}{2.887612in}}%
\pgfpathlineto{\pgfqpoint{5.297231in}{2.893291in}}%
\pgfpathlineto{\pgfqpoint{5.297484in}{2.892862in}}%
\pgfpathlineto{\pgfqpoint{5.299758in}{2.884412in}}%
\pgfpathlineto{\pgfqpoint{5.301022in}{2.885230in}}%
\pgfpathlineto{\pgfqpoint{5.302875in}{2.888726in}}%
\pgfpathlineto{\pgfqpoint{5.303886in}{2.895216in}}%
\pgfpathlineto{\pgfqpoint{5.304560in}{2.894348in}}%
\pgfpathlineto{\pgfqpoint{5.305571in}{2.893274in}}%
\pgfpathlineto{\pgfqpoint{5.306077in}{2.893961in}}%
\pgfpathlineto{\pgfqpoint{5.306498in}{2.894209in}}%
\pgfpathlineto{\pgfqpoint{5.306919in}{2.893255in}}%
\pgfpathlineto{\pgfqpoint{5.309783in}{2.880327in}}%
\pgfpathlineto{\pgfqpoint{5.312479in}{2.867876in}}%
\pgfpathlineto{\pgfqpoint{5.314922in}{2.861532in}}%
\pgfpathlineto{\pgfqpoint{5.317113in}{2.862894in}}%
\pgfpathlineto{\pgfqpoint{5.318124in}{2.865961in}}%
\pgfpathlineto{\pgfqpoint{5.318882in}{2.865755in}}%
\pgfpathlineto{\pgfqpoint{5.319303in}{2.865628in}}%
\pgfpathlineto{\pgfqpoint{5.319471in}{2.865044in}}%
\pgfpathlineto{\pgfqpoint{5.320904in}{2.855314in}}%
\pgfpathlineto{\pgfqpoint{5.321830in}{2.857095in}}%
\pgfpathlineto{\pgfqpoint{5.324947in}{2.872394in}}%
\pgfpathlineto{\pgfqpoint{5.329749in}{2.899026in}}%
\pgfpathlineto{\pgfqpoint{5.330423in}{2.900600in}}%
\pgfpathlineto{\pgfqpoint{5.333540in}{2.930748in}}%
\pgfpathlineto{\pgfqpoint{5.334298in}{2.927878in}}%
\pgfpathlineto{\pgfqpoint{5.334804in}{2.926547in}}%
\pgfpathlineto{\pgfqpoint{5.335309in}{2.928030in}}%
\pgfpathlineto{\pgfqpoint{5.336994in}{2.938593in}}%
\pgfpathlineto{\pgfqpoint{5.337752in}{2.937653in}}%
\pgfpathlineto{\pgfqpoint{5.339185in}{2.934002in}}%
\pgfpathlineto{\pgfqpoint{5.345924in}{2.891934in}}%
\pgfpathlineto{\pgfqpoint{5.346261in}{2.892086in}}%
\pgfpathlineto{\pgfqpoint{5.347104in}{2.892238in}}%
\pgfpathlineto{\pgfqpoint{5.347272in}{2.891508in}}%
\pgfpathlineto{\pgfqpoint{5.348199in}{2.874565in}}%
\pgfpathlineto{\pgfqpoint{5.349968in}{2.863189in}}%
\pgfpathlineto{\pgfqpoint{5.350136in}{2.862926in}}%
\pgfpathlineto{\pgfqpoint{5.350558in}{2.864509in}}%
\pgfpathlineto{\pgfqpoint{5.351653in}{2.869922in}}%
\pgfpathlineto{\pgfqpoint{5.352158in}{2.868384in}}%
\pgfpathlineto{\pgfqpoint{5.353253in}{2.860027in}}%
\pgfpathlineto{\pgfqpoint{5.354096in}{2.862336in}}%
\pgfpathlineto{\pgfqpoint{5.355612in}{2.864965in}}%
\pgfpathlineto{\pgfqpoint{5.356286in}{2.863515in}}%
\pgfpathlineto{\pgfqpoint{5.356792in}{2.862941in}}%
\pgfpathlineto{\pgfqpoint{5.357213in}{2.863822in}}%
\pgfpathlineto{\pgfqpoint{5.358814in}{2.864970in}}%
\pgfpathlineto{\pgfqpoint{5.360499in}{2.866078in}}%
\pgfpathlineto{\pgfqpoint{5.362773in}{2.869379in}}%
\pgfpathlineto{\pgfqpoint{5.363110in}{2.869209in}}%
\pgfpathlineto{\pgfqpoint{5.363700in}{2.870098in}}%
\pgfpathlineto{\pgfqpoint{5.364795in}{2.873903in}}%
\pgfpathlineto{\pgfqpoint{5.365553in}{2.873426in}}%
\pgfpathlineto{\pgfqpoint{5.366817in}{2.875096in}}%
\pgfpathlineto{\pgfqpoint{5.367238in}{2.873708in}}%
\pgfpathlineto{\pgfqpoint{5.369091in}{2.866161in}}%
\pgfpathlineto{\pgfqpoint{5.369428in}{2.866444in}}%
\pgfpathlineto{\pgfqpoint{5.370102in}{2.867098in}}%
\pgfpathlineto{\pgfqpoint{5.370524in}{2.866297in}}%
\pgfpathlineto{\pgfqpoint{5.372293in}{2.864043in}}%
\pgfpathlineto{\pgfqpoint{5.372545in}{2.864274in}}%
\pgfpathlineto{\pgfqpoint{5.373641in}{2.865361in}}%
\pgfpathlineto{\pgfqpoint{5.374062in}{2.864725in}}%
\pgfpathlineto{\pgfqpoint{5.374652in}{2.863908in}}%
\pgfpathlineto{\pgfqpoint{5.375157in}{2.864853in}}%
\pgfpathlineto{\pgfqpoint{5.377010in}{2.866410in}}%
\pgfpathlineto{\pgfqpoint{5.377516in}{2.868762in}}%
\pgfpathlineto{\pgfqpoint{5.380212in}{2.885975in}}%
\pgfpathlineto{\pgfqpoint{5.380549in}{2.887070in}}%
\pgfpathlineto{\pgfqpoint{5.381223in}{2.885643in}}%
\pgfpathlineto{\pgfqpoint{5.381644in}{2.887382in}}%
\pgfpathlineto{\pgfqpoint{5.382908in}{2.895335in}}%
\pgfpathlineto{\pgfqpoint{5.383666in}{2.895064in}}%
\pgfpathlineto{\pgfqpoint{5.384508in}{2.897810in}}%
\pgfpathlineto{\pgfqpoint{5.386277in}{2.906091in}}%
\pgfpathlineto{\pgfqpoint{5.386699in}{2.904973in}}%
\pgfpathlineto{\pgfqpoint{5.387373in}{2.903056in}}%
\pgfpathlineto{\pgfqpoint{5.387962in}{2.904196in}}%
\pgfpathlineto{\pgfqpoint{5.390574in}{2.910827in}}%
\pgfpathlineto{\pgfqpoint{5.391669in}{2.910235in}}%
\pgfpathlineto{\pgfqpoint{5.392427in}{2.907924in}}%
\pgfpathlineto{\pgfqpoint{5.392680in}{2.907456in}}%
\pgfpathlineto{\pgfqpoint{5.393185in}{2.909019in}}%
\pgfpathlineto{\pgfqpoint{5.393775in}{2.910919in}}%
\pgfpathlineto{\pgfqpoint{5.394196in}{2.909279in}}%
\pgfpathlineto{\pgfqpoint{5.399251in}{2.874186in}}%
\pgfpathlineto{\pgfqpoint{5.399756in}{2.873453in}}%
\pgfpathlineto{\pgfqpoint{5.400515in}{2.873923in}}%
\pgfpathlineto{\pgfqpoint{5.401189in}{2.872416in}}%
\pgfpathlineto{\pgfqpoint{5.403463in}{2.862229in}}%
\pgfpathlineto{\pgfqpoint{5.403969in}{2.863670in}}%
\pgfpathlineto{\pgfqpoint{5.404980in}{2.867455in}}%
\pgfpathlineto{\pgfqpoint{5.405569in}{2.867038in}}%
\pgfpathlineto{\pgfqpoint{5.406580in}{2.864343in}}%
\pgfpathlineto{\pgfqpoint{5.407086in}{2.866336in}}%
\pgfpathlineto{\pgfqpoint{5.408181in}{2.870955in}}%
\pgfpathlineto{\pgfqpoint{5.408855in}{2.870523in}}%
\pgfpathlineto{\pgfqpoint{5.410624in}{2.870007in}}%
\pgfpathlineto{\pgfqpoint{5.410792in}{2.871042in}}%
\pgfpathlineto{\pgfqpoint{5.412646in}{2.875963in}}%
\pgfpathlineto{\pgfqpoint{5.413910in}{2.875875in}}%
\pgfpathlineto{\pgfqpoint{5.415089in}{2.876901in}}%
\pgfpathlineto{\pgfqpoint{5.416100in}{2.879546in}}%
\pgfpathlineto{\pgfqpoint{5.417027in}{2.877954in}}%
\pgfpathlineto{\pgfqpoint{5.417532in}{2.879042in}}%
\pgfpathlineto{\pgfqpoint{5.419133in}{2.882298in}}%
\pgfpathlineto{\pgfqpoint{5.419385in}{2.882017in}}%
\pgfpathlineto{\pgfqpoint{5.421829in}{2.877629in}}%
\pgfpathlineto{\pgfqpoint{5.422587in}{2.878138in}}%
\pgfpathlineto{\pgfqpoint{5.423092in}{2.878319in}}%
\pgfpathlineto{\pgfqpoint{5.423345in}{2.877474in}}%
\pgfpathlineto{\pgfqpoint{5.424861in}{2.867830in}}%
\pgfpathlineto{\pgfqpoint{5.425535in}{2.870711in}}%
\pgfpathlineto{\pgfqpoint{5.427810in}{2.878965in}}%
\pgfpathlineto{\pgfqpoint{5.429663in}{2.881265in}}%
\pgfpathlineto{\pgfqpoint{5.432191in}{2.894867in}}%
\pgfpathlineto{\pgfqpoint{5.432359in}{2.894962in}}%
\pgfpathlineto{\pgfqpoint{5.432696in}{2.893825in}}%
\pgfpathlineto{\pgfqpoint{5.433286in}{2.892369in}}%
\pgfpathlineto{\pgfqpoint{5.434044in}{2.892764in}}%
\pgfpathlineto{\pgfqpoint{5.436066in}{2.888572in}}%
\pgfpathlineto{\pgfqpoint{5.436487in}{2.889340in}}%
\pgfpathlineto{\pgfqpoint{5.442300in}{2.912565in}}%
\pgfpathlineto{\pgfqpoint{5.443395in}{2.909863in}}%
\pgfpathlineto{\pgfqpoint{5.443732in}{2.910012in}}%
\pgfpathlineto{\pgfqpoint{5.443985in}{2.909357in}}%
\pgfpathlineto{\pgfqpoint{5.444743in}{2.906644in}}%
\pgfpathlineto{\pgfqpoint{5.445333in}{2.907821in}}%
\pgfpathlineto{\pgfqpoint{5.445754in}{2.908119in}}%
\pgfpathlineto{\pgfqpoint{5.446091in}{2.907300in}}%
\pgfpathlineto{\pgfqpoint{5.449545in}{2.891673in}}%
\pgfpathlineto{\pgfqpoint{5.450135in}{2.892098in}}%
\pgfpathlineto{\pgfqpoint{5.453673in}{2.890394in}}%
\pgfpathlineto{\pgfqpoint{5.455105in}{2.883100in}}%
\pgfpathlineto{\pgfqpoint{5.456622in}{2.876748in}}%
\pgfpathlineto{\pgfqpoint{5.457043in}{2.876988in}}%
\pgfpathlineto{\pgfqpoint{5.457548in}{2.876085in}}%
\pgfpathlineto{\pgfqpoint{5.458728in}{2.866334in}}%
\pgfpathlineto{\pgfqpoint{5.460413in}{2.860035in}}%
\pgfpathlineto{\pgfqpoint{5.460750in}{2.859046in}}%
\pgfpathlineto{\pgfqpoint{5.461423in}{2.860576in}}%
\pgfpathlineto{\pgfqpoint{5.462097in}{2.858605in}}%
\pgfpathlineto{\pgfqpoint{5.463445in}{2.858168in}}%
\pgfpathlineto{\pgfqpoint{5.464288in}{2.859490in}}%
\pgfpathlineto{\pgfqpoint{5.466141in}{2.862812in}}%
\pgfpathlineto{\pgfqpoint{5.466225in}{2.862775in}}%
\pgfpathlineto{\pgfqpoint{5.467068in}{2.862628in}}%
\pgfpathlineto{\pgfqpoint{5.467321in}{2.863120in}}%
\pgfpathlineto{\pgfqpoint{5.471786in}{2.872929in}}%
\pgfpathlineto{\pgfqpoint{5.472544in}{2.874250in}}%
\pgfpathlineto{\pgfqpoint{5.474566in}{2.876789in}}%
\pgfpathlineto{\pgfqpoint{5.475408in}{2.876724in}}%
\pgfpathlineto{\pgfqpoint{5.475577in}{2.877512in}}%
\pgfpathlineto{\pgfqpoint{5.477514in}{2.888161in}}%
\pgfpathlineto{\pgfqpoint{5.478020in}{2.886461in}}%
\pgfpathlineto{\pgfqpoint{5.478862in}{2.882726in}}%
\pgfpathlineto{\pgfqpoint{5.479452in}{2.884314in}}%
\pgfpathlineto{\pgfqpoint{5.480884in}{2.892893in}}%
\pgfpathlineto{\pgfqpoint{5.482232in}{2.890199in}}%
\pgfpathlineto{\pgfqpoint{5.484085in}{2.891700in}}%
\pgfpathlineto{\pgfqpoint{5.484422in}{2.890913in}}%
\pgfpathlineto{\pgfqpoint{5.487118in}{2.884540in}}%
\pgfpathlineto{\pgfqpoint{5.487792in}{2.883037in}}%
\pgfpathlineto{\pgfqpoint{5.488382in}{2.881970in}}%
\pgfpathlineto{\pgfqpoint{5.488803in}{2.883031in}}%
\pgfpathlineto{\pgfqpoint{5.489561in}{2.884693in}}%
\pgfpathlineto{\pgfqpoint{5.490151in}{2.883940in}}%
\pgfpathlineto{\pgfqpoint{5.490825in}{2.883251in}}%
\pgfpathlineto{\pgfqpoint{5.491162in}{2.884065in}}%
\pgfpathlineto{\pgfqpoint{5.492004in}{2.886618in}}%
\pgfpathlineto{\pgfqpoint{5.492510in}{2.885201in}}%
\pgfpathlineto{\pgfqpoint{5.494532in}{2.879463in}}%
\pgfpathlineto{\pgfqpoint{5.494784in}{2.879548in}}%
\pgfpathlineto{\pgfqpoint{5.495037in}{2.880175in}}%
\pgfpathlineto{\pgfqpoint{5.495627in}{2.881543in}}%
\pgfpathlineto{\pgfqpoint{5.496132in}{2.880132in}}%
\pgfpathlineto{\pgfqpoint{5.499418in}{2.872229in}}%
\pgfpathlineto{\pgfqpoint{5.496806in}{2.880645in}}%
\pgfpathlineto{\pgfqpoint{5.499755in}{2.873087in}}%
\pgfpathlineto{\pgfqpoint{5.502029in}{2.883105in}}%
\pgfpathlineto{\pgfqpoint{5.502535in}{2.882932in}}%
\pgfpathlineto{\pgfqpoint{5.503293in}{2.884718in}}%
\pgfpathlineto{\pgfqpoint{5.504051in}{2.885931in}}%
\pgfpathlineto{\pgfqpoint{5.504725in}{2.885581in}}%
\pgfpathlineto{\pgfqpoint{5.505652in}{2.884450in}}%
\pgfpathlineto{\pgfqpoint{5.507084in}{2.881154in}}%
\pgfpathlineto{\pgfqpoint{5.507421in}{2.881356in}}%
\pgfpathlineto{\pgfqpoint{5.507842in}{2.881596in}}%
\pgfpathlineto{\pgfqpoint{5.508179in}{2.880799in}}%
\pgfpathlineto{\pgfqpoint{5.512307in}{2.865454in}}%
\pgfpathlineto{\pgfqpoint{5.513065in}{2.867166in}}%
\pgfpathlineto{\pgfqpoint{5.513487in}{2.865803in}}%
\pgfpathlineto{\pgfqpoint{5.514076in}{2.864012in}}%
\pgfpathlineto{\pgfqpoint{5.514750in}{2.864700in}}%
\pgfpathlineto{\pgfqpoint{5.515930in}{2.865954in}}%
\pgfpathlineto{\pgfqpoint{5.516267in}{2.866072in}}%
\pgfpathlineto{\pgfqpoint{5.516519in}{2.865458in}}%
\pgfpathlineto{\pgfqpoint{5.518625in}{2.854555in}}%
\pgfpathlineto{\pgfqpoint{5.519299in}{2.856902in}}%
\pgfpathlineto{\pgfqpoint{5.519636in}{2.857445in}}%
\pgfpathlineto{\pgfqpoint{5.520142in}{2.856378in}}%
\pgfpathlineto{\pgfqpoint{5.520984in}{2.855301in}}%
\pgfpathlineto{\pgfqpoint{5.521574in}{2.855571in}}%
\pgfpathlineto{\pgfqpoint{5.524354in}{2.855540in}}%
\pgfpathlineto{\pgfqpoint{5.526544in}{2.853590in}}%
\pgfpathlineto{\pgfqpoint{5.528988in}{2.854090in}}%
\pgfpathlineto{\pgfqpoint{5.530672in}{2.858179in}}%
\pgfpathlineto{\pgfqpoint{5.531009in}{2.857377in}}%
\pgfpathlineto{\pgfqpoint{5.532610in}{2.854545in}}%
\pgfpathlineto{\pgfqpoint{5.532779in}{2.854772in}}%
\pgfpathlineto{\pgfqpoint{5.533453in}{2.860064in}}%
\pgfpathlineto{\pgfqpoint{5.534969in}{2.865205in}}%
\pgfpathlineto{\pgfqpoint{5.535390in}{2.865476in}}%
\pgfpathlineto{\pgfqpoint{5.535727in}{2.864679in}}%
\pgfpathlineto{\pgfqpoint{5.536654in}{2.862180in}}%
\pgfpathlineto{\pgfqpoint{5.537159in}{2.863369in}}%
\pgfpathlineto{\pgfqpoint{5.538928in}{2.865666in}}%
\pgfpathlineto{\pgfqpoint{5.542130in}{2.870750in}}%
\pgfpathlineto{\pgfqpoint{5.544657in}{2.874273in}}%
\pgfpathlineto{\pgfqpoint{5.545331in}{2.875732in}}%
\pgfpathlineto{\pgfqpoint{5.545921in}{2.878047in}}%
\pgfpathlineto{\pgfqpoint{5.546510in}{2.876323in}}%
\pgfpathlineto{\pgfqpoint{5.547606in}{2.872249in}}%
\pgfpathlineto{\pgfqpoint{5.548280in}{2.872850in}}%
\pgfpathlineto{\pgfqpoint{5.549627in}{2.873192in}}%
\pgfpathlineto{\pgfqpoint{5.549796in}{2.872805in}}%
\pgfpathlineto{\pgfqpoint{5.552576in}{2.865433in}}%
\pgfpathlineto{\pgfqpoint{5.553503in}{2.866264in}}%
\pgfpathlineto{\pgfqpoint{5.554008in}{2.865691in}}%
\pgfpathlineto{\pgfqpoint{5.554092in}{2.865475in}}%
\pgfpathlineto{\pgfqpoint{5.554851in}{2.864308in}}%
\pgfpathlineto{\pgfqpoint{5.555356in}{2.865043in}}%
\pgfpathlineto{\pgfqpoint{5.556114in}{2.866379in}}%
\pgfpathlineto{\pgfqpoint{5.556704in}{2.865586in}}%
\pgfpathlineto{\pgfqpoint{5.557209in}{2.866007in}}%
\pgfpathlineto{\pgfqpoint{5.557378in}{2.866336in}}%
\pgfpathlineto{\pgfqpoint{5.557968in}{2.866941in}}%
\pgfpathlineto{\pgfqpoint{5.558473in}{2.866211in}}%
\pgfpathlineto{\pgfqpoint{5.559821in}{2.863655in}}%
\pgfpathlineto{\pgfqpoint{5.560495in}{2.864904in}}%
\pgfpathlineto{\pgfqpoint{5.561422in}{2.867695in}}%
\pgfpathlineto{\pgfqpoint{5.562011in}{2.870061in}}%
\pgfpathlineto{\pgfqpoint{5.562517in}{2.868179in}}%
\pgfpathlineto{\pgfqpoint{5.563444in}{2.865116in}}%
\pgfpathlineto{\pgfqpoint{5.564033in}{2.865364in}}%
\pgfpathlineto{\pgfqpoint{5.564876in}{2.864723in}}%
\pgfpathlineto{\pgfqpoint{5.565213in}{2.865488in}}%
\pgfpathlineto{\pgfqpoint{5.567150in}{2.869560in}}%
\pgfpathlineto{\pgfqpoint{5.567403in}{2.869181in}}%
\pgfpathlineto{\pgfqpoint{5.568077in}{2.867704in}}%
\pgfpathlineto{\pgfqpoint{5.568582in}{2.869167in}}%
\pgfpathlineto{\pgfqpoint{5.570604in}{2.875171in}}%
\pgfpathlineto{\pgfqpoint{5.570857in}{2.875029in}}%
\pgfpathlineto{\pgfqpoint{5.571110in}{2.875165in}}%
\pgfpathlineto{\pgfqpoint{5.571278in}{2.875771in}}%
\pgfpathlineto{\pgfqpoint{5.573721in}{2.887710in}}%
\pgfpathlineto{\pgfqpoint{5.574311in}{2.886540in}}%
\pgfpathlineto{\pgfqpoint{5.574732in}{2.885846in}}%
\pgfpathlineto{\pgfqpoint{5.575154in}{2.887140in}}%
\pgfpathlineto{\pgfqpoint{5.576670in}{2.891531in}}%
\pgfpathlineto{\pgfqpoint{5.577007in}{2.891230in}}%
\pgfpathlineto{\pgfqpoint{5.577681in}{2.892091in}}%
\pgfpathlineto{\pgfqpoint{5.577934in}{2.892075in}}%
\pgfpathlineto{\pgfqpoint{5.578186in}{2.891316in}}%
\pgfpathlineto{\pgfqpoint{5.578776in}{2.888898in}}%
\pgfpathlineto{\pgfqpoint{5.579282in}{2.891248in}}%
\pgfpathlineto{\pgfqpoint{5.581809in}{2.901214in}}%
\pgfpathlineto{\pgfqpoint{5.582314in}{2.900244in}}%
\pgfpathlineto{\pgfqpoint{5.584336in}{2.896597in}}%
\pgfpathlineto{\pgfqpoint{5.584842in}{2.897693in}}%
\pgfpathlineto{\pgfqpoint{5.585516in}{2.899267in}}%
\pgfpathlineto{\pgfqpoint{5.586190in}{2.898605in}}%
\pgfpathlineto{\pgfqpoint{5.586864in}{2.896778in}}%
\pgfpathlineto{\pgfqpoint{5.588970in}{2.884906in}}%
\pgfpathlineto{\pgfqpoint{5.589896in}{2.879515in}}%
\pgfpathlineto{\pgfqpoint{5.590318in}{2.880743in}}%
\pgfpathlineto{\pgfqpoint{5.591497in}{2.887590in}}%
\pgfpathlineto{\pgfqpoint{5.592087in}{2.885280in}}%
\pgfpathlineto{\pgfqpoint{5.593266in}{2.881552in}}%
\pgfpathlineto{\pgfqpoint{5.593687in}{2.882038in}}%
\pgfpathlineto{\pgfqpoint{5.594024in}{2.882388in}}%
\pgfpathlineto{\pgfqpoint{5.594446in}{2.881460in}}%
\pgfpathlineto{\pgfqpoint{5.599163in}{2.869688in}}%
\pgfpathlineto{\pgfqpoint{5.600090in}{2.868431in}}%
\pgfpathlineto{\pgfqpoint{5.603375in}{2.859447in}}%
\pgfpathlineto{\pgfqpoint{5.605819in}{2.858340in}}%
\pgfpathlineto{\pgfqpoint{5.606998in}{2.857654in}}%
\pgfpathlineto{\pgfqpoint{5.608177in}{2.851039in}}%
\pgfpathlineto{\pgfqpoint{5.609104in}{2.852913in}}%
\pgfpathlineto{\pgfqpoint{5.609947in}{2.853985in}}%
\pgfpathlineto{\pgfqpoint{5.610368in}{2.853107in}}%
\pgfpathlineto{\pgfqpoint{5.611379in}{2.850730in}}%
\pgfpathlineto{\pgfqpoint{5.611800in}{2.851686in}}%
\pgfpathlineto{\pgfqpoint{5.612305in}{2.852892in}}%
\pgfpathlineto{\pgfqpoint{5.612895in}{2.851666in}}%
\pgfpathlineto{\pgfqpoint{5.616349in}{2.846451in}}%
\pgfpathlineto{\pgfqpoint{5.617107in}{2.847540in}}%
\pgfpathlineto{\pgfqpoint{5.618034in}{2.848714in}}%
\pgfpathlineto{\pgfqpoint{5.618540in}{2.848539in}}%
\pgfpathlineto{\pgfqpoint{5.620393in}{2.849498in}}%
\pgfpathlineto{\pgfqpoint{5.623763in}{2.857705in}}%
\pgfpathlineto{\pgfqpoint{5.624268in}{2.857101in}}%
\pgfpathlineto{\pgfqpoint{5.628312in}{2.851981in}}%
\pgfpathlineto{\pgfqpoint{5.629913in}{2.851498in}}%
\pgfpathlineto{\pgfqpoint{5.629997in}{2.851619in}}%
\pgfpathlineto{\pgfqpoint{5.631597in}{2.856801in}}%
\pgfpathlineto{\pgfqpoint{5.633788in}{2.866830in}}%
\pgfpathlineto{\pgfqpoint{5.634714in}{2.866459in}}%
\pgfpathlineto{\pgfqpoint{5.636315in}{2.867233in}}%
\pgfpathlineto{\pgfqpoint{5.637495in}{2.866208in}}%
\pgfpathlineto{\pgfqpoint{5.639938in}{2.862579in}}%
\pgfpathlineto{\pgfqpoint{5.640275in}{2.862570in}}%
\pgfpathlineto{\pgfqpoint{5.640612in}{2.863342in}}%
\pgfpathlineto{\pgfqpoint{5.642296in}{2.865006in}}%
\pgfpathlineto{\pgfqpoint{5.647351in}{2.864563in}}%
\pgfpathlineto{\pgfqpoint{5.648025in}{2.862281in}}%
\pgfpathlineto{\pgfqpoint{5.651395in}{2.852378in}}%
\pgfpathlineto{\pgfqpoint{5.655691in}{2.852686in}}%
\pgfpathlineto{\pgfqpoint{5.657123in}{2.851682in}}%
\pgfpathlineto{\pgfqpoint{5.657376in}{2.851869in}}%
\pgfpathlineto{\pgfqpoint{5.660325in}{2.856168in}}%
\pgfpathlineto{\pgfqpoint{5.662431in}{2.861600in}}%
\pgfpathlineto{\pgfqpoint{5.663273in}{2.860900in}}%
\pgfpathlineto{\pgfqpoint{5.664621in}{2.862123in}}%
\pgfpathlineto{\pgfqpoint{5.665379in}{2.862835in}}%
\pgfpathlineto{\pgfqpoint{5.665885in}{2.862196in}}%
\pgfpathlineto{\pgfqpoint{5.668833in}{2.859460in}}%
\pgfpathlineto{\pgfqpoint{5.668918in}{2.859539in}}%
\pgfpathlineto{\pgfqpoint{5.673635in}{2.868638in}}%
\pgfpathlineto{\pgfqpoint{5.674057in}{2.869288in}}%
\pgfpathlineto{\pgfqpoint{5.674562in}{2.868141in}}%
\pgfpathlineto{\pgfqpoint{5.674899in}{2.867979in}}%
\pgfpathlineto{\pgfqpoint{5.675320in}{2.868826in}}%
\pgfpathlineto{\pgfqpoint{5.676079in}{2.869840in}}%
\pgfpathlineto{\pgfqpoint{5.676500in}{2.868974in}}%
\pgfpathlineto{\pgfqpoint{5.678606in}{2.866411in}}%
\pgfpathlineto{\pgfqpoint{5.678774in}{2.866565in}}%
\pgfpathlineto{\pgfqpoint{5.683071in}{2.873290in}}%
\pgfpathlineto{\pgfqpoint{5.684840in}{2.874611in}}%
\pgfpathlineto{\pgfqpoint{5.685598in}{2.876589in}}%
\pgfpathlineto{\pgfqpoint{5.686188in}{2.875195in}}%
\pgfpathlineto{\pgfqpoint{5.689473in}{2.866585in}}%
\pgfpathlineto{\pgfqpoint{5.689810in}{2.867030in}}%
\pgfpathlineto{\pgfqpoint{5.690484in}{2.868118in}}%
\pgfpathlineto{\pgfqpoint{5.691074in}{2.867327in}}%
\pgfpathlineto{\pgfqpoint{5.693938in}{2.862309in}}%
\pgfpathlineto{\pgfqpoint{5.694444in}{2.864034in}}%
\pgfpathlineto{\pgfqpoint{5.695034in}{2.865535in}}%
\pgfpathlineto{\pgfqpoint{5.695623in}{2.864356in}}%
\pgfpathlineto{\pgfqpoint{5.696887in}{2.859369in}}%
\pgfpathlineto{\pgfqpoint{5.697729in}{2.860134in}}%
\pgfpathlineto{\pgfqpoint{5.697982in}{2.860083in}}%
\pgfpathlineto{\pgfqpoint{5.698235in}{2.860748in}}%
\pgfpathlineto{\pgfqpoint{5.699077in}{2.863832in}}%
\pgfpathlineto{\pgfqpoint{5.699920in}{2.863568in}}%
\pgfpathlineto{\pgfqpoint{5.700762in}{2.866914in}}%
\pgfpathlineto{\pgfqpoint{5.701352in}{2.868819in}}%
\pgfpathlineto{\pgfqpoint{5.701942in}{2.867568in}}%
\pgfpathlineto{\pgfqpoint{5.702700in}{2.866077in}}%
\pgfpathlineto{\pgfqpoint{5.703374in}{2.866658in}}%
\pgfpathlineto{\pgfqpoint{5.705311in}{2.866055in}}%
\pgfpathlineto{\pgfqpoint{5.706744in}{2.864946in}}%
\pgfpathlineto{\pgfqpoint{5.706996in}{2.865909in}}%
\pgfpathlineto{\pgfqpoint{5.708260in}{2.873098in}}%
\pgfpathlineto{\pgfqpoint{5.709018in}{2.872573in}}%
\pgfpathlineto{\pgfqpoint{5.713230in}{2.871570in}}%
\pgfpathlineto{\pgfqpoint{5.715336in}{2.867914in}}%
\pgfpathlineto{\pgfqpoint{5.717106in}{2.863274in}}%
\pgfpathlineto{\pgfqpoint{5.719043in}{2.859382in}}%
\pgfpathlineto{\pgfqpoint{5.719464in}{2.859784in}}%
\pgfpathlineto{\pgfqpoint{5.723340in}{2.864944in}}%
\pgfpathlineto{\pgfqpoint{5.723508in}{2.864657in}}%
\pgfpathlineto{\pgfqpoint{5.724603in}{2.856001in}}%
\pgfpathlineto{\pgfqpoint{5.725699in}{2.859152in}}%
\pgfpathlineto{\pgfqpoint{5.727552in}{2.861911in}}%
\pgfpathlineto{\pgfqpoint{5.728057in}{2.860694in}}%
\pgfpathlineto{\pgfqpoint{5.728731in}{2.858886in}}%
\pgfpathlineto{\pgfqpoint{5.729321in}{2.859975in}}%
\pgfpathlineto{\pgfqpoint{5.731343in}{2.864105in}}%
\pgfpathlineto{\pgfqpoint{5.731596in}{2.863909in}}%
\pgfpathlineto{\pgfqpoint{5.732859in}{2.862723in}}%
\pgfpathlineto{\pgfqpoint{5.733281in}{2.863390in}}%
\pgfpathlineto{\pgfqpoint{5.734881in}{2.871874in}}%
\pgfpathlineto{\pgfqpoint{5.736229in}{2.877632in}}%
\pgfpathlineto{\pgfqpoint{5.736566in}{2.877279in}}%
\pgfpathlineto{\pgfqpoint{5.737409in}{2.877082in}}%
\pgfpathlineto{\pgfqpoint{5.737661in}{2.877683in}}%
\pgfpathlineto{\pgfqpoint{5.740947in}{2.889060in}}%
\pgfpathlineto{\pgfqpoint{5.741115in}{2.888903in}}%
\pgfpathlineto{\pgfqpoint{5.743390in}{2.883223in}}%
\pgfpathlineto{\pgfqpoint{5.743811in}{2.885312in}}%
\pgfpathlineto{\pgfqpoint{5.744906in}{2.892453in}}%
\pgfpathlineto{\pgfqpoint{5.745580in}{2.891511in}}%
\pgfpathlineto{\pgfqpoint{5.745833in}{2.890818in}}%
\pgfpathlineto{\pgfqpoint{5.745833in}{2.890818in}}%
\pgfusepath{stroke}%
\end{pgfscope}%
\begin{pgfscope}%
\pgfpathrectangle{\pgfqpoint{0.691161in}{2.672776in}}{\pgfqpoint{5.054672in}{0.902317in}}%
\pgfusepath{clip}%
\pgfsetbuttcap%
\pgfsetroundjoin%
\pgfsetlinewidth{2.007500pt}%
\definecolor{currentstroke}{rgb}{0.172549,0.627451,0.172549}%
\pgfsetstrokecolor{currentstroke}%
\pgfsetdash{{7.400000pt}{3.200000pt}}{0.000000pt}%
\pgfpathmoveto{\pgfqpoint{0.691161in}{3.063859in}}%
\pgfpathlineto{\pgfqpoint{5.745833in}{3.063859in}}%
\pgfusepath{stroke}%
\end{pgfscope}%
\begin{pgfscope}%
\pgfpathrectangle{\pgfqpoint{0.691161in}{2.672776in}}{\pgfqpoint{5.054672in}{0.902317in}}%
\pgfusepath{clip}%
\pgfsetbuttcap%
\pgfsetroundjoin%
\pgfsetlinewidth{2.007500pt}%
\definecolor{currentstroke}{rgb}{0.839216,0.152941,0.156863}%
\pgfsetstrokecolor{currentstroke}%
\pgfsetdash{{2.000000pt}{3.300000pt}}{0.000000pt}%
\pgfpathmoveto{\pgfqpoint{1.609342in}{2.713790in}}%
\pgfpathlineto{\pgfqpoint{1.609342in}{3.534079in}}%
\pgfusepath{stroke}%
\end{pgfscope}%
\begin{pgfscope}%
\pgfpathrectangle{\pgfqpoint{0.691161in}{2.672776in}}{\pgfqpoint{5.054672in}{0.902317in}}%
\pgfusepath{clip}%
\pgfsetbuttcap%
\pgfsetroundjoin%
\pgfsetlinewidth{2.007500pt}%
\definecolor{currentstroke}{rgb}{1.000000,0.498039,0.054902}%
\pgfsetstrokecolor{currentstroke}%
\pgfsetdash{{2.000000pt}{3.300000pt}}{0.000000pt}%
\pgfpathmoveto{\pgfqpoint{3.455477in}{2.713790in}}%
\pgfpathlineto{\pgfqpoint{3.455477in}{3.534079in}}%
\pgfusepath{stroke}%
\end{pgfscope}%
\begin{pgfscope}%
\pgfsetrectcap%
\pgfsetmiterjoin%
\pgfsetlinewidth{0.803000pt}%
\definecolor{currentstroke}{rgb}{0.737255,0.737255,0.737255}%
\pgfsetstrokecolor{currentstroke}%
\pgfsetdash{}{0pt}%
\pgfpathmoveto{\pgfqpoint{0.691161in}{2.672776in}}%
\pgfpathlineto{\pgfqpoint{0.691161in}{3.575093in}}%
\pgfusepath{stroke}%
\end{pgfscope}%
\begin{pgfscope}%
\pgfsetrectcap%
\pgfsetmiterjoin%
\pgfsetlinewidth{0.803000pt}%
\definecolor{currentstroke}{rgb}{0.737255,0.737255,0.737255}%
\pgfsetstrokecolor{currentstroke}%
\pgfsetdash{}{0pt}%
\pgfpathmoveto{\pgfqpoint{5.745833in}{2.672776in}}%
\pgfpathlineto{\pgfqpoint{5.745833in}{3.575093in}}%
\pgfusepath{stroke}%
\end{pgfscope}%
\begin{pgfscope}%
\pgfsetrectcap%
\pgfsetmiterjoin%
\pgfsetlinewidth{0.803000pt}%
\definecolor{currentstroke}{rgb}{0.737255,0.737255,0.737255}%
\pgfsetstrokecolor{currentstroke}%
\pgfsetdash{}{0pt}%
\pgfpathmoveto{\pgfqpoint{0.691161in}{2.672776in}}%
\pgfpathlineto{\pgfqpoint{5.745833in}{2.672776in}}%
\pgfusepath{stroke}%
\end{pgfscope}%
\begin{pgfscope}%
\pgfsetrectcap%
\pgfsetmiterjoin%
\pgfsetlinewidth{0.803000pt}%
\definecolor{currentstroke}{rgb}{0.737255,0.737255,0.737255}%
\pgfsetstrokecolor{currentstroke}%
\pgfsetdash{}{0pt}%
\pgfpathmoveto{\pgfqpoint{0.691161in}{3.575093in}}%
\pgfpathlineto{\pgfqpoint{5.745833in}{3.575093in}}%
\pgfusepath{stroke}%
\end{pgfscope}%
\begin{pgfscope}%
\pgfsetbuttcap%
\pgfsetmiterjoin%
\definecolor{currentfill}{rgb}{0.933333,0.933333,0.933333}%
\pgfsetfillcolor{currentfill}%
\pgfsetfillopacity{0.800000}%
\pgfsetlinewidth{0.501875pt}%
\definecolor{currentstroke}{rgb}{0.800000,0.800000,0.800000}%
\pgfsetstrokecolor{currentstroke}%
\pgfsetstrokeopacity{0.800000}%
\pgfsetdash{}{0pt}%
\pgfpathmoveto{\pgfqpoint{4.404343in}{2.883149in}}%
\pgfpathlineto{\pgfqpoint{5.648611in}{2.883149in}}%
\pgfpathquadraticcurveto{\pgfqpoint{5.676389in}{2.883149in}}{\pgfqpoint{5.676389in}{2.910927in}}%
\pgfpathlineto{\pgfqpoint{5.676389in}{3.477871in}}%
\pgfpathquadraticcurveto{\pgfqpoint{5.676389in}{3.505649in}}{\pgfqpoint{5.648611in}{3.505649in}}%
\pgfpathlineto{\pgfqpoint{4.404343in}{3.505649in}}%
\pgfpathquadraticcurveto{\pgfqpoint{4.376566in}{3.505649in}}{\pgfqpoint{4.376566in}{3.477871in}}%
\pgfpathlineto{\pgfqpoint{4.376566in}{2.910927in}}%
\pgfpathquadraticcurveto{\pgfqpoint{4.376566in}{2.883149in}}{\pgfqpoint{4.404343in}{2.883149in}}%
\pgfpathlineto{\pgfqpoint{4.404343in}{2.883149in}}%
\pgfpathclose%
\pgfusepath{stroke,fill}%
\end{pgfscope}%
\begin{pgfscope}%
\pgfsetbuttcap%
\pgfsetroundjoin%
\pgfsetlinewidth{2.007500pt}%
\definecolor{currentstroke}{rgb}{0.172549,0.627451,0.172549}%
\pgfsetstrokecolor{currentstroke}%
\pgfsetdash{{7.400000pt}{3.200000pt}}{0.000000pt}%
\pgfpathmoveto{\pgfqpoint{4.432121in}{3.401482in}}%
\pgfpathlineto{\pgfqpoint{4.709899in}{3.401482in}}%
\pgfusepath{stroke}%
\end{pgfscope}%
\begin{pgfscope}%
\definecolor{textcolor}{rgb}{0.000000,0.000000,0.000000}%
\pgfsetstrokecolor{textcolor}%
\pgfsetfillcolor{textcolor}%
\pgftext[x=4.821010in,y=3.352871in,left,base]{\color{textcolor}\rmfamily\fontsize{10.000000}{12.000000}\selectfont Seuil = 5}%
\end{pgfscope}%
\begin{pgfscope}%
\pgfsetbuttcap%
\pgfsetroundjoin%
\pgfsetlinewidth{2.007500pt}%
\definecolor{currentstroke}{rgb}{0.839216,0.152941,0.156863}%
\pgfsetstrokecolor{currentstroke}%
\pgfsetdash{{2.000000pt}{3.300000pt}}{0.000000pt}%
\pgfpathmoveto{\pgfqpoint{4.432121in}{3.207871in}}%
\pgfpathlineto{\pgfqpoint{4.709899in}{3.207871in}}%
\pgfusepath{stroke}%
\end{pgfscope}%
\begin{pgfscope}%
\definecolor{textcolor}{rgb}{0.000000,0.000000,0.000000}%
\pgfsetstrokecolor{textcolor}%
\pgfsetfillcolor{textcolor}%
\pgftext[x=4.821010in,y=3.159260in,left,base]{\color{textcolor}\rmfamily\fontsize{10.000000}{12.000000}\selectfont \(\displaystyle t_1\) = 54.49 s}%
\end{pgfscope}%
\begin{pgfscope}%
\pgfsetbuttcap%
\pgfsetroundjoin%
\pgfsetlinewidth{2.007500pt}%
\definecolor{currentstroke}{rgb}{1.000000,0.498039,0.054902}%
\pgfsetstrokecolor{currentstroke}%
\pgfsetdash{{2.000000pt}{3.300000pt}}{0.000000pt}%
\pgfpathmoveto{\pgfqpoint{4.432121in}{3.014260in}}%
\pgfpathlineto{\pgfqpoint{4.709899in}{3.014260in}}%
\pgfusepath{stroke}%
\end{pgfscope}%
\begin{pgfscope}%
\definecolor{textcolor}{rgb}{0.000000,0.000000,0.000000}%
\pgfsetstrokecolor{textcolor}%
\pgfsetfillcolor{textcolor}%
\pgftext[x=4.821010in,y=2.965649in,left,base]{\color{textcolor}\rmfamily\fontsize{10.000000}{12.000000}\selectfont \(\displaystyle t_2\) = 164.06 s}%
\end{pgfscope}%
\begin{pgfscope}%
\pgfsetbuttcap%
\pgfsetmiterjoin%
\definecolor{currentfill}{rgb}{0.933333,0.933333,0.933333}%
\pgfsetfillcolor{currentfill}%
\pgfsetlinewidth{0.000000pt}%
\definecolor{currentstroke}{rgb}{0.000000,0.000000,0.000000}%
\pgfsetstrokecolor{currentstroke}%
\pgfsetstrokeopacity{0.000000}%
\pgfsetdash{}{0pt}%
\pgfpathmoveto{\pgfqpoint{0.691161in}{1.608471in}}%
\pgfpathlineto{\pgfqpoint{5.745833in}{1.608471in}}%
\pgfpathlineto{\pgfqpoint{5.745833in}{2.510788in}}%
\pgfpathlineto{\pgfqpoint{0.691161in}{2.510788in}}%
\pgfpathlineto{\pgfqpoint{0.691161in}{1.608471in}}%
\pgfpathclose%
\pgfusepath{fill}%
\end{pgfscope}%
\begin{pgfscope}%
\pgfpathrectangle{\pgfqpoint{0.691161in}{1.608471in}}{\pgfqpoint{5.054672in}{0.902317in}}%
\pgfusepath{clip}%
\pgfsetbuttcap%
\pgfsetroundjoin%
\pgfsetlinewidth{0.501875pt}%
\definecolor{currentstroke}{rgb}{0.698039,0.698039,0.698039}%
\pgfsetstrokecolor{currentstroke}%
\pgfsetdash{{1.850000pt}{0.800000pt}}{0.000000pt}%
\pgfpathmoveto{\pgfqpoint{0.691161in}{1.608471in}}%
\pgfpathlineto{\pgfqpoint{0.691161in}{2.510788in}}%
\pgfusepath{stroke}%
\end{pgfscope}%
\begin{pgfscope}%
\pgfsetbuttcap%
\pgfsetroundjoin%
\definecolor{currentfill}{rgb}{0.000000,0.000000,0.000000}%
\pgfsetfillcolor{currentfill}%
\pgfsetlinewidth{0.803000pt}%
\definecolor{currentstroke}{rgb}{0.000000,0.000000,0.000000}%
\pgfsetstrokecolor{currentstroke}%
\pgfsetdash{}{0pt}%
\pgfsys@defobject{currentmarker}{\pgfqpoint{0.000000in}{0.000000in}}{\pgfqpoint{0.000000in}{0.048611in}}{%
\pgfpathmoveto{\pgfqpoint{0.000000in}{0.000000in}}%
\pgfpathlineto{\pgfqpoint{0.000000in}{0.048611in}}%
\pgfusepath{stroke,fill}%
}%
\begin{pgfscope}%
\pgfsys@transformshift{0.691161in}{1.608471in}%
\pgfsys@useobject{currentmarker}{}%
\end{pgfscope}%
\end{pgfscope}%
\begin{pgfscope}%
\pgfpathrectangle{\pgfqpoint{0.691161in}{1.608471in}}{\pgfqpoint{5.054672in}{0.902317in}}%
\pgfusepath{clip}%
\pgfsetbuttcap%
\pgfsetroundjoin%
\pgfsetlinewidth{0.501875pt}%
\definecolor{currentstroke}{rgb}{0.698039,0.698039,0.698039}%
\pgfsetstrokecolor{currentstroke}%
\pgfsetdash{{1.850000pt}{0.800000pt}}{0.000000pt}%
\pgfpathmoveto{\pgfqpoint{1.533607in}{1.608471in}}%
\pgfpathlineto{\pgfqpoint{1.533607in}{2.510788in}}%
\pgfusepath{stroke}%
\end{pgfscope}%
\begin{pgfscope}%
\pgfsetbuttcap%
\pgfsetroundjoin%
\definecolor{currentfill}{rgb}{0.000000,0.000000,0.000000}%
\pgfsetfillcolor{currentfill}%
\pgfsetlinewidth{0.803000pt}%
\definecolor{currentstroke}{rgb}{0.000000,0.000000,0.000000}%
\pgfsetstrokecolor{currentstroke}%
\pgfsetdash{}{0pt}%
\pgfsys@defobject{currentmarker}{\pgfqpoint{0.000000in}{0.000000in}}{\pgfqpoint{0.000000in}{0.048611in}}{%
\pgfpathmoveto{\pgfqpoint{0.000000in}{0.000000in}}%
\pgfpathlineto{\pgfqpoint{0.000000in}{0.048611in}}%
\pgfusepath{stroke,fill}%
}%
\begin{pgfscope}%
\pgfsys@transformshift{1.533607in}{1.608471in}%
\pgfsys@useobject{currentmarker}{}%
\end{pgfscope}%
\end{pgfscope}%
\begin{pgfscope}%
\pgfpathrectangle{\pgfqpoint{0.691161in}{1.608471in}}{\pgfqpoint{5.054672in}{0.902317in}}%
\pgfusepath{clip}%
\pgfsetbuttcap%
\pgfsetroundjoin%
\pgfsetlinewidth{0.501875pt}%
\definecolor{currentstroke}{rgb}{0.698039,0.698039,0.698039}%
\pgfsetstrokecolor{currentstroke}%
\pgfsetdash{{1.850000pt}{0.800000pt}}{0.000000pt}%
\pgfpathmoveto{\pgfqpoint{2.376052in}{1.608471in}}%
\pgfpathlineto{\pgfqpoint{2.376052in}{2.510788in}}%
\pgfusepath{stroke}%
\end{pgfscope}%
\begin{pgfscope}%
\pgfsetbuttcap%
\pgfsetroundjoin%
\definecolor{currentfill}{rgb}{0.000000,0.000000,0.000000}%
\pgfsetfillcolor{currentfill}%
\pgfsetlinewidth{0.803000pt}%
\definecolor{currentstroke}{rgb}{0.000000,0.000000,0.000000}%
\pgfsetstrokecolor{currentstroke}%
\pgfsetdash{}{0pt}%
\pgfsys@defobject{currentmarker}{\pgfqpoint{0.000000in}{0.000000in}}{\pgfqpoint{0.000000in}{0.048611in}}{%
\pgfpathmoveto{\pgfqpoint{0.000000in}{0.000000in}}%
\pgfpathlineto{\pgfqpoint{0.000000in}{0.048611in}}%
\pgfusepath{stroke,fill}%
}%
\begin{pgfscope}%
\pgfsys@transformshift{2.376052in}{1.608471in}%
\pgfsys@useobject{currentmarker}{}%
\end{pgfscope}%
\end{pgfscope}%
\begin{pgfscope}%
\pgfpathrectangle{\pgfqpoint{0.691161in}{1.608471in}}{\pgfqpoint{5.054672in}{0.902317in}}%
\pgfusepath{clip}%
\pgfsetbuttcap%
\pgfsetroundjoin%
\pgfsetlinewidth{0.501875pt}%
\definecolor{currentstroke}{rgb}{0.698039,0.698039,0.698039}%
\pgfsetstrokecolor{currentstroke}%
\pgfsetdash{{1.850000pt}{0.800000pt}}{0.000000pt}%
\pgfpathmoveto{\pgfqpoint{3.218497in}{1.608471in}}%
\pgfpathlineto{\pgfqpoint{3.218497in}{2.510788in}}%
\pgfusepath{stroke}%
\end{pgfscope}%
\begin{pgfscope}%
\pgfsetbuttcap%
\pgfsetroundjoin%
\definecolor{currentfill}{rgb}{0.000000,0.000000,0.000000}%
\pgfsetfillcolor{currentfill}%
\pgfsetlinewidth{0.803000pt}%
\definecolor{currentstroke}{rgb}{0.000000,0.000000,0.000000}%
\pgfsetstrokecolor{currentstroke}%
\pgfsetdash{}{0pt}%
\pgfsys@defobject{currentmarker}{\pgfqpoint{0.000000in}{0.000000in}}{\pgfqpoint{0.000000in}{0.048611in}}{%
\pgfpathmoveto{\pgfqpoint{0.000000in}{0.000000in}}%
\pgfpathlineto{\pgfqpoint{0.000000in}{0.048611in}}%
\pgfusepath{stroke,fill}%
}%
\begin{pgfscope}%
\pgfsys@transformshift{3.218497in}{1.608471in}%
\pgfsys@useobject{currentmarker}{}%
\end{pgfscope}%
\end{pgfscope}%
\begin{pgfscope}%
\pgfpathrectangle{\pgfqpoint{0.691161in}{1.608471in}}{\pgfqpoint{5.054672in}{0.902317in}}%
\pgfusepath{clip}%
\pgfsetbuttcap%
\pgfsetroundjoin%
\pgfsetlinewidth{0.501875pt}%
\definecolor{currentstroke}{rgb}{0.698039,0.698039,0.698039}%
\pgfsetstrokecolor{currentstroke}%
\pgfsetdash{{1.850000pt}{0.800000pt}}{0.000000pt}%
\pgfpathmoveto{\pgfqpoint{4.060942in}{1.608471in}}%
\pgfpathlineto{\pgfqpoint{4.060942in}{2.510788in}}%
\pgfusepath{stroke}%
\end{pgfscope}%
\begin{pgfscope}%
\pgfsetbuttcap%
\pgfsetroundjoin%
\definecolor{currentfill}{rgb}{0.000000,0.000000,0.000000}%
\pgfsetfillcolor{currentfill}%
\pgfsetlinewidth{0.803000pt}%
\definecolor{currentstroke}{rgb}{0.000000,0.000000,0.000000}%
\pgfsetstrokecolor{currentstroke}%
\pgfsetdash{}{0pt}%
\pgfsys@defobject{currentmarker}{\pgfqpoint{0.000000in}{0.000000in}}{\pgfqpoint{0.000000in}{0.048611in}}{%
\pgfpathmoveto{\pgfqpoint{0.000000in}{0.000000in}}%
\pgfpathlineto{\pgfqpoint{0.000000in}{0.048611in}}%
\pgfusepath{stroke,fill}%
}%
\begin{pgfscope}%
\pgfsys@transformshift{4.060942in}{1.608471in}%
\pgfsys@useobject{currentmarker}{}%
\end{pgfscope}%
\end{pgfscope}%
\begin{pgfscope}%
\pgfpathrectangle{\pgfqpoint{0.691161in}{1.608471in}}{\pgfqpoint{5.054672in}{0.902317in}}%
\pgfusepath{clip}%
\pgfsetbuttcap%
\pgfsetroundjoin%
\pgfsetlinewidth{0.501875pt}%
\definecolor{currentstroke}{rgb}{0.698039,0.698039,0.698039}%
\pgfsetstrokecolor{currentstroke}%
\pgfsetdash{{1.850000pt}{0.800000pt}}{0.000000pt}%
\pgfpathmoveto{\pgfqpoint{4.903388in}{1.608471in}}%
\pgfpathlineto{\pgfqpoint{4.903388in}{2.510788in}}%
\pgfusepath{stroke}%
\end{pgfscope}%
\begin{pgfscope}%
\pgfsetbuttcap%
\pgfsetroundjoin%
\definecolor{currentfill}{rgb}{0.000000,0.000000,0.000000}%
\pgfsetfillcolor{currentfill}%
\pgfsetlinewidth{0.803000pt}%
\definecolor{currentstroke}{rgb}{0.000000,0.000000,0.000000}%
\pgfsetstrokecolor{currentstroke}%
\pgfsetdash{}{0pt}%
\pgfsys@defobject{currentmarker}{\pgfqpoint{0.000000in}{0.000000in}}{\pgfqpoint{0.000000in}{0.048611in}}{%
\pgfpathmoveto{\pgfqpoint{0.000000in}{0.000000in}}%
\pgfpathlineto{\pgfqpoint{0.000000in}{0.048611in}}%
\pgfusepath{stroke,fill}%
}%
\begin{pgfscope}%
\pgfsys@transformshift{4.903388in}{1.608471in}%
\pgfsys@useobject{currentmarker}{}%
\end{pgfscope}%
\end{pgfscope}%
\begin{pgfscope}%
\pgfpathrectangle{\pgfqpoint{0.691161in}{1.608471in}}{\pgfqpoint{5.054672in}{0.902317in}}%
\pgfusepath{clip}%
\pgfsetbuttcap%
\pgfsetroundjoin%
\pgfsetlinewidth{0.501875pt}%
\definecolor{currentstroke}{rgb}{0.698039,0.698039,0.698039}%
\pgfsetstrokecolor{currentstroke}%
\pgfsetdash{{1.850000pt}{0.800000pt}}{0.000000pt}%
\pgfpathmoveto{\pgfqpoint{5.745833in}{1.608471in}}%
\pgfpathlineto{\pgfqpoint{5.745833in}{2.510788in}}%
\pgfusepath{stroke}%
\end{pgfscope}%
\begin{pgfscope}%
\pgfsetbuttcap%
\pgfsetroundjoin%
\definecolor{currentfill}{rgb}{0.000000,0.000000,0.000000}%
\pgfsetfillcolor{currentfill}%
\pgfsetlinewidth{0.803000pt}%
\definecolor{currentstroke}{rgb}{0.000000,0.000000,0.000000}%
\pgfsetstrokecolor{currentstroke}%
\pgfsetdash{}{0pt}%
\pgfsys@defobject{currentmarker}{\pgfqpoint{0.000000in}{0.000000in}}{\pgfqpoint{0.000000in}{0.048611in}}{%
\pgfpathmoveto{\pgfqpoint{0.000000in}{0.000000in}}%
\pgfpathlineto{\pgfqpoint{0.000000in}{0.048611in}}%
\pgfusepath{stroke,fill}%
}%
\begin{pgfscope}%
\pgfsys@transformshift{5.745833in}{1.608471in}%
\pgfsys@useobject{currentmarker}{}%
\end{pgfscope}%
\end{pgfscope}%
\begin{pgfscope}%
\pgfpathrectangle{\pgfqpoint{0.691161in}{1.608471in}}{\pgfqpoint{5.054672in}{0.902317in}}%
\pgfusepath{clip}%
\pgfsetbuttcap%
\pgfsetroundjoin%
\pgfsetlinewidth{0.501875pt}%
\definecolor{currentstroke}{rgb}{0.698039,0.698039,0.698039}%
\pgfsetstrokecolor{currentstroke}%
\pgfsetdash{{1.850000pt}{0.800000pt}}{0.000000pt}%
\pgfpathmoveto{\pgfqpoint{0.691161in}{1.766670in}}%
\pgfpathlineto{\pgfqpoint{5.745833in}{1.766670in}}%
\pgfusepath{stroke}%
\end{pgfscope}%
\begin{pgfscope}%
\pgfsetbuttcap%
\pgfsetroundjoin%
\definecolor{currentfill}{rgb}{0.000000,0.000000,0.000000}%
\pgfsetfillcolor{currentfill}%
\pgfsetlinewidth{0.803000pt}%
\definecolor{currentstroke}{rgb}{0.000000,0.000000,0.000000}%
\pgfsetstrokecolor{currentstroke}%
\pgfsetdash{}{0pt}%
\pgfsys@defobject{currentmarker}{\pgfqpoint{0.000000in}{0.000000in}}{\pgfqpoint{0.048611in}{0.000000in}}{%
\pgfpathmoveto{\pgfqpoint{0.000000in}{0.000000in}}%
\pgfpathlineto{\pgfqpoint{0.048611in}{0.000000in}}%
\pgfusepath{stroke,fill}%
}%
\begin{pgfscope}%
\pgfsys@transformshift{0.691161in}{1.766670in}%
\pgfsys@useobject{currentmarker}{}%
\end{pgfscope}%
\end{pgfscope}%
\begin{pgfscope}%
\definecolor{textcolor}{rgb}{0.000000,0.000000,0.000000}%
\pgfsetstrokecolor{textcolor}%
\pgfsetfillcolor{textcolor}%
\pgftext[x=0.573105in, y=1.718475in, left, base]{\color{textcolor}\rmfamily\fontsize{10.000000}{12.000000}\selectfont \(\displaystyle {0}\)}%
\end{pgfscope}%
\begin{pgfscope}%
\pgfpathrectangle{\pgfqpoint{0.691161in}{1.608471in}}{\pgfqpoint{5.054672in}{0.902317in}}%
\pgfusepath{clip}%
\pgfsetbuttcap%
\pgfsetroundjoin%
\pgfsetlinewidth{0.501875pt}%
\definecolor{currentstroke}{rgb}{0.698039,0.698039,0.698039}%
\pgfsetstrokecolor{currentstroke}%
\pgfsetdash{{1.850000pt}{0.800000pt}}{0.000000pt}%
\pgfpathmoveto{\pgfqpoint{0.691161in}{2.340100in}}%
\pgfpathlineto{\pgfqpoint{5.745833in}{2.340100in}}%
\pgfusepath{stroke}%
\end{pgfscope}%
\begin{pgfscope}%
\pgfsetbuttcap%
\pgfsetroundjoin%
\definecolor{currentfill}{rgb}{0.000000,0.000000,0.000000}%
\pgfsetfillcolor{currentfill}%
\pgfsetlinewidth{0.803000pt}%
\definecolor{currentstroke}{rgb}{0.000000,0.000000,0.000000}%
\pgfsetstrokecolor{currentstroke}%
\pgfsetdash{}{0pt}%
\pgfsys@defobject{currentmarker}{\pgfqpoint{0.000000in}{0.000000in}}{\pgfqpoint{0.048611in}{0.000000in}}{%
\pgfpathmoveto{\pgfqpoint{0.000000in}{0.000000in}}%
\pgfpathlineto{\pgfqpoint{0.048611in}{0.000000in}}%
\pgfusepath{stroke,fill}%
}%
\begin{pgfscope}%
\pgfsys@transformshift{0.691161in}{2.340100in}%
\pgfsys@useobject{currentmarker}{}%
\end{pgfscope}%
\end{pgfscope}%
\begin{pgfscope}%
\definecolor{textcolor}{rgb}{0.000000,0.000000,0.000000}%
\pgfsetstrokecolor{textcolor}%
\pgfsetfillcolor{textcolor}%
\pgftext[x=0.503661in, y=2.291906in, left, base]{\color{textcolor}\rmfamily\fontsize{10.000000}{12.000000}\selectfont \(\displaystyle {10}\)}%
\end{pgfscope}%
\begin{pgfscope}%
\definecolor{textcolor}{rgb}{0.000000,0.000000,0.000000}%
\pgfsetstrokecolor{textcolor}%
\pgfsetfillcolor{textcolor}%
\pgftext[x=0.448105in,y=2.059630in,,bottom,rotate=90.000000]{\color{textcolor}\rmfamily\fontsize{12.000000}{14.400000}\selectfont Récursif}%
\end{pgfscope}%
\begin{pgfscope}%
\pgfpathrectangle{\pgfqpoint{0.691161in}{1.608471in}}{\pgfqpoint{5.054672in}{0.902317in}}%
\pgfusepath{clip}%
\pgfsetrectcap%
\pgfsetroundjoin%
\pgfsetlinewidth{1.505625pt}%
\definecolor{currentstroke}{rgb}{0.121569,0.466667,0.705882}%
\pgfsetstrokecolor{currentstroke}%
\pgfsetdash{}{0pt}%
\pgfpathmoveto{\pgfqpoint{0.691161in}{1.766670in}}%
\pgfpathlineto{\pgfqpoint{1.028055in}{1.766670in}}%
\pgfpathlineto{\pgfqpoint{1.029235in}{1.877291in}}%
\pgfpathlineto{\pgfqpoint{1.030330in}{1.885652in}}%
\pgfpathlineto{\pgfqpoint{1.031341in}{1.883525in}}%
\pgfpathlineto{\pgfqpoint{1.032941in}{1.880678in}}%
\pgfpathlineto{\pgfqpoint{1.033026in}{1.880782in}}%
\pgfpathlineto{\pgfqpoint{1.033784in}{1.882863in}}%
\pgfpathlineto{\pgfqpoint{1.034205in}{1.881395in}}%
\pgfpathlineto{\pgfqpoint{1.037238in}{1.867473in}}%
\pgfpathlineto{\pgfqpoint{1.037575in}{1.868390in}}%
\pgfpathlineto{\pgfqpoint{1.038586in}{1.876415in}}%
\pgfpathlineto{\pgfqpoint{1.040186in}{1.886835in}}%
\pgfpathlineto{\pgfqpoint{1.040439in}{1.886361in}}%
\pgfpathlineto{\pgfqpoint{1.044651in}{1.871531in}}%
\pgfpathlineto{\pgfqpoint{1.045073in}{1.870285in}}%
\pgfpathlineto{\pgfqpoint{1.045578in}{1.872004in}}%
\pgfpathlineto{\pgfqpoint{1.046168in}{1.876714in}}%
\pgfpathlineto{\pgfqpoint{1.048442in}{1.903170in}}%
\pgfpathlineto{\pgfqpoint{1.048779in}{1.902646in}}%
\pgfpathlineto{\pgfqpoint{1.051054in}{1.886900in}}%
\pgfpathlineto{\pgfqpoint{1.054087in}{1.870258in}}%
\pgfpathlineto{\pgfqpoint{1.054508in}{1.870743in}}%
\pgfpathlineto{\pgfqpoint{1.055013in}{1.871839in}}%
\pgfpathlineto{\pgfqpoint{1.055856in}{1.876108in}}%
\pgfpathlineto{\pgfqpoint{1.057372in}{1.884627in}}%
\pgfpathlineto{\pgfqpoint{1.057709in}{1.883783in}}%
\pgfpathlineto{\pgfqpoint{1.060489in}{1.872792in}}%
\pgfpathlineto{\pgfqpoint{1.060910in}{1.874747in}}%
\pgfpathlineto{\pgfqpoint{1.061921in}{1.878574in}}%
\pgfpathlineto{\pgfqpoint{1.062343in}{1.878131in}}%
\pgfpathlineto{\pgfqpoint{1.064112in}{1.870759in}}%
\pgfpathlineto{\pgfqpoint{1.064533in}{1.874592in}}%
\pgfpathlineto{\pgfqpoint{1.066049in}{1.885356in}}%
\pgfpathlineto{\pgfqpoint{1.066386in}{1.884408in}}%
\pgfpathlineto{\pgfqpoint{1.069588in}{1.869138in}}%
\pgfpathlineto{\pgfqpoint{1.070177in}{1.871324in}}%
\pgfpathlineto{\pgfqpoint{1.070514in}{1.872121in}}%
\pgfpathlineto{\pgfqpoint{1.071020in}{1.870236in}}%
\pgfpathlineto{\pgfqpoint{1.074895in}{1.854990in}}%
\pgfpathlineto{\pgfqpoint{1.082477in}{1.834019in}}%
\pgfpathlineto{\pgfqpoint{1.082561in}{1.834093in}}%
\pgfpathlineto{\pgfqpoint{1.083067in}{1.837323in}}%
\pgfpathlineto{\pgfqpoint{1.084583in}{1.843002in}}%
\pgfpathlineto{\pgfqpoint{1.084667in}{1.842956in}}%
\pgfpathlineto{\pgfqpoint{1.087953in}{1.837641in}}%
\pgfpathlineto{\pgfqpoint{1.089722in}{1.833363in}}%
\pgfpathlineto{\pgfqpoint{1.090312in}{1.834039in}}%
\pgfpathlineto{\pgfqpoint{1.090817in}{1.836486in}}%
\pgfpathlineto{\pgfqpoint{1.093682in}{1.851006in}}%
\pgfpathlineto{\pgfqpoint{1.094440in}{1.850028in}}%
\pgfpathlineto{\pgfqpoint{1.096377in}{1.847066in}}%
\pgfpathlineto{\pgfqpoint{1.096630in}{1.847529in}}%
\pgfpathlineto{\pgfqpoint{1.097473in}{1.850446in}}%
\pgfpathlineto{\pgfqpoint{1.098147in}{1.849372in}}%
\pgfpathlineto{\pgfqpoint{1.099831in}{1.845612in}}%
\pgfpathlineto{\pgfqpoint{1.100337in}{1.847941in}}%
\pgfpathlineto{\pgfqpoint{1.101938in}{1.855592in}}%
\pgfpathlineto{\pgfqpoint{1.102443in}{1.854539in}}%
\pgfpathlineto{\pgfqpoint{1.103875in}{1.854171in}}%
\pgfpathlineto{\pgfqpoint{1.105897in}{1.853248in}}%
\pgfpathlineto{\pgfqpoint{1.108340in}{1.852336in}}%
\pgfpathlineto{\pgfqpoint{1.108424in}{1.852436in}}%
\pgfpathlineto{\pgfqpoint{1.109098in}{1.852883in}}%
\pgfpathlineto{\pgfqpoint{1.109604in}{1.852277in}}%
\pgfpathlineto{\pgfqpoint{1.115838in}{1.839980in}}%
\pgfpathlineto{\pgfqpoint{1.116259in}{1.839622in}}%
\pgfpathlineto{\pgfqpoint{1.117017in}{1.840141in}}%
\pgfpathlineto{\pgfqpoint{1.118028in}{1.838236in}}%
\pgfpathlineto{\pgfqpoint{1.118618in}{1.839740in}}%
\pgfpathlineto{\pgfqpoint{1.119629in}{1.844261in}}%
\pgfpathlineto{\pgfqpoint{1.120050in}{1.845285in}}%
\pgfpathlineto{\pgfqpoint{1.120724in}{1.844388in}}%
\pgfpathlineto{\pgfqpoint{1.122999in}{1.838289in}}%
\pgfpathlineto{\pgfqpoint{1.124852in}{1.832253in}}%
\pgfpathlineto{\pgfqpoint{1.125779in}{1.832791in}}%
\pgfpathlineto{\pgfqpoint{1.127042in}{1.831343in}}%
\pgfpathlineto{\pgfqpoint{1.128643in}{1.827797in}}%
\pgfpathlineto{\pgfqpoint{1.128980in}{1.828839in}}%
\pgfpathlineto{\pgfqpoint{1.131339in}{1.840677in}}%
\pgfpathlineto{\pgfqpoint{1.131929in}{1.839776in}}%
\pgfpathlineto{\pgfqpoint{1.134961in}{1.831155in}}%
\pgfpathlineto{\pgfqpoint{1.135720in}{1.832057in}}%
\pgfpathlineto{\pgfqpoint{1.136309in}{1.830918in}}%
\pgfpathlineto{\pgfqpoint{1.136646in}{1.830379in}}%
\pgfpathlineto{\pgfqpoint{1.137152in}{1.831707in}}%
\pgfpathlineto{\pgfqpoint{1.137657in}{1.832323in}}%
\pgfpathlineto{\pgfqpoint{1.138331in}{1.831773in}}%
\pgfpathlineto{\pgfqpoint{1.140185in}{1.827332in}}%
\pgfpathlineto{\pgfqpoint{1.142796in}{1.821579in}}%
\pgfpathlineto{\pgfqpoint{1.143133in}{1.822056in}}%
\pgfpathlineto{\pgfqpoint{1.149873in}{1.842269in}}%
\pgfpathlineto{\pgfqpoint{1.150884in}{1.841705in}}%
\pgfpathlineto{\pgfqpoint{1.151979in}{1.840517in}}%
\pgfpathlineto{\pgfqpoint{1.153158in}{1.837611in}}%
\pgfpathlineto{\pgfqpoint{1.154169in}{1.838298in}}%
\pgfpathlineto{\pgfqpoint{1.155096in}{1.838297in}}%
\pgfpathlineto{\pgfqpoint{1.155349in}{1.837945in}}%
\pgfpathlineto{\pgfqpoint{1.155854in}{1.837407in}}%
\pgfpathlineto{\pgfqpoint{1.156275in}{1.838227in}}%
\pgfpathlineto{\pgfqpoint{1.157623in}{1.839357in}}%
\pgfpathlineto{\pgfqpoint{1.157707in}{1.839300in}}%
\pgfpathlineto{\pgfqpoint{1.158634in}{1.836515in}}%
\pgfpathlineto{\pgfqpoint{1.160488in}{1.831439in}}%
\pgfpathlineto{\pgfqpoint{1.160909in}{1.832164in}}%
\pgfpathlineto{\pgfqpoint{1.162172in}{1.833011in}}%
\pgfpathlineto{\pgfqpoint{1.162509in}{1.832515in}}%
\pgfpathlineto{\pgfqpoint{1.166722in}{1.824373in}}%
\pgfpathlineto{\pgfqpoint{1.169165in}{1.823434in}}%
\pgfpathlineto{\pgfqpoint{1.170260in}{1.822042in}}%
\pgfpathlineto{\pgfqpoint{1.173461in}{1.816069in}}%
\pgfpathlineto{\pgfqpoint{1.174051in}{1.816636in}}%
\pgfpathlineto{\pgfqpoint{1.175736in}{1.815360in}}%
\pgfpathlineto{\pgfqpoint{1.179274in}{1.811923in}}%
\pgfpathlineto{\pgfqpoint{1.179527in}{1.812135in}}%
\pgfpathlineto{\pgfqpoint{1.180201in}{1.813508in}}%
\pgfpathlineto{\pgfqpoint{1.182560in}{1.819488in}}%
\pgfpathlineto{\pgfqpoint{1.183571in}{1.818016in}}%
\pgfpathlineto{\pgfqpoint{1.187109in}{1.810387in}}%
\pgfpathlineto{\pgfqpoint{1.187783in}{1.811348in}}%
\pgfpathlineto{\pgfqpoint{1.188541in}{1.817655in}}%
\pgfpathlineto{\pgfqpoint{1.189131in}{1.819831in}}%
\pgfpathlineto{\pgfqpoint{1.189805in}{1.819144in}}%
\pgfpathlineto{\pgfqpoint{1.191574in}{1.816135in}}%
\pgfpathlineto{\pgfqpoint{1.191911in}{1.817191in}}%
\pgfpathlineto{\pgfqpoint{1.193090in}{1.822427in}}%
\pgfpathlineto{\pgfqpoint{1.193680in}{1.822001in}}%
\pgfpathlineto{\pgfqpoint{1.195196in}{1.818762in}}%
\pgfpathlineto{\pgfqpoint{1.195702in}{1.818039in}}%
\pgfpathlineto{\pgfqpoint{1.195954in}{1.819457in}}%
\pgfpathlineto{\pgfqpoint{1.197892in}{1.831995in}}%
\pgfpathlineto{\pgfqpoint{1.198145in}{1.831595in}}%
\pgfpathlineto{\pgfqpoint{1.200335in}{1.829049in}}%
\pgfpathlineto{\pgfqpoint{1.201430in}{1.826832in}}%
\pgfpathlineto{\pgfqpoint{1.203452in}{1.823800in}}%
\pgfpathlineto{\pgfqpoint{1.205137in}{1.820515in}}%
\pgfpathlineto{\pgfqpoint{1.206064in}{1.819663in}}%
\pgfpathlineto{\pgfqpoint{1.206232in}{1.820175in}}%
\pgfpathlineto{\pgfqpoint{1.207327in}{1.835895in}}%
\pgfpathlineto{\pgfqpoint{1.209434in}{1.862163in}}%
\pgfpathlineto{\pgfqpoint{1.210192in}{1.860605in}}%
\pgfpathlineto{\pgfqpoint{1.211371in}{1.855550in}}%
\pgfpathlineto{\pgfqpoint{1.212129in}{1.856817in}}%
\pgfpathlineto{\pgfqpoint{1.213056in}{1.861117in}}%
\pgfpathlineto{\pgfqpoint{1.213646in}{1.859249in}}%
\pgfpathlineto{\pgfqpoint{1.215752in}{1.852432in}}%
\pgfpathlineto{\pgfqpoint{1.216089in}{1.853414in}}%
\pgfpathlineto{\pgfqpoint{1.217858in}{1.861430in}}%
\pgfpathlineto{\pgfqpoint{1.218195in}{1.860603in}}%
\pgfpathlineto{\pgfqpoint{1.222155in}{1.847887in}}%
\pgfpathlineto{\pgfqpoint{1.224092in}{1.841202in}}%
\pgfpathlineto{\pgfqpoint{1.224935in}{1.839756in}}%
\pgfpathlineto{\pgfqpoint{1.225272in}{1.840560in}}%
\pgfpathlineto{\pgfqpoint{1.226282in}{1.842920in}}%
\pgfpathlineto{\pgfqpoint{1.226788in}{1.842218in}}%
\pgfpathlineto{\pgfqpoint{1.231084in}{1.833440in}}%
\pgfpathlineto{\pgfqpoint{1.231337in}{1.833865in}}%
\pgfpathlineto{\pgfqpoint{1.232854in}{1.836030in}}%
\pgfpathlineto{\pgfqpoint{1.233275in}{1.835406in}}%
\pgfpathlineto{\pgfqpoint{1.240604in}{1.819432in}}%
\pgfpathlineto{\pgfqpoint{1.240688in}{1.819586in}}%
\pgfpathlineto{\pgfqpoint{1.241615in}{1.829702in}}%
\pgfpathlineto{\pgfqpoint{1.242542in}{1.834886in}}%
\pgfpathlineto{\pgfqpoint{1.243047in}{1.833642in}}%
\pgfpathlineto{\pgfqpoint{1.244648in}{1.829246in}}%
\pgfpathlineto{\pgfqpoint{1.245153in}{1.830189in}}%
\pgfpathlineto{\pgfqpoint{1.245996in}{1.832571in}}%
\pgfpathlineto{\pgfqpoint{1.247175in}{1.835346in}}%
\pgfpathlineto{\pgfqpoint{1.247512in}{1.835114in}}%
\pgfpathlineto{\pgfqpoint{1.250292in}{1.829044in}}%
\pgfpathlineto{\pgfqpoint{1.257790in}{1.817982in}}%
\pgfpathlineto{\pgfqpoint{1.259559in}{1.820200in}}%
\pgfpathlineto{\pgfqpoint{1.260233in}{1.818757in}}%
\pgfpathlineto{\pgfqpoint{1.261834in}{1.816296in}}%
\pgfpathlineto{\pgfqpoint{1.262255in}{1.817286in}}%
\pgfpathlineto{\pgfqpoint{1.263350in}{1.824537in}}%
\pgfpathlineto{\pgfqpoint{1.263856in}{1.827781in}}%
\pgfpathlineto{\pgfqpoint{1.264530in}{1.825892in}}%
\pgfpathlineto{\pgfqpoint{1.265877in}{1.825105in}}%
\pgfpathlineto{\pgfqpoint{1.267647in}{1.822137in}}%
\pgfpathlineto{\pgfqpoint{1.269584in}{1.818514in}}%
\pgfpathlineto{\pgfqpoint{1.269837in}{1.819205in}}%
\pgfpathlineto{\pgfqpoint{1.272364in}{1.830742in}}%
\pgfpathlineto{\pgfqpoint{1.272870in}{1.830244in}}%
\pgfpathlineto{\pgfqpoint{1.274218in}{1.827391in}}%
\pgfpathlineto{\pgfqpoint{1.274976in}{1.827750in}}%
\pgfpathlineto{\pgfqpoint{1.276492in}{1.825342in}}%
\pgfpathlineto{\pgfqpoint{1.279694in}{1.818266in}}%
\pgfpathlineto{\pgfqpoint{1.279778in}{1.818370in}}%
\pgfpathlineto{\pgfqpoint{1.281126in}{1.825520in}}%
\pgfpathlineto{\pgfqpoint{1.281884in}{1.827327in}}%
\pgfpathlineto{\pgfqpoint{1.282558in}{1.826764in}}%
\pgfpathlineto{\pgfqpoint{1.284495in}{1.822337in}}%
\pgfpathlineto{\pgfqpoint{1.284664in}{1.822151in}}%
\pgfpathlineto{\pgfqpoint{1.285001in}{1.823617in}}%
\pgfpathlineto{\pgfqpoint{1.287444in}{1.835782in}}%
\pgfpathlineto{\pgfqpoint{1.288371in}{1.835181in}}%
\pgfpathlineto{\pgfqpoint{1.288708in}{1.834793in}}%
\pgfpathlineto{\pgfqpoint{1.289382in}{1.835584in}}%
\pgfpathlineto{\pgfqpoint{1.289971in}{1.834564in}}%
\pgfpathlineto{\pgfqpoint{1.295447in}{1.822785in}}%
\pgfpathlineto{\pgfqpoint{1.295700in}{1.822570in}}%
\pgfpathlineto{\pgfqpoint{1.296037in}{1.823567in}}%
\pgfpathlineto{\pgfqpoint{1.296795in}{1.827068in}}%
\pgfpathlineto{\pgfqpoint{1.297385in}{1.825430in}}%
\pgfpathlineto{\pgfqpoint{1.298312in}{1.822914in}}%
\pgfpathlineto{\pgfqpoint{1.298649in}{1.824579in}}%
\pgfpathlineto{\pgfqpoint{1.300586in}{1.832923in}}%
\pgfpathlineto{\pgfqpoint{1.302440in}{1.837685in}}%
\pgfpathlineto{\pgfqpoint{1.302692in}{1.837410in}}%
\pgfpathlineto{\pgfqpoint{1.306146in}{1.829259in}}%
\pgfpathlineto{\pgfqpoint{1.306652in}{1.830290in}}%
\pgfpathlineto{\pgfqpoint{1.307326in}{1.831528in}}%
\pgfpathlineto{\pgfqpoint{1.307915in}{1.831046in}}%
\pgfpathlineto{\pgfqpoint{1.309516in}{1.830639in}}%
\pgfpathlineto{\pgfqpoint{1.310527in}{1.834010in}}%
\pgfpathlineto{\pgfqpoint{1.311369in}{1.835555in}}%
\pgfpathlineto{\pgfqpoint{1.311875in}{1.834640in}}%
\pgfpathlineto{\pgfqpoint{1.312549in}{1.833323in}}%
\pgfpathlineto{\pgfqpoint{1.313139in}{1.834368in}}%
\pgfpathlineto{\pgfqpoint{1.313644in}{1.837304in}}%
\pgfpathlineto{\pgfqpoint{1.314571in}{1.843357in}}%
\pgfpathlineto{\pgfqpoint{1.315160in}{1.842030in}}%
\pgfpathlineto{\pgfqpoint{1.321647in}{1.826669in}}%
\pgfpathlineto{\pgfqpoint{1.324006in}{1.821672in}}%
\pgfpathlineto{\pgfqpoint{1.326870in}{1.816982in}}%
\pgfpathlineto{\pgfqpoint{1.327966in}{1.817870in}}%
\pgfpathlineto{\pgfqpoint{1.329061in}{1.821099in}}%
\pgfpathlineto{\pgfqpoint{1.329735in}{1.823336in}}%
\pgfpathlineto{\pgfqpoint{1.330409in}{1.822402in}}%
\pgfpathlineto{\pgfqpoint{1.332852in}{1.827901in}}%
\pgfpathlineto{\pgfqpoint{1.334537in}{1.833996in}}%
\pgfpathlineto{\pgfqpoint{1.334958in}{1.833134in}}%
\pgfpathlineto{\pgfqpoint{1.337485in}{1.828973in}}%
\pgfpathlineto{\pgfqpoint{1.338328in}{1.830248in}}%
\pgfpathlineto{\pgfqpoint{1.339339in}{1.830190in}}%
\pgfpathlineto{\pgfqpoint{1.339507in}{1.829841in}}%
\pgfpathlineto{\pgfqpoint{1.341529in}{1.826520in}}%
\pgfpathlineto{\pgfqpoint{1.341782in}{1.826859in}}%
\pgfpathlineto{\pgfqpoint{1.342456in}{1.827947in}}%
\pgfpathlineto{\pgfqpoint{1.342961in}{1.826973in}}%
\pgfpathlineto{\pgfqpoint{1.346415in}{1.818078in}}%
\pgfpathlineto{\pgfqpoint{1.347173in}{1.818741in}}%
\pgfpathlineto{\pgfqpoint{1.348353in}{1.819832in}}%
\pgfpathlineto{\pgfqpoint{1.348606in}{1.819673in}}%
\pgfpathlineto{\pgfqpoint{1.349701in}{1.817570in}}%
\pgfpathlineto{\pgfqpoint{1.352565in}{1.811208in}}%
\pgfpathlineto{\pgfqpoint{1.352818in}{1.812131in}}%
\pgfpathlineto{\pgfqpoint{1.355598in}{1.826335in}}%
\pgfpathlineto{\pgfqpoint{1.355682in}{1.826320in}}%
\pgfpathlineto{\pgfqpoint{1.356525in}{1.827284in}}%
\pgfpathlineto{\pgfqpoint{1.357198in}{1.826502in}}%
\pgfpathlineto{\pgfqpoint{1.363601in}{1.813322in}}%
\pgfpathlineto{\pgfqpoint{1.363770in}{1.813422in}}%
\pgfpathlineto{\pgfqpoint{1.365202in}{1.812939in}}%
\pgfpathlineto{\pgfqpoint{1.367392in}{1.809546in}}%
\pgfpathlineto{\pgfqpoint{1.367729in}{1.810703in}}%
\pgfpathlineto{\pgfqpoint{1.368656in}{1.815194in}}%
\pgfpathlineto{\pgfqpoint{1.369330in}{1.814104in}}%
\pgfpathlineto{\pgfqpoint{1.369835in}{1.813107in}}%
\pgfpathlineto{\pgfqpoint{1.370172in}{1.814265in}}%
\pgfpathlineto{\pgfqpoint{1.372699in}{1.822829in}}%
\pgfpathlineto{\pgfqpoint{1.373458in}{1.823618in}}%
\pgfpathlineto{\pgfqpoint{1.374637in}{1.827628in}}%
\pgfpathlineto{\pgfqpoint{1.375227in}{1.826936in}}%
\pgfpathlineto{\pgfqpoint{1.379439in}{1.818622in}}%
\pgfpathlineto{\pgfqpoint{1.379692in}{1.818264in}}%
\pgfpathlineto{\pgfqpoint{1.380029in}{1.819582in}}%
\pgfpathlineto{\pgfqpoint{1.381882in}{1.830775in}}%
\pgfpathlineto{\pgfqpoint{1.382472in}{1.830087in}}%
\pgfpathlineto{\pgfqpoint{1.389464in}{1.820866in}}%
\pgfpathlineto{\pgfqpoint{1.390896in}{1.818646in}}%
\pgfpathlineto{\pgfqpoint{1.391318in}{1.818983in}}%
\pgfpathlineto{\pgfqpoint{1.391823in}{1.820089in}}%
\pgfpathlineto{\pgfqpoint{1.393255in}{1.825625in}}%
\pgfpathlineto{\pgfqpoint{1.393676in}{1.825089in}}%
\pgfpathlineto{\pgfqpoint{1.395024in}{1.821826in}}%
\pgfpathlineto{\pgfqpoint{1.395867in}{1.822480in}}%
\pgfpathlineto{\pgfqpoint{1.397130in}{1.825403in}}%
\pgfpathlineto{\pgfqpoint{1.397804in}{1.823946in}}%
\pgfpathlineto{\pgfqpoint{1.399068in}{1.821924in}}%
\pgfpathlineto{\pgfqpoint{1.399405in}{1.822292in}}%
\pgfpathlineto{\pgfqpoint{1.400247in}{1.822185in}}%
\pgfpathlineto{\pgfqpoint{1.400500in}{1.821833in}}%
\pgfpathlineto{\pgfqpoint{1.405386in}{1.816299in}}%
\pgfpathlineto{\pgfqpoint{1.405471in}{1.816367in}}%
\pgfpathlineto{\pgfqpoint{1.407071in}{1.817711in}}%
\pgfpathlineto{\pgfqpoint{1.407408in}{1.817083in}}%
\pgfpathlineto{\pgfqpoint{1.409009in}{1.815938in}}%
\pgfpathlineto{\pgfqpoint{1.409936in}{1.818622in}}%
\pgfpathlineto{\pgfqpoint{1.410778in}{1.820653in}}%
\pgfpathlineto{\pgfqpoint{1.411283in}{1.819844in}}%
\pgfpathlineto{\pgfqpoint{1.412294in}{1.817689in}}%
\pgfpathlineto{\pgfqpoint{1.412800in}{1.818891in}}%
\pgfpathlineto{\pgfqpoint{1.415496in}{1.828669in}}%
\pgfpathlineto{\pgfqpoint{1.417433in}{1.840504in}}%
\pgfpathlineto{\pgfqpoint{1.418023in}{1.838803in}}%
\pgfpathlineto{\pgfqpoint{1.421646in}{1.826819in}}%
\pgfpathlineto{\pgfqpoint{1.422235in}{1.827100in}}%
\pgfpathlineto{\pgfqpoint{1.423078in}{1.826353in}}%
\pgfpathlineto{\pgfqpoint{1.423330in}{1.827448in}}%
\pgfpathlineto{\pgfqpoint{1.424426in}{1.829808in}}%
\pgfpathlineto{\pgfqpoint{1.424847in}{1.829575in}}%
\pgfpathlineto{\pgfqpoint{1.426700in}{1.826172in}}%
\pgfpathlineto{\pgfqpoint{1.428132in}{1.822709in}}%
\pgfpathlineto{\pgfqpoint{1.428722in}{1.823098in}}%
\pgfpathlineto{\pgfqpoint{1.429143in}{1.826128in}}%
\pgfpathlineto{\pgfqpoint{1.430660in}{1.840200in}}%
\pgfpathlineto{\pgfqpoint{1.430997in}{1.839531in}}%
\pgfpathlineto{\pgfqpoint{1.432850in}{1.836009in}}%
\pgfpathlineto{\pgfqpoint{1.433019in}{1.836252in}}%
\pgfpathlineto{\pgfqpoint{1.434030in}{1.840450in}}%
\pgfpathlineto{\pgfqpoint{1.434703in}{1.843315in}}%
\pgfpathlineto{\pgfqpoint{1.435293in}{1.841777in}}%
\pgfpathlineto{\pgfqpoint{1.436220in}{1.839744in}}%
\pgfpathlineto{\pgfqpoint{1.436557in}{1.840924in}}%
\pgfpathlineto{\pgfqpoint{1.438242in}{1.844103in}}%
\pgfpathlineto{\pgfqpoint{1.439758in}{1.843323in}}%
\pgfpathlineto{\pgfqpoint{1.445908in}{1.826471in}}%
\pgfpathlineto{\pgfqpoint{1.447087in}{1.825245in}}%
\pgfpathlineto{\pgfqpoint{1.448772in}{1.821606in}}%
\pgfpathlineto{\pgfqpoint{1.454669in}{1.809633in}}%
\pgfpathlineto{\pgfqpoint{1.454838in}{1.809903in}}%
\pgfpathlineto{\pgfqpoint{1.458460in}{1.819631in}}%
\pgfpathlineto{\pgfqpoint{1.458966in}{1.819122in}}%
\pgfpathlineto{\pgfqpoint{1.461914in}{1.812630in}}%
\pgfpathlineto{\pgfqpoint{1.463010in}{1.810525in}}%
\pgfpathlineto{\pgfqpoint{1.463768in}{1.811059in}}%
\pgfpathlineto{\pgfqpoint{1.464779in}{1.810609in}}%
\pgfpathlineto{\pgfqpoint{1.464863in}{1.810467in}}%
\pgfpathlineto{\pgfqpoint{1.467390in}{1.807317in}}%
\pgfpathlineto{\pgfqpoint{1.467643in}{1.807662in}}%
\pgfpathlineto{\pgfqpoint{1.468654in}{1.808531in}}%
\pgfpathlineto{\pgfqpoint{1.469075in}{1.808144in}}%
\pgfpathlineto{\pgfqpoint{1.471266in}{1.806611in}}%
\pgfpathlineto{\pgfqpoint{1.472192in}{1.807598in}}%
\pgfpathlineto{\pgfqpoint{1.476236in}{1.820417in}}%
\pgfpathlineto{\pgfqpoint{1.477584in}{1.818675in}}%
\pgfpathlineto{\pgfqpoint{1.478848in}{1.817329in}}%
\pgfpathlineto{\pgfqpoint{1.479185in}{1.818456in}}%
\pgfpathlineto{\pgfqpoint{1.479943in}{1.820694in}}%
\pgfpathlineto{\pgfqpoint{1.480532in}{1.819562in}}%
\pgfpathlineto{\pgfqpoint{1.482639in}{1.815746in}}%
\pgfpathlineto{\pgfqpoint{1.483144in}{1.816615in}}%
\pgfpathlineto{\pgfqpoint{1.483481in}{1.816917in}}%
\pgfpathlineto{\pgfqpoint{1.483987in}{1.815990in}}%
\pgfpathlineto{\pgfqpoint{1.485671in}{1.814780in}}%
\pgfpathlineto{\pgfqpoint{1.486514in}{1.815671in}}%
\pgfpathlineto{\pgfqpoint{1.488873in}{1.827136in}}%
\pgfpathlineto{\pgfqpoint{1.489968in}{1.832743in}}%
\pgfpathlineto{\pgfqpoint{1.490558in}{1.832063in}}%
\pgfpathlineto{\pgfqpoint{1.491400in}{1.833346in}}%
\pgfpathlineto{\pgfqpoint{1.491905in}{1.832416in}}%
\pgfpathlineto{\pgfqpoint{1.492916in}{1.834610in}}%
\pgfpathlineto{\pgfqpoint{1.496286in}{1.856064in}}%
\pgfpathlineto{\pgfqpoint{1.496960in}{1.854589in}}%
\pgfpathlineto{\pgfqpoint{1.498814in}{1.849970in}}%
\pgfpathlineto{\pgfqpoint{1.500330in}{1.848399in}}%
\pgfpathlineto{\pgfqpoint{1.500498in}{1.848604in}}%
\pgfpathlineto{\pgfqpoint{1.501088in}{1.848832in}}%
\pgfpathlineto{\pgfqpoint{1.501425in}{1.848149in}}%
\pgfpathlineto{\pgfqpoint{1.503026in}{1.844676in}}%
\pgfpathlineto{\pgfqpoint{1.503363in}{1.845553in}}%
\pgfpathlineto{\pgfqpoint{1.505974in}{1.860880in}}%
\pgfpathlineto{\pgfqpoint{1.506985in}{1.858729in}}%
\pgfpathlineto{\pgfqpoint{1.507322in}{1.857986in}}%
\pgfpathlineto{\pgfqpoint{1.507912in}{1.859289in}}%
\pgfpathlineto{\pgfqpoint{1.509091in}{1.861209in}}%
\pgfpathlineto{\pgfqpoint{1.509513in}{1.859812in}}%
\pgfpathlineto{\pgfqpoint{1.511787in}{1.851531in}}%
\pgfpathlineto{\pgfqpoint{1.512124in}{1.851885in}}%
\pgfpathlineto{\pgfqpoint{1.513809in}{1.855085in}}%
\pgfpathlineto{\pgfqpoint{1.514652in}{1.859350in}}%
\pgfpathlineto{\pgfqpoint{1.515241in}{1.858319in}}%
\pgfpathlineto{\pgfqpoint{1.517516in}{1.851962in}}%
\pgfpathlineto{\pgfqpoint{1.518106in}{1.854769in}}%
\pgfpathlineto{\pgfqpoint{1.518695in}{1.856305in}}%
\pgfpathlineto{\pgfqpoint{1.519453in}{1.855924in}}%
\pgfpathlineto{\pgfqpoint{1.520212in}{1.854749in}}%
\pgfpathlineto{\pgfqpoint{1.522992in}{1.849644in}}%
\pgfpathlineto{\pgfqpoint{1.525435in}{1.848094in}}%
\pgfpathlineto{\pgfqpoint{1.527204in}{1.842881in}}%
\pgfpathlineto{\pgfqpoint{1.527709in}{1.843410in}}%
\pgfpathlineto{\pgfqpoint{1.528973in}{1.842955in}}%
\pgfpathlineto{\pgfqpoint{1.529394in}{1.842665in}}%
\pgfpathlineto{\pgfqpoint{1.529563in}{1.842198in}}%
\pgfpathlineto{\pgfqpoint{1.533522in}{1.830681in}}%
\pgfpathlineto{\pgfqpoint{1.533944in}{1.831894in}}%
\pgfpathlineto{\pgfqpoint{1.534449in}{1.832773in}}%
\pgfpathlineto{\pgfqpoint{1.535039in}{1.831782in}}%
\pgfpathlineto{\pgfqpoint{1.538156in}{1.823537in}}%
\pgfpathlineto{\pgfqpoint{1.538745in}{1.825336in}}%
\pgfpathlineto{\pgfqpoint{1.541189in}{1.830702in}}%
\pgfpathlineto{\pgfqpoint{1.541441in}{1.830335in}}%
\pgfpathlineto{\pgfqpoint{1.545064in}{1.820995in}}%
\pgfpathlineto{\pgfqpoint{1.545401in}{1.823393in}}%
\pgfpathlineto{\pgfqpoint{1.546917in}{1.862369in}}%
\pgfpathlineto{\pgfqpoint{1.548434in}{1.885769in}}%
\pgfpathlineto{\pgfqpoint{1.548686in}{1.885523in}}%
\pgfpathlineto{\pgfqpoint{1.549697in}{1.881378in}}%
\pgfpathlineto{\pgfqpoint{1.551972in}{1.870175in}}%
\pgfpathlineto{\pgfqpoint{1.552562in}{1.870850in}}%
\pgfpathlineto{\pgfqpoint{1.554246in}{1.872961in}}%
\pgfpathlineto{\pgfqpoint{1.555510in}{1.873123in}}%
\pgfpathlineto{\pgfqpoint{1.555679in}{1.872864in}}%
\pgfpathlineto{\pgfqpoint{1.561576in}{1.856817in}}%
\pgfpathlineto{\pgfqpoint{1.562671in}{1.858594in}}%
\pgfpathlineto{\pgfqpoint{1.564019in}{1.860186in}}%
\pgfpathlineto{\pgfqpoint{1.565030in}{1.864969in}}%
\pgfpathlineto{\pgfqpoint{1.565619in}{1.863687in}}%
\pgfpathlineto{\pgfqpoint{1.567136in}{1.857611in}}%
\pgfpathlineto{\pgfqpoint{1.567641in}{1.860816in}}%
\pgfpathlineto{\pgfqpoint{1.568315in}{1.864692in}}%
\pgfpathlineto{\pgfqpoint{1.568905in}{1.862673in}}%
\pgfpathlineto{\pgfqpoint{1.570758in}{1.857720in}}%
\pgfpathlineto{\pgfqpoint{1.571011in}{1.858464in}}%
\pgfpathlineto{\pgfqpoint{1.572359in}{1.877193in}}%
\pgfpathlineto{\pgfqpoint{1.573707in}{1.873500in}}%
\pgfpathlineto{\pgfqpoint{1.575560in}{1.869383in}}%
\pgfpathlineto{\pgfqpoint{1.576403in}{1.867257in}}%
\pgfpathlineto{\pgfqpoint{1.577245in}{1.865217in}}%
\pgfpathlineto{\pgfqpoint{1.577582in}{1.866296in}}%
\pgfpathlineto{\pgfqpoint{1.578425in}{1.870834in}}%
\pgfpathlineto{\pgfqpoint{1.579014in}{1.869300in}}%
\pgfpathlineto{\pgfqpoint{1.579857in}{1.866058in}}%
\pgfpathlineto{\pgfqpoint{1.580194in}{1.868445in}}%
\pgfpathlineto{\pgfqpoint{1.581794in}{1.878432in}}%
\pgfpathlineto{\pgfqpoint{1.582047in}{1.877791in}}%
\pgfpathlineto{\pgfqpoint{1.582384in}{1.876657in}}%
\pgfpathlineto{\pgfqpoint{1.582721in}{1.878560in}}%
\pgfpathlineto{\pgfqpoint{1.583479in}{1.885803in}}%
\pgfpathlineto{\pgfqpoint{1.584153in}{1.883295in}}%
\pgfpathlineto{\pgfqpoint{1.584996in}{1.879514in}}%
\pgfpathlineto{\pgfqpoint{1.585417in}{1.881857in}}%
\pgfpathlineto{\pgfqpoint{1.586091in}{1.885566in}}%
\pgfpathlineto{\pgfqpoint{1.586681in}{1.883944in}}%
\pgfpathlineto{\pgfqpoint{1.588534in}{1.878126in}}%
\pgfpathlineto{\pgfqpoint{1.588702in}{1.878878in}}%
\pgfpathlineto{\pgfqpoint{1.589376in}{1.904467in}}%
\pgfpathlineto{\pgfqpoint{1.591230in}{1.935611in}}%
\pgfpathlineto{\pgfqpoint{1.591567in}{1.939277in}}%
\pgfpathlineto{\pgfqpoint{1.592241in}{1.934250in}}%
\pgfpathlineto{\pgfqpoint{1.596621in}{1.905752in}}%
\pgfpathlineto{\pgfqpoint{1.597632in}{1.904827in}}%
\pgfpathlineto{\pgfqpoint{1.597801in}{1.905372in}}%
\pgfpathlineto{\pgfqpoint{1.598896in}{1.920314in}}%
\pgfpathlineto{\pgfqpoint{1.600412in}{1.919545in}}%
\pgfpathlineto{\pgfqpoint{1.600665in}{1.919515in}}%
\pgfpathlineto{\pgfqpoint{1.600918in}{1.920237in}}%
\pgfpathlineto{\pgfqpoint{1.602771in}{1.933874in}}%
\pgfpathlineto{\pgfqpoint{1.606647in}{2.019532in}}%
\pgfpathlineto{\pgfqpoint{1.608079in}{2.040026in}}%
\pgfpathlineto{\pgfqpoint{1.608247in}{2.040356in}}%
\pgfpathlineto{\pgfqpoint{1.608668in}{2.038733in}}%
\pgfpathlineto{\pgfqpoint{1.610353in}{2.033916in}}%
\pgfpathlineto{\pgfqpoint{1.610606in}{2.034940in}}%
\pgfpathlineto{\pgfqpoint{1.611280in}{2.038348in}}%
\pgfpathlineto{\pgfqpoint{1.611954in}{2.036818in}}%
\pgfpathlineto{\pgfqpoint{1.613049in}{2.025750in}}%
\pgfpathlineto{\pgfqpoint{1.618441in}{1.976568in}}%
\pgfpathlineto{\pgfqpoint{1.619789in}{1.967724in}}%
\pgfpathlineto{\pgfqpoint{1.623411in}{1.935818in}}%
\pgfpathlineto{\pgfqpoint{1.623832in}{1.939573in}}%
\pgfpathlineto{\pgfqpoint{1.626444in}{1.970770in}}%
\pgfpathlineto{\pgfqpoint{1.626949in}{1.967796in}}%
\pgfpathlineto{\pgfqpoint{1.627960in}{1.963082in}}%
\pgfpathlineto{\pgfqpoint{1.628382in}{1.965085in}}%
\pgfpathlineto{\pgfqpoint{1.629140in}{1.989972in}}%
\pgfpathlineto{\pgfqpoint{1.630319in}{2.039775in}}%
\pgfpathlineto{\pgfqpoint{1.630993in}{2.036873in}}%
\pgfpathlineto{\pgfqpoint{1.636048in}{2.007093in}}%
\pgfpathlineto{\pgfqpoint{1.636385in}{2.006545in}}%
\pgfpathlineto{\pgfqpoint{1.636806in}{2.008275in}}%
\pgfpathlineto{\pgfqpoint{1.639165in}{2.031991in}}%
\pgfpathlineto{\pgfqpoint{1.640007in}{2.025642in}}%
\pgfpathlineto{\pgfqpoint{1.640513in}{2.023388in}}%
\pgfpathlineto{\pgfqpoint{1.641187in}{2.024458in}}%
\pgfpathlineto{\pgfqpoint{1.641861in}{2.027868in}}%
\pgfpathlineto{\pgfqpoint{1.642956in}{2.058267in}}%
\pgfpathlineto{\pgfqpoint{1.645568in}{2.151511in}}%
\pgfpathlineto{\pgfqpoint{1.645820in}{2.150584in}}%
\pgfpathlineto{\pgfqpoint{1.646494in}{2.144618in}}%
\pgfpathlineto{\pgfqpoint{1.646915in}{2.149000in}}%
\pgfpathlineto{\pgfqpoint{1.647758in}{2.196015in}}%
\pgfpathlineto{\pgfqpoint{1.649611in}{2.318452in}}%
\pgfpathlineto{\pgfqpoint{1.650032in}{2.314119in}}%
\pgfpathlineto{\pgfqpoint{1.650875in}{2.299659in}}%
\pgfpathlineto{\pgfqpoint{1.651465in}{2.306924in}}%
\pgfpathlineto{\pgfqpoint{1.653065in}{2.352590in}}%
\pgfpathlineto{\pgfqpoint{1.653823in}{2.343793in}}%
\pgfpathlineto{\pgfqpoint{1.655845in}{2.309636in}}%
\pgfpathlineto{\pgfqpoint{1.656267in}{2.310235in}}%
\pgfpathlineto{\pgfqpoint{1.656604in}{2.309575in}}%
\pgfpathlineto{\pgfqpoint{1.657362in}{2.297472in}}%
\pgfpathlineto{\pgfqpoint{1.661658in}{2.217718in}}%
\pgfpathlineto{\pgfqpoint{1.664775in}{2.172035in}}%
\pgfpathlineto{\pgfqpoint{1.669914in}{2.085118in}}%
\pgfpathlineto{\pgfqpoint{1.670167in}{2.087066in}}%
\pgfpathlineto{\pgfqpoint{1.672442in}{2.112668in}}%
\pgfpathlineto{\pgfqpoint{1.672610in}{2.112446in}}%
\pgfpathlineto{\pgfqpoint{1.673621in}{2.103381in}}%
\pgfpathlineto{\pgfqpoint{1.674126in}{2.110545in}}%
\pgfpathlineto{\pgfqpoint{1.675222in}{2.174549in}}%
\pgfpathlineto{\pgfqpoint{1.676822in}{2.271221in}}%
\pgfpathlineto{\pgfqpoint{1.677328in}{2.264649in}}%
\pgfpathlineto{\pgfqpoint{1.678339in}{2.244504in}}%
\pgfpathlineto{\pgfqpoint{1.678844in}{2.254197in}}%
\pgfpathlineto{\pgfqpoint{1.680866in}{2.324397in}}%
\pgfpathlineto{\pgfqpoint{1.681456in}{2.314382in}}%
\pgfpathlineto{\pgfqpoint{1.682804in}{2.284032in}}%
\pgfpathlineto{\pgfqpoint{1.683393in}{2.290697in}}%
\pgfpathlineto{\pgfqpoint{1.684657in}{2.310523in}}%
\pgfpathlineto{\pgfqpoint{1.685162in}{2.306352in}}%
\pgfpathlineto{\pgfqpoint{1.701253in}{2.033860in}}%
\pgfpathlineto{\pgfqpoint{1.701843in}{2.039786in}}%
\pgfpathlineto{\pgfqpoint{1.703107in}{2.053184in}}%
\pgfpathlineto{\pgfqpoint{1.703528in}{2.051724in}}%
\pgfpathlineto{\pgfqpoint{1.704454in}{2.043231in}}%
\pgfpathlineto{\pgfqpoint{1.704876in}{2.049354in}}%
\pgfpathlineto{\pgfqpoint{1.705971in}{2.121704in}}%
\pgfpathlineto{\pgfqpoint{1.707571in}{2.183587in}}%
\pgfpathlineto{\pgfqpoint{1.707824in}{2.182337in}}%
\pgfpathlineto{\pgfqpoint{1.708835in}{2.170862in}}%
\pgfpathlineto{\pgfqpoint{1.709341in}{2.175970in}}%
\pgfpathlineto{\pgfqpoint{1.711026in}{2.212330in}}%
\pgfpathlineto{\pgfqpoint{1.711868in}{2.201928in}}%
\pgfpathlineto{\pgfqpoint{1.717512in}{2.095729in}}%
\pgfpathlineto{\pgfqpoint{1.717765in}{2.096461in}}%
\pgfpathlineto{\pgfqpoint{1.718692in}{2.111775in}}%
\pgfpathlineto{\pgfqpoint{1.719787in}{2.126899in}}%
\pgfpathlineto{\pgfqpoint{1.720292in}{2.123603in}}%
\pgfpathlineto{\pgfqpoint{1.721303in}{2.110996in}}%
\pgfpathlineto{\pgfqpoint{1.721809in}{2.115866in}}%
\pgfpathlineto{\pgfqpoint{1.723578in}{2.150184in}}%
\pgfpathlineto{\pgfqpoint{1.724252in}{2.143916in}}%
\pgfpathlineto{\pgfqpoint{1.730065in}{2.062922in}}%
\pgfpathlineto{\pgfqpoint{1.733603in}{2.031973in}}%
\pgfpathlineto{\pgfqpoint{1.733856in}{2.033709in}}%
\pgfpathlineto{\pgfqpoint{1.735288in}{2.071012in}}%
\pgfpathlineto{\pgfqpoint{1.736720in}{2.103615in}}%
\pgfpathlineto{\pgfqpoint{1.737141in}{2.100941in}}%
\pgfpathlineto{\pgfqpoint{1.750283in}{1.960785in}}%
\pgfpathlineto{\pgfqpoint{1.757528in}{1.902328in}}%
\pgfpathlineto{\pgfqpoint{1.757613in}{1.902359in}}%
\pgfpathlineto{\pgfqpoint{1.758118in}{1.905399in}}%
\pgfpathlineto{\pgfqpoint{1.759972in}{1.920480in}}%
\pgfpathlineto{\pgfqpoint{1.760393in}{1.919422in}}%
\pgfpathlineto{\pgfqpoint{1.761656in}{1.912629in}}%
\pgfpathlineto{\pgfqpoint{1.762246in}{1.915500in}}%
\pgfpathlineto{\pgfqpoint{1.764437in}{1.928984in}}%
\pgfpathlineto{\pgfqpoint{1.764858in}{1.927791in}}%
\pgfpathlineto{\pgfqpoint{1.774883in}{1.872805in}}%
\pgfpathlineto{\pgfqpoint{1.782970in}{1.845616in}}%
\pgfpathlineto{\pgfqpoint{1.789541in}{1.823122in}}%
\pgfpathlineto{\pgfqpoint{1.795102in}{1.809417in}}%
\pgfpathlineto{\pgfqpoint{1.796028in}{1.810097in}}%
\pgfpathlineto{\pgfqpoint{1.800156in}{1.817365in}}%
\pgfpathlineto{\pgfqpoint{1.801083in}{1.828776in}}%
\pgfpathlineto{\pgfqpoint{1.802347in}{1.840387in}}%
\pgfpathlineto{\pgfqpoint{1.802768in}{1.839593in}}%
\pgfpathlineto{\pgfqpoint{1.804200in}{1.836483in}}%
\pgfpathlineto{\pgfqpoint{1.804621in}{1.837246in}}%
\pgfpathlineto{\pgfqpoint{1.806643in}{1.846202in}}%
\pgfpathlineto{\pgfqpoint{1.807401in}{1.844801in}}%
\pgfpathlineto{\pgfqpoint{1.809255in}{1.840321in}}%
\pgfpathlineto{\pgfqpoint{1.809592in}{1.841007in}}%
\pgfpathlineto{\pgfqpoint{1.810687in}{1.843637in}}%
\pgfpathlineto{\pgfqpoint{1.811192in}{1.842719in}}%
\pgfpathlineto{\pgfqpoint{1.815910in}{1.831059in}}%
\pgfpathlineto{\pgfqpoint{1.823492in}{1.812876in}}%
\pgfpathlineto{\pgfqpoint{1.824840in}{1.810609in}}%
\pgfpathlineto{\pgfqpoint{1.825767in}{1.808790in}}%
\pgfpathlineto{\pgfqpoint{1.826272in}{1.809718in}}%
\pgfpathlineto{\pgfqpoint{1.826778in}{1.810462in}}%
\pgfpathlineto{\pgfqpoint{1.827367in}{1.809542in}}%
\pgfpathlineto{\pgfqpoint{1.831242in}{1.803788in}}%
\pgfpathlineto{\pgfqpoint{1.832001in}{1.804901in}}%
\pgfpathlineto{\pgfqpoint{1.832675in}{1.804392in}}%
\pgfpathlineto{\pgfqpoint{1.832759in}{1.804234in}}%
\pgfpathlineto{\pgfqpoint{1.835455in}{1.800407in}}%
\pgfpathlineto{\pgfqpoint{1.835792in}{1.800688in}}%
\pgfpathlineto{\pgfqpoint{1.836550in}{1.800884in}}%
\pgfpathlineto{\pgfqpoint{1.836971in}{1.800391in}}%
\pgfpathlineto{\pgfqpoint{1.838740in}{1.800378in}}%
\pgfpathlineto{\pgfqpoint{1.839498in}{1.803319in}}%
\pgfpathlineto{\pgfqpoint{1.841857in}{1.813184in}}%
\pgfpathlineto{\pgfqpoint{1.842952in}{1.814544in}}%
\pgfpathlineto{\pgfqpoint{1.844469in}{1.823924in}}%
\pgfpathlineto{\pgfqpoint{1.845733in}{1.832139in}}%
\pgfpathlineto{\pgfqpoint{1.846238in}{1.831340in}}%
\pgfpathlineto{\pgfqpoint{1.846912in}{1.829827in}}%
\pgfpathlineto{\pgfqpoint{1.847417in}{1.831199in}}%
\pgfpathlineto{\pgfqpoint{1.849439in}{1.839063in}}%
\pgfpathlineto{\pgfqpoint{1.849861in}{1.838382in}}%
\pgfpathlineto{\pgfqpoint{1.851545in}{1.833792in}}%
\pgfpathlineto{\pgfqpoint{1.851967in}{1.835362in}}%
\pgfpathlineto{\pgfqpoint{1.853146in}{1.840671in}}%
\pgfpathlineto{\pgfqpoint{1.853652in}{1.839938in}}%
\pgfpathlineto{\pgfqpoint{1.859717in}{1.827830in}}%
\pgfpathlineto{\pgfqpoint{1.861234in}{1.827332in}}%
\pgfpathlineto{\pgfqpoint{1.863003in}{1.824568in}}%
\pgfpathlineto{\pgfqpoint{1.863424in}{1.825817in}}%
\pgfpathlineto{\pgfqpoint{1.864772in}{1.834587in}}%
\pgfpathlineto{\pgfqpoint{1.865614in}{1.833153in}}%
\pgfpathlineto{\pgfqpoint{1.866878in}{1.830653in}}%
\pgfpathlineto{\pgfqpoint{1.867299in}{1.831495in}}%
\pgfpathlineto{\pgfqpoint{1.868479in}{1.836359in}}%
\pgfpathlineto{\pgfqpoint{1.869152in}{1.834658in}}%
\pgfpathlineto{\pgfqpoint{1.874207in}{1.821382in}}%
\pgfpathlineto{\pgfqpoint{1.877240in}{1.813896in}}%
\pgfpathlineto{\pgfqpoint{1.877661in}{1.814983in}}%
\pgfpathlineto{\pgfqpoint{1.878756in}{1.817874in}}%
\pgfpathlineto{\pgfqpoint{1.879346in}{1.817586in}}%
\pgfpathlineto{\pgfqpoint{1.880526in}{1.816610in}}%
\pgfpathlineto{\pgfqpoint{1.880862in}{1.817798in}}%
\pgfpathlineto{\pgfqpoint{1.883306in}{1.829982in}}%
\pgfpathlineto{\pgfqpoint{1.884148in}{1.828926in}}%
\pgfpathlineto{\pgfqpoint{1.884653in}{1.830187in}}%
\pgfpathlineto{\pgfqpoint{1.887097in}{1.843075in}}%
\pgfpathlineto{\pgfqpoint{1.887855in}{1.841366in}}%
\pgfpathlineto{\pgfqpoint{1.892488in}{1.829430in}}%
\pgfpathlineto{\pgfqpoint{1.893920in}{1.825404in}}%
\pgfpathlineto{\pgfqpoint{1.894173in}{1.825138in}}%
\pgfpathlineto{\pgfqpoint{1.894594in}{1.826179in}}%
\pgfpathlineto{\pgfqpoint{1.896027in}{1.839359in}}%
\pgfpathlineto{\pgfqpoint{1.897037in}{1.844352in}}%
\pgfpathlineto{\pgfqpoint{1.897543in}{1.843623in}}%
\pgfpathlineto{\pgfqpoint{1.898385in}{1.841686in}}%
\pgfpathlineto{\pgfqpoint{1.898722in}{1.842967in}}%
\pgfpathlineto{\pgfqpoint{1.900154in}{1.852328in}}%
\pgfpathlineto{\pgfqpoint{1.900913in}{1.850881in}}%
\pgfpathlineto{\pgfqpoint{1.902176in}{1.846903in}}%
\pgfpathlineto{\pgfqpoint{1.902598in}{1.848878in}}%
\pgfpathlineto{\pgfqpoint{1.903524in}{1.855041in}}%
\pgfpathlineto{\pgfqpoint{1.904198in}{1.853133in}}%
\pgfpathlineto{\pgfqpoint{1.904704in}{1.852093in}}%
\pgfpathlineto{\pgfqpoint{1.905041in}{1.853368in}}%
\pgfpathlineto{\pgfqpoint{1.905967in}{1.868089in}}%
\pgfpathlineto{\pgfqpoint{1.907652in}{1.902980in}}%
\pgfpathlineto{\pgfqpoint{1.908326in}{1.899802in}}%
\pgfpathlineto{\pgfqpoint{1.908916in}{1.897635in}}%
\pgfpathlineto{\pgfqpoint{1.909337in}{1.899590in}}%
\pgfpathlineto{\pgfqpoint{1.911275in}{1.918375in}}%
\pgfpathlineto{\pgfqpoint{1.912033in}{1.914033in}}%
\pgfpathlineto{\pgfqpoint{1.916666in}{1.885607in}}%
\pgfpathlineto{\pgfqpoint{1.922142in}{1.858882in}}%
\pgfpathlineto{\pgfqpoint{1.924080in}{1.854785in}}%
\pgfpathlineto{\pgfqpoint{1.936717in}{1.813534in}}%
\pgfpathlineto{\pgfqpoint{1.942782in}{1.805370in}}%
\pgfpathlineto{\pgfqpoint{1.943035in}{1.805884in}}%
\pgfpathlineto{\pgfqpoint{1.944299in}{1.814447in}}%
\pgfpathlineto{\pgfqpoint{1.945984in}{1.824922in}}%
\pgfpathlineto{\pgfqpoint{1.946657in}{1.824138in}}%
\pgfpathlineto{\pgfqpoint{1.947668in}{1.822236in}}%
\pgfpathlineto{\pgfqpoint{1.948090in}{1.823258in}}%
\pgfpathlineto{\pgfqpoint{1.950870in}{1.833623in}}%
\pgfpathlineto{\pgfqpoint{1.951207in}{1.833176in}}%
\pgfpathlineto{\pgfqpoint{1.957441in}{1.820877in}}%
\pgfpathlineto{\pgfqpoint{1.959715in}{1.817882in}}%
\pgfpathlineto{\pgfqpoint{1.959968in}{1.818111in}}%
\pgfpathlineto{\pgfqpoint{1.961737in}{1.821342in}}%
\pgfpathlineto{\pgfqpoint{1.962495in}{1.820082in}}%
\pgfpathlineto{\pgfqpoint{1.963843in}{1.817552in}}%
\pgfpathlineto{\pgfqpoint{1.964265in}{1.818243in}}%
\pgfpathlineto{\pgfqpoint{1.966034in}{1.822673in}}%
\pgfpathlineto{\pgfqpoint{1.966539in}{1.822066in}}%
\pgfpathlineto{\pgfqpoint{1.968140in}{1.818928in}}%
\pgfpathlineto{\pgfqpoint{1.968814in}{1.819703in}}%
\pgfpathlineto{\pgfqpoint{1.970414in}{1.820552in}}%
\pgfpathlineto{\pgfqpoint{1.970667in}{1.820077in}}%
\pgfpathlineto{\pgfqpoint{1.976733in}{1.808322in}}%
\pgfpathlineto{\pgfqpoint{1.979429in}{1.805358in}}%
\pgfpathlineto{\pgfqpoint{1.980271in}{1.806595in}}%
\pgfpathlineto{\pgfqpoint{1.981198in}{1.806000in}}%
\pgfpathlineto{\pgfqpoint{1.981787in}{1.805586in}}%
\pgfpathlineto{\pgfqpoint{1.982209in}{1.806356in}}%
\pgfpathlineto{\pgfqpoint{1.983220in}{1.814219in}}%
\pgfpathlineto{\pgfqpoint{1.985241in}{1.836002in}}%
\pgfpathlineto{\pgfqpoint{1.985915in}{1.834706in}}%
\pgfpathlineto{\pgfqpoint{1.986589in}{1.833122in}}%
\pgfpathlineto{\pgfqpoint{1.987011in}{1.834443in}}%
\pgfpathlineto{\pgfqpoint{1.988106in}{1.852467in}}%
\pgfpathlineto{\pgfqpoint{1.990043in}{1.875117in}}%
\pgfpathlineto{\pgfqpoint{1.990296in}{1.874655in}}%
\pgfpathlineto{\pgfqpoint{1.991307in}{1.870886in}}%
\pgfpathlineto{\pgfqpoint{1.991813in}{1.872775in}}%
\pgfpathlineto{\pgfqpoint{1.994003in}{1.895396in}}%
\pgfpathlineto{\pgfqpoint{1.995098in}{1.891061in}}%
\pgfpathlineto{\pgfqpoint{1.995435in}{1.892563in}}%
\pgfpathlineto{\pgfqpoint{1.996530in}{1.915101in}}%
\pgfpathlineto{\pgfqpoint{1.997625in}{1.929523in}}%
\pgfpathlineto{\pgfqpoint{1.998131in}{1.927638in}}%
\pgfpathlineto{\pgfqpoint{1.998973in}{1.922984in}}%
\pgfpathlineto{\pgfqpoint{1.999395in}{1.925679in}}%
\pgfpathlineto{\pgfqpoint{2.001416in}{1.952364in}}%
\pgfpathlineto{\pgfqpoint{2.002090in}{1.948310in}}%
\pgfpathlineto{\pgfqpoint{2.002933in}{1.944169in}}%
\pgfpathlineto{\pgfqpoint{2.003354in}{1.945979in}}%
\pgfpathlineto{\pgfqpoint{2.004870in}{1.960403in}}%
\pgfpathlineto{\pgfqpoint{2.005544in}{1.956323in}}%
\pgfpathlineto{\pgfqpoint{2.006303in}{1.952338in}}%
\pgfpathlineto{\pgfqpoint{2.006808in}{1.954087in}}%
\pgfpathlineto{\pgfqpoint{2.008577in}{1.973659in}}%
\pgfpathlineto{\pgfqpoint{2.009588in}{1.966808in}}%
\pgfpathlineto{\pgfqpoint{2.013716in}{1.935428in}}%
\pgfpathlineto{\pgfqpoint{2.016917in}{1.915637in}}%
\pgfpathlineto{\pgfqpoint{2.017928in}{1.912352in}}%
\pgfpathlineto{\pgfqpoint{2.026521in}{1.862675in}}%
\pgfpathlineto{\pgfqpoint{2.028796in}{1.856625in}}%
\pgfpathlineto{\pgfqpoint{2.030649in}{1.854054in}}%
\pgfpathlineto{\pgfqpoint{2.033008in}{1.844579in}}%
\pgfpathlineto{\pgfqpoint{2.034188in}{1.841143in}}%
\pgfpathlineto{\pgfqpoint{2.034693in}{1.842126in}}%
\pgfpathlineto{\pgfqpoint{2.035957in}{1.850093in}}%
\pgfpathlineto{\pgfqpoint{2.037305in}{1.860758in}}%
\pgfpathlineto{\pgfqpoint{2.037894in}{1.859151in}}%
\pgfpathlineto{\pgfqpoint{2.038484in}{1.857672in}}%
\pgfpathlineto{\pgfqpoint{2.038989in}{1.859209in}}%
\pgfpathlineto{\pgfqpoint{2.040337in}{1.873954in}}%
\pgfpathlineto{\pgfqpoint{2.041433in}{1.882877in}}%
\pgfpathlineto{\pgfqpoint{2.041938in}{1.881275in}}%
\pgfpathlineto{\pgfqpoint{2.043876in}{1.873983in}}%
\pgfpathlineto{\pgfqpoint{2.044381in}{1.874210in}}%
\pgfpathlineto{\pgfqpoint{2.045139in}{1.873968in}}%
\pgfpathlineto{\pgfqpoint{2.045308in}{1.873522in}}%
\pgfpathlineto{\pgfqpoint{2.047582in}{1.861719in}}%
\pgfpathlineto{\pgfqpoint{2.050278in}{1.853482in}}%
\pgfpathlineto{\pgfqpoint{2.051121in}{1.852251in}}%
\pgfpathlineto{\pgfqpoint{2.053143in}{1.845719in}}%
\pgfpathlineto{\pgfqpoint{2.053817in}{1.846888in}}%
\pgfpathlineto{\pgfqpoint{2.055249in}{1.850161in}}%
\pgfpathlineto{\pgfqpoint{2.055754in}{1.849450in}}%
\pgfpathlineto{\pgfqpoint{2.065611in}{1.822945in}}%
\pgfpathlineto{\pgfqpoint{2.066285in}{1.824005in}}%
\pgfpathlineto{\pgfqpoint{2.067380in}{1.826318in}}%
\pgfpathlineto{\pgfqpoint{2.067970in}{1.825465in}}%
\pgfpathlineto{\pgfqpoint{2.069149in}{1.822897in}}%
\pgfpathlineto{\pgfqpoint{2.069739in}{1.823940in}}%
\pgfpathlineto{\pgfqpoint{2.071592in}{1.828309in}}%
\pgfpathlineto{\pgfqpoint{2.072182in}{1.827155in}}%
\pgfpathlineto{\pgfqpoint{2.073277in}{1.825116in}}%
\pgfpathlineto{\pgfqpoint{2.073698in}{1.825700in}}%
\pgfpathlineto{\pgfqpoint{2.075720in}{1.831933in}}%
\pgfpathlineto{\pgfqpoint{2.076563in}{1.830134in}}%
\pgfpathlineto{\pgfqpoint{2.078079in}{1.827902in}}%
\pgfpathlineto{\pgfqpoint{2.078332in}{1.828028in}}%
\pgfpathlineto{\pgfqpoint{2.079343in}{1.827789in}}%
\pgfpathlineto{\pgfqpoint{2.079511in}{1.827432in}}%
\pgfpathlineto{\pgfqpoint{2.083808in}{1.817854in}}%
\pgfpathlineto{\pgfqpoint{2.084229in}{1.818355in}}%
\pgfpathlineto{\pgfqpoint{2.086335in}{1.820336in}}%
\pgfpathlineto{\pgfqpoint{2.086419in}{1.820286in}}%
\pgfpathlineto{\pgfqpoint{2.087599in}{1.817835in}}%
\pgfpathlineto{\pgfqpoint{2.088946in}{1.816581in}}%
\pgfpathlineto{\pgfqpoint{2.089199in}{1.816833in}}%
\pgfpathlineto{\pgfqpoint{2.090716in}{1.819847in}}%
\pgfpathlineto{\pgfqpoint{2.091474in}{1.818388in}}%
\pgfpathlineto{\pgfqpoint{2.094254in}{1.813803in}}%
\pgfpathlineto{\pgfqpoint{2.095939in}{1.813254in}}%
\pgfpathlineto{\pgfqpoint{2.104026in}{1.802736in}}%
\pgfpathlineto{\pgfqpoint{2.104700in}{1.802322in}}%
\pgfpathlineto{\pgfqpoint{2.105121in}{1.802891in}}%
\pgfpathlineto{\pgfqpoint{2.106385in}{1.803927in}}%
\pgfpathlineto{\pgfqpoint{2.106722in}{1.803615in}}%
\pgfpathlineto{\pgfqpoint{2.107901in}{1.802289in}}%
\pgfpathlineto{\pgfqpoint{2.108323in}{1.803274in}}%
\pgfpathlineto{\pgfqpoint{2.110429in}{1.808424in}}%
\pgfpathlineto{\pgfqpoint{2.110766in}{1.808048in}}%
\pgfpathlineto{\pgfqpoint{2.113293in}{1.805737in}}%
\pgfpathlineto{\pgfqpoint{2.114810in}{1.804849in}}%
\pgfpathlineto{\pgfqpoint{2.120117in}{1.799350in}}%
\pgfpathlineto{\pgfqpoint{2.122981in}{1.799590in}}%
\pgfpathlineto{\pgfqpoint{2.123908in}{1.798636in}}%
\pgfpathlineto{\pgfqpoint{2.124245in}{1.799412in}}%
\pgfpathlineto{\pgfqpoint{2.125509in}{1.809179in}}%
\pgfpathlineto{\pgfqpoint{2.127025in}{1.819813in}}%
\pgfpathlineto{\pgfqpoint{2.127530in}{1.819450in}}%
\pgfpathlineto{\pgfqpoint{2.127952in}{1.820260in}}%
\pgfpathlineto{\pgfqpoint{2.128794in}{1.829067in}}%
\pgfpathlineto{\pgfqpoint{2.130648in}{1.862058in}}%
\pgfpathlineto{\pgfqpoint{2.131574in}{1.860542in}}%
\pgfpathlineto{\pgfqpoint{2.132080in}{1.866027in}}%
\pgfpathlineto{\pgfqpoint{2.134523in}{1.917064in}}%
\pgfpathlineto{\pgfqpoint{2.135449in}{1.913108in}}%
\pgfpathlineto{\pgfqpoint{2.135618in}{1.913003in}}%
\pgfpathlineto{\pgfqpoint{2.135955in}{1.914033in}}%
\pgfpathlineto{\pgfqpoint{2.137303in}{1.928352in}}%
\pgfpathlineto{\pgfqpoint{2.137977in}{1.932654in}}%
\pgfpathlineto{\pgfqpoint{2.138567in}{1.930355in}}%
\pgfpathlineto{\pgfqpoint{2.140167in}{1.920983in}}%
\pgfpathlineto{\pgfqpoint{2.140757in}{1.922315in}}%
\pgfpathlineto{\pgfqpoint{2.142358in}{1.931494in}}%
\pgfpathlineto{\pgfqpoint{2.143200in}{1.928693in}}%
\pgfpathlineto{\pgfqpoint{2.144632in}{1.921493in}}%
\pgfpathlineto{\pgfqpoint{2.145138in}{1.923218in}}%
\pgfpathlineto{\pgfqpoint{2.147244in}{1.938807in}}%
\pgfpathlineto{\pgfqpoint{2.147833in}{1.936304in}}%
\pgfpathlineto{\pgfqpoint{2.150108in}{1.927123in}}%
\pgfpathlineto{\pgfqpoint{2.151119in}{1.925984in}}%
\pgfpathlineto{\pgfqpoint{2.152804in}{1.914068in}}%
\pgfpathlineto{\pgfqpoint{2.154741in}{1.909412in}}%
\pgfpathlineto{\pgfqpoint{2.155331in}{1.907964in}}%
\pgfpathlineto{\pgfqpoint{2.166030in}{1.858291in}}%
\pgfpathlineto{\pgfqpoint{2.168221in}{1.857284in}}%
\pgfpathlineto{\pgfqpoint{2.170242in}{1.850739in}}%
\pgfpathlineto{\pgfqpoint{2.171001in}{1.851816in}}%
\pgfpathlineto{\pgfqpoint{2.172938in}{1.856739in}}%
\pgfpathlineto{\pgfqpoint{2.173612in}{1.855099in}}%
\pgfpathlineto{\pgfqpoint{2.175550in}{1.848734in}}%
\pgfpathlineto{\pgfqpoint{2.176140in}{1.849334in}}%
\pgfpathlineto{\pgfqpoint{2.177319in}{1.853651in}}%
\pgfpathlineto{\pgfqpoint{2.177656in}{1.854161in}}%
\pgfpathlineto{\pgfqpoint{2.178161in}{1.853178in}}%
\pgfpathlineto{\pgfqpoint{2.180183in}{1.846832in}}%
\pgfpathlineto{\pgfqpoint{2.180857in}{1.847920in}}%
\pgfpathlineto{\pgfqpoint{2.182289in}{1.851724in}}%
\pgfpathlineto{\pgfqpoint{2.182879in}{1.850589in}}%
\pgfpathlineto{\pgfqpoint{2.184985in}{1.844021in}}%
\pgfpathlineto{\pgfqpoint{2.185575in}{1.845399in}}%
\pgfpathlineto{\pgfqpoint{2.186586in}{1.847497in}}%
\pgfpathlineto{\pgfqpoint{2.187091in}{1.846911in}}%
\pgfpathlineto{\pgfqpoint{2.189197in}{1.840932in}}%
\pgfpathlineto{\pgfqpoint{2.190377in}{1.841956in}}%
\pgfpathlineto{\pgfqpoint{2.191388in}{1.841204in}}%
\pgfpathlineto{\pgfqpoint{2.201160in}{1.815287in}}%
\pgfpathlineto{\pgfqpoint{2.201750in}{1.816116in}}%
\pgfpathlineto{\pgfqpoint{2.203351in}{1.819478in}}%
\pgfpathlineto{\pgfqpoint{2.203940in}{1.818508in}}%
\pgfpathlineto{\pgfqpoint{2.204614in}{1.818003in}}%
\pgfpathlineto{\pgfqpoint{2.204951in}{1.818629in}}%
\pgfpathlineto{\pgfqpoint{2.206046in}{1.826340in}}%
\pgfpathlineto{\pgfqpoint{2.206973in}{1.830331in}}%
\pgfpathlineto{\pgfqpoint{2.207479in}{1.829417in}}%
\pgfpathlineto{\pgfqpoint{2.208489in}{1.827479in}}%
\pgfpathlineto{\pgfqpoint{2.208995in}{1.828318in}}%
\pgfpathlineto{\pgfqpoint{2.210511in}{1.831343in}}%
\pgfpathlineto{\pgfqpoint{2.211017in}{1.830682in}}%
\pgfpathlineto{\pgfqpoint{2.218767in}{1.813237in}}%
\pgfpathlineto{\pgfqpoint{2.219189in}{1.813493in}}%
\pgfpathlineto{\pgfqpoint{2.219947in}{1.813215in}}%
\pgfpathlineto{\pgfqpoint{2.220115in}{1.812908in}}%
\pgfpathlineto{\pgfqpoint{2.221126in}{1.811392in}}%
\pgfpathlineto{\pgfqpoint{2.221547in}{1.812084in}}%
\pgfpathlineto{\pgfqpoint{2.223569in}{1.818080in}}%
\pgfpathlineto{\pgfqpoint{2.224243in}{1.817052in}}%
\pgfpathlineto{\pgfqpoint{2.228034in}{1.811280in}}%
\pgfpathlineto{\pgfqpoint{2.230814in}{1.806968in}}%
\pgfpathlineto{\pgfqpoint{2.231320in}{1.807842in}}%
\pgfpathlineto{\pgfqpoint{2.232583in}{1.811068in}}%
\pgfpathlineto{\pgfqpoint{2.233173in}{1.810284in}}%
\pgfpathlineto{\pgfqpoint{2.234521in}{1.808907in}}%
\pgfpathlineto{\pgfqpoint{2.234858in}{1.809285in}}%
\pgfpathlineto{\pgfqpoint{2.236122in}{1.811970in}}%
\pgfpathlineto{\pgfqpoint{2.236796in}{1.810916in}}%
\pgfpathlineto{\pgfqpoint{2.242019in}{1.801226in}}%
\pgfpathlineto{\pgfqpoint{2.242187in}{1.801388in}}%
\pgfpathlineto{\pgfqpoint{2.242945in}{1.804472in}}%
\pgfpathlineto{\pgfqpoint{2.244883in}{1.820285in}}%
\pgfpathlineto{\pgfqpoint{2.245978in}{1.819533in}}%
\pgfpathlineto{\pgfqpoint{2.246736in}{1.827013in}}%
\pgfpathlineto{\pgfqpoint{2.248927in}{1.859692in}}%
\pgfpathlineto{\pgfqpoint{2.249685in}{1.857275in}}%
\pgfpathlineto{\pgfqpoint{2.250949in}{1.854659in}}%
\pgfpathlineto{\pgfqpoint{2.251370in}{1.854781in}}%
\pgfpathlineto{\pgfqpoint{2.252128in}{1.853754in}}%
\pgfpathlineto{\pgfqpoint{2.255835in}{1.841208in}}%
\pgfpathlineto{\pgfqpoint{2.256340in}{1.843052in}}%
\pgfpathlineto{\pgfqpoint{2.258362in}{1.858249in}}%
\pgfpathlineto{\pgfqpoint{2.259120in}{1.856065in}}%
\pgfpathlineto{\pgfqpoint{2.259542in}{1.855426in}}%
\pgfpathlineto{\pgfqpoint{2.259879in}{1.856531in}}%
\pgfpathlineto{\pgfqpoint{2.261058in}{1.871572in}}%
\pgfpathlineto{\pgfqpoint{2.262153in}{1.881158in}}%
\pgfpathlineto{\pgfqpoint{2.262659in}{1.879818in}}%
\pgfpathlineto{\pgfqpoint{2.263922in}{1.875268in}}%
\pgfpathlineto{\pgfqpoint{2.264344in}{1.876404in}}%
\pgfpathlineto{\pgfqpoint{2.265607in}{1.881985in}}%
\pgfpathlineto{\pgfqpoint{2.266197in}{1.880518in}}%
\pgfpathlineto{\pgfqpoint{2.280266in}{1.826933in}}%
\pgfpathlineto{\pgfqpoint{2.280856in}{1.828615in}}%
\pgfpathlineto{\pgfqpoint{2.282709in}{1.836474in}}%
\pgfpathlineto{\pgfqpoint{2.283214in}{1.835462in}}%
\pgfpathlineto{\pgfqpoint{2.284899in}{1.832450in}}%
\pgfpathlineto{\pgfqpoint{2.285236in}{1.832701in}}%
\pgfpathlineto{\pgfqpoint{2.286163in}{1.833155in}}%
\pgfpathlineto{\pgfqpoint{2.286500in}{1.832546in}}%
\pgfpathlineto{\pgfqpoint{2.297367in}{1.806978in}}%
\pgfpathlineto{\pgfqpoint{2.298547in}{1.806057in}}%
\pgfpathlineto{\pgfqpoint{2.303349in}{1.799773in}}%
\pgfpathlineto{\pgfqpoint{2.304444in}{1.798762in}}%
\pgfpathlineto{\pgfqpoint{2.305792in}{1.797221in}}%
\pgfpathlineto{\pgfqpoint{2.306213in}{1.797999in}}%
\pgfpathlineto{\pgfqpoint{2.308151in}{1.803776in}}%
\pgfpathlineto{\pgfqpoint{2.308740in}{1.803125in}}%
\pgfpathlineto{\pgfqpoint{2.310088in}{1.801338in}}%
\pgfpathlineto{\pgfqpoint{2.310510in}{1.802051in}}%
\pgfpathlineto{\pgfqpoint{2.311857in}{1.804758in}}%
\pgfpathlineto{\pgfqpoint{2.312363in}{1.804160in}}%
\pgfpathlineto{\pgfqpoint{2.314216in}{1.801949in}}%
\pgfpathlineto{\pgfqpoint{2.314638in}{1.802230in}}%
\pgfpathlineto{\pgfqpoint{2.315817in}{1.801614in}}%
\pgfpathlineto{\pgfqpoint{2.317755in}{1.800730in}}%
\pgfpathlineto{\pgfqpoint{2.319355in}{1.801645in}}%
\pgfpathlineto{\pgfqpoint{2.319776in}{1.800961in}}%
\pgfpathlineto{\pgfqpoint{2.320535in}{1.800307in}}%
\pgfpathlineto{\pgfqpoint{2.320956in}{1.801097in}}%
\pgfpathlineto{\pgfqpoint{2.322220in}{1.804612in}}%
\pgfpathlineto{\pgfqpoint{2.322894in}{1.803805in}}%
\pgfpathlineto{\pgfqpoint{2.323483in}{1.803298in}}%
\pgfpathlineto{\pgfqpoint{2.323904in}{1.804191in}}%
\pgfpathlineto{\pgfqpoint{2.324831in}{1.813537in}}%
\pgfpathlineto{\pgfqpoint{2.326348in}{1.825429in}}%
\pgfpathlineto{\pgfqpoint{2.326685in}{1.824991in}}%
\pgfpathlineto{\pgfqpoint{2.327358in}{1.823854in}}%
\pgfpathlineto{\pgfqpoint{2.327780in}{1.824779in}}%
\pgfpathlineto{\pgfqpoint{2.330054in}{1.840137in}}%
\pgfpathlineto{\pgfqpoint{2.331318in}{1.837119in}}%
\pgfpathlineto{\pgfqpoint{2.335699in}{1.824126in}}%
\pgfpathlineto{\pgfqpoint{2.336036in}{1.824375in}}%
\pgfpathlineto{\pgfqpoint{2.337552in}{1.825195in}}%
\pgfpathlineto{\pgfqpoint{2.337805in}{1.824934in}}%
\pgfpathlineto{\pgfqpoint{2.338984in}{1.823041in}}%
\pgfpathlineto{\pgfqpoint{2.339658in}{1.824039in}}%
\pgfpathlineto{\pgfqpoint{2.340248in}{1.824106in}}%
\pgfpathlineto{\pgfqpoint{2.340669in}{1.823547in}}%
\pgfpathlineto{\pgfqpoint{2.343533in}{1.820687in}}%
\pgfpathlineto{\pgfqpoint{2.344713in}{1.819992in}}%
\pgfpathlineto{\pgfqpoint{2.346566in}{1.816445in}}%
\pgfpathlineto{\pgfqpoint{2.347156in}{1.816855in}}%
\pgfpathlineto{\pgfqpoint{2.348083in}{1.816038in}}%
\pgfpathlineto{\pgfqpoint{2.356507in}{1.801025in}}%
\pgfpathlineto{\pgfqpoint{2.357181in}{1.801127in}}%
\pgfpathlineto{\pgfqpoint{2.357434in}{1.801640in}}%
\pgfpathlineto{\pgfqpoint{2.358613in}{1.808674in}}%
\pgfpathlineto{\pgfqpoint{2.359540in}{1.811523in}}%
\pgfpathlineto{\pgfqpoint{2.360045in}{1.810834in}}%
\pgfpathlineto{\pgfqpoint{2.361141in}{1.809292in}}%
\pgfpathlineto{\pgfqpoint{2.361562in}{1.809923in}}%
\pgfpathlineto{\pgfqpoint{2.363584in}{1.812567in}}%
\pgfpathlineto{\pgfqpoint{2.363836in}{1.812389in}}%
\pgfpathlineto{\pgfqpoint{2.365437in}{1.809262in}}%
\pgfpathlineto{\pgfqpoint{2.369565in}{1.802032in}}%
\pgfpathlineto{\pgfqpoint{2.369649in}{1.802082in}}%
\pgfpathlineto{\pgfqpoint{2.370576in}{1.804570in}}%
\pgfpathlineto{\pgfqpoint{2.371334in}{1.805762in}}%
\pgfpathlineto{\pgfqpoint{2.371840in}{1.805253in}}%
\pgfpathlineto{\pgfqpoint{2.375968in}{1.800310in}}%
\pgfpathlineto{\pgfqpoint{2.379590in}{1.796994in}}%
\pgfpathlineto{\pgfqpoint{2.382286in}{1.799855in}}%
\pgfpathlineto{\pgfqpoint{2.384139in}{1.824071in}}%
\pgfpathlineto{\pgfqpoint{2.384898in}{1.827960in}}%
\pgfpathlineto{\pgfqpoint{2.385487in}{1.826830in}}%
\pgfpathlineto{\pgfqpoint{2.385993in}{1.825868in}}%
\pgfpathlineto{\pgfqpoint{2.386414in}{1.827338in}}%
\pgfpathlineto{\pgfqpoint{2.389110in}{1.853305in}}%
\pgfpathlineto{\pgfqpoint{2.390373in}{1.849007in}}%
\pgfpathlineto{\pgfqpoint{2.395849in}{1.831908in}}%
\pgfpathlineto{\pgfqpoint{2.398208in}{1.825665in}}%
\pgfpathlineto{\pgfqpoint{2.398798in}{1.827468in}}%
\pgfpathlineto{\pgfqpoint{2.400314in}{1.831932in}}%
\pgfpathlineto{\pgfqpoint{2.400651in}{1.831627in}}%
\pgfpathlineto{\pgfqpoint{2.401746in}{1.829615in}}%
\pgfpathlineto{\pgfqpoint{2.402168in}{1.830819in}}%
\pgfpathlineto{\pgfqpoint{2.404274in}{1.845150in}}%
\pgfpathlineto{\pgfqpoint{2.405285in}{1.842193in}}%
\pgfpathlineto{\pgfqpoint{2.405874in}{1.840925in}}%
\pgfpathlineto{\pgfqpoint{2.406380in}{1.841864in}}%
\pgfpathlineto{\pgfqpoint{2.408402in}{1.846698in}}%
\pgfpathlineto{\pgfqpoint{2.408739in}{1.846263in}}%
\pgfpathlineto{\pgfqpoint{2.417669in}{1.824956in}}%
\pgfpathlineto{\pgfqpoint{2.417753in}{1.824995in}}%
\pgfpathlineto{\pgfqpoint{2.418595in}{1.827401in}}%
\pgfpathlineto{\pgfqpoint{2.419943in}{1.830526in}}%
\pgfpathlineto{\pgfqpoint{2.420364in}{1.829982in}}%
\pgfpathlineto{\pgfqpoint{2.422049in}{1.826489in}}%
\pgfpathlineto{\pgfqpoint{2.422639in}{1.827381in}}%
\pgfpathlineto{\pgfqpoint{2.423650in}{1.828841in}}%
\pgfpathlineto{\pgfqpoint{2.424155in}{1.828172in}}%
\pgfpathlineto{\pgfqpoint{2.426683in}{1.825016in}}%
\pgfpathlineto{\pgfqpoint{2.427862in}{1.822438in}}%
\pgfpathlineto{\pgfqpoint{2.431737in}{1.814313in}}%
\pgfpathlineto{\pgfqpoint{2.435191in}{1.809463in}}%
\pgfpathlineto{\pgfqpoint{2.436371in}{1.811112in}}%
\pgfpathlineto{\pgfqpoint{2.438224in}{1.817619in}}%
\pgfpathlineto{\pgfqpoint{2.438982in}{1.816356in}}%
\pgfpathlineto{\pgfqpoint{2.439741in}{1.815621in}}%
\pgfpathlineto{\pgfqpoint{2.440162in}{1.816315in}}%
\pgfpathlineto{\pgfqpoint{2.442100in}{1.824092in}}%
\pgfpathlineto{\pgfqpoint{2.443195in}{1.821675in}}%
\pgfpathlineto{\pgfqpoint{2.444121in}{1.820726in}}%
\pgfpathlineto{\pgfqpoint{2.444627in}{1.821143in}}%
\pgfpathlineto{\pgfqpoint{2.445301in}{1.820999in}}%
\pgfpathlineto{\pgfqpoint{2.445554in}{1.820537in}}%
\pgfpathlineto{\pgfqpoint{2.446817in}{1.818082in}}%
\pgfpathlineto{\pgfqpoint{2.447323in}{1.819080in}}%
\pgfpathlineto{\pgfqpoint{2.449260in}{1.822585in}}%
\pgfpathlineto{\pgfqpoint{2.449513in}{1.822306in}}%
\pgfpathlineto{\pgfqpoint{2.460802in}{1.804195in}}%
\pgfpathlineto{\pgfqpoint{2.462234in}{1.804443in}}%
\pgfpathlineto{\pgfqpoint{2.463413in}{1.804016in}}%
\pgfpathlineto{\pgfqpoint{2.465941in}{1.800629in}}%
\pgfpathlineto{\pgfqpoint{2.466530in}{1.801307in}}%
\pgfpathlineto{\pgfqpoint{2.468215in}{1.801704in}}%
\pgfpathlineto{\pgfqpoint{2.470069in}{1.798573in}}%
\pgfpathlineto{\pgfqpoint{2.473270in}{1.795389in}}%
\pgfpathlineto{\pgfqpoint{2.477061in}{1.793324in}}%
\pgfpathlineto{\pgfqpoint{2.478156in}{1.792758in}}%
\pgfpathlineto{\pgfqpoint{2.479336in}{1.791454in}}%
\pgfpathlineto{\pgfqpoint{2.479673in}{1.792113in}}%
\pgfpathlineto{\pgfqpoint{2.480852in}{1.801384in}}%
\pgfpathlineto{\pgfqpoint{2.483042in}{1.821648in}}%
\pgfpathlineto{\pgfqpoint{2.483801in}{1.820253in}}%
\pgfpathlineto{\pgfqpoint{2.484390in}{1.819507in}}%
\pgfpathlineto{\pgfqpoint{2.484812in}{1.820324in}}%
\pgfpathlineto{\pgfqpoint{2.487086in}{1.830036in}}%
\pgfpathlineto{\pgfqpoint{2.488013in}{1.827877in}}%
\pgfpathlineto{\pgfqpoint{2.491972in}{1.817257in}}%
\pgfpathlineto{\pgfqpoint{2.492225in}{1.817451in}}%
\pgfpathlineto{\pgfqpoint{2.493067in}{1.819917in}}%
\pgfpathlineto{\pgfqpoint{2.494584in}{1.828095in}}%
\pgfpathlineto{\pgfqpoint{2.495258in}{1.826860in}}%
\pgfpathlineto{\pgfqpoint{2.495763in}{1.826112in}}%
\pgfpathlineto{\pgfqpoint{2.496100in}{1.827171in}}%
\pgfpathlineto{\pgfqpoint{2.497364in}{1.842433in}}%
\pgfpathlineto{\pgfqpoint{2.498796in}{1.853508in}}%
\pgfpathlineto{\pgfqpoint{2.499217in}{1.852487in}}%
\pgfpathlineto{\pgfqpoint{2.505114in}{1.834409in}}%
\pgfpathlineto{\pgfqpoint{2.505957in}{1.833837in}}%
\pgfpathlineto{\pgfqpoint{2.506294in}{1.834267in}}%
\pgfpathlineto{\pgfqpoint{2.507389in}{1.840214in}}%
\pgfpathlineto{\pgfqpoint{2.507810in}{1.841046in}}%
\pgfpathlineto{\pgfqpoint{2.508316in}{1.839916in}}%
\pgfpathlineto{\pgfqpoint{2.509411in}{1.837002in}}%
\pgfpathlineto{\pgfqpoint{2.509832in}{1.838118in}}%
\pgfpathlineto{\pgfqpoint{2.511517in}{1.844623in}}%
\pgfpathlineto{\pgfqpoint{2.512022in}{1.843899in}}%
\pgfpathlineto{\pgfqpoint{2.513539in}{1.840028in}}%
\pgfpathlineto{\pgfqpoint{2.514129in}{1.840970in}}%
\pgfpathlineto{\pgfqpoint{2.515055in}{1.842861in}}%
\pgfpathlineto{\pgfqpoint{2.515561in}{1.841804in}}%
\pgfpathlineto{\pgfqpoint{2.522722in}{1.820352in}}%
\pgfpathlineto{\pgfqpoint{2.522806in}{1.820378in}}%
\pgfpathlineto{\pgfqpoint{2.524912in}{1.821184in}}%
\pgfpathlineto{\pgfqpoint{2.525249in}{1.820576in}}%
\pgfpathlineto{\pgfqpoint{2.535358in}{1.802421in}}%
\pgfpathlineto{\pgfqpoint{2.537127in}{1.801874in}}%
\pgfpathlineto{\pgfqpoint{2.538475in}{1.800734in}}%
\pgfpathlineto{\pgfqpoint{2.538897in}{1.801114in}}%
\pgfpathlineto{\pgfqpoint{2.540581in}{1.802877in}}%
\pgfpathlineto{\pgfqpoint{2.541087in}{1.802098in}}%
\pgfpathlineto{\pgfqpoint{2.542519in}{1.800410in}}%
\pgfpathlineto{\pgfqpoint{2.542940in}{1.800790in}}%
\pgfpathlineto{\pgfqpoint{2.545299in}{1.801480in}}%
\pgfpathlineto{\pgfqpoint{2.549427in}{1.797085in}}%
\pgfpathlineto{\pgfqpoint{2.550017in}{1.798294in}}%
\pgfpathlineto{\pgfqpoint{2.552039in}{1.801894in}}%
\pgfpathlineto{\pgfqpoint{2.552291in}{1.801728in}}%
\pgfpathlineto{\pgfqpoint{2.553808in}{1.800183in}}%
\pgfpathlineto{\pgfqpoint{2.554229in}{1.801031in}}%
\pgfpathlineto{\pgfqpoint{2.556672in}{1.806877in}}%
\pgfpathlineto{\pgfqpoint{2.557009in}{1.806603in}}%
\pgfpathlineto{\pgfqpoint{2.563580in}{1.797051in}}%
\pgfpathlineto{\pgfqpoint{2.564170in}{1.798024in}}%
\pgfpathlineto{\pgfqpoint{2.566360in}{1.805478in}}%
\pgfpathlineto{\pgfqpoint{2.567371in}{1.804210in}}%
\pgfpathlineto{\pgfqpoint{2.573521in}{1.794939in}}%
\pgfpathlineto{\pgfqpoint{2.575796in}{1.794431in}}%
\pgfpathlineto{\pgfqpoint{2.577144in}{1.794342in}}%
\pgfpathlineto{\pgfqpoint{2.577228in}{1.794223in}}%
\pgfpathlineto{\pgfqpoint{2.585063in}{1.784719in}}%
\pgfpathlineto{\pgfqpoint{2.588011in}{1.783772in}}%
\pgfpathlineto{\pgfqpoint{2.589780in}{1.783543in}}%
\pgfpathlineto{\pgfqpoint{2.589864in}{1.783676in}}%
\pgfpathlineto{\pgfqpoint{2.591718in}{1.786473in}}%
\pgfpathlineto{\pgfqpoint{2.592392in}{1.786062in}}%
\pgfpathlineto{\pgfqpoint{2.597783in}{1.784729in}}%
\pgfpathlineto{\pgfqpoint{2.600395in}{1.786058in}}%
\pgfpathlineto{\pgfqpoint{2.604607in}{1.802489in}}%
\pgfpathlineto{\pgfqpoint{2.606292in}{1.809373in}}%
\pgfpathlineto{\pgfqpoint{2.606461in}{1.809238in}}%
\pgfpathlineto{\pgfqpoint{2.610252in}{1.802389in}}%
\pgfpathlineto{\pgfqpoint{2.611852in}{1.800991in}}%
\pgfpathlineto{\pgfqpoint{2.612021in}{1.801139in}}%
\pgfpathlineto{\pgfqpoint{2.612779in}{1.803909in}}%
\pgfpathlineto{\pgfqpoint{2.614127in}{1.809788in}}%
\pgfpathlineto{\pgfqpoint{2.614717in}{1.809106in}}%
\pgfpathlineto{\pgfqpoint{2.615391in}{1.808405in}}%
\pgfpathlineto{\pgfqpoint{2.615812in}{1.809307in}}%
\pgfpathlineto{\pgfqpoint{2.617328in}{1.819946in}}%
\pgfpathlineto{\pgfqpoint{2.618255in}{1.823746in}}%
\pgfpathlineto{\pgfqpoint{2.618760in}{1.822995in}}%
\pgfpathlineto{\pgfqpoint{2.620192in}{1.820615in}}%
\pgfpathlineto{\pgfqpoint{2.620698in}{1.820960in}}%
\pgfpathlineto{\pgfqpoint{2.621793in}{1.820361in}}%
\pgfpathlineto{\pgfqpoint{2.626427in}{1.809663in}}%
\pgfpathlineto{\pgfqpoint{2.629628in}{1.804503in}}%
\pgfpathlineto{\pgfqpoint{2.629712in}{1.804558in}}%
\pgfpathlineto{\pgfqpoint{2.631481in}{1.807383in}}%
\pgfpathlineto{\pgfqpoint{2.632324in}{1.805920in}}%
\pgfpathlineto{\pgfqpoint{2.633082in}{1.804968in}}%
\pgfpathlineto{\pgfqpoint{2.633503in}{1.805694in}}%
\pgfpathlineto{\pgfqpoint{2.635862in}{1.815027in}}%
\pgfpathlineto{\pgfqpoint{2.636957in}{1.812914in}}%
\pgfpathlineto{\pgfqpoint{2.637968in}{1.811481in}}%
\pgfpathlineto{\pgfqpoint{2.638389in}{1.812043in}}%
\pgfpathlineto{\pgfqpoint{2.640580in}{1.816908in}}%
\pgfpathlineto{\pgfqpoint{2.641338in}{1.815457in}}%
\pgfpathlineto{\pgfqpoint{2.643191in}{1.812854in}}%
\pgfpathlineto{\pgfqpoint{2.643444in}{1.813008in}}%
\pgfpathlineto{\pgfqpoint{2.644960in}{1.812987in}}%
\pgfpathlineto{\pgfqpoint{2.646730in}{1.809239in}}%
\pgfpathlineto{\pgfqpoint{2.649257in}{1.806519in}}%
\pgfpathlineto{\pgfqpoint{2.651616in}{1.803442in}}%
\pgfpathlineto{\pgfqpoint{2.651953in}{1.803795in}}%
\pgfpathlineto{\pgfqpoint{2.653301in}{1.805649in}}%
\pgfpathlineto{\pgfqpoint{2.653806in}{1.804941in}}%
\pgfpathlineto{\pgfqpoint{2.654985in}{1.803176in}}%
\pgfpathlineto{\pgfqpoint{2.655407in}{1.804043in}}%
\pgfpathlineto{\pgfqpoint{2.657092in}{1.808078in}}%
\pgfpathlineto{\pgfqpoint{2.657513in}{1.807704in}}%
\pgfpathlineto{\pgfqpoint{2.658861in}{1.806257in}}%
\pgfpathlineto{\pgfqpoint{2.659366in}{1.806851in}}%
\pgfpathlineto{\pgfqpoint{2.660546in}{1.807411in}}%
\pgfpathlineto{\pgfqpoint{2.660883in}{1.807053in}}%
\pgfpathlineto{\pgfqpoint{2.662062in}{1.805686in}}%
\pgfpathlineto{\pgfqpoint{2.662567in}{1.806395in}}%
\pgfpathlineto{\pgfqpoint{2.664337in}{1.809038in}}%
\pgfpathlineto{\pgfqpoint{2.664842in}{1.808542in}}%
\pgfpathlineto{\pgfqpoint{2.677310in}{1.792652in}}%
\pgfpathlineto{\pgfqpoint{2.678153in}{1.793499in}}%
\pgfpathlineto{\pgfqpoint{2.680259in}{1.796366in}}%
\pgfpathlineto{\pgfqpoint{2.680427in}{1.796250in}}%
\pgfpathlineto{\pgfqpoint{2.687841in}{1.790200in}}%
\pgfpathlineto{\pgfqpoint{2.688009in}{1.790419in}}%
\pgfpathlineto{\pgfqpoint{2.690789in}{1.796615in}}%
\pgfpathlineto{\pgfqpoint{2.691800in}{1.795189in}}%
\pgfpathlineto{\pgfqpoint{2.695086in}{1.792044in}}%
\pgfpathlineto{\pgfqpoint{2.700309in}{1.786898in}}%
\pgfpathlineto{\pgfqpoint{2.701910in}{1.787011in}}%
\pgfpathlineto{\pgfqpoint{2.703173in}{1.786759in}}%
\pgfpathlineto{\pgfqpoint{2.709913in}{1.781554in}}%
\pgfpathlineto{\pgfqpoint{2.713283in}{1.781473in}}%
\pgfpathlineto{\pgfqpoint{2.717748in}{1.788640in}}%
\pgfpathlineto{\pgfqpoint{2.719433in}{1.794822in}}%
\pgfpathlineto{\pgfqpoint{2.719770in}{1.794583in}}%
\pgfpathlineto{\pgfqpoint{2.720949in}{1.793847in}}%
\pgfpathlineto{\pgfqpoint{2.721202in}{1.794318in}}%
\pgfpathlineto{\pgfqpoint{2.723476in}{1.800598in}}%
\pgfpathlineto{\pgfqpoint{2.724150in}{1.799743in}}%
\pgfpathlineto{\pgfqpoint{2.726256in}{1.798239in}}%
\pgfpathlineto{\pgfqpoint{2.727436in}{1.797548in}}%
\pgfpathlineto{\pgfqpoint{2.731143in}{1.793126in}}%
\pgfpathlineto{\pgfqpoint{2.731648in}{1.793610in}}%
\pgfpathlineto{\pgfqpoint{2.733501in}{1.793706in}}%
\pgfpathlineto{\pgfqpoint{2.734512in}{1.793657in}}%
\pgfpathlineto{\pgfqpoint{2.734765in}{1.794124in}}%
\pgfpathlineto{\pgfqpoint{2.736029in}{1.799120in}}%
\pgfpathlineto{\pgfqpoint{2.737377in}{1.804371in}}%
\pgfpathlineto{\pgfqpoint{2.737798in}{1.804021in}}%
\pgfpathlineto{\pgfqpoint{2.738640in}{1.803258in}}%
\pgfpathlineto{\pgfqpoint{2.738977in}{1.803894in}}%
\pgfpathlineto{\pgfqpoint{2.740072in}{1.810946in}}%
\pgfpathlineto{\pgfqpoint{2.741757in}{1.819375in}}%
\pgfpathlineto{\pgfqpoint{2.742094in}{1.819078in}}%
\pgfpathlineto{\pgfqpoint{2.743442in}{1.816740in}}%
\pgfpathlineto{\pgfqpoint{2.743863in}{1.817826in}}%
\pgfpathlineto{\pgfqpoint{2.745801in}{1.824958in}}%
\pgfpathlineto{\pgfqpoint{2.746307in}{1.824244in}}%
\pgfpathlineto{\pgfqpoint{2.747233in}{1.822845in}}%
\pgfpathlineto{\pgfqpoint{2.747739in}{1.823711in}}%
\pgfpathlineto{\pgfqpoint{2.749087in}{1.828100in}}%
\pgfpathlineto{\pgfqpoint{2.749761in}{1.826827in}}%
\pgfpathlineto{\pgfqpoint{2.750519in}{1.825602in}}%
\pgfpathlineto{\pgfqpoint{2.750940in}{1.826638in}}%
\pgfpathlineto{\pgfqpoint{2.752541in}{1.835905in}}%
\pgfpathlineto{\pgfqpoint{2.753467in}{1.833758in}}%
\pgfpathlineto{\pgfqpoint{2.754141in}{1.832744in}}%
\pgfpathlineto{\pgfqpoint{2.754647in}{1.833493in}}%
\pgfpathlineto{\pgfqpoint{2.755573in}{1.835164in}}%
\pgfpathlineto{\pgfqpoint{2.756079in}{1.834407in}}%
\pgfpathlineto{\pgfqpoint{2.757258in}{1.831802in}}%
\pgfpathlineto{\pgfqpoint{2.757848in}{1.832812in}}%
\pgfpathlineto{\pgfqpoint{2.759112in}{1.835564in}}%
\pgfpathlineto{\pgfqpoint{2.759617in}{1.834549in}}%
\pgfpathlineto{\pgfqpoint{2.760375in}{1.832881in}}%
\pgfpathlineto{\pgfqpoint{2.760881in}{1.834236in}}%
\pgfpathlineto{\pgfqpoint{2.762650in}{1.844467in}}%
\pgfpathlineto{\pgfqpoint{2.763408in}{1.842591in}}%
\pgfpathlineto{\pgfqpoint{2.764251in}{1.840863in}}%
\pgfpathlineto{\pgfqpoint{2.764672in}{1.841715in}}%
\pgfpathlineto{\pgfqpoint{2.765599in}{1.844571in}}%
\pgfpathlineto{\pgfqpoint{2.766188in}{1.843194in}}%
\pgfpathlineto{\pgfqpoint{2.767115in}{1.840933in}}%
\pgfpathlineto{\pgfqpoint{2.767620in}{1.841816in}}%
\pgfpathlineto{\pgfqpoint{2.768968in}{1.844882in}}%
\pgfpathlineto{\pgfqpoint{2.769474in}{1.843786in}}%
\pgfpathlineto{\pgfqpoint{2.770232in}{1.842224in}}%
\pgfpathlineto{\pgfqpoint{2.770653in}{1.843393in}}%
\pgfpathlineto{\pgfqpoint{2.772170in}{1.852862in}}%
\pgfpathlineto{\pgfqpoint{2.772928in}{1.850788in}}%
\pgfpathlineto{\pgfqpoint{2.773602in}{1.849142in}}%
\pgfpathlineto{\pgfqpoint{2.774107in}{1.850271in}}%
\pgfpathlineto{\pgfqpoint{2.775876in}{1.857589in}}%
\pgfpathlineto{\pgfqpoint{2.776550in}{1.855680in}}%
\pgfpathlineto{\pgfqpoint{2.778235in}{1.851286in}}%
\pgfpathlineto{\pgfqpoint{2.778488in}{1.851704in}}%
\pgfpathlineto{\pgfqpoint{2.780257in}{1.860315in}}%
\pgfpathlineto{\pgfqpoint{2.781268in}{1.857620in}}%
\pgfpathlineto{\pgfqpoint{2.784048in}{1.850117in}}%
\pgfpathlineto{\pgfqpoint{2.785396in}{1.846887in}}%
\pgfpathlineto{\pgfqpoint{2.792978in}{1.821386in}}%
\pgfpathlineto{\pgfqpoint{2.798875in}{1.806962in}}%
\pgfpathlineto{\pgfqpoint{2.799296in}{1.807021in}}%
\pgfpathlineto{\pgfqpoint{2.799633in}{1.807752in}}%
\pgfpathlineto{\pgfqpoint{2.800728in}{1.809536in}}%
\pgfpathlineto{\pgfqpoint{2.801234in}{1.808972in}}%
\pgfpathlineto{\pgfqpoint{2.802329in}{1.807375in}}%
\pgfpathlineto{\pgfqpoint{2.802750in}{1.808497in}}%
\pgfpathlineto{\pgfqpoint{2.805025in}{1.818606in}}%
\pgfpathlineto{\pgfqpoint{2.805530in}{1.817941in}}%
\pgfpathlineto{\pgfqpoint{2.809321in}{1.812975in}}%
\pgfpathlineto{\pgfqpoint{2.812354in}{1.815510in}}%
\pgfpathlineto{\pgfqpoint{2.814460in}{1.829386in}}%
\pgfpathlineto{\pgfqpoint{2.815471in}{1.826645in}}%
\pgfpathlineto{\pgfqpoint{2.819178in}{1.818578in}}%
\pgfpathlineto{\pgfqpoint{2.821368in}{1.814470in}}%
\pgfpathlineto{\pgfqpoint{2.821958in}{1.815840in}}%
\pgfpathlineto{\pgfqpoint{2.823222in}{1.819658in}}%
\pgfpathlineto{\pgfqpoint{2.823727in}{1.818873in}}%
\pgfpathlineto{\pgfqpoint{2.824907in}{1.816532in}}%
\pgfpathlineto{\pgfqpoint{2.825412in}{1.817626in}}%
\pgfpathlineto{\pgfqpoint{2.826844in}{1.820513in}}%
\pgfpathlineto{\pgfqpoint{2.827266in}{1.820040in}}%
\pgfpathlineto{\pgfqpoint{2.828276in}{1.818511in}}%
\pgfpathlineto{\pgfqpoint{2.828782in}{1.819177in}}%
\pgfpathlineto{\pgfqpoint{2.830214in}{1.822267in}}%
\pgfpathlineto{\pgfqpoint{2.830888in}{1.821103in}}%
\pgfpathlineto{\pgfqpoint{2.831646in}{1.820271in}}%
\pgfpathlineto{\pgfqpoint{2.832067in}{1.820873in}}%
\pgfpathlineto{\pgfqpoint{2.833078in}{1.827118in}}%
\pgfpathlineto{\pgfqpoint{2.833752in}{1.829425in}}%
\pgfpathlineto{\pgfqpoint{2.834342in}{1.828200in}}%
\pgfpathlineto{\pgfqpoint{2.834932in}{1.827481in}}%
\pgfpathlineto{\pgfqpoint{2.835353in}{1.828478in}}%
\pgfpathlineto{\pgfqpoint{2.840155in}{1.847531in}}%
\pgfpathlineto{\pgfqpoint{2.840323in}{1.847433in}}%
\pgfpathlineto{\pgfqpoint{2.841334in}{1.845231in}}%
\pgfpathlineto{\pgfqpoint{2.841756in}{1.846965in}}%
\pgfpathlineto{\pgfqpoint{2.843356in}{1.857642in}}%
\pgfpathlineto{\pgfqpoint{2.844030in}{1.856141in}}%
\pgfpathlineto{\pgfqpoint{2.845125in}{1.853263in}}%
\pgfpathlineto{\pgfqpoint{2.845631in}{1.853990in}}%
\pgfpathlineto{\pgfqpoint{2.846810in}{1.857957in}}%
\pgfpathlineto{\pgfqpoint{2.847484in}{1.856147in}}%
\pgfpathlineto{\pgfqpoint{2.853381in}{1.834816in}}%
\pgfpathlineto{\pgfqpoint{2.854055in}{1.835753in}}%
\pgfpathlineto{\pgfqpoint{2.855487in}{1.837556in}}%
\pgfpathlineto{\pgfqpoint{2.855824in}{1.837029in}}%
\pgfpathlineto{\pgfqpoint{2.857172in}{1.833902in}}%
\pgfpathlineto{\pgfqpoint{2.857678in}{1.835113in}}%
\pgfpathlineto{\pgfqpoint{2.858689in}{1.838391in}}%
\pgfpathlineto{\pgfqpoint{2.859194in}{1.837288in}}%
\pgfpathlineto{\pgfqpoint{2.860711in}{1.833044in}}%
\pgfpathlineto{\pgfqpoint{2.861216in}{1.833879in}}%
\pgfpathlineto{\pgfqpoint{2.861890in}{1.834786in}}%
\pgfpathlineto{\pgfqpoint{2.862395in}{1.833869in}}%
\pgfpathlineto{\pgfqpoint{2.868293in}{1.818614in}}%
\pgfpathlineto{\pgfqpoint{2.868714in}{1.819318in}}%
\pgfpathlineto{\pgfqpoint{2.869725in}{1.825620in}}%
\pgfpathlineto{\pgfqpoint{2.870736in}{1.831545in}}%
\pgfpathlineto{\pgfqpoint{2.871241in}{1.830445in}}%
\pgfpathlineto{\pgfqpoint{2.871999in}{1.828912in}}%
\pgfpathlineto{\pgfqpoint{2.872421in}{1.830359in}}%
\pgfpathlineto{\pgfqpoint{2.874442in}{1.844272in}}%
\pgfpathlineto{\pgfqpoint{2.875201in}{1.842203in}}%
\pgfpathlineto{\pgfqpoint{2.876043in}{1.840417in}}%
\pgfpathlineto{\pgfqpoint{2.876549in}{1.841257in}}%
\pgfpathlineto{\pgfqpoint{2.878065in}{1.845268in}}%
\pgfpathlineto{\pgfqpoint{2.878570in}{1.844215in}}%
\pgfpathlineto{\pgfqpoint{2.884805in}{1.825823in}}%
\pgfpathlineto{\pgfqpoint{2.884973in}{1.825976in}}%
\pgfpathlineto{\pgfqpoint{2.885731in}{1.830296in}}%
\pgfpathlineto{\pgfqpoint{2.887416in}{1.844973in}}%
\pgfpathlineto{\pgfqpoint{2.888174in}{1.843182in}}%
\pgfpathlineto{\pgfqpoint{2.888427in}{1.842785in}}%
\pgfpathlineto{\pgfqpoint{2.888848in}{1.844369in}}%
\pgfpathlineto{\pgfqpoint{2.890449in}{1.868960in}}%
\pgfpathlineto{\pgfqpoint{2.891544in}{1.879314in}}%
\pgfpathlineto{\pgfqpoint{2.892050in}{1.877460in}}%
\pgfpathlineto{\pgfqpoint{2.892892in}{1.874267in}}%
\pgfpathlineto{\pgfqpoint{2.893397in}{1.875985in}}%
\pgfpathlineto{\pgfqpoint{2.894577in}{1.882272in}}%
\pgfpathlineto{\pgfqpoint{2.895167in}{1.880784in}}%
\pgfpathlineto{\pgfqpoint{2.896515in}{1.875651in}}%
\pgfpathlineto{\pgfqpoint{2.897188in}{1.876739in}}%
\pgfpathlineto{\pgfqpoint{2.897525in}{1.876806in}}%
\pgfpathlineto{\pgfqpoint{2.897862in}{1.876154in}}%
\pgfpathlineto{\pgfqpoint{2.903338in}{1.861961in}}%
\pgfpathlineto{\pgfqpoint{2.903507in}{1.862112in}}%
\pgfpathlineto{\pgfqpoint{2.904770in}{1.864854in}}%
\pgfpathlineto{\pgfqpoint{2.905444in}{1.863259in}}%
\pgfpathlineto{\pgfqpoint{2.907129in}{1.858109in}}%
\pgfpathlineto{\pgfqpoint{2.907635in}{1.859052in}}%
\pgfpathlineto{\pgfqpoint{2.908646in}{1.862060in}}%
\pgfpathlineto{\pgfqpoint{2.909151in}{1.860977in}}%
\pgfpathlineto{\pgfqpoint{2.918587in}{1.830627in}}%
\pgfpathlineto{\pgfqpoint{2.921788in}{1.824872in}}%
\pgfpathlineto{\pgfqpoint{2.922462in}{1.826985in}}%
\pgfpathlineto{\pgfqpoint{2.924231in}{1.834692in}}%
\pgfpathlineto{\pgfqpoint{2.924821in}{1.833404in}}%
\pgfpathlineto{\pgfqpoint{2.926169in}{1.829938in}}%
\pgfpathlineto{\pgfqpoint{2.926674in}{1.830580in}}%
\pgfpathlineto{\pgfqpoint{2.927180in}{1.831129in}}%
\pgfpathlineto{\pgfqpoint{2.927685in}{1.830273in}}%
\pgfpathlineto{\pgfqpoint{2.933919in}{1.815750in}}%
\pgfpathlineto{\pgfqpoint{2.935436in}{1.814472in}}%
\pgfpathlineto{\pgfqpoint{2.937289in}{1.811928in}}%
\pgfpathlineto{\pgfqpoint{2.937626in}{1.812158in}}%
\pgfpathlineto{\pgfqpoint{2.938468in}{1.812288in}}%
\pgfpathlineto{\pgfqpoint{2.938805in}{1.811748in}}%
\pgfpathlineto{\pgfqpoint{2.943102in}{1.805301in}}%
\pgfpathlineto{\pgfqpoint{2.949083in}{1.795266in}}%
\pgfpathlineto{\pgfqpoint{2.950852in}{1.794096in}}%
\pgfpathlineto{\pgfqpoint{2.951610in}{1.793430in}}%
\pgfpathlineto{\pgfqpoint{2.952032in}{1.794173in}}%
\pgfpathlineto{\pgfqpoint{2.957002in}{1.811630in}}%
\pgfpathlineto{\pgfqpoint{2.959361in}{1.836172in}}%
\pgfpathlineto{\pgfqpoint{2.959698in}{1.835603in}}%
\pgfpathlineto{\pgfqpoint{2.960793in}{1.832892in}}%
\pgfpathlineto{\pgfqpoint{2.961214in}{1.834331in}}%
\pgfpathlineto{\pgfqpoint{2.963657in}{1.854527in}}%
\pgfpathlineto{\pgfqpoint{2.964500in}{1.852238in}}%
\pgfpathlineto{\pgfqpoint{2.966016in}{1.847681in}}%
\pgfpathlineto{\pgfqpoint{2.966522in}{1.848244in}}%
\pgfpathlineto{\pgfqpoint{2.967196in}{1.848952in}}%
\pgfpathlineto{\pgfqpoint{2.967617in}{1.848188in}}%
\pgfpathlineto{\pgfqpoint{2.981012in}{1.809777in}}%
\pgfpathlineto{\pgfqpoint{2.985729in}{1.801293in}}%
\pgfpathlineto{\pgfqpoint{2.987414in}{1.800670in}}%
\pgfpathlineto{\pgfqpoint{2.988594in}{1.799275in}}%
\pgfpathlineto{\pgfqpoint{2.989099in}{1.799888in}}%
\pgfpathlineto{\pgfqpoint{2.989857in}{1.800269in}}%
\pgfpathlineto{\pgfqpoint{2.990279in}{1.799832in}}%
\pgfpathlineto{\pgfqpoint{2.992975in}{1.798184in}}%
\pgfpathlineto{\pgfqpoint{2.995333in}{1.798091in}}%
\pgfpathlineto{\pgfqpoint{2.997692in}{1.796135in}}%
\pgfpathlineto{\pgfqpoint{2.998619in}{1.797144in}}%
\pgfpathlineto{\pgfqpoint{3.000641in}{1.800111in}}%
\pgfpathlineto{\pgfqpoint{3.000725in}{1.800066in}}%
\pgfpathlineto{\pgfqpoint{3.002494in}{1.798942in}}%
\pgfpathlineto{\pgfqpoint{3.002915in}{1.799678in}}%
\pgfpathlineto{\pgfqpoint{3.005190in}{1.803306in}}%
\pgfpathlineto{\pgfqpoint{3.005611in}{1.802891in}}%
\pgfpathlineto{\pgfqpoint{3.007717in}{1.801382in}}%
\pgfpathlineto{\pgfqpoint{3.007802in}{1.801470in}}%
\pgfpathlineto{\pgfqpoint{3.010919in}{1.804528in}}%
\pgfpathlineto{\pgfqpoint{3.011087in}{1.804358in}}%
\pgfpathlineto{\pgfqpoint{3.012688in}{1.802219in}}%
\pgfpathlineto{\pgfqpoint{3.013193in}{1.803251in}}%
\pgfpathlineto{\pgfqpoint{3.015047in}{1.813787in}}%
\pgfpathlineto{\pgfqpoint{3.015721in}{1.816146in}}%
\pgfpathlineto{\pgfqpoint{3.016310in}{1.815216in}}%
\pgfpathlineto{\pgfqpoint{3.017321in}{1.813539in}}%
\pgfpathlineto{\pgfqpoint{3.017742in}{1.814583in}}%
\pgfpathlineto{\pgfqpoint{3.019175in}{1.821419in}}%
\pgfpathlineto{\pgfqpoint{3.019933in}{1.820170in}}%
\pgfpathlineto{\pgfqpoint{3.021112in}{1.818186in}}%
\pgfpathlineto{\pgfqpoint{3.021533in}{1.818835in}}%
\pgfpathlineto{\pgfqpoint{3.023218in}{1.822947in}}%
\pgfpathlineto{\pgfqpoint{3.023808in}{1.821770in}}%
\pgfpathlineto{\pgfqpoint{3.025072in}{1.819230in}}%
\pgfpathlineto{\pgfqpoint{3.025493in}{1.820064in}}%
\pgfpathlineto{\pgfqpoint{3.026757in}{1.823311in}}%
\pgfpathlineto{\pgfqpoint{3.027346in}{1.822610in}}%
\pgfpathlineto{\pgfqpoint{3.029789in}{1.820033in}}%
\pgfpathlineto{\pgfqpoint{3.031390in}{1.819389in}}%
\pgfpathlineto{\pgfqpoint{3.032654in}{1.818129in}}%
\pgfpathlineto{\pgfqpoint{3.032991in}{1.818436in}}%
\pgfpathlineto{\pgfqpoint{3.034339in}{1.821015in}}%
\pgfpathlineto{\pgfqpoint{3.035013in}{1.819783in}}%
\pgfpathlineto{\pgfqpoint{3.038382in}{1.814413in}}%
\pgfpathlineto{\pgfqpoint{3.041247in}{1.807927in}}%
\pgfpathlineto{\pgfqpoint{3.044785in}{1.802962in}}%
\pgfpathlineto{\pgfqpoint{3.045543in}{1.804206in}}%
\pgfpathlineto{\pgfqpoint{3.047312in}{1.809547in}}%
\pgfpathlineto{\pgfqpoint{3.048070in}{1.808453in}}%
\pgfpathlineto{\pgfqpoint{3.048576in}{1.808356in}}%
\pgfpathlineto{\pgfqpoint{3.048913in}{1.809173in}}%
\pgfpathlineto{\pgfqpoint{3.049671in}{1.817047in}}%
\pgfpathlineto{\pgfqpoint{3.051272in}{1.840183in}}%
\pgfpathlineto{\pgfqpoint{3.051946in}{1.838686in}}%
\pgfpathlineto{\pgfqpoint{3.052283in}{1.838027in}}%
\pgfpathlineto{\pgfqpoint{3.052704in}{1.839597in}}%
\pgfpathlineto{\pgfqpoint{3.054978in}{1.862345in}}%
\pgfpathlineto{\pgfqpoint{3.055905in}{1.859056in}}%
\pgfpathlineto{\pgfqpoint{3.057759in}{1.856379in}}%
\pgfpathlineto{\pgfqpoint{3.059106in}{1.851970in}}%
\pgfpathlineto{\pgfqpoint{3.066099in}{1.828091in}}%
\pgfpathlineto{\pgfqpoint{3.067194in}{1.828985in}}%
\pgfpathlineto{\pgfqpoint{3.067615in}{1.827971in}}%
\pgfpathlineto{\pgfqpoint{3.069132in}{1.824275in}}%
\pgfpathlineto{\pgfqpoint{3.069637in}{1.825025in}}%
\pgfpathlineto{\pgfqpoint{3.071069in}{1.827286in}}%
\pgfpathlineto{\pgfqpoint{3.071490in}{1.826805in}}%
\pgfpathlineto{\pgfqpoint{3.073681in}{1.824624in}}%
\pgfpathlineto{\pgfqpoint{3.074776in}{1.823798in}}%
\pgfpathlineto{\pgfqpoint{3.079072in}{1.815904in}}%
\pgfpathlineto{\pgfqpoint{3.079831in}{1.816829in}}%
\pgfpathlineto{\pgfqpoint{3.080842in}{1.819473in}}%
\pgfpathlineto{\pgfqpoint{3.081431in}{1.818393in}}%
\pgfpathlineto{\pgfqpoint{3.082105in}{1.817446in}}%
\pgfpathlineto{\pgfqpoint{3.082442in}{1.818366in}}%
\pgfpathlineto{\pgfqpoint{3.084464in}{1.829176in}}%
\pgfpathlineto{\pgfqpoint{3.085222in}{1.827894in}}%
\pgfpathlineto{\pgfqpoint{3.087160in}{1.825716in}}%
\pgfpathlineto{\pgfqpoint{3.088929in}{1.821764in}}%
\pgfpathlineto{\pgfqpoint{3.091456in}{1.815945in}}%
\pgfpathlineto{\pgfqpoint{3.091962in}{1.816831in}}%
\pgfpathlineto{\pgfqpoint{3.093899in}{1.824823in}}%
\pgfpathlineto{\pgfqpoint{3.094910in}{1.822557in}}%
\pgfpathlineto{\pgfqpoint{3.095500in}{1.821743in}}%
\pgfpathlineto{\pgfqpoint{3.096006in}{1.822739in}}%
\pgfpathlineto{\pgfqpoint{3.098027in}{1.830724in}}%
\pgfpathlineto{\pgfqpoint{3.098786in}{1.829260in}}%
\pgfpathlineto{\pgfqpoint{3.104683in}{1.814093in}}%
\pgfpathlineto{\pgfqpoint{3.108811in}{1.806268in}}%
\pgfpathlineto{\pgfqpoint{3.110580in}{1.804352in}}%
\pgfpathlineto{\pgfqpoint{3.111170in}{1.803983in}}%
\pgfpathlineto{\pgfqpoint{3.111591in}{1.804834in}}%
\pgfpathlineto{\pgfqpoint{3.112770in}{1.811832in}}%
\pgfpathlineto{\pgfqpoint{3.113950in}{1.819228in}}%
\pgfpathlineto{\pgfqpoint{3.114539in}{1.818277in}}%
\pgfpathlineto{\pgfqpoint{3.115045in}{1.817720in}}%
\pgfpathlineto{\pgfqpoint{3.115382in}{1.818538in}}%
\pgfpathlineto{\pgfqpoint{3.116309in}{1.828532in}}%
\pgfpathlineto{\pgfqpoint{3.117825in}{1.842643in}}%
\pgfpathlineto{\pgfqpoint{3.118246in}{1.841955in}}%
\pgfpathlineto{\pgfqpoint{3.118920in}{1.840782in}}%
\pgfpathlineto{\pgfqpoint{3.119341in}{1.841625in}}%
\pgfpathlineto{\pgfqpoint{3.120352in}{1.850218in}}%
\pgfpathlineto{\pgfqpoint{3.121447in}{1.857599in}}%
\pgfpathlineto{\pgfqpoint{3.121953in}{1.856587in}}%
\pgfpathlineto{\pgfqpoint{3.123217in}{1.852780in}}%
\pgfpathlineto{\pgfqpoint{3.123806in}{1.853956in}}%
\pgfpathlineto{\pgfqpoint{3.124564in}{1.854826in}}%
\pgfpathlineto{\pgfqpoint{3.125070in}{1.854209in}}%
\pgfpathlineto{\pgfqpoint{3.126418in}{1.849640in}}%
\pgfpathlineto{\pgfqpoint{3.129872in}{1.838543in}}%
\pgfpathlineto{\pgfqpoint{3.130209in}{1.838739in}}%
\pgfpathlineto{\pgfqpoint{3.130377in}{1.839202in}}%
\pgfpathlineto{\pgfqpoint{3.131388in}{1.847211in}}%
\pgfpathlineto{\pgfqpoint{3.132820in}{1.858276in}}%
\pgfpathlineto{\pgfqpoint{3.133326in}{1.857102in}}%
\pgfpathlineto{\pgfqpoint{3.134253in}{1.854183in}}%
\pgfpathlineto{\pgfqpoint{3.134674in}{1.855739in}}%
\pgfpathlineto{\pgfqpoint{3.136274in}{1.867341in}}%
\pgfpathlineto{\pgfqpoint{3.136948in}{1.865431in}}%
\pgfpathlineto{\pgfqpoint{3.137959in}{1.861837in}}%
\pgfpathlineto{\pgfqpoint{3.138549in}{1.863067in}}%
\pgfpathlineto{\pgfqpoint{3.139055in}{1.864038in}}%
\pgfpathlineto{\pgfqpoint{3.139560in}{1.862754in}}%
\pgfpathlineto{\pgfqpoint{3.140992in}{1.857508in}}%
\pgfpathlineto{\pgfqpoint{3.141498in}{1.858714in}}%
\pgfpathlineto{\pgfqpoint{3.143014in}{1.865679in}}%
\pgfpathlineto{\pgfqpoint{3.143688in}{1.863677in}}%
\pgfpathlineto{\pgfqpoint{3.144446in}{1.861354in}}%
\pgfpathlineto{\pgfqpoint{3.144952in}{1.862726in}}%
\pgfpathlineto{\pgfqpoint{3.146889in}{1.877622in}}%
\pgfpathlineto{\pgfqpoint{3.147816in}{1.873787in}}%
\pgfpathlineto{\pgfqpoint{3.149838in}{1.868341in}}%
\pgfpathlineto{\pgfqpoint{3.151270in}{1.862172in}}%
\pgfpathlineto{\pgfqpoint{3.154892in}{1.847899in}}%
\pgfpathlineto{\pgfqpoint{3.155061in}{1.847949in}}%
\pgfpathlineto{\pgfqpoint{3.155903in}{1.847185in}}%
\pgfpathlineto{\pgfqpoint{3.166939in}{1.820712in}}%
\pgfpathlineto{\pgfqpoint{3.168456in}{1.817461in}}%
\pgfpathlineto{\pgfqpoint{3.170899in}{1.813744in}}%
\pgfpathlineto{\pgfqpoint{3.173005in}{1.812948in}}%
\pgfpathlineto{\pgfqpoint{3.175027in}{1.811431in}}%
\pgfpathlineto{\pgfqpoint{3.177386in}{1.811431in}}%
\pgfpathlineto{\pgfqpoint{3.188169in}{1.795237in}}%
\pgfpathlineto{\pgfqpoint{3.188590in}{1.795729in}}%
\pgfpathlineto{\pgfqpoint{3.189180in}{1.795899in}}%
\pgfpathlineto{\pgfqpoint{3.189601in}{1.795418in}}%
\pgfpathlineto{\pgfqpoint{3.191286in}{1.793563in}}%
\pgfpathlineto{\pgfqpoint{3.191707in}{1.794264in}}%
\pgfpathlineto{\pgfqpoint{3.193898in}{1.798545in}}%
\pgfpathlineto{\pgfqpoint{3.194235in}{1.798268in}}%
\pgfpathlineto{\pgfqpoint{3.204934in}{1.788709in}}%
\pgfpathlineto{\pgfqpoint{3.207124in}{1.788821in}}%
\pgfpathlineto{\pgfqpoint{3.210157in}{1.787000in}}%
\pgfpathlineto{\pgfqpoint{3.213779in}{1.785516in}}%
\pgfpathlineto{\pgfqpoint{3.219087in}{1.786133in}}%
\pgfpathlineto{\pgfqpoint{3.221109in}{1.790302in}}%
\pgfpathlineto{\pgfqpoint{3.222120in}{1.789507in}}%
\pgfpathlineto{\pgfqpoint{3.223552in}{1.790202in}}%
\pgfpathlineto{\pgfqpoint{3.225068in}{1.794565in}}%
\pgfpathlineto{\pgfqpoint{3.227259in}{1.800289in}}%
\pgfpathlineto{\pgfqpoint{3.227427in}{1.800177in}}%
\pgfpathlineto{\pgfqpoint{3.228691in}{1.799084in}}%
\pgfpathlineto{\pgfqpoint{3.228943in}{1.799705in}}%
\pgfpathlineto{\pgfqpoint{3.229954in}{1.808956in}}%
\pgfpathlineto{\pgfqpoint{3.231387in}{1.817881in}}%
\pgfpathlineto{\pgfqpoint{3.231724in}{1.817511in}}%
\pgfpathlineto{\pgfqpoint{3.232229in}{1.817057in}}%
\pgfpathlineto{\pgfqpoint{3.232650in}{1.818091in}}%
\pgfpathlineto{\pgfqpoint{3.234925in}{1.826101in}}%
\pgfpathlineto{\pgfqpoint{3.235599in}{1.824833in}}%
\pgfpathlineto{\pgfqpoint{3.236188in}{1.824136in}}%
\pgfpathlineto{\pgfqpoint{3.236610in}{1.825029in}}%
\pgfpathlineto{\pgfqpoint{3.238210in}{1.834334in}}%
\pgfpathlineto{\pgfqpoint{3.239221in}{1.832105in}}%
\pgfpathlineto{\pgfqpoint{3.240485in}{1.830205in}}%
\pgfpathlineto{\pgfqpoint{3.240822in}{1.830507in}}%
\pgfpathlineto{\pgfqpoint{3.241749in}{1.831877in}}%
\pgfpathlineto{\pgfqpoint{3.242254in}{1.830891in}}%
\pgfpathlineto{\pgfqpoint{3.248320in}{1.815568in}}%
\pgfpathlineto{\pgfqpoint{3.250089in}{1.811704in}}%
\pgfpathlineto{\pgfqpoint{3.251942in}{1.810177in}}%
\pgfpathlineto{\pgfqpoint{3.252953in}{1.808646in}}%
\pgfpathlineto{\pgfqpoint{3.254807in}{1.805809in}}%
\pgfpathlineto{\pgfqpoint{3.255228in}{1.806040in}}%
\pgfpathlineto{\pgfqpoint{3.256239in}{1.805391in}}%
\pgfpathlineto{\pgfqpoint{3.261041in}{1.797910in}}%
\pgfpathlineto{\pgfqpoint{3.261209in}{1.798123in}}%
\pgfpathlineto{\pgfqpoint{3.263231in}{1.804884in}}%
\pgfpathlineto{\pgfqpoint{3.263905in}{1.806705in}}%
\pgfpathlineto{\pgfqpoint{3.264495in}{1.805968in}}%
\pgfpathlineto{\pgfqpoint{3.265000in}{1.805822in}}%
\pgfpathlineto{\pgfqpoint{3.265337in}{1.806693in}}%
\pgfpathlineto{\pgfqpoint{3.266601in}{1.818864in}}%
\pgfpathlineto{\pgfqpoint{3.267780in}{1.824328in}}%
\pgfpathlineto{\pgfqpoint{3.268201in}{1.823781in}}%
\pgfpathlineto{\pgfqpoint{3.269634in}{1.821344in}}%
\pgfpathlineto{\pgfqpoint{3.270223in}{1.821729in}}%
\pgfpathlineto{\pgfqpoint{3.271150in}{1.823606in}}%
\pgfpathlineto{\pgfqpoint{3.271655in}{1.824529in}}%
\pgfpathlineto{\pgfqpoint{3.272245in}{1.823394in}}%
\pgfpathlineto{\pgfqpoint{3.276373in}{1.814093in}}%
\pgfpathlineto{\pgfqpoint{3.284713in}{1.799185in}}%
\pgfpathlineto{\pgfqpoint{3.287662in}{1.796150in}}%
\pgfpathlineto{\pgfqpoint{3.288252in}{1.797488in}}%
\pgfpathlineto{\pgfqpoint{3.290442in}{1.802266in}}%
\pgfpathlineto{\pgfqpoint{3.291369in}{1.802526in}}%
\pgfpathlineto{\pgfqpoint{3.291453in}{1.802762in}}%
\pgfpathlineto{\pgfqpoint{3.292717in}{1.809666in}}%
\pgfpathlineto{\pgfqpoint{3.294738in}{1.827449in}}%
\pgfpathlineto{\pgfqpoint{3.295075in}{1.827146in}}%
\pgfpathlineto{\pgfqpoint{3.296255in}{1.824708in}}%
\pgfpathlineto{\pgfqpoint{3.296845in}{1.826059in}}%
\pgfpathlineto{\pgfqpoint{3.298782in}{1.833378in}}%
\pgfpathlineto{\pgfqpoint{3.299456in}{1.831784in}}%
\pgfpathlineto{\pgfqpoint{3.304342in}{1.818158in}}%
\pgfpathlineto{\pgfqpoint{3.310071in}{1.805541in}}%
\pgfpathlineto{\pgfqpoint{3.312683in}{1.801744in}}%
\pgfpathlineto{\pgfqpoint{3.315884in}{1.798786in}}%
\pgfpathlineto{\pgfqpoint{3.317653in}{1.798680in}}%
\pgfpathlineto{\pgfqpoint{3.320349in}{1.796476in}}%
\pgfpathlineto{\pgfqpoint{3.326162in}{1.791674in}}%
\pgfpathlineto{\pgfqpoint{3.326330in}{1.791922in}}%
\pgfpathlineto{\pgfqpoint{3.327425in}{1.794597in}}%
\pgfpathlineto{\pgfqpoint{3.328183in}{1.793805in}}%
\pgfpathlineto{\pgfqpoint{3.330542in}{1.791886in}}%
\pgfpathlineto{\pgfqpoint{3.330795in}{1.792269in}}%
\pgfpathlineto{\pgfqpoint{3.332311in}{1.792810in}}%
\pgfpathlineto{\pgfqpoint{3.334081in}{1.790782in}}%
\pgfpathlineto{\pgfqpoint{3.339388in}{1.786994in}}%
\pgfpathlineto{\pgfqpoint{3.342758in}{1.785734in}}%
\pgfpathlineto{\pgfqpoint{3.342842in}{1.785862in}}%
\pgfpathlineto{\pgfqpoint{3.344190in}{1.786862in}}%
\pgfpathlineto{\pgfqpoint{3.344527in}{1.786574in}}%
\pgfpathlineto{\pgfqpoint{3.345201in}{1.786329in}}%
\pgfpathlineto{\pgfqpoint{3.345538in}{1.786988in}}%
\pgfpathlineto{\pgfqpoint{3.350171in}{1.806447in}}%
\pgfpathlineto{\pgfqpoint{3.351098in}{1.811947in}}%
\pgfpathlineto{\pgfqpoint{3.351603in}{1.811307in}}%
\pgfpathlineto{\pgfqpoint{3.352446in}{1.810092in}}%
\pgfpathlineto{\pgfqpoint{3.352783in}{1.811091in}}%
\pgfpathlineto{\pgfqpoint{3.359354in}{1.843338in}}%
\pgfpathlineto{\pgfqpoint{3.359691in}{1.842898in}}%
\pgfpathlineto{\pgfqpoint{3.361039in}{1.839482in}}%
\pgfpathlineto{\pgfqpoint{3.361544in}{1.841079in}}%
\pgfpathlineto{\pgfqpoint{3.363145in}{1.854190in}}%
\pgfpathlineto{\pgfqpoint{3.364156in}{1.851379in}}%
\pgfpathlineto{\pgfqpoint{3.364493in}{1.850771in}}%
\pgfpathlineto{\pgfqpoint{3.364998in}{1.851991in}}%
\pgfpathlineto{\pgfqpoint{3.366936in}{1.862633in}}%
\pgfpathlineto{\pgfqpoint{3.367778in}{1.860087in}}%
\pgfpathlineto{\pgfqpoint{3.368368in}{1.858610in}}%
\pgfpathlineto{\pgfqpoint{3.368874in}{1.859944in}}%
\pgfpathlineto{\pgfqpoint{3.369548in}{1.861389in}}%
\pgfpathlineto{\pgfqpoint{3.370053in}{1.860381in}}%
\pgfpathlineto{\pgfqpoint{3.375697in}{1.847004in}}%
\pgfpathlineto{\pgfqpoint{3.377130in}{1.842207in}}%
\pgfpathlineto{\pgfqpoint{3.377635in}{1.844844in}}%
\pgfpathlineto{\pgfqpoint{3.379573in}{1.860062in}}%
\pgfpathlineto{\pgfqpoint{3.379994in}{1.859181in}}%
\pgfpathlineto{\pgfqpoint{3.380331in}{1.858638in}}%
\pgfpathlineto{\pgfqpoint{3.380752in}{1.860180in}}%
\pgfpathlineto{\pgfqpoint{3.381932in}{1.877218in}}%
\pgfpathlineto{\pgfqpoint{3.383279in}{1.895864in}}%
\pgfpathlineto{\pgfqpoint{3.383785in}{1.894236in}}%
\pgfpathlineto{\pgfqpoint{3.384375in}{1.892365in}}%
\pgfpathlineto{\pgfqpoint{3.384796in}{1.893961in}}%
\pgfpathlineto{\pgfqpoint{3.386649in}{1.913016in}}%
\pgfpathlineto{\pgfqpoint{3.387660in}{1.907446in}}%
\pgfpathlineto{\pgfqpoint{3.388081in}{1.906159in}}%
\pgfpathlineto{\pgfqpoint{3.388587in}{1.907789in}}%
\pgfpathlineto{\pgfqpoint{3.389261in}{1.910212in}}%
\pgfpathlineto{\pgfqpoint{3.389850in}{1.908900in}}%
\pgfpathlineto{\pgfqpoint{3.391535in}{1.901787in}}%
\pgfpathlineto{\pgfqpoint{3.392125in}{1.903629in}}%
\pgfpathlineto{\pgfqpoint{3.393220in}{1.909391in}}%
\pgfpathlineto{\pgfqpoint{3.393726in}{1.907312in}}%
\pgfpathlineto{\pgfqpoint{3.395748in}{1.897944in}}%
\pgfpathlineto{\pgfqpoint{3.396085in}{1.898322in}}%
\pgfpathlineto{\pgfqpoint{3.396506in}{1.898713in}}%
\pgfpathlineto{\pgfqpoint{3.396927in}{1.897580in}}%
\pgfpathlineto{\pgfqpoint{3.402150in}{1.877073in}}%
\pgfpathlineto{\pgfqpoint{3.403330in}{1.874907in}}%
\pgfpathlineto{\pgfqpoint{3.407710in}{1.858664in}}%
\pgfpathlineto{\pgfqpoint{3.408300in}{1.860207in}}%
\pgfpathlineto{\pgfqpoint{3.409311in}{1.863776in}}%
\pgfpathlineto{\pgfqpoint{3.409901in}{1.862615in}}%
\pgfpathlineto{\pgfqpoint{3.410069in}{1.862492in}}%
\pgfpathlineto{\pgfqpoint{3.410406in}{1.863677in}}%
\pgfpathlineto{\pgfqpoint{3.411417in}{1.879512in}}%
\pgfpathlineto{\pgfqpoint{3.412933in}{1.909479in}}%
\pgfpathlineto{\pgfqpoint{3.413523in}{1.906833in}}%
\pgfpathlineto{\pgfqpoint{3.414029in}{1.905082in}}%
\pgfpathlineto{\pgfqpoint{3.414450in}{1.907906in}}%
\pgfpathlineto{\pgfqpoint{3.416556in}{1.941391in}}%
\pgfpathlineto{\pgfqpoint{3.417567in}{1.934570in}}%
\pgfpathlineto{\pgfqpoint{3.419505in}{1.927717in}}%
\pgfpathlineto{\pgfqpoint{3.420515in}{1.922191in}}%
\pgfpathlineto{\pgfqpoint{3.426750in}{1.880119in}}%
\pgfpathlineto{\pgfqpoint{3.429867in}{1.870616in}}%
\pgfpathlineto{\pgfqpoint{3.430456in}{1.873448in}}%
\pgfpathlineto{\pgfqpoint{3.432310in}{1.896217in}}%
\pgfpathlineto{\pgfqpoint{3.433405in}{1.892521in}}%
\pgfpathlineto{\pgfqpoint{3.433826in}{1.895068in}}%
\pgfpathlineto{\pgfqpoint{3.435174in}{1.927731in}}%
\pgfpathlineto{\pgfqpoint{3.436353in}{1.945425in}}%
\pgfpathlineto{\pgfqpoint{3.436859in}{1.942847in}}%
\pgfpathlineto{\pgfqpoint{3.437533in}{1.939450in}}%
\pgfpathlineto{\pgfqpoint{3.438038in}{1.942288in}}%
\pgfpathlineto{\pgfqpoint{3.439639in}{1.963030in}}%
\pgfpathlineto{\pgfqpoint{3.440397in}{1.958299in}}%
\pgfpathlineto{\pgfqpoint{3.441492in}{1.951583in}}%
\pgfpathlineto{\pgfqpoint{3.442082in}{1.952304in}}%
\pgfpathlineto{\pgfqpoint{3.442335in}{1.952268in}}%
\pgfpathlineto{\pgfqpoint{3.442588in}{1.951539in}}%
\pgfpathlineto{\pgfqpoint{3.447137in}{1.931439in}}%
\pgfpathlineto{\pgfqpoint{3.447558in}{1.932876in}}%
\pgfpathlineto{\pgfqpoint{3.449074in}{1.943128in}}%
\pgfpathlineto{\pgfqpoint{3.449748in}{1.939987in}}%
\pgfpathlineto{\pgfqpoint{3.450085in}{1.938557in}}%
\pgfpathlineto{\pgfqpoint{3.450507in}{1.941154in}}%
\pgfpathlineto{\pgfqpoint{3.451686in}{1.976388in}}%
\pgfpathlineto{\pgfqpoint{3.452697in}{1.996300in}}%
\pgfpathlineto{\pgfqpoint{3.453202in}{1.993046in}}%
\pgfpathlineto{\pgfqpoint{3.453792in}{1.989531in}}%
\pgfpathlineto{\pgfqpoint{3.454213in}{1.993071in}}%
\pgfpathlineto{\pgfqpoint{3.455393in}{2.034521in}}%
\pgfpathlineto{\pgfqpoint{3.456235in}{2.051724in}}%
\pgfpathlineto{\pgfqpoint{3.456825in}{2.047590in}}%
\pgfpathlineto{\pgfqpoint{3.457752in}{2.038798in}}%
\pgfpathlineto{\pgfqpoint{3.458257in}{2.042131in}}%
\pgfpathlineto{\pgfqpoint{3.459436in}{2.054388in}}%
\pgfpathlineto{\pgfqpoint{3.459942in}{2.051443in}}%
\pgfpathlineto{\pgfqpoint{3.461290in}{2.038210in}}%
\pgfpathlineto{\pgfqpoint{3.461880in}{2.041584in}}%
\pgfpathlineto{\pgfqpoint{3.462722in}{2.046677in}}%
\pgfpathlineto{\pgfqpoint{3.463227in}{2.044643in}}%
\pgfpathlineto{\pgfqpoint{3.464828in}{2.029667in}}%
\pgfpathlineto{\pgfqpoint{3.465671in}{2.032004in}}%
\pgfpathlineto{\pgfqpoint{3.466008in}{2.032249in}}%
\pgfpathlineto{\pgfqpoint{3.466345in}{2.031360in}}%
\pgfpathlineto{\pgfqpoint{3.467440in}{2.020455in}}%
\pgfpathlineto{\pgfqpoint{3.473168in}{1.957575in}}%
\pgfpathlineto{\pgfqpoint{3.473505in}{1.956620in}}%
\pgfpathlineto{\pgfqpoint{3.474011in}{1.958661in}}%
\pgfpathlineto{\pgfqpoint{3.475022in}{1.965841in}}%
\pgfpathlineto{\pgfqpoint{3.475611in}{1.963014in}}%
\pgfpathlineto{\pgfqpoint{3.476622in}{1.955941in}}%
\pgfpathlineto{\pgfqpoint{3.477128in}{1.959420in}}%
\pgfpathlineto{\pgfqpoint{3.478560in}{1.978391in}}%
\pgfpathlineto{\pgfqpoint{3.479234in}{1.974750in}}%
\pgfpathlineto{\pgfqpoint{3.479908in}{1.970110in}}%
\pgfpathlineto{\pgfqpoint{3.480329in}{1.973630in}}%
\pgfpathlineto{\pgfqpoint{3.482435in}{2.024822in}}%
\pgfpathlineto{\pgfqpoint{3.483530in}{2.014900in}}%
\pgfpathlineto{\pgfqpoint{3.484036in}{2.012253in}}%
\pgfpathlineto{\pgfqpoint{3.484541in}{2.014540in}}%
\pgfpathlineto{\pgfqpoint{3.485805in}{2.022520in}}%
\pgfpathlineto{\pgfqpoint{3.486310in}{2.020321in}}%
\pgfpathlineto{\pgfqpoint{3.489259in}{1.985807in}}%
\pgfpathlineto{\pgfqpoint{3.492882in}{1.955668in}}%
\pgfpathlineto{\pgfqpoint{3.493892in}{1.951320in}}%
\pgfpathlineto{\pgfqpoint{3.494398in}{1.951842in}}%
\pgfpathlineto{\pgfqpoint{3.494819in}{1.952188in}}%
\pgfpathlineto{\pgfqpoint{3.495240in}{1.951302in}}%
\pgfpathlineto{\pgfqpoint{3.496588in}{1.940855in}}%
\pgfpathlineto{\pgfqpoint{3.499284in}{1.926718in}}%
\pgfpathlineto{\pgfqpoint{3.508130in}{1.876011in}}%
\pgfpathlineto{\pgfqpoint{3.508214in}{1.876057in}}%
\pgfpathlineto{\pgfqpoint{3.508888in}{1.878549in}}%
\pgfpathlineto{\pgfqpoint{3.510404in}{1.886919in}}%
\pgfpathlineto{\pgfqpoint{3.510994in}{1.885298in}}%
\pgfpathlineto{\pgfqpoint{3.514195in}{1.874273in}}%
\pgfpathlineto{\pgfqpoint{3.515796in}{1.867725in}}%
\pgfpathlineto{\pgfqpoint{3.520008in}{1.848352in}}%
\pgfpathlineto{\pgfqpoint{3.520345in}{1.848969in}}%
\pgfpathlineto{\pgfqpoint{3.522114in}{1.856382in}}%
\pgfpathlineto{\pgfqpoint{3.522873in}{1.854418in}}%
\pgfpathlineto{\pgfqpoint{3.523462in}{1.852979in}}%
\pgfpathlineto{\pgfqpoint{3.523884in}{1.854566in}}%
\pgfpathlineto{\pgfqpoint{3.525147in}{1.874292in}}%
\pgfpathlineto{\pgfqpoint{3.526242in}{1.884800in}}%
\pgfpathlineto{\pgfqpoint{3.526748in}{1.883225in}}%
\pgfpathlineto{\pgfqpoint{3.527169in}{1.882019in}}%
\pgfpathlineto{\pgfqpoint{3.527590in}{1.884078in}}%
\pgfpathlineto{\pgfqpoint{3.529359in}{1.908933in}}%
\pgfpathlineto{\pgfqpoint{3.530455in}{1.904380in}}%
\pgfpathlineto{\pgfqpoint{3.530623in}{1.904295in}}%
\pgfpathlineto{\pgfqpoint{3.530876in}{1.905064in}}%
\pgfpathlineto{\pgfqpoint{3.531971in}{1.918310in}}%
\pgfpathlineto{\pgfqpoint{3.532645in}{1.923055in}}%
\pgfpathlineto{\pgfqpoint{3.533235in}{1.920425in}}%
\pgfpathlineto{\pgfqpoint{3.533740in}{1.918179in}}%
\pgfpathlineto{\pgfqpoint{3.534246in}{1.921134in}}%
\pgfpathlineto{\pgfqpoint{3.536099in}{1.949451in}}%
\pgfpathlineto{\pgfqpoint{3.537026in}{1.944345in}}%
\pgfpathlineto{\pgfqpoint{3.537278in}{1.943712in}}%
\pgfpathlineto{\pgfqpoint{3.537700in}{1.945766in}}%
\pgfpathlineto{\pgfqpoint{3.539385in}{1.963805in}}%
\pgfpathlineto{\pgfqpoint{3.540143in}{1.959407in}}%
\pgfpathlineto{\pgfqpoint{3.541069in}{1.953684in}}%
\pgfpathlineto{\pgfqpoint{3.541575in}{1.955792in}}%
\pgfpathlineto{\pgfqpoint{3.542839in}{1.966123in}}%
\pgfpathlineto{\pgfqpoint{3.543428in}{1.963488in}}%
\pgfpathlineto{\pgfqpoint{3.544860in}{1.953373in}}%
\pgfpathlineto{\pgfqpoint{3.545450in}{1.954794in}}%
\pgfpathlineto{\pgfqpoint{3.546040in}{1.956124in}}%
\pgfpathlineto{\pgfqpoint{3.546461in}{1.954783in}}%
\pgfpathlineto{\pgfqpoint{3.554212in}{1.907519in}}%
\pgfpathlineto{\pgfqpoint{3.554886in}{1.910181in}}%
\pgfpathlineto{\pgfqpoint{3.555896in}{1.915904in}}%
\pgfpathlineto{\pgfqpoint{3.556402in}{1.913851in}}%
\pgfpathlineto{\pgfqpoint{3.557329in}{1.908957in}}%
\pgfpathlineto{\pgfqpoint{3.557834in}{1.911281in}}%
\pgfpathlineto{\pgfqpoint{3.559350in}{1.927859in}}%
\pgfpathlineto{\pgfqpoint{3.560109in}{1.923782in}}%
\pgfpathlineto{\pgfqpoint{3.560783in}{1.920602in}}%
\pgfpathlineto{\pgfqpoint{3.561288in}{1.922517in}}%
\pgfpathlineto{\pgfqpoint{3.562636in}{1.932437in}}%
\pgfpathlineto{\pgfqpoint{3.563226in}{1.929915in}}%
\pgfpathlineto{\pgfqpoint{3.564321in}{1.923784in}}%
\pgfpathlineto{\pgfqpoint{3.564826in}{1.925650in}}%
\pgfpathlineto{\pgfqpoint{3.565922in}{1.930005in}}%
\pgfpathlineto{\pgfqpoint{3.566427in}{1.928614in}}%
\pgfpathlineto{\pgfqpoint{3.567775in}{1.921188in}}%
\pgfpathlineto{\pgfqpoint{3.568449in}{1.923498in}}%
\pgfpathlineto{\pgfqpoint{3.569291in}{1.926737in}}%
\pgfpathlineto{\pgfqpoint{3.569797in}{1.925255in}}%
\pgfpathlineto{\pgfqpoint{3.574683in}{1.903746in}}%
\pgfpathlineto{\pgfqpoint{3.575947in}{1.901117in}}%
\pgfpathlineto{\pgfqpoint{3.583444in}{1.864269in}}%
\pgfpathlineto{\pgfqpoint{3.583866in}{1.864729in}}%
\pgfpathlineto{\pgfqpoint{3.584624in}{1.865590in}}%
\pgfpathlineto{\pgfqpoint{3.585045in}{1.864835in}}%
\pgfpathlineto{\pgfqpoint{3.586477in}{1.859765in}}%
\pgfpathlineto{\pgfqpoint{3.587067in}{1.861419in}}%
\pgfpathlineto{\pgfqpoint{3.588668in}{1.869385in}}%
\pgfpathlineto{\pgfqpoint{3.589257in}{1.867783in}}%
\pgfpathlineto{\pgfqpoint{3.590437in}{1.863717in}}%
\pgfpathlineto{\pgfqpoint{3.590942in}{1.865028in}}%
\pgfpathlineto{\pgfqpoint{3.592122in}{1.868154in}}%
\pgfpathlineto{\pgfqpoint{3.592543in}{1.867509in}}%
\pgfpathlineto{\pgfqpoint{3.593975in}{1.862665in}}%
\pgfpathlineto{\pgfqpoint{3.594565in}{1.864443in}}%
\pgfpathlineto{\pgfqpoint{3.596334in}{1.875468in}}%
\pgfpathlineto{\pgfqpoint{3.597092in}{1.873044in}}%
\pgfpathlineto{\pgfqpoint{3.597850in}{1.870958in}}%
\pgfpathlineto{\pgfqpoint{3.598271in}{1.872057in}}%
\pgfpathlineto{\pgfqpoint{3.599872in}{1.882301in}}%
\pgfpathlineto{\pgfqpoint{3.600715in}{1.879478in}}%
\pgfpathlineto{\pgfqpoint{3.603579in}{1.869675in}}%
\pgfpathlineto{\pgfqpoint{3.612677in}{1.831290in}}%
\pgfpathlineto{\pgfqpoint{3.614194in}{1.828783in}}%
\pgfpathlineto{\pgfqpoint{3.614446in}{1.828906in}}%
\pgfpathlineto{\pgfqpoint{3.616468in}{1.830709in}}%
\pgfpathlineto{\pgfqpoint{3.617058in}{1.829330in}}%
\pgfpathlineto{\pgfqpoint{3.618153in}{1.827075in}}%
\pgfpathlineto{\pgfqpoint{3.618574in}{1.827995in}}%
\pgfpathlineto{\pgfqpoint{3.620765in}{1.838071in}}%
\pgfpathlineto{\pgfqpoint{3.621607in}{1.836032in}}%
\pgfpathlineto{\pgfqpoint{3.622618in}{1.833923in}}%
\pgfpathlineto{\pgfqpoint{3.623039in}{1.834570in}}%
\pgfpathlineto{\pgfqpoint{3.624808in}{1.839522in}}%
\pgfpathlineto{\pgfqpoint{3.625482in}{1.838364in}}%
\pgfpathlineto{\pgfqpoint{3.628263in}{1.833466in}}%
\pgfpathlineto{\pgfqpoint{3.629526in}{1.831530in}}%
\pgfpathlineto{\pgfqpoint{3.638625in}{1.809099in}}%
\pgfpathlineto{\pgfqpoint{3.643932in}{1.801740in}}%
\pgfpathlineto{\pgfqpoint{3.646207in}{1.799498in}}%
\pgfpathlineto{\pgfqpoint{3.647133in}{1.798997in}}%
\pgfpathlineto{\pgfqpoint{3.647555in}{1.799448in}}%
\pgfpathlineto{\pgfqpoint{3.648650in}{1.800294in}}%
\pgfpathlineto{\pgfqpoint{3.649071in}{1.799814in}}%
\pgfpathlineto{\pgfqpoint{3.650082in}{1.798889in}}%
\pgfpathlineto{\pgfqpoint{3.650503in}{1.799632in}}%
\pgfpathlineto{\pgfqpoint{3.652356in}{1.805917in}}%
\pgfpathlineto{\pgfqpoint{3.653283in}{1.804759in}}%
\pgfpathlineto{\pgfqpoint{3.653789in}{1.804834in}}%
\pgfpathlineto{\pgfqpoint{3.654041in}{1.805313in}}%
\pgfpathlineto{\pgfqpoint{3.655726in}{1.809236in}}%
\pgfpathlineto{\pgfqpoint{3.656400in}{1.808294in}}%
\pgfpathlineto{\pgfqpoint{3.658254in}{1.806943in}}%
\pgfpathlineto{\pgfqpoint{3.660107in}{1.805333in}}%
\pgfpathlineto{\pgfqpoint{3.665751in}{1.796403in}}%
\pgfpathlineto{\pgfqpoint{3.667773in}{1.795858in}}%
\pgfpathlineto{\pgfqpoint{3.669795in}{1.794475in}}%
\pgfpathlineto{\pgfqpoint{3.670048in}{1.794722in}}%
\pgfpathlineto{\pgfqpoint{3.671648in}{1.795878in}}%
\pgfpathlineto{\pgfqpoint{3.672070in}{1.795394in}}%
\pgfpathlineto{\pgfqpoint{3.674260in}{1.793650in}}%
\pgfpathlineto{\pgfqpoint{3.674429in}{1.793780in}}%
\pgfpathlineto{\pgfqpoint{3.676282in}{1.793954in}}%
\pgfpathlineto{\pgfqpoint{3.680494in}{1.789835in}}%
\pgfpathlineto{\pgfqpoint{3.681168in}{1.790702in}}%
\pgfpathlineto{\pgfqpoint{3.682769in}{1.790797in}}%
\pgfpathlineto{\pgfqpoint{3.685886in}{1.789694in}}%
\pgfpathlineto{\pgfqpoint{3.688666in}{1.786982in}}%
\pgfpathlineto{\pgfqpoint{3.695153in}{1.782087in}}%
\pgfpathlineto{\pgfqpoint{3.701218in}{1.778907in}}%
\pgfpathlineto{\pgfqpoint{3.720173in}{1.774072in}}%
\pgfpathlineto{\pgfqpoint{3.722448in}{1.775266in}}%
\pgfpathlineto{\pgfqpoint{3.727250in}{1.774609in}}%
\pgfpathlineto{\pgfqpoint{3.739044in}{1.773444in}}%
\pgfpathlineto{\pgfqpoint{3.740897in}{1.773586in}}%
\pgfpathlineto{\pgfqpoint{3.744351in}{1.774008in}}%
\pgfpathlineto{\pgfqpoint{3.745699in}{1.775206in}}%
\pgfpathlineto{\pgfqpoint{3.748648in}{1.778555in}}%
\pgfpathlineto{\pgfqpoint{3.749996in}{1.779020in}}%
\pgfpathlineto{\pgfqpoint{3.752523in}{1.783289in}}%
\pgfpathlineto{\pgfqpoint{3.753450in}{1.782551in}}%
\pgfpathlineto{\pgfqpoint{3.757662in}{1.780347in}}%
\pgfpathlineto{\pgfqpoint{3.762211in}{1.781012in}}%
\pgfpathlineto{\pgfqpoint{3.768108in}{1.783476in}}%
\pgfpathlineto{\pgfqpoint{3.769878in}{1.788831in}}%
\pgfpathlineto{\pgfqpoint{3.770299in}{1.788558in}}%
\pgfpathlineto{\pgfqpoint{3.771562in}{1.788199in}}%
\pgfpathlineto{\pgfqpoint{3.771815in}{1.788492in}}%
\pgfpathlineto{\pgfqpoint{3.772995in}{1.788643in}}%
\pgfpathlineto{\pgfqpoint{3.773163in}{1.788481in}}%
\pgfpathlineto{\pgfqpoint{3.778134in}{1.784833in}}%
\pgfpathlineto{\pgfqpoint{3.780408in}{1.783552in}}%
\pgfpathlineto{\pgfqpoint{3.793466in}{1.778753in}}%
\pgfpathlineto{\pgfqpoint{3.795404in}{1.778916in}}%
\pgfpathlineto{\pgfqpoint{3.796330in}{1.779968in}}%
\pgfpathlineto{\pgfqpoint{3.799195in}{1.785757in}}%
\pgfpathlineto{\pgfqpoint{3.799616in}{1.785463in}}%
\pgfpathlineto{\pgfqpoint{3.800964in}{1.785099in}}%
\pgfpathlineto{\pgfqpoint{3.801217in}{1.785496in}}%
\pgfpathlineto{\pgfqpoint{3.803323in}{1.790814in}}%
\pgfpathlineto{\pgfqpoint{3.804334in}{1.789931in}}%
\pgfpathlineto{\pgfqpoint{3.805766in}{1.790017in}}%
\pgfpathlineto{\pgfqpoint{3.807956in}{1.790826in}}%
\pgfpathlineto{\pgfqpoint{3.812168in}{1.789266in}}%
\pgfpathlineto{\pgfqpoint{3.813516in}{1.788402in}}%
\pgfpathlineto{\pgfqpoint{3.820761in}{1.782117in}}%
\pgfpathlineto{\pgfqpoint{3.827164in}{1.778817in}}%
\pgfpathlineto{\pgfqpoint{3.829354in}{1.777966in}}%
\pgfpathlineto{\pgfqpoint{3.836683in}{1.775773in}}%
\pgfpathlineto{\pgfqpoint{3.841317in}{1.778811in}}%
\pgfpathlineto{\pgfqpoint{3.842749in}{1.779026in}}%
\pgfpathlineto{\pgfqpoint{3.845024in}{1.779181in}}%
\pgfpathlineto{\pgfqpoint{3.846709in}{1.779325in}}%
\pgfpathlineto{\pgfqpoint{3.851342in}{1.777683in}}%
\pgfpathlineto{\pgfqpoint{3.852943in}{1.780565in}}%
\pgfpathlineto{\pgfqpoint{3.854543in}{1.781007in}}%
\pgfpathlineto{\pgfqpoint{3.855638in}{1.781922in}}%
\pgfpathlineto{\pgfqpoint{3.858166in}{1.788175in}}%
\pgfpathlineto{\pgfqpoint{3.859429in}{1.787209in}}%
\pgfpathlineto{\pgfqpoint{3.860525in}{1.787957in}}%
\pgfpathlineto{\pgfqpoint{3.861620in}{1.787911in}}%
\pgfpathlineto{\pgfqpoint{3.861788in}{1.787747in}}%
\pgfpathlineto{\pgfqpoint{3.867433in}{1.783503in}}%
\pgfpathlineto{\pgfqpoint{3.869623in}{1.784137in}}%
\pgfpathlineto{\pgfqpoint{3.870971in}{1.784542in}}%
\pgfpathlineto{\pgfqpoint{3.872150in}{1.787460in}}%
\pgfpathlineto{\pgfqpoint{3.873330in}{1.789345in}}%
\pgfpathlineto{\pgfqpoint{3.873751in}{1.789046in}}%
\pgfpathlineto{\pgfqpoint{3.875015in}{1.788561in}}%
\pgfpathlineto{\pgfqpoint{3.875352in}{1.788968in}}%
\pgfpathlineto{\pgfqpoint{3.877121in}{1.790091in}}%
\pgfpathlineto{\pgfqpoint{3.877289in}{1.789958in}}%
\pgfpathlineto{\pgfqpoint{3.885208in}{1.784174in}}%
\pgfpathlineto{\pgfqpoint{3.886219in}{1.784926in}}%
\pgfpathlineto{\pgfqpoint{3.888157in}{1.788209in}}%
\pgfpathlineto{\pgfqpoint{3.888831in}{1.787597in}}%
\pgfpathlineto{\pgfqpoint{3.889589in}{1.787525in}}%
\pgfpathlineto{\pgfqpoint{3.889926in}{1.787993in}}%
\pgfpathlineto{\pgfqpoint{3.891779in}{1.790163in}}%
\pgfpathlineto{\pgfqpoint{3.892201in}{1.789798in}}%
\pgfpathlineto{\pgfqpoint{3.894559in}{1.788760in}}%
\pgfpathlineto{\pgfqpoint{3.897255in}{1.786446in}}%
\pgfpathlineto{\pgfqpoint{3.902226in}{1.782926in}}%
\pgfpathlineto{\pgfqpoint{3.906017in}{1.783073in}}%
\pgfpathlineto{\pgfqpoint{3.907112in}{1.784258in}}%
\pgfpathlineto{\pgfqpoint{3.907617in}{1.783872in}}%
\pgfpathlineto{\pgfqpoint{3.909808in}{1.783271in}}%
\pgfpathlineto{\pgfqpoint{3.911914in}{1.783253in}}%
\pgfpathlineto{\pgfqpoint{3.917305in}{1.779867in}}%
\pgfpathlineto{\pgfqpoint{3.919664in}{1.780756in}}%
\pgfpathlineto{\pgfqpoint{3.921181in}{1.781129in}}%
\pgfpathlineto{\pgfqpoint{3.922276in}{1.784642in}}%
\pgfpathlineto{\pgfqpoint{3.923203in}{1.786123in}}%
\pgfpathlineto{\pgfqpoint{3.923708in}{1.785768in}}%
\pgfpathlineto{\pgfqpoint{3.924803in}{1.786080in}}%
\pgfpathlineto{\pgfqpoint{3.924887in}{1.786206in}}%
\pgfpathlineto{\pgfqpoint{3.926572in}{1.789117in}}%
\pgfpathlineto{\pgfqpoint{3.927415in}{1.788313in}}%
\pgfpathlineto{\pgfqpoint{3.930953in}{1.786733in}}%
\pgfpathlineto{\pgfqpoint{3.932470in}{1.786435in}}%
\pgfpathlineto{\pgfqpoint{3.934154in}{1.785644in}}%
\pgfpathlineto{\pgfqpoint{3.934407in}{1.785986in}}%
\pgfpathlineto{\pgfqpoint{3.936008in}{1.786666in}}%
\pgfpathlineto{\pgfqpoint{3.936092in}{1.786608in}}%
\pgfpathlineto{\pgfqpoint{3.948055in}{1.778876in}}%
\pgfpathlineto{\pgfqpoint{3.950666in}{1.777512in}}%
\pgfpathlineto{\pgfqpoint{3.955974in}{1.775580in}}%
\pgfpathlineto{\pgfqpoint{3.959512in}{1.775397in}}%
\pgfpathlineto{\pgfqpoint{3.961450in}{1.775358in}}%
\pgfpathlineto{\pgfqpoint{3.963808in}{1.775433in}}%
\pgfpathlineto{\pgfqpoint{3.966336in}{1.777485in}}%
\pgfpathlineto{\pgfqpoint{3.984364in}{1.773884in}}%
\pgfpathlineto{\pgfqpoint{3.985544in}{1.776258in}}%
\pgfpathlineto{\pgfqpoint{3.988745in}{1.781625in}}%
\pgfpathlineto{\pgfqpoint{3.992452in}{1.787268in}}%
\pgfpathlineto{\pgfqpoint{3.994136in}{1.787454in}}%
\pgfpathlineto{\pgfqpoint{3.997927in}{1.784956in}}%
\pgfpathlineto{\pgfqpoint{3.999444in}{1.784487in}}%
\pgfpathlineto{\pgfqpoint{4.000286in}{1.784703in}}%
\pgfpathlineto{\pgfqpoint{4.000455in}{1.785043in}}%
\pgfpathlineto{\pgfqpoint{4.002477in}{1.792435in}}%
\pgfpathlineto{\pgfqpoint{4.004162in}{1.791692in}}%
\pgfpathlineto{\pgfqpoint{4.006436in}{1.794720in}}%
\pgfpathlineto{\pgfqpoint{4.007194in}{1.793704in}}%
\pgfpathlineto{\pgfqpoint{4.012923in}{1.787033in}}%
\pgfpathlineto{\pgfqpoint{4.030025in}{1.784497in}}%
\pgfpathlineto{\pgfqpoint{4.035248in}{1.781568in}}%
\pgfpathlineto{\pgfqpoint{4.037607in}{1.780838in}}%
\pgfpathlineto{\pgfqpoint{4.043504in}{1.778232in}}%
\pgfpathlineto{\pgfqpoint{4.044936in}{1.778711in}}%
\pgfpathlineto{\pgfqpoint{4.046537in}{1.778505in}}%
\pgfpathlineto{\pgfqpoint{4.047463in}{1.779797in}}%
\pgfpathlineto{\pgfqpoint{4.049317in}{1.784767in}}%
\pgfpathlineto{\pgfqpoint{4.049906in}{1.784330in}}%
\pgfpathlineto{\pgfqpoint{4.050665in}{1.784284in}}%
\pgfpathlineto{\pgfqpoint{4.050917in}{1.784709in}}%
\pgfpathlineto{\pgfqpoint{4.053360in}{1.787500in}}%
\pgfpathlineto{\pgfqpoint{4.055214in}{1.785698in}}%
\pgfpathlineto{\pgfqpoint{4.061785in}{1.780340in}}%
\pgfpathlineto{\pgfqpoint{4.066924in}{1.777936in}}%
\pgfpathlineto{\pgfqpoint{4.072315in}{1.777343in}}%
\pgfpathlineto{\pgfqpoint{4.074506in}{1.778891in}}%
\pgfpathlineto{\pgfqpoint{4.077454in}{1.778581in}}%
\pgfpathlineto{\pgfqpoint{4.080824in}{1.776827in}}%
\pgfpathlineto{\pgfqpoint{4.087142in}{1.774518in}}%
\pgfpathlineto{\pgfqpoint{4.088743in}{1.775496in}}%
\pgfpathlineto{\pgfqpoint{4.091018in}{1.777667in}}%
\pgfpathlineto{\pgfqpoint{4.091944in}{1.778657in}}%
\pgfpathlineto{\pgfqpoint{4.094472in}{1.784512in}}%
\pgfpathlineto{\pgfqpoint{4.095061in}{1.784101in}}%
\pgfpathlineto{\pgfqpoint{4.099358in}{1.782272in}}%
\pgfpathlineto{\pgfqpoint{4.103739in}{1.782101in}}%
\pgfpathlineto{\pgfqpoint{4.105760in}{1.783471in}}%
\pgfpathlineto{\pgfqpoint{4.109046in}{1.782548in}}%
\pgfpathlineto{\pgfqpoint{4.109130in}{1.782728in}}%
\pgfpathlineto{\pgfqpoint{4.112247in}{1.787424in}}%
\pgfpathlineto{\pgfqpoint{4.114185in}{1.787672in}}%
\pgfpathlineto{\pgfqpoint{4.116965in}{1.795315in}}%
\pgfpathlineto{\pgfqpoint{4.118144in}{1.793847in}}%
\pgfpathlineto{\pgfqpoint{4.118903in}{1.793480in}}%
\pgfpathlineto{\pgfqpoint{4.119240in}{1.793987in}}%
\pgfpathlineto{\pgfqpoint{4.121514in}{1.800728in}}%
\pgfpathlineto{\pgfqpoint{4.122525in}{1.799428in}}%
\pgfpathlineto{\pgfqpoint{4.123199in}{1.799216in}}%
\pgfpathlineto{\pgfqpoint{4.123620in}{1.799794in}}%
\pgfpathlineto{\pgfqpoint{4.125726in}{1.801399in}}%
\pgfpathlineto{\pgfqpoint{4.127411in}{1.799031in}}%
\pgfpathlineto{\pgfqpoint{4.134488in}{1.790564in}}%
\pgfpathlineto{\pgfqpoint{4.136678in}{1.790319in}}%
\pgfpathlineto{\pgfqpoint{4.137521in}{1.790036in}}%
\pgfpathlineto{\pgfqpoint{4.137858in}{1.790563in}}%
\pgfpathlineto{\pgfqpoint{4.140132in}{1.797860in}}%
\pgfpathlineto{\pgfqpoint{4.141396in}{1.796310in}}%
\pgfpathlineto{\pgfqpoint{4.143671in}{1.795709in}}%
\pgfpathlineto{\pgfqpoint{4.146619in}{1.791816in}}%
\pgfpathlineto{\pgfqpoint{4.151168in}{1.787651in}}%
\pgfpathlineto{\pgfqpoint{4.153443in}{1.786736in}}%
\pgfpathlineto{\pgfqpoint{4.157487in}{1.785178in}}%
\pgfpathlineto{\pgfqpoint{4.164058in}{1.780527in}}%
\pgfpathlineto{\pgfqpoint{4.170882in}{1.777985in}}%
\pgfpathlineto{\pgfqpoint{4.179390in}{1.775067in}}%
\pgfpathlineto{\pgfqpoint{4.183350in}{1.774624in}}%
\pgfpathlineto{\pgfqpoint{4.185624in}{1.776392in}}%
\pgfpathlineto{\pgfqpoint{4.187309in}{1.777020in}}%
\pgfpathlineto{\pgfqpoint{4.190089in}{1.777751in}}%
\pgfpathlineto{\pgfqpoint{4.199103in}{1.774920in}}%
\pgfpathlineto{\pgfqpoint{4.205843in}{1.773605in}}%
\pgfpathlineto{\pgfqpoint{4.212667in}{1.773024in}}%
\pgfpathlineto{\pgfqpoint{4.215615in}{1.773988in}}%
\pgfpathlineto{\pgfqpoint{4.217806in}{1.774316in}}%
\pgfpathlineto{\pgfqpoint{4.219491in}{1.774740in}}%
\pgfpathlineto{\pgfqpoint{4.222692in}{1.774472in}}%
\pgfpathlineto{\pgfqpoint{4.229263in}{1.775299in}}%
\pgfpathlineto{\pgfqpoint{4.234318in}{1.774216in}}%
\pgfpathlineto{\pgfqpoint{4.238193in}{1.773387in}}%
\pgfpathlineto{\pgfqpoint{4.242742in}{1.773117in}}%
\pgfpathlineto{\pgfqpoint{4.244680in}{1.773398in}}%
\pgfpathlineto{\pgfqpoint{4.246028in}{1.774007in}}%
\pgfpathlineto{\pgfqpoint{4.248471in}{1.776025in}}%
\pgfpathlineto{\pgfqpoint{4.251335in}{1.776355in}}%
\pgfpathlineto{\pgfqpoint{4.252936in}{1.776259in}}%
\pgfpathlineto{\pgfqpoint{4.259423in}{1.774366in}}%
\pgfpathlineto{\pgfqpoint{4.267426in}{1.772227in}}%
\pgfpathlineto{\pgfqpoint{4.277956in}{1.773008in}}%
\pgfpathlineto{\pgfqpoint{4.284527in}{1.774635in}}%
\pgfpathlineto{\pgfqpoint{4.286886in}{1.774766in}}%
\pgfpathlineto{\pgfqpoint{4.289666in}{1.776115in}}%
\pgfpathlineto{\pgfqpoint{4.292278in}{1.776147in}}%
\pgfpathlineto{\pgfqpoint{4.294552in}{1.777576in}}%
\pgfpathlineto{\pgfqpoint{4.300618in}{1.775387in}}%
\pgfpathlineto{\pgfqpoint{4.316035in}{1.775757in}}%
\pgfpathlineto{\pgfqpoint{4.317467in}{1.776057in}}%
\pgfpathlineto{\pgfqpoint{4.319320in}{1.776238in}}%
\pgfpathlineto{\pgfqpoint{4.321258in}{1.776984in}}%
\pgfpathlineto{\pgfqpoint{4.326144in}{1.775500in}}%
\pgfpathlineto{\pgfqpoint{4.332968in}{1.774255in}}%
\pgfpathlineto{\pgfqpoint{4.335917in}{1.775424in}}%
\pgfpathlineto{\pgfqpoint{4.338275in}{1.775788in}}%
\pgfpathlineto{\pgfqpoint{4.340213in}{1.776279in}}%
\pgfpathlineto{\pgfqpoint{4.349733in}{1.774525in}}%
\pgfpathlineto{\pgfqpoint{4.352260in}{1.774669in}}%
\pgfpathlineto{\pgfqpoint{4.355630in}{1.775423in}}%
\pgfpathlineto{\pgfqpoint{4.360347in}{1.774908in}}%
\pgfpathlineto{\pgfqpoint{4.376522in}{1.777400in}}%
\pgfpathlineto{\pgfqpoint{4.382335in}{1.784898in}}%
\pgfpathlineto{\pgfqpoint{4.383936in}{1.789116in}}%
\pgfpathlineto{\pgfqpoint{4.384357in}{1.788835in}}%
\pgfpathlineto{\pgfqpoint{4.386716in}{1.787967in}}%
\pgfpathlineto{\pgfqpoint{4.388485in}{1.788127in}}%
\pgfpathlineto{\pgfqpoint{4.391097in}{1.786189in}}%
\pgfpathlineto{\pgfqpoint{4.394130in}{1.784677in}}%
\pgfpathlineto{\pgfqpoint{4.395899in}{1.784477in}}%
\pgfpathlineto{\pgfqpoint{4.404323in}{1.781731in}}%
\pgfpathlineto{\pgfqpoint{4.406682in}{1.783064in}}%
\pgfpathlineto{\pgfqpoint{4.408535in}{1.782765in}}%
\pgfpathlineto{\pgfqpoint{4.408620in}{1.782879in}}%
\pgfpathlineto{\pgfqpoint{4.410052in}{1.784035in}}%
\pgfpathlineto{\pgfqpoint{4.410473in}{1.783718in}}%
\pgfpathlineto{\pgfqpoint{4.412411in}{1.783677in}}%
\pgfpathlineto{\pgfqpoint{4.413758in}{1.783110in}}%
\pgfpathlineto{\pgfqpoint{4.417634in}{1.781084in}}%
\pgfpathlineto{\pgfqpoint{4.419740in}{1.780653in}}%
\pgfpathlineto{\pgfqpoint{4.422267in}{1.780249in}}%
\pgfpathlineto{\pgfqpoint{4.425384in}{1.781196in}}%
\pgfpathlineto{\pgfqpoint{4.429512in}{1.780081in}}%
\pgfpathlineto{\pgfqpoint{4.431366in}{1.780048in}}%
\pgfpathlineto{\pgfqpoint{4.432461in}{1.780718in}}%
\pgfpathlineto{\pgfqpoint{4.433724in}{1.781607in}}%
\pgfpathlineto{\pgfqpoint{4.434061in}{1.781390in}}%
\pgfpathlineto{\pgfqpoint{4.435157in}{1.781163in}}%
\pgfpathlineto{\pgfqpoint{4.435409in}{1.781673in}}%
\pgfpathlineto{\pgfqpoint{4.439200in}{1.789788in}}%
\pgfpathlineto{\pgfqpoint{4.441896in}{1.797903in}}%
\pgfpathlineto{\pgfqpoint{4.443076in}{1.796399in}}%
\pgfpathlineto{\pgfqpoint{4.445350in}{1.795244in}}%
\pgfpathlineto{\pgfqpoint{4.451332in}{1.793829in}}%
\pgfpathlineto{\pgfqpoint{4.452848in}{1.793460in}}%
\pgfpathlineto{\pgfqpoint{4.454449in}{1.793251in}}%
\pgfpathlineto{\pgfqpoint{4.455797in}{1.792694in}}%
\pgfpathlineto{\pgfqpoint{4.458577in}{1.791181in}}%
\pgfpathlineto{\pgfqpoint{4.467254in}{1.786672in}}%
\pgfpathlineto{\pgfqpoint{4.468517in}{1.785928in}}%
\pgfpathlineto{\pgfqpoint{4.470034in}{1.785975in}}%
\pgfpathlineto{\pgfqpoint{4.472308in}{1.785627in}}%
\pgfpathlineto{\pgfqpoint{4.476268in}{1.783545in}}%
\pgfpathlineto{\pgfqpoint{4.478795in}{1.783978in}}%
\pgfpathlineto{\pgfqpoint{4.481070in}{1.785667in}}%
\pgfpathlineto{\pgfqpoint{4.481828in}{1.785898in}}%
\pgfpathlineto{\pgfqpoint{4.481997in}{1.786260in}}%
\pgfpathlineto{\pgfqpoint{4.484271in}{1.789056in}}%
\pgfpathlineto{\pgfqpoint{4.485619in}{1.788866in}}%
\pgfpathlineto{\pgfqpoint{4.485788in}{1.789295in}}%
\pgfpathlineto{\pgfqpoint{4.488315in}{1.794841in}}%
\pgfpathlineto{\pgfqpoint{4.488568in}{1.794668in}}%
\pgfpathlineto{\pgfqpoint{4.490758in}{1.793997in}}%
\pgfpathlineto{\pgfqpoint{4.492443in}{1.793049in}}%
\pgfpathlineto{\pgfqpoint{4.497582in}{1.789764in}}%
\pgfpathlineto{\pgfqpoint{4.499435in}{1.789446in}}%
\pgfpathlineto{\pgfqpoint{4.500615in}{1.789118in}}%
\pgfpathlineto{\pgfqpoint{4.500867in}{1.789498in}}%
\pgfpathlineto{\pgfqpoint{4.503479in}{1.793753in}}%
\pgfpathlineto{\pgfqpoint{4.503900in}{1.793395in}}%
\pgfpathlineto{\pgfqpoint{4.506933in}{1.791413in}}%
\pgfpathlineto{\pgfqpoint{4.508618in}{1.789699in}}%
\pgfpathlineto{\pgfqpoint{4.510977in}{1.787736in}}%
\pgfpathlineto{\pgfqpoint{4.511314in}{1.788287in}}%
\pgfpathlineto{\pgfqpoint{4.513336in}{1.794329in}}%
\pgfpathlineto{\pgfqpoint{4.514346in}{1.793144in}}%
\pgfpathlineto{\pgfqpoint{4.516031in}{1.792861in}}%
\pgfpathlineto{\pgfqpoint{4.518137in}{1.792140in}}%
\pgfpathlineto{\pgfqpoint{4.531532in}{1.780079in}}%
\pgfpathlineto{\pgfqpoint{4.536756in}{1.778231in}}%
\pgfpathlineto{\pgfqpoint{4.543748in}{1.775998in}}%
\pgfpathlineto{\pgfqpoint{4.553015in}{1.774616in}}%
\pgfpathlineto{\pgfqpoint{4.559502in}{1.778571in}}%
\pgfpathlineto{\pgfqpoint{4.565399in}{1.776668in}}%
\pgfpathlineto{\pgfqpoint{4.567926in}{1.776509in}}%
\pgfpathlineto{\pgfqpoint{4.573739in}{1.774897in}}%
\pgfpathlineto{\pgfqpoint{4.578372in}{1.774913in}}%
\pgfpathlineto{\pgfqpoint{4.581658in}{1.777067in}}%
\pgfpathlineto{\pgfqpoint{4.582753in}{1.778185in}}%
\pgfpathlineto{\pgfqpoint{4.585028in}{1.781398in}}%
\pgfpathlineto{\pgfqpoint{4.585112in}{1.781359in}}%
\pgfpathlineto{\pgfqpoint{4.592441in}{1.778862in}}%
\pgfpathlineto{\pgfqpoint{4.594379in}{1.778701in}}%
\pgfpathlineto{\pgfqpoint{4.596148in}{1.778907in}}%
\pgfpathlineto{\pgfqpoint{4.600697in}{1.777565in}}%
\pgfpathlineto{\pgfqpoint{4.606005in}{1.775929in}}%
\pgfpathlineto{\pgfqpoint{4.622348in}{1.773064in}}%
\pgfpathlineto{\pgfqpoint{4.624960in}{1.774843in}}%
\pgfpathlineto{\pgfqpoint{4.626981in}{1.775223in}}%
\pgfpathlineto{\pgfqpoint{4.630183in}{1.781386in}}%
\pgfpathlineto{\pgfqpoint{4.630857in}{1.780956in}}%
\pgfpathlineto{\pgfqpoint{4.632205in}{1.780480in}}%
\pgfpathlineto{\pgfqpoint{4.632457in}{1.780960in}}%
\pgfpathlineto{\pgfqpoint{4.634900in}{1.783979in}}%
\pgfpathlineto{\pgfqpoint{4.645768in}{1.779250in}}%
\pgfpathlineto{\pgfqpoint{4.646021in}{1.779804in}}%
\pgfpathlineto{\pgfqpoint{4.648295in}{1.782181in}}%
\pgfpathlineto{\pgfqpoint{4.650907in}{1.782410in}}%
\pgfpathlineto{\pgfqpoint{4.652592in}{1.782212in}}%
\pgfpathlineto{\pgfqpoint{4.660595in}{1.777376in}}%
\pgfpathlineto{\pgfqpoint{4.662027in}{1.778490in}}%
\pgfpathlineto{\pgfqpoint{4.663796in}{1.778898in}}%
\pgfpathlineto{\pgfqpoint{4.666492in}{1.778593in}}%
\pgfpathlineto{\pgfqpoint{4.680561in}{1.778131in}}%
\pgfpathlineto{\pgfqpoint{4.681909in}{1.780080in}}%
\pgfpathlineto{\pgfqpoint{4.682330in}{1.779850in}}%
\pgfpathlineto{\pgfqpoint{4.683846in}{1.780092in}}%
\pgfpathlineto{\pgfqpoint{4.685279in}{1.780959in}}%
\pgfpathlineto{\pgfqpoint{4.685531in}{1.780787in}}%
\pgfpathlineto{\pgfqpoint{4.687722in}{1.780483in}}%
\pgfpathlineto{\pgfqpoint{4.693113in}{1.782503in}}%
\pgfpathlineto{\pgfqpoint{4.693872in}{1.784090in}}%
\pgfpathlineto{\pgfqpoint{4.696062in}{1.789670in}}%
\pgfpathlineto{\pgfqpoint{4.696315in}{1.789530in}}%
\pgfpathlineto{\pgfqpoint{4.697747in}{1.789901in}}%
\pgfpathlineto{\pgfqpoint{4.700021in}{1.792567in}}%
\pgfpathlineto{\pgfqpoint{4.700695in}{1.791748in}}%
\pgfpathlineto{\pgfqpoint{4.706424in}{1.785705in}}%
\pgfpathlineto{\pgfqpoint{4.709373in}{1.784462in}}%
\pgfpathlineto{\pgfqpoint{4.713753in}{1.783662in}}%
\pgfpathlineto{\pgfqpoint{4.716449in}{1.786775in}}%
\pgfpathlineto{\pgfqpoint{4.718808in}{1.785766in}}%
\pgfpathlineto{\pgfqpoint{4.719145in}{1.786377in}}%
\pgfpathlineto{\pgfqpoint{4.720409in}{1.787469in}}%
\pgfpathlineto{\pgfqpoint{4.720746in}{1.787214in}}%
\pgfpathlineto{\pgfqpoint{4.723189in}{1.786243in}}%
\pgfpathlineto{\pgfqpoint{4.724705in}{1.785603in}}%
\pgfpathlineto{\pgfqpoint{4.732371in}{1.780293in}}%
\pgfpathlineto{\pgfqpoint{4.736247in}{1.780594in}}%
\pgfpathlineto{\pgfqpoint{4.740796in}{1.779159in}}%
\pgfpathlineto{\pgfqpoint{4.744250in}{1.779726in}}%
\pgfpathlineto{\pgfqpoint{4.745598in}{1.781004in}}%
\pgfpathlineto{\pgfqpoint{4.745935in}{1.780816in}}%
\pgfpathlineto{\pgfqpoint{4.747198in}{1.780958in}}%
\pgfpathlineto{\pgfqpoint{4.747283in}{1.781093in}}%
\pgfpathlineto{\pgfqpoint{4.749304in}{1.783118in}}%
\pgfpathlineto{\pgfqpoint{4.749557in}{1.782949in}}%
\pgfpathlineto{\pgfqpoint{4.751832in}{1.782632in}}%
\pgfpathlineto{\pgfqpoint{4.754612in}{1.781873in}}%
\pgfpathlineto{\pgfqpoint{4.757982in}{1.780298in}}%
\pgfpathlineto{\pgfqpoint{4.758150in}{1.780535in}}%
\pgfpathlineto{\pgfqpoint{4.760172in}{1.783655in}}%
\pgfpathlineto{\pgfqpoint{4.760930in}{1.783073in}}%
\pgfpathlineto{\pgfqpoint{4.762362in}{1.783173in}}%
\pgfpathlineto{\pgfqpoint{4.762447in}{1.783308in}}%
\pgfpathlineto{\pgfqpoint{4.764131in}{1.785572in}}%
\pgfpathlineto{\pgfqpoint{4.764637in}{1.785182in}}%
\pgfpathlineto{\pgfqpoint{4.765732in}{1.785063in}}%
\pgfpathlineto{\pgfqpoint{4.765985in}{1.785389in}}%
\pgfpathlineto{\pgfqpoint{4.768175in}{1.787909in}}%
\pgfpathlineto{\pgfqpoint{4.768765in}{1.787419in}}%
\pgfpathlineto{\pgfqpoint{4.771292in}{1.786982in}}%
\pgfpathlineto{\pgfqpoint{4.773735in}{1.787380in}}%
\pgfpathlineto{\pgfqpoint{4.775252in}{1.786773in}}%
\pgfpathlineto{\pgfqpoint{4.775505in}{1.787171in}}%
\pgfpathlineto{\pgfqpoint{4.777611in}{1.793258in}}%
\pgfpathlineto{\pgfqpoint{4.778706in}{1.792084in}}%
\pgfpathlineto{\pgfqpoint{4.779296in}{1.792101in}}%
\pgfpathlineto{\pgfqpoint{4.779548in}{1.792580in}}%
\pgfpathlineto{\pgfqpoint{4.781991in}{1.798200in}}%
\pgfpathlineto{\pgfqpoint{4.782581in}{1.797634in}}%
\pgfpathlineto{\pgfqpoint{4.783845in}{1.796558in}}%
\pgfpathlineto{\pgfqpoint{4.784266in}{1.797002in}}%
\pgfpathlineto{\pgfqpoint{4.786119in}{1.797770in}}%
\pgfpathlineto{\pgfqpoint{4.786204in}{1.797702in}}%
\pgfpathlineto{\pgfqpoint{4.789236in}{1.793392in}}%
\pgfpathlineto{\pgfqpoint{4.796060in}{1.785633in}}%
\pgfpathlineto{\pgfqpoint{4.799009in}{1.784091in}}%
\pgfpathlineto{\pgfqpoint{4.804400in}{1.782770in}}%
\pgfpathlineto{\pgfqpoint{4.809539in}{1.779326in}}%
\pgfpathlineto{\pgfqpoint{4.811224in}{1.779536in}}%
\pgfpathlineto{\pgfqpoint{4.812825in}{1.778982in}}%
\pgfpathlineto{\pgfqpoint{4.814847in}{1.779045in}}%
\pgfpathlineto{\pgfqpoint{4.816532in}{1.778471in}}%
\pgfpathlineto{\pgfqpoint{4.825546in}{1.774204in}}%
\pgfpathlineto{\pgfqpoint{4.836582in}{1.773084in}}%
\pgfpathlineto{\pgfqpoint{4.838772in}{1.775254in}}%
\pgfpathlineto{\pgfqpoint{4.841468in}{1.776472in}}%
\pgfpathlineto{\pgfqpoint{4.843153in}{1.776098in}}%
\pgfpathlineto{\pgfqpoint{4.846944in}{1.775550in}}%
\pgfpathlineto{\pgfqpoint{4.851072in}{1.775148in}}%
\pgfpathlineto{\pgfqpoint{4.858064in}{1.774052in}}%
\pgfpathlineto{\pgfqpoint{4.866236in}{1.780408in}}%
\pgfpathlineto{\pgfqpoint{4.867331in}{1.781230in}}%
\pgfpathlineto{\pgfqpoint{4.869016in}{1.782182in}}%
\pgfpathlineto{\pgfqpoint{4.869100in}{1.782129in}}%
\pgfpathlineto{\pgfqpoint{4.873649in}{1.780407in}}%
\pgfpathlineto{\pgfqpoint{4.875082in}{1.780040in}}%
\pgfpathlineto{\pgfqpoint{4.880305in}{1.778339in}}%
\pgfpathlineto{\pgfqpoint{4.885022in}{1.776281in}}%
\pgfpathlineto{\pgfqpoint{4.891004in}{1.774811in}}%
\pgfpathlineto{\pgfqpoint{4.893194in}{1.775207in}}%
\pgfpathlineto{\pgfqpoint{4.914003in}{1.771785in}}%
\pgfpathlineto{\pgfqpoint{4.919984in}{1.775901in}}%
\pgfpathlineto{\pgfqpoint{4.921332in}{1.776443in}}%
\pgfpathlineto{\pgfqpoint{4.923269in}{1.778224in}}%
\pgfpathlineto{\pgfqpoint{4.925460in}{1.779774in}}%
\pgfpathlineto{\pgfqpoint{4.928661in}{1.782307in}}%
\pgfpathlineto{\pgfqpoint{4.929672in}{1.782055in}}%
\pgfpathlineto{\pgfqpoint{4.929925in}{1.782503in}}%
\pgfpathlineto{\pgfqpoint{4.932031in}{1.784808in}}%
\pgfpathlineto{\pgfqpoint{4.933463in}{1.784147in}}%
\pgfpathlineto{\pgfqpoint{4.934558in}{1.784332in}}%
\pgfpathlineto{\pgfqpoint{4.934642in}{1.784461in}}%
\pgfpathlineto{\pgfqpoint{4.937170in}{1.789957in}}%
\pgfpathlineto{\pgfqpoint{4.938265in}{1.788813in}}%
\pgfpathlineto{\pgfqpoint{4.939866in}{1.788330in}}%
\pgfpathlineto{\pgfqpoint{4.939950in}{1.788418in}}%
\pgfpathlineto{\pgfqpoint{4.941635in}{1.790356in}}%
\pgfpathlineto{\pgfqpoint{4.942393in}{1.789716in}}%
\pgfpathlineto{\pgfqpoint{4.955282in}{1.781045in}}%
\pgfpathlineto{\pgfqpoint{4.956546in}{1.781514in}}%
\pgfpathlineto{\pgfqpoint{4.958231in}{1.782391in}}%
\pgfpathlineto{\pgfqpoint{4.962443in}{1.780745in}}%
\pgfpathlineto{\pgfqpoint{4.962612in}{1.781071in}}%
\pgfpathlineto{\pgfqpoint{4.963623in}{1.782237in}}%
\pgfpathlineto{\pgfqpoint{4.964128in}{1.781918in}}%
\pgfpathlineto{\pgfqpoint{4.968003in}{1.781005in}}%
\pgfpathlineto{\pgfqpoint{4.969351in}{1.780842in}}%
\pgfpathlineto{\pgfqpoint{4.977270in}{1.776290in}}%
\pgfpathlineto{\pgfqpoint{4.979545in}{1.776890in}}%
\pgfpathlineto{\pgfqpoint{4.981651in}{1.777970in}}%
\pgfpathlineto{\pgfqpoint{4.985947in}{1.781018in}}%
\pgfpathlineto{\pgfqpoint{4.989149in}{1.780389in}}%
\pgfpathlineto{\pgfqpoint{4.991844in}{1.781685in}}%
\pgfpathlineto{\pgfqpoint{4.993277in}{1.782233in}}%
\pgfpathlineto{\pgfqpoint{4.995298in}{1.783061in}}%
\pgfpathlineto{\pgfqpoint{5.004060in}{1.779077in}}%
\pgfpathlineto{\pgfqpoint{5.006335in}{1.779410in}}%
\pgfpathlineto{\pgfqpoint{5.007598in}{1.780440in}}%
\pgfpathlineto{\pgfqpoint{5.008019in}{1.780157in}}%
\pgfpathlineto{\pgfqpoint{5.008862in}{1.780094in}}%
\pgfpathlineto{\pgfqpoint{5.009115in}{1.780582in}}%
\pgfpathlineto{\pgfqpoint{5.010884in}{1.781963in}}%
\pgfpathlineto{\pgfqpoint{5.012906in}{1.781905in}}%
\pgfpathlineto{\pgfqpoint{5.013917in}{1.782770in}}%
\pgfpathlineto{\pgfqpoint{5.014927in}{1.784569in}}%
\pgfpathlineto{\pgfqpoint{5.015517in}{1.784105in}}%
\pgfpathlineto{\pgfqpoint{5.017539in}{1.782980in}}%
\pgfpathlineto{\pgfqpoint{5.017792in}{1.783315in}}%
\pgfpathlineto{\pgfqpoint{5.019729in}{1.786789in}}%
\pgfpathlineto{\pgfqpoint{5.020488in}{1.786132in}}%
\pgfpathlineto{\pgfqpoint{5.022257in}{1.785265in}}%
\pgfpathlineto{\pgfqpoint{5.022509in}{1.785584in}}%
\pgfpathlineto{\pgfqpoint{5.023857in}{1.786904in}}%
\pgfpathlineto{\pgfqpoint{5.024279in}{1.786533in}}%
\pgfpathlineto{\pgfqpoint{5.027480in}{1.784687in}}%
\pgfpathlineto{\pgfqpoint{5.028575in}{1.785160in}}%
\pgfpathlineto{\pgfqpoint{5.030260in}{1.785785in}}%
\pgfpathlineto{\pgfqpoint{5.032872in}{1.785401in}}%
\pgfpathlineto{\pgfqpoint{5.034725in}{1.787848in}}%
\pgfpathlineto{\pgfqpoint{5.035652in}{1.790158in}}%
\pgfpathlineto{\pgfqpoint{5.036241in}{1.789583in}}%
\pgfpathlineto{\pgfqpoint{5.041296in}{1.785252in}}%
\pgfpathlineto{\pgfqpoint{5.045508in}{1.784678in}}%
\pgfpathlineto{\pgfqpoint{5.047362in}{1.785101in}}%
\pgfpathlineto{\pgfqpoint{5.049805in}{1.784231in}}%
\pgfpathlineto{\pgfqpoint{5.050057in}{1.784837in}}%
\pgfpathlineto{\pgfqpoint{5.051658in}{1.788401in}}%
\pgfpathlineto{\pgfqpoint{5.052164in}{1.788136in}}%
\pgfpathlineto{\pgfqpoint{5.062273in}{1.782040in}}%
\pgfpathlineto{\pgfqpoint{5.065558in}{1.781374in}}%
\pgfpathlineto{\pgfqpoint{5.069097in}{1.781276in}}%
\pgfpathlineto{\pgfqpoint{5.070529in}{1.781237in}}%
\pgfpathlineto{\pgfqpoint{5.079290in}{1.778775in}}%
\pgfpathlineto{\pgfqpoint{5.083166in}{1.781318in}}%
\pgfpathlineto{\pgfqpoint{5.085861in}{1.782217in}}%
\pgfpathlineto{\pgfqpoint{5.089231in}{1.780790in}}%
\pgfpathlineto{\pgfqpoint{5.094707in}{1.786939in}}%
\pgfpathlineto{\pgfqpoint{5.095634in}{1.790269in}}%
\pgfpathlineto{\pgfqpoint{5.096223in}{1.789793in}}%
\pgfpathlineto{\pgfqpoint{5.102036in}{1.786429in}}%
\pgfpathlineto{\pgfqpoint{5.102710in}{1.787773in}}%
\pgfpathlineto{\pgfqpoint{5.103637in}{1.789593in}}%
\pgfpathlineto{\pgfqpoint{5.104311in}{1.789385in}}%
\pgfpathlineto{\pgfqpoint{5.106164in}{1.790255in}}%
\pgfpathlineto{\pgfqpoint{5.106838in}{1.790039in}}%
\pgfpathlineto{\pgfqpoint{5.107091in}{1.789719in}}%
\pgfpathlineto{\pgfqpoint{5.108607in}{1.789653in}}%
\pgfpathlineto{\pgfqpoint{5.110377in}{1.789663in}}%
\pgfpathlineto{\pgfqpoint{5.114504in}{1.786509in}}%
\pgfpathlineto{\pgfqpoint{5.117453in}{1.788562in}}%
\pgfpathlineto{\pgfqpoint{5.120486in}{1.792298in}}%
\pgfpathlineto{\pgfqpoint{5.120570in}{1.792253in}}%
\pgfpathlineto{\pgfqpoint{5.130090in}{1.786054in}}%
\pgfpathlineto{\pgfqpoint{5.131775in}{1.785305in}}%
\pgfpathlineto{\pgfqpoint{5.134049in}{1.784698in}}%
\pgfpathlineto{\pgfqpoint{5.134302in}{1.785551in}}%
\pgfpathlineto{\pgfqpoint{5.135060in}{1.786794in}}%
\pgfpathlineto{\pgfqpoint{5.135650in}{1.786405in}}%
\pgfpathlineto{\pgfqpoint{5.137503in}{1.785931in}}%
\pgfpathlineto{\pgfqpoint{5.137672in}{1.786174in}}%
\pgfpathlineto{\pgfqpoint{5.140368in}{1.789279in}}%
\pgfpathlineto{\pgfqpoint{5.141968in}{1.788686in}}%
\pgfpathlineto{\pgfqpoint{5.144580in}{1.786595in}}%
\pgfpathlineto{\pgfqpoint{5.144917in}{1.787035in}}%
\pgfpathlineto{\pgfqpoint{5.146939in}{1.791584in}}%
\pgfpathlineto{\pgfqpoint{5.147697in}{1.793771in}}%
\pgfpathlineto{\pgfqpoint{5.148287in}{1.793063in}}%
\pgfpathlineto{\pgfqpoint{5.150308in}{1.792042in}}%
\pgfpathlineto{\pgfqpoint{5.152583in}{1.790166in}}%
\pgfpathlineto{\pgfqpoint{5.157722in}{1.786190in}}%
\pgfpathlineto{\pgfqpoint{5.161007in}{1.784449in}}%
\pgfpathlineto{\pgfqpoint{5.164714in}{1.782315in}}%
\pgfpathlineto{\pgfqpoint{5.168842in}{1.780573in}}%
\pgfpathlineto{\pgfqpoint{5.171959in}{1.783155in}}%
\pgfpathlineto{\pgfqpoint{5.176677in}{1.781071in}}%
\pgfpathlineto{\pgfqpoint{5.177941in}{1.780613in}}%
\pgfpathlineto{\pgfqpoint{5.184006in}{1.778006in}}%
\pgfpathlineto{\pgfqpoint{5.185691in}{1.778119in}}%
\pgfpathlineto{\pgfqpoint{5.187544in}{1.778637in}}%
\pgfpathlineto{\pgfqpoint{5.188977in}{1.779504in}}%
\pgfpathlineto{\pgfqpoint{5.189229in}{1.779338in}}%
\pgfpathlineto{\pgfqpoint{5.195885in}{1.776222in}}%
\pgfpathlineto{\pgfqpoint{5.198749in}{1.776570in}}%
\pgfpathlineto{\pgfqpoint{5.200097in}{1.776303in}}%
\pgfpathlineto{\pgfqpoint{5.203214in}{1.775942in}}%
\pgfpathlineto{\pgfqpoint{5.206668in}{1.776210in}}%
\pgfpathlineto{\pgfqpoint{5.216188in}{1.776466in}}%
\pgfpathlineto{\pgfqpoint{5.217114in}{1.777974in}}%
\pgfpathlineto{\pgfqpoint{5.218210in}{1.779471in}}%
\pgfpathlineto{\pgfqpoint{5.218631in}{1.779209in}}%
\pgfpathlineto{\pgfqpoint{5.219810in}{1.778856in}}%
\pgfpathlineto{\pgfqpoint{5.220063in}{1.779403in}}%
\pgfpathlineto{\pgfqpoint{5.221411in}{1.782717in}}%
\pgfpathlineto{\pgfqpoint{5.222001in}{1.782326in}}%
\pgfpathlineto{\pgfqpoint{5.222927in}{1.782414in}}%
\pgfpathlineto{\pgfqpoint{5.223180in}{1.782753in}}%
\pgfpathlineto{\pgfqpoint{5.224444in}{1.782559in}}%
\pgfpathlineto{\pgfqpoint{5.226971in}{1.782074in}}%
\pgfpathlineto{\pgfqpoint{5.228235in}{1.781564in}}%
\pgfpathlineto{\pgfqpoint{5.232868in}{1.779318in}}%
\pgfpathlineto{\pgfqpoint{5.242051in}{1.777656in}}%
\pgfpathlineto{\pgfqpoint{5.245084in}{1.776867in}}%
\pgfpathlineto{\pgfqpoint{5.271115in}{1.775380in}}%
\pgfpathlineto{\pgfqpoint{5.273727in}{1.780352in}}%
\pgfpathlineto{\pgfqpoint{5.274822in}{1.784106in}}%
\pgfpathlineto{\pgfqpoint{5.275327in}{1.783717in}}%
\pgfpathlineto{\pgfqpoint{5.276338in}{1.783340in}}%
\pgfpathlineto{\pgfqpoint{5.276591in}{1.783861in}}%
\pgfpathlineto{\pgfqpoint{5.278697in}{1.787355in}}%
\pgfpathlineto{\pgfqpoint{5.278781in}{1.787326in}}%
\pgfpathlineto{\pgfqpoint{5.281309in}{1.787375in}}%
\pgfpathlineto{\pgfqpoint{5.282572in}{1.788693in}}%
\pgfpathlineto{\pgfqpoint{5.282994in}{1.788385in}}%
\pgfpathlineto{\pgfqpoint{5.288132in}{1.785878in}}%
\pgfpathlineto{\pgfqpoint{5.291250in}{1.788407in}}%
\pgfpathlineto{\pgfqpoint{5.294114in}{1.794370in}}%
\pgfpathlineto{\pgfqpoint{5.294619in}{1.793780in}}%
\pgfpathlineto{\pgfqpoint{5.294956in}{1.793574in}}%
\pgfpathlineto{\pgfqpoint{5.295462in}{1.794446in}}%
\pgfpathlineto{\pgfqpoint{5.297062in}{1.795506in}}%
\pgfpathlineto{\pgfqpoint{5.297231in}{1.795369in}}%
\pgfpathlineto{\pgfqpoint{5.302538in}{1.790059in}}%
\pgfpathlineto{\pgfqpoint{5.302791in}{1.790502in}}%
\pgfpathlineto{\pgfqpoint{5.303970in}{1.793764in}}%
\pgfpathlineto{\pgfqpoint{5.304644in}{1.793003in}}%
\pgfpathlineto{\pgfqpoint{5.305908in}{1.792574in}}%
\pgfpathlineto{\pgfqpoint{5.306077in}{1.792718in}}%
\pgfpathlineto{\pgfqpoint{5.307340in}{1.793362in}}%
\pgfpathlineto{\pgfqpoint{5.307677in}{1.792944in}}%
\pgfpathlineto{\pgfqpoint{5.311131in}{1.789815in}}%
\pgfpathlineto{\pgfqpoint{5.316944in}{1.784830in}}%
\pgfpathlineto{\pgfqpoint{5.317450in}{1.785704in}}%
\pgfpathlineto{\pgfqpoint{5.318376in}{1.786116in}}%
\pgfpathlineto{\pgfqpoint{5.318797in}{1.785831in}}%
\pgfpathlineto{\pgfqpoint{5.321830in}{1.784774in}}%
\pgfpathlineto{\pgfqpoint{5.321915in}{1.784901in}}%
\pgfpathlineto{\pgfqpoint{5.323010in}{1.789227in}}%
\pgfpathlineto{\pgfqpoint{5.326042in}{1.796465in}}%
\pgfpathlineto{\pgfqpoint{5.326211in}{1.796307in}}%
\pgfpathlineto{\pgfqpoint{5.326632in}{1.795960in}}%
\pgfpathlineto{\pgfqpoint{5.326969in}{1.796780in}}%
\pgfpathlineto{\pgfqpoint{5.328401in}{1.802782in}}%
\pgfpathlineto{\pgfqpoint{5.329160in}{1.801869in}}%
\pgfpathlineto{\pgfqpoint{5.330339in}{1.800297in}}%
\pgfpathlineto{\pgfqpoint{5.330592in}{1.801124in}}%
\pgfpathlineto{\pgfqpoint{5.333119in}{1.810648in}}%
\pgfpathlineto{\pgfqpoint{5.333877in}{1.809558in}}%
\pgfpathlineto{\pgfqpoint{5.335141in}{1.807839in}}%
\pgfpathlineto{\pgfqpoint{5.335478in}{1.808456in}}%
\pgfpathlineto{\pgfqpoint{5.336489in}{1.811270in}}%
\pgfpathlineto{\pgfqpoint{5.337079in}{1.810379in}}%
\pgfpathlineto{\pgfqpoint{5.341628in}{1.802891in}}%
\pgfpathlineto{\pgfqpoint{5.343481in}{1.800535in}}%
\pgfpathlineto{\pgfqpoint{5.347609in}{1.794973in}}%
\pgfpathlineto{\pgfqpoint{5.350389in}{1.793147in}}%
\pgfpathlineto{\pgfqpoint{5.350810in}{1.794217in}}%
\pgfpathlineto{\pgfqpoint{5.351653in}{1.795176in}}%
\pgfpathlineto{\pgfqpoint{5.352158in}{1.794876in}}%
\pgfpathlineto{\pgfqpoint{5.353085in}{1.794654in}}%
\pgfpathlineto{\pgfqpoint{5.353422in}{1.795281in}}%
\pgfpathlineto{\pgfqpoint{5.354012in}{1.795787in}}%
\pgfpathlineto{\pgfqpoint{5.354686in}{1.795338in}}%
\pgfpathlineto{\pgfqpoint{5.356118in}{1.794072in}}%
\pgfpathlineto{\pgfqpoint{5.358308in}{1.792905in}}%
\pgfpathlineto{\pgfqpoint{5.363784in}{1.790570in}}%
\pgfpathlineto{\pgfqpoint{5.363868in}{1.790730in}}%
\pgfpathlineto{\pgfqpoint{5.364795in}{1.791950in}}%
\pgfpathlineto{\pgfqpoint{5.365300in}{1.791350in}}%
\pgfpathlineto{\pgfqpoint{5.368670in}{1.788958in}}%
\pgfpathlineto{\pgfqpoint{5.369681in}{1.788876in}}%
\pgfpathlineto{\pgfqpoint{5.369934in}{1.789332in}}%
\pgfpathlineto{\pgfqpoint{5.370776in}{1.790095in}}%
\pgfpathlineto{\pgfqpoint{5.371282in}{1.789652in}}%
\pgfpathlineto{\pgfqpoint{5.374483in}{1.787859in}}%
\pgfpathlineto{\pgfqpoint{5.375663in}{1.788439in}}%
\pgfpathlineto{\pgfqpoint{5.376926in}{1.788144in}}%
\pgfpathlineto{\pgfqpoint{5.377347in}{1.788618in}}%
\pgfpathlineto{\pgfqpoint{5.377432in}{1.788942in}}%
\pgfpathlineto{\pgfqpoint{5.380296in}{1.797499in}}%
\pgfpathlineto{\pgfqpoint{5.382908in}{1.801934in}}%
\pgfpathlineto{\pgfqpoint{5.383413in}{1.801395in}}%
\pgfpathlineto{\pgfqpoint{5.384424in}{1.801059in}}%
\pgfpathlineto{\pgfqpoint{5.384677in}{1.801512in}}%
\pgfpathlineto{\pgfqpoint{5.386193in}{1.803482in}}%
\pgfpathlineto{\pgfqpoint{5.386530in}{1.803183in}}%
\pgfpathlineto{\pgfqpoint{5.389142in}{1.800810in}}%
\pgfpathlineto{\pgfqpoint{5.389394in}{1.801066in}}%
\pgfpathlineto{\pgfqpoint{5.390490in}{1.801655in}}%
\pgfpathlineto{\pgfqpoint{5.390827in}{1.801239in}}%
\pgfpathlineto{\pgfqpoint{5.392764in}{1.798863in}}%
\pgfpathlineto{\pgfqpoint{5.393270in}{1.799589in}}%
\pgfpathlineto{\pgfqpoint{5.393691in}{1.799736in}}%
\pgfpathlineto{\pgfqpoint{5.394112in}{1.799099in}}%
\pgfpathlineto{\pgfqpoint{5.398746in}{1.793396in}}%
\pgfpathlineto{\pgfqpoint{5.403884in}{1.791095in}}%
\pgfpathlineto{\pgfqpoint{5.404137in}{1.791812in}}%
\pgfpathlineto{\pgfqpoint{5.404980in}{1.793182in}}%
\pgfpathlineto{\pgfqpoint{5.405485in}{1.792731in}}%
\pgfpathlineto{\pgfqpoint{5.406412in}{1.791977in}}%
\pgfpathlineto{\pgfqpoint{5.406749in}{1.792772in}}%
\pgfpathlineto{\pgfqpoint{5.408012in}{1.796082in}}%
\pgfpathlineto{\pgfqpoint{5.408602in}{1.795394in}}%
\pgfpathlineto{\pgfqpoint{5.410203in}{1.794706in}}%
\pgfpathlineto{\pgfqpoint{5.410287in}{1.794775in}}%
\pgfpathlineto{\pgfqpoint{5.412814in}{1.796158in}}%
\pgfpathlineto{\pgfqpoint{5.412899in}{1.796073in}}%
\pgfpathlineto{\pgfqpoint{5.415173in}{1.793665in}}%
\pgfpathlineto{\pgfqpoint{5.415679in}{1.794471in}}%
\pgfpathlineto{\pgfqpoint{5.416268in}{1.795018in}}%
\pgfpathlineto{\pgfqpoint{5.416774in}{1.794407in}}%
\pgfpathlineto{\pgfqpoint{5.417279in}{1.794186in}}%
\pgfpathlineto{\pgfqpoint{5.417869in}{1.794752in}}%
\pgfpathlineto{\pgfqpoint{5.419217in}{1.793991in}}%
\pgfpathlineto{\pgfqpoint{5.421492in}{1.792929in}}%
\pgfpathlineto{\pgfqpoint{5.422671in}{1.792410in}}%
\pgfpathlineto{\pgfqpoint{5.425114in}{1.790852in}}%
\pgfpathlineto{\pgfqpoint{5.425367in}{1.791343in}}%
\pgfpathlineto{\pgfqpoint{5.428315in}{1.798127in}}%
\pgfpathlineto{\pgfqpoint{5.428905in}{1.797597in}}%
\pgfpathlineto{\pgfqpoint{5.429326in}{1.797835in}}%
\pgfpathlineto{\pgfqpoint{5.429495in}{1.798294in}}%
\pgfpathlineto{\pgfqpoint{5.430590in}{1.801068in}}%
\pgfpathlineto{\pgfqpoint{5.431264in}{1.800685in}}%
\pgfpathlineto{\pgfqpoint{5.431938in}{1.802368in}}%
\pgfpathlineto{\pgfqpoint{5.432528in}{1.803254in}}%
\pgfpathlineto{\pgfqpoint{5.433033in}{1.802458in}}%
\pgfpathlineto{\pgfqpoint{5.436571in}{1.798012in}}%
\pgfpathlineto{\pgfqpoint{5.436824in}{1.798216in}}%
\pgfpathlineto{\pgfqpoint{5.438509in}{1.801315in}}%
\pgfpathlineto{\pgfqpoint{5.439351in}{1.802194in}}%
\pgfpathlineto{\pgfqpoint{5.439857in}{1.801678in}}%
\pgfpathlineto{\pgfqpoint{5.440699in}{1.800933in}}%
\pgfpathlineto{\pgfqpoint{5.441036in}{1.801532in}}%
\pgfpathlineto{\pgfqpoint{5.443985in}{1.807150in}}%
\pgfpathlineto{\pgfqpoint{5.446344in}{1.806572in}}%
\pgfpathlineto{\pgfqpoint{5.450135in}{1.802094in}}%
\pgfpathlineto{\pgfqpoint{5.459149in}{1.791788in}}%
\pgfpathlineto{\pgfqpoint{5.460918in}{1.792083in}}%
\pgfpathlineto{\pgfqpoint{5.461929in}{1.794177in}}%
\pgfpathlineto{\pgfqpoint{5.462603in}{1.793650in}}%
\pgfpathlineto{\pgfqpoint{5.463614in}{1.792871in}}%
\pgfpathlineto{\pgfqpoint{5.464035in}{1.793683in}}%
\pgfpathlineto{\pgfqpoint{5.464962in}{1.795336in}}%
\pgfpathlineto{\pgfqpoint{5.465551in}{1.794819in}}%
\pgfpathlineto{\pgfqpoint{5.467489in}{1.793640in}}%
\pgfpathlineto{\pgfqpoint{5.467826in}{1.794153in}}%
\pgfpathlineto{\pgfqpoint{5.469174in}{1.795507in}}%
\pgfpathlineto{\pgfqpoint{5.469595in}{1.795096in}}%
\pgfpathlineto{\pgfqpoint{5.472123in}{1.793441in}}%
\pgfpathlineto{\pgfqpoint{5.475577in}{1.792948in}}%
\pgfpathlineto{\pgfqpoint{5.475661in}{1.793210in}}%
\pgfpathlineto{\pgfqpoint{5.477935in}{1.798509in}}%
\pgfpathlineto{\pgfqpoint{5.478104in}{1.798429in}}%
\pgfpathlineto{\pgfqpoint{5.479789in}{1.798022in}}%
\pgfpathlineto{\pgfqpoint{5.479873in}{1.798130in}}%
\pgfpathlineto{\pgfqpoint{5.481221in}{1.803026in}}%
\pgfpathlineto{\pgfqpoint{5.482653in}{1.801402in}}%
\pgfpathlineto{\pgfqpoint{5.485939in}{1.797362in}}%
\pgfpathlineto{\pgfqpoint{5.488634in}{1.793823in}}%
\pgfpathlineto{\pgfqpoint{5.488971in}{1.794254in}}%
\pgfpathlineto{\pgfqpoint{5.489645in}{1.794901in}}%
\pgfpathlineto{\pgfqpoint{5.490235in}{1.794269in}}%
\pgfpathlineto{\pgfqpoint{5.490993in}{1.793745in}}%
\pgfpathlineto{\pgfqpoint{5.491415in}{1.794404in}}%
\pgfpathlineto{\pgfqpoint{5.494363in}{1.797297in}}%
\pgfpathlineto{\pgfqpoint{5.496722in}{1.797864in}}%
\pgfpathlineto{\pgfqpoint{5.497480in}{1.798918in}}%
\pgfpathlineto{\pgfqpoint{5.498070in}{1.798619in}}%
\pgfpathlineto{\pgfqpoint{5.499755in}{1.797464in}}%
\pgfpathlineto{\pgfqpoint{5.500092in}{1.798632in}}%
\pgfpathlineto{\pgfqpoint{5.501861in}{1.801144in}}%
\pgfpathlineto{\pgfqpoint{5.501945in}{1.801092in}}%
\pgfpathlineto{\pgfqpoint{5.513571in}{1.790427in}}%
\pgfpathlineto{\pgfqpoint{5.517530in}{1.790567in}}%
\pgfpathlineto{\pgfqpoint{5.518710in}{1.789983in}}%
\pgfpathlineto{\pgfqpoint{5.519047in}{1.790576in}}%
\pgfpathlineto{\pgfqpoint{5.519721in}{1.791144in}}%
\pgfpathlineto{\pgfqpoint{5.520310in}{1.790629in}}%
\pgfpathlineto{\pgfqpoint{5.528819in}{1.783045in}}%
\pgfpathlineto{\pgfqpoint{5.528903in}{1.783140in}}%
\pgfpathlineto{\pgfqpoint{5.532357in}{1.786807in}}%
\pgfpathlineto{\pgfqpoint{5.533031in}{1.787650in}}%
\pgfpathlineto{\pgfqpoint{5.534211in}{1.793039in}}%
\pgfpathlineto{\pgfqpoint{5.535053in}{1.792343in}}%
\pgfpathlineto{\pgfqpoint{5.545499in}{1.790955in}}%
\pgfpathlineto{\pgfqpoint{5.546342in}{1.792644in}}%
\pgfpathlineto{\pgfqpoint{5.546932in}{1.791944in}}%
\pgfpathlineto{\pgfqpoint{5.548954in}{1.791384in}}%
\pgfpathlineto{\pgfqpoint{5.550133in}{1.792019in}}%
\pgfpathlineto{\pgfqpoint{5.551144in}{1.793192in}}%
\pgfpathlineto{\pgfqpoint{5.551649in}{1.792622in}}%
\pgfpathlineto{\pgfqpoint{5.554345in}{1.791046in}}%
\pgfpathlineto{\pgfqpoint{5.555777in}{1.791262in}}%
\pgfpathlineto{\pgfqpoint{5.556788in}{1.790663in}}%
\pgfpathlineto{\pgfqpoint{5.560242in}{1.789956in}}%
\pgfpathlineto{\pgfqpoint{5.562011in}{1.792894in}}%
\pgfpathlineto{\pgfqpoint{5.562517in}{1.793012in}}%
\pgfpathlineto{\pgfqpoint{5.563022in}{1.792392in}}%
\pgfpathlineto{\pgfqpoint{5.564876in}{1.791168in}}%
\pgfpathlineto{\pgfqpoint{5.565044in}{1.791354in}}%
\pgfpathlineto{\pgfqpoint{5.569425in}{1.795887in}}%
\pgfpathlineto{\pgfqpoint{5.570183in}{1.797249in}}%
\pgfpathlineto{\pgfqpoint{5.572289in}{1.803721in}}%
\pgfpathlineto{\pgfqpoint{5.572879in}{1.805052in}}%
\pgfpathlineto{\pgfqpoint{5.573553in}{1.804528in}}%
\pgfpathlineto{\pgfqpoint{5.574901in}{1.802679in}}%
\pgfpathlineto{\pgfqpoint{5.575491in}{1.804102in}}%
\pgfpathlineto{\pgfqpoint{5.575996in}{1.804583in}}%
\pgfpathlineto{\pgfqpoint{5.576670in}{1.804202in}}%
\pgfpathlineto{\pgfqpoint{5.577849in}{1.804700in}}%
\pgfpathlineto{\pgfqpoint{5.579450in}{1.806338in}}%
\pgfpathlineto{\pgfqpoint{5.581725in}{1.809288in}}%
\pgfpathlineto{\pgfqpoint{5.582062in}{1.809388in}}%
\pgfpathlineto{\pgfqpoint{5.582483in}{1.808688in}}%
\pgfpathlineto{\pgfqpoint{5.583241in}{1.807917in}}%
\pgfpathlineto{\pgfqpoint{5.583747in}{1.808450in}}%
\pgfpathlineto{\pgfqpoint{5.586527in}{1.808545in}}%
\pgfpathlineto{\pgfqpoint{5.590065in}{1.802780in}}%
\pgfpathlineto{\pgfqpoint{5.590570in}{1.804570in}}%
\pgfpathlineto{\pgfqpoint{5.591329in}{1.806881in}}%
\pgfpathlineto{\pgfqpoint{5.591918in}{1.805828in}}%
\pgfpathlineto{\pgfqpoint{5.593435in}{1.803716in}}%
\pgfpathlineto{\pgfqpoint{5.593940in}{1.804078in}}%
\pgfpathlineto{\pgfqpoint{5.594951in}{1.803295in}}%
\pgfpathlineto{\pgfqpoint{5.597478in}{1.801328in}}%
\pgfpathlineto{\pgfqpoint{5.599332in}{1.801007in}}%
\pgfpathlineto{\pgfqpoint{5.603797in}{1.795133in}}%
\pgfpathlineto{\pgfqpoint{5.607925in}{1.791274in}}%
\pgfpathlineto{\pgfqpoint{5.611294in}{1.790345in}}%
\pgfpathlineto{\pgfqpoint{5.612053in}{1.791269in}}%
\pgfpathlineto{\pgfqpoint{5.612727in}{1.791323in}}%
\pgfpathlineto{\pgfqpoint{5.613148in}{1.790926in}}%
\pgfpathlineto{\pgfqpoint{5.617697in}{1.787963in}}%
\pgfpathlineto{\pgfqpoint{5.620730in}{1.787695in}}%
\pgfpathlineto{\pgfqpoint{5.620814in}{1.787961in}}%
\pgfpathlineto{\pgfqpoint{5.621657in}{1.789929in}}%
\pgfpathlineto{\pgfqpoint{5.622415in}{1.789595in}}%
\pgfpathlineto{\pgfqpoint{5.623426in}{1.790456in}}%
\pgfpathlineto{\pgfqpoint{5.623931in}{1.789981in}}%
\pgfpathlineto{\pgfqpoint{5.628480in}{1.785593in}}%
\pgfpathlineto{\pgfqpoint{5.628902in}{1.786255in}}%
\pgfpathlineto{\pgfqpoint{5.630418in}{1.786447in}}%
\pgfpathlineto{\pgfqpoint{5.631092in}{1.787548in}}%
\pgfpathlineto{\pgfqpoint{5.634462in}{1.794296in}}%
\pgfpathlineto{\pgfqpoint{5.640359in}{1.792344in}}%
\pgfpathlineto{\pgfqpoint{5.642044in}{1.791984in}}%
\pgfpathlineto{\pgfqpoint{5.645750in}{1.789063in}}%
\pgfpathlineto{\pgfqpoint{5.647098in}{1.788634in}}%
\pgfpathlineto{\pgfqpoint{5.650384in}{1.787578in}}%
\pgfpathlineto{\pgfqpoint{5.652490in}{1.788064in}}%
\pgfpathlineto{\pgfqpoint{5.654175in}{1.787286in}}%
\pgfpathlineto{\pgfqpoint{5.654596in}{1.788182in}}%
\pgfpathlineto{\pgfqpoint{5.655270in}{1.788570in}}%
\pgfpathlineto{\pgfqpoint{5.655776in}{1.788174in}}%
\pgfpathlineto{\pgfqpoint{5.657208in}{1.787442in}}%
\pgfpathlineto{\pgfqpoint{5.657460in}{1.787901in}}%
\pgfpathlineto{\pgfqpoint{5.658893in}{1.788870in}}%
\pgfpathlineto{\pgfqpoint{5.659061in}{1.788784in}}%
\pgfpathlineto{\pgfqpoint{5.660241in}{1.789327in}}%
\pgfpathlineto{\pgfqpoint{5.664453in}{1.790853in}}%
\pgfpathlineto{\pgfqpoint{5.666559in}{1.789580in}}%
\pgfpathlineto{\pgfqpoint{5.668833in}{1.787543in}}%
\pgfpathlineto{\pgfqpoint{5.669339in}{1.788351in}}%
\pgfpathlineto{\pgfqpoint{5.674141in}{1.793232in}}%
\pgfpathlineto{\pgfqpoint{5.675489in}{1.794448in}}%
\pgfpathlineto{\pgfqpoint{5.675994in}{1.794566in}}%
\pgfpathlineto{\pgfqpoint{5.676415in}{1.794000in}}%
\pgfpathlineto{\pgfqpoint{5.678185in}{1.793708in}}%
\pgfpathlineto{\pgfqpoint{5.679111in}{1.795662in}}%
\pgfpathlineto{\pgfqpoint{5.679870in}{1.797489in}}%
\pgfpathlineto{\pgfqpoint{5.680459in}{1.796658in}}%
\pgfpathlineto{\pgfqpoint{5.681723in}{1.796698in}}%
\pgfpathlineto{\pgfqpoint{5.683492in}{1.796208in}}%
\pgfpathlineto{\pgfqpoint{5.684756in}{1.795147in}}%
\pgfpathlineto{\pgfqpoint{5.685177in}{1.795572in}}%
\pgfpathlineto{\pgfqpoint{5.685935in}{1.795776in}}%
\pgfpathlineto{\pgfqpoint{5.686356in}{1.795233in}}%
\pgfpathlineto{\pgfqpoint{5.689558in}{1.792195in}}%
\pgfpathlineto{\pgfqpoint{5.689642in}{1.792255in}}%
\pgfpathlineto{\pgfqpoint{5.690906in}{1.793628in}}%
\pgfpathlineto{\pgfqpoint{5.691495in}{1.792955in}}%
\pgfpathlineto{\pgfqpoint{5.693938in}{1.790721in}}%
\pgfpathlineto{\pgfqpoint{5.694275in}{1.791495in}}%
\pgfpathlineto{\pgfqpoint{5.695455in}{1.793242in}}%
\pgfpathlineto{\pgfqpoint{5.695876in}{1.793086in}}%
\pgfpathlineto{\pgfqpoint{5.698235in}{1.793348in}}%
\pgfpathlineto{\pgfqpoint{5.699330in}{1.796373in}}%
\pgfpathlineto{\pgfqpoint{5.700004in}{1.795958in}}%
\pgfpathlineto{\pgfqpoint{5.700678in}{1.797125in}}%
\pgfpathlineto{\pgfqpoint{5.701352in}{1.798121in}}%
\pgfpathlineto{\pgfqpoint{5.701942in}{1.797469in}}%
\pgfpathlineto{\pgfqpoint{5.706575in}{1.792131in}}%
\pgfpathlineto{\pgfqpoint{5.706744in}{1.792434in}}%
\pgfpathlineto{\pgfqpoint{5.708007in}{1.799298in}}%
\pgfpathlineto{\pgfqpoint{5.709187in}{1.797706in}}%
\pgfpathlineto{\pgfqpoint{5.710956in}{1.796271in}}%
\pgfpathlineto{\pgfqpoint{5.711377in}{1.797301in}}%
\pgfpathlineto{\pgfqpoint{5.711967in}{1.797917in}}%
\pgfpathlineto{\pgfqpoint{5.712556in}{1.797349in}}%
\pgfpathlineto{\pgfqpoint{5.717358in}{1.792422in}}%
\pgfpathlineto{\pgfqpoint{5.724688in}{1.791781in}}%
\pgfpathlineto{\pgfqpoint{5.726541in}{1.794011in}}%
\pgfpathlineto{\pgfqpoint{5.729574in}{1.794604in}}%
\pgfpathlineto{\pgfqpoint{5.730500in}{1.795584in}}%
\pgfpathlineto{\pgfqpoint{5.731006in}{1.795024in}}%
\pgfpathlineto{\pgfqpoint{5.733196in}{1.792433in}}%
\pgfpathlineto{\pgfqpoint{5.733533in}{1.793179in}}%
\pgfpathlineto{\pgfqpoint{5.736061in}{1.800789in}}%
\pgfpathlineto{\pgfqpoint{5.736482in}{1.800607in}}%
\pgfpathlineto{\pgfqpoint{5.737746in}{1.800730in}}%
\pgfpathlineto{\pgfqpoint{5.737830in}{1.801002in}}%
\pgfpathlineto{\pgfqpoint{5.740104in}{1.806677in}}%
\pgfpathlineto{\pgfqpoint{5.740273in}{1.806589in}}%
\pgfpathlineto{\pgfqpoint{5.743474in}{1.803281in}}%
\pgfpathlineto{\pgfqpoint{5.743811in}{1.804495in}}%
\pgfpathlineto{\pgfqpoint{5.744822in}{1.808887in}}%
\pgfpathlineto{\pgfqpoint{5.745412in}{1.807858in}}%
\pgfpathlineto{\pgfqpoint{5.745833in}{1.806909in}}%
\pgfpathlineto{\pgfqpoint{5.745833in}{1.806909in}}%
\pgfusepath{stroke}%
\end{pgfscope}%
\begin{pgfscope}%
\pgfpathrectangle{\pgfqpoint{0.691161in}{1.608471in}}{\pgfqpoint{5.054672in}{0.902317in}}%
\pgfusepath{clip}%
\pgfsetbuttcap%
\pgfsetroundjoin%
\pgfsetlinewidth{2.007500pt}%
\definecolor{currentstroke}{rgb}{0.172549,0.627451,0.172549}%
\pgfsetstrokecolor{currentstroke}%
\pgfsetdash{{7.400000pt}{3.200000pt}}{0.000000pt}%
\pgfpathmoveto{\pgfqpoint{0.691161in}{2.053385in}}%
\pgfpathlineto{\pgfqpoint{5.745833in}{2.053385in}}%
\pgfusepath{stroke}%
\end{pgfscope}%
\begin{pgfscope}%
\pgfpathrectangle{\pgfqpoint{0.691161in}{1.608471in}}{\pgfqpoint{5.054672in}{0.902317in}}%
\pgfusepath{clip}%
\pgfsetbuttcap%
\pgfsetroundjoin%
\pgfsetlinewidth{2.007500pt}%
\definecolor{currentstroke}{rgb}{0.839216,0.152941,0.156863}%
\pgfsetstrokecolor{currentstroke}%
\pgfsetdash{{2.000000pt}{3.300000pt}}{0.000000pt}%
\pgfpathmoveto{\pgfqpoint{1.642872in}{1.649486in}}%
\pgfpathlineto{\pgfqpoint{1.642872in}{2.469774in}}%
\pgfusepath{stroke}%
\end{pgfscope}%
\begin{pgfscope}%
\pgfpathrectangle{\pgfqpoint{0.691161in}{1.608471in}}{\pgfqpoint{5.054672in}{0.902317in}}%
\pgfusepath{clip}%
\pgfsetbuttcap%
\pgfsetroundjoin%
\pgfsetlinewidth{2.007500pt}%
\definecolor{currentstroke}{rgb}{1.000000,0.498039,0.054902}%
\pgfsetstrokecolor{currentstroke}%
\pgfsetdash{{2.000000pt}{3.300000pt}}{0.000000pt}%
\pgfpathmoveto{\pgfqpoint{3.459184in}{1.649486in}}%
\pgfpathlineto{\pgfqpoint{3.459184in}{2.469774in}}%
\pgfusepath{stroke}%
\end{pgfscope}%
\begin{pgfscope}%
\pgfsetrectcap%
\pgfsetmiterjoin%
\pgfsetlinewidth{0.803000pt}%
\definecolor{currentstroke}{rgb}{0.737255,0.737255,0.737255}%
\pgfsetstrokecolor{currentstroke}%
\pgfsetdash{}{0pt}%
\pgfpathmoveto{\pgfqpoint{0.691161in}{1.608471in}}%
\pgfpathlineto{\pgfqpoint{0.691161in}{2.510788in}}%
\pgfusepath{stroke}%
\end{pgfscope}%
\begin{pgfscope}%
\pgfsetrectcap%
\pgfsetmiterjoin%
\pgfsetlinewidth{0.803000pt}%
\definecolor{currentstroke}{rgb}{0.737255,0.737255,0.737255}%
\pgfsetstrokecolor{currentstroke}%
\pgfsetdash{}{0pt}%
\pgfpathmoveto{\pgfqpoint{5.745833in}{1.608471in}}%
\pgfpathlineto{\pgfqpoint{5.745833in}{2.510788in}}%
\pgfusepath{stroke}%
\end{pgfscope}%
\begin{pgfscope}%
\pgfsetrectcap%
\pgfsetmiterjoin%
\pgfsetlinewidth{0.803000pt}%
\definecolor{currentstroke}{rgb}{0.737255,0.737255,0.737255}%
\pgfsetstrokecolor{currentstroke}%
\pgfsetdash{}{0pt}%
\pgfpathmoveto{\pgfqpoint{0.691161in}{1.608471in}}%
\pgfpathlineto{\pgfqpoint{5.745833in}{1.608471in}}%
\pgfusepath{stroke}%
\end{pgfscope}%
\begin{pgfscope}%
\pgfsetrectcap%
\pgfsetmiterjoin%
\pgfsetlinewidth{0.803000pt}%
\definecolor{currentstroke}{rgb}{0.737255,0.737255,0.737255}%
\pgfsetstrokecolor{currentstroke}%
\pgfsetdash{}{0pt}%
\pgfpathmoveto{\pgfqpoint{0.691161in}{2.510788in}}%
\pgfpathlineto{\pgfqpoint{5.745833in}{2.510788in}}%
\pgfusepath{stroke}%
\end{pgfscope}%
\begin{pgfscope}%
\pgfsetbuttcap%
\pgfsetmiterjoin%
\definecolor{currentfill}{rgb}{0.933333,0.933333,0.933333}%
\pgfsetfillcolor{currentfill}%
\pgfsetfillopacity{0.800000}%
\pgfsetlinewidth{0.501875pt}%
\definecolor{currentstroke}{rgb}{0.800000,0.800000,0.800000}%
\pgfsetstrokecolor{currentstroke}%
\pgfsetstrokeopacity{0.800000}%
\pgfsetdash{}{0pt}%
\pgfpathmoveto{\pgfqpoint{4.404343in}{1.818844in}}%
\pgfpathlineto{\pgfqpoint{5.648611in}{1.818844in}}%
\pgfpathquadraticcurveto{\pgfqpoint{5.676389in}{1.818844in}}{\pgfqpoint{5.676389in}{1.846622in}}%
\pgfpathlineto{\pgfqpoint{5.676389in}{2.413566in}}%
\pgfpathquadraticcurveto{\pgfqpoint{5.676389in}{2.441344in}}{\pgfqpoint{5.648611in}{2.441344in}}%
\pgfpathlineto{\pgfqpoint{4.404343in}{2.441344in}}%
\pgfpathquadraticcurveto{\pgfqpoint{4.376566in}{2.441344in}}{\pgfqpoint{4.376566in}{2.413566in}}%
\pgfpathlineto{\pgfqpoint{4.376566in}{1.846622in}}%
\pgfpathquadraticcurveto{\pgfqpoint{4.376566in}{1.818844in}}{\pgfqpoint{4.404343in}{1.818844in}}%
\pgfpathlineto{\pgfqpoint{4.404343in}{1.818844in}}%
\pgfpathclose%
\pgfusepath{stroke,fill}%
\end{pgfscope}%
\begin{pgfscope}%
\pgfsetbuttcap%
\pgfsetroundjoin%
\pgfsetlinewidth{2.007500pt}%
\definecolor{currentstroke}{rgb}{0.172549,0.627451,0.172549}%
\pgfsetstrokecolor{currentstroke}%
\pgfsetdash{{7.400000pt}{3.200000pt}}{0.000000pt}%
\pgfpathmoveto{\pgfqpoint{4.432121in}{2.337177in}}%
\pgfpathlineto{\pgfqpoint{4.709899in}{2.337177in}}%
\pgfusepath{stroke}%
\end{pgfscope}%
\begin{pgfscope}%
\definecolor{textcolor}{rgb}{0.000000,0.000000,0.000000}%
\pgfsetstrokecolor{textcolor}%
\pgfsetfillcolor{textcolor}%
\pgftext[x=4.821010in,y=2.288566in,left,base]{\color{textcolor}\rmfamily\fontsize{10.000000}{12.000000}\selectfont Seuil = 5}%
\end{pgfscope}%
\begin{pgfscope}%
\pgfsetbuttcap%
\pgfsetroundjoin%
\pgfsetlinewidth{2.007500pt}%
\definecolor{currentstroke}{rgb}{0.839216,0.152941,0.156863}%
\pgfsetstrokecolor{currentstroke}%
\pgfsetdash{{2.000000pt}{3.300000pt}}{0.000000pt}%
\pgfpathmoveto{\pgfqpoint{4.432121in}{2.143566in}}%
\pgfpathlineto{\pgfqpoint{4.709899in}{2.143566in}}%
\pgfusepath{stroke}%
\end{pgfscope}%
\begin{pgfscope}%
\definecolor{textcolor}{rgb}{0.000000,0.000000,0.000000}%
\pgfsetstrokecolor{textcolor}%
\pgfsetfillcolor{textcolor}%
\pgftext[x=4.821010in,y=2.094955in,left,base]{\color{textcolor}\rmfamily\fontsize{10.000000}{12.000000}\selectfont \(\displaystyle t_1\) = 56.48 s}%
\end{pgfscope}%
\begin{pgfscope}%
\pgfsetbuttcap%
\pgfsetroundjoin%
\pgfsetlinewidth{2.007500pt}%
\definecolor{currentstroke}{rgb}{1.000000,0.498039,0.054902}%
\pgfsetstrokecolor{currentstroke}%
\pgfsetdash{{2.000000pt}{3.300000pt}}{0.000000pt}%
\pgfpathmoveto{\pgfqpoint{4.432121in}{1.949955in}}%
\pgfpathlineto{\pgfqpoint{4.709899in}{1.949955in}}%
\pgfusepath{stroke}%
\end{pgfscope}%
\begin{pgfscope}%
\definecolor{textcolor}{rgb}{0.000000,0.000000,0.000000}%
\pgfsetstrokecolor{textcolor}%
\pgfsetfillcolor{textcolor}%
\pgftext[x=4.821010in,y=1.901344in,left,base]{\color{textcolor}\rmfamily\fontsize{10.000000}{12.000000}\selectfont \(\displaystyle t_2\) = 164.28 s}%
\end{pgfscope}%
\begin{pgfscope}%
\pgfsetbuttcap%
\pgfsetmiterjoin%
\definecolor{currentfill}{rgb}{0.933333,0.933333,0.933333}%
\pgfsetfillcolor{currentfill}%
\pgfsetlinewidth{0.000000pt}%
\definecolor{currentstroke}{rgb}{0.000000,0.000000,0.000000}%
\pgfsetstrokecolor{currentstroke}%
\pgfsetstrokeopacity{0.000000}%
\pgfsetdash{}{0pt}%
\pgfpathmoveto{\pgfqpoint{0.691161in}{0.544166in}}%
\pgfpathlineto{\pgfqpoint{5.745833in}{0.544166in}}%
\pgfpathlineto{\pgfqpoint{5.745833in}{1.446484in}}%
\pgfpathlineto{\pgfqpoint{0.691161in}{1.446484in}}%
\pgfpathlineto{\pgfqpoint{0.691161in}{0.544166in}}%
\pgfpathclose%
\pgfusepath{fill}%
\end{pgfscope}%
\begin{pgfscope}%
\pgfpathrectangle{\pgfqpoint{0.691161in}{0.544166in}}{\pgfqpoint{5.054672in}{0.902317in}}%
\pgfusepath{clip}%
\pgfsetbuttcap%
\pgfsetroundjoin%
\pgfsetlinewidth{0.501875pt}%
\definecolor{currentstroke}{rgb}{0.698039,0.698039,0.698039}%
\pgfsetstrokecolor{currentstroke}%
\pgfsetdash{{1.850000pt}{0.800000pt}}{0.000000pt}%
\pgfpathmoveto{\pgfqpoint{0.691161in}{0.544166in}}%
\pgfpathlineto{\pgfqpoint{0.691161in}{1.446484in}}%
\pgfusepath{stroke}%
\end{pgfscope}%
\begin{pgfscope}%
\pgfsetbuttcap%
\pgfsetroundjoin%
\definecolor{currentfill}{rgb}{0.000000,0.000000,0.000000}%
\pgfsetfillcolor{currentfill}%
\pgfsetlinewidth{0.803000pt}%
\definecolor{currentstroke}{rgb}{0.000000,0.000000,0.000000}%
\pgfsetstrokecolor{currentstroke}%
\pgfsetdash{}{0pt}%
\pgfsys@defobject{currentmarker}{\pgfqpoint{0.000000in}{0.000000in}}{\pgfqpoint{0.000000in}{0.048611in}}{%
\pgfpathmoveto{\pgfqpoint{0.000000in}{0.000000in}}%
\pgfpathlineto{\pgfqpoint{0.000000in}{0.048611in}}%
\pgfusepath{stroke,fill}%
}%
\begin{pgfscope}%
\pgfsys@transformshift{0.691161in}{0.544166in}%
\pgfsys@useobject{currentmarker}{}%
\end{pgfscope}%
\end{pgfscope}%
\begin{pgfscope}%
\definecolor{textcolor}{rgb}{0.000000,0.000000,0.000000}%
\pgfsetstrokecolor{textcolor}%
\pgfsetfillcolor{textcolor}%
\pgftext[x=0.691161in,y=0.495555in,,top]{\color{textcolor}\rmfamily\fontsize{10.000000}{12.000000}\selectfont \(\displaystyle {0}\)}%
\end{pgfscope}%
\begin{pgfscope}%
\pgfpathrectangle{\pgfqpoint{0.691161in}{0.544166in}}{\pgfqpoint{5.054672in}{0.902317in}}%
\pgfusepath{clip}%
\pgfsetbuttcap%
\pgfsetroundjoin%
\pgfsetlinewidth{0.501875pt}%
\definecolor{currentstroke}{rgb}{0.698039,0.698039,0.698039}%
\pgfsetstrokecolor{currentstroke}%
\pgfsetdash{{1.850000pt}{0.800000pt}}{0.000000pt}%
\pgfpathmoveto{\pgfqpoint{1.533607in}{0.544166in}}%
\pgfpathlineto{\pgfqpoint{1.533607in}{1.446484in}}%
\pgfusepath{stroke}%
\end{pgfscope}%
\begin{pgfscope}%
\pgfsetbuttcap%
\pgfsetroundjoin%
\definecolor{currentfill}{rgb}{0.000000,0.000000,0.000000}%
\pgfsetfillcolor{currentfill}%
\pgfsetlinewidth{0.803000pt}%
\definecolor{currentstroke}{rgb}{0.000000,0.000000,0.000000}%
\pgfsetstrokecolor{currentstroke}%
\pgfsetdash{}{0pt}%
\pgfsys@defobject{currentmarker}{\pgfqpoint{0.000000in}{0.000000in}}{\pgfqpoint{0.000000in}{0.048611in}}{%
\pgfpathmoveto{\pgfqpoint{0.000000in}{0.000000in}}%
\pgfpathlineto{\pgfqpoint{0.000000in}{0.048611in}}%
\pgfusepath{stroke,fill}%
}%
\begin{pgfscope}%
\pgfsys@transformshift{1.533607in}{0.544166in}%
\pgfsys@useobject{currentmarker}{}%
\end{pgfscope}%
\end{pgfscope}%
\begin{pgfscope}%
\definecolor{textcolor}{rgb}{0.000000,0.000000,0.000000}%
\pgfsetstrokecolor{textcolor}%
\pgfsetfillcolor{textcolor}%
\pgftext[x=1.533607in,y=0.495555in,,top]{\color{textcolor}\rmfamily\fontsize{10.000000}{12.000000}\selectfont \(\displaystyle {50}\)}%
\end{pgfscope}%
\begin{pgfscope}%
\pgfpathrectangle{\pgfqpoint{0.691161in}{0.544166in}}{\pgfqpoint{5.054672in}{0.902317in}}%
\pgfusepath{clip}%
\pgfsetbuttcap%
\pgfsetroundjoin%
\pgfsetlinewidth{0.501875pt}%
\definecolor{currentstroke}{rgb}{0.698039,0.698039,0.698039}%
\pgfsetstrokecolor{currentstroke}%
\pgfsetdash{{1.850000pt}{0.800000pt}}{0.000000pt}%
\pgfpathmoveto{\pgfqpoint{2.376052in}{0.544166in}}%
\pgfpathlineto{\pgfqpoint{2.376052in}{1.446484in}}%
\pgfusepath{stroke}%
\end{pgfscope}%
\begin{pgfscope}%
\pgfsetbuttcap%
\pgfsetroundjoin%
\definecolor{currentfill}{rgb}{0.000000,0.000000,0.000000}%
\pgfsetfillcolor{currentfill}%
\pgfsetlinewidth{0.803000pt}%
\definecolor{currentstroke}{rgb}{0.000000,0.000000,0.000000}%
\pgfsetstrokecolor{currentstroke}%
\pgfsetdash{}{0pt}%
\pgfsys@defobject{currentmarker}{\pgfqpoint{0.000000in}{0.000000in}}{\pgfqpoint{0.000000in}{0.048611in}}{%
\pgfpathmoveto{\pgfqpoint{0.000000in}{0.000000in}}%
\pgfpathlineto{\pgfqpoint{0.000000in}{0.048611in}}%
\pgfusepath{stroke,fill}%
}%
\begin{pgfscope}%
\pgfsys@transformshift{2.376052in}{0.544166in}%
\pgfsys@useobject{currentmarker}{}%
\end{pgfscope}%
\end{pgfscope}%
\begin{pgfscope}%
\definecolor{textcolor}{rgb}{0.000000,0.000000,0.000000}%
\pgfsetstrokecolor{textcolor}%
\pgfsetfillcolor{textcolor}%
\pgftext[x=2.376052in,y=0.495555in,,top]{\color{textcolor}\rmfamily\fontsize{10.000000}{12.000000}\selectfont \(\displaystyle {100}\)}%
\end{pgfscope}%
\begin{pgfscope}%
\pgfpathrectangle{\pgfqpoint{0.691161in}{0.544166in}}{\pgfqpoint{5.054672in}{0.902317in}}%
\pgfusepath{clip}%
\pgfsetbuttcap%
\pgfsetroundjoin%
\pgfsetlinewidth{0.501875pt}%
\definecolor{currentstroke}{rgb}{0.698039,0.698039,0.698039}%
\pgfsetstrokecolor{currentstroke}%
\pgfsetdash{{1.850000pt}{0.800000pt}}{0.000000pt}%
\pgfpathmoveto{\pgfqpoint{3.218497in}{0.544166in}}%
\pgfpathlineto{\pgfqpoint{3.218497in}{1.446484in}}%
\pgfusepath{stroke}%
\end{pgfscope}%
\begin{pgfscope}%
\pgfsetbuttcap%
\pgfsetroundjoin%
\definecolor{currentfill}{rgb}{0.000000,0.000000,0.000000}%
\pgfsetfillcolor{currentfill}%
\pgfsetlinewidth{0.803000pt}%
\definecolor{currentstroke}{rgb}{0.000000,0.000000,0.000000}%
\pgfsetstrokecolor{currentstroke}%
\pgfsetdash{}{0pt}%
\pgfsys@defobject{currentmarker}{\pgfqpoint{0.000000in}{0.000000in}}{\pgfqpoint{0.000000in}{0.048611in}}{%
\pgfpathmoveto{\pgfqpoint{0.000000in}{0.000000in}}%
\pgfpathlineto{\pgfqpoint{0.000000in}{0.048611in}}%
\pgfusepath{stroke,fill}%
}%
\begin{pgfscope}%
\pgfsys@transformshift{3.218497in}{0.544166in}%
\pgfsys@useobject{currentmarker}{}%
\end{pgfscope}%
\end{pgfscope}%
\begin{pgfscope}%
\definecolor{textcolor}{rgb}{0.000000,0.000000,0.000000}%
\pgfsetstrokecolor{textcolor}%
\pgfsetfillcolor{textcolor}%
\pgftext[x=3.218497in,y=0.495555in,,top]{\color{textcolor}\rmfamily\fontsize{10.000000}{12.000000}\selectfont \(\displaystyle {150}\)}%
\end{pgfscope}%
\begin{pgfscope}%
\pgfpathrectangle{\pgfqpoint{0.691161in}{0.544166in}}{\pgfqpoint{5.054672in}{0.902317in}}%
\pgfusepath{clip}%
\pgfsetbuttcap%
\pgfsetroundjoin%
\pgfsetlinewidth{0.501875pt}%
\definecolor{currentstroke}{rgb}{0.698039,0.698039,0.698039}%
\pgfsetstrokecolor{currentstroke}%
\pgfsetdash{{1.850000pt}{0.800000pt}}{0.000000pt}%
\pgfpathmoveto{\pgfqpoint{4.060942in}{0.544166in}}%
\pgfpathlineto{\pgfqpoint{4.060942in}{1.446484in}}%
\pgfusepath{stroke}%
\end{pgfscope}%
\begin{pgfscope}%
\pgfsetbuttcap%
\pgfsetroundjoin%
\definecolor{currentfill}{rgb}{0.000000,0.000000,0.000000}%
\pgfsetfillcolor{currentfill}%
\pgfsetlinewidth{0.803000pt}%
\definecolor{currentstroke}{rgb}{0.000000,0.000000,0.000000}%
\pgfsetstrokecolor{currentstroke}%
\pgfsetdash{}{0pt}%
\pgfsys@defobject{currentmarker}{\pgfqpoint{0.000000in}{0.000000in}}{\pgfqpoint{0.000000in}{0.048611in}}{%
\pgfpathmoveto{\pgfqpoint{0.000000in}{0.000000in}}%
\pgfpathlineto{\pgfqpoint{0.000000in}{0.048611in}}%
\pgfusepath{stroke,fill}%
}%
\begin{pgfscope}%
\pgfsys@transformshift{4.060942in}{0.544166in}%
\pgfsys@useobject{currentmarker}{}%
\end{pgfscope}%
\end{pgfscope}%
\begin{pgfscope}%
\definecolor{textcolor}{rgb}{0.000000,0.000000,0.000000}%
\pgfsetstrokecolor{textcolor}%
\pgfsetfillcolor{textcolor}%
\pgftext[x=4.060942in,y=0.495555in,,top]{\color{textcolor}\rmfamily\fontsize{10.000000}{12.000000}\selectfont \(\displaystyle {200}\)}%
\end{pgfscope}%
\begin{pgfscope}%
\pgfpathrectangle{\pgfqpoint{0.691161in}{0.544166in}}{\pgfqpoint{5.054672in}{0.902317in}}%
\pgfusepath{clip}%
\pgfsetbuttcap%
\pgfsetroundjoin%
\pgfsetlinewidth{0.501875pt}%
\definecolor{currentstroke}{rgb}{0.698039,0.698039,0.698039}%
\pgfsetstrokecolor{currentstroke}%
\pgfsetdash{{1.850000pt}{0.800000pt}}{0.000000pt}%
\pgfpathmoveto{\pgfqpoint{4.903388in}{0.544166in}}%
\pgfpathlineto{\pgfqpoint{4.903388in}{1.446484in}}%
\pgfusepath{stroke}%
\end{pgfscope}%
\begin{pgfscope}%
\pgfsetbuttcap%
\pgfsetroundjoin%
\definecolor{currentfill}{rgb}{0.000000,0.000000,0.000000}%
\pgfsetfillcolor{currentfill}%
\pgfsetlinewidth{0.803000pt}%
\definecolor{currentstroke}{rgb}{0.000000,0.000000,0.000000}%
\pgfsetstrokecolor{currentstroke}%
\pgfsetdash{}{0pt}%
\pgfsys@defobject{currentmarker}{\pgfqpoint{0.000000in}{0.000000in}}{\pgfqpoint{0.000000in}{0.048611in}}{%
\pgfpathmoveto{\pgfqpoint{0.000000in}{0.000000in}}%
\pgfpathlineto{\pgfqpoint{0.000000in}{0.048611in}}%
\pgfusepath{stroke,fill}%
}%
\begin{pgfscope}%
\pgfsys@transformshift{4.903388in}{0.544166in}%
\pgfsys@useobject{currentmarker}{}%
\end{pgfscope}%
\end{pgfscope}%
\begin{pgfscope}%
\definecolor{textcolor}{rgb}{0.000000,0.000000,0.000000}%
\pgfsetstrokecolor{textcolor}%
\pgfsetfillcolor{textcolor}%
\pgftext[x=4.903388in,y=0.495555in,,top]{\color{textcolor}\rmfamily\fontsize{10.000000}{12.000000}\selectfont \(\displaystyle {250}\)}%
\end{pgfscope}%
\begin{pgfscope}%
\pgfpathrectangle{\pgfqpoint{0.691161in}{0.544166in}}{\pgfqpoint{5.054672in}{0.902317in}}%
\pgfusepath{clip}%
\pgfsetbuttcap%
\pgfsetroundjoin%
\pgfsetlinewidth{0.501875pt}%
\definecolor{currentstroke}{rgb}{0.698039,0.698039,0.698039}%
\pgfsetstrokecolor{currentstroke}%
\pgfsetdash{{1.850000pt}{0.800000pt}}{0.000000pt}%
\pgfpathmoveto{\pgfqpoint{5.745833in}{0.544166in}}%
\pgfpathlineto{\pgfqpoint{5.745833in}{1.446484in}}%
\pgfusepath{stroke}%
\end{pgfscope}%
\begin{pgfscope}%
\pgfsetbuttcap%
\pgfsetroundjoin%
\definecolor{currentfill}{rgb}{0.000000,0.000000,0.000000}%
\pgfsetfillcolor{currentfill}%
\pgfsetlinewidth{0.803000pt}%
\definecolor{currentstroke}{rgb}{0.000000,0.000000,0.000000}%
\pgfsetstrokecolor{currentstroke}%
\pgfsetdash{}{0pt}%
\pgfsys@defobject{currentmarker}{\pgfqpoint{0.000000in}{0.000000in}}{\pgfqpoint{0.000000in}{0.048611in}}{%
\pgfpathmoveto{\pgfqpoint{0.000000in}{0.000000in}}%
\pgfpathlineto{\pgfqpoint{0.000000in}{0.048611in}}%
\pgfusepath{stroke,fill}%
}%
\begin{pgfscope}%
\pgfsys@transformshift{5.745833in}{0.544166in}%
\pgfsys@useobject{currentmarker}{}%
\end{pgfscope}%
\end{pgfscope}%
\begin{pgfscope}%
\definecolor{textcolor}{rgb}{0.000000,0.000000,0.000000}%
\pgfsetstrokecolor{textcolor}%
\pgfsetfillcolor{textcolor}%
\pgftext[x=5.745833in,y=0.495555in,,top]{\color{textcolor}\rmfamily\fontsize{10.000000}{12.000000}\selectfont \(\displaystyle {300}\)}%
\end{pgfscope}%
\begin{pgfscope}%
\definecolor{textcolor}{rgb}{0.000000,0.000000,0.000000}%
\pgfsetstrokecolor{textcolor}%
\pgfsetfillcolor{textcolor}%
\pgftext[x=3.218497in,y=0.316666in,,top]{\color{textcolor}\rmfamily\fontsize{12.000000}{14.400000}\selectfont Temps [s]}%
\end{pgfscope}%
\begin{pgfscope}%
\pgfpathrectangle{\pgfqpoint{0.691161in}{0.544166in}}{\pgfqpoint{5.054672in}{0.902317in}}%
\pgfusepath{clip}%
\pgfsetbuttcap%
\pgfsetroundjoin%
\pgfsetlinewidth{0.501875pt}%
\definecolor{currentstroke}{rgb}{0.698039,0.698039,0.698039}%
\pgfsetstrokecolor{currentstroke}%
\pgfsetdash{{1.850000pt}{0.800000pt}}{0.000000pt}%
\pgfpathmoveto{\pgfqpoint{0.691161in}{0.656888in}}%
\pgfpathlineto{\pgfqpoint{5.745833in}{0.656888in}}%
\pgfusepath{stroke}%
\end{pgfscope}%
\begin{pgfscope}%
\pgfsetbuttcap%
\pgfsetroundjoin%
\definecolor{currentfill}{rgb}{0.000000,0.000000,0.000000}%
\pgfsetfillcolor{currentfill}%
\pgfsetlinewidth{0.803000pt}%
\definecolor{currentstroke}{rgb}{0.000000,0.000000,0.000000}%
\pgfsetstrokecolor{currentstroke}%
\pgfsetdash{}{0pt}%
\pgfsys@defobject{currentmarker}{\pgfqpoint{0.000000in}{0.000000in}}{\pgfqpoint{0.048611in}{0.000000in}}{%
\pgfpathmoveto{\pgfqpoint{0.000000in}{0.000000in}}%
\pgfpathlineto{\pgfqpoint{0.048611in}{0.000000in}}%
\pgfusepath{stroke,fill}%
}%
\begin{pgfscope}%
\pgfsys@transformshift{0.691161in}{0.656888in}%
\pgfsys@useobject{currentmarker}{}%
\end{pgfscope}%
\end{pgfscope}%
\begin{pgfscope}%
\definecolor{textcolor}{rgb}{0.000000,0.000000,0.000000}%
\pgfsetstrokecolor{textcolor}%
\pgfsetfillcolor{textcolor}%
\pgftext[x=0.573105in, y=0.608694in, left, base]{\color{textcolor}\rmfamily\fontsize{10.000000}{12.000000}\selectfont \(\displaystyle {0}\)}%
\end{pgfscope}%
\begin{pgfscope}%
\pgfpathrectangle{\pgfqpoint{0.691161in}{0.544166in}}{\pgfqpoint{5.054672in}{0.902317in}}%
\pgfusepath{clip}%
\pgfsetbuttcap%
\pgfsetroundjoin%
\pgfsetlinewidth{0.501875pt}%
\definecolor{currentstroke}{rgb}{0.698039,0.698039,0.698039}%
\pgfsetstrokecolor{currentstroke}%
\pgfsetdash{{1.850000pt}{0.800000pt}}{0.000000pt}%
\pgfpathmoveto{\pgfqpoint{0.691161in}{1.031179in}}%
\pgfpathlineto{\pgfqpoint{5.745833in}{1.031179in}}%
\pgfusepath{stroke}%
\end{pgfscope}%
\begin{pgfscope}%
\pgfsetbuttcap%
\pgfsetroundjoin%
\definecolor{currentfill}{rgb}{0.000000,0.000000,0.000000}%
\pgfsetfillcolor{currentfill}%
\pgfsetlinewidth{0.803000pt}%
\definecolor{currentstroke}{rgb}{0.000000,0.000000,0.000000}%
\pgfsetstrokecolor{currentstroke}%
\pgfsetdash{}{0pt}%
\pgfsys@defobject{currentmarker}{\pgfqpoint{0.000000in}{0.000000in}}{\pgfqpoint{0.048611in}{0.000000in}}{%
\pgfpathmoveto{\pgfqpoint{0.000000in}{0.000000in}}%
\pgfpathlineto{\pgfqpoint{0.048611in}{0.000000in}}%
\pgfusepath{stroke,fill}%
}%
\begin{pgfscope}%
\pgfsys@transformshift{0.691161in}{1.031179in}%
\pgfsys@useobject{currentmarker}{}%
\end{pgfscope}%
\end{pgfscope}%
\begin{pgfscope}%
\definecolor{textcolor}{rgb}{0.000000,0.000000,0.000000}%
\pgfsetstrokecolor{textcolor}%
\pgfsetfillcolor{textcolor}%
\pgftext[x=0.434216in, y=0.982984in, left, base]{\color{textcolor}\rmfamily\fontsize{10.000000}{12.000000}\selectfont \(\displaystyle {100}\)}%
\end{pgfscope}%
\begin{pgfscope}%
\pgfpathrectangle{\pgfqpoint{0.691161in}{0.544166in}}{\pgfqpoint{5.054672in}{0.902317in}}%
\pgfusepath{clip}%
\pgfsetbuttcap%
\pgfsetroundjoin%
\pgfsetlinewidth{0.501875pt}%
\definecolor{currentstroke}{rgb}{0.698039,0.698039,0.698039}%
\pgfsetstrokecolor{currentstroke}%
\pgfsetdash{{1.850000pt}{0.800000pt}}{0.000000pt}%
\pgfpathmoveto{\pgfqpoint{0.691161in}{1.405469in}}%
\pgfpathlineto{\pgfqpoint{5.745833in}{1.405469in}}%
\pgfusepath{stroke}%
\end{pgfscope}%
\begin{pgfscope}%
\pgfsetbuttcap%
\pgfsetroundjoin%
\definecolor{currentfill}{rgb}{0.000000,0.000000,0.000000}%
\pgfsetfillcolor{currentfill}%
\pgfsetlinewidth{0.803000pt}%
\definecolor{currentstroke}{rgb}{0.000000,0.000000,0.000000}%
\pgfsetstrokecolor{currentstroke}%
\pgfsetdash{}{0pt}%
\pgfsys@defobject{currentmarker}{\pgfqpoint{0.000000in}{0.000000in}}{\pgfqpoint{0.048611in}{0.000000in}}{%
\pgfpathmoveto{\pgfqpoint{0.000000in}{0.000000in}}%
\pgfpathlineto{\pgfqpoint{0.048611in}{0.000000in}}%
\pgfusepath{stroke,fill}%
}%
\begin{pgfscope}%
\pgfsys@transformshift{0.691161in}{1.405469in}%
\pgfsys@useobject{currentmarker}{}%
\end{pgfscope}%
\end{pgfscope}%
\begin{pgfscope}%
\definecolor{textcolor}{rgb}{0.000000,0.000000,0.000000}%
\pgfsetstrokecolor{textcolor}%
\pgfsetfillcolor{textcolor}%
\pgftext[x=0.434216in, y=1.357275in, left, base]{\color{textcolor}\rmfamily\fontsize{10.000000}{12.000000}\selectfont \(\displaystyle {200}\)}%
\end{pgfscope}%
\begin{pgfscope}%
\definecolor{textcolor}{rgb}{0.000000,0.000000,0.000000}%
\pgfsetstrokecolor{textcolor}%
\pgfsetfillcolor{textcolor}%
\pgftext[x=0.378661in,y=0.995325in,,bottom,rotate=90.000000]{\color{textcolor}\rmfamily\fontsize{12.000000}{14.400000}\selectfont Baer}%
\end{pgfscope}%
\begin{pgfscope}%
\pgfpathrectangle{\pgfqpoint{0.691161in}{0.544166in}}{\pgfqpoint{5.054672in}{0.902317in}}%
\pgfusepath{clip}%
\pgfsetrectcap%
\pgfsetroundjoin%
\pgfsetlinewidth{1.505625pt}%
\definecolor{currentstroke}{rgb}{0.121569,0.466667,0.705882}%
\pgfsetstrokecolor{currentstroke}%
\pgfsetdash{}{0pt}%
\pgfpathmoveto{\pgfqpoint{0.691161in}{0.656888in}}%
\pgfpathlineto{\pgfqpoint{0.707926in}{0.656888in}}%
\pgfpathlineto{\pgfqpoint{0.708094in}{0.653592in}}%
\pgfpathlineto{\pgfqpoint{0.708347in}{0.673966in}}%
\pgfpathlineto{\pgfqpoint{0.708600in}{0.699912in}}%
\pgfpathlineto{\pgfqpoint{0.709274in}{0.652729in}}%
\pgfpathlineto{\pgfqpoint{0.709611in}{0.666703in}}%
\pgfpathlineto{\pgfqpoint{0.710116in}{0.651964in}}%
\pgfpathlineto{\pgfqpoint{0.710537in}{0.655425in}}%
\pgfpathlineto{\pgfqpoint{0.710874in}{0.651899in}}%
\pgfpathlineto{\pgfqpoint{0.711043in}{0.654831in}}%
\pgfpathlineto{\pgfqpoint{0.711380in}{0.710591in}}%
\pgfpathlineto{\pgfqpoint{0.711801in}{0.651696in}}%
\pgfpathlineto{\pgfqpoint{0.712138in}{0.653550in}}%
\pgfpathlineto{\pgfqpoint{0.712475in}{0.651840in}}%
\pgfpathlineto{\pgfqpoint{0.713318in}{0.652709in}}%
\pgfpathlineto{\pgfqpoint{0.713992in}{0.652435in}}%
\pgfpathlineto{\pgfqpoint{0.714160in}{0.659192in}}%
\pgfpathlineto{\pgfqpoint{0.715171in}{0.754843in}}%
\pgfpathlineto{\pgfqpoint{0.714834in}{0.652458in}}%
\pgfpathlineto{\pgfqpoint{0.715339in}{0.709134in}}%
\pgfpathlineto{\pgfqpoint{0.715592in}{0.652586in}}%
\pgfpathlineto{\pgfqpoint{0.716013in}{0.741440in}}%
\pgfpathlineto{\pgfqpoint{0.716519in}{0.663973in}}%
\pgfpathlineto{\pgfqpoint{0.716772in}{0.763385in}}%
\pgfpathlineto{\pgfqpoint{0.717361in}{0.654633in}}%
\pgfpathlineto{\pgfqpoint{0.717614in}{0.661053in}}%
\pgfpathlineto{\pgfqpoint{0.717867in}{0.674153in}}%
\pgfpathlineto{\pgfqpoint{0.718625in}{0.654502in}}%
\pgfpathlineto{\pgfqpoint{0.719046in}{0.652567in}}%
\pgfpathlineto{\pgfqpoint{0.719720in}{0.654127in}}%
\pgfpathlineto{\pgfqpoint{0.719973in}{0.655676in}}%
\pgfpathlineto{\pgfqpoint{0.720394in}{0.652523in}}%
\pgfpathlineto{\pgfqpoint{0.720647in}{0.652564in}}%
\pgfpathlineto{\pgfqpoint{0.722332in}{0.653329in}}%
\pgfpathlineto{\pgfqpoint{0.722584in}{0.670302in}}%
\pgfpathlineto{\pgfqpoint{0.723848in}{0.799367in}}%
\pgfpathlineto{\pgfqpoint{0.723343in}{0.655727in}}%
\pgfpathlineto{\pgfqpoint{0.724017in}{0.741103in}}%
\pgfpathlineto{\pgfqpoint{0.724438in}{0.654300in}}%
\pgfpathlineto{\pgfqpoint{0.725280in}{0.666089in}}%
\pgfpathlineto{\pgfqpoint{0.725533in}{0.654257in}}%
\pgfpathlineto{\pgfqpoint{0.725954in}{0.707078in}}%
\pgfpathlineto{\pgfqpoint{0.727555in}{0.653706in}}%
\pgfpathlineto{\pgfqpoint{0.728145in}{0.656606in}}%
\pgfpathlineto{\pgfqpoint{0.728987in}{0.656870in}}%
\pgfpathlineto{\pgfqpoint{0.729408in}{0.653629in}}%
\pgfpathlineto{\pgfqpoint{0.730588in}{0.654508in}}%
\pgfpathlineto{\pgfqpoint{0.730672in}{0.654742in}}%
\pgfpathlineto{\pgfqpoint{0.731009in}{0.653613in}}%
\pgfpathlineto{\pgfqpoint{0.731514in}{0.653832in}}%
\pgfpathlineto{\pgfqpoint{0.733115in}{0.654482in}}%
\pgfpathlineto{\pgfqpoint{0.733284in}{0.655000in}}%
\pgfpathlineto{\pgfqpoint{0.734379in}{0.671830in}}%
\pgfpathlineto{\pgfqpoint{0.733873in}{0.654657in}}%
\pgfpathlineto{\pgfqpoint{0.734631in}{0.661381in}}%
\pgfpathlineto{\pgfqpoint{0.735390in}{0.672308in}}%
\pgfpathlineto{\pgfqpoint{0.735811in}{0.654681in}}%
\pgfpathlineto{\pgfqpoint{0.735895in}{0.654623in}}%
\pgfpathlineto{\pgfqpoint{0.735979in}{0.655355in}}%
\pgfpathlineto{\pgfqpoint{0.736232in}{0.662742in}}%
\pgfpathlineto{\pgfqpoint{0.736653in}{0.654321in}}%
\pgfpathlineto{\pgfqpoint{0.737075in}{0.657143in}}%
\pgfpathlineto{\pgfqpoint{0.737327in}{0.654306in}}%
\pgfpathlineto{\pgfqpoint{0.737833in}{0.658002in}}%
\pgfpathlineto{\pgfqpoint{0.738170in}{0.656540in}}%
\pgfpathlineto{\pgfqpoint{0.738591in}{0.654085in}}%
\pgfpathlineto{\pgfqpoint{0.738928in}{0.658260in}}%
\pgfpathlineto{\pgfqpoint{0.739265in}{0.678160in}}%
\pgfpathlineto{\pgfqpoint{0.739770in}{0.653831in}}%
\pgfpathlineto{\pgfqpoint{0.740276in}{0.671142in}}%
\pgfpathlineto{\pgfqpoint{0.740360in}{0.670974in}}%
\pgfpathlineto{\pgfqpoint{0.740950in}{0.653354in}}%
\pgfpathlineto{\pgfqpoint{0.741287in}{0.676523in}}%
\pgfpathlineto{\pgfqpoint{0.741455in}{0.695896in}}%
\pgfpathlineto{\pgfqpoint{0.742045in}{0.653476in}}%
\pgfpathlineto{\pgfqpoint{0.742466in}{0.686682in}}%
\pgfpathlineto{\pgfqpoint{0.742550in}{0.690577in}}%
\pgfpathlineto{\pgfqpoint{0.742887in}{0.659856in}}%
\pgfpathlineto{\pgfqpoint{0.743983in}{0.652495in}}%
\pgfpathlineto{\pgfqpoint{0.743477in}{0.675222in}}%
\pgfpathlineto{\pgfqpoint{0.744067in}{0.652681in}}%
\pgfpathlineto{\pgfqpoint{0.744320in}{0.653888in}}%
\pgfpathlineto{\pgfqpoint{0.744657in}{0.652495in}}%
\pgfpathlineto{\pgfqpoint{0.745162in}{0.652866in}}%
\pgfpathlineto{\pgfqpoint{0.745499in}{0.653261in}}%
\pgfpathlineto{\pgfqpoint{0.746678in}{0.661553in}}%
\pgfpathlineto{\pgfqpoint{0.746173in}{0.653090in}}%
\pgfpathlineto{\pgfqpoint{0.746931in}{0.656765in}}%
\pgfpathlineto{\pgfqpoint{0.748195in}{0.652449in}}%
\pgfpathlineto{\pgfqpoint{0.749964in}{0.654091in}}%
\pgfpathlineto{\pgfqpoint{0.750132in}{0.655779in}}%
\pgfpathlineto{\pgfqpoint{0.750554in}{0.652402in}}%
\pgfpathlineto{\pgfqpoint{0.750975in}{0.652953in}}%
\pgfpathlineto{\pgfqpoint{0.752407in}{0.653392in}}%
\pgfpathlineto{\pgfqpoint{0.752828in}{0.654548in}}%
\pgfpathlineto{\pgfqpoint{0.754092in}{0.676608in}}%
\pgfpathlineto{\pgfqpoint{0.753671in}{0.653980in}}%
\pgfpathlineto{\pgfqpoint{0.754260in}{0.665504in}}%
\pgfpathlineto{\pgfqpoint{0.754597in}{0.653446in}}%
\pgfpathlineto{\pgfqpoint{0.755440in}{0.653870in}}%
\pgfpathlineto{\pgfqpoint{0.757546in}{0.655331in}}%
\pgfpathlineto{\pgfqpoint{0.757630in}{0.655840in}}%
\pgfpathlineto{\pgfqpoint{0.758051in}{0.654708in}}%
\pgfpathlineto{\pgfqpoint{0.758557in}{0.654778in}}%
\pgfpathlineto{\pgfqpoint{0.759062in}{0.654746in}}%
\pgfpathlineto{\pgfqpoint{0.759315in}{0.655348in}}%
\pgfpathlineto{\pgfqpoint{0.759989in}{0.655584in}}%
\pgfpathlineto{\pgfqpoint{0.759652in}{0.654449in}}%
\pgfpathlineto{\pgfqpoint{0.760242in}{0.654800in}}%
\pgfpathlineto{\pgfqpoint{0.760410in}{0.654413in}}%
\pgfpathlineto{\pgfqpoint{0.760494in}{0.654848in}}%
\pgfpathlineto{\pgfqpoint{0.760831in}{0.702337in}}%
\pgfpathlineto{\pgfqpoint{0.761337in}{0.691910in}}%
\pgfpathlineto{\pgfqpoint{0.761590in}{0.950313in}}%
\pgfpathlineto{\pgfqpoint{0.762095in}{0.656246in}}%
\pgfpathlineto{\pgfqpoint{0.762432in}{0.694486in}}%
\pgfpathlineto{\pgfqpoint{0.762516in}{0.703976in}}%
\pgfpathlineto{\pgfqpoint{0.763106in}{0.655235in}}%
\pgfpathlineto{\pgfqpoint{0.763190in}{0.655440in}}%
\pgfpathlineto{\pgfqpoint{0.764622in}{0.685292in}}%
\pgfpathlineto{\pgfqpoint{0.764875in}{0.664796in}}%
\pgfpathlineto{\pgfqpoint{0.765970in}{0.654794in}}%
\pgfpathlineto{\pgfqpoint{0.766055in}{0.654904in}}%
\pgfpathlineto{\pgfqpoint{0.766392in}{0.661112in}}%
\pgfpathlineto{\pgfqpoint{0.766897in}{0.654760in}}%
\pgfpathlineto{\pgfqpoint{0.767150in}{0.655096in}}%
\pgfpathlineto{\pgfqpoint{0.767740in}{0.657930in}}%
\pgfpathlineto{\pgfqpoint{0.768161in}{0.655018in}}%
\pgfpathlineto{\pgfqpoint{0.768413in}{0.654704in}}%
\pgfpathlineto{\pgfqpoint{0.768582in}{0.656295in}}%
\pgfpathlineto{\pgfqpoint{0.768835in}{0.664471in}}%
\pgfpathlineto{\pgfqpoint{0.769424in}{0.654334in}}%
\pgfpathlineto{\pgfqpoint{0.769593in}{0.654402in}}%
\pgfpathlineto{\pgfqpoint{0.771109in}{0.655417in}}%
\pgfpathlineto{\pgfqpoint{0.771446in}{0.654964in}}%
\pgfpathlineto{\pgfqpoint{0.771699in}{0.655071in}}%
\pgfpathlineto{\pgfqpoint{0.771783in}{0.655668in}}%
\pgfpathlineto{\pgfqpoint{0.772626in}{0.654622in}}%
\pgfpathlineto{\pgfqpoint{0.773131in}{0.662909in}}%
\pgfpathlineto{\pgfqpoint{0.774395in}{0.654370in}}%
\pgfpathlineto{\pgfqpoint{0.774648in}{0.654429in}}%
\pgfpathlineto{\pgfqpoint{0.777933in}{0.655430in}}%
\pgfpathlineto{\pgfqpoint{0.778186in}{0.659891in}}%
\pgfpathlineto{\pgfqpoint{0.778607in}{0.654219in}}%
\pgfpathlineto{\pgfqpoint{0.779028in}{0.655241in}}%
\pgfpathlineto{\pgfqpoint{0.780208in}{0.653698in}}%
\pgfpathlineto{\pgfqpoint{0.780376in}{0.654209in}}%
\pgfpathlineto{\pgfqpoint{0.780629in}{0.658818in}}%
\pgfpathlineto{\pgfqpoint{0.781050in}{0.653732in}}%
\pgfpathlineto{\pgfqpoint{0.781556in}{0.655321in}}%
\pgfpathlineto{\pgfqpoint{0.781808in}{0.653077in}}%
\pgfpathlineto{\pgfqpoint{0.782145in}{0.661002in}}%
\pgfpathlineto{\pgfqpoint{0.783241in}{0.682407in}}%
\pgfpathlineto{\pgfqpoint{0.782819in}{0.653445in}}%
\pgfpathlineto{\pgfqpoint{0.783325in}{0.678439in}}%
\pgfpathlineto{\pgfqpoint{0.783830in}{0.653181in}}%
\pgfpathlineto{\pgfqpoint{0.784504in}{0.670287in}}%
\pgfpathlineto{\pgfqpoint{0.785852in}{0.652786in}}%
\pgfpathlineto{\pgfqpoint{0.786273in}{0.653786in}}%
\pgfpathlineto{\pgfqpoint{0.786610in}{0.653016in}}%
\pgfpathlineto{\pgfqpoint{0.787032in}{0.655047in}}%
\pgfpathlineto{\pgfqpoint{0.787369in}{0.653078in}}%
\pgfpathlineto{\pgfqpoint{0.787621in}{0.670983in}}%
\pgfpathlineto{\pgfqpoint{0.788548in}{0.780912in}}%
\pgfpathlineto{\pgfqpoint{0.788211in}{0.654529in}}%
\pgfpathlineto{\pgfqpoint{0.788716in}{0.739449in}}%
\pgfpathlineto{\pgfqpoint{0.789138in}{0.654377in}}%
\pgfpathlineto{\pgfqpoint{0.789980in}{0.655335in}}%
\pgfpathlineto{\pgfqpoint{0.790233in}{0.654105in}}%
\pgfpathlineto{\pgfqpoint{0.790401in}{0.655078in}}%
\pgfpathlineto{\pgfqpoint{0.791581in}{0.846346in}}%
\pgfpathlineto{\pgfqpoint{0.791665in}{0.875620in}}%
\pgfpathlineto{\pgfqpoint{0.792086in}{0.674914in}}%
\pgfpathlineto{\pgfqpoint{0.792255in}{0.656956in}}%
\pgfpathlineto{\pgfqpoint{0.792760in}{0.742899in}}%
\pgfpathlineto{\pgfqpoint{0.792929in}{0.756767in}}%
\pgfpathlineto{\pgfqpoint{0.793266in}{0.663861in}}%
\pgfpathlineto{\pgfqpoint{0.793434in}{0.655899in}}%
\pgfpathlineto{\pgfqpoint{0.793687in}{0.718169in}}%
\pgfpathlineto{\pgfqpoint{0.793855in}{0.750419in}}%
\pgfpathlineto{\pgfqpoint{0.794361in}{0.655086in}}%
\pgfpathlineto{\pgfqpoint{0.794614in}{0.669878in}}%
\pgfpathlineto{\pgfqpoint{0.794698in}{0.675548in}}%
\pgfpathlineto{\pgfqpoint{0.795287in}{0.654706in}}%
\pgfpathlineto{\pgfqpoint{0.795709in}{0.668967in}}%
\pgfpathlineto{\pgfqpoint{0.797057in}{0.654176in}}%
\pgfpathlineto{\pgfqpoint{0.796551in}{0.669495in}}%
\pgfpathlineto{\pgfqpoint{0.797225in}{0.655889in}}%
\pgfpathlineto{\pgfqpoint{0.797478in}{0.665347in}}%
\pgfpathlineto{\pgfqpoint{0.798068in}{0.653853in}}%
\pgfpathlineto{\pgfqpoint{0.798320in}{0.656111in}}%
\pgfpathlineto{\pgfqpoint{0.799584in}{0.669449in}}%
\pgfpathlineto{\pgfqpoint{0.799078in}{0.653532in}}%
\pgfpathlineto{\pgfqpoint{0.799668in}{0.667877in}}%
\pgfpathlineto{\pgfqpoint{0.800932in}{0.652977in}}%
\pgfpathlineto{\pgfqpoint{0.801185in}{0.653892in}}%
\pgfpathlineto{\pgfqpoint{0.802364in}{0.660000in}}%
\pgfpathlineto{\pgfqpoint{0.801943in}{0.652947in}}%
\pgfpathlineto{\pgfqpoint{0.802448in}{0.658271in}}%
\pgfpathlineto{\pgfqpoint{0.803291in}{0.661104in}}%
\pgfpathlineto{\pgfqpoint{0.803712in}{0.652232in}}%
\pgfpathlineto{\pgfqpoint{0.803880in}{0.656944in}}%
\pgfpathlineto{\pgfqpoint{0.804049in}{0.671976in}}%
\pgfpathlineto{\pgfqpoint{0.804470in}{0.651839in}}%
\pgfpathlineto{\pgfqpoint{0.804976in}{0.667192in}}%
\pgfpathlineto{\pgfqpoint{0.805397in}{0.651699in}}%
\pgfpathlineto{\pgfqpoint{0.806324in}{0.652420in}}%
\pgfpathlineto{\pgfqpoint{0.806660in}{0.651887in}}%
\pgfpathlineto{\pgfqpoint{0.806997in}{0.653976in}}%
\pgfpathlineto{\pgfqpoint{0.808345in}{0.691784in}}%
\pgfpathlineto{\pgfqpoint{0.807671in}{0.651845in}}%
\pgfpathlineto{\pgfqpoint{0.808514in}{0.668943in}}%
\pgfpathlineto{\pgfqpoint{0.808682in}{0.652239in}}%
\pgfpathlineto{\pgfqpoint{0.808935in}{0.709439in}}%
\pgfpathlineto{\pgfqpoint{0.809104in}{0.837636in}}%
\pgfpathlineto{\pgfqpoint{0.809609in}{0.653886in}}%
\pgfpathlineto{\pgfqpoint{0.810030in}{0.732637in}}%
\pgfpathlineto{\pgfqpoint{0.812473in}{0.653255in}}%
\pgfpathlineto{\pgfqpoint{0.812895in}{0.655839in}}%
\pgfpathlineto{\pgfqpoint{0.813737in}{0.653634in}}%
\pgfpathlineto{\pgfqpoint{0.814832in}{0.654415in}}%
\pgfpathlineto{\pgfqpoint{0.815927in}{0.660642in}}%
\pgfpathlineto{\pgfqpoint{0.815506in}{0.654002in}}%
\pgfpathlineto{\pgfqpoint{0.816012in}{0.658702in}}%
\pgfpathlineto{\pgfqpoint{0.816264in}{0.653921in}}%
\pgfpathlineto{\pgfqpoint{0.817191in}{0.655632in}}%
\pgfpathlineto{\pgfqpoint{0.817528in}{0.654309in}}%
\pgfpathlineto{\pgfqpoint{0.817781in}{0.657387in}}%
\pgfpathlineto{\pgfqpoint{0.818118in}{0.683869in}}%
\pgfpathlineto{\pgfqpoint{0.818623in}{0.654426in}}%
\pgfpathlineto{\pgfqpoint{0.819044in}{0.671857in}}%
\pgfpathlineto{\pgfqpoint{0.819550in}{0.653765in}}%
\pgfpathlineto{\pgfqpoint{0.820392in}{0.653931in}}%
\pgfpathlineto{\pgfqpoint{0.821909in}{0.655176in}}%
\pgfpathlineto{\pgfqpoint{0.822920in}{0.669108in}}%
\pgfpathlineto{\pgfqpoint{0.822498in}{0.654478in}}%
\pgfpathlineto{\pgfqpoint{0.823172in}{0.656528in}}%
\pgfpathlineto{\pgfqpoint{0.824183in}{0.654180in}}%
\pgfpathlineto{\pgfqpoint{0.823762in}{0.657266in}}%
\pgfpathlineto{\pgfqpoint{0.824352in}{0.654210in}}%
\pgfpathlineto{\pgfqpoint{0.826289in}{0.655047in}}%
\pgfpathlineto{\pgfqpoint{0.826626in}{0.671076in}}%
\pgfpathlineto{\pgfqpoint{0.827132in}{0.654248in}}%
\pgfpathlineto{\pgfqpoint{0.827722in}{0.667283in}}%
\pgfpathlineto{\pgfqpoint{0.828817in}{0.653859in}}%
\pgfpathlineto{\pgfqpoint{0.828985in}{0.654015in}}%
\pgfpathlineto{\pgfqpoint{0.830249in}{0.659024in}}%
\pgfpathlineto{\pgfqpoint{0.830417in}{0.656542in}}%
\pgfpathlineto{\pgfqpoint{0.830670in}{0.653861in}}%
\pgfpathlineto{\pgfqpoint{0.831091in}{0.664224in}}%
\pgfpathlineto{\pgfqpoint{0.831176in}{0.664094in}}%
\pgfpathlineto{\pgfqpoint{0.831597in}{0.653550in}}%
\pgfpathlineto{\pgfqpoint{0.832018in}{0.669269in}}%
\pgfpathlineto{\pgfqpoint{0.833282in}{0.696654in}}%
\pgfpathlineto{\pgfqpoint{0.832692in}{0.653726in}}%
\pgfpathlineto{\pgfqpoint{0.833366in}{0.695274in}}%
\pgfpathlineto{\pgfqpoint{0.833871in}{0.653688in}}%
\pgfpathlineto{\pgfqpoint{0.834293in}{0.698649in}}%
\pgfpathlineto{\pgfqpoint{0.835472in}{0.871610in}}%
\pgfpathlineto{\pgfqpoint{0.834967in}{0.654626in}}%
\pgfpathlineto{\pgfqpoint{0.835641in}{0.761279in}}%
\pgfpathlineto{\pgfqpoint{0.835978in}{0.655890in}}%
\pgfpathlineto{\pgfqpoint{0.836904in}{0.658491in}}%
\pgfpathlineto{\pgfqpoint{0.837073in}{0.654835in}}%
\pgfpathlineto{\pgfqpoint{0.837494in}{0.674761in}}%
\pgfpathlineto{\pgfqpoint{0.837578in}{0.680744in}}%
\pgfpathlineto{\pgfqpoint{0.837999in}{0.654092in}}%
\pgfpathlineto{\pgfqpoint{0.838336in}{0.657191in}}%
\pgfpathlineto{\pgfqpoint{0.838421in}{0.658138in}}%
\pgfpathlineto{\pgfqpoint{0.838842in}{0.653853in}}%
\pgfpathlineto{\pgfqpoint{0.839263in}{0.655700in}}%
\pgfpathlineto{\pgfqpoint{0.840611in}{0.654131in}}%
\pgfpathlineto{\pgfqpoint{0.840695in}{0.654420in}}%
\pgfpathlineto{\pgfqpoint{0.842043in}{0.669989in}}%
\pgfpathlineto{\pgfqpoint{0.841538in}{0.654083in}}%
\pgfpathlineto{\pgfqpoint{0.842212in}{0.663377in}}%
\pgfpathlineto{\pgfqpoint{0.842970in}{0.674438in}}%
\pgfpathlineto{\pgfqpoint{0.843475in}{0.653563in}}%
\pgfpathlineto{\pgfqpoint{0.843560in}{0.653449in}}%
\pgfpathlineto{\pgfqpoint{0.843644in}{0.653825in}}%
\pgfpathlineto{\pgfqpoint{0.844065in}{0.671613in}}%
\pgfpathlineto{\pgfqpoint{0.844571in}{0.653316in}}%
\pgfpathlineto{\pgfqpoint{0.844992in}{0.660676in}}%
\pgfpathlineto{\pgfqpoint{0.845497in}{0.653109in}}%
\pgfpathlineto{\pgfqpoint{0.845750in}{0.659205in}}%
\pgfpathlineto{\pgfqpoint{0.846929in}{0.688686in}}%
\pgfpathlineto{\pgfqpoint{0.846340in}{0.652940in}}%
\pgfpathlineto{\pgfqpoint{0.847014in}{0.682988in}}%
\pgfpathlineto{\pgfqpoint{0.847435in}{0.652733in}}%
\pgfpathlineto{\pgfqpoint{0.847940in}{0.683944in}}%
\pgfpathlineto{\pgfqpoint{0.848277in}{0.656516in}}%
\pgfpathlineto{\pgfqpoint{0.848446in}{0.651917in}}%
\pgfpathlineto{\pgfqpoint{0.848783in}{0.680852in}}%
\pgfpathlineto{\pgfqpoint{0.849794in}{0.739145in}}%
\pgfpathlineto{\pgfqpoint{0.849372in}{0.652222in}}%
\pgfpathlineto{\pgfqpoint{0.849962in}{0.692878in}}%
\pgfpathlineto{\pgfqpoint{0.851142in}{0.651798in}}%
\pgfpathlineto{\pgfqpoint{0.851479in}{0.652669in}}%
\pgfpathlineto{\pgfqpoint{0.852068in}{0.652241in}}%
\pgfpathlineto{\pgfqpoint{0.852658in}{0.654355in}}%
\pgfpathlineto{\pgfqpoint{0.852995in}{0.652287in}}%
\pgfpathlineto{\pgfqpoint{0.854006in}{0.652861in}}%
\pgfpathlineto{\pgfqpoint{0.854343in}{0.652936in}}%
\pgfpathlineto{\pgfqpoint{0.854596in}{0.653558in}}%
\pgfpathlineto{\pgfqpoint{0.855607in}{0.654198in}}%
\pgfpathlineto{\pgfqpoint{0.855185in}{0.653226in}}%
\pgfpathlineto{\pgfqpoint{0.855691in}{0.653981in}}%
\pgfpathlineto{\pgfqpoint{0.856870in}{0.653468in}}%
\pgfpathlineto{\pgfqpoint{0.856954in}{0.653507in}}%
\pgfpathlineto{\pgfqpoint{0.859735in}{0.654921in}}%
\pgfpathlineto{\pgfqpoint{0.860998in}{0.669566in}}%
\pgfpathlineto{\pgfqpoint{0.860493in}{0.654391in}}%
\pgfpathlineto{\pgfqpoint{0.861082in}{0.667119in}}%
\pgfpathlineto{\pgfqpoint{0.861504in}{0.654380in}}%
\pgfpathlineto{\pgfqpoint{0.862009in}{0.671506in}}%
\pgfpathlineto{\pgfqpoint{0.862346in}{0.655721in}}%
\pgfpathlineto{\pgfqpoint{0.862515in}{0.654460in}}%
\pgfpathlineto{\pgfqpoint{0.862683in}{0.660757in}}%
\pgfpathlineto{\pgfqpoint{0.862852in}{0.680725in}}%
\pgfpathlineto{\pgfqpoint{0.863357in}{0.654219in}}%
\pgfpathlineto{\pgfqpoint{0.863694in}{0.655159in}}%
\pgfpathlineto{\pgfqpoint{0.863778in}{0.655344in}}%
\pgfpathlineto{\pgfqpoint{0.864200in}{0.654507in}}%
\pgfpathlineto{\pgfqpoint{0.864705in}{0.654813in}}%
\pgfpathlineto{\pgfqpoint{0.865969in}{0.655072in}}%
\pgfpathlineto{\pgfqpoint{0.868075in}{0.655675in}}%
\pgfpathlineto{\pgfqpoint{0.869170in}{0.712556in}}%
\pgfpathlineto{\pgfqpoint{0.868664in}{0.654899in}}%
\pgfpathlineto{\pgfqpoint{0.869423in}{0.668479in}}%
\pgfpathlineto{\pgfqpoint{0.869591in}{0.655566in}}%
\pgfpathlineto{\pgfqpoint{0.869844in}{0.742941in}}%
\pgfpathlineto{\pgfqpoint{0.870012in}{0.976499in}}%
\pgfpathlineto{\pgfqpoint{0.870518in}{0.657765in}}%
\pgfpathlineto{\pgfqpoint{0.870939in}{0.787184in}}%
\pgfpathlineto{\pgfqpoint{0.871023in}{0.796695in}}%
\pgfpathlineto{\pgfqpoint{0.871192in}{0.731524in}}%
\pgfpathlineto{\pgfqpoint{0.872455in}{0.655687in}}%
\pgfpathlineto{\pgfqpoint{0.872540in}{0.655558in}}%
\pgfpathlineto{\pgfqpoint{0.872624in}{0.656174in}}%
\pgfpathlineto{\pgfqpoint{0.872961in}{0.665315in}}%
\pgfpathlineto{\pgfqpoint{0.873551in}{0.655210in}}%
\pgfpathlineto{\pgfqpoint{0.873635in}{0.655248in}}%
\pgfpathlineto{\pgfqpoint{0.874899in}{0.655106in}}%
\pgfpathlineto{\pgfqpoint{0.879616in}{0.655962in}}%
\pgfpathlineto{\pgfqpoint{0.880796in}{0.663087in}}%
\pgfpathlineto{\pgfqpoint{0.880374in}{0.655633in}}%
\pgfpathlineto{\pgfqpoint{0.880964in}{0.660580in}}%
\pgfpathlineto{\pgfqpoint{0.882144in}{0.655296in}}%
\pgfpathlineto{\pgfqpoint{0.881722in}{0.663991in}}%
\pgfpathlineto{\pgfqpoint{0.882228in}{0.655714in}}%
\pgfpathlineto{\pgfqpoint{0.882565in}{0.663980in}}%
\pgfpathlineto{\pgfqpoint{0.883070in}{0.655082in}}%
\pgfpathlineto{\pgfqpoint{0.883492in}{0.659814in}}%
\pgfpathlineto{\pgfqpoint{0.884671in}{0.654748in}}%
\pgfpathlineto{\pgfqpoint{0.884839in}{0.655368in}}%
\pgfpathlineto{\pgfqpoint{0.885092in}{0.657186in}}%
\pgfpathlineto{\pgfqpoint{0.885598in}{0.654676in}}%
\pgfpathlineto{\pgfqpoint{0.885850in}{0.654999in}}%
\pgfpathlineto{\pgfqpoint{0.887367in}{0.656279in}}%
\pgfpathlineto{\pgfqpoint{0.888209in}{0.654534in}}%
\pgfpathlineto{\pgfqpoint{0.888546in}{0.698310in}}%
\pgfpathlineto{\pgfqpoint{0.888630in}{0.706905in}}%
\pgfpathlineto{\pgfqpoint{0.889136in}{0.653847in}}%
\pgfpathlineto{\pgfqpoint{0.889220in}{0.653444in}}%
\pgfpathlineto{\pgfqpoint{0.889473in}{0.657642in}}%
\pgfpathlineto{\pgfqpoint{0.889641in}{0.662136in}}%
\pgfpathlineto{\pgfqpoint{0.890063in}{0.653285in}}%
\pgfpathlineto{\pgfqpoint{0.890484in}{0.655078in}}%
\pgfpathlineto{\pgfqpoint{0.890568in}{0.655496in}}%
\pgfpathlineto{\pgfqpoint{0.890989in}{0.653400in}}%
\pgfpathlineto{\pgfqpoint{0.891410in}{0.654119in}}%
\pgfpathlineto{\pgfqpoint{0.894528in}{0.653914in}}%
\pgfpathlineto{\pgfqpoint{0.894865in}{0.655108in}}%
\pgfpathlineto{\pgfqpoint{0.895201in}{0.659434in}}%
\pgfpathlineto{\pgfqpoint{0.895623in}{0.653637in}}%
\pgfpathlineto{\pgfqpoint{0.895875in}{0.654013in}}%
\pgfpathlineto{\pgfqpoint{0.897139in}{0.666040in}}%
\pgfpathlineto{\pgfqpoint{0.896634in}{0.653426in}}%
\pgfpathlineto{\pgfqpoint{0.897392in}{0.657841in}}%
\pgfpathlineto{\pgfqpoint{0.898066in}{0.662431in}}%
\pgfpathlineto{\pgfqpoint{0.898571in}{0.653540in}}%
\pgfpathlineto{\pgfqpoint{0.898908in}{0.655402in}}%
\pgfpathlineto{\pgfqpoint{0.900425in}{0.654151in}}%
\pgfpathlineto{\pgfqpoint{0.902699in}{0.656632in}}%
\pgfpathlineto{\pgfqpoint{0.904216in}{0.692823in}}%
\pgfpathlineto{\pgfqpoint{0.903710in}{0.655356in}}%
\pgfpathlineto{\pgfqpoint{0.904384in}{0.681851in}}%
\pgfpathlineto{\pgfqpoint{0.904890in}{0.654634in}}%
\pgfpathlineto{\pgfqpoint{0.905564in}{0.670063in}}%
\pgfpathlineto{\pgfqpoint{0.905648in}{0.671990in}}%
\pgfpathlineto{\pgfqpoint{0.905985in}{0.656062in}}%
\pgfpathlineto{\pgfqpoint{0.907080in}{0.653442in}}%
\pgfpathlineto{\pgfqpoint{0.906575in}{0.666755in}}%
\pgfpathlineto{\pgfqpoint{0.907164in}{0.653460in}}%
\pgfpathlineto{\pgfqpoint{0.907838in}{0.654357in}}%
\pgfpathlineto{\pgfqpoint{0.908428in}{0.653970in}}%
\pgfpathlineto{\pgfqpoint{0.908765in}{0.668007in}}%
\pgfpathlineto{\pgfqpoint{0.909018in}{0.697006in}}%
\pgfpathlineto{\pgfqpoint{0.909523in}{0.653591in}}%
\pgfpathlineto{\pgfqpoint{0.909944in}{0.691340in}}%
\pgfpathlineto{\pgfqpoint{0.910029in}{0.691408in}}%
\pgfpathlineto{\pgfqpoint{0.910702in}{0.653405in}}%
\pgfpathlineto{\pgfqpoint{0.910955in}{0.693296in}}%
\pgfpathlineto{\pgfqpoint{0.911208in}{0.797284in}}%
\pgfpathlineto{\pgfqpoint{0.911713in}{0.654994in}}%
\pgfpathlineto{\pgfqpoint{0.911966in}{0.665533in}}%
\pgfpathlineto{\pgfqpoint{0.912303in}{0.709482in}}%
\pgfpathlineto{\pgfqpoint{0.912809in}{0.653302in}}%
\pgfpathlineto{\pgfqpoint{0.912977in}{0.655784in}}%
\pgfpathlineto{\pgfqpoint{0.913146in}{0.659343in}}%
\pgfpathlineto{\pgfqpoint{0.913567in}{0.653223in}}%
\pgfpathlineto{\pgfqpoint{0.913988in}{0.653439in}}%
\pgfpathlineto{\pgfqpoint{0.915336in}{0.654801in}}%
\pgfpathlineto{\pgfqpoint{0.916178in}{0.654021in}}%
\pgfpathlineto{\pgfqpoint{0.916684in}{0.666676in}}%
\pgfpathlineto{\pgfqpoint{0.917863in}{0.742445in}}%
\pgfpathlineto{\pgfqpoint{0.917358in}{0.654044in}}%
\pgfpathlineto{\pgfqpoint{0.918032in}{0.691422in}}%
\pgfpathlineto{\pgfqpoint{0.919211in}{0.653180in}}%
\pgfpathlineto{\pgfqpoint{0.920306in}{0.653578in}}%
\pgfpathlineto{\pgfqpoint{0.921907in}{0.655728in}}%
\pgfpathlineto{\pgfqpoint{0.923171in}{0.682412in}}%
\pgfpathlineto{\pgfqpoint{0.922665in}{0.654559in}}%
\pgfpathlineto{\pgfqpoint{0.923423in}{0.665135in}}%
\pgfpathlineto{\pgfqpoint{0.923845in}{0.654268in}}%
\pgfpathlineto{\pgfqpoint{0.924434in}{0.671449in}}%
\pgfpathlineto{\pgfqpoint{0.925024in}{0.653645in}}%
\pgfpathlineto{\pgfqpoint{0.925361in}{0.668240in}}%
\pgfpathlineto{\pgfqpoint{0.926119in}{0.653717in}}%
\pgfpathlineto{\pgfqpoint{0.926625in}{0.687627in}}%
\pgfpathlineto{\pgfqpoint{0.926709in}{0.690851in}}%
\pgfpathlineto{\pgfqpoint{0.926962in}{0.666240in}}%
\pgfpathlineto{\pgfqpoint{0.927214in}{0.653791in}}%
\pgfpathlineto{\pgfqpoint{0.927551in}{0.707476in}}%
\pgfpathlineto{\pgfqpoint{0.927636in}{0.717522in}}%
\pgfpathlineto{\pgfqpoint{0.928141in}{0.653726in}}%
\pgfpathlineto{\pgfqpoint{0.928394in}{0.666870in}}%
\pgfpathlineto{\pgfqpoint{0.928647in}{0.690303in}}%
\pgfpathlineto{\pgfqpoint{0.929152in}{0.653624in}}%
\pgfpathlineto{\pgfqpoint{0.929573in}{0.688168in}}%
\pgfpathlineto{\pgfqpoint{0.930837in}{0.653373in}}%
\pgfpathlineto{\pgfqpoint{0.930416in}{0.693449in}}%
\pgfpathlineto{\pgfqpoint{0.931005in}{0.656485in}}%
\pgfpathlineto{\pgfqpoint{0.931258in}{0.667585in}}%
\pgfpathlineto{\pgfqpoint{0.931932in}{0.652436in}}%
\pgfpathlineto{\pgfqpoint{0.932016in}{0.652492in}}%
\pgfpathlineto{\pgfqpoint{0.932438in}{0.654667in}}%
\pgfpathlineto{\pgfqpoint{0.933364in}{0.652887in}}%
\pgfpathlineto{\pgfqpoint{0.933533in}{0.652470in}}%
\pgfpathlineto{\pgfqpoint{0.934459in}{0.653074in}}%
\pgfpathlineto{\pgfqpoint{0.935470in}{0.654953in}}%
\pgfpathlineto{\pgfqpoint{0.936650in}{0.666370in}}%
\pgfpathlineto{\pgfqpoint{0.936229in}{0.653385in}}%
\pgfpathlineto{\pgfqpoint{0.936734in}{0.664759in}}%
\pgfpathlineto{\pgfqpoint{0.937829in}{0.652457in}}%
\pgfpathlineto{\pgfqpoint{0.937998in}{0.653073in}}%
\pgfpathlineto{\pgfqpoint{0.938250in}{0.656507in}}%
\pgfpathlineto{\pgfqpoint{0.938756in}{0.652174in}}%
\pgfpathlineto{\pgfqpoint{0.939093in}{0.652998in}}%
\pgfpathlineto{\pgfqpoint{0.939346in}{0.653548in}}%
\pgfpathlineto{\pgfqpoint{0.939683in}{0.652451in}}%
\pgfpathlineto{\pgfqpoint{0.940188in}{0.652884in}}%
\pgfpathlineto{\pgfqpoint{0.943811in}{0.657008in}}%
\pgfpathlineto{\pgfqpoint{0.944316in}{0.654567in}}%
\pgfpathlineto{\pgfqpoint{0.944569in}{0.673384in}}%
\pgfpathlineto{\pgfqpoint{0.945664in}{0.773817in}}%
\pgfpathlineto{\pgfqpoint{0.945159in}{0.655752in}}%
\pgfpathlineto{\pgfqpoint{0.945748in}{0.755607in}}%
\pgfpathlineto{\pgfqpoint{0.946338in}{0.656089in}}%
\pgfpathlineto{\pgfqpoint{0.947096in}{0.669603in}}%
\pgfpathlineto{\pgfqpoint{0.947349in}{0.655127in}}%
\pgfpathlineto{\pgfqpoint{0.947854in}{0.700124in}}%
\pgfpathlineto{\pgfqpoint{0.948276in}{0.655380in}}%
\pgfpathlineto{\pgfqpoint{0.948360in}{0.655308in}}%
\pgfpathlineto{\pgfqpoint{0.948444in}{0.656624in}}%
\pgfpathlineto{\pgfqpoint{0.948865in}{0.702778in}}%
\pgfpathlineto{\pgfqpoint{0.949371in}{0.655210in}}%
\pgfpathlineto{\pgfqpoint{0.949792in}{0.682490in}}%
\pgfpathlineto{\pgfqpoint{0.950466in}{0.654611in}}%
\pgfpathlineto{\pgfqpoint{0.951224in}{0.659810in}}%
\pgfpathlineto{\pgfqpoint{0.951982in}{0.662931in}}%
\pgfpathlineto{\pgfqpoint{0.952488in}{0.653866in}}%
\pgfpathlineto{\pgfqpoint{0.952741in}{0.655288in}}%
\pgfpathlineto{\pgfqpoint{0.953836in}{0.666359in}}%
\pgfpathlineto{\pgfqpoint{0.953414in}{0.653898in}}%
\pgfpathlineto{\pgfqpoint{0.954004in}{0.659502in}}%
\pgfpathlineto{\pgfqpoint{0.955184in}{0.653783in}}%
\pgfpathlineto{\pgfqpoint{0.955521in}{0.654844in}}%
\pgfpathlineto{\pgfqpoint{0.956447in}{0.660034in}}%
\pgfpathlineto{\pgfqpoint{0.955942in}{0.653779in}}%
\pgfpathlineto{\pgfqpoint{0.956784in}{0.656014in}}%
\pgfpathlineto{\pgfqpoint{0.957037in}{0.653760in}}%
\pgfpathlineto{\pgfqpoint{0.957374in}{0.661680in}}%
\pgfpathlineto{\pgfqpoint{0.957711in}{0.690000in}}%
\pgfpathlineto{\pgfqpoint{0.958216in}{0.653131in}}%
\pgfpathlineto{\pgfqpoint{0.958553in}{0.667990in}}%
\pgfpathlineto{\pgfqpoint{0.958638in}{0.668850in}}%
\pgfpathlineto{\pgfqpoint{0.958806in}{0.663257in}}%
\pgfpathlineto{\pgfqpoint{0.959649in}{0.666738in}}%
\pgfpathlineto{\pgfqpoint{0.960070in}{0.651491in}}%
\pgfpathlineto{\pgfqpoint{0.960238in}{0.656337in}}%
\pgfpathlineto{\pgfqpoint{0.960491in}{0.678247in}}%
\pgfpathlineto{\pgfqpoint{0.960996in}{0.650733in}}%
\pgfpathlineto{\pgfqpoint{0.961249in}{0.652949in}}%
\pgfpathlineto{\pgfqpoint{0.961333in}{0.653016in}}%
\pgfpathlineto{\pgfqpoint{0.961418in}{0.652184in}}%
\pgfpathlineto{\pgfqpoint{0.961755in}{0.650901in}}%
\pgfpathlineto{\pgfqpoint{0.962344in}{0.652795in}}%
\pgfpathlineto{\pgfqpoint{0.962513in}{0.655808in}}%
\pgfpathlineto{\pgfqpoint{0.962934in}{0.651154in}}%
\pgfpathlineto{\pgfqpoint{0.963440in}{0.653222in}}%
\pgfpathlineto{\pgfqpoint{0.963524in}{0.653349in}}%
\pgfpathlineto{\pgfqpoint{0.963692in}{0.652153in}}%
\pgfpathlineto{\pgfqpoint{0.963861in}{0.651294in}}%
\pgfpathlineto{\pgfqpoint{0.964198in}{0.655775in}}%
\pgfpathlineto{\pgfqpoint{0.964703in}{0.653044in}}%
\pgfpathlineto{\pgfqpoint{0.965883in}{0.720989in}}%
\pgfpathlineto{\pgfqpoint{0.965461in}{0.651957in}}%
\pgfpathlineto{\pgfqpoint{0.966051in}{0.681361in}}%
\pgfpathlineto{\pgfqpoint{0.966388in}{0.651404in}}%
\pgfpathlineto{\pgfqpoint{0.967231in}{0.654090in}}%
\pgfpathlineto{\pgfqpoint{0.967399in}{0.651427in}}%
\pgfpathlineto{\pgfqpoint{0.967736in}{0.666122in}}%
\pgfpathlineto{\pgfqpoint{0.967905in}{0.676838in}}%
\pgfpathlineto{\pgfqpoint{0.968578in}{0.651530in}}%
\pgfpathlineto{\pgfqpoint{0.968663in}{0.651824in}}%
\pgfpathlineto{\pgfqpoint{0.969000in}{0.669722in}}%
\pgfpathlineto{\pgfqpoint{0.969168in}{0.681074in}}%
\pgfpathlineto{\pgfqpoint{0.969758in}{0.651556in}}%
\pgfpathlineto{\pgfqpoint{0.969926in}{0.652998in}}%
\pgfpathlineto{\pgfqpoint{0.970348in}{0.698683in}}%
\pgfpathlineto{\pgfqpoint{0.970853in}{0.651377in}}%
\pgfpathlineto{\pgfqpoint{0.971274in}{0.675682in}}%
\pgfpathlineto{\pgfqpoint{0.972538in}{0.650577in}}%
\pgfpathlineto{\pgfqpoint{0.972622in}{0.651682in}}%
\pgfpathlineto{\pgfqpoint{0.972959in}{0.691864in}}%
\pgfpathlineto{\pgfqpoint{0.973465in}{0.650408in}}%
\pgfpathlineto{\pgfqpoint{0.974054in}{0.675377in}}%
\pgfpathlineto{\pgfqpoint{0.974560in}{0.650198in}}%
\pgfpathlineto{\pgfqpoint{0.975065in}{0.679780in}}%
\pgfpathlineto{\pgfqpoint{0.975150in}{0.681232in}}%
\pgfpathlineto{\pgfqpoint{0.975318in}{0.665425in}}%
\pgfpathlineto{\pgfqpoint{0.975739in}{0.650648in}}%
\pgfpathlineto{\pgfqpoint{0.976497in}{0.657411in}}%
\pgfpathlineto{\pgfqpoint{0.976834in}{0.650256in}}%
\pgfpathlineto{\pgfqpoint{0.977761in}{0.650864in}}%
\pgfpathlineto{\pgfqpoint{0.979025in}{0.664328in}}%
\pgfpathlineto{\pgfqpoint{0.978604in}{0.650725in}}%
\pgfpathlineto{\pgfqpoint{0.979193in}{0.655434in}}%
\pgfpathlineto{\pgfqpoint{0.979362in}{0.650111in}}%
\pgfpathlineto{\pgfqpoint{0.979783in}{0.674655in}}%
\pgfpathlineto{\pgfqpoint{0.980878in}{0.715921in}}%
\pgfpathlineto{\pgfqpoint{0.980288in}{0.650147in}}%
\pgfpathlineto{\pgfqpoint{0.981047in}{0.687109in}}%
\pgfpathlineto{\pgfqpoint{0.981299in}{0.650442in}}%
\pgfpathlineto{\pgfqpoint{0.981636in}{0.757707in}}%
\pgfpathlineto{\pgfqpoint{0.981805in}{0.824642in}}%
\pgfpathlineto{\pgfqpoint{0.982226in}{0.652387in}}%
\pgfpathlineto{\pgfqpoint{0.982647in}{0.748557in}}%
\pgfpathlineto{\pgfqpoint{0.984079in}{0.652109in}}%
\pgfpathlineto{\pgfqpoint{0.985259in}{0.652747in}}%
\pgfpathlineto{\pgfqpoint{0.985596in}{0.653813in}}%
\pgfpathlineto{\pgfqpoint{0.985680in}{0.654438in}}%
\pgfpathlineto{\pgfqpoint{0.986017in}{0.652887in}}%
\pgfpathlineto{\pgfqpoint{0.986607in}{0.653352in}}%
\pgfpathlineto{\pgfqpoint{0.987365in}{0.653538in}}%
\pgfpathlineto{\pgfqpoint{0.987112in}{0.653763in}}%
\pgfpathlineto{\pgfqpoint{0.987534in}{0.653715in}}%
\pgfpathlineto{\pgfqpoint{0.988629in}{0.656256in}}%
\pgfpathlineto{\pgfqpoint{0.988797in}{0.654994in}}%
\pgfpathlineto{\pgfqpoint{0.989050in}{0.653962in}}%
\pgfpathlineto{\pgfqpoint{0.989471in}{0.655640in}}%
\pgfpathlineto{\pgfqpoint{0.989977in}{0.654188in}}%
\pgfpathlineto{\pgfqpoint{0.990314in}{0.656029in}}%
\pgfpathlineto{\pgfqpoint{0.991325in}{0.664892in}}%
\pgfpathlineto{\pgfqpoint{0.990903in}{0.654381in}}%
\pgfpathlineto{\pgfqpoint{0.991493in}{0.660155in}}%
\pgfpathlineto{\pgfqpoint{0.991830in}{0.654303in}}%
\pgfpathlineto{\pgfqpoint{0.992757in}{0.654453in}}%
\pgfpathlineto{\pgfqpoint{0.995621in}{0.656152in}}%
\pgfpathlineto{\pgfqpoint{0.996211in}{0.654947in}}%
\pgfpathlineto{\pgfqpoint{0.996716in}{0.657399in}}%
\pgfpathlineto{\pgfqpoint{0.997053in}{0.655055in}}%
\pgfpathlineto{\pgfqpoint{0.997980in}{0.655315in}}%
\pgfpathlineto{\pgfqpoint{0.998907in}{0.662407in}}%
\pgfpathlineto{\pgfqpoint{1.000086in}{0.734223in}}%
\pgfpathlineto{\pgfqpoint{0.999580in}{0.655017in}}%
\pgfpathlineto{\pgfqpoint{1.000254in}{0.705531in}}%
\pgfpathlineto{\pgfqpoint{1.000676in}{0.653769in}}%
\pgfpathlineto{\pgfqpoint{1.001518in}{0.653960in}}%
\pgfpathlineto{\pgfqpoint{1.001602in}{0.653145in}}%
\pgfpathlineto{\pgfqpoint{1.001771in}{0.657157in}}%
\pgfpathlineto{\pgfqpoint{1.002024in}{0.675524in}}%
\pgfpathlineto{\pgfqpoint{1.002698in}{0.653043in}}%
\pgfpathlineto{\pgfqpoint{1.002782in}{0.653255in}}%
\pgfpathlineto{\pgfqpoint{1.003034in}{0.665457in}}%
\pgfpathlineto{\pgfqpoint{1.003287in}{0.687869in}}%
\pgfpathlineto{\pgfqpoint{1.003793in}{0.652587in}}%
\pgfpathlineto{\pgfqpoint{1.004130in}{0.666864in}}%
\pgfpathlineto{\pgfqpoint{1.004298in}{0.670299in}}%
\pgfpathlineto{\pgfqpoint{1.004804in}{0.652043in}}%
\pgfpathlineto{\pgfqpoint{1.004972in}{0.656111in}}%
\pgfpathlineto{\pgfqpoint{1.005225in}{0.681518in}}%
\pgfpathlineto{\pgfqpoint{1.005730in}{0.651958in}}%
\pgfpathlineto{\pgfqpoint{1.006067in}{0.659039in}}%
\pgfpathlineto{\pgfqpoint{1.006320in}{0.668083in}}%
\pgfpathlineto{\pgfqpoint{1.006741in}{0.651841in}}%
\pgfpathlineto{\pgfqpoint{1.007247in}{0.662958in}}%
\pgfpathlineto{\pgfqpoint{1.007331in}{0.664948in}}%
\pgfpathlineto{\pgfqpoint{1.007836in}{0.651807in}}%
\pgfpathlineto{\pgfqpoint{1.008089in}{0.659145in}}%
\pgfpathlineto{\pgfqpoint{1.008173in}{0.662153in}}%
\pgfpathlineto{\pgfqpoint{1.008595in}{0.651886in}}%
\pgfpathlineto{\pgfqpoint{1.009100in}{0.656071in}}%
\pgfpathlineto{\pgfqpoint{1.009437in}{0.652040in}}%
\pgfpathlineto{\pgfqpoint{1.009774in}{0.663021in}}%
\pgfpathlineto{\pgfqpoint{1.010953in}{0.732889in}}%
\pgfpathlineto{\pgfqpoint{1.010448in}{0.651918in}}%
\pgfpathlineto{\pgfqpoint{1.011122in}{0.691140in}}%
\pgfpathlineto{\pgfqpoint{1.011543in}{0.651580in}}%
\pgfpathlineto{\pgfqpoint{1.012133in}{0.687868in}}%
\pgfpathlineto{\pgfqpoint{1.013397in}{0.906436in}}%
\pgfpathlineto{\pgfqpoint{1.012891in}{0.652687in}}%
\pgfpathlineto{\pgfqpoint{1.013565in}{0.813172in}}%
\pgfpathlineto{\pgfqpoint{1.014408in}{0.835573in}}%
\pgfpathlineto{\pgfqpoint{1.014829in}{0.655200in}}%
\pgfpathlineto{\pgfqpoint{1.014913in}{0.654742in}}%
\pgfpathlineto{\pgfqpoint{1.014997in}{0.658415in}}%
\pgfpathlineto{\pgfqpoint{1.015250in}{0.698906in}}%
\pgfpathlineto{\pgfqpoint{1.015755in}{0.654047in}}%
\pgfpathlineto{\pgfqpoint{1.016092in}{0.660678in}}%
\pgfpathlineto{\pgfqpoint{1.016514in}{0.654074in}}%
\pgfpathlineto{\pgfqpoint{1.017693in}{0.654608in}}%
\pgfpathlineto{\pgfqpoint{1.018030in}{0.654463in}}%
\pgfpathlineto{\pgfqpoint{1.018283in}{0.655566in}}%
\pgfpathlineto{\pgfqpoint{1.018367in}{0.655948in}}%
\pgfpathlineto{\pgfqpoint{1.018788in}{0.654533in}}%
\pgfpathlineto{\pgfqpoint{1.019209in}{0.654822in}}%
\pgfpathlineto{\pgfqpoint{1.020557in}{0.654859in}}%
\pgfpathlineto{\pgfqpoint{1.021990in}{0.655322in}}%
\pgfpathlineto{\pgfqpoint{1.022242in}{0.658102in}}%
\pgfpathlineto{\pgfqpoint{1.022663in}{0.655058in}}%
\pgfpathlineto{\pgfqpoint{1.023253in}{0.656895in}}%
\pgfpathlineto{\pgfqpoint{1.024433in}{0.655034in}}%
\pgfpathlineto{\pgfqpoint{1.024517in}{0.655048in}}%
\pgfpathlineto{\pgfqpoint{1.024854in}{0.656754in}}%
\pgfpathlineto{\pgfqpoint{1.025781in}{0.658520in}}%
\pgfpathlineto{\pgfqpoint{1.025359in}{0.655017in}}%
\pgfpathlineto{\pgfqpoint{1.025949in}{0.657388in}}%
\pgfpathlineto{\pgfqpoint{1.027044in}{0.654899in}}%
\pgfpathlineto{\pgfqpoint{1.026707in}{0.658344in}}%
\pgfpathlineto{\pgfqpoint{1.027213in}{0.655066in}}%
\pgfpathlineto{\pgfqpoint{1.028308in}{0.658857in}}%
\pgfpathlineto{\pgfqpoint{1.027802in}{0.654964in}}%
\pgfpathlineto{\pgfqpoint{1.028476in}{0.657318in}}%
\pgfpathlineto{\pgfqpoint{1.028729in}{0.655025in}}%
\pgfpathlineto{\pgfqpoint{1.029066in}{0.659472in}}%
\pgfpathlineto{\pgfqpoint{1.029319in}{0.670606in}}%
\pgfpathlineto{\pgfqpoint{1.029824in}{0.655172in}}%
\pgfpathlineto{\pgfqpoint{1.030245in}{0.664025in}}%
\pgfpathlineto{\pgfqpoint{1.032436in}{0.653154in}}%
\pgfpathlineto{\pgfqpoint{1.032604in}{0.654760in}}%
\pgfpathlineto{\pgfqpoint{1.033026in}{0.691739in}}%
\pgfpathlineto{\pgfqpoint{1.033531in}{0.653367in}}%
\pgfpathlineto{\pgfqpoint{1.033700in}{0.654986in}}%
\pgfpathlineto{\pgfqpoint{1.033868in}{0.655959in}}%
\pgfpathlineto{\pgfqpoint{1.034458in}{0.652847in}}%
\pgfpathlineto{\pgfqpoint{1.034542in}{0.652876in}}%
\pgfpathlineto{\pgfqpoint{1.036143in}{0.654061in}}%
\pgfpathlineto{\pgfqpoint{1.037322in}{0.672299in}}%
\pgfpathlineto{\pgfqpoint{1.036817in}{0.653228in}}%
\pgfpathlineto{\pgfqpoint{1.037491in}{0.662355in}}%
\pgfpathlineto{\pgfqpoint{1.037743in}{0.652995in}}%
\pgfpathlineto{\pgfqpoint{1.037996in}{0.697660in}}%
\pgfpathlineto{\pgfqpoint{1.038164in}{0.801264in}}%
\pgfpathlineto{\pgfqpoint{1.038670in}{0.654323in}}%
\pgfpathlineto{\pgfqpoint{1.039091in}{0.738282in}}%
\pgfpathlineto{\pgfqpoint{1.039175in}{0.743747in}}%
\pgfpathlineto{\pgfqpoint{1.039344in}{0.696352in}}%
\pgfpathlineto{\pgfqpoint{1.039765in}{0.653679in}}%
\pgfpathlineto{\pgfqpoint{1.040608in}{0.655000in}}%
\pgfpathlineto{\pgfqpoint{1.040860in}{0.653288in}}%
\pgfpathlineto{\pgfqpoint{1.041787in}{0.653544in}}%
\pgfpathlineto{\pgfqpoint{1.042124in}{0.654562in}}%
\pgfpathlineto{\pgfqpoint{1.042377in}{0.663000in}}%
\pgfpathlineto{\pgfqpoint{1.042798in}{0.653726in}}%
\pgfpathlineto{\pgfqpoint{1.043219in}{0.656989in}}%
\pgfpathlineto{\pgfqpoint{1.043893in}{0.653813in}}%
\pgfpathlineto{\pgfqpoint{1.044651in}{0.654100in}}%
\pgfpathlineto{\pgfqpoint{1.044988in}{0.656538in}}%
\pgfpathlineto{\pgfqpoint{1.045494in}{0.654284in}}%
\pgfpathlineto{\pgfqpoint{1.045746in}{0.667649in}}%
\pgfpathlineto{\pgfqpoint{1.047010in}{0.778357in}}%
\pgfpathlineto{\pgfqpoint{1.046505in}{0.655227in}}%
\pgfpathlineto{\pgfqpoint{1.047094in}{0.758086in}}%
\pgfpathlineto{\pgfqpoint{1.048358in}{0.654737in}}%
\pgfpathlineto{\pgfqpoint{1.048442in}{0.654692in}}%
\pgfpathlineto{\pgfqpoint{1.048527in}{0.655182in}}%
\pgfpathlineto{\pgfqpoint{1.048695in}{0.657503in}}%
\pgfpathlineto{\pgfqpoint{1.049200in}{0.654535in}}%
\pgfpathlineto{\pgfqpoint{1.049537in}{0.654560in}}%
\pgfpathlineto{\pgfqpoint{1.053918in}{0.655849in}}%
\pgfpathlineto{\pgfqpoint{1.054845in}{0.669563in}}%
\pgfpathlineto{\pgfqpoint{1.054424in}{0.655095in}}%
\pgfpathlineto{\pgfqpoint{1.055013in}{0.661862in}}%
\pgfpathlineto{\pgfqpoint{1.055266in}{0.655083in}}%
\pgfpathlineto{\pgfqpoint{1.055603in}{0.690161in}}%
\pgfpathlineto{\pgfqpoint{1.055687in}{0.692419in}}%
\pgfpathlineto{\pgfqpoint{1.055772in}{0.680847in}}%
\pgfpathlineto{\pgfqpoint{1.056446in}{0.692522in}}%
\pgfpathlineto{\pgfqpoint{1.056951in}{0.655120in}}%
\pgfpathlineto{\pgfqpoint{1.057119in}{0.655825in}}%
\pgfpathlineto{\pgfqpoint{1.057372in}{0.661928in}}%
\pgfpathlineto{\pgfqpoint{1.057878in}{0.654954in}}%
\pgfpathlineto{\pgfqpoint{1.058130in}{0.655010in}}%
\pgfpathlineto{\pgfqpoint{1.060237in}{0.655999in}}%
\pgfpathlineto{\pgfqpoint{1.061584in}{0.667419in}}%
\pgfpathlineto{\pgfqpoint{1.061079in}{0.655107in}}%
\pgfpathlineto{\pgfqpoint{1.061753in}{0.662267in}}%
\pgfpathlineto{\pgfqpoint{1.062090in}{0.654721in}}%
\pgfpathlineto{\pgfqpoint{1.063017in}{0.654896in}}%
\pgfpathlineto{\pgfqpoint{1.063943in}{0.655647in}}%
\pgfpathlineto{\pgfqpoint{1.064701in}{0.655505in}}%
\pgfpathlineto{\pgfqpoint{1.064954in}{0.707708in}}%
\pgfpathlineto{\pgfqpoint{1.065123in}{0.743592in}}%
\pgfpathlineto{\pgfqpoint{1.065628in}{0.655118in}}%
\pgfpathlineto{\pgfqpoint{1.065881in}{0.664490in}}%
\pgfpathlineto{\pgfqpoint{1.066049in}{0.670912in}}%
\pgfpathlineto{\pgfqpoint{1.066555in}{0.654639in}}%
\pgfpathlineto{\pgfqpoint{1.066808in}{0.654886in}}%
\pgfpathlineto{\pgfqpoint{1.069419in}{0.656439in}}%
\pgfpathlineto{\pgfqpoint{1.069672in}{0.662951in}}%
\pgfpathlineto{\pgfqpoint{1.070177in}{0.654374in}}%
\pgfpathlineto{\pgfqpoint{1.070599in}{0.660189in}}%
\pgfpathlineto{\pgfqpoint{1.071020in}{0.654073in}}%
\pgfpathlineto{\pgfqpoint{1.071862in}{0.654581in}}%
\pgfpathlineto{\pgfqpoint{1.073968in}{0.655353in}}%
\pgfpathlineto{\pgfqpoint{1.074221in}{0.658698in}}%
\pgfpathlineto{\pgfqpoint{1.074558in}{0.654930in}}%
\pgfpathlineto{\pgfqpoint{1.075148in}{0.657175in}}%
\pgfpathlineto{\pgfqpoint{1.075232in}{0.657623in}}%
\pgfpathlineto{\pgfqpoint{1.075738in}{0.654798in}}%
\pgfpathlineto{\pgfqpoint{1.075822in}{0.654799in}}%
\pgfpathlineto{\pgfqpoint{1.076243in}{0.657590in}}%
\pgfpathlineto{\pgfqpoint{1.077170in}{0.655564in}}%
\pgfpathlineto{\pgfqpoint{1.077422in}{0.654779in}}%
\pgfpathlineto{\pgfqpoint{1.078349in}{0.655059in}}%
\pgfpathlineto{\pgfqpoint{1.081466in}{0.656242in}}%
\pgfpathlineto{\pgfqpoint{1.082140in}{0.655043in}}%
\pgfpathlineto{\pgfqpoint{1.082477in}{0.682465in}}%
\pgfpathlineto{\pgfqpoint{1.083741in}{0.756188in}}%
\pgfpathlineto{\pgfqpoint{1.083235in}{0.656402in}}%
\pgfpathlineto{\pgfqpoint{1.083825in}{0.735037in}}%
\pgfpathlineto{\pgfqpoint{1.084415in}{0.654413in}}%
\pgfpathlineto{\pgfqpoint{1.085089in}{0.656340in}}%
\pgfpathlineto{\pgfqpoint{1.086015in}{0.653942in}}%
\pgfpathlineto{\pgfqpoint{1.086268in}{0.655885in}}%
\pgfpathlineto{\pgfqpoint{1.086605in}{0.672196in}}%
\pgfpathlineto{\pgfqpoint{1.087026in}{0.653959in}}%
\pgfpathlineto{\pgfqpoint{1.087532in}{0.667024in}}%
\pgfpathlineto{\pgfqpoint{1.088711in}{0.653629in}}%
\pgfpathlineto{\pgfqpoint{1.088795in}{0.653925in}}%
\pgfpathlineto{\pgfqpoint{1.089722in}{0.674658in}}%
\pgfpathlineto{\pgfqpoint{1.089385in}{0.653502in}}%
\pgfpathlineto{\pgfqpoint{1.089975in}{0.659042in}}%
\pgfpathlineto{\pgfqpoint{1.090143in}{0.653587in}}%
\pgfpathlineto{\pgfqpoint{1.090396in}{0.690808in}}%
\pgfpathlineto{\pgfqpoint{1.090565in}{0.780267in}}%
\pgfpathlineto{\pgfqpoint{1.091070in}{0.654913in}}%
\pgfpathlineto{\pgfqpoint{1.091491in}{0.701164in}}%
\pgfpathlineto{\pgfqpoint{1.091660in}{0.729509in}}%
\pgfpathlineto{\pgfqpoint{1.092249in}{0.654855in}}%
\pgfpathlineto{\pgfqpoint{1.092671in}{0.728389in}}%
\pgfpathlineto{\pgfqpoint{1.092755in}{0.727560in}}%
\pgfpathlineto{\pgfqpoint{1.093345in}{0.654313in}}%
\pgfpathlineto{\pgfqpoint{1.094187in}{0.654785in}}%
\pgfpathlineto{\pgfqpoint{1.094440in}{0.653757in}}%
\pgfpathlineto{\pgfqpoint{1.095030in}{0.656341in}}%
\pgfpathlineto{\pgfqpoint{1.095198in}{0.657787in}}%
\pgfpathlineto{\pgfqpoint{1.095703in}{0.653564in}}%
\pgfpathlineto{\pgfqpoint{1.095872in}{0.653713in}}%
\pgfpathlineto{\pgfqpoint{1.096209in}{0.656808in}}%
\pgfpathlineto{\pgfqpoint{1.096546in}{0.671623in}}%
\pgfpathlineto{\pgfqpoint{1.097051in}{0.653242in}}%
\pgfpathlineto{\pgfqpoint{1.097473in}{0.665596in}}%
\pgfpathlineto{\pgfqpoint{1.098231in}{0.652200in}}%
\pgfpathlineto{\pgfqpoint{1.099494in}{0.652667in}}%
\pgfpathlineto{\pgfqpoint{1.099916in}{0.671998in}}%
\pgfpathlineto{\pgfqpoint{1.100000in}{0.673714in}}%
\pgfpathlineto{\pgfqpoint{1.100253in}{0.657377in}}%
\pgfpathlineto{\pgfqpoint{1.100421in}{0.653013in}}%
\pgfpathlineto{\pgfqpoint{1.100674in}{0.672925in}}%
\pgfpathlineto{\pgfqpoint{1.100927in}{0.744129in}}%
\pgfpathlineto{\pgfqpoint{1.101516in}{0.653179in}}%
\pgfpathlineto{\pgfqpoint{1.101685in}{0.654317in}}%
\pgfpathlineto{\pgfqpoint{1.102022in}{0.659707in}}%
\pgfpathlineto{\pgfqpoint{1.102696in}{0.653839in}}%
\pgfpathlineto{\pgfqpoint{1.103033in}{0.652310in}}%
\pgfpathlineto{\pgfqpoint{1.103285in}{0.655285in}}%
\pgfpathlineto{\pgfqpoint{1.103622in}{0.668999in}}%
\pgfpathlineto{\pgfqpoint{1.104128in}{0.652115in}}%
\pgfpathlineto{\pgfqpoint{1.104549in}{0.662092in}}%
\pgfpathlineto{\pgfqpoint{1.104633in}{0.664400in}}%
\pgfpathlineto{\pgfqpoint{1.105223in}{0.651930in}}%
\pgfpathlineto{\pgfqpoint{1.105476in}{0.653485in}}%
\pgfpathlineto{\pgfqpoint{1.105813in}{0.664552in}}%
\pgfpathlineto{\pgfqpoint{1.106403in}{0.651462in}}%
\pgfpathlineto{\pgfqpoint{1.106487in}{0.651466in}}%
\pgfpathlineto{\pgfqpoint{1.106740in}{0.653022in}}%
\pgfpathlineto{\pgfqpoint{1.107076in}{0.663691in}}%
\pgfpathlineto{\pgfqpoint{1.107666in}{0.651450in}}%
\pgfpathlineto{\pgfqpoint{1.107750in}{0.651679in}}%
\pgfpathlineto{\pgfqpoint{1.108003in}{0.658659in}}%
\pgfpathlineto{\pgfqpoint{1.108256in}{0.675820in}}%
\pgfpathlineto{\pgfqpoint{1.108761in}{0.651407in}}%
\pgfpathlineto{\pgfqpoint{1.109098in}{0.658013in}}%
\pgfpathlineto{\pgfqpoint{1.109351in}{0.669151in}}%
\pgfpathlineto{\pgfqpoint{1.109772in}{0.650837in}}%
\pgfpathlineto{\pgfqpoint{1.110109in}{0.651893in}}%
\pgfpathlineto{\pgfqpoint{1.110362in}{0.651082in}}%
\pgfpathlineto{\pgfqpoint{1.111120in}{0.652271in}}%
\pgfpathlineto{\pgfqpoint{1.111541in}{0.651892in}}%
\pgfpathlineto{\pgfqpoint{1.111794in}{0.653454in}}%
\pgfpathlineto{\pgfqpoint{1.112131in}{0.669289in}}%
\pgfpathlineto{\pgfqpoint{1.112637in}{0.651971in}}%
\pgfpathlineto{\pgfqpoint{1.112889in}{0.653747in}}%
\pgfpathlineto{\pgfqpoint{1.113058in}{0.654591in}}%
\pgfpathlineto{\pgfqpoint{1.113732in}{0.652173in}}%
\pgfpathlineto{\pgfqpoint{1.113816in}{0.652191in}}%
\pgfpathlineto{\pgfqpoint{1.114574in}{0.652855in}}%
\pgfpathlineto{\pgfqpoint{1.114995in}{0.668823in}}%
\pgfpathlineto{\pgfqpoint{1.115417in}{0.652514in}}%
\pgfpathlineto{\pgfqpoint{1.116091in}{0.660461in}}%
\pgfpathlineto{\pgfqpoint{1.116596in}{0.652473in}}%
\pgfpathlineto{\pgfqpoint{1.117102in}{0.661052in}}%
\pgfpathlineto{\pgfqpoint{1.117523in}{0.652866in}}%
\pgfpathlineto{\pgfqpoint{1.117944in}{0.651704in}}%
\pgfpathlineto{\pgfqpoint{1.118113in}{0.653292in}}%
\pgfpathlineto{\pgfqpoint{1.118450in}{0.651648in}}%
\pgfpathlineto{\pgfqpoint{1.118618in}{0.656926in}}%
\pgfpathlineto{\pgfqpoint{1.118955in}{0.762558in}}%
\pgfpathlineto{\pgfqpoint{1.119545in}{0.653056in}}%
\pgfpathlineto{\pgfqpoint{1.119713in}{0.659199in}}%
\pgfpathlineto{\pgfqpoint{1.119966in}{0.694603in}}%
\pgfpathlineto{\pgfqpoint{1.120556in}{0.652035in}}%
\pgfpathlineto{\pgfqpoint{1.120808in}{0.659598in}}%
\pgfpathlineto{\pgfqpoint{1.120977in}{0.667506in}}%
\pgfpathlineto{\pgfqpoint{1.121398in}{0.652013in}}%
\pgfpathlineto{\pgfqpoint{1.121819in}{0.656179in}}%
\pgfpathlineto{\pgfqpoint{1.122241in}{0.652222in}}%
\pgfpathlineto{\pgfqpoint{1.123083in}{0.652683in}}%
\pgfpathlineto{\pgfqpoint{1.124599in}{0.653994in}}%
\pgfpathlineto{\pgfqpoint{1.125863in}{0.676282in}}%
\pgfpathlineto{\pgfqpoint{1.125273in}{0.653632in}}%
\pgfpathlineto{\pgfqpoint{1.126032in}{0.663862in}}%
\pgfpathlineto{\pgfqpoint{1.126368in}{0.653509in}}%
\pgfpathlineto{\pgfqpoint{1.127211in}{0.653566in}}%
\pgfpathlineto{\pgfqpoint{1.127632in}{0.654332in}}%
\pgfpathlineto{\pgfqpoint{1.128222in}{0.653477in}}%
\pgfpathlineto{\pgfqpoint{1.128559in}{0.665236in}}%
\pgfpathlineto{\pgfqpoint{1.129738in}{0.779815in}}%
\pgfpathlineto{\pgfqpoint{1.129233in}{0.654185in}}%
\pgfpathlineto{\pgfqpoint{1.129991in}{0.706643in}}%
\pgfpathlineto{\pgfqpoint{1.130665in}{0.741103in}}%
\pgfpathlineto{\pgfqpoint{1.131170in}{0.654051in}}%
\pgfpathlineto{\pgfqpoint{1.131255in}{0.653997in}}%
\pgfpathlineto{\pgfqpoint{1.131339in}{0.654551in}}%
\pgfpathlineto{\pgfqpoint{1.131592in}{0.660819in}}%
\pgfpathlineto{\pgfqpoint{1.132097in}{0.653885in}}%
\pgfpathlineto{\pgfqpoint{1.132350in}{0.654044in}}%
\pgfpathlineto{\pgfqpoint{1.134540in}{0.655794in}}%
\pgfpathlineto{\pgfqpoint{1.134877in}{0.658521in}}%
\pgfpathlineto{\pgfqpoint{1.135383in}{0.654554in}}%
\pgfpathlineto{\pgfqpoint{1.135467in}{0.654550in}}%
\pgfpathlineto{\pgfqpoint{1.136394in}{0.655249in}}%
\pgfpathlineto{\pgfqpoint{1.137657in}{0.669278in}}%
\pgfpathlineto{\pgfqpoint{1.137152in}{0.654998in}}%
\pgfpathlineto{\pgfqpoint{1.137826in}{0.663214in}}%
\pgfpathlineto{\pgfqpoint{1.138247in}{0.654731in}}%
\pgfpathlineto{\pgfqpoint{1.139089in}{0.655500in}}%
\pgfpathlineto{\pgfqpoint{1.139174in}{0.655589in}}%
\pgfpathlineto{\pgfqpoint{1.139511in}{0.654830in}}%
\pgfpathlineto{\pgfqpoint{1.139932in}{0.654826in}}%
\pgfpathlineto{\pgfqpoint{1.142543in}{0.655917in}}%
\pgfpathlineto{\pgfqpoint{1.143386in}{0.655192in}}%
\pgfpathlineto{\pgfqpoint{1.144060in}{0.669336in}}%
\pgfpathlineto{\pgfqpoint{1.144565in}{0.654848in}}%
\pgfpathlineto{\pgfqpoint{1.144902in}{0.667778in}}%
\pgfpathlineto{\pgfqpoint{1.146250in}{0.710684in}}%
\pgfpathlineto{\pgfqpoint{1.145745in}{0.654949in}}%
\pgfpathlineto{\pgfqpoint{1.146334in}{0.701568in}}%
\pgfpathlineto{\pgfqpoint{1.146756in}{0.654926in}}%
\pgfpathlineto{\pgfqpoint{1.147261in}{0.704709in}}%
\pgfpathlineto{\pgfqpoint{1.147682in}{0.656332in}}%
\pgfpathlineto{\pgfqpoint{1.147851in}{0.654471in}}%
\pgfpathlineto{\pgfqpoint{1.148104in}{0.665313in}}%
\pgfpathlineto{\pgfqpoint{1.148441in}{0.709303in}}%
\pgfpathlineto{\pgfqpoint{1.148862in}{0.654515in}}%
\pgfpathlineto{\pgfqpoint{1.149283in}{0.681803in}}%
\pgfpathlineto{\pgfqpoint{1.149451in}{0.675669in}}%
\pgfpathlineto{\pgfqpoint{1.149873in}{0.653751in}}%
\pgfpathlineto{\pgfqpoint{1.150799in}{0.653958in}}%
\pgfpathlineto{\pgfqpoint{1.150968in}{0.653273in}}%
\pgfpathlineto{\pgfqpoint{1.151136in}{0.656459in}}%
\pgfpathlineto{\pgfqpoint{1.151389in}{0.666760in}}%
\pgfpathlineto{\pgfqpoint{1.151810in}{0.652994in}}%
\pgfpathlineto{\pgfqpoint{1.152232in}{0.655948in}}%
\pgfpathlineto{\pgfqpoint{1.152653in}{0.652942in}}%
\pgfpathlineto{\pgfqpoint{1.152906in}{0.655705in}}%
\pgfpathlineto{\pgfqpoint{1.154001in}{0.680256in}}%
\pgfpathlineto{\pgfqpoint{1.153495in}{0.652921in}}%
\pgfpathlineto{\pgfqpoint{1.154169in}{0.662786in}}%
\pgfpathlineto{\pgfqpoint{1.155264in}{0.652724in}}%
\pgfpathlineto{\pgfqpoint{1.154843in}{0.672184in}}%
\pgfpathlineto{\pgfqpoint{1.155349in}{0.652775in}}%
\pgfpathlineto{\pgfqpoint{1.155770in}{0.663975in}}%
\pgfpathlineto{\pgfqpoint{1.156360in}{0.652523in}}%
\pgfpathlineto{\pgfqpoint{1.156444in}{0.652484in}}%
\pgfpathlineto{\pgfqpoint{1.156528in}{0.652839in}}%
\pgfpathlineto{\pgfqpoint{1.156949in}{0.675051in}}%
\pgfpathlineto{\pgfqpoint{1.157539in}{0.651459in}}%
\pgfpathlineto{\pgfqpoint{1.157960in}{0.653030in}}%
\pgfpathlineto{\pgfqpoint{1.158466in}{0.650923in}}%
\pgfpathlineto{\pgfqpoint{1.160151in}{0.651698in}}%
\pgfpathlineto{\pgfqpoint{1.160488in}{0.663786in}}%
\pgfpathlineto{\pgfqpoint{1.161414in}{0.680454in}}%
\pgfpathlineto{\pgfqpoint{1.160993in}{0.651745in}}%
\pgfpathlineto{\pgfqpoint{1.161583in}{0.664861in}}%
\pgfpathlineto{\pgfqpoint{1.161835in}{0.651551in}}%
\pgfpathlineto{\pgfqpoint{1.162762in}{0.651836in}}%
\pgfpathlineto{\pgfqpoint{1.163015in}{0.653614in}}%
\pgfpathlineto{\pgfqpoint{1.163268in}{0.651716in}}%
\pgfpathlineto{\pgfqpoint{1.163857in}{0.652099in}}%
\pgfpathlineto{\pgfqpoint{1.164868in}{0.654148in}}%
\pgfpathlineto{\pgfqpoint{1.165205in}{0.652979in}}%
\pgfpathlineto{\pgfqpoint{1.165542in}{0.673592in}}%
\pgfpathlineto{\pgfqpoint{1.165963in}{0.652330in}}%
\pgfpathlineto{\pgfqpoint{1.166385in}{0.656217in}}%
\pgfpathlineto{\pgfqpoint{1.166637in}{0.657357in}}%
\pgfpathlineto{\pgfqpoint{1.167059in}{0.652615in}}%
\pgfpathlineto{\pgfqpoint{1.167143in}{0.652306in}}%
\pgfpathlineto{\pgfqpoint{1.167311in}{0.655371in}}%
\pgfpathlineto{\pgfqpoint{1.167564in}{0.671764in}}%
\pgfpathlineto{\pgfqpoint{1.168070in}{0.652421in}}%
\pgfpathlineto{\pgfqpoint{1.168407in}{0.654978in}}%
\pgfpathlineto{\pgfqpoint{1.168743in}{0.665944in}}%
\pgfpathlineto{\pgfqpoint{1.169249in}{0.652469in}}%
\pgfpathlineto{\pgfqpoint{1.169670in}{0.664210in}}%
\pgfpathlineto{\pgfqpoint{1.170091in}{0.652306in}}%
\pgfpathlineto{\pgfqpoint{1.171018in}{0.652383in}}%
\pgfpathlineto{\pgfqpoint{1.173124in}{0.656490in}}%
\pgfpathlineto{\pgfqpoint{1.173208in}{0.657014in}}%
\pgfpathlineto{\pgfqpoint{1.173714in}{0.653743in}}%
\pgfpathlineto{\pgfqpoint{1.173798in}{0.653503in}}%
\pgfpathlineto{\pgfqpoint{1.173967in}{0.654407in}}%
\pgfpathlineto{\pgfqpoint{1.174388in}{0.676312in}}%
\pgfpathlineto{\pgfqpoint{1.174893in}{0.653148in}}%
\pgfpathlineto{\pgfqpoint{1.175483in}{0.665116in}}%
\pgfpathlineto{\pgfqpoint{1.176494in}{0.652520in}}%
\pgfpathlineto{\pgfqpoint{1.176073in}{0.670416in}}%
\pgfpathlineto{\pgfqpoint{1.176662in}{0.655047in}}%
\pgfpathlineto{\pgfqpoint{1.176915in}{0.663481in}}%
\pgfpathlineto{\pgfqpoint{1.177336in}{0.652435in}}%
\pgfpathlineto{\pgfqpoint{1.177842in}{0.661206in}}%
\pgfpathlineto{\pgfqpoint{1.178432in}{0.651659in}}%
\pgfpathlineto{\pgfqpoint{1.178853in}{0.659164in}}%
\pgfpathlineto{\pgfqpoint{1.179527in}{0.651692in}}%
\pgfpathlineto{\pgfqpoint{1.179864in}{0.739989in}}%
\pgfpathlineto{\pgfqpoint{1.180875in}{0.806229in}}%
\pgfpathlineto{\pgfqpoint{1.180453in}{0.653088in}}%
\pgfpathlineto{\pgfqpoint{1.180959in}{0.773333in}}%
\pgfpathlineto{\pgfqpoint{1.181380in}{0.653281in}}%
\pgfpathlineto{\pgfqpoint{1.182223in}{0.654049in}}%
\pgfpathlineto{\pgfqpoint{1.182307in}{0.652960in}}%
\pgfpathlineto{\pgfqpoint{1.182728in}{0.659356in}}%
\pgfpathlineto{\pgfqpoint{1.182812in}{0.661955in}}%
\pgfpathlineto{\pgfqpoint{1.183318in}{0.652833in}}%
\pgfpathlineto{\pgfqpoint{1.183571in}{0.653180in}}%
\pgfpathlineto{\pgfqpoint{1.184666in}{0.654111in}}%
\pgfpathlineto{\pgfqpoint{1.184834in}{0.653602in}}%
\pgfpathlineto{\pgfqpoint{1.186014in}{0.654032in}}%
\pgfpathlineto{\pgfqpoint{1.186856in}{0.654875in}}%
\pgfpathlineto{\pgfqpoint{1.187530in}{0.654377in}}%
\pgfpathlineto{\pgfqpoint{1.187783in}{0.680001in}}%
\pgfpathlineto{\pgfqpoint{1.188035in}{0.812276in}}%
\pgfpathlineto{\pgfqpoint{1.188457in}{0.655501in}}%
\pgfpathlineto{\pgfqpoint{1.188878in}{0.704940in}}%
\pgfpathlineto{\pgfqpoint{1.189468in}{0.654208in}}%
\pgfpathlineto{\pgfqpoint{1.190394in}{0.654643in}}%
\pgfpathlineto{\pgfqpoint{1.190900in}{0.654304in}}%
\pgfpathlineto{\pgfqpoint{1.191321in}{0.654694in}}%
\pgfpathlineto{\pgfqpoint{1.191658in}{0.666723in}}%
\pgfpathlineto{\pgfqpoint{1.192753in}{0.689627in}}%
\pgfpathlineto{\pgfqpoint{1.192248in}{0.654966in}}%
\pgfpathlineto{\pgfqpoint{1.192837in}{0.681932in}}%
\pgfpathlineto{\pgfqpoint{1.194101in}{0.654135in}}%
\pgfpathlineto{\pgfqpoint{1.195533in}{0.654947in}}%
\pgfpathlineto{\pgfqpoint{1.196628in}{0.709882in}}%
\pgfpathlineto{\pgfqpoint{1.196797in}{0.760268in}}%
\pgfpathlineto{\pgfqpoint{1.197302in}{0.654624in}}%
\pgfpathlineto{\pgfqpoint{1.197555in}{0.659393in}}%
\pgfpathlineto{\pgfqpoint{1.197808in}{0.670586in}}%
\pgfpathlineto{\pgfqpoint{1.198145in}{0.654440in}}%
\pgfpathlineto{\pgfqpoint{1.198650in}{0.658976in}}%
\pgfpathlineto{\pgfqpoint{1.199830in}{0.654517in}}%
\pgfpathlineto{\pgfqpoint{1.199914in}{0.654544in}}%
\pgfpathlineto{\pgfqpoint{1.200251in}{0.660036in}}%
\pgfpathlineto{\pgfqpoint{1.200672in}{0.654356in}}%
\pgfpathlineto{\pgfqpoint{1.201093in}{0.654933in}}%
\pgfpathlineto{\pgfqpoint{1.201767in}{0.654321in}}%
\pgfpathlineto{\pgfqpoint{1.202357in}{0.654557in}}%
\pgfpathlineto{\pgfqpoint{1.202694in}{0.654315in}}%
\pgfpathlineto{\pgfqpoint{1.202947in}{0.655734in}}%
\pgfpathlineto{\pgfqpoint{1.203115in}{0.658054in}}%
\pgfpathlineto{\pgfqpoint{1.203621in}{0.654034in}}%
\pgfpathlineto{\pgfqpoint{1.204042in}{0.656344in}}%
\pgfpathlineto{\pgfqpoint{1.205053in}{0.654154in}}%
\pgfpathlineto{\pgfqpoint{1.205221in}{0.654660in}}%
\pgfpathlineto{\pgfqpoint{1.205811in}{0.654386in}}%
\pgfpathlineto{\pgfqpoint{1.206064in}{0.661712in}}%
\pgfpathlineto{\pgfqpoint{1.207243in}{0.911899in}}%
\pgfpathlineto{\pgfqpoint{1.206738in}{0.655804in}}%
\pgfpathlineto{\pgfqpoint{1.207412in}{0.772521in}}%
\pgfpathlineto{\pgfqpoint{1.207749in}{0.656553in}}%
\pgfpathlineto{\pgfqpoint{1.208591in}{0.657582in}}%
\pgfpathlineto{\pgfqpoint{1.209686in}{0.655129in}}%
\pgfpathlineto{\pgfqpoint{1.209181in}{0.666671in}}%
\pgfpathlineto{\pgfqpoint{1.209771in}{0.655166in}}%
\pgfpathlineto{\pgfqpoint{1.211455in}{0.656162in}}%
\pgfpathlineto{\pgfqpoint{1.211540in}{0.655910in}}%
\pgfpathlineto{\pgfqpoint{1.211792in}{0.655220in}}%
\pgfpathlineto{\pgfqpoint{1.211961in}{0.658812in}}%
\pgfpathlineto{\pgfqpoint{1.212214in}{0.677241in}}%
\pgfpathlineto{\pgfqpoint{1.212635in}{0.655046in}}%
\pgfpathlineto{\pgfqpoint{1.213056in}{0.659989in}}%
\pgfpathlineto{\pgfqpoint{1.213477in}{0.654989in}}%
\pgfpathlineto{\pgfqpoint{1.214657in}{0.655321in}}%
\pgfpathlineto{\pgfqpoint{1.215668in}{0.656916in}}%
\pgfpathlineto{\pgfqpoint{1.216931in}{0.672983in}}%
\pgfpathlineto{\pgfqpoint{1.216426in}{0.655367in}}%
\pgfpathlineto{\pgfqpoint{1.217100in}{0.665831in}}%
\pgfpathlineto{\pgfqpoint{1.218364in}{0.654878in}}%
\pgfpathlineto{\pgfqpoint{1.220554in}{0.655552in}}%
\pgfpathlineto{\pgfqpoint{1.221481in}{0.656875in}}%
\pgfpathlineto{\pgfqpoint{1.221059in}{0.654950in}}%
\pgfpathlineto{\pgfqpoint{1.221649in}{0.656336in}}%
\pgfpathlineto{\pgfqpoint{1.222913in}{0.654811in}}%
\pgfpathlineto{\pgfqpoint{1.224766in}{0.655622in}}%
\pgfpathlineto{\pgfqpoint{1.226114in}{0.677766in}}%
\pgfpathlineto{\pgfqpoint{1.225524in}{0.654654in}}%
\pgfpathlineto{\pgfqpoint{1.226198in}{0.669924in}}%
\pgfpathlineto{\pgfqpoint{1.227209in}{0.654338in}}%
\pgfpathlineto{\pgfqpoint{1.227378in}{0.654597in}}%
\pgfpathlineto{\pgfqpoint{1.228473in}{0.657567in}}%
\pgfpathlineto{\pgfqpoint{1.228052in}{0.654233in}}%
\pgfpathlineto{\pgfqpoint{1.228641in}{0.656751in}}%
\pgfpathlineto{\pgfqpoint{1.229315in}{0.654148in}}%
\pgfpathlineto{\pgfqpoint{1.229652in}{0.657148in}}%
\pgfpathlineto{\pgfqpoint{1.229737in}{0.658877in}}%
\pgfpathlineto{\pgfqpoint{1.230158in}{0.654270in}}%
\pgfpathlineto{\pgfqpoint{1.230579in}{0.654302in}}%
\pgfpathlineto{\pgfqpoint{1.230832in}{0.655271in}}%
\pgfpathlineto{\pgfqpoint{1.232011in}{0.687219in}}%
\pgfpathlineto{\pgfqpoint{1.231506in}{0.654197in}}%
\pgfpathlineto{\pgfqpoint{1.232180in}{0.671925in}}%
\pgfpathlineto{\pgfqpoint{1.233359in}{0.653904in}}%
\pgfpathlineto{\pgfqpoint{1.234875in}{0.655055in}}%
\pgfpathlineto{\pgfqpoint{1.234033in}{0.653389in}}%
\pgfpathlineto{\pgfqpoint{1.235212in}{0.654802in}}%
\pgfpathlineto{\pgfqpoint{1.235549in}{0.653864in}}%
\pgfpathlineto{\pgfqpoint{1.235971in}{0.655466in}}%
\pgfpathlineto{\pgfqpoint{1.236055in}{0.655481in}}%
\pgfpathlineto{\pgfqpoint{1.236308in}{0.656428in}}%
\pgfpathlineto{\pgfqpoint{1.236645in}{0.654176in}}%
\pgfpathlineto{\pgfqpoint{1.236982in}{0.653497in}}%
\pgfpathlineto{\pgfqpoint{1.237150in}{0.655278in}}%
\pgfpathlineto{\pgfqpoint{1.237319in}{0.660801in}}%
\pgfpathlineto{\pgfqpoint{1.237908in}{0.653421in}}%
\pgfpathlineto{\pgfqpoint{1.238245in}{0.655856in}}%
\pgfpathlineto{\pgfqpoint{1.238414in}{0.654804in}}%
\pgfpathlineto{\pgfqpoint{1.238582in}{0.653622in}}%
\pgfpathlineto{\pgfqpoint{1.239593in}{0.653963in}}%
\pgfpathlineto{\pgfqpoint{1.240436in}{0.654443in}}%
\pgfpathlineto{\pgfqpoint{1.240688in}{0.674335in}}%
\pgfpathlineto{\pgfqpoint{1.241615in}{0.888745in}}%
\pgfpathlineto{\pgfqpoint{1.241194in}{0.657222in}}%
\pgfpathlineto{\pgfqpoint{1.241868in}{0.740336in}}%
\pgfpathlineto{\pgfqpoint{1.242963in}{0.654912in}}%
\pgfpathlineto{\pgfqpoint{1.243047in}{0.654957in}}%
\pgfpathlineto{\pgfqpoint{1.244732in}{0.656133in}}%
\pgfpathlineto{\pgfqpoint{1.244816in}{0.656082in}}%
\pgfpathlineto{\pgfqpoint{1.245069in}{0.655381in}}%
\pgfpathlineto{\pgfqpoint{1.245238in}{0.657142in}}%
\pgfpathlineto{\pgfqpoint{1.245574in}{0.686059in}}%
\pgfpathlineto{\pgfqpoint{1.246164in}{0.655432in}}%
\pgfpathlineto{\pgfqpoint{1.246248in}{0.655472in}}%
\pgfpathlineto{\pgfqpoint{1.246585in}{0.658082in}}%
\pgfpathlineto{\pgfqpoint{1.246922in}{0.672152in}}%
\pgfpathlineto{\pgfqpoint{1.247344in}{0.654925in}}%
\pgfpathlineto{\pgfqpoint{1.247596in}{0.655789in}}%
\pgfpathlineto{\pgfqpoint{1.249113in}{0.656288in}}%
\pgfpathlineto{\pgfqpoint{1.249281in}{0.657972in}}%
\pgfpathlineto{\pgfqpoint{1.249702in}{0.655197in}}%
\pgfpathlineto{\pgfqpoint{1.250208in}{0.657194in}}%
\pgfpathlineto{\pgfqpoint{1.250966in}{0.658816in}}%
\pgfpathlineto{\pgfqpoint{1.251556in}{0.655193in}}%
\pgfpathlineto{\pgfqpoint{1.252314in}{0.656223in}}%
\pgfpathlineto{\pgfqpoint{1.252483in}{0.656705in}}%
\pgfpathlineto{\pgfqpoint{1.252988in}{0.655198in}}%
\pgfpathlineto{\pgfqpoint{1.253241in}{0.655182in}}%
\pgfpathlineto{\pgfqpoint{1.253662in}{0.655830in}}%
\pgfpathlineto{\pgfqpoint{1.253830in}{0.656804in}}%
\pgfpathlineto{\pgfqpoint{1.254252in}{0.655200in}}%
\pgfpathlineto{\pgfqpoint{1.254757in}{0.655821in}}%
\pgfpathlineto{\pgfqpoint{1.256021in}{0.655326in}}%
\pgfpathlineto{\pgfqpoint{1.256611in}{0.658061in}}%
\pgfpathlineto{\pgfqpoint{1.257116in}{0.655161in}}%
\pgfpathlineto{\pgfqpoint{1.257453in}{0.655894in}}%
\pgfpathlineto{\pgfqpoint{1.258295in}{0.655358in}}%
\pgfpathlineto{\pgfqpoint{1.258548in}{0.671416in}}%
\pgfpathlineto{\pgfqpoint{1.258717in}{0.704712in}}%
\pgfpathlineto{\pgfqpoint{1.259138in}{0.653935in}}%
\pgfpathlineto{\pgfqpoint{1.259559in}{0.670663in}}%
\pgfpathlineto{\pgfqpoint{1.260739in}{0.653696in}}%
\pgfpathlineto{\pgfqpoint{1.260823in}{0.653639in}}%
\pgfpathlineto{\pgfqpoint{1.260991in}{0.654786in}}%
\pgfpathlineto{\pgfqpoint{1.262002in}{0.676501in}}%
\pgfpathlineto{\pgfqpoint{1.261497in}{0.653436in}}%
\pgfpathlineto{\pgfqpoint{1.262171in}{0.664167in}}%
\pgfpathlineto{\pgfqpoint{1.262423in}{0.653852in}}%
\pgfpathlineto{\pgfqpoint{1.262592in}{0.673539in}}%
\pgfpathlineto{\pgfqpoint{1.262929in}{0.862905in}}%
\pgfpathlineto{\pgfqpoint{1.263434in}{0.656535in}}%
\pgfpathlineto{\pgfqpoint{1.263687in}{0.691925in}}%
\pgfpathlineto{\pgfqpoint{1.263771in}{0.694085in}}%
\pgfpathlineto{\pgfqpoint{1.263856in}{0.681528in}}%
\pgfpathlineto{\pgfqpoint{1.264193in}{0.654346in}}%
\pgfpathlineto{\pgfqpoint{1.265035in}{0.660890in}}%
\pgfpathlineto{\pgfqpoint{1.266214in}{0.654262in}}%
\pgfpathlineto{\pgfqpoint{1.265793in}{0.666991in}}%
\pgfpathlineto{\pgfqpoint{1.266299in}{0.654680in}}%
\pgfpathlineto{\pgfqpoint{1.266636in}{0.660610in}}%
\pgfpathlineto{\pgfqpoint{1.266973in}{0.654175in}}%
\pgfpathlineto{\pgfqpoint{1.267394in}{0.654716in}}%
\pgfpathlineto{\pgfqpoint{1.267899in}{0.654345in}}%
\pgfpathlineto{\pgfqpoint{1.268489in}{0.654424in}}%
\pgfpathlineto{\pgfqpoint{1.268742in}{0.654962in}}%
\pgfpathlineto{\pgfqpoint{1.269331in}{0.654419in}}%
\pgfpathlineto{\pgfqpoint{1.269584in}{0.663759in}}%
\pgfpathlineto{\pgfqpoint{1.270848in}{0.728492in}}%
\pgfpathlineto{\pgfqpoint{1.270258in}{0.654847in}}%
\pgfpathlineto{\pgfqpoint{1.270932in}{0.722214in}}%
\pgfpathlineto{\pgfqpoint{1.272280in}{0.653889in}}%
\pgfpathlineto{\pgfqpoint{1.272533in}{0.655155in}}%
\pgfpathlineto{\pgfqpoint{1.272785in}{0.658609in}}%
\pgfpathlineto{\pgfqpoint{1.273038in}{0.653619in}}%
\pgfpathlineto{\pgfqpoint{1.273544in}{0.653997in}}%
\pgfpathlineto{\pgfqpoint{1.273628in}{0.653739in}}%
\pgfpathlineto{\pgfqpoint{1.273881in}{0.654939in}}%
\pgfpathlineto{\pgfqpoint{1.275144in}{0.660609in}}%
\pgfpathlineto{\pgfqpoint{1.274639in}{0.653726in}}%
\pgfpathlineto{\pgfqpoint{1.275229in}{0.660491in}}%
\pgfpathlineto{\pgfqpoint{1.275650in}{0.653768in}}%
\pgfpathlineto{\pgfqpoint{1.276576in}{0.654038in}}%
\pgfpathlineto{\pgfqpoint{1.279609in}{0.655465in}}%
\pgfpathlineto{\pgfqpoint{1.280283in}{0.654884in}}%
\pgfpathlineto{\pgfqpoint{1.280536in}{0.683089in}}%
\pgfpathlineto{\pgfqpoint{1.280704in}{0.750745in}}%
\pgfpathlineto{\pgfqpoint{1.281210in}{0.654760in}}%
\pgfpathlineto{\pgfqpoint{1.281631in}{0.689564in}}%
\pgfpathlineto{\pgfqpoint{1.282052in}{0.654354in}}%
\pgfpathlineto{\pgfqpoint{1.282979in}{0.654394in}}%
\pgfpathlineto{\pgfqpoint{1.284495in}{0.655333in}}%
\pgfpathlineto{\pgfqpoint{1.285843in}{0.684216in}}%
\pgfpathlineto{\pgfqpoint{1.285338in}{0.654947in}}%
\pgfpathlineto{\pgfqpoint{1.286180in}{0.671546in}}%
\pgfpathlineto{\pgfqpoint{1.286602in}{0.654354in}}%
\pgfpathlineto{\pgfqpoint{1.287023in}{0.685095in}}%
\pgfpathlineto{\pgfqpoint{1.287697in}{0.653911in}}%
\pgfpathlineto{\pgfqpoint{1.289213in}{0.658775in}}%
\pgfpathlineto{\pgfqpoint{1.289382in}{0.676063in}}%
\pgfpathlineto{\pgfqpoint{1.289887in}{0.653782in}}%
\pgfpathlineto{\pgfqpoint{1.290224in}{0.654846in}}%
\pgfpathlineto{\pgfqpoint{1.291067in}{0.655033in}}%
\pgfpathlineto{\pgfqpoint{1.290730in}{0.653739in}}%
\pgfpathlineto{\pgfqpoint{1.291151in}{0.654887in}}%
\pgfpathlineto{\pgfqpoint{1.291656in}{0.653957in}}%
\pgfpathlineto{\pgfqpoint{1.292330in}{0.654514in}}%
\pgfpathlineto{\pgfqpoint{1.293510in}{0.658490in}}%
\pgfpathlineto{\pgfqpoint{1.293762in}{0.654951in}}%
\pgfpathlineto{\pgfqpoint{1.294858in}{0.653892in}}%
\pgfpathlineto{\pgfqpoint{1.295026in}{0.653925in}}%
\pgfpathlineto{\pgfqpoint{1.295447in}{0.654980in}}%
\pgfpathlineto{\pgfqpoint{1.296711in}{0.669173in}}%
\pgfpathlineto{\pgfqpoint{1.296374in}{0.654320in}}%
\pgfpathlineto{\pgfqpoint{1.296879in}{0.659427in}}%
\pgfpathlineto{\pgfqpoint{1.297806in}{0.653601in}}%
\pgfpathlineto{\pgfqpoint{1.298059in}{0.653742in}}%
\pgfpathlineto{\pgfqpoint{1.298227in}{0.655954in}}%
\pgfpathlineto{\pgfqpoint{1.299575in}{0.716383in}}%
\pgfpathlineto{\pgfqpoint{1.298986in}{0.655093in}}%
\pgfpathlineto{\pgfqpoint{1.299659in}{0.702524in}}%
\pgfpathlineto{\pgfqpoint{1.300502in}{0.711404in}}%
\pgfpathlineto{\pgfqpoint{1.300923in}{0.653552in}}%
\pgfpathlineto{\pgfqpoint{1.301092in}{0.656322in}}%
\pgfpathlineto{\pgfqpoint{1.301429in}{0.689280in}}%
\pgfpathlineto{\pgfqpoint{1.302018in}{0.653417in}}%
\pgfpathlineto{\pgfqpoint{1.302187in}{0.655443in}}%
\pgfpathlineto{\pgfqpoint{1.302440in}{0.668747in}}%
\pgfpathlineto{\pgfqpoint{1.302945in}{0.653092in}}%
\pgfpathlineto{\pgfqpoint{1.303366in}{0.659619in}}%
\pgfpathlineto{\pgfqpoint{1.304630in}{0.653388in}}%
\pgfpathlineto{\pgfqpoint{1.304967in}{0.653541in}}%
\pgfpathlineto{\pgfqpoint{1.305725in}{0.654536in}}%
\pgfpathlineto{\pgfqpoint{1.305978in}{0.656834in}}%
\pgfpathlineto{\pgfqpoint{1.306736in}{0.654364in}}%
\pgfpathlineto{\pgfqpoint{1.306820in}{0.654310in}}%
\pgfpathlineto{\pgfqpoint{1.306989in}{0.655160in}}%
\pgfpathlineto{\pgfqpoint{1.308252in}{0.676073in}}%
\pgfpathlineto{\pgfqpoint{1.307915in}{0.653724in}}%
\pgfpathlineto{\pgfqpoint{1.308421in}{0.663853in}}%
\pgfpathlineto{\pgfqpoint{1.308674in}{0.653412in}}%
\pgfpathlineto{\pgfqpoint{1.309095in}{0.670347in}}%
\pgfpathlineto{\pgfqpoint{1.309600in}{0.655050in}}%
\pgfpathlineto{\pgfqpoint{1.310022in}{0.653274in}}%
\pgfpathlineto{\pgfqpoint{1.310190in}{0.655194in}}%
\pgfpathlineto{\pgfqpoint{1.310527in}{0.731112in}}%
\pgfpathlineto{\pgfqpoint{1.311032in}{0.653061in}}%
\pgfpathlineto{\pgfqpoint{1.311369in}{0.662510in}}%
\pgfpathlineto{\pgfqpoint{1.311454in}{0.662843in}}%
\pgfpathlineto{\pgfqpoint{1.311538in}{0.660829in}}%
\pgfpathlineto{\pgfqpoint{1.312128in}{0.652414in}}%
\pgfpathlineto{\pgfqpoint{1.312802in}{0.653747in}}%
\pgfpathlineto{\pgfqpoint{1.312970in}{0.652955in}}%
\pgfpathlineto{\pgfqpoint{1.313139in}{0.658759in}}%
\pgfpathlineto{\pgfqpoint{1.313476in}{0.761321in}}%
\pgfpathlineto{\pgfqpoint{1.313981in}{0.654291in}}%
\pgfpathlineto{\pgfqpoint{1.314402in}{0.704255in}}%
\pgfpathlineto{\pgfqpoint{1.314908in}{0.652747in}}%
\pgfpathlineto{\pgfqpoint{1.315750in}{0.653059in}}%
\pgfpathlineto{\pgfqpoint{1.318446in}{0.654315in}}%
\pgfpathlineto{\pgfqpoint{1.319710in}{0.659916in}}%
\pgfpathlineto{\pgfqpoint{1.319878in}{0.663606in}}%
\pgfpathlineto{\pgfqpoint{1.320299in}{0.654606in}}%
\pgfpathlineto{\pgfqpoint{1.320721in}{0.660966in}}%
\pgfpathlineto{\pgfqpoint{1.321984in}{0.654383in}}%
\pgfpathlineto{\pgfqpoint{1.322069in}{0.654411in}}%
\pgfpathlineto{\pgfqpoint{1.322406in}{0.655246in}}%
\pgfpathlineto{\pgfqpoint{1.323332in}{0.654450in}}%
\pgfpathlineto{\pgfqpoint{1.325607in}{0.655556in}}%
\pgfpathlineto{\pgfqpoint{1.326955in}{0.659826in}}%
\pgfpathlineto{\pgfqpoint{1.326281in}{0.655213in}}%
\pgfpathlineto{\pgfqpoint{1.327039in}{0.659561in}}%
\pgfpathlineto{\pgfqpoint{1.327460in}{0.655276in}}%
\pgfpathlineto{\pgfqpoint{1.327713in}{0.660826in}}%
\pgfpathlineto{\pgfqpoint{1.328892in}{0.722463in}}%
\pgfpathlineto{\pgfqpoint{1.328471in}{0.655293in}}%
\pgfpathlineto{\pgfqpoint{1.329061in}{0.693247in}}%
\pgfpathlineto{\pgfqpoint{1.329988in}{0.655345in}}%
\pgfpathlineto{\pgfqpoint{1.330240in}{0.657700in}}%
\pgfpathlineto{\pgfqpoint{1.331588in}{0.692064in}}%
\pgfpathlineto{\pgfqpoint{1.330830in}{0.654835in}}%
\pgfpathlineto{\pgfqpoint{1.331672in}{0.682011in}}%
\pgfpathlineto{\pgfqpoint{1.331925in}{0.654420in}}%
\pgfpathlineto{\pgfqpoint{1.332346in}{0.762324in}}%
\pgfpathlineto{\pgfqpoint{1.334115in}{0.654522in}}%
\pgfpathlineto{\pgfqpoint{1.334789in}{0.656362in}}%
\pgfpathlineto{\pgfqpoint{1.335632in}{0.653584in}}%
\pgfpathlineto{\pgfqpoint{1.336053in}{0.653700in}}%
\pgfpathlineto{\pgfqpoint{1.336306in}{0.655287in}}%
\pgfpathlineto{\pgfqpoint{1.337570in}{0.664779in}}%
\pgfpathlineto{\pgfqpoint{1.336980in}{0.653724in}}%
\pgfpathlineto{\pgfqpoint{1.337654in}{0.662993in}}%
\pgfpathlineto{\pgfqpoint{1.338075in}{0.653913in}}%
\pgfpathlineto{\pgfqpoint{1.338328in}{0.668758in}}%
\pgfpathlineto{\pgfqpoint{1.338496in}{0.679278in}}%
\pgfpathlineto{\pgfqpoint{1.339086in}{0.653419in}}%
\pgfpathlineto{\pgfqpoint{1.339254in}{0.653898in}}%
\pgfpathlineto{\pgfqpoint{1.339423in}{0.654050in}}%
\pgfpathlineto{\pgfqpoint{1.340097in}{0.653337in}}%
\pgfpathlineto{\pgfqpoint{1.340181in}{0.653330in}}%
\pgfpathlineto{\pgfqpoint{1.340518in}{0.654448in}}%
\pgfpathlineto{\pgfqpoint{1.341613in}{0.674980in}}%
\pgfpathlineto{\pgfqpoint{1.341108in}{0.653206in}}%
\pgfpathlineto{\pgfqpoint{1.341866in}{0.658654in}}%
\pgfpathlineto{\pgfqpoint{1.342119in}{0.652739in}}%
\pgfpathlineto{\pgfqpoint{1.343045in}{0.653408in}}%
\pgfpathlineto{\pgfqpoint{1.343635in}{0.652621in}}%
\pgfpathlineto{\pgfqpoint{1.343888in}{0.654128in}}%
\pgfpathlineto{\pgfqpoint{1.343972in}{0.654161in}}%
\pgfpathlineto{\pgfqpoint{1.344056in}{0.653526in}}%
\pgfpathlineto{\pgfqpoint{1.344309in}{0.652872in}}%
\pgfpathlineto{\pgfqpoint{1.345152in}{0.653458in}}%
\pgfpathlineto{\pgfqpoint{1.346162in}{0.654343in}}%
\pgfpathlineto{\pgfqpoint{1.346752in}{0.653761in}}%
\pgfpathlineto{\pgfqpoint{1.347005in}{0.668929in}}%
\pgfpathlineto{\pgfqpoint{1.347173in}{0.697023in}}%
\pgfpathlineto{\pgfqpoint{1.347595in}{0.653814in}}%
\pgfpathlineto{\pgfqpoint{1.348100in}{0.669528in}}%
\pgfpathlineto{\pgfqpoint{1.349364in}{0.653526in}}%
\pgfpathlineto{\pgfqpoint{1.351301in}{0.654090in}}%
\pgfpathlineto{\pgfqpoint{1.352481in}{0.656849in}}%
\pgfpathlineto{\pgfqpoint{1.353576in}{0.844329in}}%
\pgfpathlineto{\pgfqpoint{1.353155in}{0.655660in}}%
\pgfpathlineto{\pgfqpoint{1.353829in}{0.706394in}}%
\pgfpathlineto{\pgfqpoint{1.354166in}{0.655554in}}%
\pgfpathlineto{\pgfqpoint{1.354587in}{0.713381in}}%
\pgfpathlineto{\pgfqpoint{1.355008in}{0.656661in}}%
\pgfpathlineto{\pgfqpoint{1.355177in}{0.655077in}}%
\pgfpathlineto{\pgfqpoint{1.355598in}{0.660913in}}%
\pgfpathlineto{\pgfqpoint{1.355851in}{0.666992in}}%
\pgfpathlineto{\pgfqpoint{1.356356in}{0.654917in}}%
\pgfpathlineto{\pgfqpoint{1.356777in}{0.663774in}}%
\pgfpathlineto{\pgfqpoint{1.357283in}{0.654582in}}%
\pgfpathlineto{\pgfqpoint{1.358125in}{0.654707in}}%
\pgfpathlineto{\pgfqpoint{1.359473in}{0.655544in}}%
\pgfpathlineto{\pgfqpoint{1.360484in}{0.660257in}}%
\pgfpathlineto{\pgfqpoint{1.360147in}{0.655109in}}%
\pgfpathlineto{\pgfqpoint{1.360653in}{0.657706in}}%
\pgfpathlineto{\pgfqpoint{1.361832in}{0.654949in}}%
\pgfpathlineto{\pgfqpoint{1.363938in}{0.655230in}}%
\pgfpathlineto{\pgfqpoint{1.364275in}{0.666169in}}%
\pgfpathlineto{\pgfqpoint{1.364696in}{0.655047in}}%
\pgfpathlineto{\pgfqpoint{1.365033in}{0.655186in}}%
\pgfpathlineto{\pgfqpoint{1.367308in}{0.656795in}}%
\pgfpathlineto{\pgfqpoint{1.368487in}{0.681581in}}%
\pgfpathlineto{\pgfqpoint{1.368066in}{0.655406in}}%
\pgfpathlineto{\pgfqpoint{1.368572in}{0.675072in}}%
\pgfpathlineto{\pgfqpoint{1.368993in}{0.654681in}}%
\pgfpathlineto{\pgfqpoint{1.369751in}{0.655316in}}%
\pgfpathlineto{\pgfqpoint{1.370425in}{0.655203in}}%
\pgfpathlineto{\pgfqpoint{1.370678in}{0.709637in}}%
\pgfpathlineto{\pgfqpoint{1.370846in}{0.736705in}}%
\pgfpathlineto{\pgfqpoint{1.371267in}{0.654587in}}%
\pgfpathlineto{\pgfqpoint{1.371773in}{0.711467in}}%
\pgfpathlineto{\pgfqpoint{1.372278in}{0.654647in}}%
\pgfpathlineto{\pgfqpoint{1.373036in}{0.668434in}}%
\pgfpathlineto{\pgfqpoint{1.373121in}{0.669572in}}%
\pgfpathlineto{\pgfqpoint{1.373458in}{0.659711in}}%
\pgfpathlineto{\pgfqpoint{1.373795in}{0.655086in}}%
\pgfpathlineto{\pgfqpoint{1.374216in}{0.665158in}}%
\pgfpathlineto{\pgfqpoint{1.374469in}{0.686718in}}%
\pgfpathlineto{\pgfqpoint{1.374890in}{0.653969in}}%
\pgfpathlineto{\pgfqpoint{1.375227in}{0.661301in}}%
\pgfpathlineto{\pgfqpoint{1.376490in}{0.653499in}}%
\pgfpathlineto{\pgfqpoint{1.376659in}{0.653822in}}%
\pgfpathlineto{\pgfqpoint{1.377838in}{0.657054in}}%
\pgfpathlineto{\pgfqpoint{1.377333in}{0.653572in}}%
\pgfpathlineto{\pgfqpoint{1.377923in}{0.656654in}}%
\pgfpathlineto{\pgfqpoint{1.379018in}{0.653473in}}%
\pgfpathlineto{\pgfqpoint{1.379186in}{0.653512in}}%
\pgfpathlineto{\pgfqpoint{1.379523in}{0.653944in}}%
\pgfpathlineto{\pgfqpoint{1.380534in}{0.678673in}}%
\pgfpathlineto{\pgfqpoint{1.380787in}{0.766126in}}%
\pgfpathlineto{\pgfqpoint{1.381377in}{0.654344in}}%
\pgfpathlineto{\pgfqpoint{1.381545in}{0.660079in}}%
\pgfpathlineto{\pgfqpoint{1.381714in}{0.678225in}}%
\pgfpathlineto{\pgfqpoint{1.382556in}{0.653134in}}%
\pgfpathlineto{\pgfqpoint{1.383567in}{0.654221in}}%
\pgfpathlineto{\pgfqpoint{1.384746in}{0.669591in}}%
\pgfpathlineto{\pgfqpoint{1.384241in}{0.653123in}}%
\pgfpathlineto{\pgfqpoint{1.384915in}{0.657347in}}%
\pgfpathlineto{\pgfqpoint{1.385083in}{0.653160in}}%
\pgfpathlineto{\pgfqpoint{1.385505in}{0.661923in}}%
\pgfpathlineto{\pgfqpoint{1.386010in}{0.653458in}}%
\pgfpathlineto{\pgfqpoint{1.387274in}{0.664940in}}%
\pgfpathlineto{\pgfqpoint{1.386853in}{0.653103in}}%
\pgfpathlineto{\pgfqpoint{1.387442in}{0.658608in}}%
\pgfpathlineto{\pgfqpoint{1.387779in}{0.653034in}}%
\pgfpathlineto{\pgfqpoint{1.388622in}{0.653276in}}%
\pgfpathlineto{\pgfqpoint{1.389970in}{0.654608in}}%
\pgfpathlineto{\pgfqpoint{1.390138in}{0.655120in}}%
\pgfpathlineto{\pgfqpoint{1.390644in}{0.654253in}}%
\pgfpathlineto{\pgfqpoint{1.390981in}{0.654374in}}%
\pgfpathlineto{\pgfqpoint{1.391318in}{0.654734in}}%
\pgfpathlineto{\pgfqpoint{1.391570in}{0.681959in}}%
\pgfpathlineto{\pgfqpoint{1.391739in}{0.709133in}}%
\pgfpathlineto{\pgfqpoint{1.392244in}{0.655141in}}%
\pgfpathlineto{\pgfqpoint{1.392750in}{0.697195in}}%
\pgfpathlineto{\pgfqpoint{1.394013in}{0.654261in}}%
\pgfpathlineto{\pgfqpoint{1.394856in}{0.655250in}}%
\pgfpathlineto{\pgfqpoint{1.395361in}{0.654466in}}%
\pgfpathlineto{\pgfqpoint{1.395614in}{0.661556in}}%
\pgfpathlineto{\pgfqpoint{1.395867in}{0.673694in}}%
\pgfpathlineto{\pgfqpoint{1.396456in}{0.654179in}}%
\pgfpathlineto{\pgfqpoint{1.396625in}{0.655637in}}%
\pgfpathlineto{\pgfqpoint{1.396962in}{0.675329in}}%
\pgfpathlineto{\pgfqpoint{1.397383in}{0.653913in}}%
\pgfpathlineto{\pgfqpoint{1.397636in}{0.654257in}}%
\pgfpathlineto{\pgfqpoint{1.397973in}{0.653624in}}%
\pgfpathlineto{\pgfqpoint{1.398310in}{0.654614in}}%
\pgfpathlineto{\pgfqpoint{1.398647in}{0.654243in}}%
\pgfpathlineto{\pgfqpoint{1.399910in}{0.669453in}}%
\pgfpathlineto{\pgfqpoint{1.399489in}{0.653963in}}%
\pgfpathlineto{\pgfqpoint{1.400079in}{0.663262in}}%
\pgfpathlineto{\pgfqpoint{1.401174in}{0.653476in}}%
\pgfpathlineto{\pgfqpoint{1.400837in}{0.666474in}}%
\pgfpathlineto{\pgfqpoint{1.401258in}{0.653620in}}%
\pgfpathlineto{\pgfqpoint{1.401595in}{0.664835in}}%
\pgfpathlineto{\pgfqpoint{1.402017in}{0.653575in}}%
\pgfpathlineto{\pgfqpoint{1.402438in}{0.655278in}}%
\pgfpathlineto{\pgfqpoint{1.402691in}{0.653694in}}%
\pgfpathlineto{\pgfqpoint{1.402943in}{0.660964in}}%
\pgfpathlineto{\pgfqpoint{1.403786in}{0.669020in}}%
\pgfpathlineto{\pgfqpoint{1.403364in}{0.653795in}}%
\pgfpathlineto{\pgfqpoint{1.403870in}{0.665003in}}%
\pgfpathlineto{\pgfqpoint{1.404207in}{0.653760in}}%
\pgfpathlineto{\pgfqpoint{1.405049in}{0.653803in}}%
\pgfpathlineto{\pgfqpoint{1.405218in}{0.656027in}}%
\pgfpathlineto{\pgfqpoint{1.406313in}{0.681751in}}%
\pgfpathlineto{\pgfqpoint{1.405808in}{0.653653in}}%
\pgfpathlineto{\pgfqpoint{1.406482in}{0.668164in}}%
\pgfpathlineto{\pgfqpoint{1.407577in}{0.652998in}}%
\pgfpathlineto{\pgfqpoint{1.407661in}{0.653033in}}%
\pgfpathlineto{\pgfqpoint{1.408672in}{0.653777in}}%
\pgfpathlineto{\pgfqpoint{1.409093in}{0.684132in}}%
\pgfpathlineto{\pgfqpoint{1.409599in}{0.652857in}}%
\pgfpathlineto{\pgfqpoint{1.409767in}{0.682677in}}%
\pgfpathlineto{\pgfqpoint{1.409936in}{0.799752in}}%
\pgfpathlineto{\pgfqpoint{1.410441in}{0.653293in}}%
\pgfpathlineto{\pgfqpoint{1.410778in}{0.667721in}}%
\pgfpathlineto{\pgfqpoint{1.411368in}{0.652511in}}%
\pgfpathlineto{\pgfqpoint{1.412294in}{0.655739in}}%
\pgfpathlineto{\pgfqpoint{1.412800in}{0.653143in}}%
\pgfpathlineto{\pgfqpoint{1.413053in}{0.662806in}}%
\pgfpathlineto{\pgfqpoint{1.414316in}{0.746667in}}%
\pgfpathlineto{\pgfqpoint{1.413895in}{0.653074in}}%
\pgfpathlineto{\pgfqpoint{1.414401in}{0.724730in}}%
\pgfpathlineto{\pgfqpoint{1.414738in}{0.653461in}}%
\pgfpathlineto{\pgfqpoint{1.415159in}{0.800154in}}%
\pgfpathlineto{\pgfqpoint{1.415664in}{0.657775in}}%
\pgfpathlineto{\pgfqpoint{1.415833in}{0.655452in}}%
\pgfpathlineto{\pgfqpoint{1.416001in}{0.663417in}}%
\pgfpathlineto{\pgfqpoint{1.416338in}{0.710568in}}%
\pgfpathlineto{\pgfqpoint{1.417012in}{0.653769in}}%
\pgfpathlineto{\pgfqpoint{1.417096in}{0.653747in}}%
\pgfpathlineto{\pgfqpoint{1.417181in}{0.654327in}}%
\pgfpathlineto{\pgfqpoint{1.417433in}{0.663485in}}%
\pgfpathlineto{\pgfqpoint{1.417939in}{0.653349in}}%
\pgfpathlineto{\pgfqpoint{1.418192in}{0.653474in}}%
\pgfpathlineto{\pgfqpoint{1.420466in}{0.654014in}}%
\pgfpathlineto{\pgfqpoint{1.421393in}{0.655511in}}%
\pgfpathlineto{\pgfqpoint{1.422488in}{0.657002in}}%
\pgfpathlineto{\pgfqpoint{1.422067in}{0.654212in}}%
\pgfpathlineto{\pgfqpoint{1.422572in}{0.656921in}}%
\pgfpathlineto{\pgfqpoint{1.423499in}{0.654556in}}%
\pgfpathlineto{\pgfqpoint{1.423162in}{0.658073in}}%
\pgfpathlineto{\pgfqpoint{1.423583in}{0.655114in}}%
\pgfpathlineto{\pgfqpoint{1.424004in}{0.674160in}}%
\pgfpathlineto{\pgfqpoint{1.424510in}{0.654002in}}%
\pgfpathlineto{\pgfqpoint{1.424847in}{0.661503in}}%
\pgfpathlineto{\pgfqpoint{1.425942in}{0.653775in}}%
\pgfpathlineto{\pgfqpoint{1.426195in}{0.654734in}}%
\pgfpathlineto{\pgfqpoint{1.426363in}{0.655676in}}%
\pgfpathlineto{\pgfqpoint{1.426616in}{0.653927in}}%
\pgfpathlineto{\pgfqpoint{1.427206in}{0.654196in}}%
\pgfpathlineto{\pgfqpoint{1.428554in}{0.654949in}}%
\pgfpathlineto{\pgfqpoint{1.428806in}{0.682035in}}%
\pgfpathlineto{\pgfqpoint{1.429902in}{0.766455in}}%
\pgfpathlineto{\pgfqpoint{1.429480in}{0.656145in}}%
\pgfpathlineto{\pgfqpoint{1.429986in}{0.751459in}}%
\pgfpathlineto{\pgfqpoint{1.430997in}{0.654544in}}%
\pgfpathlineto{\pgfqpoint{1.431249in}{0.654982in}}%
\pgfpathlineto{\pgfqpoint{1.432092in}{0.658019in}}%
\pgfpathlineto{\pgfqpoint{1.431586in}{0.654774in}}%
\pgfpathlineto{\pgfqpoint{1.432345in}{0.655008in}}%
\pgfpathlineto{\pgfqpoint{1.432429in}{0.654878in}}%
\pgfpathlineto{\pgfqpoint{1.432597in}{0.655512in}}%
\pgfpathlineto{\pgfqpoint{1.433271in}{0.655057in}}%
\pgfpathlineto{\pgfqpoint{1.433524in}{0.685044in}}%
\pgfpathlineto{\pgfqpoint{1.433693in}{0.715025in}}%
\pgfpathlineto{\pgfqpoint{1.434198in}{0.655399in}}%
\pgfpathlineto{\pgfqpoint{1.434535in}{0.664814in}}%
\pgfpathlineto{\pgfqpoint{1.434619in}{0.666666in}}%
\pgfpathlineto{\pgfqpoint{1.435125in}{0.654815in}}%
\pgfpathlineto{\pgfqpoint{1.435377in}{0.655836in}}%
\pgfpathlineto{\pgfqpoint{1.435883in}{0.654695in}}%
\pgfpathlineto{\pgfqpoint{1.436304in}{0.671134in}}%
\pgfpathlineto{\pgfqpoint{1.437147in}{0.698154in}}%
\pgfpathlineto{\pgfqpoint{1.436725in}{0.654728in}}%
\pgfpathlineto{\pgfqpoint{1.437315in}{0.669670in}}%
\pgfpathlineto{\pgfqpoint{1.438326in}{0.653855in}}%
\pgfpathlineto{\pgfqpoint{1.437905in}{0.673295in}}%
\pgfpathlineto{\pgfqpoint{1.438494in}{0.659536in}}%
\pgfpathlineto{\pgfqpoint{1.438663in}{0.668493in}}%
\pgfpathlineto{\pgfqpoint{1.439168in}{0.653582in}}%
\pgfpathlineto{\pgfqpoint{1.439505in}{0.654880in}}%
\pgfpathlineto{\pgfqpoint{1.439590in}{0.655274in}}%
\pgfpathlineto{\pgfqpoint{1.440179in}{0.653692in}}%
\pgfpathlineto{\pgfqpoint{1.440348in}{0.653723in}}%
\pgfpathlineto{\pgfqpoint{1.443549in}{0.654977in}}%
\pgfpathlineto{\pgfqpoint{1.443718in}{0.656302in}}%
\pgfpathlineto{\pgfqpoint{1.444223in}{0.654335in}}%
\pgfpathlineto{\pgfqpoint{1.444560in}{0.654341in}}%
\pgfpathlineto{\pgfqpoint{1.445318in}{0.655121in}}%
\pgfpathlineto{\pgfqpoint{1.445487in}{0.656145in}}%
\pgfpathlineto{\pgfqpoint{1.446076in}{0.654538in}}%
\pgfpathlineto{\pgfqpoint{1.446329in}{0.654682in}}%
\pgfpathlineto{\pgfqpoint{1.446750in}{0.655987in}}%
\pgfpathlineto{\pgfqpoint{1.446919in}{0.657355in}}%
\pgfpathlineto{\pgfqpoint{1.447424in}{0.654290in}}%
\pgfpathlineto{\pgfqpoint{1.447761in}{0.655807in}}%
\pgfpathlineto{\pgfqpoint{1.448857in}{0.654413in}}%
\pgfpathlineto{\pgfqpoint{1.449194in}{0.655456in}}%
\pgfpathlineto{\pgfqpoint{1.453069in}{0.655405in}}%
\pgfpathlineto{\pgfqpoint{1.454585in}{0.658006in}}%
\pgfpathlineto{\pgfqpoint{1.455343in}{0.656379in}}%
\pgfpathlineto{\pgfqpoint{1.455596in}{0.699193in}}%
\pgfpathlineto{\pgfqpoint{1.455849in}{1.081536in}}%
\pgfpathlineto{\pgfqpoint{1.456439in}{0.657092in}}%
\pgfpathlineto{\pgfqpoint{1.456691in}{0.711738in}}%
\pgfpathlineto{\pgfqpoint{1.456860in}{0.738809in}}%
\pgfpathlineto{\pgfqpoint{1.457281in}{0.655527in}}%
\pgfpathlineto{\pgfqpoint{1.457702in}{0.680807in}}%
\pgfpathlineto{\pgfqpoint{1.458882in}{0.655082in}}%
\pgfpathlineto{\pgfqpoint{1.458966in}{0.655110in}}%
\pgfpathlineto{\pgfqpoint{1.459219in}{0.657065in}}%
\pgfpathlineto{\pgfqpoint{1.459640in}{0.654943in}}%
\pgfpathlineto{\pgfqpoint{1.460230in}{0.655879in}}%
\pgfpathlineto{\pgfqpoint{1.461493in}{0.655169in}}%
\pgfpathlineto{\pgfqpoint{1.462841in}{0.656336in}}%
\pgfpathlineto{\pgfqpoint{1.463684in}{0.680176in}}%
\pgfpathlineto{\pgfqpoint{1.463262in}{0.655487in}}%
\pgfpathlineto{\pgfqpoint{1.463936in}{0.659989in}}%
\pgfpathlineto{\pgfqpoint{1.464863in}{0.655257in}}%
\pgfpathlineto{\pgfqpoint{1.464526in}{0.663480in}}%
\pgfpathlineto{\pgfqpoint{1.465116in}{0.656453in}}%
\pgfpathlineto{\pgfqpoint{1.465284in}{0.657563in}}%
\pgfpathlineto{\pgfqpoint{1.465537in}{0.655207in}}%
\pgfpathlineto{\pgfqpoint{1.466127in}{0.655224in}}%
\pgfpathlineto{\pgfqpoint{1.466379in}{0.656742in}}%
\pgfpathlineto{\pgfqpoint{1.467306in}{0.664920in}}%
\pgfpathlineto{\pgfqpoint{1.466801in}{0.655252in}}%
\pgfpathlineto{\pgfqpoint{1.467559in}{0.658475in}}%
\pgfpathlineto{\pgfqpoint{1.467812in}{0.655144in}}%
\pgfpathlineto{\pgfqpoint{1.468149in}{0.664758in}}%
\pgfpathlineto{\pgfqpoint{1.468317in}{0.673900in}}%
\pgfpathlineto{\pgfqpoint{1.468738in}{0.654739in}}%
\pgfpathlineto{\pgfqpoint{1.469075in}{0.656197in}}%
\pgfpathlineto{\pgfqpoint{1.470086in}{0.654586in}}%
\pgfpathlineto{\pgfqpoint{1.470255in}{0.654737in}}%
\pgfpathlineto{\pgfqpoint{1.471097in}{0.654382in}}%
\pgfpathlineto{\pgfqpoint{1.471434in}{0.668763in}}%
\pgfpathlineto{\pgfqpoint{1.472361in}{0.699576in}}%
\pgfpathlineto{\pgfqpoint{1.471940in}{0.654324in}}%
\pgfpathlineto{\pgfqpoint{1.472529in}{0.672393in}}%
\pgfpathlineto{\pgfqpoint{1.473287in}{0.679936in}}%
\pgfpathlineto{\pgfqpoint{1.473709in}{0.653687in}}%
\pgfpathlineto{\pgfqpoint{1.473793in}{0.653530in}}%
\pgfpathlineto{\pgfqpoint{1.473877in}{0.654161in}}%
\pgfpathlineto{\pgfqpoint{1.474972in}{0.748014in}}%
\pgfpathlineto{\pgfqpoint{1.475057in}{0.788224in}}%
\pgfpathlineto{\pgfqpoint{1.475731in}{0.655083in}}%
\pgfpathlineto{\pgfqpoint{1.475815in}{0.655786in}}%
\pgfpathlineto{\pgfqpoint{1.476068in}{0.661790in}}%
\pgfpathlineto{\pgfqpoint{1.476826in}{0.654556in}}%
\pgfpathlineto{\pgfqpoint{1.478089in}{0.653919in}}%
\pgfpathlineto{\pgfqpoint{1.478511in}{0.654982in}}%
\pgfpathlineto{\pgfqpoint{1.478848in}{0.668307in}}%
\pgfpathlineto{\pgfqpoint{1.478932in}{0.670922in}}%
\pgfpathlineto{\pgfqpoint{1.479437in}{0.653828in}}%
\pgfpathlineto{\pgfqpoint{1.479606in}{0.655229in}}%
\pgfpathlineto{\pgfqpoint{1.479859in}{0.673548in}}%
\pgfpathlineto{\pgfqpoint{1.480448in}{0.653133in}}%
\pgfpathlineto{\pgfqpoint{1.480617in}{0.653221in}}%
\pgfpathlineto{\pgfqpoint{1.482302in}{0.654999in}}%
\pgfpathlineto{\pgfqpoint{1.482554in}{0.658962in}}%
\pgfpathlineto{\pgfqpoint{1.483144in}{0.653535in}}%
\pgfpathlineto{\pgfqpoint{1.483397in}{0.654960in}}%
\pgfpathlineto{\pgfqpoint{1.483481in}{0.655096in}}%
\pgfpathlineto{\pgfqpoint{1.483734in}{0.653509in}}%
\pgfpathlineto{\pgfqpoint{1.483987in}{0.653338in}}%
\pgfpathlineto{\pgfqpoint{1.484323in}{0.653956in}}%
\pgfpathlineto{\pgfqpoint{1.485756in}{0.666471in}}%
\pgfpathlineto{\pgfqpoint{1.485082in}{0.653701in}}%
\pgfpathlineto{\pgfqpoint{1.485924in}{0.661201in}}%
\pgfpathlineto{\pgfqpoint{1.486261in}{0.653640in}}%
\pgfpathlineto{\pgfqpoint{1.486430in}{0.664803in}}%
\pgfpathlineto{\pgfqpoint{1.486682in}{0.737203in}}%
\pgfpathlineto{\pgfqpoint{1.487104in}{0.653859in}}%
\pgfpathlineto{\pgfqpoint{1.487525in}{0.699976in}}%
\pgfpathlineto{\pgfqpoint{1.487778in}{0.653494in}}%
\pgfpathlineto{\pgfqpoint{1.488114in}{0.784890in}}%
\pgfpathlineto{\pgfqpoint{1.488620in}{0.667873in}}%
\pgfpathlineto{\pgfqpoint{1.488873in}{0.784513in}}%
\pgfpathlineto{\pgfqpoint{1.489378in}{0.654688in}}%
\pgfpathlineto{\pgfqpoint{1.489715in}{0.668599in}}%
\pgfpathlineto{\pgfqpoint{1.489799in}{0.670342in}}%
\pgfpathlineto{\pgfqpoint{1.490136in}{0.655093in}}%
\pgfpathlineto{\pgfqpoint{1.490221in}{0.654242in}}%
\pgfpathlineto{\pgfqpoint{1.490473in}{0.660564in}}%
\pgfpathlineto{\pgfqpoint{1.490642in}{0.673478in}}%
\pgfpathlineto{\pgfqpoint{1.490979in}{0.654210in}}%
\pgfpathlineto{\pgfqpoint{1.491484in}{0.657701in}}%
\pgfpathlineto{\pgfqpoint{1.491653in}{0.654140in}}%
\pgfpathlineto{\pgfqpoint{1.491990in}{0.666990in}}%
\pgfpathlineto{\pgfqpoint{1.492495in}{0.655963in}}%
\pgfpathlineto{\pgfqpoint{1.492832in}{0.706576in}}%
\pgfpathlineto{\pgfqpoint{1.493253in}{0.654527in}}%
\pgfpathlineto{\pgfqpoint{1.493759in}{0.685594in}}%
\pgfpathlineto{\pgfqpoint{1.494096in}{0.654115in}}%
\pgfpathlineto{\pgfqpoint{1.494517in}{0.736092in}}%
\pgfpathlineto{\pgfqpoint{1.494938in}{0.655700in}}%
\pgfpathlineto{\pgfqpoint{1.495275in}{0.785415in}}%
\pgfpathlineto{\pgfqpoint{1.495781in}{0.654426in}}%
\pgfpathlineto{\pgfqpoint{1.496118in}{0.671745in}}%
\pgfpathlineto{\pgfqpoint{1.497466in}{0.653823in}}%
\pgfpathlineto{\pgfqpoint{1.497971in}{0.654955in}}%
\pgfpathlineto{\pgfqpoint{1.498898in}{0.653530in}}%
\pgfpathlineto{\pgfqpoint{1.498561in}{0.656921in}}%
\pgfpathlineto{\pgfqpoint{1.499066in}{0.653923in}}%
\pgfpathlineto{\pgfqpoint{1.500330in}{0.661689in}}%
\pgfpathlineto{\pgfqpoint{1.499824in}{0.653433in}}%
\pgfpathlineto{\pgfqpoint{1.500414in}{0.661271in}}%
\pgfpathlineto{\pgfqpoint{1.501594in}{0.652810in}}%
\pgfpathlineto{\pgfqpoint{1.501762in}{0.652830in}}%
\pgfpathlineto{\pgfqpoint{1.502689in}{0.653408in}}%
\pgfpathlineto{\pgfqpoint{1.503700in}{0.653157in}}%
\pgfpathlineto{\pgfqpoint{1.504037in}{0.675042in}}%
\pgfpathlineto{\pgfqpoint{1.504289in}{0.720555in}}%
\pgfpathlineto{\pgfqpoint{1.504711in}{0.653560in}}%
\pgfpathlineto{\pgfqpoint{1.505132in}{0.706920in}}%
\pgfpathlineto{\pgfqpoint{1.506311in}{0.653165in}}%
\pgfpathlineto{\pgfqpoint{1.506396in}{0.653611in}}%
\pgfpathlineto{\pgfqpoint{1.507491in}{0.664411in}}%
\pgfpathlineto{\pgfqpoint{1.506985in}{0.653297in}}%
\pgfpathlineto{\pgfqpoint{1.507659in}{0.657321in}}%
\pgfpathlineto{\pgfqpoint{1.508417in}{0.668416in}}%
\pgfpathlineto{\pgfqpoint{1.508839in}{0.653091in}}%
\pgfpathlineto{\pgfqpoint{1.509176in}{0.655852in}}%
\pgfpathlineto{\pgfqpoint{1.510187in}{0.653441in}}%
\pgfpathlineto{\pgfqpoint{1.511534in}{0.654226in}}%
\pgfpathlineto{\pgfqpoint{1.512208in}{0.654045in}}%
\pgfpathlineto{\pgfqpoint{1.512461in}{0.661308in}}%
\pgfpathlineto{\pgfqpoint{1.513556in}{0.694492in}}%
\pgfpathlineto{\pgfqpoint{1.513135in}{0.653823in}}%
\pgfpathlineto{\pgfqpoint{1.513725in}{0.682478in}}%
\pgfpathlineto{\pgfqpoint{1.514483in}{0.683896in}}%
\pgfpathlineto{\pgfqpoint{1.514988in}{0.653437in}}%
\pgfpathlineto{\pgfqpoint{1.515157in}{0.654052in}}%
\pgfpathlineto{\pgfqpoint{1.515494in}{0.659630in}}%
\pgfpathlineto{\pgfqpoint{1.515915in}{0.653292in}}%
\pgfpathlineto{\pgfqpoint{1.516252in}{0.654841in}}%
\pgfpathlineto{\pgfqpoint{1.517179in}{0.653591in}}%
\pgfpathlineto{\pgfqpoint{1.517432in}{0.654071in}}%
\pgfpathlineto{\pgfqpoint{1.518358in}{0.678102in}}%
\pgfpathlineto{\pgfqpoint{1.518443in}{0.685947in}}%
\pgfpathlineto{\pgfqpoint{1.518864in}{0.653482in}}%
\pgfpathlineto{\pgfqpoint{1.519285in}{0.667254in}}%
\pgfpathlineto{\pgfqpoint{1.519622in}{0.653161in}}%
\pgfpathlineto{\pgfqpoint{1.520549in}{0.653326in}}%
\pgfpathlineto{\pgfqpoint{1.522234in}{0.653932in}}%
\pgfpathlineto{\pgfqpoint{1.522655in}{0.665651in}}%
\pgfpathlineto{\pgfqpoint{1.523160in}{0.653379in}}%
\pgfpathlineto{\pgfqpoint{1.523834in}{0.662971in}}%
\pgfpathlineto{\pgfqpoint{1.524171in}{0.653353in}}%
\pgfpathlineto{\pgfqpoint{1.524508in}{0.670676in}}%
\pgfpathlineto{\pgfqpoint{1.524677in}{0.679619in}}%
\pgfpathlineto{\pgfqpoint{1.525014in}{0.653346in}}%
\pgfpathlineto{\pgfqpoint{1.525435in}{0.653712in}}%
\pgfpathlineto{\pgfqpoint{1.525856in}{0.653612in}}%
\pgfpathlineto{\pgfqpoint{1.526193in}{0.654365in}}%
\pgfpathlineto{\pgfqpoint{1.527204in}{0.665946in}}%
\pgfpathlineto{\pgfqpoint{1.526698in}{0.653741in}}%
\pgfpathlineto{\pgfqpoint{1.527372in}{0.658445in}}%
\pgfpathlineto{\pgfqpoint{1.527709in}{0.653575in}}%
\pgfpathlineto{\pgfqpoint{1.528383in}{0.661151in}}%
\pgfpathlineto{\pgfqpoint{1.528468in}{0.661616in}}%
\pgfpathlineto{\pgfqpoint{1.528636in}{0.657474in}}%
\pgfpathlineto{\pgfqpoint{1.528973in}{0.653248in}}%
\pgfpathlineto{\pgfqpoint{1.529900in}{0.653444in}}%
\pgfpathlineto{\pgfqpoint{1.531753in}{0.653708in}}%
\pgfpathlineto{\pgfqpoint{1.533017in}{0.654487in}}%
\pgfpathlineto{\pgfqpoint{1.534365in}{0.682307in}}%
\pgfpathlineto{\pgfqpoint{1.534028in}{0.654331in}}%
\pgfpathlineto{\pgfqpoint{1.534533in}{0.669096in}}%
\pgfpathlineto{\pgfqpoint{1.535628in}{0.653569in}}%
\pgfpathlineto{\pgfqpoint{1.535713in}{0.653598in}}%
\pgfpathlineto{\pgfqpoint{1.537398in}{0.654773in}}%
\pgfpathlineto{\pgfqpoint{1.537987in}{0.654147in}}%
\pgfpathlineto{\pgfqpoint{1.538156in}{0.664084in}}%
\pgfpathlineto{\pgfqpoint{1.539251in}{0.732564in}}%
\pgfpathlineto{\pgfqpoint{1.538830in}{0.654334in}}%
\pgfpathlineto{\pgfqpoint{1.539335in}{0.707186in}}%
\pgfpathlineto{\pgfqpoint{1.539672in}{0.654320in}}%
\pgfpathlineto{\pgfqpoint{1.540599in}{0.655205in}}%
\pgfpathlineto{\pgfqpoint{1.540683in}{0.654480in}}%
\pgfpathlineto{\pgfqpoint{1.540767in}{0.656687in}}%
\pgfpathlineto{\pgfqpoint{1.541020in}{0.706163in}}%
\pgfpathlineto{\pgfqpoint{1.541441in}{0.654161in}}%
\pgfpathlineto{\pgfqpoint{1.541863in}{0.656665in}}%
\pgfpathlineto{\pgfqpoint{1.542873in}{0.654174in}}%
\pgfpathlineto{\pgfqpoint{1.543042in}{0.654407in}}%
\pgfpathlineto{\pgfqpoint{1.544137in}{0.655469in}}%
\pgfpathlineto{\pgfqpoint{1.544306in}{0.654862in}}%
\pgfpathlineto{\pgfqpoint{1.544811in}{0.654562in}}%
\pgfpathlineto{\pgfqpoint{1.544980in}{0.655018in}}%
\pgfpathlineto{\pgfqpoint{1.545317in}{0.704398in}}%
\pgfpathlineto{\pgfqpoint{1.545906in}{0.678964in}}%
\pgfpathlineto{\pgfqpoint{1.546243in}{1.089590in}}%
\pgfpathlineto{\pgfqpoint{1.546833in}{0.658355in}}%
\pgfpathlineto{\pgfqpoint{1.547170in}{0.831027in}}%
\pgfpathlineto{\pgfqpoint{1.547254in}{0.843647in}}%
\pgfpathlineto{\pgfqpoint{1.547423in}{0.738443in}}%
\pgfpathlineto{\pgfqpoint{1.548602in}{0.655922in}}%
\pgfpathlineto{\pgfqpoint{1.549023in}{0.657105in}}%
\pgfpathlineto{\pgfqpoint{1.550034in}{0.655703in}}%
\pgfpathlineto{\pgfqpoint{1.551803in}{0.656355in}}%
\pgfpathlineto{\pgfqpoint{1.552814in}{0.661774in}}%
\pgfpathlineto{\pgfqpoint{1.552393in}{0.655812in}}%
\pgfpathlineto{\pgfqpoint{1.553067in}{0.657638in}}%
\pgfpathlineto{\pgfqpoint{1.553825in}{0.662884in}}%
\pgfpathlineto{\pgfqpoint{1.554246in}{0.655481in}}%
\pgfpathlineto{\pgfqpoint{1.554499in}{0.657961in}}%
\pgfpathlineto{\pgfqpoint{1.554668in}{0.659936in}}%
\pgfpathlineto{\pgfqpoint{1.555173in}{0.655375in}}%
\pgfpathlineto{\pgfqpoint{1.555426in}{0.655415in}}%
\pgfpathlineto{\pgfqpoint{1.557616in}{0.656241in}}%
\pgfpathlineto{\pgfqpoint{1.557953in}{0.658623in}}%
\pgfpathlineto{\pgfqpoint{1.558374in}{0.655588in}}%
\pgfpathlineto{\pgfqpoint{1.558711in}{0.656384in}}%
\pgfpathlineto{\pgfqpoint{1.558796in}{0.656610in}}%
\pgfpathlineto{\pgfqpoint{1.559133in}{0.655514in}}%
\pgfpathlineto{\pgfqpoint{1.559554in}{0.655738in}}%
\pgfpathlineto{\pgfqpoint{1.560986in}{0.655447in}}%
\pgfpathlineto{\pgfqpoint{1.562081in}{0.656067in}}%
\pgfpathlineto{\pgfqpoint{1.562334in}{0.661460in}}%
\pgfpathlineto{\pgfqpoint{1.562755in}{0.655417in}}%
\pgfpathlineto{\pgfqpoint{1.563261in}{0.657863in}}%
\pgfpathlineto{\pgfqpoint{1.564524in}{0.654198in}}%
\pgfpathlineto{\pgfqpoint{1.564946in}{0.654447in}}%
\pgfpathlineto{\pgfqpoint{1.565282in}{0.655437in}}%
\pgfpathlineto{\pgfqpoint{1.565619in}{0.660605in}}%
\pgfpathlineto{\pgfqpoint{1.565956in}{0.653328in}}%
\pgfpathlineto{\pgfqpoint{1.566293in}{0.654514in}}%
\pgfpathlineto{\pgfqpoint{1.566378in}{0.654581in}}%
\pgfpathlineto{\pgfqpoint{1.566630in}{0.653485in}}%
\pgfpathlineto{\pgfqpoint{1.566799in}{0.653439in}}%
\pgfpathlineto{\pgfqpoint{1.566883in}{0.653663in}}%
\pgfpathlineto{\pgfqpoint{1.567220in}{0.671896in}}%
\pgfpathlineto{\pgfqpoint{1.567304in}{0.677144in}}%
\pgfpathlineto{\pgfqpoint{1.567810in}{0.654126in}}%
\pgfpathlineto{\pgfqpoint{1.568063in}{0.657628in}}%
\pgfpathlineto{\pgfqpoint{1.568147in}{0.658843in}}%
\pgfpathlineto{\pgfqpoint{1.568652in}{0.652605in}}%
\pgfpathlineto{\pgfqpoint{1.569158in}{0.657925in}}%
\pgfpathlineto{\pgfqpoint{1.569242in}{0.658245in}}%
\pgfpathlineto{\pgfqpoint{1.569410in}{0.655875in}}%
\pgfpathlineto{\pgfqpoint{1.570337in}{0.652635in}}%
\pgfpathlineto{\pgfqpoint{1.570084in}{0.656791in}}%
\pgfpathlineto{\pgfqpoint{1.570590in}{0.653573in}}%
\pgfpathlineto{\pgfqpoint{1.571854in}{0.687489in}}%
\pgfpathlineto{\pgfqpoint{1.572106in}{0.768701in}}%
\pgfpathlineto{\pgfqpoint{1.572864in}{0.652761in}}%
\pgfpathlineto{\pgfqpoint{1.573033in}{0.653274in}}%
\pgfpathlineto{\pgfqpoint{1.574212in}{0.667153in}}%
\pgfpathlineto{\pgfqpoint{1.573791in}{0.652241in}}%
\pgfpathlineto{\pgfqpoint{1.574381in}{0.659239in}}%
\pgfpathlineto{\pgfqpoint{1.574634in}{0.652508in}}%
\pgfpathlineto{\pgfqpoint{1.575560in}{0.653146in}}%
\pgfpathlineto{\pgfqpoint{1.576487in}{0.652970in}}%
\pgfpathlineto{\pgfqpoint{1.576066in}{0.654042in}}%
\pgfpathlineto{\pgfqpoint{1.576655in}{0.653361in}}%
\pgfpathlineto{\pgfqpoint{1.577582in}{0.656609in}}%
\pgfpathlineto{\pgfqpoint{1.576992in}{0.653181in}}%
\pgfpathlineto{\pgfqpoint{1.578172in}{0.655050in}}%
\pgfpathlineto{\pgfqpoint{1.578930in}{0.653590in}}%
\pgfpathlineto{\pgfqpoint{1.579099in}{0.654505in}}%
\pgfpathlineto{\pgfqpoint{1.579604in}{0.652715in}}%
\pgfpathlineto{\pgfqpoint{1.579773in}{0.663460in}}%
\pgfpathlineto{\pgfqpoint{1.580868in}{0.781433in}}%
\pgfpathlineto{\pgfqpoint{1.580446in}{0.655074in}}%
\pgfpathlineto{\pgfqpoint{1.581036in}{0.754493in}}%
\pgfpathlineto{\pgfqpoint{1.582131in}{0.653870in}}%
\pgfpathlineto{\pgfqpoint{1.582300in}{0.654209in}}%
\pgfpathlineto{\pgfqpoint{1.583395in}{0.681172in}}%
\pgfpathlineto{\pgfqpoint{1.583732in}{0.664074in}}%
\pgfpathlineto{\pgfqpoint{1.584069in}{0.653362in}}%
\pgfpathlineto{\pgfqpoint{1.584911in}{0.655445in}}%
\pgfpathlineto{\pgfqpoint{1.586175in}{0.674746in}}%
\pgfpathlineto{\pgfqpoint{1.585585in}{0.654546in}}%
\pgfpathlineto{\pgfqpoint{1.586344in}{0.669095in}}%
\pgfpathlineto{\pgfqpoint{1.587523in}{0.653186in}}%
\pgfpathlineto{\pgfqpoint{1.587607in}{0.653219in}}%
\pgfpathlineto{\pgfqpoint{1.588450in}{0.655106in}}%
\pgfpathlineto{\pgfqpoint{1.588702in}{0.768816in}}%
\pgfpathlineto{\pgfqpoint{1.588871in}{0.858191in}}%
\pgfpathlineto{\pgfqpoint{1.589376in}{0.661147in}}%
\pgfpathlineto{\pgfqpoint{1.589713in}{0.750379in}}%
\pgfpathlineto{\pgfqpoint{1.590219in}{0.654918in}}%
\pgfpathlineto{\pgfqpoint{1.591230in}{0.666246in}}%
\pgfpathlineto{\pgfqpoint{1.591483in}{0.685717in}}%
\pgfpathlineto{\pgfqpoint{1.592072in}{0.654474in}}%
\pgfpathlineto{\pgfqpoint{1.592241in}{0.654558in}}%
\pgfpathlineto{\pgfqpoint{1.594768in}{0.655410in}}%
\pgfpathlineto{\pgfqpoint{1.595021in}{0.659138in}}%
\pgfpathlineto{\pgfqpoint{1.595442in}{0.654359in}}%
\pgfpathlineto{\pgfqpoint{1.595779in}{0.654945in}}%
\pgfpathlineto{\pgfqpoint{1.597295in}{0.654639in}}%
\pgfpathlineto{\pgfqpoint{1.597632in}{0.654784in}}%
\pgfpathlineto{\pgfqpoint{1.597717in}{0.655208in}}%
\pgfpathlineto{\pgfqpoint{1.598728in}{0.663246in}}%
\pgfpathlineto{\pgfqpoint{1.598896in}{0.660221in}}%
\pgfpathlineto{\pgfqpoint{1.599149in}{0.654524in}}%
\pgfpathlineto{\pgfqpoint{1.600075in}{0.655744in}}%
\pgfpathlineto{\pgfqpoint{1.600497in}{0.671650in}}%
\pgfpathlineto{\pgfqpoint{1.601086in}{0.654780in}}%
\pgfpathlineto{\pgfqpoint{1.601929in}{0.654696in}}%
\pgfpathlineto{\pgfqpoint{1.602266in}{0.655019in}}%
\pgfpathlineto{\pgfqpoint{1.603782in}{0.658024in}}%
\pgfpathlineto{\pgfqpoint{1.604203in}{0.699119in}}%
\pgfpathlineto{\pgfqpoint{1.604793in}{0.657985in}}%
\pgfpathlineto{\pgfqpoint{1.605467in}{0.654423in}}%
\pgfpathlineto{\pgfqpoint{1.605720in}{0.656380in}}%
\pgfpathlineto{\pgfqpoint{1.606141in}{0.714060in}}%
\pgfpathlineto{\pgfqpoint{1.607068in}{0.671839in}}%
\pgfpathlineto{\pgfqpoint{1.607405in}{0.654465in}}%
\pgfpathlineto{\pgfqpoint{1.608163in}{0.658851in}}%
\pgfpathlineto{\pgfqpoint{1.608584in}{0.718987in}}%
\pgfpathlineto{\pgfqpoint{1.609174in}{0.655557in}}%
\pgfpathlineto{\pgfqpoint{1.609932in}{0.653535in}}%
\pgfpathlineto{\pgfqpoint{1.609595in}{0.655730in}}%
\pgfpathlineto{\pgfqpoint{1.610269in}{0.655294in}}%
\pgfpathlineto{\pgfqpoint{1.610438in}{0.656372in}}%
\pgfpathlineto{\pgfqpoint{1.610859in}{0.654521in}}%
\pgfpathlineto{\pgfqpoint{1.611196in}{0.654619in}}%
\pgfpathlineto{\pgfqpoint{1.611785in}{0.653441in}}%
\pgfpathlineto{\pgfqpoint{1.612291in}{0.654658in}}%
\pgfpathlineto{\pgfqpoint{1.612712in}{0.657042in}}%
\pgfpathlineto{\pgfqpoint{1.613133in}{0.653056in}}%
\pgfpathlineto{\pgfqpoint{1.614060in}{0.652816in}}%
\pgfpathlineto{\pgfqpoint{1.614229in}{0.652918in}}%
\pgfpathlineto{\pgfqpoint{1.615324in}{0.654996in}}%
\pgfpathlineto{\pgfqpoint{1.616335in}{0.678304in}}%
\pgfpathlineto{\pgfqpoint{1.615745in}{0.653455in}}%
\pgfpathlineto{\pgfqpoint{1.616587in}{0.660685in}}%
\pgfpathlineto{\pgfqpoint{1.616924in}{0.653129in}}%
\pgfpathlineto{\pgfqpoint{1.617767in}{0.653278in}}%
\pgfpathlineto{\pgfqpoint{1.618104in}{0.653595in}}%
\pgfpathlineto{\pgfqpoint{1.618357in}{0.657328in}}%
\pgfpathlineto{\pgfqpoint{1.618778in}{0.653367in}}%
\pgfpathlineto{\pgfqpoint{1.619199in}{0.653804in}}%
\pgfpathlineto{\pgfqpoint{1.620631in}{0.653710in}}%
\pgfpathlineto{\pgfqpoint{1.621811in}{0.654362in}}%
\pgfpathlineto{\pgfqpoint{1.622063in}{0.655167in}}%
\pgfpathlineto{\pgfqpoint{1.622906in}{0.654437in}}%
\pgfpathlineto{\pgfqpoint{1.623158in}{0.654769in}}%
\pgfpathlineto{\pgfqpoint{1.623832in}{0.672011in}}%
\pgfpathlineto{\pgfqpoint{1.624759in}{0.665719in}}%
\pgfpathlineto{\pgfqpoint{1.625096in}{0.654387in}}%
\pgfpathlineto{\pgfqpoint{1.625602in}{0.679327in}}%
\pgfpathlineto{\pgfqpoint{1.625854in}{0.664472in}}%
\pgfpathlineto{\pgfqpoint{1.625939in}{0.663495in}}%
\pgfpathlineto{\pgfqpoint{1.626023in}{0.667086in}}%
\pgfpathlineto{\pgfqpoint{1.626444in}{0.769149in}}%
\pgfpathlineto{\pgfqpoint{1.627034in}{0.656293in}}%
\pgfpathlineto{\pgfqpoint{1.627455in}{0.653763in}}%
\pgfpathlineto{\pgfqpoint{1.628045in}{0.655410in}}%
\pgfpathlineto{\pgfqpoint{1.628719in}{0.681951in}}%
\pgfpathlineto{\pgfqpoint{1.629140in}{0.735101in}}%
\pgfpathlineto{\pgfqpoint{1.630067in}{0.708397in}}%
\pgfpathlineto{\pgfqpoint{1.630825in}{0.654209in}}%
\pgfpathlineto{\pgfqpoint{1.631246in}{0.669368in}}%
\pgfpathlineto{\pgfqpoint{1.631414in}{0.678090in}}%
\pgfpathlineto{\pgfqpoint{1.631920in}{0.653151in}}%
\pgfpathlineto{\pgfqpoint{1.632088in}{0.653683in}}%
\pgfpathlineto{\pgfqpoint{1.633184in}{0.667627in}}%
\pgfpathlineto{\pgfqpoint{1.632510in}{0.653212in}}%
\pgfpathlineto{\pgfqpoint{1.633521in}{0.657731in}}%
\pgfpathlineto{\pgfqpoint{1.634700in}{0.653068in}}%
\pgfpathlineto{\pgfqpoint{1.634110in}{0.659588in}}%
\pgfpathlineto{\pgfqpoint{1.634784in}{0.653189in}}%
\pgfpathlineto{\pgfqpoint{1.635121in}{0.655450in}}%
\pgfpathlineto{\pgfqpoint{1.635964in}{0.653432in}}%
\pgfpathlineto{\pgfqpoint{1.636301in}{0.654478in}}%
\pgfpathlineto{\pgfqpoint{1.636890in}{0.656246in}}%
\pgfpathlineto{\pgfqpoint{1.637564in}{0.655275in}}%
\pgfpathlineto{\pgfqpoint{1.637986in}{0.655905in}}%
\pgfpathlineto{\pgfqpoint{1.638070in}{0.656171in}}%
\pgfpathlineto{\pgfqpoint{1.639249in}{0.665513in}}%
\pgfpathlineto{\pgfqpoint{1.639586in}{0.681064in}}%
\pgfpathlineto{\pgfqpoint{1.640176in}{0.661756in}}%
\pgfpathlineto{\pgfqpoint{1.640850in}{0.651733in}}%
\pgfpathlineto{\pgfqpoint{1.641524in}{0.652911in}}%
\pgfpathlineto{\pgfqpoint{1.642619in}{0.670692in}}%
\pgfpathlineto{\pgfqpoint{1.643209in}{0.740184in}}%
\pgfpathlineto{\pgfqpoint{1.644135in}{0.701586in}}%
\pgfpathlineto{\pgfqpoint{1.644809in}{0.660038in}}%
\pgfpathlineto{\pgfqpoint{1.645483in}{0.669668in}}%
\pgfpathlineto{\pgfqpoint{1.645989in}{0.777500in}}%
\pgfpathlineto{\pgfqpoint{1.646578in}{0.684121in}}%
\pgfpathlineto{\pgfqpoint{1.646831in}{0.674103in}}%
\pgfpathlineto{\pgfqpoint{1.647252in}{0.704132in}}%
\pgfpathlineto{\pgfqpoint{1.647842in}{0.757746in}}%
\pgfpathlineto{\pgfqpoint{1.648432in}{0.713359in}}%
\pgfpathlineto{\pgfqpoint{1.649696in}{0.668835in}}%
\pgfpathlineto{\pgfqpoint{1.650117in}{0.683808in}}%
\pgfpathlineto{\pgfqpoint{1.650538in}{0.706966in}}%
\pgfpathlineto{\pgfqpoint{1.651043in}{0.672559in}}%
\pgfpathlineto{\pgfqpoint{1.651296in}{0.667138in}}%
\pgfpathlineto{\pgfqpoint{1.651970in}{0.676854in}}%
\pgfpathlineto{\pgfqpoint{1.652054in}{0.676434in}}%
\pgfpathlineto{\pgfqpoint{1.653318in}{0.659325in}}%
\pgfpathlineto{\pgfqpoint{1.653655in}{0.669252in}}%
\pgfpathlineto{\pgfqpoint{1.653992in}{0.690571in}}%
\pgfpathlineto{\pgfqpoint{1.654497in}{0.655821in}}%
\pgfpathlineto{\pgfqpoint{1.654919in}{0.649742in}}%
\pgfpathlineto{\pgfqpoint{1.655593in}{0.653638in}}%
\pgfpathlineto{\pgfqpoint{1.656519in}{0.653241in}}%
\pgfpathlineto{\pgfqpoint{1.657193in}{0.665481in}}%
\pgfpathlineto{\pgfqpoint{1.657530in}{0.676924in}}%
\pgfpathlineto{\pgfqpoint{1.657951in}{0.656723in}}%
\pgfpathlineto{\pgfqpoint{1.659131in}{0.641591in}}%
\pgfpathlineto{\pgfqpoint{1.659299in}{0.641926in}}%
\pgfpathlineto{\pgfqpoint{1.660142in}{0.651063in}}%
\pgfpathlineto{\pgfqpoint{1.660479in}{0.664984in}}%
\pgfpathlineto{\pgfqpoint{1.661069in}{0.643907in}}%
\pgfpathlineto{\pgfqpoint{1.661321in}{0.642618in}}%
\pgfpathlineto{\pgfqpoint{1.661911in}{0.645191in}}%
\pgfpathlineto{\pgfqpoint{1.662079in}{0.644823in}}%
\pgfpathlineto{\pgfqpoint{1.663259in}{0.644031in}}%
\pgfpathlineto{\pgfqpoint{1.663343in}{0.644079in}}%
\pgfpathlineto{\pgfqpoint{1.667555in}{0.650104in}}%
\pgfpathlineto{\pgfqpoint{1.668987in}{0.651704in}}%
\pgfpathlineto{\pgfqpoint{1.669577in}{0.652963in}}%
\pgfpathlineto{\pgfqpoint{1.669914in}{0.667706in}}%
\pgfpathlineto{\pgfqpoint{1.670335in}{0.702782in}}%
\pgfpathlineto{\pgfqpoint{1.671178in}{0.694863in}}%
\pgfpathlineto{\pgfqpoint{1.671599in}{0.653614in}}%
\pgfpathlineto{\pgfqpoint{1.672442in}{0.658625in}}%
\pgfpathlineto{\pgfqpoint{1.673031in}{0.709677in}}%
\pgfpathlineto{\pgfqpoint{1.673200in}{0.720147in}}%
\pgfpathlineto{\pgfqpoint{1.673874in}{0.686825in}}%
\pgfpathlineto{\pgfqpoint{1.673958in}{0.686110in}}%
\pgfpathlineto{\pgfqpoint{1.674295in}{0.691850in}}%
\pgfpathlineto{\pgfqpoint{1.675306in}{0.735777in}}%
\pgfpathlineto{\pgfqpoint{1.675811in}{0.716206in}}%
\pgfpathlineto{\pgfqpoint{1.677580in}{0.658873in}}%
\pgfpathlineto{\pgfqpoint{1.678002in}{0.664220in}}%
\pgfpathlineto{\pgfqpoint{1.678507in}{0.708937in}}%
\pgfpathlineto{\pgfqpoint{1.679265in}{0.673853in}}%
\pgfpathlineto{\pgfqpoint{1.679518in}{0.674767in}}%
\pgfpathlineto{\pgfqpoint{1.680866in}{0.687168in}}%
\pgfpathlineto{\pgfqpoint{1.681034in}{0.680753in}}%
\pgfpathlineto{\pgfqpoint{1.681877in}{0.652776in}}%
\pgfpathlineto{\pgfqpoint{1.682382in}{0.661572in}}%
\pgfpathlineto{\pgfqpoint{1.682888in}{0.674219in}}%
\pgfpathlineto{\pgfqpoint{1.683562in}{0.666404in}}%
\pgfpathlineto{\pgfqpoint{1.684488in}{0.663717in}}%
\pgfpathlineto{\pgfqpoint{1.686679in}{0.642580in}}%
\pgfpathlineto{\pgfqpoint{1.687184in}{0.648395in}}%
\pgfpathlineto{\pgfqpoint{1.688448in}{0.652492in}}%
\pgfpathlineto{\pgfqpoint{1.688532in}{0.652288in}}%
\pgfpathlineto{\pgfqpoint{1.689459in}{0.634327in}}%
\pgfpathlineto{\pgfqpoint{1.690723in}{0.635650in}}%
\pgfpathlineto{\pgfqpoint{1.694261in}{0.646950in}}%
\pgfpathlineto{\pgfqpoint{1.696451in}{0.651654in}}%
\pgfpathlineto{\pgfqpoint{1.697462in}{0.651098in}}%
\pgfpathlineto{\pgfqpoint{1.700663in}{0.653864in}}%
\pgfpathlineto{\pgfqpoint{1.701253in}{0.672566in}}%
\pgfpathlineto{\pgfqpoint{1.701506in}{0.683845in}}%
\pgfpathlineto{\pgfqpoint{1.702180in}{0.671855in}}%
\pgfpathlineto{\pgfqpoint{1.702348in}{0.672196in}}%
\pgfpathlineto{\pgfqpoint{1.702517in}{0.671322in}}%
\pgfpathlineto{\pgfqpoint{1.703107in}{0.661150in}}%
\pgfpathlineto{\pgfqpoint{1.703528in}{0.668342in}}%
\pgfpathlineto{\pgfqpoint{1.704707in}{0.769535in}}%
\pgfpathlineto{\pgfqpoint{1.705718in}{0.720836in}}%
\pgfpathlineto{\pgfqpoint{1.707403in}{0.668164in}}%
\pgfpathlineto{\pgfqpoint{1.707571in}{0.669787in}}%
\pgfpathlineto{\pgfqpoint{1.708245in}{0.697405in}}%
\pgfpathlineto{\pgfqpoint{1.708751in}{0.674439in}}%
\pgfpathlineto{\pgfqpoint{1.709256in}{0.662958in}}%
\pgfpathlineto{\pgfqpoint{1.709930in}{0.667725in}}%
\pgfpathlineto{\pgfqpoint{1.710857in}{0.666998in}}%
\pgfpathlineto{\pgfqpoint{1.711868in}{0.654222in}}%
\pgfpathlineto{\pgfqpoint{1.713300in}{0.651652in}}%
\pgfpathlineto{\pgfqpoint{1.713806in}{0.652684in}}%
\pgfpathlineto{\pgfqpoint{1.713974in}{0.652920in}}%
\pgfpathlineto{\pgfqpoint{1.714395in}{0.651344in}}%
\pgfpathlineto{\pgfqpoint{1.715743in}{0.651015in}}%
\pgfpathlineto{\pgfqpoint{1.716754in}{0.651781in}}%
\pgfpathlineto{\pgfqpoint{1.717344in}{0.665830in}}%
\pgfpathlineto{\pgfqpoint{1.718523in}{0.659901in}}%
\pgfpathlineto{\pgfqpoint{1.719113in}{0.659039in}}%
\pgfpathlineto{\pgfqpoint{1.719955in}{0.656489in}}%
\pgfpathlineto{\pgfqpoint{1.720377in}{0.658374in}}%
\pgfpathlineto{\pgfqpoint{1.721303in}{0.674211in}}%
\pgfpathlineto{\pgfqpoint{1.721977in}{0.666407in}}%
\pgfpathlineto{\pgfqpoint{1.722567in}{0.668313in}}%
\pgfpathlineto{\pgfqpoint{1.723072in}{0.666203in}}%
\pgfpathlineto{\pgfqpoint{1.725600in}{0.646873in}}%
\pgfpathlineto{\pgfqpoint{1.726105in}{0.647973in}}%
\pgfpathlineto{\pgfqpoint{1.727116in}{0.651214in}}%
\pgfpathlineto{\pgfqpoint{1.727453in}{0.649324in}}%
\pgfpathlineto{\pgfqpoint{1.727790in}{0.648385in}}%
\pgfpathlineto{\pgfqpoint{1.728464in}{0.649686in}}%
\pgfpathlineto{\pgfqpoint{1.730065in}{0.653967in}}%
\pgfpathlineto{\pgfqpoint{1.730570in}{0.651943in}}%
\pgfpathlineto{\pgfqpoint{1.732592in}{0.649634in}}%
\pgfpathlineto{\pgfqpoint{1.732676in}{0.649713in}}%
\pgfpathlineto{\pgfqpoint{1.733013in}{0.653904in}}%
\pgfpathlineto{\pgfqpoint{1.733687in}{0.719151in}}%
\pgfpathlineto{\pgfqpoint{1.734445in}{0.670299in}}%
\pgfpathlineto{\pgfqpoint{1.734530in}{0.670213in}}%
\pgfpathlineto{\pgfqpoint{1.734698in}{0.671006in}}%
\pgfpathlineto{\pgfqpoint{1.735372in}{0.692001in}}%
\pgfpathlineto{\pgfqpoint{1.736299in}{0.677129in}}%
\pgfpathlineto{\pgfqpoint{1.738068in}{0.650360in}}%
\pgfpathlineto{\pgfqpoint{1.739163in}{0.651769in}}%
\pgfpathlineto{\pgfqpoint{1.739753in}{0.655487in}}%
\pgfpathlineto{\pgfqpoint{1.740680in}{0.653658in}}%
\pgfpathlineto{\pgfqpoint{1.741101in}{0.652186in}}%
\pgfpathlineto{\pgfqpoint{1.741859in}{0.653256in}}%
\pgfpathlineto{\pgfqpoint{1.742364in}{0.653978in}}%
\pgfpathlineto{\pgfqpoint{1.742701in}{0.652641in}}%
\pgfpathlineto{\pgfqpoint{1.743881in}{0.652318in}}%
\pgfpathlineto{\pgfqpoint{1.743965in}{0.652354in}}%
\pgfpathlineto{\pgfqpoint{1.747588in}{0.653326in}}%
\pgfpathlineto{\pgfqpoint{1.748767in}{0.653467in}}%
\pgfpathlineto{\pgfqpoint{1.749020in}{0.654991in}}%
\pgfpathlineto{\pgfqpoint{1.749273in}{0.658086in}}%
\pgfpathlineto{\pgfqpoint{1.749778in}{0.653690in}}%
\pgfpathlineto{\pgfqpoint{1.750115in}{0.655302in}}%
\pgfpathlineto{\pgfqpoint{1.750452in}{0.653808in}}%
\pgfpathlineto{\pgfqpoint{1.750873in}{0.653583in}}%
\pgfpathlineto{\pgfqpoint{1.751547in}{0.654030in}}%
\pgfpathlineto{\pgfqpoint{1.753316in}{0.654058in}}%
\pgfpathlineto{\pgfqpoint{1.753401in}{0.653800in}}%
\pgfpathlineto{\pgfqpoint{1.754664in}{0.652272in}}%
\pgfpathlineto{\pgfqpoint{1.754159in}{0.654873in}}%
\pgfpathlineto{\pgfqpoint{1.754748in}{0.652290in}}%
\pgfpathlineto{\pgfqpoint{1.756602in}{0.657309in}}%
\pgfpathlineto{\pgfqpoint{1.757613in}{0.690382in}}%
\pgfpathlineto{\pgfqpoint{1.757950in}{0.749571in}}%
\pgfpathlineto{\pgfqpoint{1.758708in}{0.697620in}}%
\pgfpathlineto{\pgfqpoint{1.759972in}{0.667276in}}%
\pgfpathlineto{\pgfqpoint{1.760898in}{0.673231in}}%
\pgfpathlineto{\pgfqpoint{1.761656in}{0.694850in}}%
\pgfpathlineto{\pgfqpoint{1.762078in}{0.676782in}}%
\pgfpathlineto{\pgfqpoint{1.763510in}{0.666253in}}%
\pgfpathlineto{\pgfqpoint{1.764100in}{0.665156in}}%
\pgfpathlineto{\pgfqpoint{1.764268in}{0.667101in}}%
\pgfpathlineto{\pgfqpoint{1.764689in}{0.679809in}}%
\pgfpathlineto{\pgfqpoint{1.765111in}{0.662410in}}%
\pgfpathlineto{\pgfqpoint{1.766458in}{0.650871in}}%
\pgfpathlineto{\pgfqpoint{1.769238in}{0.650818in}}%
\pgfpathlineto{\pgfqpoint{1.770334in}{0.651974in}}%
\pgfpathlineto{\pgfqpoint{1.770671in}{0.653725in}}%
\pgfpathlineto{\pgfqpoint{1.771092in}{0.651797in}}%
\pgfpathlineto{\pgfqpoint{1.771513in}{0.653489in}}%
\pgfpathlineto{\pgfqpoint{1.771934in}{0.650444in}}%
\pgfpathlineto{\pgfqpoint{1.772356in}{0.655503in}}%
\pgfpathlineto{\pgfqpoint{1.772524in}{0.654138in}}%
\pgfpathlineto{\pgfqpoint{1.772777in}{0.651480in}}%
\pgfpathlineto{\pgfqpoint{1.773029in}{0.661472in}}%
\pgfpathlineto{\pgfqpoint{1.773282in}{0.680333in}}%
\pgfpathlineto{\pgfqpoint{1.773956in}{0.649135in}}%
\pgfpathlineto{\pgfqpoint{1.774546in}{0.649894in}}%
\pgfpathlineto{\pgfqpoint{1.776652in}{0.652707in}}%
\pgfpathlineto{\pgfqpoint{1.777242in}{0.653056in}}%
\pgfpathlineto{\pgfqpoint{1.777579in}{0.652317in}}%
\pgfpathlineto{\pgfqpoint{1.778421in}{0.652178in}}%
\pgfpathlineto{\pgfqpoint{1.778590in}{0.652453in}}%
\pgfpathlineto{\pgfqpoint{1.779011in}{0.657905in}}%
\pgfpathlineto{\pgfqpoint{1.779938in}{0.653678in}}%
\pgfpathlineto{\pgfqpoint{1.780443in}{0.652717in}}%
\pgfpathlineto{\pgfqpoint{1.780780in}{0.653541in}}%
\pgfpathlineto{\pgfqpoint{1.781201in}{0.663251in}}%
\pgfpathlineto{\pgfqpoint{1.781707in}{0.652422in}}%
\pgfpathlineto{\pgfqpoint{1.782465in}{0.651607in}}%
\pgfpathlineto{\pgfqpoint{1.782128in}{0.652791in}}%
\pgfpathlineto{\pgfqpoint{1.782886in}{0.652234in}}%
\pgfpathlineto{\pgfqpoint{1.783307in}{0.656253in}}%
\pgfpathlineto{\pgfqpoint{1.783981in}{0.652372in}}%
\pgfpathlineto{\pgfqpoint{1.786087in}{0.654315in}}%
\pgfpathlineto{\pgfqpoint{1.786340in}{0.652825in}}%
\pgfpathlineto{\pgfqpoint{1.786677in}{0.652515in}}%
\pgfpathlineto{\pgfqpoint{1.787435in}{0.652873in}}%
\pgfpathlineto{\pgfqpoint{1.787857in}{0.661332in}}%
\pgfpathlineto{\pgfqpoint{1.788783in}{0.654557in}}%
\pgfpathlineto{\pgfqpoint{1.789036in}{0.653343in}}%
\pgfpathlineto{\pgfqpoint{1.789878in}{0.654097in}}%
\pgfpathlineto{\pgfqpoint{1.790131in}{0.655096in}}%
\pgfpathlineto{\pgfqpoint{1.790384in}{0.651960in}}%
\pgfpathlineto{\pgfqpoint{1.790552in}{0.650725in}}%
\pgfpathlineto{\pgfqpoint{1.791058in}{0.653040in}}%
\pgfpathlineto{\pgfqpoint{1.791563in}{0.651213in}}%
\pgfpathlineto{\pgfqpoint{1.794512in}{0.655031in}}%
\pgfpathlineto{\pgfqpoint{1.794933in}{0.720459in}}%
\pgfpathlineto{\pgfqpoint{1.795102in}{0.742606in}}%
\pgfpathlineto{\pgfqpoint{1.795691in}{0.663711in}}%
\pgfpathlineto{\pgfqpoint{1.795860in}{0.662750in}}%
\pgfpathlineto{\pgfqpoint{1.796197in}{0.667275in}}%
\pgfpathlineto{\pgfqpoint{1.796534in}{0.681988in}}%
\pgfpathlineto{\pgfqpoint{1.797376in}{0.674903in}}%
\pgfpathlineto{\pgfqpoint{1.798050in}{0.661715in}}%
\pgfpathlineto{\pgfqpoint{1.798303in}{0.669805in}}%
\pgfpathlineto{\pgfqpoint{1.798808in}{0.891472in}}%
\pgfpathlineto{\pgfqpoint{1.799398in}{0.678234in}}%
\pgfpathlineto{\pgfqpoint{1.799819in}{0.662563in}}%
\pgfpathlineto{\pgfqpoint{1.800325in}{0.696221in}}%
\pgfpathlineto{\pgfqpoint{1.800662in}{0.736154in}}%
\pgfpathlineto{\pgfqpoint{1.801420in}{0.694417in}}%
\pgfpathlineto{\pgfqpoint{1.803610in}{0.655107in}}%
\pgfpathlineto{\pgfqpoint{1.804537in}{0.661709in}}%
\pgfpathlineto{\pgfqpoint{1.804958in}{0.671081in}}%
\pgfpathlineto{\pgfqpoint{1.805716in}{0.665591in}}%
\pgfpathlineto{\pgfqpoint{1.806980in}{0.653835in}}%
\pgfpathlineto{\pgfqpoint{1.807233in}{0.654331in}}%
\pgfpathlineto{\pgfqpoint{1.807654in}{0.662356in}}%
\pgfpathlineto{\pgfqpoint{1.808159in}{0.653725in}}%
\pgfpathlineto{\pgfqpoint{1.808749in}{0.650961in}}%
\pgfpathlineto{\pgfqpoint{1.809170in}{0.654329in}}%
\pgfpathlineto{\pgfqpoint{1.809507in}{0.662114in}}%
\pgfpathlineto{\pgfqpoint{1.810266in}{0.655683in}}%
\pgfpathlineto{\pgfqpoint{1.811613in}{0.646229in}}%
\pgfpathlineto{\pgfqpoint{1.812203in}{0.647556in}}%
\pgfpathlineto{\pgfqpoint{1.812540in}{0.650512in}}%
\pgfpathlineto{\pgfqpoint{1.812877in}{0.656757in}}%
\pgfpathlineto{\pgfqpoint{1.813383in}{0.648438in}}%
\pgfpathlineto{\pgfqpoint{1.813551in}{0.648661in}}%
\pgfpathlineto{\pgfqpoint{1.814057in}{0.648337in}}%
\pgfpathlineto{\pgfqpoint{1.814983in}{0.649225in}}%
\pgfpathlineto{\pgfqpoint{1.816584in}{0.650602in}}%
\pgfpathlineto{\pgfqpoint{1.817089in}{0.654286in}}%
\pgfpathlineto{\pgfqpoint{1.817679in}{0.650575in}}%
\pgfpathlineto{\pgfqpoint{1.817932in}{0.651464in}}%
\pgfpathlineto{\pgfqpoint{1.818353in}{0.649851in}}%
\pgfpathlineto{\pgfqpoint{1.818774in}{0.650581in}}%
\pgfpathlineto{\pgfqpoint{1.819111in}{0.649305in}}%
\pgfpathlineto{\pgfqpoint{1.819364in}{0.651525in}}%
\pgfpathlineto{\pgfqpoint{1.819701in}{0.658628in}}%
\pgfpathlineto{\pgfqpoint{1.820375in}{0.650067in}}%
\pgfpathlineto{\pgfqpoint{1.820628in}{0.649845in}}%
\pgfpathlineto{\pgfqpoint{1.820965in}{0.650998in}}%
\pgfpathlineto{\pgfqpoint{1.821386in}{0.654845in}}%
\pgfpathlineto{\pgfqpoint{1.821976in}{0.649962in}}%
\pgfpathlineto{\pgfqpoint{1.822734in}{0.649388in}}%
\pgfpathlineto{\pgfqpoint{1.822986in}{0.650345in}}%
\pgfpathlineto{\pgfqpoint{1.824166in}{0.672899in}}%
\pgfpathlineto{\pgfqpoint{1.824419in}{0.658762in}}%
\pgfpathlineto{\pgfqpoint{1.824756in}{0.649907in}}%
\pgfpathlineto{\pgfqpoint{1.825430in}{0.652306in}}%
\pgfpathlineto{\pgfqpoint{1.825851in}{0.803182in}}%
\pgfpathlineto{\pgfqpoint{1.826778in}{0.702245in}}%
\pgfpathlineto{\pgfqpoint{1.827199in}{0.652428in}}%
\pgfpathlineto{\pgfqpoint{1.828125in}{0.656144in}}%
\pgfpathlineto{\pgfqpoint{1.828378in}{0.663624in}}%
\pgfpathlineto{\pgfqpoint{1.828968in}{0.652508in}}%
\pgfpathlineto{\pgfqpoint{1.829052in}{0.652510in}}%
\pgfpathlineto{\pgfqpoint{1.829473in}{0.653016in}}%
\pgfpathlineto{\pgfqpoint{1.829810in}{0.656436in}}%
\pgfpathlineto{\pgfqpoint{1.830569in}{0.653369in}}%
\pgfpathlineto{\pgfqpoint{1.830737in}{0.653305in}}%
\pgfpathlineto{\pgfqpoint{1.830905in}{0.654239in}}%
\pgfpathlineto{\pgfqpoint{1.831327in}{0.672891in}}%
\pgfpathlineto{\pgfqpoint{1.832338in}{0.666605in}}%
\pgfpathlineto{\pgfqpoint{1.832843in}{0.652788in}}%
\pgfpathlineto{\pgfqpoint{1.833686in}{0.653138in}}%
\pgfpathlineto{\pgfqpoint{1.835202in}{0.659357in}}%
\pgfpathlineto{\pgfqpoint{1.835370in}{0.662144in}}%
\pgfpathlineto{\pgfqpoint{1.835960in}{0.655714in}}%
\pgfpathlineto{\pgfqpoint{1.836381in}{0.661142in}}%
\pgfpathlineto{\pgfqpoint{1.836803in}{0.653101in}}%
\pgfpathlineto{\pgfqpoint{1.837224in}{0.661995in}}%
\pgfpathlineto{\pgfqpoint{1.837645in}{0.656568in}}%
\pgfpathlineto{\pgfqpoint{1.838066in}{0.748640in}}%
\pgfpathlineto{\pgfqpoint{1.838740in}{0.662697in}}%
\pgfpathlineto{\pgfqpoint{1.838824in}{0.661830in}}%
\pgfpathlineto{\pgfqpoint{1.839077in}{0.666430in}}%
\pgfpathlineto{\pgfqpoint{1.839498in}{0.686388in}}%
\pgfpathlineto{\pgfqpoint{1.840257in}{0.677393in}}%
\pgfpathlineto{\pgfqpoint{1.840762in}{0.656144in}}%
\pgfpathlineto{\pgfqpoint{1.841520in}{0.661394in}}%
\pgfpathlineto{\pgfqpoint{1.841857in}{0.752978in}}%
\pgfpathlineto{\pgfqpoint{1.842026in}{0.810206in}}%
\pgfpathlineto{\pgfqpoint{1.842700in}{0.666665in}}%
\pgfpathlineto{\pgfqpoint{1.843121in}{0.657239in}}%
\pgfpathlineto{\pgfqpoint{1.843795in}{0.664267in}}%
\pgfpathlineto{\pgfqpoint{1.844216in}{0.673744in}}%
\pgfpathlineto{\pgfqpoint{1.844974in}{0.669833in}}%
\pgfpathlineto{\pgfqpoint{1.845059in}{0.669633in}}%
\pgfpathlineto{\pgfqpoint{1.845227in}{0.670901in}}%
\pgfpathlineto{\pgfqpoint{1.845396in}{0.673503in}}%
\pgfpathlineto{\pgfqpoint{1.845733in}{0.662741in}}%
\pgfpathlineto{\pgfqpoint{1.846238in}{0.653254in}}%
\pgfpathlineto{\pgfqpoint{1.846575in}{0.661589in}}%
\pgfpathlineto{\pgfqpoint{1.846996in}{0.725475in}}%
\pgfpathlineto{\pgfqpoint{1.847670in}{0.660082in}}%
\pgfpathlineto{\pgfqpoint{1.848681in}{0.654268in}}%
\pgfpathlineto{\pgfqpoint{1.849271in}{0.657440in}}%
\pgfpathlineto{\pgfqpoint{1.849608in}{0.676624in}}%
\pgfpathlineto{\pgfqpoint{1.850197in}{0.651342in}}%
\pgfpathlineto{\pgfqpoint{1.850956in}{0.649239in}}%
\pgfpathlineto{\pgfqpoint{1.851293in}{0.649721in}}%
\pgfpathlineto{\pgfqpoint{1.851545in}{0.662804in}}%
\pgfpathlineto{\pgfqpoint{1.851798in}{0.684395in}}%
\pgfpathlineto{\pgfqpoint{1.852304in}{0.659726in}}%
\pgfpathlineto{\pgfqpoint{1.852641in}{0.662933in}}%
\pgfpathlineto{\pgfqpoint{1.854831in}{0.646505in}}%
\pgfpathlineto{\pgfqpoint{1.855421in}{0.648940in}}%
\pgfpathlineto{\pgfqpoint{1.856516in}{0.649602in}}%
\pgfpathlineto{\pgfqpoint{1.856600in}{0.649429in}}%
\pgfpathlineto{\pgfqpoint{1.856853in}{0.648906in}}%
\pgfpathlineto{\pgfqpoint{1.857358in}{0.650285in}}%
\pgfpathlineto{\pgfqpoint{1.857527in}{0.650114in}}%
\pgfpathlineto{\pgfqpoint{1.857779in}{0.652338in}}%
\pgfpathlineto{\pgfqpoint{1.858201in}{0.660481in}}%
\pgfpathlineto{\pgfqpoint{1.858706in}{0.649733in}}%
\pgfpathlineto{\pgfqpoint{1.859212in}{0.649018in}}%
\pgfpathlineto{\pgfqpoint{1.859380in}{0.649735in}}%
\pgfpathlineto{\pgfqpoint{1.859633in}{0.653344in}}%
\pgfpathlineto{\pgfqpoint{1.860475in}{0.650670in}}%
\pgfpathlineto{\pgfqpoint{1.860897in}{0.650210in}}%
\pgfpathlineto{\pgfqpoint{1.861402in}{0.651220in}}%
\pgfpathlineto{\pgfqpoint{1.861739in}{0.663776in}}%
\pgfpathlineto{\pgfqpoint{1.861907in}{0.673104in}}%
\pgfpathlineto{\pgfqpoint{1.862581in}{0.651190in}}%
\pgfpathlineto{\pgfqpoint{1.862666in}{0.651189in}}%
\pgfpathlineto{\pgfqpoint{1.862918in}{0.652745in}}%
\pgfpathlineto{\pgfqpoint{1.864519in}{0.705284in}}%
\pgfpathlineto{\pgfqpoint{1.864688in}{0.683248in}}%
\pgfpathlineto{\pgfqpoint{1.865193in}{0.650020in}}%
\pgfpathlineto{\pgfqpoint{1.865867in}{0.655726in}}%
\pgfpathlineto{\pgfqpoint{1.866372in}{0.672870in}}%
\pgfpathlineto{\pgfqpoint{1.867131in}{0.661216in}}%
\pgfpathlineto{\pgfqpoint{1.868310in}{0.685069in}}%
\pgfpathlineto{\pgfqpoint{1.868479in}{0.706354in}}%
\pgfpathlineto{\pgfqpoint{1.869068in}{0.648026in}}%
\pgfpathlineto{\pgfqpoint{1.869152in}{0.648206in}}%
\pgfpathlineto{\pgfqpoint{1.869911in}{0.653403in}}%
\pgfpathlineto{\pgfqpoint{1.870163in}{0.663336in}}%
\pgfpathlineto{\pgfqpoint{1.870837in}{0.649477in}}%
\pgfpathlineto{\pgfqpoint{1.870922in}{0.649515in}}%
\pgfpathlineto{\pgfqpoint{1.872270in}{0.651470in}}%
\pgfpathlineto{\pgfqpoint{1.872522in}{0.657427in}}%
\pgfpathlineto{\pgfqpoint{1.872944in}{0.650164in}}%
\pgfpathlineto{\pgfqpoint{1.873365in}{0.652198in}}%
\pgfpathlineto{\pgfqpoint{1.873786in}{0.650592in}}%
\pgfpathlineto{\pgfqpoint{1.874713in}{0.650943in}}%
\pgfpathlineto{\pgfqpoint{1.876903in}{0.654770in}}%
\pgfpathlineto{\pgfqpoint{1.877408in}{0.684483in}}%
\pgfpathlineto{\pgfqpoint{1.878082in}{0.659244in}}%
\pgfpathlineto{\pgfqpoint{1.878335in}{0.660742in}}%
\pgfpathlineto{\pgfqpoint{1.878672in}{0.654735in}}%
\pgfpathlineto{\pgfqpoint{1.879009in}{0.652645in}}%
\pgfpathlineto{\pgfqpoint{1.879683in}{0.655773in}}%
\pgfpathlineto{\pgfqpoint{1.880273in}{0.712281in}}%
\pgfpathlineto{\pgfqpoint{1.880947in}{0.665356in}}%
\pgfpathlineto{\pgfqpoint{1.881115in}{0.663060in}}%
\pgfpathlineto{\pgfqpoint{1.881705in}{0.669308in}}%
\pgfpathlineto{\pgfqpoint{1.881873in}{0.668682in}}%
\pgfpathlineto{\pgfqpoint{1.882632in}{0.661069in}}%
\pgfpathlineto{\pgfqpoint{1.883221in}{0.655693in}}%
\pgfpathlineto{\pgfqpoint{1.883558in}{0.663756in}}%
\pgfpathlineto{\pgfqpoint{1.884064in}{0.744365in}}%
\pgfpathlineto{\pgfqpoint{1.884653in}{0.668202in}}%
\pgfpathlineto{\pgfqpoint{1.884990in}{0.660857in}}%
\pgfpathlineto{\pgfqpoint{1.885749in}{0.666231in}}%
\pgfpathlineto{\pgfqpoint{1.886338in}{0.666020in}}%
\pgfpathlineto{\pgfqpoint{1.886423in}{0.665482in}}%
\pgfpathlineto{\pgfqpoint{1.887097in}{0.656294in}}%
\pgfpathlineto{\pgfqpoint{1.887518in}{0.663113in}}%
\pgfpathlineto{\pgfqpoint{1.887686in}{0.667046in}}%
\pgfpathlineto{\pgfqpoint{1.888192in}{0.654262in}}%
\pgfpathlineto{\pgfqpoint{1.889034in}{0.651126in}}%
\pgfpathlineto{\pgfqpoint{1.889371in}{0.651764in}}%
\pgfpathlineto{\pgfqpoint{1.890382in}{0.653090in}}%
\pgfpathlineto{\pgfqpoint{1.890635in}{0.652643in}}%
\pgfpathlineto{\pgfqpoint{1.891393in}{0.650980in}}%
\pgfpathlineto{\pgfqpoint{1.891730in}{0.652249in}}%
\pgfpathlineto{\pgfqpoint{1.892825in}{0.655226in}}%
\pgfpathlineto{\pgfqpoint{1.892320in}{0.650952in}}%
\pgfpathlineto{\pgfqpoint{1.892909in}{0.654414in}}%
\pgfpathlineto{\pgfqpoint{1.893499in}{0.649466in}}%
\pgfpathlineto{\pgfqpoint{1.893836in}{0.652584in}}%
\pgfpathlineto{\pgfqpoint{1.894257in}{0.681224in}}%
\pgfpathlineto{\pgfqpoint{1.895268in}{0.669908in}}%
\pgfpathlineto{\pgfqpoint{1.895605in}{0.672036in}}%
\pgfpathlineto{\pgfqpoint{1.896279in}{0.669852in}}%
\pgfpathlineto{\pgfqpoint{1.897374in}{0.652443in}}%
\pgfpathlineto{\pgfqpoint{1.897711in}{0.663015in}}%
\pgfpathlineto{\pgfqpoint{1.898554in}{0.715755in}}%
\pgfpathlineto{\pgfqpoint{1.898975in}{0.680363in}}%
\pgfpathlineto{\pgfqpoint{1.900407in}{0.651019in}}%
\pgfpathlineto{\pgfqpoint{1.900491in}{0.650989in}}%
\pgfpathlineto{\pgfqpoint{1.900660in}{0.651810in}}%
\pgfpathlineto{\pgfqpoint{1.901081in}{0.654706in}}%
\pgfpathlineto{\pgfqpoint{1.901587in}{0.650701in}}%
\pgfpathlineto{\pgfqpoint{1.901671in}{0.650425in}}%
\pgfpathlineto{\pgfqpoint{1.901839in}{0.651682in}}%
\pgfpathlineto{\pgfqpoint{1.902261in}{0.692472in}}%
\pgfpathlineto{\pgfqpoint{1.902429in}{0.712659in}}%
\pgfpathlineto{\pgfqpoint{1.902935in}{0.668473in}}%
\pgfpathlineto{\pgfqpoint{1.903440in}{0.706667in}}%
\pgfpathlineto{\pgfqpoint{1.904282in}{0.664530in}}%
\pgfpathlineto{\pgfqpoint{1.904872in}{0.692134in}}%
\pgfpathlineto{\pgfqpoint{1.906220in}{0.724419in}}%
\pgfpathlineto{\pgfqpoint{1.906641in}{0.716411in}}%
\pgfpathlineto{\pgfqpoint{1.906894in}{0.710679in}}%
\pgfpathlineto{\pgfqpoint{1.907231in}{0.727400in}}%
\pgfpathlineto{\pgfqpoint{1.907400in}{0.736140in}}%
\pgfpathlineto{\pgfqpoint{1.907736in}{0.705035in}}%
\pgfpathlineto{\pgfqpoint{1.908410in}{0.657719in}}%
\pgfpathlineto{\pgfqpoint{1.909000in}{0.672001in}}%
\pgfpathlineto{\pgfqpoint{1.909084in}{0.673157in}}%
\pgfpathlineto{\pgfqpoint{1.909421in}{0.665493in}}%
\pgfpathlineto{\pgfqpoint{1.909843in}{0.660813in}}%
\pgfpathlineto{\pgfqpoint{1.910517in}{0.664346in}}%
\pgfpathlineto{\pgfqpoint{1.910601in}{0.663985in}}%
\pgfpathlineto{\pgfqpoint{1.910854in}{0.667777in}}%
\pgfpathlineto{\pgfqpoint{1.911275in}{0.689511in}}%
\pgfpathlineto{\pgfqpoint{1.911696in}{0.656550in}}%
\pgfpathlineto{\pgfqpoint{1.912201in}{0.642870in}}%
\pgfpathlineto{\pgfqpoint{1.912960in}{0.645152in}}%
\pgfpathlineto{\pgfqpoint{1.914813in}{0.648246in}}%
\pgfpathlineto{\pgfqpoint{1.915234in}{0.647674in}}%
\pgfpathlineto{\pgfqpoint{1.915908in}{0.649136in}}%
\pgfpathlineto{\pgfqpoint{1.916919in}{0.650499in}}%
\pgfpathlineto{\pgfqpoint{1.917425in}{0.650310in}}%
\pgfpathlineto{\pgfqpoint{1.918941in}{0.650941in}}%
\pgfpathlineto{\pgfqpoint{1.924670in}{0.654195in}}%
\pgfpathlineto{\pgfqpoint{1.926355in}{0.654783in}}%
\pgfpathlineto{\pgfqpoint{1.928124in}{0.655467in}}%
\pgfpathlineto{\pgfqpoint{1.928292in}{0.654809in}}%
\pgfpathlineto{\pgfqpoint{1.929387in}{0.649280in}}%
\pgfpathlineto{\pgfqpoint{1.929724in}{0.651280in}}%
\pgfpathlineto{\pgfqpoint{1.930904in}{0.664423in}}%
\pgfpathlineto{\pgfqpoint{1.930314in}{0.650237in}}%
\pgfpathlineto{\pgfqpoint{1.930988in}{0.662402in}}%
\pgfpathlineto{\pgfqpoint{1.932252in}{0.649942in}}%
\pgfpathlineto{\pgfqpoint{1.931830in}{0.663229in}}%
\pgfpathlineto{\pgfqpoint{1.932336in}{0.650159in}}%
\pgfpathlineto{\pgfqpoint{1.932673in}{0.661206in}}%
\pgfpathlineto{\pgfqpoint{1.933178in}{0.650055in}}%
\pgfpathlineto{\pgfqpoint{1.933431in}{0.650268in}}%
\pgfpathlineto{\pgfqpoint{1.933684in}{0.652240in}}%
\pgfpathlineto{\pgfqpoint{1.933852in}{0.654937in}}%
\pgfpathlineto{\pgfqpoint{1.934189in}{0.651436in}}%
\pgfpathlineto{\pgfqpoint{1.934695in}{0.651879in}}%
\pgfpathlineto{\pgfqpoint{1.934863in}{0.650348in}}%
\pgfpathlineto{\pgfqpoint{1.935284in}{0.657581in}}%
\pgfpathlineto{\pgfqpoint{1.935537in}{0.663691in}}%
\pgfpathlineto{\pgfqpoint{1.936211in}{0.655818in}}%
\pgfpathlineto{\pgfqpoint{1.936295in}{0.655881in}}%
\pgfpathlineto{\pgfqpoint{1.936464in}{0.656166in}}%
\pgfpathlineto{\pgfqpoint{1.936632in}{0.654603in}}%
\pgfpathlineto{\pgfqpoint{1.936801in}{0.653322in}}%
\pgfpathlineto{\pgfqpoint{1.937138in}{0.659473in}}%
\pgfpathlineto{\pgfqpoint{1.938317in}{0.681979in}}%
\pgfpathlineto{\pgfqpoint{1.938486in}{0.673884in}}%
\pgfpathlineto{\pgfqpoint{1.938991in}{0.651107in}}%
\pgfpathlineto{\pgfqpoint{1.939749in}{0.655020in}}%
\pgfpathlineto{\pgfqpoint{1.940086in}{0.662897in}}%
\pgfpathlineto{\pgfqpoint{1.940592in}{0.649631in}}%
\pgfpathlineto{\pgfqpoint{1.940845in}{0.648730in}}%
\pgfpathlineto{\pgfqpoint{1.941434in}{0.650871in}}%
\pgfpathlineto{\pgfqpoint{1.942108in}{0.656791in}}%
\pgfpathlineto{\pgfqpoint{1.942614in}{0.767432in}}%
\pgfpathlineto{\pgfqpoint{1.942951in}{0.903283in}}%
\pgfpathlineto{\pgfqpoint{1.943625in}{0.728873in}}%
\pgfpathlineto{\pgfqpoint{1.945057in}{0.700001in}}%
\pgfpathlineto{\pgfqpoint{1.946489in}{0.655349in}}%
\pgfpathlineto{\pgfqpoint{1.945562in}{0.701353in}}%
\pgfpathlineto{\pgfqpoint{1.947163in}{0.676562in}}%
\pgfpathlineto{\pgfqpoint{1.947584in}{0.704746in}}%
\pgfpathlineto{\pgfqpoint{1.948258in}{0.680878in}}%
\pgfpathlineto{\pgfqpoint{1.949859in}{0.657297in}}%
\pgfpathlineto{\pgfqpoint{1.950111in}{0.658238in}}%
\pgfpathlineto{\pgfqpoint{1.950954in}{0.664549in}}%
\pgfpathlineto{\pgfqpoint{1.951207in}{0.660029in}}%
\pgfpathlineto{\pgfqpoint{1.952133in}{0.650707in}}%
\pgfpathlineto{\pgfqpoint{1.952639in}{0.650999in}}%
\pgfpathlineto{\pgfqpoint{1.953313in}{0.652485in}}%
\pgfpathlineto{\pgfqpoint{1.953818in}{0.654650in}}%
\pgfpathlineto{\pgfqpoint{1.954408in}{0.652578in}}%
\pgfpathlineto{\pgfqpoint{1.954661in}{0.652903in}}%
\pgfpathlineto{\pgfqpoint{1.954998in}{0.651602in}}%
\pgfpathlineto{\pgfqpoint{1.955250in}{0.651112in}}%
\pgfpathlineto{\pgfqpoint{1.956093in}{0.651727in}}%
\pgfpathlineto{\pgfqpoint{1.957525in}{0.651110in}}%
\pgfpathlineto{\pgfqpoint{1.957693in}{0.651622in}}%
\pgfpathlineto{\pgfqpoint{1.958367in}{0.660506in}}%
\pgfpathlineto{\pgfqpoint{1.958789in}{0.652861in}}%
\pgfpathlineto{\pgfqpoint{1.959294in}{0.650303in}}%
\pgfpathlineto{\pgfqpoint{1.959800in}{0.652869in}}%
\pgfpathlineto{\pgfqpoint{1.960474in}{0.658141in}}%
\pgfpathlineto{\pgfqpoint{1.961148in}{0.657065in}}%
\pgfpathlineto{\pgfqpoint{1.961906in}{0.652659in}}%
\pgfpathlineto{\pgfqpoint{1.962243in}{0.656243in}}%
\pgfpathlineto{\pgfqpoint{1.962580in}{0.663007in}}%
\pgfpathlineto{\pgfqpoint{1.963085in}{0.651048in}}%
\pgfpathlineto{\pgfqpoint{1.963338in}{0.650012in}}%
\pgfpathlineto{\pgfqpoint{1.963675in}{0.653581in}}%
\pgfpathlineto{\pgfqpoint{1.964180in}{0.674190in}}%
\pgfpathlineto{\pgfqpoint{1.965191in}{0.670775in}}%
\pgfpathlineto{\pgfqpoint{1.966034in}{0.655086in}}%
\pgfpathlineto{\pgfqpoint{1.966371in}{0.662749in}}%
\pgfpathlineto{\pgfqpoint{1.966708in}{0.681782in}}%
\pgfpathlineto{\pgfqpoint{1.967213in}{0.647427in}}%
\pgfpathlineto{\pgfqpoint{1.967297in}{0.646642in}}%
\pgfpathlineto{\pgfqpoint{1.967634in}{0.651232in}}%
\pgfpathlineto{\pgfqpoint{1.968056in}{0.679141in}}%
\pgfpathlineto{\pgfqpoint{1.968814in}{0.658578in}}%
\pgfpathlineto{\pgfqpoint{1.969235in}{0.651863in}}%
\pgfpathlineto{\pgfqpoint{1.969656in}{0.658849in}}%
\pgfpathlineto{\pgfqpoint{1.969909in}{0.656997in}}%
\pgfpathlineto{\pgfqpoint{1.970330in}{0.687580in}}%
\pgfpathlineto{\pgfqpoint{1.970667in}{0.649728in}}%
\pgfpathlineto{\pgfqpoint{1.971004in}{0.638532in}}%
\pgfpathlineto{\pgfqpoint{1.971847in}{0.642664in}}%
\pgfpathlineto{\pgfqpoint{1.973531in}{0.646324in}}%
\pgfpathlineto{\pgfqpoint{1.975890in}{0.655866in}}%
\pgfpathlineto{\pgfqpoint{1.976649in}{0.653017in}}%
\pgfpathlineto{\pgfqpoint{1.977070in}{0.651051in}}%
\pgfpathlineto{\pgfqpoint{1.977659in}{0.653559in}}%
\pgfpathlineto{\pgfqpoint{1.978081in}{0.656379in}}%
\pgfpathlineto{\pgfqpoint{1.978586in}{0.653348in}}%
\pgfpathlineto{\pgfqpoint{1.978755in}{0.652905in}}%
\pgfpathlineto{\pgfqpoint{1.979092in}{0.654769in}}%
\pgfpathlineto{\pgfqpoint{1.979597in}{0.662346in}}%
\pgfpathlineto{\pgfqpoint{1.980524in}{0.659582in}}%
\pgfpathlineto{\pgfqpoint{1.980692in}{0.660669in}}%
\pgfpathlineto{\pgfqpoint{1.981113in}{0.702595in}}%
\pgfpathlineto{\pgfqpoint{1.981366in}{0.736566in}}%
\pgfpathlineto{\pgfqpoint{1.981956in}{0.683677in}}%
\pgfpathlineto{\pgfqpoint{1.982209in}{0.699814in}}%
\pgfpathlineto{\pgfqpoint{1.983557in}{0.755907in}}%
\pgfpathlineto{\pgfqpoint{1.983641in}{0.754337in}}%
\pgfpathlineto{\pgfqpoint{1.985831in}{0.659974in}}%
\pgfpathlineto{\pgfqpoint{1.986168in}{0.664700in}}%
\pgfpathlineto{\pgfqpoint{1.986842in}{0.701274in}}%
\pgfpathlineto{\pgfqpoint{1.987179in}{0.728093in}}%
\pgfpathlineto{\pgfqpoint{1.987853in}{0.695128in}}%
\pgfpathlineto{\pgfqpoint{1.990465in}{0.662347in}}%
\pgfpathlineto{\pgfqpoint{1.990717in}{0.664830in}}%
\pgfpathlineto{\pgfqpoint{1.991307in}{0.677931in}}%
\pgfpathlineto{\pgfqpoint{1.992318in}{0.674225in}}%
\pgfpathlineto{\pgfqpoint{1.992823in}{0.675647in}}%
\pgfpathlineto{\pgfqpoint{1.993245in}{0.673787in}}%
\pgfpathlineto{\pgfqpoint{1.993329in}{0.673611in}}%
\pgfpathlineto{\pgfqpoint{1.993497in}{0.674847in}}%
\pgfpathlineto{\pgfqpoint{1.993834in}{0.683276in}}%
\pgfpathlineto{\pgfqpoint{1.994171in}{0.670816in}}%
\pgfpathlineto{\pgfqpoint{1.994424in}{0.663701in}}%
\pgfpathlineto{\pgfqpoint{1.994761in}{0.687220in}}%
\pgfpathlineto{\pgfqpoint{1.995267in}{0.753779in}}%
\pgfpathlineto{\pgfqpoint{1.995941in}{0.713362in}}%
\pgfpathlineto{\pgfqpoint{1.997710in}{0.683566in}}%
\pgfpathlineto{\pgfqpoint{1.998299in}{0.661337in}}%
\pgfpathlineto{\pgfqpoint{1.998721in}{0.682218in}}%
\pgfpathlineto{\pgfqpoint{1.999142in}{0.726534in}}%
\pgfpathlineto{\pgfqpoint{1.999816in}{0.689720in}}%
\pgfpathlineto{\pgfqpoint{2.000995in}{0.666990in}}%
\pgfpathlineto{\pgfqpoint{2.001332in}{0.676539in}}%
\pgfpathlineto{\pgfqpoint{2.001585in}{0.683145in}}%
\pgfpathlineto{\pgfqpoint{2.002175in}{0.671156in}}%
\pgfpathlineto{\pgfqpoint{2.002849in}{0.650426in}}%
\pgfpathlineto{\pgfqpoint{2.003354in}{0.664156in}}%
\pgfpathlineto{\pgfqpoint{2.005207in}{0.710236in}}%
\pgfpathlineto{\pgfqpoint{2.005460in}{0.704789in}}%
\pgfpathlineto{\pgfqpoint{2.006471in}{0.641925in}}%
\pgfpathlineto{\pgfqpoint{2.007145in}{0.674278in}}%
\pgfpathlineto{\pgfqpoint{2.008240in}{0.698703in}}%
\pgfpathlineto{\pgfqpoint{2.008409in}{0.688654in}}%
\pgfpathlineto{\pgfqpoint{2.009335in}{0.594590in}}%
\pgfpathlineto{\pgfqpoint{2.010346in}{0.596069in}}%
\pgfpathlineto{\pgfqpoint{2.010852in}{0.608335in}}%
\pgfpathlineto{\pgfqpoint{2.012789in}{0.631097in}}%
\pgfpathlineto{\pgfqpoint{2.017760in}{0.653226in}}%
\pgfpathlineto{\pgfqpoint{2.017844in}{0.653268in}}%
\pgfpathlineto{\pgfqpoint{2.018013in}{0.652534in}}%
\pgfpathlineto{\pgfqpoint{2.018602in}{0.649851in}}%
\pgfpathlineto{\pgfqpoint{2.019360in}{0.651014in}}%
\pgfpathlineto{\pgfqpoint{2.019950in}{0.653402in}}%
\pgfpathlineto{\pgfqpoint{2.021130in}{0.652838in}}%
\pgfpathlineto{\pgfqpoint{2.023404in}{0.653564in}}%
\pgfpathlineto{\pgfqpoint{2.025089in}{0.654079in}}%
\pgfpathlineto{\pgfqpoint{2.025173in}{0.653877in}}%
\pgfpathlineto{\pgfqpoint{2.026016in}{0.651396in}}%
\pgfpathlineto{\pgfqpoint{2.026606in}{0.652312in}}%
\pgfpathlineto{\pgfqpoint{2.027279in}{0.654349in}}%
\pgfpathlineto{\pgfqpoint{2.028712in}{0.661075in}}%
\pgfpathlineto{\pgfqpoint{2.029217in}{0.660422in}}%
\pgfpathlineto{\pgfqpoint{2.031070in}{0.652795in}}%
\pgfpathlineto{\pgfqpoint{2.032418in}{0.650363in}}%
\pgfpathlineto{\pgfqpoint{2.032503in}{0.650381in}}%
\pgfpathlineto{\pgfqpoint{2.033345in}{0.652624in}}%
\pgfpathlineto{\pgfqpoint{2.033851in}{0.682488in}}%
\pgfpathlineto{\pgfqpoint{2.035704in}{0.747563in}}%
\pgfpathlineto{\pgfqpoint{2.035872in}{0.754018in}}%
\pgfpathlineto{\pgfqpoint{2.036378in}{0.727475in}}%
\pgfpathlineto{\pgfqpoint{2.038063in}{0.690831in}}%
\pgfpathlineto{\pgfqpoint{2.038231in}{0.691654in}}%
\pgfpathlineto{\pgfqpoint{2.038484in}{0.688244in}}%
\pgfpathlineto{\pgfqpoint{2.039074in}{0.674807in}}%
\pgfpathlineto{\pgfqpoint{2.039663in}{0.684979in}}%
\pgfpathlineto{\pgfqpoint{2.040169in}{0.691218in}}%
\pgfpathlineto{\pgfqpoint{2.040843in}{0.686910in}}%
\pgfpathlineto{\pgfqpoint{2.041180in}{0.684346in}}%
\pgfpathlineto{\pgfqpoint{2.042528in}{0.652906in}}%
\pgfpathlineto{\pgfqpoint{2.043791in}{0.653764in}}%
\pgfpathlineto{\pgfqpoint{2.044802in}{0.652991in}}%
\pgfpathlineto{\pgfqpoint{2.044971in}{0.653331in}}%
\pgfpathlineto{\pgfqpoint{2.045476in}{0.658239in}}%
\pgfpathlineto{\pgfqpoint{2.045813in}{0.652943in}}%
\pgfpathlineto{\pgfqpoint{2.046403in}{0.648377in}}%
\pgfpathlineto{\pgfqpoint{2.047077in}{0.648898in}}%
\pgfpathlineto{\pgfqpoint{2.048593in}{0.650936in}}%
\pgfpathlineto{\pgfqpoint{2.048930in}{0.653083in}}%
\pgfpathlineto{\pgfqpoint{2.049604in}{0.650459in}}%
\pgfpathlineto{\pgfqpoint{2.051795in}{0.649068in}}%
\pgfpathlineto{\pgfqpoint{2.052047in}{0.649766in}}%
\pgfpathlineto{\pgfqpoint{2.054322in}{0.656236in}}%
\pgfpathlineto{\pgfqpoint{2.054490in}{0.656085in}}%
\pgfpathlineto{\pgfqpoint{2.055754in}{0.653172in}}%
\pgfpathlineto{\pgfqpoint{2.056091in}{0.654612in}}%
\pgfpathlineto{\pgfqpoint{2.056260in}{0.655161in}}%
\pgfpathlineto{\pgfqpoint{2.056765in}{0.652653in}}%
\pgfpathlineto{\pgfqpoint{2.057102in}{0.651744in}}%
\pgfpathlineto{\pgfqpoint{2.057692in}{0.653094in}}%
\pgfpathlineto{\pgfqpoint{2.057776in}{0.653086in}}%
\pgfpathlineto{\pgfqpoint{2.058281in}{0.651046in}}%
\pgfpathlineto{\pgfqpoint{2.058534in}{0.653319in}}%
\pgfpathlineto{\pgfqpoint{2.058703in}{0.655852in}}%
\pgfpathlineto{\pgfqpoint{2.059124in}{0.643805in}}%
\pgfpathlineto{\pgfqpoint{2.059377in}{0.641592in}}%
\pgfpathlineto{\pgfqpoint{2.060135in}{0.644561in}}%
\pgfpathlineto{\pgfqpoint{2.063336in}{0.655148in}}%
\pgfpathlineto{\pgfqpoint{2.063589in}{0.652505in}}%
\pgfpathlineto{\pgfqpoint{2.064263in}{0.649036in}}%
\pgfpathlineto{\pgfqpoint{2.064684in}{0.651289in}}%
\pgfpathlineto{\pgfqpoint{2.065779in}{0.663124in}}%
\pgfpathlineto{\pgfqpoint{2.067127in}{0.675765in}}%
\pgfpathlineto{\pgfqpoint{2.067211in}{0.675306in}}%
\pgfpathlineto{\pgfqpoint{2.068138in}{0.649966in}}%
\pgfpathlineto{\pgfqpoint{2.068559in}{0.664651in}}%
\pgfpathlineto{\pgfqpoint{2.069065in}{0.739763in}}%
\pgfpathlineto{\pgfqpoint{2.069654in}{0.665730in}}%
\pgfpathlineto{\pgfqpoint{2.070160in}{0.663064in}}%
\pgfpathlineto{\pgfqpoint{2.070497in}{0.667508in}}%
\pgfpathlineto{\pgfqpoint{2.071676in}{0.694258in}}%
\pgfpathlineto{\pgfqpoint{2.072013in}{0.677779in}}%
\pgfpathlineto{\pgfqpoint{2.072435in}{0.664181in}}%
\pgfpathlineto{\pgfqpoint{2.073277in}{0.666746in}}%
\pgfpathlineto{\pgfqpoint{2.074288in}{0.679754in}}%
\pgfpathlineto{\pgfqpoint{2.075046in}{0.672626in}}%
\pgfpathlineto{\pgfqpoint{2.076141in}{0.658268in}}%
\pgfpathlineto{\pgfqpoint{2.076478in}{0.665611in}}%
\pgfpathlineto{\pgfqpoint{2.076731in}{0.674117in}}%
\pgfpathlineto{\pgfqpoint{2.077236in}{0.653919in}}%
\pgfpathlineto{\pgfqpoint{2.078500in}{0.648852in}}%
\pgfpathlineto{\pgfqpoint{2.078837in}{0.647396in}}%
\pgfpathlineto{\pgfqpoint{2.079174in}{0.651034in}}%
\pgfpathlineto{\pgfqpoint{2.079511in}{0.654566in}}%
\pgfpathlineto{\pgfqpoint{2.079764in}{0.647203in}}%
\pgfpathlineto{\pgfqpoint{2.080269in}{0.632983in}}%
\pgfpathlineto{\pgfqpoint{2.081027in}{0.636457in}}%
\pgfpathlineto{\pgfqpoint{2.081617in}{0.636111in}}%
\pgfpathlineto{\pgfqpoint{2.082881in}{0.648724in}}%
\pgfpathlineto{\pgfqpoint{2.083386in}{0.659884in}}%
\pgfpathlineto{\pgfqpoint{2.084145in}{0.655232in}}%
\pgfpathlineto{\pgfqpoint{2.084397in}{0.656357in}}%
\pgfpathlineto{\pgfqpoint{2.085071in}{0.654596in}}%
\pgfpathlineto{\pgfqpoint{2.086166in}{0.640857in}}%
\pgfpathlineto{\pgfqpoint{2.086503in}{0.651020in}}%
\pgfpathlineto{\pgfqpoint{2.086925in}{0.681965in}}%
\pgfpathlineto{\pgfqpoint{2.087430in}{0.643855in}}%
\pgfpathlineto{\pgfqpoint{2.087767in}{0.638795in}}%
\pgfpathlineto{\pgfqpoint{2.088525in}{0.644025in}}%
\pgfpathlineto{\pgfqpoint{2.089283in}{0.663046in}}%
\pgfpathlineto{\pgfqpoint{2.090800in}{0.684277in}}%
\pgfpathlineto{\pgfqpoint{2.090884in}{0.681299in}}%
\pgfpathlineto{\pgfqpoint{2.091474in}{0.635629in}}%
\pgfpathlineto{\pgfqpoint{2.092316in}{0.644186in}}%
\pgfpathlineto{\pgfqpoint{2.092569in}{0.649338in}}%
\pgfpathlineto{\pgfqpoint{2.093411in}{0.645388in}}%
\pgfpathlineto{\pgfqpoint{2.094338in}{0.651251in}}%
\pgfpathlineto{\pgfqpoint{2.095012in}{0.649378in}}%
\pgfpathlineto{\pgfqpoint{2.095686in}{0.645723in}}%
\pgfpathlineto{\pgfqpoint{2.096360in}{0.647084in}}%
\pgfpathlineto{\pgfqpoint{2.096781in}{0.645333in}}%
\pgfpathlineto{\pgfqpoint{2.097034in}{0.647917in}}%
\pgfpathlineto{\pgfqpoint{2.097708in}{0.676601in}}%
\pgfpathlineto{\pgfqpoint{2.098213in}{0.649787in}}%
\pgfpathlineto{\pgfqpoint{2.098887in}{0.644928in}}%
\pgfpathlineto{\pgfqpoint{2.099224in}{0.651522in}}%
\pgfpathlineto{\pgfqpoint{2.100067in}{0.659830in}}%
\pgfpathlineto{\pgfqpoint{2.099646in}{0.650426in}}%
\pgfpathlineto{\pgfqpoint{2.100235in}{0.653822in}}%
\pgfpathlineto{\pgfqpoint{2.100656in}{0.642005in}}%
\pgfpathlineto{\pgfqpoint{2.101330in}{0.654404in}}%
\pgfpathlineto{\pgfqpoint{2.101667in}{0.675966in}}%
\pgfpathlineto{\pgfqpoint{2.102173in}{0.645332in}}%
\pgfpathlineto{\pgfqpoint{2.102341in}{0.646844in}}%
\pgfpathlineto{\pgfqpoint{2.102594in}{0.650356in}}%
\pgfpathlineto{\pgfqpoint{2.102931in}{0.644667in}}%
\pgfpathlineto{\pgfqpoint{2.103352in}{0.645852in}}%
\pgfpathlineto{\pgfqpoint{2.103774in}{0.648308in}}%
\pgfpathlineto{\pgfqpoint{2.104110in}{0.698522in}}%
\pgfpathlineto{\pgfqpoint{2.104363in}{0.733667in}}%
\pgfpathlineto{\pgfqpoint{2.105037in}{0.666394in}}%
\pgfpathlineto{\pgfqpoint{2.106722in}{0.651063in}}%
\pgfpathlineto{\pgfqpoint{2.106891in}{0.651628in}}%
\pgfpathlineto{\pgfqpoint{2.107312in}{0.673310in}}%
\pgfpathlineto{\pgfqpoint{2.107817in}{0.794175in}}%
\pgfpathlineto{\pgfqpoint{2.108407in}{0.686025in}}%
\pgfpathlineto{\pgfqpoint{2.109839in}{0.662639in}}%
\pgfpathlineto{\pgfqpoint{2.109923in}{0.662698in}}%
\pgfpathlineto{\pgfqpoint{2.110260in}{0.673461in}}%
\pgfpathlineto{\pgfqpoint{2.110513in}{0.688009in}}%
\pgfpathlineto{\pgfqpoint{2.111019in}{0.656036in}}%
\pgfpathlineto{\pgfqpoint{2.111440in}{0.650167in}}%
\pgfpathlineto{\pgfqpoint{2.112198in}{0.652357in}}%
\pgfpathlineto{\pgfqpoint{2.112956in}{0.650983in}}%
\pgfpathlineto{\pgfqpoint{2.113630in}{0.651250in}}%
\pgfpathlineto{\pgfqpoint{2.114304in}{0.650245in}}%
\pgfpathlineto{\pgfqpoint{2.114641in}{0.648416in}}%
\pgfpathlineto{\pgfqpoint{2.115484in}{0.649481in}}%
\pgfpathlineto{\pgfqpoint{2.115905in}{0.651020in}}%
\pgfpathlineto{\pgfqpoint{2.117084in}{0.650509in}}%
\pgfpathlineto{\pgfqpoint{2.117337in}{0.651628in}}%
\pgfpathlineto{\pgfqpoint{2.117758in}{0.666133in}}%
\pgfpathlineto{\pgfqpoint{2.118432in}{0.652058in}}%
\pgfpathlineto{\pgfqpoint{2.118769in}{0.652800in}}%
\pgfpathlineto{\pgfqpoint{2.119106in}{0.651066in}}%
\pgfpathlineto{\pgfqpoint{2.119359in}{0.652144in}}%
\pgfpathlineto{\pgfqpoint{2.119611in}{0.654298in}}%
\pgfpathlineto{\pgfqpoint{2.120538in}{0.653043in}}%
\pgfpathlineto{\pgfqpoint{2.120875in}{0.653122in}}%
\pgfpathlineto{\pgfqpoint{2.121128in}{0.653932in}}%
\pgfpathlineto{\pgfqpoint{2.121970in}{0.655374in}}%
\pgfpathlineto{\pgfqpoint{2.122223in}{0.654144in}}%
\pgfpathlineto{\pgfqpoint{2.122729in}{0.650765in}}%
\pgfpathlineto{\pgfqpoint{2.123234in}{0.654577in}}%
\pgfpathlineto{\pgfqpoint{2.123908in}{0.738109in}}%
\pgfpathlineto{\pgfqpoint{2.123992in}{0.742611in}}%
\pgfpathlineto{\pgfqpoint{2.124413in}{0.706085in}}%
\pgfpathlineto{\pgfqpoint{2.124666in}{0.696594in}}%
\pgfpathlineto{\pgfqpoint{2.125340in}{0.713103in}}%
\pgfpathlineto{\pgfqpoint{2.125424in}{0.713217in}}%
\pgfpathlineto{\pgfqpoint{2.125509in}{0.712671in}}%
\pgfpathlineto{\pgfqpoint{2.126435in}{0.692294in}}%
\pgfpathlineto{\pgfqpoint{2.126688in}{0.702361in}}%
\pgfpathlineto{\pgfqpoint{2.127278in}{0.770182in}}%
\pgfpathlineto{\pgfqpoint{2.127783in}{0.701075in}}%
\pgfpathlineto{\pgfqpoint{2.128204in}{0.680184in}}%
\pgfpathlineto{\pgfqpoint{2.128710in}{0.705234in}}%
\pgfpathlineto{\pgfqpoint{2.129047in}{0.724882in}}%
\pgfpathlineto{\pgfqpoint{2.129721in}{0.701957in}}%
\pgfpathlineto{\pgfqpoint{2.129889in}{0.702688in}}%
\pgfpathlineto{\pgfqpoint{2.130058in}{0.699275in}}%
\pgfpathlineto{\pgfqpoint{2.130816in}{0.676037in}}%
\pgfpathlineto{\pgfqpoint{2.131237in}{0.688961in}}%
\pgfpathlineto{\pgfqpoint{2.131743in}{0.718320in}}%
\pgfpathlineto{\pgfqpoint{2.132332in}{0.690891in}}%
\pgfpathlineto{\pgfqpoint{2.133765in}{0.675603in}}%
\pgfpathlineto{\pgfqpoint{2.133933in}{0.676011in}}%
\pgfpathlineto{\pgfqpoint{2.134607in}{0.686824in}}%
\pgfpathlineto{\pgfqpoint{2.135028in}{0.679794in}}%
\pgfpathlineto{\pgfqpoint{2.136039in}{0.657708in}}%
\pgfpathlineto{\pgfqpoint{2.136629in}{0.659593in}}%
\pgfpathlineto{\pgfqpoint{2.137050in}{0.660820in}}%
\pgfpathlineto{\pgfqpoint{2.137808in}{0.660372in}}%
\pgfpathlineto{\pgfqpoint{2.138145in}{0.656251in}}%
\pgfpathlineto{\pgfqpoint{2.139072in}{0.645549in}}%
\pgfpathlineto{\pgfqpoint{2.139493in}{0.649459in}}%
\pgfpathlineto{\pgfqpoint{2.139746in}{0.651496in}}%
\pgfpathlineto{\pgfqpoint{2.140251in}{0.644715in}}%
\pgfpathlineto{\pgfqpoint{2.140420in}{0.644315in}}%
\pgfpathlineto{\pgfqpoint{2.140757in}{0.646876in}}%
\pgfpathlineto{\pgfqpoint{2.141262in}{0.651258in}}%
\pgfpathlineto{\pgfqpoint{2.141936in}{0.648710in}}%
\pgfpathlineto{\pgfqpoint{2.142863in}{0.636940in}}%
\pgfpathlineto{\pgfqpoint{2.143284in}{0.644376in}}%
\pgfpathlineto{\pgfqpoint{2.143621in}{0.655420in}}%
\pgfpathlineto{\pgfqpoint{2.144211in}{0.640772in}}%
\pgfpathlineto{\pgfqpoint{2.144464in}{0.639017in}}%
\pgfpathlineto{\pgfqpoint{2.144885in}{0.645387in}}%
\pgfpathlineto{\pgfqpoint{2.145475in}{0.661778in}}%
\pgfpathlineto{\pgfqpoint{2.146233in}{0.656007in}}%
\pgfpathlineto{\pgfqpoint{2.146991in}{0.643630in}}%
\pgfpathlineto{\pgfqpoint{2.147412in}{0.650592in}}%
\pgfpathlineto{\pgfqpoint{2.147833in}{0.662542in}}%
\pgfpathlineto{\pgfqpoint{2.148255in}{0.645344in}}%
\pgfpathlineto{\pgfqpoint{2.148844in}{0.637074in}}%
\pgfpathlineto{\pgfqpoint{2.149518in}{0.638536in}}%
\pgfpathlineto{\pgfqpoint{2.150276in}{0.639463in}}%
\pgfpathlineto{\pgfqpoint{2.151203in}{0.641883in}}%
\pgfpathlineto{\pgfqpoint{2.151456in}{0.640498in}}%
\pgfpathlineto{\pgfqpoint{2.152214in}{0.626076in}}%
\pgfpathlineto{\pgfqpoint{2.152972in}{0.632863in}}%
\pgfpathlineto{\pgfqpoint{2.154994in}{0.645784in}}%
\pgfpathlineto{\pgfqpoint{2.155415in}{0.664151in}}%
\pgfpathlineto{\pgfqpoint{2.155921in}{0.641349in}}%
\pgfpathlineto{\pgfqpoint{2.156258in}{0.633569in}}%
\pgfpathlineto{\pgfqpoint{2.157016in}{0.638971in}}%
\pgfpathlineto{\pgfqpoint{2.158532in}{0.644936in}}%
\pgfpathlineto{\pgfqpoint{2.158701in}{0.644810in}}%
\pgfpathlineto{\pgfqpoint{2.159291in}{0.644502in}}%
\pgfpathlineto{\pgfqpoint{2.159712in}{0.645187in}}%
\pgfpathlineto{\pgfqpoint{2.162913in}{0.650062in}}%
\pgfpathlineto{\pgfqpoint{2.163587in}{0.649047in}}%
\pgfpathlineto{\pgfqpoint{2.164177in}{0.650074in}}%
\pgfpathlineto{\pgfqpoint{2.166873in}{0.659912in}}%
\pgfpathlineto{\pgfqpoint{2.167547in}{0.656429in}}%
\pgfpathlineto{\pgfqpoint{2.168221in}{0.652014in}}%
\pgfpathlineto{\pgfqpoint{2.168642in}{0.656369in}}%
\pgfpathlineto{\pgfqpoint{2.170158in}{0.669819in}}%
\pgfpathlineto{\pgfqpoint{2.170411in}{0.668893in}}%
\pgfpathlineto{\pgfqpoint{2.170579in}{0.667966in}}%
\pgfpathlineto{\pgfqpoint{2.171169in}{0.670583in}}%
\pgfpathlineto{\pgfqpoint{2.171927in}{0.683530in}}%
\pgfpathlineto{\pgfqpoint{2.172770in}{0.708545in}}%
\pgfpathlineto{\pgfqpoint{2.173191in}{0.693216in}}%
\pgfpathlineto{\pgfqpoint{2.174033in}{0.642860in}}%
\pgfpathlineto{\pgfqpoint{2.174539in}{0.653165in}}%
\pgfpathlineto{\pgfqpoint{2.175044in}{0.684922in}}%
\pgfpathlineto{\pgfqpoint{2.175803in}{0.666327in}}%
\pgfpathlineto{\pgfqpoint{2.176140in}{0.671493in}}%
\pgfpathlineto{\pgfqpoint{2.177487in}{0.718984in}}%
\pgfpathlineto{\pgfqpoint{2.177740in}{0.699908in}}%
\pgfpathlineto{\pgfqpoint{2.178583in}{0.640616in}}%
\pgfpathlineto{\pgfqpoint{2.179088in}{0.648271in}}%
\pgfpathlineto{\pgfqpoint{2.179594in}{0.698206in}}%
\pgfpathlineto{\pgfqpoint{2.180436in}{0.662737in}}%
\pgfpathlineto{\pgfqpoint{2.181194in}{0.685557in}}%
\pgfpathlineto{\pgfqpoint{2.181952in}{0.676105in}}%
\pgfpathlineto{\pgfqpoint{2.183469in}{0.633116in}}%
\pgfpathlineto{\pgfqpoint{2.184227in}{0.641250in}}%
\pgfpathlineto{\pgfqpoint{2.184985in}{0.670351in}}%
\pgfpathlineto{\pgfqpoint{2.185322in}{0.682813in}}%
\pgfpathlineto{\pgfqpoint{2.185996in}{0.670440in}}%
\pgfpathlineto{\pgfqpoint{2.187934in}{0.634309in}}%
\pgfpathlineto{\pgfqpoint{2.188439in}{0.647765in}}%
\pgfpathlineto{\pgfqpoint{2.188945in}{0.666601in}}%
\pgfpathlineto{\pgfqpoint{2.189534in}{0.651809in}}%
\pgfpathlineto{\pgfqpoint{2.190798in}{0.643491in}}%
\pgfpathlineto{\pgfqpoint{2.191051in}{0.645800in}}%
\pgfpathlineto{\pgfqpoint{2.191304in}{0.650114in}}%
\pgfpathlineto{\pgfqpoint{2.191725in}{0.639661in}}%
\pgfpathlineto{\pgfqpoint{2.192567in}{0.632544in}}%
\pgfpathlineto{\pgfqpoint{2.192988in}{0.634448in}}%
\pgfpathlineto{\pgfqpoint{2.195179in}{0.647457in}}%
\pgfpathlineto{\pgfqpoint{2.195347in}{0.644745in}}%
\pgfpathlineto{\pgfqpoint{2.195769in}{0.639113in}}%
\pgfpathlineto{\pgfqpoint{2.196527in}{0.642597in}}%
\pgfpathlineto{\pgfqpoint{2.198717in}{0.647550in}}%
\pgfpathlineto{\pgfqpoint{2.200318in}{0.650370in}}%
\pgfpathlineto{\pgfqpoint{2.200739in}{0.666605in}}%
\pgfpathlineto{\pgfqpoint{2.201076in}{0.677731in}}%
\pgfpathlineto{\pgfqpoint{2.201750in}{0.664093in}}%
\pgfpathlineto{\pgfqpoint{2.202087in}{0.666479in}}%
\pgfpathlineto{\pgfqpoint{2.203182in}{0.700031in}}%
\pgfpathlineto{\pgfqpoint{2.203603in}{0.774467in}}%
\pgfpathlineto{\pgfqpoint{2.204109in}{0.680946in}}%
\pgfpathlineto{\pgfqpoint{2.204446in}{0.664526in}}%
\pgfpathlineto{\pgfqpoint{2.205035in}{0.689571in}}%
\pgfpathlineto{\pgfqpoint{2.205457in}{0.706854in}}%
\pgfpathlineto{\pgfqpoint{2.206131in}{0.694130in}}%
\pgfpathlineto{\pgfqpoint{2.206468in}{0.694525in}}%
\pgfpathlineto{\pgfqpoint{2.206720in}{0.689219in}}%
\pgfpathlineto{\pgfqpoint{2.207984in}{0.664129in}}%
\pgfpathlineto{\pgfqpoint{2.208489in}{0.665362in}}%
\pgfpathlineto{\pgfqpoint{2.210259in}{0.657616in}}%
\pgfpathlineto{\pgfqpoint{2.213291in}{0.649690in}}%
\pgfpathlineto{\pgfqpoint{2.211185in}{0.658169in}}%
\pgfpathlineto{\pgfqpoint{2.213460in}{0.649803in}}%
\pgfpathlineto{\pgfqpoint{2.213797in}{0.649955in}}%
\pgfpathlineto{\pgfqpoint{2.214134in}{0.649000in}}%
\pgfpathlineto{\pgfqpoint{2.214976in}{0.648570in}}%
\pgfpathlineto{\pgfqpoint{2.215313in}{0.648809in}}%
\pgfpathlineto{\pgfqpoint{2.217082in}{0.651372in}}%
\pgfpathlineto{\pgfqpoint{2.217419in}{0.650142in}}%
\pgfpathlineto{\pgfqpoint{2.217841in}{0.649075in}}%
\pgfpathlineto{\pgfqpoint{2.218430in}{0.650682in}}%
\pgfpathlineto{\pgfqpoint{2.220958in}{0.668770in}}%
\pgfpathlineto{\pgfqpoint{2.221969in}{0.663244in}}%
\pgfpathlineto{\pgfqpoint{2.222811in}{0.662457in}}%
\pgfpathlineto{\pgfqpoint{2.223148in}{0.661414in}}%
\pgfpathlineto{\pgfqpoint{2.223401in}{0.662845in}}%
\pgfpathlineto{\pgfqpoint{2.224075in}{0.696882in}}%
\pgfpathlineto{\pgfqpoint{2.224496in}{0.667888in}}%
\pgfpathlineto{\pgfqpoint{2.225423in}{0.648795in}}%
\pgfpathlineto{\pgfqpoint{2.225844in}{0.649037in}}%
\pgfpathlineto{\pgfqpoint{2.226686in}{0.649181in}}%
\pgfpathlineto{\pgfqpoint{2.226771in}{0.649972in}}%
\pgfpathlineto{\pgfqpoint{2.227360in}{0.663734in}}%
\pgfpathlineto{\pgfqpoint{2.227866in}{0.649837in}}%
\pgfpathlineto{\pgfqpoint{2.228118in}{0.648856in}}%
\pgfpathlineto{\pgfqpoint{2.228961in}{0.649656in}}%
\pgfpathlineto{\pgfqpoint{2.229466in}{0.650734in}}%
\pgfpathlineto{\pgfqpoint{2.232920in}{0.694649in}}%
\pgfpathlineto{\pgfqpoint{2.233173in}{0.684902in}}%
\pgfpathlineto{\pgfqpoint{2.233931in}{0.646506in}}%
\pgfpathlineto{\pgfqpoint{2.234521in}{0.652502in}}%
\pgfpathlineto{\pgfqpoint{2.235869in}{0.674139in}}%
\pgfpathlineto{\pgfqpoint{2.236037in}{0.670095in}}%
\pgfpathlineto{\pgfqpoint{2.237048in}{0.643419in}}%
\pgfpathlineto{\pgfqpoint{2.237638in}{0.644979in}}%
\pgfpathlineto{\pgfqpoint{2.239576in}{0.650044in}}%
\pgfpathlineto{\pgfqpoint{2.240334in}{0.649900in}}%
\pgfpathlineto{\pgfqpoint{2.240924in}{0.653161in}}%
\pgfpathlineto{\pgfqpoint{2.241261in}{0.652896in}}%
\pgfpathlineto{\pgfqpoint{2.241766in}{0.658443in}}%
\pgfpathlineto{\pgfqpoint{2.242608in}{0.699445in}}%
\pgfpathlineto{\pgfqpoint{2.243198in}{0.747071in}}%
\pgfpathlineto{\pgfqpoint{2.243788in}{0.716180in}}%
\pgfpathlineto{\pgfqpoint{2.245136in}{0.677256in}}%
\pgfpathlineto{\pgfqpoint{2.245473in}{0.691100in}}%
\pgfpathlineto{\pgfqpoint{2.245978in}{0.741610in}}%
\pgfpathlineto{\pgfqpoint{2.246652in}{0.705330in}}%
\pgfpathlineto{\pgfqpoint{2.247242in}{0.710012in}}%
\pgfpathlineto{\pgfqpoint{2.247579in}{0.704489in}}%
\pgfpathlineto{\pgfqpoint{2.250696in}{0.654077in}}%
\pgfpathlineto{\pgfqpoint{2.251538in}{0.653579in}}%
\pgfpathlineto{\pgfqpoint{2.251875in}{0.654176in}}%
\pgfpathlineto{\pgfqpoint{2.252212in}{0.655808in}}%
\pgfpathlineto{\pgfqpoint{2.252718in}{0.653092in}}%
\pgfpathlineto{\pgfqpoint{2.254066in}{0.652142in}}%
\pgfpathlineto{\pgfqpoint{2.255077in}{0.653033in}}%
\pgfpathlineto{\pgfqpoint{2.255751in}{0.663181in}}%
\pgfpathlineto{\pgfqpoint{2.256256in}{0.671570in}}%
\pgfpathlineto{\pgfqpoint{2.256930in}{0.665021in}}%
\pgfpathlineto{\pgfqpoint{2.258025in}{0.659622in}}%
\pgfpathlineto{\pgfqpoint{2.258362in}{0.663134in}}%
\pgfpathlineto{\pgfqpoint{2.259879in}{0.675148in}}%
\pgfpathlineto{\pgfqpoint{2.260721in}{0.679291in}}%
\pgfpathlineto{\pgfqpoint{2.261142in}{0.677836in}}%
\pgfpathlineto{\pgfqpoint{2.263080in}{0.655489in}}%
\pgfpathlineto{\pgfqpoint{2.263585in}{0.650591in}}%
\pgfpathlineto{\pgfqpoint{2.264091in}{0.656740in}}%
\pgfpathlineto{\pgfqpoint{2.265270in}{0.661039in}}%
\pgfpathlineto{\pgfqpoint{2.265355in}{0.660335in}}%
\pgfpathlineto{\pgfqpoint{2.266281in}{0.645184in}}%
\pgfpathlineto{\pgfqpoint{2.267039in}{0.649958in}}%
\pgfpathlineto{\pgfqpoint{2.269482in}{0.645308in}}%
\pgfpathlineto{\pgfqpoint{2.272178in}{0.645631in}}%
\pgfpathlineto{\pgfqpoint{2.276475in}{0.652712in}}%
\pgfpathlineto{\pgfqpoint{2.278834in}{0.655661in}}%
\pgfpathlineto{\pgfqpoint{2.279339in}{0.653902in}}%
\pgfpathlineto{\pgfqpoint{2.279592in}{0.653815in}}%
\pgfpathlineto{\pgfqpoint{2.279845in}{0.654748in}}%
\pgfpathlineto{\pgfqpoint{2.280350in}{0.673371in}}%
\pgfpathlineto{\pgfqpoint{2.280603in}{0.681340in}}%
\pgfpathlineto{\pgfqpoint{2.281108in}{0.673308in}}%
\pgfpathlineto{\pgfqpoint{2.281445in}{0.674612in}}%
\pgfpathlineto{\pgfqpoint{2.282372in}{0.681299in}}%
\pgfpathlineto{\pgfqpoint{2.282877in}{0.704387in}}%
\pgfpathlineto{\pgfqpoint{2.283299in}{0.681884in}}%
\pgfpathlineto{\pgfqpoint{2.284141in}{0.656110in}}%
\pgfpathlineto{\pgfqpoint{2.284731in}{0.658578in}}%
\pgfpathlineto{\pgfqpoint{2.286331in}{0.663181in}}%
\pgfpathlineto{\pgfqpoint{2.286416in}{0.662585in}}%
\pgfpathlineto{\pgfqpoint{2.287258in}{0.652737in}}%
\pgfpathlineto{\pgfqpoint{2.288185in}{0.653220in}}%
\pgfpathlineto{\pgfqpoint{2.293408in}{0.653994in}}%
\pgfpathlineto{\pgfqpoint{2.295767in}{0.653473in}}%
\pgfpathlineto{\pgfqpoint{2.295851in}{0.653608in}}%
\pgfpathlineto{\pgfqpoint{2.297957in}{0.655434in}}%
\pgfpathlineto{\pgfqpoint{2.298968in}{0.654039in}}%
\pgfpathlineto{\pgfqpoint{2.299389in}{0.655232in}}%
\pgfpathlineto{\pgfqpoint{2.299726in}{0.656123in}}%
\pgfpathlineto{\pgfqpoint{2.300063in}{0.654199in}}%
\pgfpathlineto{\pgfqpoint{2.301158in}{0.651840in}}%
\pgfpathlineto{\pgfqpoint{2.301411in}{0.652011in}}%
\pgfpathlineto{\pgfqpoint{2.301832in}{0.653783in}}%
\pgfpathlineto{\pgfqpoint{2.302338in}{0.718251in}}%
\pgfpathlineto{\pgfqpoint{2.303096in}{0.662005in}}%
\pgfpathlineto{\pgfqpoint{2.303602in}{0.666812in}}%
\pgfpathlineto{\pgfqpoint{2.303939in}{0.660941in}}%
\pgfpathlineto{\pgfqpoint{2.304360in}{0.657034in}}%
\pgfpathlineto{\pgfqpoint{2.305118in}{0.659459in}}%
\pgfpathlineto{\pgfqpoint{2.305371in}{0.664309in}}%
\pgfpathlineto{\pgfqpoint{2.306129in}{0.787953in}}%
\pgfpathlineto{\pgfqpoint{2.306971in}{0.701269in}}%
\pgfpathlineto{\pgfqpoint{2.309330in}{0.650318in}}%
\pgfpathlineto{\pgfqpoint{2.309751in}{0.657641in}}%
\pgfpathlineto{\pgfqpoint{2.310341in}{0.699312in}}%
\pgfpathlineto{\pgfqpoint{2.311099in}{0.674863in}}%
\pgfpathlineto{\pgfqpoint{2.311268in}{0.675527in}}%
\pgfpathlineto{\pgfqpoint{2.311521in}{0.670827in}}%
\pgfpathlineto{\pgfqpoint{2.313121in}{0.647721in}}%
\pgfpathlineto{\pgfqpoint{2.313542in}{0.651343in}}%
\pgfpathlineto{\pgfqpoint{2.313964in}{0.659656in}}%
\pgfpathlineto{\pgfqpoint{2.314722in}{0.654078in}}%
\pgfpathlineto{\pgfqpoint{2.314890in}{0.653782in}}%
\pgfpathlineto{\pgfqpoint{2.314975in}{0.653176in}}%
\pgfpathlineto{\pgfqpoint{2.315649in}{0.648719in}}%
\pgfpathlineto{\pgfqpoint{2.315985in}{0.653036in}}%
\pgfpathlineto{\pgfqpoint{2.316491in}{0.692569in}}%
\pgfpathlineto{\pgfqpoint{2.317081in}{0.653744in}}%
\pgfpathlineto{\pgfqpoint{2.317502in}{0.650311in}}%
\pgfpathlineto{\pgfqpoint{2.318092in}{0.654816in}}%
\pgfpathlineto{\pgfqpoint{2.318513in}{0.654201in}}%
\pgfpathlineto{\pgfqpoint{2.318850in}{0.665520in}}%
\pgfpathlineto{\pgfqpoint{2.319187in}{0.692719in}}%
\pgfpathlineto{\pgfqpoint{2.319776in}{0.650748in}}%
\pgfpathlineto{\pgfqpoint{2.319945in}{0.648543in}}%
\pgfpathlineto{\pgfqpoint{2.320366in}{0.659937in}}%
\pgfpathlineto{\pgfqpoint{2.320872in}{0.704512in}}%
\pgfpathlineto{\pgfqpoint{2.321714in}{0.681475in}}%
\pgfpathlineto{\pgfqpoint{2.321967in}{0.688144in}}%
\pgfpathlineto{\pgfqpoint{2.322557in}{0.680256in}}%
\pgfpathlineto{\pgfqpoint{2.322809in}{0.681115in}}%
\pgfpathlineto{\pgfqpoint{2.323231in}{0.673301in}}%
\pgfpathlineto{\pgfqpoint{2.323399in}{0.679318in}}%
\pgfpathlineto{\pgfqpoint{2.324241in}{0.829576in}}%
\pgfpathlineto{\pgfqpoint{2.324915in}{0.734593in}}%
\pgfpathlineto{\pgfqpoint{2.326348in}{0.677013in}}%
\pgfpathlineto{\pgfqpoint{2.326769in}{0.682025in}}%
\pgfpathlineto{\pgfqpoint{2.327190in}{0.685217in}}%
\pgfpathlineto{\pgfqpoint{2.327864in}{0.682061in}}%
\pgfpathlineto{\pgfqpoint{2.330139in}{0.656009in}}%
\pgfpathlineto{\pgfqpoint{2.330981in}{0.647484in}}%
\pgfpathlineto{\pgfqpoint{2.331486in}{0.647670in}}%
\pgfpathlineto{\pgfqpoint{2.335446in}{0.651279in}}%
\pgfpathlineto{\pgfqpoint{2.336036in}{0.652732in}}%
\pgfpathlineto{\pgfqpoint{2.336794in}{0.652067in}}%
\pgfpathlineto{\pgfqpoint{2.337552in}{0.650516in}}%
\pgfpathlineto{\pgfqpoint{2.337889in}{0.651426in}}%
\pgfpathlineto{\pgfqpoint{2.338395in}{0.666397in}}%
\pgfpathlineto{\pgfqpoint{2.338647in}{0.677095in}}%
\pgfpathlineto{\pgfqpoint{2.339237in}{0.654942in}}%
\pgfpathlineto{\pgfqpoint{2.340585in}{0.650604in}}%
\pgfpathlineto{\pgfqpoint{2.341259in}{0.651610in}}%
\pgfpathlineto{\pgfqpoint{2.341849in}{0.658051in}}%
\pgfpathlineto{\pgfqpoint{2.342101in}{0.659905in}}%
\pgfpathlineto{\pgfqpoint{2.342607in}{0.653647in}}%
\pgfpathlineto{\pgfqpoint{2.343365in}{0.652769in}}%
\pgfpathlineto{\pgfqpoint{2.343786in}{0.653077in}}%
\pgfpathlineto{\pgfqpoint{2.344207in}{0.654257in}}%
\pgfpathlineto{\pgfqpoint{2.344629in}{0.660087in}}%
\pgfpathlineto{\pgfqpoint{2.345134in}{0.653186in}}%
\pgfpathlineto{\pgfqpoint{2.345640in}{0.651009in}}%
\pgfpathlineto{\pgfqpoint{2.346145in}{0.653985in}}%
\pgfpathlineto{\pgfqpoint{2.346566in}{0.658923in}}%
\pgfpathlineto{\pgfqpoint{2.347409in}{0.657498in}}%
\pgfpathlineto{\pgfqpoint{2.347493in}{0.657546in}}%
\pgfpathlineto{\pgfqpoint{2.347661in}{0.656460in}}%
\pgfpathlineto{\pgfqpoint{2.348588in}{0.649923in}}%
\pgfpathlineto{\pgfqpoint{2.349178in}{0.650197in}}%
\pgfpathlineto{\pgfqpoint{2.349431in}{0.650340in}}%
\pgfpathlineto{\pgfqpoint{2.349852in}{0.649441in}}%
\pgfpathlineto{\pgfqpoint{2.350863in}{0.649507in}}%
\pgfpathlineto{\pgfqpoint{2.351031in}{0.649749in}}%
\pgfpathlineto{\pgfqpoint{2.353727in}{0.652602in}}%
\pgfpathlineto{\pgfqpoint{2.355833in}{0.653224in}}%
\pgfpathlineto{\pgfqpoint{2.357939in}{0.723329in}}%
\pgfpathlineto{\pgfqpoint{2.358613in}{0.700172in}}%
\pgfpathlineto{\pgfqpoint{2.359371in}{0.686981in}}%
\pgfpathlineto{\pgfqpoint{2.360214in}{0.658570in}}%
\pgfpathlineto{\pgfqpoint{2.360804in}{0.668231in}}%
\pgfpathlineto{\pgfqpoint{2.361056in}{0.673440in}}%
\pgfpathlineto{\pgfqpoint{2.361646in}{0.662602in}}%
\pgfpathlineto{\pgfqpoint{2.364173in}{0.653803in}}%
\pgfpathlineto{\pgfqpoint{2.366953in}{0.652788in}}%
\pgfpathlineto{\pgfqpoint{2.367543in}{0.653695in}}%
\pgfpathlineto{\pgfqpoint{2.367880in}{0.655735in}}%
\pgfpathlineto{\pgfqpoint{2.368470in}{0.652626in}}%
\pgfpathlineto{\pgfqpoint{2.369144in}{0.653125in}}%
\pgfpathlineto{\pgfqpoint{2.369818in}{0.668135in}}%
\pgfpathlineto{\pgfqpoint{2.369986in}{0.670006in}}%
\pgfpathlineto{\pgfqpoint{2.370576in}{0.663846in}}%
\pgfpathlineto{\pgfqpoint{2.370660in}{0.664082in}}%
\pgfpathlineto{\pgfqpoint{2.370913in}{0.665234in}}%
\pgfpathlineto{\pgfqpoint{2.371250in}{0.661406in}}%
\pgfpathlineto{\pgfqpoint{2.372682in}{0.650025in}}%
\pgfpathlineto{\pgfqpoint{2.373019in}{0.650588in}}%
\pgfpathlineto{\pgfqpoint{2.375209in}{0.656732in}}%
\pgfpathlineto{\pgfqpoint{2.375546in}{0.668114in}}%
\pgfpathlineto{\pgfqpoint{2.376052in}{0.650925in}}%
\pgfpathlineto{\pgfqpoint{2.376220in}{0.649908in}}%
\pgfpathlineto{\pgfqpoint{2.377147in}{0.650485in}}%
\pgfpathlineto{\pgfqpoint{2.377905in}{0.653087in}}%
\pgfpathlineto{\pgfqpoint{2.378326in}{0.668362in}}%
\pgfpathlineto{\pgfqpoint{2.379085in}{0.656359in}}%
\pgfpathlineto{\pgfqpoint{2.379506in}{0.656872in}}%
\pgfpathlineto{\pgfqpoint{2.380601in}{0.671376in}}%
\pgfpathlineto{\pgfqpoint{2.381865in}{0.762093in}}%
\pgfpathlineto{\pgfqpoint{2.382454in}{0.732168in}}%
\pgfpathlineto{\pgfqpoint{2.383887in}{0.704853in}}%
\pgfpathlineto{\pgfqpoint{2.384308in}{0.705922in}}%
\pgfpathlineto{\pgfqpoint{2.384982in}{0.666897in}}%
\pgfpathlineto{\pgfqpoint{2.385403in}{0.657578in}}%
\pgfpathlineto{\pgfqpoint{2.385740in}{0.671408in}}%
\pgfpathlineto{\pgfqpoint{2.386161in}{0.724710in}}%
\pgfpathlineto{\pgfqpoint{2.386835in}{0.672746in}}%
\pgfpathlineto{\pgfqpoint{2.387088in}{0.671460in}}%
\pgfpathlineto{\pgfqpoint{2.387762in}{0.674225in}}%
\pgfpathlineto{\pgfqpoint{2.387846in}{0.674436in}}%
\pgfpathlineto{\pgfqpoint{2.388267in}{0.672695in}}%
\pgfpathlineto{\pgfqpoint{2.388520in}{0.671738in}}%
\pgfpathlineto{\pgfqpoint{2.388857in}{0.674898in}}%
\pgfpathlineto{\pgfqpoint{2.389025in}{0.676098in}}%
\pgfpathlineto{\pgfqpoint{2.389278in}{0.670273in}}%
\pgfpathlineto{\pgfqpoint{2.390626in}{0.650947in}}%
\pgfpathlineto{\pgfqpoint{2.390879in}{0.650995in}}%
\pgfpathlineto{\pgfqpoint{2.393912in}{0.651112in}}%
\pgfpathlineto{\pgfqpoint{2.397787in}{0.652179in}}%
\pgfpathlineto{\pgfqpoint{2.398461in}{0.664781in}}%
\pgfpathlineto{\pgfqpoint{2.399472in}{0.656609in}}%
\pgfpathlineto{\pgfqpoint{2.400230in}{0.653577in}}%
\pgfpathlineto{\pgfqpoint{2.400651in}{0.656049in}}%
\pgfpathlineto{\pgfqpoint{2.402505in}{0.668793in}}%
\pgfpathlineto{\pgfqpoint{2.403516in}{0.671386in}}%
\pgfpathlineto{\pgfqpoint{2.403853in}{0.669541in}}%
\pgfpathlineto{\pgfqpoint{2.405032in}{0.655338in}}%
\pgfpathlineto{\pgfqpoint{2.405453in}{0.662298in}}%
\pgfpathlineto{\pgfqpoint{2.405959in}{0.670596in}}%
\pgfpathlineto{\pgfqpoint{2.406464in}{0.661834in}}%
\pgfpathlineto{\pgfqpoint{2.408065in}{0.657217in}}%
\pgfpathlineto{\pgfqpoint{2.408149in}{0.657448in}}%
\pgfpathlineto{\pgfqpoint{2.408739in}{0.668743in}}%
\pgfpathlineto{\pgfqpoint{2.408991in}{0.658585in}}%
\pgfpathlineto{\pgfqpoint{2.409497in}{0.647560in}}%
\pgfpathlineto{\pgfqpoint{2.410255in}{0.647680in}}%
\pgfpathlineto{\pgfqpoint{2.410845in}{0.648734in}}%
\pgfpathlineto{\pgfqpoint{2.411013in}{0.649209in}}%
\pgfpathlineto{\pgfqpoint{2.411687in}{0.648208in}}%
\pgfpathlineto{\pgfqpoint{2.411772in}{0.648198in}}%
\pgfpathlineto{\pgfqpoint{2.414299in}{0.647105in}}%
\pgfpathlineto{\pgfqpoint{2.414552in}{0.646728in}}%
\pgfpathlineto{\pgfqpoint{2.415141in}{0.647727in}}%
\pgfpathlineto{\pgfqpoint{2.415563in}{0.651444in}}%
\pgfpathlineto{\pgfqpoint{2.415899in}{0.656707in}}%
\pgfpathlineto{\pgfqpoint{2.416489in}{0.648451in}}%
\pgfpathlineto{\pgfqpoint{2.417079in}{0.649225in}}%
\pgfpathlineto{\pgfqpoint{2.417416in}{0.652667in}}%
\pgfpathlineto{\pgfqpoint{2.417921in}{0.662916in}}%
\pgfpathlineto{\pgfqpoint{2.418764in}{0.659747in}}%
\pgfpathlineto{\pgfqpoint{2.420533in}{0.656274in}}%
\pgfpathlineto{\pgfqpoint{2.420954in}{0.658415in}}%
\pgfpathlineto{\pgfqpoint{2.421375in}{0.656837in}}%
\pgfpathlineto{\pgfqpoint{2.422892in}{0.662683in}}%
\pgfpathlineto{\pgfqpoint{2.423229in}{0.663266in}}%
\pgfpathlineto{\pgfqpoint{2.423481in}{0.662174in}}%
\pgfpathlineto{\pgfqpoint{2.424408in}{0.654192in}}%
\pgfpathlineto{\pgfqpoint{2.424829in}{0.658159in}}%
\pgfpathlineto{\pgfqpoint{2.425672in}{0.673850in}}%
\pgfpathlineto{\pgfqpoint{2.426009in}{0.660567in}}%
\pgfpathlineto{\pgfqpoint{2.427694in}{0.639947in}}%
\pgfpathlineto{\pgfqpoint{2.427778in}{0.639985in}}%
\pgfpathlineto{\pgfqpoint{2.428452in}{0.642932in}}%
\pgfpathlineto{\pgfqpoint{2.429126in}{0.640673in}}%
\pgfpathlineto{\pgfqpoint{2.430221in}{0.642954in}}%
\pgfpathlineto{\pgfqpoint{2.430558in}{0.653474in}}%
\pgfpathlineto{\pgfqpoint{2.430895in}{0.680240in}}%
\pgfpathlineto{\pgfqpoint{2.431485in}{0.641047in}}%
\pgfpathlineto{\pgfqpoint{2.431569in}{0.640749in}}%
\pgfpathlineto{\pgfqpoint{2.432243in}{0.641966in}}%
\pgfpathlineto{\pgfqpoint{2.434349in}{0.661981in}}%
\pgfpathlineto{\pgfqpoint{2.434686in}{0.669519in}}%
\pgfpathlineto{\pgfqpoint{2.435276in}{0.656894in}}%
\pgfpathlineto{\pgfqpoint{2.435613in}{0.652672in}}%
\pgfpathlineto{\pgfqpoint{2.436034in}{0.659354in}}%
\pgfpathlineto{\pgfqpoint{2.436708in}{0.729410in}}%
\pgfpathlineto{\pgfqpoint{2.437635in}{0.698272in}}%
\pgfpathlineto{\pgfqpoint{2.438224in}{0.667052in}}%
\pgfpathlineto{\pgfqpoint{2.438730in}{0.694615in}}%
\pgfpathlineto{\pgfqpoint{2.438982in}{0.703293in}}%
\pgfpathlineto{\pgfqpoint{2.439488in}{0.676501in}}%
\pgfpathlineto{\pgfqpoint{2.439825in}{0.670210in}}%
\pgfpathlineto{\pgfqpoint{2.440330in}{0.681003in}}%
\pgfpathlineto{\pgfqpoint{2.440836in}{0.701230in}}%
\pgfpathlineto{\pgfqpoint{2.441678in}{0.696380in}}%
\pgfpathlineto{\pgfqpoint{2.441763in}{0.696759in}}%
\pgfpathlineto{\pgfqpoint{2.441931in}{0.692714in}}%
\pgfpathlineto{\pgfqpoint{2.443784in}{0.653399in}}%
\pgfpathlineto{\pgfqpoint{2.443869in}{0.653447in}}%
\pgfpathlineto{\pgfqpoint{2.445048in}{0.654648in}}%
\pgfpathlineto{\pgfqpoint{2.444627in}{0.652644in}}%
\pgfpathlineto{\pgfqpoint{2.445132in}{0.654424in}}%
\pgfpathlineto{\pgfqpoint{2.445891in}{0.650963in}}%
\pgfpathlineto{\pgfqpoint{2.446228in}{0.654388in}}%
\pgfpathlineto{\pgfqpoint{2.446817in}{0.696844in}}%
\pgfpathlineto{\pgfqpoint{2.447491in}{0.660011in}}%
\pgfpathlineto{\pgfqpoint{2.448839in}{0.649733in}}%
\pgfpathlineto{\pgfqpoint{2.449260in}{0.656210in}}%
\pgfpathlineto{\pgfqpoint{2.449429in}{0.656851in}}%
\pgfpathlineto{\pgfqpoint{2.449597in}{0.655167in}}%
\pgfpathlineto{\pgfqpoint{2.450187in}{0.621982in}}%
\pgfpathlineto{\pgfqpoint{2.451114in}{0.633723in}}%
\pgfpathlineto{\pgfqpoint{2.451535in}{0.642906in}}%
\pgfpathlineto{\pgfqpoint{2.452377in}{0.638479in}}%
\pgfpathlineto{\pgfqpoint{2.453473in}{0.642060in}}%
\pgfpathlineto{\pgfqpoint{2.454989in}{0.647850in}}%
\pgfpathlineto{\pgfqpoint{2.455242in}{0.646855in}}%
\pgfpathlineto{\pgfqpoint{2.455663in}{0.645143in}}%
\pgfpathlineto{\pgfqpoint{2.456253in}{0.646676in}}%
\pgfpathlineto{\pgfqpoint{2.457938in}{0.649818in}}%
\pgfpathlineto{\pgfqpoint{2.458527in}{0.650775in}}%
\pgfpathlineto{\pgfqpoint{2.459117in}{0.670469in}}%
\pgfpathlineto{\pgfqpoint{2.459538in}{0.651660in}}%
\pgfpathlineto{\pgfqpoint{2.459875in}{0.647428in}}%
\pgfpathlineto{\pgfqpoint{2.460633in}{0.650134in}}%
\pgfpathlineto{\pgfqpoint{2.462739in}{0.656722in}}%
\pgfpathlineto{\pgfqpoint{2.462908in}{0.656035in}}%
\pgfpathlineto{\pgfqpoint{2.464003in}{0.649374in}}%
\pgfpathlineto{\pgfqpoint{2.464509in}{0.651053in}}%
\pgfpathlineto{\pgfqpoint{2.465688in}{0.664370in}}%
\pgfpathlineto{\pgfqpoint{2.466783in}{0.681577in}}%
\pgfpathlineto{\pgfqpoint{2.466952in}{0.676128in}}%
\pgfpathlineto{\pgfqpoint{2.467710in}{0.656246in}}%
\pgfpathlineto{\pgfqpoint{2.468131in}{0.670241in}}%
\pgfpathlineto{\pgfqpoint{2.468384in}{0.682908in}}%
\pgfpathlineto{\pgfqpoint{2.468889in}{0.645653in}}%
\pgfpathlineto{\pgfqpoint{2.469226in}{0.638441in}}%
\pgfpathlineto{\pgfqpoint{2.470069in}{0.643116in}}%
\pgfpathlineto{\pgfqpoint{2.470658in}{0.665470in}}%
\pgfpathlineto{\pgfqpoint{2.471585in}{0.652892in}}%
\pgfpathlineto{\pgfqpoint{2.472006in}{0.647476in}}%
\pgfpathlineto{\pgfqpoint{2.472596in}{0.644580in}}%
\pgfpathlineto{\pgfqpoint{2.473102in}{0.647315in}}%
\pgfpathlineto{\pgfqpoint{2.473270in}{0.648306in}}%
\pgfpathlineto{\pgfqpoint{2.473607in}{0.646858in}}%
\pgfpathlineto{\pgfqpoint{2.474197in}{0.647377in}}%
\pgfpathlineto{\pgfqpoint{2.474449in}{0.646665in}}%
\pgfpathlineto{\pgfqpoint{2.474955in}{0.648484in}}%
\pgfpathlineto{\pgfqpoint{2.477398in}{0.673164in}}%
\pgfpathlineto{\pgfqpoint{2.477651in}{0.678493in}}%
\pgfpathlineto{\pgfqpoint{2.477988in}{0.663652in}}%
\pgfpathlineto{\pgfqpoint{2.478409in}{0.645636in}}%
\pgfpathlineto{\pgfqpoint{2.478999in}{0.664699in}}%
\pgfpathlineto{\pgfqpoint{2.479504in}{0.961661in}}%
\pgfpathlineto{\pgfqpoint{2.480262in}{0.716734in}}%
\pgfpathlineto{\pgfqpoint{2.480684in}{0.706842in}}%
\pgfpathlineto{\pgfqpoint{2.481357in}{0.714489in}}%
\pgfpathlineto{\pgfqpoint{2.481526in}{0.715204in}}%
\pgfpathlineto{\pgfqpoint{2.481779in}{0.710723in}}%
\pgfpathlineto{\pgfqpoint{2.483716in}{0.661172in}}%
\pgfpathlineto{\pgfqpoint{2.484222in}{0.665726in}}%
\pgfpathlineto{\pgfqpoint{2.484475in}{0.665958in}}%
\pgfpathlineto{\pgfqpoint{2.484812in}{0.664824in}}%
\pgfpathlineto{\pgfqpoint{2.485233in}{0.662720in}}%
\pgfpathlineto{\pgfqpoint{2.485991in}{0.663836in}}%
\pgfpathlineto{\pgfqpoint{2.486833in}{0.664526in}}%
\pgfpathlineto{\pgfqpoint{2.487086in}{0.665816in}}%
\pgfpathlineto{\pgfqpoint{2.487339in}{0.661503in}}%
\pgfpathlineto{\pgfqpoint{2.488013in}{0.651804in}}%
\pgfpathlineto{\pgfqpoint{2.488771in}{0.652054in}}%
\pgfpathlineto{\pgfqpoint{2.490372in}{0.652012in}}%
\pgfpathlineto{\pgfqpoint{2.490877in}{0.652064in}}%
\pgfpathlineto{\pgfqpoint{2.491046in}{0.652685in}}%
\pgfpathlineto{\pgfqpoint{2.491635in}{0.657636in}}%
\pgfpathlineto{\pgfqpoint{2.492309in}{0.654033in}}%
\pgfpathlineto{\pgfqpoint{2.492646in}{0.656968in}}%
\pgfpathlineto{\pgfqpoint{2.493236in}{0.673687in}}%
\pgfpathlineto{\pgfqpoint{2.494247in}{0.668765in}}%
\pgfpathlineto{\pgfqpoint{2.494921in}{0.658199in}}%
\pgfpathlineto{\pgfqpoint{2.495258in}{0.665491in}}%
\pgfpathlineto{\pgfqpoint{2.496016in}{0.723598in}}%
\pgfpathlineto{\pgfqpoint{2.496774in}{0.691175in}}%
\pgfpathlineto{\pgfqpoint{2.498122in}{0.673443in}}%
\pgfpathlineto{\pgfqpoint{2.498375in}{0.676343in}}%
\pgfpathlineto{\pgfqpoint{2.498880in}{0.704578in}}%
\pgfpathlineto{\pgfqpoint{2.499302in}{0.671805in}}%
\pgfpathlineto{\pgfqpoint{2.499976in}{0.649681in}}%
\pgfpathlineto{\pgfqpoint{2.500565in}{0.650320in}}%
\pgfpathlineto{\pgfqpoint{2.504440in}{0.656315in}}%
\pgfpathlineto{\pgfqpoint{2.504777in}{0.659997in}}%
\pgfpathlineto{\pgfqpoint{2.505367in}{0.653510in}}%
\pgfpathlineto{\pgfqpoint{2.505704in}{0.652832in}}%
\pgfpathlineto{\pgfqpoint{2.506125in}{0.654331in}}%
\pgfpathlineto{\pgfqpoint{2.507642in}{0.668278in}}%
\pgfpathlineto{\pgfqpoint{2.507979in}{0.663840in}}%
\pgfpathlineto{\pgfqpoint{2.508737in}{0.650568in}}%
\pgfpathlineto{\pgfqpoint{2.509242in}{0.656740in}}%
\pgfpathlineto{\pgfqpoint{2.509664in}{0.666498in}}%
\pgfpathlineto{\pgfqpoint{2.510338in}{0.658460in}}%
\pgfpathlineto{\pgfqpoint{2.510675in}{0.657420in}}%
\pgfpathlineto{\pgfqpoint{2.511517in}{0.652374in}}%
\pgfpathlineto{\pgfqpoint{2.512022in}{0.655677in}}%
\pgfpathlineto{\pgfqpoint{2.512275in}{0.657295in}}%
\pgfpathlineto{\pgfqpoint{2.512781in}{0.652249in}}%
\pgfpathlineto{\pgfqpoint{2.513202in}{0.651254in}}%
\pgfpathlineto{\pgfqpoint{2.513623in}{0.653537in}}%
\pgfpathlineto{\pgfqpoint{2.515224in}{0.661704in}}%
\pgfpathlineto{\pgfqpoint{2.515392in}{0.659683in}}%
\pgfpathlineto{\pgfqpoint{2.516235in}{0.643446in}}%
\pgfpathlineto{\pgfqpoint{2.516993in}{0.645164in}}%
\pgfpathlineto{\pgfqpoint{2.518341in}{0.648778in}}%
\pgfpathlineto{\pgfqpoint{2.518678in}{0.651693in}}%
\pgfpathlineto{\pgfqpoint{2.519268in}{0.647075in}}%
\pgfpathlineto{\pgfqpoint{2.520447in}{0.647621in}}%
\pgfpathlineto{\pgfqpoint{2.522300in}{0.651549in}}%
\pgfpathlineto{\pgfqpoint{2.523901in}{0.655383in}}%
\pgfpathlineto{\pgfqpoint{2.524406in}{0.652206in}}%
\pgfpathlineto{\pgfqpoint{2.524996in}{0.650203in}}%
\pgfpathlineto{\pgfqpoint{2.525502in}{0.651653in}}%
\pgfpathlineto{\pgfqpoint{2.525923in}{0.656029in}}%
\pgfpathlineto{\pgfqpoint{2.526428in}{0.651027in}}%
\pgfpathlineto{\pgfqpoint{2.526765in}{0.650222in}}%
\pgfpathlineto{\pgfqpoint{2.527608in}{0.650689in}}%
\pgfpathlineto{\pgfqpoint{2.528956in}{0.654711in}}%
\pgfpathlineto{\pgfqpoint{2.529208in}{0.652619in}}%
\pgfpathlineto{\pgfqpoint{2.529461in}{0.651221in}}%
\pgfpathlineto{\pgfqpoint{2.530304in}{0.652139in}}%
\pgfpathlineto{\pgfqpoint{2.531483in}{0.653947in}}%
\pgfpathlineto{\pgfqpoint{2.532831in}{0.660905in}}%
\pgfpathlineto{\pgfqpoint{2.532999in}{0.659888in}}%
\pgfpathlineto{\pgfqpoint{2.533758in}{0.650493in}}%
\pgfpathlineto{\pgfqpoint{2.534347in}{0.652623in}}%
\pgfpathlineto{\pgfqpoint{2.534937in}{0.704938in}}%
\pgfpathlineto{\pgfqpoint{2.535611in}{0.659068in}}%
\pgfpathlineto{\pgfqpoint{2.536201in}{0.656470in}}%
\pgfpathlineto{\pgfqpoint{2.536538in}{0.655342in}}%
\pgfpathlineto{\pgfqpoint{2.536875in}{0.659091in}}%
\pgfpathlineto{\pgfqpoint{2.537717in}{0.684836in}}%
\pgfpathlineto{\pgfqpoint{2.538223in}{0.670497in}}%
\pgfpathlineto{\pgfqpoint{2.538728in}{0.661299in}}%
\pgfpathlineto{\pgfqpoint{2.539402in}{0.667193in}}%
\pgfpathlineto{\pgfqpoint{2.540244in}{0.682324in}}%
\pgfpathlineto{\pgfqpoint{2.540581in}{0.713546in}}%
\pgfpathlineto{\pgfqpoint{2.541087in}{0.654207in}}%
\pgfpathlineto{\pgfqpoint{2.541424in}{0.648901in}}%
\pgfpathlineto{\pgfqpoint{2.541929in}{0.657329in}}%
\pgfpathlineto{\pgfqpoint{2.542351in}{0.679372in}}%
\pgfpathlineto{\pgfqpoint{2.543109in}{0.661235in}}%
\pgfpathlineto{\pgfqpoint{2.543867in}{0.653760in}}%
\pgfpathlineto{\pgfqpoint{2.544709in}{0.658704in}}%
\pgfpathlineto{\pgfqpoint{2.545046in}{0.666850in}}%
\pgfpathlineto{\pgfqpoint{2.545215in}{0.669008in}}%
\pgfpathlineto{\pgfqpoint{2.545468in}{0.655952in}}%
\pgfpathlineto{\pgfqpoint{2.545805in}{0.646784in}}%
\pgfpathlineto{\pgfqpoint{2.546647in}{0.649290in}}%
\pgfpathlineto{\pgfqpoint{2.546815in}{0.648523in}}%
\pgfpathlineto{\pgfqpoint{2.547152in}{0.653136in}}%
\pgfpathlineto{\pgfqpoint{2.547574in}{0.659958in}}%
\pgfpathlineto{\pgfqpoint{2.547995in}{0.648295in}}%
\pgfpathlineto{\pgfqpoint{2.548416in}{0.643958in}}%
\pgfpathlineto{\pgfqpoint{2.548922in}{0.649528in}}%
\pgfpathlineto{\pgfqpoint{2.549680in}{0.689935in}}%
\pgfpathlineto{\pgfqpoint{2.550691in}{0.676910in}}%
\pgfpathlineto{\pgfqpoint{2.551617in}{0.658778in}}%
\pgfpathlineto{\pgfqpoint{2.551870in}{0.666523in}}%
\pgfpathlineto{\pgfqpoint{2.552376in}{0.741188in}}%
\pgfpathlineto{\pgfqpoint{2.552881in}{0.659616in}}%
\pgfpathlineto{\pgfqpoint{2.553218in}{0.643541in}}%
\pgfpathlineto{\pgfqpoint{2.553724in}{0.674350in}}%
\pgfpathlineto{\pgfqpoint{2.554145in}{0.722655in}}%
\pgfpathlineto{\pgfqpoint{2.554903in}{0.692498in}}%
\pgfpathlineto{\pgfqpoint{2.555408in}{0.677470in}}%
\pgfpathlineto{\pgfqpoint{2.556335in}{0.656682in}}%
\pgfpathlineto{\pgfqpoint{2.556756in}{0.663625in}}%
\pgfpathlineto{\pgfqpoint{2.556925in}{0.666359in}}%
\pgfpathlineto{\pgfqpoint{2.557346in}{0.655189in}}%
\pgfpathlineto{\pgfqpoint{2.558525in}{0.643850in}}%
\pgfpathlineto{\pgfqpoint{2.558778in}{0.644135in}}%
\pgfpathlineto{\pgfqpoint{2.559789in}{0.648829in}}%
\pgfpathlineto{\pgfqpoint{2.560126in}{0.653864in}}%
\pgfpathlineto{\pgfqpoint{2.560800in}{0.647558in}}%
\pgfpathlineto{\pgfqpoint{2.561053in}{0.647263in}}%
\pgfpathlineto{\pgfqpoint{2.561727in}{0.648214in}}%
\pgfpathlineto{\pgfqpoint{2.562401in}{0.650366in}}%
\pgfpathlineto{\pgfqpoint{2.564338in}{0.683751in}}%
\pgfpathlineto{\pgfqpoint{2.564760in}{0.700267in}}%
\pgfpathlineto{\pgfqpoint{2.565518in}{0.691233in}}%
\pgfpathlineto{\pgfqpoint{2.566950in}{0.649125in}}%
\pgfpathlineto{\pgfqpoint{2.568298in}{0.649868in}}%
\pgfpathlineto{\pgfqpoint{2.568551in}{0.648782in}}%
\pgfpathlineto{\pgfqpoint{2.569393in}{0.649404in}}%
\pgfpathlineto{\pgfqpoint{2.570320in}{0.650883in}}%
\pgfpathlineto{\pgfqpoint{2.570741in}{0.653888in}}%
\pgfpathlineto{\pgfqpoint{2.571499in}{0.651330in}}%
\pgfpathlineto{\pgfqpoint{2.571836in}{0.652684in}}%
\pgfpathlineto{\pgfqpoint{2.572847in}{0.652026in}}%
\pgfpathlineto{\pgfqpoint{2.573184in}{0.654097in}}%
\pgfpathlineto{\pgfqpoint{2.573605in}{0.667374in}}%
\pgfpathlineto{\pgfqpoint{2.574279in}{0.654527in}}%
\pgfpathlineto{\pgfqpoint{2.574532in}{0.655087in}}%
\pgfpathlineto{\pgfqpoint{2.574869in}{0.653496in}}%
\pgfpathlineto{\pgfqpoint{2.575037in}{0.653156in}}%
\pgfpathlineto{\pgfqpoint{2.575290in}{0.654662in}}%
\pgfpathlineto{\pgfqpoint{2.576722in}{0.664717in}}%
\pgfpathlineto{\pgfqpoint{2.576807in}{0.663545in}}%
\pgfpathlineto{\pgfqpoint{2.577396in}{0.651755in}}%
\pgfpathlineto{\pgfqpoint{2.578239in}{0.652095in}}%
\pgfpathlineto{\pgfqpoint{2.584894in}{0.655138in}}%
\pgfpathlineto{\pgfqpoint{2.585315in}{0.687136in}}%
\pgfpathlineto{\pgfqpoint{2.585989in}{0.656198in}}%
\pgfpathlineto{\pgfqpoint{2.587084in}{0.658499in}}%
\pgfpathlineto{\pgfqpoint{2.586663in}{0.654162in}}%
\pgfpathlineto{\pgfqpoint{2.587169in}{0.658009in}}%
\pgfpathlineto{\pgfqpoint{2.587421in}{0.656177in}}%
\pgfpathlineto{\pgfqpoint{2.587590in}{0.661625in}}%
\pgfpathlineto{\pgfqpoint{2.587927in}{0.698910in}}%
\pgfpathlineto{\pgfqpoint{2.588601in}{0.655709in}}%
\pgfpathlineto{\pgfqpoint{2.588769in}{0.654952in}}%
\pgfpathlineto{\pgfqpoint{2.589190in}{0.659086in}}%
\pgfpathlineto{\pgfqpoint{2.591128in}{0.694074in}}%
\pgfpathlineto{\pgfqpoint{2.591212in}{0.693330in}}%
\pgfpathlineto{\pgfqpoint{2.591971in}{0.653350in}}%
\pgfpathlineto{\pgfqpoint{2.592729in}{0.659171in}}%
\pgfpathlineto{\pgfqpoint{2.593150in}{0.774529in}}%
\pgfpathlineto{\pgfqpoint{2.593824in}{0.656946in}}%
\pgfpathlineto{\pgfqpoint{2.594161in}{0.659128in}}%
\pgfpathlineto{\pgfqpoint{2.594414in}{0.655561in}}%
\pgfpathlineto{\pgfqpoint{2.594666in}{0.654451in}}%
\pgfpathlineto{\pgfqpoint{2.595172in}{0.656549in}}%
\pgfpathlineto{\pgfqpoint{2.595509in}{0.655631in}}%
\pgfpathlineto{\pgfqpoint{2.595846in}{0.658297in}}%
\pgfpathlineto{\pgfqpoint{2.597109in}{0.661907in}}%
\pgfpathlineto{\pgfqpoint{2.596520in}{0.658221in}}%
\pgfpathlineto{\pgfqpoint{2.597194in}{0.661317in}}%
\pgfpathlineto{\pgfqpoint{2.597783in}{0.654401in}}%
\pgfpathlineto{\pgfqpoint{2.598542in}{0.656191in}}%
\pgfpathlineto{\pgfqpoint{2.598963in}{0.675769in}}%
\pgfpathlineto{\pgfqpoint{2.599384in}{0.704649in}}%
\pgfpathlineto{\pgfqpoint{2.599974in}{0.669684in}}%
\pgfpathlineto{\pgfqpoint{2.600058in}{0.669453in}}%
\pgfpathlineto{\pgfqpoint{2.600227in}{0.671149in}}%
\pgfpathlineto{\pgfqpoint{2.600732in}{0.693030in}}%
\pgfpathlineto{\pgfqpoint{2.601996in}{0.747246in}}%
\pgfpathlineto{\pgfqpoint{2.602248in}{0.732426in}}%
\pgfpathlineto{\pgfqpoint{2.602838in}{0.683262in}}%
\pgfpathlineto{\pgfqpoint{2.603428in}{0.722252in}}%
\pgfpathlineto{\pgfqpoint{2.603681in}{0.731786in}}%
\pgfpathlineto{\pgfqpoint{2.604186in}{0.700407in}}%
\pgfpathlineto{\pgfqpoint{2.605702in}{0.663244in}}%
\pgfpathlineto{\pgfqpoint{2.605955in}{0.666870in}}%
\pgfpathlineto{\pgfqpoint{2.606461in}{0.690395in}}%
\pgfpathlineto{\pgfqpoint{2.606882in}{0.660203in}}%
\pgfpathlineto{\pgfqpoint{2.607303in}{0.649882in}}%
\pgfpathlineto{\pgfqpoint{2.608061in}{0.650655in}}%
\pgfpathlineto{\pgfqpoint{2.609409in}{0.651602in}}%
\pgfpathlineto{\pgfqpoint{2.609662in}{0.651047in}}%
\pgfpathlineto{\pgfqpoint{2.609915in}{0.651067in}}%
\pgfpathlineto{\pgfqpoint{2.610336in}{0.651771in}}%
\pgfpathlineto{\pgfqpoint{2.612189in}{0.660142in}}%
\pgfpathlineto{\pgfqpoint{2.612610in}{0.673206in}}%
\pgfpathlineto{\pgfqpoint{2.613453in}{0.668932in}}%
\pgfpathlineto{\pgfqpoint{2.613706in}{0.670296in}}%
\pgfpathlineto{\pgfqpoint{2.613958in}{0.667634in}}%
\pgfpathlineto{\pgfqpoint{2.614380in}{0.658224in}}%
\pgfpathlineto{\pgfqpoint{2.614969in}{0.669223in}}%
\pgfpathlineto{\pgfqpoint{2.616317in}{0.674678in}}%
\pgfpathlineto{\pgfqpoint{2.616907in}{0.675989in}}%
\pgfpathlineto{\pgfqpoint{2.617244in}{0.674760in}}%
\pgfpathlineto{\pgfqpoint{2.618845in}{0.655646in}}%
\pgfpathlineto{\pgfqpoint{2.619266in}{0.649029in}}%
\pgfpathlineto{\pgfqpoint{2.620024in}{0.651535in}}%
\pgfpathlineto{\pgfqpoint{2.621456in}{0.648892in}}%
\pgfpathlineto{\pgfqpoint{2.621625in}{0.649070in}}%
\pgfpathlineto{\pgfqpoint{2.622214in}{0.652952in}}%
\pgfpathlineto{\pgfqpoint{2.622720in}{0.648892in}}%
\pgfpathlineto{\pgfqpoint{2.623562in}{0.649417in}}%
\pgfpathlineto{\pgfqpoint{2.623646in}{0.649541in}}%
\pgfpathlineto{\pgfqpoint{2.625163in}{0.649951in}}%
\pgfpathlineto{\pgfqpoint{2.629122in}{0.651559in}}%
\pgfpathlineto{\pgfqpoint{2.629544in}{0.655220in}}%
\pgfpathlineto{\pgfqpoint{2.629965in}{0.661648in}}%
\pgfpathlineto{\pgfqpoint{2.630892in}{0.659760in}}%
\pgfpathlineto{\pgfqpoint{2.632155in}{0.652855in}}%
\pgfpathlineto{\pgfqpoint{2.632576in}{0.656095in}}%
\pgfpathlineto{\pgfqpoint{2.634430in}{0.677154in}}%
\pgfpathlineto{\pgfqpoint{2.635525in}{0.688352in}}%
\pgfpathlineto{\pgfqpoint{2.635778in}{0.680839in}}%
\pgfpathlineto{\pgfqpoint{2.636704in}{0.651422in}}%
\pgfpathlineto{\pgfqpoint{2.637210in}{0.656478in}}%
\pgfpathlineto{\pgfqpoint{2.637715in}{0.682722in}}%
\pgfpathlineto{\pgfqpoint{2.638389in}{0.662313in}}%
\pgfpathlineto{\pgfqpoint{2.638558in}{0.661988in}}%
\pgfpathlineto{\pgfqpoint{2.639063in}{0.663715in}}%
\pgfpathlineto{\pgfqpoint{2.640411in}{0.672152in}}%
\pgfpathlineto{\pgfqpoint{2.640580in}{0.668550in}}%
\pgfpathlineto{\pgfqpoint{2.641338in}{0.651704in}}%
\pgfpathlineto{\pgfqpoint{2.642096in}{0.652682in}}%
\pgfpathlineto{\pgfqpoint{2.642602in}{0.650308in}}%
\pgfpathlineto{\pgfqpoint{2.643107in}{0.652541in}}%
\pgfpathlineto{\pgfqpoint{2.643360in}{0.654078in}}%
\pgfpathlineto{\pgfqpoint{2.644118in}{0.651740in}}%
\pgfpathlineto{\pgfqpoint{2.646140in}{0.641067in}}%
\pgfpathlineto{\pgfqpoint{2.646224in}{0.641136in}}%
\pgfpathlineto{\pgfqpoint{2.646814in}{0.644792in}}%
\pgfpathlineto{\pgfqpoint{2.647235in}{0.654819in}}%
\pgfpathlineto{\pgfqpoint{2.648077in}{0.648994in}}%
\pgfpathlineto{\pgfqpoint{2.648330in}{0.648006in}}%
\pgfpathlineto{\pgfqpoint{2.648920in}{0.645997in}}%
\pgfpathlineto{\pgfqpoint{2.649510in}{0.646863in}}%
\pgfpathlineto{\pgfqpoint{2.652795in}{0.667287in}}%
\pgfpathlineto{\pgfqpoint{2.653301in}{0.676421in}}%
\pgfpathlineto{\pgfqpoint{2.653553in}{0.667045in}}%
\pgfpathlineto{\pgfqpoint{2.654227in}{0.645543in}}%
\pgfpathlineto{\pgfqpoint{2.654648in}{0.659129in}}%
\pgfpathlineto{\pgfqpoint{2.655154in}{0.767374in}}%
\pgfpathlineto{\pgfqpoint{2.655828in}{0.683674in}}%
\pgfpathlineto{\pgfqpoint{2.656249in}{0.681222in}}%
\pgfpathlineto{\pgfqpoint{2.656923in}{0.658063in}}%
\pgfpathlineto{\pgfqpoint{2.657513in}{0.671101in}}%
\pgfpathlineto{\pgfqpoint{2.657934in}{0.681604in}}%
\pgfpathlineto{\pgfqpoint{2.658524in}{0.668641in}}%
\pgfpathlineto{\pgfqpoint{2.658776in}{0.666565in}}%
\pgfpathlineto{\pgfqpoint{2.660377in}{0.651920in}}%
\pgfpathlineto{\pgfqpoint{2.660714in}{0.655488in}}%
\pgfpathlineto{\pgfqpoint{2.661725in}{0.688258in}}%
\pgfpathlineto{\pgfqpoint{2.662230in}{0.666954in}}%
\pgfpathlineto{\pgfqpoint{2.662652in}{0.661783in}}%
\pgfpathlineto{\pgfqpoint{2.663494in}{0.662037in}}%
\pgfpathlineto{\pgfqpoint{2.663831in}{0.661467in}}%
\pgfpathlineto{\pgfqpoint{2.664758in}{0.643361in}}%
\pgfpathlineto{\pgfqpoint{2.665095in}{0.652266in}}%
\pgfpathlineto{\pgfqpoint{2.665685in}{0.713620in}}%
\pgfpathlineto{\pgfqpoint{2.666106in}{0.655244in}}%
\pgfpathlineto{\pgfqpoint{2.667369in}{0.636094in}}%
\pgfpathlineto{\pgfqpoint{2.667454in}{0.636295in}}%
\pgfpathlineto{\pgfqpoint{2.668633in}{0.642942in}}%
\pgfpathlineto{\pgfqpoint{2.668886in}{0.647010in}}%
\pgfpathlineto{\pgfqpoint{2.669391in}{0.640789in}}%
\pgfpathlineto{\pgfqpoint{2.669644in}{0.641794in}}%
\pgfpathlineto{\pgfqpoint{2.671919in}{0.648312in}}%
\pgfpathlineto{\pgfqpoint{2.672003in}{0.648273in}}%
\pgfpathlineto{\pgfqpoint{2.672340in}{0.648119in}}%
\pgfpathlineto{\pgfqpoint{2.672761in}{0.649045in}}%
\pgfpathlineto{\pgfqpoint{2.674614in}{0.653488in}}%
\pgfpathlineto{\pgfqpoint{2.676552in}{0.651904in}}%
\pgfpathlineto{\pgfqpoint{2.675373in}{0.654128in}}%
\pgfpathlineto{\pgfqpoint{2.676721in}{0.652117in}}%
\pgfpathlineto{\pgfqpoint{2.677058in}{0.654483in}}%
\pgfpathlineto{\pgfqpoint{2.677647in}{0.674444in}}%
\pgfpathlineto{\pgfqpoint{2.678658in}{0.665658in}}%
\pgfpathlineto{\pgfqpoint{2.678995in}{0.663955in}}%
\pgfpathlineto{\pgfqpoint{2.679838in}{0.658954in}}%
\pgfpathlineto{\pgfqpoint{2.680175in}{0.661942in}}%
\pgfpathlineto{\pgfqpoint{2.680849in}{0.664486in}}%
\pgfpathlineto{\pgfqpoint{2.681186in}{0.660933in}}%
\pgfpathlineto{\pgfqpoint{2.682786in}{0.652428in}}%
\pgfpathlineto{\pgfqpoint{2.683207in}{0.651233in}}%
\pgfpathlineto{\pgfqpoint{2.683629in}{0.653298in}}%
\pgfpathlineto{\pgfqpoint{2.683966in}{0.654605in}}%
\pgfpathlineto{\pgfqpoint{2.684471in}{0.652595in}}%
\pgfpathlineto{\pgfqpoint{2.684724in}{0.653126in}}%
\pgfpathlineto{\pgfqpoint{2.686746in}{0.667708in}}%
\pgfpathlineto{\pgfqpoint{2.687083in}{0.660305in}}%
\pgfpathlineto{\pgfqpoint{2.687251in}{0.656655in}}%
\pgfpathlineto{\pgfqpoint{2.687672in}{0.671298in}}%
\pgfpathlineto{\pgfqpoint{2.688262in}{0.721146in}}%
\pgfpathlineto{\pgfqpoint{2.688936in}{0.692890in}}%
\pgfpathlineto{\pgfqpoint{2.690115in}{0.669825in}}%
\pgfpathlineto{\pgfqpoint{2.690452in}{0.679248in}}%
\pgfpathlineto{\pgfqpoint{2.690789in}{0.699101in}}%
\pgfpathlineto{\pgfqpoint{2.691211in}{0.654023in}}%
\pgfpathlineto{\pgfqpoint{2.691463in}{0.644668in}}%
\pgfpathlineto{\pgfqpoint{2.692306in}{0.649617in}}%
\pgfpathlineto{\pgfqpoint{2.692559in}{0.652331in}}%
\pgfpathlineto{\pgfqpoint{2.693317in}{0.648148in}}%
\pgfpathlineto{\pgfqpoint{2.695002in}{0.647378in}}%
\pgfpathlineto{\pgfqpoint{2.695339in}{0.648145in}}%
\pgfpathlineto{\pgfqpoint{2.698287in}{0.652588in}}%
\pgfpathlineto{\pgfqpoint{2.700393in}{0.654331in}}%
\pgfpathlineto{\pgfqpoint{2.702247in}{0.662870in}}%
\pgfpathlineto{\pgfqpoint{2.702415in}{0.666990in}}%
\pgfpathlineto{\pgfqpoint{2.702921in}{0.653012in}}%
\pgfpathlineto{\pgfqpoint{2.703173in}{0.652168in}}%
\pgfpathlineto{\pgfqpoint{2.704016in}{0.652893in}}%
\pgfpathlineto{\pgfqpoint{2.707301in}{0.654115in}}%
\pgfpathlineto{\pgfqpoint{2.707723in}{0.653752in}}%
\pgfpathlineto{\pgfqpoint{2.707807in}{0.653334in}}%
\pgfpathlineto{\pgfqpoint{2.708060in}{0.652513in}}%
\pgfpathlineto{\pgfqpoint{2.708481in}{0.655018in}}%
\pgfpathlineto{\pgfqpoint{2.708818in}{0.653924in}}%
\pgfpathlineto{\pgfqpoint{2.709660in}{0.653580in}}%
\pgfpathlineto{\pgfqpoint{2.709829in}{0.654170in}}%
\pgfpathlineto{\pgfqpoint{2.711092in}{0.680633in}}%
\pgfpathlineto{\pgfqpoint{2.711429in}{0.719579in}}%
\pgfpathlineto{\pgfqpoint{2.712103in}{0.675790in}}%
\pgfpathlineto{\pgfqpoint{2.712609in}{0.661373in}}%
\pgfpathlineto{\pgfqpoint{2.713114in}{0.672630in}}%
\pgfpathlineto{\pgfqpoint{2.713620in}{0.715848in}}%
\pgfpathlineto{\pgfqpoint{2.714462in}{0.690739in}}%
\pgfpathlineto{\pgfqpoint{2.714546in}{0.689781in}}%
\pgfpathlineto{\pgfqpoint{2.714799in}{0.695866in}}%
\pgfpathlineto{\pgfqpoint{2.715220in}{0.743458in}}%
\pgfpathlineto{\pgfqpoint{2.715642in}{0.693521in}}%
\pgfpathlineto{\pgfqpoint{2.716147in}{0.656761in}}%
\pgfpathlineto{\pgfqpoint{2.716652in}{0.691695in}}%
\pgfpathlineto{\pgfqpoint{2.717242in}{0.745078in}}%
\pgfpathlineto{\pgfqpoint{2.717832in}{0.703743in}}%
\pgfpathlineto{\pgfqpoint{2.718843in}{0.680970in}}%
\pgfpathlineto{\pgfqpoint{2.719264in}{0.690911in}}%
\pgfpathlineto{\pgfqpoint{2.719517in}{0.700378in}}%
\pgfpathlineto{\pgfqpoint{2.719938in}{0.666404in}}%
\pgfpathlineto{\pgfqpoint{2.720443in}{0.655558in}}%
\pgfpathlineto{\pgfqpoint{2.720949in}{0.670238in}}%
\pgfpathlineto{\pgfqpoint{2.721286in}{0.687729in}}%
\pgfpathlineto{\pgfqpoint{2.722044in}{0.673815in}}%
\pgfpathlineto{\pgfqpoint{2.722381in}{0.669253in}}%
\pgfpathlineto{\pgfqpoint{2.724150in}{0.653433in}}%
\pgfpathlineto{\pgfqpoint{2.725330in}{0.648981in}}%
\pgfpathlineto{\pgfqpoint{2.725582in}{0.649334in}}%
\pgfpathlineto{\pgfqpoint{2.726762in}{0.650439in}}%
\pgfpathlineto{\pgfqpoint{2.726846in}{0.650370in}}%
\pgfpathlineto{\pgfqpoint{2.727436in}{0.644696in}}%
\pgfpathlineto{\pgfqpoint{2.728447in}{0.646414in}}%
\pgfpathlineto{\pgfqpoint{2.728952in}{0.647987in}}%
\pgfpathlineto{\pgfqpoint{2.729795in}{0.647346in}}%
\pgfpathlineto{\pgfqpoint{2.730553in}{0.649984in}}%
\pgfpathlineto{\pgfqpoint{2.730974in}{0.659165in}}%
\pgfpathlineto{\pgfqpoint{2.731732in}{0.652732in}}%
\pgfpathlineto{\pgfqpoint{2.731901in}{0.652835in}}%
\pgfpathlineto{\pgfqpoint{2.732069in}{0.651852in}}%
\pgfpathlineto{\pgfqpoint{2.732406in}{0.650183in}}%
\pgfpathlineto{\pgfqpoint{2.733164in}{0.651558in}}%
\pgfpathlineto{\pgfqpoint{2.733501in}{0.669309in}}%
\pgfpathlineto{\pgfqpoint{2.733838in}{0.711504in}}%
\pgfpathlineto{\pgfqpoint{2.734512in}{0.665073in}}%
\pgfpathlineto{\pgfqpoint{2.734849in}{0.657929in}}%
\pgfpathlineto{\pgfqpoint{2.735439in}{0.672626in}}%
\pgfpathlineto{\pgfqpoint{2.735860in}{0.694265in}}%
\pgfpathlineto{\pgfqpoint{2.736703in}{0.688159in}}%
\pgfpathlineto{\pgfqpoint{2.737629in}{0.664641in}}%
\pgfpathlineto{\pgfqpoint{2.738051in}{0.676632in}}%
\pgfpathlineto{\pgfqpoint{2.739398in}{0.692907in}}%
\pgfpathlineto{\pgfqpoint{2.739651in}{0.698637in}}%
\pgfpathlineto{\pgfqpoint{2.740325in}{0.689048in}}%
\pgfpathlineto{\pgfqpoint{2.742937in}{0.652161in}}%
\pgfpathlineto{\pgfqpoint{2.743189in}{0.654843in}}%
\pgfpathlineto{\pgfqpoint{2.743779in}{0.687709in}}%
\pgfpathlineto{\pgfqpoint{2.744622in}{0.667617in}}%
\pgfpathlineto{\pgfqpoint{2.745380in}{0.658829in}}%
\pgfpathlineto{\pgfqpoint{2.745717in}{0.664153in}}%
\pgfpathlineto{\pgfqpoint{2.746307in}{0.704987in}}%
\pgfpathlineto{\pgfqpoint{2.746728in}{0.663700in}}%
\pgfpathlineto{\pgfqpoint{2.747149in}{0.650058in}}%
\pgfpathlineto{\pgfqpoint{2.747823in}{0.664366in}}%
\pgfpathlineto{\pgfqpoint{2.748497in}{0.659469in}}%
\pgfpathlineto{\pgfqpoint{2.748750in}{0.663806in}}%
\pgfpathlineto{\pgfqpoint{2.749255in}{0.697422in}}%
\pgfpathlineto{\pgfqpoint{2.749761in}{0.668444in}}%
\pgfpathlineto{\pgfqpoint{2.750266in}{0.650309in}}%
\pgfpathlineto{\pgfqpoint{2.750519in}{0.678019in}}%
\pgfpathlineto{\pgfqpoint{2.751108in}{0.748604in}}%
\pgfpathlineto{\pgfqpoint{2.751867in}{0.720255in}}%
\pgfpathlineto{\pgfqpoint{2.752204in}{0.734799in}}%
\pgfpathlineto{\pgfqpoint{2.752625in}{0.717293in}}%
\pgfpathlineto{\pgfqpoint{2.752878in}{0.719058in}}%
\pgfpathlineto{\pgfqpoint{2.753299in}{0.665699in}}%
\pgfpathlineto{\pgfqpoint{2.753720in}{0.624768in}}%
\pgfpathlineto{\pgfqpoint{2.754562in}{0.637380in}}%
\pgfpathlineto{\pgfqpoint{2.756079in}{0.623910in}}%
\pgfpathlineto{\pgfqpoint{2.756163in}{0.625223in}}%
\pgfpathlineto{\pgfqpoint{2.756753in}{0.680339in}}%
\pgfpathlineto{\pgfqpoint{2.756921in}{0.696622in}}%
\pgfpathlineto{\pgfqpoint{2.757511in}{0.642401in}}%
\pgfpathlineto{\pgfqpoint{2.757764in}{0.636933in}}%
\pgfpathlineto{\pgfqpoint{2.758606in}{0.641492in}}%
\pgfpathlineto{\pgfqpoint{2.759027in}{0.659197in}}%
\pgfpathlineto{\pgfqpoint{2.760544in}{0.727682in}}%
\pgfpathlineto{\pgfqpoint{2.760712in}{0.725480in}}%
\pgfpathlineto{\pgfqpoint{2.762060in}{0.687796in}}%
\pgfpathlineto{\pgfqpoint{2.762397in}{0.698692in}}%
\pgfpathlineto{\pgfqpoint{2.762903in}{0.749015in}}%
\pgfpathlineto{\pgfqpoint{2.763240in}{0.695714in}}%
\pgfpathlineto{\pgfqpoint{2.763829in}{0.627118in}}%
\pgfpathlineto{\pgfqpoint{2.764419in}{0.659878in}}%
\pgfpathlineto{\pgfqpoint{2.765683in}{0.696281in}}%
\pgfpathlineto{\pgfqpoint{2.765093in}{0.649506in}}%
\pgfpathlineto{\pgfqpoint{2.765851in}{0.680948in}}%
\pgfpathlineto{\pgfqpoint{2.766272in}{0.639363in}}%
\pgfpathlineto{\pgfqpoint{2.766946in}{0.681076in}}%
\pgfpathlineto{\pgfqpoint{2.767115in}{0.673806in}}%
\pgfpathlineto{\pgfqpoint{2.767705in}{0.633918in}}%
\pgfpathlineto{\pgfqpoint{2.768463in}{0.638001in}}%
\pgfpathlineto{\pgfqpoint{2.768631in}{0.640607in}}%
\pgfpathlineto{\pgfqpoint{2.769053in}{0.720060in}}%
\pgfpathlineto{\pgfqpoint{2.769305in}{0.787113in}}%
\pgfpathlineto{\pgfqpoint{2.769979in}{0.665089in}}%
\pgfpathlineto{\pgfqpoint{2.770232in}{0.673019in}}%
\pgfpathlineto{\pgfqpoint{2.770737in}{0.722910in}}%
\pgfpathlineto{\pgfqpoint{2.771664in}{0.701139in}}%
\pgfpathlineto{\pgfqpoint{2.771833in}{0.705734in}}%
\pgfpathlineto{\pgfqpoint{2.772254in}{0.683864in}}%
\pgfpathlineto{\pgfqpoint{2.772675in}{0.657118in}}%
\pgfpathlineto{\pgfqpoint{2.773265in}{0.684862in}}%
\pgfpathlineto{\pgfqpoint{2.773518in}{0.695282in}}%
\pgfpathlineto{\pgfqpoint{2.773939in}{0.658705in}}%
\pgfpathlineto{\pgfqpoint{2.774276in}{0.648215in}}%
\pgfpathlineto{\pgfqpoint{2.774950in}{0.658721in}}%
\pgfpathlineto{\pgfqpoint{2.775034in}{0.658063in}}%
\pgfpathlineto{\pgfqpoint{2.775202in}{0.656589in}}%
\pgfpathlineto{\pgfqpoint{2.775539in}{0.663279in}}%
\pgfpathlineto{\pgfqpoint{2.775624in}{0.663987in}}%
\pgfpathlineto{\pgfqpoint{2.775792in}{0.660313in}}%
\pgfpathlineto{\pgfqpoint{2.777645in}{0.607247in}}%
\pgfpathlineto{\pgfqpoint{2.777898in}{0.610990in}}%
\pgfpathlineto{\pgfqpoint{2.778404in}{0.656816in}}%
\pgfpathlineto{\pgfqpoint{2.778825in}{0.712126in}}%
\pgfpathlineto{\pgfqpoint{2.779583in}{0.689060in}}%
\pgfpathlineto{\pgfqpoint{2.781100in}{0.620475in}}%
\pgfpathlineto{\pgfqpoint{2.781942in}{0.627055in}}%
\pgfpathlineto{\pgfqpoint{2.782532in}{0.639098in}}%
\pgfpathlineto{\pgfqpoint{2.783543in}{0.635593in}}%
\pgfpathlineto{\pgfqpoint{2.786070in}{0.644564in}}%
\pgfpathlineto{\pgfqpoint{2.786323in}{0.643483in}}%
\pgfpathlineto{\pgfqpoint{2.786575in}{0.642061in}}%
\pgfpathlineto{\pgfqpoint{2.787165in}{0.644657in}}%
\pgfpathlineto{\pgfqpoint{2.789019in}{0.650914in}}%
\pgfpathlineto{\pgfqpoint{2.789355in}{0.648940in}}%
\pgfpathlineto{\pgfqpoint{2.789524in}{0.648394in}}%
\pgfpathlineto{\pgfqpoint{2.790029in}{0.649345in}}%
\pgfpathlineto{\pgfqpoint{2.790366in}{0.649282in}}%
\pgfpathlineto{\pgfqpoint{2.794747in}{0.653792in}}%
\pgfpathlineto{\pgfqpoint{2.795168in}{0.655773in}}%
\pgfpathlineto{\pgfqpoint{2.795842in}{0.653738in}}%
\pgfpathlineto{\pgfqpoint{2.796601in}{0.652689in}}%
\pgfpathlineto{\pgfqpoint{2.797359in}{0.651747in}}%
\pgfpathlineto{\pgfqpoint{2.797611in}{0.653104in}}%
\pgfpathlineto{\pgfqpoint{2.798959in}{0.680537in}}%
\pgfpathlineto{\pgfqpoint{2.799296in}{0.730139in}}%
\pgfpathlineto{\pgfqpoint{2.800055in}{0.686399in}}%
\pgfpathlineto{\pgfqpoint{2.800307in}{0.688890in}}%
\pgfpathlineto{\pgfqpoint{2.800560in}{0.679280in}}%
\pgfpathlineto{\pgfqpoint{2.801234in}{0.659563in}}%
\pgfpathlineto{\pgfqpoint{2.801655in}{0.674670in}}%
\pgfpathlineto{\pgfqpoint{2.802582in}{0.774193in}}%
\pgfpathlineto{\pgfqpoint{2.803172in}{0.707434in}}%
\pgfpathlineto{\pgfqpoint{2.804688in}{0.660110in}}%
\pgfpathlineto{\pgfqpoint{2.804856in}{0.661947in}}%
\pgfpathlineto{\pgfqpoint{2.805362in}{0.712180in}}%
\pgfpathlineto{\pgfqpoint{2.805867in}{0.664767in}}%
\pgfpathlineto{\pgfqpoint{2.807047in}{0.654084in}}%
\pgfpathlineto{\pgfqpoint{2.807131in}{0.654114in}}%
\pgfpathlineto{\pgfqpoint{2.807805in}{0.655681in}}%
\pgfpathlineto{\pgfqpoint{2.808311in}{0.668254in}}%
\pgfpathlineto{\pgfqpoint{2.808900in}{0.657453in}}%
\pgfpathlineto{\pgfqpoint{2.809574in}{0.654847in}}%
\pgfpathlineto{\pgfqpoint{2.810164in}{0.655019in}}%
\pgfpathlineto{\pgfqpoint{2.810417in}{0.654871in}}%
\pgfpathlineto{\pgfqpoint{2.810585in}{0.655544in}}%
\pgfpathlineto{\pgfqpoint{2.810922in}{0.671332in}}%
\pgfpathlineto{\pgfqpoint{2.811343in}{0.734289in}}%
\pgfpathlineto{\pgfqpoint{2.811933in}{0.673385in}}%
\pgfpathlineto{\pgfqpoint{2.812438in}{0.663525in}}%
\pgfpathlineto{\pgfqpoint{2.813028in}{0.671571in}}%
\pgfpathlineto{\pgfqpoint{2.814292in}{0.718037in}}%
\pgfpathlineto{\pgfqpoint{2.814545in}{0.694520in}}%
\pgfpathlineto{\pgfqpoint{2.815387in}{0.648142in}}%
\pgfpathlineto{\pgfqpoint{2.815893in}{0.649479in}}%
\pgfpathlineto{\pgfqpoint{2.817240in}{0.652550in}}%
\pgfpathlineto{\pgfqpoint{2.817493in}{0.651717in}}%
\pgfpathlineto{\pgfqpoint{2.818167in}{0.649228in}}%
\pgfpathlineto{\pgfqpoint{2.818841in}{0.650137in}}%
\pgfpathlineto{\pgfqpoint{2.820947in}{0.653625in}}%
\pgfpathlineto{\pgfqpoint{2.822969in}{0.661991in}}%
\pgfpathlineto{\pgfqpoint{2.823980in}{0.651788in}}%
\pgfpathlineto{\pgfqpoint{2.824485in}{0.656635in}}%
\pgfpathlineto{\pgfqpoint{2.824991in}{0.670297in}}%
\pgfpathlineto{\pgfqpoint{2.825581in}{0.656905in}}%
\pgfpathlineto{\pgfqpoint{2.826929in}{0.653501in}}%
\pgfpathlineto{\pgfqpoint{2.827266in}{0.655654in}}%
\pgfpathlineto{\pgfqpoint{2.827855in}{0.670809in}}%
\pgfpathlineto{\pgfqpoint{2.828361in}{0.657806in}}%
\pgfpathlineto{\pgfqpoint{2.828698in}{0.654518in}}%
\pgfpathlineto{\pgfqpoint{2.829456in}{0.656780in}}%
\pgfpathlineto{\pgfqpoint{2.829624in}{0.656741in}}%
\pgfpathlineto{\pgfqpoint{2.829877in}{0.657793in}}%
\pgfpathlineto{\pgfqpoint{2.830635in}{0.672052in}}%
\pgfpathlineto{\pgfqpoint{2.830972in}{0.679324in}}%
\pgfpathlineto{\pgfqpoint{2.831394in}{0.663552in}}%
\pgfpathlineto{\pgfqpoint{2.831646in}{0.657021in}}%
\pgfpathlineto{\pgfqpoint{2.832152in}{0.676383in}}%
\pgfpathlineto{\pgfqpoint{2.833668in}{0.764348in}}%
\pgfpathlineto{\pgfqpoint{2.833837in}{0.749456in}}%
\pgfpathlineto{\pgfqpoint{2.834679in}{0.659023in}}%
\pgfpathlineto{\pgfqpoint{2.835353in}{0.684250in}}%
\pgfpathlineto{\pgfqpoint{2.835606in}{0.692987in}}%
\pgfpathlineto{\pgfqpoint{2.836364in}{0.681197in}}%
\pgfpathlineto{\pgfqpoint{2.836869in}{0.672877in}}%
\pgfpathlineto{\pgfqpoint{2.837122in}{0.680259in}}%
\pgfpathlineto{\pgfqpoint{2.837880in}{0.777383in}}%
\pgfpathlineto{\pgfqpoint{2.838386in}{0.710859in}}%
\pgfpathlineto{\pgfqpoint{2.839818in}{0.658910in}}%
\pgfpathlineto{\pgfqpoint{2.839986in}{0.657574in}}%
\pgfpathlineto{\pgfqpoint{2.840323in}{0.663170in}}%
\pgfpathlineto{\pgfqpoint{2.841419in}{0.724453in}}%
\pgfpathlineto{\pgfqpoint{2.841840in}{0.696360in}}%
\pgfpathlineto{\pgfqpoint{2.843440in}{0.650064in}}%
\pgfpathlineto{\pgfqpoint{2.843525in}{0.649781in}}%
\pgfpathlineto{\pgfqpoint{2.843693in}{0.652177in}}%
\pgfpathlineto{\pgfqpoint{2.844199in}{0.661511in}}%
\pgfpathlineto{\pgfqpoint{2.844620in}{0.652061in}}%
\pgfpathlineto{\pgfqpoint{2.845125in}{0.640242in}}%
\pgfpathlineto{\pgfqpoint{2.845799in}{0.648512in}}%
\pgfpathlineto{\pgfqpoint{2.846810in}{0.655956in}}%
\pgfpathlineto{\pgfqpoint{2.846305in}{0.646587in}}%
\pgfpathlineto{\pgfqpoint{2.846979in}{0.652195in}}%
\pgfpathlineto{\pgfqpoint{2.847653in}{0.624917in}}%
\pgfpathlineto{\pgfqpoint{2.848411in}{0.632140in}}%
\pgfpathlineto{\pgfqpoint{2.852033in}{0.646398in}}%
\pgfpathlineto{\pgfqpoint{2.852792in}{0.651638in}}%
\pgfpathlineto{\pgfqpoint{2.853381in}{0.658539in}}%
\pgfpathlineto{\pgfqpoint{2.853887in}{0.652106in}}%
\pgfpathlineto{\pgfqpoint{2.855066in}{0.651383in}}%
\pgfpathlineto{\pgfqpoint{2.855150in}{0.651444in}}%
\pgfpathlineto{\pgfqpoint{2.855487in}{0.653736in}}%
\pgfpathlineto{\pgfqpoint{2.855909in}{0.661538in}}%
\pgfpathlineto{\pgfqpoint{2.856498in}{0.653720in}}%
\pgfpathlineto{\pgfqpoint{2.856751in}{0.652072in}}%
\pgfpathlineto{\pgfqpoint{2.857172in}{0.657384in}}%
\pgfpathlineto{\pgfqpoint{2.858773in}{0.676723in}}%
\pgfpathlineto{\pgfqpoint{2.858941in}{0.671003in}}%
\pgfpathlineto{\pgfqpoint{2.859784in}{0.645502in}}%
\pgfpathlineto{\pgfqpoint{2.860374in}{0.652750in}}%
\pgfpathlineto{\pgfqpoint{2.860879in}{0.667746in}}%
\pgfpathlineto{\pgfqpoint{2.861806in}{0.664050in}}%
\pgfpathlineto{\pgfqpoint{2.863743in}{0.647286in}}%
\pgfpathlineto{\pgfqpoint{2.864502in}{0.648316in}}%
\pgfpathlineto{\pgfqpoint{2.865597in}{0.649649in}}%
\pgfpathlineto{\pgfqpoint{2.865934in}{0.648759in}}%
\pgfpathlineto{\pgfqpoint{2.866945in}{0.648672in}}%
\pgfpathlineto{\pgfqpoint{2.867029in}{0.648988in}}%
\pgfpathlineto{\pgfqpoint{2.867450in}{0.660087in}}%
\pgfpathlineto{\pgfqpoint{2.867871in}{0.675366in}}%
\pgfpathlineto{\pgfqpoint{2.868545in}{0.662966in}}%
\pgfpathlineto{\pgfqpoint{2.868714in}{0.663494in}}%
\pgfpathlineto{\pgfqpoint{2.869135in}{0.684686in}}%
\pgfpathlineto{\pgfqpoint{2.870567in}{0.736633in}}%
\pgfpathlineto{\pgfqpoint{2.870651in}{0.732224in}}%
\pgfpathlineto{\pgfqpoint{2.871494in}{0.656329in}}%
\pgfpathlineto{\pgfqpoint{2.872084in}{0.692534in}}%
\pgfpathlineto{\pgfqpoint{2.872336in}{0.710398in}}%
\pgfpathlineto{\pgfqpoint{2.873095in}{0.681301in}}%
\pgfpathlineto{\pgfqpoint{2.875622in}{0.656075in}}%
\pgfpathlineto{\pgfqpoint{2.876970in}{0.657987in}}%
\pgfpathlineto{\pgfqpoint{2.877896in}{0.658509in}}%
\pgfpathlineto{\pgfqpoint{2.878065in}{0.657824in}}%
\pgfpathlineto{\pgfqpoint{2.880171in}{0.649051in}}%
\pgfpathlineto{\pgfqpoint{2.881856in}{0.646584in}}%
\pgfpathlineto{\pgfqpoint{2.884299in}{0.648284in}}%
\pgfpathlineto{\pgfqpoint{2.884720in}{0.660035in}}%
\pgfpathlineto{\pgfqpoint{2.886152in}{0.685351in}}%
\pgfpathlineto{\pgfqpoint{2.886237in}{0.685316in}}%
\pgfpathlineto{\pgfqpoint{2.886658in}{0.683207in}}%
\pgfpathlineto{\pgfqpoint{2.886911in}{0.686794in}}%
\pgfpathlineto{\pgfqpoint{2.887079in}{0.690298in}}%
\pgfpathlineto{\pgfqpoint{2.887585in}{0.677230in}}%
\pgfpathlineto{\pgfqpoint{2.887753in}{0.675335in}}%
\pgfpathlineto{\pgfqpoint{2.888090in}{0.685644in}}%
\pgfpathlineto{\pgfqpoint{2.888764in}{0.743252in}}%
\pgfpathlineto{\pgfqpoint{2.889269in}{0.696684in}}%
\pgfpathlineto{\pgfqpoint{2.890617in}{0.684950in}}%
\pgfpathlineto{\pgfqpoint{2.890786in}{0.683797in}}%
\pgfpathlineto{\pgfqpoint{2.891039in}{0.688045in}}%
\pgfpathlineto{\pgfqpoint{2.891376in}{0.696968in}}%
\pgfpathlineto{\pgfqpoint{2.891797in}{0.677476in}}%
\pgfpathlineto{\pgfqpoint{2.892471in}{0.656628in}}%
\pgfpathlineto{\pgfqpoint{2.892976in}{0.666344in}}%
\pgfpathlineto{\pgfqpoint{2.893229in}{0.669279in}}%
\pgfpathlineto{\pgfqpoint{2.893819in}{0.661606in}}%
\pgfpathlineto{\pgfqpoint{2.894240in}{0.662695in}}%
\pgfpathlineto{\pgfqpoint{2.894324in}{0.661979in}}%
\pgfpathlineto{\pgfqpoint{2.895167in}{0.651673in}}%
\pgfpathlineto{\pgfqpoint{2.895672in}{0.657540in}}%
\pgfpathlineto{\pgfqpoint{2.896093in}{0.664573in}}%
\pgfpathlineto{\pgfqpoint{2.896599in}{0.655884in}}%
\pgfpathlineto{\pgfqpoint{2.898115in}{0.647750in}}%
\pgfpathlineto{\pgfqpoint{2.898536in}{0.646737in}}%
\pgfpathlineto{\pgfqpoint{2.898873in}{0.648400in}}%
\pgfpathlineto{\pgfqpoint{2.899547in}{0.660493in}}%
\pgfpathlineto{\pgfqpoint{2.899969in}{0.648692in}}%
\pgfpathlineto{\pgfqpoint{2.901064in}{0.641010in}}%
\pgfpathlineto{\pgfqpoint{2.901401in}{0.642127in}}%
\pgfpathlineto{\pgfqpoint{2.901906in}{0.646110in}}%
\pgfpathlineto{\pgfqpoint{2.902412in}{0.641918in}}%
\pgfpathlineto{\pgfqpoint{2.902664in}{0.640805in}}%
\pgfpathlineto{\pgfqpoint{2.903170in}{0.643882in}}%
\pgfpathlineto{\pgfqpoint{2.904434in}{0.649577in}}%
\pgfpathlineto{\pgfqpoint{2.904602in}{0.649048in}}%
\pgfpathlineto{\pgfqpoint{2.905192in}{0.645559in}}%
\pgfpathlineto{\pgfqpoint{2.905613in}{0.648871in}}%
\pgfpathlineto{\pgfqpoint{2.906034in}{0.655339in}}%
\pgfpathlineto{\pgfqpoint{2.906540in}{0.645824in}}%
\pgfpathlineto{\pgfqpoint{2.906792in}{0.644646in}}%
\pgfpathlineto{\pgfqpoint{2.907214in}{0.649013in}}%
\pgfpathlineto{\pgfqpoint{2.908477in}{0.659492in}}%
\pgfpathlineto{\pgfqpoint{2.908561in}{0.658802in}}%
\pgfpathlineto{\pgfqpoint{2.909320in}{0.639115in}}%
\pgfpathlineto{\pgfqpoint{2.909909in}{0.653217in}}%
\pgfpathlineto{\pgfqpoint{2.909994in}{0.655439in}}%
\pgfpathlineto{\pgfqpoint{2.910331in}{0.640661in}}%
\pgfpathlineto{\pgfqpoint{2.910752in}{0.629275in}}%
\pgfpathlineto{\pgfqpoint{2.911510in}{0.632875in}}%
\pgfpathlineto{\pgfqpoint{2.912605in}{0.645658in}}%
\pgfpathlineto{\pgfqpoint{2.913111in}{0.639005in}}%
\pgfpathlineto{\pgfqpoint{2.914037in}{0.632802in}}%
\pgfpathlineto{\pgfqpoint{2.914374in}{0.634752in}}%
\pgfpathlineto{\pgfqpoint{2.915975in}{0.643892in}}%
\pgfpathlineto{\pgfqpoint{2.916396in}{0.641584in}}%
\pgfpathlineto{\pgfqpoint{2.916565in}{0.641475in}}%
\pgfpathlineto{\pgfqpoint{2.916902in}{0.642747in}}%
\pgfpathlineto{\pgfqpoint{2.918334in}{0.651546in}}%
\pgfpathlineto{\pgfqpoint{2.919766in}{0.674614in}}%
\pgfpathlineto{\pgfqpoint{2.920356in}{0.647566in}}%
\pgfpathlineto{\pgfqpoint{2.921282in}{0.659618in}}%
\pgfpathlineto{\pgfqpoint{2.921704in}{0.699374in}}%
\pgfpathlineto{\pgfqpoint{2.922462in}{0.671917in}}%
\pgfpathlineto{\pgfqpoint{2.922630in}{0.670805in}}%
\pgfpathlineto{\pgfqpoint{2.923220in}{0.674317in}}%
\pgfpathlineto{\pgfqpoint{2.923978in}{0.687862in}}%
\pgfpathlineto{\pgfqpoint{2.924231in}{0.703987in}}%
\pgfpathlineto{\pgfqpoint{2.924736in}{0.663767in}}%
\pgfpathlineto{\pgfqpoint{2.925410in}{0.650898in}}%
\pgfpathlineto{\pgfqpoint{2.925916in}{0.658300in}}%
\pgfpathlineto{\pgfqpoint{2.927264in}{0.669602in}}%
\pgfpathlineto{\pgfqpoint{2.928106in}{0.650368in}}%
\pgfpathlineto{\pgfqpoint{2.929201in}{0.653326in}}%
\pgfpathlineto{\pgfqpoint{2.929791in}{0.651600in}}%
\pgfpathlineto{\pgfqpoint{2.930465in}{0.651816in}}%
\pgfpathlineto{\pgfqpoint{2.931644in}{0.652532in}}%
\pgfpathlineto{\pgfqpoint{2.932150in}{0.657228in}}%
\pgfpathlineto{\pgfqpoint{2.932655in}{0.652166in}}%
\pgfpathlineto{\pgfqpoint{2.932992in}{0.651737in}}%
\pgfpathlineto{\pgfqpoint{2.933498in}{0.652705in}}%
\pgfpathlineto{\pgfqpoint{2.935183in}{0.665773in}}%
\pgfpathlineto{\pgfqpoint{2.935436in}{0.660494in}}%
\pgfpathlineto{\pgfqpoint{2.936025in}{0.650730in}}%
\pgfpathlineto{\pgfqpoint{2.936699in}{0.652534in}}%
\pgfpathlineto{\pgfqpoint{2.938553in}{0.656821in}}%
\pgfpathlineto{\pgfqpoint{2.938721in}{0.656055in}}%
\pgfpathlineto{\pgfqpoint{2.939311in}{0.650544in}}%
\pgfpathlineto{\pgfqpoint{2.940069in}{0.651982in}}%
\pgfpathlineto{\pgfqpoint{2.941333in}{0.651048in}}%
\pgfpathlineto{\pgfqpoint{2.941501in}{0.652041in}}%
\pgfpathlineto{\pgfqpoint{2.942681in}{0.661282in}}%
\pgfpathlineto{\pgfqpoint{2.942933in}{0.655532in}}%
\pgfpathlineto{\pgfqpoint{2.943354in}{0.648764in}}%
\pgfpathlineto{\pgfqpoint{2.944197in}{0.649122in}}%
\pgfpathlineto{\pgfqpoint{2.944365in}{0.649294in}}%
\pgfpathlineto{\pgfqpoint{2.944450in}{0.649755in}}%
\pgfpathlineto{\pgfqpoint{2.945461in}{0.655603in}}%
\pgfpathlineto{\pgfqpoint{2.944955in}{0.649246in}}%
\pgfpathlineto{\pgfqpoint{2.945798in}{0.651200in}}%
\pgfpathlineto{\pgfqpoint{2.946050in}{0.649778in}}%
\pgfpathlineto{\pgfqpoint{2.946893in}{0.650656in}}%
\pgfpathlineto{\pgfqpoint{2.948578in}{0.653117in}}%
\pgfpathlineto{\pgfqpoint{2.948999in}{0.663113in}}%
\pgfpathlineto{\pgfqpoint{2.950010in}{0.660583in}}%
\pgfpathlineto{\pgfqpoint{2.950684in}{0.651505in}}%
\pgfpathlineto{\pgfqpoint{2.951105in}{0.658222in}}%
\pgfpathlineto{\pgfqpoint{2.951610in}{0.767802in}}%
\pgfpathlineto{\pgfqpoint{2.952284in}{0.675857in}}%
\pgfpathlineto{\pgfqpoint{2.952706in}{0.674426in}}%
\pgfpathlineto{\pgfqpoint{2.952874in}{0.678152in}}%
\pgfpathlineto{\pgfqpoint{2.953295in}{0.699646in}}%
\pgfpathlineto{\pgfqpoint{2.954306in}{0.698141in}}%
\pgfpathlineto{\pgfqpoint{2.954896in}{0.673600in}}%
\pgfpathlineto{\pgfqpoint{2.955401in}{0.693675in}}%
\pgfpathlineto{\pgfqpoint{2.956581in}{0.739307in}}%
\pgfpathlineto{\pgfqpoint{2.957086in}{0.724689in}}%
\pgfpathlineto{\pgfqpoint{2.958771in}{0.673198in}}%
\pgfpathlineto{\pgfqpoint{2.959277in}{0.684742in}}%
\pgfpathlineto{\pgfqpoint{2.959445in}{0.689129in}}%
\pgfpathlineto{\pgfqpoint{2.959782in}{0.670708in}}%
\pgfpathlineto{\pgfqpoint{2.960372in}{0.654305in}}%
\pgfpathlineto{\pgfqpoint{2.960877in}{0.670844in}}%
\pgfpathlineto{\pgfqpoint{2.961299in}{0.707007in}}%
\pgfpathlineto{\pgfqpoint{2.961973in}{0.678424in}}%
\pgfpathlineto{\pgfqpoint{2.962562in}{0.669113in}}%
\pgfpathlineto{\pgfqpoint{2.964668in}{0.650242in}}%
\pgfpathlineto{\pgfqpoint{2.965174in}{0.648608in}}%
\pgfpathlineto{\pgfqpoint{2.965764in}{0.650279in}}%
\pgfpathlineto{\pgfqpoint{2.966185in}{0.651990in}}%
\pgfpathlineto{\pgfqpoint{2.966943in}{0.650502in}}%
\pgfpathlineto{\pgfqpoint{2.967364in}{0.651787in}}%
\pgfpathlineto{\pgfqpoint{2.967533in}{0.650032in}}%
\pgfpathlineto{\pgfqpoint{2.968122in}{0.642517in}}%
\pgfpathlineto{\pgfqpoint{2.968796in}{0.646497in}}%
\pgfpathlineto{\pgfqpoint{2.969218in}{0.652716in}}%
\pgfpathlineto{\pgfqpoint{2.969807in}{0.645637in}}%
\pgfpathlineto{\pgfqpoint{2.969892in}{0.645566in}}%
\pgfpathlineto{\pgfqpoint{2.970144in}{0.646769in}}%
\pgfpathlineto{\pgfqpoint{2.970228in}{0.647131in}}%
\pgfpathlineto{\pgfqpoint{2.970734in}{0.645179in}}%
\pgfpathlineto{\pgfqpoint{2.970902in}{0.645251in}}%
\pgfpathlineto{\pgfqpoint{2.976715in}{0.654645in}}%
\pgfpathlineto{\pgfqpoint{2.976968in}{0.653711in}}%
\pgfpathlineto{\pgfqpoint{2.977221in}{0.653063in}}%
\pgfpathlineto{\pgfqpoint{2.977642in}{0.655242in}}%
\pgfpathlineto{\pgfqpoint{2.977810in}{0.655854in}}%
\pgfpathlineto{\pgfqpoint{2.978400in}{0.652892in}}%
\pgfpathlineto{\pgfqpoint{2.981770in}{0.652861in}}%
\pgfpathlineto{\pgfqpoint{2.983708in}{0.656324in}}%
\pgfpathlineto{\pgfqpoint{2.984213in}{0.653952in}}%
\pgfpathlineto{\pgfqpoint{2.984719in}{0.653897in}}%
\pgfpathlineto{\pgfqpoint{2.984971in}{0.654548in}}%
\pgfpathlineto{\pgfqpoint{2.986740in}{0.678322in}}%
\pgfpathlineto{\pgfqpoint{2.987077in}{0.664323in}}%
\pgfpathlineto{\pgfqpoint{2.987499in}{0.655935in}}%
\pgfpathlineto{\pgfqpoint{2.987920in}{0.669717in}}%
\pgfpathlineto{\pgfqpoint{2.988510in}{0.718620in}}%
\pgfpathlineto{\pgfqpoint{2.989015in}{0.679851in}}%
\pgfpathlineto{\pgfqpoint{2.990447in}{0.652865in}}%
\pgfpathlineto{\pgfqpoint{2.989520in}{0.687554in}}%
\pgfpathlineto{\pgfqpoint{2.990531in}{0.652899in}}%
\pgfpathlineto{\pgfqpoint{2.990784in}{0.654489in}}%
\pgfpathlineto{\pgfqpoint{2.991121in}{0.681700in}}%
\pgfpathlineto{\pgfqpoint{2.991374in}{0.714543in}}%
\pgfpathlineto{\pgfqpoint{2.992048in}{0.657585in}}%
\pgfpathlineto{\pgfqpoint{2.992638in}{0.656695in}}%
\pgfpathlineto{\pgfqpoint{2.992975in}{0.658365in}}%
\pgfpathlineto{\pgfqpoint{2.994322in}{0.681054in}}%
\pgfpathlineto{\pgfqpoint{2.994659in}{0.728427in}}%
\pgfpathlineto{\pgfqpoint{2.995249in}{0.651701in}}%
\pgfpathlineto{\pgfqpoint{2.995502in}{0.652465in}}%
\pgfpathlineto{\pgfqpoint{2.995923in}{0.651955in}}%
\pgfpathlineto{\pgfqpoint{2.996344in}{0.663363in}}%
\pgfpathlineto{\pgfqpoint{2.996681in}{0.678162in}}%
\pgfpathlineto{\pgfqpoint{2.997271in}{0.655034in}}%
\pgfpathlineto{\pgfqpoint{2.997776in}{0.653289in}}%
\pgfpathlineto{\pgfqpoint{2.997945in}{0.654661in}}%
\pgfpathlineto{\pgfqpoint{2.998450in}{0.691393in}}%
\pgfpathlineto{\pgfqpoint{2.999377in}{0.672265in}}%
\pgfpathlineto{\pgfqpoint{3.000220in}{0.654922in}}%
\pgfpathlineto{\pgfqpoint{3.000725in}{0.663507in}}%
\pgfpathlineto{\pgfqpoint{3.001315in}{0.674550in}}%
\pgfpathlineto{\pgfqpoint{3.002073in}{0.671038in}}%
\pgfpathlineto{\pgfqpoint{3.002157in}{0.671485in}}%
\pgfpathlineto{\pgfqpoint{3.002326in}{0.668743in}}%
\pgfpathlineto{\pgfqpoint{3.002663in}{0.663182in}}%
\pgfpathlineto{\pgfqpoint{3.003168in}{0.672648in}}%
\pgfpathlineto{\pgfqpoint{3.003505in}{0.679346in}}%
\pgfpathlineto{\pgfqpoint{3.004179in}{0.675843in}}%
\pgfpathlineto{\pgfqpoint{3.005021in}{0.646104in}}%
\pgfpathlineto{\pgfqpoint{3.005358in}{0.664728in}}%
\pgfpathlineto{\pgfqpoint{3.005864in}{0.737756in}}%
\pgfpathlineto{\pgfqpoint{3.006369in}{0.649686in}}%
\pgfpathlineto{\pgfqpoint{3.007128in}{0.643890in}}%
\pgfpathlineto{\pgfqpoint{3.007465in}{0.649987in}}%
\pgfpathlineto{\pgfqpoint{3.007970in}{0.678331in}}%
\pgfpathlineto{\pgfqpoint{3.008812in}{0.669464in}}%
\pgfpathlineto{\pgfqpoint{3.009571in}{0.643592in}}%
\pgfpathlineto{\pgfqpoint{3.010497in}{0.654091in}}%
\pgfpathlineto{\pgfqpoint{3.011003in}{0.741893in}}%
\pgfpathlineto{\pgfqpoint{3.011508in}{0.652273in}}%
\pgfpathlineto{\pgfqpoint{3.011845in}{0.641493in}}%
\pgfpathlineto{\pgfqpoint{3.012435in}{0.651388in}}%
\pgfpathlineto{\pgfqpoint{3.012856in}{0.705489in}}%
\pgfpathlineto{\pgfqpoint{3.013867in}{0.697767in}}%
\pgfpathlineto{\pgfqpoint{3.015636in}{0.768846in}}%
\pgfpathlineto{\pgfqpoint{3.015805in}{0.744015in}}%
\pgfpathlineto{\pgfqpoint{3.016731in}{0.648533in}}%
\pgfpathlineto{\pgfqpoint{3.017153in}{0.658907in}}%
\pgfpathlineto{\pgfqpoint{3.017742in}{0.717960in}}%
\pgfpathlineto{\pgfqpoint{3.018753in}{0.698564in}}%
\pgfpathlineto{\pgfqpoint{3.019764in}{0.647595in}}%
\pgfpathlineto{\pgfqpoint{3.020691in}{0.667585in}}%
\pgfpathlineto{\pgfqpoint{3.020859in}{0.666092in}}%
\pgfpathlineto{\pgfqpoint{3.021618in}{0.657204in}}%
\pgfpathlineto{\pgfqpoint{3.022292in}{0.658137in}}%
\pgfpathlineto{\pgfqpoint{3.022713in}{0.658760in}}%
\pgfpathlineto{\pgfqpoint{3.023134in}{0.669012in}}%
\pgfpathlineto{\pgfqpoint{3.023555in}{0.655509in}}%
\pgfpathlineto{\pgfqpoint{3.024145in}{0.644017in}}%
\pgfpathlineto{\pgfqpoint{3.024650in}{0.651635in}}%
\pgfpathlineto{\pgfqpoint{3.025409in}{0.671968in}}%
\pgfpathlineto{\pgfqpoint{3.026167in}{0.662013in}}%
\pgfpathlineto{\pgfqpoint{3.026504in}{0.654598in}}%
\pgfpathlineto{\pgfqpoint{3.027515in}{0.641795in}}%
\pgfpathlineto{\pgfqpoint{3.027852in}{0.643423in}}%
\pgfpathlineto{\pgfqpoint{3.028526in}{0.669281in}}%
\pgfpathlineto{\pgfqpoint{3.028947in}{0.646302in}}%
\pgfpathlineto{\pgfqpoint{3.029705in}{0.641076in}}%
\pgfpathlineto{\pgfqpoint{3.030126in}{0.642362in}}%
\pgfpathlineto{\pgfqpoint{3.030885in}{0.651146in}}%
\pgfpathlineto{\pgfqpoint{3.031390in}{0.690192in}}%
\pgfpathlineto{\pgfqpoint{3.031980in}{0.650434in}}%
\pgfpathlineto{\pgfqpoint{3.032401in}{0.644311in}}%
\pgfpathlineto{\pgfqpoint{3.032991in}{0.651500in}}%
\pgfpathlineto{\pgfqpoint{3.034086in}{0.667266in}}%
\pgfpathlineto{\pgfqpoint{3.034254in}{0.660429in}}%
\pgfpathlineto{\pgfqpoint{3.035013in}{0.637062in}}%
\pgfpathlineto{\pgfqpoint{3.035602in}{0.639179in}}%
\pgfpathlineto{\pgfqpoint{3.036192in}{0.655913in}}%
\pgfpathlineto{\pgfqpoint{3.036697in}{0.639925in}}%
\pgfpathlineto{\pgfqpoint{3.036866in}{0.638901in}}%
\pgfpathlineto{\pgfqpoint{3.037793in}{0.640022in}}%
\pgfpathlineto{\pgfqpoint{3.041499in}{0.654583in}}%
\pgfpathlineto{\pgfqpoint{3.041836in}{0.663891in}}%
\pgfpathlineto{\pgfqpoint{3.042426in}{0.649607in}}%
\pgfpathlineto{\pgfqpoint{3.043858in}{0.650431in}}%
\pgfpathlineto{\pgfqpoint{3.044364in}{0.662519in}}%
\pgfpathlineto{\pgfqpoint{3.044701in}{0.679385in}}%
\pgfpathlineto{\pgfqpoint{3.045459in}{0.664838in}}%
\pgfpathlineto{\pgfqpoint{3.047396in}{0.701798in}}%
\pgfpathlineto{\pgfqpoint{3.047818in}{0.686627in}}%
\pgfpathlineto{\pgfqpoint{3.048323in}{0.665773in}}%
\pgfpathlineto{\pgfqpoint{3.048744in}{0.686239in}}%
\pgfpathlineto{\pgfqpoint{3.049418in}{0.823720in}}%
\pgfpathlineto{\pgfqpoint{3.050092in}{0.735388in}}%
\pgfpathlineto{\pgfqpoint{3.051693in}{0.660860in}}%
\pgfpathlineto{\pgfqpoint{3.051861in}{0.662852in}}%
\pgfpathlineto{\pgfqpoint{3.052535in}{0.709234in}}%
\pgfpathlineto{\pgfqpoint{3.053209in}{0.674364in}}%
\pgfpathlineto{\pgfqpoint{3.054557in}{0.662971in}}%
\pgfpathlineto{\pgfqpoint{3.054726in}{0.664183in}}%
\pgfpathlineto{\pgfqpoint{3.055063in}{0.670320in}}%
\pgfpathlineto{\pgfqpoint{3.055568in}{0.659310in}}%
\pgfpathlineto{\pgfqpoint{3.057000in}{0.652165in}}%
\pgfpathlineto{\pgfqpoint{3.058264in}{0.651316in}}%
\pgfpathlineto{\pgfqpoint{3.058433in}{0.651983in}}%
\pgfpathlineto{\pgfqpoint{3.058854in}{0.654744in}}%
\pgfpathlineto{\pgfqpoint{3.059275in}{0.650706in}}%
\pgfpathlineto{\pgfqpoint{3.060539in}{0.649899in}}%
\pgfpathlineto{\pgfqpoint{3.062813in}{0.651212in}}%
\pgfpathlineto{\pgfqpoint{3.065004in}{0.652321in}}%
\pgfpathlineto{\pgfqpoint{3.065930in}{0.653862in}}%
\pgfpathlineto{\pgfqpoint{3.067110in}{0.656301in}}%
\pgfpathlineto{\pgfqpoint{3.067278in}{0.655579in}}%
\pgfpathlineto{\pgfqpoint{3.068121in}{0.653474in}}%
\pgfpathlineto{\pgfqpoint{3.068542in}{0.653930in}}%
\pgfpathlineto{\pgfqpoint{3.070311in}{0.657212in}}%
\pgfpathlineto{\pgfqpoint{3.070479in}{0.656794in}}%
\pgfpathlineto{\pgfqpoint{3.070985in}{0.655385in}}%
\pgfpathlineto{\pgfqpoint{3.071322in}{0.657244in}}%
\pgfpathlineto{\pgfqpoint{3.071996in}{0.685730in}}%
\pgfpathlineto{\pgfqpoint{3.072164in}{0.692300in}}%
\pgfpathlineto{\pgfqpoint{3.072586in}{0.661233in}}%
\pgfpathlineto{\pgfqpoint{3.073007in}{0.653266in}}%
\pgfpathlineto{\pgfqpoint{3.073765in}{0.655123in}}%
\pgfpathlineto{\pgfqpoint{3.073934in}{0.655044in}}%
\pgfpathlineto{\pgfqpoint{3.074102in}{0.655616in}}%
\pgfpathlineto{\pgfqpoint{3.074523in}{0.660011in}}%
\pgfpathlineto{\pgfqpoint{3.074944in}{0.653674in}}%
\pgfpathlineto{\pgfqpoint{3.076208in}{0.650420in}}%
\pgfpathlineto{\pgfqpoint{3.076545in}{0.650013in}}%
\pgfpathlineto{\pgfqpoint{3.076966in}{0.651319in}}%
\pgfpathlineto{\pgfqpoint{3.077725in}{0.657439in}}%
\pgfpathlineto{\pgfqpoint{3.078146in}{0.714435in}}%
\pgfpathlineto{\pgfqpoint{3.078735in}{0.655099in}}%
\pgfpathlineto{\pgfqpoint{3.078988in}{0.652672in}}%
\pgfpathlineto{\pgfqpoint{3.079409in}{0.661087in}}%
\pgfpathlineto{\pgfqpoint{3.080842in}{0.717072in}}%
\pgfpathlineto{\pgfqpoint{3.081094in}{0.697649in}}%
\pgfpathlineto{\pgfqpoint{3.081516in}{0.663678in}}%
\pgfpathlineto{\pgfqpoint{3.082021in}{0.700087in}}%
\pgfpathlineto{\pgfqpoint{3.082442in}{0.770234in}}%
\pgfpathlineto{\pgfqpoint{3.083032in}{0.696185in}}%
\pgfpathlineto{\pgfqpoint{3.084717in}{0.655241in}}%
\pgfpathlineto{\pgfqpoint{3.085222in}{0.663222in}}%
\pgfpathlineto{\pgfqpoint{3.085559in}{0.672042in}}%
\pgfpathlineto{\pgfqpoint{3.085980in}{0.656986in}}%
\pgfpathlineto{\pgfqpoint{3.087328in}{0.646600in}}%
\pgfpathlineto{\pgfqpoint{3.087581in}{0.646605in}}%
\pgfpathlineto{\pgfqpoint{3.087834in}{0.647247in}}%
\pgfpathlineto{\pgfqpoint{3.088508in}{0.655050in}}%
\pgfpathlineto{\pgfqpoint{3.088929in}{0.647219in}}%
\pgfpathlineto{\pgfqpoint{3.089013in}{0.646870in}}%
\pgfpathlineto{\pgfqpoint{3.089434in}{0.648099in}}%
\pgfpathlineto{\pgfqpoint{3.089771in}{0.648066in}}%
\pgfpathlineto{\pgfqpoint{3.090698in}{0.650687in}}%
\pgfpathlineto{\pgfqpoint{3.091204in}{0.683410in}}%
\pgfpathlineto{\pgfqpoint{3.091962in}{0.659669in}}%
\pgfpathlineto{\pgfqpoint{3.093899in}{0.677273in}}%
\pgfpathlineto{\pgfqpoint{3.094068in}{0.681805in}}%
\pgfpathlineto{\pgfqpoint{3.094405in}{0.662455in}}%
\pgfpathlineto{\pgfqpoint{3.094910in}{0.646733in}}%
\pgfpathlineto{\pgfqpoint{3.095500in}{0.659001in}}%
\pgfpathlineto{\pgfqpoint{3.096595in}{0.668703in}}%
\pgfpathlineto{\pgfqpoint{3.096848in}{0.667618in}}%
\pgfpathlineto{\pgfqpoint{3.097438in}{0.662339in}}%
\pgfpathlineto{\pgfqpoint{3.099207in}{0.644318in}}%
\pgfpathlineto{\pgfqpoint{3.099544in}{0.644277in}}%
\pgfpathlineto{\pgfqpoint{3.099797in}{0.644949in}}%
\pgfpathlineto{\pgfqpoint{3.100302in}{0.648006in}}%
\pgfpathlineto{\pgfqpoint{3.101397in}{0.647734in}}%
\pgfpathlineto{\pgfqpoint{3.105272in}{0.651696in}}%
\pgfpathlineto{\pgfqpoint{3.108137in}{0.655225in}}%
\pgfpathlineto{\pgfqpoint{3.108558in}{0.653812in}}%
\pgfpathlineto{\pgfqpoint{3.110074in}{0.653808in}}%
\pgfpathlineto{\pgfqpoint{3.110159in}{0.654175in}}%
\pgfpathlineto{\pgfqpoint{3.110496in}{0.666546in}}%
\pgfpathlineto{\pgfqpoint{3.110917in}{0.701554in}}%
\pgfpathlineto{\pgfqpoint{3.111591in}{0.666362in}}%
\pgfpathlineto{\pgfqpoint{3.111759in}{0.665379in}}%
\pgfpathlineto{\pgfqpoint{3.112096in}{0.673186in}}%
\pgfpathlineto{\pgfqpoint{3.113613in}{0.708712in}}%
\pgfpathlineto{\pgfqpoint{3.113781in}{0.706624in}}%
\pgfpathlineto{\pgfqpoint{3.114708in}{0.668924in}}%
\pgfpathlineto{\pgfqpoint{3.115298in}{0.689386in}}%
\pgfpathlineto{\pgfqpoint{3.115887in}{0.705498in}}%
\pgfpathlineto{\pgfqpoint{3.116477in}{0.693922in}}%
\pgfpathlineto{\pgfqpoint{3.117572in}{0.672663in}}%
\pgfpathlineto{\pgfqpoint{3.117993in}{0.684323in}}%
\pgfpathlineto{\pgfqpoint{3.118246in}{0.689459in}}%
\pgfpathlineto{\pgfqpoint{3.118667in}{0.674518in}}%
\pgfpathlineto{\pgfqpoint{3.119173in}{0.660187in}}%
\pgfpathlineto{\pgfqpoint{3.119763in}{0.669673in}}%
\pgfpathlineto{\pgfqpoint{3.120858in}{0.677386in}}%
\pgfpathlineto{\pgfqpoint{3.121026in}{0.674607in}}%
\pgfpathlineto{\pgfqpoint{3.122374in}{0.655143in}}%
\pgfpathlineto{\pgfqpoint{3.122543in}{0.656205in}}%
\pgfpathlineto{\pgfqpoint{3.123048in}{0.659308in}}%
\pgfpathlineto{\pgfqpoint{3.123385in}{0.654711in}}%
\pgfpathlineto{\pgfqpoint{3.125323in}{0.642132in}}%
\pgfpathlineto{\pgfqpoint{3.125575in}{0.642253in}}%
\pgfpathlineto{\pgfqpoint{3.125744in}{0.642985in}}%
\pgfpathlineto{\pgfqpoint{3.125912in}{0.643518in}}%
\pgfpathlineto{\pgfqpoint{3.126165in}{0.640946in}}%
\pgfpathlineto{\pgfqpoint{3.127345in}{0.638456in}}%
\pgfpathlineto{\pgfqpoint{3.126839in}{0.642899in}}%
\pgfpathlineto{\pgfqpoint{3.127429in}{0.638622in}}%
\pgfpathlineto{\pgfqpoint{3.129282in}{0.648922in}}%
\pgfpathlineto{\pgfqpoint{3.131051in}{0.671511in}}%
\pgfpathlineto{\pgfqpoint{3.131220in}{0.671333in}}%
\pgfpathlineto{\pgfqpoint{3.132231in}{0.668639in}}%
\pgfpathlineto{\pgfqpoint{3.132399in}{0.670566in}}%
\pgfpathlineto{\pgfqpoint{3.132820in}{0.687118in}}%
\pgfpathlineto{\pgfqpoint{3.133242in}{0.665265in}}%
\pgfpathlineto{\pgfqpoint{3.133747in}{0.655526in}}%
\pgfpathlineto{\pgfqpoint{3.134168in}{0.667170in}}%
\pgfpathlineto{\pgfqpoint{3.134674in}{0.709803in}}%
\pgfpathlineto{\pgfqpoint{3.135432in}{0.678217in}}%
\pgfpathlineto{\pgfqpoint{3.135769in}{0.670904in}}%
\pgfpathlineto{\pgfqpoint{3.137370in}{0.652342in}}%
\pgfpathlineto{\pgfqpoint{3.139223in}{0.671905in}}%
\pgfpathlineto{\pgfqpoint{3.139476in}{0.661733in}}%
\pgfpathlineto{\pgfqpoint{3.139981in}{0.646891in}}%
\pgfpathlineto{\pgfqpoint{3.140655in}{0.654891in}}%
\pgfpathlineto{\pgfqpoint{3.141835in}{0.663542in}}%
\pgfpathlineto{\pgfqpoint{3.142003in}{0.663338in}}%
\pgfpathlineto{\pgfqpoint{3.142761in}{0.661929in}}%
\pgfpathlineto{\pgfqpoint{3.143014in}{0.663547in}}%
\pgfpathlineto{\pgfqpoint{3.145878in}{0.696261in}}%
\pgfpathlineto{\pgfqpoint{3.146131in}{0.694262in}}%
\pgfpathlineto{\pgfqpoint{3.146300in}{0.691814in}}%
\pgfpathlineto{\pgfqpoint{3.146552in}{0.700988in}}%
\pgfpathlineto{\pgfqpoint{3.146805in}{0.724972in}}%
\pgfpathlineto{\pgfqpoint{3.147142in}{0.673516in}}%
\pgfpathlineto{\pgfqpoint{3.147647in}{0.611365in}}%
\pgfpathlineto{\pgfqpoint{3.148406in}{0.636488in}}%
\pgfpathlineto{\pgfqpoint{3.148658in}{0.649402in}}%
\pgfpathlineto{\pgfqpoint{3.149332in}{0.631913in}}%
\pgfpathlineto{\pgfqpoint{3.149417in}{0.632442in}}%
\pgfpathlineto{\pgfqpoint{3.149754in}{0.634800in}}%
\pgfpathlineto{\pgfqpoint{3.150343in}{0.631528in}}%
\pgfpathlineto{\pgfqpoint{3.150428in}{0.631726in}}%
\pgfpathlineto{\pgfqpoint{3.154556in}{0.655292in}}%
\pgfpathlineto{\pgfqpoint{3.155651in}{0.650715in}}%
\pgfpathlineto{\pgfqpoint{3.156240in}{0.648163in}}%
\pgfpathlineto{\pgfqpoint{3.156999in}{0.649084in}}%
\pgfpathlineto{\pgfqpoint{3.157420in}{0.657005in}}%
\pgfpathlineto{\pgfqpoint{3.157588in}{0.659813in}}%
\pgfpathlineto{\pgfqpoint{3.158178in}{0.650420in}}%
\pgfpathlineto{\pgfqpoint{3.158684in}{0.650770in}}%
\pgfpathlineto{\pgfqpoint{3.158936in}{0.651129in}}%
\pgfpathlineto{\pgfqpoint{3.159863in}{0.654268in}}%
\pgfpathlineto{\pgfqpoint{3.160116in}{0.659534in}}%
\pgfpathlineto{\pgfqpoint{3.160705in}{0.650834in}}%
\pgfpathlineto{\pgfqpoint{3.160790in}{0.650922in}}%
\pgfpathlineto{\pgfqpoint{3.161716in}{0.653184in}}%
\pgfpathlineto{\pgfqpoint{3.162138in}{0.661454in}}%
\pgfpathlineto{\pgfqpoint{3.162811in}{0.653634in}}%
\pgfpathlineto{\pgfqpoint{3.163064in}{0.654372in}}%
\pgfpathlineto{\pgfqpoint{3.163485in}{0.651664in}}%
\pgfpathlineto{\pgfqpoint{3.164159in}{0.649018in}}%
\pgfpathlineto{\pgfqpoint{3.164581in}{0.651085in}}%
\pgfpathlineto{\pgfqpoint{3.164833in}{0.653262in}}%
\pgfpathlineto{\pgfqpoint{3.165507in}{0.649157in}}%
\pgfpathlineto{\pgfqpoint{3.165844in}{0.648813in}}%
\pgfpathlineto{\pgfqpoint{3.166181in}{0.649894in}}%
\pgfpathlineto{\pgfqpoint{3.167782in}{0.666674in}}%
\pgfpathlineto{\pgfqpoint{3.167950in}{0.659783in}}%
\pgfpathlineto{\pgfqpoint{3.168287in}{0.649691in}}%
\pgfpathlineto{\pgfqpoint{3.169130in}{0.650466in}}%
\pgfpathlineto{\pgfqpoint{3.169467in}{0.659100in}}%
\pgfpathlineto{\pgfqpoint{3.169804in}{0.672354in}}%
\pgfpathlineto{\pgfqpoint{3.170478in}{0.655435in}}%
\pgfpathlineto{\pgfqpoint{3.171320in}{0.653792in}}%
\pgfpathlineto{\pgfqpoint{3.171657in}{0.654575in}}%
\pgfpathlineto{\pgfqpoint{3.172415in}{0.657019in}}%
\pgfpathlineto{\pgfqpoint{3.172837in}{0.714227in}}%
\pgfpathlineto{\pgfqpoint{3.173005in}{0.733599in}}%
\pgfpathlineto{\pgfqpoint{3.173595in}{0.668687in}}%
\pgfpathlineto{\pgfqpoint{3.174606in}{0.651056in}}%
\pgfpathlineto{\pgfqpoint{3.174858in}{0.655157in}}%
\pgfpathlineto{\pgfqpoint{3.175195in}{0.668176in}}%
\pgfpathlineto{\pgfqpoint{3.175954in}{0.659521in}}%
\pgfpathlineto{\pgfqpoint{3.176712in}{0.652549in}}%
\pgfpathlineto{\pgfqpoint{3.177217in}{0.653592in}}%
\pgfpathlineto{\pgfqpoint{3.177723in}{0.669560in}}%
\pgfpathlineto{\pgfqpoint{3.178060in}{0.651807in}}%
\pgfpathlineto{\pgfqpoint{3.178312in}{0.648549in}}%
\pgfpathlineto{\pgfqpoint{3.179071in}{0.654150in}}%
\pgfpathlineto{\pgfqpoint{3.179323in}{0.660192in}}%
\pgfpathlineto{\pgfqpoint{3.180166in}{0.653877in}}%
\pgfpathlineto{\pgfqpoint{3.180924in}{0.653244in}}%
\pgfpathlineto{\pgfqpoint{3.181177in}{0.654256in}}%
\pgfpathlineto{\pgfqpoint{3.181345in}{0.654865in}}%
\pgfpathlineto{\pgfqpoint{3.181598in}{0.651676in}}%
\pgfpathlineto{\pgfqpoint{3.182019in}{0.649252in}}%
\pgfpathlineto{\pgfqpoint{3.182777in}{0.649728in}}%
\pgfpathlineto{\pgfqpoint{3.185979in}{0.653601in}}%
\pgfpathlineto{\pgfqpoint{3.186316in}{0.666236in}}%
\pgfpathlineto{\pgfqpoint{3.186821in}{0.652547in}}%
\pgfpathlineto{\pgfqpoint{3.187074in}{0.653120in}}%
\pgfpathlineto{\pgfqpoint{3.187579in}{0.652526in}}%
\pgfpathlineto{\pgfqpoint{3.187832in}{0.655504in}}%
\pgfpathlineto{\pgfqpoint{3.188169in}{0.683826in}}%
\pgfpathlineto{\pgfqpoint{3.189012in}{0.669493in}}%
\pgfpathlineto{\pgfqpoint{3.190359in}{0.651518in}}%
\pgfpathlineto{\pgfqpoint{3.190444in}{0.651869in}}%
\pgfpathlineto{\pgfqpoint{3.191033in}{0.669282in}}%
\pgfpathlineto{\pgfqpoint{3.191623in}{0.702911in}}%
\pgfpathlineto{\pgfqpoint{3.192381in}{0.686586in}}%
\pgfpathlineto{\pgfqpoint{3.192550in}{0.689154in}}%
\pgfpathlineto{\pgfqpoint{3.192803in}{0.676945in}}%
\pgfpathlineto{\pgfqpoint{3.193392in}{0.658754in}}%
\pgfpathlineto{\pgfqpoint{3.193813in}{0.672495in}}%
\pgfpathlineto{\pgfqpoint{3.194235in}{0.751313in}}%
\pgfpathlineto{\pgfqpoint{3.194740in}{0.658403in}}%
\pgfpathlineto{\pgfqpoint{3.195330in}{0.653294in}}%
\pgfpathlineto{\pgfqpoint{3.195920in}{0.655403in}}%
\pgfpathlineto{\pgfqpoint{3.196088in}{0.654697in}}%
\pgfpathlineto{\pgfqpoint{3.196257in}{0.656891in}}%
\pgfpathlineto{\pgfqpoint{3.196678in}{0.672031in}}%
\pgfpathlineto{\pgfqpoint{3.197183in}{0.652592in}}%
\pgfpathlineto{\pgfqpoint{3.197267in}{0.652339in}}%
\pgfpathlineto{\pgfqpoint{3.197689in}{0.653157in}}%
\pgfpathlineto{\pgfqpoint{3.198110in}{0.652879in}}%
\pgfpathlineto{\pgfqpoint{3.199121in}{0.660375in}}%
\pgfpathlineto{\pgfqpoint{3.199374in}{0.682939in}}%
\pgfpathlineto{\pgfqpoint{3.199963in}{0.653042in}}%
\pgfpathlineto{\pgfqpoint{3.200132in}{0.654144in}}%
\pgfpathlineto{\pgfqpoint{3.200385in}{0.656913in}}%
\pgfpathlineto{\pgfqpoint{3.201059in}{0.651526in}}%
\pgfpathlineto{\pgfqpoint{3.202406in}{0.652396in}}%
\pgfpathlineto{\pgfqpoint{3.202659in}{0.652898in}}%
\pgfpathlineto{\pgfqpoint{3.203080in}{0.652042in}}%
\pgfpathlineto{\pgfqpoint{3.203502in}{0.652278in}}%
\pgfpathlineto{\pgfqpoint{3.204934in}{0.652932in}}%
\pgfpathlineto{\pgfqpoint{3.205271in}{0.670953in}}%
\pgfpathlineto{\pgfqpoint{3.205439in}{0.680470in}}%
\pgfpathlineto{\pgfqpoint{3.206197in}{0.657405in}}%
\pgfpathlineto{\pgfqpoint{3.207461in}{0.652127in}}%
\pgfpathlineto{\pgfqpoint{3.207630in}{0.652214in}}%
\pgfpathlineto{\pgfqpoint{3.208219in}{0.653739in}}%
\pgfpathlineto{\pgfqpoint{3.208641in}{0.661135in}}%
\pgfpathlineto{\pgfqpoint{3.209230in}{0.653978in}}%
\pgfpathlineto{\pgfqpoint{3.209567in}{0.652650in}}%
\pgfpathlineto{\pgfqpoint{3.210325in}{0.653102in}}%
\pgfpathlineto{\pgfqpoint{3.211252in}{0.659181in}}%
\pgfpathlineto{\pgfqpoint{3.211589in}{0.687663in}}%
\pgfpathlineto{\pgfqpoint{3.212263in}{0.653530in}}%
\pgfpathlineto{\pgfqpoint{3.212516in}{0.652701in}}%
\pgfpathlineto{\pgfqpoint{3.213442in}{0.653115in}}%
\pgfpathlineto{\pgfqpoint{3.214201in}{0.658208in}}%
\pgfpathlineto{\pgfqpoint{3.214622in}{0.686845in}}%
\pgfpathlineto{\pgfqpoint{3.215296in}{0.658480in}}%
\pgfpathlineto{\pgfqpoint{3.216391in}{0.651822in}}%
\pgfpathlineto{\pgfqpoint{3.216559in}{0.653433in}}%
\pgfpathlineto{\pgfqpoint{3.217570in}{0.740625in}}%
\pgfpathlineto{\pgfqpoint{3.217739in}{0.790063in}}%
\pgfpathlineto{\pgfqpoint{3.218413in}{0.663423in}}%
\pgfpathlineto{\pgfqpoint{3.218750in}{0.657214in}}%
\pgfpathlineto{\pgfqpoint{3.219255in}{0.667188in}}%
\pgfpathlineto{\pgfqpoint{3.220266in}{0.681864in}}%
\pgfpathlineto{\pgfqpoint{3.220435in}{0.678708in}}%
\pgfpathlineto{\pgfqpoint{3.221024in}{0.655418in}}%
\pgfpathlineto{\pgfqpoint{3.221614in}{0.667441in}}%
\pgfpathlineto{\pgfqpoint{3.221783in}{0.670195in}}%
\pgfpathlineto{\pgfqpoint{3.222120in}{0.656591in}}%
\pgfpathlineto{\pgfqpoint{3.222457in}{0.651175in}}%
\pgfpathlineto{\pgfqpoint{3.223131in}{0.657226in}}%
\pgfpathlineto{\pgfqpoint{3.224478in}{0.673578in}}%
\pgfpathlineto{\pgfqpoint{3.224900in}{0.684933in}}%
\pgfpathlineto{\pgfqpoint{3.225574in}{0.675501in}}%
\pgfpathlineto{\pgfqpoint{3.226079in}{0.662508in}}%
\pgfpathlineto{\pgfqpoint{3.226753in}{0.667425in}}%
\pgfpathlineto{\pgfqpoint{3.227090in}{0.756012in}}%
\pgfpathlineto{\pgfqpoint{3.227259in}{0.801991in}}%
\pgfpathlineto{\pgfqpoint{3.227933in}{0.651579in}}%
\pgfpathlineto{\pgfqpoint{3.228185in}{0.649376in}}%
\pgfpathlineto{\pgfqpoint{3.228522in}{0.656510in}}%
\pgfpathlineto{\pgfqpoint{3.229112in}{0.757443in}}%
\pgfpathlineto{\pgfqpoint{3.229954in}{0.707345in}}%
\pgfpathlineto{\pgfqpoint{3.230965in}{0.673481in}}%
\pgfpathlineto{\pgfqpoint{3.231471in}{0.694114in}}%
\pgfpathlineto{\pgfqpoint{3.232060in}{0.718886in}}%
\pgfpathlineto{\pgfqpoint{3.232313in}{0.700452in}}%
\pgfpathlineto{\pgfqpoint{3.233408in}{0.654688in}}%
\pgfpathlineto{\pgfqpoint{3.233661in}{0.659025in}}%
\pgfpathlineto{\pgfqpoint{3.234841in}{0.708605in}}%
\pgfpathlineto{\pgfqpoint{3.235009in}{0.727362in}}%
\pgfpathlineto{\pgfqpoint{3.235599in}{0.654160in}}%
\pgfpathlineto{\pgfqpoint{3.235851in}{0.650208in}}%
\pgfpathlineto{\pgfqpoint{3.236357in}{0.660896in}}%
\pgfpathlineto{\pgfqpoint{3.237789in}{0.679690in}}%
\pgfpathlineto{\pgfqpoint{3.239558in}{0.648473in}}%
\pgfpathlineto{\pgfqpoint{3.240232in}{0.646158in}}%
\pgfpathlineto{\pgfqpoint{3.240653in}{0.647385in}}%
\pgfpathlineto{\pgfqpoint{3.241833in}{0.656157in}}%
\pgfpathlineto{\pgfqpoint{3.242086in}{0.650383in}}%
\pgfpathlineto{\pgfqpoint{3.242591in}{0.642368in}}%
\pgfpathlineto{\pgfqpoint{3.243349in}{0.644168in}}%
\pgfpathlineto{\pgfqpoint{3.248741in}{0.652897in}}%
\pgfpathlineto{\pgfqpoint{3.248909in}{0.652611in}}%
\pgfpathlineto{\pgfqpoint{3.249752in}{0.652536in}}%
\pgfpathlineto{\pgfqpoint{3.250089in}{0.652834in}}%
\pgfpathlineto{\pgfqpoint{3.252111in}{0.655121in}}%
\pgfpathlineto{\pgfqpoint{3.252616in}{0.657819in}}%
\pgfpathlineto{\pgfqpoint{3.252953in}{0.653680in}}%
\pgfpathlineto{\pgfqpoint{3.253122in}{0.653079in}}%
\pgfpathlineto{\pgfqpoint{3.253880in}{0.653964in}}%
\pgfpathlineto{\pgfqpoint{3.254722in}{0.658790in}}%
\pgfpathlineto{\pgfqpoint{3.255396in}{0.657314in}}%
\pgfpathlineto{\pgfqpoint{3.255817in}{0.662625in}}%
\pgfpathlineto{\pgfqpoint{3.255902in}{0.663581in}}%
\pgfpathlineto{\pgfqpoint{3.256154in}{0.656464in}}%
\pgfpathlineto{\pgfqpoint{3.256491in}{0.651352in}}%
\pgfpathlineto{\pgfqpoint{3.257250in}{0.653731in}}%
\pgfpathlineto{\pgfqpoint{3.258429in}{0.680406in}}%
\pgfpathlineto{\pgfqpoint{3.258008in}{0.652166in}}%
\pgfpathlineto{\pgfqpoint{3.258598in}{0.672559in}}%
\pgfpathlineto{\pgfqpoint{3.259356in}{0.692242in}}%
\pgfpathlineto{\pgfqpoint{3.259861in}{0.649784in}}%
\pgfpathlineto{\pgfqpoint{3.260451in}{0.649430in}}%
\pgfpathlineto{\pgfqpoint{3.260535in}{0.650061in}}%
\pgfpathlineto{\pgfqpoint{3.260788in}{0.680459in}}%
\pgfpathlineto{\pgfqpoint{3.261209in}{0.780600in}}%
\pgfpathlineto{\pgfqpoint{3.261883in}{0.684455in}}%
\pgfpathlineto{\pgfqpoint{3.262052in}{0.680198in}}%
\pgfpathlineto{\pgfqpoint{3.262304in}{0.668151in}}%
\pgfpathlineto{\pgfqpoint{3.262641in}{0.709313in}}%
\pgfpathlineto{\pgfqpoint{3.263736in}{0.839177in}}%
\pgfpathlineto{\pgfqpoint{3.263231in}{0.704540in}}%
\pgfpathlineto{\pgfqpoint{3.263905in}{0.788838in}}%
\pgfpathlineto{\pgfqpoint{3.264747in}{0.660805in}}%
\pgfpathlineto{\pgfqpoint{3.265169in}{0.681778in}}%
\pgfpathlineto{\pgfqpoint{3.265506in}{0.733300in}}%
\pgfpathlineto{\pgfqpoint{3.266264in}{0.690215in}}%
\pgfpathlineto{\pgfqpoint{3.269718in}{0.654164in}}%
\pgfpathlineto{\pgfqpoint{3.270308in}{0.653261in}}%
\pgfpathlineto{\pgfqpoint{3.270560in}{0.655208in}}%
\pgfpathlineto{\pgfqpoint{3.271740in}{0.689581in}}%
\pgfpathlineto{\pgfqpoint{3.271992in}{0.671582in}}%
\pgfpathlineto{\pgfqpoint{3.272414in}{0.649246in}}%
\pgfpathlineto{\pgfqpoint{3.273172in}{0.652101in}}%
\pgfpathlineto{\pgfqpoint{3.273425in}{0.654446in}}%
\pgfpathlineto{\pgfqpoint{3.274014in}{0.649716in}}%
\pgfpathlineto{\pgfqpoint{3.274099in}{0.649946in}}%
\pgfpathlineto{\pgfqpoint{3.274351in}{0.651490in}}%
\pgfpathlineto{\pgfqpoint{3.274941in}{0.648968in}}%
\pgfpathlineto{\pgfqpoint{3.275025in}{0.648984in}}%
\pgfpathlineto{\pgfqpoint{3.275531in}{0.650814in}}%
\pgfpathlineto{\pgfqpoint{3.276710in}{0.649525in}}%
\pgfpathlineto{\pgfqpoint{3.277468in}{0.649633in}}%
\pgfpathlineto{\pgfqpoint{3.277553in}{0.650066in}}%
\pgfpathlineto{\pgfqpoint{3.277890in}{0.654888in}}%
\pgfpathlineto{\pgfqpoint{3.278311in}{0.649883in}}%
\pgfpathlineto{\pgfqpoint{3.278648in}{0.650545in}}%
\pgfpathlineto{\pgfqpoint{3.280333in}{0.655556in}}%
\pgfpathlineto{\pgfqpoint{3.280501in}{0.653761in}}%
\pgfpathlineto{\pgfqpoint{3.280754in}{0.651581in}}%
\pgfpathlineto{\pgfqpoint{3.281681in}{0.652333in}}%
\pgfpathlineto{\pgfqpoint{3.282102in}{0.661421in}}%
\pgfpathlineto{\pgfqpoint{3.282944in}{0.653812in}}%
\pgfpathlineto{\pgfqpoint{3.283702in}{0.652018in}}%
\pgfpathlineto{\pgfqpoint{3.283955in}{0.653339in}}%
\pgfpathlineto{\pgfqpoint{3.284208in}{0.656324in}}%
\pgfpathlineto{\pgfqpoint{3.284713in}{0.652401in}}%
\pgfpathlineto{\pgfqpoint{3.285135in}{0.655206in}}%
\pgfpathlineto{\pgfqpoint{3.285472in}{0.652325in}}%
\pgfpathlineto{\pgfqpoint{3.286314in}{0.653673in}}%
\pgfpathlineto{\pgfqpoint{3.288841in}{0.697902in}}%
\pgfpathlineto{\pgfqpoint{3.289094in}{0.730235in}}%
\pgfpathlineto{\pgfqpoint{3.289600in}{0.658564in}}%
\pgfpathlineto{\pgfqpoint{3.289768in}{0.653759in}}%
\pgfpathlineto{\pgfqpoint{3.290358in}{0.660503in}}%
\pgfpathlineto{\pgfqpoint{3.290863in}{0.863961in}}%
\pgfpathlineto{\pgfqpoint{3.291453in}{0.675090in}}%
\pgfpathlineto{\pgfqpoint{3.291958in}{0.661052in}}%
\pgfpathlineto{\pgfqpoint{3.292295in}{0.677706in}}%
\pgfpathlineto{\pgfqpoint{3.292632in}{0.746412in}}%
\pgfpathlineto{\pgfqpoint{3.293475in}{0.691968in}}%
\pgfpathlineto{\pgfqpoint{3.295328in}{0.655802in}}%
\pgfpathlineto{\pgfqpoint{3.295412in}{0.655902in}}%
\pgfpathlineto{\pgfqpoint{3.295834in}{0.664538in}}%
\pgfpathlineto{\pgfqpoint{3.296171in}{0.674033in}}%
\pgfpathlineto{\pgfqpoint{3.296760in}{0.659329in}}%
\pgfpathlineto{\pgfqpoint{3.297350in}{0.657154in}}%
\pgfpathlineto{\pgfqpoint{3.297687in}{0.659903in}}%
\pgfpathlineto{\pgfqpoint{3.298866in}{0.691793in}}%
\pgfpathlineto{\pgfqpoint{3.299119in}{0.668972in}}%
\pgfpathlineto{\pgfqpoint{3.299625in}{0.650142in}}%
\pgfpathlineto{\pgfqpoint{3.300383in}{0.650939in}}%
\pgfpathlineto{\pgfqpoint{3.300551in}{0.651396in}}%
\pgfpathlineto{\pgfqpoint{3.301057in}{0.650392in}}%
\pgfpathlineto{\pgfqpoint{3.301394in}{0.650522in}}%
\pgfpathlineto{\pgfqpoint{3.304427in}{0.651373in}}%
\pgfpathlineto{\pgfqpoint{3.304764in}{0.653186in}}%
\pgfpathlineto{\pgfqpoint{3.305774in}{0.652629in}}%
\pgfpathlineto{\pgfqpoint{3.307122in}{0.652454in}}%
\pgfpathlineto{\pgfqpoint{3.311840in}{0.656260in}}%
\pgfpathlineto{\pgfqpoint{3.312009in}{0.657373in}}%
\pgfpathlineto{\pgfqpoint{3.312514in}{0.654665in}}%
\pgfpathlineto{\pgfqpoint{3.312767in}{0.654757in}}%
\pgfpathlineto{\pgfqpoint{3.313441in}{0.656054in}}%
\pgfpathlineto{\pgfqpoint{3.314620in}{0.660628in}}%
\pgfpathlineto{\pgfqpoint{3.314789in}{0.659838in}}%
\pgfpathlineto{\pgfqpoint{3.315126in}{0.658804in}}%
\pgfpathlineto{\pgfqpoint{3.315547in}{0.661849in}}%
\pgfpathlineto{\pgfqpoint{3.316726in}{0.701757in}}%
\pgfpathlineto{\pgfqpoint{3.316895in}{0.684475in}}%
\pgfpathlineto{\pgfqpoint{3.317316in}{0.651076in}}%
\pgfpathlineto{\pgfqpoint{3.318074in}{0.655850in}}%
\pgfpathlineto{\pgfqpoint{3.318327in}{0.679965in}}%
\pgfpathlineto{\pgfqpoint{3.318580in}{0.739713in}}%
\pgfpathlineto{\pgfqpoint{3.319338in}{0.656536in}}%
\pgfpathlineto{\pgfqpoint{3.320601in}{0.651283in}}%
\pgfpathlineto{\pgfqpoint{3.320686in}{0.651400in}}%
\pgfpathlineto{\pgfqpoint{3.321949in}{0.660262in}}%
\pgfpathlineto{\pgfqpoint{3.322286in}{0.671049in}}%
\pgfpathlineto{\pgfqpoint{3.322708in}{0.650552in}}%
\pgfpathlineto{\pgfqpoint{3.322792in}{0.650586in}}%
\pgfpathlineto{\pgfqpoint{3.323887in}{0.659986in}}%
\pgfpathlineto{\pgfqpoint{3.324056in}{0.653207in}}%
\pgfpathlineto{\pgfqpoint{3.324224in}{0.650136in}}%
\pgfpathlineto{\pgfqpoint{3.324392in}{0.665189in}}%
\pgfpathlineto{\pgfqpoint{3.324645in}{0.859886in}}%
\pgfpathlineto{\pgfqpoint{3.325151in}{0.655710in}}%
\pgfpathlineto{\pgfqpoint{3.325488in}{0.676713in}}%
\pgfpathlineto{\pgfqpoint{3.325825in}{0.651796in}}%
\pgfpathlineto{\pgfqpoint{3.326077in}{0.674899in}}%
\pgfpathlineto{\pgfqpoint{3.326330in}{0.791220in}}%
\pgfpathlineto{\pgfqpoint{3.326836in}{0.667721in}}%
\pgfpathlineto{\pgfqpoint{3.327173in}{0.704690in}}%
\pgfpathlineto{\pgfqpoint{3.327425in}{0.675892in}}%
\pgfpathlineto{\pgfqpoint{3.327931in}{0.652188in}}%
\pgfpathlineto{\pgfqpoint{3.328605in}{0.653477in}}%
\pgfpathlineto{\pgfqpoint{3.329026in}{0.667116in}}%
\pgfpathlineto{\pgfqpoint{3.329531in}{0.652401in}}%
\pgfpathlineto{\pgfqpoint{3.329616in}{0.652572in}}%
\pgfpathlineto{\pgfqpoint{3.329953in}{0.652175in}}%
\pgfpathlineto{\pgfqpoint{3.330205in}{0.653875in}}%
\pgfpathlineto{\pgfqpoint{3.330542in}{0.673166in}}%
\pgfpathlineto{\pgfqpoint{3.331385in}{0.660905in}}%
\pgfpathlineto{\pgfqpoint{3.331890in}{0.652382in}}%
\pgfpathlineto{\pgfqpoint{3.332648in}{0.655224in}}%
\pgfpathlineto{\pgfqpoint{3.332901in}{0.653930in}}%
\pgfpathlineto{\pgfqpoint{3.333238in}{0.652650in}}%
\pgfpathlineto{\pgfqpoint{3.333912in}{0.654271in}}%
\pgfpathlineto{\pgfqpoint{3.335007in}{0.665890in}}%
\pgfpathlineto{\pgfqpoint{3.334502in}{0.653154in}}%
\pgfpathlineto{\pgfqpoint{3.335176in}{0.659552in}}%
\pgfpathlineto{\pgfqpoint{3.335429in}{0.653218in}}%
\pgfpathlineto{\pgfqpoint{3.335934in}{0.668878in}}%
\pgfpathlineto{\pgfqpoint{3.336355in}{0.653350in}}%
\pgfpathlineto{\pgfqpoint{3.336608in}{0.673997in}}%
\pgfpathlineto{\pgfqpoint{3.336776in}{0.698351in}}%
\pgfpathlineto{\pgfqpoint{3.337282in}{0.653902in}}%
\pgfpathlineto{\pgfqpoint{3.337787in}{0.689638in}}%
\pgfpathlineto{\pgfqpoint{3.339051in}{0.652568in}}%
\pgfpathlineto{\pgfqpoint{3.339135in}{0.652690in}}%
\pgfpathlineto{\pgfqpoint{3.339557in}{0.666747in}}%
\pgfpathlineto{\pgfqpoint{3.340315in}{0.653584in}}%
\pgfpathlineto{\pgfqpoint{3.340399in}{0.653607in}}%
\pgfpathlineto{\pgfqpoint{3.340652in}{0.652564in}}%
\pgfpathlineto{\pgfqpoint{3.341241in}{0.652370in}}%
\pgfpathlineto{\pgfqpoint{3.341578in}{0.652877in}}%
\pgfpathlineto{\pgfqpoint{3.342000in}{0.657964in}}%
\pgfpathlineto{\pgfqpoint{3.342168in}{0.660143in}}%
\pgfpathlineto{\pgfqpoint{3.342926in}{0.655274in}}%
\pgfpathlineto{\pgfqpoint{3.343348in}{0.655498in}}%
\pgfpathlineto{\pgfqpoint{3.343432in}{0.655821in}}%
\pgfpathlineto{\pgfqpoint{3.344190in}{0.660558in}}%
\pgfpathlineto{\pgfqpoint{3.344527in}{0.656667in}}%
\pgfpathlineto{\pgfqpoint{3.344611in}{0.656025in}}%
\pgfpathlineto{\pgfqpoint{3.344864in}{0.659693in}}%
\pgfpathlineto{\pgfqpoint{3.345538in}{0.703618in}}%
\pgfpathlineto{\pgfqpoint{3.346380in}{0.680565in}}%
\pgfpathlineto{\pgfqpoint{3.347054in}{0.669293in}}%
\pgfpathlineto{\pgfqpoint{3.347223in}{0.667206in}}%
\pgfpathlineto{\pgfqpoint{3.347475in}{0.678601in}}%
\pgfpathlineto{\pgfqpoint{3.347981in}{0.741505in}}%
\pgfpathlineto{\pgfqpoint{3.348486in}{0.676649in}}%
\pgfpathlineto{\pgfqpoint{3.348908in}{0.663155in}}%
\pgfpathlineto{\pgfqpoint{3.349329in}{0.687120in}}%
\pgfpathlineto{\pgfqpoint{3.349750in}{0.733930in}}%
\pgfpathlineto{\pgfqpoint{3.350677in}{0.724673in}}%
\pgfpathlineto{\pgfqpoint{3.351688in}{0.651288in}}%
\pgfpathlineto{\pgfqpoint{3.352193in}{0.678776in}}%
\pgfpathlineto{\pgfqpoint{3.352530in}{0.727605in}}%
\pgfpathlineto{\pgfqpoint{3.353204in}{0.670462in}}%
\pgfpathlineto{\pgfqpoint{3.353373in}{0.669754in}}%
\pgfpathlineto{\pgfqpoint{3.353878in}{0.673697in}}%
\pgfpathlineto{\pgfqpoint{3.353962in}{0.674167in}}%
\pgfpathlineto{\pgfqpoint{3.354552in}{0.673093in}}%
\pgfpathlineto{\pgfqpoint{3.355394in}{0.655549in}}%
\pgfpathlineto{\pgfqpoint{3.355731in}{0.664723in}}%
\pgfpathlineto{\pgfqpoint{3.356153in}{0.758070in}}%
\pgfpathlineto{\pgfqpoint{3.356742in}{0.656473in}}%
\pgfpathlineto{\pgfqpoint{3.357079in}{0.652027in}}%
\pgfpathlineto{\pgfqpoint{3.357501in}{0.660062in}}%
\pgfpathlineto{\pgfqpoint{3.357838in}{0.687313in}}%
\pgfpathlineto{\pgfqpoint{3.358764in}{0.673144in}}%
\pgfpathlineto{\pgfqpoint{3.359607in}{0.650709in}}%
\pgfpathlineto{\pgfqpoint{3.360281in}{0.655037in}}%
\pgfpathlineto{\pgfqpoint{3.360786in}{0.653569in}}%
\pgfpathlineto{\pgfqpoint{3.360955in}{0.654972in}}%
\pgfpathlineto{\pgfqpoint{3.362808in}{0.690751in}}%
\pgfpathlineto{\pgfqpoint{3.363061in}{0.674255in}}%
\pgfpathlineto{\pgfqpoint{3.363566in}{0.645896in}}%
\pgfpathlineto{\pgfqpoint{3.364072in}{0.672427in}}%
\pgfpathlineto{\pgfqpoint{3.364409in}{0.695444in}}%
\pgfpathlineto{\pgfqpoint{3.364914in}{0.657336in}}%
\pgfpathlineto{\pgfqpoint{3.365167in}{0.652330in}}%
\pgfpathlineto{\pgfqpoint{3.365841in}{0.660869in}}%
\pgfpathlineto{\pgfqpoint{3.365925in}{0.660810in}}%
\pgfpathlineto{\pgfqpoint{3.367273in}{0.648088in}}%
\pgfpathlineto{\pgfqpoint{3.366431in}{0.661113in}}%
\pgfpathlineto{\pgfqpoint{3.367610in}{0.653247in}}%
\pgfpathlineto{\pgfqpoint{3.368284in}{0.692395in}}%
\pgfpathlineto{\pgfqpoint{3.368705in}{0.654473in}}%
\pgfpathlineto{\pgfqpoint{3.370053in}{0.630555in}}%
\pgfpathlineto{\pgfqpoint{3.370558in}{0.636076in}}%
\pgfpathlineto{\pgfqpoint{3.371148in}{0.680290in}}%
\pgfpathlineto{\pgfqpoint{3.371738in}{0.642687in}}%
\pgfpathlineto{\pgfqpoint{3.372412in}{0.652573in}}%
\pgfpathlineto{\pgfqpoint{3.373086in}{0.633837in}}%
\pgfpathlineto{\pgfqpoint{3.373254in}{0.632556in}}%
\pgfpathlineto{\pgfqpoint{3.373844in}{0.636104in}}%
\pgfpathlineto{\pgfqpoint{3.375613in}{0.646548in}}%
\pgfpathlineto{\pgfqpoint{3.375950in}{0.641981in}}%
\pgfpathlineto{\pgfqpoint{3.376203in}{0.640473in}}%
\pgfpathlineto{\pgfqpoint{3.376708in}{0.646000in}}%
\pgfpathlineto{\pgfqpoint{3.377130in}{0.721428in}}%
\pgfpathlineto{\pgfqpoint{3.377382in}{0.768146in}}%
\pgfpathlineto{\pgfqpoint{3.378056in}{0.678531in}}%
\pgfpathlineto{\pgfqpoint{3.379236in}{0.655543in}}%
\pgfpathlineto{\pgfqpoint{3.379488in}{0.664002in}}%
\pgfpathlineto{\pgfqpoint{3.380078in}{0.757846in}}%
\pgfpathlineto{\pgfqpoint{3.380668in}{0.686562in}}%
\pgfpathlineto{\pgfqpoint{3.381005in}{0.669306in}}%
\pgfpathlineto{\pgfqpoint{3.381679in}{0.695058in}}%
\pgfpathlineto{\pgfqpoint{3.381932in}{0.701270in}}%
\pgfpathlineto{\pgfqpoint{3.382353in}{0.693666in}}%
\pgfpathlineto{\pgfqpoint{3.382690in}{0.694616in}}%
\pgfpathlineto{\pgfqpoint{3.383364in}{0.667088in}}%
\pgfpathlineto{\pgfqpoint{3.383869in}{0.691849in}}%
\pgfpathlineto{\pgfqpoint{3.384038in}{0.698375in}}%
\pgfpathlineto{\pgfqpoint{3.384543in}{0.671770in}}%
\pgfpathlineto{\pgfqpoint{3.384964in}{0.664203in}}%
\pgfpathlineto{\pgfqpoint{3.385723in}{0.668311in}}%
\pgfpathlineto{\pgfqpoint{3.385975in}{0.667585in}}%
\pgfpathlineto{\pgfqpoint{3.386144in}{0.669041in}}%
\pgfpathlineto{\pgfqpoint{3.386649in}{0.683948in}}%
\pgfpathlineto{\pgfqpoint{3.386986in}{0.669624in}}%
\pgfpathlineto{\pgfqpoint{3.387492in}{0.653993in}}%
\pgfpathlineto{\pgfqpoint{3.388166in}{0.661973in}}%
\pgfpathlineto{\pgfqpoint{3.388587in}{0.655083in}}%
\pgfpathlineto{\pgfqpoint{3.390019in}{0.645467in}}%
\pgfpathlineto{\pgfqpoint{3.390187in}{0.646026in}}%
\pgfpathlineto{\pgfqpoint{3.390609in}{0.666900in}}%
\pgfpathlineto{\pgfqpoint{3.390777in}{0.674329in}}%
\pgfpathlineto{\pgfqpoint{3.391367in}{0.647481in}}%
\pgfpathlineto{\pgfqpoint{3.391535in}{0.646399in}}%
\pgfpathlineto{\pgfqpoint{3.391957in}{0.650624in}}%
\pgfpathlineto{\pgfqpoint{3.393473in}{0.671398in}}%
\pgfpathlineto{\pgfqpoint{3.393641in}{0.662302in}}%
\pgfpathlineto{\pgfqpoint{3.394400in}{0.629405in}}%
\pgfpathlineto{\pgfqpoint{3.394989in}{0.632301in}}%
\pgfpathlineto{\pgfqpoint{3.396759in}{0.646990in}}%
\pgfpathlineto{\pgfqpoint{3.397096in}{0.641067in}}%
\pgfpathlineto{\pgfqpoint{3.397432in}{0.636813in}}%
\pgfpathlineto{\pgfqpoint{3.398106in}{0.641613in}}%
\pgfpathlineto{\pgfqpoint{3.398612in}{0.654896in}}%
\pgfpathlineto{\pgfqpoint{3.399370in}{0.644960in}}%
\pgfpathlineto{\pgfqpoint{3.399539in}{0.645410in}}%
\pgfpathlineto{\pgfqpoint{3.400044in}{0.643826in}}%
\pgfpathlineto{\pgfqpoint{3.400297in}{0.644186in}}%
\pgfpathlineto{\pgfqpoint{3.401055in}{0.648555in}}%
\pgfpathlineto{\pgfqpoint{3.401476in}{0.658652in}}%
\pgfpathlineto{\pgfqpoint{3.402066in}{0.647523in}}%
\pgfpathlineto{\pgfqpoint{3.402319in}{0.646589in}}%
\pgfpathlineto{\pgfqpoint{3.403245in}{0.646821in}}%
\pgfpathlineto{\pgfqpoint{3.403919in}{0.645623in}}%
\pgfpathlineto{\pgfqpoint{3.404256in}{0.647461in}}%
\pgfpathlineto{\pgfqpoint{3.404930in}{0.651190in}}%
\pgfpathlineto{\pgfqpoint{3.405436in}{0.648365in}}%
\pgfpathlineto{\pgfqpoint{3.405604in}{0.647828in}}%
\pgfpathlineto{\pgfqpoint{3.405857in}{0.651190in}}%
\pgfpathlineto{\pgfqpoint{3.406110in}{0.655151in}}%
\pgfpathlineto{\pgfqpoint{3.406699in}{0.647730in}}%
\pgfpathlineto{\pgfqpoint{3.406784in}{0.647835in}}%
\pgfpathlineto{\pgfqpoint{3.407205in}{0.656085in}}%
\pgfpathlineto{\pgfqpoint{3.407795in}{0.661953in}}%
\pgfpathlineto{\pgfqpoint{3.408469in}{0.661219in}}%
\pgfpathlineto{\pgfqpoint{3.408974in}{0.662287in}}%
\pgfpathlineto{\pgfqpoint{3.409311in}{0.689043in}}%
\pgfpathlineto{\pgfqpoint{3.409816in}{0.857646in}}%
\pgfpathlineto{\pgfqpoint{3.410406in}{0.710150in}}%
\pgfpathlineto{\pgfqpoint{3.410743in}{0.686302in}}%
\pgfpathlineto{\pgfqpoint{3.411417in}{0.714651in}}%
\pgfpathlineto{\pgfqpoint{3.411501in}{0.715543in}}%
\pgfpathlineto{\pgfqpoint{3.411838in}{0.708895in}}%
\pgfpathlineto{\pgfqpoint{3.413439in}{0.668424in}}%
\pgfpathlineto{\pgfqpoint{3.412597in}{0.712998in}}%
\pgfpathlineto{\pgfqpoint{3.413776in}{0.674594in}}%
\pgfpathlineto{\pgfqpoint{3.414197in}{0.688085in}}%
\pgfpathlineto{\pgfqpoint{3.414871in}{0.675726in}}%
\pgfpathlineto{\pgfqpoint{3.415292in}{0.676546in}}%
\pgfpathlineto{\pgfqpoint{3.416472in}{0.669746in}}%
\pgfpathlineto{\pgfqpoint{3.418157in}{0.651485in}}%
\pgfpathlineto{\pgfqpoint{3.419673in}{0.650758in}}%
\pgfpathlineto{\pgfqpoint{3.420347in}{0.650016in}}%
\pgfpathlineto{\pgfqpoint{3.421021in}{0.648573in}}%
\pgfpathlineto{\pgfqpoint{3.421611in}{0.649135in}}%
\pgfpathlineto{\pgfqpoint{3.423464in}{0.649594in}}%
\pgfpathlineto{\pgfqpoint{3.425991in}{0.651738in}}%
\pgfpathlineto{\pgfqpoint{3.428350in}{0.655216in}}%
\pgfpathlineto{\pgfqpoint{3.428434in}{0.654900in}}%
\pgfpathlineto{\pgfqpoint{3.428940in}{0.653257in}}%
\pgfpathlineto{\pgfqpoint{3.429445in}{0.655046in}}%
\pgfpathlineto{\pgfqpoint{3.432141in}{0.680288in}}%
\pgfpathlineto{\pgfqpoint{3.432562in}{0.677665in}}%
\pgfpathlineto{\pgfqpoint{3.432731in}{0.678429in}}%
\pgfpathlineto{\pgfqpoint{3.433321in}{0.716297in}}%
\pgfpathlineto{\pgfqpoint{3.433573in}{0.725498in}}%
\pgfpathlineto{\pgfqpoint{3.434247in}{0.705298in}}%
\pgfpathlineto{\pgfqpoint{3.435764in}{0.679788in}}%
\pgfpathlineto{\pgfqpoint{3.436185in}{0.689125in}}%
\pgfpathlineto{\pgfqpoint{3.436438in}{0.699988in}}%
\pgfpathlineto{\pgfqpoint{3.436943in}{0.674005in}}%
\pgfpathlineto{\pgfqpoint{3.437364in}{0.665763in}}%
\pgfpathlineto{\pgfqpoint{3.438123in}{0.670605in}}%
\pgfpathlineto{\pgfqpoint{3.438628in}{0.669648in}}%
\pgfpathlineto{\pgfqpoint{3.438965in}{0.668631in}}%
\pgfpathlineto{\pgfqpoint{3.439471in}{0.670676in}}%
\pgfpathlineto{\pgfqpoint{3.439555in}{0.670795in}}%
\pgfpathlineto{\pgfqpoint{3.439639in}{0.670230in}}%
\pgfpathlineto{\pgfqpoint{3.441914in}{0.650280in}}%
\pgfpathlineto{\pgfqpoint{3.443935in}{0.645017in}}%
\pgfpathlineto{\pgfqpoint{3.442419in}{0.650559in}}%
\pgfpathlineto{\pgfqpoint{3.444272in}{0.646348in}}%
\pgfpathlineto{\pgfqpoint{3.444441in}{0.646982in}}%
\pgfpathlineto{\pgfqpoint{3.444946in}{0.644838in}}%
\pgfpathlineto{\pgfqpoint{3.445452in}{0.643550in}}%
\pgfpathlineto{\pgfqpoint{3.445789in}{0.646055in}}%
\pgfpathlineto{\pgfqpoint{3.446547in}{0.672600in}}%
\pgfpathlineto{\pgfqpoint{3.447221in}{0.654462in}}%
\pgfpathlineto{\pgfqpoint{3.448569in}{0.648214in}}%
\pgfpathlineto{\pgfqpoint{3.448737in}{0.647873in}}%
\pgfpathlineto{\pgfqpoint{3.448906in}{0.649285in}}%
\pgfpathlineto{\pgfqpoint{3.449411in}{0.688354in}}%
\pgfpathlineto{\pgfqpoint{3.450001in}{0.759376in}}%
\pgfpathlineto{\pgfqpoint{3.450675in}{0.727242in}}%
\pgfpathlineto{\pgfqpoint{3.451939in}{0.708061in}}%
\pgfpathlineto{\pgfqpoint{3.452107in}{0.710055in}}%
\pgfpathlineto{\pgfqpoint{3.452360in}{0.711865in}}%
\pgfpathlineto{\pgfqpoint{3.452781in}{0.705621in}}%
\pgfpathlineto{\pgfqpoint{3.453792in}{0.685821in}}%
\pgfpathlineto{\pgfqpoint{3.454213in}{0.696841in}}%
\pgfpathlineto{\pgfqpoint{3.454635in}{0.710777in}}%
\pgfpathlineto{\pgfqpoint{3.455308in}{0.698502in}}%
\pgfpathlineto{\pgfqpoint{3.458763in}{0.659956in}}%
\pgfpathlineto{\pgfqpoint{3.459689in}{0.655938in}}%
\pgfpathlineto{\pgfqpoint{3.459942in}{0.655245in}}%
\pgfpathlineto{\pgfqpoint{3.460700in}{0.656479in}}%
\pgfpathlineto{\pgfqpoint{3.461121in}{0.659167in}}%
\pgfpathlineto{\pgfqpoint{3.461458in}{0.654902in}}%
\pgfpathlineto{\pgfqpoint{3.463059in}{0.641908in}}%
\pgfpathlineto{\pgfqpoint{3.463143in}{0.641925in}}%
\pgfpathlineto{\pgfqpoint{3.463564in}{0.643532in}}%
\pgfpathlineto{\pgfqpoint{3.464070in}{0.650336in}}%
\pgfpathlineto{\pgfqpoint{3.464491in}{0.643181in}}%
\pgfpathlineto{\pgfqpoint{3.466260in}{0.616217in}}%
\pgfpathlineto{\pgfqpoint{3.466597in}{0.617867in}}%
\pgfpathlineto{\pgfqpoint{3.468788in}{0.635747in}}%
\pgfpathlineto{\pgfqpoint{3.471146in}{0.646645in}}%
\pgfpathlineto{\pgfqpoint{3.473084in}{0.652364in}}%
\pgfpathlineto{\pgfqpoint{3.475274in}{0.670343in}}%
\pgfpathlineto{\pgfqpoint{3.475527in}{0.667985in}}%
\pgfpathlineto{\pgfqpoint{3.475948in}{0.661009in}}%
\pgfpathlineto{\pgfqpoint{3.476370in}{0.670989in}}%
\pgfpathlineto{\pgfqpoint{3.477044in}{0.694564in}}%
\pgfpathlineto{\pgfqpoint{3.477718in}{0.682783in}}%
\pgfpathlineto{\pgfqpoint{3.478139in}{0.683241in}}%
\pgfpathlineto{\pgfqpoint{3.478644in}{0.674825in}}%
\pgfpathlineto{\pgfqpoint{3.478981in}{0.670497in}}%
\pgfpathlineto{\pgfqpoint{3.479318in}{0.678801in}}%
\pgfpathlineto{\pgfqpoint{3.480076in}{0.746345in}}%
\pgfpathlineto{\pgfqpoint{3.480835in}{0.707121in}}%
\pgfpathlineto{\pgfqpoint{3.483615in}{0.661996in}}%
\pgfpathlineto{\pgfqpoint{3.484120in}{0.662392in}}%
\pgfpathlineto{\pgfqpoint{3.484373in}{0.661098in}}%
\pgfpathlineto{\pgfqpoint{3.488417in}{0.643786in}}%
\pgfpathlineto{\pgfqpoint{3.488585in}{0.643827in}}%
\pgfpathlineto{\pgfqpoint{3.489765in}{0.645539in}}%
\pgfpathlineto{\pgfqpoint{3.491281in}{0.647255in}}%
\pgfpathlineto{\pgfqpoint{3.494145in}{0.651425in}}%
\pgfpathlineto{\pgfqpoint{3.495577in}{0.651095in}}%
\pgfpathlineto{\pgfqpoint{3.496504in}{0.651819in}}%
\pgfpathlineto{\pgfqpoint{3.498020in}{0.651667in}}%
\pgfpathlineto{\pgfqpoint{3.500042in}{0.650669in}}%
\pgfpathlineto{\pgfqpoint{3.502148in}{0.647439in}}%
\pgfpathlineto{\pgfqpoint{3.503412in}{0.646669in}}%
\pgfpathlineto{\pgfqpoint{3.503581in}{0.647036in}}%
\pgfpathlineto{\pgfqpoint{3.505097in}{0.650889in}}%
\pgfpathlineto{\pgfqpoint{3.505602in}{0.650429in}}%
\pgfpathlineto{\pgfqpoint{3.506361in}{0.648921in}}%
\pgfpathlineto{\pgfqpoint{3.506613in}{0.649806in}}%
\pgfpathlineto{\pgfqpoint{3.507119in}{0.668377in}}%
\pgfpathlineto{\pgfqpoint{3.508467in}{0.680376in}}%
\pgfpathlineto{\pgfqpoint{3.509309in}{0.690455in}}%
\pgfpathlineto{\pgfqpoint{3.509815in}{0.685794in}}%
\pgfpathlineto{\pgfqpoint{3.510910in}{0.667439in}}%
\pgfpathlineto{\pgfqpoint{3.512174in}{0.651419in}}%
\pgfpathlineto{\pgfqpoint{3.512426in}{0.652057in}}%
\pgfpathlineto{\pgfqpoint{3.513437in}{0.656001in}}%
\pgfpathlineto{\pgfqpoint{3.514111in}{0.655414in}}%
\pgfpathlineto{\pgfqpoint{3.515880in}{0.650603in}}%
\pgfpathlineto{\pgfqpoint{3.516133in}{0.650777in}}%
\pgfpathlineto{\pgfqpoint{3.519419in}{0.654557in}}%
\pgfpathlineto{\pgfqpoint{3.520261in}{0.677802in}}%
\pgfpathlineto{\pgfqpoint{3.521356in}{0.669475in}}%
\pgfpathlineto{\pgfqpoint{3.521946in}{0.666103in}}%
\pgfpathlineto{\pgfqpoint{3.522283in}{0.669577in}}%
\pgfpathlineto{\pgfqpoint{3.524052in}{0.724064in}}%
\pgfpathlineto{\pgfqpoint{3.524726in}{0.705723in}}%
\pgfpathlineto{\pgfqpoint{3.525821in}{0.686793in}}%
\pgfpathlineto{\pgfqpoint{3.526327in}{0.695225in}}%
\pgfpathlineto{\pgfqpoint{3.527253in}{0.706887in}}%
\pgfpathlineto{\pgfqpoint{3.527590in}{0.700130in}}%
\pgfpathlineto{\pgfqpoint{3.528685in}{0.678442in}}%
\pgfpathlineto{\pgfqpoint{3.529191in}{0.679087in}}%
\pgfpathlineto{\pgfqpoint{3.529781in}{0.677541in}}%
\pgfpathlineto{\pgfqpoint{3.530792in}{0.666552in}}%
\pgfpathlineto{\pgfqpoint{3.531971in}{0.667412in}}%
\pgfpathlineto{\pgfqpoint{3.533066in}{0.676450in}}%
\pgfpathlineto{\pgfqpoint{3.533824in}{0.692052in}}%
\pgfpathlineto{\pgfqpoint{3.534667in}{0.684725in}}%
\pgfpathlineto{\pgfqpoint{3.535594in}{0.679456in}}%
\pgfpathlineto{\pgfqpoint{3.535931in}{0.682638in}}%
\pgfpathlineto{\pgfqpoint{3.536689in}{0.713260in}}%
\pgfpathlineto{\pgfqpoint{3.537363in}{0.693397in}}%
\pgfpathlineto{\pgfqpoint{3.539048in}{0.658662in}}%
\pgfpathlineto{\pgfqpoint{3.539385in}{0.664836in}}%
\pgfpathlineto{\pgfqpoint{3.539806in}{0.680448in}}%
\pgfpathlineto{\pgfqpoint{3.540143in}{0.658678in}}%
\pgfpathlineto{\pgfqpoint{3.540732in}{0.599342in}}%
\pgfpathlineto{\pgfqpoint{3.541322in}{0.642530in}}%
\pgfpathlineto{\pgfqpoint{3.541575in}{0.654820in}}%
\pgfpathlineto{\pgfqpoint{3.542249in}{0.638012in}}%
\pgfpathlineto{\pgfqpoint{3.542333in}{0.638234in}}%
\pgfpathlineto{\pgfqpoint{3.542417in}{0.638636in}}%
\pgfpathlineto{\pgfqpoint{3.542586in}{0.636725in}}%
\pgfpathlineto{\pgfqpoint{3.543934in}{0.585181in}}%
\pgfpathlineto{\pgfqpoint{3.544523in}{0.598549in}}%
\pgfpathlineto{\pgfqpoint{3.546124in}{0.612783in}}%
\pgfpathlineto{\pgfqpoint{3.546714in}{0.603031in}}%
\pgfpathlineto{\pgfqpoint{3.547304in}{0.611477in}}%
\pgfpathlineto{\pgfqpoint{3.550084in}{0.639819in}}%
\pgfpathlineto{\pgfqpoint{3.552695in}{0.647681in}}%
\pgfpathlineto{\pgfqpoint{3.554886in}{0.665190in}}%
\pgfpathlineto{\pgfqpoint{3.555054in}{0.664448in}}%
\pgfpathlineto{\pgfqpoint{3.555223in}{0.663287in}}%
\pgfpathlineto{\pgfqpoint{3.555559in}{0.667450in}}%
\pgfpathlineto{\pgfqpoint{3.556065in}{0.680994in}}%
\pgfpathlineto{\pgfqpoint{3.556570in}{0.668335in}}%
\pgfpathlineto{\pgfqpoint{3.556739in}{0.666518in}}%
\pgfpathlineto{\pgfqpoint{3.557076in}{0.674889in}}%
\pgfpathlineto{\pgfqpoint{3.557834in}{0.714234in}}%
\pgfpathlineto{\pgfqpoint{3.558508in}{0.698597in}}%
\pgfpathlineto{\pgfqpoint{3.558592in}{0.698057in}}%
\pgfpathlineto{\pgfqpoint{3.558845in}{0.703045in}}%
\pgfpathlineto{\pgfqpoint{3.559098in}{0.707529in}}%
\pgfpathlineto{\pgfqpoint{3.559435in}{0.693471in}}%
\pgfpathlineto{\pgfqpoint{3.559940in}{0.672085in}}%
\pgfpathlineto{\pgfqpoint{3.560530in}{0.690200in}}%
\pgfpathlineto{\pgfqpoint{3.560614in}{0.691253in}}%
\pgfpathlineto{\pgfqpoint{3.560951in}{0.681856in}}%
\pgfpathlineto{\pgfqpoint{3.562131in}{0.671033in}}%
\pgfpathlineto{\pgfqpoint{3.562299in}{0.672746in}}%
\pgfpathlineto{\pgfqpoint{3.562636in}{0.678759in}}%
\pgfpathlineto{\pgfqpoint{3.563057in}{0.667701in}}%
\pgfpathlineto{\pgfqpoint{3.563478in}{0.660617in}}%
\pgfpathlineto{\pgfqpoint{3.563984in}{0.671765in}}%
\pgfpathlineto{\pgfqpoint{3.564237in}{0.676198in}}%
\pgfpathlineto{\pgfqpoint{3.564742in}{0.661661in}}%
\pgfpathlineto{\pgfqpoint{3.566090in}{0.648096in}}%
\pgfpathlineto{\pgfqpoint{3.566343in}{0.651351in}}%
\pgfpathlineto{\pgfqpoint{3.567017in}{0.669910in}}%
\pgfpathlineto{\pgfqpoint{3.567606in}{0.658393in}}%
\pgfpathlineto{\pgfqpoint{3.569881in}{0.608260in}}%
\pgfpathlineto{\pgfqpoint{3.570302in}{0.624896in}}%
\pgfpathlineto{\pgfqpoint{3.570639in}{0.638676in}}%
\pgfpathlineto{\pgfqpoint{3.571060in}{0.604587in}}%
\pgfpathlineto{\pgfqpoint{3.571397in}{0.590688in}}%
\pgfpathlineto{\pgfqpoint{3.572156in}{0.602759in}}%
\pgfpathlineto{\pgfqpoint{3.574346in}{0.635547in}}%
\pgfpathlineto{\pgfqpoint{3.574767in}{0.628633in}}%
\pgfpathlineto{\pgfqpoint{3.575525in}{0.629146in}}%
\pgfpathlineto{\pgfqpoint{3.576031in}{0.625257in}}%
\pgfpathlineto{\pgfqpoint{3.576199in}{0.624413in}}%
\pgfpathlineto{\pgfqpoint{3.576536in}{0.628144in}}%
\pgfpathlineto{\pgfqpoint{3.577042in}{0.638692in}}%
\pgfpathlineto{\pgfqpoint{3.577716in}{0.630423in}}%
\pgfpathlineto{\pgfqpoint{3.578137in}{0.632351in}}%
\pgfpathlineto{\pgfqpoint{3.580664in}{0.641063in}}%
\pgfpathlineto{\pgfqpoint{3.582939in}{0.650359in}}%
\pgfpathlineto{\pgfqpoint{3.583444in}{0.656380in}}%
\pgfpathlineto{\pgfqpoint{3.584287in}{0.654689in}}%
\pgfpathlineto{\pgfqpoint{3.585551in}{0.657976in}}%
\pgfpathlineto{\pgfqpoint{3.587741in}{0.696499in}}%
\pgfpathlineto{\pgfqpoint{3.588162in}{0.691930in}}%
\pgfpathlineto{\pgfqpoint{3.589763in}{0.664399in}}%
\pgfpathlineto{\pgfqpoint{3.590015in}{0.673485in}}%
\pgfpathlineto{\pgfqpoint{3.590437in}{0.687281in}}%
\pgfpathlineto{\pgfqpoint{3.591026in}{0.672655in}}%
\pgfpathlineto{\pgfqpoint{3.592543in}{0.652668in}}%
\pgfpathlineto{\pgfqpoint{3.592880in}{0.660547in}}%
\pgfpathlineto{\pgfqpoint{3.593554in}{0.699182in}}%
\pgfpathlineto{\pgfqpoint{3.594396in}{0.684254in}}%
\pgfpathlineto{\pgfqpoint{3.595070in}{0.695421in}}%
\pgfpathlineto{\pgfqpoint{3.595744in}{0.688562in}}%
\pgfpathlineto{\pgfqpoint{3.596165in}{0.682648in}}%
\pgfpathlineto{\pgfqpoint{3.596671in}{0.688769in}}%
\pgfpathlineto{\pgfqpoint{3.597261in}{0.708872in}}%
\pgfpathlineto{\pgfqpoint{3.597682in}{0.688165in}}%
\pgfpathlineto{\pgfqpoint{3.598019in}{0.675768in}}%
\pgfpathlineto{\pgfqpoint{3.598524in}{0.696278in}}%
\pgfpathlineto{\pgfqpoint{3.599535in}{0.714982in}}%
\pgfpathlineto{\pgfqpoint{3.599704in}{0.705900in}}%
\pgfpathlineto{\pgfqpoint{3.600715in}{0.618825in}}%
\pgfpathlineto{\pgfqpoint{3.601557in}{0.623598in}}%
\pgfpathlineto{\pgfqpoint{3.602147in}{0.609430in}}%
\pgfpathlineto{\pgfqpoint{3.602905in}{0.616991in}}%
\pgfpathlineto{\pgfqpoint{3.605095in}{0.638747in}}%
\pgfpathlineto{\pgfqpoint{3.605264in}{0.638055in}}%
\pgfpathlineto{\pgfqpoint{3.605432in}{0.637635in}}%
\pgfpathlineto{\pgfqpoint{3.605938in}{0.639795in}}%
\pgfpathlineto{\pgfqpoint{3.608971in}{0.648804in}}%
\pgfpathlineto{\pgfqpoint{3.613267in}{0.656967in}}%
\pgfpathlineto{\pgfqpoint{3.613857in}{0.655543in}}%
\pgfpathlineto{\pgfqpoint{3.614615in}{0.656469in}}%
\pgfpathlineto{\pgfqpoint{3.617058in}{0.664539in}}%
\pgfpathlineto{\pgfqpoint{3.618490in}{0.714587in}}%
\pgfpathlineto{\pgfqpoint{3.619417in}{0.698154in}}%
\pgfpathlineto{\pgfqpoint{3.621270in}{0.669934in}}%
\pgfpathlineto{\pgfqpoint{3.621691in}{0.661643in}}%
\pgfpathlineto{\pgfqpoint{3.622450in}{0.666337in}}%
\pgfpathlineto{\pgfqpoint{3.623376in}{0.673124in}}%
\pgfpathlineto{\pgfqpoint{3.624050in}{0.670685in}}%
\pgfpathlineto{\pgfqpoint{3.624808in}{0.658753in}}%
\pgfpathlineto{\pgfqpoint{3.626156in}{0.653135in}}%
\pgfpathlineto{\pgfqpoint{3.626746in}{0.655013in}}%
\pgfpathlineto{\pgfqpoint{3.626999in}{0.653079in}}%
\pgfpathlineto{\pgfqpoint{3.629779in}{0.631344in}}%
\pgfpathlineto{\pgfqpoint{3.630706in}{0.636826in}}%
\pgfpathlineto{\pgfqpoint{3.633907in}{0.648569in}}%
\pgfpathlineto{\pgfqpoint{3.637782in}{0.653534in}}%
\pgfpathlineto{\pgfqpoint{3.638035in}{0.653106in}}%
\pgfpathlineto{\pgfqpoint{3.638119in}{0.653042in}}%
\pgfpathlineto{\pgfqpoint{3.638288in}{0.653631in}}%
\pgfpathlineto{\pgfqpoint{3.638709in}{0.659899in}}%
\pgfpathlineto{\pgfqpoint{3.639299in}{0.652678in}}%
\pgfpathlineto{\pgfqpoint{3.639551in}{0.652391in}}%
\pgfpathlineto{\pgfqpoint{3.640057in}{0.653160in}}%
\pgfpathlineto{\pgfqpoint{3.640731in}{0.659315in}}%
\pgfpathlineto{\pgfqpoint{3.641910in}{0.658311in}}%
\pgfpathlineto{\pgfqpoint{3.642500in}{0.654854in}}%
\pgfpathlineto{\pgfqpoint{3.642837in}{0.657675in}}%
\pgfpathlineto{\pgfqpoint{3.643174in}{0.662168in}}%
\pgfpathlineto{\pgfqpoint{3.643679in}{0.657072in}}%
\pgfpathlineto{\pgfqpoint{3.643932in}{0.657529in}}%
\pgfpathlineto{\pgfqpoint{3.644943in}{0.665146in}}%
\pgfpathlineto{\pgfqpoint{3.645533in}{0.674015in}}%
\pgfpathlineto{\pgfqpoint{3.645954in}{0.665537in}}%
\pgfpathlineto{\pgfqpoint{3.646122in}{0.663404in}}%
\pgfpathlineto{\pgfqpoint{3.646375in}{0.673251in}}%
\pgfpathlineto{\pgfqpoint{3.646881in}{0.704544in}}%
\pgfpathlineto{\pgfqpoint{3.647723in}{0.693688in}}%
\pgfpathlineto{\pgfqpoint{3.647891in}{0.691510in}}%
\pgfpathlineto{\pgfqpoint{3.648397in}{0.697812in}}%
\pgfpathlineto{\pgfqpoint{3.650503in}{0.747322in}}%
\pgfpathlineto{\pgfqpoint{3.649155in}{0.696035in}}%
\pgfpathlineto{\pgfqpoint{3.650924in}{0.737185in}}%
\pgfpathlineto{\pgfqpoint{3.652356in}{0.695212in}}%
\pgfpathlineto{\pgfqpoint{3.652609in}{0.698376in}}%
\pgfpathlineto{\pgfqpoint{3.652946in}{0.704221in}}%
\pgfpathlineto{\pgfqpoint{3.653452in}{0.694407in}}%
\pgfpathlineto{\pgfqpoint{3.655389in}{0.669389in}}%
\pgfpathlineto{\pgfqpoint{3.656569in}{0.654667in}}%
\pgfpathlineto{\pgfqpoint{3.656990in}{0.660691in}}%
\pgfpathlineto{\pgfqpoint{3.657074in}{0.661198in}}%
\pgfpathlineto{\pgfqpoint{3.657327in}{0.657381in}}%
\pgfpathlineto{\pgfqpoint{3.658759in}{0.642680in}}%
\pgfpathlineto{\pgfqpoint{3.658928in}{0.642799in}}%
\pgfpathlineto{\pgfqpoint{3.659433in}{0.643785in}}%
\pgfpathlineto{\pgfqpoint{3.659770in}{0.647397in}}%
\pgfpathlineto{\pgfqpoint{3.660191in}{0.641023in}}%
\pgfpathlineto{\pgfqpoint{3.660360in}{0.639883in}}%
\pgfpathlineto{\pgfqpoint{3.661118in}{0.642048in}}%
\pgfpathlineto{\pgfqpoint{3.665583in}{0.652332in}}%
\pgfpathlineto{\pgfqpoint{3.667099in}{0.652788in}}%
\pgfpathlineto{\pgfqpoint{3.667268in}{0.652469in}}%
\pgfpathlineto{\pgfqpoint{3.667689in}{0.651887in}}%
\pgfpathlineto{\pgfqpoint{3.668110in}{0.652758in}}%
\pgfpathlineto{\pgfqpoint{3.670553in}{0.660496in}}%
\pgfpathlineto{\pgfqpoint{3.671143in}{0.662920in}}%
\pgfpathlineto{\pgfqpoint{3.671311in}{0.664062in}}%
\pgfpathlineto{\pgfqpoint{3.671733in}{0.659898in}}%
\pgfpathlineto{\pgfqpoint{3.672491in}{0.650707in}}%
\pgfpathlineto{\pgfqpoint{3.673081in}{0.654116in}}%
\pgfpathlineto{\pgfqpoint{3.674429in}{0.668207in}}%
\pgfpathlineto{\pgfqpoint{3.675271in}{0.662543in}}%
\pgfpathlineto{\pgfqpoint{3.676282in}{0.652636in}}%
\pgfpathlineto{\pgfqpoint{3.676703in}{0.658297in}}%
\pgfpathlineto{\pgfqpoint{3.679315in}{0.647220in}}%
\pgfpathlineto{\pgfqpoint{3.679483in}{0.647098in}}%
\pgfpathlineto{\pgfqpoint{3.679652in}{0.647898in}}%
\pgfpathlineto{\pgfqpoint{3.679989in}{0.670763in}}%
\pgfpathlineto{\pgfqpoint{3.680326in}{0.695407in}}%
\pgfpathlineto{\pgfqpoint{3.681084in}{0.674214in}}%
\pgfpathlineto{\pgfqpoint{3.681337in}{0.672818in}}%
\pgfpathlineto{\pgfqpoint{3.682263in}{0.648930in}}%
\pgfpathlineto{\pgfqpoint{3.682684in}{0.658422in}}%
\pgfpathlineto{\pgfqpoint{3.683274in}{0.704562in}}%
\pgfpathlineto{\pgfqpoint{3.683780in}{0.660387in}}%
\pgfpathlineto{\pgfqpoint{3.684538in}{0.649012in}}%
\pgfpathlineto{\pgfqpoint{3.685043in}{0.649770in}}%
\pgfpathlineto{\pgfqpoint{3.686054in}{0.654297in}}%
\pgfpathlineto{\pgfqpoint{3.686391in}{0.650285in}}%
\pgfpathlineto{\pgfqpoint{3.687149in}{0.645150in}}%
\pgfpathlineto{\pgfqpoint{3.687655in}{0.646683in}}%
\pgfpathlineto{\pgfqpoint{3.688666in}{0.648327in}}%
\pgfpathlineto{\pgfqpoint{3.689003in}{0.647799in}}%
\pgfpathlineto{\pgfqpoint{3.689424in}{0.647838in}}%
\pgfpathlineto{\pgfqpoint{3.689761in}{0.648441in}}%
\pgfpathlineto{\pgfqpoint{3.690351in}{0.651111in}}%
\pgfpathlineto{\pgfqpoint{3.691867in}{0.704479in}}%
\pgfpathlineto{\pgfqpoint{3.692120in}{0.672235in}}%
\pgfpathlineto{\pgfqpoint{3.692541in}{0.650873in}}%
\pgfpathlineto{\pgfqpoint{3.693299in}{0.651206in}}%
\pgfpathlineto{\pgfqpoint{3.693636in}{0.653674in}}%
\pgfpathlineto{\pgfqpoint{3.694647in}{0.652170in}}%
\pgfpathlineto{\pgfqpoint{3.694984in}{0.650998in}}%
\pgfpathlineto{\pgfqpoint{3.695742in}{0.651617in}}%
\pgfpathlineto{\pgfqpoint{3.696332in}{0.653750in}}%
\pgfpathlineto{\pgfqpoint{3.696838in}{0.651754in}}%
\pgfpathlineto{\pgfqpoint{3.697933in}{0.651996in}}%
\pgfpathlineto{\pgfqpoint{3.698017in}{0.652177in}}%
\pgfpathlineto{\pgfqpoint{3.699449in}{0.655708in}}%
\pgfpathlineto{\pgfqpoint{3.699618in}{0.654741in}}%
\pgfpathlineto{\pgfqpoint{3.700039in}{0.652700in}}%
\pgfpathlineto{\pgfqpoint{3.700629in}{0.654867in}}%
\pgfpathlineto{\pgfqpoint{3.700713in}{0.654639in}}%
\pgfpathlineto{\pgfqpoint{3.701050in}{0.653688in}}%
\pgfpathlineto{\pgfqpoint{3.701471in}{0.655968in}}%
\pgfpathlineto{\pgfqpoint{3.701555in}{0.656115in}}%
\pgfpathlineto{\pgfqpoint{3.701808in}{0.654409in}}%
\pgfpathlineto{\pgfqpoint{3.702061in}{0.653618in}}%
\pgfpathlineto{\pgfqpoint{3.702398in}{0.657266in}}%
\pgfpathlineto{\pgfqpoint{3.702735in}{0.667903in}}%
\pgfpathlineto{\pgfqpoint{3.703409in}{0.655955in}}%
\pgfpathlineto{\pgfqpoint{3.704841in}{0.653522in}}%
\pgfpathlineto{\pgfqpoint{3.703746in}{0.656127in}}%
\pgfpathlineto{\pgfqpoint{3.705262in}{0.653666in}}%
\pgfpathlineto{\pgfqpoint{3.706357in}{0.654849in}}%
\pgfpathlineto{\pgfqpoint{3.706610in}{0.657120in}}%
\pgfpathlineto{\pgfqpoint{3.707115in}{0.652976in}}%
\pgfpathlineto{\pgfqpoint{3.707284in}{0.653330in}}%
\pgfpathlineto{\pgfqpoint{3.707958in}{0.654352in}}%
\pgfpathlineto{\pgfqpoint{3.709053in}{0.694773in}}%
\pgfpathlineto{\pgfqpoint{3.709137in}{0.705486in}}%
\pgfpathlineto{\pgfqpoint{3.709643in}{0.644421in}}%
\pgfpathlineto{\pgfqpoint{3.709727in}{0.644881in}}%
\pgfpathlineto{\pgfqpoint{3.711496in}{0.666458in}}%
\pgfpathlineto{\pgfqpoint{3.711749in}{0.659094in}}%
\pgfpathlineto{\pgfqpoint{3.712339in}{0.651692in}}%
\pgfpathlineto{\pgfqpoint{3.713097in}{0.652846in}}%
\pgfpathlineto{\pgfqpoint{3.713265in}{0.653663in}}%
\pgfpathlineto{\pgfqpoint{3.713686in}{0.649329in}}%
\pgfpathlineto{\pgfqpoint{3.714108in}{0.645262in}}%
\pgfpathlineto{\pgfqpoint{3.714866in}{0.647031in}}%
\pgfpathlineto{\pgfqpoint{3.715371in}{0.651101in}}%
\pgfpathlineto{\pgfqpoint{3.715961in}{0.687539in}}%
\pgfpathlineto{\pgfqpoint{3.716972in}{0.674303in}}%
\pgfpathlineto{\pgfqpoint{3.717477in}{0.646010in}}%
\pgfpathlineto{\pgfqpoint{3.718404in}{0.657590in}}%
\pgfpathlineto{\pgfqpoint{3.718910in}{0.832075in}}%
\pgfpathlineto{\pgfqpoint{3.719668in}{0.703228in}}%
\pgfpathlineto{\pgfqpoint{3.720005in}{0.706378in}}%
\pgfpathlineto{\pgfqpoint{3.721437in}{0.672537in}}%
\pgfpathlineto{\pgfqpoint{3.721942in}{0.657891in}}%
\pgfpathlineto{\pgfqpoint{3.722532in}{0.667574in}}%
\pgfpathlineto{\pgfqpoint{3.723038in}{0.691601in}}%
\pgfpathlineto{\pgfqpoint{3.723459in}{0.665654in}}%
\pgfpathlineto{\pgfqpoint{3.724722in}{0.654242in}}%
\pgfpathlineto{\pgfqpoint{3.724807in}{0.654250in}}%
\pgfpathlineto{\pgfqpoint{3.725144in}{0.655091in}}%
\pgfpathlineto{\pgfqpoint{3.725312in}{0.653516in}}%
\pgfpathlineto{\pgfqpoint{3.725902in}{0.650362in}}%
\pgfpathlineto{\pgfqpoint{3.726492in}{0.652740in}}%
\pgfpathlineto{\pgfqpoint{3.726744in}{0.654400in}}%
\pgfpathlineto{\pgfqpoint{3.727418in}{0.651478in}}%
\pgfpathlineto{\pgfqpoint{3.728598in}{0.650762in}}%
\pgfpathlineto{\pgfqpoint{3.728092in}{0.652069in}}%
\pgfpathlineto{\pgfqpoint{3.728682in}{0.650828in}}%
\pgfpathlineto{\pgfqpoint{3.730451in}{0.651601in}}%
\pgfpathlineto{\pgfqpoint{3.731125in}{0.653151in}}%
\pgfpathlineto{\pgfqpoint{3.731294in}{0.653502in}}%
\pgfpathlineto{\pgfqpoint{3.732136in}{0.652775in}}%
\pgfpathlineto{\pgfqpoint{3.733315in}{0.651756in}}%
\pgfpathlineto{\pgfqpoint{3.732726in}{0.653696in}}%
\pgfpathlineto{\pgfqpoint{3.733568in}{0.652515in}}%
\pgfpathlineto{\pgfqpoint{3.734832in}{0.655058in}}%
\pgfpathlineto{\pgfqpoint{3.736264in}{0.666057in}}%
\pgfpathlineto{\pgfqpoint{3.736432in}{0.662398in}}%
\pgfpathlineto{\pgfqpoint{3.737191in}{0.651394in}}%
\pgfpathlineto{\pgfqpoint{3.737696in}{0.656670in}}%
\pgfpathlineto{\pgfqpoint{3.738286in}{0.761966in}}%
\pgfpathlineto{\pgfqpoint{3.739044in}{0.681430in}}%
\pgfpathlineto{\pgfqpoint{3.739128in}{0.681866in}}%
\pgfpathlineto{\pgfqpoint{3.739297in}{0.678473in}}%
\pgfpathlineto{\pgfqpoint{3.739887in}{0.661674in}}%
\pgfpathlineto{\pgfqpoint{3.740476in}{0.673019in}}%
\pgfpathlineto{\pgfqpoint{3.740729in}{0.682581in}}%
\pgfpathlineto{\pgfqpoint{3.741234in}{0.657763in}}%
\pgfpathlineto{\pgfqpoint{3.741319in}{0.657630in}}%
\pgfpathlineto{\pgfqpoint{3.741403in}{0.658288in}}%
\pgfpathlineto{\pgfqpoint{3.741908in}{0.677639in}}%
\pgfpathlineto{\pgfqpoint{3.742751in}{0.664503in}}%
\pgfpathlineto{\pgfqpoint{3.743256in}{0.655551in}}%
\pgfpathlineto{\pgfqpoint{3.743678in}{0.670153in}}%
\pgfpathlineto{\pgfqpoint{3.744267in}{0.730087in}}%
\pgfpathlineto{\pgfqpoint{3.744773in}{0.677918in}}%
\pgfpathlineto{\pgfqpoint{3.745025in}{0.669219in}}%
\pgfpathlineto{\pgfqpoint{3.745531in}{0.693254in}}%
\pgfpathlineto{\pgfqpoint{3.746036in}{0.743307in}}%
\pgfpathlineto{\pgfqpoint{3.746795in}{0.712733in}}%
\pgfpathlineto{\pgfqpoint{3.748816in}{0.655812in}}%
\pgfpathlineto{\pgfqpoint{3.749153in}{0.657497in}}%
\pgfpathlineto{\pgfqpoint{3.749912in}{0.677254in}}%
\pgfpathlineto{\pgfqpoint{3.750501in}{0.718590in}}%
\pgfpathlineto{\pgfqpoint{3.751260in}{0.697466in}}%
\pgfpathlineto{\pgfqpoint{3.752944in}{0.646157in}}%
\pgfpathlineto{\pgfqpoint{3.753787in}{0.648719in}}%
\pgfpathlineto{\pgfqpoint{3.754208in}{0.647006in}}%
\pgfpathlineto{\pgfqpoint{3.754545in}{0.646282in}}%
\pgfpathlineto{\pgfqpoint{3.755219in}{0.647568in}}%
\pgfpathlineto{\pgfqpoint{3.759431in}{0.661729in}}%
\pgfpathlineto{\pgfqpoint{3.759852in}{0.664335in}}%
\pgfpathlineto{\pgfqpoint{3.760274in}{0.660515in}}%
\pgfpathlineto{\pgfqpoint{3.761032in}{0.650342in}}%
\pgfpathlineto{\pgfqpoint{3.761874in}{0.652270in}}%
\pgfpathlineto{\pgfqpoint{3.762548in}{0.653564in}}%
\pgfpathlineto{\pgfqpoint{3.763475in}{0.667600in}}%
\pgfpathlineto{\pgfqpoint{3.763728in}{0.673317in}}%
\pgfpathlineto{\pgfqpoint{3.764233in}{0.658629in}}%
\pgfpathlineto{\pgfqpoint{3.764654in}{0.655410in}}%
\pgfpathlineto{\pgfqpoint{3.765413in}{0.656166in}}%
\pgfpathlineto{\pgfqpoint{3.765665in}{0.655961in}}%
\pgfpathlineto{\pgfqpoint{3.765834in}{0.656844in}}%
\pgfpathlineto{\pgfqpoint{3.766424in}{0.682722in}}%
\pgfpathlineto{\pgfqpoint{3.768024in}{0.754113in}}%
\pgfpathlineto{\pgfqpoint{3.768108in}{0.753785in}}%
\pgfpathlineto{\pgfqpoint{3.770299in}{0.658808in}}%
\pgfpathlineto{\pgfqpoint{3.770467in}{0.659764in}}%
\pgfpathlineto{\pgfqpoint{3.770973in}{0.672420in}}%
\pgfpathlineto{\pgfqpoint{3.771562in}{0.661030in}}%
\pgfpathlineto{\pgfqpoint{3.773163in}{0.648491in}}%
\pgfpathlineto{\pgfqpoint{3.773669in}{0.647455in}}%
\pgfpathlineto{\pgfqpoint{3.774258in}{0.648600in}}%
\pgfpathlineto{\pgfqpoint{3.775522in}{0.650426in}}%
\pgfpathlineto{\pgfqpoint{3.774848in}{0.648242in}}%
\pgfpathlineto{\pgfqpoint{3.775606in}{0.650287in}}%
\pgfpathlineto{\pgfqpoint{3.776112in}{0.649253in}}%
\pgfpathlineto{\pgfqpoint{3.776701in}{0.650101in}}%
\pgfpathlineto{\pgfqpoint{3.777291in}{0.653966in}}%
\pgfpathlineto{\pgfqpoint{3.778302in}{0.651509in}}%
\pgfpathlineto{\pgfqpoint{3.779060in}{0.650810in}}%
\pgfpathlineto{\pgfqpoint{3.779397in}{0.651277in}}%
\pgfpathlineto{\pgfqpoint{3.779987in}{0.655593in}}%
\pgfpathlineto{\pgfqpoint{3.780577in}{0.651622in}}%
\pgfpathlineto{\pgfqpoint{3.780998in}{0.652861in}}%
\pgfpathlineto{\pgfqpoint{3.782009in}{0.654350in}}%
\pgfpathlineto{\pgfqpoint{3.782177in}{0.653828in}}%
\pgfpathlineto{\pgfqpoint{3.782767in}{0.651564in}}%
\pgfpathlineto{\pgfqpoint{3.783441in}{0.652594in}}%
\pgfpathlineto{\pgfqpoint{3.787063in}{0.652330in}}%
\pgfpathlineto{\pgfqpoint{3.788664in}{0.653494in}}%
\pgfpathlineto{\pgfqpoint{3.788833in}{0.652521in}}%
\pgfpathlineto{\pgfqpoint{3.789507in}{0.650589in}}%
\pgfpathlineto{\pgfqpoint{3.790012in}{0.651320in}}%
\pgfpathlineto{\pgfqpoint{3.790349in}{0.669135in}}%
\pgfpathlineto{\pgfqpoint{3.790517in}{0.677205in}}%
\pgfpathlineto{\pgfqpoint{3.791107in}{0.653124in}}%
\pgfpathlineto{\pgfqpoint{3.791191in}{0.653138in}}%
\pgfpathlineto{\pgfqpoint{3.791528in}{0.653161in}}%
\pgfpathlineto{\pgfqpoint{3.791613in}{0.653615in}}%
\pgfpathlineto{\pgfqpoint{3.794730in}{0.700365in}}%
\pgfpathlineto{\pgfqpoint{3.795067in}{0.753024in}}%
\pgfpathlineto{\pgfqpoint{3.795909in}{0.718711in}}%
\pgfpathlineto{\pgfqpoint{3.796162in}{0.725330in}}%
\pgfpathlineto{\pgfqpoint{3.796583in}{0.708942in}}%
\pgfpathlineto{\pgfqpoint{3.798436in}{0.672320in}}%
\pgfpathlineto{\pgfqpoint{3.798689in}{0.678989in}}%
\pgfpathlineto{\pgfqpoint{3.799026in}{0.699539in}}%
\pgfpathlineto{\pgfqpoint{3.799616in}{0.667845in}}%
\pgfpathlineto{\pgfqpoint{3.800121in}{0.655545in}}%
\pgfpathlineto{\pgfqpoint{3.800711in}{0.663223in}}%
\pgfpathlineto{\pgfqpoint{3.801469in}{0.706648in}}%
\pgfpathlineto{\pgfqpoint{3.802649in}{0.695594in}}%
\pgfpathlineto{\pgfqpoint{3.803660in}{0.652174in}}%
\pgfpathlineto{\pgfqpoint{3.804081in}{0.660126in}}%
\pgfpathlineto{\pgfqpoint{3.804418in}{0.677567in}}%
\pgfpathlineto{\pgfqpoint{3.805008in}{0.653249in}}%
\pgfpathlineto{\pgfqpoint{3.805176in}{0.652436in}}%
\pgfpathlineto{\pgfqpoint{3.805681in}{0.655533in}}%
\pgfpathlineto{\pgfqpoint{3.806018in}{0.656686in}}%
\pgfpathlineto{\pgfqpoint{3.806692in}{0.655724in}}%
\pgfpathlineto{\pgfqpoint{3.808714in}{0.642877in}}%
\pgfpathlineto{\pgfqpoint{3.809136in}{0.644679in}}%
\pgfpathlineto{\pgfqpoint{3.809641in}{0.648346in}}%
\pgfpathlineto{\pgfqpoint{3.810315in}{0.645803in}}%
\pgfpathlineto{\pgfqpoint{3.810568in}{0.645684in}}%
\pgfpathlineto{\pgfqpoint{3.810820in}{0.646518in}}%
\pgfpathlineto{\pgfqpoint{3.812337in}{0.651286in}}%
\pgfpathlineto{\pgfqpoint{3.812505in}{0.650251in}}%
\pgfpathlineto{\pgfqpoint{3.813095in}{0.645143in}}%
\pgfpathlineto{\pgfqpoint{3.813853in}{0.646935in}}%
\pgfpathlineto{\pgfqpoint{3.816802in}{0.652808in}}%
\pgfpathlineto{\pgfqpoint{3.817054in}{0.651916in}}%
\pgfpathlineto{\pgfqpoint{3.817223in}{0.651641in}}%
\pgfpathlineto{\pgfqpoint{3.817560in}{0.652923in}}%
\pgfpathlineto{\pgfqpoint{3.817728in}{0.653839in}}%
\pgfpathlineto{\pgfqpoint{3.818065in}{0.652222in}}%
\pgfpathlineto{\pgfqpoint{3.818571in}{0.652374in}}%
\pgfpathlineto{\pgfqpoint{3.818992in}{0.653420in}}%
\pgfpathlineto{\pgfqpoint{3.819245in}{0.654484in}}%
\pgfpathlineto{\pgfqpoint{3.819835in}{0.652313in}}%
\pgfpathlineto{\pgfqpoint{3.819919in}{0.652348in}}%
\pgfpathlineto{\pgfqpoint{3.820256in}{0.653432in}}%
\pgfpathlineto{\pgfqpoint{3.820761in}{0.651740in}}%
\pgfpathlineto{\pgfqpoint{3.822025in}{0.651120in}}%
\pgfpathlineto{\pgfqpoint{3.822193in}{0.651708in}}%
\pgfpathlineto{\pgfqpoint{3.823794in}{0.660602in}}%
\pgfpathlineto{\pgfqpoint{3.824131in}{0.655262in}}%
\pgfpathlineto{\pgfqpoint{3.824552in}{0.650589in}}%
\pgfpathlineto{\pgfqpoint{3.825310in}{0.652754in}}%
\pgfpathlineto{\pgfqpoint{3.827417in}{0.661520in}}%
\pgfpathlineto{\pgfqpoint{3.827922in}{0.775078in}}%
\pgfpathlineto{\pgfqpoint{3.828428in}{0.653489in}}%
\pgfpathlineto{\pgfqpoint{3.828512in}{0.652936in}}%
\pgfpathlineto{\pgfqpoint{3.829354in}{0.654173in}}%
\pgfpathlineto{\pgfqpoint{3.829775in}{0.653645in}}%
\pgfpathlineto{\pgfqpoint{3.830028in}{0.660225in}}%
\pgfpathlineto{\pgfqpoint{3.830281in}{0.671523in}}%
\pgfpathlineto{\pgfqpoint{3.830702in}{0.653063in}}%
\pgfpathlineto{\pgfqpoint{3.831039in}{0.654023in}}%
\pgfpathlineto{\pgfqpoint{3.831376in}{0.653204in}}%
\pgfpathlineto{\pgfqpoint{3.831629in}{0.654703in}}%
\pgfpathlineto{\pgfqpoint{3.832724in}{0.661858in}}%
\pgfpathlineto{\pgfqpoint{3.832134in}{0.653051in}}%
\pgfpathlineto{\pgfqpoint{3.832892in}{0.658864in}}%
\pgfpathlineto{\pgfqpoint{3.833398in}{0.654596in}}%
\pgfpathlineto{\pgfqpoint{3.833903in}{0.659583in}}%
\pgfpathlineto{\pgfqpoint{3.835251in}{0.680795in}}%
\pgfpathlineto{\pgfqpoint{3.835420in}{0.672700in}}%
\pgfpathlineto{\pgfqpoint{3.835841in}{0.655676in}}%
\pgfpathlineto{\pgfqpoint{3.836262in}{0.679309in}}%
\pgfpathlineto{\pgfqpoint{3.836599in}{0.708815in}}%
\pgfpathlineto{\pgfqpoint{3.837273in}{0.667626in}}%
\pgfpathlineto{\pgfqpoint{3.838031in}{0.659966in}}%
\pgfpathlineto{\pgfqpoint{3.838368in}{0.657483in}}%
\pgfpathlineto{\pgfqpoint{3.838537in}{0.662245in}}%
\pgfpathlineto{\pgfqpoint{3.839127in}{0.774110in}}%
\pgfpathlineto{\pgfqpoint{3.839632in}{0.670085in}}%
\pgfpathlineto{\pgfqpoint{3.839969in}{0.661433in}}%
\pgfpathlineto{\pgfqpoint{3.840474in}{0.679632in}}%
\pgfpathlineto{\pgfqpoint{3.841570in}{0.701300in}}%
\pgfpathlineto{\pgfqpoint{3.841738in}{0.690647in}}%
\pgfpathlineto{\pgfqpoint{3.842496in}{0.651865in}}%
\pgfpathlineto{\pgfqpoint{3.842918in}{0.668899in}}%
\pgfpathlineto{\pgfqpoint{3.843255in}{0.709637in}}%
\pgfpathlineto{\pgfqpoint{3.843844in}{0.652744in}}%
\pgfpathlineto{\pgfqpoint{3.844181in}{0.652481in}}%
\pgfpathlineto{\pgfqpoint{3.844434in}{0.654077in}}%
\pgfpathlineto{\pgfqpoint{3.845698in}{0.669054in}}%
\pgfpathlineto{\pgfqpoint{3.845950in}{0.661771in}}%
\pgfpathlineto{\pgfqpoint{3.846709in}{0.649482in}}%
\pgfpathlineto{\pgfqpoint{3.847298in}{0.650020in}}%
\pgfpathlineto{\pgfqpoint{3.848730in}{0.651404in}}%
\pgfpathlineto{\pgfqpoint{3.848899in}{0.650524in}}%
\pgfpathlineto{\pgfqpoint{3.849152in}{0.649823in}}%
\pgfpathlineto{\pgfqpoint{3.849910in}{0.650465in}}%
\pgfpathlineto{\pgfqpoint{3.850163in}{0.660008in}}%
\pgfpathlineto{\pgfqpoint{3.850500in}{0.684827in}}%
\pgfpathlineto{\pgfqpoint{3.851174in}{0.657494in}}%
\pgfpathlineto{\pgfqpoint{3.851258in}{0.657438in}}%
\pgfpathlineto{\pgfqpoint{3.851342in}{0.657932in}}%
\pgfpathlineto{\pgfqpoint{3.852184in}{0.679416in}}%
\pgfpathlineto{\pgfqpoint{3.853111in}{0.669429in}}%
\pgfpathlineto{\pgfqpoint{3.853617in}{0.658089in}}%
\pgfpathlineto{\pgfqpoint{3.854038in}{0.673331in}}%
\pgfpathlineto{\pgfqpoint{3.854459in}{0.729918in}}%
\pgfpathlineto{\pgfqpoint{3.855049in}{0.671608in}}%
\pgfpathlineto{\pgfqpoint{3.855302in}{0.662599in}}%
\pgfpathlineto{\pgfqpoint{3.855807in}{0.680969in}}%
\pgfpathlineto{\pgfqpoint{3.856228in}{0.711587in}}%
\pgfpathlineto{\pgfqpoint{3.856986in}{0.692545in}}%
\pgfpathlineto{\pgfqpoint{3.859177in}{0.650813in}}%
\pgfpathlineto{\pgfqpoint{3.859766in}{0.658289in}}%
\pgfpathlineto{\pgfqpoint{3.860019in}{0.662451in}}%
\pgfpathlineto{\pgfqpoint{3.860440in}{0.657715in}}%
\pgfpathlineto{\pgfqpoint{3.860946in}{0.662041in}}%
\pgfpathlineto{\pgfqpoint{3.861451in}{0.650916in}}%
\pgfpathlineto{\pgfqpoint{3.861873in}{0.649128in}}%
\pgfpathlineto{\pgfqpoint{3.862631in}{0.649758in}}%
\pgfpathlineto{\pgfqpoint{3.864653in}{0.650104in}}%
\pgfpathlineto{\pgfqpoint{3.865832in}{0.650826in}}%
\pgfpathlineto{\pgfqpoint{3.867348in}{0.656137in}}%
\pgfpathlineto{\pgfqpoint{3.868022in}{0.654317in}}%
\pgfpathlineto{\pgfqpoint{3.868949in}{0.651524in}}%
\pgfpathlineto{\pgfqpoint{3.869286in}{0.653713in}}%
\pgfpathlineto{\pgfqpoint{3.869960in}{0.690415in}}%
\pgfpathlineto{\pgfqpoint{3.870466in}{0.658596in}}%
\pgfpathlineto{\pgfqpoint{3.870803in}{0.654673in}}%
\pgfpathlineto{\pgfqpoint{3.871308in}{0.664208in}}%
\pgfpathlineto{\pgfqpoint{3.871813in}{0.680501in}}%
\pgfpathlineto{\pgfqpoint{3.872572in}{0.675330in}}%
\pgfpathlineto{\pgfqpoint{3.873920in}{0.661053in}}%
\pgfpathlineto{\pgfqpoint{3.874341in}{0.664605in}}%
\pgfpathlineto{\pgfqpoint{3.874678in}{0.668893in}}%
\pgfpathlineto{\pgfqpoint{3.875352in}{0.662640in}}%
\pgfpathlineto{\pgfqpoint{3.878216in}{0.648700in}}%
\pgfpathlineto{\pgfqpoint{3.879648in}{0.650094in}}%
\pgfpathlineto{\pgfqpoint{3.880743in}{0.649293in}}%
\pgfpathlineto{\pgfqpoint{3.881165in}{0.649646in}}%
\pgfpathlineto{\pgfqpoint{3.884871in}{0.655535in}}%
\pgfpathlineto{\pgfqpoint{3.885461in}{0.676079in}}%
\pgfpathlineto{\pgfqpoint{3.886725in}{0.673758in}}%
\pgfpathlineto{\pgfqpoint{3.887651in}{0.682285in}}%
\pgfpathlineto{\pgfqpoint{3.887820in}{0.677977in}}%
\pgfpathlineto{\pgfqpoint{3.888494in}{0.654244in}}%
\pgfpathlineto{\pgfqpoint{3.888915in}{0.670311in}}%
\pgfpathlineto{\pgfqpoint{3.889336in}{0.697490in}}%
\pgfpathlineto{\pgfqpoint{3.889926in}{0.667146in}}%
\pgfpathlineto{\pgfqpoint{3.891105in}{0.663803in}}%
\pgfpathlineto{\pgfqpoint{3.891274in}{0.664327in}}%
\pgfpathlineto{\pgfqpoint{3.891779in}{0.671781in}}%
\pgfpathlineto{\pgfqpoint{3.892116in}{0.664343in}}%
\pgfpathlineto{\pgfqpoint{3.893464in}{0.653092in}}%
\pgfpathlineto{\pgfqpoint{3.894728in}{0.650326in}}%
\pgfpathlineto{\pgfqpoint{3.894981in}{0.650389in}}%
\pgfpathlineto{\pgfqpoint{3.895233in}{0.650366in}}%
\pgfpathlineto{\pgfqpoint{3.895402in}{0.651203in}}%
\pgfpathlineto{\pgfqpoint{3.895739in}{0.653583in}}%
\pgfpathlineto{\pgfqpoint{3.896160in}{0.648702in}}%
\pgfpathlineto{\pgfqpoint{3.896413in}{0.647633in}}%
\pgfpathlineto{\pgfqpoint{3.897340in}{0.648034in}}%
\pgfpathlineto{\pgfqpoint{3.897761in}{0.648087in}}%
\pgfpathlineto{\pgfqpoint{3.897929in}{0.648860in}}%
\pgfpathlineto{\pgfqpoint{3.898098in}{0.649584in}}%
\pgfpathlineto{\pgfqpoint{3.898519in}{0.647714in}}%
\pgfpathlineto{\pgfqpoint{3.898940in}{0.648679in}}%
\pgfpathlineto{\pgfqpoint{3.900204in}{0.649399in}}%
\pgfpathlineto{\pgfqpoint{3.900794in}{0.652380in}}%
\pgfpathlineto{\pgfqpoint{3.901636in}{0.650727in}}%
\pgfpathlineto{\pgfqpoint{3.901889in}{0.650528in}}%
\pgfpathlineto{\pgfqpoint{3.902394in}{0.651453in}}%
\pgfpathlineto{\pgfqpoint{3.902478in}{0.651482in}}%
\pgfpathlineto{\pgfqpoint{3.902900in}{0.652762in}}%
\pgfpathlineto{\pgfqpoint{3.903489in}{0.658195in}}%
\pgfpathlineto{\pgfqpoint{3.904163in}{0.654250in}}%
\pgfpathlineto{\pgfqpoint{3.904500in}{0.653660in}}%
\pgfpathlineto{\pgfqpoint{3.905006in}{0.655027in}}%
\pgfpathlineto{\pgfqpoint{3.905595in}{0.669842in}}%
\pgfpathlineto{\pgfqpoint{3.906775in}{0.714474in}}%
\pgfpathlineto{\pgfqpoint{3.907028in}{0.689400in}}%
\pgfpathlineto{\pgfqpoint{3.907702in}{0.650606in}}%
\pgfpathlineto{\pgfqpoint{3.908376in}{0.657657in}}%
\pgfpathlineto{\pgfqpoint{3.908713in}{0.679311in}}%
\pgfpathlineto{\pgfqpoint{3.909386in}{0.657504in}}%
\pgfpathlineto{\pgfqpoint{3.909471in}{0.657658in}}%
\pgfpathlineto{\pgfqpoint{3.911156in}{0.663490in}}%
\pgfpathlineto{\pgfqpoint{3.911324in}{0.661752in}}%
\pgfpathlineto{\pgfqpoint{3.911998in}{0.650823in}}%
\pgfpathlineto{\pgfqpoint{3.912756in}{0.651670in}}%
\pgfpathlineto{\pgfqpoint{3.914610in}{0.652912in}}%
\pgfpathlineto{\pgfqpoint{3.914694in}{0.652635in}}%
\pgfpathlineto{\pgfqpoint{3.914947in}{0.652113in}}%
\pgfpathlineto{\pgfqpoint{3.915115in}{0.653572in}}%
\pgfpathlineto{\pgfqpoint{3.915452in}{0.666591in}}%
\pgfpathlineto{\pgfqpoint{3.916042in}{0.651793in}}%
\pgfpathlineto{\pgfqpoint{3.916126in}{0.651830in}}%
\pgfpathlineto{\pgfqpoint{3.916716in}{0.652512in}}%
\pgfpathlineto{\pgfqpoint{3.917221in}{0.696508in}}%
\pgfpathlineto{\pgfqpoint{3.918316in}{0.668131in}}%
\pgfpathlineto{\pgfqpoint{3.919159in}{0.651630in}}%
\pgfpathlineto{\pgfqpoint{3.919580in}{0.659277in}}%
\pgfpathlineto{\pgfqpoint{3.920001in}{0.717694in}}%
\pgfpathlineto{\pgfqpoint{3.920675in}{0.659211in}}%
\pgfpathlineto{\pgfqpoint{3.920760in}{0.658641in}}%
\pgfpathlineto{\pgfqpoint{3.921096in}{0.663770in}}%
\pgfpathlineto{\pgfqpoint{3.921686in}{0.723268in}}%
\pgfpathlineto{\pgfqpoint{3.922866in}{0.704434in}}%
\pgfpathlineto{\pgfqpoint{3.924551in}{0.662138in}}%
\pgfpathlineto{\pgfqpoint{3.924719in}{0.662046in}}%
\pgfpathlineto{\pgfqpoint{3.924887in}{0.662874in}}%
\pgfpathlineto{\pgfqpoint{3.926067in}{0.684437in}}%
\pgfpathlineto{\pgfqpoint{3.926235in}{0.694462in}}%
\pgfpathlineto{\pgfqpoint{3.926825in}{0.660896in}}%
\pgfpathlineto{\pgfqpoint{3.927752in}{0.651675in}}%
\pgfpathlineto{\pgfqpoint{3.928257in}{0.651997in}}%
\pgfpathlineto{\pgfqpoint{3.929100in}{0.652792in}}%
\pgfpathlineto{\pgfqpoint{3.929689in}{0.655281in}}%
\pgfpathlineto{\pgfqpoint{3.930026in}{0.658067in}}%
\pgfpathlineto{\pgfqpoint{3.930616in}{0.653631in}}%
\pgfpathlineto{\pgfqpoint{3.932470in}{0.647445in}}%
\pgfpathlineto{\pgfqpoint{3.932806in}{0.647732in}}%
\pgfpathlineto{\pgfqpoint{3.933228in}{0.650257in}}%
\pgfpathlineto{\pgfqpoint{3.933986in}{0.666781in}}%
\pgfpathlineto{\pgfqpoint{3.934576in}{0.654769in}}%
\pgfpathlineto{\pgfqpoint{3.934997in}{0.654727in}}%
\pgfpathlineto{\pgfqpoint{3.935081in}{0.654154in}}%
\pgfpathlineto{\pgfqpoint{3.936934in}{0.646564in}}%
\pgfpathlineto{\pgfqpoint{3.938872in}{0.649307in}}%
\pgfpathlineto{\pgfqpoint{3.939125in}{0.649769in}}%
\pgfpathlineto{\pgfqpoint{3.939799in}{0.648682in}}%
\pgfpathlineto{\pgfqpoint{3.940136in}{0.648548in}}%
\pgfpathlineto{\pgfqpoint{3.940641in}{0.649336in}}%
\pgfpathlineto{\pgfqpoint{3.942916in}{0.654045in}}%
\pgfpathlineto{\pgfqpoint{3.943337in}{0.651795in}}%
\pgfpathlineto{\pgfqpoint{3.943758in}{0.649629in}}%
\pgfpathlineto{\pgfqpoint{3.944516in}{0.651034in}}%
\pgfpathlineto{\pgfqpoint{3.945864in}{0.660605in}}%
\pgfpathlineto{\pgfqpoint{3.946117in}{0.655047in}}%
\pgfpathlineto{\pgfqpoint{3.946454in}{0.650977in}}%
\pgfpathlineto{\pgfqpoint{3.947212in}{0.653350in}}%
\pgfpathlineto{\pgfqpoint{3.948560in}{0.657436in}}%
\pgfpathlineto{\pgfqpoint{3.948897in}{0.660058in}}%
\pgfpathlineto{\pgfqpoint{3.949234in}{0.654263in}}%
\pgfpathlineto{\pgfqpoint{3.949487in}{0.652161in}}%
\pgfpathlineto{\pgfqpoint{3.950414in}{0.652766in}}%
\pgfpathlineto{\pgfqpoint{3.951593in}{0.652174in}}%
\pgfpathlineto{\pgfqpoint{3.952183in}{0.650297in}}%
\pgfpathlineto{\pgfqpoint{3.952435in}{0.652200in}}%
\pgfpathlineto{\pgfqpoint{3.952772in}{0.658124in}}%
\pgfpathlineto{\pgfqpoint{3.953362in}{0.650682in}}%
\pgfpathlineto{\pgfqpoint{3.953699in}{0.649592in}}%
\pgfpathlineto{\pgfqpoint{3.954289in}{0.651973in}}%
\pgfpathlineto{\pgfqpoint{3.954794in}{0.674492in}}%
\pgfpathlineto{\pgfqpoint{3.955637in}{0.663729in}}%
\pgfpathlineto{\pgfqpoint{3.956226in}{0.650887in}}%
\pgfpathlineto{\pgfqpoint{3.956900in}{0.652336in}}%
\pgfpathlineto{\pgfqpoint{3.957322in}{0.663854in}}%
\pgfpathlineto{\pgfqpoint{3.958754in}{0.689539in}}%
\pgfpathlineto{\pgfqpoint{3.957827in}{0.658709in}}%
\pgfpathlineto{\pgfqpoint{3.959175in}{0.684306in}}%
\pgfpathlineto{\pgfqpoint{3.960439in}{0.663058in}}%
\pgfpathlineto{\pgfqpoint{3.961028in}{0.652341in}}%
\pgfpathlineto{\pgfqpoint{3.961450in}{0.660021in}}%
\pgfpathlineto{\pgfqpoint{3.961787in}{0.681529in}}%
\pgfpathlineto{\pgfqpoint{3.962376in}{0.653176in}}%
\pgfpathlineto{\pgfqpoint{3.962713in}{0.652090in}}%
\pgfpathlineto{\pgfqpoint{3.962966in}{0.654827in}}%
\pgfpathlineto{\pgfqpoint{3.963387in}{0.712078in}}%
\pgfpathlineto{\pgfqpoint{3.963724in}{0.839863in}}%
\pgfpathlineto{\pgfqpoint{3.964482in}{0.715372in}}%
\pgfpathlineto{\pgfqpoint{3.964735in}{0.727126in}}%
\pgfpathlineto{\pgfqpoint{3.964988in}{0.709620in}}%
\pgfpathlineto{\pgfqpoint{3.965830in}{0.654121in}}%
\pgfpathlineto{\pgfqpoint{3.966504in}{0.655455in}}%
\pgfpathlineto{\pgfqpoint{3.967094in}{0.665878in}}%
\pgfpathlineto{\pgfqpoint{3.967599in}{0.655267in}}%
\pgfpathlineto{\pgfqpoint{3.967936in}{0.653201in}}%
\pgfpathlineto{\pgfqpoint{3.968442in}{0.656616in}}%
\pgfpathlineto{\pgfqpoint{3.968695in}{0.661034in}}%
\pgfpathlineto{\pgfqpoint{3.969453in}{0.656969in}}%
\pgfpathlineto{\pgfqpoint{3.971222in}{0.651128in}}%
\pgfpathlineto{\pgfqpoint{3.971475in}{0.650699in}}%
\pgfpathlineto{\pgfqpoint{3.972149in}{0.651665in}}%
\pgfpathlineto{\pgfqpoint{3.973412in}{0.652933in}}%
\pgfpathlineto{\pgfqpoint{3.973581in}{0.652391in}}%
\pgfpathlineto{\pgfqpoint{3.974002in}{0.651425in}}%
\pgfpathlineto{\pgfqpoint{3.974760in}{0.651908in}}%
\pgfpathlineto{\pgfqpoint{3.976445in}{0.653229in}}%
\pgfpathlineto{\pgfqpoint{3.978636in}{0.656265in}}%
\pgfpathlineto{\pgfqpoint{3.978804in}{0.657243in}}%
\pgfpathlineto{\pgfqpoint{3.979478in}{0.654854in}}%
\pgfpathlineto{\pgfqpoint{3.979562in}{0.654850in}}%
\pgfpathlineto{\pgfqpoint{3.980152in}{0.655411in}}%
\pgfpathlineto{\pgfqpoint{3.981921in}{0.659794in}}%
\pgfpathlineto{\pgfqpoint{3.982427in}{0.665983in}}%
\pgfpathlineto{\pgfqpoint{3.982679in}{0.660759in}}%
\pgfpathlineto{\pgfqpoint{3.983185in}{0.651366in}}%
\pgfpathlineto{\pgfqpoint{3.983690in}{0.662634in}}%
\pgfpathlineto{\pgfqpoint{3.984196in}{0.976098in}}%
\pgfpathlineto{\pgfqpoint{3.984870in}{0.703851in}}%
\pgfpathlineto{\pgfqpoint{3.985375in}{0.684711in}}%
\pgfpathlineto{\pgfqpoint{3.985881in}{0.703011in}}%
\pgfpathlineto{\pgfqpoint{3.986049in}{0.707020in}}%
\pgfpathlineto{\pgfqpoint{3.986723in}{0.695169in}}%
\pgfpathlineto{\pgfqpoint{3.986976in}{0.691154in}}%
\pgfpathlineto{\pgfqpoint{3.987228in}{0.697305in}}%
\pgfpathlineto{\pgfqpoint{3.987565in}{0.718031in}}%
\pgfpathlineto{\pgfqpoint{3.988071in}{0.689256in}}%
\pgfpathlineto{\pgfqpoint{3.988492in}{0.672094in}}%
\pgfpathlineto{\pgfqpoint{3.989250in}{0.682479in}}%
\pgfpathlineto{\pgfqpoint{3.989419in}{0.683167in}}%
\pgfpathlineto{\pgfqpoint{3.989840in}{0.679147in}}%
\pgfpathlineto{\pgfqpoint{3.989924in}{0.679104in}}%
\pgfpathlineto{\pgfqpoint{3.990009in}{0.679555in}}%
\pgfpathlineto{\pgfqpoint{3.990345in}{0.683273in}}%
\pgfpathlineto{\pgfqpoint{3.990682in}{0.675916in}}%
\pgfpathlineto{\pgfqpoint{3.991609in}{0.658772in}}%
\pgfpathlineto{\pgfqpoint{3.992199in}{0.659120in}}%
\pgfpathlineto{\pgfqpoint{3.992536in}{0.657837in}}%
\pgfpathlineto{\pgfqpoint{3.993378in}{0.659463in}}%
\pgfpathlineto{\pgfqpoint{3.994389in}{0.650492in}}%
\pgfpathlineto{\pgfqpoint{3.996243in}{0.651051in}}%
\pgfpathlineto{\pgfqpoint{3.996748in}{0.654328in}}%
\pgfpathlineto{\pgfqpoint{3.997506in}{0.652229in}}%
\pgfpathlineto{\pgfqpoint{3.998096in}{0.650630in}}%
\pgfpathlineto{\pgfqpoint{3.998770in}{0.651033in}}%
\pgfpathlineto{\pgfqpoint{3.999191in}{0.652303in}}%
\pgfpathlineto{\pgfqpoint{4.000455in}{0.678549in}}%
\pgfpathlineto{\pgfqpoint{4.000960in}{0.717331in}}%
\pgfpathlineto{\pgfqpoint{4.001803in}{0.700767in}}%
\pgfpathlineto{\pgfqpoint{4.002140in}{0.684282in}}%
\pgfpathlineto{\pgfqpoint{4.002982in}{0.653764in}}%
\pgfpathlineto{\pgfqpoint{4.003488in}{0.662836in}}%
\pgfpathlineto{\pgfqpoint{4.004414in}{0.653718in}}%
\pgfpathlineto{\pgfqpoint{4.005341in}{0.658327in}}%
\pgfpathlineto{\pgfqpoint{4.005425in}{0.658212in}}%
\pgfpathlineto{\pgfqpoint{4.005678in}{0.659277in}}%
\pgfpathlineto{\pgfqpoint{4.006183in}{0.662218in}}%
\pgfpathlineto{\pgfqpoint{4.006436in}{0.658252in}}%
\pgfpathlineto{\pgfqpoint{4.007026in}{0.646838in}}%
\pgfpathlineto{\pgfqpoint{4.007784in}{0.648277in}}%
\pgfpathlineto{\pgfqpoint{4.010059in}{0.655922in}}%
\pgfpathlineto{\pgfqpoint{4.010480in}{0.651887in}}%
\pgfpathlineto{\pgfqpoint{4.011744in}{0.651623in}}%
\pgfpathlineto{\pgfqpoint{4.012333in}{0.652383in}}%
\pgfpathlineto{\pgfqpoint{4.013176in}{0.655952in}}%
\pgfpathlineto{\pgfqpoint{4.013681in}{0.654108in}}%
\pgfpathlineto{\pgfqpoint{4.014524in}{0.652647in}}%
\pgfpathlineto{\pgfqpoint{4.015198in}{0.651754in}}%
\pgfpathlineto{\pgfqpoint{4.015703in}{0.652183in}}%
\pgfpathlineto{\pgfqpoint{4.016209in}{0.656368in}}%
\pgfpathlineto{\pgfqpoint{4.016546in}{0.661916in}}%
\pgfpathlineto{\pgfqpoint{4.017135in}{0.654374in}}%
\pgfpathlineto{\pgfqpoint{4.017388in}{0.654064in}}%
\pgfpathlineto{\pgfqpoint{4.018062in}{0.654810in}}%
\pgfpathlineto{\pgfqpoint{4.018904in}{0.659502in}}%
\pgfpathlineto{\pgfqpoint{4.019157in}{0.661993in}}%
\pgfpathlineto{\pgfqpoint{4.019578in}{0.653178in}}%
\pgfpathlineto{\pgfqpoint{4.019831in}{0.651592in}}%
\pgfpathlineto{\pgfqpoint{4.020252in}{0.656323in}}%
\pgfpathlineto{\pgfqpoint{4.020674in}{0.673022in}}%
\pgfpathlineto{\pgfqpoint{4.021347in}{0.656143in}}%
\pgfpathlineto{\pgfqpoint{4.022190in}{0.656944in}}%
\pgfpathlineto{\pgfqpoint{4.023538in}{0.665348in}}%
\pgfpathlineto{\pgfqpoint{4.023791in}{0.660617in}}%
\pgfpathlineto{\pgfqpoint{4.024296in}{0.649867in}}%
\pgfpathlineto{\pgfqpoint{4.024802in}{0.662898in}}%
\pgfpathlineto{\pgfqpoint{4.025138in}{0.674251in}}%
\pgfpathlineto{\pgfqpoint{4.025812in}{0.661608in}}%
\pgfpathlineto{\pgfqpoint{4.027245in}{0.650632in}}%
\pgfpathlineto{\pgfqpoint{4.027582in}{0.646518in}}%
\pgfpathlineto{\pgfqpoint{4.027834in}{0.653927in}}%
\pgfpathlineto{\pgfqpoint{4.028256in}{0.702403in}}%
\pgfpathlineto{\pgfqpoint{4.028761in}{0.640780in}}%
\pgfpathlineto{\pgfqpoint{4.029772in}{0.631308in}}%
\pgfpathlineto{\pgfqpoint{4.029940in}{0.632639in}}%
\pgfpathlineto{\pgfqpoint{4.031036in}{0.657432in}}%
\pgfpathlineto{\pgfqpoint{4.031288in}{0.673971in}}%
\pgfpathlineto{\pgfqpoint{4.031962in}{0.639563in}}%
\pgfpathlineto{\pgfqpoint{4.033816in}{0.648021in}}%
\pgfpathlineto{\pgfqpoint{4.034658in}{0.687309in}}%
\pgfpathlineto{\pgfqpoint{4.035079in}{0.659572in}}%
\pgfpathlineto{\pgfqpoint{4.036343in}{0.642677in}}%
\pgfpathlineto{\pgfqpoint{4.036427in}{0.642808in}}%
\pgfpathlineto{\pgfqpoint{4.037354in}{0.647718in}}%
\pgfpathlineto{\pgfqpoint{4.038028in}{0.682166in}}%
\pgfpathlineto{\pgfqpoint{4.038618in}{0.651744in}}%
\pgfpathlineto{\pgfqpoint{4.039039in}{0.648476in}}%
\pgfpathlineto{\pgfqpoint{4.039460in}{0.653308in}}%
\pgfpathlineto{\pgfqpoint{4.040134in}{0.673298in}}%
\pgfpathlineto{\pgfqpoint{4.040471in}{0.653357in}}%
\pgfpathlineto{\pgfqpoint{4.040808in}{0.644920in}}%
\pgfpathlineto{\pgfqpoint{4.041650in}{0.646772in}}%
\pgfpathlineto{\pgfqpoint{4.043757in}{0.655068in}}%
\pgfpathlineto{\pgfqpoint{4.045441in}{0.717789in}}%
\pgfpathlineto{\pgfqpoint{4.045610in}{0.705793in}}%
\pgfpathlineto{\pgfqpoint{4.046368in}{0.669424in}}%
\pgfpathlineto{\pgfqpoint{4.046874in}{0.691784in}}%
\pgfpathlineto{\pgfqpoint{4.047379in}{0.768361in}}%
\pgfpathlineto{\pgfqpoint{4.048053in}{0.717964in}}%
\pgfpathlineto{\pgfqpoint{4.049738in}{0.658959in}}%
\pgfpathlineto{\pgfqpoint{4.050243in}{0.670182in}}%
\pgfpathlineto{\pgfqpoint{4.050665in}{0.693166in}}%
\pgfpathlineto{\pgfqpoint{4.051339in}{0.669429in}}%
\pgfpathlineto{\pgfqpoint{4.051676in}{0.670890in}}%
\pgfpathlineto{\pgfqpoint{4.051928in}{0.666474in}}%
\pgfpathlineto{\pgfqpoint{4.052518in}{0.653584in}}%
\pgfpathlineto{\pgfqpoint{4.053276in}{0.655446in}}%
\pgfpathlineto{\pgfqpoint{4.053529in}{0.656463in}}%
\pgfpathlineto{\pgfqpoint{4.053866in}{0.653260in}}%
\pgfpathlineto{\pgfqpoint{4.054371in}{0.649349in}}%
\pgfpathlineto{\pgfqpoint{4.055130in}{0.650369in}}%
\pgfpathlineto{\pgfqpoint{4.057910in}{0.652097in}}%
\pgfpathlineto{\pgfqpoint{4.061027in}{0.652426in}}%
\pgfpathlineto{\pgfqpoint{4.064312in}{0.656128in}}%
\pgfpathlineto{\pgfqpoint{4.064986in}{0.654312in}}%
\pgfpathlineto{\pgfqpoint{4.068187in}{0.656471in}}%
\pgfpathlineto{\pgfqpoint{4.068946in}{0.670631in}}%
\pgfpathlineto{\pgfqpoint{4.069788in}{0.661310in}}%
\pgfpathlineto{\pgfqpoint{4.070378in}{0.655483in}}%
\pgfpathlineto{\pgfqpoint{4.070715in}{0.661264in}}%
\pgfpathlineto{\pgfqpoint{4.071304in}{0.827890in}}%
\pgfpathlineto{\pgfqpoint{4.072231in}{0.706275in}}%
\pgfpathlineto{\pgfqpoint{4.074253in}{0.656876in}}%
\pgfpathlineto{\pgfqpoint{4.074422in}{0.657995in}}%
\pgfpathlineto{\pgfqpoint{4.075180in}{0.687744in}}%
\pgfpathlineto{\pgfqpoint{4.075601in}{0.661264in}}%
\pgfpathlineto{\pgfqpoint{4.076780in}{0.652547in}}%
\pgfpathlineto{\pgfqpoint{4.076865in}{0.652580in}}%
\pgfpathlineto{\pgfqpoint{4.077791in}{0.654214in}}%
\pgfpathlineto{\pgfqpoint{4.078044in}{0.652586in}}%
\pgfpathlineto{\pgfqpoint{4.078886in}{0.650747in}}%
\pgfpathlineto{\pgfqpoint{4.079223in}{0.651134in}}%
\pgfpathlineto{\pgfqpoint{4.079560in}{0.652462in}}%
\pgfpathlineto{\pgfqpoint{4.079982in}{0.650925in}}%
\pgfpathlineto{\pgfqpoint{4.080403in}{0.651348in}}%
\pgfpathlineto{\pgfqpoint{4.082425in}{0.653006in}}%
\pgfpathlineto{\pgfqpoint{4.082930in}{0.652511in}}%
\pgfpathlineto{\pgfqpoint{4.086216in}{0.654824in}}%
\pgfpathlineto{\pgfqpoint{4.086553in}{0.669445in}}%
\pgfpathlineto{\pgfqpoint{4.086805in}{0.685709in}}%
\pgfpathlineto{\pgfqpoint{4.087479in}{0.655505in}}%
\pgfpathlineto{\pgfqpoint{4.087648in}{0.654925in}}%
\pgfpathlineto{\pgfqpoint{4.087985in}{0.657100in}}%
\pgfpathlineto{\pgfqpoint{4.088743in}{0.691485in}}%
\pgfpathlineto{\pgfqpoint{4.090091in}{0.682657in}}%
\pgfpathlineto{\pgfqpoint{4.090260in}{0.679532in}}%
\pgfpathlineto{\pgfqpoint{4.090596in}{0.693227in}}%
\pgfpathlineto{\pgfqpoint{4.090933in}{0.717249in}}%
\pgfpathlineto{\pgfqpoint{4.091523in}{0.684481in}}%
\pgfpathlineto{\pgfqpoint{4.091607in}{0.685697in}}%
\pgfpathlineto{\pgfqpoint{4.092113in}{0.714746in}}%
\pgfpathlineto{\pgfqpoint{4.092787in}{0.692225in}}%
\pgfpathlineto{\pgfqpoint{4.095567in}{0.653562in}}%
\pgfpathlineto{\pgfqpoint{4.096325in}{0.653963in}}%
\pgfpathlineto{\pgfqpoint{4.097336in}{0.653989in}}%
\pgfpathlineto{\pgfqpoint{4.097505in}{0.653558in}}%
\pgfpathlineto{\pgfqpoint{4.098094in}{0.651980in}}%
\pgfpathlineto{\pgfqpoint{4.098515in}{0.653448in}}%
\pgfpathlineto{\pgfqpoint{4.098852in}{0.656345in}}%
\pgfpathlineto{\pgfqpoint{4.099442in}{0.651633in}}%
\pgfpathlineto{\pgfqpoint{4.100790in}{0.650887in}}%
\pgfpathlineto{\pgfqpoint{4.100959in}{0.651226in}}%
\pgfpathlineto{\pgfqpoint{4.101548in}{0.657110in}}%
\pgfpathlineto{\pgfqpoint{4.101969in}{0.651540in}}%
\pgfpathlineto{\pgfqpoint{4.102306in}{0.649847in}}%
\pgfpathlineto{\pgfqpoint{4.102728in}{0.653213in}}%
\pgfpathlineto{\pgfqpoint{4.103317in}{0.661738in}}%
\pgfpathlineto{\pgfqpoint{4.104076in}{0.659848in}}%
\pgfpathlineto{\pgfqpoint{4.105760in}{0.647646in}}%
\pgfpathlineto{\pgfqpoint{4.106603in}{0.652327in}}%
\pgfpathlineto{\pgfqpoint{4.106940in}{0.659284in}}%
\pgfpathlineto{\pgfqpoint{4.107530in}{0.649571in}}%
\pgfpathlineto{\pgfqpoint{4.108119in}{0.650068in}}%
\pgfpathlineto{\pgfqpoint{4.108204in}{0.650411in}}%
\pgfpathlineto{\pgfqpoint{4.108541in}{0.662439in}}%
\pgfpathlineto{\pgfqpoint{4.109046in}{0.694131in}}%
\pgfpathlineto{\pgfqpoint{4.109888in}{0.686732in}}%
\pgfpathlineto{\pgfqpoint{4.110141in}{0.685075in}}%
\pgfpathlineto{\pgfqpoint{4.111995in}{0.653283in}}%
\pgfpathlineto{\pgfqpoint{4.112416in}{0.660986in}}%
\pgfpathlineto{\pgfqpoint{4.112921in}{0.723782in}}%
\pgfpathlineto{\pgfqpoint{4.113679in}{0.673824in}}%
\pgfpathlineto{\pgfqpoint{4.113764in}{0.673368in}}%
\pgfpathlineto{\pgfqpoint{4.114016in}{0.676301in}}%
\pgfpathlineto{\pgfqpoint{4.114690in}{0.696248in}}%
\pgfpathlineto{\pgfqpoint{4.115786in}{0.692338in}}%
\pgfpathlineto{\pgfqpoint{4.116207in}{0.690027in}}%
\pgfpathlineto{\pgfqpoint{4.116460in}{0.700054in}}%
\pgfpathlineto{\pgfqpoint{4.116797in}{0.729835in}}%
\pgfpathlineto{\pgfqpoint{4.117218in}{0.681120in}}%
\pgfpathlineto{\pgfqpoint{4.117976in}{0.646532in}}%
\pgfpathlineto{\pgfqpoint{4.118481in}{0.651395in}}%
\pgfpathlineto{\pgfqpoint{4.119408in}{0.699179in}}%
\pgfpathlineto{\pgfqpoint{4.120503in}{0.686664in}}%
\pgfpathlineto{\pgfqpoint{4.121935in}{0.651322in}}%
\pgfpathlineto{\pgfqpoint{4.122441in}{0.668784in}}%
\pgfpathlineto{\pgfqpoint{4.122694in}{0.680188in}}%
\pgfpathlineto{\pgfqpoint{4.123368in}{0.660753in}}%
\pgfpathlineto{\pgfqpoint{4.125474in}{0.637521in}}%
\pgfpathlineto{\pgfqpoint{4.125895in}{0.644630in}}%
\pgfpathlineto{\pgfqpoint{4.126316in}{0.662398in}}%
\pgfpathlineto{\pgfqpoint{4.126822in}{0.637106in}}%
\pgfpathlineto{\pgfqpoint{4.126990in}{0.635011in}}%
\pgfpathlineto{\pgfqpoint{4.127748in}{0.639487in}}%
\pgfpathlineto{\pgfqpoint{4.129770in}{0.644427in}}%
\pgfpathlineto{\pgfqpoint{4.130191in}{0.644468in}}%
\pgfpathlineto{\pgfqpoint{4.130444in}{0.645136in}}%
\pgfpathlineto{\pgfqpoint{4.134319in}{0.658103in}}%
\pgfpathlineto{\pgfqpoint{4.134825in}{0.655546in}}%
\pgfpathlineto{\pgfqpoint{4.136089in}{0.652542in}}%
\pgfpathlineto{\pgfqpoint{4.136257in}{0.653121in}}%
\pgfpathlineto{\pgfqpoint{4.136931in}{0.669574in}}%
\pgfpathlineto{\pgfqpoint{4.138532in}{0.703907in}}%
\pgfpathlineto{\pgfqpoint{4.138869in}{0.697478in}}%
\pgfpathlineto{\pgfqpoint{4.141396in}{0.654459in}}%
\pgfpathlineto{\pgfqpoint{4.142070in}{0.655353in}}%
\pgfpathlineto{\pgfqpoint{4.145355in}{0.651952in}}%
\pgfpathlineto{\pgfqpoint{4.148725in}{0.654452in}}%
\pgfpathlineto{\pgfqpoint{4.148978in}{0.653301in}}%
\pgfpathlineto{\pgfqpoint{4.149905in}{0.652404in}}%
\pgfpathlineto{\pgfqpoint{4.150073in}{0.652710in}}%
\pgfpathlineto{\pgfqpoint{4.150494in}{0.656708in}}%
\pgfpathlineto{\pgfqpoint{4.151253in}{0.653739in}}%
\pgfpathlineto{\pgfqpoint{4.151590in}{0.654017in}}%
\pgfpathlineto{\pgfqpoint{4.152011in}{0.652973in}}%
\pgfpathlineto{\pgfqpoint{4.153106in}{0.653529in}}%
\pgfpathlineto{\pgfqpoint{4.155633in}{0.657565in}}%
\pgfpathlineto{\pgfqpoint{4.156813in}{0.659608in}}%
\pgfpathlineto{\pgfqpoint{4.156897in}{0.659342in}}%
\pgfpathlineto{\pgfqpoint{4.157992in}{0.651458in}}%
\pgfpathlineto{\pgfqpoint{4.158245in}{0.657085in}}%
\pgfpathlineto{\pgfqpoint{4.158666in}{0.682425in}}%
\pgfpathlineto{\pgfqpoint{4.159172in}{0.652353in}}%
\pgfpathlineto{\pgfqpoint{4.160351in}{0.649527in}}%
\pgfpathlineto{\pgfqpoint{4.160688in}{0.649253in}}%
\pgfpathlineto{\pgfqpoint{4.161025in}{0.650011in}}%
\pgfpathlineto{\pgfqpoint{4.162289in}{0.655518in}}%
\pgfpathlineto{\pgfqpoint{4.161783in}{0.648964in}}%
\pgfpathlineto{\pgfqpoint{4.162373in}{0.654754in}}%
\pgfpathlineto{\pgfqpoint{4.163552in}{0.647165in}}%
\pgfpathlineto{\pgfqpoint{4.163215in}{0.656104in}}%
\pgfpathlineto{\pgfqpoint{4.163721in}{0.647600in}}%
\pgfpathlineto{\pgfqpoint{4.164226in}{0.675692in}}%
\pgfpathlineto{\pgfqpoint{4.164900in}{0.651260in}}%
\pgfpathlineto{\pgfqpoint{4.165237in}{0.652132in}}%
\pgfpathlineto{\pgfqpoint{4.165574in}{0.650589in}}%
\pgfpathlineto{\pgfqpoint{4.165827in}{0.649993in}}%
\pgfpathlineto{\pgfqpoint{4.166164in}{0.652386in}}%
\pgfpathlineto{\pgfqpoint{4.166754in}{0.659488in}}%
\pgfpathlineto{\pgfqpoint{4.167091in}{0.651471in}}%
\pgfpathlineto{\pgfqpoint{4.168017in}{0.646323in}}%
\pgfpathlineto{\pgfqpoint{4.168270in}{0.648874in}}%
\pgfpathlineto{\pgfqpoint{4.170039in}{0.703762in}}%
\pgfpathlineto{\pgfqpoint{4.170376in}{0.683340in}}%
\pgfpathlineto{\pgfqpoint{4.171808in}{0.646491in}}%
\pgfpathlineto{\pgfqpoint{4.171892in}{0.646219in}}%
\pgfpathlineto{\pgfqpoint{4.172482in}{0.647706in}}%
\pgfpathlineto{\pgfqpoint{4.174336in}{0.652456in}}%
\pgfpathlineto{\pgfqpoint{4.174841in}{0.665837in}}%
\pgfpathlineto{\pgfqpoint{4.175768in}{0.662441in}}%
\pgfpathlineto{\pgfqpoint{4.176779in}{0.648833in}}%
\pgfpathlineto{\pgfqpoint{4.177116in}{0.650152in}}%
\pgfpathlineto{\pgfqpoint{4.177621in}{0.659593in}}%
\pgfpathlineto{\pgfqpoint{4.178295in}{0.651362in}}%
\pgfpathlineto{\pgfqpoint{4.178548in}{0.650887in}}%
\pgfpathlineto{\pgfqpoint{4.178801in}{0.652867in}}%
\pgfpathlineto{\pgfqpoint{4.179390in}{0.674595in}}%
\pgfpathlineto{\pgfqpoint{4.179811in}{0.652202in}}%
\pgfpathlineto{\pgfqpoint{4.180064in}{0.648442in}}%
\pgfpathlineto{\pgfqpoint{4.180991in}{0.649254in}}%
\pgfpathlineto{\pgfqpoint{4.181244in}{0.650329in}}%
\pgfpathlineto{\pgfqpoint{4.182844in}{0.693149in}}%
\pgfpathlineto{\pgfqpoint{4.183181in}{0.814018in}}%
\pgfpathlineto{\pgfqpoint{4.183939in}{0.704716in}}%
\pgfpathlineto{\pgfqpoint{4.184024in}{0.705262in}}%
\pgfpathlineto{\pgfqpoint{4.184192in}{0.700744in}}%
\pgfpathlineto{\pgfqpoint{4.185119in}{0.655825in}}%
\pgfpathlineto{\pgfqpoint{4.185540in}{0.673164in}}%
\pgfpathlineto{\pgfqpoint{4.185961in}{0.759727in}}%
\pgfpathlineto{\pgfqpoint{4.186551in}{0.668529in}}%
\pgfpathlineto{\pgfqpoint{4.187225in}{0.659743in}}%
\pgfpathlineto{\pgfqpoint{4.187730in}{0.663123in}}%
\pgfpathlineto{\pgfqpoint{4.188404in}{0.663619in}}%
\pgfpathlineto{\pgfqpoint{4.188573in}{0.662415in}}%
\pgfpathlineto{\pgfqpoint{4.189247in}{0.657087in}}%
\pgfpathlineto{\pgfqpoint{4.189500in}{0.661417in}}%
\pgfpathlineto{\pgfqpoint{4.189752in}{0.668079in}}%
\pgfpathlineto{\pgfqpoint{4.190342in}{0.652112in}}%
\pgfpathlineto{\pgfqpoint{4.191016in}{0.651350in}}%
\pgfpathlineto{\pgfqpoint{4.191437in}{0.652187in}}%
\pgfpathlineto{\pgfqpoint{4.191774in}{0.654234in}}%
\pgfpathlineto{\pgfqpoint{4.192701in}{0.653252in}}%
\pgfpathlineto{\pgfqpoint{4.193543in}{0.651867in}}%
\pgfpathlineto{\pgfqpoint{4.193880in}{0.653011in}}%
\pgfpathlineto{\pgfqpoint{4.194217in}{0.655359in}}%
\pgfpathlineto{\pgfqpoint{4.194891in}{0.652264in}}%
\pgfpathlineto{\pgfqpoint{4.195565in}{0.651867in}}%
\pgfpathlineto{\pgfqpoint{4.195986in}{0.652501in}}%
\pgfpathlineto{\pgfqpoint{4.198261in}{0.651632in}}%
\pgfpathlineto{\pgfqpoint{4.199356in}{0.652058in}}%
\pgfpathlineto{\pgfqpoint{4.200957in}{0.653348in}}%
\pgfpathlineto{\pgfqpoint{4.201378in}{0.662410in}}%
\pgfpathlineto{\pgfqpoint{4.202052in}{0.653720in}}%
\pgfpathlineto{\pgfqpoint{4.203063in}{0.652957in}}%
\pgfpathlineto{\pgfqpoint{4.203484in}{0.651962in}}%
\pgfpathlineto{\pgfqpoint{4.204074in}{0.653288in}}%
\pgfpathlineto{\pgfqpoint{4.205759in}{0.655131in}}%
\pgfpathlineto{\pgfqpoint{4.205843in}{0.654800in}}%
\pgfpathlineto{\pgfqpoint{4.207191in}{0.650156in}}%
\pgfpathlineto{\pgfqpoint{4.207359in}{0.650232in}}%
\pgfpathlineto{\pgfqpoint{4.210139in}{0.653182in}}%
\pgfpathlineto{\pgfqpoint{4.211150in}{0.672303in}}%
\pgfpathlineto{\pgfqpoint{4.211656in}{0.781979in}}%
\pgfpathlineto{\pgfqpoint{4.212330in}{0.701583in}}%
\pgfpathlineto{\pgfqpoint{4.213341in}{0.667368in}}%
\pgfpathlineto{\pgfqpoint{4.213678in}{0.685572in}}%
\pgfpathlineto{\pgfqpoint{4.214604in}{0.737034in}}%
\pgfpathlineto{\pgfqpoint{4.214183in}{0.685038in}}%
\pgfpathlineto{\pgfqpoint{4.214857in}{0.712174in}}%
\pgfpathlineto{\pgfqpoint{4.216205in}{0.653947in}}%
\pgfpathlineto{\pgfqpoint{4.216458in}{0.653468in}}%
\pgfpathlineto{\pgfqpoint{4.216795in}{0.655711in}}%
\pgfpathlineto{\pgfqpoint{4.218564in}{0.743878in}}%
\pgfpathlineto{\pgfqpoint{4.218817in}{0.691572in}}%
\pgfpathlineto{\pgfqpoint{4.219491in}{0.652920in}}%
\pgfpathlineto{\pgfqpoint{4.220080in}{0.653217in}}%
\pgfpathlineto{\pgfqpoint{4.220502in}{0.655816in}}%
\pgfpathlineto{\pgfqpoint{4.220754in}{0.658291in}}%
\pgfpathlineto{\pgfqpoint{4.221344in}{0.654547in}}%
\pgfpathlineto{\pgfqpoint{4.221428in}{0.654547in}}%
\pgfpathlineto{\pgfqpoint{4.222523in}{0.655520in}}%
\pgfpathlineto{\pgfqpoint{4.223703in}{0.667241in}}%
\pgfpathlineto{\pgfqpoint{4.224040in}{0.687349in}}%
\pgfpathlineto{\pgfqpoint{4.224545in}{0.654714in}}%
\pgfpathlineto{\pgfqpoint{4.224967in}{0.651005in}}%
\pgfpathlineto{\pgfqpoint{4.225640in}{0.652209in}}%
\pgfpathlineto{\pgfqpoint{4.226062in}{0.694714in}}%
\pgfpathlineto{\pgfqpoint{4.226230in}{0.703857in}}%
\pgfpathlineto{\pgfqpoint{4.226820in}{0.663924in}}%
\pgfpathlineto{\pgfqpoint{4.227073in}{0.667290in}}%
\pgfpathlineto{\pgfqpoint{4.227410in}{0.658825in}}%
\pgfpathlineto{\pgfqpoint{4.227915in}{0.650698in}}%
\pgfpathlineto{\pgfqpoint{4.228589in}{0.653016in}}%
\pgfpathlineto{\pgfqpoint{4.228842in}{0.654810in}}%
\pgfpathlineto{\pgfqpoint{4.229431in}{0.650299in}}%
\pgfpathlineto{\pgfqpoint{4.229600in}{0.650045in}}%
\pgfpathlineto{\pgfqpoint{4.230021in}{0.651705in}}%
\pgfpathlineto{\pgfqpoint{4.231369in}{0.652704in}}%
\pgfpathlineto{\pgfqpoint{4.232127in}{0.653956in}}%
\pgfpathlineto{\pgfqpoint{4.232464in}{0.656715in}}%
\pgfpathlineto{\pgfqpoint{4.232970in}{0.651916in}}%
\pgfpathlineto{\pgfqpoint{4.233138in}{0.651657in}}%
\pgfpathlineto{\pgfqpoint{4.233728in}{0.652799in}}%
\pgfpathlineto{\pgfqpoint{4.235329in}{0.658538in}}%
\pgfpathlineto{\pgfqpoint{4.235581in}{0.668018in}}%
\pgfpathlineto{\pgfqpoint{4.236255in}{0.652009in}}%
\pgfpathlineto{\pgfqpoint{4.236761in}{0.649836in}}%
\pgfpathlineto{\pgfqpoint{4.237519in}{0.650169in}}%
\pgfpathlineto{\pgfqpoint{4.238867in}{0.653111in}}%
\pgfpathlineto{\pgfqpoint{4.240299in}{0.673208in}}%
\pgfpathlineto{\pgfqpoint{4.239625in}{0.651564in}}%
\pgfpathlineto{\pgfqpoint{4.240383in}{0.669654in}}%
\pgfpathlineto{\pgfqpoint{4.241310in}{0.651995in}}%
\pgfpathlineto{\pgfqpoint{4.241647in}{0.655323in}}%
\pgfpathlineto{\pgfqpoint{4.243585in}{0.703596in}}%
\pgfpathlineto{\pgfqpoint{4.243753in}{0.693795in}}%
\pgfpathlineto{\pgfqpoint{4.244427in}{0.648408in}}%
\pgfpathlineto{\pgfqpoint{4.245017in}{0.673233in}}%
\pgfpathlineto{\pgfqpoint{4.245438in}{0.846330in}}%
\pgfpathlineto{\pgfqpoint{4.246112in}{0.685319in}}%
\pgfpathlineto{\pgfqpoint{4.247123in}{0.662127in}}%
\pgfpathlineto{\pgfqpoint{4.247713in}{0.664512in}}%
\pgfpathlineto{\pgfqpoint{4.248386in}{0.696921in}}%
\pgfpathlineto{\pgfqpoint{4.248808in}{0.668036in}}%
\pgfpathlineto{\pgfqpoint{4.249313in}{0.651636in}}%
\pgfpathlineto{\pgfqpoint{4.249987in}{0.654520in}}%
\pgfpathlineto{\pgfqpoint{4.250493in}{0.682815in}}%
\pgfpathlineto{\pgfqpoint{4.251251in}{0.663623in}}%
\pgfpathlineto{\pgfqpoint{4.251841in}{0.665638in}}%
\pgfpathlineto{\pgfqpoint{4.252009in}{0.663726in}}%
\pgfpathlineto{\pgfqpoint{4.252683in}{0.651010in}}%
\pgfpathlineto{\pgfqpoint{4.253525in}{0.652955in}}%
\pgfpathlineto{\pgfqpoint{4.254031in}{0.658779in}}%
\pgfpathlineto{\pgfqpoint{4.254536in}{0.652395in}}%
\pgfpathlineto{\pgfqpoint{4.255800in}{0.650203in}}%
\pgfpathlineto{\pgfqpoint{4.255968in}{0.650290in}}%
\pgfpathlineto{\pgfqpoint{4.258075in}{0.652295in}}%
\pgfpathlineto{\pgfqpoint{4.258412in}{0.651476in}}%
\pgfpathlineto{\pgfqpoint{4.258749in}{0.650814in}}%
\pgfpathlineto{\pgfqpoint{4.259170in}{0.652044in}}%
\pgfpathlineto{\pgfqpoint{4.259507in}{0.651412in}}%
\pgfpathlineto{\pgfqpoint{4.260602in}{0.653133in}}%
\pgfpathlineto{\pgfqpoint{4.260855in}{0.655297in}}%
\pgfpathlineto{\pgfqpoint{4.261360in}{0.652475in}}%
\pgfpathlineto{\pgfqpoint{4.261613in}{0.652603in}}%
\pgfpathlineto{\pgfqpoint{4.263719in}{0.654762in}}%
\pgfpathlineto{\pgfqpoint{4.263803in}{0.654430in}}%
\pgfpathlineto{\pgfqpoint{4.264477in}{0.655145in}}%
\pgfpathlineto{\pgfqpoint{4.265067in}{0.653428in}}%
\pgfpathlineto{\pgfqpoint{4.266162in}{0.655116in}}%
\pgfpathlineto{\pgfqpoint{4.267089in}{0.654452in}}%
\pgfpathlineto{\pgfqpoint{4.267594in}{0.675004in}}%
\pgfpathlineto{\pgfqpoint{4.267931in}{0.729479in}}%
\pgfpathlineto{\pgfqpoint{4.268774in}{0.691153in}}%
\pgfpathlineto{\pgfqpoint{4.269700in}{0.655034in}}%
\pgfpathlineto{\pgfqpoint{4.270122in}{0.671643in}}%
\pgfpathlineto{\pgfqpoint{4.270374in}{0.718328in}}%
\pgfpathlineto{\pgfqpoint{4.271133in}{0.655249in}}%
\pgfpathlineto{\pgfqpoint{4.271891in}{0.652672in}}%
\pgfpathlineto{\pgfqpoint{4.272312in}{0.653256in}}%
\pgfpathlineto{\pgfqpoint{4.273828in}{0.652560in}}%
\pgfpathlineto{\pgfqpoint{4.273913in}{0.652949in}}%
\pgfpathlineto{\pgfqpoint{4.275176in}{0.682987in}}%
\pgfpathlineto{\pgfqpoint{4.275429in}{0.716189in}}%
\pgfpathlineto{\pgfqpoint{4.276103in}{0.658037in}}%
\pgfpathlineto{\pgfqpoint{4.276187in}{0.657780in}}%
\pgfpathlineto{\pgfqpoint{4.276440in}{0.659691in}}%
\pgfpathlineto{\pgfqpoint{4.277030in}{0.668257in}}%
\pgfpathlineto{\pgfqpoint{4.277872in}{0.664610in}}%
\pgfpathlineto{\pgfqpoint{4.278125in}{0.664825in}}%
\pgfpathlineto{\pgfqpoint{4.278293in}{0.663409in}}%
\pgfpathlineto{\pgfqpoint{4.278967in}{0.655986in}}%
\pgfpathlineto{\pgfqpoint{4.279220in}{0.661661in}}%
\pgfpathlineto{\pgfqpoint{4.279641in}{0.713680in}}%
\pgfpathlineto{\pgfqpoint{4.280231in}{0.658680in}}%
\pgfpathlineto{\pgfqpoint{4.280568in}{0.656464in}}%
\pgfpathlineto{\pgfqpoint{4.280989in}{0.664674in}}%
\pgfpathlineto{\pgfqpoint{4.281747in}{0.676623in}}%
\pgfpathlineto{\pgfqpoint{4.282253in}{0.673901in}}%
\pgfpathlineto{\pgfqpoint{4.282927in}{0.660688in}}%
\pgfpathlineto{\pgfqpoint{4.283264in}{0.656327in}}%
\pgfpathlineto{\pgfqpoint{4.283685in}{0.665486in}}%
\pgfpathlineto{\pgfqpoint{4.283853in}{0.669860in}}%
\pgfpathlineto{\pgfqpoint{4.284275in}{0.645077in}}%
\pgfpathlineto{\pgfqpoint{4.284359in}{0.645105in}}%
\pgfpathlineto{\pgfqpoint{4.285370in}{0.651066in}}%
\pgfpathlineto{\pgfqpoint{4.286718in}{0.681032in}}%
\pgfpathlineto{\pgfqpoint{4.287055in}{0.673908in}}%
\pgfpathlineto{\pgfqpoint{4.287139in}{0.673629in}}%
\pgfpathlineto{\pgfqpoint{4.287223in}{0.674873in}}%
\pgfpathlineto{\pgfqpoint{4.287476in}{0.683797in}}%
\pgfpathlineto{\pgfqpoint{4.287729in}{0.665623in}}%
\pgfpathlineto{\pgfqpoint{4.287981in}{0.651590in}}%
\pgfpathlineto{\pgfqpoint{4.288487in}{0.680156in}}%
\pgfpathlineto{\pgfqpoint{4.288740in}{0.671205in}}%
\pgfpathlineto{\pgfqpoint{4.288824in}{0.668485in}}%
\pgfpathlineto{\pgfqpoint{4.289077in}{0.693558in}}%
\pgfpathlineto{\pgfqpoint{4.289329in}{0.728249in}}%
\pgfpathlineto{\pgfqpoint{4.289919in}{0.646871in}}%
\pgfpathlineto{\pgfqpoint{4.290256in}{0.643246in}}%
\pgfpathlineto{\pgfqpoint{4.290677in}{0.653820in}}%
\pgfpathlineto{\pgfqpoint{4.292446in}{0.760753in}}%
\pgfpathlineto{\pgfqpoint{4.291435in}{0.649779in}}%
\pgfpathlineto{\pgfqpoint{4.292699in}{0.721850in}}%
\pgfpathlineto{\pgfqpoint{4.294216in}{0.646690in}}%
\pgfpathlineto{\pgfqpoint{4.293205in}{0.754180in}}%
\pgfpathlineto{\pgfqpoint{4.294300in}{0.646768in}}%
\pgfpathlineto{\pgfqpoint{4.295479in}{0.652649in}}%
\pgfpathlineto{\pgfqpoint{4.294889in}{0.646486in}}%
\pgfpathlineto{\pgfqpoint{4.295816in}{0.648213in}}%
\pgfpathlineto{\pgfqpoint{4.295900in}{0.647705in}}%
\pgfpathlineto{\pgfqpoint{4.296153in}{0.652973in}}%
\pgfpathlineto{\pgfqpoint{4.296406in}{0.674120in}}%
\pgfpathlineto{\pgfqpoint{4.296911in}{0.647965in}}%
\pgfpathlineto{\pgfqpoint{4.297164in}{0.648862in}}%
\pgfpathlineto{\pgfqpoint{4.297417in}{0.651164in}}%
\pgfpathlineto{\pgfqpoint{4.297754in}{0.648632in}}%
\pgfpathlineto{\pgfqpoint{4.298343in}{0.650415in}}%
\pgfpathlineto{\pgfqpoint{4.300281in}{0.652338in}}%
\pgfpathlineto{\pgfqpoint{4.300365in}{0.652288in}}%
\pgfpathlineto{\pgfqpoint{4.300618in}{0.652260in}}%
\pgfpathlineto{\pgfqpoint{4.300787in}{0.652896in}}%
\pgfpathlineto{\pgfqpoint{4.301124in}{0.656676in}}%
\pgfpathlineto{\pgfqpoint{4.302050in}{0.654281in}}%
\pgfpathlineto{\pgfqpoint{4.302471in}{0.651599in}}%
\pgfpathlineto{\pgfqpoint{4.302893in}{0.655764in}}%
\pgfpathlineto{\pgfqpoint{4.303398in}{0.673229in}}%
\pgfpathlineto{\pgfqpoint{4.303904in}{0.656579in}}%
\pgfpathlineto{\pgfqpoint{4.304493in}{0.653671in}}%
\pgfpathlineto{\pgfqpoint{4.304830in}{0.658178in}}%
\pgfpathlineto{\pgfqpoint{4.305841in}{0.684956in}}%
\pgfpathlineto{\pgfqpoint{4.306178in}{0.664532in}}%
\pgfpathlineto{\pgfqpoint{4.306684in}{0.652414in}}%
\pgfpathlineto{\pgfqpoint{4.307442in}{0.654675in}}%
\pgfpathlineto{\pgfqpoint{4.308537in}{0.652280in}}%
\pgfpathlineto{\pgfqpoint{4.308874in}{0.652838in}}%
\pgfpathlineto{\pgfqpoint{4.309464in}{0.654419in}}%
\pgfpathlineto{\pgfqpoint{4.309801in}{0.669353in}}%
\pgfpathlineto{\pgfqpoint{4.310306in}{0.648807in}}%
\pgfpathlineto{\pgfqpoint{4.310475in}{0.648037in}}%
\pgfpathlineto{\pgfqpoint{4.310980in}{0.649245in}}%
\pgfpathlineto{\pgfqpoint{4.311317in}{0.649113in}}%
\pgfpathlineto{\pgfqpoint{4.312665in}{0.651804in}}%
\pgfpathlineto{\pgfqpoint{4.312834in}{0.653074in}}%
\pgfpathlineto{\pgfqpoint{4.313339in}{0.650410in}}%
\pgfpathlineto{\pgfqpoint{4.313676in}{0.651303in}}%
\pgfpathlineto{\pgfqpoint{4.314013in}{0.650663in}}%
\pgfpathlineto{\pgfqpoint{4.314097in}{0.650978in}}%
\pgfpathlineto{\pgfqpoint{4.314350in}{0.673414in}}%
\pgfpathlineto{\pgfqpoint{4.314771in}{0.745197in}}%
\pgfpathlineto{\pgfqpoint{4.315529in}{0.689687in}}%
\pgfpathlineto{\pgfqpoint{4.315951in}{0.688557in}}%
\pgfpathlineto{\pgfqpoint{4.316288in}{0.715931in}}%
\pgfpathlineto{\pgfqpoint{4.316540in}{0.748863in}}%
\pgfpathlineto{\pgfqpoint{4.317046in}{0.666050in}}%
\pgfpathlineto{\pgfqpoint{4.317551in}{0.651888in}}%
\pgfpathlineto{\pgfqpoint{4.318225in}{0.654929in}}%
\pgfpathlineto{\pgfqpoint{4.318731in}{0.675115in}}%
\pgfpathlineto{\pgfqpoint{4.319657in}{0.663878in}}%
\pgfpathlineto{\pgfqpoint{4.320079in}{0.663006in}}%
\pgfpathlineto{\pgfqpoint{4.320163in}{0.663918in}}%
\pgfpathlineto{\pgfqpoint{4.320584in}{0.684928in}}%
\pgfpathlineto{\pgfqpoint{4.321005in}{0.657059in}}%
\pgfpathlineto{\pgfqpoint{4.321426in}{0.651722in}}%
\pgfpathlineto{\pgfqpoint{4.322185in}{0.653972in}}%
\pgfpathlineto{\pgfqpoint{4.322437in}{0.657656in}}%
\pgfpathlineto{\pgfqpoint{4.322859in}{0.653489in}}%
\pgfpathlineto{\pgfqpoint{4.323280in}{0.654738in}}%
\pgfpathlineto{\pgfqpoint{4.323617in}{0.653043in}}%
\pgfpathlineto{\pgfqpoint{4.323785in}{0.652730in}}%
\pgfpathlineto{\pgfqpoint{4.324459in}{0.653401in}}%
\pgfpathlineto{\pgfqpoint{4.324796in}{0.662531in}}%
\pgfpathlineto{\pgfqpoint{4.324965in}{0.669125in}}%
\pgfpathlineto{\pgfqpoint{4.325554in}{0.653215in}}%
\pgfpathlineto{\pgfqpoint{4.325723in}{0.653467in}}%
\pgfpathlineto{\pgfqpoint{4.326902in}{0.655200in}}%
\pgfpathlineto{\pgfqpoint{4.326144in}{0.653066in}}%
\pgfpathlineto{\pgfqpoint{4.327155in}{0.653644in}}%
\pgfpathlineto{\pgfqpoint{4.328082in}{0.653020in}}%
\pgfpathlineto{\pgfqpoint{4.328250in}{0.653686in}}%
\pgfpathlineto{\pgfqpoint{4.328672in}{0.667858in}}%
\pgfpathlineto{\pgfqpoint{4.329345in}{0.653897in}}%
\pgfpathlineto{\pgfqpoint{4.329598in}{0.653834in}}%
\pgfpathlineto{\pgfqpoint{4.329851in}{0.653064in}}%
\pgfpathlineto{\pgfqpoint{4.330441in}{0.652769in}}%
\pgfpathlineto{\pgfqpoint{4.330609in}{0.653495in}}%
\pgfpathlineto{\pgfqpoint{4.330946in}{0.659812in}}%
\pgfpathlineto{\pgfqpoint{4.331536in}{0.651613in}}%
\pgfpathlineto{\pgfqpoint{4.332294in}{0.653330in}}%
\pgfpathlineto{\pgfqpoint{4.332800in}{0.701689in}}%
\pgfpathlineto{\pgfqpoint{4.333558in}{0.662100in}}%
\pgfpathlineto{\pgfqpoint{4.333810in}{0.663314in}}%
\pgfpathlineto{\pgfqpoint{4.334147in}{0.658102in}}%
\pgfpathlineto{\pgfqpoint{4.334484in}{0.653739in}}%
\pgfpathlineto{\pgfqpoint{4.335411in}{0.654842in}}%
\pgfpathlineto{\pgfqpoint{4.335664in}{0.655741in}}%
\pgfpathlineto{\pgfqpoint{4.336001in}{0.653400in}}%
\pgfpathlineto{\pgfqpoint{4.336338in}{0.651619in}}%
\pgfpathlineto{\pgfqpoint{4.336927in}{0.655162in}}%
\pgfpathlineto{\pgfqpoint{4.337433in}{0.748982in}}%
\pgfpathlineto{\pgfqpoint{4.338191in}{0.666061in}}%
\pgfpathlineto{\pgfqpoint{4.338612in}{0.665145in}}%
\pgfpathlineto{\pgfqpoint{4.339876in}{0.652786in}}%
\pgfpathlineto{\pgfqpoint{4.340550in}{0.653268in}}%
\pgfpathlineto{\pgfqpoint{4.341055in}{0.658056in}}%
\pgfpathlineto{\pgfqpoint{4.341561in}{0.662704in}}%
\pgfpathlineto{\pgfqpoint{4.341982in}{0.656165in}}%
\pgfpathlineto{\pgfqpoint{4.342319in}{0.654672in}}%
\pgfpathlineto{\pgfqpoint{4.342993in}{0.657063in}}%
\pgfpathlineto{\pgfqpoint{4.344004in}{0.667830in}}%
\pgfpathlineto{\pgfqpoint{4.343414in}{0.656942in}}%
\pgfpathlineto{\pgfqpoint{4.344173in}{0.660607in}}%
\pgfpathlineto{\pgfqpoint{4.344594in}{0.650217in}}%
\pgfpathlineto{\pgfqpoint{4.345352in}{0.651632in}}%
\pgfpathlineto{\pgfqpoint{4.346700in}{0.653415in}}%
\pgfpathlineto{\pgfqpoint{4.346279in}{0.651345in}}%
\pgfpathlineto{\pgfqpoint{4.346784in}{0.653337in}}%
\pgfpathlineto{\pgfqpoint{4.347290in}{0.651120in}}%
\pgfpathlineto{\pgfqpoint{4.347711in}{0.653034in}}%
\pgfpathlineto{\pgfqpoint{4.348048in}{0.655111in}}%
\pgfpathlineto{\pgfqpoint{4.348637in}{0.652364in}}%
\pgfpathlineto{\pgfqpoint{4.348806in}{0.653041in}}%
\pgfpathlineto{\pgfqpoint{4.349227in}{0.655688in}}%
\pgfpathlineto{\pgfqpoint{4.349564in}{0.651063in}}%
\pgfpathlineto{\pgfqpoint{4.349733in}{0.650869in}}%
\pgfpathlineto{\pgfqpoint{4.349817in}{0.651365in}}%
\pgfpathlineto{\pgfqpoint{4.351081in}{0.687463in}}%
\pgfpathlineto{\pgfqpoint{4.351418in}{0.764174in}}%
\pgfpathlineto{\pgfqpoint{4.352007in}{0.660424in}}%
\pgfpathlineto{\pgfqpoint{4.353271in}{0.655682in}}%
\pgfpathlineto{\pgfqpoint{4.353776in}{0.654954in}}%
\pgfpathlineto{\pgfqpoint{4.354198in}{0.655433in}}%
\pgfpathlineto{\pgfqpoint{4.354450in}{0.662023in}}%
\pgfpathlineto{\pgfqpoint{4.354872in}{0.704137in}}%
\pgfpathlineto{\pgfqpoint{4.355377in}{0.653204in}}%
\pgfpathlineto{\pgfqpoint{4.355546in}{0.650500in}}%
\pgfpathlineto{\pgfqpoint{4.356388in}{0.655714in}}%
\pgfpathlineto{\pgfqpoint{4.356641in}{0.671061in}}%
\pgfpathlineto{\pgfqpoint{4.357146in}{0.655128in}}%
\pgfpathlineto{\pgfqpoint{4.357483in}{0.659723in}}%
\pgfpathlineto{\pgfqpoint{4.357652in}{0.657930in}}%
\pgfpathlineto{\pgfqpoint{4.358157in}{0.650994in}}%
\pgfpathlineto{\pgfqpoint{4.358663in}{0.659773in}}%
\pgfpathlineto{\pgfqpoint{4.358915in}{0.680638in}}%
\pgfpathlineto{\pgfqpoint{4.359589in}{0.651855in}}%
\pgfpathlineto{\pgfqpoint{4.359674in}{0.651894in}}%
\pgfpathlineto{\pgfqpoint{4.360263in}{0.653148in}}%
\pgfpathlineto{\pgfqpoint{4.361358in}{0.655926in}}%
\pgfpathlineto{\pgfqpoint{4.361527in}{0.654892in}}%
\pgfpathlineto{\pgfqpoint{4.361948in}{0.652067in}}%
\pgfpathlineto{\pgfqpoint{4.362369in}{0.655399in}}%
\pgfpathlineto{\pgfqpoint{4.362875in}{0.684281in}}%
\pgfpathlineto{\pgfqpoint{4.363549in}{0.660635in}}%
\pgfpathlineto{\pgfqpoint{4.364307in}{0.657427in}}%
\pgfpathlineto{\pgfqpoint{4.364644in}{0.660299in}}%
\pgfpathlineto{\pgfqpoint{4.365486in}{0.680334in}}%
\pgfpathlineto{\pgfqpoint{4.365823in}{0.661664in}}%
\pgfpathlineto{\pgfqpoint{4.366160in}{0.652010in}}%
\pgfpathlineto{\pgfqpoint{4.367003in}{0.653076in}}%
\pgfpathlineto{\pgfqpoint{4.367087in}{0.653144in}}%
\pgfpathlineto{\pgfqpoint{4.367340in}{0.652213in}}%
\pgfpathlineto{\pgfqpoint{4.369362in}{0.647173in}}%
\pgfpathlineto{\pgfqpoint{4.369530in}{0.646873in}}%
\pgfpathlineto{\pgfqpoint{4.370036in}{0.648388in}}%
\pgfpathlineto{\pgfqpoint{4.370120in}{0.648341in}}%
\pgfpathlineto{\pgfqpoint{4.371215in}{0.649444in}}%
\pgfpathlineto{\pgfqpoint{4.371468in}{0.650942in}}%
\pgfpathlineto{\pgfqpoint{4.371805in}{0.694472in}}%
\pgfpathlineto{\pgfqpoint{4.372900in}{0.756917in}}%
\pgfpathlineto{\pgfqpoint{4.372479in}{0.666276in}}%
\pgfpathlineto{\pgfqpoint{4.372984in}{0.750154in}}%
\pgfpathlineto{\pgfqpoint{4.373658in}{0.655558in}}%
\pgfpathlineto{\pgfqpoint{4.374416in}{0.660950in}}%
\pgfpathlineto{\pgfqpoint{4.375343in}{0.657475in}}%
\pgfpathlineto{\pgfqpoint{4.375511in}{0.658858in}}%
\pgfpathlineto{\pgfqpoint{4.376691in}{0.726453in}}%
\pgfpathlineto{\pgfqpoint{4.377196in}{0.869501in}}%
\pgfpathlineto{\pgfqpoint{4.377702in}{0.700351in}}%
\pgfpathlineto{\pgfqpoint{4.378966in}{0.661632in}}%
\pgfpathlineto{\pgfqpoint{4.379555in}{0.660331in}}%
\pgfpathlineto{\pgfqpoint{4.379724in}{0.662177in}}%
\pgfpathlineto{\pgfqpoint{4.380482in}{0.706668in}}%
\pgfpathlineto{\pgfqpoint{4.381156in}{0.673804in}}%
\pgfpathlineto{\pgfqpoint{4.381577in}{0.665706in}}%
\pgfpathlineto{\pgfqpoint{4.382083in}{0.675941in}}%
\pgfpathlineto{\pgfqpoint{4.382420in}{0.679422in}}%
\pgfpathlineto{\pgfqpoint{4.383262in}{0.678615in}}%
\pgfpathlineto{\pgfqpoint{4.385031in}{0.650882in}}%
\pgfpathlineto{\pgfqpoint{4.385958in}{0.657003in}}%
\pgfpathlineto{\pgfqpoint{4.386126in}{0.657468in}}%
\pgfpathlineto{\pgfqpoint{4.386463in}{0.654731in}}%
\pgfpathlineto{\pgfqpoint{4.388232in}{0.648511in}}%
\pgfpathlineto{\pgfqpoint{4.389496in}{0.648444in}}%
\pgfpathlineto{\pgfqpoint{4.389580in}{0.648550in}}%
\pgfpathlineto{\pgfqpoint{4.393708in}{0.653972in}}%
\pgfpathlineto{\pgfqpoint{4.394045in}{0.653571in}}%
\pgfpathlineto{\pgfqpoint{4.394972in}{0.652224in}}%
\pgfpathlineto{\pgfqpoint{4.395309in}{0.651671in}}%
\pgfpathlineto{\pgfqpoint{4.396067in}{0.652227in}}%
\pgfpathlineto{\pgfqpoint{4.396488in}{0.654926in}}%
\pgfpathlineto{\pgfqpoint{4.396825in}{0.661821in}}%
\pgfpathlineto{\pgfqpoint{4.397415in}{0.652612in}}%
\pgfpathlineto{\pgfqpoint{4.397921in}{0.652469in}}%
\pgfpathlineto{\pgfqpoint{4.398173in}{0.653214in}}%
\pgfpathlineto{\pgfqpoint{4.399353in}{0.670631in}}%
\pgfpathlineto{\pgfqpoint{4.399774in}{0.706879in}}%
\pgfpathlineto{\pgfqpoint{4.400279in}{0.662841in}}%
\pgfpathlineto{\pgfqpoint{4.401627in}{0.653699in}}%
\pgfpathlineto{\pgfqpoint{4.402133in}{0.649996in}}%
\pgfpathlineto{\pgfqpoint{4.402891in}{0.651349in}}%
\pgfpathlineto{\pgfqpoint{4.403228in}{0.663040in}}%
\pgfpathlineto{\pgfqpoint{4.403986in}{0.706872in}}%
\pgfpathlineto{\pgfqpoint{4.404407in}{0.677705in}}%
\pgfpathlineto{\pgfqpoint{4.406176in}{0.655352in}}%
\pgfpathlineto{\pgfqpoint{4.406766in}{0.651467in}}%
\pgfpathlineto{\pgfqpoint{4.407103in}{0.655509in}}%
\pgfpathlineto{\pgfqpoint{4.408620in}{0.673617in}}%
\pgfpathlineto{\pgfqpoint{4.408872in}{0.671942in}}%
\pgfpathlineto{\pgfqpoint{4.409041in}{0.670396in}}%
\pgfpathlineto{\pgfqpoint{4.409294in}{0.676377in}}%
\pgfpathlineto{\pgfqpoint{4.409631in}{0.698740in}}%
\pgfpathlineto{\pgfqpoint{4.410304in}{0.670879in}}%
\pgfpathlineto{\pgfqpoint{4.411315in}{0.648518in}}%
\pgfpathlineto{\pgfqpoint{4.411737in}{0.650921in}}%
\pgfpathlineto{\pgfqpoint{4.412916in}{0.672093in}}%
\pgfpathlineto{\pgfqpoint{4.413253in}{0.657991in}}%
\pgfpathlineto{\pgfqpoint{4.413674in}{0.645068in}}%
\pgfpathlineto{\pgfqpoint{4.414517in}{0.646932in}}%
\pgfpathlineto{\pgfqpoint{4.415780in}{0.649855in}}%
\pgfpathlineto{\pgfqpoint{4.415022in}{0.646412in}}%
\pgfpathlineto{\pgfqpoint{4.416454in}{0.648225in}}%
\pgfpathlineto{\pgfqpoint{4.416707in}{0.648853in}}%
\pgfpathlineto{\pgfqpoint{4.417297in}{0.655110in}}%
\pgfpathlineto{\pgfqpoint{4.418139in}{0.652561in}}%
\pgfpathlineto{\pgfqpoint{4.418223in}{0.652680in}}%
\pgfpathlineto{\pgfqpoint{4.418392in}{0.651169in}}%
\pgfpathlineto{\pgfqpoint{4.419824in}{0.646040in}}%
\pgfpathlineto{\pgfqpoint{4.420414in}{0.647524in}}%
\pgfpathlineto{\pgfqpoint{4.420751in}{0.659924in}}%
\pgfpathlineto{\pgfqpoint{4.421004in}{0.669550in}}%
\pgfpathlineto{\pgfqpoint{4.421846in}{0.659290in}}%
\pgfpathlineto{\pgfqpoint{4.423110in}{0.655526in}}%
\pgfpathlineto{\pgfqpoint{4.423447in}{0.654618in}}%
\pgfpathlineto{\pgfqpoint{4.423784in}{0.650210in}}%
\pgfpathlineto{\pgfqpoint{4.424205in}{0.657915in}}%
\pgfpathlineto{\pgfqpoint{4.424458in}{0.655527in}}%
\pgfpathlineto{\pgfqpoint{4.424626in}{0.657903in}}%
\pgfpathlineto{\pgfqpoint{4.424879in}{0.666824in}}%
\pgfpathlineto{\pgfqpoint{4.425468in}{0.647231in}}%
\pgfpathlineto{\pgfqpoint{4.425553in}{0.647163in}}%
\pgfpathlineto{\pgfqpoint{4.425890in}{0.647931in}}%
\pgfpathlineto{\pgfqpoint{4.426058in}{0.648075in}}%
\pgfpathlineto{\pgfqpoint{4.426564in}{0.649123in}}%
\pgfpathlineto{\pgfqpoint{4.426816in}{0.647779in}}%
\pgfpathlineto{\pgfqpoint{4.426985in}{0.647131in}}%
\pgfpathlineto{\pgfqpoint{4.427322in}{0.650561in}}%
\pgfpathlineto{\pgfqpoint{4.427827in}{0.661998in}}%
\pgfpathlineto{\pgfqpoint{4.428501in}{0.652446in}}%
\pgfpathlineto{\pgfqpoint{4.428923in}{0.656883in}}%
\pgfpathlineto{\pgfqpoint{4.429259in}{0.669516in}}%
\pgfpathlineto{\pgfqpoint{4.430186in}{0.664025in}}%
\pgfpathlineto{\pgfqpoint{4.430439in}{0.668813in}}%
\pgfpathlineto{\pgfqpoint{4.432124in}{0.768821in}}%
\pgfpathlineto{\pgfqpoint{4.432377in}{0.720453in}}%
\pgfpathlineto{\pgfqpoint{4.433135in}{0.722854in}}%
\pgfpathlineto{\pgfqpoint{4.433809in}{0.649021in}}%
\pgfpathlineto{\pgfqpoint{4.433893in}{0.648848in}}%
\pgfpathlineto{\pgfqpoint{4.434061in}{0.650734in}}%
\pgfpathlineto{\pgfqpoint{4.434651in}{0.688378in}}%
\pgfpathlineto{\pgfqpoint{4.435409in}{0.815413in}}%
\pgfpathlineto{\pgfqpoint{4.435999in}{0.728695in}}%
\pgfpathlineto{\pgfqpoint{4.437515in}{0.677631in}}%
\pgfpathlineto{\pgfqpoint{4.437684in}{0.676648in}}%
\pgfpathlineto{\pgfqpoint{4.438189in}{0.661258in}}%
\pgfpathlineto{\pgfqpoint{4.438611in}{0.677533in}}%
\pgfpathlineto{\pgfqpoint{4.438948in}{0.687093in}}%
\pgfpathlineto{\pgfqpoint{4.439537in}{0.668566in}}%
\pgfpathlineto{\pgfqpoint{4.439790in}{0.669415in}}%
\pgfpathlineto{\pgfqpoint{4.440548in}{0.677746in}}%
\pgfpathlineto{\pgfqpoint{4.441475in}{0.675198in}}%
\pgfpathlineto{\pgfqpoint{4.443328in}{0.647495in}}%
\pgfpathlineto{\pgfqpoint{4.443834in}{0.648583in}}%
\pgfpathlineto{\pgfqpoint{4.444087in}{0.648985in}}%
\pgfpathlineto{\pgfqpoint{4.444845in}{0.648104in}}%
\pgfpathlineto{\pgfqpoint{4.446867in}{0.647785in}}%
\pgfpathlineto{\pgfqpoint{4.448299in}{0.648644in}}%
\pgfpathlineto{\pgfqpoint{4.449562in}{0.657095in}}%
\pgfpathlineto{\pgfqpoint{4.450321in}{0.653201in}}%
\pgfpathlineto{\pgfqpoint{4.451921in}{0.649351in}}%
\pgfpathlineto{\pgfqpoint{4.452764in}{0.650504in}}%
\pgfpathlineto{\pgfqpoint{4.453016in}{0.654723in}}%
\pgfpathlineto{\pgfqpoint{4.453522in}{0.684452in}}%
\pgfpathlineto{\pgfqpoint{4.454027in}{0.655941in}}%
\pgfpathlineto{\pgfqpoint{4.455375in}{0.648819in}}%
\pgfpathlineto{\pgfqpoint{4.455797in}{0.648067in}}%
\pgfpathlineto{\pgfqpoint{4.456218in}{0.649760in}}%
\pgfpathlineto{\pgfqpoint{4.456807in}{0.654856in}}%
\pgfpathlineto{\pgfqpoint{4.457650in}{0.652802in}}%
\pgfpathlineto{\pgfqpoint{4.458071in}{0.652095in}}%
\pgfpathlineto{\pgfqpoint{4.458324in}{0.658673in}}%
\pgfpathlineto{\pgfqpoint{4.458829in}{0.733949in}}%
\pgfpathlineto{\pgfqpoint{4.459335in}{0.653190in}}%
\pgfpathlineto{\pgfqpoint{4.459672in}{0.644166in}}%
\pgfpathlineto{\pgfqpoint{4.460514in}{0.646265in}}%
\pgfpathlineto{\pgfqpoint{4.462283in}{0.658035in}}%
\pgfpathlineto{\pgfqpoint{4.462705in}{0.664847in}}%
\pgfpathlineto{\pgfqpoint{4.463042in}{0.652556in}}%
\pgfpathlineto{\pgfqpoint{4.463379in}{0.646795in}}%
\pgfpathlineto{\pgfqpoint{4.464221in}{0.649802in}}%
\pgfpathlineto{\pgfqpoint{4.464726in}{0.646847in}}%
\pgfpathlineto{\pgfqpoint{4.465063in}{0.651208in}}%
\pgfpathlineto{\pgfqpoint{4.465316in}{0.654274in}}%
\pgfpathlineto{\pgfqpoint{4.465990in}{0.651028in}}%
\pgfpathlineto{\pgfqpoint{4.466159in}{0.651418in}}%
\pgfpathlineto{\pgfqpoint{4.467675in}{0.663662in}}%
\pgfpathlineto{\pgfqpoint{4.467928in}{0.657278in}}%
\pgfpathlineto{\pgfqpoint{4.468180in}{0.652423in}}%
\pgfpathlineto{\pgfqpoint{4.468854in}{0.662413in}}%
\pgfpathlineto{\pgfqpoint{4.469023in}{0.665316in}}%
\pgfpathlineto{\pgfqpoint{4.469613in}{0.654366in}}%
\pgfpathlineto{\pgfqpoint{4.470792in}{0.648644in}}%
\pgfpathlineto{\pgfqpoint{4.471382in}{0.649538in}}%
\pgfpathlineto{\pgfqpoint{4.472393in}{0.669444in}}%
\pgfpathlineto{\pgfqpoint{4.472645in}{0.680155in}}%
\pgfpathlineto{\pgfqpoint{4.473067in}{0.643946in}}%
\pgfpathlineto{\pgfqpoint{4.473151in}{0.642914in}}%
\pgfpathlineto{\pgfqpoint{4.473404in}{0.653081in}}%
\pgfpathlineto{\pgfqpoint{4.473825in}{0.735201in}}%
\pgfpathlineto{\pgfqpoint{4.474415in}{0.647387in}}%
\pgfpathlineto{\pgfqpoint{4.475257in}{0.646178in}}%
\pgfpathlineto{\pgfqpoint{4.474752in}{0.648594in}}%
\pgfpathlineto{\pgfqpoint{4.475510in}{0.646931in}}%
\pgfpathlineto{\pgfqpoint{4.475762in}{0.647819in}}%
\pgfpathlineto{\pgfqpoint{4.476184in}{0.645653in}}%
\pgfpathlineto{\pgfqpoint{4.476436in}{0.646400in}}%
\pgfpathlineto{\pgfqpoint{4.476942in}{0.644855in}}%
\pgfpathlineto{\pgfqpoint{4.477110in}{0.646148in}}%
\pgfpathlineto{\pgfqpoint{4.477363in}{0.660418in}}%
\pgfpathlineto{\pgfqpoint{4.477869in}{0.833107in}}%
\pgfpathlineto{\pgfqpoint{4.478458in}{0.678861in}}%
\pgfpathlineto{\pgfqpoint{4.479469in}{0.657412in}}%
\pgfpathlineto{\pgfqpoint{4.479722in}{0.660677in}}%
\pgfpathlineto{\pgfqpoint{4.480649in}{0.688616in}}%
\pgfpathlineto{\pgfqpoint{4.481323in}{0.773693in}}%
\pgfpathlineto{\pgfqpoint{4.481828in}{0.712300in}}%
\pgfpathlineto{\pgfqpoint{4.483260in}{0.654543in}}%
\pgfpathlineto{\pgfqpoint{4.483766in}{0.653699in}}%
\pgfpathlineto{\pgfqpoint{4.484018in}{0.655369in}}%
\pgfpathlineto{\pgfqpoint{4.485619in}{0.697864in}}%
\pgfpathlineto{\pgfqpoint{4.486209in}{0.676584in}}%
\pgfpathlineto{\pgfqpoint{4.487809in}{0.656004in}}%
\pgfpathlineto{\pgfqpoint{4.488062in}{0.660005in}}%
\pgfpathlineto{\pgfqpoint{4.488568in}{0.697915in}}%
\pgfpathlineto{\pgfqpoint{4.489073in}{0.654582in}}%
\pgfpathlineto{\pgfqpoint{4.489410in}{0.649385in}}%
\pgfpathlineto{\pgfqpoint{4.490253in}{0.649768in}}%
\pgfpathlineto{\pgfqpoint{4.494044in}{0.647345in}}%
\pgfpathlineto{\pgfqpoint{4.495307in}{0.650026in}}%
\pgfpathlineto{\pgfqpoint{4.497245in}{0.652245in}}%
\pgfpathlineto{\pgfqpoint{4.498424in}{0.653679in}}%
\pgfpathlineto{\pgfqpoint{4.498930in}{0.671030in}}%
\pgfpathlineto{\pgfqpoint{4.499435in}{0.654076in}}%
\pgfpathlineto{\pgfqpoint{4.499772in}{0.652317in}}%
\pgfpathlineto{\pgfqpoint{4.500278in}{0.654982in}}%
\pgfpathlineto{\pgfqpoint{4.501120in}{0.670365in}}%
\pgfpathlineto{\pgfqpoint{4.501963in}{0.667474in}}%
\pgfpathlineto{\pgfqpoint{4.503226in}{0.654993in}}%
\pgfpathlineto{\pgfqpoint{4.503647in}{0.660853in}}%
\pgfpathlineto{\pgfqpoint{4.503816in}{0.663270in}}%
\pgfpathlineto{\pgfqpoint{4.504237in}{0.651774in}}%
\pgfpathlineto{\pgfqpoint{4.504743in}{0.650175in}}%
\pgfpathlineto{\pgfqpoint{4.505417in}{0.650653in}}%
\pgfpathlineto{\pgfqpoint{4.505754in}{0.652099in}}%
\pgfpathlineto{\pgfqpoint{4.507270in}{0.683056in}}%
\pgfpathlineto{\pgfqpoint{4.507523in}{0.666876in}}%
\pgfpathlineto{\pgfqpoint{4.508449in}{0.648779in}}%
\pgfpathlineto{\pgfqpoint{4.508786in}{0.648958in}}%
\pgfpathlineto{\pgfqpoint{4.509966in}{0.651037in}}%
\pgfpathlineto{\pgfqpoint{4.510640in}{0.664878in}}%
\pgfpathlineto{\pgfqpoint{4.511229in}{0.715121in}}%
\pgfpathlineto{\pgfqpoint{4.512325in}{0.702349in}}%
\pgfpathlineto{\pgfqpoint{4.512493in}{0.699390in}}%
\pgfpathlineto{\pgfqpoint{4.512830in}{0.710440in}}%
\pgfpathlineto{\pgfqpoint{4.513083in}{0.718181in}}%
\pgfpathlineto{\pgfqpoint{4.513420in}{0.685655in}}%
\pgfpathlineto{\pgfqpoint{4.514009in}{0.651169in}}%
\pgfpathlineto{\pgfqpoint{4.514683in}{0.655193in}}%
\pgfpathlineto{\pgfqpoint{4.514936in}{0.654161in}}%
\pgfpathlineto{\pgfqpoint{4.515526in}{0.651340in}}%
\pgfpathlineto{\pgfqpoint{4.516368in}{0.651481in}}%
\pgfpathlineto{\pgfqpoint{4.518137in}{0.649867in}}%
\pgfpathlineto{\pgfqpoint{4.518474in}{0.650214in}}%
\pgfpathlineto{\pgfqpoint{4.522013in}{0.653236in}}%
\pgfpathlineto{\pgfqpoint{4.523529in}{0.653242in}}%
\pgfpathlineto{\pgfqpoint{4.527152in}{0.654080in}}%
\pgfpathlineto{\pgfqpoint{4.528247in}{0.654754in}}%
\pgfpathlineto{\pgfqpoint{4.528584in}{0.656298in}}%
\pgfpathlineto{\pgfqpoint{4.529342in}{0.654884in}}%
\pgfpathlineto{\pgfqpoint{4.530184in}{0.653905in}}%
\pgfpathlineto{\pgfqpoint{4.530521in}{0.652825in}}%
\pgfpathlineto{\pgfqpoint{4.531364in}{0.653553in}}%
\pgfpathlineto{\pgfqpoint{4.531701in}{0.664880in}}%
\pgfpathlineto{\pgfqpoint{4.532122in}{0.722579in}}%
\pgfpathlineto{\pgfqpoint{4.532796in}{0.666355in}}%
\pgfpathlineto{\pgfqpoint{4.534228in}{0.656667in}}%
\pgfpathlineto{\pgfqpoint{4.534312in}{0.656745in}}%
\pgfpathlineto{\pgfqpoint{4.534397in}{0.656800in}}%
\pgfpathlineto{\pgfqpoint{4.534565in}{0.655603in}}%
\pgfpathlineto{\pgfqpoint{4.534986in}{0.651953in}}%
\pgfpathlineto{\pgfqpoint{4.535913in}{0.652140in}}%
\pgfpathlineto{\pgfqpoint{4.536334in}{0.653239in}}%
\pgfpathlineto{\pgfqpoint{4.536671in}{0.668709in}}%
\pgfpathlineto{\pgfqpoint{4.537092in}{0.724124in}}%
\pgfpathlineto{\pgfqpoint{4.537598in}{0.660175in}}%
\pgfpathlineto{\pgfqpoint{4.538019in}{0.705522in}}%
\pgfpathlineto{\pgfqpoint{4.538103in}{0.705348in}}%
\pgfpathlineto{\pgfqpoint{4.538777in}{0.652031in}}%
\pgfpathlineto{\pgfqpoint{4.539620in}{0.655102in}}%
\pgfpathlineto{\pgfqpoint{4.539788in}{0.652811in}}%
\pgfpathlineto{\pgfqpoint{4.540125in}{0.661932in}}%
\pgfpathlineto{\pgfqpoint{4.540631in}{0.693489in}}%
\pgfpathlineto{\pgfqpoint{4.541136in}{0.658369in}}%
\pgfpathlineto{\pgfqpoint{4.541305in}{0.654072in}}%
\pgfpathlineto{\pgfqpoint{4.541642in}{0.677225in}}%
\pgfpathlineto{\pgfqpoint{4.541894in}{0.712154in}}%
\pgfpathlineto{\pgfqpoint{4.542568in}{0.651721in}}%
\pgfpathlineto{\pgfqpoint{4.542737in}{0.656530in}}%
\pgfpathlineto{\pgfqpoint{4.543074in}{0.713775in}}%
\pgfpathlineto{\pgfqpoint{4.543579in}{0.652056in}}%
\pgfpathlineto{\pgfqpoint{4.544169in}{0.697274in}}%
\pgfpathlineto{\pgfqpoint{4.545011in}{0.653148in}}%
\pgfpathlineto{\pgfqpoint{4.545854in}{0.654942in}}%
\pgfpathlineto{\pgfqpoint{4.546022in}{0.656126in}}%
\pgfpathlineto{\pgfqpoint{4.546444in}{0.650562in}}%
\pgfpathlineto{\pgfqpoint{4.546612in}{0.650029in}}%
\pgfpathlineto{\pgfqpoint{4.547286in}{0.651548in}}%
\pgfpathlineto{\pgfqpoint{4.547623in}{0.662362in}}%
\pgfpathlineto{\pgfqpoint{4.548213in}{0.649275in}}%
\pgfpathlineto{\pgfqpoint{4.548297in}{0.649324in}}%
\pgfpathlineto{\pgfqpoint{4.549476in}{0.652160in}}%
\pgfpathlineto{\pgfqpoint{4.551498in}{0.671877in}}%
\pgfpathlineto{\pgfqpoint{4.550319in}{0.652037in}}%
\pgfpathlineto{\pgfqpoint{4.551667in}{0.669210in}}%
\pgfpathlineto{\pgfqpoint{4.552425in}{0.657029in}}%
\pgfpathlineto{\pgfqpoint{4.552930in}{0.661197in}}%
\pgfpathlineto{\pgfqpoint{4.553773in}{0.711043in}}%
\pgfpathlineto{\pgfqpoint{4.554952in}{0.686566in}}%
\pgfpathlineto{\pgfqpoint{4.555626in}{0.650919in}}%
\pgfpathlineto{\pgfqpoint{4.556132in}{0.672002in}}%
\pgfpathlineto{\pgfqpoint{4.556384in}{0.693930in}}%
\pgfpathlineto{\pgfqpoint{4.557058in}{0.660399in}}%
\pgfpathlineto{\pgfqpoint{4.557143in}{0.661012in}}%
\pgfpathlineto{\pgfqpoint{4.558406in}{0.687886in}}%
\pgfpathlineto{\pgfqpoint{4.558659in}{0.678712in}}%
\pgfpathlineto{\pgfqpoint{4.559502in}{0.649000in}}%
\pgfpathlineto{\pgfqpoint{4.560007in}{0.651211in}}%
\pgfpathlineto{\pgfqpoint{4.560344in}{0.655528in}}%
\pgfpathlineto{\pgfqpoint{4.560934in}{0.651138in}}%
\pgfpathlineto{\pgfqpoint{4.561102in}{0.651260in}}%
\pgfpathlineto{\pgfqpoint{4.562366in}{0.654563in}}%
\pgfpathlineto{\pgfqpoint{4.562619in}{0.651895in}}%
\pgfpathlineto{\pgfqpoint{4.562871in}{0.650205in}}%
\pgfpathlineto{\pgfqpoint{4.563798in}{0.651010in}}%
\pgfpathlineto{\pgfqpoint{4.564556in}{0.658690in}}%
\pgfpathlineto{\pgfqpoint{4.565314in}{0.654114in}}%
\pgfpathlineto{\pgfqpoint{4.566410in}{0.652225in}}%
\pgfpathlineto{\pgfqpoint{4.566747in}{0.653212in}}%
\pgfpathlineto{\pgfqpoint{4.566999in}{0.653993in}}%
\pgfpathlineto{\pgfqpoint{4.567252in}{0.651256in}}%
\pgfpathlineto{\pgfqpoint{4.567589in}{0.649308in}}%
\pgfpathlineto{\pgfqpoint{4.568431in}{0.649750in}}%
\pgfpathlineto{\pgfqpoint{4.568768in}{0.649848in}}%
\pgfpathlineto{\pgfqpoint{4.569021in}{0.650465in}}%
\pgfpathlineto{\pgfqpoint{4.571212in}{0.656989in}}%
\pgfpathlineto{\pgfqpoint{4.571380in}{0.655355in}}%
\pgfpathlineto{\pgfqpoint{4.571801in}{0.650117in}}%
\pgfpathlineto{\pgfqpoint{4.572644in}{0.651025in}}%
\pgfpathlineto{\pgfqpoint{4.573149in}{0.652322in}}%
\pgfpathlineto{\pgfqpoint{4.574413in}{0.658540in}}%
\pgfpathlineto{\pgfqpoint{4.574666in}{0.656640in}}%
\pgfpathlineto{\pgfqpoint{4.574750in}{0.656094in}}%
\pgfpathlineto{\pgfqpoint{4.575003in}{0.661378in}}%
\pgfpathlineto{\pgfqpoint{4.575255in}{0.670121in}}%
\pgfpathlineto{\pgfqpoint{4.575761in}{0.644967in}}%
\pgfpathlineto{\pgfqpoint{4.576350in}{0.646023in}}%
\pgfpathlineto{\pgfqpoint{4.576603in}{0.659338in}}%
\pgfpathlineto{\pgfqpoint{4.576940in}{0.721341in}}%
\pgfpathlineto{\pgfqpoint{4.577783in}{0.682563in}}%
\pgfpathlineto{\pgfqpoint{4.578709in}{0.740902in}}%
\pgfpathlineto{\pgfqpoint{4.579467in}{0.712227in}}%
\pgfpathlineto{\pgfqpoint{4.580310in}{0.676455in}}%
\pgfpathlineto{\pgfqpoint{4.580731in}{0.697769in}}%
\pgfpathlineto{\pgfqpoint{4.580984in}{0.718351in}}%
\pgfpathlineto{\pgfqpoint{4.581489in}{0.665535in}}%
\pgfpathlineto{\pgfqpoint{4.581742in}{0.659886in}}%
\pgfpathlineto{\pgfqpoint{4.582163in}{0.680276in}}%
\pgfpathlineto{\pgfqpoint{4.582669in}{0.743943in}}%
\pgfpathlineto{\pgfqpoint{4.583343in}{0.701487in}}%
\pgfpathlineto{\pgfqpoint{4.586376in}{0.650031in}}%
\pgfpathlineto{\pgfqpoint{4.586544in}{0.649727in}}%
\pgfpathlineto{\pgfqpoint{4.586881in}{0.650240in}}%
\pgfpathlineto{\pgfqpoint{4.587386in}{0.650181in}}%
\pgfpathlineto{\pgfqpoint{4.591009in}{0.654388in}}%
\pgfpathlineto{\pgfqpoint{4.591430in}{0.657522in}}%
\pgfpathlineto{\pgfqpoint{4.592188in}{0.655251in}}%
\pgfpathlineto{\pgfqpoint{4.592862in}{0.654980in}}%
\pgfpathlineto{\pgfqpoint{4.593031in}{0.655799in}}%
\pgfpathlineto{\pgfqpoint{4.593284in}{0.658894in}}%
\pgfpathlineto{\pgfqpoint{4.593705in}{0.652066in}}%
\pgfpathlineto{\pgfqpoint{4.594042in}{0.650246in}}%
\pgfpathlineto{\pgfqpoint{4.594463in}{0.654634in}}%
\pgfpathlineto{\pgfqpoint{4.594884in}{0.673449in}}%
\pgfpathlineto{\pgfqpoint{4.595558in}{0.654845in}}%
\pgfpathlineto{\pgfqpoint{4.595642in}{0.654708in}}%
\pgfpathlineto{\pgfqpoint{4.595895in}{0.656342in}}%
\pgfpathlineto{\pgfqpoint{4.595979in}{0.656709in}}%
\pgfpathlineto{\pgfqpoint{4.596232in}{0.654066in}}%
\pgfpathlineto{\pgfqpoint{4.596569in}{0.650871in}}%
\pgfpathlineto{\pgfqpoint{4.597075in}{0.654404in}}%
\pgfpathlineto{\pgfqpoint{4.597327in}{0.654006in}}%
\pgfpathlineto{\pgfqpoint{4.597496in}{0.653738in}}%
\pgfpathlineto{\pgfqpoint{4.597833in}{0.655092in}}%
\pgfpathlineto{\pgfqpoint{4.598086in}{0.656029in}}%
\pgfpathlineto{\pgfqpoint{4.598507in}{0.652325in}}%
\pgfpathlineto{\pgfqpoint{4.598675in}{0.652064in}}%
\pgfpathlineto{\pgfqpoint{4.599181in}{0.653716in}}%
\pgfpathlineto{\pgfqpoint{4.599686in}{0.651558in}}%
\pgfpathlineto{\pgfqpoint{4.601287in}{0.649776in}}%
\pgfpathlineto{\pgfqpoint{4.601540in}{0.650015in}}%
\pgfpathlineto{\pgfqpoint{4.601624in}{0.650336in}}%
\pgfpathlineto{\pgfqpoint{4.602045in}{0.657705in}}%
\pgfpathlineto{\pgfqpoint{4.602466in}{0.669331in}}%
\pgfpathlineto{\pgfqpoint{4.603056in}{0.655471in}}%
\pgfpathlineto{\pgfqpoint{4.603393in}{0.653774in}}%
\pgfpathlineto{\pgfqpoint{4.603814in}{0.659417in}}%
\pgfpathlineto{\pgfqpoint{4.603898in}{0.660517in}}%
\pgfpathlineto{\pgfqpoint{4.604657in}{0.657367in}}%
\pgfpathlineto{\pgfqpoint{4.606173in}{0.646358in}}%
\pgfpathlineto{\pgfqpoint{4.606510in}{0.648297in}}%
\pgfpathlineto{\pgfqpoint{4.607942in}{0.662976in}}%
\pgfpathlineto{\pgfqpoint{4.608279in}{0.683688in}}%
\pgfpathlineto{\pgfqpoint{4.608869in}{0.647334in}}%
\pgfpathlineto{\pgfqpoint{4.609122in}{0.662538in}}%
\pgfpathlineto{\pgfqpoint{4.609374in}{0.695605in}}%
\pgfpathlineto{\pgfqpoint{4.609796in}{0.647043in}}%
\pgfpathlineto{\pgfqpoint{4.610132in}{0.651323in}}%
\pgfpathlineto{\pgfqpoint{4.611312in}{0.655636in}}%
\pgfpathlineto{\pgfqpoint{4.610806in}{0.648159in}}%
\pgfpathlineto{\pgfqpoint{4.611396in}{0.655474in}}%
\pgfpathlineto{\pgfqpoint{4.613502in}{0.646560in}}%
\pgfpathlineto{\pgfqpoint{4.611986in}{0.656079in}}%
\pgfpathlineto{\pgfqpoint{4.614008in}{0.647260in}}%
\pgfpathlineto{\pgfqpoint{4.614345in}{0.649199in}}%
\pgfpathlineto{\pgfqpoint{4.614682in}{0.681543in}}%
\pgfpathlineto{\pgfqpoint{4.614934in}{0.759016in}}%
\pgfpathlineto{\pgfqpoint{4.615524in}{0.650297in}}%
\pgfpathlineto{\pgfqpoint{4.615693in}{0.654715in}}%
\pgfpathlineto{\pgfqpoint{4.616114in}{0.692870in}}%
\pgfpathlineto{\pgfqpoint{4.616619in}{0.648846in}}%
\pgfpathlineto{\pgfqpoint{4.616704in}{0.648867in}}%
\pgfpathlineto{\pgfqpoint{4.618388in}{0.653237in}}%
\pgfpathlineto{\pgfqpoint{4.619399in}{0.660144in}}%
\pgfpathlineto{\pgfqpoint{4.618978in}{0.650010in}}%
\pgfpathlineto{\pgfqpoint{4.619652in}{0.657437in}}%
\pgfpathlineto{\pgfqpoint{4.620158in}{0.650158in}}%
\pgfpathlineto{\pgfqpoint{4.620663in}{0.656645in}}%
\pgfpathlineto{\pgfqpoint{4.622095in}{0.701356in}}%
\pgfpathlineto{\pgfqpoint{4.622432in}{0.717768in}}%
\pgfpathlineto{\pgfqpoint{4.623106in}{0.695614in}}%
\pgfpathlineto{\pgfqpoint{4.623359in}{0.698604in}}%
\pgfpathlineto{\pgfqpoint{4.623443in}{0.696821in}}%
\pgfpathlineto{\pgfqpoint{4.624286in}{0.655081in}}%
\pgfpathlineto{\pgfqpoint{4.625044in}{0.658456in}}%
\pgfpathlineto{\pgfqpoint{4.625465in}{0.651482in}}%
\pgfpathlineto{\pgfqpoint{4.625970in}{0.657376in}}%
\pgfpathlineto{\pgfqpoint{4.626560in}{0.801907in}}%
\pgfpathlineto{\pgfqpoint{4.627318in}{0.683147in}}%
\pgfpathlineto{\pgfqpoint{4.627655in}{0.677449in}}%
\pgfpathlineto{\pgfqpoint{4.627824in}{0.682319in}}%
\pgfpathlineto{\pgfqpoint{4.628245in}{0.728028in}}%
\pgfpathlineto{\pgfqpoint{4.629003in}{0.696340in}}%
\pgfpathlineto{\pgfqpoint{4.630941in}{0.652102in}}%
\pgfpathlineto{\pgfqpoint{4.631109in}{0.652249in}}%
\pgfpathlineto{\pgfqpoint{4.631699in}{0.653798in}}%
\pgfpathlineto{\pgfqpoint{4.632036in}{0.726395in}}%
\pgfpathlineto{\pgfqpoint{4.632205in}{0.756111in}}%
\pgfpathlineto{\pgfqpoint{4.632794in}{0.667535in}}%
\pgfpathlineto{\pgfqpoint{4.632879in}{0.668437in}}%
\pgfpathlineto{\pgfqpoint{4.632963in}{0.669151in}}%
\pgfpathlineto{\pgfqpoint{4.633215in}{0.663546in}}%
\pgfpathlineto{\pgfqpoint{4.633552in}{0.657825in}}%
\pgfpathlineto{\pgfqpoint{4.634311in}{0.660125in}}%
\pgfpathlineto{\pgfqpoint{4.634563in}{0.664208in}}%
\pgfpathlineto{\pgfqpoint{4.635069in}{0.655200in}}%
\pgfpathlineto{\pgfqpoint{4.636248in}{0.650702in}}%
\pgfpathlineto{\pgfqpoint{4.636417in}{0.650804in}}%
\pgfpathlineto{\pgfqpoint{4.638102in}{0.652538in}}%
\pgfpathlineto{\pgfqpoint{4.638186in}{0.652309in}}%
\pgfpathlineto{\pgfqpoint{4.638691in}{0.649606in}}%
\pgfpathlineto{\pgfqpoint{4.639450in}{0.650560in}}%
\pgfpathlineto{\pgfqpoint{4.641893in}{0.654751in}}%
\pgfpathlineto{\pgfqpoint{4.642145in}{0.661318in}}%
\pgfpathlineto{\pgfqpoint{4.642651in}{0.652388in}}%
\pgfpathlineto{\pgfqpoint{4.643072in}{0.660119in}}%
\pgfpathlineto{\pgfqpoint{4.643578in}{0.651879in}}%
\pgfpathlineto{\pgfqpoint{4.644589in}{0.654621in}}%
\pgfpathlineto{\pgfqpoint{4.645262in}{0.679717in}}%
\pgfpathlineto{\pgfqpoint{4.646021in}{0.664705in}}%
\pgfpathlineto{\pgfqpoint{4.646273in}{0.665395in}}%
\pgfpathlineto{\pgfqpoint{4.646442in}{0.663955in}}%
\pgfpathlineto{\pgfqpoint{4.646947in}{0.657194in}}%
\pgfpathlineto{\pgfqpoint{4.647706in}{0.660273in}}%
\pgfpathlineto{\pgfqpoint{4.648211in}{0.695557in}}%
\pgfpathlineto{\pgfqpoint{4.648380in}{0.704796in}}%
\pgfpathlineto{\pgfqpoint{4.648801in}{0.664083in}}%
\pgfpathlineto{\pgfqpoint{4.649306in}{0.652445in}}%
\pgfpathlineto{\pgfqpoint{4.649980in}{0.656338in}}%
\pgfpathlineto{\pgfqpoint{4.650233in}{0.658212in}}%
\pgfpathlineto{\pgfqpoint{4.651075in}{0.656545in}}%
\pgfpathlineto{\pgfqpoint{4.651328in}{0.656209in}}%
\pgfpathlineto{\pgfqpoint{4.651581in}{0.657680in}}%
\pgfpathlineto{\pgfqpoint{4.652086in}{0.673863in}}%
\pgfpathlineto{\pgfqpoint{4.652507in}{0.654634in}}%
\pgfpathlineto{\pgfqpoint{4.652844in}{0.652070in}}%
\pgfpathlineto{\pgfqpoint{4.653687in}{0.652478in}}%
\pgfpathlineto{\pgfqpoint{4.655962in}{0.652938in}}%
\pgfpathlineto{\pgfqpoint{4.657141in}{0.653847in}}%
\pgfpathlineto{\pgfqpoint{4.657309in}{0.654239in}}%
\pgfpathlineto{\pgfqpoint{4.657983in}{0.652802in}}%
\pgfpathlineto{\pgfqpoint{4.659500in}{0.653110in}}%
\pgfpathlineto{\pgfqpoint{4.660089in}{0.654002in}}%
\pgfpathlineto{\pgfqpoint{4.660848in}{0.661673in}}%
\pgfpathlineto{\pgfqpoint{4.661690in}{0.658102in}}%
\pgfpathlineto{\pgfqpoint{4.662617in}{0.660131in}}%
\pgfpathlineto{\pgfqpoint{4.663122in}{0.677946in}}%
\pgfpathlineto{\pgfqpoint{4.663544in}{0.659203in}}%
\pgfpathlineto{\pgfqpoint{4.663881in}{0.653694in}}%
\pgfpathlineto{\pgfqpoint{4.664554in}{0.661597in}}%
\pgfpathlineto{\pgfqpoint{4.665902in}{0.651421in}}%
\pgfpathlineto{\pgfqpoint{4.666492in}{0.652808in}}%
\pgfpathlineto{\pgfqpoint{4.666661in}{0.656006in}}%
\pgfpathlineto{\pgfqpoint{4.667082in}{0.820857in}}%
\pgfpathlineto{\pgfqpoint{4.667166in}{0.836160in}}%
\pgfpathlineto{\pgfqpoint{4.667503in}{0.700538in}}%
\pgfpathlineto{\pgfqpoint{4.667840in}{0.657226in}}%
\pgfpathlineto{\pgfqpoint{4.668682in}{0.659870in}}%
\pgfpathlineto{\pgfqpoint{4.669778in}{0.653644in}}%
\pgfpathlineto{\pgfqpoint{4.670115in}{0.654049in}}%
\pgfpathlineto{\pgfqpoint{4.670536in}{0.653479in}}%
\pgfpathlineto{\pgfqpoint{4.670704in}{0.654613in}}%
\pgfpathlineto{\pgfqpoint{4.672305in}{0.695031in}}%
\pgfpathlineto{\pgfqpoint{4.672473in}{0.681463in}}%
\pgfpathlineto{\pgfqpoint{4.673063in}{0.652259in}}%
\pgfpathlineto{\pgfqpoint{4.673821in}{0.652340in}}%
\pgfpathlineto{\pgfqpoint{4.675422in}{0.654098in}}%
\pgfpathlineto{\pgfqpoint{4.676012in}{0.663032in}}%
\pgfpathlineto{\pgfqpoint{4.676433in}{0.653861in}}%
\pgfpathlineto{\pgfqpoint{4.677444in}{0.651806in}}%
\pgfpathlineto{\pgfqpoint{4.677023in}{0.656019in}}%
\pgfpathlineto{\pgfqpoint{4.677612in}{0.652436in}}%
\pgfpathlineto{\pgfqpoint{4.678455in}{0.665985in}}%
\pgfpathlineto{\pgfqpoint{4.678708in}{0.694399in}}%
\pgfpathlineto{\pgfqpoint{4.679466in}{0.656436in}}%
\pgfpathlineto{\pgfqpoint{4.679550in}{0.656168in}}%
\pgfpathlineto{\pgfqpoint{4.679803in}{0.658520in}}%
\pgfpathlineto{\pgfqpoint{4.681572in}{0.681659in}}%
\pgfpathlineto{\pgfqpoint{4.681656in}{0.679550in}}%
\pgfpathlineto{\pgfqpoint{4.682667in}{0.653666in}}%
\pgfpathlineto{\pgfqpoint{4.683172in}{0.657511in}}%
\pgfpathlineto{\pgfqpoint{4.684352in}{0.668508in}}%
\pgfpathlineto{\pgfqpoint{4.684520in}{0.672460in}}%
\pgfpathlineto{\pgfqpoint{4.685026in}{0.656329in}}%
\pgfpathlineto{\pgfqpoint{4.685531in}{0.650361in}}%
\pgfpathlineto{\pgfqpoint{4.686121in}{0.655934in}}%
\pgfpathlineto{\pgfqpoint{4.686205in}{0.656314in}}%
\pgfpathlineto{\pgfqpoint{4.686627in}{0.653677in}}%
\pgfpathlineto{\pgfqpoint{4.687048in}{0.651495in}}%
\pgfpathlineto{\pgfqpoint{4.687722in}{0.653496in}}%
\pgfpathlineto{\pgfqpoint{4.687890in}{0.653200in}}%
\pgfpathlineto{\pgfqpoint{4.688143in}{0.655601in}}%
\pgfpathlineto{\pgfqpoint{4.688901in}{0.708803in}}%
\pgfpathlineto{\pgfqpoint{4.689575in}{0.669840in}}%
\pgfpathlineto{\pgfqpoint{4.691765in}{0.649788in}}%
\pgfpathlineto{\pgfqpoint{4.692187in}{0.655258in}}%
\pgfpathlineto{\pgfqpoint{4.693703in}{0.739220in}}%
\pgfpathlineto{\pgfqpoint{4.694124in}{0.700394in}}%
\pgfpathlineto{\pgfqpoint{4.695304in}{0.666375in}}%
\pgfpathlineto{\pgfqpoint{4.695641in}{0.682579in}}%
\pgfpathlineto{\pgfqpoint{4.696062in}{0.779674in}}%
\pgfpathlineto{\pgfqpoint{4.696652in}{0.690353in}}%
\pgfpathlineto{\pgfqpoint{4.697241in}{0.650105in}}%
\pgfpathlineto{\pgfqpoint{4.697915in}{0.655921in}}%
\pgfpathlineto{\pgfqpoint{4.698673in}{0.657687in}}%
\pgfpathlineto{\pgfqpoint{4.699010in}{0.656880in}}%
\pgfpathlineto{\pgfqpoint{4.699263in}{0.655902in}}%
\pgfpathlineto{\pgfqpoint{4.699516in}{0.658046in}}%
\pgfpathlineto{\pgfqpoint{4.700021in}{0.670748in}}%
\pgfpathlineto{\pgfqpoint{4.700358in}{0.654410in}}%
\pgfpathlineto{\pgfqpoint{4.700611in}{0.645150in}}%
\pgfpathlineto{\pgfqpoint{4.701538in}{0.647314in}}%
\pgfpathlineto{\pgfqpoint{4.703560in}{0.649626in}}%
\pgfpathlineto{\pgfqpoint{4.704655in}{0.650185in}}%
\pgfpathlineto{\pgfqpoint{4.704992in}{0.655101in}}%
\pgfpathlineto{\pgfqpoint{4.705919in}{0.651597in}}%
\pgfpathlineto{\pgfqpoint{4.707351in}{0.653796in}}%
\pgfpathlineto{\pgfqpoint{4.708446in}{0.654396in}}%
\pgfpathlineto{\pgfqpoint{4.707940in}{0.653431in}}%
\pgfpathlineto{\pgfqpoint{4.708530in}{0.654219in}}%
\pgfpathlineto{\pgfqpoint{4.709036in}{0.652318in}}%
\pgfpathlineto{\pgfqpoint{4.709457in}{0.654595in}}%
\pgfpathlineto{\pgfqpoint{4.710552in}{0.666380in}}%
\pgfpathlineto{\pgfqpoint{4.710047in}{0.654094in}}%
\pgfpathlineto{\pgfqpoint{4.710805in}{0.658763in}}%
\pgfpathlineto{\pgfqpoint{4.711057in}{0.652881in}}%
\pgfpathlineto{\pgfqpoint{4.711984in}{0.653532in}}%
\pgfpathlineto{\pgfqpoint{4.712405in}{0.655531in}}%
\pgfpathlineto{\pgfqpoint{4.712911in}{0.664854in}}%
\pgfpathlineto{\pgfqpoint{4.713332in}{0.682231in}}%
\pgfpathlineto{\pgfqpoint{4.714090in}{0.668951in}}%
\pgfpathlineto{\pgfqpoint{4.714764in}{0.660340in}}%
\pgfpathlineto{\pgfqpoint{4.715775in}{0.664641in}}%
\pgfpathlineto{\pgfqpoint{4.716196in}{0.683284in}}%
\pgfpathlineto{\pgfqpoint{4.716449in}{0.702139in}}%
\pgfpathlineto{\pgfqpoint{4.716955in}{0.656553in}}%
\pgfpathlineto{\pgfqpoint{4.717292in}{0.652117in}}%
\pgfpathlineto{\pgfqpoint{4.718050in}{0.656662in}}%
\pgfpathlineto{\pgfqpoint{4.718555in}{0.688487in}}%
\pgfpathlineto{\pgfqpoint{4.719313in}{0.668321in}}%
\pgfpathlineto{\pgfqpoint{4.719650in}{0.665381in}}%
\pgfpathlineto{\pgfqpoint{4.719903in}{0.662678in}}%
\pgfpathlineto{\pgfqpoint{4.720240in}{0.671978in}}%
\pgfpathlineto{\pgfqpoint{4.720409in}{0.678458in}}%
\pgfpathlineto{\pgfqpoint{4.720830in}{0.655719in}}%
\pgfpathlineto{\pgfqpoint{4.721167in}{0.648350in}}%
\pgfpathlineto{\pgfqpoint{4.721925in}{0.651102in}}%
\pgfpathlineto{\pgfqpoint{4.722346in}{0.698863in}}%
\pgfpathlineto{\pgfqpoint{4.723104in}{0.657409in}}%
\pgfpathlineto{\pgfqpoint{4.723189in}{0.658052in}}%
\pgfpathlineto{\pgfqpoint{4.723526in}{0.652100in}}%
\pgfpathlineto{\pgfqpoint{4.724200in}{0.648400in}}%
\pgfpathlineto{\pgfqpoint{4.724789in}{0.648513in}}%
\pgfpathlineto{\pgfqpoint{4.724958in}{0.648339in}}%
\pgfpathlineto{\pgfqpoint{4.725547in}{0.649284in}}%
\pgfpathlineto{\pgfqpoint{4.727569in}{0.651645in}}%
\pgfpathlineto{\pgfqpoint{4.728665in}{0.652480in}}%
\pgfpathlineto{\pgfqpoint{4.728243in}{0.650933in}}%
\pgfpathlineto{\pgfqpoint{4.728749in}{0.652080in}}%
\pgfpathlineto{\pgfqpoint{4.729002in}{0.650838in}}%
\pgfpathlineto{\pgfqpoint{4.729338in}{0.653737in}}%
\pgfpathlineto{\pgfqpoint{4.729591in}{0.660668in}}%
\pgfpathlineto{\pgfqpoint{4.730012in}{0.651372in}}%
\pgfpathlineto{\pgfqpoint{4.730434in}{0.653891in}}%
\pgfpathlineto{\pgfqpoint{4.730518in}{0.654072in}}%
\pgfpathlineto{\pgfqpoint{4.730686in}{0.652325in}}%
\pgfpathlineto{\pgfqpoint{4.730855in}{0.651675in}}%
\pgfpathlineto{\pgfqpoint{4.731360in}{0.655163in}}%
\pgfpathlineto{\pgfqpoint{4.731950in}{0.653890in}}%
\pgfpathlineto{\pgfqpoint{4.732708in}{0.659541in}}%
\pgfpathlineto{\pgfqpoint{4.733045in}{0.658340in}}%
\pgfpathlineto{\pgfqpoint{4.733298in}{0.664988in}}%
\pgfpathlineto{\pgfqpoint{4.733466in}{0.671104in}}%
\pgfpathlineto{\pgfqpoint{4.733972in}{0.652629in}}%
\pgfpathlineto{\pgfqpoint{4.734140in}{0.652986in}}%
\pgfpathlineto{\pgfqpoint{4.735488in}{0.656962in}}%
\pgfpathlineto{\pgfqpoint{4.735573in}{0.656016in}}%
\pgfpathlineto{\pgfqpoint{4.736078in}{0.652581in}}%
\pgfpathlineto{\pgfqpoint{4.736499in}{0.658341in}}%
\pgfpathlineto{\pgfqpoint{4.737005in}{0.713517in}}%
\pgfpathlineto{\pgfqpoint{4.737594in}{0.657427in}}%
\pgfpathlineto{\pgfqpoint{4.737847in}{0.660915in}}%
\pgfpathlineto{\pgfqpoint{4.738353in}{0.654794in}}%
\pgfpathlineto{\pgfqpoint{4.738521in}{0.655384in}}%
\pgfpathlineto{\pgfqpoint{4.738858in}{0.657634in}}%
\pgfpathlineto{\pgfqpoint{4.739195in}{0.653731in}}%
\pgfpathlineto{\pgfqpoint{4.739532in}{0.650541in}}%
\pgfpathlineto{\pgfqpoint{4.740290in}{0.653542in}}%
\pgfpathlineto{\pgfqpoint{4.740880in}{0.652971in}}%
\pgfpathlineto{\pgfqpoint{4.741301in}{0.675830in}}%
\pgfpathlineto{\pgfqpoint{4.741470in}{0.684510in}}%
\pgfpathlineto{\pgfqpoint{4.742059in}{0.652884in}}%
\pgfpathlineto{\pgfqpoint{4.742649in}{0.651201in}}%
\pgfpathlineto{\pgfqpoint{4.743070in}{0.652850in}}%
\pgfpathlineto{\pgfqpoint{4.743407in}{0.691224in}}%
\pgfpathlineto{\pgfqpoint{4.743660in}{0.725491in}}%
\pgfpathlineto{\pgfqpoint{4.744418in}{0.684298in}}%
\pgfpathlineto{\pgfqpoint{4.744587in}{0.685397in}}%
\pgfpathlineto{\pgfqpoint{4.744839in}{0.681490in}}%
\pgfpathlineto{\pgfqpoint{4.746187in}{0.663528in}}%
\pgfpathlineto{\pgfqpoint{4.746356in}{0.665464in}}%
\pgfpathlineto{\pgfqpoint{4.747030in}{0.689078in}}%
\pgfpathlineto{\pgfqpoint{4.747620in}{0.670214in}}%
\pgfpathlineto{\pgfqpoint{4.747957in}{0.667181in}}%
\pgfpathlineto{\pgfqpoint{4.748462in}{0.661842in}}%
\pgfpathlineto{\pgfqpoint{4.748715in}{0.667437in}}%
\pgfpathlineto{\pgfqpoint{4.750147in}{0.708457in}}%
\pgfpathlineto{\pgfqpoint{4.749389in}{0.658475in}}%
\pgfpathlineto{\pgfqpoint{4.750231in}{0.699032in}}%
\pgfpathlineto{\pgfqpoint{4.751326in}{0.647074in}}%
\pgfpathlineto{\pgfqpoint{4.751579in}{0.648785in}}%
\pgfpathlineto{\pgfqpoint{4.751916in}{0.651740in}}%
\pgfpathlineto{\pgfqpoint{4.752674in}{0.649876in}}%
\pgfpathlineto{\pgfqpoint{4.753264in}{0.647364in}}%
\pgfpathlineto{\pgfqpoint{4.754022in}{0.647888in}}%
\pgfpathlineto{\pgfqpoint{4.754275in}{0.647423in}}%
\pgfpathlineto{\pgfqpoint{4.754949in}{0.648469in}}%
\pgfpathlineto{\pgfqpoint{4.757139in}{0.651601in}}%
\pgfpathlineto{\pgfqpoint{4.759498in}{0.688059in}}%
\pgfpathlineto{\pgfqpoint{4.759835in}{0.712350in}}%
\pgfpathlineto{\pgfqpoint{4.760256in}{0.668499in}}%
\pgfpathlineto{\pgfqpoint{4.760846in}{0.649622in}}%
\pgfpathlineto{\pgfqpoint{4.761520in}{0.650235in}}%
\pgfpathlineto{\pgfqpoint{4.761773in}{0.651410in}}%
\pgfpathlineto{\pgfqpoint{4.762278in}{0.672464in}}%
\pgfpathlineto{\pgfqpoint{4.762531in}{0.695371in}}%
\pgfpathlineto{\pgfqpoint{4.763373in}{0.675515in}}%
\pgfpathlineto{\pgfqpoint{4.764721in}{0.656374in}}%
\pgfpathlineto{\pgfqpoint{4.764974in}{0.661231in}}%
\pgfpathlineto{\pgfqpoint{4.765479in}{0.676070in}}%
\pgfpathlineto{\pgfqpoint{4.765985in}{0.661154in}}%
\pgfpathlineto{\pgfqpoint{4.766406in}{0.660179in}}%
\pgfpathlineto{\pgfqpoint{4.766743in}{0.662803in}}%
\pgfpathlineto{\pgfqpoint{4.768175in}{0.692601in}}%
\pgfpathlineto{\pgfqpoint{4.768428in}{0.679736in}}%
\pgfpathlineto{\pgfqpoint{4.769776in}{0.650659in}}%
\pgfpathlineto{\pgfqpoint{4.770197in}{0.651390in}}%
\pgfpathlineto{\pgfqpoint{4.770618in}{0.647318in}}%
\pgfpathlineto{\pgfqpoint{4.770871in}{0.654186in}}%
\pgfpathlineto{\pgfqpoint{4.771040in}{0.659717in}}%
\pgfpathlineto{\pgfqpoint{4.771966in}{0.654259in}}%
\pgfpathlineto{\pgfqpoint{4.772724in}{0.648768in}}%
\pgfpathlineto{\pgfqpoint{4.772977in}{0.645702in}}%
\pgfpathlineto{\pgfqpoint{4.773651in}{0.649785in}}%
\pgfpathlineto{\pgfqpoint{4.773904in}{0.653012in}}%
\pgfpathlineto{\pgfqpoint{4.774241in}{0.643148in}}%
\pgfpathlineto{\pgfqpoint{4.774578in}{0.636874in}}%
\pgfpathlineto{\pgfqpoint{4.775083in}{0.651340in}}%
\pgfpathlineto{\pgfqpoint{4.775505in}{0.745569in}}%
\pgfpathlineto{\pgfqpoint{4.776515in}{0.707367in}}%
\pgfpathlineto{\pgfqpoint{4.776937in}{0.709025in}}%
\pgfpathlineto{\pgfqpoint{4.777442in}{0.686047in}}%
\pgfpathlineto{\pgfqpoint{4.778369in}{0.647709in}}%
\pgfpathlineto{\pgfqpoint{4.778874in}{0.669340in}}%
\pgfpathlineto{\pgfqpoint{4.779464in}{0.708760in}}%
\pgfpathlineto{\pgfqpoint{4.780138in}{0.686517in}}%
\pgfpathlineto{\pgfqpoint{4.781149in}{0.667508in}}%
\pgfpathlineto{\pgfqpoint{4.782413in}{0.650926in}}%
\pgfpathlineto{\pgfqpoint{4.782665in}{0.652511in}}%
\pgfpathlineto{\pgfqpoint{4.782834in}{0.652185in}}%
\pgfpathlineto{\pgfqpoint{4.783002in}{0.653153in}}%
\pgfpathlineto{\pgfqpoint{4.783508in}{0.667721in}}%
\pgfpathlineto{\pgfqpoint{4.784182in}{0.654628in}}%
\pgfpathlineto{\pgfqpoint{4.785867in}{0.646651in}}%
\pgfpathlineto{\pgfqpoint{4.786793in}{0.642028in}}%
\pgfpathlineto{\pgfqpoint{4.787214in}{0.644789in}}%
\pgfpathlineto{\pgfqpoint{4.787383in}{0.646510in}}%
\pgfpathlineto{\pgfqpoint{4.787888in}{0.641543in}}%
\pgfpathlineto{\pgfqpoint{4.788141in}{0.642611in}}%
\pgfpathlineto{\pgfqpoint{4.789995in}{0.647157in}}%
\pgfpathlineto{\pgfqpoint{4.793449in}{0.654815in}}%
\pgfpathlineto{\pgfqpoint{4.793786in}{0.652025in}}%
\pgfpathlineto{\pgfqpoint{4.793870in}{0.651776in}}%
\pgfpathlineto{\pgfqpoint{4.794207in}{0.654150in}}%
\pgfpathlineto{\pgfqpoint{4.794544in}{0.658111in}}%
\pgfpathlineto{\pgfqpoint{4.795133in}{0.651865in}}%
\pgfpathlineto{\pgfqpoint{4.797577in}{0.653975in}}%
\pgfpathlineto{\pgfqpoint{4.798082in}{0.670011in}}%
\pgfpathlineto{\pgfqpoint{4.798840in}{0.657796in}}%
\pgfpathlineto{\pgfqpoint{4.799093in}{0.656799in}}%
\pgfpathlineto{\pgfqpoint{4.799598in}{0.654284in}}%
\pgfpathlineto{\pgfqpoint{4.799935in}{0.658672in}}%
\pgfpathlineto{\pgfqpoint{4.801199in}{0.730384in}}%
\pgfpathlineto{\pgfqpoint{4.801536in}{0.691447in}}%
\pgfpathlineto{\pgfqpoint{4.802126in}{0.653107in}}%
\pgfpathlineto{\pgfqpoint{4.802800in}{0.653614in}}%
\pgfpathlineto{\pgfqpoint{4.803137in}{0.655023in}}%
\pgfpathlineto{\pgfqpoint{4.803642in}{0.664077in}}%
\pgfpathlineto{\pgfqpoint{4.804569in}{0.660435in}}%
\pgfpathlineto{\pgfqpoint{4.804990in}{0.653877in}}%
\pgfpathlineto{\pgfqpoint{4.805917in}{0.654200in}}%
\pgfpathlineto{\pgfqpoint{4.807770in}{0.653782in}}%
\pgfpathlineto{\pgfqpoint{4.807854in}{0.654079in}}%
\pgfpathlineto{\pgfqpoint{4.808191in}{0.658185in}}%
\pgfpathlineto{\pgfqpoint{4.808613in}{0.653773in}}%
\pgfpathlineto{\pgfqpoint{4.809034in}{0.655902in}}%
\pgfpathlineto{\pgfqpoint{4.809455in}{0.653378in}}%
\pgfpathlineto{\pgfqpoint{4.809876in}{0.657765in}}%
\pgfpathlineto{\pgfqpoint{4.810466in}{0.667584in}}%
\pgfpathlineto{\pgfqpoint{4.810971in}{0.659640in}}%
\pgfpathlineto{\pgfqpoint{4.811561in}{0.663742in}}%
\pgfpathlineto{\pgfqpoint{4.811730in}{0.660781in}}%
\pgfpathlineto{\pgfqpoint{4.812319in}{0.653148in}}%
\pgfpathlineto{\pgfqpoint{4.812993in}{0.653650in}}%
\pgfpathlineto{\pgfqpoint{4.813415in}{0.661425in}}%
\pgfpathlineto{\pgfqpoint{4.813752in}{0.684726in}}%
\pgfpathlineto{\pgfqpoint{4.814425in}{0.659775in}}%
\pgfpathlineto{\pgfqpoint{4.815858in}{0.652498in}}%
\pgfpathlineto{\pgfqpoint{4.814931in}{0.661604in}}%
\pgfpathlineto{\pgfqpoint{4.816026in}{0.652612in}}%
\pgfpathlineto{\pgfqpoint{4.816532in}{0.654233in}}%
\pgfpathlineto{\pgfqpoint{4.817627in}{0.653196in}}%
\pgfpathlineto{\pgfqpoint{4.819143in}{0.656819in}}%
\pgfpathlineto{\pgfqpoint{4.819396in}{0.654525in}}%
\pgfpathlineto{\pgfqpoint{4.819986in}{0.651733in}}%
\pgfpathlineto{\pgfqpoint{4.820660in}{0.652391in}}%
\pgfpathlineto{\pgfqpoint{4.822681in}{0.653319in}}%
\pgfpathlineto{\pgfqpoint{4.826557in}{0.654744in}}%
\pgfpathlineto{\pgfqpoint{4.826894in}{0.667738in}}%
\pgfpathlineto{\pgfqpoint{4.827989in}{0.748023in}}%
\pgfpathlineto{\pgfqpoint{4.827483in}{0.663985in}}%
\pgfpathlineto{\pgfqpoint{4.828326in}{0.688806in}}%
\pgfpathlineto{\pgfqpoint{4.829589in}{0.654538in}}%
\pgfpathlineto{\pgfqpoint{4.830853in}{0.655078in}}%
\pgfpathlineto{\pgfqpoint{4.832285in}{0.658482in}}%
\pgfpathlineto{\pgfqpoint{4.832370in}{0.657798in}}%
\pgfpathlineto{\pgfqpoint{4.832791in}{0.655301in}}%
\pgfpathlineto{\pgfqpoint{4.833633in}{0.655495in}}%
\pgfpathlineto{\pgfqpoint{4.835402in}{0.655852in}}%
\pgfpathlineto{\pgfqpoint{4.835655in}{0.678496in}}%
\pgfpathlineto{\pgfqpoint{4.835992in}{0.811161in}}%
\pgfpathlineto{\pgfqpoint{4.836750in}{0.674401in}}%
\pgfpathlineto{\pgfqpoint{4.839362in}{0.655350in}}%
\pgfpathlineto{\pgfqpoint{4.839530in}{0.655472in}}%
\pgfpathlineto{\pgfqpoint{4.839783in}{0.660407in}}%
\pgfpathlineto{\pgfqpoint{4.840289in}{0.756323in}}%
\pgfpathlineto{\pgfqpoint{4.840878in}{0.661475in}}%
\pgfpathlineto{\pgfqpoint{4.842984in}{0.654321in}}%
\pgfpathlineto{\pgfqpoint{4.841215in}{0.663105in}}%
\pgfpathlineto{\pgfqpoint{4.843321in}{0.656724in}}%
\pgfpathlineto{\pgfqpoint{4.843658in}{0.670228in}}%
\pgfpathlineto{\pgfqpoint{4.844248in}{0.655275in}}%
\pgfpathlineto{\pgfqpoint{4.844501in}{0.660179in}}%
\pgfpathlineto{\pgfqpoint{4.844669in}{0.663070in}}%
\pgfpathlineto{\pgfqpoint{4.845175in}{0.654259in}}%
\pgfpathlineto{\pgfqpoint{4.845343in}{0.654352in}}%
\pgfpathlineto{\pgfqpoint{4.845849in}{0.655013in}}%
\pgfpathlineto{\pgfqpoint{4.846270in}{0.661529in}}%
\pgfpathlineto{\pgfqpoint{4.846944in}{0.655158in}}%
\pgfpathlineto{\pgfqpoint{4.847112in}{0.655654in}}%
\pgfpathlineto{\pgfqpoint{4.847618in}{0.654026in}}%
\pgfpathlineto{\pgfqpoint{4.847786in}{0.654113in}}%
\pgfpathlineto{\pgfqpoint{4.848376in}{0.655048in}}%
\pgfpathlineto{\pgfqpoint{4.848713in}{0.658816in}}%
\pgfpathlineto{\pgfqpoint{4.849387in}{0.654651in}}%
\pgfpathlineto{\pgfqpoint{4.852336in}{0.652853in}}%
\pgfpathlineto{\pgfqpoint{4.852504in}{0.653269in}}%
\pgfpathlineto{\pgfqpoint{4.853262in}{0.652952in}}%
\pgfpathlineto{\pgfqpoint{4.854357in}{0.658623in}}%
\pgfpathlineto{\pgfqpoint{4.856295in}{0.653730in}}%
\pgfpathlineto{\pgfqpoint{4.856548in}{0.653827in}}%
\pgfpathlineto{\pgfqpoint{4.857306in}{0.654600in}}%
\pgfpathlineto{\pgfqpoint{4.857559in}{0.694875in}}%
\pgfpathlineto{\pgfqpoint{4.857896in}{0.844511in}}%
\pgfpathlineto{\pgfqpoint{4.858570in}{0.667867in}}%
\pgfpathlineto{\pgfqpoint{4.859412in}{0.658410in}}%
\pgfpathlineto{\pgfqpoint{4.859918in}{0.663087in}}%
\pgfpathlineto{\pgfqpoint{4.860254in}{0.662293in}}%
\pgfpathlineto{\pgfqpoint{4.861013in}{0.665744in}}%
\pgfpathlineto{\pgfqpoint{4.861771in}{0.681812in}}%
\pgfpathlineto{\pgfqpoint{4.862276in}{0.670928in}}%
\pgfpathlineto{\pgfqpoint{4.863119in}{0.662711in}}%
\pgfpathlineto{\pgfqpoint{4.863456in}{0.669437in}}%
\pgfpathlineto{\pgfqpoint{4.864719in}{0.785852in}}%
\pgfpathlineto{\pgfqpoint{4.865056in}{0.705825in}}%
\pgfpathlineto{\pgfqpoint{4.865730in}{0.653034in}}%
\pgfpathlineto{\pgfqpoint{4.866320in}{0.653670in}}%
\pgfpathlineto{\pgfqpoint{4.867078in}{0.662872in}}%
\pgfpathlineto{\pgfqpoint{4.868089in}{0.659571in}}%
\pgfpathlineto{\pgfqpoint{4.868510in}{0.654925in}}%
\pgfpathlineto{\pgfqpoint{4.869016in}{0.662962in}}%
\pgfpathlineto{\pgfqpoint{4.869100in}{0.664563in}}%
\pgfpathlineto{\pgfqpoint{4.869606in}{0.654121in}}%
\pgfpathlineto{\pgfqpoint{4.870280in}{0.650600in}}%
\pgfpathlineto{\pgfqpoint{4.870869in}{0.650995in}}%
\pgfpathlineto{\pgfqpoint{4.872638in}{0.652973in}}%
\pgfpathlineto{\pgfqpoint{4.874155in}{0.666941in}}%
\pgfpathlineto{\pgfqpoint{4.874239in}{0.666616in}}%
\pgfpathlineto{\pgfqpoint{4.875082in}{0.649151in}}%
\pgfpathlineto{\pgfqpoint{4.876092in}{0.650371in}}%
\pgfpathlineto{\pgfqpoint{4.878451in}{0.652856in}}%
\pgfpathlineto{\pgfqpoint{4.878704in}{0.652281in}}%
\pgfpathlineto{\pgfqpoint{4.878873in}{0.652167in}}%
\pgfpathlineto{\pgfqpoint{4.879294in}{0.652962in}}%
\pgfpathlineto{\pgfqpoint{4.879631in}{0.654634in}}%
\pgfpathlineto{\pgfqpoint{4.880557in}{0.653987in}}%
\pgfpathlineto{\pgfqpoint{4.881905in}{0.652084in}}%
\pgfpathlineto{\pgfqpoint{4.881990in}{0.652596in}}%
\pgfpathlineto{\pgfqpoint{4.882242in}{0.655819in}}%
\pgfpathlineto{\pgfqpoint{4.882664in}{0.651707in}}%
\pgfpathlineto{\pgfqpoint{4.883001in}{0.652156in}}%
\pgfpathlineto{\pgfqpoint{4.883506in}{0.651724in}}%
\pgfpathlineto{\pgfqpoint{4.884096in}{0.652279in}}%
\pgfpathlineto{\pgfqpoint{4.885022in}{0.654706in}}%
\pgfpathlineto{\pgfqpoint{4.885359in}{0.659164in}}%
\pgfpathlineto{\pgfqpoint{4.886033in}{0.653235in}}%
\pgfpathlineto{\pgfqpoint{4.887213in}{0.653494in}}%
\pgfpathlineto{\pgfqpoint{4.888813in}{0.654999in}}%
\pgfpathlineto{\pgfqpoint{4.889319in}{0.667921in}}%
\pgfpathlineto{\pgfqpoint{4.890077in}{0.658507in}}%
\pgfpathlineto{\pgfqpoint{4.890414in}{0.654016in}}%
\pgfpathlineto{\pgfqpoint{4.890583in}{0.656916in}}%
\pgfpathlineto{\pgfqpoint{4.890920in}{0.707014in}}%
\pgfpathlineto{\pgfqpoint{4.891846in}{0.694201in}}%
\pgfpathlineto{\pgfqpoint{4.892352in}{0.652367in}}%
\pgfpathlineto{\pgfqpoint{4.893110in}{0.653952in}}%
\pgfpathlineto{\pgfqpoint{4.893447in}{0.674147in}}%
\pgfpathlineto{\pgfqpoint{4.893615in}{0.688529in}}%
\pgfpathlineto{\pgfqpoint{4.894037in}{0.651086in}}%
\pgfpathlineto{\pgfqpoint{4.894374in}{0.653628in}}%
\pgfpathlineto{\pgfqpoint{4.894626in}{0.651534in}}%
\pgfpathlineto{\pgfqpoint{4.894963in}{0.656949in}}%
\pgfpathlineto{\pgfqpoint{4.895300in}{0.668313in}}%
\pgfpathlineto{\pgfqpoint{4.895974in}{0.655353in}}%
\pgfpathlineto{\pgfqpoint{4.896227in}{0.656473in}}%
\pgfpathlineto{\pgfqpoint{4.897238in}{0.684917in}}%
\pgfpathlineto{\pgfqpoint{4.897406in}{0.699570in}}%
\pgfpathlineto{\pgfqpoint{4.898080in}{0.660604in}}%
\pgfpathlineto{\pgfqpoint{4.898754in}{0.662148in}}%
\pgfpathlineto{\pgfqpoint{4.899260in}{0.651448in}}%
\pgfpathlineto{\pgfqpoint{4.899512in}{0.652953in}}%
\pgfpathlineto{\pgfqpoint{4.899765in}{0.654700in}}%
\pgfpathlineto{\pgfqpoint{4.900439in}{0.651566in}}%
\pgfpathlineto{\pgfqpoint{4.900860in}{0.652505in}}%
\pgfpathlineto{\pgfqpoint{4.902124in}{0.663094in}}%
\pgfpathlineto{\pgfqpoint{4.901619in}{0.651259in}}%
\pgfpathlineto{\pgfqpoint{4.902293in}{0.658051in}}%
\pgfpathlineto{\pgfqpoint{4.902545in}{0.651548in}}%
\pgfpathlineto{\pgfqpoint{4.903051in}{0.673531in}}%
\pgfpathlineto{\pgfqpoint{4.905157in}{0.650724in}}%
\pgfpathlineto{\pgfqpoint{4.905241in}{0.651163in}}%
\pgfpathlineto{\pgfqpoint{4.905494in}{0.734578in}}%
\pgfpathlineto{\pgfqpoint{4.905662in}{0.785721in}}%
\pgfpathlineto{\pgfqpoint{4.906420in}{0.656470in}}%
\pgfpathlineto{\pgfqpoint{4.906757in}{0.662003in}}%
\pgfpathlineto{\pgfqpoint{4.907094in}{0.654755in}}%
\pgfpathlineto{\pgfqpoint{4.907431in}{0.651911in}}%
\pgfpathlineto{\pgfqpoint{4.908021in}{0.656513in}}%
\pgfpathlineto{\pgfqpoint{4.908190in}{0.659679in}}%
\pgfpathlineto{\pgfqpoint{4.908695in}{0.652165in}}%
\pgfpathlineto{\pgfqpoint{4.908948in}{0.652338in}}%
\pgfpathlineto{\pgfqpoint{4.909790in}{0.653645in}}%
\pgfpathlineto{\pgfqpoint{4.911138in}{0.671061in}}%
\pgfpathlineto{\pgfqpoint{4.910633in}{0.652759in}}%
\pgfpathlineto{\pgfqpoint{4.911307in}{0.662335in}}%
\pgfpathlineto{\pgfqpoint{4.911559in}{0.652426in}}%
\pgfpathlineto{\pgfqpoint{4.911896in}{0.680033in}}%
\pgfpathlineto{\pgfqpoint{4.912065in}{0.706922in}}%
\pgfpathlineto{\pgfqpoint{4.912655in}{0.653724in}}%
\pgfpathlineto{\pgfqpoint{4.912907in}{0.670216in}}%
\pgfpathlineto{\pgfqpoint{4.914087in}{0.726613in}}%
\pgfpathlineto{\pgfqpoint{4.913329in}{0.666620in}}%
\pgfpathlineto{\pgfqpoint{4.914255in}{0.699321in}}%
\pgfpathlineto{\pgfqpoint{4.915519in}{0.656279in}}%
\pgfpathlineto{\pgfqpoint{4.915603in}{0.656610in}}%
\pgfpathlineto{\pgfqpoint{4.916277in}{0.672238in}}%
\pgfpathlineto{\pgfqpoint{4.916698in}{0.895856in}}%
\pgfpathlineto{\pgfqpoint{4.917288in}{0.671882in}}%
\pgfpathlineto{\pgfqpoint{4.918299in}{0.658388in}}%
\pgfpathlineto{\pgfqpoint{4.918552in}{0.659344in}}%
\pgfpathlineto{\pgfqpoint{4.918804in}{0.662049in}}%
\pgfpathlineto{\pgfqpoint{4.919226in}{0.655674in}}%
\pgfpathlineto{\pgfqpoint{4.919647in}{0.653525in}}%
\pgfpathlineto{\pgfqpoint{4.920321in}{0.654463in}}%
\pgfpathlineto{\pgfqpoint{4.920658in}{0.678995in}}%
\pgfpathlineto{\pgfqpoint{4.920995in}{0.725943in}}%
\pgfpathlineto{\pgfqpoint{4.921585in}{0.662468in}}%
\pgfpathlineto{\pgfqpoint{4.921669in}{0.663266in}}%
\pgfpathlineto{\pgfqpoint{4.921921in}{0.667359in}}%
\pgfpathlineto{\pgfqpoint{4.922427in}{0.655178in}}%
\pgfpathlineto{\pgfqpoint{4.922680in}{0.653669in}}%
\pgfpathlineto{\pgfqpoint{4.923438in}{0.655995in}}%
\pgfpathlineto{\pgfqpoint{4.924449in}{0.685477in}}%
\pgfpathlineto{\pgfqpoint{4.924870in}{0.664102in}}%
\pgfpathlineto{\pgfqpoint{4.925881in}{0.654395in}}%
\pgfpathlineto{\pgfqpoint{4.926134in}{0.657892in}}%
\pgfpathlineto{\pgfqpoint{4.927482in}{0.711181in}}%
\pgfpathlineto{\pgfqpoint{4.927819in}{0.683213in}}%
\pgfpathlineto{\pgfqpoint{4.928324in}{0.649797in}}%
\pgfpathlineto{\pgfqpoint{4.929082in}{0.650515in}}%
\pgfpathlineto{\pgfqpoint{4.929504in}{0.671135in}}%
\pgfpathlineto{\pgfqpoint{4.929756in}{0.681092in}}%
\pgfpathlineto{\pgfqpoint{4.930346in}{0.660273in}}%
\pgfpathlineto{\pgfqpoint{4.930430in}{0.661295in}}%
\pgfpathlineto{\pgfqpoint{4.930683in}{0.665242in}}%
\pgfpathlineto{\pgfqpoint{4.931104in}{0.652812in}}%
\pgfpathlineto{\pgfqpoint{4.931357in}{0.650101in}}%
\pgfpathlineto{\pgfqpoint{4.932199in}{0.651981in}}%
\pgfpathlineto{\pgfqpoint{4.932368in}{0.652776in}}%
\pgfpathlineto{\pgfqpoint{4.933042in}{0.650265in}}%
\pgfpathlineto{\pgfqpoint{4.933379in}{0.653054in}}%
\pgfpathlineto{\pgfqpoint{4.935064in}{0.671591in}}%
\pgfpathlineto{\pgfqpoint{4.935653in}{0.669911in}}%
\pgfpathlineto{\pgfqpoint{4.936917in}{0.659658in}}%
\pgfpathlineto{\pgfqpoint{4.937928in}{0.645160in}}%
\pgfpathlineto{\pgfqpoint{4.938518in}{0.648061in}}%
\pgfpathlineto{\pgfqpoint{4.940877in}{0.665492in}}%
\pgfpathlineto{\pgfqpoint{4.941129in}{0.671998in}}%
\pgfpathlineto{\pgfqpoint{4.941550in}{0.651487in}}%
\pgfpathlineto{\pgfqpoint{4.941887in}{0.644913in}}%
\pgfpathlineto{\pgfqpoint{4.942646in}{0.647792in}}%
\pgfpathlineto{\pgfqpoint{4.942983in}{0.657459in}}%
\pgfpathlineto{\pgfqpoint{4.943404in}{0.646775in}}%
\pgfpathlineto{\pgfqpoint{4.943741in}{0.648225in}}%
\pgfpathlineto{\pgfqpoint{4.944499in}{0.649578in}}%
\pgfpathlineto{\pgfqpoint{4.945173in}{0.648277in}}%
\pgfpathlineto{\pgfqpoint{4.946858in}{0.649469in}}%
\pgfpathlineto{\pgfqpoint{4.950480in}{0.655308in}}%
\pgfpathlineto{\pgfqpoint{4.950902in}{0.680199in}}%
\pgfpathlineto{\pgfqpoint{4.951323in}{0.652664in}}%
\pgfpathlineto{\pgfqpoint{4.951491in}{0.652866in}}%
\pgfpathlineto{\pgfqpoint{4.952081in}{0.654472in}}%
\pgfpathlineto{\pgfqpoint{4.952502in}{0.672162in}}%
\pgfpathlineto{\pgfqpoint{4.953513in}{0.670014in}}%
\pgfpathlineto{\pgfqpoint{4.954103in}{0.652837in}}%
\pgfpathlineto{\pgfqpoint{4.954777in}{0.657450in}}%
\pgfpathlineto{\pgfqpoint{4.956293in}{0.680525in}}%
\pgfpathlineto{\pgfqpoint{4.955535in}{0.655138in}}%
\pgfpathlineto{\pgfqpoint{4.956462in}{0.677541in}}%
\pgfpathlineto{\pgfqpoint{4.958062in}{0.654819in}}%
\pgfpathlineto{\pgfqpoint{4.958905in}{0.653452in}}%
\pgfpathlineto{\pgfqpoint{4.959158in}{0.654659in}}%
\pgfpathlineto{\pgfqpoint{4.959410in}{0.657225in}}%
\pgfpathlineto{\pgfqpoint{4.960253in}{0.654551in}}%
\pgfpathlineto{\pgfqpoint{4.960590in}{0.654105in}}%
\pgfpathlineto{\pgfqpoint{4.960842in}{0.655474in}}%
\pgfpathlineto{\pgfqpoint{4.961516in}{0.652508in}}%
\pgfpathlineto{\pgfqpoint{4.962022in}{0.683995in}}%
\pgfpathlineto{\pgfqpoint{4.962359in}{0.752746in}}%
\pgfpathlineto{\pgfqpoint{4.962864in}{0.673321in}}%
\pgfpathlineto{\pgfqpoint{4.963286in}{0.723472in}}%
\pgfpathlineto{\pgfqpoint{4.963960in}{0.652356in}}%
\pgfpathlineto{\pgfqpoint{4.964970in}{0.662220in}}%
\pgfpathlineto{\pgfqpoint{4.965307in}{0.651951in}}%
\pgfpathlineto{\pgfqpoint{4.966150in}{0.653229in}}%
\pgfpathlineto{\pgfqpoint{4.966571in}{0.659245in}}%
\pgfpathlineto{\pgfqpoint{4.967245in}{0.653007in}}%
\pgfpathlineto{\pgfqpoint{4.967329in}{0.652957in}}%
\pgfpathlineto{\pgfqpoint{4.967414in}{0.653504in}}%
\pgfpathlineto{\pgfqpoint{4.968509in}{0.677136in}}%
\pgfpathlineto{\pgfqpoint{4.968761in}{0.664443in}}%
\pgfpathlineto{\pgfqpoint{4.969098in}{0.650793in}}%
\pgfpathlineto{\pgfqpoint{4.969941in}{0.652688in}}%
\pgfpathlineto{\pgfqpoint{4.970531in}{0.652177in}}%
\pgfpathlineto{\pgfqpoint{4.970699in}{0.652795in}}%
\pgfpathlineto{\pgfqpoint{4.970952in}{0.656552in}}%
\pgfpathlineto{\pgfqpoint{4.971457in}{0.652071in}}%
\pgfpathlineto{\pgfqpoint{4.971878in}{0.653662in}}%
\pgfpathlineto{\pgfqpoint{4.972215in}{0.652403in}}%
\pgfpathlineto{\pgfqpoint{4.973058in}{0.652998in}}%
\pgfpathlineto{\pgfqpoint{4.974996in}{0.654282in}}%
\pgfpathlineto{\pgfqpoint{4.975754in}{0.655002in}}%
\pgfpathlineto{\pgfqpoint{4.976006in}{0.656734in}}%
\pgfpathlineto{\pgfqpoint{4.976596in}{0.654333in}}%
\pgfpathlineto{\pgfqpoint{4.976765in}{0.654401in}}%
\pgfpathlineto{\pgfqpoint{4.978787in}{0.656428in}}%
\pgfpathlineto{\pgfqpoint{4.980219in}{0.683899in}}%
\pgfpathlineto{\pgfqpoint{4.980977in}{0.735264in}}%
\pgfpathlineto{\pgfqpoint{4.981314in}{0.702173in}}%
\pgfpathlineto{\pgfqpoint{4.982578in}{0.656730in}}%
\pgfpathlineto{\pgfqpoint{4.982662in}{0.658237in}}%
\pgfpathlineto{\pgfqpoint{4.984094in}{0.731670in}}%
\pgfpathlineto{\pgfqpoint{4.984262in}{0.708327in}}%
\pgfpathlineto{\pgfqpoint{4.985105in}{0.653840in}}%
\pgfpathlineto{\pgfqpoint{4.985610in}{0.654243in}}%
\pgfpathlineto{\pgfqpoint{4.987969in}{0.654790in}}%
\pgfpathlineto{\pgfqpoint{4.988222in}{0.656703in}}%
\pgfpathlineto{\pgfqpoint{4.988896in}{0.700358in}}%
\pgfpathlineto{\pgfqpoint{4.989570in}{0.663477in}}%
\pgfpathlineto{\pgfqpoint{4.989823in}{0.660714in}}%
\pgfpathlineto{\pgfqpoint{4.990581in}{0.662386in}}%
\pgfpathlineto{\pgfqpoint{4.990918in}{0.668582in}}%
\pgfpathlineto{\pgfqpoint{4.991339in}{0.656889in}}%
\pgfpathlineto{\pgfqpoint{4.991760in}{0.652135in}}%
\pgfpathlineto{\pgfqpoint{4.992518in}{0.654910in}}%
\pgfpathlineto{\pgfqpoint{4.993361in}{0.656609in}}%
\pgfpathlineto{\pgfqpoint{4.994035in}{0.655733in}}%
\pgfpathlineto{\pgfqpoint{4.995046in}{0.650180in}}%
\pgfpathlineto{\pgfqpoint{4.995383in}{0.653491in}}%
\pgfpathlineto{\pgfqpoint{4.995635in}{0.657611in}}%
\pgfpathlineto{\pgfqpoint{4.996225in}{0.649036in}}%
\pgfpathlineto{\pgfqpoint{4.996309in}{0.648956in}}%
\pgfpathlineto{\pgfqpoint{4.996646in}{0.649732in}}%
\pgfpathlineto{\pgfqpoint{4.996983in}{0.649807in}}%
\pgfpathlineto{\pgfqpoint{5.000269in}{0.654235in}}%
\pgfpathlineto{\pgfqpoint{5.001617in}{0.662871in}}%
\pgfpathlineto{\pgfqpoint{5.001027in}{0.653539in}}%
\pgfpathlineto{\pgfqpoint{5.001870in}{0.658339in}}%
\pgfpathlineto{\pgfqpoint{5.002038in}{0.656587in}}%
\pgfpathlineto{\pgfqpoint{5.002291in}{0.661420in}}%
\pgfpathlineto{\pgfqpoint{5.002459in}{0.669511in}}%
\pgfpathlineto{\pgfqpoint{5.002965in}{0.652036in}}%
\pgfpathlineto{\pgfqpoint{5.003217in}{0.652555in}}%
\pgfpathlineto{\pgfqpoint{5.004397in}{0.665363in}}%
\pgfpathlineto{\pgfqpoint{5.004734in}{0.656958in}}%
\pgfpathlineto{\pgfqpoint{5.005071in}{0.653724in}}%
\pgfpathlineto{\pgfqpoint{5.005661in}{0.658187in}}%
\pgfpathlineto{\pgfqpoint{5.007008in}{0.696316in}}%
\pgfpathlineto{\pgfqpoint{5.007177in}{0.715129in}}%
\pgfpathlineto{\pgfqpoint{5.007767in}{0.649693in}}%
\pgfpathlineto{\pgfqpoint{5.007935in}{0.649211in}}%
\pgfpathlineto{\pgfqpoint{5.008188in}{0.650334in}}%
\pgfpathlineto{\pgfqpoint{5.008441in}{0.711618in}}%
\pgfpathlineto{\pgfqpoint{5.008693in}{0.893162in}}%
\pgfpathlineto{\pgfqpoint{5.009367in}{0.666451in}}%
\pgfpathlineto{\pgfqpoint{5.009452in}{0.669966in}}%
\pgfpathlineto{\pgfqpoint{5.009704in}{0.682377in}}%
\pgfpathlineto{\pgfqpoint{5.010294in}{0.653544in}}%
\pgfpathlineto{\pgfqpoint{5.011221in}{0.662644in}}%
\pgfpathlineto{\pgfqpoint{5.011726in}{0.717971in}}%
\pgfpathlineto{\pgfqpoint{5.012232in}{0.655996in}}%
\pgfpathlineto{\pgfqpoint{5.012990in}{0.653720in}}%
\pgfpathlineto{\pgfqpoint{5.013243in}{0.654849in}}%
\pgfpathlineto{\pgfqpoint{5.014759in}{0.705520in}}%
\pgfpathlineto{\pgfqpoint{5.015012in}{0.676667in}}%
\pgfpathlineto{\pgfqpoint{5.015601in}{0.651936in}}%
\pgfpathlineto{\pgfqpoint{5.016275in}{0.652257in}}%
\pgfpathlineto{\pgfqpoint{5.017371in}{0.656885in}}%
\pgfpathlineto{\pgfqpoint{5.017792in}{0.666677in}}%
\pgfpathlineto{\pgfqpoint{5.018718in}{0.663403in}}%
\pgfpathlineto{\pgfqpoint{5.019055in}{0.666073in}}%
\pgfpathlineto{\pgfqpoint{5.019308in}{0.668053in}}%
\pgfpathlineto{\pgfqpoint{5.019561in}{0.662248in}}%
\pgfpathlineto{\pgfqpoint{5.020151in}{0.650157in}}%
\pgfpathlineto{\pgfqpoint{5.020909in}{0.650737in}}%
\pgfpathlineto{\pgfqpoint{5.023605in}{0.662626in}}%
\pgfpathlineto{\pgfqpoint{5.023942in}{0.658781in}}%
\pgfpathlineto{\pgfqpoint{5.024447in}{0.650185in}}%
\pgfpathlineto{\pgfqpoint{5.025121in}{0.655902in}}%
\pgfpathlineto{\pgfqpoint{5.025290in}{0.658826in}}%
\pgfpathlineto{\pgfqpoint{5.025795in}{0.650730in}}%
\pgfpathlineto{\pgfqpoint{5.026132in}{0.654967in}}%
\pgfpathlineto{\pgfqpoint{5.026469in}{0.649789in}}%
\pgfpathlineto{\pgfqpoint{5.027227in}{0.651717in}}%
\pgfpathlineto{\pgfqpoint{5.027817in}{0.704560in}}%
\pgfpathlineto{\pgfqpoint{5.028407in}{0.659287in}}%
\pgfpathlineto{\pgfqpoint{5.029839in}{0.651741in}}%
\pgfpathlineto{\pgfqpoint{5.030428in}{0.650063in}}%
\pgfpathlineto{\pgfqpoint{5.030934in}{0.651923in}}%
\pgfpathlineto{\pgfqpoint{5.031187in}{0.651362in}}%
\pgfpathlineto{\pgfqpoint{5.031355in}{0.652284in}}%
\pgfpathlineto{\pgfqpoint{5.031608in}{0.690799in}}%
\pgfpathlineto{\pgfqpoint{5.031776in}{0.736787in}}%
\pgfpathlineto{\pgfqpoint{5.032535in}{0.654072in}}%
\pgfpathlineto{\pgfqpoint{5.033714in}{0.652840in}}%
\pgfpathlineto{\pgfqpoint{5.034051in}{0.653955in}}%
\pgfpathlineto{\pgfqpoint{5.034388in}{0.682278in}}%
\pgfpathlineto{\pgfqpoint{5.035483in}{0.857843in}}%
\pgfpathlineto{\pgfqpoint{5.034978in}{0.676538in}}%
\pgfpathlineto{\pgfqpoint{5.035736in}{0.735970in}}%
\pgfpathlineto{\pgfqpoint{5.036157in}{0.652512in}}%
\pgfpathlineto{\pgfqpoint{5.037000in}{0.653069in}}%
\pgfpathlineto{\pgfqpoint{5.037421in}{0.664111in}}%
\pgfpathlineto{\pgfqpoint{5.037926in}{0.652530in}}%
\pgfpathlineto{\pgfqpoint{5.038095in}{0.652988in}}%
\pgfpathlineto{\pgfqpoint{5.038347in}{0.653546in}}%
\pgfpathlineto{\pgfqpoint{5.038600in}{0.652468in}}%
\pgfpathlineto{\pgfqpoint{5.039190in}{0.652805in}}%
\pgfpathlineto{\pgfqpoint{5.039611in}{0.653743in}}%
\pgfpathlineto{\pgfqpoint{5.040706in}{0.658299in}}%
\pgfpathlineto{\pgfqpoint{5.040117in}{0.653195in}}%
\pgfpathlineto{\pgfqpoint{5.040875in}{0.656559in}}%
\pgfpathlineto{\pgfqpoint{5.041633in}{0.653594in}}%
\pgfpathlineto{\pgfqpoint{5.042138in}{0.653829in}}%
\pgfpathlineto{\pgfqpoint{5.044076in}{0.655299in}}%
\pgfpathlineto{\pgfqpoint{5.044666in}{0.674981in}}%
\pgfpathlineto{\pgfqpoint{5.045255in}{0.658319in}}%
\pgfpathlineto{\pgfqpoint{5.046266in}{0.654397in}}%
\pgfpathlineto{\pgfqpoint{5.046519in}{0.654844in}}%
\pgfpathlineto{\pgfqpoint{5.047783in}{0.654083in}}%
\pgfpathlineto{\pgfqpoint{5.048710in}{0.654207in}}%
\pgfpathlineto{\pgfqpoint{5.048204in}{0.654719in}}%
\pgfpathlineto{\pgfqpoint{5.048878in}{0.654568in}}%
\pgfpathlineto{\pgfqpoint{5.049636in}{0.660859in}}%
\pgfpathlineto{\pgfqpoint{5.050226in}{0.672368in}}%
\pgfpathlineto{\pgfqpoint{5.050984in}{0.667431in}}%
\pgfpathlineto{\pgfqpoint{5.052837in}{0.651939in}}%
\pgfpathlineto{\pgfqpoint{5.053427in}{0.652278in}}%
\pgfpathlineto{\pgfqpoint{5.053511in}{0.652574in}}%
\pgfpathlineto{\pgfqpoint{5.053933in}{0.660962in}}%
\pgfpathlineto{\pgfqpoint{5.054101in}{0.664726in}}%
\pgfpathlineto{\pgfqpoint{5.054691in}{0.653301in}}%
\pgfpathlineto{\pgfqpoint{5.054775in}{0.653342in}}%
\pgfpathlineto{\pgfqpoint{5.055028in}{0.653734in}}%
\pgfpathlineto{\pgfqpoint{5.055449in}{0.652495in}}%
\pgfpathlineto{\pgfqpoint{5.055786in}{0.652507in}}%
\pgfpathlineto{\pgfqpoint{5.055870in}{0.652773in}}%
\pgfpathlineto{\pgfqpoint{5.057134in}{0.668072in}}%
\pgfpathlineto{\pgfqpoint{5.057302in}{0.659631in}}%
\pgfpathlineto{\pgfqpoint{5.057639in}{0.651671in}}%
\pgfpathlineto{\pgfqpoint{5.058482in}{0.653057in}}%
\pgfpathlineto{\pgfqpoint{5.059914in}{0.651828in}}%
\pgfpathlineto{\pgfqpoint{5.061599in}{0.652479in}}%
\pgfpathlineto{\pgfqpoint{5.062610in}{0.678948in}}%
\pgfpathlineto{\pgfqpoint{5.062863in}{0.768602in}}%
\pgfpathlineto{\pgfqpoint{5.063452in}{0.653451in}}%
\pgfpathlineto{\pgfqpoint{5.063705in}{0.681880in}}%
\pgfpathlineto{\pgfqpoint{5.063874in}{0.718348in}}%
\pgfpathlineto{\pgfqpoint{5.064379in}{0.653341in}}%
\pgfpathlineto{\pgfqpoint{5.064800in}{0.700111in}}%
\pgfpathlineto{\pgfqpoint{5.065053in}{0.674083in}}%
\pgfpathlineto{\pgfqpoint{5.066148in}{0.652662in}}%
\pgfpathlineto{\pgfqpoint{5.066317in}{0.653578in}}%
\pgfpathlineto{\pgfqpoint{5.066738in}{0.687554in}}%
\pgfpathlineto{\pgfqpoint{5.067833in}{0.674602in}}%
\pgfpathlineto{\pgfqpoint{5.068254in}{0.652961in}}%
\pgfpathlineto{\pgfqpoint{5.068760in}{0.682860in}}%
\pgfpathlineto{\pgfqpoint{5.069012in}{0.667527in}}%
\pgfpathlineto{\pgfqpoint{5.069686in}{0.680775in}}%
\pgfpathlineto{\pgfqpoint{5.070276in}{0.652169in}}%
\pgfpathlineto{\pgfqpoint{5.070445in}{0.653323in}}%
\pgfpathlineto{\pgfqpoint{5.071793in}{0.686082in}}%
\pgfpathlineto{\pgfqpoint{5.071034in}{0.652641in}}%
\pgfpathlineto{\pgfqpoint{5.071961in}{0.668576in}}%
\pgfpathlineto{\pgfqpoint{5.072382in}{0.652290in}}%
\pgfpathlineto{\pgfqpoint{5.073225in}{0.652325in}}%
\pgfpathlineto{\pgfqpoint{5.073477in}{0.651750in}}%
\pgfpathlineto{\pgfqpoint{5.073730in}{0.653428in}}%
\pgfpathlineto{\pgfqpoint{5.074910in}{0.687137in}}%
\pgfpathlineto{\pgfqpoint{5.074488in}{0.652570in}}%
\pgfpathlineto{\pgfqpoint{5.075078in}{0.669515in}}%
\pgfpathlineto{\pgfqpoint{5.075499in}{0.651537in}}%
\pgfpathlineto{\pgfqpoint{5.076257in}{0.651951in}}%
\pgfpathlineto{\pgfqpoint{5.077184in}{0.656893in}}%
\pgfpathlineto{\pgfqpoint{5.077437in}{0.666031in}}%
\pgfpathlineto{\pgfqpoint{5.078027in}{0.651119in}}%
\pgfpathlineto{\pgfqpoint{5.078111in}{0.651154in}}%
\pgfpathlineto{\pgfqpoint{5.078616in}{0.650798in}}%
\pgfpathlineto{\pgfqpoint{5.079038in}{0.656491in}}%
\pgfpathlineto{\pgfqpoint{5.079206in}{0.659480in}}%
\pgfpathlineto{\pgfqpoint{5.079796in}{0.654131in}}%
\pgfpathlineto{\pgfqpoint{5.080048in}{0.654192in}}%
\pgfpathlineto{\pgfqpoint{5.080217in}{0.654097in}}%
\pgfpathlineto{\pgfqpoint{5.080385in}{0.653290in}}%
\pgfpathlineto{\pgfqpoint{5.081396in}{0.651568in}}%
\pgfpathlineto{\pgfqpoint{5.080891in}{0.653636in}}%
\pgfpathlineto{\pgfqpoint{5.081649in}{0.652451in}}%
\pgfpathlineto{\pgfqpoint{5.081986in}{0.668505in}}%
\pgfpathlineto{\pgfqpoint{5.082323in}{0.714458in}}%
\pgfpathlineto{\pgfqpoint{5.082913in}{0.657177in}}%
\pgfpathlineto{\pgfqpoint{5.083081in}{0.653739in}}%
\pgfpathlineto{\pgfqpoint{5.083334in}{0.665951in}}%
\pgfpathlineto{\pgfqpoint{5.084513in}{0.727860in}}%
\pgfpathlineto{\pgfqpoint{5.084008in}{0.650283in}}%
\pgfpathlineto{\pgfqpoint{5.084598in}{0.707536in}}%
\pgfpathlineto{\pgfqpoint{5.085861in}{0.650819in}}%
\pgfpathlineto{\pgfqpoint{5.086030in}{0.651434in}}%
\pgfpathlineto{\pgfqpoint{5.087209in}{0.672835in}}%
\pgfpathlineto{\pgfqpoint{5.087462in}{0.661000in}}%
\pgfpathlineto{\pgfqpoint{5.087799in}{0.650934in}}%
\pgfpathlineto{\pgfqpoint{5.088726in}{0.651304in}}%
\pgfpathlineto{\pgfqpoint{5.088978in}{0.665472in}}%
\pgfpathlineto{\pgfqpoint{5.089315in}{0.697993in}}%
\pgfpathlineto{\pgfqpoint{5.089905in}{0.658475in}}%
\pgfpathlineto{\pgfqpoint{5.089989in}{0.659031in}}%
\pgfpathlineto{\pgfqpoint{5.090326in}{0.664620in}}%
\pgfpathlineto{\pgfqpoint{5.090748in}{0.651973in}}%
\pgfpathlineto{\pgfqpoint{5.090916in}{0.650842in}}%
\pgfpathlineto{\pgfqpoint{5.091253in}{0.656287in}}%
\pgfpathlineto{\pgfqpoint{5.091674in}{0.653312in}}%
\pgfpathlineto{\pgfqpoint{5.092011in}{0.694139in}}%
\pgfpathlineto{\pgfqpoint{5.092264in}{0.773464in}}%
\pgfpathlineto{\pgfqpoint{5.092938in}{0.653838in}}%
\pgfpathlineto{\pgfqpoint{5.093696in}{0.651264in}}%
\pgfpathlineto{\pgfqpoint{5.093359in}{0.655168in}}%
\pgfpathlineto{\pgfqpoint{5.093865in}{0.652352in}}%
\pgfpathlineto{\pgfqpoint{5.094286in}{0.721033in}}%
\pgfpathlineto{\pgfqpoint{5.094370in}{0.729294in}}%
\pgfpathlineto{\pgfqpoint{5.094876in}{0.672907in}}%
\pgfpathlineto{\pgfqpoint{5.095297in}{0.723840in}}%
\pgfpathlineto{\pgfqpoint{5.095634in}{0.667629in}}%
\pgfpathlineto{\pgfqpoint{5.095971in}{0.651126in}}%
\pgfpathlineto{\pgfqpoint{5.096813in}{0.654550in}}%
\pgfpathlineto{\pgfqpoint{5.096982in}{0.657291in}}%
\pgfpathlineto{\pgfqpoint{5.097403in}{0.651423in}}%
\pgfpathlineto{\pgfqpoint{5.097824in}{0.653642in}}%
\pgfpathlineto{\pgfqpoint{5.097908in}{0.653837in}}%
\pgfpathlineto{\pgfqpoint{5.098161in}{0.652007in}}%
\pgfpathlineto{\pgfqpoint{5.098414in}{0.651452in}}%
\pgfpathlineto{\pgfqpoint{5.098751in}{0.652940in}}%
\pgfpathlineto{\pgfqpoint{5.099256in}{0.651902in}}%
\pgfpathlineto{\pgfqpoint{5.102121in}{0.655676in}}%
\pgfpathlineto{\pgfqpoint{5.103553in}{0.708102in}}%
\pgfpathlineto{\pgfqpoint{5.103890in}{0.680386in}}%
\pgfpathlineto{\pgfqpoint{5.105069in}{0.652985in}}%
\pgfpathlineto{\pgfqpoint{5.105153in}{0.653157in}}%
\pgfpathlineto{\pgfqpoint{5.106249in}{0.668014in}}%
\pgfpathlineto{\pgfqpoint{5.106417in}{0.683371in}}%
\pgfpathlineto{\pgfqpoint{5.106922in}{0.652634in}}%
\pgfpathlineto{\pgfqpoint{5.107259in}{0.660409in}}%
\pgfpathlineto{\pgfqpoint{5.108355in}{0.704098in}}%
\pgfpathlineto{\pgfqpoint{5.107849in}{0.653185in}}%
\pgfpathlineto{\pgfqpoint{5.108607in}{0.688493in}}%
\pgfpathlineto{\pgfqpoint{5.109113in}{0.653713in}}%
\pgfpathlineto{\pgfqpoint{5.109955in}{0.654929in}}%
\pgfpathlineto{\pgfqpoint{5.110208in}{0.652361in}}%
\pgfpathlineto{\pgfqpoint{5.111135in}{0.652680in}}%
\pgfpathlineto{\pgfqpoint{5.112651in}{0.655305in}}%
\pgfpathlineto{\pgfqpoint{5.112820in}{0.656163in}}%
\pgfpathlineto{\pgfqpoint{5.113241in}{0.653773in}}%
\pgfpathlineto{\pgfqpoint{5.113578in}{0.653926in}}%
\pgfpathlineto{\pgfqpoint{5.114252in}{0.653877in}}%
\pgfpathlineto{\pgfqpoint{5.114336in}{0.654303in}}%
\pgfpathlineto{\pgfqpoint{5.116274in}{0.675658in}}%
\pgfpathlineto{\pgfqpoint{5.116526in}{0.666785in}}%
\pgfpathlineto{\pgfqpoint{5.117116in}{0.653996in}}%
\pgfpathlineto{\pgfqpoint{5.117790in}{0.656612in}}%
\pgfpathlineto{\pgfqpoint{5.118295in}{0.657466in}}%
\pgfpathlineto{\pgfqpoint{5.118548in}{0.662157in}}%
\pgfpathlineto{\pgfqpoint{5.119054in}{0.651690in}}%
\pgfpathlineto{\pgfqpoint{5.119222in}{0.652964in}}%
\pgfpathlineto{\pgfqpoint{5.120486in}{0.682255in}}%
\pgfpathlineto{\pgfqpoint{5.120654in}{0.671509in}}%
\pgfpathlineto{\pgfqpoint{5.121160in}{0.651538in}}%
\pgfpathlineto{\pgfqpoint{5.121918in}{0.652057in}}%
\pgfpathlineto{\pgfqpoint{5.123350in}{0.652801in}}%
\pgfpathlineto{\pgfqpoint{5.123687in}{0.653853in}}%
\pgfpathlineto{\pgfqpoint{5.124698in}{0.653283in}}%
\pgfpathlineto{\pgfqpoint{5.126720in}{0.654242in}}%
\pgfpathlineto{\pgfqpoint{5.127141in}{0.703797in}}%
\pgfpathlineto{\pgfqpoint{5.128236in}{0.682464in}}%
\pgfpathlineto{\pgfqpoint{5.129247in}{0.652371in}}%
\pgfpathlineto{\pgfqpoint{5.129416in}{0.655831in}}%
\pgfpathlineto{\pgfqpoint{5.129753in}{0.700577in}}%
\pgfpathlineto{\pgfqpoint{5.130342in}{0.652791in}}%
\pgfpathlineto{\pgfqpoint{5.130427in}{0.653074in}}%
\pgfpathlineto{\pgfqpoint{5.130764in}{0.656399in}}%
\pgfpathlineto{\pgfqpoint{5.131185in}{0.651838in}}%
\pgfpathlineto{\pgfqpoint{5.131522in}{0.653061in}}%
\pgfpathlineto{\pgfqpoint{5.132449in}{0.659213in}}%
\pgfpathlineto{\pgfqpoint{5.131943in}{0.652496in}}%
\pgfpathlineto{\pgfqpoint{5.132701in}{0.653635in}}%
\pgfpathlineto{\pgfqpoint{5.132870in}{0.652560in}}%
\pgfpathlineto{\pgfqpoint{5.133460in}{0.654129in}}%
\pgfpathlineto{\pgfqpoint{5.133881in}{0.723470in}}%
\pgfpathlineto{\pgfqpoint{5.134807in}{0.674642in}}%
\pgfpathlineto{\pgfqpoint{5.135481in}{0.653306in}}%
\pgfpathlineto{\pgfqpoint{5.136155in}{0.655365in}}%
\pgfpathlineto{\pgfqpoint{5.136492in}{0.664865in}}%
\pgfpathlineto{\pgfqpoint{5.137082in}{0.654447in}}%
\pgfpathlineto{\pgfqpoint{5.137166in}{0.654454in}}%
\pgfpathlineto{\pgfqpoint{5.137419in}{0.655924in}}%
\pgfpathlineto{\pgfqpoint{5.137840in}{0.668884in}}%
\pgfpathlineto{\pgfqpoint{5.138683in}{0.662206in}}%
\pgfpathlineto{\pgfqpoint{5.139104in}{0.653413in}}%
\pgfpathlineto{\pgfqpoint{5.139946in}{0.654705in}}%
\pgfpathlineto{\pgfqpoint{5.141294in}{0.652976in}}%
\pgfpathlineto{\pgfqpoint{5.141631in}{0.654781in}}%
\pgfpathlineto{\pgfqpoint{5.141800in}{0.655646in}}%
\pgfpathlineto{\pgfqpoint{5.142305in}{0.652020in}}%
\pgfpathlineto{\pgfqpoint{5.143485in}{0.652025in}}%
\pgfpathlineto{\pgfqpoint{5.143569in}{0.652196in}}%
\pgfpathlineto{\pgfqpoint{5.144159in}{0.664357in}}%
\pgfpathlineto{\pgfqpoint{5.144327in}{0.667966in}}%
\pgfpathlineto{\pgfqpoint{5.144917in}{0.655789in}}%
\pgfpathlineto{\pgfqpoint{5.145843in}{0.652539in}}%
\pgfpathlineto{\pgfqpoint{5.145338in}{0.657151in}}%
\pgfpathlineto{\pgfqpoint{5.146012in}{0.656015in}}%
\pgfpathlineto{\pgfqpoint{5.147444in}{0.759727in}}%
\pgfpathlineto{\pgfqpoint{5.147613in}{0.716342in}}%
\pgfpathlineto{\pgfqpoint{5.148287in}{0.652610in}}%
\pgfpathlineto{\pgfqpoint{5.148876in}{0.652992in}}%
\pgfpathlineto{\pgfqpoint{5.149213in}{0.660484in}}%
\pgfpathlineto{\pgfqpoint{5.149382in}{0.664918in}}%
\pgfpathlineto{\pgfqpoint{5.149887in}{0.653728in}}%
\pgfpathlineto{\pgfqpoint{5.150140in}{0.654255in}}%
\pgfpathlineto{\pgfqpoint{5.150982in}{0.652749in}}%
\pgfpathlineto{\pgfqpoint{5.151488in}{0.653796in}}%
\pgfpathlineto{\pgfqpoint{5.151825in}{0.657809in}}%
\pgfpathlineto{\pgfqpoint{5.152499in}{0.652963in}}%
\pgfpathlineto{\pgfqpoint{5.153510in}{0.653639in}}%
\pgfpathlineto{\pgfqpoint{5.153762in}{0.656422in}}%
\pgfpathlineto{\pgfqpoint{5.154605in}{0.653845in}}%
\pgfpathlineto{\pgfqpoint{5.154942in}{0.657757in}}%
\pgfpathlineto{\pgfqpoint{5.155279in}{0.671446in}}%
\pgfpathlineto{\pgfqpoint{5.155784in}{0.653833in}}%
\pgfpathlineto{\pgfqpoint{5.155953in}{0.654140in}}%
\pgfpathlineto{\pgfqpoint{5.157638in}{0.654944in}}%
\pgfpathlineto{\pgfqpoint{5.157722in}{0.654819in}}%
\pgfpathlineto{\pgfqpoint{5.158059in}{0.654343in}}%
\pgfpathlineto{\pgfqpoint{5.158480in}{0.655560in}}%
\pgfpathlineto{\pgfqpoint{5.159238in}{0.657397in}}%
\pgfpathlineto{\pgfqpoint{5.159491in}{0.655421in}}%
\pgfpathlineto{\pgfqpoint{5.160755in}{0.654055in}}%
\pgfpathlineto{\pgfqpoint{5.161007in}{0.653854in}}%
\pgfpathlineto{\pgfqpoint{5.161344in}{0.655381in}}%
\pgfpathlineto{\pgfqpoint{5.162018in}{0.654408in}}%
\pgfpathlineto{\pgfqpoint{5.162861in}{0.660810in}}%
\pgfpathlineto{\pgfqpoint{5.164546in}{0.653017in}}%
\pgfpathlineto{\pgfqpoint{5.164630in}{0.653940in}}%
\pgfpathlineto{\pgfqpoint{5.164967in}{0.665747in}}%
\pgfpathlineto{\pgfqpoint{5.165472in}{0.651491in}}%
\pgfpathlineto{\pgfqpoint{5.165641in}{0.651913in}}%
\pgfpathlineto{\pgfqpoint{5.166736in}{0.653452in}}%
\pgfpathlineto{\pgfqpoint{5.166315in}{0.650770in}}%
\pgfpathlineto{\pgfqpoint{5.166820in}{0.652826in}}%
\pgfpathlineto{\pgfqpoint{5.166989in}{0.651767in}}%
\pgfpathlineto{\pgfqpoint{5.167326in}{0.657203in}}%
\pgfpathlineto{\pgfqpoint{5.167579in}{0.663050in}}%
\pgfpathlineto{\pgfqpoint{5.168084in}{0.651016in}}%
\pgfpathlineto{\pgfqpoint{5.168168in}{0.651050in}}%
\pgfpathlineto{\pgfqpoint{5.168421in}{0.653703in}}%
\pgfpathlineto{\pgfqpoint{5.168758in}{0.740004in}}%
\pgfpathlineto{\pgfqpoint{5.168926in}{0.795245in}}%
\pgfpathlineto{\pgfqpoint{5.169432in}{0.671960in}}%
\pgfpathlineto{\pgfqpoint{5.169685in}{0.689111in}}%
\pgfpathlineto{\pgfqpoint{5.169769in}{0.693490in}}%
\pgfpathlineto{\pgfqpoint{5.170190in}{0.660503in}}%
\pgfpathlineto{\pgfqpoint{5.170443in}{0.652705in}}%
\pgfpathlineto{\pgfqpoint{5.171370in}{0.655231in}}%
\pgfpathlineto{\pgfqpoint{5.171707in}{0.660892in}}%
\pgfpathlineto{\pgfqpoint{5.172296in}{0.652094in}}%
\pgfpathlineto{\pgfqpoint{5.172886in}{0.652435in}}%
\pgfpathlineto{\pgfqpoint{5.173139in}{0.659628in}}%
\pgfpathlineto{\pgfqpoint{5.173223in}{0.662824in}}%
\pgfpathlineto{\pgfqpoint{5.173644in}{0.652325in}}%
\pgfpathlineto{\pgfqpoint{5.173981in}{0.652365in}}%
\pgfpathlineto{\pgfqpoint{5.174150in}{0.652248in}}%
\pgfpathlineto{\pgfqpoint{5.174487in}{0.653428in}}%
\pgfpathlineto{\pgfqpoint{5.174655in}{0.654027in}}%
\pgfpathlineto{\pgfqpoint{5.174992in}{0.652969in}}%
\pgfpathlineto{\pgfqpoint{5.175498in}{0.653013in}}%
\pgfpathlineto{\pgfqpoint{5.175835in}{0.653102in}}%
\pgfpathlineto{\pgfqpoint{5.175919in}{0.653345in}}%
\pgfpathlineto{\pgfqpoint{5.176256in}{0.664376in}}%
\pgfpathlineto{\pgfqpoint{5.177098in}{0.694947in}}%
\pgfpathlineto{\pgfqpoint{5.176677in}{0.656243in}}%
\pgfpathlineto{\pgfqpoint{5.177267in}{0.670721in}}%
\pgfpathlineto{\pgfqpoint{5.177519in}{0.653028in}}%
\pgfpathlineto{\pgfqpoint{5.178446in}{0.653422in}}%
\pgfpathlineto{\pgfqpoint{5.178783in}{0.662646in}}%
\pgfpathlineto{\pgfqpoint{5.179204in}{0.653296in}}%
\pgfpathlineto{\pgfqpoint{5.179710in}{0.657845in}}%
\pgfpathlineto{\pgfqpoint{5.179962in}{0.653535in}}%
\pgfpathlineto{\pgfqpoint{5.180889in}{0.653717in}}%
\pgfpathlineto{\pgfqpoint{5.183501in}{0.654951in}}%
\pgfpathlineto{\pgfqpoint{5.184427in}{0.663207in}}%
\pgfpathlineto{\pgfqpoint{5.184764in}{0.696981in}}%
\pgfpathlineto{\pgfqpoint{5.185438in}{0.654104in}}%
\pgfpathlineto{\pgfqpoint{5.186786in}{0.654297in}}%
\pgfpathlineto{\pgfqpoint{5.187039in}{0.660990in}}%
\pgfpathlineto{\pgfqpoint{5.188640in}{0.700699in}}%
\pgfpathlineto{\pgfqpoint{5.189735in}{0.653365in}}%
\pgfpathlineto{\pgfqpoint{5.190746in}{0.654008in}}%
\pgfpathlineto{\pgfqpoint{5.191925in}{0.653484in}}%
\pgfpathlineto{\pgfqpoint{5.191251in}{0.655092in}}%
\pgfpathlineto{\pgfqpoint{5.192262in}{0.653971in}}%
\pgfpathlineto{\pgfqpoint{5.192515in}{0.654420in}}%
\pgfpathlineto{\pgfqpoint{5.192852in}{0.653478in}}%
\pgfpathlineto{\pgfqpoint{5.193442in}{0.653910in}}%
\pgfpathlineto{\pgfqpoint{5.193779in}{0.653989in}}%
\pgfpathlineto{\pgfqpoint{5.193863in}{0.654297in}}%
\pgfpathlineto{\pgfqpoint{5.194031in}{0.656296in}}%
\pgfpathlineto{\pgfqpoint{5.194453in}{0.654131in}}%
\pgfpathlineto{\pgfqpoint{5.194958in}{0.655043in}}%
\pgfpathlineto{\pgfqpoint{5.195379in}{0.654003in}}%
\pgfpathlineto{\pgfqpoint{5.195632in}{0.655982in}}%
\pgfpathlineto{\pgfqpoint{5.196727in}{0.660209in}}%
\pgfpathlineto{\pgfqpoint{5.196222in}{0.654333in}}%
\pgfpathlineto{\pgfqpoint{5.196811in}{0.658807in}}%
\pgfpathlineto{\pgfqpoint{5.197233in}{0.654114in}}%
\pgfpathlineto{\pgfqpoint{5.197654in}{0.660338in}}%
\pgfpathlineto{\pgfqpoint{5.197991in}{0.711422in}}%
\pgfpathlineto{\pgfqpoint{5.198581in}{0.660327in}}%
\pgfpathlineto{\pgfqpoint{5.199002in}{0.690802in}}%
\pgfpathlineto{\pgfqpoint{5.199676in}{0.653278in}}%
\pgfpathlineto{\pgfqpoint{5.200434in}{0.654444in}}%
\pgfpathlineto{\pgfqpoint{5.200518in}{0.654883in}}%
\pgfpathlineto{\pgfqpoint{5.200855in}{0.653552in}}%
\pgfpathlineto{\pgfqpoint{5.201445in}{0.653914in}}%
\pgfpathlineto{\pgfqpoint{5.201950in}{0.653886in}}%
\pgfpathlineto{\pgfqpoint{5.202035in}{0.654130in}}%
\pgfpathlineto{\pgfqpoint{5.202287in}{0.675122in}}%
\pgfpathlineto{\pgfqpoint{5.203382in}{0.790876in}}%
\pgfpathlineto{\pgfqpoint{5.202961in}{0.657415in}}%
\pgfpathlineto{\pgfqpoint{5.203467in}{0.752501in}}%
\pgfpathlineto{\pgfqpoint{5.203888in}{0.654266in}}%
\pgfpathlineto{\pgfqpoint{5.204730in}{0.661052in}}%
\pgfpathlineto{\pgfqpoint{5.205657in}{0.654533in}}%
\pgfpathlineto{\pgfqpoint{5.205320in}{0.662251in}}%
\pgfpathlineto{\pgfqpoint{5.205910in}{0.656174in}}%
\pgfpathlineto{\pgfqpoint{5.206836in}{0.659188in}}%
\pgfpathlineto{\pgfqpoint{5.206415in}{0.654989in}}%
\pgfpathlineto{\pgfqpoint{5.207005in}{0.656926in}}%
\pgfpathlineto{\pgfqpoint{5.207510in}{0.654401in}}%
\pgfpathlineto{\pgfqpoint{5.208269in}{0.654636in}}%
\pgfpathlineto{\pgfqpoint{5.209532in}{0.655579in}}%
\pgfpathlineto{\pgfqpoint{5.210459in}{0.660676in}}%
\pgfpathlineto{\pgfqpoint{5.210038in}{0.654802in}}%
\pgfpathlineto{\pgfqpoint{5.210627in}{0.655590in}}%
\pgfpathlineto{\pgfqpoint{5.210796in}{0.654143in}}%
\pgfpathlineto{\pgfqpoint{5.211049in}{0.661212in}}%
\pgfpathlineto{\pgfqpoint{5.211301in}{0.676774in}}%
\pgfpathlineto{\pgfqpoint{5.211723in}{0.655744in}}%
\pgfpathlineto{\pgfqpoint{5.212060in}{0.657500in}}%
\pgfpathlineto{\pgfqpoint{5.212481in}{0.664606in}}%
\pgfpathlineto{\pgfqpoint{5.212818in}{0.654690in}}%
\pgfpathlineto{\pgfqpoint{5.212986in}{0.653451in}}%
\pgfpathlineto{\pgfqpoint{5.213408in}{0.659816in}}%
\pgfpathlineto{\pgfqpoint{5.213913in}{0.653567in}}%
\pgfpathlineto{\pgfqpoint{5.214418in}{0.681588in}}%
\pgfpathlineto{\pgfqpoint{5.214671in}{0.716148in}}%
\pgfpathlineto{\pgfqpoint{5.215177in}{0.655517in}}%
\pgfpathlineto{\pgfqpoint{5.215345in}{0.656738in}}%
\pgfpathlineto{\pgfqpoint{5.215935in}{0.653664in}}%
\pgfpathlineto{\pgfqpoint{5.216862in}{0.665426in}}%
\pgfpathlineto{\pgfqpoint{5.217283in}{0.661715in}}%
\pgfpathlineto{\pgfqpoint{5.217451in}{0.664916in}}%
\pgfpathlineto{\pgfqpoint{5.217873in}{0.682620in}}%
\pgfpathlineto{\pgfqpoint{5.218294in}{0.659612in}}%
\pgfpathlineto{\pgfqpoint{5.218799in}{0.652011in}}%
\pgfpathlineto{\pgfqpoint{5.219389in}{0.654249in}}%
\pgfpathlineto{\pgfqpoint{5.220063in}{0.712062in}}%
\pgfpathlineto{\pgfqpoint{5.221074in}{0.683280in}}%
\pgfpathlineto{\pgfqpoint{5.221748in}{0.652278in}}%
\pgfpathlineto{\pgfqpoint{5.222337in}{0.664737in}}%
\pgfpathlineto{\pgfqpoint{5.222590in}{0.694981in}}%
\pgfpathlineto{\pgfqpoint{5.223180in}{0.655100in}}%
\pgfpathlineto{\pgfqpoint{5.223348in}{0.657537in}}%
\pgfpathlineto{\pgfqpoint{5.223685in}{0.662312in}}%
\pgfpathlineto{\pgfqpoint{5.224107in}{0.652014in}}%
\pgfpathlineto{\pgfqpoint{5.224865in}{0.652808in}}%
\pgfpathlineto{\pgfqpoint{5.225286in}{0.651725in}}%
\pgfpathlineto{\pgfqpoint{5.225539in}{0.652644in}}%
\pgfpathlineto{\pgfqpoint{5.226381in}{0.664688in}}%
\pgfpathlineto{\pgfqpoint{5.226718in}{0.655167in}}%
\pgfpathlineto{\pgfqpoint{5.226971in}{0.651561in}}%
\pgfpathlineto{\pgfqpoint{5.227308in}{0.666129in}}%
\pgfpathlineto{\pgfqpoint{5.227392in}{0.669763in}}%
\pgfpathlineto{\pgfqpoint{5.227898in}{0.649591in}}%
\pgfpathlineto{\pgfqpoint{5.227982in}{0.649696in}}%
\pgfpathlineto{\pgfqpoint{5.230341in}{0.652605in}}%
\pgfpathlineto{\pgfqpoint{5.230509in}{0.652055in}}%
\pgfpathlineto{\pgfqpoint{5.230678in}{0.651965in}}%
\pgfpathlineto{\pgfqpoint{5.230762in}{0.652256in}}%
\pgfpathlineto{\pgfqpoint{5.231183in}{0.672055in}}%
\pgfpathlineto{\pgfqpoint{5.232110in}{0.657411in}}%
\pgfpathlineto{\pgfqpoint{5.232531in}{0.651908in}}%
\pgfpathlineto{\pgfqpoint{5.233458in}{0.654621in}}%
\pgfpathlineto{\pgfqpoint{5.234637in}{0.668803in}}%
\pgfpathlineto{\pgfqpoint{5.234216in}{0.652406in}}%
\pgfpathlineto{\pgfqpoint{5.234721in}{0.665338in}}%
\pgfpathlineto{\pgfqpoint{5.235058in}{0.652924in}}%
\pgfpathlineto{\pgfqpoint{5.235395in}{0.678431in}}%
\pgfpathlineto{\pgfqpoint{5.235564in}{0.727288in}}%
\pgfpathlineto{\pgfqpoint{5.236069in}{0.653127in}}%
\pgfpathlineto{\pgfqpoint{5.236491in}{0.693052in}}%
\pgfpathlineto{\pgfqpoint{5.236996in}{0.652388in}}%
\pgfpathlineto{\pgfqpoint{5.238849in}{0.652940in}}%
\pgfpathlineto{\pgfqpoint{5.239523in}{0.653255in}}%
\pgfpathlineto{\pgfqpoint{5.239608in}{0.653538in}}%
\pgfpathlineto{\pgfqpoint{5.239945in}{0.667352in}}%
\pgfpathlineto{\pgfqpoint{5.241124in}{0.723772in}}%
\pgfpathlineto{\pgfqpoint{5.240619in}{0.654482in}}%
\pgfpathlineto{\pgfqpoint{5.241293in}{0.694952in}}%
\pgfpathlineto{\pgfqpoint{5.241629in}{0.653302in}}%
\pgfpathlineto{\pgfqpoint{5.242556in}{0.653409in}}%
\pgfpathlineto{\pgfqpoint{5.242725in}{0.655814in}}%
\pgfpathlineto{\pgfqpoint{5.242977in}{0.668829in}}%
\pgfpathlineto{\pgfqpoint{5.243483in}{0.653187in}}%
\pgfpathlineto{\pgfqpoint{5.243736in}{0.653423in}}%
\pgfpathlineto{\pgfqpoint{5.245252in}{0.654402in}}%
\pgfpathlineto{\pgfqpoint{5.245505in}{0.653815in}}%
\pgfpathlineto{\pgfqpoint{5.245673in}{0.654549in}}%
\pgfpathlineto{\pgfqpoint{5.246010in}{0.663878in}}%
\pgfpathlineto{\pgfqpoint{5.246431in}{0.653237in}}%
\pgfpathlineto{\pgfqpoint{5.246684in}{0.653867in}}%
\pgfpathlineto{\pgfqpoint{5.247948in}{0.655257in}}%
\pgfpathlineto{\pgfqpoint{5.247105in}{0.653135in}}%
\pgfpathlineto{\pgfqpoint{5.248116in}{0.654811in}}%
\pgfpathlineto{\pgfqpoint{5.249296in}{0.653652in}}%
\pgfpathlineto{\pgfqpoint{5.249380in}{0.653686in}}%
\pgfpathlineto{\pgfqpoint{5.250981in}{0.656105in}}%
\pgfpathlineto{\pgfqpoint{5.252244in}{0.670322in}}%
\pgfpathlineto{\pgfqpoint{5.251739in}{0.653939in}}%
\pgfpathlineto{\pgfqpoint{5.252413in}{0.662102in}}%
\pgfpathlineto{\pgfqpoint{5.252666in}{0.653587in}}%
\pgfpathlineto{\pgfqpoint{5.252918in}{0.672529in}}%
\pgfpathlineto{\pgfqpoint{5.253171in}{0.778498in}}%
\pgfpathlineto{\pgfqpoint{5.253676in}{0.656179in}}%
\pgfpathlineto{\pgfqpoint{5.254098in}{0.746618in}}%
\pgfpathlineto{\pgfqpoint{5.255698in}{0.653974in}}%
\pgfpathlineto{\pgfqpoint{5.256288in}{0.654400in}}%
\pgfpathlineto{\pgfqpoint{5.256793in}{0.653965in}}%
\pgfpathlineto{\pgfqpoint{5.257467in}{0.654332in}}%
\pgfpathlineto{\pgfqpoint{5.258647in}{0.655215in}}%
\pgfpathlineto{\pgfqpoint{5.258900in}{0.656900in}}%
\pgfpathlineto{\pgfqpoint{5.259405in}{0.654823in}}%
\pgfpathlineto{\pgfqpoint{5.259658in}{0.654903in}}%
\pgfpathlineto{\pgfqpoint{5.261090in}{0.657447in}}%
\pgfpathlineto{\pgfqpoint{5.261511in}{0.667465in}}%
\pgfpathlineto{\pgfqpoint{5.262017in}{0.655645in}}%
\pgfpathlineto{\pgfqpoint{5.262101in}{0.655909in}}%
\pgfpathlineto{\pgfqpoint{5.262354in}{0.658080in}}%
\pgfpathlineto{\pgfqpoint{5.263028in}{0.654834in}}%
\pgfpathlineto{\pgfqpoint{5.264039in}{0.655713in}}%
\pgfpathlineto{\pgfqpoint{5.265555in}{0.660240in}}%
\pgfpathlineto{\pgfqpoint{5.264712in}{0.655178in}}%
\pgfpathlineto{\pgfqpoint{5.265639in}{0.659861in}}%
\pgfpathlineto{\pgfqpoint{5.265976in}{0.654745in}}%
\pgfpathlineto{\pgfqpoint{5.266650in}{0.660853in}}%
\pgfpathlineto{\pgfqpoint{5.266819in}{0.659312in}}%
\pgfpathlineto{\pgfqpoint{5.267661in}{0.664537in}}%
\pgfpathlineto{\pgfqpoint{5.268082in}{0.654416in}}%
\pgfpathlineto{\pgfqpoint{5.269683in}{0.656177in}}%
\pgfpathlineto{\pgfqpoint{5.270188in}{0.655517in}}%
\pgfpathlineto{\pgfqpoint{5.270441in}{0.674406in}}%
\pgfpathlineto{\pgfqpoint{5.270778in}{0.738898in}}%
\pgfpathlineto{\pgfqpoint{5.271452in}{0.656776in}}%
\pgfpathlineto{\pgfqpoint{5.272210in}{0.654080in}}%
\pgfpathlineto{\pgfqpoint{5.272884in}{0.692236in}}%
\pgfpathlineto{\pgfqpoint{5.273053in}{0.701930in}}%
\pgfpathlineto{\pgfqpoint{5.273811in}{0.685422in}}%
\pgfpathlineto{\pgfqpoint{5.273895in}{0.683197in}}%
\pgfpathlineto{\pgfqpoint{5.274148in}{0.701405in}}%
\pgfpathlineto{\pgfqpoint{5.274485in}{0.775611in}}%
\pgfpathlineto{\pgfqpoint{5.274990in}{0.660112in}}%
\pgfpathlineto{\pgfqpoint{5.275159in}{0.653909in}}%
\pgfpathlineto{\pgfqpoint{5.276001in}{0.655713in}}%
\pgfpathlineto{\pgfqpoint{5.276422in}{0.766945in}}%
\pgfpathlineto{\pgfqpoint{5.277096in}{0.659109in}}%
\pgfpathlineto{\pgfqpoint{5.277265in}{0.661124in}}%
\pgfpathlineto{\pgfqpoint{5.277855in}{0.654838in}}%
\pgfpathlineto{\pgfqpoint{5.279540in}{0.653989in}}%
\pgfpathlineto{\pgfqpoint{5.279792in}{0.654748in}}%
\pgfpathlineto{\pgfqpoint{5.280550in}{0.660411in}}%
\pgfpathlineto{\pgfqpoint{5.280972in}{0.656336in}}%
\pgfpathlineto{\pgfqpoint{5.281814in}{0.654865in}}%
\pgfpathlineto{\pgfqpoint{5.282067in}{0.655646in}}%
\pgfpathlineto{\pgfqpoint{5.282572in}{0.665443in}}%
\pgfpathlineto{\pgfqpoint{5.282994in}{0.656105in}}%
\pgfpathlineto{\pgfqpoint{5.283331in}{0.652528in}}%
\pgfpathlineto{\pgfqpoint{5.284173in}{0.653452in}}%
\pgfpathlineto{\pgfqpoint{5.284594in}{0.652802in}}%
\pgfpathlineto{\pgfqpoint{5.285100in}{0.653771in}}%
\pgfpathlineto{\pgfqpoint{5.286532in}{0.664505in}}%
\pgfpathlineto{\pgfqpoint{5.285774in}{0.652750in}}%
\pgfpathlineto{\pgfqpoint{5.286785in}{0.657885in}}%
\pgfpathlineto{\pgfqpoint{5.287122in}{0.652939in}}%
\pgfpathlineto{\pgfqpoint{5.287374in}{0.661615in}}%
\pgfpathlineto{\pgfqpoint{5.287711in}{0.693329in}}%
\pgfpathlineto{\pgfqpoint{5.288385in}{0.652814in}}%
\pgfpathlineto{\pgfqpoint{5.289312in}{0.653635in}}%
\pgfpathlineto{\pgfqpoint{5.290407in}{0.658686in}}%
\pgfpathlineto{\pgfqpoint{5.290576in}{0.655475in}}%
\pgfpathlineto{\pgfqpoint{5.290744in}{0.653028in}}%
\pgfpathlineto{\pgfqpoint{5.291165in}{0.664194in}}%
\pgfpathlineto{\pgfqpoint{5.291586in}{0.657279in}}%
\pgfpathlineto{\pgfqpoint{5.291839in}{0.671614in}}%
\pgfpathlineto{\pgfqpoint{5.292260in}{0.799360in}}%
\pgfpathlineto{\pgfqpoint{5.292850in}{0.664826in}}%
\pgfpathlineto{\pgfqpoint{5.293271in}{0.657023in}}%
\pgfpathlineto{\pgfqpoint{5.293861in}{0.666815in}}%
\pgfpathlineto{\pgfqpoint{5.293945in}{0.668557in}}%
\pgfpathlineto{\pgfqpoint{5.294367in}{0.655076in}}%
\pgfpathlineto{\pgfqpoint{5.294535in}{0.653273in}}%
\pgfpathlineto{\pgfqpoint{5.294872in}{0.663969in}}%
\pgfpathlineto{\pgfqpoint{5.296051in}{0.763794in}}%
\pgfpathlineto{\pgfqpoint{5.295462in}{0.655036in}}%
\pgfpathlineto{\pgfqpoint{5.296220in}{0.709459in}}%
\pgfpathlineto{\pgfqpoint{5.296641in}{0.653707in}}%
\pgfpathlineto{\pgfqpoint{5.297484in}{0.654380in}}%
\pgfpathlineto{\pgfqpoint{5.297821in}{0.653134in}}%
\pgfpathlineto{\pgfqpoint{5.298663in}{0.653417in}}%
\pgfpathlineto{\pgfqpoint{5.300601in}{0.654858in}}%
\pgfpathlineto{\pgfqpoint{5.300853in}{0.660191in}}%
\pgfpathlineto{\pgfqpoint{5.301527in}{0.653837in}}%
\pgfpathlineto{\pgfqpoint{5.301612in}{0.653891in}}%
\pgfpathlineto{\pgfqpoint{5.302454in}{0.656516in}}%
\pgfpathlineto{\pgfqpoint{5.302791in}{0.670680in}}%
\pgfpathlineto{\pgfqpoint{5.303296in}{0.656442in}}%
\pgfpathlineto{\pgfqpoint{5.303633in}{0.662008in}}%
\pgfpathlineto{\pgfqpoint{5.303718in}{0.663104in}}%
\pgfpathlineto{\pgfqpoint{5.304139in}{0.656366in}}%
\pgfpathlineto{\pgfqpoint{5.304560in}{0.653462in}}%
\pgfpathlineto{\pgfqpoint{5.305318in}{0.654793in}}%
\pgfpathlineto{\pgfqpoint{5.305740in}{0.653977in}}%
\pgfpathlineto{\pgfqpoint{5.306498in}{0.654188in}}%
\pgfpathlineto{\pgfqpoint{5.307003in}{0.655538in}}%
\pgfpathlineto{\pgfqpoint{5.307340in}{0.656533in}}%
\pgfpathlineto{\pgfqpoint{5.307593in}{0.654096in}}%
\pgfpathlineto{\pgfqpoint{5.307761in}{0.653424in}}%
\pgfpathlineto{\pgfqpoint{5.308688in}{0.653909in}}%
\pgfpathlineto{\pgfqpoint{5.309278in}{0.658138in}}%
\pgfpathlineto{\pgfqpoint{5.309783in}{0.653774in}}%
\pgfpathlineto{\pgfqpoint{5.311721in}{0.656193in}}%
\pgfpathlineto{\pgfqpoint{5.311889in}{0.658552in}}%
\pgfpathlineto{\pgfqpoint{5.312395in}{0.654472in}}%
\pgfpathlineto{\pgfqpoint{5.312732in}{0.654693in}}%
\pgfpathlineto{\pgfqpoint{5.313069in}{0.653405in}}%
\pgfpathlineto{\pgfqpoint{5.313322in}{0.652768in}}%
\pgfpathlineto{\pgfqpoint{5.314164in}{0.653462in}}%
\pgfpathlineto{\pgfqpoint{5.316439in}{0.653862in}}%
\pgfpathlineto{\pgfqpoint{5.316860in}{0.689382in}}%
\pgfpathlineto{\pgfqpoint{5.317787in}{0.661114in}}%
\pgfpathlineto{\pgfqpoint{5.318629in}{0.653110in}}%
\pgfpathlineto{\pgfqpoint{5.319471in}{0.653903in}}%
\pgfpathlineto{\pgfqpoint{5.320230in}{0.657988in}}%
\pgfpathlineto{\pgfqpoint{5.320567in}{0.654091in}}%
\pgfpathlineto{\pgfqpoint{5.321156in}{0.652900in}}%
\pgfpathlineto{\pgfqpoint{5.321409in}{0.654081in}}%
\pgfpathlineto{\pgfqpoint{5.322083in}{0.687350in}}%
\pgfpathlineto{\pgfqpoint{5.322336in}{0.715415in}}%
\pgfpathlineto{\pgfqpoint{5.322757in}{0.684311in}}%
\pgfpathlineto{\pgfqpoint{5.323094in}{0.691596in}}%
\pgfpathlineto{\pgfqpoint{5.323515in}{0.660387in}}%
\pgfpathlineto{\pgfqpoint{5.324273in}{0.677867in}}%
\pgfpathlineto{\pgfqpoint{5.324695in}{1.054476in}}%
\pgfpathlineto{\pgfqpoint{5.325284in}{0.674730in}}%
\pgfpathlineto{\pgfqpoint{5.326379in}{0.655431in}}%
\pgfpathlineto{\pgfqpoint{5.325874in}{0.679550in}}%
\pgfpathlineto{\pgfqpoint{5.326464in}{0.655971in}}%
\pgfpathlineto{\pgfqpoint{5.326885in}{0.680088in}}%
\pgfpathlineto{\pgfqpoint{5.327643in}{0.660729in}}%
\pgfpathlineto{\pgfqpoint{5.328317in}{0.666782in}}%
\pgfpathlineto{\pgfqpoint{5.328823in}{0.661690in}}%
\pgfpathlineto{\pgfqpoint{5.330002in}{0.654410in}}%
\pgfpathlineto{\pgfqpoint{5.330086in}{0.654933in}}%
\pgfpathlineto{\pgfqpoint{5.330339in}{0.714269in}}%
\pgfpathlineto{\pgfqpoint{5.330507in}{0.772346in}}%
\pgfpathlineto{\pgfqpoint{5.331181in}{0.662381in}}%
\pgfpathlineto{\pgfqpoint{5.331350in}{0.670415in}}%
\pgfpathlineto{\pgfqpoint{5.331518in}{0.678144in}}%
\pgfpathlineto{\pgfqpoint{5.332108in}{0.654134in}}%
\pgfpathlineto{\pgfqpoint{5.332192in}{0.654198in}}%
\pgfpathlineto{\pgfqpoint{5.333372in}{0.660164in}}%
\pgfpathlineto{\pgfqpoint{5.333625in}{0.656589in}}%
\pgfpathlineto{\pgfqpoint{5.334298in}{0.653654in}}%
\pgfpathlineto{\pgfqpoint{5.334804in}{0.654044in}}%
\pgfpathlineto{\pgfqpoint{5.336405in}{0.658187in}}%
\pgfpathlineto{\pgfqpoint{5.336657in}{0.656337in}}%
\pgfpathlineto{\pgfqpoint{5.338174in}{0.652666in}}%
\pgfpathlineto{\pgfqpoint{5.338426in}{0.652922in}}%
\pgfpathlineto{\pgfqpoint{5.338511in}{0.653653in}}%
\pgfpathlineto{\pgfqpoint{5.339353in}{0.652830in}}%
\pgfpathlineto{\pgfqpoint{5.339943in}{0.660306in}}%
\pgfpathlineto{\pgfqpoint{5.340448in}{0.652724in}}%
\pgfpathlineto{\pgfqpoint{5.341375in}{0.653603in}}%
\pgfpathlineto{\pgfqpoint{5.341543in}{0.654691in}}%
\pgfpathlineto{\pgfqpoint{5.342049in}{0.652946in}}%
\pgfpathlineto{\pgfqpoint{5.342554in}{0.654180in}}%
\pgfpathlineto{\pgfqpoint{5.343397in}{0.653327in}}%
\pgfpathlineto{\pgfqpoint{5.343902in}{0.653924in}}%
\pgfpathlineto{\pgfqpoint{5.345082in}{0.654646in}}%
\pgfpathlineto{\pgfqpoint{5.345334in}{0.654419in}}%
\pgfpathlineto{\pgfqpoint{5.346514in}{0.655324in}}%
\pgfpathlineto{\pgfqpoint{5.346767in}{0.655990in}}%
\pgfpathlineto{\pgfqpoint{5.347441in}{0.654632in}}%
\pgfpathlineto{\pgfqpoint{5.348452in}{0.653117in}}%
\pgfpathlineto{\pgfqpoint{5.348620in}{0.653942in}}%
\pgfpathlineto{\pgfqpoint{5.348957in}{0.661052in}}%
\pgfpathlineto{\pgfqpoint{5.349631in}{0.652395in}}%
\pgfpathlineto{\pgfqpoint{5.349968in}{0.653718in}}%
\pgfpathlineto{\pgfqpoint{5.350473in}{0.674485in}}%
\pgfpathlineto{\pgfqpoint{5.351316in}{0.666140in}}%
\pgfpathlineto{\pgfqpoint{5.351906in}{0.652236in}}%
\pgfpathlineto{\pgfqpoint{5.352411in}{0.661971in}}%
\pgfpathlineto{\pgfqpoint{5.352748in}{0.762713in}}%
\pgfpathlineto{\pgfqpoint{5.353422in}{0.657547in}}%
\pgfpathlineto{\pgfqpoint{5.353590in}{0.659733in}}%
\pgfpathlineto{\pgfqpoint{5.353843in}{0.666655in}}%
\pgfpathlineto{\pgfqpoint{5.354349in}{0.653030in}}%
\pgfpathlineto{\pgfqpoint{5.354517in}{0.653978in}}%
\pgfpathlineto{\pgfqpoint{5.355360in}{0.660086in}}%
\pgfpathlineto{\pgfqpoint{5.354938in}{0.653588in}}%
\pgfpathlineto{\pgfqpoint{5.355612in}{0.654600in}}%
\pgfpathlineto{\pgfqpoint{5.355781in}{0.652755in}}%
\pgfpathlineto{\pgfqpoint{5.356455in}{0.654668in}}%
\pgfpathlineto{\pgfqpoint{5.356876in}{0.707118in}}%
\pgfpathlineto{\pgfqpoint{5.357718in}{0.666265in}}%
\pgfpathlineto{\pgfqpoint{5.358055in}{0.652742in}}%
\pgfpathlineto{\pgfqpoint{5.358982in}{0.654189in}}%
\pgfpathlineto{\pgfqpoint{5.359319in}{0.668795in}}%
\pgfpathlineto{\pgfqpoint{5.359825in}{0.652820in}}%
\pgfpathlineto{\pgfqpoint{5.360414in}{0.660829in}}%
\pgfpathlineto{\pgfqpoint{5.360667in}{0.652893in}}%
\pgfpathlineto{\pgfqpoint{5.360920in}{0.674319in}}%
\pgfpathlineto{\pgfqpoint{5.361172in}{0.752752in}}%
\pgfpathlineto{\pgfqpoint{5.361762in}{0.653689in}}%
\pgfpathlineto{\pgfqpoint{5.361931in}{0.655526in}}%
\pgfpathlineto{\pgfqpoint{5.362268in}{0.667069in}}%
\pgfpathlineto{\pgfqpoint{5.362773in}{0.652155in}}%
\pgfpathlineto{\pgfqpoint{5.362857in}{0.652194in}}%
\pgfpathlineto{\pgfqpoint{5.363110in}{0.652902in}}%
\pgfpathlineto{\pgfqpoint{5.364626in}{0.681267in}}%
\pgfpathlineto{\pgfqpoint{5.364879in}{0.666633in}}%
\pgfpathlineto{\pgfqpoint{5.365216in}{0.651192in}}%
\pgfpathlineto{\pgfqpoint{5.366059in}{0.652726in}}%
\pgfpathlineto{\pgfqpoint{5.366311in}{0.651982in}}%
\pgfpathlineto{\pgfqpoint{5.366648in}{0.654581in}}%
\pgfpathlineto{\pgfqpoint{5.366901in}{0.658807in}}%
\pgfpathlineto{\pgfqpoint{5.367407in}{0.651282in}}%
\pgfpathlineto{\pgfqpoint{5.367575in}{0.651485in}}%
\pgfpathlineto{\pgfqpoint{5.369428in}{0.655650in}}%
\pgfpathlineto{\pgfqpoint{5.370776in}{0.690342in}}%
\pgfpathlineto{\pgfqpoint{5.371029in}{0.663217in}}%
\pgfpathlineto{\pgfqpoint{5.371366in}{0.651776in}}%
\pgfpathlineto{\pgfqpoint{5.372209in}{0.652653in}}%
\pgfpathlineto{\pgfqpoint{5.372377in}{0.653228in}}%
\pgfpathlineto{\pgfqpoint{5.373219in}{0.652446in}}%
\pgfpathlineto{\pgfqpoint{5.373472in}{0.652567in}}%
\pgfpathlineto{\pgfqpoint{5.373641in}{0.653226in}}%
\pgfpathlineto{\pgfqpoint{5.373978in}{0.653088in}}%
\pgfpathlineto{\pgfqpoint{5.374230in}{0.656386in}}%
\pgfpathlineto{\pgfqpoint{5.374652in}{0.687885in}}%
\pgfpathlineto{\pgfqpoint{5.375326in}{0.655715in}}%
\pgfpathlineto{\pgfqpoint{5.376168in}{0.656332in}}%
\pgfpathlineto{\pgfqpoint{5.376673in}{0.652528in}}%
\pgfpathlineto{\pgfqpoint{5.376926in}{0.653676in}}%
\pgfpathlineto{\pgfqpoint{5.377263in}{0.709054in}}%
\pgfpathlineto{\pgfqpoint{5.378443in}{0.784118in}}%
\pgfpathlineto{\pgfqpoint{5.377937in}{0.669790in}}%
\pgfpathlineto{\pgfqpoint{5.378611in}{0.765588in}}%
\pgfpathlineto{\pgfqpoint{5.380043in}{0.656571in}}%
\pgfpathlineto{\pgfqpoint{5.380127in}{0.656641in}}%
\pgfpathlineto{\pgfqpoint{5.380380in}{0.657467in}}%
\pgfpathlineto{\pgfqpoint{5.380717in}{0.655131in}}%
\pgfpathlineto{\pgfqpoint{5.380886in}{0.654155in}}%
\pgfpathlineto{\pgfqpoint{5.381391in}{0.658556in}}%
\pgfpathlineto{\pgfqpoint{5.381560in}{0.660093in}}%
\pgfpathlineto{\pgfqpoint{5.382318in}{0.657351in}}%
\pgfpathlineto{\pgfqpoint{5.382402in}{0.657100in}}%
\pgfpathlineto{\pgfqpoint{5.382571in}{0.658504in}}%
\pgfpathlineto{\pgfqpoint{5.383160in}{0.670917in}}%
\pgfpathlineto{\pgfqpoint{5.383582in}{0.655906in}}%
\pgfpathlineto{\pgfqpoint{5.383834in}{0.653331in}}%
\pgfpathlineto{\pgfqpoint{5.384592in}{0.657839in}}%
\pgfpathlineto{\pgfqpoint{5.386025in}{0.653559in}}%
\pgfpathlineto{\pgfqpoint{5.386446in}{0.655740in}}%
\pgfpathlineto{\pgfqpoint{5.387036in}{0.658529in}}%
\pgfpathlineto{\pgfqpoint{5.387288in}{0.654267in}}%
\pgfpathlineto{\pgfqpoint{5.388468in}{0.652456in}}%
\pgfpathlineto{\pgfqpoint{5.388720in}{0.652528in}}%
\pgfpathlineto{\pgfqpoint{5.388805in}{0.652900in}}%
\pgfpathlineto{\pgfqpoint{5.389142in}{0.655963in}}%
\pgfpathlineto{\pgfqpoint{5.389900in}{0.653230in}}%
\pgfpathlineto{\pgfqpoint{5.390237in}{0.654212in}}%
\pgfpathlineto{\pgfqpoint{5.390574in}{0.665254in}}%
\pgfpathlineto{\pgfqpoint{5.391164in}{0.653086in}}%
\pgfpathlineto{\pgfqpoint{5.391416in}{0.651099in}}%
\pgfpathlineto{\pgfqpoint{5.391837in}{0.656974in}}%
\pgfpathlineto{\pgfqpoint{5.392343in}{0.651947in}}%
\pgfpathlineto{\pgfqpoint{5.392764in}{0.691271in}}%
\pgfpathlineto{\pgfqpoint{5.393691in}{0.675418in}}%
\pgfpathlineto{\pgfqpoint{5.394028in}{0.650588in}}%
\pgfpathlineto{\pgfqpoint{5.394955in}{0.656864in}}%
\pgfpathlineto{\pgfqpoint{5.395123in}{0.661269in}}%
\pgfpathlineto{\pgfqpoint{5.395628in}{0.649272in}}%
\pgfpathlineto{\pgfqpoint{5.395965in}{0.655593in}}%
\pgfpathlineto{\pgfqpoint{5.396302in}{0.647896in}}%
\pgfpathlineto{\pgfqpoint{5.397482in}{0.649761in}}%
\pgfpathlineto{\pgfqpoint{5.397903in}{0.651793in}}%
\pgfpathlineto{\pgfqpoint{5.398577in}{0.650735in}}%
\pgfpathlineto{\pgfqpoint{5.398830in}{0.688665in}}%
\pgfpathlineto{\pgfqpoint{5.399083in}{0.788085in}}%
\pgfpathlineto{\pgfqpoint{5.399504in}{0.652565in}}%
\pgfpathlineto{\pgfqpoint{5.400009in}{0.738439in}}%
\pgfpathlineto{\pgfqpoint{5.401189in}{0.652642in}}%
\pgfpathlineto{\pgfqpoint{5.401273in}{0.652688in}}%
\pgfpathlineto{\pgfqpoint{5.402115in}{0.654628in}}%
\pgfpathlineto{\pgfqpoint{5.402368in}{0.658619in}}%
\pgfpathlineto{\pgfqpoint{5.402789in}{0.653463in}}%
\pgfpathlineto{\pgfqpoint{5.403126in}{0.653833in}}%
\pgfpathlineto{\pgfqpoint{5.403379in}{0.653420in}}%
\pgfpathlineto{\pgfqpoint{5.403632in}{0.655410in}}%
\pgfpathlineto{\pgfqpoint{5.404643in}{0.736117in}}%
\pgfpathlineto{\pgfqpoint{5.404727in}{0.756950in}}%
\pgfpathlineto{\pgfqpoint{5.405232in}{0.653585in}}%
\pgfpathlineto{\pgfqpoint{5.405401in}{0.654471in}}%
\pgfpathlineto{\pgfqpoint{5.406412in}{0.688997in}}%
\pgfpathlineto{\pgfqpoint{5.405906in}{0.653431in}}%
\pgfpathlineto{\pgfqpoint{5.406749in}{0.664307in}}%
\pgfpathlineto{\pgfqpoint{5.407170in}{0.656362in}}%
\pgfpathlineto{\pgfqpoint{5.407928in}{0.660583in}}%
\pgfpathlineto{\pgfqpoint{5.408602in}{0.653158in}}%
\pgfpathlineto{\pgfqpoint{5.409360in}{0.658490in}}%
\pgfpathlineto{\pgfqpoint{5.409529in}{0.656724in}}%
\pgfpathlineto{\pgfqpoint{5.409782in}{0.653819in}}%
\pgfpathlineto{\pgfqpoint{5.409950in}{0.661412in}}%
\pgfpathlineto{\pgfqpoint{5.410287in}{0.750333in}}%
\pgfpathlineto{\pgfqpoint{5.410877in}{0.655021in}}%
\pgfpathlineto{\pgfqpoint{5.410961in}{0.655418in}}%
\pgfpathlineto{\pgfqpoint{5.411214in}{0.658036in}}%
\pgfpathlineto{\pgfqpoint{5.411635in}{0.653738in}}%
\pgfpathlineto{\pgfqpoint{5.411972in}{0.654575in}}%
\pgfpathlineto{\pgfqpoint{5.412309in}{0.654116in}}%
\pgfpathlineto{\pgfqpoint{5.412477in}{0.654716in}}%
\pgfpathlineto{\pgfqpoint{5.412814in}{0.659813in}}%
\pgfpathlineto{\pgfqpoint{5.413320in}{0.653554in}}%
\pgfpathlineto{\pgfqpoint{5.413488in}{0.653573in}}%
\pgfpathlineto{\pgfqpoint{5.415847in}{0.655043in}}%
\pgfpathlineto{\pgfqpoint{5.416184in}{0.660695in}}%
\pgfpathlineto{\pgfqpoint{5.416690in}{0.653615in}}%
\pgfpathlineto{\pgfqpoint{5.416858in}{0.653847in}}%
\pgfpathlineto{\pgfqpoint{5.418206in}{0.673358in}}%
\pgfpathlineto{\pgfqpoint{5.418543in}{0.660758in}}%
\pgfpathlineto{\pgfqpoint{5.419638in}{0.653134in}}%
\pgfpathlineto{\pgfqpoint{5.419722in}{0.653151in}}%
\pgfpathlineto{\pgfqpoint{5.419891in}{0.654313in}}%
\pgfpathlineto{\pgfqpoint{5.420059in}{0.658705in}}%
\pgfpathlineto{\pgfqpoint{5.420565in}{0.653128in}}%
\pgfpathlineto{\pgfqpoint{5.420902in}{0.653357in}}%
\pgfpathlineto{\pgfqpoint{5.421744in}{0.654675in}}%
\pgfpathlineto{\pgfqpoint{5.422166in}{0.653333in}}%
\pgfpathlineto{\pgfqpoint{5.423766in}{0.653920in}}%
\pgfpathlineto{\pgfqpoint{5.424524in}{0.656798in}}%
\pgfpathlineto{\pgfqpoint{5.424946in}{0.687700in}}%
\pgfpathlineto{\pgfqpoint{5.425535in}{0.656889in}}%
\pgfpathlineto{\pgfqpoint{5.426378in}{0.655257in}}%
\pgfpathlineto{\pgfqpoint{5.426799in}{0.655551in}}%
\pgfpathlineto{\pgfqpoint{5.427136in}{0.659070in}}%
\pgfpathlineto{\pgfqpoint{5.428989in}{0.691779in}}%
\pgfpathlineto{\pgfqpoint{5.429326in}{0.689104in}}%
\pgfpathlineto{\pgfqpoint{5.430927in}{0.650666in}}%
\pgfpathlineto{\pgfqpoint{5.431348in}{0.663566in}}%
\pgfpathlineto{\pgfqpoint{5.432443in}{0.702544in}}%
\pgfpathlineto{\pgfqpoint{5.431938in}{0.657715in}}%
\pgfpathlineto{\pgfqpoint{5.432528in}{0.693750in}}%
\pgfpathlineto{\pgfqpoint{5.432949in}{0.649988in}}%
\pgfpathlineto{\pgfqpoint{5.433791in}{0.650945in}}%
\pgfpathlineto{\pgfqpoint{5.434212in}{0.669874in}}%
\pgfpathlineto{\pgfqpoint{5.434634in}{0.650457in}}%
\pgfpathlineto{\pgfqpoint{5.435055in}{0.656124in}}%
\pgfpathlineto{\pgfqpoint{5.435392in}{0.650676in}}%
\pgfpathlineto{\pgfqpoint{5.435982in}{0.655700in}}%
\pgfpathlineto{\pgfqpoint{5.436319in}{0.672497in}}%
\pgfpathlineto{\pgfqpoint{5.436993in}{0.652496in}}%
\pgfpathlineto{\pgfqpoint{5.437498in}{0.653905in}}%
\pgfpathlineto{\pgfqpoint{5.437835in}{0.657645in}}%
\pgfpathlineto{\pgfqpoint{5.438677in}{0.655093in}}%
\pgfpathlineto{\pgfqpoint{5.439267in}{0.654152in}}%
\pgfpathlineto{\pgfqpoint{5.439604in}{0.651285in}}%
\pgfpathlineto{\pgfqpoint{5.440110in}{0.656047in}}%
\pgfpathlineto{\pgfqpoint{5.440531in}{0.685126in}}%
\pgfpathlineto{\pgfqpoint{5.441373in}{0.667462in}}%
\pgfpathlineto{\pgfqpoint{5.441710in}{0.663109in}}%
\pgfpathlineto{\pgfqpoint{5.441963in}{0.667523in}}%
\pgfpathlineto{\pgfqpoint{5.442300in}{0.690290in}}%
\pgfpathlineto{\pgfqpoint{5.442637in}{0.657446in}}%
\pgfpathlineto{\pgfqpoint{5.442890in}{0.644728in}}%
\pgfpathlineto{\pgfqpoint{5.443311in}{0.676700in}}%
\pgfpathlineto{\pgfqpoint{5.443648in}{0.650676in}}%
\pgfpathlineto{\pgfqpoint{5.443901in}{0.770825in}}%
\pgfpathlineto{\pgfqpoint{5.444069in}{0.937344in}}%
\pgfpathlineto{\pgfqpoint{5.444912in}{0.653298in}}%
\pgfpathlineto{\pgfqpoint{5.445249in}{0.657393in}}%
\pgfpathlineto{\pgfqpoint{5.445838in}{0.651768in}}%
\pgfpathlineto{\pgfqpoint{5.446259in}{0.651989in}}%
\pgfpathlineto{\pgfqpoint{5.446428in}{0.652581in}}%
\pgfpathlineto{\pgfqpoint{5.447523in}{0.667103in}}%
\pgfpathlineto{\pgfqpoint{5.447692in}{0.680758in}}%
\pgfpathlineto{\pgfqpoint{5.448197in}{0.652071in}}%
\pgfpathlineto{\pgfqpoint{5.448450in}{0.652837in}}%
\pgfpathlineto{\pgfqpoint{5.448787in}{0.652528in}}%
\pgfpathlineto{\pgfqpoint{5.449966in}{0.657704in}}%
\pgfpathlineto{\pgfqpoint{5.450135in}{0.663192in}}%
\pgfpathlineto{\pgfqpoint{5.450640in}{0.652678in}}%
\pgfpathlineto{\pgfqpoint{5.450977in}{0.654664in}}%
\pgfpathlineto{\pgfqpoint{5.451314in}{0.652912in}}%
\pgfpathlineto{\pgfqpoint{5.452241in}{0.653435in}}%
\pgfpathlineto{\pgfqpoint{5.452831in}{0.653070in}}%
\pgfpathlineto{\pgfqpoint{5.453167in}{0.659093in}}%
\pgfpathlineto{\pgfqpoint{5.453336in}{0.663077in}}%
\pgfpathlineto{\pgfqpoint{5.453757in}{0.653322in}}%
\pgfpathlineto{\pgfqpoint{5.454178in}{0.659586in}}%
\pgfpathlineto{\pgfqpoint{5.454600in}{0.653135in}}%
\pgfpathlineto{\pgfqpoint{5.455526in}{0.653606in}}%
\pgfpathlineto{\pgfqpoint{5.456032in}{0.654467in}}%
\pgfpathlineto{\pgfqpoint{5.457043in}{0.660178in}}%
\pgfpathlineto{\pgfqpoint{5.456537in}{0.653868in}}%
\pgfpathlineto{\pgfqpoint{5.457211in}{0.655969in}}%
\pgfpathlineto{\pgfqpoint{5.457380in}{0.653942in}}%
\pgfpathlineto{\pgfqpoint{5.457717in}{0.666825in}}%
\pgfpathlineto{\pgfqpoint{5.457885in}{0.673634in}}%
\pgfpathlineto{\pgfqpoint{5.458306in}{0.654377in}}%
\pgfpathlineto{\pgfqpoint{5.458728in}{0.663494in}}%
\pgfpathlineto{\pgfqpoint{5.458812in}{0.664632in}}%
\pgfpathlineto{\pgfqpoint{5.459317in}{0.656513in}}%
\pgfpathlineto{\pgfqpoint{5.460160in}{0.654406in}}%
\pgfpathlineto{\pgfqpoint{5.460497in}{0.655260in}}%
\pgfpathlineto{\pgfqpoint{5.461423in}{0.665809in}}%
\pgfpathlineto{\pgfqpoint{5.461845in}{0.751787in}}%
\pgfpathlineto{\pgfqpoint{5.462434in}{0.653062in}}%
\pgfpathlineto{\pgfqpoint{5.463108in}{0.652977in}}%
\pgfpathlineto{\pgfqpoint{5.462771in}{0.653659in}}%
\pgfpathlineto{\pgfqpoint{5.463193in}{0.653441in}}%
\pgfpathlineto{\pgfqpoint{5.464709in}{0.666566in}}%
\pgfpathlineto{\pgfqpoint{5.464877in}{0.662941in}}%
\pgfpathlineto{\pgfqpoint{5.465383in}{0.652575in}}%
\pgfpathlineto{\pgfqpoint{5.466141in}{0.653860in}}%
\pgfpathlineto{\pgfqpoint{5.466562in}{0.659162in}}%
\pgfpathlineto{\pgfqpoint{5.467405in}{0.656263in}}%
\pgfpathlineto{\pgfqpoint{5.467573in}{0.656337in}}%
\pgfpathlineto{\pgfqpoint{5.467995in}{0.655189in}}%
\pgfpathlineto{\pgfqpoint{5.468753in}{0.654052in}}%
\pgfpathlineto{\pgfqpoint{5.468921in}{0.655810in}}%
\pgfpathlineto{\pgfqpoint{5.469342in}{0.669248in}}%
\pgfpathlineto{\pgfqpoint{5.469848in}{0.651384in}}%
\pgfpathlineto{\pgfqpoint{5.470101in}{0.654107in}}%
\pgfpathlineto{\pgfqpoint{5.471364in}{0.694633in}}%
\pgfpathlineto{\pgfqpoint{5.470859in}{0.652864in}}%
\pgfpathlineto{\pgfqpoint{5.471533in}{0.669728in}}%
\pgfpathlineto{\pgfqpoint{5.471786in}{0.651201in}}%
\pgfpathlineto{\pgfqpoint{5.472207in}{0.689485in}}%
\pgfpathlineto{\pgfqpoint{5.472712in}{0.651476in}}%
\pgfpathlineto{\pgfqpoint{5.472965in}{0.694357in}}%
\pgfpathlineto{\pgfqpoint{5.473133in}{0.731489in}}%
\pgfpathlineto{\pgfqpoint{5.473639in}{0.650607in}}%
\pgfpathlineto{\pgfqpoint{5.473976in}{0.686110in}}%
\pgfpathlineto{\pgfqpoint{5.474060in}{0.686424in}}%
\pgfpathlineto{\pgfqpoint{5.474566in}{0.650412in}}%
\pgfpathlineto{\pgfqpoint{5.475071in}{0.693596in}}%
\pgfpathlineto{\pgfqpoint{5.475240in}{0.708280in}}%
\pgfpathlineto{\pgfqpoint{5.475914in}{0.662598in}}%
\pgfpathlineto{\pgfqpoint{5.477261in}{0.651363in}}%
\pgfpathlineto{\pgfqpoint{5.477598in}{0.651203in}}%
\pgfpathlineto{\pgfqpoint{5.477767in}{0.652020in}}%
\pgfpathlineto{\pgfqpoint{5.479115in}{0.684349in}}%
\pgfpathlineto{\pgfqpoint{5.478609in}{0.650299in}}%
\pgfpathlineto{\pgfqpoint{5.479283in}{0.661860in}}%
\pgfpathlineto{\pgfqpoint{5.479452in}{0.651574in}}%
\pgfpathlineto{\pgfqpoint{5.479620in}{0.675679in}}%
\pgfpathlineto{\pgfqpoint{5.479957in}{1.056473in}}%
\pgfpathlineto{\pgfqpoint{5.480547in}{0.657318in}}%
\pgfpathlineto{\pgfqpoint{5.480715in}{0.671949in}}%
\pgfpathlineto{\pgfqpoint{5.480968in}{0.731918in}}%
\pgfpathlineto{\pgfqpoint{5.481558in}{0.654209in}}%
\pgfpathlineto{\pgfqpoint{5.481726in}{0.655209in}}%
\pgfpathlineto{\pgfqpoint{5.481895in}{0.658500in}}%
\pgfpathlineto{\pgfqpoint{5.482737in}{0.654119in}}%
\pgfpathlineto{\pgfqpoint{5.483327in}{0.655046in}}%
\pgfpathlineto{\pgfqpoint{5.483496in}{0.656035in}}%
\pgfpathlineto{\pgfqpoint{5.484085in}{0.654106in}}%
\pgfpathlineto{\pgfqpoint{5.484338in}{0.654290in}}%
\pgfpathlineto{\pgfqpoint{5.484675in}{0.656711in}}%
\pgfpathlineto{\pgfqpoint{5.485180in}{0.654089in}}%
\pgfpathlineto{\pgfqpoint{5.485602in}{0.655071in}}%
\pgfpathlineto{\pgfqpoint{5.487455in}{0.654421in}}%
\pgfpathlineto{\pgfqpoint{5.488382in}{0.655802in}}%
\pgfpathlineto{\pgfqpoint{5.488634in}{0.661681in}}%
\pgfpathlineto{\pgfqpoint{5.489140in}{0.654883in}}%
\pgfpathlineto{\pgfqpoint{5.489561in}{0.658151in}}%
\pgfpathlineto{\pgfqpoint{5.489645in}{0.658318in}}%
\pgfpathlineto{\pgfqpoint{5.489898in}{0.656768in}}%
\pgfpathlineto{\pgfqpoint{5.490319in}{0.654618in}}%
\pgfpathlineto{\pgfqpoint{5.490909in}{0.656500in}}%
\pgfpathlineto{\pgfqpoint{5.490993in}{0.657073in}}%
\pgfpathlineto{\pgfqpoint{5.491415in}{0.655118in}}%
\pgfpathlineto{\pgfqpoint{5.491836in}{0.655300in}}%
\pgfpathlineto{\pgfqpoint{5.492257in}{0.654943in}}%
\pgfpathlineto{\pgfqpoint{5.492931in}{0.655334in}}%
\pgfpathlineto{\pgfqpoint{5.493858in}{0.658952in}}%
\pgfpathlineto{\pgfqpoint{5.494363in}{0.668504in}}%
\pgfpathlineto{\pgfqpoint{5.494784in}{0.657327in}}%
\pgfpathlineto{\pgfqpoint{5.495879in}{0.654718in}}%
\pgfpathlineto{\pgfqpoint{5.495964in}{0.654764in}}%
\pgfpathlineto{\pgfqpoint{5.496216in}{0.658586in}}%
\pgfpathlineto{\pgfqpoint{5.497396in}{0.693156in}}%
\pgfpathlineto{\pgfqpoint{5.496806in}{0.655603in}}%
\pgfpathlineto{\pgfqpoint{5.497649in}{0.672691in}}%
\pgfpathlineto{\pgfqpoint{5.498154in}{0.652281in}}%
\pgfpathlineto{\pgfqpoint{5.498912in}{0.652308in}}%
\pgfpathlineto{\pgfqpoint{5.499249in}{0.653354in}}%
\pgfpathlineto{\pgfqpoint{5.500344in}{0.686721in}}%
\pgfpathlineto{\pgfqpoint{5.500681in}{0.789766in}}%
\pgfpathlineto{\pgfqpoint{5.501355in}{0.654504in}}%
\pgfpathlineto{\pgfqpoint{5.501440in}{0.654189in}}%
\pgfpathlineto{\pgfqpoint{5.501608in}{0.656522in}}%
\pgfpathlineto{\pgfqpoint{5.501861in}{0.661269in}}%
\pgfpathlineto{\pgfqpoint{5.502282in}{0.652943in}}%
\pgfpathlineto{\pgfqpoint{5.502619in}{0.653911in}}%
\pgfpathlineto{\pgfqpoint{5.503714in}{0.654569in}}%
\pgfpathlineto{\pgfqpoint{5.503293in}{0.653215in}}%
\pgfpathlineto{\pgfqpoint{5.503798in}{0.654525in}}%
\pgfpathlineto{\pgfqpoint{5.504472in}{0.652685in}}%
\pgfpathlineto{\pgfqpoint{5.505231in}{0.653022in}}%
\pgfpathlineto{\pgfqpoint{5.507084in}{0.654199in}}%
\pgfpathlineto{\pgfqpoint{5.507421in}{0.660611in}}%
\pgfpathlineto{\pgfqpoint{5.507926in}{0.653927in}}%
\pgfpathlineto{\pgfqpoint{5.508432in}{0.658435in}}%
\pgfpathlineto{\pgfqpoint{5.509864in}{0.653742in}}%
\pgfpathlineto{\pgfqpoint{5.511380in}{0.654713in}}%
\pgfpathlineto{\pgfqpoint{5.511970in}{0.654590in}}%
\pgfpathlineto{\pgfqpoint{5.512223in}{0.660879in}}%
\pgfpathlineto{\pgfqpoint{5.512981in}{0.654821in}}%
\pgfpathlineto{\pgfqpoint{5.513571in}{0.685514in}}%
\pgfpathlineto{\pgfqpoint{5.513655in}{0.689360in}}%
\pgfpathlineto{\pgfqpoint{5.513992in}{0.661668in}}%
\pgfpathlineto{\pgfqpoint{5.514245in}{0.655100in}}%
\pgfpathlineto{\pgfqpoint{5.514666in}{0.680749in}}%
\pgfpathlineto{\pgfqpoint{5.514834in}{0.690305in}}%
\pgfpathlineto{\pgfqpoint{5.515340in}{0.654636in}}%
\pgfpathlineto{\pgfqpoint{5.515424in}{0.654662in}}%
\pgfpathlineto{\pgfqpoint{5.515845in}{0.667669in}}%
\pgfpathlineto{\pgfqpoint{5.516435in}{0.654255in}}%
\pgfpathlineto{\pgfqpoint{5.516856in}{0.655170in}}%
\pgfpathlineto{\pgfqpoint{5.517193in}{0.657146in}}%
\pgfpathlineto{\pgfqpoint{5.517783in}{0.653895in}}%
\pgfpathlineto{\pgfqpoint{5.517952in}{0.654375in}}%
\pgfpathlineto{\pgfqpoint{5.518625in}{0.671455in}}%
\pgfpathlineto{\pgfqpoint{5.519047in}{0.655465in}}%
\pgfpathlineto{\pgfqpoint{5.520310in}{0.653714in}}%
\pgfpathlineto{\pgfqpoint{5.519468in}{0.657668in}}%
\pgfpathlineto{\pgfqpoint{5.520563in}{0.653845in}}%
\pgfpathlineto{\pgfqpoint{5.521911in}{0.657304in}}%
\pgfpathlineto{\pgfqpoint{5.522080in}{0.655004in}}%
\pgfpathlineto{\pgfqpoint{5.523090in}{0.653481in}}%
\pgfpathlineto{\pgfqpoint{5.522753in}{0.655999in}}%
\pgfpathlineto{\pgfqpoint{5.523259in}{0.653552in}}%
\pgfpathlineto{\pgfqpoint{5.524438in}{0.654630in}}%
\pgfpathlineto{\pgfqpoint{5.524860in}{0.653844in}}%
\pgfpathlineto{\pgfqpoint{5.528566in}{0.655625in}}%
\pgfpathlineto{\pgfqpoint{5.528988in}{0.668141in}}%
\pgfpathlineto{\pgfqpoint{5.529662in}{0.656059in}}%
\pgfpathlineto{\pgfqpoint{5.529830in}{0.655770in}}%
\pgfpathlineto{\pgfqpoint{5.530420in}{0.656667in}}%
\pgfpathlineto{\pgfqpoint{5.530925in}{0.690685in}}%
\pgfpathlineto{\pgfqpoint{5.531346in}{0.655307in}}%
\pgfpathlineto{\pgfqpoint{5.531515in}{0.653965in}}%
\pgfpathlineto{\pgfqpoint{5.531768in}{0.661666in}}%
\pgfpathlineto{\pgfqpoint{5.532357in}{0.653750in}}%
\pgfpathlineto{\pgfqpoint{5.532779in}{0.937187in}}%
\pgfpathlineto{\pgfqpoint{5.532863in}{0.989052in}}%
\pgfpathlineto{\pgfqpoint{5.533368in}{0.671396in}}%
\pgfpathlineto{\pgfqpoint{5.534548in}{0.655027in}}%
\pgfpathlineto{\pgfqpoint{5.533958in}{0.688208in}}%
\pgfpathlineto{\pgfqpoint{5.534632in}{0.655101in}}%
\pgfpathlineto{\pgfqpoint{5.535896in}{0.655330in}}%
\pgfpathlineto{\pgfqpoint{5.536317in}{0.655938in}}%
\pgfpathlineto{\pgfqpoint{5.536738in}{0.668093in}}%
\pgfpathlineto{\pgfqpoint{5.537328in}{0.655637in}}%
\pgfpathlineto{\pgfqpoint{5.537412in}{0.655730in}}%
\pgfpathlineto{\pgfqpoint{5.537749in}{0.656533in}}%
\pgfpathlineto{\pgfqpoint{5.538170in}{0.655124in}}%
\pgfpathlineto{\pgfqpoint{5.538339in}{0.655111in}}%
\pgfpathlineto{\pgfqpoint{5.539687in}{0.655559in}}%
\pgfpathlineto{\pgfqpoint{5.540108in}{0.665482in}}%
\pgfpathlineto{\pgfqpoint{5.540698in}{0.655037in}}%
\pgfpathlineto{\pgfqpoint{5.540866in}{0.656724in}}%
\pgfpathlineto{\pgfqpoint{5.542130in}{0.699058in}}%
\pgfpathlineto{\pgfqpoint{5.541708in}{0.654925in}}%
\pgfpathlineto{\pgfqpoint{5.542298in}{0.682595in}}%
\pgfpathlineto{\pgfqpoint{5.542635in}{0.654616in}}%
\pgfpathlineto{\pgfqpoint{5.543562in}{0.655643in}}%
\pgfpathlineto{\pgfqpoint{5.543983in}{0.653918in}}%
\pgfpathlineto{\pgfqpoint{5.544657in}{0.654868in}}%
\pgfpathlineto{\pgfqpoint{5.544994in}{0.656878in}}%
\pgfpathlineto{\pgfqpoint{5.546258in}{0.670834in}}%
\pgfpathlineto{\pgfqpoint{5.545752in}{0.654834in}}%
\pgfpathlineto{\pgfqpoint{5.546342in}{0.667945in}}%
\pgfpathlineto{\pgfqpoint{5.546763in}{0.653184in}}%
\pgfpathlineto{\pgfqpoint{5.547606in}{0.653573in}}%
\pgfpathlineto{\pgfqpoint{5.548195in}{0.653859in}}%
\pgfpathlineto{\pgfqpoint{5.548364in}{0.654266in}}%
\pgfpathlineto{\pgfqpoint{5.548869in}{0.659047in}}%
\pgfpathlineto{\pgfqpoint{5.549375in}{0.653852in}}%
\pgfpathlineto{\pgfqpoint{5.550470in}{0.653747in}}%
\pgfpathlineto{\pgfqpoint{5.550638in}{0.654277in}}%
\pgfpathlineto{\pgfqpoint{5.551060in}{0.658735in}}%
\pgfpathlineto{\pgfqpoint{5.551481in}{0.652135in}}%
\pgfpathlineto{\pgfqpoint{5.552155in}{0.652374in}}%
\pgfpathlineto{\pgfqpoint{5.552323in}{0.652940in}}%
\pgfpathlineto{\pgfqpoint{5.552745in}{0.656814in}}%
\pgfpathlineto{\pgfqpoint{5.553334in}{0.652410in}}%
\pgfpathlineto{\pgfqpoint{5.554851in}{0.652932in}}%
\pgfpathlineto{\pgfqpoint{5.555862in}{0.658570in}}%
\pgfpathlineto{\pgfqpoint{5.556114in}{0.675181in}}%
\pgfpathlineto{\pgfqpoint{5.556536in}{0.651991in}}%
\pgfpathlineto{\pgfqpoint{5.557041in}{0.670814in}}%
\pgfpathlineto{\pgfqpoint{5.557462in}{0.652904in}}%
\pgfpathlineto{\pgfqpoint{5.558052in}{0.675368in}}%
\pgfpathlineto{\pgfqpoint{5.559316in}{0.650675in}}%
\pgfpathlineto{\pgfqpoint{5.559568in}{0.652545in}}%
\pgfpathlineto{\pgfqpoint{5.559821in}{0.663174in}}%
\pgfpathlineto{\pgfqpoint{5.560664in}{0.652020in}}%
\pgfpathlineto{\pgfqpoint{5.560832in}{0.653377in}}%
\pgfpathlineto{\pgfqpoint{5.562433in}{0.697142in}}%
\pgfpathlineto{\pgfqpoint{5.562517in}{0.687309in}}%
\pgfpathlineto{\pgfqpoint{5.563107in}{0.650694in}}%
\pgfpathlineto{\pgfqpoint{5.563781in}{0.654082in}}%
\pgfpathlineto{\pgfqpoint{5.564033in}{0.650402in}}%
\pgfpathlineto{\pgfqpoint{5.564286in}{0.665576in}}%
\pgfpathlineto{\pgfqpoint{5.564539in}{0.753244in}}%
\pgfpathlineto{\pgfqpoint{5.565297in}{0.655324in}}%
\pgfpathlineto{\pgfqpoint{5.565465in}{0.654393in}}%
\pgfpathlineto{\pgfqpoint{5.565802in}{0.658461in}}%
\pgfpathlineto{\pgfqpoint{5.566055in}{0.737453in}}%
\pgfpathlineto{\pgfqpoint{5.566308in}{0.906385in}}%
\pgfpathlineto{\pgfqpoint{5.566898in}{0.656410in}}%
\pgfpathlineto{\pgfqpoint{5.567066in}{0.659679in}}%
\pgfpathlineto{\pgfqpoint{5.567403in}{0.683178in}}%
\pgfpathlineto{\pgfqpoint{5.567824in}{0.653576in}}%
\pgfpathlineto{\pgfqpoint{5.568330in}{0.671043in}}%
\pgfpathlineto{\pgfqpoint{5.568498in}{0.666745in}}%
\pgfpathlineto{\pgfqpoint{5.568835in}{0.654911in}}%
\pgfpathlineto{\pgfqpoint{5.569172in}{0.689027in}}%
\pgfpathlineto{\pgfqpoint{5.569425in}{0.716636in}}%
\pgfpathlineto{\pgfqpoint{5.570015in}{0.656270in}}%
\pgfpathlineto{\pgfqpoint{5.570857in}{0.652797in}}%
\pgfpathlineto{\pgfqpoint{5.570520in}{0.665945in}}%
\pgfpathlineto{\pgfqpoint{5.571026in}{0.655374in}}%
\pgfpathlineto{\pgfqpoint{5.571363in}{0.677496in}}%
\pgfpathlineto{\pgfqpoint{5.572205in}{0.661811in}}%
\pgfpathlineto{\pgfqpoint{5.573047in}{0.687106in}}%
\pgfpathlineto{\pgfqpoint{5.573300in}{0.666073in}}%
\pgfpathlineto{\pgfqpoint{5.573637in}{0.651374in}}%
\pgfpathlineto{\pgfqpoint{5.574564in}{0.652000in}}%
\pgfpathlineto{\pgfqpoint{5.574985in}{0.677423in}}%
\pgfpathlineto{\pgfqpoint{5.575828in}{0.656119in}}%
\pgfpathlineto{\pgfqpoint{5.576249in}{0.651338in}}%
\pgfpathlineto{\pgfqpoint{5.576838in}{0.657038in}}%
\pgfpathlineto{\pgfqpoint{5.577175in}{0.715988in}}%
\pgfpathlineto{\pgfqpoint{5.577681in}{0.651715in}}%
\pgfpathlineto{\pgfqpoint{5.577934in}{0.658770in}}%
\pgfpathlineto{\pgfqpoint{5.578860in}{0.685517in}}%
\pgfpathlineto{\pgfqpoint{5.578439in}{0.650694in}}%
\pgfpathlineto{\pgfqpoint{5.579113in}{0.662522in}}%
\pgfpathlineto{\pgfqpoint{5.579282in}{0.653276in}}%
\pgfpathlineto{\pgfqpoint{5.579619in}{0.687102in}}%
\pgfpathlineto{\pgfqpoint{5.579787in}{0.723452in}}%
\pgfpathlineto{\pgfqpoint{5.580545in}{0.651850in}}%
\pgfpathlineto{\pgfqpoint{5.580882in}{0.650909in}}%
\pgfpathlineto{\pgfqpoint{5.581640in}{0.651691in}}%
\pgfpathlineto{\pgfqpoint{5.582062in}{0.656061in}}%
\pgfpathlineto{\pgfqpoint{5.582483in}{0.649885in}}%
\pgfpathlineto{\pgfqpoint{5.582820in}{0.650899in}}%
\pgfpathlineto{\pgfqpoint{5.584168in}{0.673785in}}%
\pgfpathlineto{\pgfqpoint{5.583578in}{0.650835in}}%
\pgfpathlineto{\pgfqpoint{5.584505in}{0.660236in}}%
\pgfpathlineto{\pgfqpoint{5.585263in}{0.651055in}}%
\pgfpathlineto{\pgfqpoint{5.585600in}{0.658104in}}%
\pgfpathlineto{\pgfqpoint{5.585937in}{0.691913in}}%
\pgfpathlineto{\pgfqpoint{5.586442in}{0.649620in}}%
\pgfpathlineto{\pgfqpoint{5.586864in}{0.677800in}}%
\pgfpathlineto{\pgfqpoint{5.587201in}{0.649650in}}%
\pgfpathlineto{\pgfqpoint{5.588127in}{0.650251in}}%
\pgfpathlineto{\pgfqpoint{5.589896in}{0.654774in}}%
\pgfpathlineto{\pgfqpoint{5.591160in}{0.678179in}}%
\pgfpathlineto{\pgfqpoint{5.591329in}{0.668604in}}%
\pgfpathlineto{\pgfqpoint{5.591750in}{0.651454in}}%
\pgfpathlineto{\pgfqpoint{5.592592in}{0.653058in}}%
\pgfpathlineto{\pgfqpoint{5.593772in}{0.652689in}}%
\pgfpathlineto{\pgfqpoint{5.593350in}{0.653717in}}%
\pgfpathlineto{\pgfqpoint{5.593856in}{0.652768in}}%
\pgfpathlineto{\pgfqpoint{5.594109in}{0.653100in}}%
\pgfpathlineto{\pgfqpoint{5.594951in}{0.652649in}}%
\pgfpathlineto{\pgfqpoint{5.595457in}{0.654021in}}%
\pgfpathlineto{\pgfqpoint{5.596383in}{0.668603in}}%
\pgfpathlineto{\pgfqpoint{5.595962in}{0.653406in}}%
\pgfpathlineto{\pgfqpoint{5.596636in}{0.657856in}}%
\pgfpathlineto{\pgfqpoint{5.597057in}{0.652172in}}%
\pgfpathlineto{\pgfqpoint{5.597563in}{0.662884in}}%
\pgfpathlineto{\pgfqpoint{5.597647in}{0.663957in}}%
\pgfpathlineto{\pgfqpoint{5.597900in}{0.654605in}}%
\pgfpathlineto{\pgfqpoint{5.597984in}{0.653908in}}%
\pgfpathlineto{\pgfqpoint{5.598152in}{0.659376in}}%
\pgfpathlineto{\pgfqpoint{5.598489in}{0.687582in}}%
\pgfpathlineto{\pgfqpoint{5.598995in}{0.652308in}}%
\pgfpathlineto{\pgfqpoint{5.599163in}{0.652482in}}%
\pgfpathlineto{\pgfqpoint{5.600680in}{0.655290in}}%
\pgfpathlineto{\pgfqpoint{5.601269in}{0.654582in}}%
\pgfpathlineto{\pgfqpoint{5.602365in}{0.663806in}}%
\pgfpathlineto{\pgfqpoint{5.601859in}{0.653408in}}%
\pgfpathlineto{\pgfqpoint{5.602449in}{0.660745in}}%
\pgfpathlineto{\pgfqpoint{5.603375in}{0.653395in}}%
\pgfpathlineto{\pgfqpoint{5.603712in}{0.654133in}}%
\pgfpathlineto{\pgfqpoint{5.604218in}{0.653424in}}%
\pgfpathlineto{\pgfqpoint{5.604892in}{0.654706in}}%
\pgfpathlineto{\pgfqpoint{5.605819in}{0.655979in}}%
\pgfpathlineto{\pgfqpoint{5.605313in}{0.653584in}}%
\pgfpathlineto{\pgfqpoint{5.605903in}{0.655108in}}%
\pgfpathlineto{\pgfqpoint{5.606745in}{0.653341in}}%
\pgfpathlineto{\pgfqpoint{5.606408in}{0.662112in}}%
\pgfpathlineto{\pgfqpoint{5.606830in}{0.653638in}}%
\pgfpathlineto{\pgfqpoint{5.608093in}{0.737768in}}%
\pgfpathlineto{\pgfqpoint{5.608346in}{0.680305in}}%
\pgfpathlineto{\pgfqpoint{5.608599in}{0.654238in}}%
\pgfpathlineto{\pgfqpoint{5.609020in}{0.757898in}}%
\pgfpathlineto{\pgfqpoint{5.609273in}{0.693182in}}%
\pgfpathlineto{\pgfqpoint{5.610452in}{0.653803in}}%
\pgfpathlineto{\pgfqpoint{5.611042in}{0.653818in}}%
\pgfpathlineto{\pgfqpoint{5.611126in}{0.654459in}}%
\pgfpathlineto{\pgfqpoint{5.611547in}{0.683510in}}%
\pgfpathlineto{\pgfqpoint{5.612558in}{0.673495in}}%
\pgfpathlineto{\pgfqpoint{5.613064in}{0.653295in}}%
\pgfpathlineto{\pgfqpoint{5.613822in}{0.653715in}}%
\pgfpathlineto{\pgfqpoint{5.616433in}{0.654801in}}%
\pgfpathlineto{\pgfqpoint{5.617697in}{0.672550in}}%
\pgfpathlineto{\pgfqpoint{5.617866in}{0.662158in}}%
\pgfpathlineto{\pgfqpoint{5.618203in}{0.653956in}}%
\pgfpathlineto{\pgfqpoint{5.619045in}{0.654255in}}%
\pgfpathlineto{\pgfqpoint{5.619550in}{0.655449in}}%
\pgfpathlineto{\pgfqpoint{5.620056in}{0.654865in}}%
\pgfpathlineto{\pgfqpoint{5.620224in}{0.666542in}}%
\pgfpathlineto{\pgfqpoint{5.620561in}{0.787485in}}%
\pgfpathlineto{\pgfqpoint{5.621151in}{0.656259in}}%
\pgfpathlineto{\pgfqpoint{5.621404in}{0.697252in}}%
\pgfpathlineto{\pgfqpoint{5.621488in}{0.706502in}}%
\pgfpathlineto{\pgfqpoint{5.621909in}{0.654264in}}%
\pgfpathlineto{\pgfqpoint{5.622162in}{0.660466in}}%
\pgfpathlineto{\pgfqpoint{5.623173in}{0.675542in}}%
\pgfpathlineto{\pgfqpoint{5.622752in}{0.654557in}}%
\pgfpathlineto{\pgfqpoint{5.623257in}{0.672339in}}%
\pgfpathlineto{\pgfqpoint{5.623763in}{0.653723in}}%
\pgfpathlineto{\pgfqpoint{5.624521in}{0.654072in}}%
\pgfpathlineto{\pgfqpoint{5.625953in}{0.654591in}}%
\pgfpathlineto{\pgfqpoint{5.628228in}{0.655756in}}%
\pgfpathlineto{\pgfqpoint{5.629239in}{0.680475in}}%
\pgfpathlineto{\pgfqpoint{5.628817in}{0.655642in}}%
\pgfpathlineto{\pgfqpoint{5.629491in}{0.658509in}}%
\pgfpathlineto{\pgfqpoint{5.630334in}{0.655037in}}%
\pgfpathlineto{\pgfqpoint{5.629997in}{0.672380in}}%
\pgfpathlineto{\pgfqpoint{5.630502in}{0.659516in}}%
\pgfpathlineto{\pgfqpoint{5.631682in}{0.801532in}}%
\pgfpathlineto{\pgfqpoint{5.631260in}{0.655488in}}%
\pgfpathlineto{\pgfqpoint{5.631850in}{0.753257in}}%
\pgfpathlineto{\pgfqpoint{5.633114in}{0.655105in}}%
\pgfpathlineto{\pgfqpoint{5.633282in}{0.657827in}}%
\pgfpathlineto{\pgfqpoint{5.633619in}{0.683876in}}%
\pgfpathlineto{\pgfqpoint{5.634209in}{0.654558in}}%
\pgfpathlineto{\pgfqpoint{5.634293in}{0.654704in}}%
\pgfpathlineto{\pgfqpoint{5.634546in}{0.655520in}}%
\pgfpathlineto{\pgfqpoint{5.634799in}{0.654577in}}%
\pgfpathlineto{\pgfqpoint{5.635388in}{0.654574in}}%
\pgfpathlineto{\pgfqpoint{5.635725in}{0.655237in}}%
\pgfpathlineto{\pgfqpoint{5.636989in}{0.660206in}}%
\pgfpathlineto{\pgfqpoint{5.636399in}{0.654434in}}%
\pgfpathlineto{\pgfqpoint{5.637073in}{0.658894in}}%
\pgfpathlineto{\pgfqpoint{5.637410in}{0.654499in}}%
\pgfpathlineto{\pgfqpoint{5.638000in}{0.661960in}}%
\pgfpathlineto{\pgfqpoint{5.638337in}{0.656276in}}%
\pgfpathlineto{\pgfqpoint{5.639095in}{0.654659in}}%
\pgfpathlineto{\pgfqpoint{5.639516in}{0.654807in}}%
\pgfpathlineto{\pgfqpoint{5.641033in}{0.655770in}}%
\pgfpathlineto{\pgfqpoint{5.641286in}{0.657761in}}%
\pgfpathlineto{\pgfqpoint{5.641791in}{0.654816in}}%
\pgfpathlineto{\pgfqpoint{5.642044in}{0.654882in}}%
\pgfpathlineto{\pgfqpoint{5.643897in}{0.654785in}}%
\pgfpathlineto{\pgfqpoint{5.644992in}{0.654881in}}%
\pgfpathlineto{\pgfqpoint{5.645077in}{0.655500in}}%
\pgfpathlineto{\pgfqpoint{5.645329in}{0.659487in}}%
\pgfpathlineto{\pgfqpoint{5.645919in}{0.654714in}}%
\pgfpathlineto{\pgfqpoint{5.646340in}{0.658466in}}%
\pgfpathlineto{\pgfqpoint{5.646846in}{0.654699in}}%
\pgfpathlineto{\pgfqpoint{5.647857in}{0.655076in}}%
\pgfpathlineto{\pgfqpoint{5.648194in}{0.657802in}}%
\pgfpathlineto{\pgfqpoint{5.648783in}{0.654828in}}%
\pgfpathlineto{\pgfqpoint{5.648868in}{0.654828in}}%
\pgfpathlineto{\pgfqpoint{5.649457in}{0.654963in}}%
\pgfpathlineto{\pgfqpoint{5.649626in}{0.654325in}}%
\pgfpathlineto{\pgfqpoint{5.650215in}{0.655803in}}%
\pgfpathlineto{\pgfqpoint{5.650721in}{0.653669in}}%
\pgfpathlineto{\pgfqpoint{5.651142in}{0.662915in}}%
\pgfpathlineto{\pgfqpoint{5.651479in}{0.686965in}}%
\pgfpathlineto{\pgfqpoint{5.652069in}{0.652847in}}%
\pgfpathlineto{\pgfqpoint{5.652237in}{0.653888in}}%
\pgfpathlineto{\pgfqpoint{5.652574in}{0.662186in}}%
\pgfpathlineto{\pgfqpoint{5.652996in}{0.652309in}}%
\pgfpathlineto{\pgfqpoint{5.653332in}{0.653698in}}%
\pgfpathlineto{\pgfqpoint{5.653838in}{0.656098in}}%
\pgfpathlineto{\pgfqpoint{5.655017in}{0.720515in}}%
\pgfpathlineto{\pgfqpoint{5.654512in}{0.654740in}}%
\pgfpathlineto{\pgfqpoint{5.655270in}{0.681376in}}%
\pgfpathlineto{\pgfqpoint{5.656534in}{0.651667in}}%
\pgfpathlineto{\pgfqpoint{5.656618in}{0.651637in}}%
\pgfpathlineto{\pgfqpoint{5.656702in}{0.652181in}}%
\pgfpathlineto{\pgfqpoint{5.657966in}{0.761844in}}%
\pgfpathlineto{\pgfqpoint{5.658050in}{0.784842in}}%
\pgfpathlineto{\pgfqpoint{5.658640in}{0.652847in}}%
\pgfpathlineto{\pgfqpoint{5.658893in}{0.659292in}}%
\pgfpathlineto{\pgfqpoint{5.659482in}{0.674887in}}%
\pgfpathlineto{\pgfqpoint{5.659819in}{0.652598in}}%
\pgfpathlineto{\pgfqpoint{5.660072in}{0.673654in}}%
\pgfpathlineto{\pgfqpoint{5.661251in}{0.802376in}}%
\pgfpathlineto{\pgfqpoint{5.660830in}{0.652893in}}%
\pgfpathlineto{\pgfqpoint{5.661336in}{0.773812in}}%
\pgfpathlineto{\pgfqpoint{5.662599in}{0.652952in}}%
\pgfpathlineto{\pgfqpoint{5.662936in}{0.654642in}}%
\pgfpathlineto{\pgfqpoint{5.664200in}{0.662378in}}%
\pgfpathlineto{\pgfqpoint{5.663779in}{0.653043in}}%
\pgfpathlineto{\pgfqpoint{5.664369in}{0.658980in}}%
\pgfpathlineto{\pgfqpoint{5.664874in}{0.652926in}}%
\pgfpathlineto{\pgfqpoint{5.665548in}{0.655591in}}%
\pgfpathlineto{\pgfqpoint{5.666812in}{0.652928in}}%
\pgfpathlineto{\pgfqpoint{5.667317in}{0.653109in}}%
\pgfpathlineto{\pgfqpoint{5.668328in}{0.653899in}}%
\pgfpathlineto{\pgfqpoint{5.668833in}{0.658286in}}%
\pgfpathlineto{\pgfqpoint{5.669507in}{0.654899in}}%
\pgfpathlineto{\pgfqpoint{5.669929in}{0.653573in}}%
\pgfpathlineto{\pgfqpoint{5.670687in}{0.654996in}}%
\pgfpathlineto{\pgfqpoint{5.671108in}{0.713322in}}%
\pgfpathlineto{\pgfqpoint{5.671698in}{0.654199in}}%
\pgfpathlineto{\pgfqpoint{5.671782in}{0.654607in}}%
\pgfpathlineto{\pgfqpoint{5.672203in}{0.709377in}}%
\pgfpathlineto{\pgfqpoint{5.672709in}{0.653465in}}%
\pgfpathlineto{\pgfqpoint{5.673130in}{0.671051in}}%
\pgfpathlineto{\pgfqpoint{5.673804in}{0.653278in}}%
\pgfpathlineto{\pgfqpoint{5.674225in}{0.663244in}}%
\pgfpathlineto{\pgfqpoint{5.674394in}{0.670852in}}%
\pgfpathlineto{\pgfqpoint{5.675152in}{0.653687in}}%
\pgfpathlineto{\pgfqpoint{5.675405in}{0.653665in}}%
\pgfpathlineto{\pgfqpoint{5.675573in}{0.654329in}}%
\pgfpathlineto{\pgfqpoint{5.675826in}{0.657749in}}%
\pgfpathlineto{\pgfqpoint{5.676247in}{0.653210in}}%
\pgfpathlineto{\pgfqpoint{5.676752in}{0.655679in}}%
\pgfpathlineto{\pgfqpoint{5.677089in}{0.653550in}}%
\pgfpathlineto{\pgfqpoint{5.677595in}{0.658066in}}%
\pgfpathlineto{\pgfqpoint{5.677932in}{0.660495in}}%
\pgfpathlineto{\pgfqpoint{5.678269in}{0.655181in}}%
\pgfpathlineto{\pgfqpoint{5.678522in}{0.654001in}}%
\pgfpathlineto{\pgfqpoint{5.679111in}{0.656167in}}%
\pgfpathlineto{\pgfqpoint{5.679280in}{0.655319in}}%
\pgfpathlineto{\pgfqpoint{5.679448in}{0.658183in}}%
\pgfpathlineto{\pgfqpoint{5.679701in}{0.680878in}}%
\pgfpathlineto{\pgfqpoint{5.680291in}{0.652988in}}%
\pgfpathlineto{\pgfqpoint{5.680459in}{0.653219in}}%
\pgfpathlineto{\pgfqpoint{5.681554in}{0.673620in}}%
\pgfpathlineto{\pgfqpoint{5.681723in}{0.697218in}}%
\pgfpathlineto{\pgfqpoint{5.682313in}{0.653188in}}%
\pgfpathlineto{\pgfqpoint{5.682481in}{0.653797in}}%
\pgfpathlineto{\pgfqpoint{5.682734in}{0.656451in}}%
\pgfpathlineto{\pgfqpoint{5.683408in}{0.652707in}}%
\pgfpathlineto{\pgfqpoint{5.684166in}{0.653407in}}%
\pgfpathlineto{\pgfqpoint{5.684503in}{0.662500in}}%
\pgfpathlineto{\pgfqpoint{5.685177in}{0.652361in}}%
\pgfpathlineto{\pgfqpoint{5.685598in}{0.653165in}}%
\pgfpathlineto{\pgfqpoint{5.685851in}{0.653766in}}%
\pgfpathlineto{\pgfqpoint{5.686272in}{0.652409in}}%
\pgfpathlineto{\pgfqpoint{5.686693in}{0.653166in}}%
\pgfpathlineto{\pgfqpoint{5.688041in}{0.652416in}}%
\pgfpathlineto{\pgfqpoint{5.688210in}{0.653482in}}%
\pgfpathlineto{\pgfqpoint{5.688294in}{0.653921in}}%
\pgfpathlineto{\pgfqpoint{5.688715in}{0.652397in}}%
\pgfpathlineto{\pgfqpoint{5.689052in}{0.652535in}}%
\pgfpathlineto{\pgfqpoint{5.689221in}{0.651851in}}%
\pgfpathlineto{\pgfqpoint{5.689473in}{0.654740in}}%
\pgfpathlineto{\pgfqpoint{5.689726in}{0.659589in}}%
\pgfpathlineto{\pgfqpoint{5.690063in}{0.653246in}}%
\pgfpathlineto{\pgfqpoint{5.690484in}{0.654015in}}%
\pgfpathlineto{\pgfqpoint{5.690821in}{0.654997in}}%
\pgfpathlineto{\pgfqpoint{5.690990in}{0.653402in}}%
\pgfpathlineto{\pgfqpoint{5.691327in}{0.651819in}}%
\pgfpathlineto{\pgfqpoint{5.692169in}{0.652467in}}%
\pgfpathlineto{\pgfqpoint{5.692590in}{0.658739in}}%
\pgfpathlineto{\pgfqpoint{5.693264in}{0.652565in}}%
\pgfpathlineto{\pgfqpoint{5.693601in}{0.653206in}}%
\pgfpathlineto{\pgfqpoint{5.693938in}{0.673766in}}%
\pgfpathlineto{\pgfqpoint{5.694949in}{0.801845in}}%
\pgfpathlineto{\pgfqpoint{5.694444in}{0.655934in}}%
\pgfpathlineto{\pgfqpoint{5.695118in}{0.730884in}}%
\pgfpathlineto{\pgfqpoint{5.696381in}{0.653829in}}%
\pgfpathlineto{\pgfqpoint{5.696634in}{0.654642in}}%
\pgfpathlineto{\pgfqpoint{5.697477in}{0.653919in}}%
\pgfpathlineto{\pgfqpoint{5.697729in}{0.670392in}}%
\pgfpathlineto{\pgfqpoint{5.697982in}{0.761193in}}%
\pgfpathlineto{\pgfqpoint{5.698572in}{0.656568in}}%
\pgfpathlineto{\pgfqpoint{5.698909in}{0.695739in}}%
\pgfpathlineto{\pgfqpoint{5.699077in}{0.730021in}}%
\pgfpathlineto{\pgfqpoint{5.699583in}{0.654238in}}%
\pgfpathlineto{\pgfqpoint{5.699920in}{0.670930in}}%
\pgfpathlineto{\pgfqpoint{5.700088in}{0.678022in}}%
\pgfpathlineto{\pgfqpoint{5.700762in}{0.654932in}}%
\pgfpathlineto{\pgfqpoint{5.700931in}{0.657108in}}%
\pgfpathlineto{\pgfqpoint{5.701352in}{0.674805in}}%
\pgfpathlineto{\pgfqpoint{5.701857in}{0.653690in}}%
\pgfpathlineto{\pgfqpoint{5.701942in}{0.653839in}}%
\pgfpathlineto{\pgfqpoint{5.702194in}{0.656183in}}%
\pgfpathlineto{\pgfqpoint{5.703037in}{0.653784in}}%
\pgfpathlineto{\pgfqpoint{5.704974in}{0.654044in}}%
\pgfpathlineto{\pgfqpoint{5.706238in}{0.654033in}}%
\pgfpathlineto{\pgfqpoint{5.706322in}{0.654388in}}%
\pgfpathlineto{\pgfqpoint{5.706575in}{0.663886in}}%
\pgfpathlineto{\pgfqpoint{5.706828in}{0.691970in}}%
\pgfpathlineto{\pgfqpoint{5.707333in}{0.657839in}}%
\pgfpathlineto{\pgfqpoint{5.707839in}{0.689501in}}%
\pgfpathlineto{\pgfqpoint{5.708428in}{0.653320in}}%
\pgfpathlineto{\pgfqpoint{5.709524in}{0.653383in}}%
\pgfpathlineto{\pgfqpoint{5.710703in}{0.655073in}}%
\pgfpathlineto{\pgfqpoint{5.711377in}{0.653946in}}%
\pgfpathlineto{\pgfqpoint{5.712135in}{0.660458in}}%
\pgfpathlineto{\pgfqpoint{5.712641in}{0.653163in}}%
\pgfpathlineto{\pgfqpoint{5.714494in}{0.654049in}}%
\pgfpathlineto{\pgfqpoint{5.715505in}{0.655473in}}%
\pgfpathlineto{\pgfqpoint{5.715842in}{0.654525in}}%
\pgfpathlineto{\pgfqpoint{5.716179in}{0.654149in}}%
\pgfpathlineto{\pgfqpoint{5.716432in}{0.656076in}}%
\pgfpathlineto{\pgfqpoint{5.716684in}{0.654517in}}%
\pgfpathlineto{\pgfqpoint{5.716853in}{0.654309in}}%
\pgfpathlineto{\pgfqpoint{5.717021in}{0.656758in}}%
\pgfpathlineto{\pgfqpoint{5.717274in}{0.671851in}}%
\pgfpathlineto{\pgfqpoint{5.717780in}{0.654453in}}%
\pgfpathlineto{\pgfqpoint{5.718117in}{0.656372in}}%
\pgfpathlineto{\pgfqpoint{5.718201in}{0.656473in}}%
\pgfpathlineto{\pgfqpoint{5.718369in}{0.655183in}}%
\pgfpathlineto{\pgfqpoint{5.718790in}{0.654608in}}%
\pgfpathlineto{\pgfqpoint{5.719043in}{0.656006in}}%
\pgfpathlineto{\pgfqpoint{5.719296in}{0.659613in}}%
\pgfpathlineto{\pgfqpoint{5.719886in}{0.655109in}}%
\pgfpathlineto{\pgfqpoint{5.720054in}{0.655203in}}%
\pgfpathlineto{\pgfqpoint{5.720475in}{0.657648in}}%
\pgfpathlineto{\pgfqpoint{5.721234in}{0.654263in}}%
\pgfpathlineto{\pgfqpoint{5.721739in}{0.662209in}}%
\pgfpathlineto{\pgfqpoint{5.722076in}{0.654179in}}%
\pgfpathlineto{\pgfqpoint{5.722413in}{0.667137in}}%
\pgfpathlineto{\pgfqpoint{5.722581in}{0.685025in}}%
\pgfpathlineto{\pgfqpoint{5.723087in}{0.654094in}}%
\pgfpathlineto{\pgfqpoint{5.723424in}{0.665554in}}%
\pgfpathlineto{\pgfqpoint{5.723761in}{0.653539in}}%
\pgfpathlineto{\pgfqpoint{5.724098in}{0.683163in}}%
\pgfpathlineto{\pgfqpoint{5.724266in}{0.701499in}}%
\pgfpathlineto{\pgfqpoint{5.724688in}{0.656056in}}%
\pgfpathlineto{\pgfqpoint{5.725025in}{0.660277in}}%
\pgfpathlineto{\pgfqpoint{5.725277in}{0.653173in}}%
\pgfpathlineto{\pgfqpoint{5.725951in}{0.657563in}}%
\pgfpathlineto{\pgfqpoint{5.726288in}{0.732826in}}%
\pgfpathlineto{\pgfqpoint{5.726962in}{0.652523in}}%
\pgfpathlineto{\pgfqpoint{5.728563in}{0.653453in}}%
\pgfpathlineto{\pgfqpoint{5.730164in}{0.658414in}}%
\pgfpathlineto{\pgfqpoint{5.730416in}{0.665868in}}%
\pgfpathlineto{\pgfqpoint{5.731090in}{0.652928in}}%
\pgfpathlineto{\pgfqpoint{5.731427in}{0.652371in}}%
\pgfpathlineto{\pgfqpoint{5.732270in}{0.652679in}}%
\pgfpathlineto{\pgfqpoint{5.732859in}{0.653657in}}%
\pgfpathlineto{\pgfqpoint{5.733365in}{0.673404in}}%
\pgfpathlineto{\pgfqpoint{5.733786in}{0.657593in}}%
\pgfpathlineto{\pgfqpoint{5.734123in}{0.761117in}}%
\pgfpathlineto{\pgfqpoint{5.734376in}{0.930566in}}%
\pgfpathlineto{\pgfqpoint{5.735050in}{0.657095in}}%
\pgfpathlineto{\pgfqpoint{5.735302in}{0.678554in}}%
\pgfpathlineto{\pgfqpoint{5.735471in}{0.699023in}}%
\pgfpathlineto{\pgfqpoint{5.736229in}{0.654598in}}%
\pgfpathlineto{\pgfqpoint{5.736398in}{0.656820in}}%
\pgfpathlineto{\pgfqpoint{5.737746in}{0.708042in}}%
\pgfpathlineto{\pgfqpoint{5.737156in}{0.654397in}}%
\pgfpathlineto{\pgfqpoint{5.737914in}{0.682146in}}%
\pgfpathlineto{\pgfqpoint{5.738251in}{0.655396in}}%
\pgfpathlineto{\pgfqpoint{5.738756in}{0.699859in}}%
\pgfpathlineto{\pgfqpoint{5.739093in}{0.670332in}}%
\pgfpathlineto{\pgfqpoint{5.739599in}{0.654615in}}%
\pgfpathlineto{\pgfqpoint{5.740104in}{0.672226in}}%
\pgfpathlineto{\pgfqpoint{5.740441in}{0.655212in}}%
\pgfpathlineto{\pgfqpoint{5.740610in}{0.653733in}}%
\pgfpathlineto{\pgfqpoint{5.740947in}{0.664249in}}%
\pgfpathlineto{\pgfqpoint{5.741537in}{0.654268in}}%
\pgfpathlineto{\pgfqpoint{5.741873in}{0.658332in}}%
\pgfpathlineto{\pgfqpoint{5.742463in}{0.653629in}}%
\pgfpathlineto{\pgfqpoint{5.742547in}{0.653648in}}%
\pgfpathlineto{\pgfqpoint{5.743390in}{0.656734in}}%
\pgfpathlineto{\pgfqpoint{5.743727in}{0.668667in}}%
\pgfpathlineto{\pgfqpoint{5.744148in}{0.655370in}}%
\pgfpathlineto{\pgfqpoint{5.744569in}{0.661207in}}%
\pgfpathlineto{\pgfqpoint{5.744654in}{0.661983in}}%
\pgfpathlineto{\pgfqpoint{5.744991in}{0.655679in}}%
\pgfpathlineto{\pgfqpoint{5.745580in}{0.652983in}}%
\pgfpathlineto{\pgfqpoint{5.745833in}{0.653024in}}%
\pgfusepath{stroke}%
\end{pgfscope}%
\begin{pgfscope}%
\pgfpathrectangle{\pgfqpoint{0.691161in}{0.544166in}}{\pgfqpoint{5.054672in}{0.902317in}}%
\pgfusepath{clip}%
\pgfsetbuttcap%
\pgfsetroundjoin%
\pgfsetlinewidth{2.007500pt}%
\definecolor{currentstroke}{rgb}{0.172549,0.627451,0.172549}%
\pgfsetstrokecolor{currentstroke}%
\pgfsetdash{{7.400000pt}{3.200000pt}}{0.000000pt}%
\pgfpathmoveto{\pgfqpoint{0.691161in}{1.405469in}}%
\pgfpathlineto{\pgfqpoint{5.745833in}{1.405469in}}%
\pgfusepath{stroke}%
\end{pgfscope}%
\begin{pgfscope}%
\pgfsetrectcap%
\pgfsetmiterjoin%
\pgfsetlinewidth{0.803000pt}%
\definecolor{currentstroke}{rgb}{0.737255,0.737255,0.737255}%
\pgfsetstrokecolor{currentstroke}%
\pgfsetdash{}{0pt}%
\pgfpathmoveto{\pgfqpoint{0.691161in}{0.544166in}}%
\pgfpathlineto{\pgfqpoint{0.691161in}{1.446484in}}%
\pgfusepath{stroke}%
\end{pgfscope}%
\begin{pgfscope}%
\pgfsetrectcap%
\pgfsetmiterjoin%
\pgfsetlinewidth{0.803000pt}%
\definecolor{currentstroke}{rgb}{0.737255,0.737255,0.737255}%
\pgfsetstrokecolor{currentstroke}%
\pgfsetdash{}{0pt}%
\pgfpathmoveto{\pgfqpoint{5.745833in}{0.544166in}}%
\pgfpathlineto{\pgfqpoint{5.745833in}{1.446484in}}%
\pgfusepath{stroke}%
\end{pgfscope}%
\begin{pgfscope}%
\pgfsetrectcap%
\pgfsetmiterjoin%
\pgfsetlinewidth{0.803000pt}%
\definecolor{currentstroke}{rgb}{0.737255,0.737255,0.737255}%
\pgfsetstrokecolor{currentstroke}%
\pgfsetdash{}{0pt}%
\pgfpathmoveto{\pgfqpoint{0.691161in}{0.544166in}}%
\pgfpathlineto{\pgfqpoint{5.745833in}{0.544166in}}%
\pgfusepath{stroke}%
\end{pgfscope}%
\begin{pgfscope}%
\pgfsetrectcap%
\pgfsetmiterjoin%
\pgfsetlinewidth{0.803000pt}%
\definecolor{currentstroke}{rgb}{0.737255,0.737255,0.737255}%
\pgfsetstrokecolor{currentstroke}%
\pgfsetdash{}{0pt}%
\pgfpathmoveto{\pgfqpoint{0.691161in}{1.446484in}}%
\pgfpathlineto{\pgfqpoint{5.745833in}{1.446484in}}%
\pgfusepath{stroke}%
\end{pgfscope}%
\begin{pgfscope}%
\pgfsetbuttcap%
\pgfsetmiterjoin%
\definecolor{currentfill}{rgb}{0.933333,0.933333,0.933333}%
\pgfsetfillcolor{currentfill}%
\pgfsetfillopacity{0.800000}%
\pgfsetlinewidth{0.501875pt}%
\definecolor{currentstroke}{rgb}{0.800000,0.800000,0.800000}%
\pgfsetstrokecolor{currentstroke}%
\pgfsetstrokeopacity{0.800000}%
\pgfsetdash{}{0pt}%
\pgfpathmoveto{\pgfqpoint{4.501944in}{1.141762in}}%
\pgfpathlineto{\pgfqpoint{5.648611in}{1.141762in}}%
\pgfpathquadraticcurveto{\pgfqpoint{5.676389in}{1.141762in}}{\pgfqpoint{5.676389in}{1.169539in}}%
\pgfpathlineto{\pgfqpoint{5.676389in}{1.349261in}}%
\pgfpathquadraticcurveto{\pgfqpoint{5.676389in}{1.377039in}}{\pgfqpoint{5.648611in}{1.377039in}}%
\pgfpathlineto{\pgfqpoint{4.501944in}{1.377039in}}%
\pgfpathquadraticcurveto{\pgfqpoint{4.474166in}{1.377039in}}{\pgfqpoint{4.474166in}{1.349261in}}%
\pgfpathlineto{\pgfqpoint{4.474166in}{1.169539in}}%
\pgfpathquadraticcurveto{\pgfqpoint{4.474166in}{1.141762in}}{\pgfqpoint{4.501944in}{1.141762in}}%
\pgfpathlineto{\pgfqpoint{4.501944in}{1.141762in}}%
\pgfpathclose%
\pgfusepath{stroke,fill}%
\end{pgfscope}%
\begin{pgfscope}%
\pgfsetbuttcap%
\pgfsetroundjoin%
\pgfsetlinewidth{2.007500pt}%
\definecolor{currentstroke}{rgb}{0.172549,0.627451,0.172549}%
\pgfsetstrokecolor{currentstroke}%
\pgfsetdash{{7.400000pt}{3.200000pt}}{0.000000pt}%
\pgfpathmoveto{\pgfqpoint{4.529722in}{1.272872in}}%
\pgfpathlineto{\pgfqpoint{4.807500in}{1.272872in}}%
\pgfusepath{stroke}%
\end{pgfscope}%
\begin{pgfscope}%
\definecolor{textcolor}{rgb}{0.000000,0.000000,0.000000}%
\pgfsetstrokecolor{textcolor}%
\pgfsetfillcolor{textcolor}%
\pgftext[x=4.918611in,y=1.224261in,left,base]{\color{textcolor}\rmfamily\fontsize{10.000000}{12.000000}\selectfont Seuil = 200}%
\end{pgfscope}%
\end{pgfpicture}%
\makeatother%
\endgroup%
}
    \caption{detect}
    \label{fig:pointage-turquie}
\end{figure}

\section{Conclusion et discussion}

L'objectif de détection d'un séisme et de pointage des ondes P et S de façon automatique a été atteint dans une certaine mesure. Nous constatons que cette méthode fonctionne particulièrement bien pour les séismes régionaux comme le montre la figure \ref{fig:pointage-strasbourg}, un séisme de magnitude 4 à de 100 km de la station. La détection permet d'isoler un intervalle relativement précis de la période de secousse. Le pointage est cohérent pour diverses fonctions caractéristiques, il y a des écarts de l'ordre de la centaine voir de la dizaine de millisecondes. Néanmoins nous constatons que la fonction caractéristique de \cite{baer1987} ne permet pas de pointer le début de la phase S, alors que le pic est très marqué pour le début de la phase P. Nous avons également essayé d'implémenter d'autres fonctions caractéristiques comme celle de Carl, sans succès, du fait du nombre important de paramètres. La méthode classique STA/LTA s'avère particulièrement efficace, et sa variante récursive montre des résultats similaires.

Mais cette méthode n'est plus aussi efficace pour des séismes lointains. En effet, nous voyons sur la figure \ref{fig:pointage-turquie} que le pointage n'est plus aussi précis pour un séisme de magnitude 8 à plus de 3000 km de la station sismique. Cette baisse de précision s'explique naturellement par la plus faible amplitude du signal, il est plus bruité. Le signal bruité obstrue donc les variations d'amplitudes liées aux débuts des phases, en particulier celles de la phases S dont le pointage semble être décalée avec plusieurs secondes de retard. 

Cependant, des ajustements peuvent être réalisés pour augmenter la précision de pointage. En effet, nous avons introduit un grand nombre de paramètres qui dépendent de la nature des séismes. Parmi ces paramètres, nous pouvons citer les largeurs des fenêtres STA et LTA, les seuils de détection et de pointage, et la durée minimale de détection. Nous pouvons également rappeler que les fonctions caractéristiques sont calculées à partir d'un signal modifié appelé enveloppe, et le choix de cette enveloppe constitue également un paramètre important. 

\section{Code source}

\subsection{Exemple}

\inputminted[fontsize=\small,linenos]{python}{./code/sample.py}

\subsection{Code Python}

\inputminted[fontsize=\footnotesize,linenos]{python}{./code/picker.py}

\subsection{Code C}

\inputminted[fontsize=\footnotesize,linenos]{c}{./code/stalta.c}

\newpage 

\bibliographystyle{plainnat}
\bibliography{references}

\end{document}